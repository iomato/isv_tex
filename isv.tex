\documentclass[ebook,11pt,twoside,onecolumn,draft,openright]{memoir}
% ebook (6x9) is not a standard paper size, but that was the request
% closest real paper size is (m)smallroyalvopaper
\usepackage[english]{babel}
\usepackage[pdftex,svgnames]{xcolor}
%\usepackage[pdftex,colorlinks=true,citecolor=black,linkcolor=black,urlcolor=black,draft=false]{hyperref}
\usepackage{lettrine}
\usepackage{needspace}
\usepackage{ledmac}
\usepackage{ifthen}
\usepackage[T1]{fontenc}
\usepackage{mathpazo}
\usepackage{cjhebrew}
\usepackage{verse}
\usepackage{calc}
\usepackage{perpage}
\usepackage{alphalph}
\usepackage{textcomp}
\usepackage{nicefrac}
\usepackage{etoolbox}
\usepackage{wallpaper}
\usepackage{tikz}
\usepackage{ifoddpage}
\let\hangpara\undefined
\let\hangparas\undefined
\let\endhangparas\undefined
\usepackage{hanging}
\usepackage{indentfirst} % requested by Charles
\scrollmode

% Allow enough space for the page numbers in the TOC.
% tocrmarg should be about 1em greater than pnumwidth
\makeatletter
\renewcommand{\@pnumwidth}{3em}
\renewcommand{\@tocrmarg}{4em}
\makeatother

\renewcommand*\sfdefault{uop} % Optima for sans font

% page style
\makepagestyle{headvref}
\makepagestyle{bookstart}
\makeevenhead{headvref}{\rightmark}{\booktab}{\thetestament}
\makeoddhead{headvref}{\thetestament}{\booktab}{\leftmark}
\makeevenfoot{headvref}{}{\thepage}{}
\makeoddfoot{headvref}{}{\thepage}{}
\makeoddhead{bookstart}{\def\thepageref{}}{\booktab}{}
\makeevenhead{bookstart}{\def\thepageref{}}{\booktab}{}
\pagestyle{headvref}

% page layout
\settrims{0pt}{0pt} % stock paper is equal to page size
% calculated using formula in memoir package
% adding another 10pt due to regular verse numbers
\settypeblocksize{*}{332pt+10pt}{*}
\setlrmargins{0.75in}{*}{*}
\setulmarginsandblock{0.5in}{0.5in}{*}
\setheaderspaces{*}{*}{0.2}
\checkandfixthelayout

% Extra page layout bits
% block height / 5:
% (9in-0.5in-0.5in)/5 = 8in/5=1.6in
\def\booktabheight{1.6in}
% 3/4 of the margin
% 0.75in * 3/4 = 0.5625in
\def\booktabwidth{0.5625in}
\def\booktabfontsize{12pt}
% size vertical margin
\def\booktabyoffset{-0.5in}
% offset to booktab on odd pages
% (full width of sheet)
\def\booktaboddxoffset{6in}

% block-side tabs
\newcounter{booktabposition}
\newcommand{\resetbooktab}{\setcounter{booktabposition}{0}}
\newcommand{\preresetbooktab}{\setcounter{booktabposition}{0}}
\newcommand{\advancebooktab}{%
  \addtocounter{booktabposition}{1}
  \ifthenelse{\equal{\value{booktabposition}}{5}}{\setcounter{booktabposition}{0}}
}
\newcommand{\considerbooktab}{\ifthenelse{\equal{\thebook}{}}{}{\booktab}}
\newcommand{\booktab}{%
  \ifthenelse{\equal{\thebook}{}}{}{
    \edef\y{\value{booktabposition}}
    \strictpagechecktrue
    \checkoddpage
    \edef\xl{\ifoddpage{0}\else{\booktaboddxoffset - \booktabwidth}\fi}
    \edef\xr{\ifoddpage{\booktabwidth}\else{\booktaboddxoffset}\fi}
    \ifoddpage{%
      \begin{tikzpicture}[remember picture,overlay]
        \node[yshift=0in] at (current page.north west)
        {\begin{tikzpicture}[remember picture, overlay]
            \path [fill=Gray] (\xl,\booktabyoffset - \y * \booktabheight) rectangle
            (\xr,\booktabyoffset - \y * \booktabheight - 1 * \booktabheight) {};
            \node[anchor=east,xshift=0.5 * \booktabwidth + 0.5 *
            \booktabfontsize,
            yshift=\booktabyoffset - \y * \booktabheight - 0.5 * \booktabheight,rectangle]
            {\rotatebox{90}{\color{white}\fontsize{\booktabfontsize}{\booktabfontsize}\selectfont\MakeUppercase{\thebook}}};
          \end{tikzpicture}
        };
      \end{tikzpicture}
    }\else{%
      \begin{tikzpicture}[remember picture,overlay]
        \node[yshift=0in] at (current page.north west)
        {\begin{tikzpicture}[remember picture, overlay]
            \path [fill=Gray] (\xl,\booktabyoffset - \y * \booktabheight) rectangle
            (\xr,\booktabyoffset - \y * \booktabheight - 1 * \booktabheight) {};
            \node[anchor=east,xshift=\booktaboddxoffset -0.5 * \booktabwidth + 0.5 *
            \booktabfontsize,
            yshift=\booktabyoffset - \y * \booktabheight - 0.5 * \booktabheight,rectangle]
            {\rotatebox{270}{\color{white}\fontsize{\booktabfontsize}{\booktabfontsize}\selectfont\MakeUppercase{\thebook}}};
          \end{tikzpicture}
        };
      \end{tikzpicture}
    }\fi
  }
}


% paragraph spacing
\setlength{\vgap}{1em}
\setlength{\vindent}{3\vgap}
\setlength{\stanzaskip}{0\baselineskip}
\setbeforesubsubsecskip{-1ex plus -.8ex minus -.2ex}
\setaftersubsubsecskip{0.7ex plus .2ex}
\setlength{\emergencystretch}{3em}
\setlength{\parindent}{0.8\parindent}

% Avoid widows and orphans (see
% http://en.wikibooks.org/wiki/LaTeX/Page_Layout#Widows_and_orphans
% for more suggestions if this doesn't work well enough)
\widowpenalty=300
\clubpenalty=300

% footnote style
\paragraphfootnotes
\renewcommand*{\thefootnote}{\textbf{\textrm{\alphalph{\value{footnote}}}}}
%\setlength{\footmarkwidth}{1em}
%\footmarkstyle{#1\hfill}
\setlength{\footmarkwidth}{0em}
%\footmarkstyle{{\scriptsize#1}\hspace{0.4em}}%\textsuperscript{#1}}
\footmarkstyle{\textsuperscript{#1}}
\renewcommand{\foottextfont}{\renewcommand{\v}{\footv}\fontfamily{ptm}\footnotesize}
\MakePerPage{footnote}

% Here we force the footnote rule and text color to black to address a bug
% where \paragraphfootnotes erroneously uses the colour of the last
% paragraph if it spans the page break.  This results in some pages
% having red footnotes.

% Make the horizontal rule for foot notes black regardless of the
% colour of the preceeding text.
% As below it makes the following footnote text black as well. 
\makeatletter
\renewcommand{\footnoterule}{%
  \color{black}%
  \kern-3\p@
  \hrule width .4\columnwidth
  \kern 2.6\p@}
\makeatother
% If the following is used instead, only the rule becomes black, and
% the text remains mixed colour, but with the red/black bugs affecting
% things, so it probably cannot be used at this time.
%\makeatletter
%\renewcommand{\footnoterule}{%
%  {\color{black}%
%  \kern-3\p@
%  \hrule width .4\columnwidth
%  \kern 2.6\p@}}}
%\makeatother

% This increases the incentive to break footnote lines prior to the footnote starting
\makeatletter
\let\old@parafootfmt\@parafootfmt
\def\@parafootfmt#1{\old@parafootfmt{#1}\relax\penalty -1000}
\makeatother

% formatting
\newcommand{\bookheader}[1]{\newpage\thispagestyle{bookstart}\advancebooktab\def\thebook{#1}}
\newcommand{\labelbook}[1]{\def\theshortbook{#1}\def\thechapter{}\def\theverse{}\def\theref{}}
\newcommand{\bookpretitle}[1]{{\centering \large \textsc{#1} \par}\addvspace{-0.6em plus 0.2em}}
\newcommand{\booktitle}[1]{%
  {\centering \Huge \textsc{\textbf{#1}}\par}\addvspace{0.4em plus 0.1em}%
  \addcontentsline{toc}{section}{#1}%
}
\newlength{\hangvwidth}
\newcommand{\verseone}{}
\renewcommand{\v}[1]{\edef\prevref{\theref}%
   \def\theverse{#1}%
   \edef\theref{\thebook{ }\thechapter:\theverse}%
   \markboth{\theref}{\theref}%
   \ifthenelse{\equal{#1}{1}}{\verseone}{\textsuperscript{#1}}}
\makeatletter
\let\old@v\v
\newcommand{\vpoem}[1]{\settowidth{\hangvwidth}{\old@v{#1}}\setlength{\hangvwidth}{0.75\hangvwidth}\ifhmode\kern\hangvwidth\vadjust{}\else\leavevmode\fi\kern-\hangvwidth\old@v{#1}\nobreak\hskip\z@skip}
\makeatother
\newcommand{\footv}[1]{\textsuperscript{#1}}
\newcommand{\fnote}[1]{\footnote{#1}}
\newcommand{\fbib}[1]{\textit{#1}}
\newcommand{\fbackref}[1]{{\scriptsize{#1}}}
%\newcommand{\poempassage}[1]{\subsubsection{\hspace{-4em}\textsf{#1}}~\\\vspace{-1.2\baselineskip}}
\newcommand{\passage}[1]{\par\addvspace{0.1em plus 0.6em}\noindent\textbf{{\textsf{#1}}}\par\nopagebreak\addvspace{0.0em plus 0.2em minus 0.0em}}
\newcommand{\poempassage}[1]{\par\addvspace{0.1em plus 0.6em}\noindent\hspace{-4em}\textbf{\textsf{#1}}\par\nopagebreak\addvspace{0.0em plus 0.2em minus 0.0em}}
\newcommand*{\ensuretwolinepar}{%
    \let\origpar\par
    \def\par{%
        \origpar
        \let\par\origpar
        \ifnum\prevgraf<2 %
            \addvspace{0.6\baselineskip}
        \fi
    }%
}
\newcommand{\chapt}[1]{\lettrine{#1}{}\ensuretwolinepar}
\newcommand{\labelchapt}[1]{\def\thechapter{#1}}
\newcommand{\poeml}{\penalty50\relax}
\newcommand{\poemll}{\penalty150\vin}
\newcommand{\poemlll}{\penalty150\vin\vin}
\newcommand{\divine}[1]{\textsc{#1}}
\newcommand{\passageinfo}[1]{\vspace{-0.3\baselineskip}\subsubsection*{\hspace{0em}\textmd{\textsf{\textit{#1}}}}}
\newcommand{\itemb}[1]{%
    \renewcommand{\footnote}{\bulletfootnote}%
    \item[#1] \bulletfoottext{}%
    \renewcommand{\footnote}{\oldfootnote}%
}
\newcommand{\heb}[1]{\textcjheb{#1}}
\newcommand{\footfract}[2]{\nicefrac{#1}{#2}}
\renewcommand{\textonehalf}{\nicefrac{1}{2}}
\newcommand{\booksection}[1]{{\vspace{\baselineskip}\centering \large \textsc{\textbf{#1}}\par\vspace{\baselineskip}}}
\newcommand{\labelpsalm}[1]{\def\thechapter{#1}\labelchapt{#1}{\centering \textbf{Psalm~#1} \par}}
\newcommand{\psalminfo}[1]{{\small \textmd{\textsf{\textit{#1}}}\par\relax}}
\newcommand{\interlude}[1]{\hfill\textit{#1}\par\relax}
\newcommand{\red}[1]{{\textcolor{FireBrick}{#1}}}

\makeatletter
\newenvironment{poetry}{%
    \renewcommand{\passage}{\poempassage}%
    \renewcommand{\v}{\vpoem}%
    \vspace{-0.7\topsep}%
    \addtolength{\vleftmargin}{0.5em}%
% interfered with quotes
%    \activatepunct%
    \begin{verse}%
}{%
    \end{verse}%
    \vspace{-0.7\topsep}%
    \penalty-100\relax% small bonus for page breaking here
}
\makeatother

\newenvironment{bulletlist}{%
    \gdef\bulletfoottext{}%
    \let\oldfootnote\footnote%
    \providecommand{\bulletfootnote}[1]%
        {\footnotemark \gdef\bulletfoottext{\footnotetext{##1}\gdef\bulletfoottext{}}}%
    \begin{enumerate}%
}{%
    \end{enumerate}%
}

\hyphenation{Naph-tuhites}
\hyphenation{quar-reled}
\hyphenation{when-ever}
\hyphenation{Eph-raim}

% Cross-references
\newcommand{\crossref}[4]{\csdef{xref#1x#2x#3}{#4}}
% Cross-references primarily from public domain Treasury of Scripture Knowledge

% Gen
\crossref{Gen}{1}{1}{Pr 8:22-\allowbreak24; 16:4 Mr 13:19 Joh 1:1-\allowbreak3 Heb 1:10 1Jo 1:1}
\crossref{Gen}{1}{2}{Job 26:7 Isa 45:18 Jer 4:23 Na 2:10}
\crossref{Gen}{1}{3}{Ps 33:6,\allowbreak9; 148:5 Mt 8:3 Joh 11:43}
\crossref{Gen}{1}{4}{1:10,\allowbreak12,\allowbreak18,\allowbreak25,\allowbreak31 Ec 2:13; 11:7}
\crossref{Gen}{1}{5}{Ge 8:22 Ps 19:2; 74:16; 104:20 Isa 45:7 Jer 33:20 1Co 3:13 Eph 5:13}
\crossref{Gen}{1}{6}{1:14,\allowbreak20; 7:11,\allowbreak12 Job 26:7,\allowbreak8,\allowbreak13; 37:11,\allowbreak18; 38:22-\allowbreak26 Ps 19:1; 33:6,\allowbreak9; 104:2}
\crossref{Gen}{1}{7}{Pr 8:28,\allowbreak29}
\crossref{Gen}{1}{8}{1:5,\allowbreak10; 5:2}
\crossref{Gen}{1}{9}{Job 26:7,\allowbreak10; 38:8-\allowbreak11 Ps 24:1,\allowbreak2; 33:7; 95:5; 104:3,\allowbreak5-\allowbreak9; 136:5,\allowbreak6}
\crossref{Gen}{1}{10}{1:4 De 32:4 Ps 104:31}
\crossref{Gen}{1}{11}{Ge 2:5 Job 28:5 Ps 104:14-\allowbreak17; 147:8 Mt 6:30 Heb 6:7}
\crossref{Gen}{1}{12}{Isa 61:11 Mr 4:28}
\crossref{Gen}{1}{13}{}
\crossref{Gen}{1}{14}{De 4:19 Job 25:3,\allowbreak5; 38:12-\allowbreak14 Ps 8:3,\allowbreak4; 19:1-\allowbreak6; 74:16,\allowbreak17; 104:19,\allowbreak20}
\crossref{Gen}{1}{15}{}
\crossref{Gen}{1}{16}{De 4:19 Jos 10:12-\allowbreak14 Job 31:26; 38:7 Ps 8:3; 19:6; 74:16}
\crossref{Gen}{1}{17}{Ge 9:13 Job 38:12 Ps 8:1,\allowbreak3 Ac 13:47}
\crossref{Gen}{1}{18}{Ps 19:6 Jer 31:35}
\crossref{Gen}{1}{19}{}
\crossref{Gen}{1}{20}{1:22; 2:19; 8:17 Ps 104:24,\allowbreak25; 148:10 Ac 17:25}
\crossref{Gen}{1}{21}{Ge 6:20; 7:14; 8:19 Job 7:12; 26:5 Ps 104:24-\allowbreak26 Eze 32:2 Jon 1:17}
\crossref{Gen}{1}{22}{1:28; 8:17; 9:1; 30:27,\allowbreak30; 35:11 Le 26:9 Job 40:15; 42:12 Ps 107:31,\allowbreak38}
\crossref{Gen}{1}{23}{}
\crossref{Gen}{1}{24}{Ge 6:20; 7:14; 8:19 Job 38:39,\allowbreak40; 39:1,\allowbreak5,\allowbreak9,\allowbreak19; 40:15 Ps 50:9,\allowbreak10}
\crossref{Gen}{1}{25}{Ge 2:19,\allowbreak20 Job 12:8-\allowbreak10; 26:13}
\crossref{Gen}{1}{26}{Ge 3:22; 11:7 Job 35:10 Ps 100:3; 149:2 Isa 64:8 Joh 5:17; 14:23}
\crossref{Gen}{1}{27}{Ps 139:14 Isa 43:7 Eph 2:10; 4:24 Col 1:15}
\crossref{Gen}{1}{28}{1:22; 8:17; 9:1,\allowbreak7; 17:16,\allowbreak20; 22:17,\allowbreak18; 24:60; 26:3,\allowbreak4,\allowbreak24; 33:5; 49:25}
\crossref{Gen}{1}{29}{Ps 24:1; 115:16 Ho 2:8 Ac 17:24,\allowbreak25,\allowbreak28 1Ti 6:17}
\crossref{Gen}{1}{30}{Ge 9:3 Job 38:39-\allowbreak41; 39:4,\allowbreak8,\allowbreak30; 40:15,\allowbreak20 Ps 104:14; 145:15,\allowbreak16; 147:9}
\crossref{Gen}{1}{31}{Job 38:7 Ps 19:1,\allowbreak2; 104:24,\allowbreak31 La 3:38 1Ti 4:4}
\crossref{Gen}{2}{1}{2:4; 1:1,\allowbreak10 Ex 20:11; 31:17 2Ki 19:15 2Ch 2:12 Ne 9:6 Job 12:9}
\crossref{Gen}{2}{2}{Ge 1:31 Ex 20:11; 23:12; 31:17 De 5:14 Isa 58:13 Joh 5:17 Heb 4:4}
\crossref{Gen}{2}{3}{Ex 16:22-\allowbreak30; 20:8-\allowbreak11; 23:12; 31:13-\allowbreak17; 34:21; 35:2,\allowbreak3 Le 23:3; 25:2,\allowbreak3}
\crossref{Gen}{2}{4}{Ge 1:4; 5:1; 10:1; 11:10; 25:12,\allowbreak19; 36:1,\allowbreak9 Ex 6:16 Job 38:28 Ps 90:1,\allowbreak2}
\crossref{Gen}{2}{5}{Ge 1:12 Ps 104:14}
\crossref{Gen}{2}{6}{2:6}
\crossref{Gen}{2}{7}{Ps 100:3; 139:14,\allowbreak15 Isa 64:8}
\crossref{Gen}{2}{8}{Ge 13:10 Eze 28:13; 31:8,\allowbreak9 Joe 2:3}
\crossref{Gen}{2}{9}{Eze 31:8,\allowbreak9,\allowbreak16,\allowbreak18}
\crossref{Gen}{2}{10}{Ps 46:4 Re 22:1}
\crossref{Gen}{2}{11}{Ge 10:7,\allowbreak29; 25:18 1Sa 15:7}
\crossref{Gen}{2}{12}{Ex 28:20; 39:13 Job 28:16 Eze 28:13}
\crossref{Gen}{2}{13}{}
\crossref{Gen}{2}{14}{Da 10:4}
\crossref{Gen}{2}{15}{2:2 Job 31:33}
\crossref{Gen}{2}{16}{1Sa 15:22}
\crossref{Gen}{2}{17}{2:9; 3:1-\allowbreak3,\allowbreak11,\allowbreak17,\allowbreak19}
\crossref{Gen}{2}{18}{Ge 1:31; 3:12 Ru 3:1 Pr 18:22 Ec 4:9-\allowbreak12 1Co 7:36}
\crossref{Gen}{2}{19}{Ge 1:20-\allowbreak25}
\crossref{Gen}{2}{20}{2:18}
\crossref{Gen}{2}{21}{Ge 15:12 1Sa 26:12 Job 4:13; 33:15 Pr 19:15 Da 8:18}
\crossref{Gen}{2}{22}{Ps 127:1 1Ti 2:13}
\crossref{Gen}{2}{23}{Ge 29:14 Jud 9:2 2Sa 5:1; 19:13 Eph 5:30}
\crossref{Gen}{2}{24}{Ge 24:58,\allowbreak59; 31:14,\allowbreak15 Ps 45:10}
\crossref{Gen}{2}{25}{Ge 3:7,\allowbreak10,\allowbreak11}
\crossref{Gen}{3}{1}{3:13-\allowbreak15 Isa 27:1 Mt 10:16 2Co 11:3,\allowbreak14 Re 12:9; 20:2}
\crossref{Gen}{3}{2}{Ps 58:4}
\crossref{Gen}{3}{3}{Ge 2:16,\allowbreak17}
\crossref{Gen}{3}{4}{Joh 8:44}
\crossref{Gen}{3}{5}{Ex 20:7 1Ki 22:6 Jer 14:13,\allowbreak14; 28:2,\allowbreak3 Eze 13:2-\allowbreak6,\allowbreak22 2Co 11:3}
\crossref{Gen}{3}{6}{Jos 7:21 Jud 16:1,\allowbreak2}
\crossref{Gen}{3}{7}{3:5 De 28:34 2Ki 6:20 Lu 16:23}
\crossref{Gen}{3}{8}{3:10 De 4:33; 5:25}
\crossref{Gen}{3}{9}{Ge 4:9; 11:5; 16:8; 18:20,\allowbreak21 Jos 7:17-\allowbreak19 Re 20:12,\allowbreak13}
\crossref{Gen}{3}{10}{Ge 2:25 Ex 3:6 Job 23:15 Ps 119:120 Isa 33:14; 57:11 1Jo 3:20}
\crossref{Gen}{3}{11}{Ge 4:10 Ps 50:21 Ro 3:20}
\crossref{Gen}{3}{12}{Ge 2:18,\allowbreak20,\allowbreak22 Ex 32:21-\allowbreak24 1Sa 15:20-\allowbreak24 Job 31:33 Pr 19:3; 28:13}
\crossref{Gen}{3}{13}{Ge 4:10-\allowbreak12; 44:15 1Sa 13:11 2Sa 3:24; 12:9-\allowbreak12 Joh 18:35}
\crossref{Gen}{3}{14}{3:1; 9:6 Ex 21:28-\allowbreak32 Le 20:25}
\crossref{Gen}{3}{15}{Nu 21:6,\allowbreak7 Am 9:3 Mr 16:18 Lu 10:19 Ac 28:3-\allowbreak6 Ro 3:13}
\crossref{Gen}{3}{16}{Ge 35:16-\allowbreak18 1Sa 4:19-\allowbreak21 Ps 48:6 Isa 13:8; 21:3; 26:17,\allowbreak18; 53:11}
\crossref{Gen}{3}{17}{1Sa 15:23,\allowbreak24 Mt 22:12; 25:26,\allowbreak27,\allowbreak45 Lu 19:22 Ro 3:19}
\crossref{Gen}{3}{18}{Jos 23:13 Job 5:5; 31:40 Pr 22:5; 24:31 Isa 5:6; 7:23; 32:13}
\crossref{Gen}{3}{19}{Ec 1:3,\allowbreak13 Eph 4:28 1Th 2:9 2Th 3:10}
\crossref{Gen}{3}{20}{Ge 2:20,\allowbreak23; 5:29; 16:11; 29:32-\allowbreak35; 35:18 Ex 2:10 1Sa 1:20 Mt 1:21,\allowbreak23}
\crossref{Gen}{3}{21}{3:7 Isa 61:10 Ro 3:22 2Co 5:2,\allowbreak3,\allowbreak21}
\crossref{Gen}{3}{22}{3:5; 1:26; 11:6,\allowbreak7 Isa 19:12,\allowbreak13; 47:12,\allowbreak13 Jer 22:23}
\crossref{Gen}{3}{23}{3:19; 2:5; 4:2,\allowbreak12; 9:20 Ec 5:9}
\crossref{Gen}{3}{24}{Ge 2:8}
\crossref{Gen}{4}{1}{Nu 31:17}
\crossref{Gen}{4}{2}{Ge 30:29-\allowbreak31; 37:13; 46:32-\allowbreak34; 47:3 Ex 3:1 Ps 78:70-\allowbreak72 Am 7:15}
\crossref{Gen}{4}{3}{Le 2:1-\allowbreak11 Nu 18:12}
\crossref{Gen}{4}{4}{Ex 13:12 Nu 18:12,\allowbreak17 Pr 3:9 Heb 9:22 1Pe 1:19,\allowbreak20 Re 13:8}
\crossref{Gen}{4}{5}{Nu 16:15 Heb 11:4}
\crossref{Gen}{4}{6}{1Ch 13:11-\allowbreak13 Job 5:2 Isa 1:18 Jer 2:5,\allowbreak31 Joh 4:1-\allowbreak4,\allowbreak8-\allowbreak11}
\crossref{Gen}{4}{7}{Ge 19:21 2Sa 24:23 2Ki 8:28 Job 42:8 Pr 18:5 Ec 8:12,\allowbreak13}
\crossref{Gen}{4}{8}{2Sa 3:27; 13:26-\allowbreak28; 20:9,\allowbreak10 Ne 6:2 Ps 36:3; 55:21 Pr 26:24-\allowbreak26}
\crossref{Gen}{4}{9}{Ge 3:9-\allowbreak11 Ps 9:12}
\crossref{Gen}{4}{10}{Ge 3:13 Jos 7:19 Ps 50:21}
\crossref{Gen}{4}{11}{4:14; 3:14 De 27:16-\allowbreak26; 28:15-\allowbreak20; 29:19-\allowbreak21 Ga 3:10}
\crossref{Gen}{4}{12}{Ge 3:17,\allowbreak18 Le 26:20 De 28:23,\allowbreak24 Ro 8:20}
\crossref{Gen}{4}{13}{}
\crossref{Gen}{4}{14}{Job 15:20-\allowbreak24 Pr 14:32; 28:1 Isa 8:22 Ho 13:3}
\crossref{Gen}{4}{15}{1Ki 16:7 Ps 59:11 Ho 1:4 Mt 26:52}
\crossref{Gen}{4}{16}{4:14; 3:8 Ex 20:18 2Ki 13:23; 24:20 Job 1:12; 2:7; 20:17 Ps 5:11; 68:2}
\crossref{Gen}{4}{17}{Ge 5:18,\allowbreak22}
\crossref{Gen}{4}{18}{Ge 5:21; 36:2}
\crossref{Gen}{4}{19}{Ge 2:18,\allowbreak24 Mt 19:4-\allowbreak6,\allowbreak8}
\crossref{Gen}{4}{20}{4:21 1Ch 2:50-\allowbreak52; 4:4,\allowbreak5 Joh 8:44 Ro 4:11,\allowbreak12}
\crossref{Gen}{4}{21}{Ro 4:11,\allowbreak12}
\crossref{Gen}{4}{22}{Ex 25:3 Nu 31:22 De 8:9; 33:25 2Ch 2:7}
\crossref{Gen}{4}{23}{Nu 23:18 Jud 9:7}
\crossref{Gen}{4}{24}{4:15}
\crossref{Gen}{4}{25}{Ge 5:3,\allowbreak4 1Ch 1:1 Lu 3:38}
\crossref{Gen}{4}{26}{4:6-\allowbreak8}
\crossref{Gen}{5}{1}{Ge 1:26,\allowbreak27 Ec 7:29; 12:1 1Co 11:7 2Co 3:18 Eph 4:24 Col 3:10}
\crossref{Gen}{5}{2}{Ge 1:27 Mal 2:15}
\crossref{Gen}{5}{3}{Job 14:4; 15:14-\allowbreak16; 25:4 Ps 14:2,\allowbreak3; 51:5 Lu 1:35 Joh 3:6 Ro 5:12}
\crossref{Gen}{5}{4}{1Ch 1:1-\allowbreak3 Lu 3:36-\allowbreak38}
\crossref{Gen}{5}{5}{5:8,\allowbreak11,\allowbreak14,\allowbreak17 etc.}
\crossref{Gen}{5}{6}{Ge 4:26}
\crossref{Gen}{5}{7}{}
\crossref{Gen}{5}{8}{5:8}
\crossref{Gen}{5}{9}{1Ch 1:2 Lu 3:37}
\crossref{Gen}{5}{10}{5:4}
\crossref{Gen}{5}{11}{5:5}
\crossref{Gen}{5}{12}{Lu 3:37}
\crossref{Gen}{5}{13}{5:4}
\crossref{Gen}{5}{14}{5:5}
\crossref{Gen}{5}{15}{1Ch 1:2}
\crossref{Gen}{5}{16}{5:4}
\crossref{Gen}{5}{17}{5:5}
\crossref{Gen}{5}{18}{Ge 4:17 1Ch 1:3}
\crossref{Gen}{5}{19}{5:4}
\crossref{Gen}{5}{20}{5:5}
\crossref{Gen}{5}{21}{Lu 3:37}
\crossref{Gen}{5}{22}{Ge 6:9; 17:1; 24:40; 48:15 Ex 16:4 Le 26:12 De 5:33; 13:4; 28:9}
\crossref{Gen}{5}{23}{5:23}
\crossref{Gen}{5}{24}{5:21}
\crossref{Gen}{5}{25}{Ge 4:18}
\crossref{Gen}{5}{26}{5:4}
\crossref{Gen}{5}{27}{5:5}
\crossref{Gen}{5}{28}{5:28}
\crossref{Gen}{5}{29}{Ge 6:8,\allowbreak9; 7:23; 9:24 Isa 54:9 Eze 14:14,\allowbreak20 Mt 24:37 Lu 3:36}
\crossref{Gen}{5}{30}{5:4}
\crossref{Gen}{5}{31}{5:5}
\crossref{Gen}{5}{32}{Ge 6:10; 7:13; 9:18,\allowbreak19,\allowbreak22-\allowbreak27; 10:1,\allowbreak21,\allowbreak32 1Ch 1:4-\allowbreak28 Lu 3:36}
\crossref{Gen}{6}{1}{Ge 1:28}
\crossref{Gen}{6}{2}{Ge 4:26 Ex 4:22,\allowbreak23 De 14:1 Ps 82:6,\allowbreak7 Isa 63:16 Mal 2:11 Joh 8:41}
\crossref{Gen}{6}{3}{Nu 11:17 Ne 9:30 Isa 5:4; 63:10 Jer 11:7,\allowbreak11 Ac 7:51 Ga 5:16,\allowbreak17}
\crossref{Gen}{6}{4}{Nu 13:33 De 2:20,\allowbreak21; 3:11 1Sa 17:4 2Sa 21:15-\allowbreak22}
\crossref{Gen}{6}{5}{Ge 13:13; 18:20,\allowbreak21 Ps 14:1-\allowbreak4; 53:2 Ro 1:28-\allowbreak31; 3:9-\allowbreak19}
\crossref{Gen}{6}{6}{Ex 32:14 Nu 23:19 De 32:36 1Sa 15:11,\allowbreak29 2Sa 24:16 1Ch 21:15}
\crossref{Gen}{6}{7}{Ps 24:1,\allowbreak2; 37:20 Pr 10:27; 16:4}
\crossref{Gen}{6}{8}{Ge 19:19 Ex 33:12-\allowbreak17 Ps 84:11; 145:20 Pr 3:4; 8:35; 12:2 Jer 31:2}
\crossref{Gen}{6}{9}{Ge 2:4; 5:1; 10:1}
\crossref{Gen}{6}{10}{Ge 5:32}
\crossref{Gen}{6}{11}{Ge 7:1; 10:9; 13:13 2Ch 34:27 Lu 1:6 Ro 2:13; 3:19}
\crossref{Gen}{6}{12}{6:8; 18:21 Job 33:27 Ps 14:2; 33:13,\allowbreak14; 53:2,\allowbreak3 Pr 15:3}
\crossref{Gen}{6}{13}{Jer 51:13 Eze 7:2-\allowbreak6 Am 8:2 1Pe 4:7}
\crossref{Gen}{6}{14}{Mt 24:38 Lu 17:27 1Pe 3:20}
\crossref{Gen}{6}{15}{Ge 7:20 De 3:11}
\crossref{Gen}{6}{16}{Ge 8:6 2Sa 6:16 2Ki 9:30}
\crossref{Gen}{6}{17}{6:13; 7:4,\allowbreak21-\allowbreak23; 9:9 Ex 14:17 Le 26:28 De 32:39 Ps 29:10 Isa 51:12}
\crossref{Gen}{6}{18}{Ge 9:9,\allowbreak11; 17:4,\allowbreak7,\allowbreak21}
\crossref{Gen}{6}{19}{Ge 7:2,\allowbreak3,\allowbreak8,\allowbreak9,\allowbreak15,\allowbreak16; 8:17 Ps 36:6}
\crossref{Gen}{6}{20}{Ge 1:20-\allowbreak24 Ac 10:11,\allowbreak12}
\crossref{Gen}{6}{21}{Ge 1:29,\allowbreak30 Job 38:41; 40:20 Ps 35:6; 104:27,\allowbreak28; 136:25; 145:16; 147:9}
\crossref{Gen}{6}{22}{Ge 7:5,\allowbreak9,\allowbreak16; 17:23 Ex 40:16,\allowbreak19,\allowbreak21,\allowbreak23,\allowbreak25,\allowbreak27,\allowbreak32 De 12:32 Mt 7:24-\allowbreak27}
\crossref{Gen}{7}{1}{7:7,\allowbreak13 Job 5:19-\allowbreak24 Ps 91:1-\allowbreak10 Pr 14:26; 18:10 Isa 26:20,\allowbreak21}
\crossref{Gen}{7}{2}{7:8; 6:19-\allowbreak21; 8:20 Le 11:1-\allowbreak47 De 14:1-\allowbreak21 Ac 10:11-\allowbreak15}
\crossref{Gen}{7}{3}{}
\crossref{Gen}{7}{4}{7:10; 2:5; 6:3; 8:10,\allowbreak12; 29:27,\allowbreak28 Job 28:25; 36:27-\allowbreak32; 37:11,\allowbreak12 Am 4:7}
\crossref{Gen}{7}{5}{Ge 6:22 Ex 39:32,\allowbreak42,\allowbreak43; 40:16 Ps 119:6 Mt 3:15 Lu 8:21 Joh 2:5}
\crossref{Gen}{7}{6}{Ge 5:32; 8:13}
\crossref{Gen}{7}{7}{7:1,\allowbreak13-\allowbreak15; 6:18 Pr 22:3 Mt 24:38 Lu 17:27 Heb 6:18; 11:7 1Pe 3:20}
\crossref{Gen}{7}{8}{7:8}
\crossref{Gen}{7}{9}{7:16; 2:19 Isa 11:6-\allowbreak9; 65:25 Jer 8:7 Ac 10:11,\allowbreak12 Ga 3:28 Col 3:11}
\crossref{Gen}{7}{10}{7:4}
\crossref{Gen}{7}{11}{Ge 1:7; 6:17; 8:2 Job 28:4; 38:8-\allowbreak11 Ps 33:7; 74:15 Pr 8:28,\allowbreak29}
\crossref{Gen}{7}{12}{7:4,\allowbreak17 Ex 24:18 De 9:9,\allowbreak18; 10:10 1Ki 19:8 Mt 4:2}
\crossref{Gen}{7}{13}{7:1,\allowbreak7-\allowbreak9; 6:18 Heb 11:7 1Pe 3:20 2Pe 2:5}
\crossref{Gen}{7}{14}{7:2,\allowbreak3,\allowbreak8,\allowbreak9}
\crossref{Gen}{7}{15}{Ge 6:20 Isa 11:6}
\crossref{Gen}{7}{16}{7:2,\allowbreak3}
\crossref{Gen}{7}{17}{7:4,\allowbreak12}
\crossref{Gen}{7}{18}{Ex 14:28 Job 22:16 Ps 69:15}
\crossref{Gen}{7}{19}{}
\crossref{Gen}{7}{20}{Ps 104:6 Jer 3:23}
\crossref{Gen}{7}{21}{7:4; 6:6,\allowbreak7,\allowbreak13,\allowbreak17 Job 22:15-\allowbreak17 Isa 24:6,\allowbreak19 Jer 4:22-\allowbreak27; 12:3,\allowbreak4 Ho 4:3}
\crossref{Gen}{7}{22}{Ge 2:7; 6:17}
\crossref{Gen}{7}{23}{Ex 14:28-\allowbreak30 Job 5:19 Ps 91:1,\allowbreak9,\allowbreak10 Pr 11:4 Eze 14:14-\allowbreak20}
\crossref{Gen}{7}{24}{Ge 8:3,\allowbreak4}
\crossref{Gen}{8}{1}{Ge 19:29; 30:22 Ex 2:24 1Sa 1:19 Ne 13:14,\allowbreak22,\allowbreak29,\allowbreak31 Job 14:13}
\crossref{Gen}{8}{2}{Ge 7:11 Pr 8:28 Jon 2:3}
\crossref{Gen}{8}{3}{Ge 7:11,\allowbreak24}
\crossref{Gen}{8}{4}{Ge 7:17-\allowbreak19}
\crossref{Gen}{8}{5}{}
\crossref{Gen}{8}{6}{Ge 6:16 Da 6:10}
\crossref{Gen}{8}{7}{Le 11:15 1Ki 17:4,\allowbreak6 Job 38:41 Ps 147:9}
\crossref{Gen}{8}{8}{8:10-\allowbreak12 So 1:15; 2:11,\allowbreak12,\allowbreak14 Mt 10:16}
\crossref{Gen}{8}{9}{De 28:65 Eze 7:16 Mt 11:28 Joh 16:33}
\crossref{Gen}{8}{10}{Ps 40:1 Isa 8:17; 26:8 Ro 8:25}
\crossref{Gen}{8}{11}{Ne 8:15 Zec 4:12-\allowbreak14 Ro 10:15}
\crossref{Gen}{8}{12}{Ps 27:14; 130:5,\allowbreak6 Isa 8:17; 25:9; 26:8; 30:18 Hab 2:3 Jas 5:7,\allowbreak8}
\crossref{Gen}{8}{13}{Ge 7:11}
\crossref{Gen}{8}{14}{Ge 7:11,\allowbreak13,\allowbreak14}
\crossref{Gen}{8}{15}{8:15}
\crossref{Gen}{8}{16}{Ge 7:1,\allowbreak7,\allowbreak13 Jos 3:17; 4:10,\allowbreak16-\allowbreak18 Ps 91:11; 121:8 Da 9:25,\allowbreak26}
\crossref{Gen}{8}{17}{Ge 7:14,\allowbreak15}
\crossref{Gen}{8}{18}{Ps 121:8}
\crossref{Gen}{8}{19}{8:19}
\crossref{Gen}{8}{20}{Ge 4:4; 12:7,\allowbreak8; 13:4; 22:9; 26:25; 33:20; 35:1,\allowbreak7 Ex 20:24,\allowbreak25; 24:4-\allowbreak8}
\crossref{Gen}{8}{21}{Le 1:9,\allowbreak13,\allowbreak17; 26:31 So 4:10,\allowbreak11 Isa 65:6 Eze 20:41 Am 5:21,\allowbreak22}
\crossref{Gen}{8}{22}{Jer 31:35; 33:20-\allowbreak26}
\crossref{Gen}{9}{1}{9:7; 1:22,\allowbreak28; 2:3; 8:17; 24:60 Ps 112:1; 128:3,\allowbreak4 Isa 51:2}
\crossref{Gen}{9}{2}{Ge 1:28; 2:19; 35:5 Le 26:6,\allowbreak22 Job 5:22,\allowbreak23 Ps 8:4-\allowbreak8; 104:20-\allowbreak23}
\crossref{Gen}{9}{3}{Le 11:1-\allowbreak47; 22:8 De 12:15; 14:3-\allowbreak21 Ac 10:12-\allowbreak15 1Ti 4:3-\allowbreak5}
\crossref{Gen}{9}{4}{Le 3:17; 7:26; 17:10-\allowbreak14; 19:26 De 12:16,\allowbreak23; 14:21; 15:23 1Sa 14:34}
\crossref{Gen}{9}{5}{Ex 21:12,\allowbreak28,\allowbreak29}
\crossref{Gen}{9}{6}{Ex 21:12-\allowbreak14; 22:2,\allowbreak3 Le 17:4; 24:17 Nu 35:25 1Ki 2:5,\allowbreak6,\allowbreak28-\allowbreak34}
\crossref{Gen}{9}{7}{9:1,\allowbreak19; 1:28; 8:17}
\crossref{Gen}{9}{8}{9:8}
\crossref{Gen}{9}{9}{9:11,\allowbreak17; 6:18; 17:7,\allowbreak8; 22:17 Isa 54:9,\allowbreak10 Jer 31:35,\allowbreak36; 33:20 Ro 1:3}
\crossref{Gen}{9}{10}{9:15,\allowbreak16; 8:1 Job 38:1-\allowbreak41; 41:1-\allowbreak34 Ps 36:5,\allowbreak6; 145:9 Jon 4:11}
\crossref{Gen}{9}{11}{Ge 8:21,\allowbreak22 Isa 54:9}
\crossref{Gen}{9}{12}{Ge 17:11 Ex 12:13; 13:16 Jos 2:12 Mt 26:26-\allowbreak28 1Co 11:23-\allowbreak25}
\crossref{Gen}{9}{13}{Eze 1:28 Re 4:3; 10:1}
\crossref{Gen}{9}{14}{}
\crossref{Gen}{9}{15}{Ex 28:12 Le 26:42-\allowbreak45 De 7:9 1Ki 8:23 Ne 9:32 Ps 106:45}
\crossref{Gen}{9}{16}{9:9-\allowbreak11; 8:21,\allowbreak22; 17:13,\allowbreak19 2Sa 23:5 Ps 89:3,\allowbreak4 Isa 54:8-\allowbreak10; 55:3 Jer 32:40}
\crossref{Gen}{9}{17}{}
\crossref{Gen}{9}{18}{9:23; 10:1 1Ch 1:4}
\crossref{Gen}{9}{19}{Ge 5:32}
\crossref{Gen}{9}{20}{Ge 3:18,\allowbreak19,\allowbreak23; 4:2; 5:29 Pr 10:11; 12:11 Ec 5:9 Isa 28:24-\allowbreak26}
\crossref{Gen}{9}{21}{Ge 6:9; 19:32-\allowbreak36 Pr 20:1; 23:31,\allowbreak32 Ec 7:20 Lu 22:3,\allowbreak4 Ro 13:13}
\crossref{Gen}{9}{22}{9:25; 10:6,\allowbreak15-\allowbreak19 1Ch 1:8,\allowbreak13-\allowbreak16}
\crossref{Gen}{9}{23}{Ex 20:12 Le 19:32 Ro 13:7 Ga 6:1 1Ti 5:1,\allowbreak17,\allowbreak19 1Pe 2:17; 4:8}
\crossref{Gen}{9}{24}{}
\crossref{Gen}{9}{25}{9:22; 3:14; 4:11; 49:7 De 27:16; 28:18 Mt 25:41 Joh 8:34}
\crossref{Gen}{9}{26}{De 33:26 Ps 144:15 Ro 9:5}
\crossref{Gen}{9}{27}{}
\crossref{Gen}{9}{28}{}
\crossref{Gen}{9}{29}{Ge 5:5,\allowbreak20,\allowbreak27,\allowbreak32; 11:11-\allowbreak25 Ps 90:10}
\crossref{Gen}{10}{1}{Ge 2:4; 5:1; 6:9 Mt 1:1}
\crossref{Gen}{10}{2}{10:21 1Ch 1:5-\allowbreak7 Isa 66:19 Eze 27:7,\allowbreak12-\allowbreak14,\allowbreak19; 38:2,\allowbreak6,\allowbreak15; 39:1}
\crossref{Gen}{10}{3}{}
\crossref{Gen}{10}{4}{Nu 24:24 Isa 23:1,\allowbreak12 Da 11:30}
\crossref{Gen}{10}{5}{10:25 Ps 72:10 Isa 24:15; 40:15; 41:5; 42:4,\allowbreak10; 49:1; 51:5; 59:18}
\crossref{Gen}{10}{6}{Ge 9:22 1Ch 1:8-\allowbreak16; 4:40 Ps 78:51; 105:23,\allowbreak27; 106:22}
\crossref{Gen}{10}{7}{Ps 72:10}
\crossref{Gen}{10}{8}{Mic 5:6}
\crossref{Gen}{10}{9}{Ge 6:4; 25:27; 27:30; 27:30 Jer 16:16 Eze 13:18 Mic 7:2}
\crossref{Gen}{10}{10}{Jer 50:21 Mic 5:6}
\crossref{Gen}{10}{11}{Nu 24:22,\allowbreak24 Ezr 4:2 Ps 83:8 Eze 27:23; 32:22 Ho 14:3}
\crossref{Gen}{10}{12}{}
\crossref{Gen}{10}{13}{1Ch 1:11,\allowbreak12 Jer 46:9 Eze 30:5}
\crossref{Gen}{10}{14}{Isa 11:11 Jer 44:1}
\crossref{Gen}{10}{15}{1Ch 1:13}
\crossref{Gen}{10}{16}{Jud 1:21 2Sa 24:18 Zec 9:7}
\crossref{Gen}{10}{17}{Ge 34:2}
\crossref{Gen}{10}{18}{Eze 27:8}
\crossref{Gen}{10}{19}{Ge 13:12-\allowbreak17; 15:18-\allowbreak21 Nu 34:2-\allowbreak15 De 32:8 Jos 12:7,\allowbreak8; 14:1-\allowbreak21:45}
\crossref{Gen}{10}{20}{10:6; 11:1-\allowbreak9}
\crossref{Gen}{10}{21}{Ge 11:10-\allowbreak26}
\crossref{Gen}{10}{22}{Ge 9:26 1Ch 1:17-\allowbreak27}
\crossref{Gen}{10}{23}{Job 1:1 Jer 25:20}
\crossref{Gen}{10}{24}{Ge 11:12-\allowbreak15}
\crossref{Gen}{10}{25}{10:21 1Ch 1:19}
\crossref{Gen}{10}{26}{1Ch 1:20-\allowbreak28}
\crossref{Gen}{10}{27}{1Ch 1:20-\allowbreak28}
\crossref{Gen}{10}{28}{Ge 25:3 1Ki 10:1 1Ch 1:20-\allowbreak28}
\crossref{Gen}{10}{29}{1Ki 9:28; 22:48 1Ch 8:18; 9:10,\allowbreak13 Job 22:24; 28:16 Ps 45:9}
\crossref{Gen}{10}{30}{Nu 23:7}
\crossref{Gen}{10}{31}{10:5,\allowbreak20 Ac 17:26}
\crossref{Gen}{10}{32}{10:1,\allowbreak20,\allowbreak31; 5:29-\allowbreak31}
\crossref{Gen}{11}{1}{Isa 19:18 Zep 3:9 Ac 2:6}
\crossref{Gen}{11}{2}{Ge 13:11}
\crossref{Gen}{11}{3}{Heb 3:13; 10:24}
\crossref{Gen}{11}{4}{De 1:28; 9:1 Da 4:11,\allowbreak22}
\crossref{Gen}{11}{5}{Ge 18:21 Ex 19:11 Ps 11:4; 33:13,\allowbreak14 Jer 23:23,\allowbreak24 Joh 3:13 Heb 4:13}
\crossref{Gen}{11}{6}{Ge 3:22 Jud 10:14 1Ki 18:27 Ec 11:9}
\crossref{Gen}{11}{7}{11:5; 1:26; 3:22 Isa 6:8}
\crossref{Gen}{11}{8}{11:4,\allowbreak9; 49:7 De 32:8 Lu 1:51}
\crossref{Gen}{11}{9}{Ge 10:25,\allowbreak32 Ac 17:26}
\crossref{Gen}{11}{10}{11:27; 10:21,\allowbreak22 1Ch 1:17-\allowbreak27 Lu 3:34-\allowbreak36}
\crossref{Gen}{11}{11}{Ge 5:4 etc.}
\crossref{Gen}{11}{12}{Lu 3:36}
\crossref{Gen}{11}{13}{11:13}
\crossref{Gen}{11}{14}{11:14}
\crossref{Gen}{11}{15}{11:15}
\crossref{Gen}{11}{16}{Ge 10:21,\allowbreak25 Nu 24:24 1Ch 1:19}
\crossref{Gen}{11}{17}{11:17}
\crossref{Gen}{11}{18}{Lu 3:35}
\crossref{Gen}{11}{19}{11:19}
\crossref{Gen}{11}{20}{Lu 3:35}
\crossref{Gen}{11}{21}{11:21}
\crossref{Gen}{11}{22}{Jos 24:2}
\crossref{Gen}{11}{23}{11:23}
\crossref{Gen}{11}{24}{Lu 3:34}
\crossref{Gen}{11}{25}{11:25}
\crossref{Gen}{11}{26}{Ge 12:4,\allowbreak5; 22:20-\allowbreak24; 29:4,\allowbreak5 Jos 24:2 1Ch 1:26,\allowbreak27}
\crossref{Gen}{11}{27}{11:31; 12:4; 13:1-\allowbreak11; 14:12; 19:1-\allowbreak29 2Pe 2:7}
\crossref{Gen}{11}{28}{Ge 15:7 Ne 9:7 Ac 7:2-\allowbreak4}
\crossref{Gen}{11}{29}{Ge 17:15; 20:12}
\crossref{Gen}{11}{30}{Ge 15:2,\allowbreak3; 16:1,\allowbreak2; 18:11,\allowbreak12; 21:1,\allowbreak2; 25:21; 29:31; 30:1,\allowbreak2 Jud 13:2}
\crossref{Gen}{11}{31}{11:26,\allowbreak27; 12:1}
\crossref{Gen}{11}{32}{11:32}
\crossref{Gen}{12}{1}{Ge 11:31,\allowbreak32; 15:7 Ne 9:7 Isa 41:9; 51:2 Eze 33:24}
\crossref{Gen}{12}{2}{Ge 13:16; 15:5; 17:5,\allowbreak6; 18:18; 22:17,\allowbreak18; 24:35; 26:4; 27:29; 28:3,\allowbreak14}
\crossref{Gen}{12}{3}{Ge 27:29 Ex 23:22 Nu 24:9 Mt 25:40,\allowbreak45}
\crossref{Gen}{12}{4}{Ge 11:27}
\crossref{Gen}{12}{5}{Ge 14:14,\allowbreak21}
\crossref{Gen}{12}{6}{Heb 11:9}
\crossref{Gen}{12}{7}{Ge 17:1; 18:1; 32:30}
\crossref{Gen}{12}{8}{Ge 28:19; 35:3,\allowbreak15,\allowbreak16 Jos 8:17; 18:22 Ne 11:31}
\crossref{Gen}{12}{9}{Ge 13:3; 24:62 Ps 105:13 Heb 11:13,\allowbreak14}
\crossref{Gen}{12}{10}{Ge 26:1; 42:5; 43:1; 47:13 Ru 1:1 2Sa 21:1 1Ki 17:1-\allowbreak18:46 2Ki 4:38}
\crossref{Gen}{12}{11}{12:14; 26:7; 29:17; 39:6,\allowbreak7 2Sa 11:2 Pr 21:30 So 1:14}
\crossref{Gen}{12}{12}{Ge 20:11; 26:7 1Sa 27:1 Pr 29:25 Mt 10:28 1Jo 1:8-\allowbreak10}
\crossref{Gen}{12}{13}{Joh 8:44 Ro 3:6-\allowbreak8; 6:23 Col 3:6}
\crossref{Gen}{12}{14}{Ge 3:6; 6:2; 39:7 Mt 5:28}
\crossref{Gen}{12}{15}{Es 2:2-\allowbreak16 Pr 29:12 Ho 7:4,\allowbreak5}
\crossref{Gen}{12}{16}{Ge 13:2; 20:14}
\crossref{Gen}{12}{17}{Ge 20:18 1Ch 16:21; 21:22 Job 34:19 Ps 105:14,\allowbreak15 Heb 13:4}
\crossref{Gen}{12}{18}{Ge 3:13; 4:10; 20:9,\allowbreak10; 26:9-\allowbreak11; 31:26; 44:15 Ex 32:21 Jos 7:19}
\crossref{Gen}{12}{19}{12:19}
\crossref{Gen}{12}{20}{Ex 18:27 1Sa 29:6-\allowbreak11 Ps 105:14,\allowbreak15 Pr 21:1}
\crossref{Gen}{13}{1}{}
\crossref{Gen}{13}{2}{Ge 24:35; 26:12,\allowbreak13 De 8:18 1Sa 2:7 Job 1:3,\allowbreak10; 22:21-\allowbreak25 Ps 112:1-\allowbreak3}
\crossref{Gen}{13}{3}{Ge 12:6,\allowbreak8,\allowbreak9}
\crossref{Gen}{13}{4}{13:18; 12:7,\allowbreak8; 35:1-\allowbreak3 Ps 26:8; 42:1,\allowbreak2; 84:1,\allowbreak2,\allowbreak10}
\crossref{Gen}{13}{5}{Ge 4:20; 25:27 Jer 49:29}
\crossref{Gen}{13}{6}{Ge 36:6,\allowbreak7 Ec 5:10,\allowbreak11 Lu 12:17,\allowbreak18 1Ti 6:9}
\crossref{Gen}{13}{7}{Ge 21:25; 26:20 Ex 2:17 1Co 3:3 Ga 5:20 Tit 3:3 Jas 3:16; 4:1}
\crossref{Gen}{13}{8}{Pr 15:1 Mt 5:9 1Co 6:6,\allowbreak7 Php 2:14 Heb 12:14 Jas 3:17,\allowbreak18}
\crossref{Gen}{13}{9}{Ge 20:15; 34:10}
\crossref{Gen}{13}{10}{Ge 3:6; 6:2 Nu 32:1 etc.}
\crossref{Gen}{13}{11}{Ge 19:17}
\crossref{Gen}{13}{12}{Ge 19:29}
\crossref{Gen}{13}{13}{Ge 15:16; 18:20; 19:4 etc.}
\crossref{Gen}{13}{14}{13:11}
\crossref{Gen}{13}{15}{Ge 12:7; 15:18; 17:7,\allowbreak8; 18:18; 24:7; 26:3,\allowbreak4; 28:4,\allowbreak13; 35:12; 48:4}
\crossref{Gen}{13}{16}{Ge 12:2,\allowbreak3; 15:5; 17:6,\allowbreak16,\allowbreak20; 18:18; 21:13; 22:17; 25:1-\allowbreak34; 26:4}
\crossref{Gen}{13}{17}{}
\crossref{Gen}{13}{18}{Ge 14:13; 18:1}
\crossref{Gen}{14}{1}{Ge 10:10; 11:2 Isa 11:11 Da 1:2 Zec 5:11}
\crossref{Gen}{14}{2}{Ge 10:19; 13:10; 19:24 Isa 1:9,\allowbreak10}
\crossref{Gen}{14}{3}{Ge 19:24 Nu 34:12 De 3:17 Jos 3:16 Ps 107:34}
\crossref{Gen}{14}{4}{Ge 9:25,\allowbreak26}
\crossref{Gen}{14}{5}{Ge 15:20 De 3:11,\allowbreak20,\allowbreak22 2Sa 5:18,\allowbreak22; 23:13 1Ch 11:15; 14:9 Isa 17:5}
\crossref{Gen}{14}{6}{Ge 36:8,\allowbreak20-\allowbreak30 De 2:12,\allowbreak22 1Ch 1:38-\allowbreak42}
\crossref{Gen}{14}{7}{Ge 36:12,\allowbreak16 Ex 17:8-\allowbreak16 Nu 14:43,\allowbreak45; 24:20 1Sa 15:1-\allowbreak35; 27:1-\allowbreak12}
\crossref{Gen}{14}{8}{14:2; 13:10; 19:20,\allowbreak22}
\crossref{Gen}{14}{9}{14:1}
\crossref{Gen}{14}{10}{Jos 8:24 Ps 83:10 Isa 24:18 Jer 48:44}
\crossref{Gen}{14}{11}{14:16,\allowbreak21; 12:5 De 28:31,\allowbreak35,\allowbreak51}
\crossref{Gen}{14}{12}{Ge 11:27; 12:5}
\crossref{Gen}{14}{13}{1Sa 4:12 Job 1:15}
\crossref{Gen}{14}{14}{Ge 11:27-\allowbreak31; 13:8 Pr 17:17; 24:11,\allowbreak12 Ga 6:1,\allowbreak2 1Jo 2:18}
\crossref{Gen}{14}{15}{Ps 112:5}
\crossref{Gen}{14}{16}{14:11,\allowbreak12; 12:2 1Sa 30:8,\allowbreak18,\allowbreak19 Isa 41:2}
\crossref{Gen}{14}{17}{Jud 11:34 1Sa 18:6 Pr 14:20; 19:4}
\crossref{Gen}{14}{18}{Ps 76:2 Heb 7:1,\allowbreak2}
\crossref{Gen}{14}{19}{Ge 27:4,\allowbreak25-\allowbreak29; 47:7,\allowbreak10; 48:9-\allowbreak16; 49:28 Nu 6:23-\allowbreak27 Mr 10:16 Heb 7:6,\allowbreak7}
\crossref{Gen}{14}{20}{Ge 9:26; 24:27 Ps 68:19; 72:17-\allowbreak19; 144:1 Eph 1:3 1Pe 1:3,\allowbreak4}
\crossref{Gen}{14}{21}{14:21}
\crossref{Gen}{14}{22}{Ex 6:8 De 32:40 Da 12:7 Re 10:5,\allowbreak6}
\crossref{Gen}{14}{23}{1Ki 13:8 2Ki 5:16,\allowbreak20 Es 9:15,\allowbreak16 2Co 11:9-\allowbreak11; 12:14}
\crossref{Gen}{14}{24}{Pr 3:27 Mt 7:12 Ro 13:7,\allowbreak8}
\crossref{Gen}{15}{1}{Ge 46:2 Nu 12:6 1Sa 9:9 Eze 1:1; 3:4; 11:24 Da 10:1-\allowbreak16 Ac 10:10-\allowbreak17}
\crossref{Gen}{15}{2}{Ge 12:1-\allowbreak3}
\crossref{Gen}{15}{3}{Ge 12:2; 13:16 Pr 13:12 Jer 12:1 Heb 10:35,\allowbreak36}
\crossref{Gen}{15}{4}{Ge 17:16; 21:12 2Sa 7:12; 16:11 2Ch 32:21 Phm 1:12}
\crossref{Gen}{15}{5}{De 1:10 Ps 147:4 Jer 33:22 Ro 9:7,\allowbreak8}
\crossref{Gen}{15}{6}{Ro 4:3-\allowbreak6,\allowbreak9,\allowbreak20-\allowbreak25 Ga 3:6-\allowbreak14 Heb 11:8 Jas 2:23}
\crossref{Gen}{15}{7}{Ge 11:28-\allowbreak31; 12:1 Ne 9:7 Ac 7:2-\allowbreak4}
\crossref{Gen}{15}{8}{Ge 24:2-\allowbreak4,\allowbreak13,\allowbreak14 Jud 6:17-\allowbreak24,\allowbreak36-\allowbreak40 1Sa 14:9,\allowbreak10 2Ki 20:8 Ps 86:17}
\crossref{Gen}{15}{9}{Ge 22:13 Le 1:3,\allowbreak10,\allowbreak14; 3:1,\allowbreak6; 9:2,\allowbreak4; 12:8; 14:22,\allowbreak30 Ps 50:5 Isa 15:5}
\crossref{Gen}{15}{10}{Jer 34:18,\allowbreak19 2Ti 2:15}
\crossref{Gen}{15}{11}{Eze 17:3,\allowbreak7 Mt 13:4}
\crossref{Gen}{15}{12}{Ge 2:21 1Sa 26:12 Job 4:13,\allowbreak14; 33:15 Da 10:8,\allowbreak9 Ac 20:9}
\crossref{Gen}{15}{13}{Ge 17:8 Ex 1:1-\allowbreak2:25 5:1-\allowbreak23 22:21 23:9 Le 19:34 De 10:19 Ps 105:11}
\crossref{Gen}{15}{14}{Ge 46:1-\allowbreak34 Ex 6:5,\allowbreak6; 7:1-\allowbreak14:31 De 4:20; 6:22; 7:18,\allowbreak19; 11:2-\allowbreak4}
\crossref{Gen}{15}{15}{Ge 25:8 Nu 20:24; 27:13 Jud 2:10 Job 5:26 Ec 12:7 Ac 13:36}
\crossref{Gen}{15}{16}{Ex 12:40}
\crossref{Gen}{15}{17}{Ex 3:2,\allowbreak3 De 4:20 Jud 6:21; 13:20 1Ch 21:26 Isa 62:1 Jer 11:4}
\crossref{Gen}{15}{18}{Ge 9:8-\allowbreak17; 17:1-\allowbreak27; 24:7 2Sa 23:5 Isa 55:3 Jer 31:31-\allowbreak34; 32:40}
\crossref{Gen}{15}{19}{Nu 24:21,\allowbreak22}
\crossref{Gen}{15}{20}{Ge 14:5 Isa 17:5}
\crossref{Gen}{15}{21}{Ge 10:15-\allowbreak19 Ex 23:23-\allowbreak28; 33:2; 34:11 De 7:1}
\crossref{Gen}{16}{1}{Ge 15:2,\allowbreak3; 21:10,\allowbreak12; 25:21 Jud 13:2 Lu 1:7,\allowbreak36}
\crossref{Gen}{16}{2}{Ge 17:16; 18:10; 20:18; 25:21; 30:2,\allowbreak3,\allowbreak9,\allowbreak22 Ps 127:3}
\crossref{Gen}{16}{3}{Ge 12:4,\allowbreak5}
\crossref{Gen}{16}{4}{1Sa 1:6-\allowbreak8 2Sa 6:16 Pr 30:20,\allowbreak21,\allowbreak23 1Co 4:6; 13:4,\allowbreak5}
\crossref{Gen}{16}{5}{Lu 10:40,\allowbreak41}
\crossref{Gen}{16}{6}{Ge 13:8,\allowbreak9 Pr 14:29; 15:1,\allowbreak17,\allowbreak18 1Pe 3:7}
\crossref{Gen}{16}{7}{Pr 15:3}
\crossref{Gen}{16}{8}{16:1,\allowbreak4 Eph 6:5-\allowbreak8 1Ti 6:1,\allowbreak2}
\crossref{Gen}{16}{9}{Ec 10:4 Eph 5:21; 6:5,\allowbreak6 Tit 2:9 1Pe 2:18-\allowbreak25; 5:5,\allowbreak6}
\crossref{Gen}{16}{10}{Ge 22:15-\allowbreak18; 31:11-\allowbreak13; 32:24-\allowbreak30; 48:15,\allowbreak16 Ex 3:2-\allowbreak6 Jud 2:1-\allowbreak3; 6:11}
\crossref{Gen}{16}{11}{Ge 17:19; 29:32-\allowbreak35 Isa 7:14 Mt 1:21-\allowbreak23 Lu 1:13,\allowbreak31,\allowbreak63}
\crossref{Gen}{16}{12}{Ge 21:20 Job 11:12; 39:5-\allowbreak8}
\crossref{Gen}{16}{13}{16:7,\allowbreak9,\allowbreak10; 22:14; 28:17:19; 32:30 Jud 6:24}
\crossref{Gen}{16}{14}{Nu 13:26}
\crossref{Gen}{16}{15}{16:11; 25:12 1Ch 1:28 Ga 4:22,\allowbreak23}
\crossref{Gen}{16}{16}{16:16}
\crossref{Gen}{17}{1}{Ge 16:16}
\crossref{Gen}{17}{2}{17:4-\allowbreak6; 9:9; 15:18 Ps 105:8-\allowbreak11 Ga 3:17,\allowbreak18}
\crossref{Gen}{17}{3}{17:17 Ex 3:6 Le 9:23,\allowbreak24 Nu 14:5; 16:22,\allowbreak45 Jos 5:14 Jud 13:20}
\crossref{Gen}{17}{4}{Ge 12:2; 13:16; 16:10; 22:17; 25:1-\allowbreak18; 32:12; 35:11; 36:1-\allowbreak43 Nu 1:1-\allowbreak54}
\crossref{Gen}{17}{5}{17:15; 32:28 Nu 13:16 2Sa 12:25 Ne 9:7 Isa 62:2-\allowbreak4; 65:15 Jer 20:3}
\crossref{Gen}{17}{6}{17:4,\allowbreak20; 35:11}
\crossref{Gen}{17}{7}{Ge 15:18; 26:24 Ex 6:4 Ps 105:8-\allowbreak11 Mic 7:20 Lu 1:54,\allowbreak55,\allowbreak72-\allowbreak75}
\crossref{Gen}{17}{8}{Ge 12:7; 13:15,\allowbreak17; 15:7-\allowbreak21 Ps 105:9,\allowbreak11}
\crossref{Gen}{17}{9}{Ps 25:10; 103:18 Isa 56:4,\allowbreak5}
\crossref{Gen}{17}{10}{17:11; 34:15 Ex 4:25; 12:48 De 10:16; 30:6 Jos 5:2,\allowbreak4 Jer 4:4; 9:25,\allowbreak26}
\crossref{Gen}{17}{11}{Ex 4:25 Jos 5:3 1Sa 18:25-\allowbreak27 2Sa 3:14}
\crossref{Gen}{17}{12}{Ge 21:4 Le 12:3 Lu 1:59; 2:21 Joh 7:22,\allowbreak23 Ac 7:8 Ro 2:28 Php 3:5}
\crossref{Gen}{17}{13}{Ge 14:14; 15:3 Ex 12:44; 21:4}
\crossref{Gen}{17}{14}{Ex 4:24-\allowbreak26; 12:15,\allowbreak19; 30:33,\allowbreak38 Le 7:20,\allowbreak21,\allowbreak25,\allowbreak27; 18:29; 19:8}
\crossref{Gen}{17}{15}{17:5; 32:28 2Sa 12:25}
\crossref{Gen}{17}{16}{Ge 1:28; 12:2; 24:60 Ro 9:9}
\crossref{Gen}{17}{17}{17:3 Le 9:24 Nu 14:5; 16:22,\allowbreak45 De 9:18,\allowbreak25 Jos 5:14; 7:6 Jud 13:20}
\crossref{Gen}{17}{18}{Jer 32:39 Ac 2:39}
\crossref{Gen}{17}{19}{17:21; 18:10-\allowbreak14; 21:2,\allowbreak3,\allowbreak6 2Ki 4:16,\allowbreak17 Lu 1:13-\allowbreak20 Ro 9:6-\allowbreak9 Ga 4:28-\allowbreak31}
\crossref{Gen}{17}{20}{Ge 16:10-\allowbreak12}
\crossref{Gen}{17}{21}{Ge 21:10-\allowbreak12; 26:2-\allowbreak5; 46:1; 48:15 Ex 2:24; 3:6 Lu 1:55,\allowbreak72 Ro 9:5,\allowbreak6,\allowbreak9}
\crossref{Gen}{17}{22}{17:3; 18:33; 35:9-\allowbreak15 Ex 20:22 Nu 12:6-\allowbreak8 De 5:4 Jud 6:21; 13:20}
\crossref{Gen}{17}{23}{17:10-\allowbreak14,\allowbreak26,\allowbreak27; 18:19; 34:24 Jos 5:2-\allowbreak9 Ps 119:60 Pr 27:1 Ec 9:10 Ac 16:3}
\crossref{Gen}{17}{24}{17:1,\allowbreak17; 12:4 Ro 4:11,\allowbreak19,\allowbreak20}
\crossref{Gen}{17}{25}{}
\crossref{Gen}{17}{26}{Ge 12:4; 22:3,\allowbreak4 Ps 119:60}
\crossref{Gen}{17}{27}{Ge 18:19}
\crossref{Gen}{18}{1}{Ge 15:1; 17:1-\allowbreak3,\allowbreak22; 26:2; 48:3 Ex 4:1 2Ch 1:7 Ac 7:2}
\crossref{Gen}{18}{2}{Jud 13:3,\allowbreak9 Heb 13:2}
\crossref{Gen}{18}{3}{Ge 32:5}
\crossref{Gen}{18}{4}{}
\crossref{Gen}{18}{5}{Jud 6:18; 13:15 Mt 6:11}
\crossref{Gen}{18}{6}{Isa 32:8 Mt 13:33 Lu 10:38-\allowbreak40 Ac 16:15 Ro 12:13 Ga 5:13}
\crossref{Gen}{18}{7}{Ge 19:3 Jud 13:15,\allowbreak16 Am 6:4 Mal 1:14 Mt 22:4 Lu 15:23,\allowbreak27,\allowbreak30}
\crossref{Gen}{18}{8}{Ge 19:3 De 32:14 Jud 5:25}
\crossref{Gen}{18}{9}{Ge 4:9}
\crossref{Gen}{18}{10}{18:13,\allowbreak14; 16:10; 22:15,\allowbreak16}
\crossref{Gen}{18}{11}{Ge 17:17,\allowbreak24 Lu 1:7,\allowbreak18,\allowbreak36 Ro 4:18-\allowbreak21 Heb 11:11,\allowbreak12,\allowbreak19}
\crossref{Gen}{18}{12}{18:13; 17:17; 21:6,\allowbreak7 Ps 126:2 Lu 1:18-\allowbreak20,\allowbreak34,\allowbreak35 Heb 11:11,\allowbreak12}
\crossref{Gen}{18}{13}{Joh 2:25}
\crossref{Gen}{18}{14}{Nu 11:23 De 7:21 1Sa 14:6 2Ki 7:1,\allowbreak2 Job 36:5; 42:2 Ps 93:1; 95:3}
\crossref{Gen}{18}{15}{Ge 4:9; 12:13 Job 2:10 Pr 28:13 Joh 18:17,\allowbreak25-\allowbreak27 Eph 4:23 Col 3:9}
\crossref{Gen}{18}{16}{Ac 15:3; 20:38; 21:5 Ro 15:24 3Jo 1:6}
\crossref{Gen}{18}{17}{2Ki 4:27 2Ch 20:7 Ps 25:14 Am 3:7 Joh 15:15 Jas 2:23}
\crossref{Gen}{18}{18}{Ge 12:2,\allowbreak3; 22:17,\allowbreak18; 26:4 Ps 72:17 Ac 3:25,\allowbreak26 Ga 3:8,\allowbreak14 Eph 1:3}
\crossref{Gen}{18}{19}{2Sa 7:20 Ps 1:6; 11:4; 34:15 Joh 10:14; 21:17 2Ti 2:19}
\crossref{Gen}{18}{20}{Ge 4:10; 19:13 Isa 3:9; 5:7 Jer 14:7 Jas 5:4}
\crossref{Gen}{18}{21}{Job 34:22 Ps 90:8 Jer 17:1,\allowbreak10 Zep 1:12 Heb 4:13}
\crossref{Gen}{18}{22}{18:2; 19:1}
\crossref{Gen}{18}{23}{Ps 73:28 Jer 30:21 Heb 10:22 Jas 5:17}
\crossref{Gen}{18}{24}{18:32 Isa 1:9 Jer 5:1 Mt 7:13,\allowbreak14}
\crossref{Gen}{18}{25}{Jer 12:1}
\crossref{Gen}{18}{26}{Isa 6:13; 10:22; 19:24; 65:8 Jer 5:1 Eze 22:30 Mt 24:22}
\crossref{Gen}{18}{27}{18:30-\allowbreak32 Ezr 9:6 Job 42:6-\allowbreak8 Isa 6:5 Lu 18:1}
\crossref{Gen}{18}{28}{Nu 14:17-\allowbreak19 1Ki 20:32,\allowbreak33 Job 23:3,\allowbreak4}
\crossref{Gen}{18}{29}{Eph 6:18 Heb 4:16}
\crossref{Gen}{18}{30}{Ge 44:18 Jud 6:39 Es 4:11-\allowbreak16 Job 40:4 Ps 9:12; 10:17; 89:7 Isa 6:5}
\crossref{Gen}{18}{31}{18:27 Mt 7:7,\allowbreak11 Lu 11:8; 18:1 Eph 6:18 Heb 4:16; 10:20-\allowbreak22}
\crossref{Gen}{18}{32}{18:30 Jud 6:39 Pr 15:8 Isa 42:6,\allowbreak7 Jas 5:15-\allowbreak17 1Jo 5:15,\allowbreak16}
\crossref{Gen}{18}{33}{18:16,\allowbreak22; 32:26}
\crossref{Gen}{19}{1}{Ge 18:1-\allowbreak5 Job 31:32 Heb 13:2}
\crossref{Gen}{19}{2}{Heb 13:2}
\crossref{Gen}{19}{3}{2Ki 4:8 Lu 11:8; 14:23; 24:28,\allowbreak29 2Co 5:14}
\crossref{Gen}{19}{4}{Pr 4:16; 6:18 Mic 7:3 Ro 3:15}
\crossref{Gen}{19}{5}{Le 18:22; 20:13 Jud 19:22 Isa 1:9; 3:9 Jer 3:3; 6:15 Eze 16:49,\allowbreak51}
\crossref{Gen}{19}{6}{Jud 19:23}
\crossref{Gen}{19}{7}{19:4 Le 18:22; 20:13 De 23:17 Jud 19:23 1Sa 30:23,\allowbreak24 Ac 17:26}
\crossref{Gen}{19}{8}{Ex 32:22}
\crossref{Gen}{19}{9}{1Sa 17:44; 25:17 Pr 9:7,\allowbreak8 Isa 65:5 Jer 3:3; 6:15; 8:12 Mt 7:6}
\crossref{Gen}{19}{10}{}
\crossref{Gen}{19}{11}{Ec 10:15 Isa 57:10 Jer 2:36}
\crossref{Gen}{19}{12}{Ge 7:1 Nu 16:26 Jos 6:22,\allowbreak23 Jer 32:39 2Pe 2:7,\allowbreak9}
\crossref{Gen}{19}{13}{Ge 13:13; 18:20 Jas 5:4}
\crossref{Gen}{19}{14}{Mt 1:18}
\crossref{Gen}{19}{15}{19:17,\allowbreak22 Nu 16:24-\allowbreak27 Pr 6:4,\allowbreak5 Lu 13:24,\allowbreak25 2Co 6:2 Heb 3:7,\allowbreak8}
\crossref{Gen}{19}{16}{Ps 119:60 Joh 6:44}
\crossref{Gen}{19}{17}{Ge 18:22}
\crossref{Gen}{19}{18}{Ge 32:26 2Ki 5:11,\allowbreak12 Isa 45:11 Joh 13:6-\allowbreak8 Ac 9:13; 10:14}
\crossref{Gen}{19}{19}{Ps 18:1-\allowbreak50; 40:1-\allowbreak17; 103:1-\allowbreak22; 106:1-\allowbreak107:43; 116:1-\allowbreak19 1Ti 1:14-\allowbreak16}
\crossref{Gen}{19}{20}{19:30 Pr 3:5-\allowbreak7 Am 3:6}
\crossref{Gen}{19}{21}{Ge 4:7 Job 42:8,\allowbreak9 Ps 34:15; 102:17; 145:19 Jer 14:10 Mt 12:20}
\crossref{Gen}{19}{22}{Ge 32:25-\allowbreak28 Ex 32:10 De 9:14 Ps 91:1-\allowbreak10 Isa 65:8 Mr 6:5 2Ti 2:13}
\crossref{Gen}{19}{23}{19:23}
\crossref{Gen}{19}{24}{De 29:23 Job 18:15 Ps 11:6 Isa 1:9; 13:19 Jer 20:16; 49:18; 50:40}
\crossref{Gen}{19}{25}{Ge 13:10; 14:3 Ps 107:34}
\crossref{Gen}{19}{26}{Nu 16:38}
\crossref{Gen}{19}{27}{Ps 5:3}
\crossref{Gen}{19}{28}{Ps 107:34 2Pe 2:7 Jude 1:7 Re 14:10,\allowbreak11; 18:9,\allowbreak18; 19:3; 21:8}
\crossref{Gen}{19}{29}{Ge 8:1; 12:2; 18:23-\allowbreak33; 30:22 De 9:5 Ne 13:14,\allowbreak22 Ps 25:7; 105:8,\allowbreak42}
\crossref{Gen}{19}{30}{19:17-\allowbreak23}
\crossref{Gen}{19}{31}{19:28 Mr 9:6}
\crossref{Gen}{19}{32}{Ge 11:3}
\crossref{Gen}{19}{33}{Le 18:6,\allowbreak7 Pr 20:1; 23:29-\allowbreak35 Hab 2:15,\allowbreak16}
\crossref{Gen}{19}{34}{Isa 3:9 Jer 3:3; 5:3; 6:15; 8:12}
\crossref{Gen}{19}{35}{Ps 8:4 Pr 24:16 Ec 7:26 Lu 21:34 1Co 10:11,\allowbreak12 1Pe 4:7}
\crossref{Gen}{19}{36}{19:8 Le 18:6,\allowbreak7 Jud 1:7 1Sa 15:33 Hab 2:15 Mt 7:2}
\crossref{Gen}{19}{37}{Nu 21:29; 22:1-\allowbreak41; 24:1-\allowbreak25 De 2:9,\allowbreak19; 23:3 Jud 3:1-\allowbreak31 Ru 4:10}
\crossref{Gen}{19}{38}{De 2:9,\allowbreak19; 23:3 Jud 10:6-\allowbreak18; 11:1-\allowbreak40 1Sa 11:1-\allowbreak15 2Sa 10:1-\allowbreak19}
\crossref{Gen}{20}{1}{Ge 13:1; 18:1; 24:62}
\crossref{Gen}{20}{2}{Ge 12:11-\allowbreak13; 26:7 2Ch 19:2; 20:37; 32:31 Pr 24:16 Ec 7:20 Ga 2:11,\allowbreak12}
\crossref{Gen}{20}{3}{Ge 28:12; 31:24; 37:5,\allowbreak9; 40:8; 41:1 etc.}
\crossref{Gen}{20}{4}{20:6,\allowbreak18}
\crossref{Gen}{20}{5}{Jos 22:22 1Ki 9:4 2Ki 20:3 1Ch 29:17 Ps 7:8; 25:21; 78:72}
\crossref{Gen}{20}{6}{20:18; 31:7; 35:5 Ex 34:24 1Sa 25:26,\allowbreak34 Ps 84:11 Pr 21:1 Ho 2:6,\allowbreak7}
\crossref{Gen}{20}{7}{}
\crossref{Gen}{20}{8}{}
\crossref{Gen}{20}{9}{Ge 12:18; 26:10 Ex 32:21,\allowbreak35 Jos 7:25 1Sa 26:18,\allowbreak19 Pr 28:10}
\crossref{Gen}{20}{10}{}
\crossref{Gen}{20}{11}{Ge 22:12; 42:18 Ne 5:15 Job 1:1; 28:28 Ps 14:4; 36:1-\allowbreak4 Pr 1:7; 2:5}
\crossref{Gen}{20}{12}{Ge 11:29; 12:13 1Th 5:22}
\crossref{Gen}{20}{13}{Ge 12:1,\allowbreak9,\allowbreak11 etc.}
\crossref{Gen}{20}{14}{20:11; 12:16}
\crossref{Gen}{20}{15}{Ge 13:9; 34:10; 47:6}
\crossref{Gen}{20}{16}{20:5 Pr 27:5}
\crossref{Gen}{20}{17}{20:7; 29:31 1Sa 5:11,\allowbreak12 Ezr 6:10 Job 42:9,\allowbreak10 Pr 15:8,\allowbreak29 Isa 45:11}
\crossref{Gen}{20}{18}{20:7; 12:17; 16:2; 30:2 1Sa 1:6; 5:10}
\crossref{Gen}{21}{1}{Ge 50:24 Ex 3:16; 4:31; 20:5 Ru 1:6 1Sa 2:21 Ps 106:4 Lu 1:68; 19:44}
\crossref{Gen}{21}{2}{2Ki 4:16,\allowbreak17 Lu 1:24,\allowbreak25,\allowbreak36 Ac 7:8 Ga 4:22 Heb 11:11}
\crossref{Gen}{21}{3}{21:6,\allowbreak12; 17:19; 22:2 Jos 24:3 Mt 1:2 Ac 7:8 Ro 9:7 Heb 11:18}
\crossref{Gen}{21}{4}{Ge 17:10-\allowbreak12 Ex 12:48 Le 12:3 De 12:32 Lu 1:6,\allowbreak59; 2:21 Joh 7:22,\allowbreak23}
\crossref{Gen}{21}{5}{Ge 17:1,\allowbreak17 Ro 4:19}
\crossref{Gen}{21}{6}{Ge 17:17; 18:12-\allowbreak15 1Sa 1:26-\allowbreak28; 2:1-\allowbreak10 Ps 113:9; 126:2 Isa 49:15,\allowbreak21}
\crossref{Gen}{21}{7}{Nu 23:23 De 4:32-\allowbreak34 Ps 86:8,\allowbreak10 Isa 49:21; 66:8 Eph 3:10}
\crossref{Gen}{21}{8}{1Sa 1:22 Ps 131:2 Ho 1:8}
\crossref{Gen}{21}{9}{Ge 16:3-\allowbreak6,\allowbreak15; 17:20}
\crossref{Gen}{21}{10}{Joh 8:35 Ga 3:18; 4:7 1Pe 1:4 1Jo 2:19}
\crossref{Gen}{21}{11}{Ge 17:18; 22:1,\allowbreak2 2Sa 18:33 Mt 10:37 Heb 12:11}
\crossref{Gen}{21}{12}{1Sa 8:7,\allowbreak9 Isa 46:10}
\crossref{Gen}{21}{13}{21:18; 16:10; 17:20; 25:12-\allowbreak18}
\crossref{Gen}{21}{14}{Ge 19:27; 22:3; 24:54; 26:31 Ps 119:60 Pr 27:14 Ec 9:10}
\crossref{Gen}{21}{15}{21:14 Ex 15:22-\allowbreak25; 17:1-\allowbreak3 2Ki 3:9 Ps 63:1 Isa 44:12 Jer 14:3}
\crossref{Gen}{21}{16}{Ge 44:34 1Ki 3:26 Es 8:6 Isa 49:15 Zec 12:10 Lu 15:20}
\crossref{Gen}{21}{17}{Ge 16:11 Ex 3:7; 22:23,\allowbreak27 2Ki 13:4,\allowbreak23 Ps 50:15; 65:2; 91:15 Mt 15:32}
\crossref{Gen}{21}{18}{21:13; 16:10; 17:20; 25:12-\allowbreak18 1Ch 1:29-\allowbreak31}
\crossref{Gen}{21}{19}{Nu 22:31 2Ki 6:17-\allowbreak20 Isa 35:5,\allowbreak6 Lu 24:16-\allowbreak31}
\crossref{Gen}{21}{20}{Ge 17:20; 28:15; 39:2,\allowbreak3,\allowbreak21 Jud 6:12; 13:24,\allowbreak25 Lu 1:80; 2:40}
\crossref{Gen}{21}{21}{Nu 10:12; 12:16; 13:3,\allowbreak26 1Sa 25:1}
\crossref{Gen}{21}{22}{Ge 20:2; 26:26}
\crossref{Gen}{21}{23}{Ge 14:22,\allowbreak23; 24:3; 26:28; 31:44,\allowbreak53 De 6:13 Jos 2:12 1Sa 20:13,\allowbreak17,\allowbreak42}
\crossref{Gen}{21}{24}{Ge 14:13 Ro 12:18 Heb 6:16}
\crossref{Gen}{21}{25}{Ge 26:15-\allowbreak22; 29:8 Ex 2:15-\allowbreak17 Jud 1:15 Pr 17:10; 25:9; 27:5 Mt 18:15}
\crossref{Gen}{21}{26}{}
\crossref{Gen}{21}{27}{Ge 14:22,\allowbreak23 Pr 17:8; 18:16,\allowbreak24; 21:14 Isa 32:8}
\crossref{Gen}{21}{28}{21:28}
\crossref{Gen}{21}{29}{Ge 33:8 Ex 12:26 1Sa 15:14}
\crossref{Gen}{21}{30}{Ge 31:44-\allowbreak48,\allowbreak52 Jos 22:27,\allowbreak28; 24:27}
\crossref{Gen}{21}{31}{Ge 26:33}
\crossref{Gen}{21}{32}{21:27; 14:13; 31:53 1Sa 18:3}
\crossref{Gen}{21}{33}{Am 8:14}
\crossref{Gen}{21}{34}{Ge 20:1 1Ch 29:15 Ps 39:12 Heb 11:9,\allowbreak13 1Pe 2:11}
\crossref{Gen}{22}{1}{Ex 15:25,\allowbreak26; 16:4 De 8:2; 13:3 Jud 2:22 2Sa 24:1 2Ch 32:31}
\crossref{Gen}{22}{2}{Ge 17:19; 21:12 Joh 3:16 Ro 5:8; 8:32 Heb 11:17 1Jo 4:9,\allowbreak10}
\crossref{Gen}{22}{3}{Ge 17:23; 21:14 Ps 119:60 Ec 9:10 Isa 26:3,\allowbreak4 Mt 10:37 Mr 10:28-\allowbreak31}
\crossref{Gen}{22}{4}{Ex 5:3; 15:22; 19:11,\allowbreak15 Le 7:17 Nu 10:33; 19:12,\allowbreak19; 31:19 Jos 1:11}
\crossref{Gen}{22}{5}{Heb 12:1}
\crossref{Gen}{22}{6}{Isa 53:6 Mt 8:17 Lu 24:26,\allowbreak27 Joh 19:17 1Pe 2:24}
\crossref{Gen}{22}{7}{Mt 26:39,\allowbreak42 Joh 18:11 Ro 8:15}
\crossref{Gen}{22}{8}{Ge 18:14 2Ch 25:9 Mt 19:26 Joh 1:29,\allowbreak36 1Pe 1:19,\allowbreak20 Re 5:6,\allowbreak12}
\crossref{Gen}{22}{9}{22:2-\allowbreak4 Mt 21:1-\allowbreak46; 26:1-\allowbreak27:66}
\crossref{Gen}{22}{10}{Isa 53:6-\allowbreak12 Heb 11:17-\allowbreak19 Jas 2:21-\allowbreak23}
\crossref{Gen}{22}{11}{22:12,\allowbreak16; 16:7,\allowbreak9,\allowbreak10; 21:17}
\crossref{Gen}{22}{12}{1Sa 15:22 Job 5:19 Jer 19:5 Mic 6:6-\allowbreak8 1Co 10:13 2Co 8:12}
\crossref{Gen}{22}{13}{22:8 Ps 40:6-\allowbreak8; 89:19,\allowbreak20 Isa 30:21 1Co 10:13 2Co 1:9,\allowbreak10}
\crossref{Gen}{22}{14}{Ge 16:13,\allowbreak14; 28:19; 32:30 Ex 17:15 Jud 6:24 1Sa 7:12 Eze 48:35}
\crossref{Gen}{22}{15}{22:11}
\crossref{Gen}{22}{16}{Ge 12:2 Ps 105:9 Isa 45:23 Jer 49:13; 51:14 Am 6:8 Lu 1:73}
\crossref{Gen}{22}{17}{Ge 12:2; 27:28,\allowbreak29; 28:3,\allowbreak14 etc.}
\crossref{Gen}{22}{18}{Ge 12:3; 18:18; 26:4 Ps 72:17 Ac 3:25 Ro 1:3 Ga 3:8,\allowbreak9,\allowbreak16,\allowbreak18,\allowbreak28,\allowbreak29}
\crossref{Gen}{22}{19}{22:5}
\crossref{Gen}{22}{20}{Pr 25:25}
\crossref{Gen}{22}{21}{Job 1:1}
\crossref{Gen}{22}{22}{}
\crossref{Gen}{22}{23}{Ge 24:15,\allowbreak24,\allowbreak47; 25:20; 28:2,\allowbreak5}
\crossref{Gen}{22}{24}{Ge 16:3; 25:6 Pr 15:25}
\crossref{Gen}{23}{1}{Ge 17:17}
\crossref{Gen}{23}{2}{23:19; 13:18 Nu 13:22 Jos 10:39; 14:14,\allowbreak15; 20:7 Jud 1:10 1Sa 20:31}
\crossref{Gen}{23}{3}{23:5,\allowbreak7; 10:15; 25:10; 27:46; 49:30 1Sa 26:6 2Sa 23:39}
\crossref{Gen}{23}{4}{Ge 17:8; 47:9 Le 25:23 1Ch 29:15 Ps 39:12; 105:12,\allowbreak13; 119:19}
\crossref{Gen}{23}{5}{}
\crossref{Gen}{23}{6}{Ge 18:12; 24:18,\allowbreak35; 31:35; 32:4,\allowbreak5,\allowbreak18; 42:10; 44:5,\allowbreak8 Ex 32:22 Ru 2:13}
\crossref{Gen}{23}{7}{Ge 18:2; 19:1 Pr 18:24 Ro 12:17,\allowbreak18 Heb 12:14 1Pe 3:8}
\crossref{Gen}{23}{8}{1Ki 2:17 Lu 7:3,\allowbreak4 Heb 7:26 1Jo 2:1,\allowbreak2}
\crossref{Gen}{23}{9}{Ro 12:17; 13:8}
\crossref{Gen}{23}{10}{23:18; 34:20,\allowbreak24 Ru 4:1-\allowbreak4 Job 29:7 Isa 28:6}
\crossref{Gen}{23}{11}{23:6 2Sa 24:20-\allowbreak24 1Ch 21:22-\allowbreak24 Isa 32:8}
\crossref{Gen}{23}{12}{23:7; 18:2; 19:1}
\crossref{Gen}{23}{13}{Ge 14:22,\allowbreak23 2Sa 24:24 Ac 20:35 Ro 13:8 Php 4:5-\allowbreak8 Col 4:5 Heb 13:5}
\crossref{Gen}{23}{14}{}
\crossref{Gen}{23}{15}{Ex 30:15 Eze 45:12}
\crossref{Gen}{23}{16}{Ge 43:21 Ezr 8:25-\allowbreak30 Job 28:15 Jer 32:9 Zec 11:12 Mt 7:12 Ro 13:8}
\crossref{Gen}{23}{17}{23:20; 25:9; 49:30-\allowbreak32; 50:13 Ac 7:16}
\crossref{Gen}{23}{18}{Ge 34:20 Ru 4:1 Jer 32:12}
\crossref{Gen}{23}{19}{Ge 3:19; 25:9,\allowbreak10; 35:27-\allowbreak29; 47:30; 49:29-\allowbreak32; 50:13,\allowbreak25 Job 30:23 Ec 6:3}
\crossref{Gen}{23}{20}{Ru 4:7-\allowbreak10 2Sa 24:24 Jer 32:10,\allowbreak11}
\crossref{Gen}{24}{1}{Ge 18:11; 21:5; 25:20 1Ki 1:1 Lu 1:7}
\crossref{Gen}{24}{2}{Ge 15:2 1Ti 5:17}
\crossref{Gen}{24}{3}{Ge 21:23; 26:28-\allowbreak31; 31:44-\allowbreak53; 50:25 Ex 20:7; 22:11; 23:13 Le 19:12}
\crossref{Gen}{24}{4}{Ge 11:25 etc.}
\crossref{Gen}{24}{5}{24:58 Ex 20:7; 9:2 Pr 13:16 Jer 4:2}
\crossref{Gen}{24}{6}{Ga 5:1 Heb 10:39; 11:9,\allowbreak13-\allowbreak16 2Pe 2:20-\allowbreak22}
\crossref{Gen}{24}{7}{Ezr 1:2 Da 2:44 Jon 1:9 Re 11:13}
\crossref{Gen}{24}{8}{Nu 30:5,\allowbreak8 Jos 2:17-\allowbreak20; 9:20 Joh 8:32}
\crossref{Gen}{24}{9}{}
\crossref{Gen}{24}{10}{24:2; 39:4-\allowbreak6,\allowbreak8,\allowbreak9,\allowbreak22,\allowbreak23}
\crossref{Gen}{24}{11}{Ge 33:13,\allowbreak14 Pr 12:10}
\crossref{Gen}{24}{12}{24:27; 15:1; 17:7,\allowbreak8; 26:24; 28:13; 31:42; 32:9 Ex 3:6,\allowbreak15 1Ki 18:36}
\crossref{Gen}{24}{13}{24:43 Ps 37:5 Pr 3:6}
\crossref{Gen}{24}{14}{Jud 6:17,\allowbreak37 1Sa 14:9}
\crossref{Gen}{24}{15}{24:45 Jud 6:36-\allowbreak40 Ps 34:15; 65:2; 145:18,\allowbreak19 Isa 58:9; 65:24}
\crossref{Gen}{24}{16}{Ge 26:7; 39:6}
\crossref{Gen}{24}{17}{1Ki 17:10 Joh 4:7,\allowbreak9}
\crossref{Gen}{24}{18}{Pr 31:26 1Pe 3:8; 4:8,\allowbreak9}
\crossref{Gen}{24}{19}{24:14,\allowbreak45,\allowbreak46 1Pe 4:9}
\crossref{Gen}{24}{20}{}
\crossref{Gen}{24}{21}{2Sa 7:18-\allowbreak20 Ps 34:1-\allowbreak6; 107:1,\allowbreak8,\allowbreak15,\allowbreak43; 116:1-\allowbreak7 Lu 2:19,\allowbreak51}
\crossref{Gen}{24}{22}{24:30 Ex 32:2,\allowbreak3 Es 5:1 Jer 2:32 1Ti 2:9,\allowbreak10 1Pe 3:3,\allowbreak8}
\crossref{Gen}{24}{23}{24:23}
\crossref{Gen}{24}{24}{24:15; 11:29; 22:20,\allowbreak23}
\crossref{Gen}{24}{25}{Ge 18:4-\allowbreak8 Jud 19:19-\allowbreak21 Isa 32:8 1Pe 4:9}
\crossref{Gen}{24}{26}{24:48,\allowbreak52; 22:5 Ex 4:31; 12:27; 34:8 1Ch 29:20 2Ch 20:18; 29:30 Ne 8:6}
\crossref{Gen}{24}{27}{24:12; 9:26; 14:20 Ex 18:10 Ru 4:14 1Sa 25:32,\allowbreak39 2Sa 18:28}
\crossref{Gen}{24}{28}{24:48,\allowbreak55,\allowbreak67; 31:33}
\crossref{Gen}{24}{29}{24:55,\allowbreak60; 29:5}
\crossref{Gen}{24}{30}{}
\crossref{Gen}{24}{31}{Ge 26:29 Jud 17:2 Ru 3:10 Ps 115:15 Pr 17:8; 18:16}
\crossref{Gen}{24}{32}{Ge 18:4; 19:2; 43:24 Jud 19:21 1Sa 25:41 Lu 7:44 Joh 13:4-\allowbreak14}
\crossref{Gen}{24}{33}{Job 23:12 Ps 132:3-\allowbreak5 Pr 22:29 Ec 9:10 Joh 4:14,\allowbreak31-\allowbreak34 Eph 6:5-\allowbreak8}
\crossref{Gen}{24}{34}{}
\crossref{Gen}{24}{35}{24:1; 12:2; 13:2; 25:11; 26:12; 49:25 Ps 18:35; 112:3 Pr 10:22; 22:4}
\crossref{Gen}{24}{36}{Ge 11:29,\allowbreak30; 17:15-\allowbreak19; 18:10-\allowbreak14; 21:1-\allowbreak7 Ro 4:19}
\crossref{Gen}{24}{37}{24:2-\allowbreak9; 6:2; 27:46 Ezr 9:1-\allowbreak3}
\crossref{Gen}{24}{38}{24:4; 12:1}
\crossref{Gen}{24}{39}{24:5}
\crossref{Gen}{24}{40}{24:7}
\crossref{Gen}{24}{41}{24:8 De 29:12}
\crossref{Gen}{24}{42}{24:12-\allowbreak14 Ac 10:7,\allowbreak8,\allowbreak22}
\crossref{Gen}{24}{43}{24:13,\allowbreak14}
\crossref{Gen}{24}{44}{Isa 32:8 1Ti 2:10 Heb 13:2 1Pe 3:8}
\crossref{Gen}{24}{45}{24:15-\allowbreak20 Isa 58:9; 65:24 Da 9:19,\allowbreak23 Ac 4:24-\allowbreak33; 10:30; 12:12-\allowbreak17}
\crossref{Gen}{24}{46}{}
\crossref{Gen}{24}{47}{24:22,\allowbreak53 Ps 45:9,\allowbreak13,\allowbreak14 Isa 62:3-\allowbreak5 Eze 16:10-\allowbreak13 Eph 5:26,\allowbreak27}
\crossref{Gen}{24}{48}{24:26,\allowbreak27,\allowbreak52}
\crossref{Gen}{24}{49}{Ge 47:29 Jos 2:14}
\crossref{Gen}{24}{50}{24:15,\allowbreak28,\allowbreak53,\allowbreak55,\allowbreak60}
\crossref{Gen}{24}{51}{Ge 20:15}
\crossref{Gen}{24}{52}{24:26,\allowbreak48 1Ch 29:20 2Ch 20:18 Ps 34:1,\allowbreak2; 95:6; 107:21,\allowbreak22; 116:1,\allowbreak2}
\crossref{Gen}{24}{53}{}
\crossref{Gen}{24}{54}{24:56,\allowbreak59; 28:5,\allowbreak6; 45:24 2Sa 18:19,\allowbreak27,\allowbreak28 Pr 22:29 Ec 7:10}
\crossref{Gen}{24}{55}{Ge 4:3 Le 25:29 Jud 14:8}
\crossref{Gen}{24}{56}{Ge 45:9-\allowbreak13 Pr 25:25}
\crossref{Gen}{24}{57}{24:57}
\crossref{Gen}{24}{58}{Ps 45:10,\allowbreak11 Lu 1:38}
\crossref{Gen}{24}{59}{24:50,\allowbreak53,\allowbreak60}
\crossref{Gen}{24}{60}{Ge 1:28; 9:1; 14:19; 17:16; 28:3; 48:15,\allowbreak16,\allowbreak20 Ru 4:11,\allowbreak12}
\crossref{Gen}{24}{61}{Ge 31:34 1Sa 30:17 Es 8:10,\allowbreak14}
\crossref{Gen}{24}{62}{Ge 16:14; 25:11}
\crossref{Gen}{24}{63}{}
\crossref{Gen}{24}{64}{Jos 15:18 Jud 1:14}
\crossref{Gen}{24}{65}{Ge 20:16 1Co 11:5,\allowbreak6,\allowbreak10 1Ti 2:9}
\crossref{Gen}{24}{66}{Mr 6:30}
\crossref{Gen}{24}{67}{Ge 18:6,\allowbreak9,\allowbreak10 So 8:2 Isa 54:1-\allowbreak5}
\crossref{Gen}{25}{1}{Ge 23:1,\allowbreak2; 28:1 1Ch 1:32,\allowbreak33}
\crossref{Gen}{25}{2}{1Ch 1:32,\allowbreak33 Jer 25:25}
\crossref{Gen}{25}{3}{1Ki 10:1 Job 6:19 Ps 72:10}
\crossref{Gen}{25}{4}{Isa 60:6}
\crossref{Gen}{25}{5}{Ge 21:10-\allowbreak12; 24:36 Ps 68:18 Mt 11:27; 28:18 Joh 3:35; 17:2}
\crossref{Gen}{25}{6}{25:1; 16:3; 30:4,\allowbreak9; 32:22; 35:22 Jud 19:1,\allowbreak2,\allowbreak4}
\crossref{Gen}{25}{7}{Ge 12:4}
\crossref{Gen}{25}{8}{25:17; 35:18; 49:33 Ac 5:5,\allowbreak10; 12:23}
\crossref{Gen}{25}{9}{Ge 21:9,\allowbreak10; 35:29}
\crossref{Gen}{25}{10}{Ge 23:16}
\crossref{Gen}{25}{11}{Ge 12:2; 17:19; 22:17; 50:24}
\crossref{Gen}{25}{12}{Ge 16:10-\allowbreak15; 17:20; 21:13 Ps 83:6}
\crossref{Gen}{25}{13}{1Ch 1:29-\allowbreak31}
\crossref{Gen}{25}{14}{Isa 21:11,\allowbreak16}
\crossref{Gen}{25}{15}{1Ch 5:19 Job 2:11}
\crossref{Gen}{25}{16}{Ge 17:20,\allowbreak23}
\crossref{Gen}{25}{17}{25:7,\allowbreak8}
\crossref{Gen}{25}{18}{Ge 2:11; 10:7,\allowbreak29; 20:1; 21:14,\allowbreak21}
\crossref{Gen}{25}{19}{1Ch 1:32 Mt 1:2 Lu 3:34 Ac 7:8}
\crossref{Gen}{25}{20}{Ge 22:23; 24:67}
\crossref{Gen}{25}{21}{1Sa 1:11,\allowbreak27 Ps 50:15; 65:2; 91:15 Isa 45:11; 58:9; 65:24 Lu 1:13}
\crossref{Gen}{25}{22}{1Sa 9:9; 10:22; 22:15; 28:6; 30:8 Eze 20:31; 36:37}
\crossref{Gen}{25}{23}{Ge 17:16; 24:60}
\crossref{Gen}{25}{24}{}
\crossref{Gen}{25}{25}{}
\crossref{Gen}{25}{26}{Ge 38:28-\allowbreak30}
\crossref{Gen}{25}{27}{Ge 10:9; 21:20; 27:3-\allowbreak5,\allowbreak40}
\crossref{Gen}{25}{28}{Ge 27:6}
\crossref{Gen}{25}{29}{Jud 8:4,\allowbreak5 1Sa 14:28,\allowbreak31 Pr 13:25 Isa 40:30,\allowbreak31}
\crossref{Gen}{25}{30}{Ge 36:1,\allowbreak9,\allowbreak43 Ex 15:15 Nu 20:14-\allowbreak21 De 23:7 2Ki 8:20}
\crossref{Gen}{25}{31}{}
\crossref{Gen}{25}{32}{Job 21:15; 22:17; 34:9 Mal 3:14}
\crossref{Gen}{25}{33}{Ge 14:22; 24:3 Mr 6:23 Heb 6:16}
\crossref{Gen}{25}{34}{Ec 8:15 Isa 22:13 1Co 15:32}
\crossref{Gen}{26}{1}{Ge 12:10}
\crossref{Gen}{26}{2}{Ge 12:7; 17:1; 18:1,\allowbreak10-\allowbreak20}
\crossref{Gen}{26}{3}{26:12,\allowbreak14; 20:1 Ps 32:8; 37:1-\allowbreak6; 39:12 Heb 11:9,\allowbreak13-\allowbreak16}
\crossref{Gen}{26}{4}{Ge 13:16; 15:5,\allowbreak18; 17:4-\allowbreak8; 18:18; 22:17 Heb 11:2}
\crossref{Gen}{26}{5}{Ge 12:4; 17:23; 18:19; 22:16,\allowbreak18 Ps 112:1,\allowbreak2; 128:1-\allowbreak6 Mt 5:19; 7:24}
\crossref{Gen}{26}{6}{Ge 20:1}
\crossref{Gen}{26}{7}{Ge 12:13; 20:2,\allowbreak5,\allowbreak12,\allowbreak13 Pr 29:25 Mt 10:28 Eph 5:25 Col 3:9}
\crossref{Gen}{26}{8}{Jud 5:28 Pr 7:6 So 2:9}
\crossref{Gen}{26}{9}{26:9}
\crossref{Gen}{26}{10}{Ge 12:18,\allowbreak19; 20:9,\allowbreak10}
\crossref{Gen}{26}{11}{Ge 20:6 Ps 105:15 Pr 6:29 Zec 2:8}
\crossref{Gen}{26}{12}{Ps 67:6; 72:16 Ec 11:6 Zec 8:12 Mt 13:8,\allowbreak23 Mr 4:8 1Co 3:6}
\crossref{Gen}{26}{13}{Ge 24:35 Ps 112:3}
\crossref{Gen}{26}{14}{Ge 12:16; 13:2 Job 1:3; 42:12 Ps 112:3; 144:13,\allowbreak14 Pr 10:22}
\crossref{Gen}{26}{15}{Ge 21:30}
\crossref{Gen}{26}{16}{Ex 1:9}
\crossref{Gen}{26}{17}{}
\crossref{Gen}{26}{18}{Ge 21:31 Nu 32:38 Ps 16:4 Ho 2:17 Zec 13:2}
\crossref{Gen}{26}{19}{So 4:15 Joh 4:10,\allowbreak11; 7:38}
\crossref{Gen}{26}{20}{Ge 21:25}
\crossref{Gen}{26}{21}{Ezr 4:6}
\crossref{Gen}{26}{22}{Ps 4:1; 18:19; 118:5}
\crossref{Gen}{26}{23}{Ge 21:31; 46:1 Jud 20:1}
\crossref{Gen}{26}{24}{Ge 15:1; 17:7; 24:12; 28:13; 31:5 Ex 3:6 Mt 22:32 Ac 7:32}
\crossref{Gen}{26}{25}{Ge 8:20; 12:7; 13:18; 22:9; 33:20; 35:1 Ex 17:15}
\crossref{Gen}{26}{26}{Ge 20:3; 21:22-\allowbreak32}
\crossref{Gen}{26}{27}{26:14,\allowbreak16 Jud 11:7 Ac 7:9,\allowbreak14,\allowbreak27,\allowbreak35 Re 3:9}
\crossref{Gen}{26}{28}{Ge 21:22,\allowbreak23; 39:5 Jos 3:7 2Ch 1:1 Isa 45:14; 60:14; 61:6,\allowbreak9 Ro 8:31}
\crossref{Gen}{26}{29}{26:11,\allowbreak14,\allowbreak15}
\crossref{Gen}{26}{30}{Ge 19:3; 21:8; 31:54 Ro 12:18 Heb 12:14 1Pe 4:9}
\crossref{Gen}{26}{31}{Ge 19:2; 21:14; 22:3; 31:55}
\crossref{Gen}{26}{32}{26:25 Pr 2:4,\allowbreak5; 10:4; 13:4 Mt 7:7}
\crossref{Gen}{26}{33}{Ge 21:31}
\crossref{Gen}{26}{34}{Ge 36:2,\allowbreak5,\allowbreak13}
\crossref{Gen}{26}{35}{Ge 6:2; 27:46; 28:1,\allowbreak2,\allowbreak8}
\crossref{Gen}{27}{1}{Ge 48:10 1Sa 3:2 Ec 12:3 Joh 9:3}
\crossref{Gen}{27}{2}{Ge 48:21 1Sa 20:3 Pr 27:1 Ec 9:10 Isa 38:1,\allowbreak3 Mr 13:35 Jas 4:14}
\crossref{Gen}{27}{3}{Ge 10:9; 25:27,\allowbreak28}
\crossref{Gen}{27}{4}{27:7,\allowbreak23,\allowbreak25,\allowbreak27; 14:19; 24:60; 28:3; 48:9,\allowbreak15-\allowbreak20; 49:28 Le 9:22,\allowbreak23 De 33:1}
\crossref{Gen}{27}{5}{27:5}
\crossref{Gen}{27}{6}{}
\crossref{Gen}{27}{7}{De 33:1 Jos 6:26 1Sa 24:19}
\crossref{Gen}{27}{8}{27:13; 25:23 Ac 4:19; 5:29 Eph 6:1}
\crossref{Gen}{27}{9}{Jud 13:15 1Sa 16:20}
\crossref{Gen}{27}{10}{}
\crossref{Gen}{27}{11}{Ge 25:25}
\crossref{Gen}{27}{12}{27:22 Job 12:16 2Co 6:8}
\crossref{Gen}{27}{13}{Ge 25:23,\allowbreak33; 43:9 1Sa 14:24-\allowbreak28,\allowbreak36-\allowbreak45; 25:24 2Sa 14:9 Mt 27:25}
\crossref{Gen}{27}{14}{27:4,\allowbreak7,\allowbreak9,\allowbreak17,\allowbreak31; 25:28 Ps 141:4 Pr 23:2,\allowbreak3 Lu 21:34}
\crossref{Gen}{27}{15}{27:27}
\crossref{Gen}{27}{16}{}
\crossref{Gen}{27}{17}{27:17}
\crossref{Gen}{27}{18}{}
\crossref{Gen}{27}{19}{27:21,\allowbreak24,\allowbreak25; 25:25; 29:23-\allowbreak25 1Ki 13:18; 14:2 Isa 28:15 Zec 13:3,\allowbreak4 Mt 26:70-\allowbreak74}
\crossref{Gen}{27}{20}{Ex 20:7 Job 13:7}
\crossref{Gen}{27}{21}{Ps 73:28 Isa 57:19 Jas 4:8}
\crossref{Gen}{27}{22}{}
\crossref{Gen}{27}{23}{27:16}
\crossref{Gen}{27}{24}{1Sa 21:2,\allowbreak13; 27:10 2Sa 14:5 Job 13:7,\allowbreak8; 15:5 Pr 12:19,\allowbreak22; 30:8}
\crossref{Gen}{27}{25}{27:4}
\crossref{Gen}{27}{26}{}
\crossref{Gen}{27}{27}{Heb 11:20}
\crossref{Gen}{27}{28}{De 11:11,\allowbreak12; 32:2; 33:13,\allowbreak28 2Sa 1:21 1Ki 17:1 Ps 65:9-\allowbreak13; 133:3}
\crossref{Gen}{27}{29}{Ge 9:25,\allowbreak26; 22:17,\allowbreak18; 49:8-\allowbreak10 2Sa 8:1-\allowbreak18; 10:1-\allowbreak19 1Ki 4:21 Ps 2:6-\allowbreak9}
\crossref{Gen}{27}{30}{}
\crossref{Gen}{27}{31}{27:4}
\crossref{Gen}{27}{32}{}
\crossref{Gen}{27}{33}{27:25}
\crossref{Gen}{27}{34}{1Sa 30:4 Pr 1:24-\allowbreak28,\allowbreak31; 19:3 Lu 13:24-\allowbreak28 Heb 12:17}
\crossref{Gen}{27}{35}{27:19-\allowbreak23 2Ki 10:19 Job 13:7 Mal 2:10 Ro 3:7,\allowbreak8 2Co 4:7 1Th 4:6}
\crossref{Gen}{27}{36}{Ge 25:26,\allowbreak31-\allowbreak34; 32:28 Joh 1:47}
\crossref{Gen}{27}{37}{27:29; 25:23 2Sa 8:14 Ro 9:10-\allowbreak12}
\crossref{Gen}{27}{38}{27:34,\allowbreak36; 49:28 Pr 1:24-\allowbreak26 Isa 32:10-\allowbreak12; 65:14 Heb 12:17}
\crossref{Gen}{27}{39}{Ge 36:6-\allowbreak8 Jos 24:4 Heb 11:20}
\crossref{Gen}{27}{40}{Ge 32:6 Mt 10:34}
\crossref{Gen}{27}{41}{Ge 4:2-\allowbreak8; 37:4,\allowbreak8 Eze 25:12-\allowbreak15; 35:5 Am 1:11,\allowbreak12 Ob 1:10-\allowbreak14 1Jo 3:12-\allowbreak15}
\crossref{Gen}{27}{42}{Ge 37:18-\allowbreak20; 42:21,\allowbreak22 1Sa 30:5 Job 20:12-\allowbreak14 Ps 64:5 Pr 2:14}
\crossref{Gen}{27}{43}{27:8,\allowbreak13; 28:7 Pr 30:17 Jer 35:14 Ac 5:29}
\crossref{Gen}{27}{44}{Ge 31:38}
\crossref{Gen}{27}{45}{Pr 19:21 La 3:37 Jas 4:13-\allowbreak15}
\crossref{Gen}{27}{46}{Nu 11:15 1Ki 19:4 Job 3:20-\allowbreak22; 7:16; 14:13 Jon 4:3,\allowbreak9}
\crossref{Gen}{28}{1}{28:3,\allowbreak4; 27:4,\allowbreak27-\allowbreak33; 48:15; 49:28 De 33:1 Jos 22:7}
\crossref{Gen}{28}{2}{Ho 12:12}
\crossref{Gen}{28}{3}{Ge 17:1-\allowbreak6; 22:17,\allowbreak18; 35:11; 43:14; 48:3 Ex 6:3 Ps 127:1 2Co 6:18}
\crossref{Gen}{28}{4}{Ge 12:1-\allowbreak3,\allowbreak7; 15:5-\allowbreak7; 17:6-\allowbreak8; 22:17,\allowbreak18 Ps 72:17 Ro 4:7,\allowbreak8 Ga 3:8,\allowbreak14}
\crossref{Gen}{28}{5}{28:2}
\crossref{Gen}{28}{6}{Ge 27:33}
\crossref{Gen}{28}{7}{Ge 27:43 Ex 20:12 Le 19:3 Pr 1:8; 30:17 Eph 6:1,\allowbreak3 Col 3:20}
\crossref{Gen}{28}{8}{28:1; 24:3; 26:34,\allowbreak35}
\crossref{Gen}{28}{9}{Ge 25:13-\allowbreak17; 36:3,\allowbreak13,\allowbreak18}
\crossref{Gen}{28}{10}{Ge 11:31; 32:10 Ho 12:12 Ac 7:2; 25:13}
\crossref{Gen}{28}{11}{28:18; 31:46 Mt 8:20 2Co 1:5}
\crossref{Gen}{28}{12}{Ge 15:1,\allowbreak12; 20:3,\allowbreak6,\allowbreak7; 37:5-\allowbreak11; 40:1-\allowbreak41:57 Nu 12:6 Job 4:12-\allowbreak21}
\crossref{Gen}{28}{13}{Ge 35:1,\allowbreak6,\allowbreak7; 48:3}
\crossref{Gen}{28}{14}{Ge 13:16; 32:12; 35:11,\allowbreak12 Nu 23:10 Ac 3:25 Re 7:4,\allowbreak9}
\crossref{Gen}{28}{15}{28:20,\allowbreak21; 26:24; 31:3; 32:9; 39:2,\allowbreak21; 46:4 Ex 3:12 Jud 6:16 Ps 46:7,\allowbreak11}
\crossref{Gen}{28}{16}{Ex 3:5; 15:11 Jos 5:15 1Sa 3:4-\allowbreak7 Job 9:11; 33:14 Ps 68:35}
\crossref{Gen}{28}{17}{Ex 3:6 Jud 13:22 Mt 17:6 Lu 2:9; 8:35 Re 1:17}
\crossref{Gen}{28}{18}{Ge 22:3 Ps 119:60 Ec 9:10}
\crossref{Gen}{28}{19}{Ge 12:8; 35:1; 48:3 Jud 1:22-\allowbreak26 1Ki 12:29 Ho 4:15; 12:4,\allowbreak5}
\crossref{Gen}{28}{20}{Ge 31:13 Le 27:1-\allowbreak34 Nu 6:1-\allowbreak20; 21:2,\allowbreak3 Jud 11:30,\allowbreak31 1Sa 1:11,\allowbreak28}
\crossref{Gen}{28}{21}{Jud 11:31 2Sa 19:24,\allowbreak30}
\crossref{Gen}{28}{22}{28:17; 12:8; 21:33; 33:20; 35:1,\allowbreak15}
\crossref{Gen}{29}{1}{Ps 119:32,\allowbreak60 Ec 9:7}
\crossref{Gen}{29}{2}{Ge 24:11,\allowbreak13 Ex 2:15,\allowbreak16 Joh 4:6,\allowbreak14}
\crossref{Gen}{29}{3}{}
\crossref{Gen}{29}{4}{Ge 11:31; 24:10; 27:43; 28:10 Ac 7:2,\allowbreak4}
\crossref{Gen}{29}{5}{Ge 24:24,\allowbreak29; 31:53}
\crossref{Gen}{29}{6}{Ge 37:14; 43:27 Ex 18:7 1Sa 17:22; 25:5 2Sa 20:9}
\crossref{Gen}{29}{7}{Ga 6:9,\allowbreak10 Eph 5:16}
\crossref{Gen}{29}{8}{29:3; 34:14; 43:32}
\crossref{Gen}{29}{9}{Ge 24:15 Ex 2:15,\allowbreak16,\allowbreak21 So 1:7,\allowbreak8}
\crossref{Gen}{29}{10}{Ex 2:17}
\crossref{Gen}{29}{11}{29:13; 27:26; 33:4; 43:30; 45:2,\allowbreak14,\allowbreak15 Ex 4:27; 18:7 Ro 16:16}
\crossref{Gen}{29}{12}{Ge 13:8; 14:14-\allowbreak16}
\crossref{Gen}{29}{13}{Ge 24:29}
\crossref{Gen}{29}{14}{29:12,\allowbreak15; 2:23; 13:8 Jud 9:2 2Sa 5:1; 19:12,\allowbreak13 Mic 7:5 Eph 5:30}
\crossref{Gen}{29}{15}{Ge 30:28; 31:7}
\crossref{Gen}{29}{16}{29:17,\allowbreak25-\allowbreak32; 30:19; 31:4; 33:2; 35:23; 46:15; 49:31 Ru 4:11}
\crossref{Gen}{29}{17}{29:6-\allowbreak12,\allowbreak18; 30:1,\allowbreak2,\allowbreak22; 35:19,\allowbreak20,\allowbreak24; 46:19-\allowbreak22; 48:7 1Sa 10:2 Jer 31:15}
\crossref{Gen}{29}{18}{29:20,\allowbreak30}
\crossref{Gen}{29}{19}{Ps 12:2 Isa 6:5,\allowbreak11}
\crossref{Gen}{29}{20}{Ge 30:26 Ho 12:12}
\crossref{Gen}{29}{21}{Mt 1:18}
\crossref{Gen}{29}{22}{Jud 14:10-\allowbreak18 Ru 4:10-\allowbreak13 Mt 22:2-\allowbreak10; 25:1-\allowbreak10 Joh 2:1-\allowbreak10 Re 19:9}
\crossref{Gen}{29}{23}{Ge 24:65; 38:14,\allowbreak15 Mic 7:5}
\crossref{Gen}{29}{24}{Ge 16:1; 24:59; 30:9-\allowbreak12; 46:18}
\crossref{Gen}{29}{25}{1Co 3:13}
\crossref{Gen}{29}{26}{29:26}
\crossref{Gen}{29}{27}{Ge 2:2,\allowbreak3; 8:10-\allowbreak12 Le 18:18 Jud 14:10,\allowbreak12 Mal 2:15 1Ti 6:10}
\crossref{Gen}{29}{28}{}
\crossref{Gen}{29}{29}{29:24; 30:3-\allowbreak8; 35:22,\allowbreak25; 37:2}
\crossref{Gen}{29}{30}{29:20,\allowbreak31; 44:20,\allowbreak27 De 21:15 Mt 6:24; 10:37 Lu 14:26 Joh 12:25}
\crossref{Gen}{29}{31}{Ex 3:7}
\crossref{Gen}{29}{32}{Ge 35:22; 37:21,\allowbreak22,\allowbreak29; 42:22,\allowbreak27; 46:8,\allowbreak9; 49:3,\allowbreak4 1Ch 5:1}
\crossref{Gen}{29}{33}{Ge 30:6,\allowbreak8,\allowbreak18,\allowbreak20}
\crossref{Gen}{29}{34}{Ge 34:25; 35:23; 46:11; 49:5-\allowbreak7 Ex 2:1; 32:26-\allowbreak29 De 33:8-\allowbreak10}
\crossref{Gen}{29}{35}{Ge 35:26; 38:1-\allowbreak30; 43:8,\allowbreak9; 44:18-\allowbreak34; 46:12; 49:8-\allowbreak12 De 33:7 1Ch 5:2}
\crossref{Gen}{30}{1}{Ge 29:31}
\crossref{Gen}{30}{2}{Ge 31:36 Ex 32:19 Mt 5:22 Mr 3:5 Eph 4:26}
\crossref{Gen}{30}{3}{30:9; 16:2,\allowbreak3}
\crossref{Gen}{30}{4}{Ge 16:3; 21:10; 22:24; 25:1,\allowbreak6; 33:2; 35:22 2Sa 12:11}
\crossref{Gen}{30}{5}{}
\crossref{Gen}{30}{6}{Ge 29:32-\allowbreak35 Ps 35:24; 43:1 La 3:59}
\crossref{Gen}{30}{7}{30:7}
\crossref{Gen}{30}{8}{Ge 23:6; 32:24 Ex 9:28 1Sa 14:15}
\crossref{Gen}{30}{9}{30:17; 29:35}
\crossref{Gen}{30}{10}{}
\crossref{Gen}{30}{11}{Ge 35:26; 46:16; 49:19 De 33:20,\allowbreak21}
\crossref{Gen}{30}{12}{}
\crossref{Gen}{30}{13}{Ge 35:26; 46:17; 49:20 De 33:24,\allowbreak25}
\crossref{Gen}{30}{14}{Ge 25:30}
\crossref{Gen}{30}{15}{Nu 16:9,\allowbreak10,\allowbreak13 Isa 7:13 Eze 16:47 1Co 4:3}
\crossref{Gen}{30}{16}{}
\crossref{Gen}{30}{17}{30:6,\allowbreak22 Ex 3:7 1Sa 1:20,\allowbreak26,\allowbreak27 Lu 1:13}
\crossref{Gen}{30}{18}{Ge 35:23; 46:13; 49:14,\allowbreak15 De 33:18 1Ch 12:32}
\crossref{Gen}{30}{19}{}
\crossref{Gen}{30}{20}{30:15; 29:34}
\crossref{Gen}{30}{21}{Ge 34:1-\allowbreak3,\allowbreak26; 46:15}
\crossref{Gen}{30}{22}{Ge 8:1; 21:1; 29:31 1Sa 1:19,\allowbreak20 Ps 105:42}
\crossref{Gen}{30}{23}{}
\crossref{Gen}{30}{24}{Ge 35:24; 37:2,\allowbreak4; 39:1-\allowbreak23; 42:6; 48:1 etc.}
\crossref{Gen}{30}{25}{Ge 24:54,\allowbreak56}
\crossref{Gen}{30}{26}{Ge 29:19,\allowbreak20,\allowbreak30; 31:26,\allowbreak31,\allowbreak41 Ho 12:12}
\crossref{Gen}{30}{27}{Ge 18:3; 33:15; 34:11; 39:3-\allowbreak5,\allowbreak21; 47:25 Ex 3:21 Nu 11:11,\allowbreak15 Ru 2:13}
\crossref{Gen}{30}{28}{Ge 29:15,\allowbreak19}
\crossref{Gen}{30}{29}{30:5; 31:6,\allowbreak38-\allowbreak40 Mt 24:45 Eph 6:5-\allowbreak8 Col 3:22-\allowbreak25 Tit 2:9,\allowbreak10}
\crossref{Gen}{30}{30}{30:43}
\crossref{Gen}{30}{31}{2Sa 21:4-\allowbreak6 Ps 118:8 Heb 13:5}
\crossref{Gen}{30}{32}{30:35; 31:8,\allowbreak10}
\crossref{Gen}{30}{33}{Ge 31:37 1Sa 26:23 2Sa 22:21 Ps 37:6}
\crossref{Gen}{30}{34}{Nu 22:29 1Co 7:7; 14:5 Ga 5:12}
\crossref{Gen}{30}{35}{Ge 31:9}
\crossref{Gen}{30}{36}{}
\crossref{Gen}{30}{37}{Ge 31:9-\allowbreak13}
\crossref{Gen}{30}{38}{}
\crossref{Gen}{30}{39}{Ge 31:9-\allowbreak12,\allowbreak38,\allowbreak40,\allowbreak42 Ex 12:35,\allowbreak36}
\crossref{Gen}{30}{40}{}
\crossref{Gen}{30}{41}{}
\crossref{Gen}{30}{42}{30:42}
\crossref{Gen}{30}{43}{30:30; 13:2; 24:35; 26:13,\allowbreak14; 28:15; 31:7,\allowbreak8,\allowbreak42; 32:10; 33:11; 36:7 Ec 2:7}
\crossref{Gen}{31}{1}{31:8,\allowbreak9 Job 31:31 Ps 57:4; 64:3,\allowbreak4; 120:3-\allowbreak5 Pr 14:30; 27:4 Ec 4:4}
\crossref{Gen}{31}{2}{Ge 4:5 De 28:54 1Sa 18:9-\allowbreak11 Da 3:19}
\crossref{Gen}{31}{3}{Ge 28:15,\allowbreak20,\allowbreak21 etc.}
\crossref{Gen}{31}{4}{}
\crossref{Gen}{31}{5}{31:2,\allowbreak3}
\crossref{Gen}{31}{6}{31:38-\allowbreak42; 30:29 Eph 6:5-\allowbreak8 Col 3:22-\allowbreak25 Tit 2:9,\allowbreak10 1Pe 2:18}
\crossref{Gen}{31}{7}{31:29; 20:6 Job 1:10 Ps 37:28; 105:14,\allowbreak15 Isa 54:17}
\crossref{Gen}{31}{8}{Ge 30:32}
\crossref{Gen}{31}{9}{31:1,\allowbreak16 Es 8:1,\allowbreak2 Ps 50:10 Pr 13:22 Mt 20:15}
\crossref{Gen}{31}{10}{31:24; 20:6; 28:12 Nu 12:6 De 13:1 1Ki 3:5}
\crossref{Gen}{31}{11}{31:5,\allowbreak13; 16:7-\allowbreak13; 18:1,\allowbreak17; 48:15,\allowbreak16}
\crossref{Gen}{31}{12}{Ge 30:37-\allowbreak43}
\crossref{Gen}{31}{13}{Ge 28:12-\allowbreak22; 35:7}
\crossref{Gen}{31}{14}{Ru 4:11}
\crossref{Gen}{31}{15}{31:41; 29:15-\allowbreak20,\allowbreak27-\allowbreak30; 30:26 Ex 21:7-\allowbreak11 Ne 5:8}
\crossref{Gen}{31}{16}{31:1,\allowbreak9; 30:35-\allowbreak43}
\crossref{Gen}{31}{17}{Ge 24:10,\allowbreak61 1Sa 30:17}
\crossref{Gen}{31}{18}{Ge 27:1,\allowbreak2,\allowbreak41; 28:21; 35:27-\allowbreak29}
\crossref{Gen}{31}{19}{31:30,\allowbreak32; 35:2 Jos 24:2 Jud 17:4,\allowbreak5; 18:14-\allowbreak24,\allowbreak31 1Sa 19:13 Eze 21:21}
\crossref{Gen}{31}{20}{}
\crossref{Gen}{31}{21}{Ge 2:14; 15:18 Jos 24:2,\allowbreak3}
\crossref{Gen}{31}{22}{Ge 30:36 Ex 14:5 etc.}
\crossref{Gen}{31}{23}{Ge 13:8; 24:27 Ex 2:11,\allowbreak13}
\crossref{Gen}{31}{24}{Ge 28:5 De 26:5 Ho 12:12}
\crossref{Gen}{31}{25}{Ge 12:8; 33:18 Heb 11:9}
\crossref{Gen}{31}{26}{31:36; 3:13; 4:10; 12:18; 20:9,\allowbreak10; 26:10 Jos 7:19 1Sa 14:43; 17:29}
\crossref{Gen}{31}{27}{31:3-\allowbreak5,\allowbreak20,\allowbreak21,\allowbreak31 Jud 6:27}
\crossref{Gen}{31}{28}{31:55; 29:13 Ex 4:27 Ru 1:9,\allowbreak14 1Ki 19:20 Ac 20:37}
\crossref{Gen}{31}{29}{Ps 52:1 Joh 19:10,\allowbreak11}
\crossref{Gen}{31}{30}{31:19 Ex 12:12 Nu 33:4 Jud 6:31; 18:24 1Sa 5:2-\allowbreak6 2Sa 5:21}
\crossref{Gen}{31}{31}{31:26,\allowbreak27; 20:11 Pr 29:25}
\crossref{Gen}{31}{32}{31:23; 13:8; 19:7; 30:33 1Sa 12:3-\allowbreak5 2Co 8:20,\allowbreak21; 12:17-\allowbreak19}
\crossref{Gen}{31}{33}{Ge 24:28,\allowbreak67}
\crossref{Gen}{31}{34}{31:17,\allowbreak19}
\crossref{Gen}{31}{35}{Ge 18:12 Ex 20:12 Le 19:3 Eph 6:1 1Pe 2:18; 3:6}
\crossref{Gen}{31}{36}{Ge 30:2; 34:7; 49:7 Nu 16:15 2Ki 5:11; 13:19 Pr 28:1 Mr 3:5 Eph 4:26}
\crossref{Gen}{31}{37}{31:32 Jos 7:23 1Sa 12:3,\allowbreak4 Mt 18:16 1Co 6:4,\allowbreak5 1Th 2:10 Heb 13:18}
\crossref{Gen}{31}{38}{31:41}
\crossref{Gen}{31}{39}{Ex 22:10,\allowbreak31 Le 22:8 1Sa 17:34,\allowbreak35 Joh 10:12,\allowbreak13}
\crossref{Gen}{31}{40}{Ex 2:19-\allowbreak22; 3:1 Ps 78:70,\allowbreak71 Ho 12:12 Lu 2:8 Joh 21:15-\allowbreak17}
\crossref{Gen}{31}{41}{31:38; 29:18-\allowbreak30; 30:33-\allowbreak40 1Co 15:10 2Co 11:26}
\crossref{Gen}{31}{42}{31:24,\allowbreak29 Ps 124:1-\allowbreak3}
\crossref{Gen}{31}{43}{}
\crossref{Gen}{31}{44}{Ge 15:18; 21:22-\allowbreak32; 26:28-\allowbreak31 1Sa 20:14-\allowbreak17}
\crossref{Gen}{31}{45}{Ge 28:18-\allowbreak22}
\crossref{Gen}{31}{46}{31:23,\allowbreak32,\allowbreak37,\allowbreak54}
\crossref{Gen}{31}{47}{}
\crossref{Gen}{31}{48}{Jos 24:27}
\crossref{Gen}{31}{49}{Jud 10:17; 11:11,\allowbreak29}
\crossref{Gen}{31}{50}{Le 18:18 Mt 19:5,\allowbreak6}
\crossref{Gen}{31}{51}{}
\crossref{Gen}{31}{52}{31:44,\allowbreak45,\allowbreak48}
\crossref{Gen}{31}{53}{Ge 11:24-\allowbreak29,\allowbreak31; 17:7; 22:20-\allowbreak24; 24:3,\allowbreak4 Ex 3:6 Jos 24:2}
\crossref{Gen}{31}{54}{Ge 21:8; 26:30; 37:25 Ex 18:12 2Sa 3:20,\allowbreak21}
\crossref{Gen}{31}{55}{31:28; 33:4 Ru 1:14}
\crossref{Gen}{32}{1}{Ps 91:11 Heb 1:4 1Co 3:22 Eph 3:10}
\crossref{Gen}{32}{2}{Jos 5:14 2Ki 6:17 Ps 34:7; 103:21; 148:2 Da 10:20 Lu 2:13}
\crossref{Gen}{32}{3}{Mal 3:1 Lu 9:52; 14:31,\allowbreak32}
\crossref{Gen}{32}{4}{32:5,\allowbreak18; 4:7; 23:6; 27:29,\allowbreak37; 33:8 Ex 32:22 1Sa 26:17 Pr 6:3; 15:1}
\crossref{Gen}{32}{5}{Ge 30:43; 31:1,\allowbreak16; 33:11 Job 6:22}
\crossref{Gen}{32}{6}{32:8,\allowbreak11; 27:40,\allowbreak41; 33:1 Am 5:19}
\crossref{Gen}{32}{7}{Ex 14:10 Ps 18:4,\allowbreak5; 31:13; 55:4,\allowbreak5; 61:2; 142:4 Mt 8:26 Joh 16:33}
\crossref{Gen}{32}{8}{Ge 33:1-\allowbreak3 Mt 10:16}
\crossref{Gen}{32}{9}{1Sa 30:6 2Ch 20:6,\allowbreak12; 32:20 Ps 34:4-\allowbreak6; 50:15; 91:15 Php 4:6,\allowbreak7}
\crossref{Gen}{32}{10}{Ge 18:27 2Sa 7:18 Job 42:5,\allowbreak6 Ps 16:2 Isa 6:5; 63:7 Da 9:8,\allowbreak9 Lu 5:8}
\crossref{Gen}{32}{11}{1Sa 12:10; 24:15 Ps 16:1; 25:20; 31:2; 43:1; 59:1,\allowbreak2; 119:134; 142:6}
\crossref{Gen}{32}{12}{32:6 Ex 32:13 Nu 23:19 1Sa 15:29 Mt 24:35 2Ti 2:13 Tit 1:2}
\crossref{Gen}{32}{13}{1Sa 25:8}
\crossref{Gen}{32}{14}{}
\crossref{Gen}{32}{15}{}
\crossref{Gen}{32}{16}{32:20; 33:8,\allowbreak9 Ps 112:5 Pr 2:11 Isa 28:26 Mt 10:16}
\crossref{Gen}{32}{17}{Ge 33:3}
\crossref{Gen}{32}{18}{32:4,\allowbreak5}
\crossref{Gen}{32}{19}{}
\crossref{Gen}{32}{20}{Ge 43:11 1Sa 25:17-\allowbreak35 Job 42:8,\allowbreak9 Pr 15:18; 16:14; 21:14}
\crossref{Gen}{32}{21}{}
\crossref{Gen}{32}{22}{Ge 29:21-\allowbreak35; 30:1-\allowbreak24; 35:18,\allowbreak22-\allowbreak26 1Ti 5:8}
\crossref{Gen}{32}{23}{32:23}
\crossref{Gen}{32}{24}{Ge 30:8 Lu 13:24; 22:44 Ro 8:26,\allowbreak27; 15:30 Eph 6:12,\allowbreak18 Col 2:1; 4:12}
\crossref{Gen}{32}{25}{Ge 19:22 Nu 14:13,\allowbreak14 Isa 41:14; 45:11 Ho 12:3,\allowbreak4 Mt 15:22-\allowbreak28}
\crossref{Gen}{32}{26}{Ex 32:10 De 9:14 So 7:5 Isa 45:11; 64:7 Lu 24:28,\allowbreak29}
\crossref{Gen}{32}{27}{}
\crossref{Gen}{32}{28}{Ge 17:5,\allowbreak15; 33:20; 35:10 Nu 13:16 2Sa 12:25 2Ki 17:34 Isa 62:2-\allowbreak4}
\crossref{Gen}{32}{29}{32:27 De 29:29 Jud 13:16-\allowbreak18 Job 11:7 Pr 30:4 Isa 9:6 Lu 1:19}
\crossref{Gen}{32}{30}{32:31}
\crossref{Gen}{32}{31}{Ge 19:15,\allowbreak23 Mal 4:2}
\crossref{Gen}{32}{32}{1Sa 5:5}
\crossref{Gen}{33}{1}{Ge 27:41,\allowbreak42; 32:6}
\crossref{Gen}{33}{2}{Ge 29:30; 30:22-\allowbreak24; 37:3 Mal 3:17}
\crossref{Gen}{33}{3}{Joh 10:4,\allowbreak11,\allowbreak12,\allowbreak15}
\crossref{Gen}{33}{4}{Ge 32:28; 43:30,\allowbreak34; 45:2,\allowbreak15 Job 2:12 Ne 1:11 Ps 34:4 Pr 16:7; 21:1}
\crossref{Gen}{33}{5}{Ge 30:2; 48:9 Ru 4:13 1Sa 1:27 1Ch 28:5 Ps 127:3 Isa 8:18 Heb 2:13}
\crossref{Gen}{33}{6}{33:6}
\crossref{Gen}{33}{7}{}
\crossref{Gen}{33}{8}{Ge 32:5; 39:5 Es 2:17}
\crossref{Gen}{33}{9}{Ge 27:39 Pr 30:15 Ec 4:8}
\crossref{Gen}{33}{10}{Ge 19:19; 47:29; 50:4 Ex 33:12,\allowbreak13 Ru 2:10 1Sa 20:3 Jer 31:2}
\crossref{Gen}{33}{11}{Ge 32:13-\allowbreak20 Jos 15:19 Jud 1:15 1Sa 25:27; 30:26 2Ki 5:15 2Co 9:5,\allowbreak6}
\crossref{Gen}{33}{12}{}
\crossref{Gen}{33}{13}{1Ch 22:5 Pr 12:10 Isa 40:11 Eze 34:15,\allowbreak16,\allowbreak23-\allowbreak25 Joh 21:15-\allowbreak17}
\crossref{Gen}{33}{14}{Isa 40:11 Mr 4:33 Ro 15:1 1Co 3:2; 9:19-\allowbreak22}
\crossref{Gen}{33}{15}{Ge 34:11; 47:25 Ru 2:13 1Sa 25:8 2Sa 16:4}
\crossref{Gen}{33}{16}{}
\crossref{Gen}{33}{17}{Jos 13:27 Jud 8:5,\allowbreak8,\allowbreak16 1Ki 7:46 Ps 60:6}
\crossref{Gen}{33}{18}{}
\crossref{Gen}{33}{19}{Ge 23:17-\allowbreak20; 49:30-\allowbreak32 Jos 24:32 Joh 4:5 Ac 7:16}
\crossref{Gen}{33}{20}{Ge 8:20; 12:7,\allowbreak8; 13:18; 21:33}
\crossref{Gen}{34}{1}{Ge 30:21; 46:15}
\crossref{Gen}{34}{2}{Ge 10:17; 33:19}
\crossref{Gen}{34}{3}{Ru 1:14 1Sa 18:1}
\crossref{Gen}{34}{4}{Ge 21:21 Jud 14:2 2Sa 13:13}
\crossref{Gen}{34}{5}{Ge 30:35; 37:13,\allowbreak14 1Sa 10:27; 16:11; 17:15 2Sa 13:22 Lu 15:25,\allowbreak29}
\crossref{Gen}{34}{6}{}
\crossref{Gen}{34}{7}{Ge 46:7 2Sa 13:21}
\crossref{Gen}{34}{8}{34:3 1Ki 11:2 Ps 63:1; 84:2; 119:20}
\crossref{Gen}{34}{9}{Ge 6:2; 19:14; 24:3; 26:34,\allowbreak35; 27:46 De 7:3}
\crossref{Gen}{34}{10}{34:21-\allowbreak23; 13:9; 20:15; 42:34; 47:27}
\crossref{Gen}{34}{11}{Ge 18:3; 33:15}
\crossref{Gen}{34}{12}{Ge 24:53; 29:18; 31:41 Ex 22:16,\allowbreak17 De 22:28,\allowbreak29 1Sa 18:25-\allowbreak27}
\crossref{Gen}{34}{13}{Ge 25:27-\allowbreak34 Jud 15:3 2Sa 13:23-\allowbreak29 Job 13:4,\allowbreak7 Ps 12:2 Pr 12:13}
\crossref{Gen}{34}{14}{Ge 17:11 Jos 5:2-\allowbreak9 1Sa 14:6; 17:26,\allowbreak36 2Sa 1:20; 15:7 1Ki 21:9}
\crossref{Gen}{34}{15}{Ga 4:12}
\crossref{Gen}{34}{16}{34:16}
\crossref{Gen}{34}{17}{34:17}
\crossref{Gen}{34}{18}{}
\crossref{Gen}{34}{19}{Ge 29:20 So 8:6 Isa 62:4}
\crossref{Gen}{34}{20}{Ge 22:17; 23:10 De 17:5 Ru 4:1 Job 29:7 Pr 31:23 Am 5:10,\allowbreak12,\allowbreak15}
\crossref{Gen}{34}{21}{34:21}
\crossref{Gen}{34}{22}{34:15-\allowbreak17}
\crossref{Gen}{34}{23}{Pr 1:12,\allowbreak13; 23:4,\allowbreak5; 28:20 Joh 2:16; 6:26,\allowbreak27 Ac 19:24-\allowbreak26}
\crossref{Gen}{34}{24}{Ge 23:10,\allowbreak18}
\crossref{Gen}{34}{25}{Jos 5:6,\allowbreak8}
\crossref{Gen}{34}{26}{De 32:42 2Sa 2:26 Isa 31:8}
\crossref{Gen}{34}{27}{Es 9:10,\allowbreak16 1Ti 6:10}
\crossref{Gen}{34}{28}{Nu 31:17 De 8:17,\allowbreak18 Job 1:15,\allowbreak16; 20:5}
\crossref{Gen}{34}{29}{}
\crossref{Gen}{34}{30}{Ge 49:5-\allowbreak7 Jos 7:25 1Ki 18:18 1Ch 2:7 Pr 11:17,\allowbreak29; 15:27}
\crossref{Gen}{34}{31}{34:13; 49:7 Pr 6:34}
\crossref{Gen}{35}{1}{Ge 22:14 De 32:36 Ps 46:1; 91:15}
\crossref{Gen}{35}{2}{Ge 18:19 Jos 24:15 Ps 101:2-\allowbreak7}
\crossref{Gen}{35}{3}{Ge 28:12,\allowbreak13; 32:7,\allowbreak24 Ps 46:1; 50:15; 66:13,\allowbreak14; 91:15; 103:1-\allowbreak5; 107:6,\allowbreak8}
\crossref{Gen}{35}{4}{Ex 32:20 De 7:5,\allowbreak25 Isa 2:20; 30:22}
\crossref{Gen}{35}{5}{Ge 34:30 Ex 15:15,\allowbreak16; 23:27; 34:24 De 11:25 Jos 2:9-\allowbreak11; 5:1 1Sa 11:7}
\crossref{Gen}{35}{6}{Ge 12:8; 28:19,\allowbreak22 Jud 1:22-\allowbreak26}
\crossref{Gen}{35}{7}{35:1,\allowbreak3 Ec 5:4,\allowbreak5}
\crossref{Gen}{35}{8}{Ge 24:59}
\crossref{Gen}{35}{9}{Ge 12:7; 17:1; 18:1; 26:2; 28:13; 31:3,\allowbreak11-\allowbreak13; 32:1,\allowbreak24-\allowbreak30; 35:1; 46:2,\allowbreak3}
\crossref{Gen}{35}{10}{Ge 17:5,\allowbreak15; 32:27,\allowbreak28 1Ki 18:31 2Ki 17:34}
\crossref{Gen}{35}{11}{Ge 17:1; 18:14; 43:14; 48:3,\allowbreak4 Ex 6:3 2Co 6:18}
\crossref{Gen}{35}{12}{Ge 12:7; 13:14-\allowbreak17; 15:18; 26:3,\allowbreak4; 28:3,\allowbreak4,\allowbreak13; 48:4 Ex 3:8 Jos 6:1-\allowbreak21:45}
\crossref{Gen}{35}{13}{Ge 11:5; 17:22; 18:33 Jud 6:21; 13:20 Lu 24:31}
\crossref{Gen}{35}{14}{35:20; 28:18,\allowbreak19 Ex 17:15 1Sa 7:12}
\crossref{Gen}{35}{15}{Ge 28:19}
\crossref{Gen}{35}{16}{2Ki 5:19}
\crossref{Gen}{35}{17}{Ge 30:24 1Sa 4:19-\allowbreak21}
\crossref{Gen}{35}{18}{Ge 30:1 1Sa 4:20,\allowbreak21 Ps 16:10 Ex 12:7 La 2:12 Lu 12:20; 23:46}
\crossref{Gen}{35}{19}{Ge 48:7}
\crossref{Gen}{35}{20}{35:9,\allowbreak14 1Sa 10:2 2Sa 18:17,\allowbreak18}
\crossref{Gen}{35}{21}{Mic 4:8 Lu 2:8}
\crossref{Gen}{35}{22}{Ge 49:4 Le 18:8 2Sa 16:21,\allowbreak22; 20:3 1Ch 5:1 1Co 5:1}
\crossref{Gen}{35}{23}{Ge 29:32-\allowbreak35; 30:18-\allowbreak20; 33:2; 46:8-\allowbreak15}
\crossref{Gen}{35}{24}{35:16-\allowbreak18; 30:22-\allowbreak24; 46:19-\allowbreak22}
\crossref{Gen}{35}{25}{Ge 30:4-\allowbreak8; 37:2; 46:23-\allowbreak25}
\crossref{Gen}{35}{26}{Ge 30:9-\allowbreak13; 46:16-\allowbreak18}
\crossref{Gen}{35}{27}{Ge 27:43-\allowbreak45; 28:5}
\crossref{Gen}{35}{28}{Ge 25:7; 47:28; 50:26}
\crossref{Gen}{35}{29}{Ge 3:19; 15:15; 25:7,\allowbreak8,\allowbreak17; 27:1,\allowbreak2; 49:33 Job 5:26 Ec 12:5-\allowbreak7}
\crossref{Gen}{36}{1}{Ge 22:17; 25:24-\allowbreak34; 27:35-\allowbreak41; 32:3-\allowbreak7 Nu 20:14-\allowbreak21 De 23:7 1Ch 1:35}
\crossref{Gen}{36}{2}{Ge 9:25; 26:34,\allowbreak35; 27:46}
\crossref{Gen}{36}{3}{Ge 25:13; 28:9}
\crossref{Gen}{36}{4}{1Ch 1:35}
\crossref{Gen}{36}{5}{36:6}
\crossref{Gen}{36}{6}{Eze 27:13 Re 18:13}
\crossref{Gen}{36}{7}{Ge 13:6,\allowbreak11}
\crossref{Gen}{36}{8}{36:20; 14:6; 32:3 De 2:5 Jos 24:4 1Ch 4:42 2Ch 20:10,\allowbreak23 Eze 35:2-\allowbreak7}
\crossref{Gen}{36}{9}{Ge 19:37}
\crossref{Gen}{36}{10}{36:3,\allowbreak4 1Ch 1:35-\allowbreak54}
\crossref{Gen}{36}{11}{36:15,\allowbreak16 1Ch 1:35,\allowbreak36}
\crossref{Gen}{36}{12}{36:22 1Ch 1:36}
\crossref{Gen}{36}{13}{36:17 1Ch 1:37}
\crossref{Gen}{36}{14}{36:2,\allowbreak5,\allowbreak18 1Ch 1:35}
\crossref{Gen}{36}{15}{Job 21:8 Ps 37:35}
\crossref{Gen}{36}{16}{Ex 15:15}
\crossref{Gen}{36}{17}{36:4,\allowbreak13 1Ch 1:37}
\crossref{Gen}{36}{18}{36:5,\allowbreak14 1Ch 1:35}
\crossref{Gen}{36}{19}{36:1}
\crossref{Gen}{36}{20}{36:2,\allowbreak22-\allowbreak30; 14:6 De 2:12,\allowbreak22 1Ch 1:38-\allowbreak42}
\crossref{Gen}{36}{21}{36:21}
\crossref{Gen}{36}{22}{1Ch 1:39}
\crossref{Gen}{36}{23}{1Ch 1:40}
\crossref{Gen}{36}{24}{Le 19:19 De 2:10 2Sa 13:29; 18:9 1Ki 1:38,\allowbreak44; 4:28 Zec 14:15}
\crossref{Gen}{36}{25}{36:21}
\crossref{Gen}{36}{26}{1Ch 1:41}
\crossref{Gen}{36}{27}{36:21 1Ch 1:38}
\crossref{Gen}{36}{28}{Job 1:1 Jer 25:20 La 4:21}
\crossref{Gen}{36}{29}{36:20,\allowbreak28 1Ch 1:41,\allowbreak42}
\crossref{Gen}{36}{30}{}
\crossref{Gen}{36}{31}{Ge 17:6,\allowbreak16; 25:23 Nu 20:14; 24:17,\allowbreak18 De 17:14-\allowbreak20; 33:5,\allowbreak29}
\crossref{Gen}{36}{32}{1Ch 1:43}
\crossref{Gen}{36}{33}{}
\crossref{Gen}{36}{34}{36:11,\allowbreak15 Job 2:11 Jer 49:7}
\crossref{Gen}{36}{35}{36:35}
\crossref{Gen}{36}{36}{36:36}
\crossref{Gen}{36}{37}{Ge 10:11 1Ch 1:48}
\crossref{Gen}{36}{38}{36:38}
\crossref{Gen}{36}{39}{1Ch 1:50}
\crossref{Gen}{36}{40}{36:15,\allowbreak16 Ex 15:15 1Ch 1:51-\allowbreak54}
\crossref{Gen}{36}{41}{36:2,\allowbreak5,\allowbreak14,\allowbreak18,\allowbreak25 1Ch 1:52}
\crossref{Gen}{36}{42}{1Ch 1:53}
\crossref{Gen}{36}{43}{36:15,\allowbreak18,\allowbreak19,\allowbreak30,\allowbreak31 Ex 15:15 Nu 20:14}
\crossref{Gen}{37}{1}{}
\crossref{Gen}{37}{2}{Ge 30:4,\allowbreak9; 35:22,\allowbreak25,\allowbreak26}
\crossref{Gen}{37}{3}{Joh 3:35; 13:22,\allowbreak23}
\crossref{Gen}{37}{4}{37:5,\allowbreak11,\allowbreak18-\allowbreak24; 4:5; 27:41; 49:23 1Sa 16:12,\allowbreak13; 17:28 Ps 38:19; 69:4}
\crossref{Gen}{37}{5}{37:9}
\crossref{Gen}{37}{6}{Ge 44:18 Jud 9:7}
\crossref{Gen}{37}{7}{Ge 42:6,\allowbreak9; 43:26; 44:14,\allowbreak19}
\crossref{Gen}{37}{8}{37:4 Ex 2:14 1Sa 10:27; 17:28 Ps 2:3-\allowbreak6; 118:22 Lu 19:14; 20:17}
\crossref{Gen}{37}{9}{37:7; 41:25,\allowbreak32}
\crossref{Gen}{37}{10}{Ge 27:29 Isa 60:14 Php 2:10,\allowbreak11}
\crossref{Gen}{37}{11}{Ge 26:14-\allowbreak16 Ps 106:16 Ec 4:4 Isa 11:13; 26:11 Mt 27:18 Mr 15:10}
\crossref{Gen}{37}{12}{37:1; 33:18; 34:25-\allowbreak31}
\crossref{Gen}{37}{13}{1Sa 17:17-\allowbreak20 Mt 10:16 Lu 20:13}
\crossref{Gen}{37}{14}{Ge 23:2; 35:27 Nu 13:22 Jos 14:13,\allowbreak15}
\crossref{Gen}{37}{15}{Ge 21:14}
\crossref{Gen}{37}{16}{Lu 19:10}
\crossref{Gen}{37}{17}{2Ki 6:13}
\crossref{Gen}{37}{18}{1Sa 19:1 Ps 31:13; 37:12,\allowbreak32; 94:21; 105:25; 109:4 Mt 21:38; 27:1}
\crossref{Gen}{37}{19}{37:5,\allowbreak11; 28:12; 49:23}
\crossref{Gen}{37}{20}{Ps 64:5 Pr 1:11,\allowbreak12,\allowbreak16; 6:17; 27:4 Tit 3:3 Joh 3:12}
\crossref{Gen}{37}{21}{Ge 35:22; 42:22}
\crossref{Gen}{37}{22}{Ge 42:22}
\crossref{Gen}{37}{23}{37:3,\allowbreak31-\allowbreak33; 42:21 Ps 22:18 Mt 27:28}
\crossref{Gen}{37}{24}{Ps 35:7 La 4:20}
\crossref{Gen}{37}{25}{Es 3:15 Ps 14:4 Pr 30:20 Am 6:6}
\crossref{Gen}{37}{26}{Ge 25:32 Ps 30:9 Jer 41:8 Mt 16:26 Ro 6:21}
\crossref{Gen}{37}{27}{37:22 Ex 21:16,\allowbreak21 Ne 5:8 Mt 16:26; 26:15 1Ti 1:10 Re 18:13}
\crossref{Gen}{37}{28}{37:25; 25:2 Ex 2:16 Nu 25:15,\allowbreak17; 31:2,\allowbreak3,\allowbreak8,\allowbreak9 Jud 6:1-\allowbreak3 Ps 83:9}
\crossref{Gen}{37}{29}{37:34; 34:13 Nu 14:6 Jud 11:35 2Ki 19:1 Job 1:20 Joe 2:13}
\crossref{Gen}{37}{30}{37:20; 42:13,\allowbreak32,\allowbreak35 Jer 31:15}
\crossref{Gen}{37}{31}{37:3,\allowbreak23 Pr 28:13}
\crossref{Gen}{37}{32}{37:3; 44:20-\allowbreak23 Lu 15:30}
\crossref{Gen}{37}{33}{37:20; 44:28 1Ki 13:24 2Ki 2:24 Pr 14:15 Joh 13:7}
\crossref{Gen}{37}{34}{37:29 Jos 7:6 2Sa 1:11; 3:31 1Ki 20:31; 21:27 2Ki 19:1 1Ch 21:16}
\crossref{Gen}{37}{35}{Ge 31:43; 35:22-\allowbreak26}
\crossref{Gen}{37}{36}{37:28; 25:1,\allowbreak2; 39:1}
\crossref{Gen}{38}{1}{Ge 19:2,\allowbreak3 Jud 4:18 2Ki 4:8 Pr 9:6; 13:20}
\crossref{Gen}{38}{2}{Ge 3:6; 6:2; 24:3; 34:2 Jud 14:2; 16:1 2Sa 11:2 2Co 6:14}
\crossref{Gen}{38}{3}{Ge 46:12 Nu 26:19}
\crossref{Gen}{38}{4}{Ge 46:12 Nu 26:19}
\crossref{Gen}{38}{5}{38:11,\allowbreak26; 46:12 Nu 26:20 1Ch 4:21}
\crossref{Gen}{38}{6}{Ge 21:21; 24:3}
\crossref{Gen}{38}{7}{Ge 46:12 Nu 26:19}
\crossref{Gen}{38}{8}{Le 18:16 Nu 36:8,\allowbreak9 De 25:5-\allowbreak10 Ru 1:11; 4:5-\allowbreak11 Mt 22:23-\allowbreak27}
\crossref{Gen}{38}{9}{De 25:6 Ru 1:11; 4:10}
\crossref{Gen}{38}{10}{Nu 11:1; 22:34 2Sa 11:27 1Ch 21:7 Pr 14:32; 24:18 Jer 44:4}
\crossref{Gen}{38}{11}{Ru 1:11,\allowbreak13}
\crossref{Gen}{38}{12}{Ge 31:19 1Sa 25:4-\allowbreak8,\allowbreak36 2Sa 13:23-\allowbreak29}
\crossref{Gen}{38}{13}{Jud 14:1}
\crossref{Gen}{38}{14}{Pr 7:12 Jer 3:2 Eze 16:25}
\crossref{Gen}{38}{15}{Ge 34:31 Le 19:29; 21:14 Nu 25:1,\allowbreak6 De 23:18 Jud 11:1; 16:1; 19:2,\allowbreak25}
\crossref{Gen}{38}{16}{2Sa 13:11}
\crossref{Gen}{38}{17}{Eze 16:33}
\crossref{Gen}{38}{18}{}
\crossref{Gen}{38}{19}{38:14 2Sa 14:2,\allowbreak5}
\crossref{Gen}{38}{20}{Ge 20:9 Le 19:17 Jud 14:20 2Sa 13:3 Lu 23:12}
\crossref{Gen}{38}{21}{38:14}
\crossref{Gen}{38}{22}{}
\crossref{Gen}{38}{23}{2Sa 12:9 Pr 6:33 Ro 6:21 2Co 4:2 Eph 5:12 Re 16:15}
\crossref{Gen}{38}{24}{Ge 34:31 Jud 19:2 Ec 7:26 Jer 2:20; 3:1,\allowbreak6,\allowbreak8 Eze 16:15,\allowbreak28,\allowbreak41}
\crossref{Gen}{38}{25}{38:18; 37:32 Ps 50:21 Jer 2:26 Ro 2:16 1Co 4:5 Re 20:12}
\crossref{Gen}{38}{26}{Ge 37:33}
\crossref{Gen}{38}{27}{Ge 25:24}
\crossref{Gen}{38}{28}{38:1}
\crossref{Gen}{38}{29}{Mt 1:3 Lu 3:33}
\crossref{Gen}{38}{30}{1Ch 9:6}
\crossref{Gen}{39}{1}{Ge 37:36; 45:4 Ps 105:17 Ac 7:9}
\crossref{Gen}{39}{2}{39:21,\allowbreak22; 21:22; 26:24,\allowbreak28; 28:15 1Sa 3:19; 16:18; 18:14,\allowbreak28 Ps 1:3}
\crossref{Gen}{39}{3}{Ge 21:22; 26:24,\allowbreak28; 30:27,\allowbreak30 1Sa 18:14,\allowbreak28 Zec 8:23 Mt 5:16}
\crossref{Gen}{39}{4}{39:21; 18:3; 19:19; 32:5; 33:8,\allowbreak10 1Sa 16:22 Ne 2:4,\allowbreak5 Pr 16:7}
\crossref{Gen}{39}{5}{Ge 12:2; 19:29; 30:27 De 28:3-\allowbreak6 2Sa 6:11,\allowbreak12 Ps 21:6; 72:17 Ac 27:24}
\crossref{Gen}{39}{6}{39:4,\allowbreak8,\allowbreak23 Lu 16:10; 19:17}
\crossref{Gen}{39}{7}{Ge 6:2 Job 31:1 Ps 119:37 Eze 23:5,\allowbreak6,\allowbreak12-\allowbreak16 Mt 5:28 2Pe 2:14}
\crossref{Gen}{39}{8}{Pr 1:10; 2:10,\allowbreak16-\allowbreak19; 5:3-\allowbreak8; 6:20-\allowbreak25,\allowbreak29,\allowbreak32,\allowbreak33; 7:5,\allowbreak25-\allowbreak27; 9:13-\allowbreak18}
\crossref{Gen}{39}{9}{Ge 24:2 Ne 6:11 Lu 12:48 1Co 4:2 Tit 2:10}
\crossref{Gen}{39}{10}{39:8 Pr 2:16; 5:3; 6:25,\allowbreak26; 7:5,\allowbreak13; 9:14,\allowbreak16; 22:14; 23:27}
\crossref{Gen}{39}{11}{Job 24:15 Pr 9:17 Jer 23:24 Mal 3:5 Eph 5:3,\allowbreak12}
\crossref{Gen}{39}{12}{39:8,\allowbreak10 Pr 7:13-\allowbreak27 Ec 7:26 Eze 16:30,\allowbreak31}
\crossref{Gen}{39}{13}{Ge 38:1}
\crossref{Gen}{39}{14}{39:17; 10:21; 14:13; 40:15 Ps 120:3 Eze 22:5}
\crossref{Gen}{39}{15}{Ge 38:1}
\crossref{Gen}{39}{16}{Ps 37:12,\allowbreak32 Jer 4:22; 9:3-\allowbreak5 Tit 3:3}
\crossref{Gen}{39}{17}{39:14 Ex 20:16; 23:1 1Ki 18:17; 21:9-\allowbreak13 Ps 37:14; 55:3; 120:2-\allowbreak4}
\crossref{Gen}{39}{18}{39:14 Ex 20:16; 23:1 Le 19:16 Pr 6:19; 12:22; 19:5}
\crossref{Gen}{39}{19}{Job 29:16 Pr 18:17; 29:12 Ac 25:16 2Th 2:11}
\crossref{Gen}{39}{20}{Ge 40:1-\allowbreak3,\allowbreak15; 41:9-\allowbreak14 Ps 76:10}
\crossref{Gen}{39}{21}{39:2; 21:22; 49:23,\allowbreak24 Isa 41:10; 43:2 Da 6:22 Ro 8:31,\allowbreak32,\allowbreak37}
\crossref{Gen}{39}{22}{39:4,\allowbreak6,\allowbreak7,\allowbreak9; 40:3,\allowbreak4 1Sa 2:30 Ps 37:3,\allowbreak11}
\crossref{Gen}{39}{23}{Ge 40:3,\allowbreak4}
\crossref{Gen}{40}{1}{Ge 39:20-\allowbreak23 Es 6:1}
\crossref{Gen}{40}{2}{Ps 76:10 Pr 16:14; 19:12,\allowbreak19; 27:4 Ac 12:20}
\crossref{Gen}{40}{3}{Ge 39:20,\allowbreak23}
\crossref{Gen}{40}{4}{Ge 37:36; 39:1,\allowbreak21-\allowbreak23 Ps 37:5}
\crossref{Gen}{40}{5}{40:8; 12:1-\allowbreak7; 20:3; 37:5-\allowbreak10; 41:1-\allowbreak7,\allowbreak11 Nu 12:6 Jud 7:13,\allowbreak14 Es 6:1}
\crossref{Gen}{40}{6}{40:8; 41:8 Da 2:1-\allowbreak3; 4:5; 5:6; 7:28; 8:27}
\crossref{Gen}{40}{7}{Jud 18:24 1Sa 1:8 2Sa 13:4 Ne 2:2 Lu 24:17}
\crossref{Gen}{40}{8}{Ge 41:15,\allowbreak16 Job 33:15,\allowbreak16 Ps 25:14 Isa 8:19 Da 2:11,\allowbreak28,\allowbreak47; 4:8}
\crossref{Gen}{40}{9}{Ge 37:5-\allowbreak10 Jud 7:13-\allowbreak15 Da 2:31; 4:8,\allowbreak10-\allowbreak18}
\crossref{Gen}{40}{10}{Nu 13:23 De 1:24}
\crossref{Gen}{40}{11}{Ge 49:11 Le 10:9 Pr 3:10}
\crossref{Gen}{40}{12}{40:18; 41:12,\allowbreak25,\allowbreak26 Jud 7:14 Da 2:36-\allowbreak45; 4:19-\allowbreak33}
\crossref{Gen}{40}{13}{Ge 7:4}
\crossref{Gen}{40}{14}{1Sa 25:31 Lu 23:42 1Co 7:21}
\crossref{Gen}{40}{15}{Ge 37:28 Ex 21:16 De 24:7 1Ti 1:10}
\crossref{Gen}{40}{16}{40:1,\allowbreak2}
\crossref{Gen}{40}{17}{}
\crossref{Gen}{40}{18}{40:12; 41:26 1Co 10:4; 11:24}
\crossref{Gen}{40}{19}{40:13}
\crossref{Gen}{40}{20}{40:13,\allowbreak19}
\crossref{Gen}{40}{21}{40:13 Ne 2:1}
\crossref{Gen}{40}{22}{40:8,\allowbreak19; 41:11-\allowbreak13,\allowbreak16 Jer 23:28 Da 2:19-\allowbreak23,\allowbreak30; 5:12 Ac 5:30}
\crossref{Gen}{40}{23}{Job 19:14 Ps 31:12; 105:19 Ec 9:15,\allowbreak16 Am 6:6}
\crossref{Gen}{41}{1}{Ge 20:3; 37:5-\allowbreak10; 40:5 Jud 7:13,\allowbreak14 Es 6:1 Job 33:15,\allowbreak16 Da 2:1-\allowbreak3}
\crossref{Gen}{41}{2}{41:17-\allowbreak27}
\crossref{Gen}{41}{3}{41:4,\allowbreak20,\allowbreak21}
\crossref{Gen}{41}{4}{1Ki 3:15}
\crossref{Gen}{41}{5}{De 32:14}
\crossref{Gen}{41}{6}{Eze 17:10; 19:12 Ho 13:15}
\crossref{Gen}{41}{7}{Ge 20:3; 37:5}
\crossref{Gen}{41}{8}{Ge 40:6 Da 2:1-\allowbreak3; 4:5,\allowbreak19; 5:6; 7:28; 8:27 Hab 3:16}
\crossref{Gen}{41}{9}{Ge 40:1-\allowbreak3,\allowbreak14,\allowbreak23}
\crossref{Gen}{41}{10}{Ge 39:20; 40:2,\allowbreak3}
\crossref{Gen}{41}{11}{Ge 40:5-\allowbreak8}
\crossref{Gen}{41}{12}{Ge 37:36; 39:1,\allowbreak20}
\crossref{Gen}{41}{13}{Ge 40:12,\allowbreak20-\allowbreak22 Jer 1:10 Eze 43:3}
\crossref{Gen}{41}{14}{1Sa 2:7,\allowbreak8 Ps 105:19-\allowbreak22; 113:7,\allowbreak8}
\crossref{Gen}{41}{15}{41:9-\allowbreak13 Ps 25:14 Da 5:12,\allowbreak16}
\crossref{Gen}{41}{16}{Ge 40:8 Nu 12:6 2Ki 6:27 Da 2:18-\allowbreak23,\allowbreak28-\allowbreak30,\allowbreak47; 4:2 Ac 3:7,\allowbreak12}
\crossref{Gen}{41}{17}{41:1-\allowbreak7}
\crossref{Gen}{41}{18}{Jer 24:1-\allowbreak3,\allowbreak5,\allowbreak8}
\crossref{Gen}{41}{19}{Ru 3:10}
\crossref{Gen}{41}{20}{}
\crossref{Gen}{41}{21}{Eze 3:3 Re 10:9,\allowbreak10}
\crossref{Gen}{41}{22}{}
\crossref{Gen}{41}{23}{41:6 2Ki 19:26 Ps 129:6,\allowbreak7 Ho 8:7; 9:16; 13:15}
\crossref{Gen}{41}{24}{41:8 Ex 8:19 Da 4:7}
\crossref{Gen}{41}{25}{41:16 Ex 9:14 Jos 11:6 Ps 98:2 Isa 41:22,\allowbreak23; 43:9 Da 2:28,\allowbreak29}
\crossref{Gen}{41}{26}{41:2,\allowbreak5,\allowbreak29,\allowbreak47,\allowbreak53; 40:18 Ex 12:11 1Co 10:4}
\crossref{Gen}{41}{27}{2Sa 24:19 2Ki 8:1}
\crossref{Gen}{41}{28}{41:16,\allowbreak25}
\crossref{Gen}{41}{29}{41:26,\allowbreak46,\allowbreak49}
\crossref{Gen}{41}{30}{41:27,\allowbreak54 2Sa 24:13 1Ki 17:1 2Ki 8:1 Lu 4:25 Jas 5:17}
\crossref{Gen}{41}{31}{1Sa 5:6 Isa 24:20}
\crossref{Gen}{41}{32}{Ge 37:7,\allowbreak9 Job 33:14,\allowbreak15 2Co 13:1}
\crossref{Gen}{41}{33}{Da 4:27}
\crossref{Gen}{41}{34}{Nu 31:14 2Ki 11:11,\allowbreak12 2Ch 34:12 Ne 11:9}
\crossref{Gen}{41}{35}{41:48,\allowbreak49,\allowbreak56; 45:6,\allowbreak7}
\crossref{Gen}{41}{36}{Ge 47:13-\allowbreak25}
\crossref{Gen}{41}{37}{Ps 105:19 Pr 10:20; 25:11 Ac 7:10}
\crossref{Gen}{41}{38}{Nu 27:18 Job 32:8 Da 4:6,\allowbreak8,\allowbreak18; 5:11,\allowbreak14; 6:3}
\crossref{Gen}{41}{39}{41:16,\allowbreak25,\allowbreak28,\allowbreak33}
\crossref{Gen}{41}{40}{Ge 39:4-\allowbreak6; 45:8,\allowbreak9,\allowbreak26 Ps 105:21,\allowbreak22 Pr 22:29 Da 2:46-\allowbreak48; 5:29; 6:3}
\crossref{Gen}{41}{41}{41:44; 39:5,\allowbreak22 Es 10:3 Pr 17:2; 22:29 Da 2:7,\allowbreak8; 4:2,\allowbreak3; 6:3 Mt 28:18}
\crossref{Gen}{41}{42}{Es 3:10,\allowbreak12; 6:7-\allowbreak12; 8:2,\allowbreak8,\allowbreak10,\allowbreak15; 10:3 Da 2:46,\allowbreak47; 5:7,\allowbreak29 Lu 15:22}
\crossref{Gen}{41}{43}{Es 6:8,\allowbreak9}
\crossref{Gen}{41}{44}{Ex 11:7}
\crossref{Gen}{41}{45}{Ge 14:18 Ex 2:16}
\crossref{Gen}{41}{46}{Ge 37:2 Nu 4:3 2Sa 5:4 Lu 3:23}
\crossref{Gen}{41}{47}{}
\crossref{Gen}{41}{48}{41:34-\allowbreak36; 47:21}
\crossref{Gen}{41}{49}{Ge 22:17 Jud 6:5; 7:12 1Sa 13:5 Job 1:3 Ps 78:27 Jer 33:22}
\crossref{Gen}{41}{50}{Ge 46:20; 48:5}
\crossref{Gen}{41}{51}{Ge 48:5,\allowbreak13,\allowbreak14,\allowbreak18-\allowbreak20 De 33:17}
\crossref{Gen}{41}{52}{Ge 29:32-\allowbreak35; 30:6-\allowbreak13; 50:23}
\crossref{Gen}{41}{53}{41:29-\allowbreak31 Ps 73:20 Lu 16:25}
\crossref{Gen}{41}{54}{41:3,\allowbreak4,\allowbreak6,\allowbreak7,\allowbreak27; 45:11 Ps 105:16 Ac 7:11}
\crossref{Gen}{41}{55}{2Ki 6:25-\allowbreak29 Jer 14:1-\allowbreak6 La 4:3-\allowbreak10}
\crossref{Gen}{41}{56}{Isa 23:17 Zec 5:3 Lu 21:35 Ac 17:26}
\crossref{Gen}{41}{57}{Ge 42:1,\allowbreak5; 50:20 De 9:28 Ps 105:16,\allowbreak17}
\crossref{Gen}{42}{1}{Ge 41:54,\allowbreak57 Ac 7:12}
\crossref{Gen}{42}{2}{Ge 43:2,\allowbreak4; 45:9}
\crossref{Gen}{42}{3}{42:5,\allowbreak13}
\crossref{Gen}{42}{4}{Ge 35:16-\allowbreak19}
\crossref{Gen}{42}{5}{Ge 12:10; 26:1; 41:57 Ac 7:11; 11:28}
\crossref{Gen}{42}{6}{Ge 41:55,\allowbreak56}
\crossref{Gen}{42}{7}{42:9-\allowbreak12,\allowbreak14-\allowbreak17,\allowbreak19,\allowbreak20 Mt 15:23-\allowbreak26}
\crossref{Gen}{42}{8}{Lu 24:16 Joh 20:14; 21:4}
\crossref{Gen}{42}{9}{Ge 37:5-\allowbreak9}
\crossref{Gen}{42}{10}{Ge 27:29,\allowbreak37; 37:8; 44:9 1Sa 26:17 1Ki 18:7}
\crossref{Gen}{42}{11}{42:19,\allowbreak33,\allowbreak34 Joh 7:18 2Co 6:4}
\crossref{Gen}{42}{12}{42:9}
\crossref{Gen}{42}{13}{42:11,\allowbreak32; 29:32-\allowbreak35; 30:6-\allowbreak24; 35:16-\allowbreak26; 43:7; 46:8-\allowbreak27 Ex 1:2-\allowbreak5}
\crossref{Gen}{42}{14}{42:9-\allowbreak11 Job 13:24; 19:11 Mt 15:21-\allowbreak28}
\crossref{Gen}{42}{15}{42:7,\allowbreak12,\allowbreak16,\allowbreak30 De 6:13 1Sa 1:26; 17:55; 20:3 Jer 5:2,\allowbreak7 Mt 5:33-\allowbreak37}
\crossref{Gen}{42}{16}{42:19}
\crossref{Gen}{42}{17}{Isa 24:22 Ac 5:18}
\crossref{Gen}{42}{18}{Ge 20:11 Le 25:43 Ne 5:9,\allowbreak15 Lu 18:2,\allowbreak4}
\crossref{Gen}{42}{19}{Ge 40:3 Isa 42:7,\allowbreak22 Jer 37:15}
\crossref{Gen}{42}{20}{42:15,\allowbreak34; 43:5,\allowbreak19; 44:23}
\crossref{Gen}{42}{21}{Ge 41:9 Nu 32:23 2Sa 12:13 1Ki 17:18 Job 33:27,\allowbreak28; 34:31,\allowbreak32}
\crossref{Gen}{42}{22}{Ge 37:21,\allowbreak22,\allowbreak29,\allowbreak30 Lu 23:41 Ro 2:15}
\crossref{Gen}{42}{23}{}
\crossref{Gen}{42}{24}{Ge 43:30 Isa 63:9 Lu 19:41 Ro 12:15 1Co 12:26 Heb 4:15}
\crossref{Gen}{42}{25}{Ge 44:1,\allowbreak2 Isa 55:1}
\crossref{Gen}{42}{26}{Ge 12:16; 24:35; 30:43; 32:5,\allowbreak15; 34:28; 44:3,\allowbreak13; 45:17,\allowbreak23 1Sa 25:18}
\crossref{Gen}{42}{27}{Ge 43:21; 44:11 Ex 4:24 Lu 2:7; 10:34}
\crossref{Gen}{42}{28}{42:36; 27:33 Le 26:36 De 28:65 1Ki 10:5 Ps 61:2 So 5:6}
\crossref{Gen}{42}{29}{42:5,\allowbreak13; 37:1; 45:17}
\crossref{Gen}{42}{30}{42:7-\allowbreak20}
\crossref{Gen}{42}{31}{42:11}
\crossref{Gen}{42}{32}{42:13}
\crossref{Gen}{42}{33}{42:15,\allowbreak19,\allowbreak20}
\crossref{Gen}{42}{34}{Ge 34:10,\allowbreak21 1Ki 10:15 Eze 17:4}
\crossref{Gen}{42}{35}{42:27,\allowbreak28; 43:21}
\crossref{Gen}{42}{36}{Ge 37:20-\allowbreak35; 43:14}
\crossref{Gen}{42}{37}{Ge 43:9; 44:32-\allowbreak34; 46:9 Mic 6:7}
\crossref{Gen}{42}{38}{42:13; 30:22-\allowbreak24; 35:16-\allowbreak18; 37:33,\allowbreak35; 44:20,\allowbreak27-\allowbreak34}
\crossref{Gen}{43}{1}{Ge 18:13; 41:54-\allowbreak57; 42:5 Ec 9:1,\allowbreak2 La 5:10 Ac 7:11-\allowbreak13}
\crossref{Gen}{43}{2}{43:4,\allowbreak20; 42:1,\allowbreak2 Pr 15:16; 16:18; 31:16 1Ti 5:8; 6:6-\allowbreak8}
\crossref{Gen}{43}{3}{Ge 42:15-\allowbreak20,\allowbreak33,\allowbreak34; 44:23}
\crossref{Gen}{43}{4}{}
\crossref{Gen}{43}{5}{Ge 42:38; 44:26 Ex 20:12}
\crossref{Gen}{43}{6}{Ge 42:38}
\crossref{Gen}{43}{7}{43:3}
\crossref{Gen}{43}{8}{Ge 42:38; 44:26 Ex 20:12}
\crossref{Gen}{43}{9}{Ge 42:37; 44:32,\allowbreak33 1Ki 1:21 Job 17:3 Ps 119:122 Phm 1:18,\allowbreak19}
\crossref{Gen}{43}{10}{Ge 19:16}
\crossref{Gen}{43}{11}{43:14 Es 4:16 Ac 21:14}
\crossref{Gen}{43}{12}{Ro 12:17; 13:8 2Co 8:21 Php 4:8 1Th 4:6; 5:21 Heb 13:8}
\crossref{Gen}{43}{13}{Ge 42:38}
\crossref{Gen}{43}{14}{Ge 17:1; 22:14; 32:11-\allowbreak28; 39:21 Ezr 7:27 Ne 1:11 Es 4:16 Ps 37:5-\allowbreak7}
\crossref{Gen}{43}{15}{43:12 Ex 22:4,\allowbreak7 Pr 6:31}
\crossref{Gen}{43}{16}{43:19; 15:2; 24:2-\allowbreak10; 39:4,\allowbreak5; 44:1}
\crossref{Gen}{43}{17}{}
\crossref{Gen}{43}{18}{Ge 42:21,\allowbreak28,\allowbreak35 Jud 13:22 Job 15:21 Ps 53:5; 73:16 Isa 7:2}
\crossref{Gen}{43}{19}{43:16,\allowbreak24 2Sa 19:17 Isa 22:15}
\crossref{Gen}{43}{20}{43:3,\allowbreak7; 42:3,\allowbreak10,\allowbreak27,\allowbreak35}
\crossref{Gen}{43}{21}{Ge 42:27-\allowbreak35}
\crossref{Gen}{43}{22}{Ro 10:3}
\crossref{Gen}{43}{23}{Jud 6:23; 19:20 1Sa 25:6 1Ch 12:18 Ezr 4:17 Lu 10:5; 24:36}
\crossref{Gen}{43}{24}{Ge 18:4; 19:2; 24:32 Lu 7:44 Joh 13:4-\allowbreak17}
\crossref{Gen}{43}{25}{43:11,\allowbreak16}
\crossref{Gen}{43}{26}{43:28; 27:29; 37:7-\allowbreak10,\allowbreak19,\allowbreak20; 42:6 Ps 72:9 Ro 14:11 Php 2:10,\allowbreak11}
\crossref{Gen}{43}{27}{Ge 37:14; 41:16 Ex 18:7 Jud 18:15 1Sa 17:22; 25:5 1Ch 18:10}
\crossref{Gen}{43}{28}{43:26; 37:7,\allowbreak9,\allowbreak10}
\crossref{Gen}{43}{29}{Ge 30:22-\allowbreak24; 35:17,\allowbreak18}
\crossref{Gen}{43}{30}{1Ki 3:26 Jer 31:20 Ho 11:8 Php 1:8; 2:1 Col 3:12 1Jo 3:17}
\crossref{Gen}{43}{31}{Ge 45:1 Isa 42:14 Jer 31:16 1Pe 3:10}
\crossref{Gen}{43}{32}{43:16; 31:54}
\crossref{Gen}{43}{33}{}
\crossref{Gen}{43}{34}{2Sa 11:8}
\crossref{Gen}{44}{1}{Ge 24:2; 43:16,\allowbreak19}
\crossref{Gen}{44}{2}{Ge 42:15,\allowbreak16,\allowbreak20; 43:32 De 8:2,\allowbreak16; 13:3 Mt 10:16 2Co 8:8}
\crossref{Gen}{44}{3}{}
\crossref{Gen}{44}{4}{De 2:16}
\crossref{Gen}{44}{5}{44:15}
\crossref{Gen}{44}{6}{44:6}
\crossref{Gen}{44}{7}{Ge 34:25-\allowbreak31; 35:22; 37:18-\allowbreak32; 38:16-\allowbreak18 Jos 22:22-\allowbreak29 2Sa 20:20}
\crossref{Gen}{44}{8}{Ge 42:21,\allowbreak27,\allowbreak35; 43:12,\allowbreak21,\allowbreak22}
\crossref{Gen}{44}{9}{Ge 31:32 Job 31:38-\allowbreak40 Ps 7:3-\allowbreak5 Ac 25:11}
\crossref{Gen}{44}{10}{44:17,\allowbreak33 Ex 22:3 Mt 18:24,\allowbreak25}
\crossref{Gen}{44}{11}{}
\crossref{Gen}{44}{12}{Ge 43:33}
\crossref{Gen}{44}{13}{Ge 37:29-\allowbreak34 Nu 14:6 2Sa 1:2,\allowbreak11; 13:19}
\crossref{Gen}{44}{14}{Ge 43:16,\allowbreak25}
\crossref{Gen}{44}{15}{44:4,\allowbreak5; 3:13; 4:10}
\crossref{Gen}{44}{16}{44:32; 43:8,\allowbreak9}
\crossref{Gen}{44}{17}{Ge 18:25; 42:18 2Sa 23:3 Ps 75:2 Pr 17:15}
\crossref{Gen}{44}{18}{Ge 18:30,\allowbreak32 2Sa 14:12 Job 33:31 Ac 2:29}
\crossref{Gen}{44}{19}{Ge 42:7-\allowbreak10; 43:7,\allowbreak29}
\crossref{Gen}{44}{20}{Ge 35:18; 37:3,\allowbreak19; 43:7,\allowbreak8; 46:21}
\crossref{Gen}{44}{21}{Ge 42:15,\allowbreak20; 43:29}
\crossref{Gen}{44}{22}{44:30; 42:38}
\crossref{Gen}{44}{23}{Ge 42:15-\allowbreak20; 43:3,\allowbreak5}
\crossref{Gen}{44}{24}{Ge 42:29-\allowbreak34}
\crossref{Gen}{44}{25}{Ge 43:2,\allowbreak5}
\crossref{Gen}{44}{26}{Ge 43:4,\allowbreak5 Lu 11:7}
\crossref{Gen}{44}{27}{Ge 29:18-\allowbreak21,\allowbreak28; 30:22-\allowbreak25; 35:16-\allowbreak18; 46:19}
\crossref{Gen}{44}{28}{Ge 37:13,\allowbreak14}
\crossref{Gen}{44}{29}{Ge 42:36,\allowbreak38; 43:14 Ps 88:3,\allowbreak4}
\crossref{Gen}{44}{30}{44:17,\allowbreak31,\allowbreak34}
\crossref{Gen}{44}{31}{1Sa 4:17,\allowbreak18 2Co 7:10 1Th 4:13}
\crossref{Gen}{44}{32}{Ge 43:8,\allowbreak9,\allowbreak16}
\crossref{Gen}{44}{33}{Ex 32:32 Ro 5:7-\allowbreak10; 9:3}
\crossref{Gen}{44}{34}{1Sa 2:33,\allowbreak34 2Ch 34:28 Es 8:6 Jer 52:10,\allowbreak11}
\crossref{Gen}{45}{1}{Ge 43:30,\allowbreak31 Isa 42:14 Jer 20:9}
\crossref{Gen}{45}{2}{}
\crossref{Gen}{45}{3}{Mt 14:27 Ac 7:13; 9:5}
\crossref{Gen}{45}{4}{Ge 37:28; 50:18 Mt 14:27 Ac 9:5}
\crossref{Gen}{45}{5}{Isa 40:1,\allowbreak2 Lu 23:34 2Co 2:7,\allowbreak11}
\crossref{Gen}{45}{6}{Ge 41:29-\allowbreak31,\allowbreak54,\allowbreak56; 47:18}
\crossref{Gen}{45}{7}{}
\crossref{Gen}{45}{8}{45:5 Joh 15:16; 19:11 Ro 9:16}
\crossref{Gen}{45}{9}{45:26-\allowbreak28}
\crossref{Gen}{45}{10}{Ge 46:29,\allowbreak34; 47:1-\allowbreak6 Ex 8:22; 9:26}
\crossref{Gen}{45}{11}{Ge 47:6,\allowbreak12 Mt 15:5,\allowbreak6 Mr 7:9-\allowbreak12 1Ti 5:4}
\crossref{Gen}{45}{12}{Ge 42:23 Lu 24:39 Joh 20:27}
\crossref{Gen}{45}{13}{Joh 17:24 1Pe 1:10-\allowbreak12 Re 21:23}
\crossref{Gen}{45}{14}{Ge 29:11; 33:4; 46:29 Ro 1:31}
\crossref{Gen}{45}{15}{45:2; 29:11,\allowbreak13; 33:4 Ex 4:27 Ru 1:9,\allowbreak14 1Sa 10:1; 20:41 2Sa 14:33}
\crossref{Gen}{45}{16}{}
\crossref{Gen}{45}{17}{Ge 42:25,\allowbreak26; 44:1,\allowbreak2}
\crossref{Gen}{45}{18}{Ge 27:28; 47:6 Nu 18:12,\allowbreak29 De 32:14 Ps 81:16; 147:14 Isa 28:1,\allowbreak4}
\crossref{Gen}{45}{19}{Isa 49:1,\allowbreak23}
\crossref{Gen}{45}{20}{De 7:16; 19:13,\allowbreak21 Isa 13:18 Eze 7:4,\allowbreak9; 9:5; 20:17}
\crossref{Gen}{45}{21}{45:19,\allowbreak27; 46:5 Nu 7:3-\allowbreak9 Eze 23:24}
\crossref{Gen}{45}{22}{Jud 14:12,\allowbreak19 2Ki 5:5,\allowbreak22,\allowbreak23 Re 6:11}
\crossref{Gen}{45}{23}{45:17}
\crossref{Gen}{45}{24}{Ge 37:22; 42:21,\allowbreak22 Ps 133:1-\allowbreak3 Joh 13:34,\allowbreak35 Eph 4:31,\allowbreak32 Php 2:2-\allowbreak5}
\crossref{Gen}{45}{25}{}
\crossref{Gen}{45}{26}{Lu 24:34}
\crossref{Gen}{45}{27}{Jud 15:19 1Sa 30:12 Ps 85:6 Isa 57:15 Ho 6:2}
\crossref{Gen}{45}{28}{Ge 46:30 Lu 2:28-\allowbreak30 Joh 16:21,\allowbreak22}
\crossref{Gen}{46}{1}{Ge 21:14,\allowbreak31,\allowbreak33; 26:22,\allowbreak23; 28:10 1Sa 3:20}
\crossref{Gen}{46}{2}{Ge 15:1,\allowbreak13; 22:11 Nu 12:6; 24:4 2Ch 26:5 Job 4:13; 33:14,\allowbreak15 Da 2:19}
\crossref{Gen}{46}{3}{Ge 28:13}
\crossref{Gen}{46}{4}{Ge 28:15; 48:21 Isa 43:1,\allowbreak2}
\crossref{Gen}{46}{5}{Ac 7:15}
\crossref{Gen}{46}{6}{Ge 15:13 Nu 20:15 De 10:22; 26:5 Jos 24:4 1Sa 12:8 Ps 105:23}
\crossref{Gen}{46}{7}{Ge 37:35 Jos 24:4 Ps 105:23 Isa 52:4}
\crossref{Gen}{46}{8}{Ge 29:1-\allowbreak30:43; 35:23; 49:1-\allowbreak33 Ex 1:1-\allowbreak5; 6:14-\allowbreak18 1Ch 2:1-\allowbreak55; 8:1-\allowbreak40}
\crossref{Gen}{46}{9}{Ex 6:14}
\crossref{Gen}{46}{10}{Ge 29:33; 34:25,\allowbreak30; 49:5-\allowbreak7 Ex 6:15 Nu 1:6,\allowbreak22,\allowbreak23; 2:12,\allowbreak13; 26:12,\allowbreak13}
\crossref{Gen}{46}{11}{Ge 29:34; 49:5-\allowbreak7 Ex 6:16 Nu 3:17-\allowbreak22; 4:1-\allowbreak49; 8:1-\allowbreak26; 26:57,\allowbreak58}
\crossref{Gen}{46}{12}{Ge 29:35; 38:1-\allowbreak3,\allowbreak7,\allowbreak10,\allowbreak24-\allowbreak30; 49:8-\allowbreak12 Nu 1:7,\allowbreak26,\allowbreak27; 26:19-\allowbreak21 De 33:7}
\crossref{Gen}{46}{13}{Ge 30:14-\allowbreak18; 35:23; 49:14,\allowbreak15 Nu 1:8,\allowbreak28-\allowbreak30; 26:23-\allowbreak25 De 33:18}
\crossref{Gen}{46}{14}{Ge 30:19,\allowbreak20; 49:13 Nu 1:9,\allowbreak30,\allowbreak31; 26:26,\allowbreak27 De 33:18,\allowbreak19 1Ch 2:1}
\crossref{Gen}{46}{15}{Ge 29:32-\allowbreak35; 30:17-\allowbreak21; 35:23; 49:3-\allowbreak15 Ex 1:2,\allowbreak3 Nu 1:1-\allowbreak54; 10:1-\allowbreak36}
\crossref{Gen}{46}{16}{Ge 30:11; 35:26; 49:19 Nu 1:11,\allowbreak24,\allowbreak25; 26:15-\allowbreak17 De 33:20,\allowbreak21 1Ch 2:2}
\crossref{Gen}{46}{17}{Ge 30:13; 35:26; 49:20 Nu 1:13,\allowbreak40,\allowbreak41; 26:44-\allowbreak46 De 33:24 1Ch 2:2}
\crossref{Gen}{46}{18}{Ge 29:24; 30:9-\allowbreak13; 35:26 Ex 1:4}
\crossref{Gen}{46}{19}{Ge 29:18; 30:24; 35:16-\allowbreak18,\allowbreak24; 44:27 Ex 1:3,\allowbreak5 1Ch 2:2}
\crossref{Gen}{46}{20}{Ge 41:50-\allowbreak52; 48:4,\allowbreak5,\allowbreak13,\allowbreak14,\allowbreak20 Nu 1:32-\allowbreak35; 26:28-\allowbreak37 De 33:13-\allowbreak17}
\crossref{Gen}{46}{21}{Ge 49:27 Nu 1:11,\allowbreak36,\allowbreak37 De 33:12 1Ch 7:6-\allowbreak12; 8:1-\allowbreak7}
\crossref{Gen}{46}{22}{46:15,\allowbreak18,\allowbreak25,\allowbreak26,\allowbreak27; 12:5}
\crossref{Gen}{46}{23}{Ge 30:6; 35:25; 49:16,\allowbreak17 Nu 1:12,\allowbreak38,\allowbreak39; 10:25 De 33:22 1Ch 2:2; 7:12}
\crossref{Gen}{46}{24}{Ge 30:7,\allowbreak8; 35:25; 49:21 Nu 1:15,\allowbreak42,\allowbreak43; 26:48-\allowbreak50 De 33:23 2Ki 15:29}
\crossref{Gen}{46}{25}{Ge 29:29; 30:3-\allowbreak8; 35:22,\allowbreak25 Ex 1:2}
\crossref{Gen}{46}{26}{Ge 35:11 Ex 1:5 Jud 8:30}
\crossref{Gen}{46}{27}{}
\crossref{Gen}{46}{28}{Ge 43:8; 44:16-\allowbreak34; 49:8}
\crossref{Gen}{46}{29}{Ge 41:43; 45:19,\allowbreak21}
\crossref{Gen}{46}{30}{Ge 45:28 Lu 2:29,\allowbreak30}
\crossref{Gen}{46}{31}{Ge 45:16-\allowbreak20; 47:1-\allowbreak3 Ac 18:3 Heb 2:11}
\crossref{Gen}{46}{32}{Ge 4:2; 31:18; 37:2; 47:3 Ex 3:1 1Sa 16:11; 17:15 Ps 78:70-\allowbreak72}
\crossref{Gen}{46}{33}{46:32; 47:2-\allowbreak4 Jon 1:8}
\crossref{Gen}{46}{34}{46:32; 30:35; 34:5; 37:12}
\crossref{Gen}{47}{1}{Ge 45:16; 46:31 Heb 2:11}
\crossref{Gen}{47}{2}{Ac 7:13 2Co 4:14 Col 1:28 Jude 1:24}
\crossref{Gen}{47}{3}{Ge 46:33,\allowbreak34 Am 7:14,\allowbreak15 Jon 1:8 2Th 3:10}
\crossref{Gen}{47}{4}{Ge 12:10; 15:13 De 26:5 Ps 105:23 Isa 52:4 Ac 7:6}
\crossref{Gen}{47}{5}{}
\crossref{Gen}{47}{6}{47:11; 13:9; 20:15; 34:10; 45:18-\allowbreak20 Pr 21:1 Joh 17:2}
\crossref{Gen}{47}{7}{47:10; 35:27 Ex 12:32 Nu 6:23,\allowbreak24 Jos 14:13 1Sa 2:20 2Sa 8:10}
\crossref{Gen}{47}{8}{}
\crossref{Gen}{47}{9}{1Ch 29:15 Ps 39:12; 119:19,\allowbreak54 2Co 5:6 Heb 11:9-\allowbreak16; 13:14}
\crossref{Gen}{47}{10}{47:7; 14:19 Nu 6:23-\allowbreak27 De 33:1 Ru 2:4 2Sa 8:10; 19:39 Ps 119:46}
\crossref{Gen}{47}{11}{47:6 Ex 1:11; 12:37 Joh 10:10,\allowbreak28; 14:2,\allowbreak23; 17:2,\allowbreak24}
\crossref{Gen}{47}{12}{Ru 4:15}
\crossref{Gen}{47}{13}{Ge 41:30,\allowbreak31 1Ki 18:5 Jer 14:1-\allowbreak6 La 2:19,\allowbreak20; 4:9 Ac 7:11}
\crossref{Gen}{47}{14}{Ge 41:56}
\crossref{Gen}{47}{15}{47:18,\allowbreak19,\allowbreak24 Jud 8:5,\allowbreak8 1Sa 21:3; 25:8 Ps 37:3 Isa 33:16 Mt 6:11}
\crossref{Gen}{47}{16}{}
\crossref{Gen}{47}{17}{Ex 9:3 1Ki 10:28 Job 2:4 Isa 31:1 Mt 6:24}
\crossref{Gen}{47}{18}{2Ki 6:26 Jer 38:9}
\crossref{Gen}{47}{19}{Ne 5:2,\allowbreak3 Job 2:4 La 1:11; 5:6,\allowbreak9 Mt 16:26 Php 3:8,\allowbreak9}
\crossref{Gen}{47}{20}{}
\crossref{Gen}{47}{21}{Ge 41:48}
\crossref{Gen}{47}{22}{Ge 14:18; 41:45,\allowbreak50}
\crossref{Gen}{47}{23}{47:19}
\crossref{Gen}{47}{24}{47:25; 41:34 Le 27:32 1Sa 8:15-\allowbreak17 Ps 41:1; 112:5}
\crossref{Gen}{47}{25}{Ge 6:19; 41:45}
\crossref{Gen}{47}{26}{47:22 Eze 7:24}
\crossref{Gen}{47}{27}{47:11}
\crossref{Gen}{47}{28}{Ge 37:2}
\crossref{Gen}{47}{29}{47:9; 3:19; 50:24 De 31:14 2Sa 7:12; 14:14 1Ki 2:1 Job 7:1}
\crossref{Gen}{47}{30}{Ge 23:19; 25:9; 49:29-\allowbreak32; 50:5-\allowbreak14,\allowbreak25 2Sa 19:37 1Ki 13:22 Ne 2:3,\allowbreak5}
\crossref{Gen}{47}{31}{Ge 24:3}
\crossref{Gen}{48}{1}{Joh 11:3}
\crossref{Gen}{48}{2}{De 3:28 1Sa 23:16 Ne 2:18 Ps 41:3 Pr 23:15 Eph 6:10}
\crossref{Gen}{48}{3}{Ge 17:1; 28:3; 35:11 Ex 6:3 Re 21:11}
\crossref{Gen}{48}{4}{Ge 12:2; 13:15,\allowbreak16; 22:17; 26:4; 28:3,\allowbreak13-\allowbreak15; 32:12; 35:11; 46:3; 47:27}
\crossref{Gen}{48}{5}{Ge 41:50-\allowbreak52; 46:20 Jos 13:7; 14:4; 16:1-\allowbreak17:18}
\crossref{Gen}{48}{6}{Jos 14:4}
\crossref{Gen}{48}{7}{Ge 25:20}
\crossref{Gen}{48}{8}{}
\crossref{Gen}{48}{9}{Ge 30:2; 33:5 Ru 4:11-\allowbreak14 1Sa 1:20,\allowbreak27; 2:20,\allowbreak21 1Ch 25:5; 26:4,\allowbreak5}
\crossref{Gen}{48}{10}{Ge 27:1 1Sa 3:2; 4:15}
\crossref{Gen}{48}{11}{Ge 37:33,\allowbreak35; 42:36; 45:26}
\crossref{Gen}{48}{12}{}
\crossref{Gen}{48}{13}{Ge 41:52 Jud 3:15; 20:16 Ec 10:2 Mt 25:41}
\crossref{Gen}{48}{14}{Ex 15:6 Ps 110:1; 118:16}
\crossref{Gen}{48}{15}{48:16; 27:4; 28:3; 49:28 De 33:1 Heb 11:21}
\crossref{Gen}{48}{16}{Ge 16:7-\allowbreak13; 28:15; 31:11-\allowbreak13 Ex 3:2-\allowbreak6; 23:20,\allowbreak21 Jud 2:1-\allowbreak4; 6:21-\allowbreak24}
\crossref{Gen}{48}{17}{48:14}
\crossref{Gen}{48}{18}{Ge 19:18 Ex 10:11 Mt 25:9 Ac 10:14; 11:8}
\crossref{Gen}{48}{19}{48:14; 17:20,\allowbreak21; 25:28 Nu 1:33-\allowbreak35; 2:19-\allowbreak21 De 33:17 Isa 7:17}
\crossref{Gen}{48}{20}{Ge 24:60; 28:3 Ru 4:11,\allowbreak12}
\crossref{Gen}{48}{21}{Ge 50:24 1Ki 2:2-\allowbreak4 Ps 146:3,\allowbreak4 Zec 1:5,\allowbreak6 Lu 2:29 Ac 13:36 2Ti 4:6}
\crossref{Gen}{48}{22}{Ge 33:19 De 21:17 Jos 24:32 1Ch 5:2 Eze 47:13 Joh 4:5}
\crossref{Gen}{49}{1}{De 31:12,\allowbreak28,\allowbreak29; 33:1-\allowbreak29 Ps 25:14; 105:15 Isa 22:14; 53:1 Da 2:47}
\crossref{Gen}{49}{2}{Ps 34:11 Pr 1:8,\allowbreak9; 4:1-\allowbreak4; 5:1; 6:20; 7:1,\allowbreak24; 8:32; 23:22,\allowbreak26}
\crossref{Gen}{49}{3}{Ge 29:32; 46:8; 48:18 Nu 1:20; 26:5 1Ch 2:1; 5:1,\allowbreak3}
\crossref{Gen}{49}{4}{Jas 1:6-\allowbreak8 2Pe 2:14; 3:16}
\crossref{Gen}{49}{5}{Ge 29:33,\allowbreak34; 34:25-\allowbreak31; 46:10,\allowbreak11 Pr 18:9}
\crossref{Gen}{49}{6}{Jud 5:21 Ps 42:5,\allowbreak11; 43:5; 103:1 Jer 4:19 Lu 12:19}
\crossref{Gen}{49}{7}{2Sa 13:15,\allowbreak22-\allowbreak28 Pr 26:24,\allowbreak25; 27:3}
\crossref{Gen}{49}{8}{Ge 29:35; 44:18-\allowbreak34; 46:12 De 33:7 1Ch 5:2 Ps 76:1 Heb 7:14}
\crossref{Gen}{49}{9}{Ho 5:4,\allowbreak14 1Co 15:24 Re 5:5}
\crossref{Gen}{49}{10}{Nu 24:17 Ps 60:7 Jer 30:21 Ho 11:12 Eze 19:11,\allowbreak14 Zec 10:11}
\crossref{Gen}{49}{11}{Isa 63:1-\allowbreak3}
\crossref{Gen}{49}{12}{Pr 23:29}
\crossref{Gen}{49}{13}{Ge 30:20 De 33:18,\allowbreak19 Jos 19:10-\allowbreak16}
\crossref{Gen}{49}{14}{Ge 30:18 De 33:18 Jos 19:17-\allowbreak23 Jud 5:15; 10:1 1Ch 12:32}
\crossref{Gen}{49}{15}{Jos 14:15 Jud 3:11 2Sa 7:1}
\crossref{Gen}{49}{16}{Ge 30:6 Nu 10:25 De 33:22 Jud 13:2,\allowbreak24,\allowbreak25; 15:20; 18:1,\allowbreak2}
\crossref{Gen}{49}{17}{Jud 14:1-\allowbreak15:20; 16:22-\allowbreak30; 18:22-\allowbreak31 1Ch 12:35}
\crossref{Gen}{49}{18}{Ps 14:7; 25:6; 40:1; 62:1,\allowbreak5; 85:7; 119:41,\allowbreak166,\allowbreak174; 123:2; 130:5}
\crossref{Gen}{49}{19}{Ge 30:11; 46:16 Nu 32:1-\allowbreak42 De 33:20,\allowbreak21 Jos 13:8 Jud 10:1-\allowbreak11:40}
\crossref{Gen}{49}{20}{Ge 30:13; 46:17 De 33:24,\allowbreak25 Jos 19:24-\allowbreak31}
\crossref{Gen}{49}{21}{Ge 30:8; 46:24 De 33:23 Jos 19:32-\allowbreak39 Jud 4:6,\allowbreak10; 5:18 Ps 18:33,\allowbreak34}
\crossref{Gen}{49}{22}{Ge 30:22-\allowbreak24; 41:52; 46:27; 48:1,\allowbreak5,\allowbreak16,\allowbreak19,\allowbreak20 Nu 32:1-\allowbreak42 De 33:17}
\crossref{Gen}{49}{23}{Ge 37:4,\allowbreak18,\allowbreak24,\allowbreak28; 39:7-\allowbreak20; 42:21 Ps 64:3; 118:13 Joh 16:33 Ac 14:22}
\crossref{Gen}{49}{24}{Ne 6:9 Ps 27:14; 28:8; 89:1 Col 1:11 2Ti 4:17}
\crossref{Gen}{49}{25}{Ge 28:13,\allowbreak21; 35:3; 43:23 De 8:17; 28:12; 33:1,\allowbreak13-\allowbreak17}
\crossref{Gen}{49}{26}{Ge 27:27-\allowbreak29,\allowbreak39,\allowbreak40; 28:3,\allowbreak4 Eph 1:3}
\crossref{Gen}{49}{27}{Ge 35:18; 46:21 De 33:12}
\crossref{Gen}{49}{28}{Nu 23:24 Es 8:7,\allowbreak9,\allowbreak11; 9:1-\allowbreak10:3 Eze 39:8-\allowbreak10 Zec 14:1-\allowbreak7}
\crossref{Gen}{49}{29}{Ro 12:6-\allowbreak21}
\crossref{Gen}{49}{30}{Ge 23:8}
\crossref{Gen}{49}{31}{Ge 23:3,\allowbreak16-\allowbreak20; 25:9; 35:29; 47:30; 50:13 Ac 7:16}
\crossref{Gen}{49}{32}{Ge 23:17-\allowbreak20}
\crossref{Gen}{49}{33}{49:1,\allowbreak24-\allowbreak26 Jos 24:27-\allowbreak29 Heb 11:22}
\crossref{Gen}{50}{1}{Ge 46:4 De 6:7,\allowbreak8 Eph 6:4}
\crossref{Gen}{50}{2}{50:26 2Ch 16:14 Mt 26:12 Mr 14:8; 16:1 Lu 24:1 Joh 12:7; 19:39,\allowbreak40}
\crossref{Gen}{50}{3}{Nu 20:29 De 21:13; 34:8}
\crossref{Gen}{50}{4}{50:10}
\crossref{Gen}{50}{5}{Ge 47:29-\allowbreak31}
\crossref{Gen}{50}{6}{Ge 48:21}
\crossref{Gen}{50}{7}{Ge 14:16}
\crossref{Gen}{50}{8}{Ex 10:8,\allowbreak9,\allowbreak26 Nu 32:24-\allowbreak27}
\crossref{Gen}{50}{9}{Ge 41:43; 46:29 Ex 14:7,\allowbreak17,\allowbreak28 2Ki 18:24 So 1:9 Ac 8:2}
\crossref{Gen}{50}{10}{50:11 De 1:1}
\crossref{Gen}{50}{11}{Ge 10:15-\allowbreak19; 13:7; 24:6; 34:30}
\crossref{Gen}{50}{12}{Ge 47:29-\allowbreak31; 49:29-\allowbreak32 Ex 20:12 Ac 7:16 Eph 6:1}
\crossref{Gen}{50}{13}{Ge 23:16-\allowbreak18; 25:9; 35:27,\allowbreak29; 49:29-\allowbreak31 2Ki 21:18}
\crossref{Gen}{50}{14}{}
\crossref{Gen}{50}{15}{Ge 27:41,\allowbreak42}
\crossref{Gen}{50}{16}{Pr 29:25}
\crossref{Gen}{50}{17}{Mt 6:12,\allowbreak14,\allowbreak15; 18:35 Lu 17:3,\allowbreak4 Eph 4:32 Col 3:12,\allowbreak13}
\crossref{Gen}{50}{18}{Ge 27:29; 37:7-\allowbreak11; 42:6; 44:14; 45:3}
\crossref{Gen}{50}{19}{Ge 45:5 Mt 14:27 Lu 24:37,\allowbreak38}
\crossref{Gen}{50}{20}{Ge 37:4,\allowbreak18-\allowbreak20 Ps 56:5}
\crossref{Gen}{50}{21}{Ge 45:10,\allowbreak11; 47:12 Mt 5:44; 6:14 Ro 12:20,\allowbreak21 1Th 5:15 1Pe 3:9}
\crossref{Gen}{50}{22}{}
\crossref{Gen}{50}{23}{Ge 48:19; 49:12 Nu 32:33,\allowbreak39 Jos 17:1 Job 42:16 Ps 128:6}
\crossref{Gen}{50}{24}{50:5; 3:19 Job 30:23 Ec 12:5,\allowbreak7 Ro 5:12 Heb 9:27}
\crossref{Gen}{50}{25}{50:5; 47:29-\allowbreak31}
\crossref{Gen}{50}{26}{50:2,\allowbreak3}

% Exod
\crossref{Exod}{1}{1}{Ex 6:14-\allowbreak16 Ge 29:31-\allowbreak35; 30:1-\allowbreak21; 35:18,\allowbreak23-\allowbreak26; 46:8-\allowbreak26; 49:3-\allowbreak27}
\crossref{Exod}{1}{2}{Ge 35:22}
\crossref{Exod}{1}{3}{Ge 35:23}
\crossref{Exod}{1}{4}{Ge 35:25 1Ch 2:2}
\crossref{Exod}{1}{5}{Ge 46:26 Jud 8:30}
\crossref{Exod}{1}{6}{Ge 50:24,\allowbreak26 Ac 7:14-\allowbreak16}
\crossref{Exod}{1}{7}{Ex 12:37 Ge 1:20,\allowbreak28; 9:1; 12:2; 13:16; 15:5; 17:4-\allowbreak6,\allowbreak16; 22:17; 26:4}
\crossref{Exod}{1}{8}{}
\crossref{Exod}{1}{9}{Nu 22:4,\allowbreak5 Job 5:2 Ps 105:24,\allowbreak25 Pr 14:28; 27:4 Ec 4:4 Tit 3:3}
\crossref{Exod}{1}{10}{Ps 10:2; 83:3,\allowbreak4 Pr 1:11}
\crossref{Exod}{1}{11}{Ex 3:7; 5:15 Ge 15:13 Nu 20:15 De 26:6}
\crossref{Exod}{1}{12}{1:9 Job 5:2 Pr 27:4 Joh 12:19 Ac 4:2-\allowbreak4; 5:28-\allowbreak33}
\crossref{Exod}{1}{13}{}
\crossref{Exod}{1}{14}{Ex 2:23; 6:9 Ge 15:13 Nu 20:15 De 4:20; 26:6 Ru 1:20 Ac 7:19,\allowbreak34}
\crossref{Exod}{1}{15}{Ge 35:17; 38:28}
\crossref{Exod}{1}{16}{1:22 Mt 21:38 Re 12:4}
\crossref{Exod}{1}{17}{Ge 20:11; 42:18 Ne 5:15 Ps 31:19 Pr 8:13; 16:6; 24:11,\allowbreak12 Ec 8:12}
\crossref{Exod}{1}{18}{2Sa 13:28 Ec 8:4}
\crossref{Exod}{1}{19}{Jos 2:4-\allowbreak24 1Sa 21:2 2Sa 17:19,\allowbreak20}
\crossref{Exod}{1}{20}{Ps 41:1,\allowbreak2; 61:5; 85:9; 103:11; 111:5; 145:19 Pr 11:18; 19:17}
\crossref{Exod}{1}{21}{1Sa 2:35; 25:28 2Sa 7:11-\allowbreak13,\allowbreak27-\allowbreak29 1Ki 2:24; 11:38 Ps 37:3}
\crossref{Exod}{1}{22}{1:16; 7:19-\allowbreak21 Ps 105:25 Pr 1:16; 4:16; 27:4 Ac 7:19 Re 16:4-\allowbreak6}
\crossref{Exod}{2}{1}{Ex 6:16-\allowbreak20 Nu 26:59 1Ch 6:1-\allowbreak3; 23:12-\allowbreak14}
\crossref{Exod}{2}{2}{Ps 112:5 Ac 7:20 Heb 11:23}
\crossref{Exod}{2}{3}{Ex 1:22 Mt 2:13,\allowbreak16 Ac 7:19}
\crossref{Exod}{2}{4}{Ex 15:20 Nu 12:1-\allowbreak15; 20:1; 26:59 Mic 6:4}
\crossref{Exod}{2}{5}{Ac 7:21}
\crossref{Exod}{2}{6}{1Ki 8:50 Ne 1:11 Ps 106:46 Pr 21:1 Ac 7:21 1Pe 3:8}
\crossref{Exod}{2}{7}{2:4; 15:20 Nu 12:1; 26:59}
\crossref{Exod}{2}{8}{Ps 27:10 Isa 46:3,\allowbreak4 Eze 16:8}
\crossref{Exod}{2}{9}{Jud 13:8}
\crossref{Exod}{2}{10}{Ge 48:5 Ac 7:21,\allowbreak22 Ga 4:5 Heb 11:24 1Jo 3:1}
\crossref{Exod}{2}{11}{Ac 7:22-\allowbreak24 Heb 11:24-\allowbreak26}
\crossref{Exod}{2}{12}{Ac 7:24-\allowbreak26}
\crossref{Exod}{2}{13}{Ac 7:26 1Co 6:7,\allowbreak8}
\crossref{Exod}{2}{14}{Ge 19:9; 37:8-\allowbreak11,\allowbreak19,\allowbreak20 Nu 16:3,\allowbreak13 Ps 2:2-\allowbreak6 Mt 21:23 Lu 12:14}
\crossref{Exod}{2}{15}{Ex 4:19 Ge 28:6,\allowbreak7 1Ki 19:1-\allowbreak3,\allowbreak13,\allowbreak14 Pr 22:3 Jer 26:21-\allowbreak23 Mt 10:23}
\crossref{Exod}{2}{16}{Ex 3:1 Ge 14:18; 41:45}
\crossref{Exod}{2}{17}{Ge 21:25; 26:15-\allowbreak22}
\crossref{Exod}{2}{18}{Ex 3:1; 4:18; 18:1-\allowbreak12}
\crossref{Exod}{2}{19}{Ge 50:11}
\crossref{Exod}{2}{20}{Ge 24:31-\allowbreak33; 18:5; 19:2,\allowbreak3; 29:13; 31:54; 43:25 Job 31:32; 42:11}
\crossref{Exod}{2}{21}{2:10 Ge 31:38-\allowbreak40 Php 4:11,\allowbreak12 1Ti 6:6 Heb 11:25; 13:5 Jas 1:10}
\crossref{Exod}{2}{22}{1Ch 23:14-\allowbreak17}
\crossref{Exod}{2}{23}{Ex 7:7 Ac 7:30}
\crossref{Exod}{2}{24}{Ex 6:5 Jud 2:18 Ne 9:27,\allowbreak28 Ps 22:5,\allowbreak24; 79:11; 102:20; 138:3}
\crossref{Exod}{2}{25}{Ex 4:31 1Sa 1:11 2Sa 16:12 Job 33:27 Lu 1:25}
\crossref{Exod}{3}{1}{Ps 78:70-\allowbreak72 Am 1:1; 7:14,\allowbreak15 Mt 4:18,\allowbreak19 Lu 2:8}
\crossref{Exod}{3}{2}{3:4,\allowbreak6 Ge 16:7-\allowbreak13; 22:15,\allowbreak16; 48:16 De 33:16 Isa 63:9 Ho 12:4,\allowbreak5}
\crossref{Exod}{3}{3}{Job 37:14 Ps 107:8; 111:2-\allowbreak4 Ac 7:31}
\crossref{Exod}{3}{4}{De 33:16}
\crossref{Exod}{3}{5}{Ex 19:12,\allowbreak21 Le 10:3 Heb 12:20}
\crossref{Exod}{3}{6}{3:14,\allowbreak15; 4:5; 29:45 Ge 12:1,\allowbreak7; 17:7,\allowbreak8; 26:24; 28:13; 31:42; 32:9}
\crossref{Exod}{3}{7}{Ex 2:23-\allowbreak25; 22:23 Ge 29:32 1Sa 9:16 Ps 22:24; 34:4,\allowbreak6; 106:44}
\crossref{Exod}{3}{8}{Ge 11:5,\allowbreak7; 18:21; 50:24 Ps 18:9-\allowbreak19; 12:5; 22:4,\allowbreak5; 34:8; 91:15}
\crossref{Exod}{3}{9}{3:7; 2:23}
\crossref{Exod}{3}{10}{1Sa 12:6 Ps 77:20; 103:6,\allowbreak7; 105:26 Isa 63:11,\allowbreak12 Ho 12:13}
\crossref{Exod}{3}{11}{Ex 4:10-\allowbreak13; 6:12 1Sa 18:18 2Sa 7:18 1Ki 3:7,\allowbreak9 Isa 6:5-\allowbreak8 Jer 1:6}
\crossref{Exod}{3}{12}{Ex 4:12,\allowbreak15 Ge 15:1; 31:3 De 31:23 Jos 1:5 Isa 41:10; 43:2 Mt 28:20}
\crossref{Exod}{3}{13}{3:14; 15:3 Ge 32:29 Jud 13:6,\allowbreak17 Pr 30:4 Isa 7:14; 9:6 Jer 23:6}
\crossref{Exod}{3}{14}{Ex 6:3 Job 11:7 Ps 68:4; 90:2 Isa 44:6 Mt 18:20; 28:20 Joh 8:58}
\crossref{Exod}{3}{15}{3:6; 4:5 Ge 17:7,\allowbreak8 De 1:11,\allowbreak35; 4:1 2Ch 28:9 Mt 22:32 Ac 7:32}
\crossref{Exod}{3}{16}{Ex 4:29; 18:12; 24:11 Ge 1:7 Mt 26:3 Ac 11:30; 20:17 1Pe 5:1}
\crossref{Exod}{3}{17}{3:9; 2:23-\allowbreak25 Ge 15:13-\allowbreak21; 46:4; 50:24}
\crossref{Exod}{3}{18}{3:16; 4:31 Jos 1:17 2Ch 30:12 Ps 110:3 Jer 26:5}
\crossref{Exod}{3}{19}{Ex 5:2; 7:4}
\crossref{Exod}{3}{20}{Ex 6:6; 7:5; 9:15 Eze 20:33}
\crossref{Exod}{3}{21}{Ex 11:3; 12:36 Ge 39:21 Ne 1:11 Ps 106:46 Pr 16:7 Ac 7:10}
\crossref{Exod}{3}{22}{Ex 11:2; 12:35,\allowbreak36 Ge 15:14 Ps 105:37}
\crossref{Exod}{4}{1}{4:31; 2:14; 3:18 Jer 1:6 Eze 3:14 Ac 7:25}
\crossref{Exod}{4}{2}{4:17,\allowbreak20 Ge 30:37 Le 27:32 Ps 110:2 Isa 11:4 Mic 7:14}
\crossref{Exod}{4}{3}{4:17; 7:10-\allowbreak15 Am 5:19}
\crossref{Exod}{4}{4}{Ge 22:1,\allowbreak2 Ps 91:13 Mr 16:18 Lu 10:19 Ac 28:3-\allowbreak6}
\crossref{Exod}{4}{5}{4:1; 3:18; 4:31; 19:9 2Ch 20:20 Isa 7:9 Joh 5:36; 11:15,\allowbreak42; 20:27,\allowbreak31}
\crossref{Exod}{4}{6}{Nu 12:10 2Ki 5:27}
\crossref{Exod}{4}{7}{Nu 12:13,\allowbreak14 De 32:39 2Ki 5:14 Mt 8:3}
\crossref{Exod}{4}{8}{4:30,\allowbreak31 Isa 28:10 Joh 12:37}
\crossref{Exod}{4}{9}{Ex 7:19}
\crossref{Exod}{4}{10}{4:1 Job 12:2 1Co 2:1-\allowbreak4 2Co 10:10; 11:6}
\crossref{Exod}{4}{11}{Ge 18:14 Ps 51:15; 94:9; 146:8 Isa 6:7; 35:5,\allowbreak6; 42:7 Jer 1:6,\allowbreak9}
\crossref{Exod}{4}{12}{Ps 25:4,\allowbreak5; 32:9; 143:10 Isa 49:2; 50:4 Jer 1:9 Mt 10:19,\allowbreak20}
\crossref{Exod}{4}{13}{4:1; 23:20 Ge 24:7; 48:16 Jud 2:1 1Ki 19:4 Jer 1:6; 20:9}
\crossref{Exod}{4}{14}{2Sa 6:7 1Ki 11:9 1Ch 21:7 Lu 9:59,\allowbreak60 Ac 15:28 Php 2:21}
\crossref{Exod}{4}{15}{Ex 7:1,\allowbreak2 2Sa 14:3 Isa 51:16; 59:21}
\crossref{Exod}{4}{16}{Ex 7:1,\allowbreak2; 18:19 Ps 82:6 Joh 10:34,\allowbreak35}
\crossref{Exod}{4}{17}{4:2; 7:9,\allowbreak19 1Co 1:27}
\crossref{Exod}{4}{18}{Ex 3:1}
\crossref{Exod}{4}{19}{Ex 2:15,\allowbreak23 Mt 2:20}
\crossref{Exod}{4}{20}{4:2,\allowbreak17; 17:9 Nu 20:8,\allowbreak9}
\crossref{Exod}{4}{21}{Ex 3:20}
\crossref{Exod}{4}{22}{Ex 19:5,\allowbreak6 De 14:1 Jer 31:9 Ho 11:1 Ro 9:4 2Co 6:18 Heb 12:23}
\crossref{Exod}{4}{23}{Ex 11:5; 12:29 Ps 78:51; 105:36; 135:8}
\crossref{Exod}{4}{24}{Ge 42:27}
\crossref{Exod}{4}{25}{Jos 5:2,\allowbreak3}
\crossref{Exod}{4}{26}{}
\crossref{Exod}{4}{27}{4:14-\allowbreak16 Ec 4:9 Ac 10:5,\allowbreak6,\allowbreak20}
\crossref{Exod}{4}{28}{4:8,\allowbreak9,\allowbreak15,\allowbreak16 Jon 3:2 Mt 21:29}
\crossref{Exod}{4}{29}{Ex 3:16; 24:1,\allowbreak11}
\crossref{Exod}{4}{30}{4:16}
\crossref{Exod}{4}{31}{4:8,\allowbreak9; 3:18 Ps 106:12,\allowbreak13 Lu 8:13}
\crossref{Exod}{5}{1}{1Ki 21:20 Ps 119:46 Eze 2:6 Jon 3:3,\allowbreak4 Mt 10:18,\allowbreak28 Ac 4:29}
\crossref{Exod}{5}{2}{Ex 3:19 2Ki 18:35 2Ch 32:15,\allowbreak19 Job 21:15 Ps 10:4; 12:4; 14:1}
\crossref{Exod}{5}{3}{Ex 3:18}
\crossref{Exod}{5}{4}{Jer 38:4 Am 7:10 Lu 23:2 Ac 16:20,\allowbreak21; 24:5}
\crossref{Exod}{5}{5}{Ex 1:7-\allowbreak11 Pr 14:28}
\crossref{Exod}{5}{6}{}
\crossref{Exod}{5}{7}{}
\crossref{Exod}{5}{8}{Ps 106:41}
\crossref{Exod}{5}{9}{}
\crossref{Exod}{5}{10}{Ex 1:11 Pr 29:12}
\crossref{Exod}{5}{11}{5:13,\allowbreak14}
\crossref{Exod}{5}{12}{Ex 15:7 Isa 5:24; 47:14 Joe 2:5 Na 1:10 Ob 1:18 1Co 3:12}
\crossref{Exod}{5}{13}{5:19; 16:4}
\crossref{Exod}{5}{14}{5:14}
\crossref{Exod}{5}{15}{5:15}
\crossref{Exod}{5}{16}{5:16}
\crossref{Exod}{5}{17}{Mt 26:8 Joh 6:27 2Th 3:10,\allowbreak11}
\crossref{Exod}{5}{18}{Eze 18:18 Da 2:9-\allowbreak13}
\crossref{Exod}{5}{19}{De 32:36 Ec 4:1; 5:8}
\crossref{Exod}{5}{20}{}
\crossref{Exod}{5}{21}{Ex 4:31; 6:9 Ge 16:5}
\crossref{Exod}{5}{22}{Ex 17:4 1Sa 30:6 Ps 73:25 Jer 12:1}
\crossref{Exod}{5}{23}{Ps 118:26 Jer 11:21 Joh 5:43}
\crossref{Exod}{6}{1}{Ex 14:13 Nu 23:23 De 32:39 2Ki 7:2,\allowbreak19 2Ch 20:17 Ps 12:5}
\crossref{Exod}{6}{2}{6:6,\allowbreak8; 14:18; 17:1; 20:2 Ge 15:7 Isa 42:8; 43:11,\allowbreak15; 44:6 Jer 9:24}
\crossref{Exod}{6}{3}{}
\crossref{Exod}{6}{4}{Ge 6:18; 15:18; 17:7,\allowbreak8,\allowbreak13; 28:4 2Sa 23:5 Isa 55:3}
\crossref{Exod}{6}{5}{Ex 2:24; 3:7 Ps 106:44 Isa 63:9}
\crossref{Exod}{6}{6}{6:2,\allowbreak8,\allowbreak29 Eze 20:7-\allowbreak9}
\crossref{Exod}{6}{7}{Ex 19:5,\allowbreak6 Ge 17:7,\allowbreak8 De 4:20; 7:6; 14:2; 26:18 2Sa 7:23,\allowbreak24 Jer 31:33}
\crossref{Exod}{6}{8}{Ex 32:13 Ge 15:18; 22:16,\allowbreak17; 26:3; 28:13; 35:12}
\crossref{Exod}{6}{9}{Ex 5:21; 14:12 Job 21:4 Pr 14:19}
\crossref{Exod}{6}{10}{6:10}
\crossref{Exod}{6}{11}{6:29; 3:10; 5:1,\allowbreak23; 7:1}
\crossref{Exod}{6}{12}{6:9; 3:13; 4:29-\allowbreak31; 5:19-\allowbreak21}
\crossref{Exod}{6}{13}{Nu 27:19,\allowbreak23 De 31:14 Ps 91:11 Mt 4:6 1Ti 1:18; 5:21; 6:13,\allowbreak17}
\crossref{Exod}{6}{14}{6:25 Jos 14:1; 19:51 1Ch 5:24; 7:2,\allowbreak7; 8:6}
\crossref{Exod}{6}{15}{Ge 46:10 Nu 26:12,\allowbreak13 1Ch 4:24}
\crossref{Exod}{6}{16}{Ge 46:11 Nu 3:17 1Ch 6:1,\allowbreak16}
\crossref{Exod}{6}{17}{Ge 46:11 Nu 3:18}
\crossref{Exod}{6}{18}{Nu 3:19}
\crossref{Exod}{6}{19}{Nu 3:20 1Ch 6:19; 23:21}
\crossref{Exod}{6}{20}{Ex 2:1,\allowbreak2 Nu 26:59}
\crossref{Exod}{6}{21}{6:24 Nu 16:1,\allowbreak32; 26:10,\allowbreak11 1Ch 6:37,\allowbreak38}
\crossref{Exod}{6}{22}{Le 10:4 Ne 3:20}
\crossref{Exod}{6}{23}{Lu 1:5}
\crossref{Exod}{6}{24}{6:21 Nu 16:1,\allowbreak32; 26:9-\allowbreak11 1Ch 6:22,\allowbreak33,\allowbreak37,\allowbreak38 Ps 84:1-\allowbreak12}
\crossref{Exod}{6}{25}{Nu 25:7-\allowbreak13; 31:6 Jos 22:13,\allowbreak31,\allowbreak32; 24:33 Jud 20:28 Ps 106:30,\allowbreak31}
\crossref{Exod}{6}{26}{6:13,\allowbreak20 Jos 24:5 1Sa 12:6,\allowbreak8 1Ch 6:3 Ps 77:20; 99:6 Mic 6:4}
\crossref{Exod}{6}{27}{Ex 5:1-\allowbreak3; 7:10}
\crossref{Exod}{6}{28}{}
\crossref{Exod}{6}{29}{6:2,\allowbreak6,\allowbreak8}
\crossref{Exod}{6}{30}{6:12; 4:10 1Co 9:16,\allowbreak17}
\crossref{Exod}{7}{1}{Ex 16:29 Ge 19:21 1Ki 17:23 2Ki 6:32 Ec 1:10}
\crossref{Exod}{7}{2}{Ex 4:15; 6:29 De 4:2 1Ki 22:14 Jer 1:7,\allowbreak17 Eze 3:10,\allowbreak17 Mt 28:20}
\crossref{Exod}{7}{3}{Ex 4:21,\allowbreak29}
\crossref{Exod}{7}{4}{Ex 9:3; 10:1; 11:9 Jud 2:15 La 3:3}
\crossref{Exod}{7}{5}{7:17; 8:10,\allowbreak22; 14:4,\allowbreak18 Ps 9:16 Eze 25:17; 28:22; 36:23; 39:7,\allowbreak22}
\crossref{Exod}{7}{6}{7:2,\allowbreak10; 12:28; 39:43; 40:16 Ge 6:22; 22:18 Ps 119:4 Joh 15:10,\allowbreak14}
\crossref{Exod}{7}{7}{Ex 2:23 Ge 41:46 De 29:5; 31:2; 34:7 Ps 90:10 Ac 7:23,\allowbreak30}
\crossref{Exod}{7}{8}{}
\crossref{Exod}{7}{9}{Isa 7:11 Mt 12:39 Joh 2:18; 6:30; 10:38}
\crossref{Exod}{7}{10}{7:9}
\crossref{Exod}{7}{11}{Ge 41:8,\allowbreak38,\allowbreak39 Isa 19:11,\allowbreak12; 47:12,\allowbreak13 Da 2:2,\allowbreak27; 4:7-\allowbreak9; 5:7,\allowbreak11}
\crossref{Exod}{7}{12}{Ex 8:18,\allowbreak19; 9:11 Ac 8:9-\allowbreak13; 13:8-\allowbreak11; 19:19,\allowbreak20 1Jo 4:4}
\crossref{Exod}{7}{13}{7:4; 4:21; 8:15; 10:1,\allowbreak20,\allowbreak27; 14:17 De 2:30 Zec 7:11,\allowbreak12 Ro 1:28; 2:5}
\crossref{Exod}{7}{14}{Ex 8:15; 10:1,\allowbreak20,\allowbreak27 Zec 7:12}
\crossref{Exod}{7}{15}{Ex 2:5; 8:20 Eze 29:3}
\crossref{Exod}{7}{16}{Ex 3:18; 5:3; 9:1,\allowbreak13; 10:3 1Sa 4:6-\allowbreak9}
\crossref{Exod}{7}{17}{7:5; 5:2; 6:7 1Sa 17:46,\allowbreak47 1Ki 20:28 2Ki 19:19 Ps 9:16; 83:18}
\crossref{Exod}{7}{18}{7:21}
\crossref{Exod}{7}{19}{Ex 8:5,\allowbreak6,\allowbreak16; 9:22,\allowbreak23,\allowbreak33; 10:12,\allowbreak21; 14:21,\allowbreak26}
\crossref{Exod}{7}{20}{Ex 17:5,\allowbreak6,\allowbreak9-\allowbreak12 Nu 20:8-\allowbreak12}
\crossref{Exod}{7}{21}{7:18 Re 8:9}
\crossref{Exod}{7}{22}{7:11; 8:7,\allowbreak8 Jer 27:18 2Ti 3:8}
\crossref{Exod}{7}{23}{Ex 9:21 De 32:46 1Sa 4:20}
\crossref{Exod}{7}{24}{7:18-\allowbreak21}
\crossref{Exod}{7}{25}{Ex 8:9,\allowbreak10; 10:23 2Sa 24:13}
\crossref{Exod}{8}{1}{Jer 1:17-\allowbreak19; 15:19-\allowbreak21 Eze 2:6,\allowbreak7}
\crossref{Exod}{8}{2}{Ex 7:14; 9:2}
\crossref{Exod}{8}{3}{Ex 12:34}
\crossref{Exod}{8}{4}{Ps 107:40 Isa 19:11,\allowbreak22; 23:9 Da 4:37 Ac 12:22,\allowbreak23}
\crossref{Exod}{8}{5}{Ex 7:19}
\crossref{Exod}{8}{6}{}
\crossref{Exod}{8}{7}{Ex 7:11,\allowbreak22 De 13:1-\allowbreak3 Mt 24:24 2Th 2:9-\allowbreak11 2Ti 3:8 Re 13:14}
\crossref{Exod}{8}{8}{Ex 5:2; 9:28; 10:17 Nu 21:7 1Sa 12:19 1Ki 13:6 Ac 8:24}
\crossref{Exod}{8}{9}{Jud 7:2 1Ki 18:25 Isa 10:15}
\crossref{Exod}{8}{10}{Pr 27:1 Jas 4:14}
\crossref{Exod}{8}{11}{8:3,\allowbreak9}
\crossref{Exod}{8}{12}{8:8,\allowbreak30; 9:33; 10:18; 32:11 1Sa 12:23 Eze 36:37 Jas 5:16-\allowbreak18}
\crossref{Exod}{8}{13}{De 34:10-\allowbreak12}
\crossref{Exod}{8}{14}{8:24; 7:21 Isa 34:2 Eze 39:11 Joe 2:20}
\crossref{Exod}{8}{15}{Ex 14:5 Ec 8:11 Isa 26:10 Jer 34:7-\allowbreak11 Ho 6:4}
\crossref{Exod}{8}{16}{8:5,\allowbreak17}
\crossref{Exod}{8}{17}{Ps 105:31 Isa 23:9 Ac 12:23}
\crossref{Exod}{8}{18}{Ex 7:11}
\crossref{Exod}{8}{19}{1Sa 6:3,\allowbreak9 Ps 8:3 Da 2:10,\allowbreak11,\allowbreak19 Mt 12:28 Lu 11:20 Joh 11:47}
\crossref{Exod}{8}{20}{Ex 7:15}
\crossref{Exod}{8}{21}{}
\crossref{Exod}{8}{22}{Ex 9:4,\allowbreak6,\allowbreak26; 10:23; 11:6,\allowbreak7; 12:13 Mal 3:18}
\crossref{Exod}{8}{23}{Ex 33:16 Le 20:24 Nu 23:9 De 33:28 Ps 111:9; 130:7 Isa 50:2}
\crossref{Exod}{8}{24}{8:21 Ps 78:45; 105:31}
\crossref{Exod}{8}{25}{8:8; 9:27; 10:16; 12:31 Re 3:9}
\crossref{Exod}{8}{26}{Ex 3:18 2Co 6:14-\allowbreak17}
\crossref{Exod}{8}{27}{Ex 3:18; 5:1}
\crossref{Exod}{8}{28}{Ho 10:2}
\crossref{Exod}{8}{29}{8:10}
\crossref{Exod}{8}{30}{8:12; 9:33 Jas 5:16}
\crossref{Exod}{8}{31}{Ps 65:2}
\crossref{Exod}{8}{32}{8:15; 4:21; 7:13,\allowbreak14 Isa 63:17 Ac 28:26,\allowbreak27 Ro 2:5 Jas 1:13,\allowbreak14}
\crossref{Exod}{9}{1}{9:13; 3:18; 4:22,\allowbreak23; 5:1; 8:1,\allowbreak20; 10:3}
\crossref{Exod}{9}{2}{Ex 4:23; 8:2; 10:4 Le 26:14-\allowbreak16,\allowbreak23,\allowbreak24,\allowbreak27,\allowbreak28 Ps 7:11,\allowbreak12; 68:21}
\crossref{Exod}{9}{3}{Ex 7:4; 8:19 1Sa 5:6-\allowbreak11; 6:9 Ac 13:11}
\crossref{Exod}{9}{4}{Ex 8:22; 10:23; 12:13 Isa 65:13,\allowbreak14 Mal 3:18}
\crossref{Exod}{9}{5}{9:18; 8:23; 10:4 Nu 16:5 Job 24:1 Ec 3:1-\allowbreak11 Jer 28:16,\allowbreak17}
\crossref{Exod}{9}{6}{9:19,\allowbreak25 Ps 78:48,\allowbreak50}
\crossref{Exod}{9}{7}{9:12; 7:14; 8:32 Job 9:4 Pr 29:1 Isa 48:4 Da 5:20 Ro 9:18}
\crossref{Exod}{9}{8}{}
\crossref{Exod}{9}{9}{Le 13:18-\allowbreak20 De 28:27,\allowbreak35 Job 2:7 Re 16:2}
\crossref{Exod}{9}{10}{De 28:27}
\crossref{Exod}{9}{11}{Ex 7:11,\allowbreak12; 8:18,\allowbreak19 Isa 47:12-\allowbreak14 2Ti 3:8,\allowbreak9 Re 16:2}
\crossref{Exod}{9}{12}{Ex 4:21; 7:13,\allowbreak14 Ps 81:11,\allowbreak12 Re 16:10,\allowbreak11}
\crossref{Exod}{9}{13}{9:1; 7:15; 8:20}
\crossref{Exod}{9}{14}{Le 26:18,\allowbreak21,\allowbreak28 De 28:15-\allowbreak17,\allowbreak59-\allowbreak61; 29:20-\allowbreak22; 32:39-\allowbreak42 1Sa 4:8}
\crossref{Exod}{9}{15}{9:3,\allowbreak6,\allowbreak16; 3:20}
\crossref{Exod}{9}{16}{Ex 14:17 Ps 83:17,\allowbreak18 Pr 16:4 Ro 9:17,\allowbreak22 1Pe 2:8,\allowbreak19 Jude 1:4}
\crossref{Exod}{9}{17}{Job 9:4; 15:25,\allowbreak26; 40:9 Isa 10:15; 26:11; 37:23,\allowbreak24,\allowbreak29; 45:9}
\crossref{Exod}{9}{18}{1Ki 19:2; 20:6 2Ki 7:1,\allowbreak18}
\crossref{Exod}{9}{19}{Hab 3:2}
\crossref{Exod}{9}{20}{Pr 16:16; 22:3,\allowbreak23 Jon 3:5,\allowbreak6 Mr 13:14-\allowbreak16 Heb 11:7}
\crossref{Exod}{9}{21}{Ex 7:23 1Sa 4:20}
\crossref{Exod}{9}{22}{Ex 7:19; 8:5,\allowbreak16 Re 16:21}
\crossref{Exod}{9}{23}{Ex 19:16; 20:18 1Sa 12:17,\allowbreak18 Job 37:1-\allowbreak5 Ps 29:3; 77:18}
\crossref{Exod}{9}{24}{9:23; 10:6 Mt 24:21}
\crossref{Exod}{9}{25}{Ps 105:33}
\crossref{Exod}{9}{26}{Ex 8:22-\allowbreak32; 9:4,\allowbreak6; 10:23; 11:7; 12:13 Isa 32:18,\allowbreak19}
\crossref{Exod}{9}{27}{Ex 10:16 Nu 22:34 1Sa 15:24,\allowbreak30; 26:21 Mt 27:4}
\crossref{Exod}{9}{28}{Ex 8:8,\allowbreak28; 10:17 Ac 8:24}
\crossref{Exod}{9}{29}{9:33 1Ki 8:22,\allowbreak38 2Ch 6:12,\allowbreak13 Ezr 9:5 Job 11:13 Ps 143:6}
\crossref{Exod}{9}{30}{Pr 16:6 Isa 26:10; 63:17}
\crossref{Exod}{9}{31}{}
\crossref{Exod}{9}{32}{Ex 10:22}
\crossref{Exod}{9}{33}{9:29; 8:12}
\crossref{Exod}{9}{34}{Ex 8:15 Ec 8:11}
\crossref{Exod}{9}{35}{Ex 7:13}
\crossref{Exod}{10}{1}{Ex 4:21; 7:13,\allowbreak14; 9:27,\allowbreak34,\allowbreak35 Ps 7:11}
\crossref{Exod}{10}{2}{Ex 13:8,\allowbreak9,\allowbreak14 De 4:9; 6:20-\allowbreak22 Ps 44:1; 71:18; 78:5,\allowbreak6 Joe 1:3 Eph 6:4}
\crossref{Exod}{10}{3}{Ex 9:17; 16:28 Nu 14:27 1Ki 18:21 Pr 1:22,\allowbreak24 Jer 13:10 Eze 5:6}
\crossref{Exod}{10}{4}{Ex 8:10,\allowbreak23; 9:5,\allowbreak18; 11:4,\allowbreak5}
\crossref{Exod}{10}{5}{10:15}
\crossref{Exod}{10}{6}{Ex 8:3,\allowbreak21}
\crossref{Exod}{10}{7}{10:3}
\crossref{Exod}{10}{8}{10:16,\allowbreak24; 12:31}
\crossref{Exod}{10}{9}{Ge 50:8 De 31:12,\allowbreak13 Jos 24:15 Ps 148:12,\allowbreak13 Ec 12:1 Eph 6:4}
\crossref{Exod}{10}{10}{Ex 12:30,\allowbreak31; 13:21}
\crossref{Exod}{10}{11}{Ps 52:3,\allowbreak4; 119:69}
\crossref{Exod}{10}{12}{Ex 7:19}
\crossref{Exod}{10}{13}{Ex 14:21 Ge 41:6 Ps 78:26; 107:25-\allowbreak28; 148:8 Jon 1:4; 4:8 Mt 8:27}
\crossref{Exod}{10}{14}{De 28:42 1Ki 8:37 Ps 78:46; 105:34,\allowbreak35 Re 9:3-\allowbreak7}
\crossref{Exod}{10}{15}{10:5 Joe 1:6,\allowbreak7; 2:1-\allowbreak11,\allowbreak25}
\crossref{Exod}{10}{16}{Ex 9:27 Nu 21:7; 22:34 1Sa 15:24,\allowbreak30; 26:21 2Sa 19:20 Job 34:31,\allowbreak32}
\crossref{Exod}{10}{17}{1Sa 15:25}
\crossref{Exod}{10}{18}{Ex 8:30}
\crossref{Exod}{10}{19}{10:13}
\crossref{Exod}{10}{20}{Ex 4:21; 7:13,\allowbreak14; 9:12; 11:10 De 2:30 Isa 6:9,\allowbreak10 Joh 12:39,\allowbreak40}
\crossref{Exod}{10}{21}{Ex 9:22}
\crossref{Exod}{10}{22}{Ex 20:21 De 4:11; 5:22 Ps 105:28 Joe 2:2,\allowbreak31 Am 4:13 Re 16:10}
\crossref{Exod}{10}{23}{Ex 8:22; 9:4,\allowbreak26; 14:20 Jos 24:7 Isa 42:16; 60:1-\allowbreak3; 65:13,\allowbreak14}
\crossref{Exod}{10}{24}{10:8,\allowbreak9; 8:28; 9:28}
\crossref{Exod}{10}{25}{Ex 29:1-\allowbreak46; 36:1-\allowbreak40:38 Le 9:22; 16:9}
\crossref{Exod}{10}{26}{Ex 12:32 Isa 23:18; 60:5-\allowbreak10 Ho 5:6 Zec 14:20 Ac 2:44,\allowbreak45 2Co 8:5}
\crossref{Exod}{10}{27}{10:1,\allowbreak20; 4:21; 14:4,\allowbreak8 Re 9:20; 16:10,\allowbreak11}
\crossref{Exod}{10}{28}{10:11}
\crossref{Exod}{10}{29}{Ex 11:4-\allowbreak8; 12:30,\allowbreak31 Heb 11:27}
\crossref{Exod}{11}{1}{Ex 9:14 Le 26:21 De 4:34 1Sa 6:4 Job 10:17 Re 16:9}
\crossref{Exod}{11}{2}{Ex 3:22; 12:1,\allowbreak2,\allowbreak35,\allowbreak36 Ge 31:9 Job 27:16,\allowbreak17 Ps 24:1; 105:37}
\crossref{Exod}{11}{3}{Ex 3:21; 12:36 Ge 39:21 Ps 106:46}
\crossref{Exod}{11}{4}{Ex 12:12,\allowbreak23,\allowbreak29 Job 34:20 Am 4:10; 5:17 Mt 25:6}
\crossref{Exod}{11}{5}{Ex 4:23; 12:12,\allowbreak29; 13:15 Ps 78:51; 105:36; 135:8; 136:10 Heb 11:28}
\crossref{Exod}{11}{6}{Ex 3:7; 12:30 Pr 21:13 Isa 15:4,\allowbreak5,\allowbreak8 Jer 31:15 La 3:8 Am 5:17}
\crossref{Exod}{11}{7}{Jos 10:21 Job 5:16}
\crossref{Exod}{11}{8}{Ex 12:31-\allowbreak33 Isa 49:23,\allowbreak26 Re 3:9}
\crossref{Exod}{11}{9}{Ex 3:19; 7:4; 10:1 Ro 9:16-\allowbreak18}
\crossref{Exod}{11}{10}{Ex 4:21; 7:13,\allowbreak14; 10:20,\allowbreak27 De 2:30 1Sa 6:6 Job 9:4 Ro 2:4,\allowbreak5; 9:22}
\crossref{Exod}{12}{1}{}
\crossref{Exod}{12}{2}{Ex 13:4; 23:15; 34:18 Le 23:5 Nu 28:16 De 16:1 Es 3:7}
\crossref{Exod}{12}{3}{Ex 4:30; 6:6; 14:15; 20:19 Le 1:2}
\crossref{Exod}{12}{4}{Jos 19:9}
\crossref{Exod}{12}{5}{Le 1:3,\allowbreak10; 22:19-\allowbreak24 De 17:1 Mal 1:7,\allowbreak8,\allowbreak14 Heb 7:26; 9:13,\allowbreak14}
\crossref{Exod}{12}{6}{Le 23:5 Nu 9:3; 28:16,\allowbreak18 De 16:1-\allowbreak6 2Ch 30:15 Eze 45:21}
\crossref{Exod}{12}{7}{12:22,\allowbreak23 Eph 1:7 Heb 9:13,\allowbreak14,\allowbreak22; 10:14,\allowbreak29; 11:28 1Pe 1:2}
\crossref{Exod}{12}{8}{Mt 26:26 Joh 6:52-\allowbreak57}
\crossref{Exod}{12}{9}{12:8 De 16:7 La 1:13}
\crossref{Exod}{12}{10}{Ex 23:18; 29:34; 34:25 Le 7:15-\allowbreak17; 22:30 De 16:4,\allowbreak5}
\crossref{Exod}{12}{11}{Mt 26:19,\allowbreak20 Lu 12:35 Eph 6:15 1Pe 1:13}
\crossref{Exod}{12}{12}{12:23; 11:4,\allowbreak5 Am 5:17}
\crossref{Exod}{12}{13}{12:23 Ge 17:11 Jos 2:12 Heb 11:28}
\crossref{Exod}{12}{14}{Ex 13:9 Nu 16:40 Jos 4:7 Ps 111:4; 135:13 Zec 6:14 Mt 26:13}
\crossref{Exod}{12}{15}{12:8; 13:6,\allowbreak7-\allowbreak10; 23:15; 34:18,\allowbreak25 Le 23:5-\allowbreak8 Nu 28:17 De 16:3,\allowbreak5,\allowbreak8}
\crossref{Exod}{12}{16}{Le 23:2,\allowbreak3,\allowbreak7,\allowbreak8,\allowbreak21,\allowbreak24,\allowbreak25,\allowbreak27,\allowbreak35 Nu 28:18,\allowbreak25; 29:1,\allowbreak12}
\crossref{Exod}{12}{17}{Ex 7:5; 13:8 Nu 20:16}
\crossref{Exod}{12}{18}{12:1,\allowbreak15 Le 23:5,\allowbreak6 Nu 28:16}
\crossref{Exod}{12}{19}{Ex 23:15; 34:18 De 16:3 1Co 5:7,\allowbreak8}
\crossref{Exod}{12}{20}{Le 23:2,\allowbreak3,\allowbreak7,\allowbreak8,\allowbreak21,\allowbreak24,\allowbreak25,\allowbreak27,\allowbreak35 Nu 28:18,\allowbreak25; 29:1,\allowbreak12}
\crossref{Exod}{12}{21}{Ex 3:16; 17:5; 19:7 Nu 11:16}
\crossref{Exod}{12}{22}{Le 14:6,\allowbreak7 Nu 19:18 Ps 51:7 Heb 9:1,\allowbreak14,\allowbreak19; 11:28; 12:24 1Pe 1:2}
\crossref{Exod}{12}{23}{12:12,\allowbreak13}
\crossref{Exod}{12}{24}{12:14 Ge 17:8-\allowbreak10}
\crossref{Exod}{12}{25}{De 4:5; 12:8,\allowbreak9; 16:5-\allowbreak9 Jos 5:10-\allowbreak12 Ps 105:44,\allowbreak45}
\crossref{Exod}{12}{26}{Ex 13:8,\allowbreak9,\allowbreak14,\allowbreak15,\allowbreak22 De 6:7; 11:19; 32:7 Jos 4:6,\allowbreak7,\allowbreak21-\allowbreak24 Ps 78:3-\allowbreak6}
\crossref{Exod}{12}{27}{12:11,\allowbreak23; 34:25 De 16:2,\allowbreak5 1Co 5:7}
\crossref{Exod}{12}{28}{Heb 11:28}
\crossref{Exod}{12}{29}{12:12; 11:4; 13:15 Job 34:20 1Th 5:2,\allowbreak3}
\crossref{Exod}{12}{30}{Ex 11:6 Pr 21:13 Am 5:17 Mt 25:6 Jas 2:13}
\crossref{Exod}{12}{31}{Ex 10:29}
\crossref{Exod}{12}{32}{Ex 10:26}
\crossref{Exod}{12}{33}{Ex 11:1 Ps 105:38}
\crossref{Exod}{12}{34}{Ex 8:3}
\crossref{Exod}{12}{35}{Ex 3:21,\allowbreak22; 11:2,\allowbreak3 Ge 15:14 Ps 105:37}
\crossref{Exod}{12}{36}{Ex 3:21; 11:3 Ge 39:21 Pr 16:7 Da 1:9 Ac 2:47; 7:10}
\crossref{Exod}{12}{37}{Nu 33:3,\allowbreak5}
\crossref{Exod}{12}{38}{Nu 11:4 Zec 8:23}
\crossref{Exod}{12}{39}{12:33; 6:1; 11:1}
\crossref{Exod}{12}{40}{Ac 13:17 Heb 11:9}
\crossref{Exod}{12}{41}{Ps 102:13 Da 9:24 Hab 2:3 Joh 7:8 Ac 1:7}
\crossref{Exod}{12}{42}{}
\crossref{Exod}{12}{43}{12:48 Le 22:10 Nu 9:14 Eph 2:12}
\crossref{Exod}{12}{44}{Ge 17:12,\allowbreak13,\allowbreak23}
\crossref{Exod}{12}{45}{Le 22:10 Eph 2:12}
\crossref{Exod}{12}{46}{1Co 12:12 Eph 2:19-\allowbreak22}
\crossref{Exod}{12}{47}{12:3,\allowbreak6 Nu 9:13}
\crossref{Exod}{12}{48}{12:43 Nu 9:14; 15:15,\allowbreak16}
\crossref{Exod}{12}{49}{Le 24:22 Nu 9:14; 15:15,\allowbreak16,\allowbreak29 Ga 3:28 Col 3:11}
\crossref{Exod}{12}{50}{De 4:1,\allowbreak2; 12:32 Mt 7:24,\allowbreak25; 28:20 Joh 2:5; 13:17; 15:14 Re 22:15}
\crossref{Exod}{12}{51}{12:41}
\crossref{Exod}{13}{1}{13:1}
\crossref{Exod}{13}{2}{}
\crossref{Exod}{13}{3}{Ex 12:42; 20:8; 23:15 De 5:15; 15:15; 16:3,\allowbreak12; 24:18,\allowbreak22 1Ch 16:12}
\crossref{Exod}{13}{4}{Ex 23:15; 34:18 De 16:1-\allowbreak3}
\crossref{Exod}{13}{5}{Ex 3:8; 34:11 Ge 15:18-\allowbreak21 De 7:1; 12:29; 19:1; 26:1 Jos 24:11}
\crossref{Exod}{13}{6}{Ex 12:15-\allowbreak20; 34:18 Le 23:8}
\crossref{Exod}{13}{7}{Ex 12:19 Mt 16:6}
\crossref{Exod}{13}{8}{13:14; 12:26,\allowbreak27 De 4:9,\allowbreak10 Ps 44:1; 78:3-\allowbreak8 Isa 38:19 Eph 6:4}
\crossref{Exod}{13}{9}{De 30:14 Jos 1:8 Isa 59:21 Ro 10:8}
\crossref{Exod}{13}{10}{Ex 12:14,\allowbreak24; 23:15 Le 23:6 De 16:3,\allowbreak4}
\crossref{Exod}{13}{11}{13:5}
\crossref{Exod}{13}{12}{13:2; 22:29; 34:19 Le 27:26 Nu 8:17; 18:15 De 15:19 Eze 44:30}
\crossref{Exod}{13}{13}{Ex 34:20 Nu 18:15-\allowbreak17}
\crossref{Exod}{13}{14}{Ex 12:26 De 6:20-\allowbreak24 Jos 4:6,\allowbreak21-\allowbreak24 Ps 145:4}
\crossref{Exod}{13}{15}{Ex 12:29}
\crossref{Exod}{13}{16}{13:9; 12:13}
\crossref{Exod}{13}{17}{Ex 14:11,\allowbreak12 Nu 14:1-\allowbreak4 De 20:8 Jud 7:3 1Ki 8:47 Lu 14:27-\allowbreak32}
\crossref{Exod}{13}{18}{Ex 14:2 Nu 33:6-\allowbreak8 De 32:10 Ps 107:7}
\crossref{Exod}{13}{19}{Ge 50:24,\allowbreak25 Jos 24:32 Ac 7:16}
\crossref{Exod}{13}{20}{Nu 33:5,\allowbreak6}
\crossref{Exod}{13}{21}{Ex 14:19-\allowbreak24; 40:34-\allowbreak38 Nu 9:15-\allowbreak23; 10:34; 14:14 De 1:33 Ne 9:12,\allowbreak19}
\crossref{Exod}{13}{22}{Ps 121:5-\allowbreak8}
\crossref{Exod}{14}{1}{Ex 12:1; 13:1}
\crossref{Exod}{14}{2}{14:9; 13:17,\allowbreak18 Nu 33:7,\allowbreak8}
\crossref{Exod}{14}{3}{Ex 7:3,\allowbreak4 De 31:21 Ps 139:2,\allowbreak4 Eze 38:10,\allowbreak11,\allowbreak17 Ac 4:28}
\crossref{Exod}{14}{4}{14:8,\allowbreak17; 4:21-\allowbreak31; 7:3,\allowbreak13,\allowbreak14 Ro 11:8}
\crossref{Exod}{14}{5}{Ex 12:33 Ps 105:25}
\crossref{Exod}{14}{6}{14:23 Nu 21:23 De 2:32; 3:1}
\crossref{Exod}{14}{7}{14:23; 15:4 Jos 17:16-\allowbreak18 Jud 4:3,\allowbreak15 Ps 20:7; 68:17 Isa 37:24}
\crossref{Exod}{14}{8}{14:4}
\crossref{Exod}{14}{9}{Ex 15:9 Jos 24:6}
\crossref{Exod}{14}{10}{Ps 53:5 Isa 7:2; 8:12,\allowbreak13; 51:12,\allowbreak13 Mt 8:26; 14:30,\allowbreak31 1Jo 4:18}
\crossref{Exod}{14}{11}{Ex 15:23,\allowbreak24; 16:2,\allowbreak3; 17:2,\allowbreak3 Nu 11:1; 14:1-\allowbreak4; 16:41 Ps 106:7,\allowbreak8}
\crossref{Exod}{14}{12}{Ex 5:21; 3:9}
\crossref{Exod}{14}{13}{Nu 14:9 De 20:3 2Ki 6:16 2Ch 20:15,\allowbreak17 Ps 27:1,\allowbreak2; 46:1-\allowbreak3}
\crossref{Exod}{14}{14}{14:25; 15:3 De 1:30; 3:22; 20:4 Jos 10:10,\allowbreak14,\allowbreak42; 23:3,\allowbreak10 Jud 5:20}
\crossref{Exod}{14}{15}{Ex 17:4 Jos 7:10 Ezr 10:4,\allowbreak5 Ne 9:9}
\crossref{Exod}{14}{16}{14:21,\allowbreak26; 4:2,\allowbreak17,\allowbreak20; 7:9,\allowbreak19}
\crossref{Exod}{14}{17}{Ge 6:17; 9:9 Le 26:28 De 32:39 Isa 48:15; 51:12 Jer 23:39}
\crossref{Exod}{14}{18}{14:4; 7:5,\allowbreak17}
\crossref{Exod}{14}{19}{14:24; 13:21; 23:20,\allowbreak21; 32:34 Nu 20:16 Isa 63:9}
\crossref{Exod}{14}{20}{Ps 18:11 Pr 4:18,\allowbreak19 Isa 8:14 2Co 2:15,\allowbreak16}
\crossref{Exod}{14}{21}{14:16}
\crossref{Exod}{14}{22}{14:29; 15:19 Nu 33:8 Ps 66:6; 78:13 Isa 63:13 1Co 10:1 Heb 11:29}
\crossref{Exod}{14}{23}{14:17; 15:9,\allowbreak19 1Ki 22:20 Ec 9:3 Isa 14:24-\allowbreak27}
\crossref{Exod}{14}{24}{1Sa 11:11}
\crossref{Exod}{14}{25}{Jud 4:15 Ps 46:9; 76:6 Jer 51:21}
\crossref{Exod}{14}{26}{14:16; 7:19; 8:5 Mt 8:27}
\crossref{Exod}{14}{27}{14:21,\allowbreak22; 15:1-\allowbreak21 Jos 4:18}
\crossref{Exod}{14}{28}{Ex 15:10 De 11:4 Ne 9:11 Ps 78:53 Hab 3:8-\allowbreak10,\allowbreak13 Heb 11:29}
\crossref{Exod}{14}{29}{14:22 Job 38:8-\allowbreak11 Ps 66:6,\allowbreak7; 77:19,\allowbreak20; 78:52,\allowbreak53 Isa 43:2}
\crossref{Exod}{14}{30}{14:13 1Sa 14:23 2Ch 32:22 Ps 106:8,\allowbreak10 Isa 63:9 Jude 1:5}
\crossref{Exod}{14}{31}{1Sa 12:18 Ps 119:120}
\crossref{Exod}{15}{1}{Jud 5:1-\allowbreak31 2Sa 22:1-\allowbreak51 Ps 106:12; 107:8,\allowbreak15,\allowbreak21,\allowbreak22 Isa 12:1-\allowbreak6}
\crossref{Exod}{15}{2}{Ps 18:1,\allowbreak2; 27:1; 28:8; 59:17; 62:6,\allowbreak7; 118:14 Hab 3:17-\allowbreak19 Php 4:13}
\crossref{Exod}{15}{3}{Ps 24:8; 45:3 Re 19:11-\allowbreak21}
\crossref{Exod}{15}{4}{Ex 14:13-\allowbreak28}
\crossref{Exod}{15}{5}{Ex 14:28 Eze 27:34 Jon 2:2 Mic 7:19 Mt 18:6}
\crossref{Exod}{15}{6}{15:11 1Ch 29:11,\allowbreak12 Ps 17:7; 44:3; 60:5; 74:11; 77:10; 89:8-\allowbreak13}
\crossref{Exod}{15}{7}{Ex 9:16 De 33:26 Ps 68:33; 148:13 Isa 5:16 Jer 10:6}
\crossref{Exod}{15}{8}{Ex 14:21 2Sa 22:16 Job 4:9 Isa 11:4; 37:7 2Th 2:8}
\crossref{Exod}{15}{9}{Ge 49:27 Jud 5:30 1Ki 19:2; 20:10 Isa 10:8-\allowbreak13; 36:20; 53:12}
\crossref{Exod}{15}{10}{Ex 14:21 Ge 8:1 Ps 74:13,\allowbreak14; 135:7; 147:18 Isa 11:15 Jer 10:13}
\crossref{Exod}{15}{11}{De 3:24; 33:26 1Sa 2:2 2Sa 7:22 1Ki 8:23 Ps 35:10; 77:19; 86:8}
\crossref{Exod}{15}{12}{15:6}
\crossref{Exod}{15}{13}{Ge 19:16 Eph 2:4}
\crossref{Exod}{15}{14}{Nu 14:14; 22:5 De 2:4,\allowbreak5 Jos 2:9,\allowbreak10; 9:24 Ps 48:6}
\crossref{Exod}{15}{15}{Ge 36:40 Nu 20:14-\allowbreak21 De 2:4 1Ch 1:51-\allowbreak54}
\crossref{Exod}{15}{16}{De 2:25; 11:25 Jos 2:9}
\crossref{Exod}{15}{17}{Ps 44:2; 78:54,\allowbreak55; 80:8 Isa 5:1-\allowbreak4 Jer 2:21; 32:41}
\crossref{Exod}{15}{18}{Ps 10:16; 29:10; 146:10 Isa 57:15 Da 2:44; 4:3; 7:14,\allowbreak27 Mt 6:13}
\crossref{Exod}{15}{19}{Ex 14:23 Pr 21:31}
\crossref{Exod}{15}{20}{Jud 4:4 1Sa 10:5 2Ki 22:14 Lu 2:36 Ac 21:9 1Co 11:5; 14:34}
\crossref{Exod}{15}{21}{1Sa 18:7 2Ch 5:13 Ps 24:7-\allowbreak10; 134:1-\allowbreak3}
\crossref{Exod}{15}{22}{Ex 3:18}
\crossref{Exod}{15}{23}{Nu 33:8}
\crossref{Exod}{15}{24}{Ex 14:11; 16:2,\allowbreak8,\allowbreak9; 17:3,\allowbreak4 Nu 11:1-\allowbreak6; 14:1-\allowbreak4; 16:11,\allowbreak41; 17:10}
\crossref{Exod}{15}{25}{Ex 14:10; 17:4 Ps 50:15; 91:15; 99:6 Jer 15:1}
\crossref{Exod}{15}{26}{Le 26:3,\allowbreak13 De 7:12,\allowbreak13,\allowbreak15; 28:1-\allowbreak15}
\crossref{Exod}{15}{27}{}
\crossref{Exod}{16}{1}{Ex 15:27 Nu 33:10-\allowbreak12}
\crossref{Exod}{16}{2}{Ex 15:24 Ge 19:4 Ps 106:7,\allowbreak13,\allowbreak25 1Co 10:10}
\crossref{Exod}{16}{3}{Nu 20:3-\allowbreak5 De 28:67 Jos 7:7 2Sa 18:33 La 4:9 Ac 26:29 1Co 4:8}
\crossref{Exod}{16}{4}{Ps 78:24,\allowbreak25; 105:40 Joh 6:31,\allowbreak32 1Co 10:3}
\crossref{Exod}{16}{5}{16:23; 35:2,\allowbreak3 Le 25:21,\allowbreak22}
\crossref{Exod}{16}{6}{16:8,\allowbreak12,\allowbreak13}
\crossref{Exod}{16}{7}{16:13}
\crossref{Exod}{16}{8}{16:9,\allowbreak12 Nu 14:27 Mt 9:4 Joh 6:41-\allowbreak43 1Co 10:10}
\crossref{Exod}{16}{9}{Nu 16:16}
\crossref{Exod}{16}{10}{16:7 Nu 14:10; 16:19,\allowbreak42}
\crossref{Exod}{16}{11}{}
\crossref{Exod}{16}{12}{16:8}
\crossref{Exod}{16}{13}{Nu 11:9}
\crossref{Exod}{16}{14}{Nu 11:7-\allowbreak9 De 8:3 Ne 9:15 Ps 78:24; 105:40}
\crossref{Exod}{16}{15}{16:31,\allowbreak33 De 8:3,\allowbreak16 Jos 5:12 Ne 9:15,\allowbreak20 Joh 6:31,\allowbreak32,\allowbreak49,\allowbreak58}
\crossref{Exod}{16}{16}{16:18,\allowbreak33,\allowbreak36}
\crossref{Exod}{16}{17}{16:18; 36:5 Le 11:42 1Ch 8:40 Ne 6:17; 9:37 Pr 28:8 Ec 6:11}
\crossref{Exod}{16}{18}{2Co 8:14,\allowbreak15}
\crossref{Exod}{16}{19}{Ex 12:10; 23:18 Mt 6:34}
\crossref{Exod}{16}{20}{Mt 6:19 Lu 12:15,\allowbreak33 Heb 13:5 Jas 5:2,\allowbreak3}
\crossref{Exod}{16}{21}{Pr 6:6-\allowbreak11 Ec 9:10; 12:1 Mt 6:33 Joh 12:35 2Co 6:2}
\crossref{Exod}{16}{22}{}
\crossref{Exod}{16}{23}{Ex 20:8-\allowbreak11; 31:15; 35:3 Ge 2:2,\allowbreak3 Le 23:3 Mr 2:27,\allowbreak28 Lu 23:56}
\crossref{Exod}{16}{24}{16:20,\allowbreak33}
\crossref{Exod}{16}{25}{16:23,\allowbreak29 Ne 9:14}
\crossref{Exod}{16}{26}{Ex 20:9-\allowbreak11 De 5:13 Eze 46:1 Lu 13:14}
\crossref{Exod}{16}{27}{Pr 20:4}
\crossref{Exod}{16}{28}{Ex 10:3 Nu 14:11; 20:12 2Ki 17:14 Ps 78:10,\allowbreak22; 81:13,\allowbreak14; 106:13}
\crossref{Exod}{16}{29}{Ex 31:13 Ne 9:14 Isa 58:13,\allowbreak14 Eze 20:12}
\crossref{Exod}{16}{30}{Le 23:3 De 5:12-\allowbreak14 Heb 4:9}
\crossref{Exod}{16}{31}{16:15}
\crossref{Exod}{16}{32}{Ps 103:1,\allowbreak2; 105:5; 111:4,\allowbreak5 Lu 22:19 Heb 2:1}
\crossref{Exod}{16}{33}{Heb 9:4}
\crossref{Exod}{16}{34}{Ex 25:16,\allowbreak21; 27:21; 30:6,\allowbreak36; 31:18; 38:21; 40:20 Nu 1:50,\allowbreak53; 17:10}
\crossref{Exod}{16}{35}{Nu 33:38 De 8:2,\allowbreak3 Ne 9:15,\allowbreak20,\allowbreak21 Ps 78:24,\allowbreak25 Joh 6:30-\allowbreak58}
\crossref{Exod}{16}{36}{16:16,\allowbreak32,\allowbreak33}
\crossref{Exod}{17}{1}{Ex 16:1 Nu 33:12-\allowbreak14}
\crossref{Exod}{17}{2}{Ex 5:21; 14:11,\allowbreak12; 15:24; 16:2,\allowbreak3 Nu 11:4-\allowbreak6; 14:2; 20:3-\allowbreak5; 21:5}
\crossref{Exod}{17}{3}{Ex 16:3}
\crossref{Exod}{17}{4}{Ex 14:15; 15:25 Nu 11:11}
\crossref{Exod}{17}{5}{Eze 2:6 Ac 20:23,\allowbreak24}
\crossref{Exod}{17}{6}{Ex 16:10}
\crossref{Exod}{17}{7}{Nu 20:13 De 9:22}
\crossref{Exod}{17}{8}{Ge 36:12,\allowbreak16 Nu 24:20 De 25:17 1Sa 15:2; 30:1 Ps 83:7}
\crossref{Exod}{17}{9}{17:13; 24:13 Nu 11:28; 13:16}
\crossref{Exod}{17}{10}{Jos 11:15 Mt 28:20 Joh 2:5; 15:14}
\crossref{Exod}{17}{11}{Ps 56:9 Lu 18:1 1Ti 2:8 Jas 5:16}
\crossref{Exod}{17}{12}{Mt 26:40-\allowbreak45 Mr 14:37-\allowbreak40 Eph 6:18 Col 4:2}
\crossref{Exod}{17}{13}{Jos 10:28,\allowbreak32,\allowbreak37,\allowbreak42; 11:12}
\crossref{Exod}{17}{14}{Ex 12:14; 13:9; 34:27 De 31:9 Jos 4:7 Job 19:23 Hag 2:2,\allowbreak3 Nu 24:20 De 25:17-\allowbreak19 1Sa 15:2,\allowbreak3,\allowbreak7,\allowbreak8,\allowbreak18; 27:8,\allowbreak9; 30:1,\allowbreak17}
\crossref{Exod}{17}{15}{Ge 22:14; 33:20 Ps 60:4}
\crossref{Exod}{17}{16}{Ps 21:8-\allowbreak11}
\crossref{Exod}{18}{1}{Ex 2:16,\allowbreak21; 3:1; 4:18 Nu 10:29 Jud 4:11}
\crossref{Exod}{18}{2}{Ex 2:21; 4:25,\allowbreak26}
\crossref{Exod}{18}{3}{Ac 7:29}
\crossref{Exod}{18}{4}{Ps 46:1 Isa 50:7-\allowbreak9 Heb 13:6}
\crossref{Exod}{18}{5}{Ex 3:1,\allowbreak12; 19:11,\allowbreak20; 24:16,\allowbreak17 1Ki 19:8}
\crossref{Exod}{18}{6}{Mt 12:47}
\crossref{Exod}{18}{7}{Ge 14:17; 46:29 Nu 22:36 Jud 11:34 1Ki 2:19 Ac 28:15}
\crossref{Exod}{18}{8}{18:1 Ne 9:9-\allowbreak15 Ps 66:16; 71:17-\allowbreak20; 105:1,\allowbreak2; 145:4-\allowbreak12}
\crossref{Exod}{18}{9}{Isa 44:23; 66:10 Ro 12:10,\allowbreak15 1Co 12:26}
\crossref{Exod}{18}{10}{Ge 14:20 2Sa 18:28 1Ki 8:15 Ps 41:13; 106:47,\allowbreak48 Lu 1:68}
\crossref{Exod}{18}{11}{Ex 9:16 1Ki 17:24 2Ki 5:15}
\crossref{Exod}{18}{12}{Ex 24:5 Ge 4:4; 8:20; 12:7; 26:25; 31:54 Job 1:5; 42:8}
\crossref{Exod}{18}{13}{Jud 5:10 Job 29:7 Isa 16:5 Joe 3:12 Mt 23:2 Ro 12:8; 13:6}
\crossref{Exod}{18}{14}{Pr 1:5; 9:9; 11:14; 12:15; 19:20; 20:18; 24:6}
\crossref{Exod}{18}{15}{18:19,\allowbreak20 Le 24:12-\allowbreak14 Nu 15:34; 27:5}
\crossref{Exod}{18}{16}{Ex 23:7; 24:14 De 17:8-\allowbreak12 2Sa 15:3 Job 31:13 Ac 18:14,\allowbreak15}
\crossref{Exod}{18}{17}{18:14 De 1:12 1Ki 3:8,\allowbreak9; 13:18 2Ch 19:6 Mt 17:4 Joh 13:6-\allowbreak10}
\crossref{Exod}{18}{18}{2Co 12:15 Php 2:30 1Th 2:8,\allowbreak9}
\crossref{Exod}{18}{19}{18:24 Pr 9:9}
\crossref{Exod}{18}{20}{18:16 De 4:1,\allowbreak5; 5:1; 6:1,\allowbreak2; 7:11 Ne 9:13,\allowbreak14}
\crossref{Exod}{18}{21}{De 1:13-\allowbreak17 Ac 6:3}
\crossref{Exod}{18}{22}{18:26 Ro 13:6}
\crossref{Exod}{18}{23}{18:18 Ge 21:10-\allowbreak12 1Sa 8:6,\allowbreak7,\allowbreak22 Ac 15:2 Ga 2:2}
\crossref{Exod}{18}{24}{18:2-\allowbreak5,\allowbreak19 Ezr 10:2,\allowbreak5 Pr 1:5 1Co 12:21}
\crossref{Exod}{18}{25}{18:21 De 1:15 Ac 6:5}
\crossref{Exod}{18}{26}{18:14,\allowbreak22}
\crossref{Exod}{18}{27}{Ge 24:59; 31:55 Nu 10:29,\allowbreak30 Jud 19:9}
\crossref{Exod}{19}{1}{Ex 12:2,\allowbreak6 Le 23:16-\allowbreak18}
\crossref{Exod}{19}{2}{Ex 17:1,\allowbreak8}
\crossref{Exod}{19}{3}{Ex 20:21; 24:15-\allowbreak18; 34:2 De 5:5-\allowbreak31 Ac 7:38}
\crossref{Exod}{19}{4}{Ex 7:1-\allowbreak14:31 De 4:9,\allowbreak33-\allowbreak36; 29:2 Isa 63:9}
\crossref{Exod}{19}{5}{Ex 23:22; 24:7 De 11:27; 28:1 Jos 24:24 1Sa 15:22 Isa 1:19}
\crossref{Exod}{19}{6}{De 33:2-\allowbreak4 Isa 61:6 Ro 12:1 1Pe 2:5,\allowbreak9 Re 1:6; 5:10; 20:6}
\crossref{Exod}{19}{7}{Ex 3:16}
\crossref{Exod}{19}{8}{Ex 20:19; 24:3,\allowbreak7 De 5:27-\allowbreak29; 26:17-\allowbreak19 Jos 24:24 Ne 10:29}
\crossref{Exod}{19}{9}{19:16; 20:21; 24:15,\allowbreak16 De 4:11 1Ki 8:12 2Ch 6:1 Ps 18:11,\allowbreak12}
\crossref{Exod}{19}{10}{19:15 Le 11:44,\allowbreak45 Jos 3:5; 7:13 1Sa 16:5 2Ch 29:5,\allowbreak34; 30:17-\allowbreak19}
\crossref{Exod}{19}{11}{19:16,\allowbreak18,\allowbreak20; 3:8; 34:5 Nu 11:17 De 33:2 Ps 18:9; 144:5 Isa 64:1,\allowbreak2}
\crossref{Exod}{19}{12}{19:21,\allowbreak23 Jos 3:4}
\crossref{Exod}{19}{13}{Ex 21:28,\allowbreak29 Le 20:15,\allowbreak16}
\crossref{Exod}{19}{14}{19:10}
\crossref{Exod}{19}{15}{Am 4:12 Mal 3:2 Mt 3:10-\allowbreak12; 24:44 2Pe 3:11,\allowbreak12}
\crossref{Exod}{19}{16}{Ex 9:23,\allowbreak28,\allowbreak29; 20:18 1Sa 12:17,\allowbreak18 Job 37:1-\allowbreak5; 38:25 Ps 18:11-\allowbreak14}
\crossref{Exod}{19}{17}{De 4:10; 5:5}
\crossref{Exod}{19}{18}{Ex 20:18 De 4:11,\allowbreak12; 5:22; 33:2 Jud 5:5 Ps 68:7,\allowbreak8; 104:32; 144:5}
\crossref{Exod}{19}{19}{19:13,\allowbreak16}
\crossref{Exod}{19}{20}{19:11 Ne 9:13 Ps 81:7}
\crossref{Exod}{19}{21}{19:12,\allowbreak13}
\crossref{Exod}{19}{22}{Ex 24:5 Le 10:1-\allowbreak3 Isa 52:11}
\crossref{Exod}{19}{23}{19:12 Jos 3:4,\allowbreak5}
\crossref{Exod}{19}{24}{19:20}
\crossref{Exod}{19}{25}{19:24}
\crossref{Exod}{20}{1}{De 4:33,\allowbreak36; 5:4,\allowbreak22 Ac 7:38,\allowbreak53}
\crossref{Exod}{20}{2}{Ge 17:7,\allowbreak8 Le 26:1,\allowbreak13 De 5:6; 6:4,\allowbreak5 2Ch 28:5 Ps 50:7; 81:10}
\crossref{Exod}{20}{3}{Ex 15:11 De 5:7; 6:5,\allowbreak14 Jos 24:18-\allowbreak24 2Ki 17:29-\allowbreak35 Ps 29:2; 73:25}
\crossref{Exod}{20}{4}{Ex 32:1,\allowbreak8,\allowbreak23; 34:17 Le 19:4; 26:1 De 4:15-\allowbreak19,\allowbreak23-\allowbreak25; 5:8; 27:15}
\crossref{Exod}{20}{5}{Ex 23:24 Le 26:1 Jos 23:7,\allowbreak16 Jud 2:19 2Ki 17:35,\allowbreak41 2Ch 25:14}
\crossref{Exod}{20}{6}{De 4:37; 5:29; 7:9 Jer 32:39,\allowbreak40 Ac 2:39 Ro 11:28,\allowbreak29}
\crossref{Exod}{20}{7}{Le 19:12; 24:11-\allowbreak16 De 5:11 Ps 50:14-\allowbreak16 Pr 30:9 Jer 4:2}
\crossref{Exod}{20}{8}{Ex 16:23-\allowbreak30; 31:13,\allowbreak14 Ge 2:3 Le 19:3; 23:3 Isa 56:4-\allowbreak6}
\crossref{Exod}{20}{9}{Ex 23:12 Lu 13:14}
\crossref{Exod}{20}{10}{Ex 31:13; 34:21}
\crossref{Exod}{20}{11}{Ex 31:17 Ge 2:2,\allowbreak3 Ps 95:4-\allowbreak7 Mr 2:27,\allowbreak28 Ac 20:7}
\crossref{Exod}{20}{12}{Ex 21:15,\allowbreak17 Le 19:3,\allowbreak32 1Ki 2:19 2Ki 2:12 Pr 1:8,\allowbreak9; 15:5; 20:20}
\crossref{Exod}{20}{13}{Ex 21:14,\allowbreak20,\allowbreak29 Ge 4:8-\allowbreak23; 9:5,\allowbreak6 Le 24:21 Nu 35:16-\allowbreak34 De 5:17}
\crossref{Exod}{20}{14}{Le 18:20; 20:10 2Sa 11:4,\allowbreak5,\allowbreak27 Pr 2:15-\allowbreak18; 6:24-\allowbreak35; 7:18-\allowbreak27}
\crossref{Exod}{20}{15}{Ex 21:16 Le 6:1-\allowbreak7; 19:11,\allowbreak13,\allowbreak35-\allowbreak37 De 24:7; 25:13-\allowbreak16 Job 20:19-\allowbreak22}
\crossref{Exod}{20}{16}{Ex 23:6,\allowbreak7 Le 19:11,\allowbreak16 De 19:15-\allowbreak21 1Sa 22:8-\allowbreak19 1Ki 21:10-\allowbreak13}
\crossref{Exod}{20}{17}{Ge 3:6; 14:23; 34:23 Jos 7:21 1Sa 15:19 Ps 10:3; 119:36}
\crossref{Exod}{20}{18}{Ex 19:16-\allowbreak18}
\crossref{Exod}{20}{19}{De 18:16 Ac 7:38}
\crossref{Exod}{20}{20}{1Sa 12:20 Isa 41:10}
\crossref{Exod}{20}{21}{Ex 19:16,\allowbreak17 De 5:5}
\crossref{Exod}{20}{22}{De 4:36 Ne 9:13 Heb 12:25,\allowbreak26}
\crossref{Exod}{20}{23}{20:3-\allowbreak5}
\crossref{Exod}{20}{24}{Joh 4:24}
\crossref{Exod}{20}{25}{De 27:5,\allowbreak6 Jos 8:31}
\crossref{Exod}{20}{26}{Le 10:3 Ps 89:7 Ec 5:1 Heb 12:28,\allowbreak29 1Pe 1:16}
\crossref{Exod}{21}{1}{Le 18:5,\allowbreak26; 19:37; 20:22 Nu 35:24; 36:13 De 5:1,\allowbreak31; 6:20 1Ki 6:12}
\crossref{Exod}{21}{2}{Ex 12:44; 22:3 Ge 27:28,\allowbreak36 Le 25:39-\allowbreak41,\allowbreak44 2Ki 4:1 Ne 5:1-\allowbreak5,\allowbreak8}
\crossref{Exod}{21}{3}{De 15:12-\allowbreak14}
\crossref{Exod}{21}{4}{Ex 4:22 Ge 14:14; 15:3; 17:13,\allowbreak27; 18:19 Ec 2:7 Jer 2:14}
\crossref{Exod}{21}{5}{De 15:16,\allowbreak17 Isa 26:13 2Co 5:14,\allowbreak15}
\crossref{Exod}{21}{6}{21:22; 12:12; 18:21-\allowbreak26; 22:8,\allowbreak9,\allowbreak28 Nu 25:5-\allowbreak8 De 1:16; 16:18}
\crossref{Exod}{21}{7}{Ne 5:5}
\crossref{Exod}{21}{8}{Ge 28:8 Jud 14:3 1Sa 8:6; 18:8}
\crossref{Exod}{21}{9}{Ex 22:17 Ge 38:11 Le 22:13}
\crossref{Exod}{21}{10}{}
\crossref{Exod}{21}{11}{21:2}
\crossref{Exod}{21}{12}{Ex 20:13 Ge 9:6 Le 24:17 Nu 35:16-\allowbreak24,\allowbreak30,\allowbreak31 De 19:11-\allowbreak13}
\crossref{Exod}{21}{13}{Nu 35:11,\allowbreak22 De 19:4-\allowbreak6,\allowbreak11 Mic 7:2}
\crossref{Exod}{21}{14}{Nu 15:30,\allowbreak31 De 1:43; 17:12,\allowbreak13; 18:22; 19:11-\allowbreak13 1Ki 2:29-\allowbreak34}
\crossref{Exod}{21}{15}{}
\crossref{Exod}{21}{16}{Ge 40:15 De 24:7 1Ti 1:10 Re 18:12}
\crossref{Exod}{21}{17}{Le 20:9 De 27:16 Pr 20:20; 30:11,\allowbreak17 Mt 15:3-\allowbreak6 Mr 7:10,\allowbreak11}
\crossref{Exod}{21}{18}{21:22; 2:13 De 25:11 2Sa 14:6}
\crossref{Exod}{21}{19}{2Sa 3:29 Zec 8:4}
\crossref{Exod}{21}{20}{21:26,\allowbreak27 De 19:21 Pr 29:19 Isa 58:3,\allowbreak4}
\crossref{Exod}{21}{21}{Le 25:45,\allowbreak46}
\crossref{Exod}{21}{22}{21:18}
\crossref{Exod}{21}{23}{Nu 35:31}
\crossref{Exod}{21}{24}{}
\crossref{Exod}{21}{25}{Pr 6:28 Isa 43:2}
\crossref{Exod}{21}{26}{21:20 De 16:19 Ne 5:5 Job 31:13-\allowbreak15 Ps 9:12; 10:14,\allowbreak18; 72:12-\allowbreak14}
\crossref{Exod}{21}{27}{21:19}
\crossref{Exod}{21}{28}{21:32 Ge 9:5,\allowbreak6 Le 20:15,\allowbreak16}
\crossref{Exod}{21}{29}{De 21:1-\allowbreak9}
\crossref{Exod}{21}{30}{21:22; 30:12 Nu 35:31-\allowbreak33 Pr 13:8}
\crossref{Exod}{21}{31}{21:31}
\crossref{Exod}{21}{32}{Ge 37:28 Zec 11:12,\allowbreak13 Mt 26:15; 27:3-\allowbreak9 Php 2:7}
\crossref{Exod}{21}{33}{Ps 9:15; 119:85 Pr 28:10 Ec 10:8 Jer 18:20,\allowbreak22}
\crossref{Exod}{21}{34}{21:29,\allowbreak30; 22:6,\allowbreak14}
\crossref{Exod}{21}{35}{21:35}
\crossref{Exod}{21}{36}{21:29}
\crossref{Exod}{22}{1}{Pr 14:4}
\crossref{Exod}{22}{2}{Job 24:14; 30:5 Ho 7:1 Joe 2:9 Mt 6:19,\allowbreak20; 24:43 1Th 5:2}
\crossref{Exod}{22}{3}{Ex 21:2 Jud 2:14; 10:7 Isa 50:1}
\crossref{Exod}{22}{4}{Ex 21:16}
\crossref{Exod}{22}{5}{22:3,\allowbreak12; 21:34 Job 20:18}
\crossref{Exod}{22}{6}{Jud 15:4,\allowbreak5 2Sa 14:30,\allowbreak31}
\crossref{Exod}{22}{7}{Pr 6:30,\allowbreak31 Jer 2:26 Joh 12:6 1Co 6:10}
\crossref{Exod}{22}{8}{22:28}
\crossref{Exod}{22}{9}{Nu 5:6,\allowbreak7 1Ki 8:31 Mt 6:14,\allowbreak15; 18:15,\allowbreak35 Lu 17:3,\allowbreak4}
\crossref{Exod}{22}{10}{Ge 39:8 Lu 12:48; 16:11 2Ti 1:12}
\crossref{Exod}{22}{11}{Le 5:1; 6:3 1Ki 2:42,\allowbreak43 Pr 30:9 Heb 6:16}
\crossref{Exod}{22}{12}{22:7 Ge 31:39}
\crossref{Exod}{22}{13}{Eze 4:14 Am 3:12 Mic 5:8 Na 2:12}
\crossref{Exod}{22}{14}{De 15:2; 23:19,\allowbreak20 Ne 5:4 Ps 37:21 Mt 5:42 Lu 6:35}
\crossref{Exod}{22}{15}{Zec 8:10}
\crossref{Exod}{22}{16}{Ge 34:2-\allowbreak4 De 22:28,\allowbreak29}
\crossref{Exod}{22}{17}{De 7:3,\allowbreak4}
\crossref{Exod}{22}{18}{Le 19:26,\allowbreak31; 20:6,\allowbreak27 De 18:10,\allowbreak11 1Sa 28:3,\allowbreak9 Isa 19:3 Ac 8:9-\allowbreak11}
\crossref{Exod}{22}{19}{Le 18:23,\allowbreak25; 20:15,\allowbreak16 De 27:21}
\crossref{Exod}{22}{20}{Nu 25:2-\allowbreak4,\allowbreak7,\allowbreak8 De 13:1-\allowbreak15; 17:2-\allowbreak5; 18:20}
\crossref{Exod}{22}{21}{Ex 23:9 Le 19:33; 25:35 De 10:19 Jer 7:6; 22:3 Zec 7:10 Mal 3:5}
\crossref{Exod}{22}{22}{De 10:18; 24:17; 27:19 Ps 94:6,\allowbreak7 Isa 1:17,\allowbreak23; 10:2 Eze 22:7}
\crossref{Exod}{22}{23}{De 15:9; 24:15 Job 31:38,\allowbreak39; 35:9 Lu 18:7}
\crossref{Exod}{22}{24}{Job 31:23 Ps 69:24; 76:7; 90:11 Na 1:6 Ro 2:5-\allowbreak9 Heb 10:31}
\crossref{Exod}{22}{25}{Le 25:35-\allowbreak37 De 23:19,\allowbreak20 2Ki 4:1,\allowbreak7 Ne 5:2-\allowbreak5,\allowbreak7,\allowbreak10,\allowbreak11 Ps 15:5}
\crossref{Exod}{22}{26}{De 24:6,\allowbreak10-\allowbreak13,\allowbreak17 Job 22:6; 24:3,\allowbreak9 Pr 20:16; 22:27 Eze 18:7,\allowbreak16}
\crossref{Exod}{22}{27}{Ex 2:23,\allowbreak24 Ps 34:6; 72:12 Isa 19:20}
\crossref{Exod}{22}{28}{22:8,\allowbreak9 Ps 32:6; 82:1-\allowbreak7; 138:1 Joh 10:34,\allowbreak35}
\crossref{Exod}{22}{29}{Ex 23:16,\allowbreak19 De 26:2-\allowbreak10 2Ki 4:42 2Ch 31:5 Pr 3:9,\allowbreak10 Eze 20:40}
\crossref{Exod}{22}{30}{De 15:19}
\crossref{Exod}{22}{31}{Ex 19:5,\allowbreak6 Le 11:45; 19:2 De 14:21 1Pe 1:15,\allowbreak16}
\crossref{Exod}{23}{1}{23:7; 20:16 Le 19:16 2Sa 16:3; 19:27 Ps 15:3; 101:5; 120:3}
\crossref{Exod}{23}{2}{Ex 32:1-\allowbreak5 Ge 6:12; 7:1; 19:4,\allowbreak7-\allowbreak9 Nu 14:1-\allowbreak10 Jos 24:15 1Sa 15:9}
\crossref{Exod}{23}{3}{Ps 82:2,\allowbreak3 Jas 3:17}
\crossref{Exod}{23}{4}{De 22:1-\allowbreak4 Job 31:29,\allowbreak30 Pr 24:17,\allowbreak18; 25:21 Mt 5:44 Lu 6:27,\allowbreak28}
\crossref{Exod}{23}{5}{De 22:4}
\crossref{Exod}{23}{6}{23:2,\allowbreak3 De 16:19; 27:19 2Ch 19:7 Job 31:13,\allowbreak21,\allowbreak22 Ps 82:3,\allowbreak4 Ec 5:8}
\crossref{Exod}{23}{7}{23:1 Le 19:11 De 19:16-\allowbreak21 Job 22:23 Pr 4:14,\allowbreak15 Isa 33:15 Lu 3:14}
\crossref{Exod}{23}{8}{De 16:19 1Sa 8:3; 12:3 Ps 26:10 Pr 15:27; 17:8,\allowbreak23; 19:4 Ec 7:7}
\crossref{Exod}{23}{9}{Ex 21:21 De 10:19; 24:14-\allowbreak18; 27:19 Ps 94:6 Eze 22:7}
\crossref{Exod}{23}{10}{Le 25:3,\allowbreak4 Ne 10:31}
\crossref{Exod}{23}{11}{Le 25:2-\allowbreak7,\allowbreak11,\allowbreak12,\allowbreak20,\allowbreak22; 26:34,\allowbreak35}
\crossref{Exod}{23}{12}{Ex 20:8-\allowbreak11; 31:15,\allowbreak16 Lu 13:14}
\crossref{Exod}{23}{13}{De 4:9,\allowbreak15 Jos 22:5; 23:11 1Ch 28:7-\allowbreak9 Ps 39:1 Eph 5:15 1Ti 4:16}
\crossref{Exod}{23}{14}{Ex 34:22 Le 23:5,\allowbreak16,\allowbreak34 De 16:16}
\crossref{Exod}{23}{15}{Ex 12:14-\allowbreak28,\allowbreak43-\allowbreak49; 13:6,\allowbreak7; 34:18 Le 23:5-\allowbreak8 Nu 9:2-\allowbreak14; 28:16-\allowbreak25}
\crossref{Exod}{23}{16}{Ex 22:29; 34:22 Le 23:9-\allowbreak21 Nu 28:26-\allowbreak31 De 16:9-\allowbreak12 Ac 2:1}
\crossref{Exod}{23}{17}{Ex 34:23 De 12:5; 16:16; 31:11 Ps 84:7 Lu 2:42}
\crossref{Exod}{23}{18}{Ex 12:8,\allowbreak15; 34:25 Le 2:11; 7:12 De 16:4}
\crossref{Exod}{23}{19}{Ex 22:29; 34:26 Le 23:10-\allowbreak17 Nu 18:12,\allowbreak13 De 12:5-\allowbreak7; 26:10 Ne 10:35}
\crossref{Exod}{23}{20}{Ex 3:2-\allowbreak6; 14:19; 32:34; 33:2,\allowbreak14 Ge 48:16 Nu 20:16 Jos 5:13; 6:2}
\crossref{Exod}{23}{21}{Ps 2:12 Mt 17:5 Heb 12:25}
\crossref{Exod}{23}{22}{Ge 12:3 Nu 24:9 De 30:7 Jer 30:20 Zec 2:8 Ac 9:4,\allowbreak5}
\crossref{Exod}{23}{23}{23:20; 32:2 Isa 5:13}
\crossref{Exod}{23}{24}{Ex 20:5}
\crossref{Exod}{23}{25}{De 6:13; 10:12,\allowbreak20; 11:13,\allowbreak14; 13:4; 28:1-\allowbreak6 Jos 22:5; 24:14,\allowbreak15,\allowbreak21,\allowbreak24}
\crossref{Exod}{23}{26}{De 7:14; 28:4 Job 21:10 Ps 107:38; 144:13 Mal 3:10,\allowbreak11}
\crossref{Exod}{23}{27}{Ex 15:14-\allowbreak16 Ge 35:5 De 2:25; 11:23,\allowbreak25 Jos 2:9-\allowbreak11 1Sa 14:15}
\crossref{Exod}{23}{28}{}
\crossref{Exod}{23}{29}{De 7:22 Jos 15:63; 16:10; 17:12,\allowbreak13 Jud 3:1-\allowbreak4}
\crossref{Exod}{23}{30}{Ge 6:17 De 7:22 Php 3:12-\allowbreak14 2Pe 3:18}
\crossref{Exod}{23}{31}{Ge 15:18 Nu 34:3-\allowbreak15 De 11:24 Jos 1:4 1Ki 4:21,\allowbreak24 Ps 72:8}
\crossref{Exod}{23}{32}{Ex 34:12,\allowbreak15 De 7:2 Jos 9:14-\allowbreak23 2Sa 21:1,\allowbreak2 Ps 106:35 2Co 6:15}
\crossref{Exod}{23}{33}{1Ki 14:16 2Ch 33:9}
\crossref{Exod}{24}{1}{24:15; 3:5; 19:9,\allowbreak20,\allowbreak24; 20:21; 34:2}
\crossref{Exod}{24}{2}{24:13,\allowbreak15,\allowbreak18; 20:21 Nu 16:5 Jer 30:21; 49:19 Heb 9:24; 10:21,\allowbreak22}
\crossref{Exod}{24}{3}{Ex 21:1-\allowbreak23:33 De 4:1,\allowbreak5,\allowbreak45; 5:1,\allowbreak31; 6:1; 11:1}
\crossref{Exod}{24}{4}{De 31:9 Jos 24:26}
\crossref{Exod}{24}{5}{Ex 19:22}
\crossref{Exod}{24}{6}{24:8; 12:7,\allowbreak22 Col 1:20 Heb 9:18; 12:24 1Pe 1:2,\allowbreak19}
\crossref{Exod}{24}{7}{24:4 Heb 9:18-\allowbreak23}
\crossref{Exod}{24}{8}{24:6 Le 8:30 Isa 52:15 Eze 36:25 Heb 9:18-\allowbreak21}
\crossref{Exod}{24}{9}{24:1}
\crossref{Exod}{24}{10}{24:10; 3:6; 33:20,\allowbreak23 Ge 32:30 Jud 13:21,\allowbreak22 1Ki 22:19 Isa 6:1-\allowbreak5}
\crossref{Exod}{24}{11}{24:1,\allowbreak9 Nu 21:18 Jud 5:13 1Ki 21:8 2Ch 23:20 Ne 2:16 Jer 14:3}
\crossref{Exod}{24}{12}{24:2,\allowbreak15,\allowbreak18}
\crossref{Exod}{24}{13}{Ex 17:9-\allowbreak14; 32:17; 33:11 Nu 11:28}
\crossref{Exod}{24}{14}{Ex 32:1 Ge 22:5 1Sa 10:8; 13:8-\allowbreak13}
\crossref{Exod}{24}{15}{Ex 19:9,\allowbreak16 2Ch 6:1 Mt 17:5}
\crossref{Exod}{24}{16}{24:17; 16:10 Le 9:23 Nu 14:10; 16:42 Eze 1:28 2Co 4:6}
\crossref{Exod}{24}{17}{Ex 3:2; 19:18 De 4:24,\allowbreak36 Eze 1:27 Na 1:6 Hab 3:4,\allowbreak5 Heb 12:18,\allowbreak29}
\crossref{Exod}{24}{18}{24:17; 9:29,\allowbreak33; 19:20 Pr 28:1}
\crossref{Exod}{25}{1}{25:1}
\crossref{Exod}{25}{2}{Ex 35:5-\allowbreak29 Nu 7:3-\allowbreak88 De 16:16,\allowbreak17 1Ch 29:1-\allowbreak30}
\crossref{Exod}{25}{3}{}
\crossref{Exod}{25}{4}{Ge 41:42 Eze 16:10 Re 19:8}
\crossref{Exod}{25}{5}{Ex 26:14}
\crossref{Exod}{25}{6}{25:37; 27:20; 40:24,\allowbreak25}
\crossref{Exod}{25}{7}{Ex 28:9-\allowbreak21}
\crossref{Exod}{25}{8}{Ex 15:2; 36:1-\allowbreak4 Le 4:6; 10:4; 21:12 Heb 9:1,\allowbreak2}
\crossref{Exod}{25}{9}{25:40 1Ch 28:11-\allowbreak19 Heb 8:5; 9:9}
\crossref{Exod}{25}{10}{}
\crossref{Exod}{25}{11}{25:24; 30:3 1Ki 6:20 2Ch 3:4}
\crossref{Exod}{25}{12}{25:15,\allowbreak26; 26:29; 27:7; 37:5; 38:7}
\crossref{Exod}{25}{13}{25:28; 27:6; 30:5; 37:4; 40:20 Nu 4:6,\allowbreak8,\allowbreak11,\allowbreak14 1Ch 15:15}
\crossref{Exod}{25}{14}{Job 17:16}
\crossref{Exod}{25}{15}{1Ki 8:8 2Ch 5:9}
\crossref{Exod}{25}{16}{Ex 16:34; 27:21; 30:6,\allowbreak36; 31:18; 32:15; 34:29; 38:21 Nu 17:4}
\crossref{Exod}{25}{17}{Ex 26:34; 37:6; 40:20 Le 16:12-\allowbreak15 1Ch 28:11 Ro 3:25 Heb 4:16; 9:5}
\crossref{Exod}{25}{18}{Ex 37:7-\allowbreak9 Ge 3:24 1Sa 4:4 1Ki 6:23-\allowbreak28; 8:6,\allowbreak7 1Ch 28:18}
\crossref{Exod}{25}{19}{}
\crossref{Exod}{25}{20}{25:18 1Ki 8:7 1Ch 28:18 2Ch 3:10}
\crossref{Exod}{25}{21}{25:17; 26:34 Ro 10:4}
\crossref{Exod}{25}{22}{Ex 20:24; 30:6,\allowbreak36; 31:18 Ge 18:33 Le 1:1; 16:2 Nu 7:89; 17:4}
\crossref{Exod}{25}{23}{Ex 37:10-\allowbreak16; 40:22,\allowbreak23 Le 24:6 Nu 3:31 1Ki 7:48 1Ch 28:16}
\crossref{Exod}{25}{24}{25:11 1Ki 6:20-\allowbreak22}
\crossref{Exod}{25}{25}{Ex 30:3; 37:2}
\crossref{Exod}{25}{26}{25:12}
\crossref{Exod}{25}{27}{25:14,\allowbreak28}
\crossref{Exod}{25}{28}{25:14,\allowbreak27 Nu 10:17 Ac 9:15}
\crossref{Exod}{25}{29}{Ex 37:16 Nu 4:7; 7:13,\allowbreak19,\allowbreak31-\allowbreak33 1Ki 7:50 2Ch 4:22 Ezr 1:9-\allowbreak11}
\crossref{Exod}{25}{30}{Ex 35:13; 39:36 Le 24:5,\allowbreak6 Nu 4:7 1Sa 21:6 1Ch 9:32; 23:29}
\crossref{Exod}{25}{31}{Ex 35:14; 37:17-\allowbreak24; 40:24,\allowbreak25 1Ki 7:49 2Ch 13:11 Zec 4:2 Heb 9:2}
\crossref{Exod}{25}{32}{}
\crossref{Exod}{25}{33}{Nu 17:4-\allowbreak8 Jer 1:11,\allowbreak12}
\crossref{Exod}{25}{34}{25:34}
\crossref{Exod}{25}{35}{}
\crossref{Exod}{25}{36}{25:18 Nu 8:4 1Ki 10:16,\allowbreak17 2Ch 9:15}
\crossref{Exod}{25}{37}{Ex 37:23 Zec 4:2 Re 1:4,\allowbreak12,\allowbreak20; 2:1; 4:5}
\crossref{Exod}{25}{38}{2Ch 4:21 Isa 6:6}
\crossref{Exod}{25}{39}{Ex 37:24 Zec 5:7}
\crossref{Exod}{25}{40}{Ex 26:30; 39:42,\allowbreak43 Nu 8:4 1Ch 28:11,\allowbreak19 Eze 43:11,\allowbreak12 Ac 7:44}
\crossref{Exod}{26}{1}{26:36; 25:4; 35:6,\allowbreak35 Re 19:8}
\crossref{Exod}{26}{2}{26:7,\allowbreak8 Nu 4:25 2Sa 7:2 1Ch 17:1}
\crossref{Exod}{26}{3}{26:9; 36:10 Joh 17:21 1Co 12:4,\allowbreak12-\allowbreak27 Eph 2:21,\allowbreak22; 4:3-\allowbreak6,\allowbreak16}
\crossref{Exod}{26}{4}{26:5,\allowbreak10,\allowbreak11; 36:11,\allowbreak12,\allowbreak17}
\crossref{Exod}{26}{5}{}
\crossref{Exod}{26}{6}{26:11,\allowbreak33; 35:11; 36:13,\allowbreak18; 39:33}
\crossref{Exod}{26}{7}{Ex 35:26; 36:14-\allowbreak18 Nu 4:25 Ps 45:13 1Pe 3:4; 5:5}
\crossref{Exod}{26}{8}{26:2,\allowbreak13}
\crossref{Exod}{26}{9}{26:3}
\crossref{Exod}{26}{10}{26:4-\allowbreak6}
\crossref{Exod}{26}{11}{26:3,\allowbreak6}
\crossref{Exod}{26}{12}{26:9}
\crossref{Exod}{26}{13}{26:2,\allowbreak8}
\crossref{Exod}{26}{14}{Ex 36:19 Nu 4:5 Ps 27:5; 121:4,\allowbreak5 Isa 4:6; 25:4}
\crossref{Exod}{26}{15}{26:18,\allowbreak22-\allowbreak29; 36:20-\allowbreak33; 40:17,\allowbreak18 Nu 4:31,\allowbreak32 Eph 2:20,\allowbreak21}
\crossref{Exod}{26}{16}{}
\crossref{Exod}{26}{17}{26:19; 36:22,\allowbreak24}
\crossref{Exod}{26}{18}{}
\crossref{Exod}{26}{19}{26:25,\allowbreak37; 27:10,\allowbreak12-\allowbreak18; 36:24-\allowbreak26; 38:27,\allowbreak30,\allowbreak31; 40:18 Nu 3:36}
\crossref{Exod}{26}{20}{}
\crossref{Exod}{26}{21}{26:19}
\crossref{Exod}{26}{22}{}
\crossref{Exod}{26}{23}{Ex 36:28}
\crossref{Exod}{26}{24}{}
\crossref{Exod}{26}{25}{}
\crossref{Exod}{26}{26}{Ex 36:31-\allowbreak38 Nu 3:36; 4:31 Ro 15:1 1Co 9:19,\allowbreak20 Ga 6:1,\allowbreak2 Eph 4:16}
\crossref{Exod}{26}{27}{}
\crossref{Exod}{26}{28}{1Ch 12:15 Ne 13:28 Job 41:28 Pr 19:26}
\crossref{Exod}{26}{29}{Ex 25:11,\allowbreak12}
\crossref{Exod}{26}{30}{Ex 40:2,\allowbreak17,\allowbreak18 Nu 10:21 Jos 18:1 Heb 8:2}
\crossref{Exod}{26}{31}{Ex 36:35; 40:3,\allowbreak21 Le 16:2,\allowbreak15 2Ch 3:14 Mt 27:51 Mr 15:38 Lu 23:45}
\crossref{Exod}{26}{32}{26:37; 36:38 Es 1:6}
\crossref{Exod}{26}{33}{Ex 27:10; 36:36}
\crossref{Exod}{26}{34}{Ex 25:21; 40:20 Heb 9:5}
\crossref{Exod}{26}{35}{Ex 40:22 Heb 9:2,\allowbreak8,\allowbreak9}
\crossref{Exod}{26}{36}{Ex 35:11; 39:33; 40:29 Nu 3:25; 9:15 2Sa 7:6 Ps 78:60}
\crossref{Exod}{26}{37}{Ex 36:38}
\crossref{Exod}{27}{1}{Ex 20:24-\allowbreak26; 24:4; 38:1-\allowbreak7; 40:10,\allowbreak29 2Sa 24:18 2Ch 4:1 Eze 43:13-\allowbreak17}
\crossref{Exod}{27}{2}{Nu 16:38,\allowbreak39 1Ki 8:64}
\crossref{Exod}{27}{3}{Le 16:12 1Ki 7:40,\allowbreak45 2Ch 4:11 Jer 52:18}
\crossref{Exod}{27}{4}{Ex 35:16; 38:4,\allowbreak5}
\crossref{Exod}{27}{5}{Ex 38:4}
\crossref{Exod}{27}{6}{Ex 35:13-\allowbreak15; 30:4 Nu 4:44}
\crossref{Exod}{27}{7}{Ex 25:28; 30:4 Nu 4:13,\allowbreak14}
\crossref{Exod}{27}{8}{Ex 25:9,\allowbreak40; 26:30-\allowbreak37 1Ch 28:11,\allowbreak19 Mt 15:9 Col 2:20-\allowbreak23 Heb 8:5}
\crossref{Exod}{27}{9}{Ex 38:9-\allowbreak20; 40:8 1Ki 6:36; 8:64 2Ch 33:5 Ps 84:10; 92:13; 100:4}
\crossref{Exod}{27}{10}{Ex 26:19-\allowbreak21}
\crossref{Exod}{27}{11}{27:11}
\crossref{Exod}{27}{12}{27:12}
\crossref{Exod}{27}{13}{}
\crossref{Exod}{27}{14}{27:9; 26:36}
\crossref{Exod}{27}{15}{}
\crossref{Exod}{27}{16}{Ex 26:31,\allowbreak36}
\crossref{Exod}{27}{17}{}
\crossref{Exod}{27}{18}{27:9-\allowbreak12}
\crossref{Exod}{27}{19}{27:3; 35:18; 38:20,\allowbreak31; 39:40 Nu 3:37; 4:32 Ezr 9:8 Ec 12:11}
\crossref{Exod}{27}{20}{Ex 25:31-\allowbreak37}
\crossref{Exod}{27}{21}{Ex 29:10,\allowbreak44 Le 3:8 Nu 8:9}
\crossref{Exod}{28}{1}{Le 8:2 Nu 16:9-\allowbreak11; 17:2-\allowbreak9 2Ch 26:18-\allowbreak21 Heb 5:1-\allowbreak5}
\crossref{Exod}{28}{2}{Ex 29:5-\allowbreak9,\allowbreak29,\allowbreak30; 31:10; 39:1,\allowbreak2; 40:13 Le 8:7-\allowbreak9,\allowbreak30 Nu 20:26-\allowbreak28}
\crossref{Exod}{28}{3}{Ex 31:3-\allowbreak6; 35:30,\allowbreak35; 36:1,\allowbreak2 Pr 2:6 Isa 28:24-\allowbreak26}
\crossref{Exod}{28}{4}{}
\crossref{Exod}{28}{5}{Ex 25:3,\allowbreak4; 39:2,\allowbreak3}
\crossref{Exod}{28}{6}{Ex 26:1}
\crossref{Exod}{28}{7}{Ex 39:4}
\crossref{Exod}{28}{8}{28:27,\allowbreak28; 29:5; 39:20,\allowbreak21 Le 8:7 Isa 11:5 1Pe 1:13 Re 1:13}
\crossref{Exod}{28}{9}{28:20; 39:13 Ge 2:12 Job 28:16 Eze 28:13}
\crossref{Exod}{28}{10}{Ex 1:1-\allowbreak4 Ge 43:33}
\crossref{Exod}{28}{11}{28:21,\allowbreak36 Jer 22:24 Zec 3:9 Eph 1:13; 4:30 2Ti 2:19 Re 7:2}
\crossref{Exod}{28}{12}{28:7 Ps 89:19 Isa 9:6; 12:2 Zec 6:13,\allowbreak14 Heb 7:25-\allowbreak28}
\crossref{Exod}{28}{13}{}
\crossref{Exod}{28}{14}{28:24; 39:15}
\crossref{Exod}{28}{15}{28:4,\allowbreak30; 39:8 Le 8:8}
\crossref{Exod}{28}{16}{}
\crossref{Exod}{28}{17}{28:9,\allowbreak11; 39:10-\allowbreak21 Mal 3:17}
\crossref{Exod}{28}{18}{Ex 24:10 Job 28:6,\allowbreak16 So 5:14 Eze 1:26; 10:1 Re 4:3}
\crossref{Exod}{28}{19}{Ex 39:12}
\crossref{Exod}{28}{20}{Eze 1:16; 10:9 Da 10:6 Re 21:20}
\crossref{Exod}{28}{21}{28:9-\allowbreak11}
\crossref{Exod}{28}{22}{28:14}
\crossref{Exod}{28}{23}{Ex 25:11-\allowbreak15}
\crossref{Exod}{28}{24}{}
\crossref{Exod}{28}{25}{28:14; 39:15}
\crossref{Exod}{28}{26}{}
\crossref{Exod}{28}{27}{28:8}
\crossref{Exod}{28}{28}{28:31,\allowbreak37; 39:30,\allowbreak31 Nu 15:38}
\crossref{Exod}{28}{29}{28:15,\allowbreak30}
\crossref{Exod}{28}{30}{Zec 6:13}
\crossref{Exod}{28}{31}{28:4,\allowbreak28; 39:22 Le 8:7}
\crossref{Exod}{28}{32}{Ex 39:28 2Ch 26:14 Ne 4:16 Job 41:26}
\crossref{Exod}{28}{33}{Ex 39:24-\allowbreak26}
\crossref{Exod}{28}{34}{Ps 89:15 So 2:3; 4:3,\allowbreak13; 6:7,\allowbreak11; 8:2 Joh 15:4-\allowbreak8,\allowbreak16 Col 1:5,\allowbreak6,\allowbreak10}
\crossref{Exod}{28}{35}{Le 16:2 Heb 9:12}
\crossref{Exod}{28}{36}{28:9,\allowbreak11}
\crossref{Exod}{28}{37}{28:28,\allowbreak31 Nu 15:38}
\crossref{Exod}{28}{38}{28:43 Le 10:17; 22:9 Nu 18:1 Isa 53:6,\allowbreak11,\allowbreak12 Eze 4:4-\allowbreak6 Joh 1:29}
\crossref{Exod}{28}{39}{28:4}
\crossref{Exod}{28}{40}{28:4; 39:27,\allowbreak29,\allowbreak41 Le 8:13 Eze 44:17,\allowbreak18}
\crossref{Exod}{28}{41}{Ex 29:7; 30:23-\allowbreak30; 40:15 Le 10:7 Isa 10:27; 61:1 Joh 3:34}
\crossref{Exod}{28}{42}{Ex 20:26; 39:28 Le 6:10; 16:4 Eze 44:18 Re 3:18}
\crossref{Exod}{28}{43}{Ex 20:26}
\crossref{Exod}{29}{1}{29:21; 20:11; 28:41 Le 8:2-\allowbreak36 Mt 6:9}
\crossref{Exod}{29}{2}{Ex 12:8 Le 2:4; 6:20-\allowbreak22; 8:2 1Co 5:7}
\crossref{Exod}{29}{3}{Le 8:2,\allowbreak26,\allowbreak31 Nu 6:17}
\crossref{Exod}{29}{4}{Ex 26:36; 40:28 Le 8:3-\allowbreak6}
\crossref{Exod}{29}{5}{Ex 28:2-\allowbreak8 Le 8:7,\allowbreak8}
\crossref{Exod}{29}{6}{Ex 28:36-\allowbreak39 Le 8:9}
\crossref{Exod}{29}{7}{Ex 28:41; 30:23-\allowbreak31 Le 8:10-\allowbreak12; 10:7; 21:10 Nu 35:25 Ps 89:20; 133:2}
\crossref{Exod}{29}{8}{Ex 28:40 Le 8:13}
\crossref{Exod}{29}{9}{Ex 28:1 Nu 16:10,\allowbreak35,\allowbreak40; 18:7 Heb 5:4,\allowbreak5,\allowbreak10; 7:11-\allowbreak14}
\crossref{Exod}{29}{10}{29:1}
\crossref{Exod}{29}{11}{Le 1:4,\allowbreak5; 8:15; 9:8,\allowbreak12}
\crossref{Exod}{29}{12}{Le 8:15; 9:9; 16:14,\allowbreak18,\allowbreak19 Heb 9:13,\allowbreak14,\allowbreak22; 10:4}
\crossref{Exod}{29}{13}{29:22 Le 3:3,\allowbreak4,\allowbreak9,\allowbreak10,\allowbreak14-\allowbreak16; 4:8,\allowbreak9,\allowbreak26,\allowbreak31,\allowbreak35; 6:12; 7:3,\allowbreak31 Ps 22:14}
\crossref{Exod}{29}{14}{Le 4:11,\allowbreak12,\allowbreak21; 8:17; 16:27 Heb 13:11-\allowbreak13}
\crossref{Exod}{29}{15}{29:3,\allowbreak19 Le 8:18-\allowbreak21}
\crossref{Exod}{29}{16}{29:11,\allowbreak12}
\crossref{Exod}{29}{17}{Le 1:9,\allowbreak13; 8:21; 9:14 Jer 4:14 Mt 23:26}
\crossref{Exod}{29}{18}{Ge 22:2,\allowbreak7,\allowbreak13 Le 9:24 1Sa 7:9 1Ki 3:4; 18:38 Ps 50:8 Isa 1:11}
\crossref{Exod}{29}{19}{29:3 Le 8:22-\allowbreak29}
\crossref{Exod}{29}{20}{Le 14:7,\allowbreak16; 16:14,\allowbreak15,\allowbreak19 Isa 52:15 Heb 9:19-\allowbreak23; 10:22; 12:24}
\crossref{Exod}{29}{21}{29:7; 30:25-\allowbreak31 Le 8:30; 14:15-\allowbreak18,\allowbreak29 Ps 133:2 Isa 11:2-\allowbreak5; 61:1-\allowbreak3}
\crossref{Exod}{29}{22}{29:13 Le 8:25-\allowbreak27}
\crossref{Exod}{29}{23}{29:2,\allowbreak3}
\crossref{Exod}{29}{24}{Le 8:27}
\crossref{Exod}{29}{25}{Le 7:29-\allowbreak31; 8:28 Ps 99:6}
\crossref{Exod}{29}{26}{Le 8:29}
\crossref{Exod}{29}{27}{Le 7:31-\allowbreak34; 8:29; 9:21; 10:15 Nu 6:20; 18:11,\allowbreak18,\allowbreak19 De 18:3}
\crossref{Exod}{29}{28}{Le 7:32-\allowbreak34; 10:14,\allowbreak15 De 18:3}
\crossref{Exod}{29}{29}{Ex 28:3,\allowbreak4}
\crossref{Exod}{29}{30}{Nu 20:28 Heb 7:26}
\crossref{Exod}{29}{31}{29:27}
\crossref{Exod}{29}{32}{Ex 24:9-\allowbreak11 Le 10:12-\allowbreak14}
\crossref{Exod}{29}{33}{Le 10:13-\allowbreak18 Ps 22:26 Joh 6:53-\allowbreak55 1Co 11:24,\allowbreak26}
\crossref{Exod}{29}{34}{29:22,\allowbreak26,\allowbreak28}
\crossref{Exod}{29}{35}{Ex 40:12-\allowbreak15 Le 8:4-\allowbreak36}
\crossref{Exod}{29}{36}{29:10-\allowbreak14 Eze 43:25,\allowbreak27; 48:18-\allowbreak20 Heb 10:11}
\crossref{Exod}{29}{37}{Ex 40:10 Da 9:24}
\crossref{Exod}{29}{38}{Nu 28:3-\allowbreak8 1Ch 16:40 2Ch 2:4; 13:11; 31:3 Ezr 3:3 Da 9:21,\allowbreak27}
\crossref{Exod}{29}{39}{2Ki 16:15 2Ch 13:11 Ps 5:3; 55:16,\allowbreak17 Eze 46:13-\allowbreak15 Lu 1:10}
\crossref{Exod}{29}{40}{Ex 16:36 Nu 15:4,\allowbreak9; 28:5,\allowbreak13}
\crossref{Exod}{29}{41}{1Ki 18:29,\allowbreak36 2Ki 16:15 Ezr 9:4,\allowbreak5 Ps 141:2 Eze 46:13-\allowbreak15}
\crossref{Exod}{29}{42}{Ex 25:22; 30:6,\allowbreak36 Le 1:1 Nu 17:4}
\crossref{Exod}{29}{43}{Ex 40:34 1Ki 8:11 2Ch 5:14; 7:1-\allowbreak3 Isa 6:1-\allowbreak3; 60:1 Eze 43:5}
\crossref{Exod}{29}{44}{Le 21:15; 22:9,\allowbreak16 Joh 10:36 Re 1:5,\allowbreak6}
\crossref{Exod}{29}{45}{Ex 15:17; 25:8 Le 26:12 Ps 68:18 Zec 2:10 Joh 14:17,\allowbreak20,\allowbreak23}
\crossref{Exod}{29}{46}{Ex 20:2 Jer 31:33}
\crossref{Exod}{30}{1}{30:7,\allowbreak8,\allowbreak10; 37:25-\allowbreak28; 40:5 Le 4:7,\allowbreak18 1Ki 6:20 2Ch 26:16 Re 8:3}
\crossref{Exod}{30}{2}{Ex 27:2}
\crossref{Exod}{30}{3}{Ex 25:11,\allowbreak24}
\crossref{Exod}{30}{4}{Ex 25:12,\allowbreak14,\allowbreak27; 26:29; 27:4,\allowbreak7}
\crossref{Exod}{30}{5}{Ex 25:13,\allowbreak27}
\crossref{Exod}{30}{6}{Ex 26:31-\allowbreak35; 40:3,\allowbreak5,\allowbreak26 Mt 27:51 Heb 9:3,\allowbreak4}
\crossref{Exod}{30}{7}{30:34-\allowbreak38}
\crossref{Exod}{30}{8}{Ro 8:34 1Th 5:17 Heb 7:25; 9:24}
\crossref{Exod}{30}{9}{Le 10:1}
\crossref{Exod}{30}{10}{Ex 29:36,\allowbreak37 Le 16:18,\allowbreak29,\allowbreak30; 23:27 Heb 1:3; 9:7,\allowbreak22,\allowbreak23,\allowbreak25}
\crossref{Exod}{30}{11}{}
\crossref{Exod}{30}{12}{Ex 38:25,\allowbreak26 Nu 1:2-\allowbreak5; 26:2-\allowbreak4 2Sa 24:1}
\crossref{Exod}{30}{13}{Le 27:25 Nu 3:47 Eze 45:12}
\crossref{Exod}{30}{14}{Nu 1:3,\allowbreak18,\allowbreak20; 14:29; 26:2; 32:11}
\crossref{Exod}{30}{15}{Job 34:19 Pr 22:2 Eph 6:9 Col 3:25}
\crossref{Exod}{30}{16}{Ex 38:25-\allowbreak31 Ne 10:32,\allowbreak33}
\crossref{Exod}{30}{17}{}
\crossref{Exod}{30}{18}{Ex 31:9; 38:8 Le 8:11 1Ki 7:23,\allowbreak38 2Ch 4:2,\allowbreak6,\allowbreak14,\allowbreak15 Zec 13:1}
\crossref{Exod}{30}{19}{Ex 40:31,\allowbreak32 Ps 26:6 Isa 52:11 Joh 13:8-\allowbreak10 1Co 6:9-\allowbreak11 Tit 3:5}
\crossref{Exod}{30}{20}{Ex 12:15 Le 10:1-\allowbreak3; 16:1,\allowbreak2 1Sa 6:19 1Ch 13:10 Ps 89:7 Ac 5:5,\allowbreak10}
\crossref{Exod}{30}{21}{Ex 28:43}
\crossref{Exod}{30}{22}{}
\crossref{Exod}{30}{23}{Ex 37:29 Ps 45:8 Pr 7:17 So 1:3,\allowbreak13; 4:14 Jer 6:20 Eze 27:19,\allowbreak22}
\crossref{Exod}{30}{24}{Ps 45:8}
\crossref{Exod}{30}{25}{1Ch 9:30}
\crossref{Exod}{30}{26}{Ex 40:9-\allowbreak15 Le 8:10-\allowbreak12 Nu 7:1,\allowbreak10 Isa 61:1 Ac 10:38 2Co 1:21,\allowbreak22}
\crossref{Exod}{30}{27}{30:27}
\crossref{Exod}{30}{28}{}
\crossref{Exod}{30}{29}{Ex 29:37 Le 6:18 Mt 23:17,\allowbreak19}
\crossref{Exod}{30}{30}{Ex 29:7-\allowbreak37; 40:15 Le 8:12,\allowbreak30 Nu 3:3}
\crossref{Exod}{30}{31}{Ex 37:29 Le 8:12; 21:10 Ps 89:20}
\crossref{Exod}{30}{32}{Le 21:10 Mt 7:6}
\crossref{Exod}{30}{33}{30:38 Lu 12:1,\allowbreak2 Heb 10:26-\allowbreak29}
\crossref{Exod}{30}{34}{30:23; 25:6; 37:29}
\crossref{Exod}{30}{35}{Pr 27:9 So 1:3; 3:6 Joh 12:3}
\crossref{Exod}{30}{36}{Ex 16:34}
\crossref{Exod}{30}{37}{30:32,\allowbreak33}
\crossref{Exod}{30}{38}{30:33}
\crossref{Exod}{31}{1}{31:1}
\crossref{Exod}{31}{2}{Ex 33:12,\allowbreak17; 35:30; 36:1 Isa 45:3,\allowbreak4 Mr 3:16-\allowbreak19 Joh 3:27}
\crossref{Exod}{31}{3}{Ex 35:31 1Ki 3:9; 7:14 Isa 28:6,\allowbreak26 1Co 12:4-\allowbreak11}
\crossref{Exod}{31}{4}{Ex 25:32-\allowbreak35; 26:1; 28:15 1Ki 7:14 2Ch 2:7,\allowbreak13,\allowbreak14}
\crossref{Exod}{31}{5}{Ex 28:9-\allowbreak21}
\crossref{Exod}{31}{6}{Ex 4:14,\allowbreak15; 6:26 Ezr 5:1,\allowbreak2 Ec 4:9-\allowbreak12 Mt 10:2-\allowbreak4 Lu 10:1 Ac 13:2}
\crossref{Exod}{31}{7}{Ex 26:1-\allowbreak37; 27:9-\allowbreak19; 36:8-\allowbreak38}
\crossref{Exod}{31}{8}{Ex 25:23-\allowbreak30; 37:10-\allowbreak16}
\crossref{Exod}{31}{9}{Ex 27:1-\allowbreak8; 38:1-\allowbreak7}
\crossref{Exod}{31}{10}{Ex 28:1-\allowbreak43; 39:1-\allowbreak43 Le 8:7,\allowbreak8,\allowbreak13 Nu 4:5-\allowbreak14}
\crossref{Exod}{31}{11}{Ex 30:23-\allowbreak33; 37:29}
\crossref{Exod}{31}{12}{}
\crossref{Exod}{31}{13}{Ex 20:8-\allowbreak11 Le 19:3,\allowbreak30; 23:3; 25:2; 26:2}
\crossref{Exod}{31}{14}{Ex 20:8 De 5:12-\allowbreak15 Ne 9:14 Isa 56:2-\allowbreak6; 58:13,\allowbreak14 Eze 20:12; 44:24}
\crossref{Exod}{31}{15}{31:17; 16:26; 20:9; 34:21 Le 23:3 Eze 46:1 Lu 13:14}
\crossref{Exod}{31}{16}{Ge 9:13; 17:11 Jer 50:5}
\crossref{Exod}{31}{17}{31:13 Eze 20:12,\allowbreak20}
\crossref{Exod}{31}{18}{Ex 24:12,\allowbreak18; 32:15,\allowbreak16; 34:1-\allowbreak4,\allowbreak28,\allowbreak29 De 4:13; 5:22; 9:9-\allowbreak11 2Co 3:3}
\crossref{Exod}{32}{1}{Ex 24:18 De 9:9 Mt 24:43 2Pe 3:4}
\crossref{Exod}{32}{2}{Ex 12:35,\allowbreak36 Ge 24:22,\allowbreak47 Jud 8:24-\allowbreak27 Eze 16:11,\allowbreak12,\allowbreak17 Ho 2:8}
\crossref{Exod}{32}{3}{Jud 17:3,\allowbreak4 Isa 40:19,\allowbreak20; 46:6 Jer 10:9}
\crossref{Exod}{32}{4}{Ex 20:23 De 9:16 Ps 106:19-\allowbreak21 Isa 44:9,\allowbreak10; 46:6 Ac 7:41; 17:29}
\crossref{Exod}{32}{5}{1Sa 14:35 2Ki 16:11 Ho 8:11,\allowbreak14}
\crossref{Exod}{32}{6}{Ex 24:4,\allowbreak5}
\crossref{Exod}{32}{7}{Ex 19:24; 33:1 De 9:12 Da 9:24}
\crossref{Exod}{32}{8}{De 9:16 Jud 2:17}
\crossref{Exod}{32}{9}{De 9:13 Jer 13:27 Ho 6:10}
\crossref{Exod}{32}{10}{Ge 18:32,\allowbreak33; 32:26-\allowbreak28 Nu 14:19,\allowbreak20; 16:22,\allowbreak45-\allowbreak48 De 9:14,\allowbreak19}
\crossref{Exod}{32}{11}{De 9:18-\allowbreak20,\allowbreak26-\allowbreak29 Ps 106:23}
\crossref{Exod}{32}{12}{Nu 14:13-\allowbreak16 De 9:28; 32:26,\allowbreak27 Jos 7:9 Ps 74:18; 79:9,\allowbreak10}
\crossref{Exod}{32}{13}{Le 26:42 De 7:8; 9:27 Lu 1:54,\allowbreak55}
\crossref{Exod}{32}{14}{De 32:26 2Sa 24:16 1Ch 21:15 Ps 106:45 Jer 18:8; 26:13,\allowbreak19}
\crossref{Exod}{32}{15}{Ex 24:18 De 9:15}
\crossref{Exod}{32}{16}{Ex 31:18; 34:1,\allowbreak4 De 9:9-\allowbreak11,\allowbreak15; 10:1 2Co 3:3,\allowbreak7 Heb 8:10}
\crossref{Exod}{32}{17}{Ex 17:9; 24:13}
\crossref{Exod}{32}{18}{Ex 15:1-\allowbreak18 Da 5:4,\allowbreak23}
\crossref{Exod}{32}{19}{32:4-\allowbreak6 De 9:16,\allowbreak17}
\crossref{Exod}{32}{20}{Pr 1:31; 14:14}
\crossref{Exod}{32}{21}{Ge 20:9; 26:10 De 13:6-\allowbreak8 1Sa 26:19 Jos 7:19-\allowbreak26 1Ki 14:16; 21:22}
\crossref{Exod}{32}{22}{Ex 14:11; 15:24; 16:2-\allowbreak4,\allowbreak20,\allowbreak28; 17:2-\allowbreak4 De 9:7,\allowbreak24}
\crossref{Exod}{32}{23}{32:1-\allowbreak4,\allowbreak8}
\crossref{Exod}{32}{24}{32:4 Ge 3:12,\allowbreak13 Lu 10:29 Ro 3:10}
\crossref{Exod}{32}{25}{De 9:20 2Ch 28:19}
\crossref{Exod}{32}{26}{Jos 5:13 2Sa 20:11 2Ki 9:32 Mt 12:30}
\crossref{Exod}{32}{27}{32:26,\allowbreak29 Nu 25:5,\allowbreak7-\allowbreak12 De 33:8,\allowbreak9 Lu 14:26 2Co 5:16}
\crossref{Exod}{32}{28}{De 33:9 Mal 2:4-\allowbreak6}
\crossref{Exod}{32}{29}{Nu 25:11-\allowbreak13 De 13:6-\allowbreak11; 33:9,\allowbreak10 1Sa 15:18-\allowbreak22 Pr 21:3}
\crossref{Exod}{32}{30}{32:31 1Sa 2:17; 12:20,\allowbreak23 2Sa 12:9 2Ki 17:21 Lu 7:47; 15:18}
\crossref{Exod}{32}{31}{Ex 34:28 De 9:18,\allowbreak19}
\crossref{Exod}{32}{32}{Nu 14:19 Da 9:18,\allowbreak19 Am 7:2 Lu 23:34}
\crossref{Exod}{32}{33}{Le 23:30 Ps 69:28 Eze 18:4}
\crossref{Exod}{32}{34}{Ex 23:20; 33:2,\allowbreak14,\allowbreak15 Nu 20:16 Isa 63:9}
\crossref{Exod}{32}{35}{32:25 2Sa 12:9,\allowbreak10 Mt 27:3-\allowbreak7 Ac 1:18; 7:41}
\crossref{Exod}{33}{1}{Ex 32:34}
\crossref{Exod}{33}{2}{Ex 23:20; 32:34; 34:11}
\crossref{Exod}{33}{3}{Ex 3:8; 13:5 Le 20:24 Nu 13:27; 14:8; 16:13 Jos 5:6 Jer 11:5}
\crossref{Exod}{33}{4}{Nu 14:1,\allowbreak39 Ho 7:14 Zec 7:3,\allowbreak5}
\crossref{Exod}{33}{5}{33:3 Nu 16:45,\allowbreak46}
\crossref{Exod}{33}{6}{33:4; 32:3 Jer 2:19}
\crossref{Exod}{33}{7}{Ps 10:1; 35:22 Pr 15:29 Isa 59:2 Ho 9:12}
\crossref{Exod}{33}{8}{Nu 16:27}
\crossref{Exod}{33}{9}{Ex 13:21,\allowbreak22 Ps 99:7}
\crossref{Exod}{33}{10}{Ex 4:31 1Ki 8:14,\allowbreak22 Lu 18:13}
\crossref{Exod}{33}{11}{33:9 Ge 32:30 Nu 12:8 De 5:4; 34:10}
\crossref{Exod}{33}{12}{33:1; 32:34}
\crossref{Exod}{33}{13}{33:17; 34:9}
\crossref{Exod}{33}{14}{Ex 13:21 Jos 1:5 Isa 63:9 Mt 28:20}
\crossref{Exod}{33}{15}{33:3; 34:9 Ps 4:6}
\crossref{Exod}{33}{16}{Nu 14:14 Mt 1:23}
\crossref{Exod}{33}{17}{Ge 18:32; 19:21 Isa 65:24 Joh 16:23 Jas 5:16 1Jo 5:14,\allowbreak15}
\crossref{Exod}{33}{18}{33:20 Ps 4:6 Joh 1:18 2Co 3:18; 4:6 1Ti 6:16 Tit 2:13 Re 21:23}
\crossref{Exod}{33}{19}{Ne 9:25 Ps 25:13}
\crossref{Exod}{33}{20}{}
\crossref{Exod}{33}{21}{De 5:31 Jos 20:4 Isa 56:5 Zec 3:7 Lu 15:1}
\crossref{Exod}{33}{22}{Ps 18:2 So 2:3 Isa 2:21; 32:2 1Co 10:4 2Co 5:19}
\crossref{Exod}{33}{23}{33:20 Job 11:7; 26:14 Joh 1:18 1Co 13:12 1Ti 6:16}
\crossref{Exod}{34}{1}{Ex 31:18; 32:16,\allowbreak19 De 10:1}
\crossref{Exod}{34}{2}{Ex 19:20,\allowbreak24; 24:12 De 9:25}
\crossref{Exod}{34}{3}{Ex 19:12,\allowbreak13,\allowbreak21 Le 16:17 1Ti 2:5 Heb 12:20}
\crossref{Exod}{34}{4}{}
\crossref{Exod}{34}{5}{Ex 19:18; 33:9 Nu 11:17,\allowbreak25 1Ki 8:10-\allowbreak12 Lu 9:34,\allowbreak35}
\crossref{Exod}{34}{6}{Ex 33:20-\allowbreak23 1Ki 19:11}
\crossref{Exod}{34}{7}{Ex 20:6 De 5:10 Ne 1:5; 9:32 Ps 86:15 Jer 32:18 Da 9:4}
\crossref{Exod}{34}{8}{Ex 4:31 Ge 17:3 2Ch 20:18}
\crossref{Exod}{34}{9}{Ex 33:13,\allowbreak17}
\crossref{Exod}{34}{10}{Ex 24:7,\allowbreak8 De 4:13; 5:2,\allowbreak3; 29:12-\allowbreak14}
\crossref{Exod}{34}{11}{De 4:1,\allowbreak2,\allowbreak40; 5:32; 6:3,\allowbreak25; 12:28,\allowbreak32; 28:1 Mt 28:20 Joh 14:21}
\crossref{Exod}{34}{12}{Ex 23:32,\allowbreak33 De 7:2 Jud 2:2}
\crossref{Exod}{34}{13}{Ex 23:24 De 7:5,\allowbreak25,\allowbreak26; 12:2,\allowbreak3 Jud 2:2; 6:25 2Ki 18:4; 23:14}
\crossref{Exod}{34}{14}{Ex 20:3-\allowbreak5 De 5:7 Mt 4:10}
\crossref{Exod}{34}{15}{34:10,\allowbreak12; 23:32 De 7:2}
\crossref{Exod}{34}{16}{Nu 25:1,\allowbreak2 De 7:3,\allowbreak4 1Ki 11:2-\allowbreak4 Ezr 9:2 Ne 13:23,\allowbreak25 2Co 6:14-\allowbreak17}
\crossref{Exod}{34}{17}{Ex 32:8 Le 19:4 Isa 46:6,\allowbreak7 Jer 10:14 Ac 17:29; 19:26}
\crossref{Exod}{34}{18}{Ex 12:15-\allowbreak20; 13:4,\allowbreak6,\allowbreak7; 23:15 Le 23:6 De 16:1-\allowbreak4 Mr 14:1 Lu 22:1}
\crossref{Exod}{34}{19}{Ex 13:2,\allowbreak12; 22:29 Nu 18:15-\allowbreak17 Eze 44:30 Lu 2:23}
\crossref{Exod}{34}{20}{Ex 13:10 Nu 18:15}
\crossref{Exod}{34}{21}{Ex 20:9-\allowbreak11; 23:12; 35:2 De 5:12-\allowbreak15 Lu 13:14; 23:56}
\crossref{Exod}{34}{22}{Ex 23:16 Nu 28:16-\allowbreak31; 29:12-\allowbreak39 De 16:10-\allowbreak15 Joh 7:2 Ac 2:1}
\crossref{Exod}{34}{23}{Ex 23:14,\allowbreak17 De 16:16 Ps 84:7}
\crossref{Exod}{34}{24}{34:11; 23:27-\allowbreak30; 33:2 Le 18:24 De 7:1 Ps 78:55; 80:8}
\crossref{Exod}{34}{25}{Ex 12:20; 23:18 De 16:3 1Co 5:7,\allowbreak8}
\crossref{Exod}{34}{26}{Ex 23:19 De 26:2,\allowbreak10 Pr 3:9,\allowbreak10 Mt 6:33 1Co 15:20 Jas 1:18}
\crossref{Exod}{34}{27}{Ex 17:14; 24:4,\allowbreak7 De 31:9}
\crossref{Exod}{34}{28}{Ex 24:18 De 9:9,\allowbreak18,\allowbreak25}
\crossref{Exod}{34}{29}{Ex 16:15 Jos 2:4; 8:14 Jud 16:20 Mr 9:6; 14:40 Lu 2:49 Joh 5:13}
\crossref{Exod}{34}{30}{Nu 12:8 Mr 9:3,\allowbreak15 Lu 5:8}
\crossref{Exod}{34}{31}{Ex 3:15; 24:1-\allowbreak3}
\crossref{Exod}{34}{32}{Ex 21:1 Nu 15:40 1Ki 22:14 Mt 28:20 1Co 11:23; 15:3}
\crossref{Exod}{34}{33}{Ro 10:4 2Co 3:13-\allowbreak18; 4:4-\allowbreak6}
\crossref{Exod}{34}{34}{2Co 3:16 Heb 4:16; 10:19-\allowbreak22}
\crossref{Exod}{34}{35}{34:29,\allowbreak30 Ec 8:1 Da 12:3 Mt 5:16; 13:43 Joh 5:35 Php 2:15}
\crossref{Exod}{35}{1}{Ex 25:1-\allowbreak40; 31:1-\allowbreak11; 34:32}
\crossref{Exod}{35}{2}{Ex 20:9,\allowbreak10; 23:12; 31:13-\allowbreak16; 34:21 Le 23:3 De 5:12-\allowbreak15 Lu 13:14}
\crossref{Exod}{35}{3}{Ex 12:16; 16:23 Nu 15:32-\allowbreak36 Isa 58:13}
\crossref{Exod}{35}{4}{Ex 25:1,\allowbreak2}
\crossref{Exod}{35}{5}{Ex 25:2-\allowbreak7 Jud 5:9 Ps 110:3 Mr 12:41-\allowbreak44 2Co 8:11,\allowbreak12; 9:7}
\crossref{Exod}{35}{6}{Ex 26:1,\allowbreak31,\allowbreak36; 28:5,\allowbreak6,\allowbreak15,\allowbreak33}
\crossref{Exod}{35}{7}{}
\crossref{Exod}{35}{8}{Ex 27:20}
\crossref{Exod}{35}{9}{Ex 25:5; 28:9,\allowbreak17-\allowbreak21; 39:6-\allowbreak14}
\crossref{Exod}{35}{10}{Ex 31:1-\allowbreak6; 36:1-\allowbreak4}
\crossref{Exod}{35}{11}{Ex 26:1,\allowbreak2-\allowbreak37; 31:7-\allowbreak9; 36:8-\allowbreak34}
\crossref{Exod}{35}{12}{Ex 25:10-\allowbreak22; 37:1-\allowbreak9}
\crossref{Exod}{35}{13}{Ex 25:23-\allowbreak30; 37:10-\allowbreak16 Le 24:5,\allowbreak6}
\crossref{Exod}{35}{14}{Ex 25:31-\allowbreak39; 37:17-\allowbreak24 Ps 148:3 Mt 5:14,\allowbreak15}
\crossref{Exod}{35}{15}{Ex 30:1-\allowbreak10,\allowbreak22-\allowbreak38; 37:25-\allowbreak28 Ps 141:2}
\crossref{Exod}{35}{16}{Ex 27:1-\allowbreak8; 38:1-\allowbreak7}
\crossref{Exod}{35}{17}{Ex 27:9-\allowbreak19; 38:9-\allowbreak20 2Sa 7:2}
\crossref{Exod}{35}{18}{}
\crossref{Exod}{35}{19}{Ex 31:10; 39:1,\allowbreak41 Nu 4:5-\allowbreak15}
\crossref{Exod}{35}{20}{35:20}
\crossref{Exod}{35}{21}{35:5,\allowbreak22,\allowbreak26,\allowbreak29; 25:2; 36:2 Jud 5:3,\allowbreak9,\allowbreak12 2Sa 7:27 1Ch 28:2,\allowbreak9}
\crossref{Exod}{35}{22}{}
\crossref{Exod}{35}{23}{35:6-\allowbreak10; 25:2-\allowbreak7 1Ch 29:8}
\crossref{Exod}{35}{24}{2Co 8:12}
\crossref{Exod}{35}{25}{Ex 28:3; 31:6; 36:1 2Ki 23:7 Pr 14:1; 31:19-\allowbreak24 Lu 8:2,\allowbreak3 Ac 9:39}
\crossref{Exod}{35}{26}{35:21,\allowbreak29; 36:8}
\crossref{Exod}{35}{27}{35:9 1Ch 29:6 Ezr 2:68}
\crossref{Exod}{35}{28}{35:8; 30:23-\allowbreak38}
\crossref{Exod}{35}{29}{35:21,\allowbreak22 1Ch 29:3,\allowbreak6,\allowbreak9,\allowbreak10,\allowbreak14,\allowbreak17 Jud 5:2,\allowbreak9 1Co 9:17 2Co 9:7}
\crossref{Exod}{35}{30}{Ex 31:2-\allowbreak6 1Ki 7:13,\allowbreak14 Isa 28:26 1Co 3:10; 12:4,\allowbreak11 Jas 1:17}
\crossref{Exod}{35}{31}{Isa 11:2-\allowbreak5; 28:26; 61:1-\allowbreak3 1Co 12:4-\allowbreak10 Col 2:3 Jas 1:17}
\crossref{Exod}{35}{32}{35:32}
\crossref{Exod}{35}{33}{}
\crossref{Exod}{35}{34}{Ezr 7:10,\allowbreak27 Ne 2:12 Jas 1:16,\allowbreak17}
\crossref{Exod}{35}{35}{35:31; 31:3,\allowbreak6 1Ki 3:12; 7:14 2Ch 2:14 Isa 28:26}
\crossref{Exod}{36}{1}{Ex 31:1-\allowbreak6; 35:30-\allowbreak35}
\crossref{Exod}{36}{2}{Ex 28:3; 31:6; 35:10,\allowbreak21-\allowbreak35 Ac 6:3,\allowbreak4; 14:23 Col 4:17 Heb 5:4}
\crossref{Exod}{36}{3}{Ex 35:5-\allowbreak21,\allowbreak27,\allowbreak29}
\crossref{Exod}{36}{4}{2Ch 24:13 Mt 24:45 Lu 12:42 1Co 3:10}
\crossref{Exod}{36}{5}{Ex 32:3 2Ch 24:14; 31:6-\allowbreak10 2Co 8:2,\allowbreak3 Php 2:21; 4:17,\allowbreak18}
\crossref{Exod}{36}{6}{Ex 35:21-\allowbreak29; 38:8 Ge 14:21; 28:22; 45:18-\allowbreak20 Le 26:10 Nu 7:1-\allowbreak88}
\crossref{Exod}{36}{7}{2Ch 31:10}
\crossref{Exod}{36}{8}{Ex 31:6; 35:10}
\crossref{Exod}{36}{9}{36:9}
\crossref{Exod}{36}{10}{Ex 26:3 Ps 122:3; 133:1 Zep 3:9 Ac 2:1 1Co 1:10; 12:20,\allowbreak27 Eph 1:23}
\crossref{Exod}{36}{11}{Ex 26:4}
\crossref{Exod}{36}{12}{Ex 26:5,\allowbreak10}
\crossref{Exod}{36}{13}{1Co 12:20 Eph 2:20-\allowbreak22 1Pe 2:4,\allowbreak5}
\crossref{Exod}{36}{14}{Ex 26:7-\allowbreak13}
\crossref{Exod}{36}{15}{36:15}
\crossref{Exod}{36}{16}{36:16}
\crossref{Exod}{36}{17}{36:17}
\crossref{Exod}{36}{18}{}
\crossref{Exod}{36}{19}{Ex 26:14}
\crossref{Exod}{36}{20}{Ex 26:15-\allowbreak25; 40:18,\allowbreak19}
\crossref{Exod}{36}{21}{}
\crossref{Exod}{36}{22}{36:22}
\crossref{Exod}{36}{23}{36:23}
\crossref{Exod}{36}{24}{36:24}
\crossref{Exod}{36}{25}{36:25}
\crossref{Exod}{36}{26}{}
\crossref{Exod}{36}{27}{Ex 26:22,\allowbreak27}
\crossref{Exod}{36}{28}{}
\crossref{Exod}{36}{29}{Ex 26:24 Ps 122:3; 133:1 Ac 2:46; 4:32 1Co 1:10; 12:13 2Co 1:10}
\crossref{Exod}{36}{30}{}
\crossref{Exod}{36}{31}{Ex 25:28; 26:26-\allowbreak29; 30:5}
\crossref{Exod}{36}{32}{Ex 26:26}
\crossref{Exod}{36}{33}{36:33}
\crossref{Exod}{36}{34}{}
\crossref{Exod}{36}{35}{Ex 26:31-\allowbreak35; 30:6; 40:21 Mt 27:51 Heb 10:20}
\crossref{Exod}{36}{36}{Jer 1:18}
\crossref{Exod}{36}{37}{}
\crossref{Exod}{36}{38}{Ex 27:10}
\crossref{Exod}{37}{1}{Ex 25:10-\allowbreak16; 26:33; 31:7; 40:3,\allowbreak20,\allowbreak21 Nu 10:33-\allowbreak36}
\crossref{Exod}{37}{2}{Ex 30:3}
\crossref{Exod}{37}{3}{}
\crossref{Exod}{37}{4}{Nu 4:14,\allowbreak15 Ac 9:15}
\crossref{Exod}{37}{5}{Nu 1:50; 4:15 2Sa 6:3-\allowbreak7}
\crossref{Exod}{37}{6}{Ex 25:17-\allowbreak22 Le 16:12-\allowbreak15 1Ch 28:11 Ro 3:25 Ga 4:4 Tit 2:14}
\crossref{Exod}{37}{7}{1Ki 6:23-\allowbreak29 Ps 80:1; 104:4 Eze 10:2}
\crossref{Exod}{37}{8}{}
\crossref{Exod}{37}{9}{Ge 3:24; 28:12 Isa 6:2 Eze 10:1-\allowbreak22 Joh 1:51 2Co 3:18 Php 3:8}
\crossref{Exod}{37}{10}{Ex 25:23-\allowbreak30; 35:13; 40:4,\allowbreak22,\allowbreak23 Eze 40:39-\allowbreak42 Mal 1:12 Joh 1:14,\allowbreak16}
\crossref{Exod}{37}{11}{}
\crossref{Exod}{37}{12}{Ge 21:16}
\crossref{Exod}{37}{13}{37:13}
\crossref{Exod}{37}{14}{37:14}
\crossref{Exod}{37}{15}{}
\crossref{Exod}{37}{16}{Ex 25:29 1Ki 7:50 2Ki 12:13 Jer 52:18,\allowbreak19 2Ti 2:20}
\crossref{Exod}{37}{17}{Ex 25:31-\allowbreak39; 40:24,\allowbreak25 Le 24:4 1Ch 28:15 2Ch 13:11 Zec 4:2,\allowbreak11}
\crossref{Exod}{37}{18}{37:18}
\crossref{Exod}{37}{19}{}
\crossref{Exod}{37}{20}{Ex 25:33 Nu 17:8 Ec 12:5 Jer 1:11}
\crossref{Exod}{37}{21}{Ex 25:35}
\crossref{Exod}{37}{22}{Ex 25:31 1Co 9:27 Col 3:5}
\crossref{Exod}{37}{23}{Ex 25:37 Nu 8:2 Zec 4:2 Re 1:12,\allowbreak20; 2:1; 4:5; 5:5}
\crossref{Exod}{37}{24}{}
\crossref{Exod}{37}{25}{Ex 30:1-\allowbreak5; 40:5,\allowbreak26,\allowbreak27 2Ch 26:16 Mt 23:19 Lu 1:9,\allowbreak10 Heb 7:25}
\crossref{Exod}{37}{26}{37:26}
\crossref{Exod}{37}{27}{37:27}
\crossref{Exod}{37}{28}{}
\crossref{Exod}{37}{29}{Ex 30:23-\allowbreak38 Ps 23:5; 92:10 Isa 11:2; 61:1,\allowbreak3 Joh 3:34 2Co 1:21,\allowbreak22}
\crossref{Exod}{38}{1}{Ex 27:1-\allowbreak8; 40:6,\allowbreak29 2Ch 4:1 Eze 43:13-\allowbreak17 Ro 8:3,\allowbreak4; 12:1 Heb 3:1}
\crossref{Exod}{38}{2}{Ex 27:2}
\crossref{Exod}{38}{3}{Ex 27:3}
\crossref{Exod}{38}{4}{}
\crossref{Exod}{38}{5}{Ex 27:4}
\crossref{Exod}{38}{6}{Ex 25:6 De 10:3}
\crossref{Exod}{38}{7}{Ac 9:15 1Co 1:24; 2:2}
\crossref{Exod}{38}{8}{Ex 30:18-\allowbreak21; 40:7,\allowbreak30-\allowbreak32 1Ki 7:23-\allowbreak26,\allowbreak38 Ps 26:6 Zec 13:1 Joh 13:10}
\crossref{Exod}{38}{9}{Ex 27:9-\allowbreak19; 40:8,\allowbreak33 1Ki 6:36 Ps 84:2,\allowbreak10; 89:7; 92:13; 100:4}
\crossref{Exod}{38}{10}{38:10}
\crossref{Exod}{38}{11}{38:11}
\crossref{Exod}{38}{12}{38:12}
\crossref{Exod}{38}{13}{}
\crossref{Exod}{38}{14}{Ex 27:14}
\crossref{Exod}{38}{15}{38:15}
\crossref{Exod}{38}{16}{38:16}
\crossref{Exod}{38}{17}{}
\crossref{Exod}{38}{18}{2Ch 3:14}
\crossref{Exod}{38}{19}{}
\crossref{Exod}{38}{20}{Ex 27:19 2Ch 3:9 Ezr 9:8 Ec 12:11 Isa 22:23; 33:20 Eph 2:21,\allowbreak22}
\crossref{Exod}{38}{21}{Ex 25:16; 26:33; 40:3 Nu 1:50,\allowbreak53; 9:15; 10:11; 17:7,\allowbreak8; 18:2 2Ch 24:6}
\crossref{Exod}{38}{22}{Ex 31:1-\allowbreak5; 35:30-\allowbreak35; 36:1-\allowbreak3}
\crossref{Exod}{38}{23}{Ex 35:34}
\crossref{Exod}{38}{24}{Ex 25:2; 29:24; 35:22}
\crossref{Exod}{38}{25}{}
\crossref{Exod}{38}{26}{Ex 30:13,\allowbreak15,\allowbreak16}
\crossref{Exod}{38}{27}{Ex 26:19,\allowbreak21,\allowbreak25,\allowbreak32}
\crossref{Exod}{38}{28}{Ex 27:17}
\crossref{Exod}{38}{29}{}
\crossref{Exod}{38}{30}{Ex 26:37; 27:10,\allowbreak17}
\crossref{Exod}{38}{31}{Ex 27:10-\allowbreak12}
\crossref{Exod}{39}{1}{Ex 25:4; 26:1; 35:23}
\crossref{Exod}{39}{2}{Ex 25:7; 28:6-\allowbreak12 Le 8:7}
\crossref{Exod}{39}{3}{Ex 26:1; 36:8}
\crossref{Exod}{39}{4}{}
\crossref{Exod}{39}{5}{Ex 28:8; 29:5 Le 8:7 Isa 11:5 Re 1:13}
\crossref{Exod}{39}{6}{Ex 25:7; 28:9; 35:9 Job 28:16 Eze 28:13}
\crossref{Exod}{39}{7}{Ex 28:12,\allowbreak29 Jos 4:7 Ne 2:20 Mr 14:9,\allowbreak22-\allowbreak25}
\crossref{Exod}{39}{8}{Ex 25:7; 28:4,\allowbreak13-\allowbreak29 Le 8:8,\allowbreak9 Ps 89:28 Isa 59:17 Eph 6:14}
\crossref{Exod}{39}{9}{}
\crossref{Exod}{39}{10}{Ex 28:16,\allowbreak17,\allowbreak21 Re 21:19-\allowbreak21}
\crossref{Exod}{39}{11}{Ex 28:18 Eze 28:13}
\crossref{Exod}{39}{12}{}
\crossref{Exod}{39}{13}{39:6}
\crossref{Exod}{39}{14}{Re 21:12}
\crossref{Exod}{39}{15}{Ex 28:14 2Ch 3:5 So 1:10 Joh 10:28; 17:12 1Pe 1:5 Jude 1:1}
\crossref{Exod}{39}{16}{Ex 25:12}
\crossref{Exod}{39}{17}{}
\crossref{Exod}{39}{18}{Ex 28:14 So 1:10}
\crossref{Exod}{39}{19}{}
\crossref{Exod}{39}{20}{Ex 26:3}
\crossref{Exod}{39}{21}{Mt 16:24 1Co 1:25,\allowbreak27}
\crossref{Exod}{39}{22}{Ex 28:31-\allowbreak35}
\crossref{Exod}{39}{23}{}
\crossref{Exod}{39}{24}{Ex 28:33}
\crossref{Exod}{39}{25}{Ex 28:33,\allowbreak34 Ps 89:15}
\crossref{Exod}{39}{26}{Ex 28:34 So 4:3,\allowbreak13; 6:7}
\crossref{Exod}{39}{27}{Ex 28:39-\allowbreak42 Le 8:13 Isa 61:10 Eze 44:18 Ro 3:22; 13:14 Ga 3:27}
\crossref{Exod}{39}{28}{Ex 28:4,\allowbreak39 Eze 44:18}
\crossref{Exod}{39}{29}{}
\crossref{Exod}{39}{30}{Ex 26:36; 28:36-\allowbreak39 1Co 1:30 2Co 5:21 Heb 1:3; 7:26}
\crossref{Exod}{39}{31}{}
\crossref{Exod}{39}{32}{39:33,\allowbreak42; 25:1-\allowbreak31:18; 35:1-\allowbreak40:38 Le 8:1-\allowbreak9:24 Nu 3:25,\allowbreak26,\allowbreak31,\allowbreak36,\allowbreak37}
\crossref{Exod}{39}{33}{Ex 25:1-\allowbreak30:38; 31:7-\allowbreak11; 35:11-\allowbreak19; 36:1-\allowbreak40:38}
\crossref{Exod}{39}{34}{}
\crossref{Exod}{39}{35}{Ex 25:17 Heb 9:5,\allowbreak8}
\crossref{Exod}{39}{36}{Ex 25:30 1Ki 7:48}
\crossref{Exod}{39}{37}{Ex 27:21 Mt 5:14-\allowbreak16 Php 2:15}
\crossref{Exod}{39}{38}{Ex 25:6; 30:7; 31:11; 35:8; 37:29 2Ch 2:4}
\crossref{Exod}{39}{39}{Ex 38:30 1Ki 8:64}
\crossref{Exod}{39}{40}{Ex 27:9; 35:17; 38:9; 40:7}
\crossref{Exod}{39}{41}{39:1; 31:10}
\crossref{Exod}{39}{42}{39:32; 23:21,\allowbreak22; 25:1-\allowbreak31:18 De 12:32 Mt 28:20 2Ti 2:15; 4:7}
\crossref{Exod}{39}{43}{Ex 40:25 Ge 1:31 Ps 104:31}
\crossref{Exod}{40}{1}{40:1}
\crossref{Exod}{40}{2}{40:17; 12:1,\allowbreak2; 13:4 Nu 7:1}
\crossref{Exod}{40}{3}{40:20,\allowbreak21; 25:10,\allowbreak22; 26:31,\allowbreak33,\allowbreak34; 35:12; 36:35,\allowbreak36; 37:1-\allowbreak9 Le 16:14}
\crossref{Exod}{40}{4}{40:22,\allowbreak25; 25:23-\allowbreak30; 26:35,\allowbreak36; 37:10-\allowbreak24}
\crossref{Exod}{40}{5}{40:26,\allowbreak27; 30:1-\allowbreak5; 35:25-\allowbreak28; 37:25-\allowbreak28 Joh 14:6 Heb 9:24; 10:19-\allowbreak22}
\crossref{Exod}{40}{6}{40:29}
\crossref{Exod}{40}{7}{40:30-\allowbreak32}
\crossref{Exod}{40}{8}{40:33}
\crossref{Exod}{40}{9}{Ex 30:23-\allowbreak33; 37:29; 39:39 Le 8:10 Nu 7:1 Ps 45:7 Isa 11:2; 61:1}
\crossref{Exod}{40}{10}{Ex 29:36,\allowbreak37 Le 8:11 Isa 11:2; 61:1 Joh 3:34; 17:19}
\crossref{Exod}{40}{11}{}
\crossref{Exod}{40}{12}{Ex 29:1-\allowbreak35 Le 8:1-\allowbreak13; 9:1-\allowbreak24 Isa 11:1-\allowbreak5; 61:1-\allowbreak3 Mt 3:16 Lu 1:35}
\crossref{Exod}{40}{13}{Ex 28:41 Isa 61:1 Joh 3:34; 17:19 Heb 10:10,\allowbreak29 1Jo 2:20,\allowbreak27}
\crossref{Exod}{40}{14}{Isa 44:3-\allowbreak5; 61:10 Joh 1:16 Ro 8:30; 13:14 1Co 1:9,\allowbreak30}
\crossref{Exod}{40}{15}{Ex 12:14; 30:31,\allowbreak33 Nu 25:13 Ps 110:4 Heb 5:1-\allowbreak14; 7:3,\allowbreak7,\allowbreak17-\allowbreak24}
\crossref{Exod}{40}{16}{40:17-\allowbreak32; 23:21,\allowbreak22; 39:42,\allowbreak43 De 4:1; 12:32 Isa 8:20 Mt 28:20}
\crossref{Exod}{40}{17}{40:1,\allowbreak2 Nu 7:1; 9:1}
\crossref{Exod}{40}{18}{40:2; 26:15-\allowbreak30; 36:20-\allowbreak34 Le 26:11 Eze 37:27,\allowbreak28 Joh 1:14 Ga 4:4}
\crossref{Exod}{40}{19}{Ex 26:1-\allowbreak14; 36:8-\allowbreak19}
\crossref{Exod}{40}{20}{Ex 16:34; 25:16-\allowbreak21; 31:18 Ps 40:8 Mt 3:15}
\crossref{Exod}{40}{21}{40:3; 26:33; 35:12}
\crossref{Exod}{40}{22}{Joh 6:53-\allowbreak57 Eph 3:8}
\crossref{Exod}{40}{23}{40:4; 25:30 Mt 12:4 Heb 9:2}
\crossref{Exod}{40}{24}{Ex 25:31-\allowbreak35; 37:17-\allowbreak24 Ps 119:105 Joh 1:1,\allowbreak5,\allowbreak9; 8:12 Re 1:20; 2:5}
\crossref{Exod}{40}{25}{40:4; 25:37 Re 4:5}
\crossref{Exod}{40}{26}{40:5; 30:1-\allowbreak10 Mt 23:19 Joh 11:42; 17:1-\allowbreak26 Heb 7:25; 10:1 1Jo 2:1}
\crossref{Exod}{40}{27}{Ex 30:7}
\crossref{Exod}{40}{28}{40:5; 26:36,\allowbreak37; 38:9-\allowbreak19 Joh 14:6; 10:9 Eph 2:18 Heb 10:19,\allowbreak20}
\crossref{Exod}{40}{29}{40:6; 27:1-\allowbreak8; 38:1-\allowbreak7 Mt 23:19 Ro 3:24-\allowbreak26 Heb 9:12; 13:5,\allowbreak6,\allowbreak10}
\crossref{Exod}{40}{30}{40:7; 30:18-\allowbreak21; 38:8 Eze 36:25 Heb 10:22}
\crossref{Exod}{40}{31}{Ps 26:6; 51:6,\allowbreak7 Joh 13:10 1Jo 1:7,\allowbreak9}
\crossref{Exod}{40}{32}{40:19; 30:19,\allowbreak20 Ps 73:19}
\crossref{Exod}{40}{33}{40:8; 27:9-\allowbreak16 Nu 1:50 Mt 16:8 1Co 12:12,\allowbreak28 Eph 4:11-\allowbreak13 Heb 9:6,\allowbreak7}
\crossref{Exod}{40}{34}{Ex 13:21,\allowbreak22; 14:19,\allowbreak20,\allowbreak24; 25:8,\allowbreak21,\allowbreak22; 29:43; 33:9 Le 16:2}
\crossref{Exod}{40}{35}{Le 16:2 1Ki 8:11 2Ch 5:14; 7:2 Isa 6:4 Re 15:8}
\crossref{Exod}{40}{36}{Ex 13:21,\allowbreak22 Nu 10:11-\allowbreak13,\allowbreak33-\allowbreak36; 19:17-\allowbreak22 Ne 9:19 Ps 78:14; 105:39}
\crossref{Exod}{40}{37}{Nu 9:19-\allowbreak22 Ps 31:15}
\crossref{Exod}{40}{38}{Ex 13:21 Nu 9:15}

% Lev
\crossref{Lev}{1}{1}{Ex 19:3; 24:1,\allowbreak2,\allowbreak12; 29:42 Joh 1:17}
\crossref{Lev}{1}{2}{Le 22:18,\allowbreak19 Ge 4:3,\allowbreak5 1Ch 16:29 Ro 12:1,\allowbreak6 Eph 5:2}
\crossref{Lev}{1}{3}{Le 6:9-\allowbreak13; 8:18,\allowbreak21 Ge 8:20; 22:2,\allowbreak8,\allowbreak13 Ex 24:5; 29:18,\allowbreak42; 32:6; 38:1}
\crossref{Lev}{1}{4}{Le 3:2,\allowbreak8,\allowbreak13; 4:4,\allowbreak15,\allowbreak24,\allowbreak29; 8:14,\allowbreak22; 16:21 Ex 29:10,\allowbreak15,\allowbreak19 Nu 8:12}
\crossref{Lev}{1}{5}{1:11; 3:2,\allowbreak8,\allowbreak13; 16:15 2Ch 29:22-\allowbreak24 Mic 6:6}
\crossref{Lev}{1}{6}{Le 7:8 Ge 3:21}
\crossref{Lev}{1}{7}{Le 6:12,\allowbreak13; 9:24; 10:1 1Ch 21:26 2Ch 7:1 Mal 1:10}
\crossref{Lev}{1}{8}{Le 8:18-\allowbreak21; 9:13,\allowbreak14 Ex 29:17,\allowbreak18 1Ki 18:23,\allowbreak33}
\crossref{Lev}{1}{9}{1:13; 8:21; 9:14 Ps 51:6 Jer 4:14 Mt 23:25-\allowbreak28}
\crossref{Lev}{1}{10}{1:2 Ge 4:4; 8:20 Isa 53:6,\allowbreak7 Joh 1:29}
\crossref{Lev}{1}{11}{1:5 Ex 40:22 Eze 8:5}
\crossref{Lev}{1}{12}{1:6-\allowbreak8}
\crossref{Lev}{1}{13}{1:9}
\crossref{Lev}{1}{14}{Le 5:7; 12:8 Mt 11:29 Lu 2:24 2Co 8:12 Heb 7:26}
\crossref{Lev}{1}{15}{Le 5:8 Ps 22:1,\allowbreak21; 69:1-\allowbreak21 Isa 53:4,\allowbreak5,\allowbreak10 Mt 26:1-\allowbreak27:66 1Jo 2:27}
\crossref{Lev}{1}{16}{Lu 1:35 1Pe 1:2}
\crossref{Lev}{1}{17}{Ge 15:10 Ps 16:10 Mt 27:50 Joh 19:30 Ro 4:25 1Pe 1:19-\allowbreak21; 3:18}
\crossref{Lev}{2}{1}{Ex 29:2 Nu 7:13,\allowbreak19 Joe 1:9; 2:14}
\crossref{Lev}{2}{2}{2:9; 5:12; 6:15; 24:7 Ex 30:16 Nu 5:18 Ne 13:14,\allowbreak22 Isa 66:3}
\crossref{Lev}{2}{3}{Le 6:16,\allowbreak17,\allowbreak26; 7:9; 10:12,\allowbreak13; 21:22 Nu 18:9 1Sa 2:28}
\crossref{Lev}{2}{4}{1Ch 23:28,\allowbreak29 Ps 22:14 Eze 46:20 Mt 26:38 Joh 12:27}
\crossref{Lev}{2}{5}{}
\crossref{Lev}{2}{6}{Le 1:6 Ps 22:1-\allowbreak21 Mr 14:1-\allowbreak15:47 Joh 18:1-\allowbreak19:42}
\crossref{Lev}{2}{7}{2:1,\allowbreak2}
\crossref{Lev}{2}{8}{}
\crossref{Lev}{2}{9}{2:2; 6:15}
\crossref{Lev}{2}{10}{2:3}
\crossref{Lev}{2}{11}{Le 6:17 Ex 12:19,\allowbreak20 Mt 16:6,\allowbreak11,\allowbreak12 Mr 8:15 Lu 12:1 1Co 5:6-\allowbreak8}
\crossref{Lev}{2}{12}{Ge 23:10,\allowbreak11,\allowbreak17 Ex 22:29; 23:10,\allowbreak11,\allowbreak19 Nu 15:20 De 26:10}
\crossref{Lev}{2}{13}{Ezr 7:22 Eze 43:24 Mt 5:13 Mr 9:49,\allowbreak50 Col 4:6}
\crossref{Lev}{2}{14}{2Ki 4:42}
\crossref{Lev}{2}{15}{2:1}
\crossref{Lev}{2}{16}{2:1,\allowbreak2,\allowbreak4-\allowbreak7,\allowbreak9,\allowbreak12 Ps 141:2 Isa 11:2-\allowbreak4; 61:1 Ro 8:26,\allowbreak27 Heb 5:7}
\crossref{Lev}{3}{1}{Le 7:11-\allowbreak21,\allowbreak29-\allowbreak34; 22:19-\allowbreak21 Ex 20:24; 24:5; 29:28 Nu 6:14; 7:17}
\crossref{Lev}{3}{2}{Le 1:4,\allowbreak5; 8:22; 16:21,\allowbreak22 Ex 29:10 Isa 53:6 2Co 5:21 1Jo 1:9,\allowbreak10}
\crossref{Lev}{3}{3}{3:16; 4:8,\allowbreak9; 7:3,\allowbreak4 Ex 29:13,\allowbreak22 De 30:6 Ps 119:70 Pr 23:26}
\crossref{Lev}{3}{4}{}
\crossref{Lev}{3}{5}{Le 1:9; 4:31,\allowbreak35; 6:12; 9:9,\allowbreak10 Ex 29:13 1Sa 2:15,\allowbreak16 1Ki 8:64}
\crossref{Lev}{3}{6}{Ga 4:4 Eph 1:10; 2:13-\allowbreak22}
\crossref{Lev}{3}{7}{3:1 1Ki 8:62 Eph 5:2,\allowbreak12 Heb 9:14}
\crossref{Lev}{3}{8}{3:2-\allowbreak5,\allowbreak13; 4:4,\allowbreak15,\allowbreak24 Isa 53:6,\allowbreak11,\allowbreak12 2Co 5:21 1Pe 2:24}
\crossref{Lev}{3}{9}{3:3,\allowbreak4 Pr 23:26 Isa 53:10}
\crossref{Lev}{3}{10}{3:4}
\crossref{Lev}{3}{11}{3:5 Ps 22:14 Isa 53:4-\allowbreak10 Ro 8:32}
\crossref{Lev}{3}{12}{3:1,\allowbreak7-\allowbreak17; 1:2,\allowbreak6,\allowbreak10; 9:3,\allowbreak15; 10:16; 22:19-\allowbreak27 Isa 53:2,\allowbreak6 Mt 25:32,\allowbreak33}
\crossref{Lev}{3}{13}{3:1-\allowbreak5,\allowbreak8 Isa 53:6,\allowbreak11,\allowbreak12 2Co 5:21 1Pe 2:24; 3:18}
\crossref{Lev}{3}{14}{3:3-\allowbreak5,\allowbreak9-\allowbreak11 Ps 22:14,\allowbreak15 Pr 23:26 Jer 20:18 Mt 22:37; 26:38}
\crossref{Lev}{3}{15}{3:4}
\crossref{Lev}{3}{16}{3:11}
\crossref{Lev}{3}{17}{Le 6:18; 7:36; 16:34; 17:7; 23:14 Nu 19:21}
\crossref{Lev}{4}{1}{4:1}
\crossref{Lev}{4}{2}{Le 5:15,\allowbreak17 Nu 15:22-\allowbreak29 De 19:4 1Sa 14:27 Ps 19:12 1Ti 1:13}
\crossref{Lev}{4}{3}{Le 8:12; 21:10-\allowbreak12 Ex 29:7,\allowbreak21}
\crossref{Lev}{4}{4}{Le 1:3 Ex 29:10,\allowbreak11}
\crossref{Lev}{4}{5}{4:16,\allowbreak17; 16:14,\allowbreak19 Nu 19:4 1Jo 1:7}
\crossref{Lev}{4}{6}{4:17,\allowbreak25,\allowbreak30,\allowbreak34; 8:15; 9:9; 16:14,\allowbreak19 Nu 19:4}
\crossref{Lev}{4}{7}{Le 8:15; 9:9; 16:18 Ex 30:1-\allowbreak10 Ps 118:27 Heb 9:21-\allowbreak15}
\crossref{Lev}{4}{8}{4:19,\allowbreak26,\allowbreak31,\allowbreak35; 3:3-\allowbreak5,\allowbreak9-\allowbreak11,\allowbreak14-\allowbreak16; 7:3-\allowbreak5; 16:25 Isa 53:10 Joh 12:27}
\crossref{Lev}{4}{9}{}
\crossref{Lev}{4}{10}{Le 23:19 Ps 32:1 1Ti 2:5,\allowbreak6}
\crossref{Lev}{4}{11}{4:21; 6:30; 8:14-\allowbreak17; 9:8-\allowbreak11; 16:27 Ex 29:14 Nu 19:5 Ps 103:12}
\crossref{Lev}{4}{12}{Le 6:10,\allowbreak11}
\crossref{Lev}{4}{13}{4:1,\allowbreak2; 5:2-\allowbreak5,\allowbreak17 Nu 15:24-\allowbreak29 Jos 7:11,\allowbreak24-\allowbreak26 1Ti 1:13 Heb 10:26-\allowbreak29}
\crossref{Lev}{4}{14}{4:3}
\crossref{Lev}{4}{15}{Ex 24:1,\allowbreak9 Nu 11:16,\allowbreak25 De 21:3-\allowbreak9}
\crossref{Lev}{4}{16}{4:5-\allowbreak12 Heb 9:12-\allowbreak14}
\crossref{Lev}{4}{17}{4:6,\allowbreak7}
\crossref{Lev}{4}{18}{4:7}
\crossref{Lev}{4}{19}{4:8-\allowbreak10,\allowbreak26,\allowbreak31,\allowbreak35; 5:6; 6:7; 12:8; 14:18 Nu 15:25 Ps 22:14 Heb 1:3}
\crossref{Lev}{4}{20}{4:3}
\crossref{Lev}{4}{21}{4:11,\allowbreak12}
\crossref{Lev}{4}{22}{4:2,\allowbreak13}
\crossref{Lev}{4}{23}{4:14; 5:4 2Ki 22:10-\allowbreak13}
\crossref{Lev}{4}{24}{4:4-\allowbreak35 Isa 53:6}
\crossref{Lev}{4}{25}{4:7,\allowbreak18,\allowbreak30,\allowbreak34; 8:10,\allowbreak15; 9:9; 16:18 Isa 40:21 Ro 3:24-\allowbreak26; 8:3,\allowbreak4; 10:4}
\crossref{Lev}{4}{26}{4:8-\allowbreak10,\allowbreak35}
\crossref{Lev}{4}{27}{4:2 Nu 15:27}
\crossref{Lev}{4}{28}{4:23,\allowbreak32; 5:6 Ge 3:15 Isa 7:14 Jer 31:22 Ro 8:3 Ga 4:4,\allowbreak5}
\crossref{Lev}{4}{29}{4:4,\allowbreak15,\allowbreak24,\allowbreak33 Heb 10:4-\allowbreak14}
\crossref{Lev}{4}{30}{4:25,\allowbreak34 Isa 42:21 Ro 8:3,\allowbreak4; 10:4 Heb 2:10}
\crossref{Lev}{4}{31}{4:8-\allowbreak10,\allowbreak19,\allowbreak26,\allowbreak35; 3:3-\allowbreak5,\allowbreak9-\allowbreak11,\allowbreak14-\allowbreak16}
\crossref{Lev}{4}{32}{4:28; 3:6,\allowbreak7; 5:6 Ex 12:3,\allowbreak5 Isa 53:7 Lu 1:35 Joh 1:29,\allowbreak36 Heb 7:26}
\crossref{Lev}{4}{33}{4:4,\allowbreak29-\allowbreak31}
\crossref{Lev}{4}{34}{4:25,\allowbreak30 Isa 42:21 Joh 17:19 Ro 8:1,\allowbreak3; 10:4 2Co 5:21 Heb 2:10}
\crossref{Lev}{4}{35}{4:31}
\crossref{Lev}{5}{1}{5:15,\allowbreak17; 4:2 Eze 18:4,\allowbreak20}
\crossref{Lev}{5}{2}{Le 7:21; 11:24,\allowbreak28,\allowbreak31,\allowbreak39 Nu 19:11-\allowbreak16 De 14:8 Isa 52:11 Hag 2:13}
\crossref{Lev}{5}{3}{Le 12:1-\allowbreak13:59; 15:1-\allowbreak33; 22:4-\allowbreak6 Nu 19:11-\allowbreak16}
\crossref{Lev}{5}{4}{Le 27:2-\allowbreak34 Jos 2:14; 9:15 Jud 9:19; 11:31; 21:7,\allowbreak18 1Sa 1:11}
\crossref{Lev}{5}{5}{Le 16:21; 26:40 Nu 5:7 Jos 7:19 Ezr 10:11,\allowbreak12 Job 33:27 Ps 32:5}
\crossref{Lev}{5}{6}{Le 4:28,\allowbreak32}
\crossref{Lev}{5}{7}{Le 1:14,\allowbreak15 Mt 3:16; 10:16 Lu 2:24}
\crossref{Lev}{5}{8}{Le 1:15 Ro 4:25 1Pe 3:18}
\crossref{Lev}{5}{9}{Le 1:5; 4:25,\allowbreak30,\allowbreak34; 7:2 Ex 12:22,\allowbreak23 Isa 42:21 Heb 2:10; 12:24}
\crossref{Lev}{5}{10}{Le 1:14-\allowbreak17 Eph 5:2}
\crossref{Lev}{5}{11}{5:7}
\crossref{Lev}{5}{12}{Le 2:2,\allowbreak9,\allowbreak16; 6:15 Nu 5:26 Ac 10:4 Eph 5:2}
\crossref{Lev}{5}{13}{5:6; 4:20,\allowbreak26,\allowbreak31}
\crossref{Lev}{5}{14}{}
\crossref{Lev}{5}{15}{5:1,\allowbreak2; 4:2}
\crossref{Lev}{5}{16}{Le 22:14 Ex 22:1,\allowbreak3,\allowbreak4 Ps 69:4 Lu 19:8 Ac 26:20}
\crossref{Lev}{5}{17}{5:15 Ps 19:12 Lu 12:48 Ro 14:23}
\crossref{Lev}{5}{18}{5:15,\allowbreak16}
\crossref{Lev}{5}{19}{Ezr 10:2 Ps 51:4 Mal 3:8 2Co 5:19-\allowbreak21}
\crossref{Lev}{6}{1}{6:1}
\crossref{Lev}{6}{2}{Le 5:15,\allowbreak19 Nu 5:6-\allowbreak8 Ps 51:4}
\crossref{Lev}{6}{3}{Ex 23:4 De 22:1-\allowbreak3}
\crossref{Lev}{6}{4}{Le 4:13-\allowbreak15; 5:3,\allowbreak4}
\crossref{Lev}{6}{5}{Le 5:16 Ex 22:1,\allowbreak4,\allowbreak7,\allowbreak9 Nu 5:7,\allowbreak8 1Sa 12:3 2Sa 12:6 Pr 6:30,\allowbreak31}
\crossref{Lev}{6}{6}{Le 5:15,\allowbreak18 Isa 53:10,\allowbreak11}
\crossref{Lev}{6}{7}{Le 4:20,\allowbreak26,\allowbreak31; 5:10,\allowbreak13,\allowbreak15,\allowbreak16,\allowbreak18 Ex 34:7 Eze 18:21-\allowbreak23,\allowbreak26,\allowbreak27}
\crossref{Lev}{6}{8}{6:8}
\crossref{Lev}{6}{9}{Le 1:1-\allowbreak17 Ex 29:38-\allowbreak42 Nu 28:3}
\crossref{Lev}{6}{10}{Le 16:4 Ex 28:39-\allowbreak43; 39:27-\allowbreak29 Eze 44:17,\allowbreak18 Re 7:13; 19:8,\allowbreak14}
\crossref{Lev}{6}{11}{Le 16:23,\allowbreak24 Eze 44:19}
\crossref{Lev}{6}{12}{Le 9:24 Nu 4:13,\allowbreak14 Mr 9:48,\allowbreak49 Heb 10:27}
\crossref{Lev}{6}{13}{Isa 6:6,\allowbreak7 Re 8:5}
\crossref{Lev}{6}{14}{Le 2:1,\allowbreak2 Nu 15:4,\allowbreak6,\allowbreak9 Joh 6:32}
\crossref{Lev}{6}{15}{Le 2:2,\allowbreak9}
\crossref{Lev}{6}{16}{Le 2:3,\allowbreak10; 5:13 Eze 44:29 1Co 9:13-\allowbreak15}
\crossref{Lev}{6}{17}{Le 2:11 1Pe 2:22}
\crossref{Lev}{6}{18}{6:29; 21:21,\allowbreak22 Nu 18:10}
\crossref{Lev}{6}{19}{}
\crossref{Lev}{6}{20}{}
\crossref{Lev}{6}{21}{Le 2:5; 7:9 1Ch 9:31}
\crossref{Lev}{6}{22}{Le 4:3 De 10:6 Heb 7:23}
\crossref{Lev}{6}{23}{6:16,\allowbreak17; 2:10}
\crossref{Lev}{6}{24}{}
\crossref{Lev}{6}{25}{Le 4:2,\allowbreak3-\allowbreak20,\allowbreak21,\allowbreak24,\allowbreak33,\allowbreak34}
\crossref{Lev}{6}{26}{Le 10:17,\allowbreak18 Nu 18:9,\allowbreak10 Eze 44:28,\allowbreak29; 46:20 Ho 4:8}
\crossref{Lev}{6}{27}{6:18 Ex 29:37; 30:29 Hag 2:12 Mt 9:21; 14:36}
\crossref{Lev}{6}{28}{Le 11:33; 15:12 Heb 9:9,\allowbreak10}
\crossref{Lev}{6}{29}{6:18 Nu 18:10}
\crossref{Lev}{6}{30}{Le 4:3-\allowbreak21; 10:18; 16:27,\allowbreak28 Heb 9:11,\allowbreak12; 13:11}
\crossref{Lev}{7}{1}{Le 5:1-\allowbreak6:7; 14:12,\allowbreak13 19:21,\allowbreak22 Nu 6:12 Eze 40:39; 44:29; 46:20}
\crossref{Lev}{7}{2}{Le 1:3,\allowbreak5,\allowbreak11; 4:24,\allowbreak29,\allowbreak33; 6:25 Nu 6:12 Eze 40:39}
\crossref{Lev}{7}{3}{Le 3:3-\allowbreak5,\allowbreak9-\allowbreak11,\allowbreak15,\allowbreak16; 4:8-\allowbreak10 Ex 29:13 Ps 51:6,\allowbreak17}
\crossref{Lev}{7}{4}{7:4}
\crossref{Lev}{7}{5}{Le 1:9,\allowbreak13; 2:2,\allowbreak9,\allowbreak16; 3:16 Ga 2:20; 5:24 1Pe 4:1,\allowbreak2}
\crossref{Lev}{7}{6}{Le 6:16-\allowbreak18,\allowbreak29 Nu 18:9,\allowbreak10}
\crossref{Lev}{7}{7}{Le 6:25,\allowbreak26; 14:13}
\crossref{Lev}{7}{8}{Le 1:6; 4:11 Ge 3:21 Ex 29:14 Nu 19:5 Ro 13:14}
\crossref{Lev}{7}{9}{Le 2:4-\allowbreak7 Nu 18:9 Eze 44:29}
\crossref{Lev}{7}{10}{Ex 16:18 2Co 8:14}
\crossref{Lev}{7}{11}{Le 3:1-\allowbreak17; 22:18-\allowbreak21 Eze 45:15}
\crossref{Lev}{7}{12}{Le 22:29 2Ch 29:31; 33:16 Ne 12:43 Ps 50:13,\allowbreak14,\allowbreak23; 103:1,\allowbreak2}
\crossref{Lev}{7}{13}{Le 23:17 Am 4:5 Mt 13:33 1Ti 4:4}
\crossref{Lev}{7}{14}{Ex 29:27,\allowbreak28 Nu 15:19-\allowbreak21; 18:24-\allowbreak28; 31:29,\allowbreak41}
\crossref{Lev}{7}{15}{Le 22:29,\allowbreak30}
\crossref{Lev}{7}{16}{Le 22:18-\allowbreak21; 23:38 Nu 15:3 De 12:6,\allowbreak11,\allowbreak17,\allowbreak26 Ps 66:13; 116:14,\allowbreak18}
\crossref{Lev}{7}{17}{Le 19:7 Ge 22:4 Ex 19:11 Ho 6:2 1Co 15:4}
\crossref{Lev}{7}{18}{Le 10:19; 19:7,\allowbreak8; 22:23,\allowbreak25 Jer 14:10,\allowbreak12 Ho 8:13 Am 5:22}
\crossref{Lev}{7}{19}{Le 11:24-\allowbreak39 Nu 19:11-\allowbreak16 Lu 11:41 Ac 10:15,\allowbreak16,\allowbreak28 Ro 14:14,\allowbreak20}
\crossref{Lev}{7}{20}{Le 15:2,\allowbreak3-\allowbreak33 1Co 11:28}
\crossref{Lev}{7}{21}{Le 5:2,\allowbreak3; 12,\allowbreak1-\allowbreak13:59; 15:1-\allowbreak33; 22:4 Nu 19:11-\allowbreak16}
\crossref{Lev}{7}{22}{}
\crossref{Lev}{7}{23}{}
\crossref{Lev}{7}{24}{Le 17:15; 22:8 Ex 22:31 De 14:21 Eze 4:14; 44:31}
\crossref{Lev}{7}{25}{7:21}
\crossref{Lev}{7}{26}{}
\crossref{Lev}{7}{27}{7:20,\allowbreak21,\allowbreak25 Heb 10:29}
\crossref{Lev}{7}{28}{7:28}
\crossref{Lev}{7}{29}{Le 3:1-\allowbreak17 Col 1:20 1Jo 1:7}
\crossref{Lev}{7}{30}{Le 3:3,\allowbreak4,\allowbreak9,\allowbreak14 Ps 110:3 Joh 10:18 2Co 8:12}
\crossref{Lev}{7}{31}{Le 3:5,\allowbreak11,\allowbreak16}
\crossref{Lev}{7}{32}{7:34; 8:25,\allowbreak26; 9:21; 10:14 Nu 6:20; 18:18,\allowbreak19 De 18:3 1Co 9:13,\allowbreak14}
\crossref{Lev}{7}{33}{7:3; 6:1-\allowbreak30; 26:1-\allowbreak46}
\crossref{Lev}{7}{34}{7:30-\allowbreak32; 10:14,\allowbreak15 Ex 29:28 Nu 18:18,\allowbreak19 De 18:3}
\crossref{Lev}{7}{35}{Le 8:10-\allowbreak12,\allowbreak30 Ex 29:7,\allowbreak21; 40:13-\allowbreak15 Isa 10:27; 61:1 Joh 3:34}
\crossref{Lev}{7}{36}{Le 8:12,\allowbreak30 Ex 40:13,\allowbreak15}
\crossref{Lev}{7}{37}{Le 1:1-\allowbreak17; 6:9-\allowbreak13 Ex 29:38-\allowbreak42}
\crossref{Lev}{7}{38}{Le 1:1,\allowbreak2}
\crossref{Lev}{8}{1}{8:1}
\crossref{Lev}{8}{2}{Ex 29:1-\allowbreak4}
\crossref{Lev}{8}{3}{Nu 20:8; 21:16 1Ch 13:5; 15:3 2Ch 5:2,\allowbreak6; 30:2,\allowbreak13,\allowbreak25 Ne 8:1}
\crossref{Lev}{8}{4}{8:9,\allowbreak13,\allowbreak17,\allowbreak29,\allowbreak35 Ex 39:1,\allowbreak5,\allowbreak7,\allowbreak21,\allowbreak26,\allowbreak29,\allowbreak31,\allowbreak32,\allowbreak42,\allowbreak43 De 12:32}
\crossref{Lev}{8}{5}{Ex 29:4-\allowbreak37}
\crossref{Lev}{8}{6}{Ex 29:4; 40:12 Ps 51:2,\allowbreak7 Isa 1:16 Eze 36:25 Zec 13:1}
\crossref{Lev}{8}{7}{Ex 28:4; 29:5; 39:1-\allowbreak7 Isa 61:3,\allowbreak10 Ro 3:22; 13:14 Ga 3:27}
\crossref{Lev}{8}{8}{Ex 28:15-\allowbreak29; 39:8-\allowbreak21 So 8:6 Isa 59:17 Eph 6:14 1Th 5:8}
\crossref{Lev}{8}{9}{Ex 28:4,\allowbreak36-\allowbreak38; 29:6; 39:28-\allowbreak30 Zec 3:5; 6:11-\allowbreak14 Php 2:9-\allowbreak11}
\crossref{Lev}{8}{10}{Ex 30:23-\allowbreak29; 40:9-\allowbreak11}
\crossref{Lev}{8}{11}{Ex 4:6,\allowbreak17; 16:14,\allowbreak19 Isa 52:15 Eze 36:25 Tit 3:6}
\crossref{Lev}{8}{12}{Le 4:3; 21:10,\allowbreak11,\allowbreak12 Ex 28:41; 29:7; 30:30 Ps 133:2}
\crossref{Lev}{8}{13}{Ex 28:40,\allowbreak41; 29:8,\allowbreak9; 40:14,\allowbreak15 Ps 132:9 Isa 61:6,\allowbreak10 1Pe 2:5,\allowbreak9}
\crossref{Lev}{8}{14}{8:2; 4:3-\allowbreak12; 16:6 Ex 29:10-\allowbreak14 Isa 53:10 Eze 43:19 Ro 8:3 2Co 5:21}
\crossref{Lev}{8}{15}{Le 1:5,\allowbreak11; 3:2,\allowbreak8 Ex 29:10,\allowbreak11}
\crossref{Lev}{8}{16}{Le 3:3-\allowbreak5; 4:8,\allowbreak9 Ex 29:13}
\crossref{Lev}{8}{17}{Le 4:11,\allowbreak12,\allowbreak21; 6:30; 16:27 Ex 29:14 Ga 3:13 Heb 13:11-\allowbreak13}
\crossref{Lev}{8}{18}{Le 1:4-\allowbreak13 Ex 29:15-\allowbreak18}
\crossref{Lev}{8}{19}{8:19}
\crossref{Lev}{8}{20}{}
\crossref{Lev}{8}{21}{Le 1:17; 2:9 Ge 8:21 Ex 29:18 Eph 5:2}
\crossref{Lev}{8}{22}{8:2,\allowbreak29; 7:37 Ex 29:19-\allowbreak31 Joh 17:19 1Co 1:30 2Co 5:21}
\crossref{Lev}{8}{23}{Le 14:14,\allowbreak17,\allowbreak28 Ex 29:20 Ro 6:13,\allowbreak19; 12:1 1Co 1:2,\allowbreak30; 6:20}
\crossref{Lev}{8}{24}{Heb 9:22}
\crossref{Lev}{8}{25}{Le 3:3-\allowbreak5,\allowbreak9 Ex 29:22-\allowbreak25 Pr 23:26 Isa 53:10}
\crossref{Lev}{8}{26}{Ex 29:23 Joh 1:14 Ac 5:12 1Ti 2:5}
\crossref{Lev}{8}{27}{Ex 29:24-\allowbreak37 Jer 30:21 Heb 9:14}
\crossref{Lev}{8}{28}{Ex 29:25 Ps 22:13,\allowbreak14 Zec 13:7 Heb 10:14-\allowbreak22}
\crossref{Lev}{8}{29}{Le 7:30-\allowbreak34 Ex 29:26,\allowbreak27 Isa 66:20 1Co 10:31 1Pe 4:11}
\crossref{Lev}{8}{30}{Ex 29:21; 30:30 Isa 61:1,\allowbreak3 Ga 5:22-\allowbreak25 Heb 2:11 1Pe 1:2}
\crossref{Lev}{8}{31}{Le 6:28; 7:15 Ex 29:31,\allowbreak32 De 12:6,\allowbreak7 1Sa 2:13-\allowbreak17 Eze 46:20-\allowbreak24}
\crossref{Lev}{8}{32}{Le 7:17 Ex 12:10; 29:34 Pr 27:1 Ec 9:10 2Co 6:2 Heb 3:13,\allowbreak14}
\crossref{Lev}{8}{33}{Le 14:8 Ex 29:30,\allowbreak35 Nu 19:12 Eze 43:25-\allowbreak27}
\crossref{Lev}{8}{34}{Heb 7:16,\allowbreak27; 10:11,\allowbreak12}
\crossref{Lev}{8}{35}{Le 14:8 Ex 29:35 Nu 19:12 Eze 43:25 2Co 7:1 Col 2:9,\allowbreak10 Heb 7:28}
\crossref{Lev}{8}{36}{}
\crossref{Lev}{9}{1}{}
\crossref{Lev}{9}{2}{9:7,\allowbreak8; 4:3; 8:14 Ex 29:1 2Co 5:21 Heb 5:3; 7:27; 10:10-\allowbreak14}
\crossref{Lev}{9}{3}{Le 4:23; 16:5,\allowbreak15 Ezr 6:17; 10:19 Isa 53:10 Ro 8:3 2Co 5:21}
\crossref{Lev}{9}{4}{Le 3:1-\allowbreak17}
\crossref{Lev}{9}{5}{Ex 19:17 De 31:12 1Ch 15:3 2Ch 5:2,\allowbreak3 Ne 8:1}
\crossref{Lev}{9}{6}{9:23 Ex 16:10; 24:16; 40:34,\allowbreak35 1Ki 8:10-\allowbreak12 2Ch 5:13,\allowbreak14}
\crossref{Lev}{9}{7}{9:2; 4:3,\allowbreak20; 8:34 1Sa 3:14 Heb 5:3; 7:27,\allowbreak28; 9:7}
\crossref{Lev}{9}{8}{Le 1:4,\allowbreak5; 4:4,\allowbreak29}
\crossref{Lev}{9}{9}{Le 4:6,\allowbreak7,\allowbreak17,\allowbreak18,\allowbreak25,\allowbreak30; 8:15; 16:18 Heb 2:10; 9:22,\allowbreak23; 10:4-\allowbreak19}
\crossref{Lev}{9}{10}{Le 3:3-\allowbreak5,\allowbreak9-\allowbreak11; 4:8-\allowbreak12,\allowbreak34,\allowbreak35; 8:16,\allowbreak17 Ps 51:17 Pr 23:26 Isa 53:10}
\crossref{Lev}{9}{11}{Le 4:11,\allowbreak12,\allowbreak21; 8:17; 16:27,\allowbreak28 Heb 13:11,\allowbreak12}
\crossref{Lev}{9}{12}{Le 1:1-\allowbreak17; 8:18-\allowbreak21 Eph 5:2,\allowbreak25-\allowbreak27}
\crossref{Lev}{9}{13}{9:13}
\crossref{Lev}{9}{14}{Le 8:21}
\crossref{Lev}{9}{15}{9:3; 4:27-\allowbreak31; 9:15 Nu 28:1-\allowbreak29:31 Isa 53:10 2Co 5:21 Tit 2:14}
\crossref{Lev}{9}{16}{9:12-\allowbreak14; 1:3-\allowbreak10; 8:18-\allowbreak21 Heb 10:1-\allowbreak22}
\crossref{Lev}{9}{17}{9:1; 2:1,\allowbreak2 Ex 29:38,\allowbreak41 Joh 6:53 Ga 2:20}
\crossref{Lev}{9}{18}{Le 3:1-\allowbreak17; 7:11-\allowbreak18 Ro 5:1,\allowbreak10 Eph 2:14-\allowbreak17 Col 1:20}
\crossref{Lev}{9}{19}{9:10; 3:5,\allowbreak16}
\crossref{Lev}{9}{20}{Le 7:29-\allowbreak34}
\crossref{Lev}{9}{21}{Le 7:24,\allowbreak26,\allowbreak30-\allowbreak34 Ex 29:27,\allowbreak28 Isa 49:3 Lu 2:13 1Pe 4:11}
\crossref{Lev}{9}{22}{}
\crossref{Lev}{9}{23}{Lu 1:21,\allowbreak22 Heb 9:24-\allowbreak28}
\crossref{Lev}{9}{24}{Ge 17:3 Nu 14:5; 16:22 1Ki 18:39 2Ch 7:3 Ezr 3:11 Mt 26:39}
\crossref{Lev}{10}{1}{Le 16:1; 22:9 Ex 6:23; 24:1,\allowbreak9; 28:1 Nu 3:3,\allowbreak4; 26:61}
\crossref{Lev}{10}{2}{Le 9:24; 16:1 Nu 3:3,\allowbreak4; 16:35; 26:61 2Sa 6:7 2Ki 1:10,\allowbreak12 1Ch 24:2}
\crossref{Lev}{10}{3}{Le 8:35; 21:6,\allowbreak8,\allowbreak15,\allowbreak17,\allowbreak21; 22:9 Ex 14:4; 19:22; 29:43,\allowbreak44 Nu 20:12}
\crossref{Lev}{10}{4}{Ex 6:18,\allowbreak22 Nu 3:19,\allowbreak30 1Ch 6:2}
\crossref{Lev}{10}{5}{}
\crossref{Lev}{10}{6}{Le 13:45; 21:1-\allowbreak15 Ex 33:5 Nu 5:18; 6:6,\allowbreak7; 14:6 De 33:9 Jer 7:29}
\crossref{Lev}{10}{7}{Le 21:12 Mt 8:21,\allowbreak22 Lu 9:60}
\crossref{Lev}{10}{8}{}
\crossref{Lev}{10}{9}{Nu 6:3,\allowbreak20 Pr 31:4,\allowbreak5 Isa 28:7 Jer 35:5,\allowbreak6 Eze 44:21 Lu 1:15}
\crossref{Lev}{10}{10}{Le 11:47; 20:25,\allowbreak26 Jer 15:19 Eze 22:26; 44:23 Tit 1:15 1Pe 1:14-\allowbreak16}
\crossref{Lev}{10}{11}{De 24:8; 33:10 2Ch 17:9; 30:22 Ne 8:2,\allowbreak8; 9:13,\allowbreak14 Jer 2:8; 18:18}
\crossref{Lev}{10}{12}{Le 2:1-\allowbreak16; 6:15-\allowbreak18; 7:9; 21:22 Ex 29:2 Nu 18:9,\allowbreak10 Eze 44:29}
\crossref{Lev}{10}{13}{Nu 18:10}
\crossref{Lev}{10}{14}{Le 7:29-\allowbreak34; 9:21 Ex 29:24-\allowbreak28 Nu 18:11 Joh 4:34}
\crossref{Lev}{10}{15}{Le 7:29,\allowbreak30,\allowbreak34}
\crossref{Lev}{10}{16}{Le 6:26,\allowbreak30; 9:3,\allowbreak15}
\crossref{Lev}{10}{17}{Le 6:26,\allowbreak29; 7:6,\allowbreak7}
\crossref{Lev}{10}{18}{Le 6:30}
\crossref{Lev}{10}{19}{Le 9:8,\allowbreak12 Heb 7:27; 9:8}
\crossref{Lev}{10}{20}{2Ch 30:18-\allowbreak20 Zec 7:8,\allowbreak9 Mt 12:3-\allowbreak7,\allowbreak20}
\crossref{Lev}{11}{1}{11:1}
\crossref{Lev}{11}{2}{De 14:3-\allowbreak8 Eze 4:14 Da 1:8 Mt 15:11 Mr 7:15-\allowbreak19 Ac 10:12,\allowbreak14}
\crossref{Lev}{11}{3}{Ps 1:1 Pr 9:6 2Co 6:17}
\crossref{Lev}{11}{4}{Ge 7:1,\allowbreak2 De 14:1-\allowbreak29 Isa 52:11 1Co 8:13 1Th 5:22 1Jo 3:4}
\crossref{Lev}{11}{5}{Job 36:14 Mt 7:26 Ro 2:18-\allowbreak24 Php 3:18,\allowbreak19 2Ti 3:5 Tit 1:16}
\crossref{Lev}{11}{6}{De 14:7}
\crossref{Lev}{11}{7}{De 14:8 Isa 65:4; 66:3,\allowbreak17 Mt 7:6 Lu 8:33; 15:15 2Pe 2:18-\allowbreak22}
\crossref{Lev}{11}{8}{Le 5:2 Isa 52:11 Ho 9:3 Mt 15:11,\allowbreak20 Mr 7:2,\allowbreak15,\allowbreak18 Ac 10:10-\allowbreak15}
\crossref{Lev}{11}{9}{De 14:9,\allowbreak10 Ac 20:21 Ga 5:6 Jas 2:18 1Jo 5:2-\allowbreak5}
\crossref{Lev}{11}{10}{Le 7:18 De 14:3 Ps 139:21,\allowbreak22 Pr 13:20; 29:27 Re 21:8}
\crossref{Lev}{11}{11}{11:11}
\crossref{Lev}{11}{12}{}
\crossref{Lev}{11}{13}{}
\crossref{Lev}{11}{14}{11:14}
\crossref{Lev}{11}{15}{Ge 8:7 1Ki 17:4,\allowbreak6 Pr 30:17 Lu 12:24}
\crossref{Lev}{11}{16}{De 14:15-\allowbreak18 Ps 102:6 Isa 13:21,\allowbreak22; 34:11-\allowbreak15 Joh 3:19-\allowbreak21}
\crossref{Lev}{11}{17}{11:17}
\crossref{Lev}{11}{18}{}
\crossref{Lev}{11}{19}{Isa 2:20; 66:17}
\crossref{Lev}{11}{20}{11:23,\allowbreak27 De 14:19 2Ki 17:28-\allowbreak41 Ps 17:14 Mt 6:24 Php 3:18,\allowbreak19}
\crossref{Lev}{11}{21}{11:21}
\crossref{Lev}{11}{22}{Ex 10:4,\allowbreak5 Isa 35:3 Mt 3:4 Mr 1:6 Ro 14:1; 15:1 Heb 5:11}
\crossref{Lev}{11}{23}{11:23}
\crossref{Lev}{11}{24}{11:8,\allowbreak27,\allowbreak28,\allowbreak31,\allowbreak38-\allowbreak40; 17:15,\allowbreak16 Isa 22:14 1Co 15:33 2Co 6:17}
\crossref{Lev}{11}{25}{11:28,\allowbreak40; 14:8; 15:5,\allowbreak7-\allowbreak11,\allowbreak13; 16:28 Ex 19:10,\allowbreak14 Nu 19:8,\allowbreak10,\allowbreak19,\allowbreak21,\allowbreak22}
\crossref{Lev}{11}{26}{}
\crossref{Lev}{11}{27}{11:20,\allowbreak23}
\crossref{Lev}{11}{28}{11:24,\allowbreak25}
\crossref{Lev}{11}{29}{11:20,\allowbreak21,\allowbreak41,\allowbreak42 Ps 10:3; 17:13,\allowbreak14 Hag 2:6 Lu 12:15; 16:14 Joh 6:26}
\crossref{Lev}{11}{30}{11:30}
\crossref{Lev}{11}{31}{11:8,\allowbreak24,\allowbreak25}
\crossref{Lev}{11}{32}{Le 6:28; 15:12 Tit 2:14; 3:5}
\crossref{Lev}{11}{33}{11:35; 14:45 Jer 48:38 2Co 5:1-\allowbreak8 Php 3:21}
\crossref{Lev}{11}{34}{Pr 15:8; 21:4,\allowbreak27; 28:8 Tit 1:15}
\crossref{Lev}{11}{35}{11:33; 6:28; 15:12 2Co 5:1-\allowbreak7}
\crossref{Lev}{11}{36}{Zec 13:1 Joh 4:14}
\crossref{Lev}{11}{37}{1Co 15:37 1Pe 1:23 1Jo 3:9; 5:18}
\crossref{Lev}{11}{38}{11:38}
\crossref{Lev}{11}{39}{11:24,\allowbreak28,\allowbreak31,\allowbreak40; 15:5,\allowbreak7 Nu 19:11,\allowbreak16}
\crossref{Lev}{11}{40}{11:25; 17:15,\allowbreak16; 22:8 Ex 22:31 De 14:21 Isa 1:16 Eze 4:14; 36:25}
\crossref{Lev}{11}{41}{11:20,\allowbreak23,\allowbreak29}
\crossref{Lev}{11}{42}{Ge 3:14,\allowbreak15 Isa 65:25 Mic 7:17 Mt 3:7; 23:23 Joh 8:44}
\crossref{Lev}{11}{43}{11:41,\allowbreak42; 20:25}
\crossref{Lev}{11}{44}{Ex 20:2}
\crossref{Lev}{11}{45}{Ex 6:7 Ps 105:43-\allowbreak45}
\crossref{Lev}{11}{46}{}
\crossref{Lev}{11}{47}{Le 10:10 Eze 44:23 Mal 3:18 Ro 14:2,\allowbreak3,\allowbreak13-\allowbreak23}
\crossref{Lev}{12}{1}{12:1}
\crossref{Lev}{12}{2}{Ge 1:28; 3:16 Job 14:4; 15:14; 25:4 Ps 51:5 Lu 2:22 Ro 5:12-\allowbreak19}
\crossref{Lev}{12}{3}{Ge 17:11,\allowbreak12 De 30:6 Lu 1:59; 2:21 Joh 7:22,\allowbreak23 Ro 3:19; 4:11,\allowbreak12}
\crossref{Lev}{12}{4}{Le 15:25-\allowbreak28 Hag 2:13 Lu 2:22,\allowbreak23}
\crossref{Lev}{12}{5}{12:2,\allowbreak4 Ge 3:13 1Ti 2:14,\allowbreak15}
\crossref{Lev}{12}{6}{Le 1:10-\allowbreak13; 5:6-\allowbreak10; 14:22; 15:14,\allowbreak29 Nu 6:10 Lu 2:22 Joh 1:29}
\crossref{Lev}{12}{7}{Le 1:4; 4:20,\allowbreak26,\allowbreak31,\allowbreak35 Job 1:5; 14:4 Ro 3:23,\allowbreak26 1Co 7:14}
\crossref{Lev}{12}{8}{Le 4:26}
\crossref{Lev}{13}{1}{}
\crossref{Lev}{13}{2}{Le 14:56 De 28:27 Isa 3:17}
\crossref{Lev}{13}{3}{13:2; 10:10 Eze 44:23 Hag 2:11 Mal 2:7 Ac 20:28 Ro 3:19,\allowbreak20; 7:7}
\crossref{Lev}{13}{4}{Nu 12:15 De 13:14 Eze 44:10 1Co 4:5 1Ti 5:24}
\crossref{Lev}{13}{5}{}
\crossref{Lev}{13}{6}{Isa 11:3,\allowbreak4; 42:3 Ro 14:1 Jude 1:22,\allowbreak23 De 32:5 Jas 3:2}
\crossref{Lev}{13}{7}{13:27,\allowbreak35,\allowbreak36 Ps 38:3 Isa 1:5,\allowbreak6 Ro 6:12-\allowbreak14 2Ti 2:16,\allowbreak17}
\crossref{Lev}{13}{8}{13:3 Mt 15:7,\allowbreak8 Ac 8:21 Php 3:18,\allowbreak19 2Pe 2:19}
\crossref{Lev}{13}{9}{}
\crossref{Lev}{13}{10}{13:3,\allowbreak4 Nu 12:10-\allowbreak12 2Ki 5:27 2Ch 26:19,\allowbreak20}
\crossref{Lev}{13}{11}{Mt 8:2-\allowbreak4 Lu 5:14}
\crossref{Lev}{13}{12}{1Ki 8:38 Job 40:4; 42:6 Isa 64:6 Joh 16:8,\allowbreak9 Ro 7:14 1Jo 1:8-\allowbreak10}
\crossref{Lev}{13}{13}{Isa 64:6 Joh 9:41}
\crossref{Lev}{13}{14}{}
\crossref{Lev}{13}{15}{Nu 22:34}
\crossref{Lev}{13}{16}{Ro 7:14-\allowbreak24 Ga 1:14-\allowbreak16 Php 3:6-\allowbreak8 1Ti 1:13-\allowbreak15}
\crossref{Lev}{13}{17}{}
\crossref{Lev}{13}{18}{Ex 9:9; 15:26 2Ki 20:7 Job 2:7 Ps 38:3-\allowbreak7 Isa 38:21}
\crossref{Lev}{13}{19}{}
\crossref{Lev}{13}{20}{13:3 Mt 12:45 Joh 5:14 2Pe 2:20}
\crossref{Lev}{13}{21}{1Co 5:5}
\crossref{Lev}{13}{22}{13:22}
\crossref{Lev}{13}{23}{Ge 38:26 2Sa 12:13 2Ch 19:2,\allowbreak3 Job 34:31,\allowbreak32; 40:4,\allowbreak5 Pr 28:13}
\crossref{Lev}{13}{24}{Isa 3:24}
\crossref{Lev}{13}{25}{13:4,\allowbreak18-\allowbreak20}
\crossref{Lev}{13}{26}{13:4,\allowbreak5,\allowbreak23}
\crossref{Lev}{13}{27}{13:2}
\crossref{Lev}{13}{28}{Ps 38:3-\allowbreak7,\allowbreak11 Jer 3:12-\allowbreak14; 8:4-\allowbreak6 Re 2:5}
\crossref{Lev}{13}{29}{1Ki 8:38; 12:28 2Ch 6:29 Ps 53:4 Isa 1:5; 5:20; 9:15 Mic 3:11}
\crossref{Lev}{13}{30}{13:34-\allowbreak37; 14:54}
\crossref{Lev}{13}{31}{13:4-\allowbreak6}
\crossref{Lev}{13}{32}{13:30 Mt 23:5 Lu 18:9-\allowbreak12 Ro 2:23}
\crossref{Lev}{13}{33}{1Pe 5:6}
\crossref{Lev}{13}{34}{1Jo 4:1 Jude 1:22 Re 2:2}
\crossref{Lev}{13}{35}{13:7,\allowbreak27 2Ti 2:16,\allowbreak17; 3:13}
\crossref{Lev}{13}{36}{}
\crossref{Lev}{13}{37}{Joh 5:22}
\crossref{Lev}{13}{38}{}
\crossref{Lev}{13}{39}{Ec 7:20 Ro 7:22-\allowbreak25 Jas 3:2}
\crossref{Lev}{13}{40}{13:41 So 5:11 Ro 6:12,\allowbreak19; 8:10 Ga 4:13}
\crossref{Lev}{13}{41}{}
\crossref{Lev}{13}{42}{2Ch 26:16-\allowbreak20}
\crossref{Lev}{13}{43}{}
\crossref{Lev}{13}{44}{Job 36:14 Mt 6:23 2Pe 2:1,\allowbreak2 2Jo 1:8-\allowbreak10}
\crossref{Lev}{13}{45}{Ge 37:29 2Sa 13:19 Job 1:20 Jer 3:25; 36:24 Joe 2:13}
\crossref{Lev}{13}{46}{Pr 30:12}
\crossref{Lev}{13}{47}{}
\crossref{Lev}{13}{48}{13:51 De 8:11 Jude 1:23 Re 3:4}
\crossref{Lev}{13}{49}{}
\crossref{Lev}{13}{50}{Eze 44:23}
\crossref{Lev}{13}{51}{Le 14:44}
\crossref{Lev}{13}{52}{Le 11:33,\allowbreak35 De 7:25,\allowbreak26 Isa 30:22 Ac 19:19,\allowbreak20}
\crossref{Lev}{13}{53}{}
\crossref{Lev}{13}{54}{Hag 1:6}
\crossref{Lev}{13}{55}{Eze 24:13 Heb 6:4-\allowbreak8 2Pe 1:9; 2:20-\allowbreak22}
\crossref{Lev}{13}{56}{Eph 4:25}
\crossref{Lev}{13}{57}{Isa 33:14 Mt 3:12; 22:7; 25:41 Re 21:8,\allowbreak27}
\crossref{Lev}{13}{58}{2Ki 5:10,\allowbreak14 Ps 51:2 2Co 7:1; 12:8 Heb 9:10 Re 1:5}
\crossref{Lev}{13}{59}{1Co 5:3-\allowbreak5 Re 19:8}
\crossref{Lev}{14}{1}{}
\crossref{Lev}{14}{2}{14:54-\allowbreak57; 13:59}
\crossref{Lev}{14}{3}{Le 13:46}
\crossref{Lev}{14}{4}{14:6,\allowbreak49-\allowbreak52 Nu 19:6}
\crossref{Lev}{14}{5}{14:50 Nu 5:17 2Co 4:7; 5:1; 13:4 Heb 2:14}
\crossref{Lev}{14}{6}{Joh 14:19 Ro 4:25; 5:10 Php 2:9-\allowbreak11 Heb 1:3 Re 1:18}
\crossref{Lev}{14}{7}{Nu 19:18,\allowbreak19 Isa 52:15 Eze 36:25 Joh 19:34 Heb 9:13,\allowbreak19,\allowbreak21}
\crossref{Lev}{14}{8}{Le 11:25; 13:6; 15:5-\allowbreak8 Ex 19:10,\allowbreak14 Nu 8:7 Re 7:14}
\crossref{Lev}{14}{9}{Nu 6:9; 8:7}
\crossref{Lev}{14}{10}{14:23; 9:1; 15:13,\allowbreak14}
\crossref{Lev}{14}{11}{Le 8:3 Ex 29:1-\allowbreak4 Nu 8:6-\allowbreak11,\allowbreak21 Eph 5:26,\allowbreak27 Jude 1:24}
\crossref{Lev}{14}{12}{Le 5:2,\allowbreak3,\allowbreak6,\allowbreak7,\allowbreak18,\allowbreak19; 6:6,\allowbreak7 Isa 53:10}
\crossref{Lev}{14}{13}{Le 1:5,\allowbreak11; 4:4,\allowbreak24 Ex 29:11}
\crossref{Lev}{14}{14}{Le 8:23,\allowbreak24 Ex 29:20 Isa 1:5 Ro 6:13,\allowbreak19; 12:1 1Co 6:20 2Co 7:1}
\crossref{Lev}{14}{15}{Ps 45:7 Joh 3:34 1Jo 2:20}
\crossref{Lev}{14}{16}{Le 4:6,\allowbreak17 Lu 17:18 1Co 10:31}
\crossref{Lev}{14}{17}{14:14; 8:30 Ex 29:20,\allowbreak21 Eze 36:27 Joh 1:16 Tit 3:3-\allowbreak6 1Pe 1:2}
\crossref{Lev}{14}{18}{Le 8:12 Ex 29:7 2Co 1:21,\allowbreak22}
\crossref{Lev}{14}{19}{14:12; 5:1,\allowbreak6; 12:6-\allowbreak8 Ro 8:3 2Co 5:21}
\crossref{Lev}{14}{20}{14:10 Eph 5:2}
\crossref{Lev}{14}{21}{Le 1:14; 5:7; 12:8 1Sa 2:8 Job 34:19 Pr 17:5; 22:2 Lu 6:20; 21:2-\allowbreak4}
\crossref{Lev}{14}{22}{}
\crossref{Lev}{14}{23}{}
\crossref{Lev}{14}{24}{14:10-\allowbreak13}
\crossref{Lev}{14}{25}{14:14-\allowbreak20 Ps 40:6 Ec 5:1}
\crossref{Lev}{14}{26}{14:26}
\crossref{Lev}{14}{27}{14:27}
\crossref{Lev}{14}{28}{14:28}
\crossref{Lev}{14}{29}{14:18,\allowbreak20 Ex 30:15,\allowbreak16 Joh 17:19 1Jo 2:1,\allowbreak2; 5:6}
\crossref{Lev}{14}{30}{14:22; 12:8; 15:14,\allowbreak15 Lu 2:24 Ro 8:3}
\crossref{Lev}{14}{31}{14:31}
\crossref{Lev}{14}{32}{14:2,\allowbreak54-\allowbreak57; 13:59}
\crossref{Lev}{14}{33}{}
\crossref{Lev}{14}{34}{Le 23:10; 25:2 Nu 35:10 De 7:2; 12:1,\allowbreak8; 19:1; 26:1; 27:3}
\crossref{Lev}{14}{35}{De 7:26 Jos 7:21 1Sa 3:12-\allowbreak14 1Ki 13:34 Ps 91:10 Pr 3:33}
\crossref{Lev}{14}{36}{1Co 15:33 2Ti 2:17,\allowbreak18 Heb 12:15 Re 18:4}
\crossref{Lev}{14}{37}{Le 13:3,\allowbreak19,\allowbreak20,\allowbreak42,\allowbreak49}
\crossref{Lev}{14}{38}{Le 13:50}
\crossref{Lev}{14}{39}{Le 13:7,\allowbreak8,\allowbreak22,\allowbreak27,\allowbreak36,\allowbreak51}
\crossref{Lev}{14}{40}{Ps 101:5,\allowbreak7,\allowbreak8 Pr 22:10; 25:4,\allowbreak5 Isa 1:25,\allowbreak26 Mt 18:17 Joh 15:2}
\crossref{Lev}{14}{41}{Job 36:13,\allowbreak14 Isa 65:4 Mt 8:28; 24:51 1Ti 1:20 Re 22:15}
\crossref{Lev}{14}{42}{Ge 18:19 Jos 24:15 2Ch 17:7-\allowbreak9; 19:5-\allowbreak7; 29:4,\allowbreak5 Ps 101:6}
\crossref{Lev}{14}{43}{Jer 6:28-\allowbreak30 Eze 24:13 Heb 6:4-\allowbreak8 2Pe 2:20,\allowbreak22 Jude 1:12}
\crossref{Lev}{14}{44}{Le 13:51,\allowbreak52 Zec 5:4}
\crossref{Lev}{14}{45}{1Ki 9:6-\allowbreak9 2Ki 10:27; 17:20-\allowbreak23; 18:4; 25:4-\allowbreak12,\allowbreak25,\allowbreak26 Jer 52:13}
\crossref{Lev}{14}{46}{Le 11:24,\allowbreak25,\allowbreak28; 15:5-\allowbreak8,\allowbreak10; 17:15; 22:6 Nu 19:7-\allowbreak10,\allowbreak21,\allowbreak22}
\crossref{Lev}{14}{47}{14:8,\allowbreak9}
\crossref{Lev}{14}{48}{}
\crossref{Lev}{14}{49}{14:4-\allowbreak7}
\crossref{Lev}{14}{50}{}
\crossref{Lev}{14}{51}{Ge 43:3}
\crossref{Lev}{14}{52}{14:52}
\crossref{Lev}{14}{53}{}
\crossref{Lev}{14}{54}{14:2,\allowbreak32; 6:9,\allowbreak14,\allowbreak25; 7:1,\allowbreak37; 11:46; 15:32 Nu 5:29; 6:13; 19:14 De 24:8}
\crossref{Lev}{14}{55}{Le 13:47-\allowbreak59}
\crossref{Lev}{14}{56}{Le 13:2}
\crossref{Lev}{14}{57}{Le 10:10 Jer 15:19 Eze 44:23}
\crossref{Lev}{15}{1}{Le 11:1; 13:1 Ps 25:14 Am 3:7 Heb 1:1}
\crossref{Lev}{15}{2}{De 4:7,\allowbreak8 Ne 9:13,\allowbreak14 Ps 78:5; 147:19,\allowbreak20 Ro 3:2}
\crossref{Lev}{15}{3}{Le 12:3 Eze 16:26; 23:20}
\crossref{Lev}{15}{4}{1Co 15:33 Eph 5:11 Tit 1:15}
\crossref{Lev}{15}{5}{Le 11:25,\allowbreak28,\allowbreak32; 13:6,\allowbreak34; 14:8,\allowbreak9,\allowbreak27,\allowbreak46,\allowbreak47; 16:26,\allowbreak28; 17:15}
\crossref{Lev}{15}{6}{Isa 1:16 Jas 4:8}
\crossref{Lev}{15}{7}{15:7}
\crossref{Lev}{15}{8}{Isa 1:16 Ga 1:8,\allowbreak9 1Ti 4:1-\allowbreak3 Tit 1:9,\allowbreak10 2Pe 2:1-\allowbreak3 Jas 4:8}
\crossref{Lev}{15}{9}{}
\crossref{Lev}{15}{10}{15:5,\allowbreak8 Ps 26:6 Jas 4:8}
\crossref{Lev}{15}{11}{}
\crossref{Lev}{15}{12}{Le 6:28; 11:32,\allowbreak33 Pr 1:21,\allowbreak23; 3:21 2Co 5:1 Php 3:21}
\crossref{Lev}{15}{13}{15:28; 8:33; 9:1; 14:8,\allowbreak10 Ex 29:35,\allowbreak37 Nu 12:14; 19:11,\allowbreak12}
\crossref{Lev}{15}{14}{15:29,\allowbreak30; 1:14; 12:6,\allowbreak8; 14:22-\allowbreak31 Nu 6:10 2Co 5:21 Heb 7:26; 10:10,\allowbreak12}
\crossref{Lev}{15}{15}{Le 5:7-\allowbreak10; 14:19,\allowbreak20,\allowbreak30,\allowbreak31}
\crossref{Lev}{15}{16}{15:5; 22:4 De 23:10,\allowbreak11 2Co 7:1 1Pe 2:11 1Jo 1:7}
\crossref{Lev}{15}{17}{}
\crossref{Lev}{15}{18}{15:5 Eph 4:17-\allowbreak19; 5:3-\allowbreak11 2Ti 2:22 1Pe 2:11}
\crossref{Lev}{15}{19}{Le 12:2,\allowbreak4; 20:18 La 1:8,\allowbreak9,\allowbreak17 Eze 36:17 Mt 15:19 Mr 5:25}
\crossref{Lev}{15}{20}{15:4-\allowbreak9 Pr 2:16-\allowbreak19; 5:3-\allowbreak13; 6:24,\allowbreak35; 7:10-\allowbreak27; 9:13-\allowbreak18; 22:27 Ec 7:26}
\crossref{Lev}{15}{21}{15:5,\allowbreak6 Isa 22:14 2Co 7:1 Heb 9:26 Re 7:14}
\crossref{Lev}{15}{22}{15:22}
\crossref{Lev}{15}{23}{15:23}
\crossref{Lev}{15}{24}{15:33; 20:18 Eze 18:6; 22:10 1Th 5:22 Heb 13:4 1Pe 2:11}
\crossref{Lev}{15}{25}{15:19-\allowbreak24 Mt 9:20 Mr 5:25; 7:20-\allowbreak23 Lu 8:43}
\crossref{Lev}{15}{26}{15:26}
\crossref{Lev}{15}{27}{15:5-\allowbreak8,\allowbreak13,\allowbreak21; 17:15,\allowbreak16 Eze 36:25,\allowbreak29 Zec 13:1 Heb 9:14; 10:22}
\crossref{Lev}{15}{28}{15:13-\allowbreak15 Mt 1:21 1Co 1:30; 6:11 Ga 3:13; 4:4 Eph 1:6,\allowbreak7}
\crossref{Lev}{15}{29}{15:14}
\crossref{Lev}{15}{30}{Le 23:19}
\crossref{Lev}{15}{31}{Le 11:47; 13:59 Nu 5:3 De 24:8 Ps 66:18 Eze 44:23 Heb 10:29}
\crossref{Lev}{15}{32}{15:1-\allowbreak18; 11:46; 13:59; 14:2,\allowbreak32,\allowbreak54-\allowbreak57 Nu 5:29; 6:13; 19:14 Eze 43:12}
\crossref{Lev}{15}{33}{15:19-\allowbreak30}
\crossref{Lev}{16}{1}{Le 10:1,\allowbreak2}
\crossref{Lev}{16}{2}{Le 23:27 Ex 26:33,\allowbreak34; 30:10; 40:20,\allowbreak21 1Ki 8:6 Heb 9:3,\allowbreak7,\allowbreak8}
\crossref{Lev}{16}{3}{Heb 9:7,\allowbreak12,\allowbreak24,\allowbreak25}
\crossref{Lev}{16}{4}{Le 8:6,\allowbreak7 Ex 29:4; 30:20; 40:12,\allowbreak31,\allowbreak32 Re 1:5,\allowbreak6}
\crossref{Lev}{16}{5}{Le 4:14; 8:2,\allowbreak14; 9:8-\allowbreak16 Nu 29:11 2Ch 29:21 Ezr 6:17 Eze 45:22,\allowbreak23}
\crossref{Lev}{16}{6}{Le 8:14-\allowbreak17 Heb 9:7}
\crossref{Lev}{16}{7}{Le 1:3; 4:4; 12:6,\allowbreak7 Mt 16:21 Ro 12:1}
\crossref{Lev}{16}{8}{Nu 26:55; 33:54 Jos 18:10,\allowbreak11 1Sa 14:41,\allowbreak42 Pr 16:33 Eze 48:29}
\crossref{Lev}{16}{9}{Ac 2:23; 4:27,\allowbreak28}
\crossref{Lev}{16}{10}{16:21,\allowbreak22}
\crossref{Lev}{16}{11}{16:3,\allowbreak6}
\crossref{Lev}{16}{12}{Le 10:1 Nu 16:18,\allowbreak46 Isa 6:6,\allowbreak7 Heb 9:14 1Jo 1:7}
\crossref{Lev}{16}{13}{Ex 30:1,\allowbreak7,\allowbreak8 Nu 16:7,\allowbreak18,\allowbreak46 Re 8:3,\allowbreak4}
\crossref{Lev}{16}{14}{Le 4:5,\allowbreak6,\allowbreak17; 8:11 Ro 3:24-\allowbreak26 Heb 9:7,\allowbreak13,\allowbreak25; 10:4,\allowbreak10-\allowbreak12,\allowbreak19; 12:24}
\crossref{Lev}{16}{15}{16:5-\allowbreak9 Heb 2:17; 5:3; 9:7,\allowbreak25,\allowbreak26}
\crossref{Lev}{16}{16}{16:18; 8:15 Ex 29:36,\allowbreak37 Eze 45:18,\allowbreak19 Joh 14:3 Heb 9:22,\allowbreak23}
\crossref{Lev}{16}{17}{Ex 34:3 Isa 53:6 Da 9:24 Lu 1:10 Ac 4:12 1Ti 2:5 Heb 1:3; 9:7}
\crossref{Lev}{16}{18}{16:16; 4:7,\allowbreak18 Ex 30:10 Joh 17:19 Heb 2:11; 5:7,\allowbreak8; 9:22,\allowbreak23}
\crossref{Lev}{16}{19}{Eze 43:18-\allowbreak22 Zec 13:1}
\crossref{Lev}{16}{20}{16:16; 6:30; 8:15 Eze 45:20 2Co 5:19 Col 1:20}
\crossref{Lev}{16}{21}{Le 1:4 Ex 29:10}
\crossref{Lev}{16}{22}{Isa 53:11,\allowbreak12 Joh 1:29 Ga 3:13 Heb 9:28 1Pe 2:24}
\crossref{Lev}{16}{23}{16:4 Eze 42:14; 44:19 Ro 8:3 Php 2:6-\allowbreak11 Heb 9:28}
\crossref{Lev}{16}{24}{16:4; 8:6; 14:9; 22:6 Ex 29:4 Heb 9:10; 10:19-\allowbreak22 Re 1:5,\allowbreak6}
\crossref{Lev}{16}{25}{16:6; 4:8-\allowbreak10,\allowbreak19 Ex 29:13}
\crossref{Lev}{16}{26}{16:10,\allowbreak21,\allowbreak22}
\crossref{Lev}{16}{27}{Le 4:11,\allowbreak12,\allowbreak21; 6:30; 8:17}
\crossref{Lev}{16}{28}{16:26}
\crossref{Lev}{16}{29}{Le 23:27-\allowbreak32 Ex 30:10 Nu 29:7 1Ki 8:2 Ezr 3:1}
\crossref{Lev}{16}{30}{Ps 51:2,\allowbreak7,\allowbreak10 Jer 33:8 Eze 36:25-\allowbreak27 Eph 5:26 Tit 2:14}
\crossref{Lev}{16}{31}{Le 23:32; 25:4 Ex 31:15; 35:2}
\crossref{Lev}{16}{32}{Le 4:3,\allowbreak5,\allowbreak16}
\crossref{Lev}{16}{33}{16:6,\allowbreak16,\allowbreak18,\allowbreak19,\allowbreak24 Ex 20:25,\allowbreak26}
\crossref{Lev}{16}{34}{Le 23:31 Nu 29:7}
\crossref{Lev}{17}{1}{17:1}
\crossref{Lev}{17}{2}{}
\crossref{Lev}{17}{3}{17:8,\allowbreak12,\allowbreak13,\allowbreak15}
\crossref{Lev}{17}{4}{Le 1:3 De 12:5,\allowbreak6,\allowbreak13,\allowbreak14 Eze 20:40 Joh 10:7,\allowbreak9; 14:6}
\crossref{Lev}{17}{5}{Ge 21:33; 22:2,\allowbreak13; 31:54 De 12:2 1Ki 14:23 2Ki 16:4; 17:10}
\crossref{Lev}{17}{6}{Le 3:2,\allowbreak8,\allowbreak13}
\crossref{Lev}{17}{7}{De 32:17 2Ch 11:15 Ps 106:37 Joh 12:31; 14:30 1Co 10:20}
\crossref{Lev}{17}{8}{17:4,\allowbreak10; 1:2,\allowbreak3 Jud 6:26 1Sa 7:9; 10:8; 16:2 2Sa 24:25 1Ki 18:30-\allowbreak38}
\crossref{Lev}{17}{9}{}
\crossref{Lev}{17}{10}{17:11; 3:17; 7:26,\allowbreak27; 19:26 Ge 9:4 De 12:16,\allowbreak23; 15:23 1Sa 14:33}
\crossref{Lev}{17}{11}{Le 8:15; 16:11,\allowbreak14-\allowbreak19 Mt 20:28; 26:28 Mr 14:24 Ro 3:25; 5:9}
\crossref{Lev}{17}{12}{Ex 12:49}
\crossref{Lev}{17}{13}{Le 7:26}
\crossref{Lev}{17}{14}{17:11,\allowbreak12 Ge 9:4 De 12:23}
\crossref{Lev}{17}{15}{Le 22:8 Ex 22:31 De 14:21 Eze 4:14; 44:31}
\crossref{Lev}{17}{16}{Le 5:1; 7:18; 19:8; 20:17,\allowbreak19,\allowbreak20 Nu 19:19,\allowbreak20 Isa 53:11 Joh 13:8}
\crossref{Lev}{18}{1}{18:1}
\crossref{Lev}{18}{2}{18:4; 11:44; 19:3,\allowbreak4,\allowbreak10,\allowbreak34; 20:7 Ge 17:7 Ex 6:7; 20:2 Ps 33:12}
\crossref{Lev}{18}{3}{Ps 106:35 Eze 20:7,\allowbreak8; 23:8 Eph 5:7-\allowbreak11 1Pe 4:2-\allowbreak4}
\crossref{Lev}{18}{4}{18:26; 19:37; 20:22 De 4:1,\allowbreak2; 6:1 Ps 105:45; 119:4 Eze 20:19; 36:27}
\crossref{Lev}{18}{5}{Eze 20:11,\allowbreak13,\allowbreak21 Lu 10:28 Ro 10:5 Ga 3:12}
\crossref{Lev}{18}{6}{18:7-\allowbreak19; 20:11,\allowbreak12,\allowbreak17-\allowbreak21}
\crossref{Lev}{18}{7}{Le 20:11 Eze 22:10}
\crossref{Lev}{18}{8}{Le 20:11 Ge 35:22; 49:4 De 22:30; 27:20 2Sa 16:21,\allowbreak22 Eze 22:10}
\crossref{Lev}{18}{9}{Le 20:17 De 27:22 2Sa 13:11-\allowbreak14 Eze 22:11}
\crossref{Lev}{18}{10}{}
\crossref{Lev}{18}{11}{2Sa 13:12 Eze 22:11}
\crossref{Lev}{18}{12}{Le 20:19 Ex 6:20}
\crossref{Lev}{18}{13}{18:13}
\crossref{Lev}{18}{14}{Le 20:20}
\crossref{Lev}{18}{15}{Le 20:12 Ge 38:18,\allowbreak19,\allowbreak26 Eze 22:11}
\crossref{Lev}{18}{16}{Le 20:21 De 25:5 Mt 14:3,\allowbreak4; 22:24 Mr 6:17; 12:19 Lu 3:19}
\crossref{Lev}{18}{17}{Le 20:14 De 27:23 Am 2:7}
\crossref{Lev}{18}{18}{Ge 4:19; 29:28 Ex 26:3}
\crossref{Lev}{18}{19}{Le 15:19,\allowbreak24; 20:18 Eze 18:6; 22:10}
\crossref{Lev}{18}{20}{Le 20:10 Ex 20:14 De 5:18; 22:22,\allowbreak25 2Sa 11:3,\allowbreak4,\allowbreak27 Pr 6:25,\allowbreak29-\allowbreak33}
\crossref{Lev}{18}{21}{1Ki 11:7,\allowbreak33 Am 5:26 Ac 7:43}
\crossref{Lev}{18}{22}{Le 20:13 Ge 19:5 Jud 19:22 1Ki 14:24 Ro 1:26,\allowbreak27 1Co 6:9 1Ti 1:10}
\crossref{Lev}{18}{23}{Le 20:15,\allowbreak16 Ex 22:19}
\crossref{Lev}{18}{24}{18:6-\allowbreak23,\allowbreak30 Jer 44:4 Mt 15:18-\allowbreak20 Mr 7:10-\allowbreak23 1Co 3:17}
\crossref{Lev}{18}{25}{Nu 35:33,\allowbreak34 Ps 106:38 Isa 24:5 Jer 2:7; 16:18 Eze 36:17,\allowbreak18}
\crossref{Lev}{18}{26}{18:5,\allowbreak30 De 4:1,\allowbreak2,\allowbreak40; 12:32 Ps 105:44,\allowbreak45 Lu 8:15; 11:28}
\crossref{Lev}{18}{27}{18:24 De 20:18; 23:18; 25:16; 27:15 1Ki 14:24 2Ki 16:3; 21:2}
\crossref{Lev}{18}{28}{18:25; 20:22 Jer 9:19 Eze 36:13,\allowbreak17 Ro 8:22 Re 3:16}
\crossref{Lev}{18}{29}{Le 17:10; 20:6}
\crossref{Lev}{18}{30}{18:3,\allowbreak26,\allowbreak27; 20:23 De 18:9-\allowbreak12}
\crossref{Lev}{19}{1}{19:1}
\crossref{Lev}{19}{2}{Le 11:44,\allowbreak45; 20:7,\allowbreak26; 21:8 Ex 19:6 Isa 6:3,\allowbreak4 Am 3:3 Mt 5:48}
\crossref{Lev}{19}{3}{Ex 20:12; 21:15,\allowbreak17 De 21:18-\allowbreak21; 27:16 Pr 1:8; 6:20,\allowbreak21; 23:22}
\crossref{Lev}{19}{4}{Le 26:1}
\crossref{Lev}{19}{5}{Le 3:1-\allowbreak17; 7:16; 22:21 Ex 24:5 2Ch 31:2 Eze 45:15-\allowbreak17; 46:2,\allowbreak12}
\crossref{Lev}{19}{6}{Le 7:11-\allowbreak17}
\crossref{Lev}{19}{7}{Isa 1:13; 65:4; 66:3 Jer 16:18}
\crossref{Lev}{19}{8}{Le 5:1}
\crossref{Lev}{19}{9}{}
\crossref{Lev}{19}{10}{Jud 8:2 Isa 17:6; 24:13 Jer 49:9 Ob 1:5 Mic 7:1}
\crossref{Lev}{19}{11}{Le 6:2 Ex 20:15,\allowbreak17; 22:1,\allowbreak7,\allowbreak10-\allowbreak12 De 5:19 Jer 6:13; 7:9-\allowbreak11}
\crossref{Lev}{19}{12}{Le 6:3 Ex 20:7 De 5:11 Ps 15:4 Jer 4:2; 7:9 Zec 5:4 Mal 3:5}
\crossref{Lev}{19}{13}{Pr 20:10; 22:22 Jer 22:3 Eze 22:29 Mr 10:19 Lu 3:13 1Th 4:6}
\crossref{Lev}{19}{14}{De 27:18 Ro 12:14; 14:13 1Co 8:8-\allowbreak13; 10:32 Re 2:14}
\crossref{Lev}{19}{15}{19:35 Ex 18:21; 23:2; 23:2,\allowbreak3,\allowbreak7,\allowbreak8 De 1:17; 16:19; 25:13-\allowbreak16; 27:19}
\crossref{Lev}{19}{16}{Ex 23:1 Ps 15:3 Pr 11:13; 20:19 Jer 6:28; 9:4 Eze 22:9 1Ti 3:11}
\crossref{Lev}{19}{17}{Ge 27:41 Pr 26:24-\allowbreak26 1Jo 2:9,\allowbreak11; 3:12-\allowbreak15}
\crossref{Lev}{19}{18}{Ex 23:4,\allowbreak5 De 32:25 2Sa 13:22,\allowbreak28 Pr 20:22 Mt 5:43,\allowbreak44}
\crossref{Lev}{19}{19}{De 22:9-\allowbreak11 Mt 9:16,\allowbreak17 Ro 11:6 2Co 6:14-\allowbreak17 Ga 3:9-\allowbreak11}
\crossref{Lev}{19}{20}{}
\crossref{Lev}{19}{21}{Le 5:1-\allowbreak6:7}
\crossref{Lev}{19}{22}{Le 4:20,\allowbreak26}
\crossref{Lev}{19}{23}{Le 14:34}
\crossref{Lev}{19}{24}{Nu 18:12,\allowbreak13 De 12:17,\allowbreak18; 14:28,\allowbreak29; 18:4 Pr 3:9}
\crossref{Lev}{19}{25}{Le 26:3,\allowbreak4 Pr 3:9,\allowbreak10 Ec 11:1,\allowbreak2 Hag 1:4-\allowbreak6,\allowbreak9-\allowbreak11; 2:18,\allowbreak19 Mal 3:8-\allowbreak10}
\crossref{Lev}{19}{26}{Le 3:17; 7:26; 17:10-\allowbreak14 De 12:23}
\crossref{Lev}{19}{27}{Le 21:5 Isa 15:2 Jer 16:6; 48:37 Eze 7:18; 44:20}
\crossref{Lev}{19}{28}{Le 21:5 De 14:1 1Ki 18:28 Jer 16:6; 48:37 Mr 5:5}
\crossref{Lev}{19}{29}{Le 21:7 De 23:17 Ho 4:12-\allowbreak14 1Co 6:15}
\crossref{Lev}{19}{30}{19:3; 26:2}
\crossref{Lev}{19}{31}{19:26; 20:6,\allowbreak7,\allowbreak27 Ex 22:18 De 18:10-\allowbreak14 1Sa 28:3,\allowbreak7-\allowbreak9 2Ki 17:17}
\crossref{Lev}{19}{32}{19:14 1Ki 2:19 Job 32:4,\allowbreak6 Pr 16:31; 20:29 Isa 3:5 La 5:12 Ro 13:7}
\crossref{Lev}{19}{33}{Ex 22:21; 23:9 De 10:18,\allowbreak19; 24:14 Mal 3:5}
\crossref{Lev}{19}{34}{19:18 Ex 12:48,\allowbreak49 De 10:19 Mt 5:43}
\crossref{Lev}{19}{35}{19:15}
\crossref{Lev}{19}{36}{Pr 11:1}
\crossref{Lev}{19}{37}{Le 18:4,\allowbreak5 De 4:1,\allowbreak2,\allowbreak5,\allowbreak6; 5:1; 6:1,\allowbreak2; 8:1 Ps 119:4,\allowbreak34 1Jo 3:22,\allowbreak23}
\crossref{Lev}{20}{1}{20:1}
\crossref{Lev}{20}{2}{Le 17:8,\allowbreak13,\allowbreak15}
\crossref{Lev}{20}{3}{Le 17:10 1Pe 3:12}
\crossref{Lev}{20}{4}{Ac 17:30}
\crossref{Lev}{20}{5}{Le 17:10}
\crossref{Lev}{20}{6}{20:27; 19:26,\allowbreak31 De 18:10-\allowbreak14 Isa 8:19}
\crossref{Lev}{20}{7}{Le 11:44; 19:2 Eph 1:4 Phm 2:12,\allowbreak13 Col 3:12 1Th 4:3,\allowbreak7 Heb 12:14}
\crossref{Lev}{20}{8}{Le 18:4,\allowbreak5; 19:37 Mt 5:19; 7:24; 12:50 Joh 13:17 Jas 1:22 Re 22:14}
\crossref{Lev}{20}{9}{20:11-\allowbreak13,\allowbreak16,\allowbreak27 Jos 2:19 Jud 9:24 2Sa 1:16 1Ki 2:32 Mt 27:25}
\crossref{Lev}{20}{10}{De 22:22-\allowbreak24 2Sa 12:13 Eze 23:45-\allowbreak47 Joh 8:4,\allowbreak5}
\crossref{Lev}{20}{11}{Le 18:8 De 27:20,\allowbreak23 Am 2:7 1Co 5:1}
\crossref{Lev}{20}{12}{Le 18:15 Ge 38:16,\allowbreak18 De 27:23}
\crossref{Lev}{20}{13}{Le 18:22 Ge 19:5 De 23:17 Jud 19:22 Ro 1:26,\allowbreak27 1Co 6:9 1Ti 1:10}
\crossref{Lev}{20}{14}{Le 18:17 De 27:23 Am 2:7}
\crossref{Lev}{20}{15}{Le 18:23 Ex 22:19 De 27:21}
\crossref{Lev}{20}{16}{Ex 19:13; 21:28,\allowbreak32 Heb 12:20}
\crossref{Lev}{20}{17}{Le 18:9 Ge 20:12 De 27:22 2Sa 13:12 Eze 22:11}
\crossref{Lev}{20}{18}{Le 15:24; 18:19 Eze 18:6; 22:10}
\crossref{Lev}{20}{19}{Le 18:12,\allowbreak13-\allowbreak30 Ex 6:20}
\crossref{Lev}{20}{20}{Le 18:14}
\crossref{Lev}{20}{21}{Le 18:16 Mt 14:3,\allowbreak4}
\crossref{Lev}{20}{22}{Le 18:4,\allowbreak5,\allowbreak26; 19:37 Ps 19:8-\allowbreak11; 105:45; 119:80,\allowbreak145,\allowbreak171 Eze 36:27}
\crossref{Lev}{20}{23}{Le 18:3,\allowbreak24,\allowbreak30 De 12:30,\allowbreak31 Jer 10:1,\allowbreak2}
\crossref{Lev}{20}{24}{Ex 3:8,\allowbreak17; 6:8}
\crossref{Lev}{20}{25}{Le 11:1-\allowbreak47 De 14:3-\allowbreak21 Ac 10:11-\allowbreak15,\allowbreak28 Eph 5:7-\allowbreak11}
\crossref{Lev}{20}{26}{20:7; 19:2 Ps 99:5,\allowbreak9 Isa 6:3; 30:11 1Pe 1:15,\allowbreak16 Re 3:7; 4:8}
\crossref{Lev}{20}{27}{20:6; 19:31 Ex 22:18 De 18:10-\allowbreak12 1Sa 28:7-\allowbreak9}
\crossref{Lev}{21}{1}{Ho 5:1 Mal 2:1,\allowbreak4}
\crossref{Lev}{21}{2}{Le 18:6 1Th 4:13}
\crossref{Lev}{21}{3}{}
\crossref{Lev}{21}{4}{}
\crossref{Lev}{21}{5}{}
\crossref{Lev}{21}{6}{21:8; 10:3 Ex 28:36; 29:44 Ezr 8:28 1Pe 2:9}
\crossref{Lev}{21}{7}{21:8 Eze 44:22 1Ti 3:11}
\crossref{Lev}{21}{8}{21:6 Ex 19:10,\allowbreak14; 28:41; 29:1,\allowbreak43,\allowbreak44}
\crossref{Lev}{21}{9}{1Sa 2:17,\allowbreak34; 3:13,\allowbreak14 Eze 9:6 Mal 2:3 Mt 11:20-\allowbreak24 1Ti 3:4,\allowbreak5}
\crossref{Lev}{21}{10}{Le 8:12; 10:7; 16:32 Ex 29:29,\allowbreak30 Nu 35:25 Ps 133:2}
\crossref{Lev}{21}{11}{21:1,\allowbreak2 Nu 6:7; 19:14 De 33:9 Mt 8:21,\allowbreak22; 12:46-\allowbreak50 Lu 9:59,\allowbreak60}
\crossref{Lev}{21}{12}{Le 10:7}
\crossref{Lev}{21}{13}{21:7 Eze 44:22 2Co 11:2 Re 14:4}
\crossref{Lev}{21}{14}{So 6:9 2Co 11:2 Eph 5:27}
\crossref{Lev}{21}{15}{Ge 18:19 Ezr 2:62; 9:2 Ne 13:23-\allowbreak29 Mal 2:11,\allowbreak15 Ro 11:16}
\crossref{Lev}{21}{16}{}
\crossref{Lev}{21}{17}{Le 22:20-\allowbreak25 1Th 2:10 1Ti 3:2 Heb 7:26}
\crossref{Lev}{21}{18}{Isa 56:10 Mt 23:16,\allowbreak17,\allowbreak19 1Ti 3:2,\allowbreak3,\allowbreak7 Tit 1:7,\allowbreak10}
\crossref{Lev}{21}{19}{}
\crossref{Lev}{21}{20}{De 23:1}
\crossref{Lev}{21}{21}{21:6,\allowbreak8,\allowbreak17}
\crossref{Lev}{21}{22}{Le 2:3,\allowbreak10; 6:16,\allowbreak17,\allowbreak29; 7:1; 24:8,\allowbreak9 Nu 18:9,\allowbreak10}
\crossref{Lev}{21}{23}{Ex 30:6-\allowbreak8; 40:26,\allowbreak27 Eze 44:9-\allowbreak14}
\crossref{Lev}{21}{24}{Mal 2:1-\allowbreak7 Col 4:17 1Ti 1:18 2Ti 2:2}
\crossref{Lev}{22}{1}{22:1}
\crossref{Lev}{22}{2}{22:3-\allowbreak6; 15:31 Nu 6:3-\allowbreak8}
\crossref{Lev}{22}{3}{}
\crossref{Lev}{22}{4}{Le 13:2,\allowbreak3,\allowbreak44-\allowbreak46}
\crossref{Lev}{22}{5}{Le 11:24,\allowbreak43,\allowbreak44}
\crossref{Lev}{22}{6}{Le 11:24,\allowbreak25; 15:5; 16:24-\allowbreak28 Nu 19:7-\allowbreak10 Hag 2:13 1Co 6:11 Heb 10:22}
\crossref{Lev}{22}{7}{Le 21:22 Nu 18:11-\allowbreak19 De 18:3,\allowbreak4 1Co 9:4,\allowbreak13,\allowbreak14}
\crossref{Lev}{22}{8}{Le 17:15 Ex 22:31 De 14:21 Eze 44:31}
\crossref{Lev}{22}{9}{}
\crossref{Lev}{22}{10}{}
\crossref{Lev}{22}{11}{Ge 17:13 Nu 18:11-\allowbreak13}
\crossref{Lev}{22}{12}{Le 21:3 Isa 40:13}
\crossref{Lev}{22}{13}{Le 10:14 Nu 18:11-\allowbreak19}
\crossref{Lev}{22}{14}{Le 5:15-\allowbreak19; 27:13,\allowbreak15}
\crossref{Lev}{22}{15}{22:9; 19:8 Nu 18:32 Eze 22:26}
\crossref{Lev}{22}{16}{22:9; 20:8}
\crossref{Lev}{22}{17}{}
\crossref{Lev}{22}{18}{Le 1:2,\allowbreak10; 17:10,\allowbreak13}
\crossref{Lev}{22}{19}{Le 1:3,\allowbreak10; 4:32 Ex 12:5 Mt 27:4,\allowbreak19,\allowbreak24,\allowbreak54 Lu 23:14,\allowbreak41,\allowbreak47 Joh 19:4}
\crossref{Lev}{22}{20}{22:25 De 15:21; 17:1 Mal 1:8,\allowbreak13,\allowbreak14}
\crossref{Lev}{22}{21}{Le 3:1,\allowbreak6; 7:11-\allowbreak38}
\crossref{Lev}{22}{22}{22:20; 21:18-\allowbreak21 Mal 1:8}
\crossref{Lev}{22}{23}{Le 21:18}
\crossref{Lev}{22}{24}{22:20 De 23:1}
\crossref{Lev}{22}{25}{Nu 15:14-\allowbreak16; 16:40 Ezr 6:8-\allowbreak10}
\crossref{Lev}{22}{26}{}
\crossref{Lev}{22}{27}{}
\crossref{Lev}{22}{28}{}
\crossref{Lev}{22}{29}{Le 7:12-\allowbreak15 Ps 107:22; 116:17 Ho 14:2 Am 4:5 Heb 13:15 1Pe 2:5}
\crossref{Lev}{22}{30}{Le 7:15-\allowbreak18; 19:7 Ex 16:19,\allowbreak20}
\crossref{Lev}{22}{31}{Le 18:4,\allowbreak5; 19:37 Nu 15:40 De 4:40 1Th 4:1,\allowbreak2}
\crossref{Lev}{22}{32}{22:2; 18:21}
\crossref{Lev}{22}{33}{Le 11:45; 19:36; 25:38 Ex 6:7; 20:2 Nu 15:41}
\crossref{Lev}{23}{1}{23:1}
\crossref{Lev}{23}{2}{Ex 32:5 Nu 10:2,\allowbreak3,\allowbreak10 2Ki 10:20 2Ch 30:5 Ps 81:3 Joe 1:14; 2:15}
\crossref{Lev}{23}{3}{Le 19:3 Ex 16:23,\allowbreak29; 20:8-\allowbreak11; 23:12; 31:15; 34:21; 35:2,\allowbreak3 De 5:13}
\crossref{Lev}{23}{4}{23:2,\allowbreak37 Ex 23:14}
\crossref{Lev}{23}{5}{Ex 12:2-\allowbreak14,\allowbreak18; 13:3-\allowbreak10; 23:15 Nu 9:2-\allowbreak7; 28:16 De 16:1-\allowbreak8 Jos 5:10}
\crossref{Lev}{23}{6}{Ex 12:15,\allowbreak16; 13:6,\allowbreak7; 34:18 Nu 28:17,\allowbreak18 De 16:8 Ac 12:3,\allowbreak4}
\crossref{Lev}{23}{7}{Nu 28:18-\allowbreak25}
\crossref{Lev}{23}{8}{Le 1:9}
\crossref{Lev}{23}{9}{}
\crossref{Lev}{23}{10}{Le 14:34}
\crossref{Lev}{23}{11}{Le 9:21; 10:14 Ex 29:24}
\crossref{Lev}{23}{12}{Le 1:10 Heb 10:10-\allowbreak12 1Pe 1:19}
\crossref{Lev}{23}{13}{Le 2:14-\allowbreak16; 14:10 Nu 15:3-\allowbreak12}
\crossref{Lev}{23}{14}{Le 19:23-\allowbreak25; 25:2,\allowbreak3 Ge 4:4,\allowbreak5 Jos 5:11,\allowbreak12}
\crossref{Lev}{23}{15}{23:10,\allowbreak11; 25:8 Ex 34:22 De 16:9,\allowbreak10}
\crossref{Lev}{23}{16}{Ac 2:1}
\crossref{Lev}{23}{17}{Nu 28:26}
\crossref{Lev}{23}{18}{23:12,\allowbreak13 Nu 28:27-\allowbreak31 Mal 1:13,\allowbreak14}
\crossref{Lev}{23}{19}{Le 4:23-\allowbreak28; 16:15 Nu 15:24; 28:30 Ro 8:3 2Co 5:21}
\crossref{Lev}{23}{20}{23:17; 7:29,\allowbreak30 Ex 29:24 Lu 2:14 Eph 2:14}
\crossref{Lev}{23}{21}{23:2,\allowbreak4 Ex 12:16 De 16:11 Isa 11:10}
\crossref{Lev}{23}{22}{Le 19:9,\allowbreak10 De 16:11-\allowbreak14; 24:19-\allowbreak21 Ru 2:3-\allowbreak7,\allowbreak15,\allowbreak16-\allowbreak23 Job 31:16-\allowbreak21}
\crossref{Lev}{23}{23}{}
\crossref{Lev}{23}{24}{Nu 10:10; 29:1-\allowbreak6 1Ch 15:28 2Ch 5:13 Ezr 3:6 Ps 81:1-\allowbreak4; 98:6}
\crossref{Lev}{23}{25}{23:8 Nu 29:1-\allowbreak6}
\crossref{Lev}{23}{26}{}
\crossref{Lev}{23}{27}{Le 16:29,\allowbreak30; 25:9 Nu 29:7-\allowbreak11}
\crossref{Lev}{23}{28}{Le 16:34 Isa 53:10 Da 9:24 Zec 3:9 Ro 5:10,\allowbreak11 Heb 9:12,\allowbreak26}
\crossref{Lev}{23}{29}{23:27,\allowbreak32 Isa 22:12 Jer 31:9 Eze 7:16}
\crossref{Lev}{23}{30}{Le 20:3,\allowbreak5,\allowbreak6 Ge 17:14 Jer 15:7 Eze 14:9 Zep 2:5 1Co 3:17}
\crossref{Lev}{23}{31}{23:28 Mt 12:12 Mr 3:4 Ro 4:4,\allowbreak5; 11:6 Eph 2:7-\allowbreak10 Heb 4:8-\allowbreak11}
\crossref{Lev}{23}{32}{Le 16:31 Mt 11:28-\allowbreak30 Heb 4:3,\allowbreak11}
\crossref{Lev}{23}{33}{}
\crossref{Lev}{23}{34}{Ex 23:16; 34:22 Nu 29:12 De 16:13-\allowbreak15 Ezr 3:4 Ne 8:14}
\crossref{Lev}{23}{35}{23:7,\allowbreak8,\allowbreak24,\allowbreak25}
\crossref{Lev}{23}{36}{Nu 29:12-\allowbreak38}
\crossref{Lev}{23}{37}{23:2,\allowbreak4 De 16:16,\allowbreak17}
\crossref{Lev}{23}{38}{23:3; 19:3 Ge 2:2,\allowbreak3 Ex 20:8-\allowbreak11}
\crossref{Lev}{23}{39}{23:34 Ex 23:16 De 16:13}
\crossref{Lev}{23}{40}{Ne 8:15 Mt 21:8}
\crossref{Lev}{23}{41}{Nu 29:12 Ne 8:18}
\crossref{Lev}{23}{42}{Ge 33:17 Nu 24:2,\allowbreak5 Ne 8:14-\allowbreak17 Jer 35:10 2Co 5:1 Heb 11:13-\allowbreak16}
\crossref{Lev}{23}{43}{Ex 13:14 De 31:10-\allowbreak13 Ps 78:5,\allowbreak6}
\crossref{Lev}{23}{44}{23:1,\allowbreak2; 21:24 Mt 18:20}
\crossref{Lev}{24}{1}{24:1}
\crossref{Lev}{24}{2}{Ex 27:20,\allowbreak21; 39:37; 40:24 Nu 8:2-\allowbreak4 1Sa 3:3,\allowbreak4}
\crossref{Lev}{24}{3}{Ex 27:21; 39:37 Zec 4:2,\allowbreak3,\allowbreak10-\allowbreak14 Col 2:9 Heb 9:2 Re 1:12-\allowbreak14; 2:18}
\crossref{Lev}{24}{4}{Ex 25:31-\allowbreak39; 31:8; 37:17-\allowbreak24; 39:37 Nu 3:31; 4:9 1Ki 7:49}
\crossref{Lev}{24}{5}{Ex 25:30; 40:23 1Ki 18:31 1Sa 21:4,\allowbreak5 Mt 12:4 Ac 26:7 Jas 1:1}
\crossref{Lev}{24}{6}{1Co 14:40}
\crossref{Lev}{24}{7}{Le 2:2 Eph 1:6 Heb 7:25 Re 8:3,\allowbreak4}
\crossref{Lev}{24}{8}{Nu 4:7 1Ch 9:32; 23:29 2Ch 2:4 Ne 10:33 Mt 12:3-\allowbreak5}
\crossref{Lev}{24}{9}{Le 8:31 1Sa 21:6 Mal 1:12 Mt 12:4 Mr 2:26 Lu 6:4}
\crossref{Lev}{24}{10}{Ex 12:38 Nu 11:4}
\crossref{Lev}{24}{11}{24:15,\allowbreak16 Ex 20:7 2Sa 12:14 1Ki 21:10,\allowbreak13 2Ki 18:30,\allowbreak35,\allowbreak37}
\crossref{Lev}{24}{12}{}
\crossref{Lev}{24}{13}{}
\crossref{Lev}{24}{14}{Le 13:46 Nu 5:2-\allowbreak4; 15:35}
\crossref{Lev}{24}{15}{Le 5:1; 20:16,\allowbreak17 Nu 9:13}
\crossref{Lev}{24}{16}{}
\crossref{Lev}{24}{17}{Ge 9:5,\allowbreak6 Ex 21:12-\allowbreak14 Nu 35:31 De 19:11,\allowbreak12}
\crossref{Lev}{24}{18}{24:21 Ex 21:34-\allowbreak36}
\crossref{Lev}{24}{19}{De 19:21 Mt 5:38; 7:2}
\crossref{Lev}{24}{20}{Ex 21:23-\allowbreak25 De 19:21 Mt 5:38}
\crossref{Lev}{24}{21}{24:18 Ex 21:33}
\crossref{Lev}{24}{22}{Le 17:10; 19:34 Ex 12:49 Nu 9:14; 15:15,\allowbreak16,\allowbreak29}
\crossref{Lev}{24}{23}{24:14-\allowbreak16 Nu 15:35,\allowbreak36 Heb 2:2,\allowbreak3; 10:28,\allowbreak29}
\crossref{Lev}{25}{1}{Ex 19:1 Nu 1:1; 10:11,\allowbreak12 Ga 4:24,\allowbreak25}
\crossref{Lev}{25}{2}{Le 14:34 De 32:8,\allowbreak49; 34:4 Ps 24:1,\allowbreak2; 115:16 Isa 8:8 Jer 27:5}
\crossref{Lev}{25}{3}{Mt 21:33-\allowbreak41 Lu 13:6-\allowbreak9}
\crossref{Lev}{25}{4}{25:20-\allowbreak23; 26:34,\allowbreak35,\allowbreak43 Ex 23:10,\allowbreak11 2Ch 36:21}
\crossref{Lev}{25}{5}{2Ki 19:29 Isa 37:30}
\crossref{Lev}{25}{6}{Ex 23:11 Ac 2:44; 4:32,\allowbreak34,\allowbreak35}
\crossref{Lev}{25}{7}{25:7}
\crossref{Lev}{25}{8}{Le 23:15 Ge 2:2}
\crossref{Lev}{25}{9}{Nu 10:10 Ps 89:15 Ac 13:38,\allowbreak39 Ro 10:18; 15:19 2Co 5:19-\allowbreak21}
\crossref{Lev}{25}{10}{Ex 20:2 Ezr 1:3 Ps 146:7 Isa 49:9,\allowbreak24,\allowbreak25; 61:1-\allowbreak3; 63:4}
\crossref{Lev}{25}{11}{25:5-\allowbreak7}
\crossref{Lev}{25}{12}{25:6,\allowbreak7}
\crossref{Lev}{25}{13}{25:10; 27:17-\allowbreak24 Nu 36:4}
\crossref{Lev}{25}{14}{25:17; 19:13 De 16:19,\allowbreak20 Jud 4:3 1Sa 12:3,\allowbreak4 2Ch 16:10 Ne 9:36,\allowbreak37}
\crossref{Lev}{25}{15}{Le 27:18-\allowbreak23 Php 4:5}
\crossref{Lev}{25}{16}{}
\crossref{Lev}{25}{17}{25:14}
\crossref{Lev}{25}{18}{Le 19:37 Ps 103:18}
\crossref{Lev}{25}{19}{Le 26:5 Ps 67:6; 85:12 Isa 30:23; 65:21,\allowbreak22 Eze 34:25-\allowbreak28; 36:30}
\crossref{Lev}{25}{20}{Nu 11:4,\allowbreak13 2Ki 6:15-\allowbreak17; 7:2 2Ch 25:9 Ps 78:19,\allowbreak20 Isa 1:2}
\crossref{Lev}{25}{21}{25:4,\allowbreak8-\allowbreak11}
\crossref{Lev}{25}{22}{2Ki 19:29 Isa 37:30}
\crossref{Lev}{25}{23}{25:10 1Ki 21:3 Eze 48:14}
\crossref{Lev}{25}{24}{25:27,\allowbreak31,\allowbreak51-\allowbreak53 Ro 8:23 1Co 1:30 Eph 1:7,\allowbreak14; 4:30}
\crossref{Lev}{25}{25}{Ru 2:20; 3:2,\allowbreak9,\allowbreak12; 4:4-\allowbreak6 Jer 32:7,\allowbreak8 2Co 8:9 Heb 2:13,\allowbreak14 Re 5:9}
\crossref{Lev}{25}{26}{}
\crossref{Lev}{25}{27}{25:50-\allowbreak53}
\crossref{Lev}{25}{28}{25:13}
\crossref{Lev}{25}{29}{}
\crossref{Lev}{25}{30}{}
\crossref{Lev}{25}{31}{Ps 49:7,\allowbreak8}
\crossref{Lev}{25}{32}{}
\crossref{Lev}{25}{33}{Nu 18:20-\allowbreak24 De 18:1,\allowbreak2}
\crossref{Lev}{25}{34}{25:23 Ac 4:36,\allowbreak37}
\crossref{Lev}{25}{35}{25:25 De 15:7,\allowbreak8 Pr 14:20,\allowbreak21; 17:5; 19:17 Mr 14:7 Joh 12:8 2Co 8:9}
\crossref{Lev}{25}{36}{Ex 22:25 De 23:19,\allowbreak20 Ne 5:7-\allowbreak10 Ps 15:5 Pr 28:8 Eze 18:8,\allowbreak13,\allowbreak17}
\crossref{Lev}{25}{37}{}
\crossref{Lev}{25}{38}{Ex 20:2}
\crossref{Lev}{25}{39}{Ex 21:2; 22:3 De 15:12 1Ki 9:22 2Ki 4:1 Ne 5:5 Jer 34:14}
\crossref{Lev}{25}{40}{Ex 21:2,\allowbreak3}
\crossref{Lev}{25}{41}{Ex 21:3 Joh 8:32 Ro 6:14 Tit 2:14}
\crossref{Lev}{25}{42}{25:55 Ro 6:22 1Co 7:21-\allowbreak23}
\crossref{Lev}{25}{43}{25:46,\allowbreak53 Ex 1:13,\allowbreak14; 2:23; 3:7,\allowbreak9; 5:14 Isa 47:6; 58:3 Eph 6:9}
\crossref{Lev}{25}{44}{Ex 12:44 Ps 2:8,\allowbreak9 Isa 14:1,\allowbreak2 Re 2:26,\allowbreak27}
\crossref{Lev}{25}{45}{Isa 56:3-\allowbreak6}
\crossref{Lev}{25}{46}{Isa 14:2}
\crossref{Lev}{25}{47}{}
\crossref{Lev}{25}{48}{25:25,\allowbreak35 Ne 5:5,\allowbreak8 Ga 4:4,\allowbreak5 Heb 2:11-\allowbreak13}
\crossref{Lev}{25}{49}{25:26}
\crossref{Lev}{25}{50}{25:27}
\crossref{Lev}{25}{51}{}
\crossref{Lev}{25}{52}{}
\crossref{Lev}{25}{53}{25:43}
\crossref{Lev}{25}{54}{25:40,\allowbreak41 Ex 21:2,\allowbreak3 Isa 49:9,\allowbreak25; 52:3}
\crossref{Lev}{25}{55}{25:42 Ex 13:3; 20:2 Ps 116:16 Isa 43:3 Lu 1:74,\allowbreak75 Ro 6:14,\allowbreak17,\allowbreak18}
\crossref{Lev}{26}{1}{Le 19:4 Ex 20:4,\allowbreak5,\allowbreak23; 23:24; 34:17 De 4:16-\allowbreak19; 5:8,\allowbreak9; 16:21,\allowbreak22; 27:15}
\crossref{Lev}{26}{2}{Le 19:30}
\crossref{Lev}{26}{3}{Le 18:4,\allowbreak5 De 11:13-\allowbreak15; 28:1-\allowbreak14 Jos 23:14,\allowbreak15 Jud 2:1,\allowbreak2 Ps 81:12-\allowbreak16}
\crossref{Lev}{26}{4}{De 28:12 1Ki 17:1 Job 5:10; 37:11-\allowbreak13; 38:25-\allowbreak28 Ps 65:9-\allowbreak13; 68:9}
\crossref{Lev}{26}{5}{Am 9:13 Mt 9:37,\allowbreak38 Joh 4:35,\allowbreak36}
\crossref{Lev}{26}{6}{1Ch 22:9 Ps 29:11; 147:14 Isa 9:7; 45:7 Jer 30:10 Ho 2:18}
\crossref{Lev}{26}{7}{1Sa 14:6 2Ch 20:5-\allowbreak7,\allowbreak12,\allowbreak14-\allowbreak17 Zec 12:8}
\crossref{Lev}{26}{8}{Nu 14:9 De 28:7; 32:30 Jos 23:10 Jud 7:19-\allowbreak21 1Sa 14:6-\allowbreak16}
\crossref{Lev}{26}{9}{Ex 2:25 2Ki 13:23 Ne 2:20 Ps 89:3; 138:6,\allowbreak7 Jer 33:3 Heb 8:9}
\crossref{Lev}{26}{10}{Le 25:22 Jos 5:11 2Ki 19:29 Lu 12:17}
\crossref{Lev}{26}{11}{Ex 25:8; 29:45 Jos 22:19 1Ki 8:13,\allowbreak27 Ps 76:2; 78:68,\allowbreak69}
\crossref{Lev}{26}{12}{Ge 3:8; 5:22,\allowbreak24; 6:9 De 23:14 2Co 6:16 Re 2:1}
\crossref{Lev}{26}{13}{Le 25:38,\allowbreak42,\allowbreak55}
\crossref{Lev}{26}{14}{26:18 De 28:15-\allowbreak68 Jer 17:27 La 1:18; 2:17 Mal 2:2 Ac 3:23}
\crossref{Lev}{26}{15}{26:43 Nu 15:31 2Sa 12:9,\allowbreak10 2Ki 17:15 2Ch 36:16 Pr 1:7,\allowbreak30}
\crossref{Lev}{26}{16}{Ps 109:6}
\crossref{Lev}{26}{17}{Le 17:10; 20:5,\allowbreak6 Ps 68:1,\allowbreak2}
\crossref{Lev}{26}{18}{26:21,\allowbreak24,\allowbreak28 1Sa 2:5 Ps 119:164 Pr 24:16 Da 3:19}
\crossref{Lev}{26}{19}{1Sa 4:3,\allowbreak11 Isa 2:12; 25:11; 26:5 Jer 13:9 Eze 7:24; 30:6 Da 4:37}
\crossref{Lev}{26}{20}{Ps 127:1 Isa 49:4 Hab 2:13 Ga 4:11}
\crossref{Lev}{26}{21}{}
\crossref{Lev}{26}{22}{26:6 De 32:24 2Ki 17:25 Jer 15:3 Eze 5:17; 14:15,\allowbreak21}
\crossref{Lev}{26}{23}{Isa 1:16-\allowbreak20 Jer 2:30; 5:3 Eze 24:13,\allowbreak14 Am 4:6-\allowbreak12}
\crossref{Lev}{26}{24}{2Sa 22:27 Job 9:4 Ps 18:26 Isa 63:10}
\crossref{Lev}{26}{25}{De 32:25,\allowbreak41 Jud 2:14-\allowbreak16 Ps 78:62-\allowbreak64 Isa 34:5,\allowbreak6 Jer 9:16}
\crossref{Lev}{26}{26}{Ps 105:16 Isa 3:1; 9:20 Jer 14:12 La 4:3-\allowbreak9 Eze 4:10,\allowbreak16; 5:16}
\crossref{Lev}{26}{27}{26:21,\allowbreak24}
\crossref{Lev}{26}{28}{Isa 27:4; 59:18; 63:3; 66:15 Jer 21:5 Eze 5:13,\allowbreak15; 8:18 Na 1:2,\allowbreak6}
\crossref{Lev}{26}{29}{}
\crossref{Lev}{26}{30}{1Ki 13:2 2Ki 23:8,\allowbreak16,\allowbreak20 2Ch 14:3-\allowbreak5; 23:17; 31:1; 34:3-\allowbreak7 Isa 27:9}
\crossref{Lev}{26}{31}{2Ki 25:4-\allowbreak10 2Ch 36:19 Ne 2:3,\allowbreak17 Isa 1:7; 24:10-\allowbreak12 Jer 4:7; 9:11}
\crossref{Lev}{26}{32}{De 29:23 Isa 1:7,\allowbreak8; 5:6,\allowbreak9; 6:11; 24:1; 32:13,\allowbreak14; 64:10 Jer 9:11}
\crossref{Lev}{26}{33}{De 4:27; 28:64-\allowbreak66 Ps 44:11 Jer 9:16 La 1:3; 4:15 Eze 12:14-\allowbreak16}
\crossref{Lev}{26}{34}{}
\crossref{Lev}{26}{35}{Isa 24:5,\allowbreak6 Ro 8:22}
\crossref{Lev}{26}{36}{Ge 35:5 De 28:65-\allowbreak67 Jos 2:9-\allowbreak11; 5:1 1Sa 17:24 2Ki 7:6,\allowbreak7}
\crossref{Lev}{26}{37}{Jud 7:22 1Sa 14:15,\allowbreak16 Isa 10:4 Jer 37:10}
\crossref{Lev}{26}{38}{De 4:27; 28:48,\allowbreak68 Isa 27:13 Jer 42:17,\allowbreak18,\allowbreak22; 44:12-\allowbreak14,\allowbreak27,\allowbreak28}
\crossref{Lev}{26}{39}{De 28:65; 30:1 Ne 1:9 Ps 32:3,\allowbreak4 Jer 3:25; 29:12 La 4:9 Eze 4:17}
\crossref{Lev}{26}{40}{Nu 5:7 De 4:29-\allowbreak31; 30:1-\allowbreak3 Jos 7:19 1Ki 8:33-\allowbreak36,\allowbreak47 Ne 9:2-\allowbreak5}
\crossref{Lev}{26}{41}{De 30:6 Jer 4:4; 6:10; 9:25,\allowbreak26 Eze 44:7 Ac 7:51 Ro 2:28,\allowbreak29}
\crossref{Lev}{26}{42}{Ge 9:16 Ex 2:24; 6:5 De 4:31 Ps 106:45 Eze 16:60 Lu 1:72}
\crossref{Lev}{26}{43}{26:34,\allowbreak35}
\crossref{Lev}{26}{44}{De 4:29-\allowbreak31 2Ki 13:23 Ne 9:31 Ps 94:14 Eze 14:22,\allowbreak23 Ro 11:2,\allowbreak26}
\crossref{Lev}{26}{45}{Ge 12:2; 15:18; 17:7,\allowbreak8 Ex 2:24; 19:5,\allowbreak6 Lu 1:72,\allowbreak73}
\crossref{Lev}{26}{46}{Le 27:34 De 6:1; 12:1; 13:4 Joh 1:17}
\crossref{Lev}{27}{1}{27:1}
\crossref{Lev}{27}{2}{Ge 28:20-\allowbreak22 Nu 6:2; 21:2 De 23:21-\allowbreak23 Jud 11:30,\allowbreak31,\allowbreak39}
\crossref{Lev}{27}{3}{27:14; 5:15; 6:6 Nu 18:16 2Ki 12:4}
\crossref{Lev}{27}{4}{}
\crossref{Lev}{27}{5}{}
\crossref{Lev}{27}{6}{Nu 3:40-\allowbreak43; 18:14-\allowbreak16}
\crossref{Lev}{27}{7}{Ps 90:10}
\crossref{Lev}{27}{8}{Le 5:7; 12:8; 14:21,\allowbreak22 Mr 14:7 Lu 21:1-\allowbreak4 2Co 8:12}
\crossref{Lev}{27}{9}{}
\crossref{Lev}{27}{10}{27:15-\allowbreak33 Jas 1:8}
\crossref{Lev}{27}{11}{De 23:18 Mal 1:14}
\crossref{Lev}{27}{12}{}
\crossref{Lev}{27}{13}{27:10,\allowbreak15,\allowbreak19; 5:16; 6:4,\allowbreak5; 22:14}
\crossref{Lev}{27}{14}{27:21; 25:29-\allowbreak31 Nu 18:14 Ps 101:2-\allowbreak7}
\crossref{Lev}{27}{15}{27:13}
\crossref{Lev}{27}{16}{Ac 4:34-\allowbreak37; 5:4}
\crossref{Lev}{27}{17}{27:17}
\crossref{Lev}{27}{18}{Le 25:15,\allowbreak16,\allowbreak27,\allowbreak51,\allowbreak52}
\crossref{Lev}{27}{19}{27:13}
\crossref{Lev}{27}{20}{}
\crossref{Lev}{27}{21}{Le 25:10,\allowbreak28,\allowbreak31}
\crossref{Lev}{27}{22}{Le 25:10,\allowbreak25}
\crossref{Lev}{27}{23}{27:12,\allowbreak18}
\crossref{Lev}{27}{24}{27:20; 25:28}
\crossref{Lev}{27}{25}{27:3}
\crossref{Lev}{27}{26}{Ex 13:2,\allowbreak12,\allowbreak13; 22:30 Nu 18:17 De 15:19}
\crossref{Lev}{27}{27}{}
\crossref{Lev}{27}{28}{}
\crossref{Lev}{27}{29}{Nu 21:2,\allowbreak3 1Sa 15:18-\allowbreak23}
\crossref{Lev}{27}{30}{Ge 14:20; 28:22 Nu 18:21-\allowbreak24 De 12:5,\allowbreak6; 14:22,\allowbreak23 2Ch 31:5,\allowbreak6,\allowbreak12}
\crossref{Lev}{27}{31}{}
\crossref{Lev}{27}{32}{}
\crossref{Lev}{27}{33}{}
\crossref{Lev}{27}{34}{Le 26:46 De 4:45 Joh 1:17}

% Num
\crossref{Num}{1}{1}{Nu 10:11,\allowbreak12 Ex 19:1 Le 27:34}
\crossref{Num}{1}{2}{Ge 49:1-\allowbreak3 Ex 1:1-\allowbreak5}
\crossref{Num}{1}{3}{Nu 14:29; 32:11 Ex 30:14}
\crossref{Num}{1}{4}{1:16; 2:3-\allowbreak31; 7:10-\allowbreak83; 10:14-\allowbreak27; 13:2-\allowbreak15; 17:3; 25:4,\allowbreak14; 34:18-\allowbreak28}
\crossref{Num}{1}{5}{Nu 2:10; 7:30; 10:18 Ge 29:32-\allowbreak35; 30:5-\allowbreak20; 35:17-\allowbreak26; 46:8-\allowbreak24; 49:1-\allowbreak33}
\crossref{Num}{1}{6}{Nu 2:12; 7:36}
\crossref{Num}{1}{7}{Nu 2:3; 7:12; 10:14 Ru 4:18 1Ch 2:10,\allowbreak11 Mt 1:2-\allowbreak5 Lu 3:32}
\crossref{Num}{1}{8}{Nu 2:5; 7:18; 10:15}
\crossref{Num}{1}{9}{Nu 2:7; 7:24; 10:16}
\crossref{Num}{1}{10}{Nu 2:18; 7:48; 10:22 1Ch 7:26,\allowbreak27}
\crossref{Num}{1}{11}{Nu 2:22; 7:60; 10:24}
\crossref{Num}{1}{12}{Nu 2:25; 7:66; 10:25}
\crossref{Num}{1}{13}{Nu 2:27; 7:72; 10:26}
\crossref{Num}{1}{14}{Nu 7:42; 10:20}
\crossref{Num}{1}{15}{Nu 2:29; 7:78; 10:27}
\crossref{Num}{1}{16}{1:4 Ex 18:21,\allowbreak25 De 1:15 1Sa 22:7; 23:23 Mic 5:2}
\crossref{Num}{1}{17}{1:5-\allowbreak15 Joh 10:3 Re 7:4-\allowbreak17}
\crossref{Num}{1}{18}{Ezr 2:59 Ne 7:61 Heb 7:3,\allowbreak6}
\crossref{Num}{1}{19}{1:2; 26:1,\allowbreak2 2Sa 24:1-\allowbreak10}
\crossref{Num}{1}{20}{Nu 2:10,\allowbreak11; 26:5-\allowbreak7 Ge 29:32; 46:9; 49:3,\allowbreak4 1Ch 5:1}
\crossref{Num}{1}{21}{Nu 2:10,\allowbreak11; 26:7}
\crossref{Num}{1}{22}{Nu 2:12,\allowbreak13; 26:12-\allowbreak14 Ge 29:33; 34:25-\allowbreak30; 42:24; 46:10; 49:5,\allowbreak6}
\crossref{Num}{1}{23}{Nu 2:13; 25:8,\allowbreak9,\allowbreak14; 26:14}
\crossref{Num}{1}{24}{Nu 2:15; 26:18}
\crossref{Num}{1}{25}{Nu 2:3,\allowbreak4; 26:19-\allowbreak22 Ge 29:35; 46:12; 49:8-\allowbreak12 2Sa 24:9 1Ch 5:2}
\crossref{Num}{1}{26}{Nu 2:3,\allowbreak4; 26:22 2Sa 24:9 2Ch 17:14-\allowbreak16}
\crossref{Num}{1}{27}{}
\crossref{Num}{1}{28}{Nu 2:5,\allowbreak6; 23:23-\allowbreak25 Ge 30:18; 46:13; 49:14,\allowbreak15}
\crossref{Num}{1}{29}{Nu 2:6; 26:25}
\crossref{Num}{1}{30}{Nu 2:7,\allowbreak8; 26:26,\allowbreak27 Ge 30:20; 46:14; 49:13}
\crossref{Num}{1}{31}{Nu 2:8; 26:27}
\crossref{Num}{1}{32}{Nu 2:18,\allowbreak19; 26:35-\allowbreak37 Ge 30:24; 37:1-\allowbreak36; 39:1-\allowbreak23; 46:20; 48:1-\allowbreak22}
\crossref{Num}{1}{33}{Ge 48:5 De 33:17}
\crossref{Num}{1}{34}{Nu 26:34 Ge 41:51; 46:20; 48:1; 50:23 De 33:17 Jos 4:12; 17:1}
\crossref{Num}{1}{35}{Nu 2:21; 26:34 Ge 48:19,\allowbreak20}
\crossref{Num}{1}{36}{Ge 35:16-\allowbreak18; 44:20; 46:21; 49:27}
\crossref{Num}{1}{37}{Nu 2:23; 26:41 Jud 20:44-\allowbreak46 2Ch 17:17}
\crossref{Num}{1}{38}{Ge 30:5,\allowbreak6; 46:23; 49:16,\allowbreak17}
\crossref{Num}{1}{39}{Nu 2:26; 26:43}
\crossref{Num}{1}{40}{Ge 30:12,\allowbreak13; 46:27; 49:20}
\crossref{Num}{1}{41}{Nu 2:28; 26:47}
\crossref{Num}{1}{42}{Ge 30:7,\allowbreak8; 46:24; 49:21}
\crossref{Num}{1}{43}{Nu 2:30; 26:50}
\crossref{Num}{1}{44}{1:2-\allowbreak16; 26:64}
\crossref{Num}{1}{45}{1:47}
\crossref{Num}{1}{46}{}
\crossref{Num}{1}{47}{1:3,\allowbreak50; 2:33; 3:1-\allowbreak51; 4:1-\allowbreak49; 8:1-\allowbreak26; 26:57-\allowbreak62 1Ch 6:1-\allowbreak81; 21:6}
\crossref{Num}{1}{48}{1:48}
\crossref{Num}{1}{49}{Nu 2:33; 26:62}
\crossref{Num}{1}{50}{Nu 3:1-\allowbreak10; 4:15,\allowbreak25-\allowbreak33 Ex 31:18; 32:26-\allowbreak29; 38:21 1Ch 23:1-\allowbreak32}
\crossref{Num}{1}{51}{Nu 4:5-\allowbreak33; 10:11,\allowbreak17-\allowbreak21}
\crossref{Num}{1}{52}{Nu 2:2,\allowbreak34; 10:1-\allowbreak36; 24:2}
\crossref{Num}{1}{53}{1:50; 3:7; 18:3 1Ti 4:13-\allowbreak16 2Ti 4:2}
\crossref{Num}{1}{54}{Nu 2:34 Ex 23:21,\allowbreak22; 39:32,\allowbreak43; 40:16,\allowbreak32 De 12:32 1Sa 15:22}
\crossref{Num}{2}{1}{2:1}
\crossref{Num}{2}{2}{2:3,\allowbreak10; 1:52; 10:14,\allowbreak18,\allowbreak22,\allowbreak25}
\crossref{Num}{2}{3}{Ge 49:8-\allowbreak10 Jud 1:1,\allowbreak2 1Ch 5:2}
\crossref{Num}{2}{4}{Nu 1:27; 26:22}
\crossref{Num}{2}{5}{Nu 1:8,\allowbreak28,\allowbreak29; 7:18,\allowbreak23; 10:15; 26:23-\allowbreak25}
\crossref{Num}{2}{6}{Nu 1:29; 26:25}
\crossref{Num}{2}{7}{Nu 1:9,\allowbreak30,\allowbreak31; 7:24,\allowbreak29; 10:16}
\crossref{Num}{2}{8}{Nu 1:31; 26:26,\allowbreak27}
\crossref{Num}{2}{9}{Nu 10:14}
\crossref{Num}{2}{10}{Ge 49:3,\allowbreak4 1Ch 5:1}
\crossref{Num}{2}{11}{Nu 1:21; 26:7}
\crossref{Num}{2}{12}{Nu 1:6; 7:36,\allowbreak41; 10:19}
\crossref{Num}{2}{13}{Nu 1:23; 26:14}
\crossref{Num}{2}{14}{Nu 1:14; 7:42,\allowbreak47; 10:20}
\crossref{Num}{2}{15}{Nu 1:25; 26:18}
\crossref{Num}{2}{16}{2:9,\allowbreak24,\allowbreak31}
\crossref{Num}{2}{17}{2:1; 1:50-\allowbreak53; 3:38; 10:17,\allowbreak21 1Co 14:40 Col 2:5}
\crossref{Num}{2}{18}{Nu 1:32; 10:22 Ge 48:5,\allowbreak14-\allowbreak20 De 33:17 Ps 80:1,\allowbreak2}
\crossref{Num}{2}{19}{Nu 1:33; 26:37}
\crossref{Num}{2}{20}{Nu 1:10; 7:54,\allowbreak59; 10:23}
\crossref{Num}{2}{21}{Nu 1:35; 26:34}
\crossref{Num}{2}{22}{Nu 1:11; 7:60,\allowbreak65; 10:24}
\crossref{Num}{2}{23}{Nu 1:37; 26:41}
\crossref{Num}{2}{24}{2:9,\allowbreak16,\allowbreak31}
\crossref{Num}{2}{25}{Nu 1:12; 7:66,\allowbreak71; 10:25}
\crossref{Num}{2}{26}{Nu 1:39; 26:43}
\crossref{Num}{2}{27}{Nu 1:13; 7:72,\allowbreak77}
\crossref{Num}{2}{28}{Nu 1:41; 26:47}
\crossref{Num}{2}{29}{Nu 1:42,\allowbreak43; 26:48-\allowbreak50}
\crossref{Num}{2}{30}{Nu 1:42,\allowbreak43; 26:50}
\crossref{Num}{2}{31}{2:9,\allowbreak16,\allowbreak24}
\crossref{Num}{2}{32}{2:9; 1:46; 11:21; 26:51 Ex 12:37; 38:26}
\crossref{Num}{2}{33}{Nu 1:47-\allowbreak49}
\crossref{Num}{2}{34}{Nu 1:54 Ex 39:42 Ps 119:6 Lu 1:6}
\crossref{Num}{3}{1}{Ge 2:4; 5:1; 10:1 Ex 6:16,\allowbreak20 Mt 1:1}
\crossref{Num}{3}{2}{Nu 26:60 Ex 6:23; 28:1 1Ch 6:3; 24:1}
\crossref{Num}{3}{3}{Ex 28:41; 40:13,\allowbreak15 Le 8:2,\allowbreak12,\allowbreak30}
\crossref{Num}{3}{4}{Nu 26:61 Le 10:1,\allowbreak2}
\crossref{Num}{3}{5}{3:5}
\crossref{Num}{3}{6}{}
\crossref{Num}{3}{7}{3:32; 8:26; 31:30 1Ch 23:28-\allowbreak32; 26:20,\allowbreak22,\allowbreak26}
\crossref{Num}{3}{8}{Nu 4:15,\allowbreak28,\allowbreak33; 10:17,\allowbreak21 1Ch 26:20-\allowbreak28 Ezr 8:24-\allowbreak30 Isa 52:11}
\crossref{Num}{3}{9}{Nu 8:19; 18:6,\allowbreak7 Eph 4:8,\allowbreak11}
\crossref{Num}{3}{10}{Nu 18:7 1Ch 6:32 Eze 44:8 Ac 6:3,\allowbreak4 Ro 12:7 1Ti 4:15,\allowbreak16}
\crossref{Num}{3}{11}{}
\crossref{Num}{3}{12}{3:11}
\crossref{Num}{3}{13}{Nu 8:16,\allowbreak17; 18:15 Ex 13:2,\allowbreak12; 22:29; 34:19 Le 27:26 Eze 44:30}
\crossref{Num}{3}{14}{3:14}
\crossref{Num}{3}{15}{3:22,\allowbreak28,\allowbreak34,\allowbreak39,\allowbreak40,\allowbreak43; 18:15,\allowbreak16; 26:62 Pr 8:17 Jer 2:2; 31:3}
\crossref{Num}{3}{16}{3:39,\allowbreak51; 4:27,\allowbreak37,\allowbreak41,\allowbreak45,\allowbreak49 Ge 45:21 De 21:5}
\crossref{Num}{3}{17}{Nu 26:57,\allowbreak58 Ge 46:11 Ex 6:16-\allowbreak19 Jos 21:1-\allowbreak45 1Ch 6:1,\allowbreak2,\allowbreak16-\allowbreak19}
\crossref{Num}{3}{18}{3:21 Ex 6:17-\allowbreak19 1Ch 6:17,\allowbreak20,\allowbreak21; 23:7-\allowbreak11; 25:4; 26:1-\allowbreak32 Ne 12:1-\allowbreak26}
\crossref{Num}{3}{19}{3:27 Ex 6:18,\allowbreak20 1Ch 6:18,\allowbreak38; 15:5,\allowbreak8-\allowbreak10,\allowbreak17-\allowbreak21; 23:12,\allowbreak13,\allowbreak18-\allowbreak20}
\crossref{Num}{3}{20}{3:33 Ex 6:19 1Ch 6:19,\allowbreak29,\allowbreak44-\allowbreak47; 15:6; 23:21-\allowbreak23; 24:27-\allowbreak30; 25:3}
\crossref{Num}{3}{21}{3:18}
\crossref{Num}{3}{22}{}
\crossref{Num}{3}{23}{Nu 1:53; 2:17}
\crossref{Num}{3}{24}{}
\crossref{Num}{3}{25}{Ex 25:9; 26:1-\allowbreak14; 36:8-\allowbreak19; 40:19}
\crossref{Num}{3}{26}{Ex 27:9-\allowbreak16; 38:9-\allowbreak16}
\crossref{Num}{3}{27}{3:19 1Ch 23:12; 26:23}
\crossref{Num}{3}{28}{Nu 4:35,\allowbreak36}
\crossref{Num}{3}{29}{3:23; 1:53; 2:10}
\crossref{Num}{3}{30}{Nu 3:30 1Ch 15:8}
\crossref{Num}{3}{31}{Nu 4:4-\allowbreak16 Ex 25:10-\allowbreak40; 31:1-\allowbreak35:29; 37:1-\allowbreak24; 39:33-\allowbreak42; 40:2-\allowbreak16,\allowbreak30}
\crossref{Num}{3}{32}{Nu 4:16,\allowbreak27; 20:25-\allowbreak28 2Ki 25:18 1Ch 9:14-\allowbreak20; 26:20-\allowbreak24}
\crossref{Num}{3}{33}{3:20 1Ch 6:19; 23:21}
\crossref{Num}{3}{34}{Nu 1:21; 2:9; 2:11}
\crossref{Num}{3}{35}{3:28,\allowbreak29; 1:53}
\crossref{Num}{3}{36}{}
\crossref{Num}{3}{37}{}
\crossref{Num}{3}{38}{3:23,\allowbreak29,\allowbreak35; 1:53; 2:3}
\crossref{Num}{3}{39}{}
\crossref{Num}{3}{40}{3:12,\allowbreak15,\allowbreak45 Ex 32:26-\allowbreak29 Ps 87:6 Isa 4:3 Lu 10:20 Php 4:3}
\crossref{Num}{3}{41}{3:12,\allowbreak45; 8:16; 18:15 Ex 24:5,\allowbreak6; 32:26-\allowbreak29 Mt 20:28 1Ti 2:6}
\crossref{Num}{3}{42}{3:42}
\crossref{Num}{3}{43}{3:39}
\crossref{Num}{3}{44}{3:44}
\crossref{Num}{3}{45}{3:12,\allowbreak40,\allowbreak41}
\crossref{Num}{3}{46}{Nu 18:15 Ex 13:13}
\crossref{Num}{3}{47}{Nu 18:16 Le 27:6}
\crossref{Num}{3}{48}{3:48}
\crossref{Num}{3}{49}{3:49}
\crossref{Num}{3}{50}{3:46,\allowbreak47 Mt 20:28 1Ti 2:5,\allowbreak6 Tit 2:14 Heb 9:12 1Pe 1:18; 3:18}
\crossref{Num}{3}{51}{3:48; 16:15 1Sa 12:3,\allowbreak4 Ac 20:33 1Co 9:12 1Pe 5:2}
\crossref{Num}{4}{1}{4:1}
\crossref{Num}{4}{2}{Nu 3:19,\allowbreak27}
\crossref{Num}{4}{3}{Nu 8:24-\allowbreak26 Ge 41:46 1Ch 23:3,\allowbreak24-\allowbreak27; 28:12,\allowbreak13 Lu 3:23 1Ti 3:6}
\crossref{Num}{4}{4}{4:15,\allowbreak19,\allowbreak24,\allowbreak30; 3:30,\allowbreak31 Mr 13:34}
\crossref{Num}{4}{5}{Nu 2:16,\allowbreak17; 10:14}
\crossref{Num}{4}{6}{4:7,\allowbreak8,\allowbreak11-\allowbreak13 Ex 35:19; 39:1,\allowbreak41}
\crossref{Num}{4}{7}{Ex 25:23-\allowbreak30; 37:10-\allowbreak16 Le 24:5-\allowbreak8}
\crossref{Num}{4}{8}{4:6,\allowbreak7,\allowbreak9,\allowbreak11-\allowbreak13}
\crossref{Num}{4}{9}{Ex 25:31-\allowbreak39; 37:17-\allowbreak24 Ps 119:105 Re 1:20; 2:1}
\crossref{Num}{4}{10}{4:6,\allowbreak12}
\crossref{Num}{4}{11}{Ex 30:1-\allowbreak19; 37:25-\allowbreak28; 39:38; 40:5,\allowbreak26,\allowbreak27}
\crossref{Num}{4}{12}{4:7,\allowbreak9; 3:8 Ex 25:9; 31:10 2Ki 25:14,\allowbreak15 1Ch 9:29 2Ch 4:11,\allowbreak16,\allowbreak19,\allowbreak22}
\crossref{Num}{4}{13}{}
\crossref{Num}{4}{14}{Ex 38:1-\allowbreak7 2Ch 4:19}
\crossref{Num}{4}{15}{Nu 7:9; 10:21 De 31:9 Jos 4:10 2Sa 6:13 1Ch 15:2,\allowbreak15}
\crossref{Num}{4}{16}{Ex 25:6; 27:20,\allowbreak21 Le 24:2}
\crossref{Num}{4}{17}{4:17}
\crossref{Num}{4}{18}{}
\crossref{Num}{4}{19}{4:4}
\crossref{Num}{4}{20}{4:15,\allowbreak19 Ex 19:21 Le 10:2 1Sa 6:19 Heb 10:19,\allowbreak20 Re 11:19}
\crossref{Num}{4}{21}{}
\crossref{Num}{4}{22}{Nu 3:18,\allowbreak21,\allowbreak24}
\crossref{Num}{4}{23}{4:3}
\crossref{Num}{4}{24}{4:15,\allowbreak19,\allowbreak27,\allowbreak31,\allowbreak32,\allowbreak47,\allowbreak49}
\crossref{Num}{4}{25}{Nu 3:25,\allowbreak26; 7:5-\allowbreak7}
\crossref{Num}{4}{26}{Ex 27:9}
\crossref{Num}{4}{27}{}
\crossref{Num}{4}{28}{4:33 1Co 12:5,\allowbreak6}
\crossref{Num}{4}{29}{Nu 3:33-\allowbreak35}
\crossref{Num}{4}{30}{4:3,\allowbreak23 Ps 110:1-\allowbreak7 1Ti 6:11,\allowbreak12 2Ti 2:4; 4:7,\allowbreak8}
\crossref{Num}{4}{31}{Nu 3:36,\allowbreak37; 7:8,\allowbreak9}
\crossref{Num}{4}{32}{Nu 3:8; 7:1 Ex 25:9; 38:17,\allowbreak21 1Ch 9:29}
\crossref{Num}{4}{33}{4:28 Jos 3:6 Isa 3:6}
\crossref{Num}{4}{34}{4:2}
\crossref{Num}{4}{35}{4:3,\allowbreak23,\allowbreak30,\allowbreak39,\allowbreak43,\allowbreak47; 8:24-\allowbreak26 1Ch 23:3,\allowbreak24,\allowbreak26,\allowbreak27; 28:13}
\crossref{Num}{4}{36}{4:36}
\crossref{Num}{4}{37}{4:37}
\crossref{Num}{4}{38}{4:38}
\crossref{Num}{4}{39}{4:39}
\crossref{Num}{4}{40}{Nu 3:32}
\crossref{Num}{4}{41}{4:22}
\crossref{Num}{4}{42}{4:42}
\crossref{Num}{4}{43}{4:43}
\crossref{Num}{4}{44}{}
\crossref{Num}{4}{45}{4:29}
\crossref{Num}{4}{46}{}
\crossref{Num}{4}{47}{4:3,\allowbreak23,\allowbreak30 1Ch 23:3,\allowbreak27}
\crossref{Num}{4}{48}{Nu 3:39 Mt 7:14; 20:16; 22:15}
\crossref{Num}{4}{49}{4:37,\allowbreak41,\allowbreak45; 1:54; 2:33; 3:51}
\crossref{Num}{5}{1}{5:1}
\crossref{Num}{5}{2}{Le 15:2-\allowbreak27}
\crossref{Num}{5}{3}{1Ki 7:3 1Co 5:7-\allowbreak13 2Co 6:17 2Th 3:6 Tit 3:10 Heb 12:15,\allowbreak16}
\crossref{Num}{5}{4}{5:4}
\crossref{Num}{5}{5}{}
\crossref{Num}{5}{6}{Le 5:1-\allowbreak4,\allowbreak17; 6:2,\allowbreak3}
\crossref{Num}{5}{7}{Le 5:5; 26:40 Jos 7:19 Job 33:27,\allowbreak28 Ps 32:5 Pr 28:13}
\crossref{Num}{5}{8}{Le 25:25,\allowbreak26}
\crossref{Num}{5}{9}{Nu 18:8,\allowbreak9,\allowbreak19 Ex 29:28 Le 6:17,\allowbreak18,\allowbreak26; 7:6-\allowbreak14; 10:13; 22:2,\allowbreak3}
\crossref{Num}{5}{10}{1Co 3:21-\allowbreak23 1Pe 2:5,\allowbreak7,\allowbreak9}
\crossref{Num}{5}{11}{5:11}
\crossref{Num}{5}{12}{5:19,\allowbreak20 Pr 2:16,\allowbreak17}
\crossref{Num}{5}{13}{Le 18:20; 20:10 Pr 7:18,\allowbreak19; 30:20}
\crossref{Num}{5}{14}{}
\crossref{Num}{5}{15}{1Ki 17:18 Eze 29:16 Heb 10:3}
\crossref{Num}{5}{16}{}
\crossref{Num}{5}{17}{Job 2:12 Jer 17:13 La 3:29 Joh 8:6,\allowbreak8}
\crossref{Num}{5}{18}{Heb 13:4 Re 2:19-\allowbreak23}
\crossref{Num}{5}{19}{Mt 26:63}
\crossref{Num}{5}{20}{}
\crossref{Num}{5}{21}{Jos 6:26 1Sa 14:24 Ne 10:29 Mt 26:74}
\crossref{Num}{5}{22}{5:27 Ps 109:18 Pr 1:31 Eze 3:3}
\crossref{Num}{5}{23}{Ex 17:14 De 31:19 2Ch 34:24 Job 31:35 Jer 51:60-\allowbreak64}
\crossref{Num}{5}{24}{Zec 5:3,\allowbreak4 Mal 3:5}
\crossref{Num}{5}{25}{5:15,\allowbreak18}
\crossref{Num}{5}{26}{Le 2:2,\allowbreak9; 5:12; 6:15}
\crossref{Num}{5}{27}{5:20 Pr 5:4-\allowbreak11 Ec 7:26 Ro 6:21 2Co 2:16 Heb 10:26-\allowbreak30 2Pe 2:10}
\crossref{Num}{5}{28}{5:19 Mic 7:7-\allowbreak10 2Co 4:17 1Pe 1:7}
\crossref{Num}{5}{29}{Le 7:11; 11:46; 13:59; 14:54-\allowbreak57; 15:32,\allowbreak33}
\crossref{Num}{5}{30}{Ge 41:8}
\crossref{Num}{5}{31}{Ps 37:6}
\crossref{Num}{6}{1}{6:1}
\crossref{Num}{6}{2}{6:5,\allowbreak6 Ex 33:16 Le 20:26 Pr 18:1 Ro 1:1 2Co 6:16 Ga 1:15}
\crossref{Num}{6}{3}{}
\crossref{Num}{6}{4}{6:5,\allowbreak8,\allowbreak9,\allowbreak12,\allowbreak13,\allowbreak18,\allowbreak19,\allowbreak21}
\crossref{Num}{6}{5}{Jud 13:5; 16:17,\allowbreak19 1Sa 1:11 La 4:7,\allowbreak8 1Co 11:10-\allowbreak15}
\crossref{Num}{6}{6}{Nu 19:11-\allowbreak16 Le 19:28 Jer 16:5,\allowbreak6 Eze 24:16-\allowbreak18 Mt 8:21,\allowbreak22}
\crossref{Num}{6}{7}{Nu 9:6 Le 21:1,\allowbreak2,\allowbreak10-\allowbreak12 Eze 44:25}
\crossref{Num}{6}{8}{2Co 6:17,\allowbreak18}
\crossref{Num}{6}{9}{Nu 19:14-\allowbreak19}
\crossref{Num}{6}{10}{Le 1:14; 5:7-\allowbreak10; 9:1-\allowbreak21; 12:6; 14:22,\allowbreak23,\allowbreak31; 15:14,\allowbreak29 Ro 4:25}
\crossref{Num}{6}{11}{Le 5:8-\allowbreak10; 14:30,\allowbreak31}
\crossref{Num}{6}{12}{Le 5:6; 14:24}
\crossref{Num}{6}{13}{Ac 21:26}
\crossref{Num}{6}{14}{Le 1:10-\allowbreak13 1Ch 15:26,\allowbreak28,\allowbreak29}
\crossref{Num}{6}{15}{Le 2:4; 8:2; 9:4 Joh 6:50-\allowbreak59}
\crossref{Num}{6}{16}{Le 23:19}
\crossref{Num}{6}{17}{Le 23:19}
\crossref{Num}{6}{18}{Lu 17:10 Eph 1:6}
\crossref{Num}{6}{19}{Le 8:31 1Sa 2:15}
\crossref{Num}{6}{20}{Nu 5:25 Ex 29:27,\allowbreak28 Le 9:21; 10:15; 23:11}
\crossref{Num}{6}{21}{Nu 5:29}
\crossref{Num}{6}{22}{6:22}
\crossref{Num}{6}{23}{Ge 14:19,\allowbreak20; 24:60; 27:27-\allowbreak29; 28:3,\allowbreak4; 47:7,\allowbreak10; 48:20 Le 9:22,\allowbreak23}
\crossref{Num}{6}{24}{Ru 2:4 Ps 134:3 1Co 14:16 Eph 6:24 Php 4:23 Re 1:4,\allowbreak5}
\crossref{Num}{6}{25}{Ps 21:6; 31:16; 67:1; 80:1-\allowbreak3,\allowbreak7,\allowbreak19; 119:135 Da 9:17}
\crossref{Num}{6}{26}{Ps 4:6; 42:5; 89:15 Ac 2:28}
\crossref{Num}{6}{27}{Ex 3:13-\allowbreak15; 6:3; 34:5-\allowbreak7 De 28:10 2Ch 7:14 Isa 43:7 Jer 14:9}
\crossref{Num}{7}{1}{Ex 40:17-\allowbreak19}
\crossref{Num}{7}{2}{Nu 1:4-\allowbreak16; 2:1-\allowbreak34; 10:1-\allowbreak36}
\crossref{Num}{7}{3}{}
\crossref{Num}{7}{4}{}
\crossref{Num}{7}{5}{Ex 25:1-\allowbreak11; 35:4-\allowbreak10 Ps 16:2,\allowbreak3 Isa 42:1-\allowbreak7; 49:1-\allowbreak8 Eph 4:11-\allowbreak13}
\crossref{Num}{7}{6}{}
\crossref{Num}{7}{7}{Nu 3:25,\allowbreak26; 4:24-\allowbreak28}
\crossref{Num}{7}{8}{Nu 3:36,\allowbreak37; 4:28-\allowbreak33}
\crossref{Num}{7}{9}{Nu 3:31; 4:4-\allowbreak16 2Sa 6:6,\allowbreak13 1Ch 15:3,\allowbreak13; 23:26}
\crossref{Num}{7}{10}{De 20:5 1Ki 8:63 2Ch 7:5,\allowbreak9 Ezr 6:16,\allowbreak17 Ne 12:27,\allowbreak43}
\crossref{Num}{7}{11}{1Co 14:33,\allowbreak40 Col 2:5}
\crossref{Num}{7}{12}{Nu 1:7; 2:3; 10:14 Ge 49:8,\allowbreak10 Ru 4:20 Mt 1:4 Lu 3:32}
\crossref{Num}{7}{13}{Ex 25:29; 37:16 1Ki 7:43,\allowbreak45 2Ki 25:14,\allowbreak15 Ezr 1:9,\allowbreak10; 8:25}
\crossref{Num}{7}{14}{Nu 4:7 Ex 37:16 1Ki 7:50 2Ki 25:14,\allowbreak15 2Ch 4:22; 24:14}
\crossref{Num}{7}{15}{Nu 25:1-\allowbreak18; 28:1-\allowbreak29:40 Le 1:1-\allowbreak17 Isa 53:4,\allowbreak10,\allowbreak11 Mt 20:28}
\crossref{Num}{7}{16}{Le 4:23,\allowbreak25}
\crossref{Num}{7}{17}{Le 3:1-\allowbreak17 2Co 5:19-\allowbreak21}
\crossref{Num}{7}{18}{Nu 1:8; 2:5}
\crossref{Num}{7}{19}{7:12-\allowbreak17}
\crossref{Num}{7}{20}{7:20}
\crossref{Num}{7}{21}{Ge 8:20 Ro 12:1 Eph 5:2}
\crossref{Num}{7}{22}{7:22}
\crossref{Num}{7}{23}{Le 7:11-\allowbreak13 1Ki 8:63 Pr 7:14 Col 1:1}
\crossref{Num}{7}{24}{Nu 1:9; 2:7}
\crossref{Num}{7}{25}{7:25}
\crossref{Num}{7}{26}{7:26}
\crossref{Num}{7}{27}{Ps 50:8-\allowbreak14; 51:16 Isa 1:11 Jer 7:22 Am 5:22}
\crossref{Num}{7}{28}{}
\crossref{Num}{7}{29}{2Sa 24:22,\allowbreak23 Isa 32:8}
\crossref{Num}{7}{30}{Nu 1:5; 2:10}
\crossref{Num}{7}{31}{7:13-\allowbreak89}
\crossref{Num}{7}{32}{Ps 66:15 Mal 1:11 Lu 1:10 Re 8:3}
\crossref{Num}{7}{33}{7:33}
\crossref{Num}{7}{34}{}
\crossref{Num}{7}{35}{}
\crossref{Num}{7}{36}{Nu 1:6; 2:12}
\crossref{Num}{7}{37}{7:13-\allowbreak89}
\crossref{Num}{7}{38}{7:38}
\crossref{Num}{7}{39}{Ex 12:5 Joh 1:29 Ac 8:32 1Pe 1:19 Re 5:6}
\crossref{Num}{7}{40}{}
\crossref{Num}{7}{41}{2Co 9:5-\allowbreak7}
\crossref{Num}{7}{42}{Nu 1:14; 2:14}
\crossref{Num}{7}{43}{7:13-\allowbreak89}
\crossref{Num}{7}{44}{7:44}
\crossref{Num}{7}{45}{Ps 66:15 Isa 53:4 2Co 5:21}
\crossref{Num}{7}{46}{}
\crossref{Num}{7}{47}{Ex 35:29}
\crossref{Num}{7}{48}{Nu 1:10; 2:18}
\crossref{Num}{7}{49}{7:13-\allowbreak89}
\crossref{Num}{7}{50}{7:50}
\crossref{Num}{7}{51}{7:51}
\crossref{Num}{7}{52}{}
\crossref{Num}{7}{53}{Ex 35:20-\allowbreak24}
\crossref{Num}{7}{54}{Nu 1:10; 2:20}
\crossref{Num}{7}{55}{7:13-\allowbreak89}
\crossref{Num}{7}{56}{7:56}
\crossref{Num}{7}{57}{7:57}
\crossref{Num}{7}{58}{}
\crossref{Num}{7}{59}{2Ch 24:8-\allowbreak11}
\crossref{Num}{7}{60}{Nu 1:11; 2:22}
\crossref{Num}{7}{61}{7:13-\allowbreak89}
\crossref{Num}{7}{62}{Ps 112:2 Isa 66:20 Da 9:27 Ro 15:16 Php 4:18 Heb 13:15}
\crossref{Num}{7}{63}{7:63}
\crossref{Num}{7}{64}{}
\crossref{Num}{7}{65}{Mr 12:41-\allowbreak44}
\crossref{Num}{7}{66}{Nu 1:12; 2:25}
\crossref{Num}{7}{67}{7:13-\allowbreak89}
\crossref{Num}{7}{68}{7:68}
\crossref{Num}{7}{69}{7:69}
\crossref{Num}{7}{70}{}
\crossref{Num}{7}{71}{2Co 8:1-\allowbreak4}
\crossref{Num}{7}{72}{Nu 1:13; 2:27}
\crossref{Num}{7}{73}{7:13-\allowbreak89}
\crossref{Num}{7}{74}{7:74}
\crossref{Num}{7}{75}{7:75}
\crossref{Num}{7}{76}{}
\crossref{Num}{7}{77}{Php 4:17,\allowbreak18}
\crossref{Num}{7}{78}{Nu 1:15; 2:29}
\crossref{Num}{7}{79}{7:13-\allowbreak89}
\crossref{Num}{7}{80}{7:80}
\crossref{Num}{7}{81}{7:81}
\crossref{Num}{7}{82}{}
\crossref{Num}{7}{83}{Ro 11:35,\allowbreak36}
\crossref{Num}{7}{84}{7:10 1Ch 29:6-\allowbreak8 Ezr 2:68,\allowbreak69 Ne 7:70-\allowbreak72 Isa 60:6-\allowbreak10 Heb 13:10}
\crossref{Num}{7}{85}{1Ch 22:14; 29:4,\allowbreak7 Ezr 8:25,\allowbreak26}
\crossref{Num}{7}{86}{7:86}
\crossref{Num}{7}{87}{}
\crossref{Num}{7}{88}{7:1,\allowbreak10,\allowbreak84}
\crossref{Num}{7}{89}{Nu 12:8 Ex 33:9-\allowbreak11}
\crossref{Num}{8}{1}{8:1}
\crossref{Num}{8}{2}{Ex 25:37; 37:18,\allowbreak19,\allowbreak23; 40:25 Le 24:1,\allowbreak2 Ps 119:105,\allowbreak130 Isa 8:20}
\crossref{Num}{8}{3}{Ps 40:7,\allowbreak8 Joh 8:28,\allowbreak29 Heb 3:1-\allowbreak6}
\crossref{Num}{8}{4}{Ex 25:31-\allowbreak39; 37:17-\allowbreak24}
\crossref{Num}{8}{5}{}
\crossref{Num}{8}{6}{Ex 19:15 2Co 7:1 Jas 4:8}
\crossref{Num}{8}{7}{Le 8:6; 14:7 Isa 52:15 Eze 36:25 Heb 9:10}
\crossref{Num}{8}{8}{Ex 29:1,\allowbreak3 Le 1:3; 8:2}
\crossref{Num}{8}{9}{Ex 29:4-\allowbreak37; 40:12}
\crossref{Num}{8}{10}{Nu 3:45 Le 1:4 Ac 6:6; 13:2,\allowbreak3 1Ti 4:14; 5:22}
\crossref{Num}{8}{11}{Nu 1:49-\allowbreak53; 3:5-\allowbreak43}
\crossref{Num}{8}{12}{Ex 29:10 Le 1:4; 8:14; 16:21}
\crossref{Num}{8}{13}{}
\crossref{Num}{8}{14}{Nu 6:2 De 10:8 Ro 1:1 Ga 1:15 Heb 7:26}
\crossref{Num}{8}{15}{8:11; 3:23-\allowbreak37; 4:3-\allowbreak32 1Ch 23:1-\allowbreak32; 25:1-\allowbreak26:32}
\crossref{Num}{8}{16}{}
\crossref{Num}{8}{17}{Nu 3:13 Ex 13:2,\allowbreak12-\allowbreak15 Lu 2:23}
\crossref{Num}{8}{18}{Nu 16:8,\allowbreak9 Mt 1:23-\allowbreak25 Lu 2:23}
\crossref{Num}{8}{19}{Nu 3:6-\allowbreak9; 18:2-\allowbreak6 1Ch 23:28-\allowbreak32 Eze 44:11-\allowbreak14}
\crossref{Num}{8}{20}{}
\crossref{Num}{8}{21}{8:7; 19:12,\allowbreak19}
\crossref{Num}{8}{22}{8:15 2Ch 30:15-\allowbreak17,\allowbreak27; 31:2; 35:8-\allowbreak15}
\crossref{Num}{8}{23}{}
\crossref{Num}{8}{24}{1Co 9:7 2Co 10:4 1Ti 1:18; 6:12 2Ti 2:3-\allowbreak5}
\crossref{Num}{8}{25}{}
\crossref{Num}{8}{26}{Nu 1:53; 3:32; 18:4; 31:30 1Ch 23:32; 26:20-\allowbreak29 Eze 44:8,\allowbreak11}
\crossref{Num}{9}{1}{Nu 1:1 Ex 40:2}
\crossref{Num}{9}{2}{Ex 12:1-\allowbreak20}
\crossref{Num}{9}{3}{2Ch 30:2,\allowbreak15}
\crossref{Num}{9}{4}{Ex 12:1-\allowbreak11 Le 23:5 De 16:1 Jos 5:10}
\crossref{Num}{9}{5}{Jos 5:10}
\crossref{Num}{9}{6}{Nu 5:2; 6:6,\allowbreak7; 19:11,\allowbreak16,\allowbreak18 Le 21:11 Joh 18:28}
\crossref{Num}{9}{7}{9:2 Ex 12:27 De 16:2 2Ch 30:17-\allowbreak19 1Co 5:7,\allowbreak8}
\crossref{Num}{9}{8}{Ex 14:13 2Ch 20:17}
\crossref{Num}{9}{9}{}
\crossref{Num}{9}{10}{9:6,\allowbreak7 Ro 15:8-\allowbreak19; 16:25,\allowbreak26 1Co 6:9-\allowbreak11 Eph 2:1,\allowbreak2,\allowbreak12,\allowbreak13; 3:6-\allowbreak9}
\crossref{Num}{9}{11}{9:3 Ex 12:2-\allowbreak14,\allowbreak43-\allowbreak49 2Ch 30:2-\allowbreak15 Joh 19:36}
\crossref{Num}{9}{12}{Ex 12:10}
\crossref{Num}{9}{13}{Nu 15:30,\allowbreak31; 19:13 Ge 17:14 Ex 12:15 Le 17:4,\allowbreak10,\allowbreak14-\allowbreak16 Heb 2:3}
\crossref{Num}{9}{14}{Ex 12:48,\allowbreak49 Le 19:10; 22:25; 24:22; 25:15 De 29:11; 31:12}
\crossref{Num}{9}{15}{Ex 40:2,\allowbreak18}
\crossref{Num}{9}{16}{9:18-\allowbreak22 Ex 13:21,\allowbreak22; 40:38 De 1:33 Ne 9:12,\allowbreak19 Ps 78:14; 105:39}
\crossref{Num}{9}{17}{Nu 10:11,\allowbreak33,\allowbreak34 Ex 40:36-\allowbreak38 Ps 80:1,\allowbreak2 Isa 49:10 Joh 10:3-\allowbreak5,\allowbreak9}
\crossref{Num}{9}{18}{9:20; 10:13 Ex 17:1 2Jo 1:6}
\crossref{Num}{9}{19}{Nu 1:52,\allowbreak53; 3:8 Zec 3:7}
\crossref{Num}{9}{20}{Ge 34:30 Job 16:22 Isa 10:19}
\crossref{Num}{9}{21}{Ne 9:12,\allowbreak19}
\crossref{Num}{9}{22}{9:17; 1:54; 8:20; 23:21,\allowbreak22 Ex 39:42; 40:16,\allowbreak36,\allowbreak37 De 1:6,\allowbreak7; 2:3,\allowbreak4}
\crossref{Num}{9}{23}{9:19 Ge 26:5 Jos 22:3 Eze 44:8 Zec 3:7}
\crossref{Num}{10}{1}{10:1}
\crossref{Num}{10}{2}{Ex 25:18,\allowbreak31 Eph 4:5}
\crossref{Num}{10}{3}{Jer 4:5 Joe 2:15,\allowbreak16}
\crossref{Num}{10}{4}{Nu 1:4-\allowbreak16; 7:2 Ex 18:21 De 1:15}
\crossref{Num}{10}{5}{10:6,\allowbreak7 Isa 58:1 Joe 2:1}
\crossref{Num}{10}{6}{Nu 2:10-\allowbreak16}
\crossref{Num}{10}{7}{10:3,\allowbreak4}
\crossref{Num}{10}{8}{Nu 31:6 Jos 6:4-\allowbreak16 1Ch 15:24; 16:6 2Ch 13:12-\allowbreak15}
\crossref{Num}{10}{9}{Nu 31:6 Jos 6:5 2Ch 13:14}
\crossref{Num}{10}{10}{Nu 29:1 Le 23:24; 25:9,\allowbreak10 1Ch 15:24,\allowbreak28; 16:42 2Ch 5:12,\allowbreak13; 7:6}
\crossref{Num}{10}{11}{Nu 9:17-\allowbreak23}
\crossref{Num}{10}{12}{Nu 33:16 Ex 13:20; 40:36,\allowbreak37 De 1:19}
\crossref{Num}{10}{13}{Nu 9:23}
\crossref{Num}{10}{14}{}
\crossref{Num}{10}{15}{Nu 1:8; 7:18}
\crossref{Num}{10}{16}{Nu 1:9; 7:24}
\crossref{Num}{10}{17}{Nu 1:51 Heb 9:11; 12:28 2Pe 1:14}
\crossref{Num}{10}{18}{Nu 2:10-\allowbreak16; 26:5-\allowbreak18}
\crossref{Num}{10}{19}{Nu 1:6; 7:36}
\crossref{Num}{10}{20}{Nu 1:14; 2:14}
\crossref{Num}{10}{21}{Nu 2:17; 3:27-\allowbreak32; 4:4-\allowbreak16; 7:9 1Ch 15:2,\allowbreak12-\allowbreak15}
\crossref{Num}{10}{22}{Nu 2:18-\allowbreak24; 26:23-\allowbreak41 Ge 48:19 Ps 80:1,\allowbreak2}
\crossref{Num}{10}{23}{Nu 1:10; 7:54}
\crossref{Num}{10}{24}{Nu 1:11; 7:60}
\crossref{Num}{10}{25}{Nu 2:25,\allowbreak28-\allowbreak31; 26:42-\allowbreak51 Ge 49:16,\allowbreak17}
\crossref{Num}{10}{26}{Nu 1:13; 7:72}
\crossref{Num}{10}{27}{Nu 1:15; 7:78}
\crossref{Num}{10}{28}{10:35,\allowbreak36; 2:34; 24:4,\allowbreak5 So 6:10 1Co 14:33,\allowbreak40 Col 2:5}
\crossref{Num}{10}{29}{Ex 2:18}
\crossref{Num}{10}{30}{Ge 12:1; 31:30 Ru 1:15-\allowbreak17 Ps 45:10 Lu 14:26 2Co 5:16}
\crossref{Num}{10}{31}{Job 29:15 Ps 32:8 1Co 12:14-\allowbreak21 Ga 6:2}
\crossref{Num}{10}{32}{Jud 1:16; 4:11 1Jo 1:3}
\crossref{Num}{10}{33}{Ex 3:1; 19:3; 24:17,\allowbreak18}
\crossref{Num}{10}{34}{Ex 13:21,\allowbreak22 Ne 9:12,\allowbreak19 Ps 105:39}
\crossref{Num}{10}{35}{Ps 68:1,\allowbreak2; 132:8 Isa 51:9}
\crossref{Num}{10}{36}{Ps 90:13-\allowbreak17}
\crossref{Num}{11}{1}{Nu 10:33; 20:2-\allowbreak5; 21:5 Ex 15:23,\allowbreak24; 16:2,\allowbreak3,\allowbreak7,\allowbreak9; 17:2,\allowbreak3 De 9:22}
\crossref{Num}{11}{2}{Nu 21:7 Ps 78:34,\allowbreak35 Jer 37:3; 42:2 Ac 8:24}
\crossref{Num}{11}{3}{De 9:22}
\crossref{Num}{11}{4}{Ex 12:38 Le 24:10,\allowbreak11 Ne 13:3}
\crossref{Num}{11}{5}{Ex 16:3 Ps 17:14 Php 3:19}
\crossref{Num}{11}{6}{Nu 21:5 2Sa 13:4}
\crossref{Num}{11}{7}{Ex 16:14,\allowbreak15,\allowbreak31 1Co 1:23,\allowbreak24 Re 2:17}
\crossref{Num}{11}{8}{Ex 16:16-\allowbreak18 Joh 6:27,\allowbreak33-\allowbreak58}
\crossref{Num}{11}{9}{Ex 16:13,\allowbreak14 De 32:2 Ps 78:23-\allowbreak25; 105:40}
\crossref{Num}{11}{10}{Nu 14:1,\allowbreak2; 16:27; 21:5 Ps 106:25}
\crossref{Num}{11}{11}{Job 10:2 Ps 130:3; 143:2 La 3:22,\allowbreak23,\allowbreak39,\allowbreak40}
\crossref{Num}{11}{12}{Isa 40:11 Eze 34:23 Joh 10:11}
\crossref{Num}{11}{13}{Mt 15:33 Mr 8:4; 9:23}
\crossref{Num}{11}{14}{Ex 18:18 De 1:9-\allowbreak12 Ps 89:19 Isa 9:6 Zec 6:13 2Co 2:16}
\crossref{Num}{11}{15}{1Ki 19:4 Job 3:20-\allowbreak22; 6:8-\allowbreak10; 7:15 Jon 4:3,\allowbreak8,\allowbreak9 Php 1:20-\allowbreak24}
\crossref{Num}{11}{16}{Ge 46:27 Ex 4:29; 24:1,\allowbreak9 Eze 8:11 Lu 10:1,\allowbreak17}
\crossref{Num}{11}{17}{11:25; 12:5 Ge 11:5; 18:21 Ex 19:11,\allowbreak20; 34:5 Joh 3:13}
\crossref{Num}{11}{18}{Ge 35:2 Ex 19:10,\allowbreak15 Jos 7:13}
\crossref{Num}{11}{19}{11:19}
\crossref{Num}{11}{20}{Ex 16:8,\allowbreak13}
\crossref{Num}{11}{21}{Nu 1:46; 2:32 Ge 12:2 Ex 12:37; 38:26}
\crossref{Num}{11}{22}{2Ki 7:2 Mt 15:33 Mr 6:37; 8:4 Lu 1:18,\allowbreak34 Joh 6:6,\allowbreak7,\allowbreak9}
\crossref{Num}{11}{23}{Nu 23:19 2Ki 7:2,\allowbreak17-\allowbreak19 Jer 44:28,\allowbreak29 Eze 12:25; 24:14 Mt 24:35}
\crossref{Num}{11}{24}{11:16,\allowbreak26}
\crossref{Num}{11}{25}{11:17; 12:5 Ex 34:5; 40:38 Ps 99:7 Lu 9:34,\allowbreak35}
\crossref{Num}{11}{26}{Ex 3:11; 4:13,\allowbreak14 1Sa 10:22; 20:26 Jer 1:6; 36:5}
\crossref{Num}{11}{27}{}
\crossref{Num}{11}{28}{Ex 17:9}
\crossref{Num}{11}{29}{1Co 3:3,\allowbreak21; 13:4 Php 2:3 Jas 3:14,\allowbreak15; 4:5; 5:9 1Pe 2:1}
\crossref{Num}{11}{30}{}
\crossref{Num}{11}{31}{Ex 10:13,\allowbreak19; 15:10 Ps 135:7}
\crossref{Num}{11}{32}{Ex 16:36 Eze 45:11}
\crossref{Num}{11}{33}{Ps 78:30,\allowbreak31; 106:14,\allowbreak15}
\crossref{Num}{11}{34}{Nu 33:16 De 9:22 1Co 10:6}
\crossref{Num}{11}{35}{Nu 33:17}
\crossref{Num}{12}{1}{Mt 10:36; 12:48 Joh 7:5; 15:20 Ga 4:16}
\crossref{Num}{12}{2}{Nu 16:3 Ex 4:30; 5:1; 7:10; 15:20,\allowbreak21 Mic 6:4}
\crossref{Num}{12}{3}{Ps 147:6; 149:4 Mt 5:5; 11:29; 21:5 2Co 10:1 1Th 2:7 Jas 3:13}
\crossref{Num}{12}{4}{Ps 76:9}
\crossref{Num}{12}{5}{Nu 11:25 Ex 34:5; 40:38 Ps 99:7}
\crossref{Num}{12}{6}{Ge 20:7 Ex 7:1 Ps 105:15 Mt 23:31,\allowbreak34,\allowbreak37 Lu 20:6 Eph 4:11}
\crossref{Num}{12}{7}{De 18:18 Ps 105:26 Mt 11:9,\allowbreak11 Ac 3:22,\allowbreak23; 7:31}
\crossref{Num}{12}{8}{Nu 14:14 Ex 33:11 De 34:10 1Ti 6:16}
\crossref{Num}{12}{9}{Nu 11:1 Ho 5:15}
\crossref{Num}{12}{10}{Ex 33:7-\allowbreak10 Eze 10:4,\allowbreak5,\allowbreak18,\allowbreak19 Ho 9:12 Mt 25:41}
\crossref{Num}{12}{11}{Ex 12:32 1Sa 2:30; 12:19; 15:24,\allowbreak25 1Ki 13:6 Jer 42:2 Ac 8:24}
\crossref{Num}{12}{12}{Ps 88:4,\allowbreak5 Eph 2:1-\allowbreak5 Col 2:13 1Ti 5:6}
\crossref{Num}{12}{13}{Nu 14:2,\allowbreak13-\allowbreak20; 16:41,\allowbreak46-\allowbreak50 Ex 32:10-\allowbreak14 1Sa 12:23; 15:11 Mt 5:44,\allowbreak45}
\crossref{Num}{12}{14}{De 25:9 Job 30:10 Isa 50:6 Mt 26:67 Heb 12:9}
\crossref{Num}{12}{15}{De 24:8,\allowbreak9}
\crossref{Num}{12}{16}{Nu 11:35; 33:18}
\crossref{Num}{13}{1}{13:1}
\crossref{Num}{13}{2}{Nu 32:8 De 1:22-\allowbreak25 Jos 2:1-\allowbreak24}
\crossref{Num}{13}{3}{Nu 12:16; 32:8 De 1:19,\allowbreak23; 9:23}
\crossref{Num}{13}{4}{1Ch 25:2}
\crossref{Num}{13}{5}{1Ch 3:22}
\crossref{Num}{13}{6}{13:30; 14:6,\allowbreak24,\allowbreak30,\allowbreak38; 26:65; 27:15-\allowbreak23; 34:19 De 31:7-\allowbreak17 Jos 14:6-\allowbreak15}
\crossref{Num}{13}{7}{}
\crossref{Num}{13}{8}{13:16}
\crossref{Num}{13}{9}{13:9}
\crossref{Num}{13}{10}{13:10}
\crossref{Num}{13}{11}{13:11}
\crossref{Num}{13}{12}{13:12}
\crossref{Num}{13}{13}{}
\crossref{Num}{13}{14}{}
\crossref{Num}{13}{15}{}
\crossref{Num}{13}{16}{Ho 1:1 Ro 9:25}
\crossref{Num}{13}{17}{13:21,\allowbreak22 Ge 12:9; 13:1 Jos 15:3 Jud 1:15}
\crossref{Num}{13}{18}{Ex 3:8 Eze 34:14}
\crossref{Num}{13}{19}{Nu 32:17,\allowbreak36 Jos 10:20; 19:29,\allowbreak35 1Sa 6:18 2Sa 24:7 2Ki 3:19; 8:12}
\crossref{Num}{13}{20}{Ne 9:25,\allowbreak35 Eze 34:14}
\crossref{Num}{13}{21}{}
\crossref{Num}{13}{22}{Jos 11:21,\allowbreak22; 15:13,\allowbreak14 Jud 1:10}
\crossref{Num}{13}{23}{13:24; 32:9 De 1:24,\allowbreak25 Jud 16:4}
\crossref{Num}{13}{24}{13:23}
\crossref{Num}{13}{25}{Nu 14:33,\allowbreak34 Ex 24:18; 34:28}
\crossref{Num}{13}{26}{13:3}
\crossref{Num}{13}{27}{Nu 14:8 Ex 3:8,\allowbreak17; 13:5; 33:3 Le 20:24 De 1:25-\allowbreak33; 6:3; 11:9}
\crossref{Num}{13}{28}{De 1:28; 2:10,\allowbreak11,\allowbreak21; 3:5; 9:1,\allowbreak2}
\crossref{Num}{13}{29}{Nu 14:43; 24:20 Ge 14:7 Ex 17:8-\allowbreak16 Jud 6:3 1Sa 14:48; 15:3-\allowbreak9; 30:1}
\crossref{Num}{13}{30}{Nu 14:6-\allowbreak9,\allowbreak24 Jos 14:6-\allowbreak8 Ps 27:1,\allowbreak2; 60:12; 118:10,\allowbreak11 Isa 41:10-\allowbreak16}
\crossref{Num}{13}{31}{Nu 32:9 De 1:28 Jos 14:8 Heb 3:19}
\crossref{Num}{13}{32}{Nu 14:36,\allowbreak37 De 1:28 Mt 23:13}
\crossref{Num}{13}{33}{13:22 De 1:28; 2:10; 3:11; 9:2 1Sa 17:4-\allowbreak7 2Sa 21:20-\allowbreak22 1Ch 11:23}
\crossref{Num}{14}{1}{Nu 11:1-\allowbreak4 De 1:45}
\crossref{Num}{14}{2}{Nu 16:41 Ex 15:24; 16:2,\allowbreak3; 17:3 De 1:27 Ps 106:24,\allowbreak45 1Co 10:10}
\crossref{Num}{14}{3}{Ps 78:40 Jer 9:3}
\crossref{Num}{14}{4}{De 17:16; 28:68 Ne 9:16,\allowbreak17 Lu 17:32 Ac 7:39 Heb 10:38,\allowbreak39; 11:15}
\crossref{Num}{14}{5}{Nu 16:4,\allowbreak22,\allowbreak45 Ge 17:3 Le 9:24 Jos 5:14; 7:10 1Ki 18:39}
\crossref{Num}{14}{6}{14:24,\allowbreak30,\allowbreak38; 13:6,\allowbreak8,\allowbreak30}
\crossref{Num}{14}{7}{Nu 13:27 De 1:25; 6:10,\allowbreak11; 8:7-\allowbreak9}
\crossref{Num}{14}{8}{De 10:15 2Sa 15:25,\allowbreak26; 22:20 1Ki 10:9 Ps 22:8; 147:10,\allowbreak11}
\crossref{Num}{14}{9}{De 9:7,\allowbreak23,\allowbreak24 Isa 1:2; 63:10 Da 9:5,\allowbreak9 Php 1:27}
\crossref{Num}{14}{10}{Ex 17:4 1Sa 30:6 Mt 23:37 Ac 7:52,\allowbreak59}
\crossref{Num}{14}{11}{14:27 Ex 10:3; 16:28 Pr 1:22 Jer 4:14 Ho 8:5 Zec 8:14}
\crossref{Num}{14}{12}{Nu 16:46-\allowbreak49; 25:9 Ex 5:3 2Sa 24:1,\allowbreak12-\allowbreak15}
\crossref{Num}{14}{13}{Ex 32:12 De 9:26-\allowbreak28; 32:27 Jos 7:8,\allowbreak9 Ps 106:23 Eze 20:9,\allowbreak14}
\crossref{Num}{14}{14}{Ex 15:14 Jos 2:9,\allowbreak10; 5:1}
\crossref{Num}{14}{15}{Jud 6:16}
\crossref{Num}{14}{16}{De 9:28; 32:26,\allowbreak27 Jos 7:9}
\crossref{Num}{14}{17}{Mic 3:8 Mt 9:6,\allowbreak8}
\crossref{Num}{14}{18}{Ex 34:6,\allowbreak7 Ps 103:8; 145:8 Jon 4:2 Mic 7:18 Na 1:2,\allowbreak3 Ro 3:24-\allowbreak26}
\crossref{Num}{14}{19}{Ex 32:32; 34:9 1Ki 8:34 Ps 51:1,\allowbreak2 Eze 20:8,\allowbreak9 Da 9:19}
\crossref{Num}{14}{20}{Ps 78:37-\allowbreak43; 106:45 Isa 48:9,\allowbreak11 Eph 4:32}
\crossref{Num}{14}{21}{De 32:40 Isa 49:18 Jer 22:24 Eze 5:11; 18:3; 33:11,\allowbreak27 Zep 2:9}
\crossref{Num}{14}{22}{14:11 De 1:31-\allowbreak35 Ps 95:9-\allowbreak11; 106:26 Heb 3:17,\allowbreak18}
\crossref{Num}{14}{23}{Nu 26:64; 32:11 De 1:35-\allowbreak45 Ne 9:23 Ps 95:11; 106:26 Eze 20:15}
\crossref{Num}{14}{24}{14:6-\allowbreak9; 13:30; 26:65 De 1:36 Jos 14:6-\allowbreak14}
\crossref{Num}{14}{25}{Nu 13:29}
\crossref{Num}{14}{26}{}
\crossref{Num}{14}{27}{14:11 Ex 16:28 Mt 17:7 Mr 9:19}
\crossref{Num}{14}{28}{14:21,\allowbreak23; 26:64,\allowbreak65; 32:11 De 1:35 Ps 90:8,\allowbreak9 Heb 3:17}
\crossref{Num}{14}{29}{14:32,\allowbreak33 1Co 10:5 Heb 3:17 Jude 1:5}
\crossref{Num}{14}{30}{Ge 14:22}
\crossref{Num}{14}{31}{Nu 26:6,\allowbreak64 De 1:39}
\crossref{Num}{14}{32}{14:29 1Co 10:5 Heb 3:17}
\crossref{Num}{14}{33}{Nu 33:38 De 1:3; 2:14}
\crossref{Num}{14}{34}{Nu 13:25 2Ch 36:21}
\crossref{Num}{14}{35}{Nu 23:19}
\crossref{Num}{14}{36}{Nu 13:31-\allowbreak33}
\crossref{Num}{14}{37}{14:12; 16:49; 25:9 Jer 28:16,\allowbreak17; 29:32 1Co 10:10 Heb 3:17 Jude 1:5}
\crossref{Num}{14}{38}{Nu 26:65 Jos 14:6-\allowbreak10}
\crossref{Num}{14}{39}{Ex 33:4 Pr 19:3 Isa 26:16 Mt 8:12 Heb 12:17}
\crossref{Num}{14}{40}{De 1:41 Ec 9:3 Mt 7:21-\allowbreak23; 26:11,\allowbreak12 Lu 13:25}
\crossref{Num}{14}{41}{14:25 2Ch 24:20}
\crossref{Num}{14}{42}{De 1:42 Jos 7:8,\allowbreak12 Ps 44:1,\allowbreak2-\allowbreak11}
\crossref{Num}{14}{43}{14:25; 13:29 Le 26:17 De 28:25}
\crossref{Num}{14}{44}{Nu 10:33 1Sa 4:3-\allowbreak11}
\crossref{Num}{14}{45}{14:43 Ex 17:16 De 1:44; 32:30 Jos 7:5,\allowbreak11,\allowbreak12}
\crossref{Num}{15}{1}{}
\crossref{Num}{15}{2}{15:18 Le 14:34; 23:10; 25:2 De 7:1,\allowbreak2; 12:1,\allowbreak9}
\crossref{Num}{15}{3}{Le 1:1-\allowbreak17}
\crossref{Num}{15}{4}{Ex 29:40 Le 2:1; 6:14; 7:9,\allowbreak10; 23:13 Isa 66:20 Mal 1:11 Ro 15:16}
\crossref{Num}{15}{5}{Nu 28:7,\allowbreak14 Jud 9:13 Ps 116:13 So 1:4 Zec 9:17 Mt 26:28,\allowbreak29}
\crossref{Num}{15}{6}{15:4; 28:12-\allowbreak14}
\crossref{Num}{15}{7}{}
\crossref{Num}{15}{8}{Le 3:1; 7:11-\allowbreak18}
\crossref{Num}{15}{9}{Nu 28:12,\allowbreak14}
\crossref{Num}{15}{10}{15:5; 6:15}
\crossref{Num}{15}{11}{15:28}
\crossref{Num}{15}{12}{15:12}
\crossref{Num}{15}{13}{15:13}
\crossref{Num}{15}{14}{}
\crossref{Num}{15}{15}{15:29; 9:14 Ex 12:49 Le 24:22 Ga 3:28 Eph 2:11-\allowbreak22 Col 3:11}
\crossref{Num}{15}{16}{15:16}
\crossref{Num}{15}{17}{}
\crossref{Num}{15}{18}{15:2 De 26:1-\allowbreak15}
\crossref{Num}{15}{19}{}
\crossref{Num}{15}{20}{Nu 18:12 Ex 23:19 De 26:2-\allowbreak10 Ne 10:37 Pr 3:9,\allowbreak10 Eze 44:30}
\crossref{Num}{15}{21}{Nu 18:26 Ex 29:28}
\crossref{Num}{15}{22}{}
\crossref{Num}{15}{23}{}
\crossref{Num}{15}{24}{Le 4:13}
\crossref{Num}{15}{25}{Le 4:20,\allowbreak26 Ro 3:25 1Jo 2:2}
\crossref{Num}{15}{26}{}
\crossref{Num}{15}{27}{Le 4:27,\allowbreak28 Ac 3:17; 17:30 1Ti 1:13}
\crossref{Num}{15}{28}{Le 4:35}
\crossref{Num}{15}{29}{15:15; 9:14 Le 16:29; 17:15 Ro 3:29,\allowbreak30}
\crossref{Num}{15}{30}{Nu 9:13; 14:44 Ge 17:14 Ex 21:14 Le 20:3,\allowbreak6,\allowbreak10 De 1:43; 17:12}
\crossref{Num}{15}{31}{Le 26:15,\allowbreak43 2Sa 12:9 Ps 119:126 Pr 13:13 Isa 30:12 1Th 4:8}
\crossref{Num}{15}{32}{}
\crossref{Num}{15}{33}{Joh 8:3-\allowbreak20}
\crossref{Num}{15}{34}{Le 24:12}
\crossref{Num}{15}{35}{Ex 31:14,\allowbreak15}
\crossref{Num}{15}{36}{Jos 7:25}
\crossref{Num}{15}{37}{}
\crossref{Num}{15}{38}{}
\crossref{Num}{15}{39}{Ex 13:9 De 6:6-\allowbreak9; 11:18-\allowbreak21,\allowbreak28-\allowbreak32 Pr 3:1}
\crossref{Num}{15}{40}{Le 11:44,\allowbreak45; 19:2 Ro 12:1 Eph 1:4 Col 1:2 1Th 4:7 1Pe 1:15,\allowbreak16}
\crossref{Num}{15}{41}{Le 22:33; 25:38 Ps 105:45 Jer 31:31-\allowbreak33; 32:37-\allowbreak41 Eze 36:25-\allowbreak27}
\crossref{Num}{16}{1}{Nu 26:9,\allowbreak10; 27:3 Ex 6:18,\allowbreak21 Jude 1:11}
\crossref{Num}{16}{2}{Nu 26:9 Ge 6:4 1Ch 5:24; 12:30 Eze 16:14; 23:10}
\crossref{Num}{16}{3}{16:11; 12:1,\allowbreak2; 14:1-\allowbreak4 Ps 106:16 Ac 7:39,\allowbreak51}
\crossref{Num}{16}{4}{16:45; 14:5; 20:6 Jos 7:6}
\crossref{Num}{16}{5}{Mal 3:18 2Ti 2:19}
\crossref{Num}{16}{6}{16:35-\allowbreak40,\allowbreak46-\allowbreak48 Le 10:1; 16:12,\allowbreak13 1Ki 18:21-\allowbreak23}
\crossref{Num}{16}{7}{16:3,\allowbreak5 Eph 1:4 2Th 2:13 1Pe 2:9}
\crossref{Num}{16}{8}{}
\crossref{Num}{16}{9}{16:13 Ge 30:15 1Sa 18:23 2Sa 7:19 Isa 7:13 Eze 34:18 1Co 4:3}
\crossref{Num}{16}{10}{Pr 13:10 Mt 20:21,\allowbreak22 Lu 22:24 Ro 12:10 Php 2:3 3Jo 1:9}
\crossref{Num}{16}{11}{16:3 1Sa 8:7 Lu 10:16 Joh 13:20 Ro 13:2}
\crossref{Num}{16}{12}{Pr 29:9 Isa 3:5 1Pe 2:13,\allowbreak14 Jude 1:8}
\crossref{Num}{16}{13}{16:9}
\crossref{Num}{16}{14}{Nu 36:8-\allowbreak10}
\crossref{Num}{16}{15}{Nu 12:3 Ex 32:19 Mt 5:22 Mr 3:5 Eph 4:26}
\crossref{Num}{16}{16}{16:6,\allowbreak7}
\crossref{Num}{16}{17}{1Sa 12:7}
\crossref{Num}{16}{18}{}
\crossref{Num}{16}{19}{16:1,\allowbreak2}
\crossref{Num}{16}{20}{}
\crossref{Num}{16}{21}{Ge 19:15-\allowbreak22 Jer 5:16 Ac 2:40 2Co 6:17 Eph 5:6,\allowbreak7 Re 18:4}
\crossref{Num}{16}{22}{16:4,\allowbreak45; 14:5}
\crossref{Num}{16}{23}{}
\crossref{Num}{16}{24}{16:21}
\crossref{Num}{16}{25}{Nu 11:16,\allowbreak17,\allowbreak25,\allowbreak30}
\crossref{Num}{16}{26}{16:21-\allowbreak24 Ge 19:12-\allowbreak14 De 13:17 Isa 52:11 Mt 10:14 Ac 8:20; 13:51}
\crossref{Num}{16}{27}{2Ki 9:30,\allowbreak31 Job 9:4; 40:10,\allowbreak11 Pr 16:18; 18:12 Isa 28:14}
\crossref{Num}{16}{28}{Ex 3:12; 4:1-\allowbreak9; 7:9 De 18:22 Zec 2:9; 4:9 Joh 5:36; 11:42; 14:11}
\crossref{Num}{16}{29}{Ex 20:5; 32:34 Job 35:15 Isa 10:3 Jer 5:9 La 4:22}
\crossref{Num}{16}{30}{16:33 Ps 55:15}
\crossref{Num}{16}{31}{Nu 26:10,\allowbreak11; 27:3 De 11:6 Ps 106:17,\allowbreak18}
\crossref{Num}{16}{32}{16:30 Ge 4:11 Isa 5:14 Re 12:16}
\crossref{Num}{16}{33}{Ps 9:15; 55:23; 69:15; 143:7 Isa 14:9,\allowbreak15 Eze 32:18,\allowbreak30}
\crossref{Num}{16}{34}{Isa 33:3 Zec 14:5 Re 6:15-\allowbreak17}
\crossref{Num}{16}{35}{Nu 11:1; 26:10 Le 10:2 Ps 106:18}
\crossref{Num}{16}{36}{}
\crossref{Num}{16}{37}{16:7,\allowbreak18}
\crossref{Num}{16}{38}{1Ki 2:23 Pr 1:18; 8:36; 20:2 Hab 2:10}
\crossref{Num}{16}{39}{}
\crossref{Num}{16}{40}{Nu 3:10,\allowbreak38; 18:4-\allowbreak7 Le 22:10 2Ch 26:18-\allowbreak20 Jude 1:11}
\crossref{Num}{16}{41}{16:1-\allowbreak7; 14:2 Ps 106:13,\allowbreak23,\allowbreak25-\allowbreak48 Isa 26:11}
\crossref{Num}{16}{42}{16:19}
\crossref{Num}{16}{43}{16:43}
\crossref{Num}{16}{44}{}
\crossref{Num}{16}{45}{16:21,\allowbreak24,\allowbreak26}
\crossref{Num}{16}{46}{Le 9:24; 10:1; 16:12,\allowbreak13 Isa 6:6,\allowbreak7 Ro 5:9,\allowbreak10 Heb 7:25-\allowbreak27; 9:25,\allowbreak26}
\crossref{Num}{16}{47}{Mt 5:44 Ro 12:21}
\crossref{Num}{16}{48}{}
\crossref{Num}{16}{49}{16:32-\allowbreak35; 25:9 1Ch 21:14 Heb 2:1-\allowbreak3; 10:28,\allowbreak29; 12:25}
\crossref{Num}{16}{50}{16:43 1Ch 21:26-\allowbreak30}
\crossref{Num}{17}{1}{17:1}
\crossref{Num}{17}{2}{Nu 1:5-\allowbreak16; 2:3-\allowbreak30; 10:14-\allowbreak27}
\crossref{Num}{17}{3}{Nu 3:2,\allowbreak3; 18:1,\allowbreak7 Ex 6:16,\allowbreak20}
\crossref{Num}{17}{4}{Ex 25:16-\allowbreak22; 29:42,\allowbreak43; 30:6,\allowbreak36}
\crossref{Num}{17}{5}{Nu 16:5}
\crossref{Num}{17}{6}{}
\crossref{Num}{17}{7}{Nu 18:2 Ex 38:21 Ac 7:44}
\crossref{Num}{17}{8}{17:5 Ge 40:10 Ps 110:2; 132:17,\allowbreak18 So 2:3 Isa 4:2 Eze 17:24}
\crossref{Num}{17}{9}{}
\crossref{Num}{17}{10}{Heb 9:4}
\crossref{Num}{17}{11}{}
\crossref{Num}{17}{12}{Nu 26:11 Ps 90:7 Pr 19:3 Isa 57:16 Heb 12:5}
\crossref{Num}{17}{13}{Nu 1:51-\allowbreak53; 18:4-\allowbreak7}
\crossref{Num}{18}{1}{Nu 17:3,\allowbreak7,\allowbreak13 Heb 4:15}
\crossref{Num}{18}{2}{Nu 3:6-\allowbreak9; 8:19,\allowbreak22}
\crossref{Num}{18}{3}{Nu 3:25,\allowbreak31,\allowbreak36; 4:19,\allowbreak20; 16:40}
\crossref{Num}{18}{4}{Nu 1:51; 3:10 1Sa 6:19 2Sa 6:6,\allowbreak7}
\crossref{Num}{18}{5}{Nu 8:2 Ex 27:21; 30:7-\allowbreak10 Le 24:3 1Ch 9:19,\allowbreak23,\allowbreak33; 24:5 1Ti 1:18}
\crossref{Num}{18}{6}{Ge 6:17; 9:9 Ex 14:17; 31:6 Isa 48:15; 51:12 Eze 34:11,\allowbreak20}
\crossref{Num}{18}{7}{18:5; 3:10}
\crossref{Num}{18}{8}{18:9 Le 6:16,\allowbreak18,\allowbreak20,\allowbreak26; 7:6,\allowbreak32-\allowbreak34; 10:14,\allowbreak15 De 12:6,\allowbreak11; 26:13}
\crossref{Num}{18}{9}{Le 2:2,\allowbreak3; 10:12,\allowbreak13}
\crossref{Num}{18}{10}{Ex 29:31,\allowbreak32 Le 6:16,\allowbreak26,\allowbreak29; 7:6; 10:13,\allowbreak17; 14:13}
\crossref{Num}{18}{11}{18:8 Ex 29:27,\allowbreak28 Le 7:14,\allowbreak30-\allowbreak34}
\crossref{Num}{18}{12}{18:29}
\crossref{Num}{18}{13}{Ex 22:29 Jer 24:2 Ho 9:10 Mic 7:1}
\crossref{Num}{18}{14}{Le 27:28 Eze 44:29}
\crossref{Num}{18}{15}{Nu 3:13 Ex 13:2,\allowbreak12; 22:29; 34:20 Le 27:26}
\crossref{Num}{18}{16}{Nu 3:47 Le 27:2-\allowbreak7}
\crossref{Num}{18}{17}{De 15:19-\allowbreak22}
\crossref{Num}{18}{18}{Ex 29:26-\allowbreak28 Le 7:31-\allowbreak34}
\crossref{Num}{18}{19}{18:8,\allowbreak11; 15:19-\allowbreak21; 31:29,\allowbreak41 Le 7:14 De 12:6 2Ch 31:4}
\crossref{Num}{18}{20}{18:23,\allowbreak24; 26:62 De 10:9; 12:12; 14:27,\allowbreak29 Jos 14:3}
\crossref{Num}{18}{21}{18:24-\allowbreak26 Le 27:30-\allowbreak32 De 12:17-\allowbreak19; 14:22-\allowbreak29 2Ch 31:5,\allowbreak6,\allowbreak12}
\crossref{Num}{18}{22}{18:7; 1:52; 3:10,\allowbreak38}
\crossref{Num}{18}{23}{Nu 3:7}
\crossref{Num}{18}{24}{Mal 3:8-\allowbreak10}
\crossref{Num}{18}{25}{}
\crossref{Num}{18}{26}{18:19}
\crossref{Num}{18}{27}{Le 6:19-\allowbreak23}
\crossref{Num}{18}{28}{Ge 14:18 Heb 6:20; 7:1-\allowbreak10}
\crossref{Num}{18}{29}{18:12}
\crossref{Num}{18}{30}{18:28 Ge 43:11 De 6:5 Pr 3:9,\allowbreak10 Mal 1:8 Mt 6:33; 10:37-\allowbreak39}
\crossref{Num}{18}{31}{De 14:22,\allowbreak23}
\crossref{Num}{18}{32}{18:22 Le 19:8}
\crossref{Num}{19}{1}{19:1}
\crossref{Num}{19}{2}{Nu 31:21 Heb 9:10}
\crossref{Num}{19}{3}{Nu 5:2; 15:36 Le 4:12,\allowbreak21; 13:45,\allowbreak46; 16:27; 24:14 Heb 13:11-\allowbreak13}
\crossref{Num}{19}{4}{Le 4:6,\allowbreak17; 16:14,\allowbreak19 Heb 9:13,\allowbreak14; 12:24 1Pe 1:2}
\crossref{Num}{19}{5}{Ex 29:14 Le 4:11,\allowbreak12,\allowbreak21 Ps 22:14 Isa 53:10}
\crossref{Num}{19}{6}{Le 14:4,\allowbreak6,\allowbreak49 Ps 51:7 Isa 1:18 Heb 9:19-\allowbreak23}
\crossref{Num}{19}{7}{19:8,\allowbreak19 Le 11:25,\allowbreak40; 14:8,\allowbreak9; 15:5; 16:26-\allowbreak28}
\crossref{Num}{19}{8}{}
\crossref{Num}{19}{9}{19:18; 9:13 2Co 5:21 Heb 7:26; 9:13}
\crossref{Num}{19}{10}{19:7,\allowbreak8,\allowbreak19}
\crossref{Num}{19}{11}{}
\crossref{Num}{19}{12}{Nu 31:19 Ex 19:11,\allowbreak15 Le 7:17 Ho 6:2 1Co 15:3,\allowbreak4}
\crossref{Num}{19}{13}{Nu 15:30 Le 5:3,\allowbreak6,\allowbreak17; 15:31 Heb 2:2,\allowbreak3; 10:29 Re 21:8; 22:11,\allowbreak15}
\crossref{Num}{19}{14}{19:14}
\crossref{Num}{19}{15}{Nu 31:20 Le 11:32; 14:36}
\crossref{Num}{19}{16}{19:11; 31:19}
\crossref{Num}{19}{17}{19:9}
\crossref{Num}{19}{18}{19:9 Ps 51:7 Eze 36:25-\allowbreak27 Joh 15:2,\allowbreak3; 17:17,\allowbreak19 1Co 1:30 Heb 9:14}
\crossref{Num}{19}{19}{Eph 5:25-\allowbreak27 Tit 2:14; 3:3-\allowbreak5 1Jo 1:7; 2:1,\allowbreak2 Jude 1:23 Re 1:5,\allowbreak6}
\crossref{Num}{19}{20}{19:13; 15:30 Ge 17:14 Mr 16:16 Ac 13:39-\allowbreak41 Ro 2:4,\allowbreak5 2Pe 3:14}
\crossref{Num}{19}{21}{Le 11:25,\allowbreak40; 16:26-\allowbreak28 Heb 7:19; 9:10,\allowbreak13,\allowbreak14; 10:4}
\crossref{Num}{19}{22}{Le 7:19 Hag 2:13}
\crossref{Num}{20}{1}{Nu 13:21; 27:14; 33:36 De 32:51}
\crossref{Num}{20}{2}{Ex 15:23,\allowbreak24; 17:1-\allowbreak4}
\crossref{Num}{20}{3}{Nu 14:1,\allowbreak2 Ex 16:2,\allowbreak3; 17:2 Job 3:10,\allowbreak11}
\crossref{Num}{20}{4}{Nu 11:5 Ex 5:21; 17:3 Ps 106:21 Ac 7:35,\allowbreak39,\allowbreak40}
\crossref{Num}{20}{5}{Nu 16:14 De 8:15 Ne 9:21 Jer 2:2,\allowbreak6 Eze 20:36}
\crossref{Num}{20}{6}{Nu 14:5; 16:4,\allowbreak22,\allowbreak45 Ex 17:4 Jos 7:6 1Ch 21:16 Ps 109:3,\allowbreak4}
\crossref{Num}{20}{7}{}
\crossref{Num}{20}{8}{Nu 21:15,\allowbreak18 Ex 4:2,\allowbreak17; 7:20; 14:16; 17:5,\allowbreak9}
\crossref{Num}{20}{9}{Nu 17:10}
\crossref{Num}{20}{10}{De 9:24 Ps 106:32,\allowbreak33 Mt 5:22 Lu 9:54,\allowbreak55 Ac 23:3-\allowbreak5 Eph 4:26}
\crossref{Num}{20}{11}{20:8 Le 10:1 1Sa 15:13,\allowbreak14,\allowbreak19,\allowbreak24 1Ki 13:21-\allowbreak24 1Ch 13:9,\allowbreak10}
\crossref{Num}{20}{12}{Nu 11:21,\allowbreak22 2Ch 20:20 Isa 7:9 Mt 17:17,\allowbreak20 Lu 1:20,\allowbreak45 Ro 4:20}
\crossref{Num}{20}{13}{De 33:8 Ps 95:8; 106:32-\allowbreak48}
\crossref{Num}{20}{14}{Jud 11:16,\allowbreak17}
\crossref{Num}{20}{15}{Ge 46:6 Ac 7:15}
\crossref{Num}{20}{16}{Ex 2:23,\allowbreak24; 3:7-\allowbreak9; 6:5; 14:10}
\crossref{Num}{20}{17}{Nu 21:1,\allowbreak22-\allowbreak24 De 2:1-\allowbreak4,\allowbreak27,\allowbreak29}
\crossref{Num}{20}{18}{20:18}
\crossref{Num}{20}{19}{De 2:6,\allowbreak28}
\crossref{Num}{20}{20}{20:18 Ge 27:41; 32:6 Jud 11:17,\allowbreak20 Ps 120:7 Eze 35:5-\allowbreak11}
\crossref{Num}{20}{21}{De 2:27,\allowbreak29}
\crossref{Num}{20}{22}{20:1,\allowbreak14,\allowbreak16; 13:26; 33:36,\allowbreak37 Eze 47:19; 48:28}
\crossref{Num}{20}{23}{}
\crossref{Num}{20}{24}{Nu 27:13; 31:2 Ge 15:15; 25:8,\allowbreak17; 35:29; 49:29,\allowbreak33 De 32:50 Jud 2:10}
\crossref{Num}{20}{25}{Nu 33:38,\allowbreak39}
\crossref{Num}{20}{26}{Ex 29:29,\allowbreak30 Isa 22:21,\allowbreak22 Heb 7:11,\allowbreak23,\allowbreak24}
\crossref{Num}{20}{27}{}
\crossref{Num}{20}{28}{20:26; 33:38-\allowbreak49 Ex 29:29,\allowbreak30}
\crossref{Num}{20}{29}{Ge 1:10 De 34:8 2Ch 35:24,\allowbreak25 Ac 8:2}
\crossref{Num}{21}{1}{Nu 33:40 Jos 12:14 Jud 1:16}
\crossref{Num}{21}{2}{Ge 28:20 Jud 11:30 1Sa 1:11 2Sa 15:7,\allowbreak8 Ps 56:12,\allowbreak13; 116:18}
\crossref{Num}{21}{3}{Ps 10:17; 91:15; 102:17}
\crossref{Num}{21}{4}{Nu 20:22,\allowbreak23,\allowbreak27; 33:41}
\crossref{Num}{21}{5}{Nu 11:1-\allowbreak6; 14:1-\allowbreak4; 16:13,\allowbreak14,\allowbreak41; 17:12 Ex 14:11; 15:24; 16:2,\allowbreak3,\allowbreak7,\allowbreak8}
\crossref{Num}{21}{6}{Ge 3:14,\allowbreak15 De 8:15 Isa 14:29; 30:6 Jer 8:17 Am 9:3,\allowbreak4 1Co 10:9}
\crossref{Num}{21}{7}{Ex 9:27,\allowbreak28 1Sa 12:19; 15:24,\allowbreak30 Ps 78:34 Mt 27:4}
\crossref{Num}{21}{8}{Ps 106:43-\allowbreak45; 145:8}
\crossref{Num}{21}{9}{2Ki 18:4 Joh 3:14,\allowbreak15; 12:32 Ro 8:3 2Co 5:21}
\crossref{Num}{21}{10}{Nu 33:43-\allowbreak45}
\crossref{Num}{21}{11}{21:11}
\crossref{Num}{21}{12}{De 2:13,\allowbreak14}
\crossref{Num}{21}{13}{21:14; 22:36 De 2:24 Jud 11:18 Isa 16:2 Jer 48:20}
\crossref{Num}{21}{14}{Jos 10:13 2Sa 1:18}
\crossref{Num}{21}{15}{21:28 De 2:9,\allowbreak18,\allowbreak29 Isa 15:1}
\crossref{Num}{21}{16}{Jud 9:21}
\crossref{Num}{21}{17}{Ex 15:1,\allowbreak2 Jud 5:1 Ps 105:2; 106:12 Isa 12:1,\allowbreak2,\allowbreak5 Jas 5:13}
\crossref{Num}{21}{18}{2Ch 17:7-\allowbreak9 Ne 3:1,\allowbreak5 1Ti 6:17,\allowbreak18}
\crossref{Num}{21}{19}{}
\crossref{Num}{21}{20}{Nu 22:1; 26:63; 33:49,\allowbreak50 De 1:5}
\crossref{Num}{21}{21}{Nu 20:14-\allowbreak19 De 2:26-\allowbreak28 Jud 11:19-\allowbreak21}
\crossref{Num}{21}{22}{Nu 20:17}
\crossref{Num}{21}{23}{De 2:30-\allowbreak32; 29:7,\allowbreak8}
\crossref{Num}{21}{24}{Nu 32:1-\allowbreak4,\allowbreak33-\allowbreak42 De 2:31-\allowbreak37; 29:7 Jos 9:10; 12:1-\allowbreak3; 13:8-\allowbreak10; 24:8}
\crossref{Num}{21}{25}{21:31; 32:33-\allowbreak42 De 2:12}
\crossref{Num}{21}{26}{}
\crossref{Num}{21}{27}{21:14 Isa 14:4 Hab 2:6}
\crossref{Num}{21}{28}{Jud 9:20 Isa 10:16 Jer 48:45,\allowbreak46 Am 1:4,\allowbreak7,\allowbreak10,\allowbreak12,\allowbreak14; 2:2,\allowbreak5}
\crossref{Num}{21}{29}{Jud 11:24 1Ki 11:7,\allowbreak33 2Ki 23:13 Jer 48:7,\allowbreak13,\allowbreak46 1Co 8:4,\allowbreak5}
\crossref{Num}{21}{30}{Ge 49:23 2Sa 11:24 Ps 18:14}
\crossref{Num}{21}{31}{Nu 32:33-\allowbreak42 De 3:16,\allowbreak17 Jos 12:1-\allowbreak6; 13:8-\allowbreak12}
\crossref{Num}{21}{32}{Nu 32:1,\allowbreak35 Isa 16:8,\allowbreak9 Jer 48:32}
\crossref{Num}{21}{33}{De 3:1-\allowbreak6; 29:7 Jos 13:12}
\crossref{Num}{21}{34}{Nu 14:9 De 3:2,\allowbreak11; 20:3; 31:6 Jos 10:8,\allowbreak25 Isa 41:13}
\crossref{Num}{21}{35}{De 3:3-\allowbreak17; 29:7,\allowbreak8 Jos 12:4-\allowbreak6; 13:12 Ps 135:10-\allowbreak12; 136:17-\allowbreak21}
\crossref{Num}{22}{1}{Nu 21:20; 33:48-\allowbreak50; 36:13 De 34:1,\allowbreak8}
\crossref{Num}{22}{2}{Nu 21:3,\allowbreak20-\allowbreak35 Jud 11:25}
\crossref{Num}{22}{3}{Ex 15:15 De 2:25 Jos 2:10,\allowbreak11,\allowbreak24; 9:24 Ps 53:5 Isa 23:5}
\crossref{Num}{22}{4}{22:7; 25:15-\allowbreak18; 31:8 Jos 13:21,\allowbreak22}
\crossref{Num}{22}{5}{De 23:4 Jos 13:22; 24:9 Ne 13:1,\allowbreak2 Mic 6:5 2Pe 2:15,\allowbreak16}
\crossref{Num}{22}{6}{Nu 23:7,\allowbreak8; 24:9 Ge 12:3; 27:29 De 23:4 Jos 24:9 1Sa 17:43 Ne 13:2}
\crossref{Num}{22}{7}{1Sa 9:7,\allowbreak8 Isa 56:11 Eze 13:19 Mic 3:11 Ro 16:18 1Ti 6:9,\allowbreak10}
\crossref{Num}{22}{8}{22:19,\allowbreak20; 12:6; 23:12 Jer 12:2 Eze 33:31}
\crossref{Num}{22}{9}{22:20 Ge 20:3; 31:24; 41:25 Da 2:45; 4:31,\allowbreak32 Mt 7:22; 24:24}
\crossref{Num}{22}{10}{22:4-\allowbreak6}
\crossref{Num}{22}{11}{}
\crossref{Num}{22}{12}{22:20 Job 33:15-\allowbreak17 Mt 27:19}
\crossref{Num}{22}{13}{22:14 De 23:5}
\crossref{Num}{22}{14}{22:13,\allowbreak37}
\crossref{Num}{22}{15}{22:7,\allowbreak8 Ac 10:7,\allowbreak8}
\crossref{Num}{22}{16}{22:16}
\crossref{Num}{22}{17}{Nu 24:11 De 16:9 Es 5:11; 7:9 Mt 4:8,\allowbreak9; 16:26}
\crossref{Num}{22}{18}{Nu 24:13 Tit 1:16}
\crossref{Num}{22}{19}{22:7,\allowbreak8 1Ti 6:9,\allowbreak10 2Pe 2:3,\allowbreak15 Jude 1:11}
\crossref{Num}{22}{20}{22:9}
\crossref{Num}{22}{21}{Pr 1:15,\allowbreak16}
\crossref{Num}{22}{22}{2Ki 10:20 Ho 1:4}
\crossref{Num}{22}{23}{2Ki 6:17 1Ch 21:16 Da 10:7 Ac 22:9 1Co 1:27-\allowbreak29 2Pe 2:16}
\crossref{Num}{22}{24}{}
\crossref{Num}{22}{25}{Job 5:13-\allowbreak15 Isa 47:12}
\crossref{Num}{22}{26}{Isa 26:11 Ho 2:6}
\crossref{Num}{22}{27}{Pr 14:16; 27:3,\allowbreak4}
\crossref{Num}{22}{28}{Ro 8:22}
\crossref{Num}{22}{29}{Pr 12:10,\allowbreak16 Ec 9:3}
\crossref{Num}{22}{30}{2Pe 2:16}
\crossref{Num}{22}{31}{Nu 24:4}
\crossref{Num}{22}{32}{22:28 De 25:4 Ps 36:6; 145:9; 147:9 Jon 4:11}
\crossref{Num}{22}{33}{Nu 14:37; 16:33-\allowbreak35 1Ki 13:24-\allowbreak28}
\crossref{Num}{22}{34}{Ex 9:27; 10:16,\allowbreak17 1Sa 15:24,\allowbreak30; 24:17; 26:21 2Sa 12:13}
\crossref{Num}{22}{35}{22:20 Ps 81:12 Isa 37:26-\allowbreak29 2Th 2:9-\allowbreak12}
\crossref{Num}{22}{36}{Ge 14:17; 18:2; 46:29 Ex 18:7 1Sa 13:10 Ac 28:15}
\crossref{Num}{22}{37}{22:16,\allowbreak17; 24:11 Ps 75:6 Mt 4:8,\allowbreak9 Lu 4:6 Joh 5:44}
\crossref{Num}{22}{38}{22:18 Ps 33:10; 76:10 Pr 19:21 Isa 44:25; 46:10; 47:12}
\crossref{Num}{22}{39}{}
\crossref{Num}{22}{40}{Nu 23:2,\allowbreak14,\allowbreak30 Ge 31:54 Pr 1:16}
\crossref{Num}{22}{41}{Nu 23:13}
\crossref{Num}{23}{1}{23:29 Eze 33:31 Jude 1:11}
\crossref{Num}{23}{2}{23:14,\allowbreak30}
\crossref{Num}{23}{3}{23:15}
\crossref{Num}{23}{4}{23:16; 22:9,\allowbreak20}
\crossref{Num}{23}{5}{23:16; 22:35 De 18:18 Pr 16:1,\allowbreak9 Isa 51:16; 59:21 Jer 1:9}
\crossref{Num}{23}{6}{}
\crossref{Num}{23}{7}{23:18; 24:3,\allowbreak15,\allowbreak23 Job 27:1; 29:1 Ps 78:2 Eze 17:2; 20:49 Mic 2:4}
\crossref{Num}{23}{8}{23:20,\allowbreak23 Isa 44:25; 47:12,\allowbreak13}
\crossref{Num}{23}{9}{Ex 19:5,\allowbreak6; 33:16 De 33:28 Es 3:8 2Co 6:17 Tit 2:14 1Pe 2:9}
\crossref{Num}{23}{10}{Ge 13:16; 22:17; 28:14}
\crossref{Num}{23}{11}{23:7,\allowbreak8; 22:11,\allowbreak17; 24:10 Ps 109:17-\allowbreak20}
\crossref{Num}{23}{12}{23:20,\allowbreak26; 22:38; 24:13 Pr 26:25 Ro 16:18 Tit 1:16}
\crossref{Num}{23}{13}{1Ki 20:23,\allowbreak28 Mic 6:5}
\crossref{Num}{23}{14}{Nu 21:20 De 3:27}
\crossref{Num}{23}{15}{23:3; 22:8}
\crossref{Num}{23}{16}{23:5; 22:35; 24:1}
\crossref{Num}{23}{17}{23:26 1Sa 3:17 Jer 37:17}
\crossref{Num}{23}{18}{Jud 3:20}
\crossref{Num}{23}{19}{1Sa 15:29 Ps 89:35 Hab 2:3 Mal 3:6 Lu 21:33 Ro 11:29}
\crossref{Num}{23}{20}{Nu 22:12 Ge 12:2; 22:17}
\crossref{Num}{23}{21}{Ps 103:12 Isa 1:18; 38:17 Jer 50:20 Ho 14:2-\allowbreak4 Mic 7:18-\allowbreak20}
\crossref{Num}{23}{22}{Nu 22:5; 24:8 Ex 9:16; 14:18; 20:2 Ps 68:35}
\crossref{Num}{23}{23}{Nu 22:6; 24:1 Ge 3:15 Mt 12:25,\allowbreak27; 16:18 Lu 10:18,\allowbreak19 Ro 16:20}
\crossref{Num}{23}{24}{Nu 24:8,\allowbreak9 Ge 49:9 De 33:20 Ps 17:12 Pr 30:30 Isa 31:4 Am 3:8}
\crossref{Num}{23}{25}{Ps 2:1-\allowbreak3}
\crossref{Num}{23}{26}{23:12,\allowbreak13; 22:18,\allowbreak38; 24:12,\allowbreak13 1Ki 22:14 2Ch 18:13 Ac 4:19,\allowbreak20; 5:29}
\crossref{Num}{23}{27}{23:13}
\crossref{Num}{23}{28}{Nu 21:20}
\crossref{Num}{23}{29}{23:1,\allowbreak2}
\crossref{Num}{23}{30}{23:30}
\crossref{Num}{24}{1}{Nu 22:13; 23:20; 31:16 1Sa 24:20; 26:2,\allowbreak25 Re 2:14}
\crossref{Num}{24}{2}{24:5; 2:2-\allowbreak34; 23:9,\allowbreak10 So 6:4,\allowbreak10}
\crossref{Num}{24}{3}{Nu 23:7,\allowbreak18}
\crossref{Num}{24}{4}{Nu 12:6 Ge 15:12 Ps 89:19 Da 8:26,\allowbreak27 Ac 10:10,\allowbreak19; 22:17}
\crossref{Num}{24}{5}{}
\crossref{Num}{24}{6}{Ge 2:8-\allowbreak10; 13:10 So 4:12-\allowbreak15; 6:11 Isa 58:11 Jer 31:12 Joe 3:18}
\crossref{Num}{24}{7}{Ps 68:26 Pr 5:16-\allowbreak18 Isa 48:1}
\crossref{Num}{24}{8}{Nu 21:5; 23:22}
\crossref{Num}{24}{9}{Ge 49:9 Job 38:39,\allowbreak40}
\crossref{Num}{24}{10}{Job 27:23 Eze 21:14,\allowbreak17; 22:13}
\crossref{Num}{24}{11}{Nu 22:17,\allowbreak37}
\crossref{Num}{24}{12}{Nu 22:18,\allowbreak38}
\crossref{Num}{24}{13}{}
\crossref{Num}{24}{14}{24:17; 31:7-\allowbreak18 Mic 6:5 Re 2:10,\allowbreak14}
\crossref{Num}{24}{15}{24:3,\allowbreak4; 23:7,\allowbreak18 Job 27:1 Mt 13:35}
\crossref{Num}{24}{16}{24:4 2Sa 23:1,\allowbreak2 1Co 8:1; 13:2}
\crossref{Num}{24}{17}{Mt 2:2-\allowbreak9 Lu 1:78 2Pe 1:19 Re 22:16}
\crossref{Num}{24}{18}{Ge 27:29,\allowbreak40 2Sa 8:14 Ps 60:1}
\crossref{Num}{24}{19}{Ge 49:10 Ps 2:1-\allowbreak12; 72:10,\allowbreak11 Isa 11:10 Mic 5:2,\allowbreak4 Mt 28:18}
\crossref{Num}{24}{20}{Jud 6:3 1Sa 14:48; 15:3-\allowbreak8; 27:8,\allowbreak9; 30:1,\allowbreak17 1Ch 4:43 Es 3:1}
\crossref{Num}{24}{21}{Ge 15:19 Jud 1:16 Job 29:18}
\crossref{Num}{24}{22}{}
\crossref{Num}{24}{23}{Nu 23:23 2Ki 5:1 Mal 3:2}
\crossref{Num}{24}{24}{Ge 10:4 Isa 23:1 Da 7:19,\allowbreak20; 8:5-\allowbreak8,\allowbreak21; 10:20; 11:30}
\crossref{Num}{24}{25}{24:11; 31:8 Jos 13:22}
\crossref{Num}{25}{1}{Nu 33:49 Jos 2:1; 3:1 Mic 6:5}
\crossref{Num}{25}{2}{Ex 34:15,\allowbreak16 Jos 22:17 1Ki 11:1-\allowbreak8 Ps 106:28 Ho 9:10 1Co 10:20}
\crossref{Num}{25}{3}{25:5 De 4:3,\allowbreak4 Jos 22:17 Ps 106:28,\allowbreak29 Ho 9:10}
\crossref{Num}{25}{4}{25:14,\allowbreak15,\allowbreak18 Ex 18:25 De 4:3 Jos 22:17; 23:2}
\crossref{Num}{25}{5}{Ex 18:21,\allowbreak25,\allowbreak26}
\crossref{Num}{25}{6}{25:14,\allowbreak15; 22:4; 31:2,\allowbreak9-\allowbreak16}
\crossref{Num}{25}{7}{Ex 6:25 Jos 22:30,\allowbreak31 Jud 20:28}
\crossref{Num}{25}{8}{25:5,\allowbreak11 Ps 106:29-\allowbreak31}
\crossref{Num}{25}{9}{}
\crossref{Num}{25}{10}{}
\crossref{Num}{25}{11}{Jos 7:25,\allowbreak26 2Sa 21:14 Ps 106:23 Joh 3:36}
\crossref{Num}{25}{12}{Nu 13:29 Mal 2:4,\allowbreak5; 3:1}
\crossref{Num}{25}{13}{1Sa 2:30 1Ki 2:27 1Ch 6:4-\allowbreak15,\allowbreak50-\allowbreak53}
\crossref{Num}{25}{14}{25:4,\allowbreak5 2Ch 19:7}
\crossref{Num}{25}{15}{Nu 31:8 Jos 13:21}
\crossref{Num}{25}{16}{25:16}
\crossref{Num}{25}{17}{}
\crossref{Num}{25}{18}{Nu 31:15,\allowbreak16 Ge 26:10 Ex 32:21,\allowbreak35 Re 2:14}
\crossref{Num}{26}{1}{Nu 25:9}
\crossref{Num}{26}{2}{Nu 1:2,\allowbreak3 Ex 30:12; 38:25,\allowbreak26}
\crossref{Num}{26}{3}{26:63; 22:1; 31:12; 33:48; 35:1 De 4:46-\allowbreak49; 34:1,\allowbreak6,\allowbreak8}
\crossref{Num}{26}{4}{Nu 1:1 1Ch 21:1}
\crossref{Num}{26}{5}{Ge 29:32; 49:2,\allowbreak3 1Ch 5:1}
\crossref{Num}{26}{6}{26:6}
\crossref{Num}{26}{7}{26:1,\allowbreak21; 2:11 Ge 46:9}
\crossref{Num}{26}{8}{26:8}
\crossref{Num}{26}{9}{Nu 1:16; 16:1,\allowbreak2-\allowbreak35 Ps 106:17 Jude 1:11}
\crossref{Num}{26}{10}{Nu 16:2,\allowbreak31-\allowbreak35,\allowbreak38; 27:3 Ex 16:35 Ps 106:17,\allowbreak18}
\crossref{Num}{26}{11}{}
\crossref{Num}{26}{12}{Ge 46:10 Ex 6:15}
\crossref{Num}{26}{13}{Ge 46:10}
\crossref{Num}{26}{14}{}
\crossref{Num}{26}{15}{Nu 2:14 Ge 46:16}
\crossref{Num}{26}{16}{Ge 46:16}
\crossref{Num}{26}{17}{Ge 46:16}
\crossref{Num}{26}{18}{Nu 1:24,\allowbreak25; 2:14,\allowbreak15}
\crossref{Num}{26}{19}{Ge 38:1-\allowbreak10; 46:12 1Ch 2:3-\allowbreak8}
\crossref{Num}{26}{20}{Ge 38:5,\allowbreak11,\allowbreak14,\allowbreak26-\allowbreak30 1Ch 4:21}
\crossref{Num}{26}{21}{26:21}
\crossref{Num}{26}{22}{Nu 1:26,\allowbreak27; 2:3,\allowbreak4 Ge 49:8 1Ch 5:2 Ps 115:14 Heb 7:14}
\crossref{Num}{26}{23}{Nu 2:5 Ge 30:17,\allowbreak18; 46:13 1Ch 7:1}
\crossref{Num}{26}{24}{Ge 46:13}
\crossref{Num}{26}{25}{Nu 1:28,\allowbreak29; 2:5,\allowbreak6}
\crossref{Num}{26}{26}{Ge 30:19,\allowbreak20; 46:14}
\crossref{Num}{26}{27}{Nu 1:30,\allowbreak31; 2:7,\allowbreak8}
\crossref{Num}{26}{28}{Ge 41:51,\allowbreak52; 46:20; 48:5,\allowbreak13-\allowbreak20}
\crossref{Num}{26}{29}{Nu 32:39,\allowbreak40; 36:1 Ge 48:14 De 3:15 Jos 17:1 Jud 5:14 1Ch 7:14-\allowbreak19}
\crossref{Num}{26}{30}{Jos 17:2 Jud 6:11,\allowbreak24,\allowbreak34; 8:2}
\crossref{Num}{26}{31}{26:31}
\crossref{Num}{26}{32}{}
\crossref{Num}{26}{33}{Nu 27:1; 36:10-\allowbreak12}
\crossref{Num}{26}{34}{Nu 1:34,\allowbreak35; 2:20,\allowbreak21}
\crossref{Num}{26}{35}{1Ch 7:20,\allowbreak21}
\crossref{Num}{26}{36}{26:36}
\crossref{Num}{26}{37}{Nu 1:32,\allowbreak33; 2:18,\allowbreak19}
\crossref{Num}{26}{38}{1Ch 7:6-\allowbreak12}
\crossref{Num}{26}{39}{Ge 46:21}
\crossref{Num}{26}{40}{1Ch 8:3}
\crossref{Num}{26}{41}{Nu 1:36,\allowbreak37; 2:22,\allowbreak23 Ge 46:21}
\crossref{Num}{26}{42}{Ge 46:23}
\crossref{Num}{26}{43}{Nu 1:38,\allowbreak39; 2:25,\allowbreak26}
\crossref{Num}{26}{44}{Ge 46:17}
\crossref{Num}{26}{45}{}
\crossref{Num}{26}{46}{Ge 46:17}
\crossref{Num}{26}{47}{Nu 1:40,\allowbreak41; 2:27,\allowbreak28}
\crossref{Num}{26}{48}{Ge 46:24 1Ch 7:13}
\crossref{Num}{26}{49}{1Ch 7:13}
\crossref{Num}{26}{50}{Nu 1:42,\allowbreak43; 2:29,\allowbreak30}
\crossref{Num}{26}{51}{}
\crossref{Num}{26}{52}{26:52}
\crossref{Num}{26}{53}{Ge 12:2,\allowbreak7 Jos 11:23; 14:1 Ps 49:14; 105:44 Eze 47:22 Da 7:27}
\crossref{Num}{26}{54}{Nu 32:3,\allowbreak5; 33:54 Jos 17:14}
\crossref{Num}{26}{55}{26:56; 33:54; 34:13 Jos 11:23; 14:2; 17:14; 18:6,\allowbreak10,\allowbreak11; 19:1,\allowbreak10}
\crossref{Num}{26}{56}{}
\crossref{Num}{26}{57}{Nu 35:2,\allowbreak3 Ge 46:11 Ex 6:16-\allowbreak19 1Ch 6:1,\allowbreak16-\allowbreak30}
\crossref{Num}{26}{58}{Nu 3:17-\allowbreak21; 16:1}
\crossref{Num}{26}{59}{Ex 2:1,\allowbreak2; 6:20 Le 18:12}
\crossref{Num}{26}{60}{Nu 3:2,\allowbreak8}
\crossref{Num}{26}{61}{Nu 3:4 Le 10:1,\allowbreak2 1Ch 24:1,\allowbreak2}
\crossref{Num}{26}{62}{Nu 1:49; 3:39; 4:27,\allowbreak48; 18:20-\allowbreak24; 35:2-\allowbreak8 De 10:9; 14:27-\allowbreak29; 18:1,\allowbreak2}
\crossref{Num}{26}{63}{26:3}
\crossref{Num}{26}{64}{Nu 1:1-\allowbreak2:34 De 2:14,\allowbreak15; 4:3,\allowbreak4 1Co 10:5}
\crossref{Num}{26}{65}{Nu 14:23,\allowbreak24,\allowbreak28-\allowbreak30,\allowbreak35,\allowbreak38 Ex 12:37 De 2:14,\allowbreak15; 32:49,\allowbreak50 Ps 90:3-\allowbreak7}
\crossref{Num}{27}{1}{Nu 26:33; 36:1-\allowbreak12 Jos 17:3-\allowbreak6 1Ch 7:15 Ga 3:28}
\crossref{Num}{27}{2}{Nu 15:33,\allowbreak34 Ex 18:13,\allowbreak14,\allowbreak19-\allowbreak26 De 17:8-\allowbreak10}
\crossref{Num}{27}{3}{Nu 14:35; 26:64,\allowbreak65}
\crossref{Num}{27}{4}{Ex 32:11 Ps 109:13 Pr 13:9}
\crossref{Num}{27}{5}{Nu 15:34 Ex 18:15-\allowbreak19; 25:22 Le 24:12,\allowbreak13 Job 23:4 Pr 3:5,\allowbreak6}
\crossref{Num}{27}{6}{Ps 68:5,\allowbreak6 Ga 3:28}
\crossref{Num}{27}{7}{Nu 36:1,\allowbreak2 Ps 68:5 Jer 49:11 Ga 3:28}
\crossref{Num}{27}{8}{27:8}
\crossref{Num}{27}{9}{27:9}
\crossref{Num}{27}{10}{}
\crossref{Num}{27}{11}{Le 25:25,\allowbreak49 Ru 4:3-\allowbreak6 Jer 32:8}
\crossref{Num}{27}{12}{Nu 33:47,\allowbreak48 De 3:27; 32:49; 34:1-\allowbreak4}
\crossref{Num}{27}{13}{Nu 31:2}
\crossref{Num}{27}{14}{Nu 20:8-\allowbreak13 De 1:37; 32:51,\allowbreak52 Ps 106:32,\allowbreak33}
\crossref{Num}{27}{15}{}
\crossref{Num}{27}{16}{Nu 16:22 Heb 12:9}
\crossref{Num}{27}{17}{De 31:2 1Sa 8:20; 18:13 2Sa 5:2 1Ki 3:7 2Ch 1:10 Joh 10:3,\allowbreak4,\allowbreak9}
\crossref{Num}{27}{18}{Nu 11:28; 13:8,\allowbreak16 Ex 17:9 De 3:28; 31:7,\allowbreak8,\allowbreak23; 34:9}
\crossref{Num}{27}{19}{De 31:7 Lu 9:1-\allowbreak5; 10:2-\allowbreak11 Ac 20:28-\allowbreak31 Col 4:17 1Ti 5:21}
\crossref{Num}{27}{20}{Nu 11:17,\allowbreak28,\allowbreak29 1Sa 10:6,\allowbreak9 2Ki 2:9,\allowbreak10,\allowbreak15 1Ch 29:23,\allowbreak25}
\crossref{Num}{27}{21}{Jos 9:14 Jud 1:1; 20:18,\allowbreak23,\allowbreak26-\allowbreak28 1Sa 22:10; 23:9; 28:6; 30:7}
\crossref{Num}{27}{22}{}
\crossref{Num}{27}{23}{27:19 De 3:28; 31:7,\allowbreak8}
\crossref{Num}{28}{1}{28:1}
\crossref{Num}{28}{2}{Le 3:11; 21:6,\allowbreak8 Mal 1:7,\allowbreak12}
\crossref{Num}{28}{3}{Ex 29:38,\allowbreak39 Le 6:9 Eze 46:13-\allowbreak15 Joh 1:29 1Pe 1:19,\allowbreak20 Re 13:8}
\crossref{Num}{28}{4}{1Ki 18:29,\allowbreak36 Ezr 9:4,\allowbreak5 Ps 141:2 Da 9:21}
\crossref{Num}{28}{5}{Nu 15:4,\allowbreak5 Ex 16:36; 29:38-\allowbreak42 Le 2:1}
\crossref{Num}{28}{6}{Ex 29:42 Le 6:9 2Ch 2:4; 31:3 Ezr 3:4 Ps 50:8 Eze 46:14}
\crossref{Num}{28}{7}{Ex 29:42}
\crossref{Num}{28}{8}{28:8}
\crossref{Num}{28}{9}{Ex 20:8-\allowbreak11 Ps 92:1-\allowbreak4 Isa 58:13 Eze 20:12 Re 1:10}
\crossref{Num}{28}{10}{Eze 46:4,\allowbreak5}
\crossref{Num}{28}{11}{Nu 10:10; 15:3-\allowbreak11 1Sa 20:5 2Ki 4:23 1Ch 23:31 2Ch 2:4 Ezr 3:5}
\crossref{Num}{28}{12}{Nu 15:4-\allowbreak12; 29:10 Eze 46:5-\allowbreak7}
\crossref{Num}{28}{13}{28:2}
\crossref{Num}{28}{14}{}
\crossref{Num}{28}{15}{28:22; 15:24 Le 4:23; 16:15 Ro 8:3 2Co 5:21}
\crossref{Num}{28}{16}{Nu 9:3-\allowbreak5 Ex 12:2-\allowbreak11,\allowbreak18,\allowbreak43-\allowbreak49 Le 23:5-\allowbreak8 De 16:1-\allowbreak8 Eze 45:21-\allowbreak24}
\crossref{Num}{28}{17}{Ex 12:15-\allowbreak17; 13:6 Le 23:6}
\crossref{Num}{28}{18}{Ex 12:16 Le 23:7,\allowbreak8}
\crossref{Num}{28}{19}{Eze 45:21-\allowbreak25}
\crossref{Num}{28}{20}{28:20}
\crossref{Num}{28}{21}{}
\crossref{Num}{28}{22}{28:15}
\crossref{Num}{28}{23}{28:3,\allowbreak10}
\crossref{Num}{28}{24}{}
\crossref{Num}{28}{25}{Ex 12:16; 13:6 Le 23:8}
\crossref{Num}{28}{26}{Ex 23:16; 34:22 Le 23:10,\allowbreak15-\allowbreak21 De 16:9-\allowbreak11 Ac 2:1-\allowbreak13 1Co 15:20}
\crossref{Num}{28}{27}{28:11,\allowbreak19 Le 23:18,\allowbreak19}
\crossref{Num}{28}{28}{28:28}
\crossref{Num}{28}{29}{28:29}
\crossref{Num}{28}{30}{28:15,\allowbreak22; 15:24 2Co 5:21 Ga 3:13 1Pe 2:24; 3:18}
\crossref{Num}{28}{31}{28:19 Mal 1:13,\allowbreak14}
\crossref{Num}{29}{1}{}
\crossref{Num}{29}{2}{29:8,\allowbreak36; 28:19,\allowbreak27 Heb 10:10-\allowbreak14}
\crossref{Num}{29}{3}{29:3}
\crossref{Num}{29}{4}{}
\crossref{Num}{29}{5}{Nu 28:15,\allowbreak22,\allowbreak30}
\crossref{Num}{29}{6}{Nu 28:11-\allowbreak15}
\crossref{Num}{29}{7}{Le 16:29-\allowbreak31; 23:27}
\crossref{Num}{29}{8}{29:2,\allowbreak13; 28:19}
\crossref{Num}{29}{9}{Nu 15:3-\allowbreak12}
\crossref{Num}{29}{10}{}
\crossref{Num}{29}{11}{Le 16:3,\allowbreak5,\allowbreak9 Isa 53:10 Da 9:24-\allowbreak26 Heb 7:27; 9:25-\allowbreak28}
\crossref{Num}{29}{12}{}
\crossref{Num}{29}{13}{29:2,\allowbreak8; 28:11,\allowbreak19,\allowbreak27 Ezr 3:4 Heb 10:12-\allowbreak14}
\crossref{Num}{29}{14}{29:14}
\crossref{Num}{29}{15}{}
\crossref{Num}{29}{16}{29:11}
\crossref{Num}{29}{17}{29:13,\allowbreak20-\allowbreak40 Ps 40:6; 50:8,\allowbreak9; 51:16,\allowbreak17; 69:31 Isa 1:11 Jer 7:22,\allowbreak23}
\crossref{Num}{29}{18}{}
\crossref{Num}{29}{19}{29:11,\allowbreak22,\allowbreak25 Am 8:14}
\crossref{Num}{29}{20}{}
\crossref{Num}{29}{21}{29:18}
\crossref{Num}{29}{22}{Ps 16:4 Joe 1:9,\allowbreak13; 2:14}
\crossref{Num}{29}{23}{29:23}
\crossref{Num}{29}{24}{}
\crossref{Num}{29}{25}{29:11 Joh 8:31 Ac 13:43 Ro 2:7 Ga 2:5; 6:9 2Th 3:13 Heb 3:14}
\crossref{Num}{29}{26}{29:26}
\crossref{Num}{29}{27}{29:27}
\crossref{Num}{29}{28}{29:28}
\crossref{Num}{29}{29}{29:29}
\crossref{Num}{29}{30}{29:30}
\crossref{Num}{29}{31}{29:31}
\crossref{Num}{29}{32}{29:32}
\crossref{Num}{29}{33}{29:33}
\crossref{Num}{29}{34}{}
\crossref{Num}{29}{35}{}
\crossref{Num}{29}{36}{29:36}
\crossref{Num}{29}{37}{29:37}
\crossref{Num}{29}{38}{}
\crossref{Num}{29}{39}{Nu 6:21 Le 7:11,\allowbreak16-\allowbreak38; 22:21-\allowbreak23; 23:28 De 12:6 1Co 10:31}
\crossref{Num}{29}{40}{Ex 40:16 De 4:5 Mt 28:20 Ac 20:27 1Co 15:3 Heb 3:2,\allowbreak5}
\crossref{Num}{30}{1}{Nu 1:4-\allowbreak16; 7:2; 34:17-\allowbreak28 Ex 18:25 De 1:13-\allowbreak17}
\crossref{Num}{30}{2}{Nu 21:2 Ge 28:20-\allowbreak22 Le 27:2-\allowbreak34 De 23:21,\allowbreak22 Jud 11:11,\allowbreak30,\allowbreak31,\allowbreak35,\allowbreak36}
\crossref{Num}{30}{3}{}
\crossref{Num}{30}{4}{30:2}
\crossref{Num}{30}{5}{Ho 6:6 Mt 15:4-\allowbreak6 Mr 7:10-\allowbreak13 Eph 6:1}
\crossref{Num}{30}{6}{Ps 56:12}
\crossref{Num}{30}{7}{30:7}
\crossref{Num}{30}{8}{Ge 3:16 1Co 7:4; 14:34 Eph 5:22-\allowbreak24}
\crossref{Num}{30}{9}{Le 21:7 Lu 2:37 Ro 7:2}
\crossref{Num}{30}{10}{30:10}
\crossref{Num}{30}{11}{}
\crossref{Num}{30}{12}{1Co 11:3}
\crossref{Num}{30}{13}{1Co 11:3,\allowbreak9 1Pe 3:1-\allowbreak6}
\crossref{Num}{30}{14}{}
\crossref{Num}{30}{15}{30:5,\allowbreak8,\allowbreak12 Le 5:1 Ga 3:28}
\crossref{Num}{30}{16}{Nu 5:29,\allowbreak30 Le 11:46,\allowbreak47; 13:59; 14:54-\allowbreak57; 15:32,\allowbreak33}
\crossref{Num}{31}{1}{31:1}
\crossref{Num}{31}{2}{31:3; 25:17,\allowbreak18 De 32:35 Jud 16:24,\allowbreak28-\allowbreak30 Ps 94:1-\allowbreak3 Isa 1:24}
\crossref{Num}{31}{3}{Ex 17:9-\allowbreak13}
\crossref{Num}{31}{4}{}
\crossref{Num}{31}{5}{}
\crossref{Num}{31}{6}{Nu 25:7-\allowbreak13}
\crossref{Num}{31}{7}{De 20:13,\allowbreak14 Jud 21:11 1Sa 27:9 1Ki 11:15,\allowbreak16}
\crossref{Num}{31}{8}{Nu 22:4 Jos 13:21,\allowbreak22}
\crossref{Num}{31}{9}{31:15,\allowbreak16 De 20:14 2Ch 28:5,\allowbreak8-\allowbreak10}
\crossref{Num}{31}{10}{Jos 6:24 1Sa 30:1 1Ki 9:16 Isa 1:7 Re 18:8}
\crossref{Num}{31}{11}{De 20:14 Jos 8:2}
\crossref{Num}{31}{12}{Nu 22:1}
\crossref{Num}{31}{13}{Ge 14:17 1Sa 15:12; 30:21}
\crossref{Num}{31}{14}{Nu 12:3 Ex 32:19,\allowbreak22 Le 10:16 1Sa 15:13,\allowbreak14 1Ki 20:42 2Ki 13:19}
\crossref{Num}{31}{15}{De 2:34; 20:13,\allowbreak16-\allowbreak18 Jos 6:21; 8:25; 10:40; 11:14 1Sa 15:3}
\crossref{Num}{31}{16}{Nu 24:14; 25:1-\allowbreak3 Pr 23:27 Ec 7:26 2Pe 2:15 Re 2:14}
\crossref{Num}{31}{17}{}
\crossref{Num}{31}{18}{}
\crossref{Num}{31}{19}{}
\crossref{Num}{31}{20}{Nu 19:14-\allowbreak16,\allowbreak22 Ge 35:2 Ex 19:10}
\crossref{Num}{31}{21}{Nu 30:16}
\crossref{Num}{31}{22}{}
\crossref{Num}{31}{23}{Isa 43:2 Zec 13:9 Mal 2:2,\allowbreak3 Mt 3:11 1Co 3:13 1Pe 1:7; 4:12}
\crossref{Num}{31}{24}{Nu 19:19 Le 11:25; 14:9; 15:13}
\crossref{Num}{31}{25}{}
\crossref{Num}{31}{26}{31:26}
\crossref{Num}{31}{27}{Jos 22:8 1Sa 30:4,\allowbreak24,\allowbreak25 Ps 68:12}
\crossref{Num}{31}{28}{Ge 14:20 Jos 6:19,\allowbreak24 2Sa 8:11,\allowbreak12 1Ch 18:11; 26:26,\allowbreak27 Pr 3:9,\allowbreak10}
\crossref{Num}{31}{29}{Nu 18:26 Ex 29:27 De 12:12,\allowbreak19}
\crossref{Num}{31}{30}{31:42-\allowbreak47}
\crossref{Num}{31}{31}{}
\crossref{Num}{31}{32}{}
\crossref{Num}{31}{33}{31:33}
\crossref{Num}{31}{34}{31:34}
\crossref{Num}{31}{35}{31:35}
\crossref{Num}{31}{36}{31:36}
\crossref{Num}{31}{37}{31:37}
\crossref{Num}{31}{38}{31:38}
\crossref{Num}{31}{39}{31:39}
\crossref{Num}{31}{40}{}
\crossref{Num}{31}{41}{31:29-\allowbreak31; 18:8,\allowbreak19 Mt 10:10 1Co 9:10-\allowbreak14 Ga 6:6 1Ti 5:17}
\crossref{Num}{31}{42}{31:42}
\crossref{Num}{31}{43}{31:43}
\crossref{Num}{31}{44}{31:44}
\crossref{Num}{31}{45}{31:45}
\crossref{Num}{31}{46}{}
\crossref{Num}{31}{47}{Nu 18:21-\allowbreak24 De 12:17-\allowbreak19 Lu 10:1-\allowbreak8 1Th 5:12,\allowbreak13}
\crossref{Num}{31}{48}{}
\crossref{Num}{31}{49}{1Sa 30:18,\allowbreak19 Ps 72:14 Joh 18:9}
\crossref{Num}{31}{50}{Ps 107:15,\allowbreak21,\allowbreak22; 116:12,\allowbreak17}
\crossref{Num}{31}{51}{Nu 7:2-\allowbreak6}
\crossref{Num}{31}{52}{31:52}
\crossref{Num}{31}{53}{De 20:14}
\crossref{Num}{31}{54}{Nu 16:40 Ex 30:16 Jos 4:7 Ps 18:49; 103:1,\allowbreak2; 115:1; 145:7 Zec 6:14}
\crossref{Num}{32}{1}{Nu 2:10-\allowbreak15; 26:5-\allowbreak7,\allowbreak15-\allowbreak18 Ge 29:32; 30:10,\allowbreak11}
\crossref{Num}{32}{2}{}
\crossref{Num}{32}{3}{32:1,\allowbreak34-\allowbreak38 Jos 13:17 Isa 15:2-\allowbreak4 Jer 48:22,\allowbreak23}
\crossref{Num}{32}{4}{Nu 21:24,\allowbreak34 De 2:24-\allowbreak35}
\crossref{Num}{32}{5}{Ge 19:19 Ru 2:10 1Sa 20:3 2Sa 14:22 Es 5:2 Jer 31:2}
\crossref{Num}{32}{6}{2Sa 11:11 1Co 13:5 Php 2:4}
\crossref{Num}{32}{7}{32:9; 21:4 De 1:28}
\crossref{Num}{32}{8}{Nu 13:2-\allowbreak26; 14:2 De 1:22,\allowbreak23 Jos 14:6,\allowbreak7}
\crossref{Num}{32}{9}{Nu 13:23-\allowbreak33; 14:1-\allowbreak10 De 1:24-\allowbreak28}
\crossref{Num}{32}{10}{Nu 14:11,\allowbreak21,\allowbreak23,\allowbreak29 De 1:34-\allowbreak40 Ps 95:11 Eze 20:15 Heb 3:8-\allowbreak19}
\crossref{Num}{32}{11}{Nu 14:28,\allowbreak29; 26:2,\allowbreak64,\allowbreak65 De 1:35; 2:14,\allowbreak15}
\crossref{Num}{32}{12}{Nu 14:24,\allowbreak30; 26:65 De 1:36 Jos 14:8,\allowbreak9}
\crossref{Num}{32}{13}{Nu 14:33-\allowbreak35 De 2:14 Ps 78:33}
\crossref{Num}{32}{14}{Ge 5:3; 8:21 Ne 9:24-\allowbreak26 Job 14:4 Ps 78:57 Isa 1:4; 57:4}
\crossref{Num}{32}{15}{Le 26:14-\allowbreak18 De 28:15-\allowbreak68; 30:17-\allowbreak19 Jos 22:16-\allowbreak18 2Ch 7:19-\allowbreak22}
\crossref{Num}{32}{16}{}
\crossref{Num}{32}{17}{32:29-\allowbreak32 De 3:18-\allowbreak20 Jos 4:12,\allowbreak13}
\crossref{Num}{32}{18}{Jos 22:4,\allowbreak5}
\crossref{Num}{32}{19}{Ge 13:10-\allowbreak12; 14:12 2Ki 10:32,\allowbreak33; 15:29 1Ch 5:25,\allowbreak26 Pr 20:21}
\crossref{Num}{32}{20}{De 3:18-\allowbreak20 Jos 1:13-\allowbreak15; 4:12,\allowbreak13; 22:2-\allowbreak4}
\crossref{Num}{32}{21}{}
\crossref{Num}{32}{22}{De 3:20 Jos 10:30,\allowbreak42; 11:23; 18:1 Ps 44:1-\allowbreak4; 78:55}
\crossref{Num}{32}{23}{Le 26:14-\allowbreak46 De 28:15-\allowbreak68}
\crossref{Num}{32}{24}{32:16,\allowbreak34-\allowbreak42}
\crossref{Num}{32}{25}{Jos 1:13,\allowbreak14}
\crossref{Num}{32}{26}{}
\crossref{Num}{32}{27}{Jos 4:12}
\crossref{Num}{32}{28}{Jos 1:13}
\crossref{Num}{32}{29}{32:20-\allowbreak23}
\crossref{Num}{32}{30}{Jos 22:19}
\crossref{Num}{32}{31}{32:31}
\crossref{Num}{32}{32}{}
\crossref{Num}{32}{33}{32:1 De 3:12-\allowbreak17; 29:8 Jos 12:6; 13:8-\allowbreak14; 22:4}
\crossref{Num}{32}{34}{32:3; 21:20; 33:45,\allowbreak46}
\crossref{Num}{32}{35}{32:1,\allowbreak3}
\crossref{Num}{32}{36}{32:3}
\crossref{Num}{32}{37}{32:3; 21:27 Isa 15:4}
\crossref{Num}{32}{38}{Isa 46:1}
\crossref{Num}{32}{39}{Nu 26:29 Ge 50:23 Jos 17:1}
\crossref{Num}{32}{40}{De 3:13-\allowbreak15 Jos 13:29-\allowbreak31; 17:1}
\crossref{Num}{32}{41}{De 3:14 Jos 13:30 1Ch 2:21-\allowbreak23}
\crossref{Num}{32}{42}{32:42}
\crossref{Num}{33}{1}{Ex 12:37,\allowbreak51; 13:18}
\crossref{Num}{33}{2}{Nu 9:17-\allowbreak23; 10:6,\allowbreak13 De 1:2; 10:11}
\crossref{Num}{33}{3}{Ge 47:11 Ex 1:11; 12:37}
\crossref{Num}{33}{4}{Ex 12:29,\allowbreak30 Ps 105:36}
\crossref{Num}{33}{5}{Ex 12:37}
\crossref{Num}{33}{6}{Ex 13:20}
\crossref{Num}{33}{7}{33:8 Ex 14:2,\allowbreak9}
\crossref{Num}{33}{8}{Ex 14:21,\allowbreak22-\allowbreak31; 15:22-\allowbreak26}
\crossref{Num}{33}{9}{Ex 15:27}
\crossref{Num}{33}{10}{Ex 16:1; 17:1}
\crossref{Num}{33}{11}{33:11}
\crossref{Num}{33}{12}{33:12}
\crossref{Num}{33}{13}{}
\crossref{Num}{33}{14}{Ex 17:1-\allowbreak8; 19:2}
\crossref{Num}{33}{15}{Ex 16:1; 19:1,\allowbreak2}
\crossref{Num}{33}{16}{Nu 10:11-\allowbreak13,\allowbreak33 De 1:6}
\crossref{Num}{33}{17}{Nu 11:35}
\crossref{Num}{33}{18}{Nu 12:16}
\crossref{Num}{33}{19}{}
\crossref{Num}{33}{20}{}
\crossref{Num}{33}{21}{De 1:1}
\crossref{Num}{33}{22}{33:22}
\crossref{Num}{33}{23}{}
\crossref{Num}{33}{24}{}
\crossref{Num}{33}{25}{33:25}
\crossref{Num}{33}{26}{33:26}
\crossref{Num}{33}{27}{33:27}
\crossref{Num}{33}{28}{33:28}
\crossref{Num}{33}{29}{}
\crossref{Num}{33}{30}{}
\crossref{Num}{33}{31}{Ge 36:27 De 10:6 1Ch 1:43}
\crossref{Num}{33}{32}{De 10:7}
\crossref{Num}{33}{33}{}
\crossref{Num}{33}{34}{}
\crossref{Num}{33}{35}{2Ch 20:36}
\crossref{Num}{33}{36}{Nu 13:21; 20:1; 27:14 De 32:51}
\crossref{Num}{33}{37}{Nu 20:22,\allowbreak23; 21:4}
\crossref{Num}{33}{38}{Nu 20:24-\allowbreak28 De 10:6; 32:50}
\crossref{Num}{33}{39}{}
\crossref{Num}{33}{40}{Nu 21:1-\allowbreak3,\allowbreak4-\allowbreak9}
\crossref{Num}{33}{41}{Nu 21:4}
\crossref{Num}{33}{42}{}
\crossref{Num}{33}{43}{Nu 21:10}
\crossref{Num}{33}{44}{Nu 21:11}
\crossref{Num}{33}{45}{}
\crossref{Num}{33}{46}{Nu 32:34 Isa 15:2 Jer 48:18}
\crossref{Num}{33}{47}{}
\crossref{Num}{33}{48}{Nu 22:1}
\crossref{Num}{33}{49}{}
\crossref{Num}{33}{50}{33:48,\allowbreak49}
\crossref{Num}{33}{51}{De 7:1; 9:1 Jos 3:17}
\crossref{Num}{33}{52}{Ex 23:24,\allowbreak31-\allowbreak33; 34:12-\allowbreak17 De 7:2-\allowbreak5,\allowbreak25,\allowbreak26; 12:2,\allowbreak3,\allowbreak30,\allowbreak31; 20:16-\allowbreak18}
\crossref{Num}{33}{53}{De 32:8 Ps 24:1,\allowbreak2; 115:16 Jer 27:5,\allowbreak6 Da 4:17,\allowbreak25,\allowbreak32 Mt 20:15}
\crossref{Num}{33}{54}{Nu 26:53-\allowbreak56}
\crossref{Num}{33}{55}{Ex 23:33 De 7:4,\allowbreak16 Jos 23:12,\allowbreak13 Jud 1:21-\allowbreak36; 2:3 Ps 106:34-\allowbreak36}
\crossref{Num}{33}{56}{Le 18:28; 20:23 De 28:63; 29:28 Jos 23:15,\allowbreak16 2Ch 36:17-\allowbreak20}
\crossref{Num}{34}{1}{34:1}
\crossref{Num}{34}{2}{Nu 33:51-\allowbreak53 Ge 12:6,\allowbreak7; 13:15-\allowbreak17; 15:16-\allowbreak21; 17:8 De 1:7,\allowbreak8 Ps 78:55}
\crossref{Num}{34}{3}{Ex 23:31 Jos 15:1-\allowbreak12 Eze 47:13,\allowbreak19-\allowbreak23}
\crossref{Num}{34}{4}{Jos 15:3 Jud 1:36}
\crossref{Num}{34}{5}{Ge 15:18 Jos 15:4,\allowbreak47 1Ki 8:65 Isa 27:12}
\crossref{Num}{34}{6}{Jos 1:4; 9:1; 15:12,\allowbreak47; 23:4 Eze 47:10,\allowbreak15,\allowbreak20}
\crossref{Num}{34}{7}{34:3,\allowbreak6,\allowbreak9,\allowbreak10}
\crossref{Num}{34}{8}{Nu 13:21 Jos 13:5,\allowbreak6 2Sa 8:9 2Ki 14:25 Jer 39:5 Eze 47:15-\allowbreak20}
\crossref{Num}{34}{9}{Eze 47:17}
\crossref{Num}{34}{10}{}
\crossref{Num}{34}{11}{2Ki 23:33; 25:6 Jer 39:5,\allowbreak6; 52:10,\allowbreak26,\allowbreak27}
\crossref{Num}{34}{12}{34:3 Ge 13:10; 14:3; 19:24-\allowbreak26}
\crossref{Num}{34}{13}{34:1 Jos 14:1,\allowbreak2}
\crossref{Num}{34}{14}{Nu 32:23,\allowbreak33 De 3:12-\allowbreak17 Jos 13:8-\allowbreak12; 14:2,\allowbreak3}
\crossref{Num}{34}{15}{Nu 32:32}
\crossref{Num}{34}{16}{}
\crossref{Num}{34}{17}{Jos 14:1; 19:51}
\crossref{Num}{34}{18}{Nu 1:4-\allowbreak16}
\crossref{Num}{34}{19}{Nu 13:30; 14:6,\allowbreak24,\allowbreak30,\allowbreak38; 26:65}
\crossref{Num}{34}{20}{34:20}
\crossref{Num}{34}{21}{34:21}
\crossref{Num}{34}{22}{34:22}
\crossref{Num}{34}{23}{34:23}
\crossref{Num}{34}{24}{34:24}
\crossref{Num}{34}{25}{34:25}
\crossref{Num}{34}{26}{34:26}
\crossref{Num}{34}{27}{34:27}
\crossref{Num}{34}{28}{34:28}
\crossref{Num}{34}{29}{34:18 Jos 19:51}
\crossref{Num}{35}{1}{Nu 22:1; 26:63; 31:12; 33:50; 36:13}
\crossref{Num}{35}{2}{Le 25:32,\allowbreak33 Jos 14:3,\allowbreak4; 21:2-\allowbreak42 Eze 45:1-\allowbreak8; 48:8,\allowbreak22 1Co 9:10-\allowbreak14}
\crossref{Num}{35}{3}{Jos 21:11 2Ch 11:14 Eze 45:2}
\crossref{Num}{35}{4}{}
\crossref{Num}{35}{5}{35:4}
\crossref{Num}{35}{6}{35:13,\allowbreak14 De 4:41-\allowbreak43 Jos 20:2-\allowbreak9; 21:3,\allowbreak13,\allowbreak21,\allowbreak27,\allowbreak32,\allowbreak36,\allowbreak38 Ps 9:9}
\crossref{Num}{35}{7}{Jos 21:3-\allowbreak42 1Ch 6:54-\allowbreak81}
\crossref{Num}{35}{8}{Ge 49:7 Ex 32:28,\allowbreak29 De 33:8-\allowbreak11 Jos 21:3}
\crossref{Num}{35}{9}{35:9}
\crossref{Num}{35}{10}{Nu 34:2 Le 14:34; 25:2 De 12:9; 19:1,\allowbreak2}
\crossref{Num}{35}{11}{35:6 Jos 20:2}
\crossref{Num}{35}{12}{35:19,\allowbreak25-\allowbreak27 De 19:6 Jos 20:3-\allowbreak6,\allowbreak9 2Sa 14:7}
\crossref{Num}{35}{13}{35:6}
\crossref{Num}{35}{14}{De 4:41-\allowbreak43; 19:8-\allowbreak10 Jos 20:7-\allowbreak9}
\crossref{Num}{35}{15}{Nu 15:16 Ex 12:49 Le 24:22 Ro 3:29 Ga 3:28}
\crossref{Num}{35}{16}{35:22-\allowbreak24 De 19:11-\allowbreak13}
\crossref{Num}{35}{17}{Ex 21:18}
\crossref{Num}{35}{18}{35:18}
\crossref{Num}{35}{19}{35:12,\allowbreak21,\allowbreak24,\allowbreak27 De 19:6,\allowbreak12 Jos 20:3,\allowbreak5}
\crossref{Num}{35}{20}{Ge 4:5,\allowbreak8 2Sa 3:27; 13:22,\allowbreak28,\allowbreak29; 20:10 1Ki 2:5,\allowbreak6,\allowbreak31-\allowbreak33 Pr 26:24}
\crossref{Num}{35}{21}{35:21}
\crossref{Num}{35}{22}{35:11 Ex 21:13 De 19:5 Jos 20:3,\allowbreak5}
\crossref{Num}{35}{23}{}
\crossref{Num}{35}{24}{35:12 Jos 20:6}
\crossref{Num}{35}{25}{35:28 Jos 20:6 Ro 3:24-\allowbreak26 Eph 2:16-\allowbreak18 Heb 4:14-\allowbreak16; 7:25-\allowbreak28}
\crossref{Num}{35}{26}{}
\crossref{Num}{35}{27}{}
\crossref{Num}{35}{28}{Joh 15:4-\allowbreak6 Ac 11:23; 27:31 Heb 3:14; 6:4-\allowbreak8; 10:26-\allowbreak30,\allowbreak39}
\crossref{Num}{35}{29}{Nu 27:1,\allowbreak11}
\crossref{Num}{35}{30}{De 17:6,\allowbreak7; 19:15 Mt 18:16 Joh 8:17,\allowbreak18 2Co 13:1 1Ti 5:19}
\crossref{Num}{35}{31}{Ge 9:5,\allowbreak6 Ex 21:14 De 19:11-\allowbreak13 2Sa 12:13 1Ki 2:28-\allowbreak34 Ps 51:14}
\crossref{Num}{35}{32}{Ac 4:12 Ga 2:21; 3:10-\allowbreak13,\allowbreak22 Re 5:9}
\crossref{Num}{35}{33}{Le 18:25 De 21:1-\allowbreak8,\allowbreak23 2Ki 23:26; 24:4 Ps 106:28 Isa 26:21}
\crossref{Num}{35}{34}{Nu 5:3 Le 20:24-\allowbreak26}
\crossref{Num}{36}{1}{Nu 26:29-\allowbreak33; 27:1 Jos 17:2,\allowbreak3 1Ch 7:14-\allowbreak16}
\crossref{Num}{36}{2}{Nu 27:1,\allowbreak7 Jos 17:3-\allowbreak6 Job 42:15}
\crossref{Num}{36}{3}{36:3}
\crossref{Num}{36}{4}{Le 25:10-\allowbreak18,\allowbreak23 Isa 61:2 Lu 4:18,\allowbreak19}
\crossref{Num}{36}{5}{Nu 27:7 De 5:28}
\crossref{Num}{36}{6}{36:12 Ge 24:3,\allowbreak57,\allowbreak58 2Co 6:14}
\crossref{Num}{36}{7}{36:9 1Ki 21:3}
\crossref{Num}{36}{8}{1Ch 23:22}
\crossref{Num}{36}{9}{36:9}
\crossref{Num}{36}{10}{Ex 39:42,\allowbreak43 Le 24:23 2Ch 30:12 Mt 28:20}
\crossref{Num}{36}{11}{Nu 27:1}
\crossref{Num}{36}{12}{36:12}
\crossref{Num}{36}{13}{Le 7:37,\allowbreak38; 11:46; 13:59; 14:54-\allowbreak57; 15:32,\allowbreak33; 27:34}

% Deut
\crossref{Deut}{1}{1}{Nu 32:5,\allowbreak19,\allowbreak32; 34:15; 35:14 Jos 9:1,\allowbreak10; 22:4,\allowbreak7}
\crossref{Deut}{1}{2}{1:44; 2:4,\allowbreak8 Nu 20:17-\allowbreak21}
\crossref{Deut}{1}{3}{Nu 20:1; 33:38}
\crossref{Deut}{1}{4}{De 2:26-\allowbreak37; 3:1-\allowbreak11 Nu 21:21-\allowbreak35 Jos 12:2-\allowbreak6; 13:10-\allowbreak12 Ne 9:22}
\crossref{Deut}{1}{5}{De 4:8; 17:18,\allowbreak19; 31:9; 32:46}
\crossref{Deut}{1}{6}{De 5:2 Ex 3:1; 17:6}
\crossref{Deut}{1}{7}{Ge 15:16-\allowbreak21 Ex 23:31 Nu 34:3-\allowbreak12 Jos 24:15 Am 2:9}
\crossref{Deut}{1}{8}{Ge 12:7; 13:14,\allowbreak15; 15:16,\allowbreak18; 17:7,\allowbreak8; 22:16-\allowbreak18; 26:3,\allowbreak4; 28:13,\allowbreak14}
\crossref{Deut}{1}{9}{Ex 18:18 Nu 11:11-\allowbreak14,\allowbreak17}
\crossref{Deut}{1}{10}{De 10:22; 28:62 Ge 15:5; 22:17; 28:14 Ex 12:37; 32:13 Nu 1:46}
\crossref{Deut}{1}{11}{2Sa 24:3 1Ch 21:3 Ps 115:14}
\crossref{Deut}{1}{12}{1:9 Ex 18:13-\allowbreak16 Nu 11:11-\allowbreak15 1Ki 3:7-\allowbreak9 Ps 89:19 2Co 2:16; 3:5}
\crossref{Deut}{1}{13}{Ex 18:21 Nu 11:16,\allowbreak17 Ac 1:21-\allowbreak23; 6:2-\allowbreak6}
\crossref{Deut}{1}{14}{}
\crossref{Deut}{1}{15}{De 16:18 Ex 18:25,\allowbreak26}
\crossref{Deut}{1}{16}{De 27:11; 31:14 Nu 27:19 1Th 2:11 1Ti 5:21; 6:17}
\crossref{Deut}{1}{17}{De 10:17; 16:19 Le 19:15 1Sa 16:7 2Sa 14:14 Pr 24:23 Lu 20:21}
\crossref{Deut}{1}{18}{De 4:5,\allowbreak40; 12:28,\allowbreak32 Mt 28:20 Ac 20:20,\allowbreak27}
\crossref{Deut}{1}{19}{De 8:15; 32:10 Nu 10:12 Jer 2:6}
\crossref{Deut}{1}{20}{1:7,\allowbreak8}
\crossref{Deut}{1}{21}{De 20:1 Nu 13:30; 14:8,\allowbreak9 Jos 1:9 Ps 27:1-\allowbreak3; 46:1,\allowbreak7,\allowbreak11 Isa 41:10}
\crossref{Deut}{1}{22}{Nu 13:1-\allowbreak20}
\crossref{Deut}{1}{23}{Nu 13:3-\allowbreak33}
\crossref{Deut}{1}{24}{Nu 13:21-\allowbreak27 Jos 2:1,\allowbreak2}
\crossref{Deut}{1}{25}{1:25}
\crossref{Deut}{1}{26}{Nu 14:1-\allowbreak4 Ps 106:24,\allowbreak25 Isa 63:10 Ac 7:51}
\crossref{Deut}{1}{27}{De 9:28 Ex 16:3,\allowbreak8 Nu 14:3; 21:5 Mt 25:24 Lu 19:21}
\crossref{Deut}{1}{28}{De 20:8}
\crossref{Deut}{1}{29}{1:21}
\crossref{Deut}{1}{30}{De 20:1-\allowbreak4 Ex 14:14,\allowbreak25 Jos 10:42 1Sa 17:45,\allowbreak46 2Ch 14:11,\allowbreak12; 32:8}
\crossref{Deut}{1}{31}{Ex 16:1-\allowbreak17:16 Ne 9:12-\allowbreak23 Ps 78:14-\allowbreak28; 105:39-\allowbreak41}
\crossref{Deut}{1}{32}{2Ch 20:20 Ps 78:22; 106:24 Isa 7:9 Heb 3:12,\allowbreak18,\allowbreak19 Jude 1:5}
\crossref{Deut}{1}{33}{Ex 13:21 Nu 10:33 Ps 77:20 Eze 20:6}
\crossref{Deut}{1}{34}{De 2:14,\allowbreak15 Nu 14:22-\allowbreak30; 32:8-\allowbreak13 Ps 95:11 Eze 20:15 Heb 3:8-\allowbreak11}
\crossref{Deut}{1}{35}{}
\crossref{Deut}{1}{36}{Nu 13:6,\allowbreak30; 26:65; 32:12; 34:19 Jos 14:6-\allowbreak14 Jud 1:12-\allowbreak15}
\crossref{Deut}{1}{37}{De 3:23-\allowbreak26; 4:21; 34:4 Nu 20:12; 27:13,\allowbreak14 Ps 106:32,\allowbreak33}
\crossref{Deut}{1}{38}{Nu 13:8,\allowbreak16; 14:30,\allowbreak38; 26:65}
\crossref{Deut}{1}{39}{Nu 14:3,\allowbreak31}
\crossref{Deut}{1}{40}{Nu 14:25}
\crossref{Deut}{1}{41}{Nu 14:39,\allowbreak40-\allowbreak45; 22:34 Pr 19:3}
\crossref{Deut}{1}{42}{Nu 14:41,\allowbreak42}
\crossref{Deut}{1}{43}{Isa 63:10 Ac 7:51 Ro 8:7,\allowbreak8}
\crossref{Deut}{1}{44}{Nu 14:45; 21:3}
\crossref{Deut}{1}{45}{Ps 78:34 Heb 12:17}
\crossref{Deut}{1}{46}{Nu 14:25,\allowbreak34; 20:1,\allowbreak22 Jud 11:16,\allowbreak17}
\crossref{Deut}{2}{1}{De 1:40 Nu 14:25}
\crossref{Deut}{2}{2}{}
\crossref{Deut}{2}{3}{2:7,\allowbreak14; 1:6}
\crossref{Deut}{2}{4}{De 23:7 Nu 20:14-\allowbreak21 Ob 1:10-\allowbreak13}
\crossref{Deut}{2}{5}{De 32:8 Ge 36:8 Jos 24:4 2Ch 20:10-\allowbreak12 Jer 27:5 Da 4:25,\allowbreak32}
\crossref{Deut}{2}{6}{2:28,\allowbreak29 Nu 20:19 Mt 7:12 Ro 12:17 2Th 3:7,\allowbreak8}
\crossref{Deut}{2}{7}{Ge 12:2; 24:35; 26:12; 30:27; 39:5 Ps 90:17}
\crossref{Deut}{2}{8}{Nu 20:20,\allowbreak21 Jud 11:18}
\crossref{Deut}{2}{9}{2:19 Ge 19:36,\allowbreak37 Ps 83:8}
\crossref{Deut}{2}{10}{}
\crossref{Deut}{2}{11}{De 1:28; 9:2 Nu 13:22,\allowbreak28,\allowbreak33}
\crossref{Deut}{2}{12}{2:22 Ge 14:6; 36:20-\allowbreak30 1Ch 1:38-\allowbreak42}
\crossref{Deut}{2}{13}{Nu 13:23}
\crossref{Deut}{2}{14}{De 1:2,\allowbreak19,\allowbreak46 Nu 13:26}
\crossref{Deut}{2}{15}{Jud 2:15 1Sa 5:6,\allowbreak9,\allowbreak11; 7:13 Ps 32:4; 78:33; 90:7-\allowbreak9; 106:26}
\crossref{Deut}{2}{16}{2:16}
\crossref{Deut}{2}{17}{}
\crossref{Deut}{2}{18}{Nu 21:15,\allowbreak23 Isa 15:1}
\crossref{Deut}{2}{19}{2:5,\allowbreak9 Ge 19:36-\allowbreak38 Jud 11:13-\allowbreak27 2Ch 20:10}
\crossref{Deut}{2}{20}{Ge 14:5}
\crossref{Deut}{2}{21}{2:10,\allowbreak11; 1:28; 3:11}
\crossref{Deut}{2}{22}{Ge 36:8}
\crossref{Deut}{2}{23}{Jos 13:3}
\crossref{Deut}{2}{24}{2:36 Nu 21:13-\allowbreak15 Jud 11:18-\allowbreak21}
\crossref{Deut}{2}{25}{De 11:25; 28:10 Ex 15:14-\allowbreak16; 23:27 Jos 2:9-\allowbreak12; 9:24 2Ki 7:6,\allowbreak7}
\crossref{Deut}{2}{26}{Jos 13:18; 21:37}
\crossref{Deut}{2}{27}{2:6 Nu 21:21-\allowbreak23 Jud 11:19}
\crossref{Deut}{2}{28}{Nu 20:19}
\crossref{Deut}{2}{29}{De 4:1,\allowbreak21,\allowbreak40; 5:16; 9:6; 25:15 Ex 20:12 Jos 1:11-\allowbreak15}
\crossref{Deut}{2}{30}{Ex 4:21; 11:10 Nu 21:23 Jos 11:19,\allowbreak20 Jud 11:20 Ro 9:17-\allowbreak23}
\crossref{Deut}{2}{31}{2:24; 1:8}
\crossref{Deut}{2}{32}{Nu 21:23-\allowbreak30 Jud 11:20-\allowbreak23 Ne 9:22 Ps 120:7; 135:11; 136:19}
\crossref{Deut}{2}{33}{De 3:2,\allowbreak3; 7:2; 20:16 Ge 14:20 Jos 21:44 Jud 1:4; 7:2}
\crossref{Deut}{2}{34}{De 7:2,\allowbreak26; 20:16-\allowbreak18 Le 27:28,\allowbreak29 Nu 21:2,\allowbreak3 Jos 7:11; 8:25,\allowbreak26; 9:24}
\crossref{Deut}{2}{35}{De 20:14 Nu 31:9-\allowbreak11 Jos 8:27}
\crossref{Deut}{2}{36}{De 3:12; 4:48 Jos 13:9 Isa 17:2 Jer 48:19}
\crossref{Deut}{2}{37}{2:5,\allowbreak9,\allowbreak19; 3:16 Jud 11:15}
\crossref{Deut}{3}{1}{De 1:4; 4:47; 29:7; 31:4 Nu 21:33-\allowbreak35 Jos 9:10; 12:4; 13:30 1Ki 4:19}
\crossref{Deut}{3}{2}{3:11; 20:3 Nu 14:9 2Ch 20:17 Isa 41:10; 43:5 Ac 18:9; 27:24}
\crossref{Deut}{3}{3}{De 2:33,\allowbreak34 Nu 21:35 Jos 13:12,\allowbreak30}
\crossref{Deut}{3}{4}{Nu 32:33-\allowbreak42 Jos 12:4; 13:30,\allowbreak31}
\crossref{Deut}{3}{5}{De 1:28 Nu 13:28 Heb 11:30}
\crossref{Deut}{3}{6}{De 2:34; 20:16-\allowbreak18 Le 27:28,\allowbreak29 Nu 21:2 Jos 11:14}
\crossref{Deut}{3}{7}{De 2:35 Jos 8:27; 11:11-\allowbreak14}
\crossref{Deut}{3}{8}{Nu 32:33-\allowbreak42 Jos 12:2-\allowbreak6; 13:9-\allowbreak12}
\crossref{Deut}{3}{9}{1Ch 5:23 Eze 27:5}
\crossref{Deut}{3}{10}{De 4:49}
\crossref{Deut}{3}{11}{Ge 14:5}
\crossref{Deut}{3}{12}{De 2:36; 4:48 Nu 32:33-\allowbreak38 Jos 12:2-\allowbreak6; 13:8-\allowbreak12,\allowbreak14-\allowbreak28 2Ki 10:33}
\crossref{Deut}{3}{13}{Nu 32:39-\allowbreak42 Jos 13:29-\allowbreak32 1Ch 5:23-\allowbreak26}
\crossref{Deut}{3}{14}{1Ch 2:21-\allowbreak23}
\crossref{Deut}{3}{15}{Ge 50:23 Nu 26:29; 32:39,\allowbreak40 Jos 17:1-\allowbreak3; 22:7}
\crossref{Deut}{3}{16}{Nu 32:33-\allowbreak38 2Sa 24:5}
\crossref{Deut}{3}{17}{Nu 34:11 Jos 12:3}
\crossref{Deut}{3}{18}{Nu 32:20-\allowbreak24 Jos 1:12-\allowbreak15; 4:12,\allowbreak13; 22:1-\allowbreak9}
\crossref{Deut}{3}{19}{}
\crossref{Deut}{3}{20}{Jos 22:4,\allowbreak8}
\crossref{Deut}{3}{21}{Nu 27:18-\allowbreak23}
\crossref{Deut}{3}{22}{Isa 43:1,\allowbreak2}
\crossref{Deut}{3}{23}{2Co 12:8,\allowbreak9}
\crossref{Deut}{3}{24}{De 11:2 Ne 9:32 Ps 106:2; 145:3,\allowbreak6 Jer 32:18-\allowbreak21}
\crossref{Deut}{3}{25}{De 4:21,\allowbreak22; 11:11,\allowbreak12 Ex 3:8 Nu 32:5 Eze 20:6}
\crossref{Deut}{3}{26}{De 1:37; 31:2; 32:51,\allowbreak52; 34:4 Nu 20:7-\allowbreak12; 27:12-\allowbreak14 Ps 106:32,\allowbreak33}
\crossref{Deut}{3}{27}{De 34:1-\allowbreak4 Nu 27:12}
\crossref{Deut}{3}{28}{De 1:38; 31:3,\allowbreak7,\allowbreak23 Nu 27:18-\allowbreak23 1Ch 22:6,\allowbreak11-\allowbreak16; 28:9,\allowbreak10,\allowbreak20 1Ti 6:13}
\crossref{Deut}{3}{29}{De 4:3,\allowbreak46; 34:6 Nu 25:3; 33:48,\allowbreak49}
\crossref{Deut}{4}{1}{Le 18:5 Eze 20:11,\allowbreak21 Ro 10:5}
\crossref{Deut}{4}{2}{De 12:32 Jos 1:7 Pr 30:6 Ec 12:13 Mt 5:18,\allowbreak43; 15:2-\allowbreak9 Mr 7:1-\allowbreak13}
\crossref{Deut}{4}{3}{Nu 25:1-\allowbreak9; 31:16 Jos 22:17 Ps 106:28,\allowbreak29 Ho 9:10}
\crossref{Deut}{4}{4}{De 10:20; 13:4 Jos 22:5; 23:8 Ru 1:14-\allowbreak17 Ps 63:8; 143:6-\allowbreak11}
\crossref{Deut}{4}{5}{4:1 Pr 22:19,\allowbreak20 Mt 28:20 Ac 20:27 1Co 11:28; 15:3 1Th 4:1,\allowbreak2}
\crossref{Deut}{4}{6}{Job 28:28 Ps 19:7; 111:10; 119:98-\allowbreak100 Pr 1:7; 14:8 Jer 8:9}
\crossref{Deut}{4}{7}{Nu 23:9,\allowbreak21 2Sa 7:23 Isa 43:4}
\crossref{Deut}{4}{8}{De 10:12,\allowbreak13 Ps 19:7-\allowbreak11; 119:86,\allowbreak96,\allowbreak127,\allowbreak128; 147:19,\allowbreak20 Ro 7:12-\allowbreak14}
\crossref{Deut}{4}{9}{4:15,\allowbreak23 Pr 3:1,\allowbreak3; 4:20-\allowbreak23 Lu 8:18 Heb 2:3 Jas 2:22}
\crossref{Deut}{4}{10}{De 5:2 Ex 19:9,\allowbreak16; 20:18 Heb 12:18,\allowbreak19,\allowbreak25}
\crossref{Deut}{4}{11}{De 5:23 Ex 19:16-\allowbreak18; 20:18,\allowbreak19}
\crossref{Deut}{4}{12}{De 5:4,\allowbreak22}
\crossref{Deut}{4}{13}{De 5:1-\allowbreak21 Ex 19:5; 24:17,\allowbreak18 Heb 9:19,\allowbreak20}
\crossref{Deut}{4}{14}{Eze 21:1-\allowbreak23:49 Ps 105:44,\allowbreak45}
\crossref{Deut}{4}{15}{4:9,\allowbreak23 Jos 23:11 1Ch 28:9,\allowbreak10 Ps 119:9 Pr 4:23,\allowbreak27 Jer 17:21}
\crossref{Deut}{4}{16}{4:8,\allowbreak9 Ex 20:4,\allowbreak5; 32:7 Ps 106:19,\allowbreak20 Ro 1:22-\allowbreak24}
\crossref{Deut}{4}{17}{Ro 1:23}
\crossref{Deut}{4}{18}{}
\crossref{Deut}{4}{19}{De 17:3 2Ki 23:4,\allowbreak5,\allowbreak11 Job 31:26,\allowbreak27 Jer 8:2 Eze 8:16 Am 5:25,\allowbreak26}
\crossref{Deut}{4}{20}{1Ki 8:51 Jer 11:4}
\crossref{Deut}{4}{21}{De 1:37; 3:26; 31:2 Nu 20:12 Ps 106:32,\allowbreak33}
\crossref{Deut}{4}{22}{De 3:25,\allowbreak27 1Ki 13:21,\allowbreak22 Am 3:2 Heb 12:6-\allowbreak10 2Pe 1:13-\allowbreak15}
\crossref{Deut}{4}{23}{4:9,\allowbreak15,\allowbreak16; 27:9 Jos 23:11 Mt 24:4 Lu 12:15; 21:8 Heb 3:12}
\crossref{Deut}{4}{24}{De 9:3; 32:22 Ex 24:17 Ps 21:9 Isa 30:33; 33:14 Jer 21:12-\allowbreak14}
\crossref{Deut}{4}{25}{De 31:16-\allowbreak18 Jud 2:8-\allowbreak15}
\crossref{Deut}{4}{26}{De 29:28 Le 18:28; 26:31-\allowbreak35 Jos 23:16 Isa 6:11; 24:1-\allowbreak3 Jer 44:22}
\crossref{Deut}{4}{27}{De 28:62-\allowbreak64 Ne 1:3,\allowbreak8,\allowbreak9 Eze 12:15; 32:26}
\crossref{Deut}{4}{28}{De 28:36,\allowbreak64 1Sa 26:19 Jer 16:13 Eze 20:32,\allowbreak39 Ac 7:42}
\crossref{Deut}{4}{29}{De 30:10 Le 26:39-\allowbreak42 2Ch 15:4,\allowbreak15 Ne 1:9 Isa 55:6,\allowbreak7 Jer 3:12-\allowbreak14}
\crossref{Deut}{4}{30}{1Ki 8:46-\allowbreak53 2Ch 6:36-\allowbreak39 Da 9:11-\allowbreak19}
\crossref{Deut}{4}{31}{Ex 34:6,\allowbreak7 Nu 14:18 2Ch 30:9 Ne 1:5; 9:31 Ps 86:5,\allowbreak15; 116:5}
\crossref{Deut}{4}{32}{Job 8:8 Ps 44:1 Joe 1:2}
\crossref{Deut}{4}{33}{4:24-\allowbreak26; 9:10 Ex 19:18,\allowbreak19; 20:18,\allowbreak19; 24:11; 33:20 Jud 6:22}
\crossref{Deut}{4}{34}{Ex 1:9; 3:10,\allowbreak17-\allowbreak20}
\crossref{Deut}{4}{35}{1Sa 17:45-\allowbreak47 1Ki 18:36,\allowbreak37 2Ki 19:19 Ps 58:11; 83:18}
\crossref{Deut}{4}{36}{4:33 Ex 19:9,\allowbreak19; 20:18-\allowbreak22; 24:16 Ne 9:13 Heb 12:18,\allowbreak25}
\crossref{Deut}{4}{37}{De 7:7-\allowbreak9; 9:5; 10:15 Ps 105:6-\allowbreak10 Isa 41:8,\allowbreak9 Jer 31:1 Mal 1:2}
\crossref{Deut}{4}{38}{De 7:1; 9:1-\allowbreak5; 11:23 Ex 23:27,\allowbreak28 Jos 3:10 Ps 44:2,\allowbreak3}
\crossref{Deut}{4}{39}{De 32:29 1Ch 28:9 Isa 1:3; 5:12 Ho 7:2}
\crossref{Deut}{4}{40}{4:1,\allowbreak6; 28:1-\allowbreak14 Le 22:31; 26:1-\allowbreak13 Jer 11:4 Joh 14:15,\allowbreak21-\allowbreak24}
\crossref{Deut}{4}{41}{Nu 35:6,\allowbreak14,\allowbreak15 Jos 20:2-\allowbreak9}
\crossref{Deut}{4}{42}{De 19:1-\allowbreak10 Nu 35:6,\allowbreak11,\allowbreak12,\allowbreak15-\allowbreak28 Heb 6:18}
\crossref{Deut}{4}{43}{}
\crossref{Deut}{4}{44}{De 6:17,\allowbreak20 1Ki 2:3 Ps 119:2,\allowbreak14,\allowbreak22,\allowbreak24,\allowbreak111}
\crossref{Deut}{4}{45}{4:47}
\crossref{Deut}{4}{46}{De 3:1-\allowbreak14; 29:7,\allowbreak8 Nu 21:33-\allowbreak35}
\crossref{Deut}{4}{47}{De 2:36; 3:12 Jos 13:24-\allowbreak33}
\crossref{Deut}{4}{48}{De 3:17; 34:1 Jos 13:20}
\crossref{Deut}{4}{49}{}
\crossref{Deut}{5}{1}{De 1:1; 29:2,\allowbreak10}
\crossref{Deut}{5}{2}{De 4:23 Ex 19:5-\allowbreak8; 24:8 Heb 8:6-\allowbreak13; 9:19-\allowbreak23}
\crossref{Deut}{5}{3}{De 29:10-\allowbreak15 Ge 17:7,\allowbreak21 Ps 105:8-\allowbreak10 Jer 32:38-\allowbreak40 Mt 13:17}
\crossref{Deut}{5}{4}{5:24-\allowbreak26; 4:33,\allowbreak36; 34:10 Ex 19:9,\allowbreak18,\allowbreak19; 20:18-\allowbreak22; 33:11 Nu 12:8}
\crossref{Deut}{5}{5}{5:27 Ge 18:22 Ex 19:16; 20:18-\allowbreak21; 24:2,\allowbreak3 Nu 16:48 Ps 106:23}
\crossref{Deut}{5}{6}{De 4:4}
\crossref{Deut}{5}{7}{Ex 20:3 Mt 4:10 Joh 5:23 1Jo 5:21}
\crossref{Deut}{5}{8}{De 4:15-\allowbreak19 Ex 20:4}
\crossref{Deut}{5}{9}{Ex 20:4-\allowbreak6}
\crossref{Deut}{5}{10}{Isa 1:16-\allowbreak19 Jer 32:18 Da 9:4 Mt 7:21-\allowbreak27 Ga 5:6 1Jo 1:7}
\crossref{Deut}{5}{11}{De 6:13}
\crossref{Deut}{5}{12}{Ex 20:8-\allowbreak11 Isa 56:6; 58:13}
\crossref{Deut}{5}{13}{Ex 23:12; 35:2,\allowbreak3 Eze 20:12 Lu 13:14-\allowbreak16; 23:56}
\crossref{Deut}{5}{14}{Ge 2:2 Ex 16:29,\allowbreak30 Heb 4:4}
\crossref{Deut}{5}{15}{De 15:15; 16:12; 24:18-\allowbreak22 Isa 51:1,\allowbreak2 Eph 2:11,\allowbreak12}
\crossref{Deut}{5}{16}{Ex 20:12 Le 19:3 Mt 15:4-\allowbreak6 Col 3:20}
\crossref{Deut}{5}{17}{Ex 20:13 Mt 5:21,\allowbreak22}
\crossref{Deut}{5}{18}{Ex 20:14 Pr 6:32,\allowbreak33 Mt 5:27,\allowbreak28 Lu 18:20 Jas 2:10,\allowbreak11}
\crossref{Deut}{5}{19}{Ex 20:15 Ro 13:9 Eph 4:28}
\crossref{Deut}{5}{20}{De 19:16-\allowbreak21}
\crossref{Deut}{5}{21}{Ex 20:17 1Ki 21:1-\allowbreak4 Mic 2:2 Hab 2:9 Lu 12:15 Ro 7:7,\allowbreak8; 13:9}
\crossref{Deut}{5}{22}{5:4; 4:12-\allowbreak15,\allowbreak36 Ex 19:18,\allowbreak19}
\crossref{Deut}{5}{23}{Ex 20:18,\allowbreak19 Heb 12:18-\allowbreak21}
\crossref{Deut}{5}{24}{5:4,\allowbreak5 Ex 19:19}
\crossref{Deut}{5}{25}{De 18:16; 33:2 2Co 3:7-\allowbreak9 Ga 3:10,\allowbreak21,\allowbreak22 Heb 12:29}
\crossref{Deut}{5}{26}{De 4:33}
\crossref{Deut}{5}{27}{Ex 20:19 Heb 12:19}
\crossref{Deut}{5}{28}{De 18:17 Nu 27:7; 36:5}
\crossref{Deut}{5}{29}{De 32:29,\allowbreak30 Ps 81:13-\allowbreak15 Isa 48:18 Jer 44:4 Eze 33:31,\allowbreak32 Mt 23:37}
\crossref{Deut}{5}{30}{}
\crossref{Deut}{5}{31}{5:1; 4:1,\allowbreak5,\allowbreak45; 6:1; 11:1; 12:1 Eze 20:11 Mal 4:4 Ga 3:19}
\crossref{Deut}{5}{32}{De 6:3,\allowbreak25; 8:1; 11:32; 24:8 2Ki 21:8 Eze 37:24}
\crossref{Deut}{5}{33}{De 10:12 Ps 119:6 Jer 7:23 Lu 1:6 Ro 2:7}
\crossref{Deut}{6}{1}{De 4:1,\allowbreak5,\allowbreak14,\allowbreak45; 5:31; 12:1 Le 27:34 Nu 36:13 Eze 37:24}
\crossref{Deut}{6}{2}{De 4:10; 10:12,\allowbreak13,\allowbreak20; 13:4 Ge 22:12 Ex 20:20 Job 28:28 Ps 111:10}
\crossref{Deut}{6}{3}{De 4:6; 5:32 Ec 8:12 Isa 3:10}
\crossref{Deut}{6}{4}{De 4:35,\allowbreak36; 5:6 1Ki 18:21 2Ki 19:5 1Ch 29:10 Isa 42:8; 44:6,\allowbreak8}
\crossref{Deut}{6}{5}{De 10:12; 11:13; 30:6 Mt 22:37 Mr 12:30,\allowbreak33 Lu 10:27 1Jo 5:3}
\crossref{Deut}{6}{6}{De 11:18; 32:46 Ps 37:31; 40:8; 119:11,\allowbreak98 Pr 2:10,\allowbreak11; 3:1-\allowbreak3,\allowbreak5; 7:3}
\crossref{Deut}{6}{7}{6:2; 4:9,\allowbreak10; 11:19 Ge 18:19 Ex 12:26,\allowbreak27; 13:14,\allowbreak15 Ps 78:4-\allowbreak6}
\crossref{Deut}{6}{8}{De 11:18 Ex 13:9,\allowbreak16 Nu 15:38,\allowbreak39 Pr 3:3; 6:21; 7:3 Mt 23:5 Heb 2:1}
\crossref{Deut}{6}{9}{De 11:20 Ex 12:7 Job 19:23-\allowbreak25 Isa 30:8; 57:8 Hab 2:2}
\crossref{Deut}{6}{10}{Ge 13:15-\allowbreak17; 15:18; 26:3; 28:13}
\crossref{Deut}{6}{11}{De 7:12-\allowbreak18; 8:10-\allowbreak20; 32:15 Jud 3:7 Pr 30:8,\allowbreak9 Jer 2:31,\allowbreak32}
\crossref{Deut}{6}{12}{6:12}
\crossref{Deut}{6}{13}{6:2; 5:29; 10:12,\allowbreak20; 13:4 Mt 4:10 Lu 4:8}
\crossref{Deut}{6}{14}{De 8:19; 11:28 Ex 34:14-\allowbreak16 Jer 25:6 1Jo 5:21}
\crossref{Deut}{6}{15}{De 4:24 Ex 20:5 Am 3:2 1Co 10:22}
\crossref{Deut}{6}{16}{Mt 4:7 Lu 4:12}
\crossref{Deut}{6}{17}{6:1,\allowbreak2; 11:13,\allowbreak22 Ex 15:26 Ps 119:4 1Co 15:58 Tit 3:8 Heb 6:11}
\crossref{Deut}{6}{18}{De 8:11; 12:25,\allowbreak28; 13:18 Ex 15:26 Ps 19:11 Isa 3:10 Eze 18:5,\allowbreak19,\allowbreak21}
\crossref{Deut}{6}{19}{Ex 23:28-\allowbreak30 Nu 33:52,\allowbreak53 Jud 2:1-\allowbreak3; 3:1-\allowbreak4}
\crossref{Deut}{6}{20}{6:7 Ex 12:26; 13:14 Jos 4:6,\allowbreak7,\allowbreak21-\allowbreak24 Pr 22:6}
\crossref{Deut}{6}{21}{De 5:6,\allowbreak15; 15:15; 26:5-\allowbreak9}
\crossref{Deut}{6}{22}{De 4:34 Ex 7:1-\allowbreak12:51; 14:1-\allowbreak31 Ps 135:9}
\crossref{Deut}{6}{23}{6:10,\allowbreak18; 1:8,\allowbreak35}
\crossref{Deut}{6}{24}{6:2}
\crossref{Deut}{6}{25}{De 24:13 Le 18:5 Ps 106:30,\allowbreak31; 119:6 Pr 12:28 Eze 20:11}
\crossref{Deut}{7}{1}{De 4:38; 6:1,\allowbreak10,\allowbreak19,\allowbreak23; 9:1,\allowbreak4; 11:29; 31:3,\allowbreak20 Ex 6:8; 15:7 Nu 14:31}
\crossref{Deut}{7}{2}{7:23,\allowbreak24; 3:3; 23:14 Ge 14:20 Jos 10:24,\allowbreak25,\allowbreak30,\allowbreak32,\allowbreak42; 21:44 Jud 1:4}
\crossref{Deut}{7}{3}{Ge 6:2,\allowbreak3 Ex 34:15,\allowbreak16 Jos 23:12,\allowbreak13 Jud 3:6,\allowbreak7 1Ki 11:2}
\crossref{Deut}{7}{4}{De 6:15; 32:16,\allowbreak17 Ex 20:5 Jud 2:11,\allowbreak20; 3:7,\allowbreak8; 10:6,\allowbreak7}
\crossref{Deut}{7}{5}{De 12:2,\allowbreak3 Ex 23:24; 34:13 2Ki 23:6-\allowbreak14}
\crossref{Deut}{7}{6}{De 14:2; 26:19; 28:9 Ex 19:5,\allowbreak6 Ps 50:5 Jer 2:3 Am 3:2 1Co 6:19,\allowbreak20}
\crossref{Deut}{7}{7}{Ps 115:1 Ro 9:11-\allowbreak15,\allowbreak18,\allowbreak21; 11:6 1Jo 3:1; 4:10}
\crossref{Deut}{7}{8}{De 4:37; 9:4,\allowbreak5; 10:15 1Sa 12:22 2Sa 22:20 Ps 44:3 Isa 43:4 Jer 31:3}
\crossref{Deut}{7}{9}{Ex 34:6,\allowbreak7 Ps 119:75; 146:6 Isa 49:7 La 3:23 1Co 1:9; 10:3}
\crossref{Deut}{7}{10}{7:9; 32:35,\allowbreak41 Ps 21:8,\allowbreak9 Pr 11:31 Isa 59:18 Na 1:2 Ro 12:19}
\crossref{Deut}{7}{11}{De 4:1; 5:32 Joh 14:15}
\crossref{Deut}{7}{12}{De 28:1 Le 26:3}
\crossref{Deut}{7}{13}{7:7; 28:4 Ex 23:25 Ps 1:3; 11:7; 144:12-\allowbreak15 Joh 14:21; 15:10; 16:27}
\crossref{Deut}{7}{14}{De 33:29 Ps 115:15; 147:19,\allowbreak20}
\crossref{Deut}{7}{15}{De 28:27,\allowbreak60 Ex 9:11; 15:26 Ps 105:36,\allowbreak37}
\crossref{Deut}{7}{16}{7:2}
\crossref{Deut}{7}{17}{De 8:17; 15:9; 18:21 Isa 14:13; 47:8; 49:21 Jer 13:22 Lu 9:47}
\crossref{Deut}{7}{18}{De 1:29; 3:6; 31:6 Ps 27:1,\allowbreak2; 46:1,\allowbreak2 Isa 41:10-\allowbreak14}
\crossref{Deut}{7}{19}{De 4:34; 11:2-\allowbreak4; 29:3 Ne 9:10,\allowbreak11 Jer 32:20,\allowbreak21 Eze 20:6-\allowbreak9}
\crossref{Deut}{7}{20}{Ex 23:28-\allowbreak30 Jos 24:12}
\crossref{Deut}{7}{21}{Nu 9:20; 14:9,\allowbreak14,\allowbreak42; 16:3; 23:21 Jos 3:10 2Ch 32:8 Ps 46:5,\allowbreak7,\allowbreak11}
\crossref{Deut}{7}{22}{}
\crossref{Deut}{7}{23}{7:2}
\crossref{Deut}{7}{24}{Jos 10:24,\allowbreak25,\allowbreak42; 12:1-\allowbreak6}
\crossref{Deut}{7}{25}{7:5; 12:3 Ex 32:20 1Ch 14:12 Isa 30:22}
\crossref{Deut}{7}{26}{De 13:17 Le 27:28,\allowbreak29 Jos 6:17-\allowbreak24; 7:1-\allowbreak25,\allowbreak11-\allowbreak26 Eze 14:7}
\crossref{Deut}{8}{1}{De 4:1; 5:32,\allowbreak33; 6:1-\allowbreak3 Ps 119:4-\allowbreak6 1Th 4:1,\allowbreak2}
\crossref{Deut}{8}{2}{De 7:18 Ps 77:11; 106:7 Eph 2:11,\allowbreak12 2Pe 1:12,\allowbreak13; 3:1,\allowbreak2}
\crossref{Deut}{8}{3}{Ex 16:2,\allowbreak3,\allowbreak12-\allowbreak35 Ps 78:23-\allowbreak25; 105:40 1Co 10:3}
\crossref{Deut}{8}{4}{De 29:5 Ne 9:21 Mt 26:25-\allowbreak30}
\crossref{Deut}{8}{5}{De 4:9,\allowbreak23 Isa 1:3 Eze 12:3; 18:28}
\crossref{Deut}{8}{6}{De 5:33 Ex 18:20 1Sa 12:24 2Ch 6:31 Ps 128:1 Lu 1:6}
\crossref{Deut}{8}{7}{De 6:10,\allowbreak11; 11:10-\allowbreak12 Ex 3:8 Ne 9:24,\allowbreak25 Ps 65:9-\allowbreak13 Eze 20:6}
\crossref{Deut}{8}{8}{De 32:14 2Sa 4:6 1Ki 5:11 Ps 81:16; 147:14 Eze 27:17}
\crossref{Deut}{8}{9}{De 33:25 Jos 22:8 1Ch 22:14 Job 28:2}
\crossref{Deut}{8}{10}{De 6:11,\allowbreak12 Ps 103:2 Mt 14:19 Joh 6:23 Ro 14:6 1Co 10:31 1Th 5:18}
\crossref{Deut}{8}{11}{Ps 106:21 Pr 1:32; 30:9 Eze 16:10-\allowbreak15 Ho 2:8,\allowbreak9}
\crossref{Deut}{8}{12}{De 28:47; 31:20; 32:15 Pr 30:9 Ho 13:5,\allowbreak6}
\crossref{Deut}{8}{13}{Ge 13:1-\allowbreak5 Job 1:3 Ps 39:6 Lu 12:13-\allowbreak21}
\crossref{Deut}{8}{14}{De 17:20 2Ch 26:16; 32:25 Jer 2:31 1Co 4:7,\allowbreak8}
\crossref{Deut}{8}{15}{De 1:19 Ps 136:16 Isa 63:12-\allowbreak14 Jer 2:6}
\crossref{Deut}{8}{16}{8:3 Ex 16:15}
\crossref{Deut}{8}{17}{De 7:17}
\crossref{Deut}{8}{18}{Ps 127:1,\allowbreak2; 144:1 Pr 10:22 Ho 2:8}
\crossref{Deut}{8}{19}{De 4:26; 28:58-\allowbreak68; 29:25-\allowbreak28; 30:18,\allowbreak19 Jos 23:13 1Sa 12:25 Da 9:2}
\crossref{Deut}{8}{20}{2Ch 36:16,\allowbreak17 Da 9:11,\allowbreak12}
\crossref{Deut}{9}{1}{De 3:18; 11:31; 27:2 Jos 1:11; 3:6,\allowbreak14,\allowbreak16; 4:5,\allowbreak19}
\crossref{Deut}{9}{2}{De 2:11,\allowbreak12,\allowbreak21}
\crossref{Deut}{9}{3}{9:6 Mt 15:10 Mr 7:14 Eph 5:17}
\crossref{Deut}{9}{4}{9:5; 7:7,\allowbreak8; 8:17 Eze 36:22,\allowbreak32 Ro 11:6,\allowbreak20 1Co 4:4,\allowbreak7 Eph 2:4,\allowbreak5}
\crossref{Deut}{9}{5}{Ge 12:7; 13:15; 15:7; 17:8; 26:4; 28:13 Ex 32:13 Eze 20:14}
\crossref{Deut}{9}{6}{9:3,\allowbreak4 Eze 20:44}
\crossref{Deut}{9}{7}{De 31:27; 32:5,\allowbreak6 Ex 14:11; 16:2; 17:2 Nu 11:4; 14:1-\allowbreak10; 16:1-\allowbreak35}
\crossref{Deut}{9}{8}{}
\crossref{Deut}{9}{9}{Ex 24:12,\allowbreak15,\allowbreak18}
\crossref{Deut}{9}{10}{Ex 31:18}
\crossref{Deut}{9}{11}{9:9 Nu 10:33 Heb 8:6-\allowbreak10; 9:4}
\crossref{Deut}{9}{12}{Ex 32:7,\allowbreak8}
\crossref{Deut}{9}{13}{Ge 11:5; 18:21 Ex 32:9,\allowbreak10 Ps 50:7 Jer 7:11; 13:27 Ho 6:10}
\crossref{Deut}{9}{14}{Ex 32:10-\allowbreak13 Isa 62:6,\allowbreak7 Jer 14:11; 15:1 Lu 11:7-\allowbreak10; 18:1-\allowbreak8}
\crossref{Deut}{9}{15}{Ex 32:14,\allowbreak15-\allowbreak35}
\crossref{Deut}{9}{16}{Ex 32:19 Ac 7:40,\allowbreak41}
\crossref{Deut}{9}{17}{}
\crossref{Deut}{9}{18}{9:9 Ex 32:10-\allowbreak14; 34:28 2Sa 12:16 Ps 106:23}
\crossref{Deut}{9}{19}{9:8 Ex 32:10,\allowbreak11 Ne 1:2-\allowbreak7 Lu 12:4,\allowbreak5}
\crossref{Deut}{9}{20}{Ex 32:2-\allowbreak5,\allowbreak21,\allowbreak35 Heb 7:26-\allowbreak28}
\crossref{Deut}{9}{21}{Ex 32:20 Isa 2:18-\allowbreak21; 30:22; 31:7 Ho 8:11}
\crossref{Deut}{9}{22}{Nu 11:1-\allowbreak5}
\crossref{Deut}{9}{23}{De 1:19-\allowbreak33 Nu 13:1-\allowbreak33}
\crossref{Deut}{9}{24}{9:6,\allowbreak7; 31:27 Ac 7:51}
\crossref{Deut}{9}{25}{9:16,\allowbreak18}
\crossref{Deut}{9}{26}{Ex 32:11-\allowbreak13; 34:9 Nu 14:13-\allowbreak19 Ps 99:6; 106:23 Jer 14:21}
\crossref{Deut}{9}{27}{Ex 3:6,\allowbreak16; 6:3-\allowbreak8; 13:5; 32:13 Jer 14:21}
\crossref{Deut}{9}{28}{Ge 41:57 Ex 6:6-\allowbreak8 1Sa 14:25}
\crossref{Deut}{9}{29}{9:26; 4:20 1Ki 8:15 Ne 1:10 Ps 95:7; 100:3 Isa 63:19}
\crossref{Deut}{10}{1}{10:4 Ex 34:1,\allowbreak2,\allowbreak4}
\crossref{Deut}{10}{2}{10:5 Ex 25:16-\allowbreak22; 40:20 1Ki 8:9 Heb 9:4}
\crossref{Deut}{10}{3}{Ex 25:5,\allowbreak10; 37:1-\allowbreak9}
\crossref{Deut}{10}{4}{De 9:10 Ex 34:28}
\crossref{Deut}{10}{5}{De 9:15 Ex 32:15; 34:29}
\crossref{Deut}{10}{6}{Nu 10:6,\allowbreak12,\allowbreak13; 33:1,\allowbreak2}
\crossref{Deut}{10}{7}{}
\crossref{Deut}{10}{8}{Ex 29:1-\allowbreak37 Le 8:9 Nu 1:47-\allowbreak53; 3:1-\allowbreak4:49 8:1-\allowbreak26 16:9,\allowbreak10 18:1-\allowbreak32}
\crossref{Deut}{10}{9}{De 18:1,\allowbreak2 Nu 18:20-\allowbreak24; 26:62 Jos 14:3 Eze 44:28}
\crossref{Deut}{10}{10}{De 9:18,\allowbreak25 Ex 24:18; 34:28}
\crossref{Deut}{10}{11}{Ex 32:34; 33:1}
\crossref{Deut}{10}{12}{Jer 7:22,\allowbreak23 Mic 6:8 Mt 11:29,\allowbreak30 1Jo 5:3}
\crossref{Deut}{10}{13}{De 6:24 Pr 9:12 Jer 32:39 Jas 1:25}
\crossref{Deut}{10}{14}{1Ki 8:27 2Ch 6:18 Ne 9:6 Ps 115:16; 148:4 Isa 66:1}
\crossref{Deut}{10}{15}{De 4:37; 7:7,\allowbreak8 Nu 14:8 Ro 9:13-\allowbreak23}
\crossref{Deut}{10}{16}{De 30:6 Le 26:41 Jer 4:4,\allowbreak14 Ro 2:28,\allowbreak29 Col 2:11}
\crossref{Deut}{10}{17}{Jos 22:22 1Ch 16:25,\allowbreak26 Ps 136:2 Da 2:47; 11:36}
\crossref{Deut}{10}{18}{Ps 68:5; 103:6; 146:9 Isa 1:17 Jer 49:11 Ho 14:3}
\crossref{Deut}{10}{19}{Ex 22:21 Le 19:33,\allowbreak34 Lu 6:35; 10:28-\allowbreak37; 17:18 Ga 6:10}
\crossref{Deut}{10}{20}{De 6:13; 13:4 Mt 4:10 Lu 4:8}
\crossref{Deut}{10}{21}{Ex 15:2 Ps 22:3 Isa 12:2-\allowbreak6; 60:19 Jer 17:14 Lu 2:32 Re 21:23}
\crossref{Deut}{10}{22}{De 1:10; 28:62 Ge 15:5 Nu 26:51,\allowbreak62 Ne 9:23 Heb 11:12}
\crossref{Deut}{11}{1}{De 6:5; 10:12; 30:16-\allowbreak20 Ps 116:1}
\crossref{Deut}{11}{2}{De 8:2-\allowbreak5}
\crossref{Deut}{11}{3}{De 4:34; 7:19 Ps 78:12,\allowbreak13; 105:27-\allowbreak45; 135:9 Jer 32:20,\allowbreak21}
\crossref{Deut}{11}{4}{Ex 14:23-\allowbreak31; 15:4,\allowbreak9,\allowbreak10,\allowbreak19 Ps 106:11 Heb 11:29}
\crossref{Deut}{11}{5}{Ps 77:20; 78:14-\allowbreak72; 105:39-\allowbreak41; 106:12-\allowbreak48}
\crossref{Deut}{11}{6}{Nu 16:1,\allowbreak31-\allowbreak33; 26:9,\allowbreak10; 27:3 Ps 106:17}
\crossref{Deut}{11}{7}{De 5:3; 7:19 Ps 106:2; 145:4-\allowbreak6,\allowbreak12; 150:2}
\crossref{Deut}{11}{8}{De 8:10,\allowbreak11; 10:12-\allowbreak15; 26:16-\allowbreak19; 28:47 Ps 116:12-\allowbreak16}
\crossref{Deut}{11}{9}{De 4:40; 5:16; 6:2 Ps 34:12-\allowbreak22 Pr 3:2,\allowbreak16; 9:11; 10:27}
\crossref{Deut}{11}{10}{}
\crossref{Deut}{11}{11}{De 8:7-\allowbreak9 Ge 27:28 Ps 65:12,\allowbreak13; 104:10-\allowbreak13 Isa 28:1 Jer 2:7 Heb 6:7}
\crossref{Deut}{11}{12}{1Ki 9:3 Ezr 5:5 Ps 33:18; 34:15 Jer 24:6}
\crossref{Deut}{11}{13}{11:8,\allowbreak22}
\crossref{Deut}{11}{14}{De 28:12 Le 26:4 Job 5:10,\allowbreak11; 37:11-\allowbreak13 Ps 65:9-\allowbreak13 Jer 14:22}
\crossref{Deut}{11}{15}{1Ki 18:5 Ps 104:14 Jer 14:5 Joe 1:18; 2:22}
\crossref{Deut}{11}{16}{De 4:9,\allowbreak23 Lu 21:8,\allowbreak34,\allowbreak36 Heb 2:1; 3:12; 4:1; 12:15}
\crossref{Deut}{11}{17}{De 6:15; 30:17,\allowbreak18}
\crossref{Deut}{11}{18}{De 6:6-\allowbreak9; 32:46 Ex 13:9,\allowbreak16 Ps 119:11 Pr 3:1; 6:20-\allowbreak23; 7:2,\allowbreak3}
\crossref{Deut}{11}{19}{De 4:9,\allowbreak10; 6:7 Ps 34:11; 78:5,\allowbreak6 Pr 2:1; 4:1-\allowbreak27 Isa 38:19}
\crossref{Deut}{11}{20}{De 6:9}
\crossref{Deut}{11}{21}{De 4:40; 5:16; 6:2 Pr 3:2,\allowbreak16; 4:10; 9:11}
\crossref{Deut}{11}{22}{11:13; 6:17}
\crossref{Deut}{11}{23}{De 4:38; 7:1,\allowbreak2,\allowbreak22,\allowbreak23; 9:1,\allowbreak5 Ex 23:27-\allowbreak30; 34:11}
\crossref{Deut}{11}{24}{Ge 15:18-\allowbreak21 Ex 23:31 Nu 34:3-\allowbreak15 Jos 1:3,\allowbreak4; 14:9 1Ki 4:21,\allowbreak24}
\crossref{Deut}{11}{25}{De 2:25; 7:24 Jos 1:5; 2:9; 5:1}
\crossref{Deut}{11}{26}{De 30:1,\allowbreak15-\allowbreak20 Ga 3:10,\allowbreak13,\allowbreak14}
\crossref{Deut}{11}{27}{De 28:1-\allowbreak14 Le 26:3-\allowbreak13 Ps 19:11 Isa 1:19; 3:10 Mt 5:3-\allowbreak12; 25:31-\allowbreak46}
\crossref{Deut}{11}{28}{De 28:15-\allowbreak68; 29:19-\allowbreak28 Le 26:14-\allowbreak32 Isa 1:20; 3:11 Mt 25:41 Ro 2:8,\allowbreak9}
\crossref{Deut}{11}{29}{De 27:12-\allowbreak26 Jos 8:30-\allowbreak35}
\crossref{Deut}{11}{30}{Ge 12:6 Jos 5:9 Jud 7:1}
\crossref{Deut}{11}{31}{De 9:1 Jos 1:11; 3:13-\allowbreak17}
\crossref{Deut}{11}{32}{De 5:32,\allowbreak33; 12:32 Ps 119:6 Mt 7:21-\allowbreak27; 28:20 Lu 1:6 Joh 15:14}
\crossref{Deut}{12}{1}{De 4:1,\allowbreak2,\allowbreak5,\allowbreak45; 6:1,\allowbreak2}
\crossref{Deut}{12}{2}{De 7:5,\allowbreak25,\allowbreak26 Ex 23:24; 34:12-\allowbreak17 Nu 33:51,\allowbreak52 Jud 2:2}
\crossref{Deut}{12}{3}{Nu 33:52 Jud 2:2 2Ch 31:1}
\crossref{Deut}{12}{4}{12:30,\allowbreak31; 16:21,\allowbreak22; 20:18 Le 20:23}
\crossref{Deut}{12}{5}{12:11; 16:2; 26:2 Jos 9:27; 18:1 1Ki 8:16,\allowbreak20,\allowbreak29; 14:21 1Ch 22:1}
\crossref{Deut}{12}{6}{Le 17:3-\allowbreak9 Eze 20:40}
\crossref{Deut}{12}{7}{12:18; 14:23,\allowbreak26; 15:20 Isa 23:18}
\crossref{Deut}{12}{8}{Nu 15:39 Jud 17:6; 21:25 Pr 21:2 Am 5:25 Ac 7:42}
\crossref{Deut}{12}{9}{De 25:19 1Ki 8:56 1Ch 23:25 Mic 2:10 Heb 4:8,\allowbreak9 1Pe 1:3,\allowbreak4}
\crossref{Deut}{12}{10}{De 3:27; 4:22; 9:1; 11:31 Jos 3:17; 4:1,\allowbreak12}
\crossref{Deut}{12}{11}{12:5,\allowbreak14,\allowbreak18,\allowbreak21,\allowbreak26; 14:23; 15:20; 16:2-\allowbreak8; 17:8; 18:6; 23:16; 26:2; 31:11}
\crossref{Deut}{12}{12}{12:7; 14:26,\allowbreak27 1Ki 8:66 2Ch 29:36; 30:21-\allowbreak26 Ne 8:10-\allowbreak12 Ps 100:1,\allowbreak2}
\crossref{Deut}{12}{13}{}
\crossref{Deut}{12}{14}{12:5,\allowbreak11 Ps 5:7; 9:11 2Co 5:19 Heb 10:19-\allowbreak22; 13:15}
\crossref{Deut}{12}{15}{De 14:26}
\crossref{Deut}{12}{16}{12:23,\allowbreak24; 15:23 Ge 9:4}
\crossref{Deut}{12}{17}{12:6,\allowbreak11; 14:22-\allowbreak29; 26:12,\allowbreak14 Le 27:30-\allowbreak32 Nu 18:21-\allowbreak24}
\crossref{Deut}{12}{18}{12:11,\allowbreak12,\allowbreak19; 14:23; 15:20}
\crossref{Deut}{12}{19}{De 14:27-\allowbreak29 2Ch 11:13,\allowbreak14; 31:4-\allowbreak21 Ne 10:34-\allowbreak39 1Co 9:10-\allowbreak14}
\crossref{Deut}{12}{20}{1Ch 4:10}
\crossref{Deut}{12}{21}{12:5,\allowbreak11; 14:23,\allowbreak24; 16:6,\allowbreak11; 26:2 Ex 20:24 1Ki 14:21 2Ch 12:13}
\crossref{Deut}{12}{22}{12:15,\allowbreak16}
\crossref{Deut}{12}{23}{Ge 9:4 Le 3:16,\allowbreak17; 17:11,\allowbreak14 Mt 20:28 Re 5:9}
\crossref{Deut}{12}{24}{12:16; 15:23}
\crossref{Deut}{12}{25}{12:28}
\crossref{Deut}{12}{26}{12:6,\allowbreak11,\allowbreak18 Nu 5:9,\allowbreak10; 18:19}
\crossref{Deut}{12}{27}{Le 1:5,\allowbreak9,\allowbreak13; 17:11}
\crossref{Deut}{12}{28}{De 24:8}
\crossref{Deut}{12}{29}{De 9:3; 19:1 Ex 23:23 Jos 23:4 Ps 78:55}
\crossref{Deut}{12}{30}{De 7:16 Ex 23:31-\allowbreak33 Le 18:3 Nu 33:52 Jud 2:2,\allowbreak3 2Ki 17:15}
\crossref{Deut}{12}{31}{12:4; 18:9 Ex 23:2 Le 18:3,\allowbreak26-\allowbreak30 2Ki 17:15-\allowbreak17; 21:2 2Ch 33:2}
\crossref{Deut}{12}{32}{De 4:2; 13:18 Jos 1:7 Pr 30:6 Mt 28:20 Re 22:18,\allowbreak19}
\crossref{Deut}{13}{1}{Jer 23:25-\allowbreak28; 27:9; 29:8,\allowbreak24}
\crossref{Deut}{13}{2}{De 18:22 Ex 7:22 1Ki 13:3 Jer 28:9 Mt 7:22,\allowbreak23; 24:24 2Co 11:13-\allowbreak15}
\crossref{Deut}{13}{3}{Isa 8:20 Ac 17:11 Eph 4:14 1Jo 4:1}
\crossref{Deut}{13}{4}{De 6:13 2Ki 23:3 2Ch 34:31 Mic 6:8 Lu 1:6 Col 1:10 1Th 4:1,\allowbreak2}
\crossref{Deut}{13}{5}{De 18:20 1Ki 18:40 Isa 9:14,\allowbreak15; 28:17,\allowbreak18 Jer 14:15; 28:15-\allowbreak17}
\crossref{Deut}{13}{6}{De 17:2,\allowbreak3; 28:54 Ge 16:5 Pr 5:20; 18:24 Mic 7:5-\allowbreak7 Mt 12:48-\allowbreak50}
\crossref{Deut}{13}{7}{}
\crossref{Deut}{13}{8}{Ex 20:3 Pr 1:10 Ga 1:8,\allowbreak9 1Jo 5:21}
\crossref{Deut}{13}{9}{De 17:2-\allowbreak7 Mt 10:37 Lu 14:26}
\crossref{Deut}{13}{10}{De 21:21 Le 20:2,\allowbreak27; 24:14-\allowbreak16,\allowbreak23 Nu 15:35,\allowbreak36 Jos 7:25 2Ch 24:21}
\crossref{Deut}{13}{11}{De 17:13; 19:20 Pr 19:25; 21:11 1Ti 5:20}
\crossref{Deut}{13}{12}{Jos 22:11-\allowbreak34 Jud 20:1,\allowbreak2-\allowbreak17}
\crossref{Deut}{13}{13}{Jud 19:22; 20:13 1Sa 2:12; 10:27; 25:17,\allowbreak25 2Sa 16:7; 20:1; 23:6}
\crossref{Deut}{13}{14}{De 17:4; 10:18 Nu 35:30 Isa 11:3,\allowbreak4 Joh 7:24 1Ti 5:19}
\crossref{Deut}{13}{15}{De 2:34; 7:2,\allowbreak16 Ex 22:20; 23:24 Le 27:28 Jos 6:17-\allowbreak21,\allowbreak24 Jud 20:48}
\crossref{Deut}{13}{16}{Jos 6:24}
\crossref{Deut}{13}{17}{De 7:26 Jos 6:18; 7:1}
\crossref{Deut}{13}{18}{De 12:25,\allowbreak28,\allowbreak32 Ps 119:6 Mt 6:33; 7:21,\allowbreak24}
\crossref{Deut}{14}{1}{Ge 6:2,\allowbreak4 Ex 4:22,\allowbreak23 Ps 82:6,\allowbreak7 Jer 3:19 Ho 1:10 Joh 1:12; 11:52}
\crossref{Deut}{14}{2}{14:21}
\crossref{Deut}{14}{3}{Le 11:43; 20:25 Isa 65:4 Eze 4:14 Ac 10:12-\allowbreak14 Ro 14:14}
\crossref{Deut}{14}{4}{Le 11:2-\allowbreak8 1Ki 4:23}
\crossref{Deut}{14}{5}{}
\crossref{Deut}{14}{6}{Ps 1:1,\allowbreak2 Pr 18:1 2Co 6:17}
\crossref{Deut}{14}{7}{Mt 7:22,\allowbreak23,\allowbreak26 2Ti 3:5 Tit 1:16 2Pe 2:18-\allowbreak22}
\crossref{Deut}{14}{8}{Isa 65:4; 66:3,\allowbreak17 Lu 15:15,\allowbreak16 2Pe 2:22}
\crossref{Deut}{14}{9}{Le 11:9-\allowbreak12}
\crossref{Deut}{14}{10}{14:10}
\crossref{Deut}{14}{11}{}
\crossref{Deut}{14}{12}{Le 11:13-\allowbreak19}
\crossref{Deut}{14}{13}{}
\crossref{Deut}{14}{14}{14:14}
\crossref{Deut}{14}{15}{Job 30:29}
\crossref{Deut}{14}{16}{}
\crossref{Deut}{14}{17}{}
\crossref{Deut}{14}{18}{}
\crossref{Deut}{14}{19}{Le 11:20-\allowbreak23 Php 3:19}
\crossref{Deut}{14}{20}{}
\crossref{Deut}{14}{21}{Le 17:15; 22:8 Eze 4:14 Ac 15:20}
\crossref{Deut}{14}{22}{De 12:6,\allowbreak17; 26:12-\allowbreak15 Le 27:30-\allowbreak33 Nu 18:21 Ne 10:37}
\crossref{Deut}{14}{23}{De 12:5-\allowbreak7,\allowbreak17,\allowbreak18}
\crossref{Deut}{14}{24}{De 11:24; 12:21 Ex 23:31}
\crossref{Deut}{14}{25}{}
\crossref{Deut}{14}{26}{Ezr 7:15-\allowbreak17,\allowbreak22 Mt 21:12 Mr 11:15 Joh 2:14-\allowbreak16}
\crossref{Deut}{14}{27}{14:29; 12:12,\allowbreak18,\allowbreak19 Ga 6:6 1Ti 5:17}
\crossref{Deut}{14}{28}{14:22; 26:12-\allowbreak15 Am 4:4}
\crossref{Deut}{14}{29}{14:27; 12:12}
\crossref{Deut}{15}{1}{}
\crossref{Deut}{15}{2}{}
\crossref{Deut}{15}{3}{De 23:20 Ex 22:25 Mt 17:25,\allowbreak26 Joh 8:35 1Co 6:6,\allowbreak7 Ga 6:10}
\crossref{Deut}{15}{4}{De 14:29; 28:1-\allowbreak8,\allowbreak11 Pr 11:24,\allowbreak25; 14:21; 28:27 Isa 58:10,\allowbreak11}
\crossref{Deut}{15}{5}{De 4:9; 11:13-\allowbreak15; 28:1-\allowbreak15 Le 26:3-\allowbreak14 Jos 1:7 Ps 19:11 Isa 1:19,\allowbreak20}
\crossref{Deut}{15}{6}{De 28:12,\allowbreak44 Ps 37:21,\allowbreak26; 112:5 Pr 22:7 Lu 6:35}
\crossref{Deut}{15}{7}{15:9 Pr 21:13 Mt 18:30 Jas 2:15,\allowbreak16 1Jo 3:16,\allowbreak17}
\crossref{Deut}{15}{8}{}
\crossref{Deut}{15}{9}{Pr 4:23 Jer 17:10 Mt 15:19 Mr 7:21,\allowbreak22 Ro 7:8,\allowbreak9 Jas 4:5}
\crossref{Deut}{15}{10}{Mt 25:40 Ac 20:35 Ro 12:8 2Co 9:5-\allowbreak7 1Ti 6:18,\allowbreak19 1Pe 4:11}
\crossref{Deut}{15}{11}{15:8 Mt 5:42 Lu 12:33 Ac 2:45; 4:32-\allowbreak35; 11:28-\allowbreak30 2Co 8:2-\allowbreak9}
\crossref{Deut}{15}{12}{15:1 Ex 21:2-\allowbreak6 Le 25:39-\allowbreak41 Jer 34:14 Joh 8:35,\allowbreak36}
\crossref{Deut}{15}{13}{}
\crossref{Deut}{15}{14}{Ne 8:10 Ps 68:10 Pr 10:22 Ac 20:35 1Co 16:2}
\crossref{Deut}{15}{15}{De 5:14,\allowbreak15; 16:12 Ex 20:2 Isa 51:1 Mt 6:14,\allowbreak15; 18:32,\allowbreak33 Eph 1:7}
\crossref{Deut}{15}{16}{Ex 21:5,\allowbreak6 Ps 40:6,\allowbreak8}
\crossref{Deut}{15}{17}{Le 25:39-\allowbreak42 1Sa 1:22}
\crossref{Deut}{15}{18}{15:10}
\crossref{Deut}{15}{19}{Ex 13:2,\allowbreak12; 34:19 Le 27:26 Nu 3:13; 18:17 Ro 8:29 Heb 12:23}
\crossref{Deut}{15}{20}{De 12:5-\allowbreak7,\allowbreak17}
\crossref{Deut}{15}{21}{De 17:1 Le 22:20,\allowbreak24 Mal 1:7,\allowbreak8}
\crossref{Deut}{15}{22}{De 12:15,\allowbreak21,\allowbreak22}
\crossref{Deut}{15}{23}{De 12:16,\allowbreak23 Le 7:26 1Sa 14:32 Eze 33:25}
\crossref{Deut}{16}{1}{Ex 12:2-\allowbreak20; 34:18 Le 23:5 Nu 9:2-\allowbreak5; 28:16}
\crossref{Deut}{16}{2}{Ex 12:5-\allowbreak7 Nu 28:16-\allowbreak19 2Ch 35:7 Mt 26:2,\allowbreak17 Mr 14:12 Lu 22:8,\allowbreak15}
\crossref{Deut}{16}{3}{Ex 12:15,\allowbreak19,\allowbreak20,\allowbreak39; 13:3-\allowbreak7; 34:18 Le 23:6 Nu 9:11; 28:17 1Co 5:8}
\crossref{Deut}{16}{4}{Ex 12:15; 13:7; 34:25}
\crossref{Deut}{16}{5}{16:2; 12:5,\allowbreak6}
\crossref{Deut}{16}{6}{Ex 12:6-\allowbreak9 Nu 9:3,\allowbreak11 Mt 26:20 Heb 1:2,\allowbreak3; 9:26 1Pe 1:19,\allowbreak20}
\crossref{Deut}{16}{7}{Ex 12:8,\allowbreak9 2Ch 35:13 Ps 22:14,\allowbreak15}
\crossref{Deut}{16}{8}{Ex 12:15,\allowbreak16; 13:7,\allowbreak8 Le 23:6-\allowbreak8 Nu 28:17-\allowbreak19}
\crossref{Deut}{16}{9}{16:10,\allowbreak16 Ex 23:16; 34:22 Le 23:15,\allowbreak16 Nu 28:26-\allowbreak30 2Ch 8:13 Ac 2:1}
\crossref{Deut}{16}{10}{16:16 Le 5:7; 12:8; 25:26}
\crossref{Deut}{16}{11}{16:14; 12:7,\allowbreak12,\allowbreak18 Isa 64:5; 66:10-\allowbreak14 Hab 3:18 Ro 5:11 2Co 1:24}
\crossref{Deut}{16}{12}{16:15; 15:15 La 3:19,\allowbreak20 Ro 6:17,\allowbreak18 Eph 2:1-\allowbreak3,\allowbreak11}
\crossref{Deut}{16}{13}{De 31:10 Ex 23:16; 34:22 Le 23:34-\allowbreak36 Nu 29:12-\allowbreak40 2Ch 5:3; 7:8-\allowbreak10}
\crossref{Deut}{16}{14}{De 12:12; 26:11 Ne 8:9-\allowbreak12 Ec 9:7 Isa 12:1-\allowbreak6; 25:6-\allowbreak8; 30:29; 35:10}
\crossref{Deut}{16}{15}{Le 23:36-\allowbreak42 Nu 29:12-\allowbreak38}
\crossref{Deut}{16}{16}{Ex 23:14-\allowbreak17; 34:22,\allowbreak23 1Ki 9:25}
\crossref{Deut}{16}{17}{}
\crossref{Deut}{16}{18}{De 1:15-\allowbreak17; 17:9,\allowbreak12; 19:17,\allowbreak18; 21:2 Ex 18:25,\allowbreak26; 21:6 1Ch 23:4; 26:29}
\crossref{Deut}{16}{19}{De 24:17; 27:19 Ex 23:2,\allowbreak6-\allowbreak8 Le 19:15 1Sa 8:3; 12:3 Job 31:21,\allowbreak22}
\crossref{Deut}{16}{20}{De 25:13-\allowbreak16 Mic 6:8 Php 4:8}
\crossref{Deut}{16}{21}{Ex 34:13 Jud 3:7 1Ki 14:15; 16:33 2Ki 17:16; 21:3 2Ch 33:3}
\crossref{Deut}{16}{22}{Ex 20:4 Le 26:1}
\crossref{Deut}{17}{1}{De 15:21}
\crossref{Deut}{17}{2}{17:5; 13:6-\allowbreak18; 29:18}
\crossref{Deut}{17}{3}{De 4:19 2Ki 21:3 Job 31:26,\allowbreak27 Jer 8:2 Eze 8:16}
\crossref{Deut}{17}{4}{De 13:12-\allowbreak14; 19:18 Pr 25:2 Joh 7:51}
\crossref{Deut}{17}{5}{De 13:10,\allowbreak11; 21:21; 22:21,\allowbreak24 Le 24:14,\allowbreak16 Jos 7:25}
\crossref{Deut}{17}{6}{De 19:15 Nu 35:30 Mt 18:16 Joh 8:17,\allowbreak18 2Co 13:1 1Ti 5:19}
\crossref{Deut}{17}{7}{De 13:9 Ac 7:58,\allowbreak59}
\crossref{Deut}{17}{8}{De 1:17 Ex 18:26 1Ki 3:16-\allowbreak28 2Ch 19:8-\allowbreak10 Hag 2:11 Mal 2:7}
\crossref{Deut}{17}{9}{Jer 18:18 Hag 2:11 Mal 2:7}
\crossref{Deut}{17}{10}{Mt 22:2,\allowbreak3}
\crossref{Deut}{17}{11}{Jos 1:7 Mal 2:8,\allowbreak9 Ro 13:1-\allowbreak6 Tit 3:1 1Pe 2:13-\allowbreak15 2Pe 2:10}
\crossref{Deut}{17}{12}{De 13:5,\allowbreak11 Nu 15:30 Ezr 10:8 Ps 19:13 Ho 4:4 Mt 10:14}
\crossref{Deut}{17}{13}{De 13:11; 19:20}
\crossref{Deut}{17}{14}{De 7:1; 12:9,\allowbreak10; 18:9; 26:1,\allowbreak9 Le 14:34 Jos 1:13}
\crossref{Deut}{17}{15}{1Sa 9:15-\allowbreak17; 10:24; 16:12,\allowbreak13 2Sa 5:2 1Ch 12:23; 22:10; 28:5}
\crossref{Deut}{17}{16}{Isa 31:1-\allowbreak3 Jer 42:14 Eze 17:15}
\crossref{Deut}{17}{17}{Ge 2:24 2Sa 3:2-\allowbreak5 1Ki 11:1-\allowbreak4 Ne 13:26 Mal 2:15 Mt 19:5}
\crossref{Deut}{17}{18}{2Ki 11:12}
\crossref{Deut}{17}{19}{De 6:6-\allowbreak9; 11:18 Jos 1:8 Ps 1:2; 119:97-\allowbreak100 Joh 5:39 2Ti 3:15-\allowbreak17}
\crossref{Deut}{17}{20}{De 8:2,\allowbreak13,\allowbreak14 2Ki 14:10 2Ch 25:19; 26:16; 32:25,\allowbreak26; 33:12,\allowbreak19,\allowbreak23; 34:27}
\crossref{Deut}{18}{1}{De 10:9; 12:19 Nu 18:20; 26:62 Jos 13:33; 18:7 1Pe 5:2-\allowbreak4}
\crossref{Deut}{18}{2}{Ge 15:1 Ps 16:5; 73:24-\allowbreak26; 84:11; 119:57 Isa 61:6 La 3:24}
\crossref{Deut}{18}{3}{De 12:27 Le 7:30-\allowbreak34}
\crossref{Deut}{18}{4}{De 26:9,\allowbreak10 Ex 22:29; 23:19 Le 23:10,\allowbreak17 Nu 18:12-\allowbreak24 2Ch 31:4-\allowbreak10}
\crossref{Deut}{18}{5}{De 10:8; 17:12 Ex 28:1-\allowbreak14 Nu 3:10; 16:5,\allowbreak9,\allowbreak10; 17:5-\allowbreak9; 25:13}
\crossref{Deut}{18}{6}{Nu 35:2,\allowbreak3}
\crossref{Deut}{18}{7}{2Ch 31:2-\allowbreak4}
\crossref{Deut}{18}{8}{Le 7:8,\allowbreak9,\allowbreak14 Ne 12:44,\allowbreak47 Lu 10:7 1Co 9:7-\allowbreak14 1Ti 5:17,\allowbreak18}
\crossref{Deut}{18}{9}{De 12:29-\allowbreak31 Le 18:26,\allowbreak27,\allowbreak30}
\crossref{Deut}{18}{10}{De 12:31 Le 18:21-\allowbreak30; 20:2-\allowbreak5 2Ki 16:3; 17:17; 21:6 2Ch 28:3}
\crossref{Deut}{18}{11}{1Sa 28:11-\allowbreak14}
\crossref{Deut}{18}{12}{De 9:4 Le 18:24,\allowbreak27}
\crossref{Deut}{18}{13}{Ge 6:9; 17:1 Job 1:1,\allowbreak8 Ps 37:37 Mt 5:48 Php 3:12,\allowbreak15 Re 3:2}
\crossref{Deut}{18}{14}{18:10 Ge 20:6 Ps 147:19,\allowbreak20 Ac 14:16}
\crossref{Deut}{18}{15}{18:18,\allowbreak19 Joh 1:45 Ac 3:22,\allowbreak23; 7:37}
\crossref{Deut}{18}{16}{De 9:10}
\crossref{Deut}{18}{17}{De 5:28}
\crossref{Deut}{18}{18}{18:15 Joh 1:45}
\crossref{Deut}{18}{19}{Mr 16:16 Ac 3:22,\allowbreak23 Heb 2:3; 3:7; 10:26; 12:25,\allowbreak26}
\crossref{Deut}{18}{20}{De 13:1-\allowbreak5 Jer 14:14,\allowbreak15; 23:13-\allowbreak15,\allowbreak31; 27:15 Eze 13:6 Mt 7:15}
\crossref{Deut}{18}{21}{1Th 5:24 1Jo 4:1-\allowbreak3 Re 2:2}
\crossref{Deut}{18}{22}{Isa 41:22 Jer 28:1-\allowbreak14}
\crossref{Deut}{19}{1}{De 6:10; 7:1,\allowbreak2; 12:1,\allowbreak29; 17:14}
\crossref{Deut}{19}{2}{}
\crossref{Deut}{19}{3}{Isa 35:8; 57:14; 62:10 Heb 12:13}
\crossref{Deut}{19}{4}{De 4:42 Nu 35:15-\allowbreak24}
\crossref{Deut}{19}{5}{2Ki 6:5-\allowbreak7}
\crossref{Deut}{19}{6}{Nu 35:12 Jos 20:5 2Sa 14:7}
\crossref{Deut}{19}{7}{19:7}
\crossref{Deut}{19}{8}{De 11:24,\allowbreak25; 12:20 Ge 15:18-\allowbreak21; 28:14 Ex 23:31; 34:24 1Ki 4:21}
\crossref{Deut}{19}{9}{De 11:22-\allowbreak25; 12:32}
\crossref{Deut}{19}{10}{19:13; 21:8 1Ki 2:31 2Ki 21:16; 24:4 Ps 94:21 Pr 6:17 Isa 59:7}
\crossref{Deut}{19}{11}{De 27:24 Ge 9:6 Ex 21:12-\allowbreak14 Nu 35:16-\allowbreak21,\allowbreak24 Pr 28:17}
\crossref{Deut}{19}{12}{1Ki 2:5,\allowbreak6,\allowbreak28-\allowbreak34}
\crossref{Deut}{19}{13}{De 7:16; 13:8; 25:12 Eze 16:5}
\crossref{Deut}{19}{14}{}
\crossref{Deut}{19}{15}{De 17:6 Nu 35:30 1Ki 21:10,\allowbreak13 Mt 18:16; 26:60,\allowbreak61 Joh 8:17}
\crossref{Deut}{19}{16}{Ex 23:1-\allowbreak7 1Ki 21:10-\allowbreak13 Ps 27:12; 35:11 Mr 14:55-\allowbreak59 Ac 6:13}
\crossref{Deut}{19}{17}{De 17:9; 21:5 Mal 2:7 Mt 23:2,\allowbreak3}
\crossref{Deut}{19}{18}{De 13:14; 17:4 2Ch 19:6,\allowbreak7 Job 19:16}
\crossref{Deut}{19}{19}{Pr 19:5,\allowbreak9 Jer 14:15 Da 6:24}
\crossref{Deut}{19}{20}{De 13:11; 17:7,\allowbreak13; 21:21 Pr 21:11 Ro 13:3,\allowbreak4 1Ti 5:20}
\crossref{Deut}{19}{21}{19:13}
\crossref{Deut}{20}{1}{De 3:21,\allowbreak22; 7:1}
\crossref{Deut}{20}{2}{Nu 10:8,\allowbreak9; 31:6 Jud 20:27,\allowbreak28 1Sa 14:18; 30:7,\allowbreak8 2Ch 13:12}
\crossref{Deut}{20}{3}{Ps 27:1-\allowbreak3 Isa 35:3,\allowbreak4; 41:10-\allowbreak14 Mt 10:16,\allowbreak28,\allowbreak31 Eph 6:11-\allowbreak18}
\crossref{Deut}{20}{4}{De 1:30; 3:22; 11:25; 32:30 Ex 14:14 Jos 10:42; 23:10 2Ch 13:12}
\crossref{Deut}{20}{5}{De 1:15; 16:18 Nu 31:14,\allowbreak48 1Sa 17:18}
\crossref{Deut}{20}{6}{De 28:1-\allowbreak30:20 Le 19:23-\allowbreak25 Jer 31:5}
\crossref{Deut}{20}{7}{De 28:30 Lu 14:18-\allowbreak20 2Ti 2:4}
\crossref{Deut}{20}{8}{De 1:28; 23:9 Jud 7:3 Lu 9:62 Ac 15:37,\allowbreak38 Re 3:16; 21:8}
\crossref{Deut}{20}{9}{20:9}
\crossref{Deut}{20}{10}{2Sa 20:18-\allowbreak22 Isa 57:19 Zec 9:10 Lu 10:5,\allowbreak6 Ac 10:36}
\crossref{Deut}{20}{11}{Le 25:42-\allowbreak46 Jos 9:22,\allowbreak23,\allowbreak27; 11:19,\allowbreak20; 16:10 Jud 1:28,\allowbreak30-\allowbreak35}
\crossref{Deut}{20}{12}{}
\crossref{Deut}{20}{13}{Nu 31:7-\allowbreak9,\allowbreak17,\allowbreak18 1Ki 11:15,\allowbreak16 Ps 2:6-\allowbreak12; 21:8,\allowbreak9; 110:1 Lu 19:27}
\crossref{Deut}{20}{14}{Nu 31:9,\allowbreak12,\allowbreak18,\allowbreak35-\allowbreak54 Jos 8:2; 11:14 2Ch 14:13-\allowbreak15; 20:25 Ps 68:12}
\crossref{Deut}{20}{15}{20:15}
\crossref{Deut}{20}{16}{De 7:1-\allowbreak4,\allowbreak16 Nu 21:2,\allowbreak3,\allowbreak35; 33:52 Jos 6:17-\allowbreak21; 9:24,\allowbreak27; 10:28,\allowbreak40}
\crossref{Deut}{20}{17}{Isa 34:5,\allowbreak6 Jer 48:10; 50:35-\allowbreak40 Eze 38:21-\allowbreak23 Re 19:18}
\crossref{Deut}{20}{18}{De 7:4,\allowbreak5; 12:30,\allowbreak31; 18:19 Ex 23:33 Jos 23:13 Jud 2:3 Ps 106:34-\allowbreak40}
\crossref{Deut}{20}{19}{Mt 3:10; 7:15-\allowbreak20; 21:19 Lu 13:7-\allowbreak9 Joh 15:2-\allowbreak8}
\crossref{Deut}{20}{20}{De 1:28 2Ch 26:15 Ec 9:14 Isa 37:33 Jer 6:6; 33:4 Eze 17:17}
\crossref{Deut}{21}{1}{Ps 5:6; 9:12 Pr 28:17 Isa 26:21 Ac 28:4}
\crossref{Deut}{21}{2}{De 16:18,\allowbreak19 Ro 13:3,\allowbreak4}
\crossref{Deut}{21}{3}{Nu 19:2 Jer 31:18 Mt 11:28-\allowbreak30 Php 2:8}
\crossref{Deut}{21}{4}{1Pe 2:21-\allowbreak24; 3:18}
\crossref{Deut}{21}{5}{De 10:8; 18:5 Nu 6:22-\allowbreak27 1Ch 23:13}
\crossref{Deut}{21}{6}{}
\crossref{Deut}{21}{7}{Nu 5:19-\allowbreak28 2Sa 16:8 Job 21:21-\allowbreak23,\allowbreak31-\allowbreak34 Ps 7:3,\allowbreak4}
\crossref{Deut}{21}{8}{Nu 35:33 2Sa 3:28 2Ki 24:4 Ps 19:12 Jer 26:15 Eze 23:3,\allowbreak24,\allowbreak25}
\crossref{Deut}{21}{9}{De 19:12,\allowbreak13}
\crossref{Deut}{21}{10}{De 20:10-\allowbreak16}
\crossref{Deut}{21}{11}{Ge 6:2; 12:14,\allowbreak15; 29:18-\allowbreak20; 34:3,\allowbreak8 Jud 14:2,\allowbreak3 Pr 6:25; 31:10,\allowbreak30}
\crossref{Deut}{21}{12}{}
\crossref{Deut}{21}{13}{Ps 45:10,\allowbreak11 Lu 14:26,\allowbreak27}
\crossref{Deut}{21}{14}{Ex 21:7-\allowbreak11}
\crossref{Deut}{21}{15}{Ge 29:18,\allowbreak20,\allowbreak30,\allowbreak31,\allowbreak33 1Sa 1:4,\allowbreak5}
\crossref{Deut}{21}{16}{1Ch 5:2; 26:10 2Ch 11:19-\allowbreak22; 21:3 Ro 8:29 Php 4:8 Heb 12:16,\allowbreak17}
\crossref{Deut}{21}{17}{Ge 25:5,\allowbreak6,\allowbreak32,\allowbreak34 1Ch 5:1,\allowbreak2}
\crossref{Deut}{21}{18}{Pr 28:24; 30:11,\allowbreak17 Isa 1:2}
\crossref{Deut}{21}{19}{21:2; 16:18; 25:7 Zec 13:3}
\crossref{Deut}{21}{20}{Pr 29:17}
\crossref{Deut}{21}{21}{De 13:10,\allowbreak11; 17:5 Le 24:16}
\crossref{Deut}{21}{22}{Jos 8:29; 10:26}
\crossref{Deut}{21}{23}{Le 18:25 Nu 35:33,\allowbreak34}
\crossref{Deut}{22}{1}{Ex 23:4 Eze 34:4,\allowbreak16 Mt 10:6; 15:24; 18:12,\allowbreak13 Lu 15:4-\allowbreak6}
\crossref{Deut}{22}{2}{Mt 7:12 1Th 4:6}
\crossref{Deut}{22}{3}{}
\crossref{Deut}{22}{4}{Ex 23:4,\allowbreak5 Mt 5:44 Lu 10:29-\allowbreak37 Ro 15:1 2Co 12:15 Ga 6:1,\allowbreak2}
\crossref{Deut}{22}{5}{1Co 11:4-\allowbreak15}
\crossref{Deut}{22}{6}{Lu 12:6}
\crossref{Deut}{22}{7}{De 4:40}
\crossref{Deut}{22}{8}{2Sa 11:2 Isa 22:1 Jer 19:13 Mt 10:27 Mr 2:4 Ac 10:9}
\crossref{Deut}{22}{9}{Le 19:19 Mt 6:24; 9:16 Ro 11:6 2Co 1:12; 11:3 Jas 1:6-\allowbreak8; 3:10}
\crossref{Deut}{22}{10}{}
\crossref{Deut}{22}{11}{Le 19:19}
\crossref{Deut}{22}{12}{Nu 15:38,\allowbreak39 Mt 23:5}
\crossref{Deut}{22}{13}{Ge 29:21,\allowbreak23,\allowbreak31 Jud 15:1,\allowbreak2 Eph 5:28,\allowbreak29}
\crossref{Deut}{22}{14}{22:19 Ex 20:16; 23:1 Pr 18:8,\allowbreak21 1Ti 5:14}
\crossref{Deut}{22}{15}{22:15}
\crossref{Deut}{22}{16}{22:16}
\crossref{Deut}{22}{17}{22:17}
\crossref{Deut}{22}{18}{}
\crossref{Deut}{22}{19}{22:29; 24:1-\allowbreak4 Mt 19:8,\allowbreak9}
\crossref{Deut}{22}{20}{}
\crossref{Deut}{22}{21}{22:22,\allowbreak24; 13:10; 17:5; 21:21 Le 24:16,\allowbreak23 Nu 15:35,\allowbreak36}
\crossref{Deut}{22}{22}{Le 20:10 Nu 5:22-\allowbreak27 Eze 23:45-\allowbreak47 Joh 8:4,\allowbreak5 Heb 13:4}
\crossref{Deut}{22}{23}{De 20:7 Mt 1:18,\allowbreak19}
\crossref{Deut}{22}{24}{De 21:14 Ge 29:21 Mt 1:20,\allowbreak24}
\crossref{Deut}{22}{25}{2Sa 13:14}
\crossref{Deut}{22}{26}{De 21:22}
\crossref{Deut}{22}{27}{1Co 13:7}
\crossref{Deut}{22}{28}{Ex 22:16,\allowbreak17}
\crossref{Deut}{22}{29}{22:19,\allowbreak24; 21:14}
\crossref{Deut}{22}{30}{Ru 3:9 Eze 16:8}
\crossref{Deut}{23}{1}{Le 21:17-\allowbreak21; 22:22-\allowbreak24 Ga 3:28}
\crossref{Deut}{23}{2}{Isa 57:3 Zec 9:6 Joh 8:41 Heb 12:8}
\crossref{Deut}{23}{3}{}
\crossref{Deut}{23}{4}{De 2:28,\allowbreak29 Ge 14:17,\allowbreak18 1Sa 25:11 1Ki 18:4 Isa 63:9 Zec 2:8}
\crossref{Deut}{23}{5}{Nu 22:35; 23:5-\allowbreak12,\allowbreak16-\allowbreak26; 24:9 Mic 6:5 Ro 8:31 2Co 4:17}
\crossref{Deut}{23}{6}{2Sa 8:2; 12:31 Ezr 9:12 Ne 13:23-\allowbreak25}
\crossref{Deut}{23}{7}{Ge 25:24-\allowbreak26,\allowbreak30 Nu 20:14 Ob 1:10-\allowbreak12 Mal 1:2}
\crossref{Deut}{23}{8}{23:1 Ro 3:29,\allowbreak30 Eph 2:12,\allowbreak13}
\crossref{Deut}{23}{9}{Jos 6:18; 7:11-\allowbreak13 Jud 20:26 2Ch 19:4; 20:3-\allowbreak13; 31:20,\allowbreak21; 32:1-\allowbreak22}
\crossref{Deut}{23}{10}{Le 15:16 Nu 5:2,\allowbreak3 1Co 5:11-\allowbreak13}
\crossref{Deut}{23}{11}{Le 11:25; 15:17-\allowbreak23}
\crossref{Deut}{23}{12}{23:12}
\crossref{Deut}{23}{13}{Eze 24:6-\allowbreak8}
\crossref{Deut}{23}{14}{Ge 17:1 Le 26:12 2Co 6:16}
\crossref{Deut}{23}{15}{}
\crossref{Deut}{23}{16}{Isa 16:3,\allowbreak4 Lu 15:15-\allowbreak24 Tit 3:2,\allowbreak3}
\crossref{Deut}{23}{17}{Ro 1:26}
\crossref{Deut}{23}{18}{Eze 16:33}
\crossref{Deut}{23}{19}{Ex 22:25 Le 25:35-\allowbreak37 Ne 5:1-\allowbreak7 Ps 15:5 Eze 18:7,\allowbreak8,\allowbreak13,\allowbreak16-\allowbreak18}
\crossref{Deut}{23}{20}{De 14:21; 15:3 Le 19:33,\allowbreak34}
\crossref{Deut}{23}{21}{23:18 Ge 28:20; 35:1-\allowbreak3 Le 27:2-\allowbreak34 Nu 30:2-\allowbreak16 Ps 56:12; 66:13,\allowbreak14}
\crossref{Deut}{23}{22}{}
\crossref{Deut}{23}{23}{Nu 30:2 Jud 11:30,\allowbreak31,\allowbreak35 1Sa 1:11 Ps 66:13,\allowbreak14; 116:18 Pr 20:25}
\crossref{Deut}{23}{24}{Ro 12:13 1Co 10:26 Heb 13:5}
\crossref{Deut}{23}{25}{Mt 12:1,\allowbreak2 Mr 2:23 Lu 6:1,\allowbreak2}
\crossref{Deut}{24}{1}{De 21:15; 22:13 Ex 21:10}
\crossref{Deut}{24}{2}{Le 21:7,\allowbreak14; 22:13 Nu 30:9 Eze 44:22 Mt 5:32 Mr 10:11 1Co 7:15}
\crossref{Deut}{24}{3}{}
\crossref{Deut}{24}{4}{Jer 3:1}
\crossref{Deut}{24}{5}{De 20:7 Ge 2:24 Mt 19:4-\allowbreak6 Mr 10:6-\allowbreak9 1Co 7:10-\allowbreak15 Eph 5:28,\allowbreak29}
\crossref{Deut}{24}{6}{De 20:19 Ge 44:30 Lu 12:15}
\crossref{Deut}{24}{7}{Ex 21:16 Eze 27:13 1Ti 1:10 Re 18:13}
\crossref{Deut}{24}{8}{Le 13:1-\allowbreak14:57 Mt 8:4 Mr 1:44 Lu 5:14; 17:14}
\crossref{Deut}{24}{9}{Lu 17:32 1Co 10:6,\allowbreak11}
\crossref{Deut}{24}{10}{De 15:8}
\crossref{Deut}{24}{11}{24:11}
\crossref{Deut}{24}{12}{24:17 Job 22:6; 24:3,\allowbreak9}
\crossref{Deut}{24}{13}{Ex 22:26,\allowbreak27 Job 24:7,\allowbreak8; 29:11-\allowbreak13; 31:16-\allowbreak20 Eze 18:7,\allowbreak12,\allowbreak16; 33:15}
\crossref{Deut}{24}{14}{Le 25:40-\allowbreak43 Job 24:10,\allowbreak11; 31:13-\allowbreak15 Pr 14:31; 22:16 Eze 22:7}
\crossref{Deut}{24}{15}{Le 19:13 Pr 3:27,\allowbreak28 Jer 22:13 Mt 20:8 Mr 10:19 Jas 5:4}
\crossref{Deut}{24}{16}{2Ki 14:5,\allowbreak6 2Ch 25:4 Jer 31:29,\allowbreak30 Eze 18:20}
\crossref{Deut}{24}{17}{De 16:19; 27:19 Ex 22:21,\allowbreak22; 23:2,\allowbreak6,\allowbreak9 1Sa 12:3,\allowbreak4 Job 22:8,\allowbreak9}
\crossref{Deut}{24}{18}{24:22; 5:15; 15:15; 16:12}
\crossref{Deut}{24}{19}{Le 19:9,\allowbreak10; 23:22 Ru 2:16 Ps 41:1}
\crossref{Deut}{24}{20}{24:20}
\crossref{Deut}{24}{21}{24:19 Le 19:9,\allowbreak10}
\crossref{Deut}{24}{22}{24:18}
\crossref{Deut}{25}{1}{De 16:18-\allowbreak20; 17:8,\allowbreak9; 19:17-\allowbreak19 Ex 23:6,\allowbreak7 2Sa 23:3 2Ch 19:6-\allowbreak10}
\crossref{Deut}{25}{2}{Mt 10:17; 27:26 Lu 12:47,\allowbreak48 Ac 5:40; 16:22-\allowbreak24 1Pe 2:20,\allowbreak24}
\crossref{Deut}{25}{3}{2Co 11:24,\allowbreak25}
\crossref{Deut}{25}{4}{Pr 12:10 1Co 9:9,\allowbreak10 1Ti 5:17,\allowbreak18}
\crossref{Deut}{25}{5}{Mt 22:24 Mr 12:19 Lu 20:28}
\crossref{Deut}{25}{6}{Ge 28:8-\allowbreak10}
\crossref{Deut}{25}{7}{}
\crossref{Deut}{25}{8}{Ru 4:6}
\crossref{Deut}{25}{9}{Nu 12:14 Job 30:10 Isa 50:6 Mt 26:67; 27:30 Mr 10:34}
\crossref{Deut}{25}{10}{}
\crossref{Deut}{25}{11}{Ro 3:8 1Ti 2:9}
\crossref{Deut}{25}{12}{De 19:13,\allowbreak21}
\crossref{Deut}{25}{13}{Le 19:35,\allowbreak36 Pr 11:1; 16:11; 20:10 Eze 45:10,\allowbreak11 Am 8:5}
\crossref{Deut}{25}{14}{}
\crossref{Deut}{25}{15}{De 4:40; 5:16,\allowbreak33; 6:18; 11:9; 17:20 Ex 20:12 Ps 34:12 Eph 6:3}
\crossref{Deut}{25}{16}{De 18:12; 22:5 Pr 11:1; 20:23 Am 8:5-\allowbreak7 1Co 6:9-\allowbreak11 1Th 4:6 Re 21:27}
\crossref{Deut}{25}{17}{Ex 17:8-\allowbreak16 Nu 24:20; 25:17,\allowbreak18}
\crossref{Deut}{25}{18}{Ne 5:9,\allowbreak15 Ps 36:1 Pr 16:6 Ro 3:18}
\crossref{Deut}{25}{19}{Jos 23:1}
\crossref{Deut}{26}{1}{De 5:31; 6:1-\allowbreak10; 7:1; 13:1,\allowbreak9; 17:14; 18:9 Nu 15:2,\allowbreak18}
\crossref{Deut}{26}{2}{De 16:10; 18:4 Ex 23:16,\allowbreak19; 34:26 Le 2:12,\allowbreak14 Nu 18:12,\allowbreak13 2Ki 4:42}
\crossref{Deut}{26}{3}{De 19:17 Heb 7:26; 10:21; 13:15 1Pe 2:5}
\crossref{Deut}{26}{4}{Mt 5:23,\allowbreak24; 23:19 Heb 13:10-\allowbreak12}
\crossref{Deut}{26}{5}{Ge 27:41; 31:40; 43:1,\allowbreak2,\allowbreak12; 45:7,\allowbreak11 Isa 51:1,\allowbreak2}
\crossref{Deut}{26}{6}{De 4:20 Ex 1:11,\allowbreak14,\allowbreak16,\allowbreak22; 5:9,\allowbreak19,\allowbreak23}
\crossref{Deut}{26}{7}{Ex 2:23-\allowbreak25; 3:1-\allowbreak4:31 6:5 Ps 50:15; 103:1,\allowbreak2; 116:1-\allowbreak4 Jer 33:2}
\crossref{Deut}{26}{8}{De 4:34; 5:15 Ex 12:37,\allowbreak41,\allowbreak51; 13:3; 14:16-\allowbreak31 Ps 78:12,\allowbreak13; 105:27-\allowbreak38}
\crossref{Deut}{26}{9}{Jos 23:14 1Sa 7:12 Ps 105:44; 107:7,\allowbreak8 Ac 26:22}
\crossref{Deut}{26}{10}{26:2; 26:17 1Ch 29:14 Ro 2:1 1Pe 4:10,\allowbreak11}
\crossref{Deut}{26}{11}{De 12:7,\allowbreak12,\allowbreak18; 16:11; 28:47 Ps 63:3-\allowbreak5; 100:1,\allowbreak2 Isa 65:14 Zec 9:17}
\crossref{Deut}{26}{12}{Le 27:30 Nu 18:24}
\crossref{Deut}{26}{13}{26:12; 14:29; 24:19-\allowbreak21 Job 31:16-\allowbreak20}
\crossref{Deut}{26}{14}{De 16:11 Le 7:20; 21:1,\allowbreak11 Ho 9:4 Mal 2:13}
\crossref{Deut}{26}{15}{26:7 1Ki 8:27,\allowbreak43 Ps 102:19,\allowbreak20 Isa 57:15; 61:1; 63:15; 66:1,\allowbreak2}
\crossref{Deut}{26}{16}{De 4:1-\allowbreak6; 6:1; 11:1,\allowbreak8; 12:1,\allowbreak32 Mt 28:20}
\crossref{Deut}{26}{17}{De 5:2,\allowbreak3 Ex 15:2; 20:19; 24:7 2Ch 34:31 Isa 12:2; 44:5 Zec 13:9}
\crossref{Deut}{26}{18}{De 7:6; 14:2; 28:9 Ex 6:7; 19:5,\allowbreak6 Jer 31:32-\allowbreak34 Eze 36:25-\allowbreak27}
\crossref{Deut}{26}{19}{De 4:7,\allowbreak8; 28:1 Ps 148:14 Isa 62:12; 66:20,\allowbreak21 Jer 13:11; 33:9}
\crossref{Deut}{27}{1}{De 4:1-\allowbreak3; 11:32; 26:16 Lu 11:28 Joh 15:14 1Th 4:1,\allowbreak2 Jas 2:10}
\crossref{Deut}{27}{2}{De 6:1; 9:1; 11:31 Jos 1:11; 4:1,\allowbreak5-\allowbreak24}
\crossref{Deut}{27}{3}{Jos 8:32 Jer 31:31-\allowbreak33 2Co 3:2,\allowbreak3 Heb 8:6-\allowbreak10; 10:16}
\crossref{Deut}{27}{4}{}
\crossref{Deut}{27}{5}{Ex 24:4 Jos 8:30,\allowbreak31 1Ki 18:31,\allowbreak32}
\crossref{Deut}{27}{6}{Le 1:1-\allowbreak17 Eph 5:2}
\crossref{Deut}{27}{7}{Le 3:1-\allowbreak17; 7:11-\allowbreak17 Ac 10:36 Ro 5:1,\allowbreak10 Eph 2:16,\allowbreak17; 2:16,\allowbreak17}
\crossref{Deut}{27}{8}{27:3}
\crossref{Deut}{27}{9}{De 26:16-\allowbreak18 Ro 6:17,\allowbreak18,\allowbreak22 1Co 6:9-\allowbreak11 Eph 5:8,\allowbreak9 1Pe 2:10,\allowbreak11}
\crossref{Deut}{27}{10}{De 10:12,\allowbreak13; 11:1,\allowbreak7,\allowbreak8 Le 19:2 Mic 4:5; 6:8 Mt 5:48 Eph 4:17-\allowbreak24}
\crossref{Deut}{27}{11}{}
\crossref{Deut}{27}{12}{Ge 29:33-\allowbreak35; 30:18,\allowbreak24; 35:18}
\crossref{Deut}{27}{13}{27:4; 11:29 Jos 8:33}
\crossref{Deut}{27}{14}{De 33:9,\allowbreak10 Jos 8:33 Ne 8:7,\allowbreak8 Da 9:11 Mal 2:7-\allowbreak9}
\crossref{Deut}{27}{15}{De 28:16-\allowbreak19 Ge 9:25 1Sa 26:19 Jer 11:3}
\crossref{Deut}{27}{16}{De 21:18-\allowbreak21 Ex 20:12; 21:17 Le 19:3 Pr 30:11-\allowbreak17 Eze 22:7 Mt 15:4-\allowbreak6}
\crossref{Deut}{27}{17}{De 19:14 Pr 22:28; 23:10,\allowbreak11}
\crossref{Deut}{27}{18}{Le 19:14 Job 29:15 Pr 28:10 Isa 56:10 Mt 15:14 Re 2:14}
\crossref{Deut}{27}{19}{De 10:18; 24:17 Ex 22:21-\allowbreak24; 23:2,\allowbreak8,\allowbreak9 Ps 82:2-\allowbreak4 Pr 17:23; 31:5}
\crossref{Deut}{27}{20}{De 22:30 Ge 35:22; 49:4 Le 18:8; 20:11 2Sa 16:22 1Ch 5:1 Eze 22:10}
\crossref{Deut}{27}{21}{Ex 22:19 Le 18:23; 20:15}
\crossref{Deut}{27}{22}{Le 18:9; 20:17 2Sa 13:1,\allowbreak8-\allowbreak14 Eze 22:11}
\crossref{Deut}{27}{23}{Le 18:17; 20:14}
\crossref{Deut}{27}{24}{De 19:11,\allowbreak12 Ex 20:13; 21:12-\allowbreak14 Le 24:17 Nu 35:31 2Sa 3:27; 11:15-\allowbreak17}
\crossref{Deut}{27}{25}{De 10:17; 16:19 Ex 23:7,\allowbreak8 Ps 15:5 Pr 1:11-\allowbreak29 Eze 22:12,\allowbreak13}
\crossref{Deut}{27}{26}{27:15; 28:15-\allowbreak68 Ps 119:21 Mt 25:41 1Co 16:22}
\crossref{Deut}{28}{1}{De 11:13; 15:5; 27:1 Ex 15:26 Le 26:3-\allowbreak13 Ps 106:3; 111:10 Isa 1:19}
\crossref{Deut}{28}{2}{28:15,\allowbreak45 Zec 1:6 1Ti 4:8}
\crossref{Deut}{28}{3}{Ps 107:36,\allowbreak37; 128:1-\allowbreak5; 144:12-\allowbreak15 Isa 65:21-\allowbreak23 Zec 8:3-\allowbreak5}
\crossref{Deut}{28}{4}{28:11; 7:13 Ge 22:17; 49:25 Le 26:9 Ps 107:38; 127:3; 128:3 Pr 10:22}
\crossref{Deut}{28}{5}{28:5}
\crossref{Deut}{28}{6}{De 31:2 Nu 27:17 2Sa 3:25 2Ch 1:10 Ps 121:8}
\crossref{Deut}{28}{7}{28:25; 32:30 Le 26:7,\allowbreak8 2Sa 22:38-\allowbreak41 Ps 89:23}
\crossref{Deut}{28}{8}{Le 25:21 Ps 42:8; 44:4; 133:3}
\crossref{Deut}{28}{9}{De 7:6; 26:18,\allowbreak19; 29:13 Ge 17:7 Ex 19:5,\allowbreak6 Ps 87:5 Isa 1:26; 62:12}
\crossref{Deut}{28}{10}{Mal 3:12}
\crossref{Deut}{28}{11}{28:4; 30:9 Le 26:9 Pr 10:22}
\crossref{Deut}{28}{12}{De 11:14 Le 26:4 Job 38:22 Ps 65:9-\allowbreak13; 135:7 Joe 2:23,\allowbreak24}
\crossref{Deut}{28}{13}{Nu 24:18,\allowbreak19 Isa 9:14,\allowbreak15}
\crossref{Deut}{28}{14}{De 5:32; 11:16,\allowbreak26-\allowbreak28 Jos 23:6 2Ki 22:2 Pr 4:26,\allowbreak27}
\crossref{Deut}{28}{15}{Le 26:14-\allowbreak46 La 2:17 Da 9:11-\allowbreak13 Mal 2:2 Ro 2:8,\allowbreak9}
\crossref{Deut}{28}{16}{28:3-\allowbreak14 Pr 3:33 Isa 24:6-\allowbreak12; 43:28 Jer 9:11; 26:6; 44:22 La 1:1}
\crossref{Deut}{28}{17}{28:5 Ps 69:22 Pr 1:32 Hag 1:6 Zec 5:3,\allowbreak4 Mal 2:2 Lu 16:25}
\crossref{Deut}{28}{18}{28:4; 5:9 Job 18:16-\allowbreak19 Ps 109:9-\allowbreak15 La 2:11,\allowbreak12,\allowbreak20 Ho 9:11-\allowbreak14}
\crossref{Deut}{28}{19}{28:6 Jud 5:6,\allowbreak7 2Ch 15:5}
\crossref{Deut}{28}{20}{Ps 7:11 Mal 2:2}
\crossref{Deut}{28}{21}{Ex 5:3 Le 26:25 Nu 14:12; 16:46-\allowbreak49; 25:9 2Sa 24:15 Jer 15:2; 16:4}
\crossref{Deut}{28}{22}{Le 26:16 2Ch 6:28 Jer 14:12}
\crossref{Deut}{28}{23}{}
\crossref{Deut}{28}{24}{}
\crossref{Deut}{28}{25}{28:7; 32:30 Le 26:17,\allowbreak36,\allowbreak37 Isa 30:17}
\crossref{Deut}{28}{26}{1Sa 17:44-\allowbreak46 Ps 79:1-\allowbreak3 Isa 34:3 Jer 7:33; 8:1; 16:4; 19:7; 34:20}
\crossref{Deut}{28}{27}{28:35 Ex 9:9,\allowbreak11; 15:26}
\crossref{Deut}{28}{28}{1Sa 16:14 Ps 60:3 Isa 6:9,\allowbreak10; 19:11-\allowbreak17; 43:19 Jer 4:9 Eze 4:17}
\crossref{Deut}{28}{29}{Job 5:14; 12:25 Ps 69:23,\allowbreak24 Isa 59:10 La 5:17 Zep 1:17}
\crossref{Deut}{28}{30}{De 20:6,\allowbreak7 Job 31:10 Jer 8:10 Ho 4:2}
\crossref{Deut}{28}{31}{Jud 6:1 Job 1:14,\allowbreak15}
\crossref{Deut}{28}{32}{28:65 Job 11:20; 17:5 Ps 69:3; 119:82,\allowbreak123 Isa 38:14 La 2:11; 4:17}
\crossref{Deut}{28}{33}{28:30,\allowbreak51 Le 26:16 Ne 9:36,\allowbreak37 Isa 1:7 Jer 5:17; 8:16}
\crossref{Deut}{28}{34}{28:28,\allowbreak68 Isa 33:14 Jer 25:15,\allowbreak16 Re 16:10,\allowbreak11}
\crossref{Deut}{28}{35}{28:27 Job 2:6,\allowbreak7 Isa 1:6; 3:17,\allowbreak24}
\crossref{Deut}{28}{36}{2Ki 17:4-\allowbreak6; 24:12-\allowbreak15; 25:6,\allowbreak7,\allowbreak11 2Ch 33:11; 36:6,\allowbreak17,\allowbreak20 Isa 39:7}
\crossref{Deut}{28}{37}{28:28; 29:22-\allowbreak28 1Ki 9:7,\allowbreak8 2Ch 7:20 Ps 44:13,\allowbreak14 Jer 24:9; 25:9}
\crossref{Deut}{28}{38}{Isa 5:10 Mic 6:15 Hag 1:6}
\crossref{Deut}{28}{39}{Joe 1:4-\allowbreak7; 2:2-\allowbreak4 Jon 4:7}
\crossref{Deut}{28}{40}{Ps 23:5; 104:15 Mic 6:15}
\crossref{Deut}{28}{41}{}
\crossref{Deut}{28}{42}{28:38,\allowbreak39 Am 7:1,\allowbreak2}
\crossref{Deut}{28}{43}{Jud 2:3,\allowbreak11-\allowbreak15; 4:2,\allowbreak3; 10:7-\allowbreak10; 14:4; 15:11,\allowbreak12 1Sa 13:3-\allowbreak7,\allowbreak19-\allowbreak23}
\crossref{Deut}{28}{44}{28:12,\allowbreak13 La 1:5}
\crossref{Deut}{28}{45}{28:5,\allowbreak15; 29:20,\allowbreak21 Le 26:28 2Ki 17:20 Pr 13:21 Isa 1:20; 65:14,\allowbreak15}
\crossref{Deut}{28}{46}{28:37,\allowbreak59; 29:20,\allowbreak28 Isa 8:18 Jer 19:8; 25:18 Eze 14:8; 23:32,\allowbreak33}
\crossref{Deut}{28}{47}{De 12:7-\allowbreak12; 16:11; 32:13-\allowbreak15 Ne 9:35 1Ti 6:17-\allowbreak19}
\crossref{Deut}{28}{48}{2Ch 12:8 Ne 9:35-\allowbreak37 Jer 5:19; 17:4 Eze 17:3,\allowbreak7,\allowbreak12}
\crossref{Deut}{28}{49}{Jer 4:13; 48:40; 49:22 La 4:19 Eze 17:3,\allowbreak12 Ho 8:1 Mt 24:28}
\crossref{Deut}{28}{50}{Pr 7:13 Ec 8:1}
\crossref{Deut}{28}{51}{28:33 Isa 1:7; 62:8}
\crossref{Deut}{28}{52}{Le 26:25 2Ki 17:1-\allowbreak6; 18:13; 24:10,\allowbreak11; 25:1-\allowbreak4 Isa 1:7; 62:8}
\crossref{Deut}{28}{53}{28:18,\allowbreak55,\allowbreak57 Le 26:29 2Ki 6:28,\allowbreak29 Jer 19:9 La 2:20; 4:10}
\crossref{Deut}{28}{54}{De 15:9 Pr 23:6; 28:22 Mt 20:15}
\crossref{Deut}{28}{55}{Jer 5:10; 34:2; 52:6}
\crossref{Deut}{28}{56}{Isa 3:16 La 4:3-\allowbreak6}
\crossref{Deut}{28}{57}{Ge 49:10 Isa 49:15}
\crossref{Deut}{28}{58}{28:15 Le 26:14,\allowbreak15 Jer 7:9,\allowbreak10,\allowbreak26-\allowbreak28}
\crossref{Deut}{28}{59}{28:46; 29:20-\allowbreak28; 31:17,\allowbreak18; 32:22,\allowbreak26 1Ki 9:7-\allowbreak9; 16:3,\allowbreak4 La 1:9,\allowbreak12}
\crossref{Deut}{28}{60}{De 7:15 Ex 15:26}
\crossref{Deut}{28}{61}{28:61}
\crossref{Deut}{28}{62}{De 10:22 Ne 9:23 Ro 9:27}
\crossref{Deut}{28}{63}{De 30:9 Isa 62:5 Jer 32:41 Mic 7:18 Zep 3:17 Lu 15:6-\allowbreak10,\allowbreak23,\allowbreak24,\allowbreak32}
\crossref{Deut}{28}{64}{De 4:27,\allowbreak28 Le 26:33 Ne 1:8 Jer 16:13; 50:17 Eze 11:16,\allowbreak17 Lu 21:24}
\crossref{Deut}{28}{65}{Ge 8:9 Isa 57:21 Eze 5:12-\allowbreak17; 20:32-\allowbreak35 Am 9:4,\allowbreak9,\allowbreak10}
\crossref{Deut}{28}{66}{28:67 La 1:13 Heb 10:27 Re 6:15-\allowbreak17}
\crossref{Deut}{28}{67}{28:34 Job 7:3,\allowbreak4 Re 9:6}
\crossref{Deut}{28}{68}{Ex 20:2 Ne 5:8 Es 7:4 Joe 3:3-\allowbreak7 Lu 21:24}
\crossref{Deut}{29}{1}{29:12,\allowbreak21,\allowbreak25 Le 26:44,\allowbreak45 2Ki 23:3 Jer 11:2,\allowbreak6; 34:18 Ac 3:25}
\crossref{Deut}{29}{2}{Ex 8:12; 19:4 Jos 24:5,\allowbreak6 Ps 78:43-\allowbreak51; 105:27-\allowbreak36}
\crossref{Deut}{29}{3}{De 4:32-\allowbreak35; 7:18,\allowbreak19 Ne 9:9-\allowbreak11}
\crossref{Deut}{29}{4}{De 2:30 Pr 20:12 Isa 6:9,\allowbreak10; 63:17 Eze 36:26 Mt 13:11-\allowbreak15 Joh 8:43}
\crossref{Deut}{29}{5}{De 1:3; 8:2}
\crossref{Deut}{29}{6}{De 8:3 Ex 16:12,\allowbreak35 Ne 9:15 Ps 78:24,\allowbreak25}
\crossref{Deut}{29}{7}{De 2:24-\allowbreak37; 3:1-\allowbreak17 Nu 21:21-\allowbreak35; 32:33-\allowbreak42 Ps 135:10-\allowbreak12; 136:17-\allowbreak22}
\crossref{Deut}{29}{8}{De 3:12,\allowbreak13 Nu 32:33}
\crossref{Deut}{29}{9}{29:1; 4:6 Jos 1:7 1Ki 2:3 Ps 25:10; 103:17,\allowbreak18 Isa 56:1,\allowbreak2,\allowbreak4-\allowbreak7}
\crossref{Deut}{29}{10}{De 4:10; 31:12,\allowbreak13 2Ch 23:16; 34:29-\allowbreak32 Ne 8:2; 9:1,\allowbreak2,\allowbreak38; 10:28}
\crossref{Deut}{29}{11}{De 5:14 Ex 12:38,\allowbreak48,\allowbreak49 Nu 11:4}
\crossref{Deut}{29}{12}{De 5:2,\allowbreak3 Ex 19:5,\allowbreak6 Jos 24:25 2Ki 11:17 2Ch 15:12-\allowbreak15}
\crossref{Deut}{29}{13}{De 7:6; 26:18,\allowbreak19; 28:9}
\crossref{Deut}{29}{14}{Jer 31:31-\allowbreak34 Heb 8:7-\allowbreak12}
\crossref{Deut}{29}{15}{De 5:3 Jer 32:39; 50:5 Ac 2:39 1Co 7:14}
\crossref{Deut}{29}{16}{De 2:4,\allowbreak9,\allowbreak19,\allowbreak24; 3:1,\allowbreak2}
\crossref{Deut}{29}{17}{29:17}
\crossref{Deut}{29}{18}{De 11:16,\allowbreak17; 13:1-\allowbreak15; 17:2-\allowbreak7 Heb 3:12}
\crossref{Deut}{29}{19}{29:12 Ge 2:17}
\crossref{Deut}{29}{20}{Ps 78:50 Pr 6:34 Isa 27:11 Jer 13:14 Eze 5:11; 7:4,\allowbreak9; 8:18; 9:10}
\crossref{Deut}{29}{21}{Jos 7:1-\allowbreak26 Eze 13:9 Mal 3:18 Mt 24:51; 25:32,\allowbreak41,\allowbreak46}
\crossref{Deut}{29}{22}{}
\crossref{Deut}{29}{23}{Job 18:15 Isa 34:9 Lu 17:29 Re 19:20}
\crossref{Deut}{29}{24}{1Ki 9:8,\allowbreak9 2Ch 7:21,\allowbreak22 Jer 22:8,\allowbreak9 La 2:15-\allowbreak17; 4:12 Eze 14:23}
\crossref{Deut}{29}{25}{Isa 47:6 Jer 40:2,\allowbreak3; 50:7}
\crossref{Deut}{29}{26}{Jud 2:12,\allowbreak13; 5:8 2Ki 17:7-\allowbreak18 2Ch 36:12-\allowbreak17 Jer 19:3-\allowbreak13; 44:2-\allowbreak6}
\crossref{Deut}{29}{27}{29:20,\allowbreak21; 27:15-\allowbreak26; 28:15-\allowbreak68 Le 26:14-\allowbreak46 Da 9:11-\allowbreak14}
\crossref{Deut}{29}{28}{De 28:25,\allowbreak36,\allowbreak64 1Ki 14:15 2Ki 17:18,\allowbreak23 2Ch 7:20 Ps 52:5 Pr 2:22}
\crossref{Deut}{29}{29}{Job 11:6,\allowbreak7; 28:28 Ps 25:14 Pr 3:32 Jer 23:18 Da 2:18,\allowbreak19,\allowbreak22}
\crossref{Deut}{30}{1}{De 4:30 Le 26:40-\allowbreak46}
\crossref{Deut}{30}{2}{De 4:28-\allowbreak31 Ne 1:9 Isa 55:6,\allowbreak7 La 3:32,\allowbreak40 Ho 3:5; 6:1,\allowbreak2; 14:1-\allowbreak3}
\crossref{Deut}{30}{3}{Ps 106:45-\allowbreak47; 126:1-\allowbreak4 Isa 56:8 Jer 29:14; 31:10 La 3:22,\allowbreak32}
\crossref{Deut}{30}{4}{De 28:64 Ne 1:9 Isa 11:11-\allowbreak16 Eze 39:25-\allowbreak29 Zep 3:19,\allowbreak20}
\crossref{Deut}{30}{5}{}
\crossref{Deut}{30}{6}{De 10:16 Jer 4:4; 9:26; 32:39 Eze 11:19,\allowbreak20; 36:26,\allowbreak27 Joh 3:3-\allowbreak7}
\crossref{Deut}{30}{7}{Nu 24:14 Ps 137:7-\allowbreak9 Isa 10:12; 14:1-\allowbreak27 Jer 25:12-\allowbreak16,\allowbreak29}
\crossref{Deut}{30}{8}{30:2 Pr 16:1 Isa 1:25,\allowbreak26 Jer 31:33; 32:39,\allowbreak40 Eze 11:19,\allowbreak20; 36:27}
\crossref{Deut}{30}{9}{De 28:4,\allowbreak11-\allowbreak14 Le 26:4,\allowbreak6,\allowbreak9,\allowbreak10}
\crossref{Deut}{30}{10}{30:2,\allowbreak8 Isa 55:2,\allowbreak3 1Co 7:19}
\crossref{Deut}{30}{11}{}
\crossref{Deut}{30}{12}{Pr 30:4 Joh 3:13 Ro 10:6,\allowbreak7}
\crossref{Deut}{30}{13}{Ac 10:22,\allowbreak33; 16:9 Ro 10:14,\allowbreak15}
\crossref{Deut}{30}{14}{Eze 2:5; 33:33 Lu 10:11,\allowbreak12 Joh 5:46 Ac 13:26,\allowbreak38-\allowbreak41; 28:23-\allowbreak28}
\crossref{Deut}{30}{15}{30:1,\allowbreak19; 11:26; 28:1-\allowbreak14; 32:47 Mr 16:16 Joh 3:16 Ga 3:13,\allowbreak14; 5:6}
\crossref{Deut}{30}{16}{30:6 Mt 22:37,\allowbreak38 1Co 7:19 1Jo 5:2,\allowbreak3}
\crossref{Deut}{30}{17}{De 29:18-\allowbreak28 1Sa 12:25 Joh 3:19-\allowbreak21}
\crossref{Deut}{30}{18}{De 8:19,\allowbreak20; 31:29 Jos 23:15,\allowbreak16 Isa 63:17,\allowbreak18}
\crossref{Deut}{30}{19}{De 4:26; 31:28; 32:1 Isa 1:2 Jer 2:12,\allowbreak13; 22:29,\allowbreak30 Mic 6:1,\allowbreak2}
\crossref{Deut}{30}{20}{30:6,\allowbreak16; 10:12; 11:22}
\crossref{Deut}{31}{1}{31:1}
\crossref{Deut}{31}{2}{De 34:7 Nu 27:17 2Sa 21:17 1Ki 3:7}
\crossref{Deut}{31}{3}{De 9:3 Ge 48:21 Ps 44:2,\allowbreak3; 146:3-\allowbreak6}
\crossref{Deut}{31}{4}{De 2:33; 3:3-\allowbreak11,\allowbreak21; 7:2,\allowbreak16 Ex 23:28-\allowbreak31}
\crossref{Deut}{31}{5}{De 7:2,\allowbreak18}
\crossref{Deut}{31}{6}{31:7,\allowbreak23; 20:4 Jos 1:6,\allowbreak7,\allowbreak9; 10:25 1Ch 22:13; 28:10,\allowbreak20 2Ch 32:7}
\crossref{Deut}{31}{7}{31:6,\allowbreak23; 1:38; 3:28 Jos 1:6 Da 10:19 Eph 6:10}
\crossref{Deut}{31}{8}{31:3; 9:3 Ex 13:21,\allowbreak22; 33:14}
\crossref{Deut}{31}{9}{31:22-\allowbreak24,\allowbreak28 Nu 33:2 Da 9:13 Mal 4:4 Mr 10:4,\allowbreak5; 12:19 Lu 20:28}
\crossref{Deut}{31}{10}{De 15:1,\allowbreak2}
\crossref{Deut}{31}{11}{De 16:16,\allowbreak17 Ex 23:16,\allowbreak17; 34:24 Ps 84:7}
\crossref{Deut}{31}{12}{De 4:10}
\crossref{Deut}{31}{13}{De 6:7; 11:2 Ps 78:4-\allowbreak8 Pr 22:6 Eph 6:4}
\crossref{Deut}{31}{14}{31:2; 34:5 Nu 27:13 Jos 23:14 2Ki 1:4 Ec 9:5 Isa 38:1}
\crossref{Deut}{31}{15}{Ex 33:9,\allowbreak10; 40:38 Ps 99:7}
\crossref{Deut}{31}{16}{Ge 25:8 2Sa 7:12 Isa 57:2 Ac 13:36}
\crossref{Deut}{31}{17}{De 29:20; 32:21,\allowbreak22 Jud 2:14,\allowbreak15 Ps 2:12; 90:11}
\crossref{Deut}{31}{18}{31:16,\allowbreak17}
\crossref{Deut}{31}{19}{31:22,\allowbreak30; 32:1-\allowbreak43,\allowbreak44,\allowbreak45}
\crossref{Deut}{31}{20}{De 6:10-\allowbreak12; 7:1; 8:7}
\crossref{Deut}{31}{21}{31:19}
\crossref{Deut}{31}{22}{31:9,\allowbreak19}
\crossref{Deut}{31}{23}{31:7,\allowbreak8,\allowbreak14 Jos 1:5-\allowbreak9}
\crossref{Deut}{31}{24}{31:9; 17:18}
\crossref{Deut}{31}{25}{31:9}
\crossref{Deut}{31}{26}{1Ki 8:9 2Ki 22:8-\allowbreak11 2Ch 34:14,\allowbreak15}
\crossref{Deut}{31}{27}{De 32:20}
\crossref{Deut}{31}{28}{31:12; 29:10 Ge 49:1,\allowbreak2 Ex 18:25 Nu 11:16,\allowbreak17}
\crossref{Deut}{31}{29}{De 32:5 Jud 2:19 Isa 1:4 Ho 9:9 Ac 20:30 2Ti 3:1-\allowbreak6 2Pe 1:14,\allowbreak15}
\crossref{Deut}{31}{30}{De 4:5 Joh 12:49 Ac 20:27 Heb 3:2,\allowbreak5}
\crossref{Deut}{32}{1}{}
\crossref{Deut}{32}{2}{2Sa 23:4 Job 29:22,\allowbreak23 Ps 72:6 Isa 55:10,\allowbreak11 Ho 6:4; 14:5}
\crossref{Deut}{32}{3}{Ex 3:13-\allowbreak16; 6:3; 20:24; 34:5-\allowbreak7 Ps 29:1,\allowbreak2; 89:16-\allowbreak18; 105:1-\allowbreak5}
\crossref{Deut}{32}{4}{32:18,\allowbreak30,\allowbreak31 1Sa 2:2 2Sa 22:2,\allowbreak3,\allowbreak32,\allowbreak47; 23:3 Ps 18:2,\allowbreak31,\allowbreak46; 61:2-\allowbreak4}
\crossref{Deut}{32}{5}{De 9:24 Ps 78:8 Isa 1:4 Mt 3:7; 16:4; 17:17 Lu 9:41 Ac 7:51}
\crossref{Deut}{32}{6}{32:18 Isa 1:2 2Co 5:14,\allowbreak15 Tit 2:11-\allowbreak14}
\crossref{Deut}{32}{7}{Ps 44:1; 77:5; 119:52 Isa 63:11}
\crossref{Deut}{32}{8}{Nu 24:16 Ps 7:17; 50:14; 82:6; 91:1; 92:8 Isa 14:14 Da 4:17; 5:18}
\crossref{Deut}{32}{9}{De 26:18,\allowbreak19 Ex 15:16; 19:5,\allowbreak6 1Sa 10:1 Ps 78:71; 135:4 Isa 43:21}
\crossref{Deut}{32}{10}{De 8:15,\allowbreak16 Ne 9:19-\allowbreak21 Ps 107:4,\allowbreak5 So 8:5 Jer 2:6 Ho 13:5}
\crossref{Deut}{32}{11}{Ex 19:4 Isa 31:5; 40:31; 46:4; 63:9 Heb 11:3 Re 12:4}
\crossref{Deut}{32}{12}{De 1:31 Ne 9:12 Ps 27:11; 78:14,\allowbreak52,\allowbreak53; 80:1; 136:16 Isa 46:4}
\crossref{Deut}{32}{13}{De 33:26,\allowbreak29 Isa 58:14 Eze 36:2}
\crossref{Deut}{32}{14}{Ge 18:8 Jud 5:25 2Sa 17:29 Job 20:17 Isa 7:15,\allowbreak22}
\crossref{Deut}{32}{15}{De 33:5,\allowbreak26 Isa 44:2}
\crossref{Deut}{32}{16}{De 5:9 1Ki 14:22 Na 1:1,\allowbreak2 1Co 10:22}
\crossref{Deut}{32}{17}{Le 17:7 Ps 106:37,\allowbreak38 1Co 10:20 1Ti 4:1 Re 9:20}
\crossref{Deut}{32}{18}{32:4,\allowbreak15 Isa 17:10}
\crossref{Deut}{32}{19}{Le 26:11 Jud 2:14 Ps 5:4; 10:3; 78:59; 106:40 Am 3:2,\allowbreak3 Zec 11:8}
\crossref{Deut}{32}{20}{De 31:17,\allowbreak18 Job 13:24; 34:29 Isa 64:7 Jer 18:17 Ho 9:12}
\crossref{Deut}{32}{21}{32:16 Ps 78:58}
\crossref{Deut}{32}{22}{De 29:20 Nu 16:35 Ps 21:9; 83:14; 97:3 Isa 66:15,\allowbreak16 Jer 4:4; 15:14}
\crossref{Deut}{32}{23}{De 28:15 Le 26:18,\allowbreak24 Isa 24:17,\allowbreak18; 26:15 Jer 15:2,\allowbreak3 Eze 14:21}
\crossref{Deut}{32}{24}{De 28:53 Jer 14:18 La 4:4-\allowbreak9; 5:10}
\crossref{Deut}{32}{25}{Le 26:36,\allowbreak37 Isa 30:16 Jer 9:21 La 1:20 Eze 7:15 2Co 7:5}
\crossref{Deut}{32}{26}{De 28:25,\allowbreak37,\allowbreak64 Le 26:33,\allowbreak38 Isa 63:16 Lu 21:24}
\crossref{Deut}{32}{27}{1Sa 12:22 Isa 37:28,\allowbreak29,\allowbreak35; 47:7 Jer 19:4 La 1:9 Eze 20:13,\allowbreak14}
\crossref{Deut}{32}{28}{32:6 Job 28:28 Ps 81:12 Pr 1:7 Isa 27:11; 29:14 Jer 4:22; 8:9}
\crossref{Deut}{32}{29}{De 5:29 Ps 81:13; 107:15,\allowbreak43 Isa 48:18,\allowbreak19 Ho 14:9 Lu 19:41,\allowbreak42}
\crossref{Deut}{32}{30}{Le 26:8 Jos 23:10 Jud 7:22,\allowbreak23 1Sa 14:15-\allowbreak17 2Ch 24:24}
\crossref{Deut}{32}{31}{Ex 14:25 Nu 23:8,\allowbreak23 1Sa 2:2; 4:8 Ezr 1:3; 6:9-\allowbreak12; 7:20,\allowbreak21}
\crossref{Deut}{32}{32}{Isa 1:10 Jer 2:21 La 4:6 Eze 16:45-\allowbreak51 Mt 11:24}
\crossref{Deut}{32}{33}{Job 20:14-\allowbreak16 Ps 58:4; 140:3 Jer 8:14}
\crossref{Deut}{32}{34}{Job 14:17 Jer 2:22 Ho 13:12 Ro 2:5 1Co 4:5 Re 20:12,\allowbreak13}
\crossref{Deut}{32}{35}{32:43 Ps 94:1 Na 1:2,\allowbreak6 Ro 12:19; 13:4 Heb 10:30}
\crossref{Deut}{32}{36}{Ps 7:8; 50:4; 96:13; 135:14}
\crossref{Deut}{32}{37}{Jud 10:14 2Ki 3:13 Jer 2:28}
\crossref{Deut}{32}{38}{Le 21:21 Ps 50:13 Eze 16:18,\allowbreak19 Ho 2:8 Zep 2:11}
\crossref{Deut}{32}{39}{Ps 102:27 Isa 41:4; 45:5,\allowbreak18,\allowbreak22; 46:4; 48:12 Heb 1:12 Re 1:11; 2:8}
\crossref{Deut}{32}{40}{Ge 14:22 Ex 6:8 Nu 14:28-\allowbreak30 Jer 4:2 Heb 6:17,\allowbreak18 Re 10:5,\allowbreak6}
\crossref{Deut}{32}{41}{Ps 7:12 Isa 27:1; 34:5,\allowbreak6; 66:16 Eze 21:9-\allowbreak15,\allowbreak20 Zep 2:12}
\crossref{Deut}{32}{42}{32:23 Ps 45:5; 68:23 Isa 34:6-\allowbreak8 Jer 16:10 Eze 35:6-\allowbreak8; 38:21,\allowbreak22}
\crossref{Deut}{32}{43}{32:35 Job 13:24 Jer 13:14 La 2:5 Lu 19:27,\allowbreak43,\allowbreak44; 21:22-\allowbreak24 Ro 12:19}
\crossref{Deut}{32}{44}{De 31:22,\allowbreak30}
\crossref{Deut}{32}{45}{}
\crossref{Deut}{32}{46}{De 6:6,\allowbreak7; 11:18 1Ch 22:19 Pr 3:1-\allowbreak4 Eze 40:4 Lu 9:44 Heb 2:1}
\crossref{Deut}{32}{47}{De 30:19 Le 18:5 Pr 3:1,\allowbreak2,\allowbreak18,\allowbreak22; 4:22 Isa 45:19 Mt 6:33 Ro 10:5,\allowbreak6}
\crossref{Deut}{32}{48}{Nu 27:12,\allowbreak13}
\crossref{Deut}{32}{49}{De 34:1}
\crossref{Deut}{32}{50}{Ge 15:15}
\crossref{Deut}{32}{51}{De 3:23-\allowbreak27 Nu 20:11,\allowbreak12,\allowbreak24; 27:14}
\crossref{Deut}{32}{52}{32:49; 34:1-\allowbreak4 Nu 27:12 Heb 11:13,\allowbreak39}
\crossref{Deut}{33}{1}{Ge 27:4,\allowbreak27-\allowbreak29; 49:1,\allowbreak28 Lu 24:50,\allowbreak51 Joh 14:27; 16:33}
\crossref{Deut}{33}{2}{Ex 19:18-\allowbreak20 Jud 5:4,\allowbreak5 Hab 3:3}
\crossref{Deut}{33}{3}{De 7:7,\allowbreak8 Ex 19:5,\allowbreak6 Ps 47:4; 147:19,\allowbreak20 Jer 31:3 Ho 11:1 Mal 1:2}
\crossref{Deut}{33}{4}{Joh 1:17; 7:19}
\crossref{Deut}{33}{5}{Ge 36:31 Ex 18:16,\allowbreak19 Nu 16:13-\allowbreak15 Jud 8:22; 9:2; 17:6}
\crossref{Deut}{33}{6}{Ge 49:3,\allowbreak4,\allowbreak8 Nu 32:31,\allowbreak32 Jos 22:1-\allowbreak9}
\crossref{Deut}{33}{7}{Ge 49:8-\allowbreak12 Jud 1:1-\allowbreak7 Ps 78:68,\allowbreak70 Mic 5:2 Mal 3:1 Heb 7:14}
\crossref{Deut}{33}{8}{Ex 28:30,\allowbreak36 Le 8:8 Nu 27:21 1Sa 28:6 Ezr 2:63 Ne 7:65}
\crossref{Deut}{33}{9}{Ex 32:25-\allowbreak29 Le 10:6; 21:11 Mt 10:37; 12:48; 22:16 Lu 14:26}
\crossref{Deut}{33}{10}{De 17:9-\allowbreak11; 24:8 Le 10:11 2Ch 17:8-\allowbreak10; 30:22 Ne 8:1-\allowbreak9,\allowbreak13-\allowbreak15,\allowbreak18}
\crossref{Deut}{33}{11}{De 18:1-\allowbreak5 Nu 18:8-\allowbreak20; 35:2-\allowbreak8}
\crossref{Deut}{33}{12}{33:27-\allowbreak29 Jos 18:11-\allowbreak28 Jud 1:21 1Ki 12:21 2Ch 11:1; 15:2}
\crossref{Deut}{33}{13}{Ge 48:5,\allowbreak9,\allowbreak15-\allowbreak20; 49:22-\allowbreak26}
\crossref{Deut}{33}{14}{De 28:8 Le 26:4 2Sa 23:4 Ps 65:9-\allowbreak13; 74:16; 84:11 Mal 4:2 Mt 5:45}
\crossref{Deut}{33}{15}{Ge 49:26 Hab 3:6 Jas 5:7}
\crossref{Deut}{33}{16}{Ps 24:1; 50:12; 89:11 Jer 8:16}
\crossref{Deut}{33}{17}{1Ch 5:1}
\crossref{Deut}{33}{18}{Ge 49:13-\allowbreak15 Jos 19:11 Jud 5:14}
\crossref{Deut}{33}{19}{Isa 2:3 Jer 50:4,\allowbreak5 Mic 4:2}
\crossref{Deut}{33}{20}{Ge 9:26,\allowbreak27 Jos 13:8,\allowbreak10,\allowbreak24-\allowbreak28 1Ch 4:10; 12:8,\allowbreak37,\allowbreak38 Ps 18:19,\allowbreak36}
\crossref{Deut}{33}{21}{Nu 32:1-\allowbreak6,\allowbreak16,\allowbreak17-\allowbreak42}
\crossref{Deut}{33}{22}{Ge 49:16,\allowbreak17 Jos 19:47 Jud 13:2,\allowbreak24,\allowbreak25; 14:6,\allowbreak19; 15:8,\allowbreak15; 16:30}
\crossref{Deut}{33}{23}{Ge 49:21 Ps 36:8; 90:14 Isa 9:1,\allowbreak2 Jer 31:14 Mt 4:13,\allowbreak16; 11:28}
\crossref{Deut}{33}{24}{Ge 49:20 Ps 115:15; 128:3,\allowbreak6}
\crossref{Deut}{33}{25}{De 8:9 Lu 15:22 Eph 6:15}
\crossref{Deut}{33}{26}{Ex 15:11 Ps 86:8 Isa 40:18,\allowbreak25; 43:11-\allowbreak13; 66:8 Jer 10:6}
\crossref{Deut}{33}{27}{1Sa 15:29 Ps 90:1,\allowbreak2; 102:24 Isa 9:6; 25:4; 57:15 Jer 10:10}
\crossref{Deut}{33}{28}{Ex 33:16 Nu 23:9 Jer 23:6; 33:16 Eze 34:25 Re 21:27; 22:14,\allowbreak15}
\crossref{Deut}{33}{29}{De 4:7,\allowbreak8 Nu 23:20-\allowbreak24; 24:5 2Sa 7:23 Ps 33:12; 144:15}
\crossref{Deut}{34}{1}{De 32:49 Nu 27:12; 33:47}
\crossref{Deut}{34}{2}{De 11:24 Ex 23:31 Nu 34:6 Jos 15:12}
\crossref{Deut}{34}{3}{Jud 1:16; 3:13 2Ch 28:15}
\crossref{Deut}{34}{4}{Ge 12:7; 13:15; 15:18-\allowbreak21; 26:3; 28:13 Ps 105:9-\allowbreak11}
\crossref{Deut}{34}{5}{Jos 1:1 Mal 4:4 Joh 8:35,\allowbreak36 2Ti 2:25 Heb 3:3-\allowbreak6 2Pe 1:1}
\crossref{Deut}{34}{6}{Jude 1:9}
\crossref{Deut}{34}{7}{De 31:2 Ac 7:23,\allowbreak30,\allowbreak36}
\crossref{Deut}{34}{8}{Ge 50:3,\allowbreak10 Nu 20:29 1Sa 25:1 Isa 57:1 Ac 8:2}
\crossref{Deut}{34}{9}{Ex 31:3 Nu 11:17 1Ki 3:9,\allowbreak12 2Ki 2:9,\allowbreak15 Isa 11:2 Da 6:3}
\crossref{Deut}{34}{10}{De 18:15-\allowbreak18 Ac 3:22,\allowbreak23; 7:37 Heb 3:5,\allowbreak6}
\crossref{Deut}{34}{11}{}
\crossref{Deut}{34}{12}{34:12}

% Josh
\crossref{Josh}{1}{1}{Jos 12:6}
\crossref{Josh}{1}{2}{1:1 Isa 42:1 Heb 3:5,\allowbreak6; 7:23,\allowbreak24}
\crossref{Josh}{1}{3}{Jos 14:9 De 11:24 Tit 1:2}
\crossref{Josh}{1}{4}{}
\crossref{Josh}{1}{5}{De 7:24; 20:4 Ps 46:11 Ro 8:31,\allowbreak37}
\crossref{Josh}{1}{6}{1:7,\allowbreak9 1Sa 4:9 1Ki 2:2 1Ch 22:13; 28:10 2Ch 32:7,\allowbreak8 Ps 27:14}
\crossref{Josh}{1}{7}{1:1; 11:15 Nu 27:23 De 31:7}
\crossref{Josh}{1}{8}{De 6:6-\allowbreak9; 11:18,\allowbreak19; 17:18,\allowbreak19; 30:14; 31:11 Ps 37:30,\allowbreak31; 40:10}
\crossref{Josh}{1}{9}{De 31:7,\allowbreak8,\allowbreak28 Jud 6:14 2Sa 13:28 Ac 4:19}
\crossref{Josh}{1}{10}{}
\crossref{Josh}{1}{11}{Jos 3:2 Ex 19:11 2Ki 20:5 Ho 6:2}
\crossref{Josh}{1}{12}{}
\crossref{Josh}{1}{13}{Jos 22:1-\allowbreak4 Nu 32:20-\allowbreak28 De 3:18}
\crossref{Josh}{1}{14}{Ex 13:18}
\crossref{Josh}{1}{15}{Nu 32:17-\allowbreak22 Ga 5:13; 6:2 Php 1:21-\allowbreak26; 2:4}
\crossref{Josh}{1}{16}{Nu 32:25 De 5:27 Ro 13:1-\allowbreak5 Tit 3:1 1Pe 2:13-\allowbreak15}
\crossref{Josh}{1}{17}{1:5 1Sa 20:13 1Ki 1:37 1Ch 28:20 Ps 20:1,\allowbreak4,\allowbreak9; 118:25,\allowbreak26}
\crossref{Josh}{1}{18}{De 17:12 1Sa 11:12 Ps 2:1-\allowbreak6 Lu 19:27 Heb 10:28,\allowbreak29; 12:25}
\crossref{Josh}{2}{1}{Nu 25:1; 33:49}
\crossref{Josh}{2}{2}{Ps 127:1 Pr 21:30 Isa 43:13 Da 4:35}
\crossref{Josh}{2}{3}{Jos 10:23 Ge 38:24 Le 24:14 Job 21:30 Joh 19:4 Ac 12:4,\allowbreak6}
\crossref{Josh}{2}{4}{Ex 1:19 2Sa 16:18,\allowbreak19; 17:19,\allowbreak20 2Ki 6:19}
\crossref{Josh}{2}{5}{2:7 Ne 13:19 Isa 60:11 Eze 47:1,\allowbreak2,\allowbreak12 Re 21:25}
\crossref{Josh}{2}{6}{2:8 Ex 1:15-\allowbreak21 De 22:8 2Sa 11:2 Mt 24:17}
\crossref{Josh}{2}{7}{Jud 3:28; 12:5}
\crossref{Josh}{2}{8}{}
\crossref{Josh}{2}{9}{Ex 18:11 2Ki 5:15 Job 19:25 Ec 8:12 Heb 11:1,\allowbreak2}
\crossref{Josh}{2}{10}{Jos 4:24 Ex 14:21-\allowbreak31; 15:14-\allowbreak16}
\crossref{Josh}{2}{11}{Jos 5:1; 7:5; 14:8 De 1:28; 20:8 Isa 13:7 Na 2:10}
\crossref{Josh}{2}{12}{Jos 9:15,\allowbreak18-\allowbreak20 1Sa 20:14,\allowbreak15,\allowbreak17; 30:15 2Ch 36:13 Jer 12:16}
\crossref{Josh}{2}{13}{}
\crossref{Josh}{2}{14}{1Ki 20:39}
\crossref{Josh}{2}{15}{1Sa 19:12-\allowbreak17 Ac 9:25 2Co 11:33}
\crossref{Josh}{2}{16}{2:22 1Sa 23:14,\allowbreak29 Ps 11:1}
\crossref{Josh}{2}{17}{2:20 Ge 24:3-\allowbreak8 Ex 20:7 Le 19:11,\allowbreak12 Nu 30:2 2Sa 21:1,\allowbreak2,\allowbreak7}
\crossref{Josh}{2}{18}{2:21 Le 14:4 Nu 4:8; 19:6 Heb 9:19}
\crossref{Josh}{2}{19}{Ex 12:13,\allowbreak23 Nu 35:26-\allowbreak28 1Ki 2:36-\allowbreak42 Mt 24:17 Ac 27:31 Php 3:9}
\crossref{Josh}{2}{20}{Pr 11:13}
\crossref{Josh}{2}{21}{2:18 Mt 7:24 Joh 2:5}
\crossref{Josh}{2}{22}{1Sa 19:10-\allowbreak12 2Sa 17:20 Ps 32:6,\allowbreak7}
\crossref{Josh}{2}{23}{}
\crossref{Josh}{2}{24}{Jos 2:9-\allowbreak11 Ps 48:5,\allowbreak6 Re 6:16,\allowbreak17}
\crossref{Josh}{3}{1}{Jos 2:1 Nu 25:1 Mic 6:5}
\crossref{Josh}{3}{2}{Jos 1:10,\allowbreak11}
\crossref{Josh}{3}{3}{Jos 3:11 Nu 10:33}
\crossref{Josh}{3}{4}{Ex 3:5; 19:12 Ps 89:7 Heb 12:28,\allowbreak29}
\crossref{Josh}{3}{5}{Jos 7:13 Ex 19:10-\allowbreak15 Le 10:3; 20:7 Nu 11:8 1Sa 16:5 Job 1:5}
\crossref{Josh}{3}{6}{Jos 3:3 Nu 14:15; 10:33 Mic 2:13 Joh 14:2,\allowbreak3 Heb 6:20}
\crossref{Josh}{3}{7}{Jos 4:14 1Ch 29:25 2Ch 1:1 Job 7:17 Ps 18:35 Joh 17:1 Php 1:20}
\crossref{Josh}{3}{8}{Jos 3:3 1Ch 15:11,\allowbreak12 2Ch 17:8,\allowbreak9; 29:4-\allowbreak11,\allowbreak15,\allowbreak27,\allowbreak30; 30:12; 31:9,\allowbreak10}
\crossref{Josh}{3}{9}{De 4:1; 12:8}
\crossref{Josh}{3}{10}{Nu 15:28-\allowbreak30 1Ki 18:36,\allowbreak37; 22:28 Ps 9:16 Isa 7:14 2Co 13:2,\allowbreak3}
\crossref{Josh}{3}{11}{Jos 3:13 Ps 24:1 Isa 54:5 Jer 10:7 Mic 4:13 Zep 2:11 Zec 4:14; 6:5}
\crossref{Josh}{3}{12}{Jos 4:2,\allowbreak9}
\crossref{Josh}{3}{13}{Jos 3:15,\allowbreak16}
\crossref{Josh}{3}{14}{Jos 3:3,\allowbreak5; 6:6 De 31:26 Jer 3:16 Ac 7:44,\allowbreak45 1Co 1:24,\allowbreak25 Heb 9:4}
\crossref{Josh}{3}{15}{Jos 3:13 Isa 26:6}
\crossref{Josh}{3}{16}{Jos 3:13 Ps 29:10; 77:19; 114:3 Mt 8:26,\allowbreak27; 14:24-\allowbreak33}
\crossref{Josh}{3}{17}{Jos 3:3-\allowbreak6}
\crossref{Josh}{4}{1}{Jos 3:17 De 27:2}
\crossref{Josh}{4}{2}{Jos 3:12 Nu 1:4-\allowbreak15; 13:2; 34:18 De 1:23 1Ki 18:31 Mt 10:15}
\crossref{Josh}{4}{3}{Jos 3:13}
\crossref{Josh}{4}{4}{Jos 4:2 Mr 3:14-\allowbreak19}
\crossref{Josh}{4}{5}{}
\crossref{Josh}{4}{6}{Jos 4:21 Ex 12:26; 13:14 De 6:20,\allowbreak21; 11:19 Ps 44:1; 71:18; 78:3-\allowbreak8}
\crossref{Josh}{4}{7}{Jos 3:13-\allowbreak16}
\crossref{Josh}{4}{8}{Jos 4:2-\allowbreak5; 1:16-\allowbreak18}
\crossref{Josh}{4}{9}{Ex 24:12; 28:21 1Ki 18:31 Ps 111:2-\allowbreak4}
\crossref{Josh}{4}{10}{Jos 3:13,\allowbreak16,\allowbreak17 Isa 28:16}
\crossref{Josh}{4}{11}{Jos 4:18; 3:8,\allowbreak17}
\crossref{Josh}{4}{12}{Jos 1:14 Nu 32:20-\allowbreak32}
\crossref{Josh}{4}{13}{Eph 6:11}
\crossref{Josh}{4}{14}{Jos 1:16-\allowbreak18; 3:7 1Co 10:2}
\crossref{Josh}{4}{15}{}
\crossref{Josh}{4}{16}{}
\crossref{Josh}{4}{17}{Ge 8:16-\allowbreak18 Da 3:26 Ac 16:23,\allowbreak35-\allowbreak39}
\crossref{Josh}{4}{18}{Jos 3:13,\allowbreak15}
\crossref{Josh}{4}{19}{Ex 12:2,\allowbreak3}
\crossref{Josh}{4}{20}{Jos 4:3,\allowbreak8}
\crossref{Josh}{4}{21}{Jos 4:6 Ps 105:2-\allowbreak5; 145:4-\allowbreak7}
\crossref{Josh}{4}{22}{Jos 3:17 Ex 14:29; 15:19 Ps 66:5,\allowbreak6 Isa 11:15,\allowbreak16; 44:27; 51:10 Re 16:12}
\crossref{Josh}{4}{23}{Ex 14:21 Ne 9:11 Ps 77:16-\allowbreak19; 78:13 Isa 43:16; 63:12-\allowbreak14}
\crossref{Josh}{4}{24}{Ex 9:16 De 28:10 1Sa 17:46 1Ki 8:42,\allowbreak43 2Ki 5:15; 19:19 Ps 106:8}
\crossref{Josh}{5}{1}{Jos 12:9-\allowbreak24; 24:15 Ge 10:15-\allowbreak19; 15:18-\allowbreak21; 48:22 Jer 11:23 2Sa 21:2}
\crossref{Josh}{5}{2}{Ge 17:10-\allowbreak14 De 10:16; 30:6 Ro 2:29; 4:11 Col 2:11}
\crossref{Josh}{5}{3}{Ge 17:23-\allowbreak27 Mt 16:24}
\crossref{Josh}{5}{4}{Nu 14:22; 26:64,\allowbreak65 De 2:16 1Co 10:5 Heb 3:17-\allowbreak19}
\crossref{Josh}{5}{5}{De 12:8,\allowbreak9 Ho 6:6,\allowbreak7 Mt 12:7 Ro 2:26 1Co 7:19 Ga 5:6; 6:15}
\crossref{Josh}{5}{6}{Nu 14:32-\allowbreak34 De 1:3; 2:7,\allowbreak14; 8:4 Ps 95:10,\allowbreak11 Jer 2:2}
\crossref{Josh}{5}{7}{Nu 14:31 De 1:39}
\crossref{Josh}{5}{8}{Ge 34:25}
\crossref{Josh}{5}{9}{Jos 24:14 Ge 34:14 Le 24:14 1Sa 14:6; 17:26,\allowbreak36 Ps 119:39 Jer 9:25}
\crossref{Josh}{5}{10}{Eze 12:3,\allowbreak6 Nu 9:1-\allowbreak5}
\crossref{Josh}{5}{11}{Ex 12:18-\allowbreak20; 13:6,\allowbreak7 Le 23:6,\allowbreak14}
\crossref{Josh}{5}{12}{Ex 16:35 Ne 9:20,\allowbreak21 Re 7:16,\allowbreak17}
\crossref{Josh}{5}{13}{Ge 33:1,\allowbreak5 Da 8:3; 10:5}
\crossref{Josh}{5}{14}{Ex 23:20-\allowbreak22 Isa 55:4 Da 10:13,\allowbreak21; 12:1 Heb 2:10 Re 12:7; 19:11-\allowbreak14}
\crossref{Josh}{5}{15}{Ex 3:5 Ac 7:32,\allowbreak33 2Pe 1:18}
\crossref{Josh}{6}{1}{Jos 2:7 2Ki 17:4}
\crossref{Josh}{6}{2}{Jos 5:13-\allowbreak15}
\crossref{Josh}{6}{3}{Jos 6:7,\allowbreak14 Nu 14:9 1Co 1:21-\allowbreak25 2Co 4:7}
\crossref{Josh}{6}{4}{Ge 2:3; 7:2,\allowbreak3 Le 4:6; 14:16; 25:8 Nu 23:1 1Ki 18:43 2Ki 5:10 Job 42:8}
\crossref{Josh}{6}{5}{Jos 6:16,\allowbreak20 Ex 19:19 2Ch 20:21,\allowbreak22}
\crossref{Josh}{6}{6}{Jos 6:8,\allowbreak13; 3:3,\allowbreak6 Ex 25:14 De 20:2-\allowbreak4 Ac 9:1}
\crossref{Josh}{6}{7}{Jos 6:3; 1:14; 4:13}
\crossref{Josh}{6}{8}{Jos 6:3,\allowbreak4 Nu 32:20}
\crossref{Josh}{6}{9}{Jos 6:13 Nu 10:25 Isa 52:11; 58:8}
\crossref{Josh}{6}{10}{Isa 42:2 Mt 12:19}
\crossref{Josh}{6}{11}{}
\crossref{Josh}{6}{12}{1Ch 15:26 Mt 24:13 Ga 6:9}
\crossref{Josh}{6}{13}{1Ch 15:26 Mt 24:13 Ga 6:9}
\crossref{Josh}{6}{14}{Jos 6:3,\allowbreak11,\allowbreak15}
\crossref{Josh}{6}{15}{Ps 119:147 Mt 28:1 2Pe 1:19}
\crossref{Josh}{6}{16}{Jos 6:5 Jud 7:20-\allowbreak22 2Ch 13:15; 20:22,\allowbreak23}
\crossref{Josh}{6}{17}{Jos 7:1 Le 27:28,\allowbreak29 Nu 21:2,\allowbreak3 1Co 2:7 Ezr 10:8 Isa 34:6}
\crossref{Josh}{6}{18}{Ro 12:9 2Co 6:17 Eph 5:11 Jas 1:27 1Jo 5:21}
\crossref{Josh}{6}{19}{2Sa 8:11 1Ch 18:11; 26:20,\allowbreak26,\allowbreak28; 28:12 2Ch 15:18; 31:12 Isa 23:17,\allowbreak18}
\crossref{Josh}{6}{20}{Jos 6:5 2Co 10:4,\allowbreak5 Heb 11:30}
\crossref{Josh}{6}{21}{Jos 9:24,\allowbreak25; 10:28,\allowbreak39; 11:14 De 2:34; 7:2,\allowbreak3,\allowbreak16; 20:16,\allowbreak17 1Sa 15:3,\allowbreak8,\allowbreak18}
\crossref{Josh}{6}{22}{Jos 6:17; 2:1}
\crossref{Josh}{6}{23}{Jos 2:18 Ge 12:2; 18:24; 19:29 Ac 27:24 Heb 11:7}
\crossref{Josh}{6}{24}{Jos 8:28 De 13:16 2Ki 25:9 Re 17:16; 18:8}
\crossref{Josh}{6}{25}{Jos 11:19,\allowbreak20 Jud 1:24,\allowbreak25 Ac 2:21 Heb 11:31}
\crossref{Josh}{6}{26}{1Ki 16:34 Mal 1:4}
\crossref{Josh}{6}{27}{Jos 1:5,\allowbreak9 Ge 39:2,\allowbreak3,\allowbreak21 De 31:6 Mt 18:20; 28:20 Ac 18:9,\allowbreak10}
\crossref{Josh}{7}{1}{Jos 7:20,\allowbreak21; 22:16 2Ch 24:18 Ezr 9:6 Da 9:7}
\crossref{Josh}{7}{2}{Jos 12:9 Ge 12:8}
\crossref{Josh}{7}{3}{Pr 13:4; 21:25 Lu 13:24 Heb 4:11; 6:11,\allowbreak12 2Pe 1:5,\allowbreak10}
\crossref{Josh}{7}{4}{Le 26:17 De 28:25; 32:30 Isa 30:17; 59:2}
\crossref{Josh}{7}{5}{De 1:44}
\crossref{Josh}{7}{6}{Ge 37:29,\allowbreak34 Nu 14:6 2Sa 13:31 Ezr 9:3-\allowbreak5 Es 4:1 Job 1:20 Ac 14:14}
\crossref{Josh}{7}{7}{Ex 5:22,\allowbreak23 Nu 14:3 2Ki 3:10 Ps 116:11 Jer 12:1,\allowbreak2 Heb 12:5}
\crossref{Josh}{7}{8}{Ezr 9:10 Hab 2:1 Ro 3:5,\allowbreak6}
\crossref{Josh}{7}{9}{Ex 32:12 Nu 14:13}
\crossref{Josh}{7}{10}{Ex 14:15 1Sa 15:22; 16:1 1Ch 22:16}
\crossref{Josh}{7}{11}{Jos 7:1,\allowbreak20,\allowbreak21}
\crossref{Josh}{7}{12}{Jos 22:18-\allowbreak20 Nu 14:45 Jud 2:4 Ps 5:4,\allowbreak5 Pr 28:1 Isa 59:2 Hab 1:13}
\crossref{Josh}{7}{13}{Jos 3:5 Ex 19:10-\allowbreak15 La 3:40,\allowbreak41 Joe 2:16,\allowbreak17 Zep 2:1,\allowbreak2}
\crossref{Josh}{7}{14}{Jos 7:17,\allowbreak18 1Sa 10:19-\allowbreak21; 14:38-\allowbreak42 Pr 16:33 Jon 1:7 Ac 1:24-\allowbreak26}
\crossref{Josh}{7}{15}{Jos 7:25,\allowbreak26 De 13:15,\allowbreak16 1Sa 14:38,\allowbreak39}
\crossref{Josh}{7}{16}{Jos 3:1 Ge 22:3 Ps 119:60 Ec 9:10}
\crossref{Josh}{7}{17}{Ge 38:30}
\crossref{Josh}{7}{18}{Nu 32:23 1Sa 14:42 Pr 13:21 Jer 2:26 Ac 5:1-\allowbreak10}
\crossref{Josh}{7}{19}{2Ti 2:25 Tit 2:2 Jas 1:20 1Pe 3:8,\allowbreak9}
\crossref{Josh}{7}{20}{Ge 42:21 Ex 10:16 1Sa 15:24,\allowbreak30 Job 7:20; 33:27 Ps 38:18}
\crossref{Josh}{7}{21}{Ge 3:6; 6:2 2Sa 11:2 Job 31:1 Ps 119:37 Pr 23:31; 28:22 Mt 5:28,\allowbreak29}
\crossref{Josh}{7}{22}{}
\crossref{Josh}{7}{23}{}
\crossref{Josh}{7}{24}{Jos 7:1 Job 20:15 Pr 15:27 Ec 5:13 Eze 22:13,\allowbreak14 1Ti 6:9,\allowbreak10}
\crossref{Josh}{7}{25}{Jos 7:11-\allowbreak13; 6:18 Ge 34:30 1Ki 18:17,\allowbreak18 1Ch 2:7 Hab 2:6-\allowbreak9 Ga 5:12}
\crossref{Josh}{7}{26}{Jos 8:29; 10:27 2Sa 18:17 La 3:52}
\crossref{Josh}{8}{1}{Jos 1:9; 7:6,\allowbreak7,\allowbreak9 De 1:21; 7:18; 31:8 Ps 27:1; 46:11 Isa 12:2; 41:10-\allowbreak16}
\crossref{Josh}{8}{2}{Jos 8:24,\allowbreak28,\allowbreak29; 6:21; 10:1,\allowbreak28 De 3:2}
\crossref{Josh}{8}{3}{Mt 24:39,\allowbreak50; 25:6 1Th 5:2 2Pe 3:10}
\crossref{Josh}{8}{4}{Jos 8:16 Jud 9:25; 20:29,\allowbreak33,\allowbreak36 1Sa 15:2,\allowbreak5 Ac 23:21}
\crossref{Josh}{8}{5}{Jos 7:5}
\crossref{Josh}{8}{6}{Jos 8:16}
\crossref{Josh}{8}{7}{Jos 8:1 2Ki 5:1 Pr 21:30,\allowbreak31}
\crossref{Josh}{8}{8}{Jos 8:28; 6:24}
\crossref{Josh}{8}{9}{Jos 8:12; 7:2 Ge 12:8}
\crossref{Josh}{8}{10}{Jos 3:1; 6:12; 7:16 Ps 119:60}
\crossref{Josh}{8}{11}{Jos 8:1-\allowbreak5}
\crossref{Josh}{8}{12}{Jos 8:2,\allowbreak3}
\crossref{Josh}{8}{13}{Jos 8:8,\allowbreak12}
\crossref{Josh}{8}{14}{Jos 8:5,\allowbreak16}
\crossref{Josh}{8}{15}{Jos 18:12}
\crossref{Josh}{8}{16}{Jud 20:36-\allowbreak39}
\crossref{Josh}{8}{17}{Jos 8:3,\allowbreak24,\allowbreak25; 11:20 De 2:30 Job 5:13 Isa 19:11-\allowbreak13}
\crossref{Josh}{8}{18}{Jos 8:7,\allowbreak26 Ex 8:5; 17:11 Job 15:25}
\crossref{Josh}{8}{19}{Jos 8:6-\allowbreak8}
\crossref{Josh}{8}{20}{Ge 19:28 Isa 34:10 Re 18:9; 19:3}
\crossref{Josh}{8}{21}{}
\crossref{Josh}{8}{22}{}
\crossref{Josh}{8}{23}{Jos 8:29; 10:17 1Sa 15:8 Re 19:20}
\crossref{Josh}{8}{24}{}
\crossref{Josh}{8}{25}{}
\crossref{Josh}{8}{26}{Jos 8:18 Ex 17:11,\allowbreak12}
\crossref{Josh}{8}{27}{Jos 8:2; 11:4 Nu 31:22,\allowbreak26 Ps 50:10 Mt 20:15}
\crossref{Josh}{8}{28}{De 13:16 2Ki 19:25 Isa 17:1; 25:2 Jer 9:11; 49:2; 50:26 Mic 3:12}
\crossref{Josh}{8}{29}{Jos 10:27}
\crossref{Josh}{8}{30}{Ge 8:20; 12:7,\allowbreak8}
\crossref{Josh}{8}{31}{Jos 8:34,\allowbreak25; 1:8 2Ki 14:6; 22:8 2Ch 25:4; 35:12 Ezr 6:18 Ne 13:1}
\crossref{Josh}{8}{32}{De 27:2,\allowbreak3,\allowbreak8}
\crossref{Josh}{8}{33}{Jos 23:2; 24:1 De 27:12,\allowbreak13; 29:10,\allowbreak11}
\crossref{Josh}{8}{34}{De 31:10-\allowbreak12 Ne 8:2,\allowbreak3; 9:3; 13:1}
\crossref{Josh}{8}{35}{De 4:2 Jer 26:2 Ac 20:27}
\crossref{Josh}{9}{1}{Jos 10:2-\allowbreak5,\allowbreak23,\allowbreak28-\allowbreak39; 11:1-\allowbreak5,\allowbreak10,\allowbreak11; 12:7-\allowbreak24}
\crossref{Josh}{9}{2}{2Ch 20:1 Ps 2:1,\allowbreak2; 83:2-\allowbreak8 Pr 11:21 Isa 8:9,\allowbreak10,\allowbreak12; 54:15 Joe 3:9-\allowbreak13}
\crossref{Josh}{9}{3}{Jos 9:17; 10:2 2Sa 21:1,\allowbreak2}
\crossref{Josh}{9}{4}{Ge 34:13 1Ki 20:31-\allowbreak33 Mt 10:16 Lu 16:8}
\crossref{Josh}{9}{5}{9:13 De 29:5; 33:25 Lu 15:22}
\crossref{Josh}{9}{6}{Jos 5:10; 10:43}
\crossref{Josh}{9}{7}{Jos 11:19 Ge 10:17; 34:2 Ex 3:8}
\crossref{Josh}{9}{8}{9:11,\allowbreak23,\allowbreak25,\allowbreak27 Ge 9:25,\allowbreak26 De 20:11 1Ki 9:20,\allowbreak21 2Ki 10:5}
\crossref{Josh}{9}{9}{De 20:15}
\crossref{Josh}{9}{10}{Nu 21:24-\allowbreak35 De 2:30-\allowbreak37; 3:1-\allowbreak7}
\crossref{Josh}{9}{11}{Es 8:17}
\crossref{Josh}{9}{12}{9:4,\allowbreak5}
\crossref{Josh}{9}{13}{}
\crossref{Josh}{9}{14}{}
\crossref{Josh}{9}{15}{Jos 2:12-\allowbreak19; 6:22-\allowbreak25; 11:19 De 20:10,\allowbreak11 2Sa 21:2 Jer 18:7,\allowbreak8}
\crossref{Josh}{9}{16}{Pr 12:19}
\crossref{Josh}{9}{17}{Jos 10:2; 18:25-\allowbreak28 1Ch 21:29 2Ch 1:3 Ezr 2:25 Ne 7:29}
\crossref{Josh}{9}{18}{2Sa 21:7 Ps 15:4 Ec 5:2,\allowbreak6; 9:2}
\crossref{Josh}{9}{19}{9:20 Ec 8:2; 9:2 Jer 4:2}
\crossref{Josh}{9}{20}{2Sa 21:1-\allowbreak6 2Ch 36:13 Pr 20:25 Eze 17:12-\allowbreak21 Zec 5:3,\allowbreak4 Mal 3:5}
\crossref{Josh}{9}{21}{9:23,\allowbreak27 De 29:11 2Ch 2:17,\allowbreak18}
\crossref{Josh}{9}{22}{Ge 3:13,\allowbreak14; 27:35,\allowbreak36,\allowbreak41-\allowbreak45; 29:25 2Co 11:3}
\crossref{Josh}{9}{23}{Ge 9:25,\allowbreak26 Le 27:28,\allowbreak29}
\crossref{Josh}{9}{24}{Ex 23:31-\allowbreak33 Nu 33:51,\allowbreak52,\allowbreak55,\allowbreak56 De 7:1,\allowbreak2,\allowbreak23,\allowbreak24; 20:15-\allowbreak17}
\crossref{Josh}{9}{25}{Ge 16:6 Jud 8:15 2Sa 24:14 Isa 47:6 Jer 26:14; 38:5}
\crossref{Josh}{9}{26}{}
\crossref{Josh}{9}{27}{9:21,\allowbreak23 1Ch 9:2 Ezr 2:43; 8:20 Ne 7:60; 11:3}
\crossref{Josh}{10}{1}{Ge 14:18 Heb 7:1}
\crossref{Josh}{10}{2}{Jos 2:9-\allowbreak13,\allowbreak24 Ex 15:14-\allowbreak16 De 11:25; 28:10 Ps 48:4-\allowbreak6 Pr 1:26,\allowbreak27}
\crossref{Josh}{10}{3}{10:1,\allowbreak5; 12:10-\allowbreak13; 15:35-\allowbreak39,\allowbreak54,\allowbreak63; 18:28}
\crossref{Josh}{10}{4}{Isa 8:9,\allowbreak10; 41:5-\allowbreak7 Ac 9:24-\allowbreak27; 21:28 Re 16:14; 20:8-\allowbreak10}
\crossref{Josh}{10}{5}{10:6}
\crossref{Josh}{10}{6}{Jos 5:10; 9:6}
\crossref{Josh}{10}{7}{Isa 8:12,\allowbreak14}
\crossref{Josh}{10}{8}{Jos 1:5-\allowbreak9; 8:1; 11:6 De 3:2; 20:1-\allowbreak4 Jud 4:14,\allowbreak15 Ps 27:1,\allowbreak2}
\crossref{Josh}{10}{9}{1Sa 11:9-\allowbreak11 Pr 22:29; 24:11,\allowbreak12 Ec 9:10 2Ti 2:3; 4:2}
\crossref{Josh}{10}{10}{Jos 11:8 Jud 4:15 1Sa 7:10-\allowbreak12 2Ch 14:12 Ps 18:14; 44:3; 78:55}
\crossref{Josh}{10}{11}{Ge 19:24 Ex 9:22-\allowbreak26 Jud 5:20 Ps 11:6; 18:12-\allowbreak14; 77:17,\allowbreak18}
\crossref{Josh}{10}{12}{10:13 De 4:19; 17:3 Job 9:7; 31:26,\allowbreak27 Ps 19:4; 74:16; 148:3}
\crossref{Josh}{10}{13}{Nu 31:2 Jud 5:2; 16:28 Es 8:13 Lu 18:7 Re 6:10}
\crossref{Josh}{10}{14}{2Ki 20:10,\allowbreak11 Isa 38:8}
\crossref{Josh}{10}{15}{10:6,\allowbreak43}
\crossref{Josh}{10}{16}{Ps 48:4-\allowbreak6; 139:7-\allowbreak10 Isa 2:10-\allowbreak12 Am 9:2 Re 6:15}
\crossref{Josh}{10}{17}{}
\crossref{Josh}{10}{18}{10:22 Jud 9:46-\allowbreak49 Job 21:30 Am 5:19; 9:1 Mt 27:66}
\crossref{Josh}{10}{19}{Ps 18:37-\allowbreak41 Jer 48:10}
\crossref{Josh}{10}{20}{10:10; 8:24 2Ch 13:17}
\crossref{Josh}{10}{21}{10:15-\allowbreak17}
\crossref{Josh}{10}{22}{1Sa 15:32}
\crossref{Josh}{10}{23}{10:1,\allowbreak3,\allowbreak5}
\crossref{Josh}{10}{24}{De 33:29 Jud 8:20 Ps 2:8-\allowbreak12; 18:40; 91:13; 107:40; 110:1,\allowbreak5}
\crossref{Josh}{10}{25}{Jos 1:9 De 31:6-\allowbreak8 1Sa 17:37 Ps 63:9; 77:11 2Co 1:10 2Ti 4:17,\allowbreak18}
\crossref{Josh}{10}{26}{Jud 8:21 1Sa 15:33}
\crossref{Josh}{10}{27}{Jos 8:29 De 21:23 2Sa 18:17}
\crossref{Josh}{10}{28}{10:32,\allowbreak35,\allowbreak37,\allowbreak39; 6:21 De 7:2,\allowbreak16; 20:16,\allowbreak17 Ps 21:8,\allowbreak9; 110:1 Lu 19:27}
\crossref{Josh}{10}{29}{10:28; 6:21; 8:2,\allowbreak29}
\crossref{Josh}{10}{30}{}
\crossref{Josh}{10}{31}{}
\crossref{Josh}{10}{32}{10:30}
\crossref{Josh}{10}{33}{}
\crossref{Josh}{10}{34}{}
\crossref{Josh}{10}{35}{10:32}
\crossref{Josh}{10}{36}{10:3,\allowbreak5; 14:13,\allowbreak14; 15:13,\allowbreak54; 21:13 Ge 13:18 Nu 13:22 Jud 1:10}
\crossref{Josh}{10}{37}{10:35}
\crossref{Josh}{10}{38}{}
\crossref{Josh}{10}{39}{10:33,\allowbreak37,\allowbreak40; 11:8 De 3:3 2Ki 10:11 Ob 1:18}
\crossref{Josh}{10}{40}{Jos 15:21-\allowbreak63; 18:21-\allowbreak28; 19:1-\allowbreak8,\allowbreak40-\allowbreak48}
\crossref{Josh}{10}{41}{Jos 14:6,\allowbreak7 Nu 13:26; 32:8; 34:4 De 9:23}
\crossref{Josh}{10}{42}{10:14 Ex 14:14,\allowbreak25 De 20:4 Ps 44:3-\allowbreak8; 46:1,\allowbreak7,\allowbreak11; 80:3; 118:6}
\crossref{Josh}{10}{43}{10:15; 4:19 1Sa 11:14}
\crossref{Josh}{11}{1}{11:10; 12:19; 19:36 Jud 4:2,\allowbreak17}
\crossref{Josh}{11}{2}{11:21; 10:6,\allowbreak40 Lu 1:39}
\crossref{Josh}{11}{3}{Jos 15:63 Nu 13:29 2Sa 24:16}
\crossref{Josh}{11}{4}{Ge 22:17; 32:12 Jud 7:12 1Sa 13:5 2Sa 17:11 1Ki 4:20}
\crossref{Josh}{11}{5}{Ps 3:1; 118:10-\allowbreak12 Isa 8:9 Re 16:14}
\crossref{Josh}{11}{6}{Jos 10:8 Ps 27:1,\allowbreak2; 46:11}
\crossref{Josh}{11}{7}{Jos 10:9 1Th 5:2,\allowbreak3}
\crossref{Josh}{11}{8}{Jos 21:44}
\crossref{Josh}{11}{9}{11:6 Eze 39:9,\allowbreak10}
\crossref{Josh}{11}{10}{11:1 Jud 4:2}
\crossref{Josh}{11}{11}{Jos 10:40}
\crossref{Josh}{11}{12}{Jos 10:28,\allowbreak30,\allowbreak32,\allowbreak35,\allowbreak37,\allowbreak39,\allowbreak40}
\crossref{Josh}{11}{13}{}
\crossref{Josh}{11}{14}{Jos 8:27 Nu 31:9 De 6:10,\allowbreak11; 20:14}
\crossref{Josh}{11}{15}{11:12 Ex 34:11-\allowbreak13}
\crossref{Josh}{11}{16}{Ge 15:18-\allowbreak21 Nu 34:2-\allowbreak13 De 34:2,\allowbreak3}
\crossref{Josh}{11}{17}{Ge 32:3 De 2:1; 33:2}
\crossref{Josh}{11}{18}{}
\crossref{Josh}{11}{19}{Jos 9:3-\allowbreak27}
\crossref{Josh}{11}{20}{Ex 4:21; 9:16 De 2:30 Jud 14:4 1Sa 2:25 1Ki 12:15; 22:20-\allowbreak23}
\crossref{Josh}{11}{21}{Jos 14:12-\allowbreak14; 15:13,\allowbreak14 Nu 13:22,\allowbreak23 De 1:28; 2:21; 9:2 Jud 1:10,\allowbreak11,\allowbreak20}
\crossref{Josh}{11}{22}{Jud 3:3 1Sa 17:4 2Sa 21:16-\allowbreak22 1Ch 18:1; 29:4-\allowbreak8}
\crossref{Josh}{11}{23}{Ex 23:27-\allowbreak31; 34:11 Nu 34:2-\allowbreak13 De 11:23-\allowbreak25}
\crossref{Josh}{12}{1}{Jos 1:15; 22:4}
\crossref{Josh}{12}{2}{Nu 21:23-\allowbreak30 De 2:24-\allowbreak37; 3:6-\allowbreak17 Ne 9:22 Ps 135:11; 136:19,\allowbreak20}
\crossref{Josh}{12}{3}{Jos 11:2 De 3:17 Joh 6:1}
\crossref{Josh}{12}{4}{Nu 21:33-\allowbreak35 De 3:1-\allowbreak7,\allowbreak10}
\crossref{Josh}{12}{5}{12:1; 11:3 De 3:8,\allowbreak9; 4:47,\allowbreak48}
\crossref{Josh}{12}{6}{Nu 21:24-\allowbreak35}
\crossref{Josh}{12}{7}{12:1; 3:17; 9:1}
\crossref{Josh}{12}{8}{Jos 10:40; 11:16}
\crossref{Josh}{12}{9}{Jos 6:2-\allowbreak21}
\crossref{Josh}{12}{10}{Jos 10:23}
\crossref{Josh}{12}{11}{}
\crossref{Josh}{12}{12}{Jos 10:3,\allowbreak23; 15:39}
\crossref{Josh}{12}{13}{Jos 10:3,\allowbreak38}
\crossref{Josh}{12}{14}{Nu 14:45; 21:3}
\crossref{Josh}{12}{15}{1Sa 22:1}
\crossref{Josh}{12}{16}{Jos 10:28}
\crossref{Josh}{12}{17}{Jos 15:34}
\crossref{Josh}{12}{18}{Jos 19:30 1Sa 4:1}
\crossref{Josh}{12}{19}{Jos 11:1}
\crossref{Josh}{12}{20}{Jos 11:1; 19:15}
\crossref{Josh}{12}{21}{Jos 17:11 Jud 5:19}
\crossref{Josh}{12}{22}{Jos 15:23; 19:37; 21:32}
\crossref{Josh}{12}{23}{Jos 11:2; 17:11}
\crossref{Josh}{12}{24}{1Ki 16:23 2Ki 15:14}
\crossref{Josh}{13}{1}{Jos 14:10; 23:1,\allowbreak2; 24:29 Ge 18:11 1Ki 1:1 Lu 1:7}
\crossref{Josh}{13}{2}{Ex 23:29-\allowbreak31 De 11:23,\allowbreak24 Jud 3:1}
\crossref{Josh}{13}{3}{Jer 2:18}
\crossref{Josh}{13}{4}{Jos 10:40; 11:3; 12:7,\allowbreak8}
\crossref{Josh}{13}{5}{De 1:7; 3:25}
\crossref{Josh}{13}{6}{Jos 11:8}
\crossref{Josh}{13}{7}{Nu 26:53-\allowbreak56; 33:54; 32:2-\allowbreak14 Eze 47:13-\allowbreak23; 48:23-\allowbreak29}
\crossref{Josh}{13}{8}{Jos 4:12; 22:4 Nu 32:33-\allowbreak42 De 3:12-\allowbreak17}
\crossref{Josh}{13}{9}{13:16; 12:2 De 3:12,\allowbreak16}
\crossref{Josh}{13}{10}{Nu 21:24-\allowbreak26}
\crossref{Josh}{13}{11}{Jos 12:2-\allowbreak5 De 4:47,\allowbreak48 1Ch 2:23}
\crossref{Josh}{13}{12}{Jos 12:4 De 3:10,\allowbreak11}
\crossref{Josh}{13}{13}{13:11; 23:12,\allowbreak13 Nu 33:55 Jud 2:1-\allowbreak3 2Sa 3:3; 13:37,\allowbreak38}
\crossref{Josh}{13}{14}{Jos 14:3,\allowbreak4 Nu 18:20-\allowbreak24 De 10:9; 12:12,\allowbreak19; 18:2}
\crossref{Josh}{13}{15}{13:15}
\crossref{Josh}{13}{16}{13:9; 12:2 Nu 21:28-\allowbreak30; 32:33-\allowbreak38 De 3:12 Isa 15:1,\allowbreak2,\allowbreak4; 16:7-\allowbreak9}
\crossref{Josh}{13}{17}{}
\crossref{Josh}{13}{18}{1Ch 6:78,\allowbreak79}
\crossref{Josh}{13}{19}{}
\crossref{Josh}{13}{20}{Nu 25:3 De 4:46}
\crossref{Josh}{13}{21}{De 3:10}
\crossref{Josh}{13}{22}{Nu 22:5-\allowbreak7; 24:1; 31:8 2Pe 2:15 Jude 1:11 Re 2:14; 19:20}
\crossref{Josh}{13}{23}{13:23}
\crossref{Josh}{13}{24}{Nu 32:34-\allowbreak36}
\crossref{Josh}{13}{25}{Nu 32:35}
\crossref{Josh}{13}{26}{Jos 20:8 Ge 31:49 Jud 10:17; 11:11,\allowbreak29 1Ki 22:3}
\crossref{Josh}{13}{27}{Nu 32:3,\allowbreak36}
\crossref{Josh}{13}{28}{13:28}
\crossref{Josh}{13}{29}{13:29}
\crossref{Josh}{13}{30}{13:26 Nu 32:39-\allowbreak41 De 3:13-\allowbreak15 1Ch 2:21-\allowbreak23}
\crossref{Josh}{13}{31}{Jos 12:4}
\crossref{Josh}{13}{32}{13:32}
\crossref{Josh}{13}{33}{}
\crossref{Josh}{14}{1}{Nu 34:17-\allowbreak29}
\crossref{Josh}{14}{2}{}
\crossref{Josh}{14}{3}{Jos 13:8 Nu 32:29-\allowbreak42 De 3:12-\allowbreak17}
\crossref{Josh}{14}{4}{Ge 48:5 1Ch 5:1,\allowbreak2}
\crossref{Josh}{14}{5}{}
\crossref{Josh}{14}{6}{Jos 4:19; 10:43}
\crossref{Josh}{14}{7}{Nu 13:6,\allowbreak16-\allowbreak20}
\crossref{Josh}{14}{8}{14:14 Nu 14:24 De 1:36 Re 14:4}
\crossref{Josh}{14}{9}{Jos 1:3 Nu 13:22; 14:22-\allowbreak24}
\crossref{Josh}{14}{10}{Jos 11:18 Nu 14:33,\allowbreak34}
\crossref{Josh}{14}{11}{De 31:2; 34:7 Ps 90:10; 103:5}
\crossref{Josh}{14}{12}{Jos 11:21,\allowbreak22 Nu 13:28,\allowbreak33}
\crossref{Josh}{14}{13}{Jos 22:6 Ge 47:7,\allowbreak10 1Sa 1:17 So 6:9}
\crossref{Josh}{14}{14}{14:8,\allowbreak9 1Co 15:58}
\crossref{Josh}{14}{15}{Jos 15:13 Ge 23:2}
\crossref{Josh}{15}{1}{Nu 33:36,\allowbreak37; 34:3-\allowbreak5 Eze 47:19}
\crossref{Josh}{15}{2}{Jos 3:16 Ge 14:3 Nu 34:3 Eze 47:8,\allowbreak18}
\crossref{Josh}{15}{3}{Nu 34:4 Jud 1:36}
\crossref{Josh}{15}{4}{Jos 13:3 Ex 23:31}
\crossref{Josh}{15}{5}{Nu 34:12}
\crossref{Josh}{15}{6}{Jos 18:17}
\crossref{Josh}{15}{7}{15:15; 10:38,\allowbreak39}
\crossref{Josh}{15}{8}{15:63; 18:28 Jud 1:8,\allowbreak21; 19:10}
\crossref{Josh}{15}{9}{Jos 18:15}
\crossref{Josh}{15}{10}{15:57 Ge 38:13 Jud 14:1,\allowbreak5}
\crossref{Josh}{15}{11}{15:45; 19:43 1Sa 5:10; 7:14 2Ki 1:2,\allowbreak3,\allowbreak6,\allowbreak16}
\crossref{Josh}{15}{12}{15:47 Nu 34:6,\allowbreak7 De 11:24 Eze 47:20}
\crossref{Josh}{15}{13}{Jos 14:6-\allowbreak15 Nu 13:30; 14:23,\allowbreak24 De 1:34-\allowbreak36}
\crossref{Josh}{15}{14}{Jos 10:36,\allowbreak37; 11:21 Nu 13:22,\allowbreak23 Jud 1:10,\allowbreak20}
\crossref{Josh}{15}{15}{Jos 10:3,\allowbreak38 Jud 1:11-\allowbreak13}
\crossref{Josh}{15}{16}{Jud 1:6,\allowbreak12,\allowbreak13}
\crossref{Josh}{15}{17}{Jud 1:13; 3:9,\allowbreak11}
\crossref{Josh}{15}{18}{Ge 24:64 1Sa 25:23}
\crossref{Josh}{15}{19}{Jud 1:14,\allowbreak15}
\crossref{Josh}{15}{20}{Ge 49:8-\allowbreak12 De 33:7}
\crossref{Josh}{15}{21}{Ne 11:25}
\crossref{Josh}{15}{22}{}
\crossref{Josh}{15}{23}{Jos 12:22 Nu 33:37 De 1:19}
\crossref{Josh}{15}{24}{1Sa 15:4}
\crossref{Josh}{15}{25}{}
\crossref{Josh}{15}{26}{}
\crossref{Josh}{15}{27}{Ne 11:26}
\crossref{Josh}{15}{28}{Jos 19:3 1Ch 4:28}
\crossref{Josh}{15}{29}{15:9-\allowbreak11; 19:3}
\crossref{Josh}{15}{30}{Jos 19:4 Nu 14:45 De 1:44 Jud 1:17}
\crossref{Josh}{15}{31}{Jos 19:5 1Sa 27:6; 30:1 1Ch 12:1}
\crossref{Josh}{15}{32}{Ne 11:29}
\crossref{Josh}{15}{33}{}
\crossref{Josh}{15}{34}{15:53; 12:17}
\crossref{Josh}{15}{35}{Jos 10:3,\allowbreak23; 12:11 Ne 11:29}
\crossref{Josh}{15}{36}{1Sa 17:52}
\crossref{Josh}{15}{37}{}
\crossref{Josh}{15}{38}{Ge 31:48,\allowbreak49 Jud 20:1; 21:5 1Sa 7:5,\allowbreak6,\allowbreak16; 10:17}
\crossref{Josh}{15}{39}{Jos 10:3,\allowbreak31,\allowbreak32; 12:11 2Ki 18:14,\allowbreak17; 19:8}
\crossref{Josh}{15}{40}{}
\crossref{Josh}{15}{41}{Jos 10:21,\allowbreak28; 12:16}
\crossref{Josh}{15}{42}{Jos 10:29; 12:15 2Ki 8:22}
\crossref{Josh}{15}{43}{}
\crossref{Josh}{15}{44}{1Sa 23:1-\allowbreak14}
\crossref{Josh}{15}{45}{Jos 13:3 1Sa 5:10; 6:17 Am 1:8 Zep 2:4 Zec 9:5-\allowbreak7}
\crossref{Josh}{15}{46}{1Sa 5:1,\allowbreak6 2Ch 26:6 Ne 13:23,\allowbreak24 Isa 20:1 Am 1:8}
\crossref{Josh}{15}{47}{Jud 16:1-\allowbreak21 Jer 47:1,\allowbreak5 Am 1:6,\allowbreak7 Zep 2:4 Ac 8:26}
\crossref{Josh}{15}{48}{Jos 21:14}
\crossref{Josh}{15}{49}{15:15 Jud 1:11}
\crossref{Josh}{15}{50}{}
\crossref{Josh}{15}{51}{Jos 10:41; 11:16}
\crossref{Josh}{15}{52}{Isa 21:11}
\crossref{Josh}{15}{53}{15:53}
\crossref{Josh}{15}{54}{15:13; 14:15 Ge 23:2}
\crossref{Josh}{15}{55}{1Sa 23:25; 25:2,\allowbreak7 2Ch 26:10 Isa 35:2}
\crossref{Josh}{15}{56}{}
\crossref{Josh}{15}{57}{15:10 Ge 38:12 Jer 14:1}
\crossref{Josh}{15}{58}{1Ch 4:39}
\crossref{Josh}{15}{59}{}
\crossref{Josh}{15}{60}{Jos 18:14 1Sa 7:1}
\crossref{Josh}{15}{61}{15:6; 18:18}
\crossref{Josh}{15}{62}{}
\crossref{Josh}{15}{63}{Jud 1:8,\allowbreak21 2Sa 5:6-\allowbreak9 1Ch 11:4-\allowbreak8 Ro 7:14-\allowbreak21}
\crossref{Josh}{16}{1}{Jos 8:15; 15:61; 18:12 2Ki 2:19-\allowbreak21}
\crossref{Josh}{16}{2}{Jos 18:13 Ge 28:19 Jud 1:22-\allowbreak26}
\crossref{Josh}{16}{3}{}
\crossref{Josh}{16}{4}{Jos 17:14}
\crossref{Josh}{16}{5}{16:2; 18:13}
\crossref{Josh}{16}{6}{Jos 17:7}
\crossref{Josh}{16}{7}{1Ch 7:28}
\crossref{Josh}{16}{8}{Jos 12:17; 17:8}
\crossref{Josh}{16}{9}{Jos 17:9}
\crossref{Josh}{16}{10}{Jos 15:63 Jud 1:29 1Ki 9:16,\allowbreak21}
\crossref{Josh}{17}{1}{Ge 41:51; 46:20; 48:18 De 21:17}
\crossref{Josh}{17}{2}{Nu 26:29-\allowbreak32}
\crossref{Josh}{17}{3}{Nu 26:33; 27:1; 36:2-\allowbreak11}
\crossref{Josh}{17}{4}{Jos 14:1 Nu 34:17-\allowbreak29}
\crossref{Josh}{17}{5}{Jos 13:29-\allowbreak31 Nu 32:30-\allowbreak42}
\crossref{Josh}{17}{6}{}
\crossref{Josh}{17}{7}{}
\crossref{Josh}{17}{8}{Jos 12:17; 15:34,\allowbreak53; 16:8}
\crossref{Josh}{17}{9}{Jos 16:9}
\crossref{Josh}{17}{10}{}
\crossref{Josh}{17}{11}{Jos 16:9 1Ch 7:29}
\crossref{Josh}{17}{12}{Jos 15:63; 16:10 Ex 23:29-\allowbreak33 Nu 33:52-\allowbreak56 Jud 1:27,\allowbreak28 Ro 6:12-\allowbreak14}
\crossref{Josh}{17}{13}{Jud 1:28 2Sa 3:1 Eph 6:10 Php 4:13 2Pe 3:18}
\crossref{Josh}{17}{14}{Ge 48:22 Nu 26:34-\allowbreak37 De 33:13-\allowbreak17}
\crossref{Josh}{17}{15}{Lu 12:48}
\crossref{Josh}{17}{16}{17:18 Jud 1:19; 4:3}
\crossref{Josh}{17}{17}{17:14}
\crossref{Josh}{17}{18}{17:15; 15:9; 20:7}
\crossref{Josh}{18}{1}{Jud 18:31 1Sa 1:3,\allowbreak24; 4:3,\allowbreak4 1Ki 2:27; 14:2,\allowbreak4 Ps 78:66}
\crossref{Josh}{18}{2}{}
\crossref{Josh}{18}{3}{Jud 18:9 Pr 2:2-\allowbreak6; 10:4; 13:4; 15:19 Ec 9:10 Zep 3:16 Mt 20:6}
\crossref{Josh}{18}{4}{18:3; 3:12; 4:2 Nu 1:4; 13:2}
\crossref{Josh}{18}{5}{Jos 15:1-\allowbreak12; 19:1-\allowbreak9}
\crossref{Josh}{18}{6}{18:8,\allowbreak10; 14:2 Nu 26:54,\allowbreak55; 33:54; 34:13 Ps 105:11 Pr 16:33; 18:18}
\crossref{Josh}{18}{7}{Jos 13:14,\allowbreak33 Nu 18:20,\allowbreak23 De 10:9; 18:1,\allowbreak2}
\crossref{Josh}{18}{8}{Ge 13:17}
\crossref{Josh}{18}{9}{Ac 13:19}
\crossref{Josh}{18}{10}{18:6,\allowbreak8 Pr 18:18 Eze 47:22; 48:29 Mt 27:35 Ac 13:19}
\crossref{Josh}{18}{11}{Jos 15:1-\allowbreak8; 16:1-\allowbreak10 De 10:1-\allowbreak22; 13:12}
\crossref{Josh}{18}{12}{Jos 2:1; 3:16; 6:1; 16:1}
\crossref{Josh}{18}{13}{Jos 16:2 Ge 28:19 Jud 1:22-\allowbreak26}
\crossref{Josh}{18}{14}{Jos 15:9,\allowbreak60 1Sa 7:1,\allowbreak2 2Sa 6:2 1Ch 13:5,\allowbreak6}
\crossref{Josh}{18}{15}{Jos 15:9}
\crossref{Josh}{18}{16}{Jos 15:8 2Ki 23:10 2Ch 28:3; 33:6 Isa 30:33 Jer 7:31,\allowbreak32; 19:2,\allowbreak6,\allowbreak11}
\crossref{Josh}{18}{17}{}
\crossref{Josh}{18}{18}{Jos 15:6,\allowbreak61}
\crossref{Josh}{18}{19}{Jos 15:2}
\crossref{Josh}{18}{20}{}
\crossref{Josh}{18}{21}{18:12; 2:1; 6:1 Lu 10:30; 19:1}
\crossref{Josh}{18}{22}{18:18; 15:6}
\crossref{Josh}{18}{23}{}
\crossref{Josh}{18}{24}{}
\crossref{Josh}{18}{25}{Jos 9:17; 10:2 1Ki 3:4,\allowbreak5; 9:2 Isa 28:21}
\crossref{Josh}{18}{26}{Jos 9:17 Ezr 2:25}
\crossref{Josh}{18}{27}{}
\crossref{Josh}{18}{28}{2Sa 21:14}
\crossref{Josh}{19}{1}{Jos 18:6-\allowbreak11}
\crossref{Josh}{19}{2}{Jos 15:28 Ge 21:31 1Ch 4:28-\allowbreak30}
\crossref{Josh}{19}{3}{Jos 15:28,\allowbreak29}
\crossref{Josh}{19}{4}{Jos 15:30}
\crossref{Josh}{19}{5}{Jos 15:31 1Sa 27:6; 30:1}
\crossref{Josh}{19}{6}{Jos 15:32}
\crossref{Josh}{19}{7}{Jos 15:32}
\crossref{Josh}{19}{8}{1Ch 4:33}
\crossref{Josh}{19}{9}{Ex 16:18 2Co 8:14,\allowbreak15}
\crossref{Josh}{19}{10}{Jos 18:6,\allowbreak11}
\crossref{Josh}{19}{11}{Jos 12:22 1Ki 4:12 1Ch 6:68}
\crossref{Josh}{19}{12}{}
\crossref{Josh}{19}{13}{}
\crossref{Josh}{19}{14}{}
\crossref{Josh}{19}{15}{Jos 21:34,\allowbreak35 Jud 1:30}
\crossref{Josh}{19}{16}{19:16}
\crossref{Josh}{19}{17}{}
\crossref{Josh}{19}{18}{1Ki 21:1,\allowbreak15,\allowbreak16 2Ki 8:29; 9:15,\allowbreak30 Ho 1:4,\allowbreak5}
\crossref{Josh}{19}{19}{}
\crossref{Josh}{19}{20}{}
\crossref{Josh}{19}{21}{Jos 21:29}
\crossref{Josh}{19}{22}{19:12 Jud 4:6 1Ch 6:77 Jer 46:18}
\crossref{Josh}{19}{23}{19:23}
\crossref{Josh}{19}{24}{}
\crossref{Josh}{19}{25}{2Sa 2:16}
\crossref{Josh}{19}{26}{1Sa 15:12 1Ki 18:20,\allowbreak42 So 7:5 Isa 33:9; 35:2; 37:24 Jer 46:18}
\crossref{Josh}{19}{27}{1Sa 5:2}
\crossref{Josh}{19}{28}{19:30}
\crossref{Josh}{19}{29}{2Sa 5:11 Isa 23:1-\allowbreak18 Eze 26:1-\allowbreak28:26}
\crossref{Josh}{19}{30}{Jos 12:18; 13:4 1Sa 4:1 1Ki 20:30}
\crossref{Josh}{19}{31}{Ge 49:20 De 33:24,\allowbreak25}
\crossref{Josh}{19}{32}{}
\crossref{Josh}{19}{33}{Jud 4:11}
\crossref{Josh}{19}{34}{De 33:23}
\crossref{Josh}{19}{35}{Ge 10:18 Nu 13:21; 34:8; 34:8 1Ki 8:65}
\crossref{Josh}{19}{36}{Jos 11:1,\allowbreak20; 12:19}
\crossref{Josh}{19}{37}{Jos 12:22; 20:7}
\crossref{Josh}{19}{38}{19:22}
\crossref{Josh}{19}{39}{19:39}
\crossref{Josh}{19}{40}{}
\crossref{Josh}{19}{41}{Jud 13:2,\allowbreak25; 16:31; 18:2 1Ch 2:53 2Ch 11:10}
\crossref{Josh}{19}{42}{}
\crossref{Josh}{19}{43}{Ge 38:12 Jud 14:1,\allowbreak2}
\crossref{Josh}{19}{44}{Jos 21:23 1Ki 15:27; 16:15}
\crossref{Josh}{19}{45}{Jos 21:24}
\crossref{Josh}{19}{46}{Jon 1:3 Ac 9:36,\allowbreak43; 10:8}
\crossref{Josh}{19}{47}{Jud 1:34,\allowbreak35; 18:1-\allowbreak29}
\crossref{Josh}{19}{48}{}
\crossref{Josh}{19}{49}{Eze 45:7,\allowbreak8}
\crossref{Josh}{19}{50}{Jos 24:30 Jud 2:9}
\crossref{Josh}{19}{51}{Jos 14:1 Nu 34:17-\allowbreak29 Ps 47:3,\allowbreak4 Mt 20:23; 25:34 Joh 14:2,\allowbreak3; 17:2}
\crossref{Josh}{20}{1}{Jos 5:14; 6:2; 7:10; 13:1-\allowbreak7}
\crossref{Josh}{20}{2}{Ex 21:13,\allowbreak14 Nu 35:6,\allowbreak11-\allowbreak14 De 4:41-\allowbreak43; 19:2-\allowbreak13 Ro 8:1,\allowbreak33,\allowbreak34}
\crossref{Josh}{20}{3}{}
\crossref{Josh}{20}{4}{Ru 4:1,\allowbreak2 Job 5:4; 29:7 Pr 31:23 Jer 38:7}
\crossref{Josh}{20}{5}{Nu 35:12,\allowbreak25}
\crossref{Josh}{20}{6}{Nu 35:12,\allowbreak24,\allowbreak25 Heb 9:26}
\crossref{Josh}{20}{7}{Jos 21:21 Ge 33:18,\allowbreak19 2Ch 10:1}
\crossref{Josh}{20}{8}{Jos 21:36 De 4:43 1Ch 6:78}
\crossref{Josh}{20}{9}{Nu 35:15}
\crossref{Josh}{21}{1}{Jos 19:51 Ex 6:14,\allowbreak25}
\crossref{Josh}{21}{2}{Jos 18:1}
\crossref{Josh}{21}{3}{Ge 49:7 De 33:8-\allowbreak10 1Ch 6:54-\allowbreak81}
\crossref{Josh}{21}{4}{21:8-\allowbreak19; 24:33 1Ch 6:54-\allowbreak60}
\crossref{Josh}{21}{5}{21:20-\allowbreak26 Ge 46:11 Ex 6:16-\allowbreak25 Nu 3:27 1Ch 6:18,\allowbreak19,\allowbreak61,\allowbreak66-\allowbreak70}
\crossref{Josh}{21}{6}{21:27-\allowbreak33 Ex 6:16,\allowbreak17 Nu 3:21,\allowbreak22 1Ch 6:62,\allowbreak71-\allowbreak76}
\crossref{Josh}{21}{7}{21:34-\allowbreak40 Ex 6:19 Nu 3:20 1Ch 6:63,\allowbreak77-\allowbreak81}
\crossref{Josh}{21}{8}{21:3; 18:6 Nu 33:54; 35:3 Pr 16:33; 18:18}
\crossref{Josh}{21}{9}{21:13-\allowbreak18 1Ch 6:65}
\crossref{Josh}{21}{10}{21:4 Ex 6:18,\allowbreak20-\allowbreak26 Nu 3:2-\allowbreak4,\allowbreak19,\allowbreak27; 4:2}
\crossref{Josh}{21}{11}{1Ch 6:55}
\crossref{Josh}{21}{12}{Jos 14:13-\allowbreak15 1Ch 6:55-\allowbreak57}
\crossref{Josh}{21}{13}{1Ch 6:56}
\crossref{Josh}{21}{14}{}
\crossref{Josh}{21}{15}{Jos 15:51 1Ch 6:58}
\crossref{Josh}{21}{16}{Jos 15:42 1Ch 6:59}
\crossref{Josh}{21}{17}{Jos 9:3; 18:25 1Ch 6:60}
\crossref{Josh}{21}{18}{1Ch 6:60}
\crossref{Josh}{21}{19}{21:19}
\crossref{Josh}{21}{20}{21:5 1Ch 6:66}
\crossref{Josh}{21}{21}{Jos 20:7 Ge 33:19 Jud 9:1 1Ki 12:1}
\crossref{Josh}{21}{22}{Jos 16:3,\allowbreak5; 18:13,\allowbreak14 1Ch 6:68}
\crossref{Josh}{21}{23}{Jos 19:44,\allowbreak45}
\crossref{Josh}{21}{24}{Jos 10:12; 19:42}
\crossref{Josh}{21}{25}{Jos 17:11 Jud 5:19}
\crossref{Josh}{21}{26}{}
\crossref{Josh}{21}{27}{21:6}
\crossref{Josh}{21}{28}{Jos 19:12 1Ch 6:72,\allowbreak73}
\crossref{Josh}{21}{29}{}
\crossref{Josh}{21}{30}{Jos 19:25-\allowbreak28}
\crossref{Josh}{21}{31}{}
\crossref{Josh}{21}{32}{Jos 19:37; 20:7 1Ch 6:76}
\crossref{Josh}{21}{33}{}
\crossref{Josh}{21}{34}{21:7 1Ch 6:77}
\crossref{Josh}{21}{35}{}
\crossref{Josh}{21}{36}{Jos 20:8 De 4:43 1Ch 6:78,\allowbreak79}
\crossref{Josh}{21}{37}{}
\crossref{Josh}{21}{38}{Jos 20:8 1Ki 22:3 1Ch 6:80}
\crossref{Josh}{21}{39}{Jos 13:17,\allowbreak21 Nu 21:26-\allowbreak30; 32:37 1Ch 6:81}
\crossref{Josh}{21}{40}{}
\crossref{Josh}{21}{41}{Ge 49:7 Nu 35:1-\allowbreak8 De 33:10}
\crossref{Josh}{21}{42}{21:42}
\crossref{Josh}{21}{43}{Ge 12:7; 13:15; 15:13-\allowbreak21; 26:3,\allowbreak4; 28:4,\allowbreak13,\allowbreak14 Ex 3:8; 23:27-\allowbreak31}
\crossref{Josh}{21}{44}{Jos 1:15; 11:23; 22:4,\allowbreak9 De 7:22-\allowbreak24; 31:3-\allowbreak5 Heb 4:9}
\crossref{Josh}{21}{45}{Jos 23:14,\allowbreak15 Nu 23:19 1Ki 8:56 1Co 1:9 1Th 5:24 Tit 1:2 Heb 6:18}
\crossref{Josh}{22}{1}{Nu 32:18-\allowbreak33 De 29:7,\allowbreak8}
\crossref{Josh}{22}{2}{Nu 32:20-\allowbreak29 De 3:16-\allowbreak20}
\crossref{Josh}{22}{3}{Php 1:23-\allowbreak27}
\crossref{Josh}{22}{4}{Jos 21:43,\allowbreak44 De 12:9}
\crossref{Josh}{22}{5}{Ex 15:26 De 4:1,\allowbreak2,\allowbreak6,\allowbreak9; 6:6-\allowbreak9,\allowbreak17; 11:22 1Ch 28:7,\allowbreak8 Ps 106:3}
\crossref{Josh}{22}{6}{22:7,\allowbreak8; 14:13 Ge 14:19; 47:7,\allowbreak10 Ex 39:43 1Sa 2:20 2Sa 6:18,\allowbreak20}
\crossref{Josh}{22}{7}{Jos 13:29-\allowbreak31; 17:1-\allowbreak12}
\crossref{Josh}{22}{8}{De 8:9-\allowbreak14,\allowbreak17,\allowbreak18 2Ch 17:5; 32:27 Pr 3:16 1Co 15:58 Heb 11:26}
\crossref{Josh}{22}{9}{Jos 13:11,\allowbreak25,\allowbreak31 Nu 32:1,\allowbreak26,\allowbreak29,\allowbreak39,\allowbreak40 De 3:15,\allowbreak16 Ps 60:7}
\crossref{Josh}{22}{10}{22:25-\allowbreak28; 4:5-\allowbreak9; 24:26,\allowbreak27 Ge 28:18; 31:46-\allowbreak52}
\crossref{Josh}{22}{11}{Le 17:8,\allowbreak9 De 12:5-\allowbreak7; 13:12-\allowbreak14 Joh 20:1,\allowbreak12}
\crossref{Josh}{22}{12}{}
\crossref{Josh}{22}{13}{De 13:14 Jud 20:12 Pr 20:18 Mt 18:15}
\crossref{Josh}{22}{14}{Ex 18:25 Nu 1:4}
\crossref{Josh}{22}{15}{22:12 Mt 18:17 1Co 1:10; 5:4 Ga 1:1,\allowbreak2}
\crossref{Josh}{22}{16}{}
\crossref{Josh}{22}{17}{Nu 25:3,\allowbreak4-\allowbreak18 De 4:3,\allowbreak4 Ps 106:28,\allowbreak29}
\crossref{Josh}{22}{18}{22:16 De 7:4 1Sa 12:14,\allowbreak20 1Ki 9:6 2Ki 17:21 2Ch 25:27; 34:33}
\crossref{Josh}{22}{19}{Ex 15:17 Le 18:25-\allowbreak28 Am 7:17 Ac 10:14,\allowbreak15; 11:8,\allowbreak9}
\crossref{Josh}{22}{20}{Jos 7:1,\allowbreak5,\allowbreak18,\allowbreak24 1Co 10:6 2Pe 2:6 Jude 1:5,\allowbreak6}
\crossref{Josh}{22}{21}{Pr 15:1; 16:1; 18:13; 24:26 Ac 11:4 Jas 1:19 1Pe 3:15}
\crossref{Josh}{22}{22}{1Ki 8:39 Job 10:7; 23:10 Ps 7:3; 44:21; 139:1-\allowbreak12 Jer 12:3; 17:10}
\crossref{Josh}{22}{23}{Ge 9:4 De 18:19 1Sa 20:16 2Ch 24:22 Ps 10:13,\allowbreak14 Eze 3:18}
\crossref{Josh}{22}{24}{Ge 18:19}
\crossref{Josh}{22}{25}{22:27 2Sa 20:1 1Ki 12:16 Ezr 4:2,\allowbreak3 Ne 2:20 Ac 8:21}
\crossref{Josh}{22}{26}{}
\crossref{Josh}{22}{27}{22:10,\allowbreak34; 24:27 Ge 31:48,\allowbreak52 1Sa 7:12}
\crossref{Josh}{22}{28}{Ex 25:40 2Ki 16:10 Eze 43:10,\allowbreak11 Heb 8:5}
\crossref{Josh}{22}{29}{Jos 24:16 Ge 44:7,\allowbreak17 1Sa 12:23 1Ki 21:3 Ro 3:6; 6:2; 9:14}
\crossref{Josh}{22}{30}{}
\crossref{Josh}{22}{31}{Jos 3:10 Le 26:11,\allowbreak12 Nu 14:41-\allowbreak43 2Ch 15:2 Ps 68:17 Isa 12:6}
\crossref{Josh}{22}{32}{22:12-\allowbreak14 Pr 25:13}
\crossref{Josh}{22}{33}{22:30 Ac 15:12,\allowbreak31 2Co 7:7 1Th 3:6-\allowbreak8}
\crossref{Josh}{22}{34}{}
\crossref{Josh}{23}{1}{Jos 11:23; 21:44; 22:4 Ps 46:9}
\crossref{Josh}{23}{2}{Jos 24:1 De 31:28 1Ch 28:1 Ac 20:17-\allowbreak35}
\crossref{Josh}{23}{3}{De 4:9 Ps 44:1,\allowbreak2 Mal 1:5}
\crossref{Josh}{23}{4}{Jos 13:2,\allowbreak6,\allowbreak7; 18:10}
\crossref{Josh}{23}{5}{23:12,\allowbreak13; 13:6 Ex 23:30,\allowbreak31; 33:2; 34:11 De 11:23}
\crossref{Josh}{23}{6}{Jos 1:7-\allowbreak9 Jer 9:3 1Co 16:13 Eph 6:10-\allowbreak19 Heb 12:4 Re 21:8}
\crossref{Josh}{23}{7}{Ex 23:13 Nu 32:38 Ps 16:4 Ho 2:17}
\crossref{Josh}{23}{8}{Jos 22:5 De 4:4; 10:20; 11:22; 13:4 Ac 11:23}
\crossref{Josh}{23}{9}{23:5; 21:43,\allowbreak44 De 11:23}
\crossref{Josh}{23}{10}{Le 26:8 De 32:30 Jud 3:31; 7:19-\allowbreak22; 15:15 1Sa 14:6,\allowbreak12-\allowbreak16}
\crossref{Josh}{23}{11}{Jos 22:5 De 4:9; 6:5-\allowbreak12 Pr 4:23 Lu 21:34 Eph 5:15 Heb 12:15}
\crossref{Josh}{23}{12}{Ps 36:3; 125:5 Isa 1:4 Eze 18:24 Zep 1:6 Mt 12:45 Joh 6:66}
\crossref{Josh}{23}{13}{Ex 23:33 Nu 33:55 De 7:16 Jud 2:2,\allowbreak3 Ps 106:35-\allowbreak39}
\crossref{Josh}{23}{14}{1Ki 2:2 Job 30:23 Ec 9:10; 12:5 Heb 9:27}
\crossref{Josh}{23}{15}{Le 26:14-\allowbreak46 De 28:15-\allowbreak68 Jud 3:8,\allowbreak12; 4:1,\allowbreak2; 6:1; 10:6,\allowbreak7; 13:1}
\crossref{Josh}{23}{16}{2Ki 24:20}
\crossref{Josh}{24}{1}{}
\crossref{Josh}{24}{2}{Ge 11:26,\allowbreak31; 12:1; 31:53 De 26:5 Isa 51:2 Eze 16:3}
\crossref{Josh}{24}{3}{Ge 12:1-\allowbreak4 Ne 9:7,\allowbreak8 Ac 7:2,\allowbreak3}
\crossref{Josh}{24}{4}{Ge 25:24-\allowbreak26}
\crossref{Josh}{24}{5}{Ex 3:10; 4:12,\allowbreak13 Ps 105:26}
\crossref{Josh}{24}{6}{Ex 12:37,\allowbreak51 Mic 6:4}
\crossref{Josh}{24}{7}{Ex 14:10}
\crossref{Josh}{24}{8}{Jos 13:10 Nu 21:21-\allowbreak35 De 2:32-\allowbreak37; 3:1-\allowbreak7 Ne 9:22 Ps 135:10,\allowbreak11}
\crossref{Josh}{24}{9}{Nu 22:5,\allowbreak6-\allowbreak21 De 23:4,\allowbreak5 Jud 11:25 Mic 6:5}
\crossref{Josh}{24}{10}{Nu 22:11,\allowbreak12,\allowbreak18-\allowbreak20,\allowbreak35; 23:3-\allowbreak12,\allowbreak15-\allowbreak26; 24:5-\allowbreak10 Isa 54:17}
\crossref{Josh}{24}{11}{Jos 3:14-\allowbreak17; 4:10-\allowbreak12,\allowbreak23 Ps 114:3,\allowbreak5}
\crossref{Josh}{24}{12}{Ex 23:28 De 7:20}
\crossref{Josh}{24}{13}{Jos 21:45}
\crossref{Josh}{24}{14}{De 10:12 1Sa 12:24 Job 1:1; 28:28 Ps 111:10; 130:4 Ho 3:5}
\crossref{Josh}{24}{15}{Ru 1:15,\allowbreak16 1Ki 18:21 Eze 20:39 Joh 6:67}
\crossref{Josh}{24}{16}{1Sa 12:23 Ro 3:6; 6:2 Heb 10:38,\allowbreak39}
\crossref{Josh}{24}{17}{24:5-\allowbreak14 Ex 19:4 De 32:11,\allowbreak12 Isa 46:4; 63:7-\allowbreak14 Am 2:9,\allowbreak10}
\crossref{Josh}{24}{18}{Ex 10:2; 15:2 Ps 116:16 Mic 4:2 Zec 8:23 Lu 1:73-\allowbreak75}
\crossref{Josh}{24}{19}{24:23 Ru 1:15 Mt 6:24 Lu 14:25-\allowbreak33}
\crossref{Josh}{24}{20}{Jos 23:12-\allowbreak15 1Ch 28:9 2Ch 15:2 Ezr 8:22 Isa 1:28; 63:10; 65:11,\allowbreak12}
\crossref{Josh}{24}{21}{Ex 19:8; 20:19; 24:3,\allowbreak7 De 5:27,\allowbreak28; 26:17 Isa 44:5}
\crossref{Josh}{24}{22}{Ps 119:11,\allowbreak173 Lu 10:42}
\crossref{Josh}{24}{23}{24:14 Ge 35:2-\allowbreak4 Ex 20:23 Jud 10:15,\allowbreak16 1Sa 7:3,\allowbreak4 Ho 14:2,\allowbreak3,\allowbreak8}
\crossref{Josh}{24}{24}{De 5:28,\allowbreak29}
\crossref{Josh}{24}{25}{Ex 15:25; 24:3,\allowbreak7,\allowbreak8 De 5:2,\allowbreak3; 29:1,\allowbreak10-\allowbreak15 2Ki 11:17 2Ch 15:12,\allowbreak15}
\crossref{Josh}{24}{26}{Ex 24:4 De 31:24-\allowbreak26}
\crossref{Josh}{24}{27}{Jos 22:27,\allowbreak28,\allowbreak34 Ge 31:44-\allowbreak52 De 4:26; 30:19; 31:19,\allowbreak21,\allowbreak26 1Sa 7:12}
\crossref{Josh}{24}{28}{Jud 2:6}
\crossref{Josh}{24}{29}{De 34:5 Jud 2:8 Ps 115:17 2Ti 4:7,\allowbreak8 Re 14:13}
\crossref{Josh}{24}{30}{Jos 19:50 Jud 2:9}
\crossref{Josh}{24}{31}{De 31:29 Jud 2:7 2Ch 24:2,\allowbreak17,\allowbreak18 Ac 20:29 Php 2:12}
\crossref{Josh}{24}{32}{Ge 50:25 Ex 13:19 Ac 7:16 Heb 11:22}
\crossref{Josh}{24}{33}{Jos 14:1 Ex 6:23,\allowbreak25 Nu 3:32; 20:26-\allowbreak28}

% Judg
\crossref{Judg}{1}{1}{Jos 24:29,\allowbreak30}
\crossref{Judg}{1}{2}{Ge 49:8-\allowbreak10 Nu 2:3; 7:12 Ps 78:68-\allowbreak70 Heb 7:14 Re 5:5; 19:11-\allowbreak16}
\crossref{Judg}{1}{3}{Ge 29:33 Jos 19:1}
\crossref{Judg}{1}{4}{Ex 23:28,\allowbreak29 De 7:2; 9:3 Jos 10:8-\allowbreak10; 11:6-\allowbreak8 1Sa 14:6,\allowbreak10}
\crossref{Judg}{1}{5}{1:5}
\crossref{Judg}{1}{6}{}
\crossref{Judg}{1}{7}{Ex 21:23-\allowbreak25 Le 24:19-\allowbreak21 1Sa 15:33 Isa 33:1 Mt 7:1,\allowbreak2}
\crossref{Judg}{1}{8}{1:21 Jos 15:63}
\crossref{Judg}{1}{9}{Jos 10:36; 11:21; 15:13-\allowbreak20}
\crossref{Judg}{1}{10}{Jos 14:15}
\crossref{Judg}{1}{11}{Jos 10:38,\allowbreak39; 15:15}
\crossref{Judg}{1}{12}{}
\crossref{Judg}{1}{13}{Jud 3:9}
\crossref{Judg}{1}{14}{Jos 15:18,\allowbreak19}
\crossref{Judg}{1}{15}{Ge 33:11 1Sa 25:18,\allowbreak27 2Co 9:5}
\crossref{Judg}{1}{16}{Jud 4:11,\allowbreak17 Nu 10:29-\allowbreak32; 24:21,\allowbreak22 1Sa 15:6 1Ch 2:15 Jer 35:2}
\crossref{Judg}{1}{17}{1:3}
\crossref{Judg}{1}{18}{Jud 3:3; 16:1,\allowbreak2,\allowbreak21 Ex 23:31}
\crossref{Judg}{1}{19}{1:2; 6:12,\allowbreak13 Ge 39:2,\allowbreak21 Jos 1:5,\allowbreak9; 14:12 2Sa 5:10 2Ki 18:7}
\crossref{Judg}{1}{20}{Nu 14:24 De 1:36 Jos 14:9-\allowbreak14; 15:13,\allowbreak14; 21:11,\allowbreak12}
\crossref{Judg}{1}{21}{Jud 19:10-\allowbreak12 Jos 15:63; 18:11-\allowbreak28 2Sa 5:6-\allowbreak9}
\crossref{Judg}{1}{22}{Nu 1:10,\allowbreak32 Jos 14:4; 16:1-\allowbreak4 1Ch 7:29 Re 7:8}
\crossref{Judg}{1}{23}{Jud 18:2 Jos 2:1; 7:2}
\crossref{Judg}{1}{24}{Jos 2:12-\allowbreak14 1Sa 30:15}
\crossref{Judg}{1}{25}{Jos 6:22-\allowbreak25}
\crossref{Judg}{1}{26}{2Ki 7:6 2Ch 1:17}
\crossref{Judg}{1}{27}{Jos 17:11-\allowbreak13}
\crossref{Judg}{1}{28}{}
\crossref{Judg}{1}{29}{Jos 16:10 1Ki 9:16}
\crossref{Judg}{1}{30}{Jos 19:15}
\crossref{Judg}{1}{31}{Jos 19:24-\allowbreak30}
\crossref{Judg}{1}{32}{Ps 106:34,\allowbreak35}
\crossref{Judg}{1}{33}{Jos 19:32-\allowbreak38}
\crossref{Judg}{1}{34}{Jud 18:1 Jos 19:47}
\crossref{Judg}{1}{35}{Jud 12:12 Jos 10:12}
\crossref{Judg}{1}{36}{Nu 34:4 Jos 15:2}
\crossref{Judg}{2}{1}{Jud 6:12; 13:3 Ge 16:7-\allowbreak10,\allowbreak13; 22:11,\allowbreak12; 48:16 Ex 3:2-\allowbreak6; 14:19; 23:20}
\crossref{Judg}{2}{2}{Ex 23:32,\allowbreak33; 34:12-\allowbreak16 Nu 33:52,\allowbreak53 De 7:2-\allowbreak4,\allowbreak16,\allowbreak25,\allowbreak26; 12:2,\allowbreak3}
\crossref{Judg}{2}{3}{2:21 Nu 33:55 Jos 23:13}
\crossref{Judg}{2}{4}{1Sa 7:6 Ezr 10:1 Pr 17:10 Jer 31:9 Zec 12:10 Lu 6:21; 7:38}
\crossref{Judg}{2}{5}{Ge 35:8 Jos 7:26}
\crossref{Judg}{2}{6}{Jos 22:6; 24:28-\allowbreak31}
\crossref{Judg}{2}{7}{Jos 24:31 2Ki 12:2 2Ch 24:2,\allowbreak14-\allowbreak22 Php 2:12}
\crossref{Judg}{2}{8}{Jos 24:29,\allowbreak30}
\crossref{Judg}{2}{9}{}
\crossref{Judg}{2}{10}{Ex 5:2 1Sa 2:12 1Ch 28:9 Job 21:14 Ps 92:5,\allowbreak6 Isa 5:12 Jer 9:3}
\crossref{Judg}{2}{11}{Jud 4:1; 6:1; 13:1 Ge 13:13; 38:7 2Ch 33:2,\allowbreak6 Ezr 8:12}
\crossref{Judg}{2}{12}{De 13:5; 29:18,\allowbreak25; 31:16,\allowbreak17; 32:15; 33:17}
\crossref{Judg}{2}{13}{2:11; 3:7; 10:6 1Sa 31:10 1Ki 11:5,\allowbreak33 2Ki 23:13 Ps 106:36}
\crossref{Judg}{2}{14}{Jud 3:7,\allowbreak8; 10:7 Le 26:28 Nu 32:14 De 28:20,\allowbreak58; 29:19,\allowbreak20; 31:17,\allowbreak18}
\crossref{Judg}{2}{15}{Jer 18:8; 21:10; 44:11,\allowbreak27 Mic 2:3}
\crossref{Judg}{2}{16}{Jud 3:9,\allowbreak10,\allowbreak15; 4:5; 6:14 1Sa 12:11 Ac 13:20}
\crossref{Judg}{2}{17}{1Sa 8:5-\allowbreak8; 12:12,\allowbreak17,\allowbreak19 2Ch 36:15,\allowbreak16 Ps 106:43}
\crossref{Judg}{2}{18}{Ex 3:12 Jos 1:5 Ac 18:9,\allowbreak10}
\crossref{Judg}{2}{19}{2:7; 3:11,\allowbreak12; 4:1; 8:33 Jos 24:31 2Ch 24:17,\allowbreak18}
\crossref{Judg}{2}{20}{2:14; 3:8; 10:7 Ex 32:10,\allowbreak11 De 32:22}
\crossref{Judg}{2}{21}{2:3; 3:3 Jos 23:13 Eze 20:24}
\crossref{Judg}{2}{22}{Jud 3:1-\allowbreak4}
\crossref{Judg}{2}{23}{2:23}
\crossref{Judg}{3}{1}{Jud 2:21,\allowbreak22 De 7:22}
\crossref{Judg}{3}{2}{Ge 2:17; 3:5,\allowbreak7 2Ch 12:8 Mt 10:34-\allowbreak39 Joh 16:33 1Co 9:26,\allowbreak27}
\crossref{Judg}{3}{3}{Jud 10:7; 14:4 Jos 13:3 1Sa 4:1,\allowbreak2; 6:18; 13:5,\allowbreak19-\allowbreak23; 29:2}
\crossref{Judg}{3}{4}{3:1; 2:22 Ex 15:25 De 33:8 1Co 11:19 2Th 2:9-\allowbreak12}
\crossref{Judg}{3}{5}{Jud 1:29-\allowbreak32 Ps 106:34-\allowbreak38}
\crossref{Judg}{3}{6}{Ex 34:16 De 7:3,\allowbreak4 1Ki 11:1-\allowbreak5 Ezr 9:11,\allowbreak12 Ne 13:23-\allowbreak27 Eze 16:3}
\crossref{Judg}{3}{7}{3:12; 2:11-\allowbreak13}
\crossref{Judg}{3}{8}{Jud 2:14,\allowbreak20 Ex 22:24 De 29:20 Ps 6:1; 85:3}
\crossref{Judg}{3}{9}{3:15; 4:3; 6:7; 10:10 1Sa 12:10 Ne 9:27 Ps 22:5; 78:34; 106:41-\allowbreak44}
\crossref{Judg}{3}{10}{Jud 6:34; 11:29; 13:25; 14:6,\allowbreak19 Nu 11:17; 27:18 1Sa 10:6; 11:6; 16:13}
\crossref{Judg}{3}{11}{3:30; 5:31; 8:28 Jos 11:23 Es 9:22}
\crossref{Judg}{3}{12}{Jud 2:19 Ho 6:4}
\crossref{Judg}{3}{13}{Jud 5:14 Ps 83:6}
\crossref{Judg}{3}{14}{Le 26:23-\allowbreak25 De 28:40,\allowbreak47,\allowbreak48}
\crossref{Judg}{3}{15}{3:9 Ps 50:15; 78:34; 90:15 Jer 29:12,\allowbreak13; 33:3}
\crossref{Judg}{3}{16}{Ps 149:6 Heb 4:12 Re 1:16; 2:12}
\crossref{Judg}{3}{17}{3:29}
\crossref{Judg}{3}{18}{}
\crossref{Judg}{3}{19}{Jos 4:20}
\crossref{Judg}{3}{20}{3:19 2Sa 12:1-\allowbreak15; 24:12 Mic 6:9}
\crossref{Judg}{3}{21}{Nu 25:7,\allowbreak8 1Sa 15:33 Job 20:25 Zec 13:3 2Co 5:16}
\crossref{Judg}{3}{22}{3:22}
\crossref{Judg}{3}{23}{}
\crossref{Judg}{3}{24}{1Sa 24:3}
\crossref{Judg}{3}{25}{}
\crossref{Judg}{3}{26}{3:19}
\crossref{Judg}{3}{27}{Jud 5:14; 6:34 1Sa 13:3 2Sa 20:22 2Ki 9:13}
\crossref{Judg}{3}{28}{Jud 4:10; 7:17}
\crossref{Judg}{3}{29}{3:17 De 32:15 Job 15:27 Ps 17:10}
\crossref{Judg}{3}{30}{3:11; 5:31}
\crossref{Judg}{3}{31}{Jud 5:6,\allowbreak8}
\crossref{Judg}{4}{1}{Jud 2:11,\allowbreak19,\allowbreak20; 3:7,\allowbreak12; 6:1; 10:6 Le 26:23-\allowbreak25 Ne 9:23-\allowbreak30}
\crossref{Judg}{4}{2}{Jud 2:14,\allowbreak15; 10:7 Isa 50:1 Mt 18:25}
\crossref{Judg}{4}{3}{Jud 3:9,\allowbreak15; 10:16 1Sa 7:8 Ps 50:15; 78:34 Jer 2:27,\allowbreak28}
\crossref{Judg}{4}{4}{Ex 15:20 2Ki 22:14 Ne 6:14 Joe 2:28,\allowbreak29 Mic 6:4 Lu 2:36}
\crossref{Judg}{4}{5}{Ge 35:8}
\crossref{Judg}{4}{6}{Jud 5:1 Heb 11:32}
\crossref{Judg}{4}{7}{Ex 14:4 Jos 11:20 Eze 38:10-\allowbreak16 Joe 3:11-\allowbreak14}
\crossref{Judg}{4}{8}{Ex 4:10-\allowbreak14 Mt 14:30,\allowbreak31}
\crossref{Judg}{4}{9}{1Sa 2:30 2Ch 26:18}
\crossref{Judg}{4}{10}{4:6; 5:18}
\crossref{Judg}{4}{11}{Jud 1:16 Nu 10:29; 24:21}
\crossref{Judg}{4}{12}{4:6 Jos 19:12,\allowbreak34 Ps 89:12 Jer 46:18}
\crossref{Judg}{4}{13}{}
\crossref{Judg}{4}{14}{Jud 19:28 Ge 19:14; 44:4 Jos 7:13 1Sa 9:26}
\crossref{Judg}{4}{15}{Jud 5:20,\allowbreak21 Jos 10:10 2Ki 7:6 2Ch 13:15-\allowbreak17 Ps 83:9,\allowbreak10 Heb 11:32}
\crossref{Judg}{4}{16}{Le 26:7,\allowbreak8 Jos 10:19,\allowbreak20; 11:8 Ps 104:35 Ro 2:12 Jas 2:13}
\crossref{Judg}{4}{17}{Job 12:19-\allowbreak21; 18:7-\allowbreak12; 40:11,\allowbreak12 Ps 37:35,\allowbreak36; 107:40 Pr 29:23}
\crossref{Judg}{4}{18}{2Ki 6:19}
\crossref{Judg}{4}{19}{Jud 5:25,\allowbreak26 Ge 24:43 1Ki 17:10 Isa 41:17 Joh 4:7}
\crossref{Judg}{4}{20}{Jos 2:3-\allowbreak5 2Sa 17:20}
\crossref{Judg}{4}{21}{Jud 3:21,\allowbreak31; 5:26; 15:15 1Sa 17:43,\allowbreak49,\allowbreak50 1Co 1:19,\allowbreak27}
\crossref{Judg}{4}{22}{2Sa 17:3,\allowbreak10-\allowbreak15}
\crossref{Judg}{4}{23}{1Ch 22:18 Ne 9:24 Ps 18:39,\allowbreak47; 47:3; 81:14 1Co 15:28 Heb 11:33}
\crossref{Judg}{4}{24}{1Sa 3:12}
\crossref{Judg}{5}{1}{}
\crossref{Judg}{5}{2}{De 32:43 2Sa 22:47,\allowbreak48 Ps 18:47; 48:11; 94:1; 97:8; 136:15,\allowbreak19,\allowbreak20}
\crossref{Judg}{5}{3}{De 32:1,\allowbreak3 Ps 2:10-\allowbreak12; 49:1,\allowbreak2; 119:46; 138:4,\allowbreak5}
\crossref{Judg}{5}{4}{De 33:2 Ps 68:7,\allowbreak8 Hab 3:3-\allowbreak6}
\crossref{Judg}{5}{5}{De 4:11 Ps 97:5; 114:4 Isa 64:1-\allowbreak3 Na 1:5 Hab 3:10}
\crossref{Judg}{5}{6}{Jud 3:31}
\crossref{Judg}{5}{7}{Es 9:19}
\crossref{Judg}{5}{8}{Jud 2:12,\allowbreak17 De 32:16,\allowbreak17}
\crossref{Judg}{5}{9}{5:2 1Ch 29:9 2Co 8:3,\allowbreak4,\allowbreak12,\allowbreak17; 9:5}
\crossref{Judg}{5}{10}{Ps 105:2; 145:5,\allowbreak11}
\crossref{Judg}{5}{11}{La 5:4,\allowbreak9}
\crossref{Judg}{5}{12}{Ps 57:8; 103:1,\allowbreak2; 108:2 Isa 51:9,\allowbreak17; 52:1,\allowbreak2; 60:1 Jer 31:26}
\crossref{Judg}{5}{13}{Ps 49:14 Isa 41:15,\allowbreak16 Eze 17:24 Da 7:18-\allowbreak27 Ro 8:37 Re 2:26,\allowbreak27}
\crossref{Judg}{5}{14}{Jud 3:27; 4:5,\allowbreak6}
\crossref{Judg}{5}{15}{1Ch 12:32}
\crossref{Judg}{5}{16}{Nu 32:1-\allowbreak5,\allowbreak24 Php 2:21; 3:19}
\crossref{Judg}{5}{17}{Jos 13:25,\allowbreak31}
\crossref{Judg}{5}{18}{Jud 4:10}
\crossref{Judg}{5}{19}{Jos 10:22-\allowbreak27; 11:1-\allowbreak15 Ps 48:4-\allowbreak6; 68:12-\allowbreak14; 118:8-\allowbreak12 Re 17:12-\allowbreak14}
\crossref{Judg}{5}{20}{Jos 10:11 1Sa 7:10 Ps 77:17,\allowbreak18}
\crossref{Judg}{5}{21}{Jud 4:7,\allowbreak13 1Ki 18:40 Ps 83:9,\allowbreak10}
\crossref{Judg}{5}{22}{}
\crossref{Judg}{5}{23}{1Sa 26:19 Jer 48:10 1Co 16:22}
\crossref{Judg}{5}{24}{Jud 4:17 Ge 14:19 Pr 31:31 Lu 1:28,\allowbreak42}
\crossref{Judg}{5}{25}{Jud 4:19-\allowbreak21}
\crossref{Judg}{5}{26}{}
\crossref{Judg}{5}{27}{Ps 52:7 Mt 7:2 Jas 2:13}
\crossref{Judg}{5}{28}{2Ki 1:2 So 2:9}
\crossref{Judg}{5}{29}{5:29}
\crossref{Judg}{5}{30}{Ex 15:9 Job 20:5}
\crossref{Judg}{5}{31}{Ps 48:4,\allowbreak5; 58:10,\allowbreak11; 68:1-\allowbreak3; 83:9-\allowbreak18; 92:9; 97:8 Re 6:10; 18:20}
\crossref{Judg}{6}{1}{Jud 2:13,\allowbreak14,\allowbreak19,\allowbreak20 Le 26:14-\allowbreak46 De 28:15-\allowbreak68 Ne 9:26-\allowbreak29 Ps 106:34-\allowbreak42}
\crossref{Judg}{6}{2}{Le 26:17 De 28:47,\allowbreak48}
\crossref{Judg}{6}{3}{Le 26:16 De 28:30-\allowbreak33,\allowbreak51 Job 31:8 Isa 65:21,\allowbreak22 Mic 6:15}
\crossref{Judg}{6}{4}{Le 26:16 De 28:30,\allowbreak33,\allowbreak51 Mic 6:15}
\crossref{Judg}{6}{5}{So 1:5 Isa 13:20}
\crossref{Judg}{6}{6}{Ps 106:43}
\crossref{Judg}{6}{7}{}
\crossref{Judg}{6}{8}{}
\crossref{Judg}{6}{9}{Ps 44:2,\allowbreak3}
\crossref{Judg}{6}{10}{Ex 20:2,\allowbreak3}
\crossref{Judg}{6}{11}{6:14-\allowbreak16; 2:1-\allowbreak5; 5:23; 13:3,\allowbreak18-\allowbreak20 Ge 48:16 Jos 18:23 Isa 63:9}
\crossref{Judg}{6}{12}{Jud 13:3 Lu 1:11,\allowbreak28}
\crossref{Judg}{6}{13}{Ge 25:22 Ex 33:14-\allowbreak16 Nu 14:14,\allowbreak15 Ro 8:31}
\crossref{Judg}{6}{14}{6:11}
\crossref{Judg}{6}{15}{Ex 3:11; 4:10 Jer 1:6 Lu 1:34}
\crossref{Judg}{6}{16}{6:12 Ex 3:12 Jos 1:5 Isa 41:10,\allowbreak14-\allowbreak16 Mt 28:20 Mr 16:20}
\crossref{Judg}{6}{17}{Ex 33:13,\allowbreak16}
\crossref{Judg}{6}{18}{Jud 13:15 Ge 18:3,\allowbreak5; 19:3}
\crossref{Judg}{6}{19}{Le 2:4}
\crossref{Judg}{6}{20}{Jud 13:19}
\crossref{Judg}{6}{21}{Jud 13:20 Le 9:24 1Ki 18:38 1Ch 21:26 2Ch 7:1}
\crossref{Judg}{6}{22}{Jud 13:21}
\crossref{Judg}{6}{23}{Ge 32:30; 43:23 Ps 85:8 Da 10:19 Joh 14:27; 20:19,\allowbreak26 Ro 1:7}
\crossref{Judg}{6}{24}{Jud 21:4 Ge 33:20 Jos 22:10,\allowbreak26-\allowbreak28}
\crossref{Judg}{6}{25}{Ge 35:2 Job 22:23 Ps 101:2}
\crossref{Judg}{6}{26}{2Sa 24:18}
\crossref{Judg}{6}{27}{De 4:1 Mt 16:24 Joh 2:5; 15:14 Ga 1:16 1Th 2:4}
\crossref{Judg}{6}{28}{6:28}
\crossref{Judg}{6}{29}{}
\crossref{Judg}{6}{30}{Jer 26:11; 50:38 Joh 16:2 Ac 26:9 Php 3:6}
\crossref{Judg}{6}{31}{De 13:5-\allowbreak18; 17:2-\allowbreak7 1Ki 18:40}
\crossref{Judg}{6}{32}{1Sa 12:11 2Sa 11:21}
\crossref{Judg}{6}{33}{Ps 3:1; 27:2,\allowbreak3; 118:10-\allowbreak12 Isa 8:9,\allowbreak10 Ro 8:35-\allowbreak39}
\crossref{Judg}{6}{34}{Jud 3:10; 13:25; 14:19; 15:14 1Sa 10:6; 11:6; 16:14 1Ch 12:18}
\crossref{Judg}{6}{35}{2Ch 30:6-\allowbreak12}
\crossref{Judg}{6}{36}{6:14,\allowbreak17-\allowbreak20 Ex 4:1-\allowbreak9 2Ki 20:9 Ps 103:13,\allowbreak14 Mt 16:1}
\crossref{Judg}{6}{37}{De 32:2 Ps 72:6 Ho 6:3,\allowbreak4; 14:5}
\crossref{Judg}{6}{38}{Isa 35:7}
\crossref{Judg}{6}{39}{Ge 18:32}
\crossref{Judg}{6}{40}{6:40}
\crossref{Judg}{7}{1}{Ge 22:3 Jos 3:1; 6:12 Ec 9:10}
\crossref{Judg}{7}{2}{1Sa 14:6 2Ch 14:11 Zec 4:6; 12:7 1Co 1:27-\allowbreak29; 2:4,\allowbreak5 2Co 4:7}
\crossref{Judg}{7}{3}{De 20:8 Mt 13:21 Lu 14:25-\allowbreak33 Re 17:14; 21:8}
\crossref{Judg}{7}{4}{Ps 33:16}
\crossref{Judg}{7}{5}{}
\crossref{Judg}{7}{6}{7:6}
\crossref{Judg}{7}{7}{7:18-\allowbreak22 1Sa 14:6 Isa 41:14-\allowbreak16}
\crossref{Judg}{7}{8}{Jud 3:27 Le 23:24; 25:9 Nu 10:9 Jos 6:4,\allowbreak20 Isa 27:13 1Co 15:52}
\crossref{Judg}{7}{9}{Ge 46:2,\allowbreak3 Job 4:13; 33:15,\allowbreak16 Mt 1:20; 2:13 Ac 18:9,\allowbreak10; 27:23}
\crossref{Judg}{7}{10}{Jud 4:8,\allowbreak9 Ex 4:10-\allowbreak14}
\crossref{Judg}{7}{11}{7:13-\allowbreak15 Ge 24:14 1Sa 14:8,\allowbreak12}
\crossref{Judg}{7}{12}{Jud 6:3,\allowbreak5,\allowbreak33 1Ki 4:30}
\crossref{Judg}{7}{13}{Jud 3:15,\allowbreak31; 4:9,\allowbreak21; 6:15 Isa 41:14,\allowbreak15 1Co 1:27}
\crossref{Judg}{7}{14}{Nu 22:38; 23:5,\allowbreak20; 24:10-\allowbreak13 Job 1:10}
\crossref{Judg}{7}{15}{Ge 40:8; 41:11}
\crossref{Judg}{7}{16}{2Co 4:7}
\crossref{Judg}{7}{17}{Jud 9:48 Mt 16:24 1Co 11:1 Heb 13:7 1Pe 5:3}
\crossref{Judg}{7}{18}{7:20}
\crossref{Judg}{7}{19}{Ex 14:24 Mt 25:6 1Th 5:2 Re 16:15}
\crossref{Judg}{7}{20}{2Co 4:7 Heb 11:4 2Pe 1:15}
\crossref{Judg}{7}{21}{Ex 14:13,\allowbreak14 2Ch 20:17 Isa 30:7,\allowbreak15}
\crossref{Judg}{7}{22}{Jos 6:4,\allowbreak16,\allowbreak20 2Co 4:7}
\crossref{Judg}{7}{23}{Jud 6:35 1Sa 14:21,\allowbreak22}
\crossref{Judg}{7}{24}{Jud 3:27 Ro 15:30 Php 1:27}
\crossref{Judg}{7}{25}{Jud 8:3 Ps 83:11,\allowbreak12}
\crossref{Judg}{8}{1}{Jud 12:1-\allowbreak6 2Sa 19:41 Job 5:2 Ec 4:4 Jas 4:5,\allowbreak6}
\crossref{Judg}{8}{2}{1Co 13:4-\allowbreak7 Ga 5:14,\allowbreak15 Php 2:2,\allowbreak3 Jas 1:19,\allowbreak20; 3:13-\allowbreak18}
\crossref{Judg}{8}{3}{Jud 7:24,\allowbreak25 Ps 44:3; 115:1; 118:14-\allowbreak16 Joh 4:37 Ro 12:3,\allowbreak6; 15:18,\allowbreak19}
\crossref{Judg}{8}{4}{1Sa 14:28,\allowbreak29,\allowbreak31,\allowbreak32; 30:10 2Co 4:8,\allowbreak9,\allowbreak16 Ga 6:9 Heb 12:1-\allowbreak4}
\crossref{Judg}{8}{5}{Ge 33:17 Ps 60:6}
\crossref{Judg}{8}{6}{Jud 5:23 Ge 25:13; 37:25,\allowbreak28 1Sa 25:10,\allowbreak11 1Ki 20:11 2Ki 14:9}
\crossref{Judg}{8}{7}{8:16}
\crossref{Judg}{8}{8}{Ge 32:30,\allowbreak31 1Ki 12:25}
\crossref{Judg}{8}{9}{1Ki 22:27,\allowbreak28}
\crossref{Judg}{8}{10}{Jud 7:12}
\crossref{Judg}{8}{11}{Jud 18:27 1Sa 15:32; 30:16 1Th 5:3}
\crossref{Judg}{8}{12}{Jos 10:16-\allowbreak18,\allowbreak22-\allowbreak25 Job 12:16-\allowbreak21; 34:19 Ps 83:11 Am 2:14}
\crossref{Judg}{8}{13}{}
\crossref{Judg}{8}{14}{Jud 1:24,\allowbreak25 1Sa 30:11-\allowbreak15}
\crossref{Judg}{8}{15}{8:6,\allowbreak7}
\crossref{Judg}{8}{16}{8:7 Pr 10:13; 19:29 Ezr 2:6}
\crossref{Judg}{8}{17}{8:9 1Ki 12:25}
\crossref{Judg}{8}{18}{Jud 4:6 Ps 89:12}
\crossref{Judg}{8}{19}{8:19}
\crossref{Judg}{8}{20}{Jos 10:24 1Sa 15:33 Ps 149:9}
\crossref{Judg}{8}{21}{Ps 83:1}
\crossref{Judg}{8}{22}{Jud 9:8-\allowbreak15 1Sa 8:5; 12:12 Joh 6:15}
\crossref{Judg}{8}{23}{Jud 2:18; 10:18; 11:9-\allowbreak11 Lu 22:24-\allowbreak27 2Co 1:24 1Pe 5:3}
\crossref{Judg}{8}{24}{Ge 24:22,\allowbreak53 Ex 12:35; 32:3 1Pe 3:3-\allowbreak5}
\crossref{Judg}{8}{25}{}
\crossref{Judg}{8}{26}{Es 8:15 Jer 10:9 Eze 27:7 Lu 16:19 Joh 19:2,\allowbreak5 Re 17:4}
\crossref{Judg}{8}{27}{Jud 17:5; 18:14,\allowbreak17 Ex 28:6-\allowbreak12 1Sa 23:9,\allowbreak10 Isa 8:20}
\crossref{Judg}{8}{28}{Ps 83:9-\allowbreak12 Isa 9:4; 10:26}
\crossref{Judg}{8}{29}{Jud 6:32 1Sa 12:11}
\crossref{Judg}{8}{30}{Jud 9:2,\allowbreak5; 10:4; 12:9,\allowbreak14 Ge 46:26 Ex 1:5 2Ki 10:1}
\crossref{Judg}{8}{31}{Jud 9:1-\allowbreak5 Ge 16:15; 22:24}
\crossref{Judg}{8}{32}{Ge 15:15; 25:8 Jos 24:29,\allowbreak30 Job 5:26; 42:17}
\crossref{Judg}{8}{33}{Jud 2:7-\allowbreak10,\allowbreak17,\allowbreak19 Jos 24:31 2Ki 12:2 2Ch 24:17,\allowbreak18}
\crossref{Judg}{8}{34}{Ps 78:11,\allowbreak42; 106:18,\allowbreak21 Ec 12:1 Jer 2:32}
\crossref{Judg}{8}{35}{Jud 9:5,\allowbreak16-\allowbreak19 Ec 9:14,\allowbreak15}
\crossref{Judg}{9}{1}{Jud 8:31}
\crossref{Judg}{9}{2}{Jud 8:30}
\crossref{Judg}{9}{3}{Ps 10:3 Pr 1:11-\allowbreak14}
\crossref{Judg}{9}{4}{9:46-\allowbreak49; 8:33}
\crossref{Judg}{9}{5}{Jud 6:24}
\crossref{Judg}{9}{6}{2Sa 5:9 2Ki 12:20}
\crossref{Judg}{9}{7}{De 11:29; 27:12 Jos 8:33 Joh 4:20}
\crossref{Judg}{9}{8}{}
\crossref{Judg}{9}{9}{Ex 29:2,\allowbreak7; 35:14 Le 2:1 1Ki 19:15,\allowbreak16 Ps 89:20; 104:15 Ac 4:27}
\crossref{Judg}{9}{10}{9:10}
\crossref{Judg}{9}{11}{Lu 13:6,\allowbreak7}
\crossref{Judg}{9}{12}{}
\crossref{Judg}{9}{13}{Nu 15:5,\allowbreak7,\allowbreak10 Ps 104:15 Pr 31:6 Ec 10:19}
\crossref{Judg}{9}{14}{2Ki 14:9}
\crossref{Judg}{9}{15}{Isa 30:2 Da 4:12 Ho 14:7 Mt 13:32}
\crossref{Judg}{9}{16}{Jud 8:35}
\crossref{Judg}{9}{17}{Jud 7:1-\allowbreak25; 8:4-\allowbreak10}
\crossref{Judg}{9}{18}{9:5,\allowbreak6; 8:35 Ps 109:4}
\crossref{Judg}{9}{19}{Isa 8:6 Php 3:3 Jas 4:16}
\crossref{Judg}{9}{20}{9:15,\allowbreak23,\allowbreak56,\allowbreak57; 7:22 2Ch 20:22,\allowbreak23 Ps 21:9,\allowbreak10; 28:4; 52:1-\allowbreak5}
\crossref{Judg}{9}{21}{}
\crossref{Judg}{9}{22}{}
\crossref{Judg}{9}{23}{9:16 Isa 33:1 Mt 7:2}
\crossref{Judg}{9}{24}{1Sa 15:33 1Ki 2:32 Es 9:25 Ps 7:16 Mt 23:34-\allowbreak36}
\crossref{Judg}{9}{25}{Jos 8:4,\allowbreak12,\allowbreak13 Pr 1:11,\allowbreak12}
\crossref{Judg}{9}{26}{Ge 13:8; 19:7}
\crossref{Judg}{9}{27}{Isa 16:9,\allowbreak10; 24:7-\allowbreak9 Jer 25:30 Am 6:3-\allowbreak6}
\crossref{Judg}{9}{28}{1Sa 25:10 2Sa 20:1 1Ki 12:16}
\crossref{Judg}{9}{29}{}
\crossref{Judg}{9}{30}{9:30}
\crossref{Judg}{9}{31}{}
\crossref{Judg}{9}{32}{Job 24:14-\allowbreak17 Ps 36:4 Pr 1:11-\allowbreak16; 4:16 Ro 3:15}
\crossref{Judg}{9}{33}{Le 25:26}
\crossref{Judg}{9}{34}{}
\crossref{Judg}{9}{35}{9:44}
\crossref{Judg}{9}{36}{}
\crossref{Judg}{9}{37}{}
\crossref{Judg}{9}{38}{9:28,\allowbreak29 2Sa 2:26,\allowbreak27 2Ki 14:8-\allowbreak14 Jer 2:28}
\crossref{Judg}{9}{39}{}
\crossref{Judg}{9}{40}{1Ki 20:18-\allowbreak21,\allowbreak30}
\crossref{Judg}{9}{41}{9:28,\allowbreak30}
\crossref{Judg}{9}{42}{9:42}
\crossref{Judg}{9}{43}{}
\crossref{Judg}{9}{44}{9:15,\allowbreak20 Ga 5:15}
\crossref{Judg}{9}{45}{9:20}
\crossref{Judg}{9}{46}{9:4,\allowbreak27; 8:33 1Ki 8:26 2Ki 1:2-\allowbreak4 Ps 115:8 Isa 28:15-\allowbreak18; 37:38}
\crossref{Judg}{9}{47}{}
\crossref{Judg}{9}{48}{Ps 68:14}
\crossref{Judg}{9}{49}{9:15,\allowbreak20 Ga 5:15 Jas 3:16}
\crossref{Judg}{9}{50}{}
\crossref{Judg}{9}{51}{9:51}
\crossref{Judg}{9}{52}{9:48,\allowbreak49 2Ki 14:10; 15:16}
\crossref{Judg}{9}{53}{9:15,\allowbreak20 2Sa 11:21; 20:21 Job 31:3 Jer 49:20; 50:45}
\crossref{Judg}{9}{54}{1Sa 31:4,\allowbreak5}
\crossref{Judg}{9}{55}{2Sa 18:16; 20:21,\allowbreak22 1Ki 22:35,\allowbreak36 Pr 22:10}
\crossref{Judg}{9}{56}{}
\crossref{Judg}{9}{57}{9:20,\allowbreak45 Jos 6:26 1Ki 16:34}
\crossref{Judg}{10}{1}{Jud 2:16; 3:9}
\crossref{Judg}{10}{2}{}
\crossref{Judg}{10}{3}{Ge 31:48 Nu 32:29}
\crossref{Judg}{10}{4}{Jud 5:10; 12:14}
\crossref{Judg}{10}{5}{}
\crossref{Judg}{10}{6}{Jud 4:1; 6:1; 13:1}
\crossref{Judg}{10}{7}{Jud 2:14 De 29:20-\allowbreak28; 31:16-\allowbreak18; 32:16-\allowbreak22 Jos 23:15,\allowbreak16 Ps 74:1}
\crossref{Judg}{10}{8}{10:5 Isa 30:13 1Th 5:3}
\crossref{Judg}{10}{9}{Jud 3:12,\allowbreak13; 6:3-\allowbreak5 2Ch 14:9; 20:1,\allowbreak2}
\crossref{Judg}{10}{10}{Jud 3:9 1Sa 12:10 Ps 106:43,\allowbreak44; 107:13,\allowbreak19,\allowbreak28}
\crossref{Judg}{10}{11}{Jud 2:1-\allowbreak3}
\crossref{Judg}{10}{12}{Jud 5:19-\allowbreak31}
\crossref{Judg}{10}{13}{Jud 2:12 De 32:15 1Ch 28:9 Jer 2:13 Jon 2:8}
\crossref{Judg}{10}{14}{De 32:26-\allowbreak28,\allowbreak37,\allowbreak38 1Ki 18:27,\allowbreak28 2Ki 3:13 Pr 1:25-\allowbreak27 Isa 10:3}
\crossref{Judg}{10}{15}{2Sa 12:13; 24:10 Job 33:27 Pr 28:13 1Jo 1:8-\allowbreak10}
\crossref{Judg}{10}{16}{2Ch 7:14; 15:8; 33:15 Jer 18:7,\allowbreak8 Eze 18:30-\allowbreak32 Ho 14:1-\allowbreak3,\allowbreak8}
\crossref{Judg}{10}{17}{Jud 11:11,\allowbreak29 Ge 31:49}
\crossref{Judg}{10}{18}{Jud 1:1; 11:5-\allowbreak8 Isa 3:1-\allowbreak8; 34:12}
\crossref{Judg}{11}{1}{Heb 11:32}
\crossref{Judg}{11}{2}{Ge 12:10 De 23:2 Ga 4:30}
\crossref{Judg}{11}{3}{Jud 9:4 1Sa 22:2; 27:2; 30:22-\allowbreak24 Job 30:1-\allowbreak10 Ac 17:5}
\crossref{Judg}{11}{4}{}
\crossref{Judg}{11}{5}{Jud 10:9,\allowbreak17,\allowbreak18}
\crossref{Judg}{11}{6}{}
\crossref{Judg}{11}{7}{Ge 26:27; 37:27; 45:4,\allowbreak5 Pr 17:17 Isa 60:14 Ac 7:9-\allowbreak14 Re 3:9}
\crossref{Judg}{11}{8}{Ex 8:8,\allowbreak28; 9:28; 10:17 1Ki 13:6 Lu 17:3,\allowbreak4}
\crossref{Judg}{11}{9}{Nu 32:20-\allowbreak29}
\crossref{Judg}{11}{10}{Ge 21:23; 31:50 1Sa 12:5 Jer 29:23; 42:5 Ro 1:9 2Co 11:31}
\crossref{Judg}{11}{11}{11:8}
\crossref{Judg}{11}{12}{2Ki 14:8-\allowbreak12}
\crossref{Judg}{11}{13}{Nu 21:24-\allowbreak26 Pr 19:5,\allowbreak9}
\crossref{Judg}{11}{14}{Ps 120:7 Ro 12:18 Heb 12:14 1Pe 3:11}
\crossref{Judg}{11}{15}{Nu 21:13-\allowbreak15,\allowbreak27-\allowbreak30 De 2:9,\allowbreak19 2Ch 20:10 Ac 24:12,\allowbreak13}
\crossref{Judg}{11}{16}{Nu 14:25 De 1:40 Jos 5:6}
\crossref{Judg}{11}{17}{Nu 20:14-\allowbreak21 De 2:4-\allowbreak8,\allowbreak29}
\crossref{Judg}{11}{18}{Nu 20:22; 21:10-\allowbreak13; 33:37-\allowbreak44 De 2:1-\allowbreak8}
\crossref{Judg}{11}{19}{Nu 21:21-\allowbreak35 De 2:26-\allowbreak34; 3:1-\allowbreak17 Jos 13:8-\allowbreak12}
\crossref{Judg}{11}{20}{Nu 21:23 De 2:32}
\crossref{Judg}{11}{21}{}
\crossref{Judg}{11}{22}{De 2:36}
\crossref{Judg}{11}{23}{11:23}
\crossref{Judg}{11}{24}{Nu 21:29 1Ki 11:7 Jer 48:7,\allowbreak46}
\crossref{Judg}{11}{25}{Nu 22:2-\allowbreak21 De 23:3,\allowbreak4 Jos 24:9,\allowbreak10 Mic 6:5}
\crossref{Judg}{11}{26}{Nu 21:25-\allowbreak30 De 2:24; 3:2,\allowbreak6 Jos 12:2,\allowbreak5; 13:10}
\crossref{Judg}{11}{27}{Ge 18:25 1Sa 2:10 Job 9:15; 23:7 Ps 7:11; 50:6; 75:7; 82:8}
\crossref{Judg}{11}{28}{2Ki 14:11 Pr 16:18}
\crossref{Judg}{11}{29}{Jud 3:10; 6:34; 13:25 Nu 11:25 1Sa 10:10; 16:13-\allowbreak15 1Ch 12:18}
\crossref{Judg}{11}{30}{Ge 28:20 Nu 30:2-\allowbreak16 1Sa 1:11 Ec 5:1,\allowbreak2,\allowbreak4,\allowbreak5}
\crossref{Judg}{11}{31}{}
\crossref{Judg}{11}{32}{Jud 1:4; 2:18; 3:10}
\crossref{Judg}{11}{33}{De 2:36}
\crossref{Judg}{11}{34}{11:11; 10:17}
\crossref{Judg}{11}{35}{Ge 37:29,\allowbreak30,\allowbreak34,\allowbreak35; 42:36-\allowbreak38 2Sa 13:30,\allowbreak31; 18:33 Job 1:20}
\crossref{Judg}{11}{36}{Jud 16:28-\allowbreak30 2Sa 18:19,\allowbreak31; 19:30 Ac 20:24; 21:13 Ro 16:4 Php 2:30}
\crossref{Judg}{11}{37}{1Sa 1:6 Lu 1:25}
\crossref{Judg}{11}{38}{}
\crossref{Judg}{11}{39}{1Sa 1:11,\allowbreak22,\allowbreak24,\allowbreak28; 2:18}
\crossref{Judg}{11}{40}{Jud 5:11}
\crossref{Judg}{12}{1}{Jud 8:1 2Sa 19:41-\allowbreak43 Ps 109:4 Ec 4:4 Joh 10:32}
\crossref{Judg}{12}{2}{Jud 11:12-\allowbreak33}
\crossref{Judg}{12}{3}{Jud 9:17 1Sa 19:5; 28:21 Job 13:14 Ps 119:109 Ro 16:4 Re 12:11}
\crossref{Judg}{12}{4}{Jud 11:10 Nu 32:39,\allowbreak40 De 3:12-\allowbreak17}
\crossref{Judg}{12}{5}{Jud 3:28; 7:24 Jos 2:7; 22:11}
\crossref{Judg}{12}{6}{Mt 26:73 Mr 14:70}
\crossref{Judg}{12}{7}{}
\crossref{Judg}{12}{8}{Ge 15:19 1Sa 16:1 Mic 5:2 Mt 2:1}
\crossref{Judg}{12}{9}{12:14; 10:4}
\crossref{Judg}{12}{10}{}
\crossref{Judg}{12}{11}{}
\crossref{Judg}{12}{12}{Jos 19:42 1Ch 6:69; 8:13}
\crossref{Judg}{12}{13}{}
\crossref{Judg}{12}{14}{Jud 5:10; 10:4}
\crossref{Judg}{12}{15}{2Sa 23:30}
\crossref{Judg}{13}{1}{Jud 2:11; 3:7; 4:1; 6:1; 10:6 Ro 2:6}
\crossref{Judg}{13}{2}{Jos 15:33; 19:41}
\crossref{Judg}{13}{3}{Jud 2:1; 6:11,\allowbreak12 Ge 16:7-\allowbreak13 Lu 1:11,\allowbreak28-\allowbreak38}
\crossref{Judg}{13}{4}{13:14 Nu 6:2,\allowbreak3 Lu 1:15}
\crossref{Judg}{13}{5}{Nu 6:2,\allowbreak3,\allowbreak5 1Sa 1:11}
\crossref{Judg}{13}{6}{De 33:1 Jos 14:6 1Sa 2:27; 9:6 1Ki 17:18,\allowbreak24 2Ki 4:9,\allowbreak16}
\crossref{Judg}{13}{7}{}
\crossref{Judg}{13}{8}{Job 34:32 Pr 3:5,\allowbreak6 Ac 9:6}
\crossref{Judg}{13}{9}{Ps 65:2 Mt 7:7-\allowbreak11}
\crossref{Judg}{13}{10}{Joh 1:41,\allowbreak42; 4:28,\allowbreak29}
\crossref{Judg}{13}{11}{}
\crossref{Judg}{13}{12}{}
\crossref{Judg}{13}{13}{}
\crossref{Judg}{13}{14}{13:4}
\crossref{Judg}{13}{15}{Jud 6:18,\allowbreak19 Ge 18:3-\allowbreak5}
\crossref{Judg}{13}{16}{}
\crossref{Judg}{13}{17}{}
\crossref{Judg}{13}{18}{13:6 Ge 32:29}
\crossref{Judg}{13}{19}{Jud 6:19,\allowbreak20 1Ki 18:30-\allowbreak38}
\crossref{Judg}{13}{20}{2Ki 2:11 Ps 47:5 Heb 1:3}
\crossref{Judg}{13}{21}{Jud 6:22 Ho 12:4,\allowbreak5}
\crossref{Judg}{13}{22}{Ge 32:30 Ex 33:20 De 4:38; 5:26 Isa 6:5}
\crossref{Judg}{13}{23}{Ec 4:9,\allowbreak10 1Co 12:21}
\crossref{Judg}{13}{24}{Heb 11:32}
\crossref{Judg}{13}{25}{Jud 3:10; 6:34; 11:29 1Sa 11:6 Mt 4:1 Joh 3:34}
\crossref{Judg}{14}{1}{Ge 38:12,\allowbreak13 Jos 15:10; 19:43}
\crossref{Judg}{14}{2}{Ge 21:21; 24:2,\allowbreak3; 34:4; 38:6 2Ki 14:9}
\crossref{Judg}{14}{3}{Ge 13:8; 21:3,\allowbreak4,\allowbreak27}
\crossref{Judg}{14}{4}{Jud 13:1; 15:11 De 28:48}
\crossref{Judg}{14}{5}{14:5}
\crossref{Judg}{14}{6}{Jud 3:10; 11:29; 13:25 1Sa 11:6}
\crossref{Judg}{14}{7}{}
\crossref{Judg}{14}{8}{Ge 29:21 Mt 1:20}
\crossref{Judg}{14}{9}{1Sa 14:25-\allowbreak30 Pr 25:15}
\crossref{Judg}{14}{10}{Ge 29:22 Es 1:7-\allowbreak22 Ec 10:19 Mt 22:2-\allowbreak4 Joh 2:9 Re 19:9}
\crossref{Judg}{14}{11}{1Sa 10:23; 16:6}
\crossref{Judg}{14}{12}{1Ki 10:1 Ps 49:4 Pr 1:6 Eze 17:2; 20:49 Mt 13:13,\allowbreak34 Lu 14:7}
\crossref{Judg}{14}{13}{}
\crossref{Judg}{14}{14}{Ge 3:15 De 8:15,\allowbreak16 1Ki 17:6 2Ch 20:2,\allowbreak25 Isa 53:10-\allowbreak12 Ro 5:3-\allowbreak5}
\crossref{Judg}{14}{15}{Jud 16:5 Ge 3:1-\allowbreak6 Pr 1:11; 5:3; 6:26 Mic 7:5}
\crossref{Judg}{14}{16}{Jud 16:15}
\crossref{Judg}{14}{17}{Jud 16:6,\allowbreak13,\allowbreak16 Ge 3:6 Job 2:9 Pr 7:21 Lu 11:8; 18:4,\allowbreak5}
\crossref{Judg}{14}{18}{}
\crossref{Judg}{14}{19}{14:6; 3:10; 13:25; 15:14 1Sa 11:6}
\crossref{Judg}{14}{20}{Jud 15:2}
\crossref{Judg}{15}{1}{Ge 38:17 Lu 15:29}
\crossref{Judg}{15}{2}{Jud 14:16,\allowbreak20 Ac 26:9}
\crossref{Judg}{15}{3}{}
\crossref{Judg}{15}{4}{}
\crossref{Judg}{15}{5}{Ex 22:6 2Sa 14:30}
\crossref{Judg}{15}{6}{Jud 12:1; 14:15 Pr 22:8 1Th 4:6}
\crossref{Judg}{15}{7}{Jud 14:4,\allowbreak19 Ro 12:19}
\crossref{Judg}{15}{8}{Isa 25:10; 63:3,\allowbreak6}
\crossref{Judg}{15}{9}{15:17,\allowbreak19}
\crossref{Judg}{15}{10}{}
\crossref{Judg}{15}{11}{Jud 13:1; 14:4 De 28:13,\allowbreak47,\allowbreak48 Ps 106:41}
\crossref{Judg}{15}{12}{Mt 27:2 Ac 7:25}
\crossref{Judg}{15}{13}{}
\crossref{Judg}{15}{14}{Jud 5:30; 16:24 Ex 14:3,\allowbreak5 1Sa 4:5 Job 20:5 Mic 7:8}
\crossref{Judg}{15}{15}{Jud 3:31; 4:21; 7:16 Le 26:8 Jos 23:10 1Sa 14:6,\allowbreak14; 17:49,\allowbreak50}
\crossref{Judg}{15}{16}{15:16}
\crossref{Judg}{15}{17}{}
\crossref{Judg}{15}{18}{Jud 8:4 Ps 22:14,\allowbreak15 Joh 19:28 2Co 4:8,\allowbreak9}
\crossref{Judg}{15}{19}{Isa 44:3}
\crossref{Judg}{15}{20}{Jud 13:1,\allowbreak5; 16:31}
\crossref{Judg}{16}{1}{Ge 38:16-\allowbreak18 Ezr 9:1,\allowbreak2}
\crossref{Judg}{16}{2}{1Sa 19:11; 23:26 Ps 118:10-\allowbreak12 Ac 9:24 2Co 11:32,\allowbreak33}
\crossref{Judg}{16}{3}{Ps 107:16 Isa 63:1-\allowbreak5 Mic 2:13 Ac 2:24}
\crossref{Judg}{16}{4}{1Ki 11:1 Ne 13:26 Pr 22:14; 23:27; 26:11; 27:22 1Co 10:6}
\crossref{Judg}{16}{5}{Jud 3:3 Jos 13:3 1Sa 29:6}
\crossref{Judg}{16}{6}{Ps 12:2 Pr 6:26; 7:21; 22:14; 26:28 Jer 9:2-\allowbreak5 Mic 7:2,\allowbreak5}
\crossref{Judg}{16}{7}{16:10 1Sa 19:17; 21:2,\allowbreak3; 27:10 Pr 12:19; 17:7 Ro 3:8 Ga 6:7 Col 3:9}
\crossref{Judg}{16}{8}{Ec 7:26}
\crossref{Judg}{16}{9}{Ps 58:9}
\crossref{Judg}{16}{10}{16:7,\allowbreak13,\allowbreak15-\allowbreak17 Pr 23:7,\allowbreak8; 24:28 Eze 33:31 Lu 22:48}
\crossref{Judg}{16}{11}{Pr 13:3,\allowbreak5; 29:25 Eph 4:25}
\crossref{Judg}{16}{12}{}
\crossref{Judg}{16}{13}{}
\crossref{Judg}{16}{14}{Ezr 9:13,\allowbreak14 Ps 106:43}
\crossref{Judg}{16}{15}{Jud 14:16 Pr 2:16; 5:3-\allowbreak14}
\crossref{Judg}{16}{16}{Pr 7:21-\allowbreak23,\allowbreak26,\allowbreak27 Lu 11:8; 18:5}
\crossref{Judg}{16}{17}{Pr 12:23; 29:12 Mic 7:5}
\crossref{Judg}{16}{18}{Ps 62:9 Pr 18:8 Jer 9:4-\allowbreak6}
\crossref{Judg}{16}{19}{Pr 7:21-\allowbreak23,\allowbreak26,\allowbreak27; 23:33,\allowbreak34 Ec 7:26}
\crossref{Judg}{16}{20}{16:3,\allowbreak9,\allowbreak14 De 32:30 Isa 42:24 Ho 7:9}
\crossref{Judg}{16}{21}{Pr 5:22; 14:14; 2:19}
\crossref{Judg}{16}{22}{Le 26:44 De 32:36 Ps 106:44,\allowbreak45; 107:13,\allowbreak14}
\crossref{Judg}{16}{23}{1Sa 5:2-\allowbreak5 Jer 2:11 Mic 4:5 Ro 1:23-\allowbreak25 1Co 8:4,\allowbreak5; 10:20}
\crossref{Judg}{16}{24}{De 32:27 Isa 37:20 Eze 20:14 Da 5:4,\allowbreak23 Hab 1:16 Re 11:10}
\crossref{Judg}{16}{25}{Jud 9:27; 18:20; 19:6,\allowbreak9 2Sa 13:28 1Ki 20:12 Es 3:15 Isa 22:13}
\crossref{Judg}{16}{26}{}
\crossref{Judg}{16}{27}{Jud 9:51 De 22:8 Jos 2:8 2Sa 11:2}
\crossref{Judg}{16}{28}{2Ch 20:12 Ps 50:15; 91:15; 116:4 La 3:31,\allowbreak32 Heb 11:32}
\crossref{Judg}{16}{29}{16:29}
\crossref{Judg}{16}{30}{Mt 16:25 Ac 20:24; 21:13 Php 2:17,\allowbreak30 Heb 12:1-\allowbreak4}
\crossref{Judg}{16}{31}{Joh 19:39-\allowbreak42}
\crossref{Judg}{17}{1}{Jud 10:1 Jos 15:9; 17:14-\allowbreak18}
\crossref{Judg}{17}{2}{Jud 5:23 De 27:16 1Sa 14:24,\allowbreak28; 26:19 Ne 13:25 Jer 48:10 Mt 26:74}
\crossref{Judg}{17}{3}{17:13; 18:5 Isa 66:3}
\crossref{Judg}{17}{4}{Isa 46:6,\allowbreak7 Jer 10:9,\allowbreak10}
\crossref{Judg}{17}{5}{Jud 8:27; 18:14 Ex 28:4,\allowbreak15 1Sa 23:6}
\crossref{Judg}{17}{6}{Jud 18:1; 19:1; 21:3,\allowbreak25 Ge 36:31 De 33:5}
\crossref{Judg}{17}{7}{Jud 19:1,\allowbreak2 Ge 35:19 Jos 19:15 Ru 1:1,\allowbreak2 Mic 5:2 Mt 2:1,\allowbreak5,\allowbreak6}
\crossref{Judg}{17}{8}{17:11 Ne 13:10,\allowbreak11}
\crossref{Judg}{17}{9}{}
\crossref{Judg}{17}{10}{17:11; 18:19 Ge 45:8 2Ki 6:21; 8:8,\allowbreak9; 13:14 Job 29:16 Isa 22:21}
\crossref{Judg}{17}{11}{}
\crossref{Judg}{17}{12}{17:5}
\crossref{Judg}{17}{13}{Pr 14:12 Isa 44:20; 66:3,\allowbreak4 Mt 15:9,\allowbreak13 Joh 16:2 Ac 26:9}
\crossref{Judg}{18}{1}{Jos 19:40-\allowbreak48}
\crossref{Judg}{18}{2}{18:8,\allowbreak11; 13:2,\allowbreak25; 16:31 Ge 42:9 Jos 19:41}
\crossref{Judg}{18}{3}{Isa 22:16}
\crossref{Judg}{18}{4}{Jud 17:10 Pr 28:21 Isa 56:11 Eze 13:19 Ho 4:8,\allowbreak9 Mal 1:10}
\crossref{Judg}{18}{5}{1Ki 22:5 2Ki 16:15 Isa 30:1 Eze 21:21 Ho 4:12 Ac 8:10}
\crossref{Judg}{18}{6}{1Ki 22:6,\allowbreak12,\allowbreak15 Jer 23:21,\allowbreak22,\allowbreak32}
\crossref{Judg}{18}{7}{Jos 19:47}
\crossref{Judg}{18}{8}{18:2,\allowbreak11; 13:2; 16:31}
\crossref{Judg}{18}{9}{Nu 13:30; 14:7-\allowbreak9 Jos 2:23,\allowbreak24}
\crossref{Judg}{18}{10}{18:7,\allowbreak27}
\crossref{Judg}{18}{11}{18:11}
\crossref{Judg}{18}{12}{Jud 13:25}
\crossref{Judg}{18}{13}{18:2,\allowbreak3; 17:1; 19:1 Jos 24:30,\allowbreak33}
\crossref{Judg}{18}{14}{1Sa 14:28}
\crossref{Judg}{18}{15}{Ge 37:14; 43:27 1Sa 17:22}
\crossref{Judg}{18}{16}{18:11}
\crossref{Judg}{18}{17}{18:2,\allowbreak14}
\crossref{Judg}{18}{18}{}
\crossref{Judg}{18}{19}{Jud 17:10 2Ki 6:21; 8:8,\allowbreak9; 13:14 Mt 23:9}
\crossref{Judg}{18}{20}{Jud 17:10 Pr 30:15 Isa 56:11 Eze 13:19 Ho 4:3 Ac 20:33 Php 3:19}
\crossref{Judg}{18}{21}{}
\crossref{Judg}{18}{22}{}
\crossref{Judg}{18}{23}{Ge 21:17 1Sa 11:5 2Sa 14:5 2Ki 6:28 Ps 114:5 Isa 22:1}
\crossref{Judg}{18}{24}{Jud 17:13 Ps 115:8 Isa 44:18-\allowbreak20 Jer 50:38; 51:17 Eze 23:5}
\crossref{Judg}{18}{25}{1Sa 30:6 2Sa 17:8 Job 3:5; 27:2}
\crossref{Judg}{18}{26}{}
\crossref{Judg}{18}{27}{18:7,\allowbreak10}
\crossref{Judg}{18}{28}{2Sa 14:6}
\crossref{Judg}{18}{29}{Jud 20:1 Ge 14:14 Jos 19:47 2Sa 17:11 1Ki 12:29,\allowbreak30; 15:20}
\crossref{Judg}{18}{30}{Ex 20:4 Le 26:1 De 17:2-\allowbreak7; 27:15; 31:16,\allowbreak29 Jos 19:40-\allowbreak48}
\crossref{Judg}{18}{31}{Jud 19:18; 21:21 Jos 18:1 1Sa 1:3; 4:4 Jer 7:12}
\crossref{Judg}{19}{1}{Jud 17:6; 18:1; 21:25}
\crossref{Judg}{19}{2}{Le 21:9 De 22:21 Eze 16:28}
\crossref{Judg}{19}{3}{Jud 15:1}
\crossref{Judg}{19}{4}{}
\crossref{Judg}{19}{5}{19:8 Ge 18:5 1Sa 14:27-\allowbreak29; 30:12 1Ki 13:7 Ps 104:15 Joh 4:34}
\crossref{Judg}{19}{6}{19:9,\allowbreak21; 9:27; 16:25 Ru 3:7 1Sa 25:36 Es 1:10 Ps 104:15 Lu 12:19}
\crossref{Judg}{19}{7}{19:7}
\crossref{Judg}{19}{8}{}
\crossref{Judg}{19}{9}{Lu 24:29}
\crossref{Judg}{19}{10}{Jud 1:8 Jos 15:8,\allowbreak63; 18:28 2Sa 5:6}
\crossref{Judg}{19}{11}{19:10; 1:21 Ge 10:16 Jos 15:63 2Sa 5:6}
\crossref{Judg}{19}{12}{}
\crossref{Judg}{19}{13}{Jos 18:25,\allowbreak26,\allowbreak28 1Sa 10:26 Isa 10:29 Ho 5:8}
\crossref{Judg}{19}{14}{}
\crossref{Judg}{19}{15}{}
\crossref{Judg}{19}{16}{Ge 3:19 Ps 104:23; 128:2 Pr 13:11; 14:23; 24:27 Ec 1:13; 5:12}
\crossref{Judg}{19}{17}{Ge 16:8; 32:17}
\crossref{Judg}{19}{18}{Jud 18:31; 20:18 Jos 18:1 1Sa 1:3,\allowbreak7}
\crossref{Judg}{19}{19}{Ge 3:19}
\crossref{Judg}{19}{20}{Jud 6:23 Ge 43:23,\allowbreak24 1Sa 25:6 1Ch 12:18 Lu 10:5,\allowbreak6 Joh 14:27}
\crossref{Judg}{19}{21}{Ge 24:32; 43:24}
\crossref{Judg}{19}{22}{19:6,\allowbreak7; 16:25}
\crossref{Judg}{19}{23}{Ge 19:6,\allowbreak7}
\crossref{Judg}{19}{24}{Ge 19:8 Ro 3:8}
\crossref{Judg}{19}{25}{Ge 4:1}
\crossref{Judg}{19}{26}{19:3,\allowbreak27 Ge 18:12 1Pe 3:6}
\crossref{Judg}{19}{27}{}
\crossref{Judg}{19}{28}{Jud 20:5 1Ki 18:29}
\crossref{Judg}{19}{29}{De 21:22,\allowbreak23}
\crossref{Judg}{19}{30}{Jud 20:7 Pr 11:14; 13:10; 15:22; 20:18; 24:6}
\crossref{Judg}{20}{1}{20:2,\allowbreak8,\allowbreak11; 21:5 De 13:12-\allowbreak18 Jos 22:12}
\crossref{Judg}{20}{2}{20:15,\allowbreak17; 8:10 2Sa 24:9 2Ki 3:26}
\crossref{Judg}{20}{3}{Pr 22:3 Mt 5:25 Lu 12:58,\allowbreak59; 14:31,\allowbreak32}
\crossref{Judg}{20}{4}{Jud 19:15-\allowbreak28}
\crossref{Judg}{20}{5}{Jud 19:22}
\crossref{Judg}{20}{6}{Jud 19:29}
\crossref{Judg}{20}{7}{Ex 19:5,\allowbreak6 De 4:6; 14:1,\allowbreak2 1Co 5:1,\allowbreak6,\allowbreak10-\allowbreak12}
\crossref{Judg}{20}{8}{20:1,\allowbreak11}
\crossref{Judg}{20}{9}{Jos 14:2 1Sa 14:41,\allowbreak42 1Ch 24:5 Ne 11:1 Pr 16:33 Jon 1:7}
\crossref{Judg}{20}{10}{Ge 24:60}
\crossref{Judg}{20}{11}{1Sa 18:1 1Ch 12:17 2Ch 5:13}
\crossref{Judg}{20}{12}{De 13:14; 20:10 Jos 22:13-\allowbreak16 Mt 18:15-\allowbreak18 Ro 12:18}
\crossref{Judg}{20}{13}{2Sa 20:21,\allowbreak22}
\crossref{Judg}{20}{14}{Nu 20:20; 21:23 2Ch 13:13 Job 15:25,\allowbreak26}
\crossref{Judg}{20}{15}{20:25,\allowbreak35,\allowbreak46,\allowbreak47 Nu 26:41}
\crossref{Judg}{20}{16}{}
\crossref{Judg}{20}{17}{20:2 Nu 1:46; 26:51 1Sa 11:8; 15:4 1Ch 21:5 2Ch 17:14-\allowbreak18}
\crossref{Judg}{20}{18}{Jud 18:31; 19:18 Jos 18:1 Joe 1:14}
\crossref{Judg}{20}{19}{Jos 3:1; 6:12; 7:16}
\crossref{Judg}{20}{20}{}
\crossref{Judg}{20}{21}{Ge 49:27 Ho 10:9}
\crossref{Judg}{20}{22}{20:15,\allowbreak17 1Sa 30:6 2Sa 11:25 Ps 64:5}
\crossref{Judg}{20}{23}{20:26,\allowbreak27 Ps 78:34-\allowbreak36 Ho 5:15}
\crossref{Judg}{20}{24}{}
\crossref{Judg}{20}{25}{20:21 Ge 18:25 Job 9:12,\allowbreak13 Ps 97:2 Ro 2:5; 3:5; 11:33}
\crossref{Judg}{20}{26}{20:18,\allowbreak23}
\crossref{Judg}{20}{27}{20:18,\allowbreak23 Nu 27:21}
\crossref{Judg}{20}{28}{De 10:8; 18:5}
\crossref{Judg}{20}{29}{20:34 Jos 8:4 2Sa 5:23}
\crossref{Judg}{20}{30}{}
\crossref{Judg}{20}{31}{Jos 8:14-\allowbreak16}
\crossref{Judg}{20}{32}{}
\crossref{Judg}{20}{33}{Jos 8:18-\allowbreak22}
\crossref{Judg}{20}{34}{20:29}
\crossref{Judg}{20}{35}{20:15,\allowbreak44-\allowbreak46 Job 20:5}
\crossref{Judg}{20}{36}{Jos 8:15-\allowbreak29}
\crossref{Judg}{20}{37}{Jos 8:19}
\crossref{Judg}{20}{38}{Ge 17:21 2Ki 4:16}
\crossref{Judg}{20}{39}{20:31}
\crossref{Judg}{20}{40}{Ge 19:28 So 3:6 Joe 2:30 Re 19:3}
\crossref{Judg}{20}{41}{Ex 15:9,\allowbreak10 Isa 13:8,\allowbreak9; 33:14 Lu 17:27,\allowbreak28; 21:26 1Th 5:3}
\crossref{Judg}{20}{42}{La 1:3 Ho 9:9; 10:9}
\crossref{Judg}{20}{43}{Jos 8:20-\allowbreak22}
\crossref{Judg}{20}{44}{}
\crossref{Judg}{20}{45}{Jos 15:32 1Ch 6:77 Zec 14:10}
\crossref{Judg}{20}{46}{20:15,\allowbreak35}
\crossref{Judg}{20}{47}{Jud 21:13 Ps 103:9,\allowbreak10 Isa 1:9 Jer 14:7 La 3:32 Hab 3:2}
\crossref{Judg}{20}{48}{De 13:15-\allowbreak17 2Ch 25:13; 28:6-\allowbreak9 Pr 18:19}
\crossref{Judg}{21}{1}{Jud 20:1,\allowbreak8,\allowbreak10 Jer 4:2}
\crossref{Judg}{21}{2}{21:12; 20:18,\allowbreak23,\allowbreak26 Jos 18:1}
\crossref{Judg}{21}{3}{De 29:24 Jos 7:7-\allowbreak9 Ps 74:1; 80:12 Pr 19:3 Isa 63:17 Jer 12:1}
\crossref{Judg}{21}{4}{Ps 78:34,\allowbreak35 Ho 5:15}
\crossref{Judg}{21}{5}{21:1,\allowbreak18; 5:23 Le 27:28,\allowbreak29 1Sa 11:7 Jer 48:10}
\crossref{Judg}{21}{6}{21:15; 11:35; 20:23 2Sa 2:26 Ho 11:8 Lu 19:41,\allowbreak42}
\crossref{Judg}{21}{7}{21:1,\allowbreak18 1Sa 14:28,\allowbreak29,\allowbreak45}
\crossref{Judg}{21}{8}{}
\crossref{Judg}{21}{9}{}
\crossref{Judg}{21}{10}{}
\crossref{Judg}{21}{11}{Nu 31:17,\allowbreak18 De 2:34}
\crossref{Judg}{21}{12}{Jud 20:18,\allowbreak23 Jos 18:1 Ps 78:60 Jer 7:12}
\crossref{Judg}{21}{13}{Jud 20:47 Jos 15:32}
\crossref{Judg}{21}{14}{21:12; 20:47 1Co 7:2}
\crossref{Judg}{21}{15}{21:6,\allowbreak17}
\crossref{Judg}{21}{16}{}
\crossref{Judg}{21}{17}{Nu 26:55; 36:7}
\crossref{Judg}{21}{18}{21:1; 11:35}
\crossref{Judg}{21}{19}{Ex 23:14-\allowbreak16 Le 23:2,\allowbreak4,\allowbreak6,\allowbreak10,\allowbreak34 Nu 10:10; 28:16,\allowbreak26; 29:12}
\crossref{Judg}{21}{20}{}
\crossref{Judg}{21}{21}{Jud 11:34 Ex 15:20 1Sa 18:6 2Sa 6:14,\allowbreak21 Ps 149:3; 150:4 Ec 3:4}
\crossref{Judg}{21}{22}{Phm 1:9-\allowbreak12}
\crossref{Judg}{21}{23}{Jud 20:48}
\crossref{Judg}{21}{24}{}
\crossref{Judg}{21}{25}{Jud 17:6; 18:1; 19:1}

% Ruth
\crossref{Ruth}{1}{1}{Jud 2:16; 12:8}
\crossref{Ruth}{1}{2}{1:20}
\crossref{Ruth}{1}{3}{2Ki 4:1 Ps 34:19 Heb 12:6,\allowbreak10,\allowbreak11}
\crossref{Ruth}{1}{4}{De 7:3; 23:3 1Ki 11:1,\allowbreak2}
\crossref{Ruth}{1}{5}{De 32:39 Ps 89:30-\allowbreak32 Jer 2:19}
\crossref{Ruth}{1}{6}{Ge 21:1; 50:25 Ex 3:16; 4:31 1Sa 2:21 Lu 1:68; 19:44 1Pe 2:12}
\crossref{Ruth}{1}{7}{2Ki 8:3}
\crossref{Ruth}{1}{8}{Jos 24:15-\allowbreak28 Lu 14:25-\allowbreak33}
\crossref{Ruth}{1}{9}{Ru 3:1}
\crossref{Ruth}{1}{10}{Ps 16:3; 119:63 Zec 8:23}
\crossref{Ruth}{1}{11}{Ge 38:11 De 25:5}
\crossref{Ruth}{1}{12}{Ge 17:17 1Ti 5:9}
\crossref{Ruth}{1}{13}{}
\crossref{Ruth}{1}{14}{Ge 31:28,\allowbreak55 1Ki 19:20 Mt 10:37; 19:22 Mr 10:21,\allowbreak22 2Ti 4:10}
\crossref{Ruth}{1}{15}{Ps 36:3; 125:5 Zep 1:6 Mt 13:20,\allowbreak21 Heb 10:38 1Jo 2:19}
\crossref{Ruth}{1}{16}{2Ki 2:2-\allowbreak6 Lu 24:28,\allowbreak29 Ac 21:13}
\crossref{Ruth}{1}{17}{1Sa 3:17; 25:22 2Sa 3:9,\allowbreak35; 19:13 1Ki 2:23; 19:2; 20:10 2Ki 6:31}
\crossref{Ruth}{1}{18}{Ac 21:14}
\crossref{Ruth}{1}{19}{Isa 23:7 La 2:15}
\crossref{Ruth}{1}{20}{Job 6:4; 19:6 Ps 73:14; 88:15 Isa 38:13 La 3:1-\allowbreak20 Heb 12:11}
\crossref{Ruth}{1}{21}{1Sa 2:7,\allowbreak8 Job 1:21}
\crossref{Ruth}{1}{22}{}
\crossref{Ruth}{2}{1}{Ru 3:2,\allowbreak12}
\crossref{Ruth}{2}{2}{Le 19:9,\allowbreak16; 23:22 De 24:19-\allowbreak21}
\crossref{Ruth}{2}{3}{1Th 4:11,\allowbreak12 2Th 3:12}
\crossref{Ruth}{2}{4}{Ps 118:26; 129:7,\allowbreak8 Lu 1:28 2Th 3:16 2Ti 4:22 2Jo 1:10,\allowbreak11}
\crossref{Ruth}{2}{5}{Ru 4:21 1Ch 2:11,\allowbreak12}
\crossref{Ruth}{2}{6}{Ru 1:16,\allowbreak19,\allowbreak22}
\crossref{Ruth}{2}{7}{Pr 15:33; 18:23 Mt 5:3 Eph 5:21 1Pe 5:5,\allowbreak6}
\crossref{Ruth}{2}{8}{1Sa 3:6,\allowbreak16 2Ki 5:13 Mt 9:2,\allowbreak22}
\crossref{Ruth}{2}{9}{Ge 20:6 Job 19:21 Ps 105:15 Pr 6:29 1Co 7:1 1Jo 5:18}
\crossref{Ruth}{2}{10}{Ge 18:2 1Sa 25:23}
\crossref{Ruth}{2}{11}{Ru 1:11,\allowbreak14-\allowbreak22 Ps 37:5,\allowbreak6}
\crossref{Ruth}{2}{12}{1Sa 24:19 Ps 19:11; 58:11 Pr 11:18; 23:18}
\crossref{Ruth}{2}{13}{Ge 33:8,\allowbreak10,\allowbreak15; 43:14 1Sa 1:18 2Sa 16:4}
\crossref{Ruth}{2}{14}{Job 31:16-\allowbreak22 Pr 11:24,\allowbreak25 Isa 32:8; 58:7,\allowbreak10,\allowbreak11 Lu 14:12-\allowbreak14}
\crossref{Ruth}{2}{15}{Jas 1:5}
\crossref{Ruth}{2}{16}{De 24:19-\allowbreak21 Ps 112:9 Pr 19:17 Mt 25:40 Ro 12:13 2Co 8:5-\allowbreak11}
\crossref{Ruth}{2}{17}{Pr 31:27}
\crossref{Ruth}{2}{18}{2:14 Joh 6:12,\allowbreak13 1Ti 5:4}
\crossref{Ruth}{2}{19}{2:10 Ps 41:1 2Co 9:13-\allowbreak15}
\crossref{Ruth}{2}{20}{Ru 3:10 2Sa 2:5 Job 29:12,\allowbreak13 2Ti 1:16-\allowbreak18}
\crossref{Ruth}{2}{21}{2:7,\allowbreak8,\allowbreak22 So 1:7,\allowbreak8}
\crossref{Ruth}{2}{22}{Pr 27:10 So 1:8}
\crossref{Ruth}{2}{23}{Pr 6:6-\allowbreak8; 13:1,\allowbreak20 1Co 15:33 Eph 6:1-\allowbreak3}
\crossref{Ruth}{3}{1}{Ru 1:9 1Co 7:36 1Ti 5:8,\allowbreak14}
\crossref{Ruth}{3}{2}{Ru 2:20-\allowbreak23 De 25:5,\allowbreak6 Heb 2:11-\allowbreak14}
\crossref{Ruth}{3}{3}{2Sa 14:2 Ps 104:15 Ec 9:8 Mt 6:17}
\crossref{Ruth}{3}{4}{}
\crossref{Ruth}{3}{5}{}
\crossref{Ruth}{3}{6}{Ex 20:12 Pr 1:8 Joh 2:5; 15:14}
\crossref{Ruth}{3}{7}{Ge 43:34 Jud 16:25; 19:6,\allowbreak9,\allowbreak22 2Sa 13:28 Es 1:10 Ps 104:15}
\crossref{Ruth}{3}{8}{Ge 27:33}
\crossref{Ruth}{3}{9}{Ru 2:10-\allowbreak13 1Sa 25:41 Lu 14:11}
\crossref{Ruth}{3}{10}{Ru 2:4,\allowbreak20 1Co 13:4,\allowbreak5}
\crossref{Ruth}{3}{11}{Pr 12:4; 31:10,\allowbreak29-\allowbreak31}
\crossref{Ruth}{3}{12}{Ru 4:1 Mt 7:12 1Th 4:6}
\crossref{Ruth}{3}{13}{Ru 2:20; 4:5 De 25:5-\allowbreak9 Mt 22:24-\allowbreak27}
\crossref{Ruth}{3}{14}{Ec 7:1 Ro 12:17; 14:16 1Co 10:32 2Co 8:21 1Th 5:22 1Pe 2:12}
\crossref{Ruth}{3}{15}{Isa 32:8 Ga 6:10}
\crossref{Ruth}{3}{16}{}
\crossref{Ruth}{3}{17}{}
\crossref{Ruth}{3}{18}{Ps 37:3-\allowbreak5 Isa 28:16; 30:7}
\crossref{Ruth}{4}{1}{De 16:18; 17:5; 21:19; 25:7 Job 29:7; 31:21 Am 5:10-\allowbreak12,\allowbreak15}
\crossref{Ruth}{4}{2}{Ex 18:21,\allowbreak22; 21:8 De 29:10; 31:28 1Ki 21:8 Pr 31:23 La 5:14}
\crossref{Ruth}{4}{3}{Ps 112:5 Pr 13:10}
\crossref{Ruth}{4}{4}{Jer 32:7-\allowbreak9,\allowbreak25 Ro 12:17 2Co 8:21 Php 4:8}
\crossref{Ruth}{4}{5}{Ru 3:12,\allowbreak13 Ge 38:8 De 25:5,\allowbreak6 Mt 22:24 Lu 20:28}
\crossref{Ruth}{4}{6}{}
\crossref{Ruth}{4}{7}{}
\crossref{Ruth}{4}{8}{}
\crossref{Ruth}{4}{9}{Ge 23:16-\allowbreak18 Jer 32:10-\allowbreak12}
\crossref{Ruth}{4}{10}{Ge 29:18,\allowbreak19,\allowbreak27 Pr 18:22; 19:14; 31:10,\allowbreak11 Ho 3:2; 12:12 Eph 5:25}
\crossref{Ruth}{4}{11}{Ge 24:60 Ps 127:3-\allowbreak5; 128:3-\allowbreak6}
\crossref{Ruth}{4}{12}{Ge 46:12 Nu 26:20-\allowbreak22}
\crossref{Ruth}{4}{13}{Ru 3:11}
\crossref{Ruth}{4}{14}{Lu 1:58 Ro 12:15 1Co 12:26}
\crossref{Ruth}{4}{15}{Ge 45:11; 47:12 Ps 55:22 Isa 46:4}
\crossref{Ruth}{4}{16}{}
\crossref{Ruth}{4}{17}{Lu 1:58-\allowbreak63}
\crossref{Ruth}{4}{18}{1Ch 2:4-\allowbreak8; 4:1 Mt 1:3 Lu 3:33}
\crossref{Ruth}{4}{19}{1Ch 2:9,\allowbreak10 Mt 1:4 Lu 3:33}
\crossref{Ruth}{4}{20}{Nu 1:7 Mt 1:4 Lu 3:32}
\crossref{Ruth}{4}{21}{1Ch 2:11}
\crossref{Ruth}{4}{22}{1Sa 16:1 Isa 11:1}

% 1Sam
\crossref{1Sam}{1}{1}{Jud 17:1; 19:1}
\crossref{1Sam}{1}{2}{Ge 4:19,\allowbreak23; 29:23-\allowbreak29 Jud 8:30 Mt 19:8}
\crossref{1Sam}{1}{3}{Ex 23:14,\allowbreak17; 34:23 De 16:16 Lu 2:41}
\crossref{1Sam}{1}{4}{Le 3:4; 7:15 De 12:5-\allowbreak7,\allowbreak17; 16:11}
\crossref{1Sam}{1}{5}{Ge 29:30,\allowbreak31 De 21:15}
\crossref{1Sam}{1}{6}{Le 18:18 Job 6:14}
\crossref{1Sam}{1}{7}{1Sa 2:19}
\crossref{1Sam}{1}{8}{2Sa 12:16,\allowbreak17 2Ki 8:12 Job 6:14 Joh 20:13,\allowbreak15 1Th 5:14}
\crossref{1Sam}{1}{9}{1Sa 3:3,\allowbreak15 2Sa 7:2 Ps 5:7; 27:4; 29:9}
\crossref{1Sam}{1}{10}{Ru 1:20 2Sa 17:8 Job 7:11; 9:18; 10:1 Isa 38:15; 54:6 La 3:15}
\crossref{1Sam}{1}{11}{Ge 28:20 Nu 21:2; 30:3-\allowbreak8 Jud 11:30 Ec 5:4}
\crossref{1Sam}{1}{12}{Lu 11:8-\allowbreak10; 18:1 Eph 6:18 Col 4:2 1Th 5:17 Jas 5:16}
\crossref{1Sam}{1}{13}{Ge 24:42-\allowbreak45 Ne 2:4 Ps 25:1 Ro 8:26}
\crossref{1Sam}{1}{14}{Jos 22:12-\allowbreak20 Job 8:2 Ps 62:3 Pr 6:9 Mt 7:1-\allowbreak3}
\crossref{1Sam}{1}{15}{Pr 15:1; 25:15}
\crossref{1Sam}{1}{16}{1Sa 2:12; 10:27; 25:25 De 13:13}
\crossref{1Sam}{1}{17}{1Sa 25:35; 29:7 Jud 18:6 2Ki 5:19 Mr 5:34 Lu 7:50; 8:48}
\crossref{1Sam}{1}{18}{Ge 32:5; 33:8,\allowbreak15 Ru 2:13}
\crossref{1Sam}{1}{19}{1Sa 9:26 Ps 5:3; 55:17; 119:147 Mr 1:35}
\crossref{1Sam}{1}{20}{}
\crossref{1Sam}{1}{21}{1:3 Ge 18:19 Jos 24:15 Ps 101:2}
\crossref{1Sam}{1}{22}{De 16:16 Lu 2:22,\allowbreak41,\allowbreak42}
\crossref{1Sam}{1}{23}{Nu 30:7-\allowbreak11}
\crossref{1Sam}{1}{24}{Nu 15:9,\allowbreak10 De 12:5,\allowbreak6,\allowbreak11; 16:16}
\crossref{1Sam}{1}{25}{Lu 2:22; 18:15,\allowbreak16}
\crossref{1Sam}{1}{26}{1Sa 17:55; 20:3 Ge 42:15 2Sa 11:11; 14:19 2Ki 2:2,\allowbreak4,\allowbreak6; 4:30}
\crossref{1Sam}{1}{27}{1:11-\allowbreak13 Mt 7:7}
\crossref{1Sam}{1}{28}{}
\crossref{1Sam}{2}{1}{Ne 11:17 Hab 3:1 Php 4:6}
\crossref{1Sam}{2}{2}{Ex 15:11 De 32:4 Ps 99:5,\allowbreak9; 111:9 Isa 6:3; 57:15 1Pe 1:16}
\crossref{1Sam}{2}{3}{Ps 94:4 Pr 8:13 Isa 37:23 Da 4:30,\allowbreak31,\allowbreak37 Mal 3:13 Jude 1:15,\allowbreak16}
\crossref{1Sam}{2}{4}{Ps 37:15,\allowbreak17; 46:9; 76:3}
\crossref{1Sam}{2}{5}{Ps 34:10 Lu 1:53; 16:25}
\crossref{1Sam}{2}{6}{De 32:39 2Ki 5:7 Job 5:18 Ps 68:20 Ho 6:1,\allowbreak2 Joh 5:25-\allowbreak29; 11:25}
\crossref{1Sam}{2}{7}{De 8:17,\allowbreak18 Job 1:21; 5:11 Ps 102:10}
\crossref{1Sam}{2}{8}{Job 2:8; 42:10-\allowbreak12 Ps 113:7,\allowbreak8 Da 4:17 Lu 1:51,\allowbreak52}
\crossref{1Sam}{2}{9}{Job 5:24 Ps 37:23,\allowbreak24; 91:11,\allowbreak12; 94:18; 121:3,\allowbreak5,\allowbreak8 Pr 16:9 1Pe 1:5}
\crossref{1Sam}{2}{10}{Ex 15:6 Jud 5:31 Ps 2:9; 21:8,\allowbreak9; 68:1,\allowbreak2; 92:9 Lu 19:27}
\crossref{1Sam}{2}{11}{2:18; 1:28; 3:1,\allowbreak15}
\crossref{1Sam}{2}{12}{Ho 4:6-\allowbreak9 Mal 2:1-\allowbreak9}
\crossref{1Sam}{2}{13}{}
\crossref{1Sam}{2}{14}{2:29 Ex 29:27,\allowbreak28 Le 7:34 Isa 56:11 Mal 1:10 2Pe 2:13-\allowbreak15}
\crossref{1Sam}{2}{15}{Le 3:3-\allowbreak5,\allowbreak16 Ro 16:18 Php 3:19 Jude 1:12}
\crossref{1Sam}{2}{16}{Le 3:16; 7:23-\allowbreak25}
\crossref{1Sam}{2}{17}{Ge 6:11; 10:9; 13:13 2Ki 21:6 Ps 51:4 Isa 3:8}
\crossref{1Sam}{2}{18}{2:11; 3:1}
\crossref{1Sam}{2}{19}{1Sa 1:3,\allowbreak21 Ex 23:14}
\crossref{1Sam}{2}{20}{Ge 14:19; 27:27-\allowbreak29 Nu 6:23-\allowbreak27 Ru 2:12; 4:11}
\crossref{1Sam}{2}{21}{1Sa 1:19,\allowbreak20 Ge 21:1 Lu 1:68}
\crossref{1Sam}{2}{22}{1Sa 8:1}
\crossref{1Sam}{2}{23}{1Ki 1:6 Ac 9:4; 14:15}
\crossref{1Sam}{2}{24}{Ac 6:3 2Co 6:8 1Ti 3:7 3Jo 1:12}
\crossref{1Sam}{2}{25}{De 17:8-\allowbreak12; 25:1-\allowbreak3}
\crossref{1Sam}{2}{26}{2:21}
\crossref{1Sam}{2}{27}{1Sa 9:4 De 33:1 Jud 6:8; 13:6 1Ki 13:1 1Ti 6:11 2Pe 1:21}
\crossref{1Sam}{2}{28}{Ex 28:1,\allowbreak4,\allowbreak6-\allowbreak30; 29:4-\allowbreak37; 39:1-\allowbreak7 Le 8:7,\allowbreak8 Nu 16:5; 17:5-\allowbreak8; 18:1-\allowbreak7}
\crossref{1Sam}{2}{29}{2:13-\allowbreak17 De 32:15 Mal 1:12,\allowbreak13}
\crossref{1Sam}{2}{30}{Ex 28:43; 29:9 Nu 25:11-\allowbreak13}
\crossref{1Sam}{2}{31}{}
\crossref{1Sam}{2}{32}{Zec 8:4}
\crossref{1Sam}{2}{33}{1Sa 22:21-\allowbreak23 1Ki 1:7,\allowbreak19; 2:26,\allowbreak27 Mt 2:16-\allowbreak18}
\crossref{1Sam}{2}{34}{1Sa 3:12 1Ki 13:3; 14:12}
\crossref{1Sam}{2}{35}{1Ki 1:8,\allowbreak45; 2:35 1Ch 29:22 Eze 34:23; 44:15,\allowbreak16 Heb 2:17; 7:26-\allowbreak28}
\crossref{1Sam}{2}{36}{1Ki 2:27 Eze 44:10-\allowbreak12}
\crossref{1Sam}{3}{1}{3:15; 2:11,\allowbreak18}
\crossref{1Sam}{3}{2}{1Sa 2:22; 4:15 Ge 27:1; 48:19 Ps 90:10 Ec 12:3}
\crossref{1Sam}{3}{3}{Ex 27:20,\allowbreak21; 30:7,\allowbreak8 Le 24:2-\allowbreak4 2Ch 13:11}
\crossref{1Sam}{3}{4}{Ge 22:1 Ex 3:4 Ps 99:6 Ac 9:4 1Co 12:6-\allowbreak11,\allowbreak28 Ga 1:15,\allowbreak16}
\crossref{1Sam}{3}{5}{3:5}
\crossref{1Sam}{3}{6}{1Sa 4:16 Ge 43:29 2Sa 18:22 Mt 9:2}
\crossref{1Sam}{3}{7}{}
\crossref{1Sam}{3}{8}{Job 33:14,\allowbreak15}
\crossref{1Sam}{3}{9}{Ex 20:19 Ps 85:8 Isa 6:8 Da 10:19 Ac 9:6}
\crossref{1Sam}{3}{10}{3:4-\allowbreak6,\allowbreak8}
\crossref{1Sam}{3}{11}{Isa 29:14 Am 3:6,\allowbreak7 Hab 1:5 Ac 13:41}
\crossref{1Sam}{3}{12}{1Sa 2:27-\allowbreak36 Nu 23:19 Jos 23:15 Zec 1:6 Lu 21:33}
\crossref{1Sam}{3}{13}{1Sa 2:27-\allowbreak30,\allowbreak31-\allowbreak36}
\crossref{1Sam}{3}{14}{1Sa 2:25 Nu 15:30,\allowbreak31 Ps 51:16 Isa 22:14 Jer 7:16; 15:1 Eze 24:13}
\crossref{1Sam}{3}{15}{1Sa 1:9 Mal 1:10}
\crossref{1Sam}{3}{16}{}
\crossref{1Sam}{3}{17}{Ps 141:5 Da 4:19 Mic 2:7}
\crossref{1Sam}{3}{18}{Ge 18:25 Jud 10:15 2Sa 16:10-\allowbreak12 Job 1:21; 2:10 Ps 39:9}
\crossref{1Sam}{3}{19}{1Sa 2:21 Jud 13:24 Lu 1:80; 2:40,\allowbreak52}
\crossref{1Sam}{3}{20}{Jud 20:1 2Sa 3:10; 17:11}
\crossref{1Sam}{3}{21}{Ge 12:7; 15:1 Nu 12:6 Am 3:7 Heb 1:1}
\crossref{1Sam}{4}{1}{}
\crossref{1Sam}{4}{2}{1Sa 17:8,\allowbreak21}
\crossref{1Sam}{4}{3}{De 29:24 Ps 74:1,\allowbreak11 Isa 50:1; 58:3}
\crossref{1Sam}{4}{4}{2Sa 6:2 2Ki 19:15 Ps 80:1; 99:1}
\crossref{1Sam}{4}{5}{}
\crossref{1Sam}{4}{6}{Ex 32:17,\allowbreak18}
\crossref{1Sam}{4}{7}{Ex 14:25; 15:14-\allowbreak16 De 32:30}
\crossref{1Sam}{4}{8}{Ex 7:5; 9:14 Ps 78:43-\allowbreak51}
\crossref{1Sam}{4}{9}{2Sa 10:12 1Co 16:13 Eph 6:10,\allowbreak11}
\crossref{1Sam}{4}{10}{4:2 Le 26:17 De 28:25 Ps 78:9,\allowbreak60-\allowbreak64}
\crossref{1Sam}{4}{11}{1Sa 2:32 Ps 78:61}
\crossref{1Sam}{4}{12}{Jos 7:6 2Sa 13:19; 15:32 Ne 9:1 Job 2:12}
\crossref{1Sam}{4}{13}{1Sa 1:9}
\crossref{1Sam}{4}{14}{4:6}
\crossref{1Sam}{4}{15}{1Sa 3:2 Ps 90:10}
\crossref{1Sam}{4}{16}{2Sa 1:4}
\crossref{1Sam}{4}{17}{4:10,\allowbreak11; 3:11}
\crossref{1Sam}{4}{18}{4:21,\allowbreak22 Ps 26:8; 42:3,\allowbreak10; 69:9 La 2:15-\allowbreak19}
\crossref{1Sam}{4}{19}{Ge 16:11; 38:24,\allowbreak25 Ex 21:22 Jud 13:5,\allowbreak7 2Sa 11:5 2Ki 8:12; 15:16}
\crossref{1Sam}{4}{20}{Ge 35:17,\allowbreak18 Joh 16:21}
\crossref{1Sam}{4}{21}{}
\crossref{1Sam}{4}{22}{Ps 137:5,\allowbreak6 Joh 2:17}
\crossref{1Sam}{5}{1}{1Sa 4:11,\allowbreak17,\allowbreak18,\allowbreak22 Ps 78:61}
\crossref{1Sam}{5}{2}{Jud 16:23 1Ch 10:10 Da 5:2,\allowbreak23 Hab 1:11,\allowbreak16}
\crossref{1Sam}{5}{3}{Ex 12:12 Ps 97:7 Isa 19:1; 46:1,\allowbreak2 Zep 2:11 Mr 3:11 Lu 10:18-\allowbreak20}
\crossref{1Sam}{5}{4}{Isa 2:18,\allowbreak19; 27:9 Jer 10:11; 50:2 Eze 6:4-\allowbreak6 Da 11:8 Mic 1:7}
\crossref{1Sam}{5}{5}{Ps 115:4-\allowbreak7; 135:15-\allowbreak18}
\crossref{1Sam}{5}{6}{5:7,\allowbreak11 Ex 9:3 Ps 32:4 Ac 13:11}
\crossref{1Sam}{5}{7}{1Sa 4:8 Ex 8:8,\allowbreak28; 9:28; 10:7; 12:33}
\crossref{1Sam}{5}{8}{Zec 12:3}
\crossref{1Sam}{5}{9}{5:6; 7:13; 12:15 De 2:15 Am 5:19; 9:1-\allowbreak4}
\crossref{1Sam}{5}{10}{Jos 15:45 Jud 1:18 2Ki 1:2 Am 1:8}
\crossref{1Sam}{5}{11}{5:6,\allowbreak9}
\crossref{1Sam}{5}{12}{1Ki 19:17 Am 5:19}
\crossref{1Sam}{6}{1}{1Sa 5:1,\allowbreak3,\allowbreak10,\allowbreak11 Ps 78:61}
\crossref{1Sam}{6}{2}{Ge 41:8 Ex 7:11 Isa 47:12,\allowbreak13 Da 2:2; 5:7 Mt 2:4}
\crossref{1Sam}{6}{3}{Ex 23:15; 34:20 De 16:16}
\crossref{1Sam}{6}{4}{6:5,\allowbreak17,\allowbreak18; 5:6,\allowbreak9 Ex 12:35 Jos 13:3 Jud 3:3}
\crossref{1Sam}{6}{5}{Jos 7:19 Ps 18:44; 66:3}
\crossref{1Sam}{6}{6}{Job 9:4 Ps 95:8 Ro 2:5 Heb 3:13}
\crossref{1Sam}{6}{7}{2Sa 6:3 1Ch 13:7}
\crossref{1Sam}{6}{8}{6:4,\allowbreak5}
\crossref{1Sam}{6}{9}{Jos 15:10; 21:16}
\crossref{1Sam}{6}{10}{}
\crossref{1Sam}{6}{11}{2Sa 6:3 1Ch 13:7; 15:13-\allowbreak15}
\crossref{1Sam}{6}{12}{6:12}
\crossref{1Sam}{6}{13}{}
\crossref{1Sam}{6}{14}{1Sa 7:9-\allowbreak17; 11:5; 20:29 Ex 20:24 Jud 6:26; 21:4 2Sa 24:18,\allowbreak22,\allowbreak25}
\crossref{1Sam}{6}{15}{}
\crossref{1Sam}{6}{16}{6:4,\allowbreak12 Jos 13:3 Jud 3:3; 16:5,\allowbreak23-\allowbreak30}
\crossref{1Sam}{6}{17}{6:4}
\crossref{1Sam}{6}{18}{6:16 Jos 13:3}
\crossref{1Sam}{6}{19}{Ex 19:21 Le 10:1-\allowbreak3 Nu 4:4,\allowbreak5,\allowbreak15,\allowbreak20 De 29:29 2Sa 6:7}
\crossref{1Sam}{6}{20}{1Sa 5:8-\allowbreak12 Nu 17:12,\allowbreak13 2Sa 6:7,\allowbreak9 1Ch 13:11-\allowbreak13 Ps 76:7 Mal 3:2}
\crossref{1Sam}{6}{21}{Jos 18:14 Jud 18:12 1Ch 13:5,\allowbreak6 Ps 78:60 Jer 7:12,\allowbreak14}
\crossref{1Sam}{7}{1}{1Sa 6:21 Jos 18:14 2Sa 6:2 1Ch 13:5,\allowbreak6 Ps 132:6}
\crossref{1Sam}{7}{2}{Jud 2:4 Jer 3:13,\allowbreak22-\allowbreak25; 31:9 Zec 12:10,\allowbreak11 Mt 5:4 2Co 7:10,\allowbreak11}
\crossref{1Sam}{7}{3}{De 30:2-\allowbreak10 1Ki 8:48 Isa 55:7 Ho 6:1,\allowbreak2; 14:1 Joe 2:12,\allowbreak13}
\crossref{1Sam}{7}{4}{Jud 2:11,\allowbreak13; 10:15,\allowbreak16 1Ki 11:33 Ho 14:3,\allowbreak8}
\crossref{1Sam}{7}{5}{Ne 9:1 Joe 2:16}
\crossref{1Sam}{7}{6}{2Ch 20:3 Ezr 8:21-\allowbreak23 Ne 9:1-\allowbreak3 Da 9:3-\allowbreak5 Joe 2:12 Jon 3:1-\allowbreak10}
\crossref{1Sam}{7}{7}{1Sa 13:6; 17:11 Ex 14:10 2Ch 20:3}
\crossref{1Sam}{7}{8}{1Sa 12:19-\allowbreak24 Isa 37:4; 62:1,\allowbreak6,\allowbreak7 Jas 5:16}
\crossref{1Sam}{7}{9}{7:17; 6:14,\allowbreak15; 9:12; 10:8; 16:2 Jud 6:26,\allowbreak28 1Ki 18:30-\allowbreak38}
\crossref{1Sam}{7}{10}{1Sa 2:10; 12:17 Ex 9:23-\allowbreak25 Jud 5:8,\allowbreak20 Ps 18:11-\allowbreak14; 77:16-\allowbreak18; 97:3,\allowbreak4}
\crossref{1Sam}{7}{11}{}
\crossref{1Sam}{7}{12}{Ge 28:18,\allowbreak19; 31:45-\allowbreak52; 35:14 Jos 4:9,\allowbreak20-\allowbreak24; 24:26,\allowbreak27 Isa 19:19}
\crossref{1Sam}{7}{13}{Jud 13:1}
\crossref{1Sam}{7}{14}{De 7:2,\allowbreak16 Jud 4:17 Ps 106:34}
\crossref{1Sam}{7}{15}{7:6; 12:1; 25:1 Jud 2:16; 3:10,\allowbreak11 Ac 13:20,\allowbreak21}
\crossref{1Sam}{7}{16}{Jud 5:10; 10:4; 12:14 Ps 75:2; 82:3,\allowbreak4}
\crossref{1Sam}{7}{17}{1Sa 1:1,\allowbreak19; 8:4; 19:18-\allowbreak23}
\crossref{1Sam}{8}{1}{De 16:18,\allowbreak19 Jud 8:22,\allowbreak23 2Ch 19:5,\allowbreak6 Ne 7:2 1Ti 5:21}
\crossref{1Sam}{8}{2}{1Ch 6:28,\allowbreak38}
\crossref{1Sam}{8}{3}{2Sa 15:4 1Ki 12:6-\allowbreak11 2Ki 21:1-\allowbreak3 Ec 2:19 Jer 22:15-\allowbreak17}
\crossref{1Sam}{8}{4}{Ex 3:16; 24:1 2Sa 5:3}
\crossref{1Sam}{8}{5}{8:6-\allowbreak8,\allowbreak19,\allowbreak20; 12:17 Nu 23:9 De 17:14,\allowbreak15 Ho 13:10,\allowbreak11 Ac 13:21}
\crossref{1Sam}{8}{6}{1Sa 12:17}
\crossref{1Sam}{8}{7}{Nu 22:20 Ps 81:11,\allowbreak12 Isa 66:4 Ho 13:10,\allowbreak11}
\crossref{1Sam}{8}{8}{Ex 14:11,\allowbreak12; 16:3; 17:2; 32:1 Nu 14:2-\allowbreak4; 16:2,\allowbreak3,\allowbreak41 De 9:24}
\crossref{1Sam}{8}{9}{8:11-\allowbreak18; 2:13; 10:25; 14:52 Eze 45:7,\allowbreak8; 46:18}
\crossref{1Sam}{8}{10}{}
\crossref{1Sam}{8}{11}{1Sa 10:25 De 17:14-\allowbreak20}
\crossref{1Sam}{8}{12}{1Ch 27:1-\allowbreak22}
\crossref{1Sam}{8}{13}{8:13}
\crossref{1Sam}{8}{14}{1Sa 22:7 1Ki 21:7,\allowbreak19 Eze 46:18}
\crossref{1Sam}{8}{15}{Ge 37:36 Isa 39:7 Da 1:3,\allowbreak7-\allowbreak10,\allowbreak18}
\crossref{1Sam}{8}{16}{8:16}
\crossref{1Sam}{8}{17}{}
\crossref{1Sam}{8}{18}{Isa 8:21}
\crossref{1Sam}{8}{19}{Ps 81:11 Jer 7:13; 44:16 Eze 33:31}
\crossref{1Sam}{8}{20}{8:5 Ex 33:16 Le 20:24-\allowbreak26 Nu 23:9 De 7:6 Ps 106:35 Joh 15:19}
\crossref{1Sam}{8}{21}{}
\crossref{1Sam}{8}{22}{8:7 Ho 13:11}
\crossref{1Sam}{9}{1}{1Sa 14:51 1Ch 8:30-\allowbreak33; 9:36-\allowbreak39 Ac 13:21}
\crossref{1Sam}{9}{2}{1Sa 16:7 Ge 6:2 2Sa 14:25,\allowbreak26 Jer 9:23}
\crossref{1Sam}{9}{3}{1Sa 10:2 Jud 5:10; 10:4}
\crossref{1Sam}{9}{4}{Jud 17:1; 19:1}
\crossref{1Sam}{9}{5}{1Sa 1:1}
\crossref{1Sam}{9}{6}{1Sa 2:27 De 33:1 1Ki 13:1 2Ki 6:6 1Ti 6:11}
\crossref{1Sam}{9}{7}{Jud 6:18; 13:15-\allowbreak17 1Ki 14:3 2Ki 4:42; 5:5; 8:8}
\crossref{1Sam}{9}{8}{9:8}
\crossref{1Sam}{9}{9}{Ge 25:22 Jud 1:1}
\crossref{1Sam}{9}{10}{2Ki 5:13,\allowbreak14}
\crossref{1Sam}{9}{11}{}
\crossref{1Sam}{9}{12}{1Sa 16:2 Ge 31:54 De 12:6,\allowbreak7 1Co 5:7,\allowbreak8}
\crossref{1Sam}{9}{13}{Mt 26:26 Mr 6:41 Lu 24:30 Joh 6:11,\allowbreak23 1Co 10:30 1Ti 4:4}
\crossref{1Sam}{9}{14}{}
\crossref{1Sam}{9}{15}{9:17; 15:1 Ps 25:14 Am 3:7 Mr 11:2-\allowbreak4; 14:13-\allowbreak16 Ac 13:21; 27:23}
\crossref{1Sam}{9}{16}{1Sa 10:1; 15:1; 16:3 1Ki 19:15,\allowbreak16 2Ki 9:3-\allowbreak6}
\crossref{1Sam}{9}{17}{1Sa 16:6-\allowbreak12 Ho 13:11}
\crossref{1Sam}{9}{18}{}
\crossref{1Sam}{9}{19}{Joh 4:29 1Co 14:25}
\crossref{1Sam}{9}{20}{9:3}
\crossref{1Sam}{9}{21}{Jud 20:46-\allowbreak48 Ps 68:27}
\crossref{1Sam}{9}{22}{Ge 43:32 Lu 14:10}
\crossref{1Sam}{9}{23}{1Sa 1:5 Ge 43:34}
\crossref{1Sam}{9}{24}{}
\crossref{1Sam}{9}{25}{9:13}
\crossref{1Sam}{9}{26}{Ge 19:14; 44:4 Jos 7:13 Jud 19:28}
\crossref{1Sam}{9}{27}{1Sa 20:38,\allowbreak39 Joh 15:14,\allowbreak15}
\crossref{1Sam}{10}{1}{1Sa 2:10; 9:16; 16:13; 24:6; 26:11 2Ki 9:3-\allowbreak6 Ac 13:21 Re 5:8}
\crossref{1Sam}{10}{2}{Ge 35:19 Jer 31:15}
\crossref{1Sam}{10}{3}{Jos 19:12,\allowbreak22 Jud 4:6,\allowbreak12; 8:18 Ps 89:12}
\crossref{1Sam}{10}{4}{Jud 18:15}
\crossref{1Sam}{10}{5}{10:10; 13:3}
\crossref{1Sam}{10}{6}{10:10; 16:13; 19:23,\allowbreak24 Nu 11:25 Jud 3:10 Mt 7:22}
\crossref{1Sam}{10}{7}{Ex 4:8 Lu 2:12 Joh 6:14}
\crossref{1Sam}{10}{8}{1Sa 11:14,\allowbreak15; 13:4,\allowbreak8-\allowbreak15; 15:33}
\crossref{1Sam}{10}{9}{10:6}
\crossref{1Sam}{10}{10}{10:5; 19:20-\allowbreak24}
\crossref{1Sam}{10}{11}{Joh 9:8,\allowbreak9 Ac 3:10}
\crossref{1Sam}{10}{12}{Isa 54:13 Joh 6:45; 7:16 Jas 1:17}
\crossref{1Sam}{10}{13}{}
\crossref{1Sam}{10}{14}{1Sa 9:3-\allowbreak10}
\crossref{1Sam}{10}{15}{}
\crossref{1Sam}{10}{16}{1Sa 9:27 Ex 4:18 Jud 14:6 Pr 29:11}
\crossref{1Sam}{10}{17}{1Sa 7:5,\allowbreak6 Jud 20:1}
\crossref{1Sam}{10}{18}{Jud 2:1; 6:8,\allowbreak9 Ne 9:9-\allowbreak12,\allowbreak27,\allowbreak28}
\crossref{1Sam}{10}{19}{1Sa 8:7-\allowbreak9,\allowbreak19; 12:12,\allowbreak17-\allowbreak19}
\crossref{1Sam}{10}{20}{1Sa 14:41 Jos 7:16-\allowbreak18 Ac 1:24-\allowbreak26}
\crossref{1Sam}{10}{21}{}
\crossref{1Sam}{10}{22}{1Sa 23:2-\allowbreak4,\allowbreak11,\allowbreak12 Nu 27:21 Jud 1:1; 20:18,\allowbreak23,\allowbreak28}
\crossref{1Sam}{10}{23}{1Sa 9:2; 16:7; 17:4}
\crossref{1Sam}{10}{24}{De 17:15 2Sa 21:6}
\crossref{1Sam}{10}{25}{1Sa 8:11-\allowbreak18 De 17:14-\allowbreak20 Eze 45:9,\allowbreak10; 46:16-\allowbreak18 Ro 13:1-\allowbreak7 1Ti 2:2}
\crossref{1Sam}{10}{26}{1Sa 11:4; 15:34 Jos 18:28 Jud 19:12-\allowbreak16; 20:14 2Sa 21:6}
\crossref{1Sam}{10}{27}{1Sa 2:12; 11:12 De 13:13 2Sa 20:1 2Ch 13:7 Ac 7:35,\allowbreak51,\allowbreak52}
\crossref{1Sam}{10}{28}{}
\crossref{1Sam}{11}{1}{1Sa 31:11-\allowbreak13 Jud 21:8,\allowbreak10-\allowbreak25}
\crossref{1Sam}{11}{2}{2Ki 18:31}
\crossref{1Sam}{11}{3}{}
\crossref{1Sam}{11}{4}{1Sa 10:26; 14:2; 15:34 2Sa 21:6}
\crossref{1Sam}{11}{5}{1Sa 9:1 1Ki 19:19 Ps 78:71}
\crossref{1Sam}{11}{6}{1Sa 10:10; 16:13 Jud 3:10; 6:34; 11:29; 13:25; 14:6}
\crossref{1Sam}{11}{7}{Jud 19:29}
\crossref{1Sam}{11}{8}{Jud 1:4,\allowbreak5}
\crossref{1Sam}{11}{9}{Ps 18:17}
\crossref{1Sam}{11}{10}{11:2,\allowbreak3}
\crossref{1Sam}{11}{11}{Ge 22:14 Ps 46:1}
\crossref{1Sam}{11}{12}{1Sa 10:27 Ps 21:8 Lu 19:27}
\crossref{1Sam}{11}{13}{1Sa 14:45 2Sa 19:22}
\crossref{1Sam}{11}{14}{1Sa 7:16; 10:8}
\crossref{1Sam}{11}{15}{1Sa 10:17}
\crossref{1Sam}{12}{1}{1Sa 8:5-\allowbreak8,\allowbreak19-\allowbreak22}
\crossref{1Sam}{12}{2}{1Sa 8:20 Nu 27:17}
\crossref{1Sam}{12}{3}{12:5; 10:1; 24:6 2Sa 1:14-\allowbreak16 Mt 22:21 Ro 13:1-\allowbreak7}
\crossref{1Sam}{12}{4}{Ps 37:5,\allowbreak6 Da 6:4 3Jo 1:12}
\crossref{1Sam}{12}{5}{Job 31:35-\allowbreak40; 42:7}
\crossref{1Sam}{12}{6}{Ex 6:26 Ne 9:9-\allowbreak14 Ps 77:19,\allowbreak20; 78:12-\allowbreak72; 99:6; 105:26,\allowbreak41}
\crossref{1Sam}{12}{7}{Isa 1:18; 5:3,\allowbreak4 Eze 18:25-\allowbreak30 Mic 6:2,\allowbreak3 Ac 17:3}
\crossref{1Sam}{12}{8}{Ge 46:5-\allowbreak7 Nu 20:15 Ac 7:15}
\crossref{1Sam}{12}{9}{De 32:18 Jud 3:7 Ps 106:21 Jer 2:32}
\crossref{1Sam}{12}{10}{1Sa 7:2 Jud 3:9,\allowbreak15; 4:3; 6:7; 10:10,\allowbreak15 Ps 78:34,\allowbreak35; 106:44 Isa 26:16}
\crossref{1Sam}{12}{11}{Jud 6:14,\allowbreak32; 8:29,\allowbreak35}
\crossref{1Sam}{12}{12}{1Sa 11:1,\allowbreak2}
\crossref{1Sam}{12}{13}{1Sa 10:24; 11:15}
\crossref{1Sam}{12}{14}{Le 20:1-\allowbreak13 De 28:1-\allowbreak14 Jos 24:14,\allowbreak20 Ps 81:12-\allowbreak15 Isa 3:10}
\crossref{1Sam}{12}{15}{Le 26:14-\allowbreak30 De 28:15-\allowbreak68 Jos 24:20 Isa 1:20; 3:11 Ro 2:8,\allowbreak9}
\crossref{1Sam}{12}{16}{12:7; 15:16 Ex 14:13,\allowbreak31}
\crossref{1Sam}{12}{17}{1Sa 7:9,\allowbreak10 Jos 10:12 Ps 99:6 Jer 15:1 Jas 5:16-\allowbreak18}
\crossref{1Sam}{12}{18}{Ex 9:23-\allowbreak25 Re 11:5,\allowbreak6}
\crossref{1Sam}{12}{19}{1Sa 7:5,\allowbreak8 Ge 20:7 Ex 9:28; 10:17 Job 42:8 Ps 78:34,\allowbreak35 Isa 26:16}
\crossref{1Sam}{12}{20}{Ex 20:19,\allowbreak20 1Pe 3:16}
\crossref{1Sam}{12}{21}{De 32:21 Jer 2:5,\allowbreak13; 10:8,\allowbreak15; 14:22; 16:19 Jon 2:8 Hab 2:18}
\crossref{1Sam}{12}{22}{De 31:17 1Ki 6:13 2Ki 21:14 1Ch 28:9 2Ch 15:2 Ps 94:14}
\crossref{1Sam}{12}{23}{Ac 12:5 Ro 1:9 Col 1:9 1Th 3:10 2Ti 1:3}
\crossref{1Sam}{12}{24}{Job 28:28 Ps 111:10 Pr 1:7 Ex 12:13 Heb 12:29}
\crossref{1Sam}{12}{25}{De 32:15-\allowbreak44 Jos 24:20 Isa 3:11}
\crossref{1Sam}{13}{1}{}
\crossref{1Sam}{13}{2}{1Sa 8:11; 14:52}
\crossref{1Sam}{13}{3}{1Sa 10:5; 14:1-\allowbreak6 2Sa 23:14}
\crossref{1Sam}{13}{4}{Ge 34:30; 46:34 Ex 5:21 Zec 11:8}
\crossref{1Sam}{13}{5}{Ge 22:17 Jos 11:4 Jud 7:12 2Ch 1:9 Isa 48:19 Jer 15:8 Ro 9:27}
\crossref{1Sam}{13}{6}{Ex 14:10-\allowbreak12 Jos 8:20 Jud 10:9; 20:41 2Sa 24:14 Php 1:23}
\crossref{1Sam}{13}{7}{Le 26:17,\allowbreak36,\allowbreak37 De 28:25}
\crossref{1Sam}{13}{8}{1Sa 10:8}
\crossref{1Sam}{13}{9}{13:12,\allowbreak13; 14:18; 15:21,\allowbreak22 De 12:6 1Ki 3:4 Ps 37:7 Pr 15:8}
\crossref{1Sam}{13}{10}{1Sa 15:13}
\crossref{1Sam}{13}{11}{Ge 3:13; 4:10 Jos 7:19 2Sa 3:24 2Ki 5:25}
\crossref{1Sam}{13}{12}{1Ki 12:26,\allowbreak27}
\crossref{1Sam}{13}{13}{2Sa 12:7-\allowbreak9 1Ki 18:18; 21:20 2Ch 16:9; 19:2; 25:15,\allowbreak16 Job 34:18}
\crossref{1Sam}{13}{14}{1Sa 2:30; 15:28}
\crossref{1Sam}{13}{15}{13:2,\allowbreak6,\allowbreak7; 14:2}
\crossref{1Sam}{13}{16}{13:3}
\crossref{1Sam}{13}{17}{1Sa 11:11}
\crossref{1Sam}{13}{18}{Jos 10:11; 16:3,\allowbreak5; 18:13,\allowbreak14 1Ch 6:68 2Ch 8:5}
\crossref{1Sam}{13}{19}{}
\crossref{1Sam}{13}{20}{}
\crossref{1Sam}{13}{21}{13:21}
\crossref{1Sam}{13}{22}{1Sa 17:47,\allowbreak50 Jud 5:8 Zec 4:6 1Co 1:27-\allowbreak29 2Co 4:7}
\crossref{1Sam}{13}{23}{13:3; 14:4}
\crossref{1Sam}{14}{1}{1Sa 25:19 Jud 6:27; 14:6 Mic 7:5}
\crossref{1Sam}{14}{2}{1Sa 13:15,\allowbreak16 Isa 10:28,\allowbreak29}
\crossref{1Sam}{14}{3}{1Sa 22:9-\allowbreak12,\allowbreak20}
\crossref{1Sam}{14}{4}{1Sa 13:23}
\crossref{1Sam}{14}{5}{14:4}
\crossref{1Sam}{14}{6}{1Sa 17:26,\allowbreak36 Ge 17:7-\allowbreak11 Jud 15:18 2Sa 1:20 Jer 9:23,\allowbreak26}
\crossref{1Sam}{14}{7}{1Sa 10:7 2Sa 7:3 Ps 46:7 Zec 8:23}
\crossref{1Sam}{14}{8}{Jud 7:9-\allowbreak14}
\crossref{1Sam}{14}{9}{Ge 24:13,\allowbreak14 Jud 6:36-\allowbreak40}
\crossref{1Sam}{14}{10}{1Sa 10:7 Ge 24:14 Jud 7:11 Isa 7:11-\allowbreak14}
\crossref{1Sam}{14}{11}{14:22; 13:6 Jud 6:2}
\crossref{1Sam}{14}{12}{Ge 24:26,\allowbreak27,\allowbreak42,\allowbreak48 Jud 4:14; 7:15 2Sa 5:24}
\crossref{1Sam}{14}{13}{Ps 18:29 Heb 11:34}
\crossref{1Sam}{14}{14}{}
\crossref{1Sam}{14}{15}{Jos 2:9 Jud 7:21 2Ki 7:6,\allowbreak7 Job 18:11 Ps 14:5}
\crossref{1Sam}{14}{16}{Ps 58:7; 68:2}
\crossref{1Sam}{14}{17}{}
\crossref{1Sam}{14}{18}{1Sa 5:2; 7:1}
\crossref{1Sam}{14}{19}{14:24; 13:11 Jos 9:14 Ps 106:13 Isa 28:16}
\crossref{1Sam}{14}{20}{}
\crossref{1Sam}{14}{21}{}
\crossref{1Sam}{14}{22}{1Sa 13:6; 31:7}
\crossref{1Sam}{14}{23}{Ex 14:30 Jud 2:18 2Ki 14:27 Ps 44:6-\allowbreak8 Ho 1:7}
\crossref{1Sam}{14}{24}{14:27-\allowbreak30 Le 27:29 Nu 21:2 De 27:15-\allowbreak26 Jos 6:17-\allowbreak19,\allowbreak26 Jud 11:30}
\crossref{1Sam}{14}{25}{De 9:28 Mt 3:5}
\crossref{1Sam}{14}{26}{Ec 9:2}
\crossref{1Sam}{14}{27}{14:29; 30:12 Pr 25:26}
\crossref{1Sam}{14}{28}{14:24,\allowbreak43}
\crossref{1Sam}{14}{29}{1Ki 18:18}
\crossref{1Sam}{14}{30}{Ec 9:18}
\crossref{1Sam}{14}{31}{Jos 10:12; 19:42}
\crossref{1Sam}{14}{32}{1Sa 15:19}
\crossref{1Sam}{14}{33}{Mt 7:5 Ro 2:1}
\crossref{1Sam}{14}{34}{14:34}
\crossref{1Sam}{14}{35}{}
\crossref{1Sam}{14}{36}{Jos 10:9-\allowbreak14,\allowbreak19 Jer 6:5}
\crossref{1Sam}{14}{37}{1Sa 23:4,\allowbreak9-\allowbreak12; 30:7,\allowbreak8 Jud 1:1; 20:18,\allowbreak28 2Sa 5:19,\allowbreak23 1Ki 22:5,\allowbreak15}
\crossref{1Sam}{14}{38}{1Sa 10:19,\allowbreak20 Jos 7:14-\allowbreak26}
\crossref{1Sam}{14}{39}{14:24,\allowbreak44; 19:6; 20:31; 22:16; 28:10 2Sa 12:5 Ec 9:2}
\crossref{1Sam}{14}{40}{14:7,\allowbreak36 2Sa 15:15}
\crossref{1Sam}{14}{41}{Pr 16:33 Ac 1:24}
\crossref{1Sam}{14}{42}{}
\crossref{1Sam}{14}{43}{Jos 7:19 Jon 1:7-\allowbreak10}
\crossref{1Sam}{14}{44}{1Sa 25:22 Ru 1:17 2Sa 3:9; 19:13}
\crossref{1Sam}{14}{45}{14:23; 19:5 Ne 9:27}
\crossref{1Sam}{14}{46}{}
\crossref{1Sam}{14}{47}{1Sa 13:1}
\crossref{1Sam}{14}{48}{1Sa 15:3-\allowbreak7 Ex 17:14 De 25:19}
\crossref{1Sam}{14}{49}{1Sa 31:2 1Ch 8:33; 9:39}
\crossref{1Sam}{14}{50}{1Sa 17:55 2Sa 2:8; 3:27}
\crossref{1Sam}{14}{51}{1Sa 9:1,\allowbreak21}
\crossref{1Sam}{14}{52}{1Sa 8:1,\allowbreak11}
\crossref{1Sam}{15}{1}{15:17,\allowbreak18; 9:16; 10:1}
\crossref{1Sam}{15}{2}{Jer 31:34 Ho 7:2 Am 8:7}
\crossref{1Sam}{15}{3}{Le 27:28,\allowbreak29 Nu 24:20 De 13:15,\allowbreak16; 20:16-\allowbreak18 Jos 6:17-\allowbreak21}
\crossref{1Sam}{15}{4}{Jos 15:24}
\crossref{1Sam}{15}{5}{15:5}
\crossref{1Sam}{15}{6}{1Sa 27:10 Nu 24:21,\allowbreak22 Jud 1:16; 4:11; 5:24 1Ch 2:55}
\crossref{1Sam}{15}{7}{1Sa 14:48 Job 21:30 Ec 8:13}
\crossref{1Sam}{15}{8}{15:3 Nu 24:7 1Ki 20:30,\allowbreak34-\allowbreak42 Es 3:1}
\crossref{1Sam}{15}{9}{15:3,\allowbreak15,\allowbreak19 Jos 7:21}
\crossref{1Sam}{15}{10}{}
\crossref{1Sam}{15}{11}{15:35 Ge 6:6 2Sa 24:16 Ps 110:4 Jer 18:7-\allowbreak10 Am 7:3 Jon 3:10; 4:2}
\crossref{1Sam}{15}{12}{1Sa 25:2 Jos 15:55 1Ki 18:42}
\crossref{1Sam}{15}{13}{1Sa 13:10 Ge 14:19 Jud 17:2 Ru 3:10}
\crossref{1Sam}{15}{14}{Ps 36:2; 50:16-\allowbreak21 Jer 2:18,\allowbreak19,\allowbreak22,\allowbreak23,\allowbreak34-\allowbreak37 Mal 3:13-\allowbreak15 Lu 19:22}
\crossref{1Sam}{15}{15}{15:9,\allowbreak21 Ge 3:12,\allowbreak13 Ex 32:22,\allowbreak23 Job 31:33 Pr 28:13}
\crossref{1Sam}{15}{16}{1Sa 9:27; 12:7 1Ki 22:16}
\crossref{1Sam}{15}{17}{1Sa 9:21; 10:22 Jud 6:15 Ho 13:1 Mt 18:4}
\crossref{1Sam}{15}{18}{Ge 13:13; 15:16 Nu 16:38 Job 31:3 Pr 10:29; 13:21}
\crossref{1Sam}{15}{19}{Pr 15:27 Jer 7:11 Hab 2:9-\allowbreak12 2Ti 4:10}
\crossref{1Sam}{15}{20}{15:13 Job 33:9; 34:5; 35:2; 40:8 Mt 19:20 Lu 10:29; 18:11 Ro 10:3}
\crossref{1Sam}{15}{21}{15:15 Ge 3:13 Ex 32:22,\allowbreak23}
\crossref{1Sam}{15}{22}{Ps 50:8,\allowbreak9; 51:16,\allowbreak17 Pr 21:3 Isa 1:11-\allowbreak17 Jer 7:22,\allowbreak23 Ho 6:6}
\crossref{1Sam}{15}{23}{1Sa 12:14,\allowbreak15 Nu 14:9 De 9:7,\allowbreak24 Jos 22:16-\allowbreak19 Job 34:37 Ps 107:11}
\crossref{1Sam}{15}{24}{15:30 Ex 9:27; 10:16 Nu 22:34 2Sa 12:13 Mt 27:4}
\crossref{1Sam}{15}{25}{Ex 10:17}
\crossref{1Sam}{15}{26}{15:31 Ge 42:38; 43:11-\allowbreak14 Lu 24:28,\allowbreak29 2Jo 1:11}
\crossref{1Sam}{15}{27}{}
\crossref{1Sam}{15}{28}{1Sa 28:17,\allowbreak18 1Ki 11:30,\allowbreak31}
\crossref{1Sam}{15}{29}{De 33:27 Ps 29:11; 68:35 Isa 45:24 Joe 3:16 2Co 12:9 Php 4:13}
\crossref{1Sam}{15}{30}{Hab 2:4 Joh 5:44; 12:43}
\crossref{1Sam}{15}{31}{}
\crossref{1Sam}{15}{32}{Jer 48:44 1Th 5:3 Re 18:7}
\crossref{1Sam}{15}{33}{Ge 9:6 Ex 17:11 Nu 14:45 Jud 1:7 Mt 7:2 Jas 2:13 Re 16:6; 18:6}
\crossref{1Sam}{15}{34}{1Sa 11:4}
\crossref{1Sam}{15}{35}{1Sa 19:24}
\crossref{1Sam}{16}{1}{1Sa 15:11,\allowbreak35 Jer 7:16; 11:14}
\crossref{1Sam}{16}{2}{Ex 3:11 1Ki 18:9-\allowbreak14 Mt 10:16 Lu 1:34}
\crossref{1Sam}{16}{3}{1Sa 9:12,\allowbreak13 2Sa 15:11 Mt 22:1-\allowbreak4}
\crossref{1Sam}{16}{4}{1Sa 21:1 2Sa 6:9 1Ki 17:18 Ho 6:5; 11:10 Lu 5:8; 8:37}
\crossref{1Sam}{16}{5}{Ex 19:10,\allowbreak14,\allowbreak15 Le 20:7,\allowbreak8 Nu 11:8 Jos 3:5; 7:13 2Ch 30:17-\allowbreak20}
\crossref{1Sam}{16}{6}{1Sa 17:13,\allowbreak22 1Ch 2:13; 27:18}
\crossref{1Sam}{16}{7}{1Sa 9:2; 10:23,\allowbreak24 2Sa 14:25 Ps 147:10,\allowbreak11 Pr 31:30}
\crossref{1Sam}{16}{8}{1Sa 17:13 1Ch 2:13}
\crossref{1Sam}{16}{9}{1Sa 17:13 2Sa 13:3}
\crossref{1Sam}{16}{10}{1Ch 2:13-\allowbreak15}
\crossref{1Sam}{16}{11}{1Sa 17:12-\allowbreak15,\allowbreak28 2Sa 7:8 1Ch 17:7 Ps 78:70,\allowbreak71}
\crossref{1Sam}{16}{12}{1Sa 17:42 So 5:10 La 4:7 Ac 7:20 Heb 11:23}
\crossref{1Sam}{16}{13}{1Sa 10:1 2Ki 9:6}
\crossref{1Sam}{16}{14}{1Sa 11:6; 18:12; 28:15 Jud 16:29 Ps 51:11 Ho 9:12}
\crossref{1Sam}{16}{15}{}
\crossref{1Sam}{16}{16}{16:21,\allowbreak22 Ge 41:46 1Ki 10:8}
\crossref{1Sam}{16}{17}{}
\crossref{1Sam}{16}{18}{1Sa 17:32-\allowbreak36 2Sa 17:8,\allowbreak10}
\crossref{1Sam}{16}{19}{16:11; 17:15,\allowbreak33,\allowbreak34 Ex 3:1-\allowbreak10 1Ki 19:19 Ps 78:70-\allowbreak72; 113:8}
\crossref{1Sam}{16}{20}{}
\crossref{1Sam}{16}{21}{Ge 41:46 De 1:38; 10:8 1Ki 10:8 Pr 22:29}
\crossref{1Sam}{16}{22}{}
\crossref{1Sam}{16}{23}{16:14,\allowbreak16}
\crossref{1Sam}{17}{1}{1Sa 7:7; 13:5; 14:46,\allowbreak52 Jud 3:3}
\crossref{1Sam}{17}{2}{17:19; 21:9}
\crossref{1Sam}{17}{3}{}
\crossref{1Sam}{17}{4}{17:23; 21:9,\allowbreak10 2Sa 21:19 1Ch 20:5}
\crossref{1Sam}{17}{5}{17:38}
\crossref{1Sam}{17}{6}{1Ki 10:16 2Ch 9:15}
\crossref{1Sam}{17}{7}{2Sa 21:19 1Ch 11:23; 20:5}
\crossref{1Sam}{17}{8}{17:26; 8:17 2Sa 11:11 1Ch 21:3}
\crossref{1Sam}{17}{9}{1Sa 11:1}
\crossref{1Sam}{17}{10}{17:25,\allowbreak26,\allowbreak36,\allowbreak45 Nu 23:7,\allowbreak8 2Sa 21:21; 23:9 Ne 2:19}
\crossref{1Sam}{17}{11}{De 31:8 Jos 1:9 Ps 27:1 Pr 28:1 Isa 51:12,\allowbreak13; 57:11}
\crossref{1Sam}{17}{12}{17:58; 16:1,\allowbreak18 Ru 4:22 Mt 1:6 Lu 3:31,\allowbreak32}
\crossref{1Sam}{17}{13}{17:28; 16:6-\allowbreak9 1Ch 2:13}
\crossref{1Sam}{17}{14}{1Sa 16:11 Ge 25:23}
\crossref{1Sam}{17}{15}{1Sa 16:11,\allowbreak19-\allowbreak23}
\crossref{1Sam}{17}{16}{Mt 4:2 Lu 4:2}
\crossref{1Sam}{17}{17}{Mt 7:11 Lu 11:13}
\crossref{1Sam}{17}{18}{1Sa 16:20}
\crossref{1Sam}{17}{19}{}
\crossref{1Sam}{17}{20}{17:28 Eph 6:1,\allowbreak2}
\crossref{1Sam}{17}{21}{}
\crossref{1Sam}{17}{22}{}
\crossref{1Sam}{17}{23}{17:4-\allowbreak10}
\crossref{1Sam}{17}{24}{1Sa 13:6,\allowbreak7}
\crossref{1Sam}{17}{25}{1Sa 18:17-\allowbreak27 Jos 15:16 Re 2:7,\allowbreak17; 3:5,\allowbreak12,\allowbreak21}
\crossref{1Sam}{17}{26}{1Sa 11:2 Jos 7:8,\allowbreak9 2Ki 19:4 Ne 5:9 Ps 44:13; 74:18; 79:12 Da 9:16}
\crossref{1Sam}{17}{27}{17:25}
\crossref{1Sam}{17}{28}{1Sa 16:13 Ge 37:4,\allowbreak8,\allowbreak11 Pr 18:19; 27:4 Ec 4:4 Mt 10:36; 27:18}
\crossref{1Sam}{17}{29}{Pr 15:1 Ac 11:2-\allowbreak4 1Co 2:15 1Pe 3:9}
\crossref{1Sam}{17}{30}{17:26,\allowbreak27}
\crossref{1Sam}{17}{31}{Pr 22:29}
\crossref{1Sam}{17}{32}{Nu 13:30; 14:9 De 20:1-\allowbreak3 Isa 35:4 Heb 12:12}
\crossref{1Sam}{17}{33}{Nu 13:31 De 9:2 Ps 11:1 Re 13:4}
\crossref{1Sam}{17}{34}{Ge 8:22}
\crossref{1Sam}{17}{35}{Jud 14:5,\allowbreak6 2Sa 23:20 Ps 91:13 Da 6:22 Am 3:12 Ac 28:4-\allowbreak6}
\crossref{1Sam}{17}{36}{17:26 Eze 32:19,\allowbreak27-\allowbreak32 Ro 2:28,\allowbreak29}
\crossref{1Sam}{17}{37}{1Sa 7:12 Ps 11:1; 18:16,\allowbreak17; 63:7; 77:11; 138:3,\allowbreak7,\allowbreak8 2Co 1:9,\allowbreak10}
\crossref{1Sam}{17}{38}{}
\crossref{1Sam}{17}{39}{Ho 1:7 Zec 4:6 2Co 10:4,\allowbreak5}
\crossref{1Sam}{17}{40}{Jud 3:31; 7:16-\allowbreak20; 15:15,\allowbreak16; 20:16 1Co 1:27-\allowbreak29}
\crossref{1Sam}{17}{41}{}
\crossref{1Sam}{17}{42}{1Ki 20:18 2Ki 18:23,\allowbreak24 Ne 4:2-\allowbreak4 Ps 123:3,\allowbreak4 2Co 11:27-\allowbreak29}
\crossref{1Sam}{17}{43}{1Sa 24:14 2Sa 3:8; 9:8; 16:9 2Ki 8:13}
\crossref{1Sam}{17}{44}{1Ki 20:10,\allowbreak11 Pr 18:12 Ec 9:11,\allowbreak12 Jer 9:23 Eze 28:2,\allowbreak9,\allowbreak10}
\crossref{1Sam}{17}{45}{Ps 44:6}
\crossref{1Sam}{17}{46}{De 7:2,\allowbreak23; 9:2,\allowbreak3 Jos 10:8}
\crossref{1Sam}{17}{47}{Ps 33:16,\allowbreak17; 44:6,\allowbreak7 Pr 21:30,\allowbreak31 Ho 1:7}
\crossref{1Sam}{17}{48}{Ps 27:1 Pr 28:1}
\crossref{1Sam}{17}{49}{1Ki 22:34 2Ki 9:24 1Co 1:27,\allowbreak28}
\crossref{1Sam}{17}{50}{17:39; 13:22}
\crossref{1Sam}{17}{51}{1Sa 21:9 2Sa 23:21 Es 7:10 Ps 7:15,\allowbreak16 Heb 2:14}
\crossref{1Sam}{17}{52}{1Sa 14:21,\allowbreak22 Jud 7:23 2Sa 23:10}
\crossref{1Sam}{17}{53}{2Ki 7:7-\allowbreak16 Jer 4:20; 30:16}
\crossref{1Sam}{17}{54}{1Sa 21:9 Ex 16:33 Jos 4:7,\allowbreak8}
\crossref{1Sam}{17}{55}{17:58; 16:21,\allowbreak22}
\crossref{1Sam}{17}{56}{}
\crossref{1Sam}{17}{57}{17:54}
\crossref{1Sam}{17}{58}{17:12; 16:18,\allowbreak19}
\crossref{1Sam}{18}{1}{18:3; 19:2; 20:17 De 13:6 2Sa 1:26 Pr 18:24}
\crossref{1Sam}{18}{2}{1Sa 16:21-\allowbreak23; 17:15}
\crossref{1Sam}{18}{3}{1Sa 20:8-\allowbreak17,\allowbreak42; 23:18 2Sa 9:1-\allowbreak3; 21:7}
\crossref{1Sam}{18}{4}{}
\crossref{1Sam}{18}{5}{18:14,\allowbreak15,\allowbreak30 Ge 39:2,\allowbreak3,\allowbreak23 Ps 1:3 Ac 7:10}
\crossref{1Sam}{18}{6}{Ex 15:20 Jud 11:34 Ps 68:25 Jer 31:11-\allowbreak13}
\crossref{1Sam}{18}{7}{Ex 15:21 Ps 24:7,\allowbreak8}
\crossref{1Sam}{18}{8}{Es 3:5 Pr 13:10; 27:4 Ec 4:4 Jas 4:5}
\crossref{1Sam}{18}{9}{Ge 4:5,\allowbreak6; 31:2 Mt 20:15 Mr 7:22 Eph 4:27 Jas 5:9}
\crossref{1Sam}{18}{10}{1Sa 16:14,\allowbreak15; 19:9; 26:19}
\crossref{1Sam}{18}{11}{1Sa 19:9,\allowbreak10; 20:33 Pr 27:4 Isa 54:17}
\crossref{1Sam}{18}{12}{18:15,\allowbreak20,\allowbreak29; 16:4 Ps 48:3-\allowbreak6; 53:5 Mr 6:20 Lu 8:37 Ac 24:25}
\crossref{1Sam}{18}{13}{18:17,\allowbreak25; 8:12; 22:7}
\crossref{1Sam}{18}{14}{18:5}
\crossref{1Sam}{18}{15}{Ps 112:5 Da 6:4,\allowbreak5 Col 4:5 Jas 1:5; 3:17}
\crossref{1Sam}{18}{16}{18:5 Lu 19:48; 20:19}
\crossref{1Sam}{18}{17}{1Sa 17:25 Ps 12:2; 55:21}
\crossref{1Sam}{18}{18}{18:23; 9:21 Ex 3:11 Ru 2:10 2Sa 7:18 Pr 15:33; 18:12 Jer 1:6}
\crossref{1Sam}{18}{19}{Jud 14:20 2Sa 21:8}
\crossref{1Sam}{18}{20}{18:28}
\crossref{1Sam}{18}{21}{Ex 10:7 Ps 7:14-\allowbreak16; 38:12 Pr 26:24-\allowbreak26; 29:5 Jer 5:26; 9:8}
\crossref{1Sam}{18}{22}{Ps 36:1-\allowbreak3; 55:21}
\crossref{1Sam}{18}{23}{1Jo 3:1}
\crossref{1Sam}{18}{24}{18:24}
\crossref{1Sam}{18}{25}{Ge 29:18; 34:12 Ex 22:16,\allowbreak17}
\crossref{1Sam}{18}{26}{18:21}
\crossref{1Sam}{18}{27}{18:13}
\crossref{1Sam}{18}{28}{1Sa 24:20; 26:25 Ge 30:27; 37:8-\allowbreak11; 39:3 Re 3:9}
\crossref{1Sam}{18}{29}{18:12,\allowbreak15 Ps 37:12-\allowbreak14 Ec 4:4 Jas 2:19}
\crossref{1Sam}{18}{30}{2Sa 11:1}
\crossref{1Sam}{19}{1}{}
\crossref{1Sam}{19}{2}{1Sa 18:1-\allowbreak3 Ps 16:3 Joh 15:17-\allowbreak19 1Jo 3:12-\allowbreak14}
\crossref{1Sam}{19}{3}{1Sa 20:9,\allowbreak13}
\crossref{1Sam}{19}{4}{1Sa 20:32; 22:14 Pr 24:11,\allowbreak12; 31:8,\allowbreak9 Jer 18:20}
\crossref{1Sam}{19}{5}{1Sa 28:21 Jud 9:17; 12:3 Ps 119:109 Ac 20:24 Php 2:30}
\crossref{1Sam}{19}{6}{1Sa 14:39; 28:10 Ps 15:4 Pr 26:24,\allowbreak25 Jer 5:2}
\crossref{1Sam}{19}{7}{1Sa 16:21; 18:2,\allowbreak10,\allowbreak13 Ge 31:2 Ex 4:10 1Ch 11:2 Isa 30:33}
\crossref{1Sam}{19}{8}{Ps 18:32-\allowbreak50; 27:3}
\crossref{1Sam}{19}{9}{1Sa 16:14; 18:10,\allowbreak11}
\crossref{1Sam}{19}{10}{19:6 Ho 6:4 Mt 12:43-\allowbreak45 Lu 11:24-\allowbreak26 2Pe 2:20-\allowbreak22}
\crossref{1Sam}{19}{11}{Ps 59:1}
\crossref{1Sam}{19}{12}{Ps 34:19}
\crossref{1Sam}{19}{13}{Ge 31:19}
\crossref{1Sam}{19}{14}{Jos 2:5 2Sa 16:17-\allowbreak19; 17:20}
\crossref{1Sam}{19}{15}{}
\crossref{1Sam}{19}{16}{}
\crossref{1Sam}{19}{17}{1Sa 22:17; 28:12 Mt 2:16}
\crossref{1Sam}{19}{18}{1Sa 7:17; 15:34; 28:3 Ps 116:11 Jas 5:16}
\crossref{1Sam}{19}{19}{1Sa 22:9,\allowbreak10; 23:19; 26:1 Pr 29:12}
\crossref{1Sam}{19}{20}{19:11,\allowbreak14 Joh 7:32,\allowbreak45}
\crossref{1Sam}{19}{21}{2Ki 1:9-\allowbreak13 Pr 27:22 Jer 13:23}
\crossref{1Sam}{19}{22}{}
\crossref{1Sam}{19}{23}{19:20; 10:10 Nu 23:5; 24:2 Mt 7:22 Joh 11:51 1Co 13:2}
\crossref{1Sam}{19}{24}{2Sa 6:14,\allowbreak20 Isa 20:2 Mic 1:8}
\crossref{1Sam}{20}{1}{1Sa 19:19-\allowbreak24; 23:26-\allowbreak28 Ps 124:6-\allowbreak8 2Pe 2:9}
\crossref{1Sam}{20}{2}{1Sa 14:45 Ge 44:7 Jos 22:29; 24:16 Lu 20:16}
\crossref{1Sam}{20}{3}{De 6:13 Jer 4:2 Heb 6:16}
\crossref{1Sam}{20}{4}{Joh 14:13}
\crossref{1Sam}{20}{5}{20:19; 19:2 Ps 55:12 Pr 22:3 Joh 8:59 Ac 17:14}
\crossref{1Sam}{20}{6}{1Sa 17:58 Joh 7:42}
\crossref{1Sam}{20}{7}{De 1:23 2Sa 17:4}
\crossref{1Sam}{20}{8}{Ge 24:49; 47:29 Jos 2:14 Ru 1:8 Pr 3:3}
\crossref{1Sam}{20}{9}{20:38,\allowbreak42; 19:2}
\crossref{1Sam}{20}{10}{20:30-\allowbreak34; 25:10,\allowbreak14,\allowbreak17 Ge 42:7,\allowbreak30 1Ki 12:13 Pr 18:23}
\crossref{1Sam}{20}{11}{}
\crossref{1Sam}{20}{12}{Pr 20:5; 25:2,\allowbreak3}
\crossref{1Sam}{20}{13}{1Sa 3:17; 25:22 Ru 1:17 2Sa 3:35; 19:13 1Ki 19:2; 20:10}
\crossref{1Sam}{20}{14}{2Sa 9:3 Eph 5:1,\allowbreak2}
\crossref{1Sam}{20}{15}{1Sa 24:21 2Sa 9:1-\allowbreak7; 21:7}
\crossref{1Sam}{20}{16}{1Sa 18:3 Ge 15:18}
\crossref{1Sam}{20}{17}{}
\crossref{1Sam}{20}{18}{20:5}
\crossref{1Sam}{20}{19}{20:5; 19:2}
\crossref{1Sam}{20}{20}{}
\crossref{1Sam}{20}{21}{Jer 4:2; 5:2; 12:16 Am 8:14}
\crossref{1Sam}{20}{22}{}
\crossref{1Sam}{20}{23}{20:14,\allowbreak15}
\crossref{1Sam}{20}{24}{Ps 50:16-\allowbreak21 Pr 4:17; 15:17; 17:1; 21:3,\allowbreak27 Isa 1:11-\allowbreak15 Zec 7:6}
\crossref{1Sam}{20}{25}{Jud 16:20}
\crossref{1Sam}{20}{26}{Le 7:21; 11:24,\allowbreak27,\allowbreak31,\allowbreak40; 15:5,\allowbreak16,\allowbreak17,\allowbreak19-\allowbreak21 Nu 19:16}
\crossref{1Sam}{20}{27}{1Sa 18:11; 19:9,\allowbreak10,\allowbreak15}
\crossref{1Sam}{20}{28}{}
\crossref{1Sam}{20}{29}{1Sa 17:28}
\crossref{1Sam}{20}{30}{Job 5:2 Pr 14:29; 19:12,\allowbreak19; 21:24; 25:28; 27:3 Jas 1:19,\allowbreak20}
\crossref{1Sam}{20}{31}{20:8; 19:6,\allowbreak11-\allowbreak15}
\crossref{1Sam}{20}{32}{1Sa 19:5 Pr 24:11,\allowbreak12; 31:8,\allowbreak9 Joh 7:51}
\crossref{1Sam}{20}{33}{}
\crossref{1Sam}{20}{34}{Ec 7:20 Eph 4:26}
\crossref{1Sam}{20}{35}{20:19 2Sa 20:5}
\crossref{1Sam}{20}{36}{20:20,\allowbreak21}
\crossref{1Sam}{20}{37}{20:37}
\crossref{1Sam}{20}{38}{Ps 55:6-\allowbreak9 Pr 6:4,\allowbreak5 Mt 24:16-\allowbreak18 Mr 13:14-\allowbreak16 Lu 17:31,\allowbreak32}
\crossref{1Sam}{20}{39}{}
\crossref{1Sam}{20}{40}{1Sa 21:8}
\crossref{1Sam}{20}{41}{1Sa 25:23 Ge 43:28 2Sa 9:6}
\crossref{1Sam}{20}{42}{20:22; 1:17 Nu 6:26 Lu 7:50 Ac 16:36}
\crossref{1Sam}{21}{1}{1Sa 14:3}
\crossref{1Sam}{21}{2}{}
\crossref{1Sam}{21}{3}{21:4 Jud 9:29 Isa 3:6}
\crossref{1Sam}{21}{4}{21:6 Ex 25:30 Le 24:5-\allowbreak9 Mt 12:3,\allowbreak4}
\crossref{1Sam}{21}{5}{Ac 9:15 1Th 4:3,\allowbreak4 2Ti 2:20,\allowbreak21 1Pe 3:17}
\crossref{1Sam}{21}{6}{Mt 12:3,\allowbreak4 Mr 2:25-\allowbreak27 Lu 6:3,\allowbreak4}
\crossref{1Sam}{21}{7}{Jer 7:9-\allowbreak11 Eze 33:31 Am 8:5 Mt 15:8 Ac 21:26,\allowbreak27}
\crossref{1Sam}{21}{8}{}
\crossref{1Sam}{21}{9}{1Sa 17:51-\allowbreak54}
\crossref{1Sam}{21}{10}{1Sa 27:1 1Ki 19:3 Jer 26:21}
\crossref{1Sam}{21}{11}{Ps 56:1}
\crossref{1Sam}{21}{12}{Ps 119:11 Lu 2:19,\allowbreak51}
\crossref{1Sam}{21}{13}{Ps 34:1}
\crossref{1Sam}{21}{14}{Ec 7:7}
\crossref{1Sam}{21}{15}{21:15}
\crossref{1Sam}{22}{1}{1Sa 21:10-\allowbreak15 Ps 34:1; 57:1}
\crossref{1Sam}{22}{2}{Jud 11:3 Mt 11:12,\allowbreak28}
\crossref{1Sam}{22}{3}{Jud 11:29}
\crossref{1Sam}{22}{4}{2Sa 23:13,\allowbreak14 1Ch 12:16}
\crossref{1Sam}{22}{5}{}
\crossref{1Sam}{22}{6}{1Sa 18:10; 19:9; 20:33}
\crossref{1Sam}{22}{7}{22:9,\allowbreak13; 18:14; 20:27,\allowbreak30; 25:10 2Sa 20:1 1Ki 12:16 Isa 11:1,\allowbreak10}
\crossref{1Sam}{22}{8}{1Sa 20:2 Job 33:16}
\crossref{1Sam}{22}{9}{1Sa 21:1-\allowbreak15}
\crossref{1Sam}{22}{10}{22:13,\allowbreak15; 23:2,\allowbreak4,\allowbreak12; 30:8 Nu 27:21}
\crossref{1Sam}{22}{11}{Ro 3:15}
\crossref{1Sam}{22}{12}{22:7,\allowbreak13}
\crossref{1Sam}{22}{13}{22:8 Ps 119:69 Am 7:10 Lu 23:2-\allowbreak5}
\crossref{1Sam}{22}{14}{1Sa 19:4,\allowbreak5; 20:32; 24:11; 26:23 2Sa 22:23-\allowbreak25 Pr 24:11,\allowbreak12; 31:8,\allowbreak9}
\crossref{1Sam}{22}{15}{Ge 20:5,\allowbreak6 2Sa 15:11 2Co 1:12 1Pe 3:16,\allowbreak17}
\crossref{1Sam}{22}{16}{1Sa 14:44; 20:31 1Ki 18:4; 19:2 Pr 28:15 Da 2:5,\allowbreak12; 3:19,\allowbreak20 Ac 12:19}
\crossref{1Sam}{22}{17}{1Sa 8:11 2Sa 15:1 1Ki 1:5}
\crossref{1Sam}{22}{18}{22:9}
\crossref{1Sam}{22}{19}{22:9,\allowbreak11; 21:1 Ne 11:32 Isa 10:32}
\crossref{1Sam}{22}{20}{1Sa 23:6; 30:7 2Sa 20:25 1Ki 2:26,\allowbreak27}
\crossref{1Sam}{22}{21}{}
\crossref{1Sam}{22}{22}{1Sa 21:1-\allowbreak9 Ps 44:22}
\crossref{1Sam}{22}{23}{1Ki 2:26 Mt 24:9 Joh 15:20; 16:2,\allowbreak3 Heb 12:1-\allowbreak3}
\crossref{1Sam}{23}{1}{Le 26:16 De 28:33,\allowbreak51 Jud 6:4,\allowbreak11 Mic 6:15}
\crossref{1Sam}{23}{2}{}
\crossref{1Sam}{23}{3}{23:15,\allowbreak23,\allowbreak26 Ps 11:1 Jer 12:5}
\crossref{1Sam}{23}{4}{1Sa 28:6 Jud 6:39}
\crossref{1Sam}{23}{5}{}
\crossref{1Sam}{23}{6}{1Sa 22:20}
\crossref{1Sam}{23}{7}{23:14; 24:4-\allowbreak6; 26:8,\allowbreak9 Ps 71:10,\allowbreak11}
\crossref{1Sam}{23}{8}{}
\crossref{1Sam}{23}{9}{Jer 11:18,\allowbreak19 Ac 9:24; 14:6; 23:16-\allowbreak18}
\crossref{1Sam}{23}{10}{23:8; 22:19 Ge 18:24 Es 3:6 Pr 28:15 Ro 3:15,\allowbreak16}
\crossref{1Sam}{23}{11}{Ps 50:15 Jer 33:3 Mt 7:7,\allowbreak8}
\crossref{1Sam}{23}{12}{Ps 31:8}
\crossref{1Sam}{23}{13}{1Sa 22:2; 25:13; 30:9,\allowbreak10}
\crossref{1Sam}{23}{14}{Ps 11:1-\allowbreak3}
\crossref{1Sam}{23}{15}{Ge 44:30}
\crossref{1Sam}{23}{16}{De 3:28 Ne 2:18 Job 4:3,\allowbreak4; 16:5 Pr 27:9,\allowbreak17 Ec 4:9-\allowbreak12}
\crossref{1Sam}{23}{17}{Isa 41:10,\allowbreak14 Heb 13:6}
\crossref{1Sam}{23}{18}{1Sa 18:3; 20:12-\allowbreak17,\allowbreak42 2Sa 9:1; 21:7}
\crossref{1Sam}{23}{19}{1Sa 22:7,\allowbreak8; 26:1 Ps 54:1}
\crossref{1Sam}{23}{20}{De 18:6 2Sa 3:21 Ps 112:10 Pr 11:23}
\crossref{1Sam}{23}{21}{1Sa 22:8 Jud 17:2 Ps 10:3 Isa 66:5 Mic 3:11}
\crossref{1Sam}{23}{22}{Job 5:13}
\crossref{1Sam}{23}{23}{Mr 14:1,\allowbreak10,\allowbreak11 Joh 18:2,\allowbreak3}
\crossref{1Sam}{23}{24}{1Sa 25:2 Jos 15:55}
\crossref{1Sam}{23}{25}{23:28 Jud 15:8}
\crossref{1Sam}{23}{26}{1Sa 19:12; 20:38 2Sa 15:14; 17:21,\allowbreak22 Ps 31:22}
\crossref{1Sam}{23}{27}{Ge 22:14 De 32:36 2Ki 19:9 Ps 116:3}
\crossref{1Sam}{23}{28}{}
\crossref{1Sam}{23}{29}{23:14,\allowbreak19}
\crossref{1Sam}{24}{1}{1Sa 23:28,\allowbreak29}
\crossref{1Sam}{24}{2}{1Sa 13:2}
\crossref{1Sam}{24}{3}{Ps 141:6}
\crossref{1Sam}{24}{4}{1Sa 26:8-\allowbreak11 2Sa 4:8 Job 31:31}
\crossref{1Sam}{24}{5}{2Sa 12:9; 24:10 2Ki 22:19 1Jo 3:20,\allowbreak21}
\crossref{1Sam}{24}{6}{1Sa 26:9-\allowbreak11 2Sa 1:14 1Ki 21:3 Job 31:29,\allowbreak30 Mt 5:44 Ro 12:14-\allowbreak21}
\crossref{1Sam}{24}{7}{Ps 7:4 Mt 5:44 Ro 12:17-\allowbreak21}
\crossref{1Sam}{24}{8}{1Sa 26:17}
\crossref{1Sam}{24}{9}{1Sa 26:19 Le 19:16 Ps 101:5; 141:6 Pr 16:28; 17:4; 18:8; 25:23}
\crossref{1Sam}{24}{10}{24:4; 26:8}
\crossref{1Sam}{24}{11}{1Sa 18:27 2Ki 5:13 Pr 15:1}
\crossref{1Sam}{24}{12}{1Sa 26:10,\allowbreak23 Ge 16:5 Jud 11:27 Job 5:8 Ps 7:8,\allowbreak9; 35:1; 43:1; 94:1}
\crossref{1Sam}{24}{13}{Mt 7:16-\allowbreak18; 12:33,\allowbreak34; 15:19}
\crossref{1Sam}{24}{14}{2Sa 6:20 1Ki 21:7}
\crossref{1Sam}{24}{15}{24:12 2Ch 24:22 Mic 1:2}
\crossref{1Sam}{24}{16}{1Sa 26:17 Job 6:25 Pr 15:1; 25:11 Lu 21:15 Ac 6:10}
\crossref{1Sam}{24}{17}{1Sa 26:21 Ge 38:26 Ex 9:27 Ps 37:6 Mt 27:4}
\crossref{1Sam}{24}{18}{24:10; 23:7; 26:23}
\crossref{1Sam}{24}{19}{1Sa 23:21; 26:25 Jud 17:2 Ps 18:20 Pr 25:21,\allowbreak22}
\crossref{1Sam}{24}{20}{1Sa 20:30,\allowbreak31; 23:17 2Sa 3:17,\allowbreak18 Job 15:25 Mt 2:3-\allowbreak6,\allowbreak13,\allowbreak16}
\crossref{1Sam}{24}{21}{1Sa 20:14-\allowbreak17 Ge 21:23; 31:48,\allowbreak53 Heb 6:16}
\crossref{1Sam}{24}{22}{Pr 26:24,\allowbreak25 Mt 10:16,\allowbreak17 Joh 2:24}
\crossref{1Sam}{25}{1}{1Sa 28:3}
\crossref{1Sam}{25}{2}{1Sa 23:24}
\crossref{1Sam}{25}{3}{Pr 14:1; 31:26,\allowbreak30,\allowbreak31}
\crossref{1Sam}{25}{4}{Ge 38:13 2Sa 13:23}
\crossref{1Sam}{25}{5}{1Sa 17:22 Ge 43:23}
\crossref{1Sam}{25}{6}{1Th 3:8 1Ti 5:6}
\crossref{1Sam}{25}{7}{25:15,\allowbreak16,\allowbreak21; 22:2 Isa 11:6-\allowbreak9 Lu 3:14 Php 2:15; 4:8}
\crossref{1Sam}{25}{8}{Ne 8:10-\allowbreak12 Es 9:19 Ec 11:2 Lu 11:41; 14:12-\allowbreak14}
\crossref{1Sam}{25}{9}{Ge 8:4 2Ki 2:15 2Ch 14:7}
\crossref{1Sam}{25}{10}{1Sa 20:30; 22:7,\allowbreak8 Ex 5:2 Jud 9:28 2Sa 20:1 1Ki 12:16 Ps 73:7,\allowbreak8}
\crossref{1Sam}{25}{11}{25:3; 24:13 De 8:17 Jud 8:6 Job 31:17 Ps 73:7,\allowbreak8 1Pe 4:9}
\crossref{1Sam}{25}{12}{2Sa 24:13 Isa 36:21,\allowbreak22 Heb 13:17}
\crossref{1Sam}{25}{13}{Jos 9:14 Pr 14:29; 16:32; 19:2,\allowbreak11; 25:8 Jas 1:19,\allowbreak20}
\crossref{1Sam}{25}{14}{Mr 15:29}
\crossref{1Sam}{25}{15}{25:7,\allowbreak21 Php 2:15}
\crossref{1Sam}{25}{16}{Ex 14:22 Job 1:10 Jer 15:20 Zec 2:5}
\crossref{1Sam}{25}{17}{1Sa 20:7,\allowbreak9,\allowbreak33 2Ch 25:16 Es 7:7}
\crossref{1Sam}{25}{18}{25:34 Nu 16:46-\allowbreak48 Pr 6:4,\allowbreak5 Mt 5:25}
\crossref{1Sam}{25}{19}{Ge 32:16,\allowbreak20}
\crossref{1Sam}{25}{20}{2Ki 4:24}
\crossref{1Sam}{25}{21}{25:13 Job 30:8 Ps 37:8 Eph 4:26,\allowbreak31 1Th 5:15 1Pe 2:21-\allowbreak23; 3:9}
\crossref{1Sam}{25}{22}{25:34}
\crossref{1Sam}{25}{23}{Jos 15:18 Jud 1:14}
\crossref{1Sam}{25}{24}{2Ki 4:37 Es 8:3 Mt 18:29}
\crossref{1Sam}{25}{25}{2Sa 13:33 Isa 42:25 Mal 2:2}
\crossref{1Sam}{25}{26}{25:34; 22:3 2Ki 2:2; 4:6}
\crossref{1Sam}{25}{27}{1Sa 30:26 Ge 33:11 2Ki 5:15 2Co 9:5}
\crossref{1Sam}{25}{28}{25:24}
\crossref{1Sam}{25}{29}{Joh 10:27-\allowbreak30; 14:19; 17:21,\allowbreak23 Col 3:3,\allowbreak4 1Pe 1:5}
\crossref{1Sam}{25}{30}{1Sa 13:14; 15:28; 23:17 Ps 89:20}
\crossref{1Sam}{25}{31}{Pr 5:12,\allowbreak13 Ro 14:21 2Co 1:12}
\crossref{1Sam}{25}{32}{}
\crossref{1Sam}{25}{33}{Ps 141:5 Pr 9:9; 17:10; 25:12; 27:21; 28:23}
\crossref{1Sam}{25}{34}{25:26}
\crossref{1Sam}{25}{35}{1Sa 20:42 2Sa 15:9 2Ki 5:19 Lu 7:50; 8:48}
\crossref{1Sam}{25}{36}{2Sa 13:23 Es 1:3-\allowbreak7 Lu 14:12}
\crossref{1Sam}{25}{37}{25:22,\allowbreak34}
\crossref{1Sam}{25}{38}{25:33; 6:9 Ex 12:29 2Ki 15:5; 19:35 2Ch 10:15 Ac 12:23}
\crossref{1Sam}{25}{39}{25:32 Jud 5:2 2Sa 22:47-\allowbreak49 Ps 58:10,\allowbreak11 Re 19:1-\allowbreak4}
\crossref{1Sam}{25}{40}{Ge 24:37,\allowbreak38,\allowbreak51}
\crossref{1Sam}{25}{41}{Ru 2:10,\allowbreak13 Pr 15:33; 18:12}
\crossref{1Sam}{25}{42}{Ge 24:61-\allowbreak67 Ps 45:10,\allowbreak11}
\crossref{1Sam}{25}{43}{Jos 15:56 2Sa 3:2}
\crossref{1Sam}{25}{44}{1Sa 18:20,\allowbreak27}
\crossref{1Sam}{26}{1}{Jos 15:24,\allowbreak55}
\crossref{1Sam}{26}{2}{1Sa 23:23-\allowbreak25; 24:17 Ps 38:12; 140:4-\allowbreak9}
\crossref{1Sam}{26}{3}{26:1; 23:19}
\crossref{1Sam}{26}{4}{Jos 2:1 Mt 10:16}
\crossref{1Sam}{26}{5}{1Sa 9:1; 14:50,\allowbreak51; 17:55 2Sa 2:8-\allowbreak12; 3:7,\allowbreak8,\allowbreak27,\allowbreak33-\allowbreak38 1Ch 9:39}
\crossref{1Sam}{26}{6}{Ge 10:15; 15:20 2Sa 11:6,\allowbreak21,\allowbreak24; 12:9; 23:39}
\crossref{1Sam}{26}{7}{1Th 5:2,\allowbreak3}
\crossref{1Sam}{26}{8}{26:23; 23:14; 24:4,\allowbreak18,\allowbreak19 Jos 21:44 Jud 1:4}
\crossref{1Sam}{26}{9}{1Sa 24:6,\allowbreak7 2Sa 1:14,\allowbreak16 Ps 105:15}
\crossref{1Sam}{26}{10}{1Sa 24:15; 25:26,\allowbreak38 Ps 94:1,\allowbreak2,\allowbreak23 Lu 18:7 Ro 12:19 Re 18:8}
\crossref{1Sam}{26}{11}{1Sa 24:6,\allowbreak12 2Sa 1:14,\allowbreak16}
\crossref{1Sam}{26}{12}{26:7; 24:4}
\crossref{1Sam}{26}{13}{1Sa 24:8 Jud 9:7}
\crossref{1Sam}{26}{14}{}
\crossref{1Sam}{26}{15}{26:8}
\crossref{1Sam}{26}{16}{1Sa 20:31 2Sa 12:5; 19:28 Ps 79:11; 102:20}
\crossref{1Sam}{26}{17}{1Sa 24:8,\allowbreak16}
\crossref{1Sam}{26}{18}{1Sa 24:9,\allowbreak11-\allowbreak14 Ps 7:3-\allowbreak5; 35:7; 69:4}
\crossref{1Sam}{26}{19}{1Sa 25:24 Ge 44:18}
\crossref{1Sam}{26}{20}{1Sa 2:9; 25:29}
\crossref{1Sam}{26}{21}{1Sa 15:24,\allowbreak30; 24:17 Ex 9:27 Nu 22:34 Mt 27:4}
\crossref{1Sam}{26}{22}{}
\crossref{1Sam}{26}{23}{1Ki 8:32 Ne 13:14 Ps 7:8,\allowbreak9; 18:20-\allowbreak26}
\crossref{1Sam}{26}{24}{Ps 18:25 Mt 5:7; 7:2}
\crossref{1Sam}{26}{25}{1Sa 24:19 Nu 24:9,\allowbreak10}
\crossref{1Sam}{27}{1}{1Sa 16:1,\allowbreak13; 23:17; 25:30 Ps 116:11 Pr 13:12 Isa 40:27-\allowbreak31; 51:12}
\crossref{1Sam}{27}{2}{1Sa 25:13; 30:8}
\crossref{1Sam}{27}{3}{1Sa 25:3,\allowbreak18-\allowbreak35,\allowbreak42,\allowbreak43; 30:5}
\crossref{1Sam}{27}{4}{1Sa 26:21}
\crossref{1Sam}{27}{5}{Ge 46:34 2Co 6:17}
\crossref{1Sam}{27}{6}{}
\crossref{1Sam}{27}{7}{}
\crossref{1Sam}{27}{8}{Jos 13:2,\allowbreak13 2Sa 13:37,\allowbreak38; 14:23,\allowbreak32; 15:8 1Ch 2:23}
\crossref{1Sam}{27}{9}{1Sa 15:7 Ge 16:7; 25:18 Ex 15:22}
\crossref{1Sam}{27}{10}{1Sa 21:2 Ge 27:19,\allowbreak20,\allowbreak24 Jos 2:4-\allowbreak6 2Sa 17:20 Ps 119:29,\allowbreak163}
\crossref{1Sam}{27}{11}{1Sa 22:22 Pr 12:19; 29:25}
\crossref{1Sam}{27}{12}{1Sa 13:4 Ge 34:30}
\crossref{1Sam}{28}{1}{1Sa 7:7; 13:5; 17:1; 29:1}
\crossref{1Sam}{28}{2}{1Sa 27:10 2Sa 16:16-\allowbreak19 Ro 12:9}
\crossref{1Sam}{28}{3}{1Sa 25:1 Isa 57:1,\allowbreak2}
\crossref{1Sam}{28}{4}{Jos 19:18 2Ki 4:8}
\crossref{1Sam}{28}{5}{Job 15:21; 18:11 Ps 48:5,\allowbreak6; 73:19 Pr 10:24 Isa 7:2; 21:3,\allowbreak4}
\crossref{1Sam}{28}{6}{1Sa 14:37 1Ch 10:14 Pr 1:,\allowbreak27,\allowbreak28 La 2:9 Eze 20:1-\allowbreak3 Joh 9:31}
\crossref{1Sam}{28}{7}{2Ki 1:2,\allowbreak3; 6:33 Isa 8:19,\allowbreak20 La 3:25,\allowbreak26 Hab 2:3}
\crossref{1Sam}{28}{8}{1Ki 14:2,\allowbreak3; 22:30,\allowbreak34 Job 24:13-\allowbreak15 Jer 23:24 Joh 3:19}
\crossref{1Sam}{28}{9}{28:3}
\crossref{1Sam}{28}{10}{1Sa 14:39; 19:6 Ge 3:5 Ex 20:7 De 18:10-\allowbreak12 2Sa 14:11 Mt 26:72}
\crossref{1Sam}{28}{11}{}
\crossref{1Sam}{28}{12}{28:3 1Ki 14:5}
\crossref{1Sam}{28}{13}{Ex 4:16; 22:28 Ps 82:6,\allowbreak7 Joh 10:34,\allowbreak35}
\crossref{1Sam}{28}{14}{1Sa 15:27 2Ki 2:8,\allowbreak13,\allowbreak14}
\crossref{1Sam}{28}{15}{28:8,\allowbreak11}
\crossref{1Sam}{28}{16}{Jud 5:31 2Ki 6:27 Ps 68:1-\allowbreak3 Re 18:20,\allowbreak24; 19:1-\allowbreak6}
\crossref{1Sam}{28}{17}{1Sa 13:13,\allowbreak14; 15:27-\allowbreak29}
\crossref{1Sam}{28}{18}{1Sa 13:9; 15:9,\allowbreak23-\allowbreak26 1Ki 20:42 1Ch 10:13 Jer 48:10}
\crossref{1Sam}{28}{19}{1Sa 12:25; 31:1-\allowbreak6 1Ki 22:20,\allowbreak28}
\crossref{1Sam}{28}{20}{}
\crossref{1Sam}{28}{21}{1Sa 19:5 Jud 12:3 Job 13:14}
\crossref{1Sam}{28}{22}{}
\crossref{1Sam}{28}{23}{1Ki 21:4 Pr 25:20}
\crossref{1Sam}{28}{24}{Ge 18:7,\allowbreak8 Lu 15:23}
\crossref{1Sam}{28}{25}{28:25}
\crossref{1Sam}{29}{1}{1Sa 28:1,\allowbreak2}
\crossref{1Sam}{29}{2}{29:6,\allowbreak7; 5:8-\allowbreak11; 6:4 Jos 13:3}
\crossref{1Sam}{29}{3}{1Sa 27:7}
\crossref{1Sam}{29}{4}{}
\crossref{1Sam}{29}{5}{1Sa 18:6,\allowbreak7; 21:11 Pr 27:14}
\crossref{1Sam}{29}{6}{1Sa 20:3; 28:10 De 10:20 Isa 65:16 Jer 12:16}
\crossref{1Sam}{29}{7}{Nu 22:34}
\crossref{1Sam}{29}{8}{1Sa 12:3; 17:29; 20:8; 26:18}
\crossref{1Sam}{29}{9}{2Sa 14:17,\allowbreak20; 19:27 Ga 4:14}
\crossref{1Sam}{29}{10}{1Sa 30:1,\allowbreak2 Ge 22:14 Ps 37:23,\allowbreak24 1Co 10:13 2Pe 2:9}
\crossref{1Sam}{29}{11}{29:1 Jos 19:18 2Sa 4:4}
\crossref{1Sam}{30}{1}{1Sa 29:11 2Sa 1:2}
\crossref{1Sam}{30}{2}{30:19; 27:11 Job 38:11 Ps 76:10 Isa 27:8,\allowbreak9}
\crossref{1Sam}{30}{3}{Ps 34:19 Heb 12:6 1Pe 1:6,\allowbreak7 Re 3:9}
\crossref{1Sam}{30}{4}{1Sa 4:13; 11:4 Ge 37:33-\allowbreak35 Nu 14:1,\allowbreak39 Jud 2:4; 21:2 Ezr 10:1}
\crossref{1Sam}{30}{5}{1Sa 1:2; 25:42,\allowbreak43; 27:3 2Sa 2:2; 3:2,\allowbreak3}
\crossref{1Sam}{30}{6}{Ge 32:7 Ps 25:17; 42:7; 116:3,\allowbreak4,\allowbreak10 2Co 1:8,\allowbreak9; 4:8; 7:5}
\crossref{1Sam}{30}{7}{1Sa 22:20,\allowbreak21; 23:2-\allowbreak9 1Ki 2:26 Mr 2:26}
\crossref{1Sam}{30}{8}{1Sa 23:2,\allowbreak4,\allowbreak10-\allowbreak12 Jud 20:18,\allowbreak23,\allowbreak28 2Sa 5:19,\allowbreak23 Pr 3:5,\allowbreak6}
\crossref{1Sam}{30}{9}{}
\crossref{1Sam}{30}{10}{30:21}
\crossref{1Sam}{30}{11}{De 15:7-\allowbreak11; 23:7 Pr 25:21 Mt 25:35 Lu 10:36,\allowbreak37 Ro 12:20,\allowbreak21}
\crossref{1Sam}{30}{12}{1Sa 14:27 Jud 15:19 Isa 40:29-\allowbreak31}
\crossref{1Sam}{30}{13}{}
\crossref{1Sam}{30}{14}{}
\crossref{1Sam}{30}{15}{1Sa 29:6 Jos 2:12; 9:15,\allowbreak19,\allowbreak20 Eze 17:13,\allowbreak16,\allowbreak19}
\crossref{1Sam}{30}{16}{Jud 1:24,\allowbreak25}
\crossref{1Sam}{30}{17}{1Sa 11:11 Jud 4:16 1Ki 20:29,\allowbreak30 Ps 18:42}
\crossref{1Sam}{30}{18}{30:18}
\crossref{1Sam}{30}{19}{30:8 Ge 14:14-\allowbreak16 Nu 31:49 Job 1:10 Ps 34:9,\allowbreak10; 91:9,\allowbreak10 Mt 6:33}
\crossref{1Sam}{30}{20}{30:26 Nu 31:9-\allowbreak12 2Ch 20:25 Isa 53:12 Ro 8:37}
\crossref{1Sam}{30}{21}{30:10}
\crossref{1Sam}{30}{22}{1Sa 22:2; 25:17,\allowbreak25 De 13:13 Jud 19:22 1Ki 21:10,\allowbreak13}
\crossref{1Sam}{30}{23}{Ge 19:7 Jud 19:23 Ac 7:2; 22:1}
\crossref{1Sam}{30}{24}{1Sa 25:13}
\crossref{1Sam}{30}{25}{1Sa 16:13}
\crossref{1Sam}{30}{26}{1Ch 12:1-\allowbreak15 Ps 35:27; 68:18 Pr 18:16-\allowbreak24 Isa 32:8}
\crossref{1Sam}{30}{27}{}
\crossref{1Sam}{30}{28}{Jos 13:16}
\crossref{1Sam}{30}{29}{}
\crossref{1Sam}{30}{30}{Jos 19:4 Jud 1:17}
\crossref{1Sam}{30}{31}{Jos 14:13,\allowbreak14 2Sa 2:1; 4:1; 15:10}
\crossref{1Sam}{31}{1}{1Sa 28:1,\allowbreak15; 29:1}
\crossref{1Sam}{31}{2}{1Sa 14:22 2Sa 1:6}
\crossref{1Sam}{31}{3}{2Sa 1:4 Am 2:14}
\crossref{1Sam}{31}{4}{Jud 9:54 1Ch 10:4}
\crossref{1Sam}{31}{5}{1Ch 10:5}
\crossref{1Sam}{31}{6}{1Sa 4:10,\allowbreak11; 11:15; 12:17,\allowbreak25; 28:19 1Ch 10:6 Ec 9:1,\allowbreak2 Ho 13:10,\allowbreak11}
\crossref{1Sam}{31}{7}{1Sa 13:6 Le 26:32,\allowbreak36 De 28:33 Jud 6:2}
\crossref{1Sam}{31}{8}{1Ch 10:8 2Ch 20:25}
\crossref{1Sam}{31}{9}{31:4; 17:51,\allowbreak54 1Ch 10:9,\allowbreak10}
\crossref{1Sam}{31}{10}{Jud 2:13}
\crossref{1Sam}{31}{11}{1Sa 11:1 2Sa 2:4}
\crossref{1Sam}{31}{12}{2Ch 16:14 Jer 34:5 Am 6:10}
\crossref{1Sam}{31}{13}{Ge 35:8 2Sa 2:4,\allowbreak5; 21:12-\allowbreak14}

% 2Sam
\crossref{2Sam}{1}{1}{1Sa 30:17-\allowbreak26}
\crossref{2Sam}{1}{2}{Ge 22:4 Es 4:16; 5:1 Ho 6:2 Mt 12:40; 16:21}
\crossref{2Sam}{1}{3}{2Ki 5:25}
\crossref{2Sam}{1}{4}{1Sa 4:16}
\crossref{2Sam}{1}{5}{Pr 14:15; 25:2}
\crossref{2Sam}{1}{6}{1:21 1Sa 28:4; 31:1}
\crossref{2Sam}{1}{7}{2Sa 9:6 Jud 9:54 1Sa 22:12 Isa 6:8}
\crossref{2Sam}{1}{8}{Ge 14:7 Ex 17:8-\allowbreak16 Nu 24:20 De 25:17-\allowbreak19 1Sa 15:3; 27:8}
\crossref{2Sam}{1}{9}{Ge 44:30}
\crossref{2Sam}{1}{10}{Jud 1:7; 9:54 1Sa 22:18; 31:4,\allowbreak5 Mt 7:2}
\crossref{2Sam}{1}{11}{2Sa 3:31; 13:31 Ge 37:29,\allowbreak34 Ac 14:14}
\crossref{2Sam}{1}{12}{Ps 35:13,\allowbreak14 Pr 24:17 Jer 9:1 Am 6:6 Mt 5:44 2Co 11:29 1Pe 3:8}
\crossref{2Sam}{1}{13}{}
\crossref{2Sam}{1}{14}{Nu 12:8 1Sa 31:4 2Pe 2:10}
\crossref{2Sam}{1}{15}{2Sa 4:10-\allowbreak12 Jud 8:20 1Sa 22:17,\allowbreak18 1Ki 2:25,\allowbreak34,\allowbreak46 Job 5:12}
\crossref{2Sam}{1}{16}{Ge 9:5,\allowbreak6 Le 20:9,\allowbreak11-\allowbreak13,\allowbreak16,\allowbreak27 De 19:10 Jos 2:19 Jud 9:24}
\crossref{2Sam}{1}{17}{1:19 Ge 50:11 2Ch 35:25 Jer 9:17-\allowbreak21}
\crossref{2Sam}{1}{18}{Ge 49:8 De 4:10}
\crossref{2Sam}{1}{19}{1:23 De 4:7,\allowbreak8 1Sa 31:8 Isa 4:2; 53:2 La 2:1 Zec 11:7,\allowbreak10}
\crossref{2Sam}{1}{20}{De 32:26,\allowbreak27 Jud 14:19; 16:23,\allowbreak24 1Sa 31:9 Mic 1:10}
\crossref{2Sam}{1}{21}{1Sa 31:1 1Ch 10:1,\allowbreak8}
\crossref{2Sam}{1}{22}{1Sa 14:6-\allowbreak14; 18:4 Isa 34:6,\allowbreak7}
\crossref{2Sam}{1}{23}{1Sa 18:1; 20:2}
\crossref{2Sam}{1}{24}{Jud 5:30 Ps 68:12 Pr 31:21 Isa 3:16-\allowbreak26 Jer 2:32 1Ti 2:9,\allowbreak10}
\crossref{2Sam}{1}{25}{1:19,\allowbreak27 La 5:16}
\crossref{2Sam}{1}{26}{1Sa 18:1-\allowbreak4; 19:2; 20:17,\allowbreak41; 23:16}
\crossref{2Sam}{1}{27}{1:19,\allowbreak25}
\crossref{2Sam}{2}{1}{2Sa 5:19,\allowbreak23 Nu 27:21 Jud 1:1 1Sa 23:2,\allowbreak4,\allowbreak9-\allowbreak12; 30:7,\allowbreak8 Ps 25:4,\allowbreak5}
\crossref{2Sam}{2}{2}{1Sa 25:42,\allowbreak43; 30:5 Lu 22:28,\allowbreak29}
\crossref{2Sam}{2}{3}{1Sa 22:2; 27:2,\allowbreak3; 30:1,\allowbreak9,\allowbreak10 1Ch 12:1-\allowbreak7}
\crossref{2Sam}{2}{4}{2:11; 19:11,\allowbreak42 Ge 49:8-\allowbreak10}
\crossref{2Sam}{2}{5}{Ru 1:8; 2:20; 3:10 1Sa 23:21; 24:19; 25:32,\allowbreak33 Ps 115:15}
\crossref{2Sam}{2}{6}{2Sa 15:20 Ps 57:3 Pr 14:22 Mt 5:7 2Ti 1:16-\allowbreak18}
\crossref{2Sam}{2}{7}{2Sa 10:12 Ge 15:1 1Sa 4:9; 31:7,\allowbreak12 1Co 16:13 Eph 6:10}
\crossref{2Sam}{2}{8}{1Sa 14:50; 17:55; 26:14}
\crossref{2Sam}{2}{9}{Nu 32:1-\allowbreak42 Jos 13:8-\allowbreak11 Ps 108:8}
\crossref{2Sam}{2}{10}{}
\crossref{2Sam}{2}{11}{2Sa 5:4,\allowbreak5 1Ki 2:11 1Ch 3:4; 29:27}
\crossref{2Sam}{2}{12}{2Sa 17:14 Ge 32:2}
\crossref{2Sam}{2}{13}{2:18; 8:16; 20:23 1Ki 1:7; 2:28-\allowbreak35 1Ch 2:16}
\crossref{2Sam}{2}{14}{2:17,\allowbreak26,\allowbreak27 Pr 10:23; 17:14; 20:18; 25:8; 26:18,\allowbreak19}
\crossref{2Sam}{2}{15}{}
\crossref{2Sam}{2}{16}{2:16}
\crossref{2Sam}{2}{17}{2Sa 3:1}
\crossref{2Sam}{2}{18}{1Ch 2:15,\allowbreak16; 11:26}
\crossref{2Sam}{2}{19}{2:21 Jos 1:7; 23:6 2Ki 22:2 Pr 4:27}
\crossref{2Sam}{2}{20}{}
\crossref{2Sam}{2}{21}{Jud 14:19}
\crossref{2Sam}{2}{22}{2Ki 14:10-\allowbreak12 Pr 29:1 Ec 6:10}
\crossref{2Sam}{2}{23}{2Sa 3:27; 4:6; 5:6; 20:10}
\crossref{2Sam}{2}{24}{2:24}
\crossref{2Sam}{2}{25}{}
\crossref{2Sam}{2}{26}{2:14 Ac 7:26}
\crossref{2Sam}{2}{27}{2:14 Pr 15:1; 17:14; 20:18; 25:8 Isa 47:7 Lu 14:31,\allowbreak32}
\crossref{2Sam}{2}{28}{}
\crossref{2Sam}{2}{29}{2:12}
\crossref{2Sam}{2}{30}{}
\crossref{2Sam}{2}{31}{}
\crossref{2Sam}{2}{32}{1Sa 17:58 1Ch 2:13-\allowbreak16 2Ch 16:14; 21:1}
\crossref{2Sam}{3}{1}{1Ki 14:30; 15:16,\allowbreak32}
\crossref{2Sam}{3}{2}{1Ch 3:1-\allowbreak4}
\crossref{2Sam}{3}{3}{1Ch 3:1}
\crossref{2Sam}{3}{4}{1Ki 1:5-\allowbreak18; 2:13-\allowbreak25}
\crossref{2Sam}{3}{5}{}
\crossref{2Sam}{3}{6}{2Sa 2:8,\allowbreak9 2Ki 10:23 2Ch 25:8 Pr 21:30 Isa 8:9,\allowbreak10 Joe 3:9-\allowbreak13}
\crossref{2Sam}{3}{7}{2Sa 21:8-\allowbreak11}
\crossref{2Sam}{3}{8}{Ps 76:10 Mr 6:18,\allowbreak19}
\crossref{2Sam}{3}{9}{3:35; 19:13 Ru 1:17 1Sa 3:17; 14:44; 25:22 1Ki 19:2}
\crossref{2Sam}{3}{10}{2Sa 17:11; 24:2 Jud 20:1 1Ki 4:25}
\crossref{2Sam}{3}{11}{3:39}
\crossref{2Sam}{3}{12}{2Sa 19:6; 20:1-\allowbreak13}
\crossref{2Sam}{3}{13}{Ge 43:3; 44:23,\allowbreak26}
\crossref{2Sam}{3}{14}{2Sa 2:10}
\crossref{2Sam}{3}{15}{1Sa 25:44}
\crossref{2Sam}{3}{16}{Pr 9:17,\allowbreak18}
\crossref{2Sam}{3}{17}{3:17}
\crossref{2Sam}{3}{18}{3:9 1Sa 13:14; 15:28; 16:1,\allowbreak12,\allowbreak13 Joh 12:42,\allowbreak43}
\crossref{2Sam}{3}{19}{1Sa 10:20,\allowbreak21 1Ch 12:29 Ps 68:27}
\crossref{2Sam}{3}{20}{Ge 26:30; 31:54 Es 1:3}
\crossref{2Sam}{3}{21}{3:10,\allowbreak12; 2:9 Php 2:21}
\crossref{2Sam}{3}{22}{3:22}
\crossref{2Sam}{3}{23}{}
\crossref{2Sam}{3}{24}{}
\crossref{2Sam}{3}{25}{3:27 2Ki 18:32}
\crossref{2Sam}{3}{26}{Pr 26:23-\allowbreak26; 27:4-\allowbreak6}
\crossref{2Sam}{3}{27}{2Sa 20:9,\allowbreak10 De 27:24 1Ki 2:5,\allowbreak32}
\crossref{2Sam}{3}{28}{Ge 9:6 Ex 21:12 Nu 35:33 De 21:1-\allowbreak9 Mt 27:24}
\crossref{2Sam}{3}{29}{2Sa 1:16 Jud 9:24,\allowbreak56,\allowbreak57 1Ki 2:31-\allowbreak34 Ac 28:4 Re 16:6}
\crossref{2Sam}{3}{30}{Pr 28:17 Ac 28:4}
\crossref{2Sam}{3}{31}{2Sa 1:2,\allowbreak11 Ge 37:29,\allowbreak34 Jos 7:6 Jud 11:35 2Ki 19:1}
\crossref{2Sam}{3}{32}{2Sa 1:12; 18:33 1Sa 30:4 Job 31:28 Pr 24:17 Lu 19:41,\allowbreak42}
\crossref{2Sam}{3}{33}{}
\crossref{2Sam}{3}{34}{Job 24:14 Ho 6:9}
\crossref{2Sam}{3}{35}{2Sa 12:17 Jer 16:7 Eze 24:17,\allowbreak22}
\crossref{2Sam}{3}{36}{2Sa 15:6,\allowbreak13 Ps 62:9 Mr 7:37; 15:11-\allowbreak13}
\crossref{2Sam}{3}{37}{}
\crossref{2Sam}{3}{38}{3:12; 2:8 1Sa 14:50,\allowbreak51 Job 32:9}
\crossref{2Sam}{3}{39}{Ex 21:12 2Ch 19:6,\allowbreak7 Ps 75:10; 101:8 Pr 20:8; 25:5}
\crossref{2Sam}{4}{1}{2Sa 17:2 Ezr 4:4 Ne 6:9 Isa 13:7; 35:3 Jer 6:24; 50:43 Zep 3:16}
\crossref{2Sam}{4}{2}{2Sa 3:22 2Ki 5:2; 6:23}
\crossref{2Sam}{4}{3}{1Sa 31:7 Ne 11:33}
\crossref{2Sam}{4}{4}{2Sa 9:3}
\crossref{2Sam}{4}{5}{2Ch 24:25; 25:27; 33:24}
\crossref{2Sam}{4}{6}{2Sa 2:23; 3:27; 20:10}
\crossref{2Sam}{4}{7}{1Sa 17:54; 31:9 2Ki 10:6,\allowbreak7 Mt 14:11 Mr 6:28,\allowbreak29}
\crossref{2Sam}{4}{8}{1Sa 18:11; 19:2-\allowbreak11,\allowbreak15; 20:1; 23:15; 25:29 Ps 63:9,\allowbreak10; 71:24}
\crossref{2Sam}{4}{9}{Ge 48:16 1Ki 1:29 Ps 31:5-\allowbreak7; 34:6,\allowbreak7,\allowbreak17,\allowbreak22; 71:23; 103:4; 106:10}
\crossref{2Sam}{4}{10}{2Sa 1:2-\allowbreak16}
\crossref{2Sam}{4}{11}{1Ki 2:32 Pr 25:26 Hab 1:4,\allowbreak12 1Jo 3:12}
\crossref{2Sam}{4}{12}{2Sa 1:15 Ps 55:23 Mt 7:2}
\crossref{2Sam}{5}{1}{1Ch 11:1-\allowbreak3; 12:23-\allowbreak40}
\crossref{2Sam}{5}{2}{Nu 27:17 1Sa 18:13,\allowbreak16; 25:28 Isa 55:4}
\crossref{2Sam}{5}{3}{Ex 3:16 1Ch 11:3}
\crossref{2Sam}{5}{4}{1Ch 26:31; 29:27}
\crossref{2Sam}{5}{5}{2Sa 2:11 1Ki 2:11 1Ch 3:4}
\crossref{2Sam}{5}{6}{Ge 14:18 Jos 10:3 Jud 1:8 Heb 7:1}
\crossref{2Sam}{5}{7}{Ps 2:6; 9:11; 48:12; 51:18; 87:2; 132:13 Isa 12:6; 59:20 Mic 4:2}
\crossref{2Sam}{5}{8}{Jos 15:16,\allowbreak17 1Sa 17:25}
\crossref{2Sam}{5}{9}{5:7}
\crossref{2Sam}{5}{10}{2Sa 3:1 Job 17:9 Pr 4:18 Isa 9:7 Da 2:44,\allowbreak45 Lu 2:52}
\crossref{2Sam}{5}{11}{1Ki 5:1,\allowbreak2,\allowbreak8,\allowbreak9 1Ch 14:1}
\crossref{2Sam}{5}{12}{2Sa 7:16 1Ch 14:2}
\crossref{2Sam}{5}{13}{Ge 25:5,\allowbreak6 De 17:17 1Ch 3:9; 14:3-\allowbreak7 2Ch 11:18-\allowbreak21; 13:21}
\crossref{2Sam}{5}{14}{1Ch 3:5-\allowbreak9; 14:4}
\crossref{2Sam}{5}{15}{1Ch 3:6; 14:5}
\crossref{2Sam}{5}{16}{1Ch 14:7}
\crossref{2Sam}{5}{17}{1Ch 14:8,\allowbreak9 Ps 2:1-\allowbreak5 Re 11:15-\allowbreak18}
\crossref{2Sam}{5}{18}{2Sa 23:13 Ge 14:5 Jos 15:8 1Ch 11:15 Isa 17:5}
\crossref{2Sam}{5}{19}{2Sa 2:1 1Sa 23:2,\allowbreak4; 30:7,\allowbreak8 Jas 4:15}
\crossref{2Sam}{5}{20}{Isa 28:21}
\crossref{2Sam}{5}{21}{De 7:5,\allowbreak25 1Sa 5:2-\allowbreak6 1Ch 14:11,\allowbreak12 Isa 37:19}
\crossref{2Sam}{5}{22}{1Ki 20:22 1Ch 14:13}
\crossref{2Sam}{5}{23}{5:19}
\crossref{2Sam}{5}{24}{2Ki 7:6}
\crossref{2Sam}{5}{25}{1Ch 14:16}
\crossref{2Sam}{6}{1}{2Sa 5:1 1Ki 8:1 1Ch 13:1-\allowbreak4 Ps 132:1-\allowbreak6}
\crossref{2Sam}{6}{2}{Jos 15:9,\allowbreak10,\allowbreak60}
\crossref{2Sam}{6}{3}{Nu 4:5-\allowbreak12; 7:9 1Sa 6:7}
\crossref{2Sam}{6}{4}{1Sa 7:1,\allowbreak2 1Ch 13:7}
\crossref{2Sam}{6}{5}{1Sa 10:5; 16:16 2Ki 3:15 1Ch 13:8; 15:10-\allowbreak24 Ps 47:5; 68:25-\allowbreak27}
\crossref{2Sam}{6}{6}{1Ch 13:9}
\crossref{2Sam}{6}{7}{Le 10:1-\allowbreak3 1Sa 6:19 1Ch 13:10; 15:2,\allowbreak13 1Co 11:30-\allowbreak32}
\crossref{2Sam}{6}{8}{1Ch 13:11,\allowbreak12 Jon 4:1,\allowbreak9}
\crossref{2Sam}{6}{9}{Nu 17:12,\allowbreak13 1Sa 5:10,\allowbreak11; 6:20 Ps 119:120 Isa 6:5 Lu 5:8,\allowbreak9}
\crossref{2Sam}{6}{10}{1Ch 13:13,\allowbreak14; 15:18; 16:5; 26:4-\allowbreak8}
\crossref{2Sam}{6}{11}{Ge 30:27; 39:5,\allowbreak23 Pr 3:9,\allowbreak10 Mal 3:10}
\crossref{2Sam}{6}{12}{Mt 10:42}
\crossref{2Sam}{6}{13}{Nu 4:15; 7:9 Jos 3:3 1Ch 15:2,\allowbreak15,\allowbreak25,\allowbreak26}
\crossref{2Sam}{6}{14}{Ex 15:20 Jud 11:34; 21:21 Ps 30:11; 149:3; 150:4 Lu 15:25}
\crossref{2Sam}{6}{15}{Ps 132:18}
\crossref{2Sam}{6}{16}{1Ch 15:29}
\crossref{2Sam}{6}{17}{1Ch 15:1; 16:1 2Ch 1:4 Ps 132:8}
\crossref{2Sam}{6}{18}{1Ki 8:55 1Ch 16:2 2Ch 6:3; 30:18,\allowbreak19,\allowbreak27 Ac 3:26}
\crossref{2Sam}{6}{19}{1Ch 16:3 2Ch 30:24; 35:7,\allowbreak8,\allowbreak12,\allowbreak13 Ne 8:10 Eze 45:17 Ac 20:35}
\crossref{2Sam}{6}{20}{6:18 Ge 18:19 Jos 24:15 1Ch 16:43 Ps 30:1}
\crossref{2Sam}{6}{21}{6:14,\allowbreak16 1Co 10:31}
\crossref{2Sam}{6}{22}{Isa 50:6; 51:7 Mt 5:11,\allowbreak12 Ac 5:41,\allowbreak42 Heb 12:2 1Pe 4:14}
\crossref{2Sam}{6}{23}{1Sa 1:6-\allowbreak8 Isa 4:1 Ho 9:11 Lu 1:25}
\crossref{2Sam}{7}{1}{1Ch 17:1-\allowbreak15 Da 4:29,\allowbreak30}
\crossref{2Sam}{7}{2}{2Sa 12:1 1Ch 29:29}
\crossref{2Sam}{7}{3}{2Ki 4:27}
\crossref{2Sam}{7}{4}{Nu 12:6 1Ch 17:3 Am 3:7}
\crossref{2Sam}{7}{5}{1Ki 5:3; 8:16-\allowbreak19 1Ch 17:4; 22:7,\allowbreak8; 23:3-\allowbreak32}
\crossref{2Sam}{7}{6}{Jos 18:1 1Ki 8:16 1Ch 17:5,\allowbreak6}
\crossref{2Sam}{7}{7}{Le 26:11,\allowbreak12}
\crossref{2Sam}{7}{8}{1Sa 16:11,\allowbreak12 1Ch 17:7 Ps 78:70}
\crossref{2Sam}{7}{9}{2Sa 5:10; 8:6,\allowbreak14; 22:30,\allowbreak34-\allowbreak38 1Sa 18:14 1Ch 17:8}
\crossref{2Sam}{7}{10}{1Ch 17:9 Ps 44:4; 80:8 Jer 18:9; 24:6 Eze 37:25-\allowbreak27 Am 9:15}
\crossref{2Sam}{7}{11}{Jud 2:14-\allowbreak16 1Sa 12:9-\allowbreak11 Ps 106:42}
\crossref{2Sam}{7}{12}{1Ki 2:1; 8:20}
\crossref{2Sam}{7}{13}{1Ki 5:5; 6:12; 8:19 1Ch 17:11,\allowbreak12; 22:9,\allowbreak10; 28:6,\allowbreak10 Zec 6:13}
\crossref{2Sam}{7}{14}{1Ch 17:13; 28:6 Ps 89:20-\allowbreak37 Mt 3:17 Heb 1:5}
\crossref{2Sam}{7}{15}{7:14,\allowbreak16 1Sa 19:24 Ps 89:28,\allowbreak34 Isa 55:3 Ac 13:34-\allowbreak37}
\crossref{2Sam}{7}{16}{7:13 Ge 49:10 2Ki 19:34 1Ch 17:13,\allowbreak14 Ps 45:6; 72:5,\allowbreak17-\allowbreak19}
\crossref{2Sam}{7}{17}{1Ch 17:15 Ac 20:20,\allowbreak27 1Co 15:3}
\crossref{2Sam}{7}{18}{1Ch 17:16 Isa 37:14}
\crossref{2Sam}{7}{19}{2Sa 12:8 Nu 16:9,\allowbreak13}
\crossref{2Sam}{7}{20}{Ge 18:19 1Sa 16:7 Ps 139:1 Joh 2:25; 21:17 Heb 4:13 Re 2:23}
\crossref{2Sam}{7}{21}{Nu 23:19 De 9:5 Jos 23:14,\allowbreak15 Ps 115:1; 138:2 Mt 24:35}
\crossref{2Sam}{7}{22}{De 3:24 1Ch 16:25 2Ch 2:5 Ps 48:1; 86:10; 96:4; 135:5; 145:3}
\crossref{2Sam}{7}{23}{De 4:7,\allowbreak8,\allowbreak32-\allowbreak34; 33:29 Ps 147:20 Ro 3:1,\allowbreak2}
\crossref{2Sam}{7}{24}{Ge 17:7 De 26:18}
\crossref{2Sam}{7}{25}{Ge 32:12 Ps 119:49 Jer 11:4,\allowbreak5 Eze 36:37}
\crossref{2Sam}{7}{26}{1Ch 17:23,\allowbreak24; 29:10-\allowbreak13 Ps 72:18,\allowbreak19; 115:1 Mt 6:9 Joh 12:28}
\crossref{2Sam}{7}{27}{Ru 4:4 1Sa 9:15}
\crossref{2Sam}{7}{28}{Nu 23:19 Joh 17:17 Tit 1:2}
\crossref{2Sam}{7}{29}{Nu 6:24-\allowbreak26 1Ch 17:27 Ps 115:12-\allowbreak15}
\crossref{2Sam}{8}{1}{2Sa 7:9; 21:15-\allowbreak22}
\crossref{2Sam}{8}{2}{Nu 24:17 Jud 3:29,\allowbreak30 1Sa 14:47 Ps 60:8; 83:6; 108:9}
\crossref{2Sam}{8}{3}{1Ch 18:3}
\crossref{2Sam}{8}{4}{De 17:16 Jos 11:6,\allowbreak9 Ps 20:7; 33:16,\allowbreak17}
\crossref{2Sam}{8}{5}{1Ki 11:23-\allowbreak25 1Ch 18:5,\allowbreak6 Isa 7:8}
\crossref{2Sam}{8}{6}{8:14; 23:14 1Sa 13:3; 14:1,\allowbreak6,\allowbreak15 2Ch 17:2 Ps 18:34-\allowbreak46}
\crossref{2Sam}{8}{7}{1Ki 10:16,\allowbreak17; 14:26,\allowbreak27 1Ch 18:7 2Ch 9:15,\allowbreak16}
\crossref{2Sam}{8}{8}{}
\crossref{2Sam}{8}{9}{1Ch 18:9}
\crossref{2Sam}{8}{10}{1Ch 18:10}
\crossref{2Sam}{8}{11}{1Ki 7:51 1Ch 18:11; 22:14-\allowbreak16; 26:26-\allowbreak28; 29:2 Mic 4:13}
\crossref{2Sam}{8}{12}{2Sa 10:11,\allowbreak14; 12:26-\allowbreak31 1Ch 18:11}
\crossref{2Sam}{8}{13}{2Sa 7:9 1Ch 18:12 Ps 60:1}
\crossref{2Sam}{8}{14}{Ge 25:23; 27:29,\allowbreak37,\allowbreak40 Nu 24:18 1Ki 22:47 1Ch 18:13 Ps 60:8,\allowbreak9}
\crossref{2Sam}{8}{15}{2Sa 3:12; 5:5}
\crossref{2Sam}{8}{16}{2Sa 19:13; 20:23 1Ch 11:6; 18:15-\allowbreak17}
\crossref{2Sam}{8}{17}{1Ch 6:8,\allowbreak53; 24:3,\allowbreak4}
\crossref{2Sam}{8}{18}{1Ki 1:44; 2:34,\allowbreak35 1Ch 18:17}
\crossref{2Sam}{9}{1}{2Sa 1:26 1Sa 18:1-\allowbreak4; 20:14-\allowbreak17,\allowbreak42; 23:16-\allowbreak18 1Ki 2:7 Pr 27:10}
\crossref{2Sam}{9}{2}{Ge 15:2,\allowbreak3; 24:2; 39:6}
\crossref{2Sam}{9}{3}{2Sa 4:4; 19:26}
\crossref{2Sam}{9}{4}{2Sa 17:27-\allowbreak29}
\crossref{2Sam}{9}{5}{}
\crossref{2Sam}{9}{6}{1Ch 8:34; 9:40}
\crossref{2Sam}{9}{7}{Ge 43:18,\allowbreak23; 50:18-\allowbreak21 1Sa 12:19,\allowbreak20,\allowbreak24 Isa 35:3,\allowbreak4 Mr 5:33,\allowbreak34}
\crossref{2Sam}{9}{8}{2Sa 3:8; 16:9 1Sa 24:14,\allowbreak15; 26:20 Mt 15:26,\allowbreak27}
\crossref{2Sam}{9}{9}{2Sa 16:4; 19:29 1Sa 9:1 Isa 32:8}
\crossref{2Sam}{9}{10}{}
\crossref{2Sam}{9}{11}{2Sa 19:17}
\crossref{2Sam}{9}{12}{1Ch 8:8,\allowbreak34-\allowbreak40; 9:40-\allowbreak44}
\crossref{2Sam}{9}{13}{9:7,\allowbreak10,\allowbreak11}
\crossref{2Sam}{10}{1}{Jud 10:7-\allowbreak9; 11:12-\allowbreak28 1Sa 11:1-\allowbreak3 1Ch 19:1-\allowbreak3}
\crossref{2Sam}{10}{2}{De 23:3-\allowbreak6 Ne 4:3-\allowbreak7; 13:1-\allowbreak3}
\crossref{2Sam}{10}{3}{}
\crossref{2Sam}{10}{4}{Isa 20:4; 47:2,\allowbreak3 Jer 41:5}
\crossref{2Sam}{10}{5}{Jos 6:24-\allowbreak26 1Ki 16:34 1Ch 19:5}
\crossref{2Sam}{10}{6}{Ge 34:30 Ex 5:21 1Sa 13:4; 27:12 1Ch 19:6,\allowbreak7}
\crossref{2Sam}{10}{7}{2Sa 23:8-\allowbreak39 1Ch 19:8-\allowbreak19}
\crossref{2Sam}{10}{8}{10:6 Nu 13:21 Jos 19:28 Jud 1:31}
\crossref{2Sam}{10}{9}{Jos 8:21,\allowbreak22 Jud 20:42,\allowbreak43}
\crossref{2Sam}{10}{10}{10:10}
\crossref{2Sam}{10}{11}{1Ch 19:9-\allowbreak12 Ne 4:20 Lu 22:32 Ro 15:1 Ga 6:2 Php 1:27,\allowbreak28}
\crossref{2Sam}{10}{12}{1Sa 4:9 1Ch 19:13 1Co 16:13}
\crossref{2Sam}{10}{13}{1Ki 20:13-\allowbreak21,\allowbreak28-\allowbreak30 1Ch 19:14,\allowbreak15 2Ch 13:5-\allowbreak16}
\crossref{2Sam}{10}{14}{}
\crossref{2Sam}{10}{15}{Ps 2:1 Isa 8:9,\allowbreak10 Mic 4:11,\allowbreak12 Zec 14:2,\allowbreak3 Re 19:19-\allowbreak21}
\crossref{2Sam}{10}{16}{2Sa 8:3-\allowbreak8 1Ch 18:3,\allowbreak5}
\crossref{2Sam}{10}{17}{1Ch 19:17}
\crossref{2Sam}{10}{18}{2Sa 8:4 Ps 18:38; 46:11}
\crossref{2Sam}{10}{19}{Ge 14:1-\allowbreak5 Jos 11:10 Jud 1:7 1Ki 20:1 Da 2:37}
\crossref{2Sam}{11}{1}{1Ch 20:1 Zec 14:3}
\crossref{2Sam}{11}{2}{2Sa 4:5,\allowbreak7 Pr 19:15; 24:33,\allowbreak34 Mt 26:40,\allowbreak41 1Th 5:6,\allowbreak7 1Pe 4:7}
\crossref{2Sam}{11}{3}{Jer 5:8 Ho 7:6,\allowbreak7 Jas 1:14,\allowbreak15}
\crossref{2Sam}{11}{4}{Ge 39:7 Job 31:9-\allowbreak11 Ps 50:18}
\crossref{2Sam}{11}{5}{De 22:22 Pr 6:34}
\crossref{2Sam}{11}{6}{Ge 4:7; 38:18-\allowbreak23 1Sa 15:30 Job 20:12-\allowbreak14 Pr 28:13 Isa 29:13}
\crossref{2Sam}{11}{7}{Ge 29:6; 37:14 1Sa 17:22}
\crossref{2Sam}{11}{8}{Ps 44:21 Isa 29:15 Lu 12:2 Heb 4:13}
\crossref{2Sam}{11}{9}{Job 5:12-\allowbreak14 Pr 21:30}
\crossref{2Sam}{11}{10}{}
\crossref{2Sam}{11}{11}{2Sa 7:2,\allowbreak6 1Sa 4:4; 14:18}
\crossref{2Sam}{11}{12}{Jer 2:22,\allowbreak23,\allowbreak37}
\crossref{2Sam}{11}{13}{Ge 19:32-\allowbreak35 Ex 32:21 Hab 2:15}
\crossref{2Sam}{11}{14}{}
\crossref{2Sam}{11}{15}{11:17 1Sa 18:17,\allowbreak21,\allowbreak25 Ps 51:4,\allowbreak14 Jer 20:13}
\crossref{2Sam}{11}{16}{11:21; 3:27; 20:9,\allowbreak10 1Sa 22:17-\allowbreak19 1Ki 2:5,\allowbreak31-\allowbreak34; 21:12-\allowbreak14}
\crossref{2Sam}{11}{17}{2Sa 12:9 Ps 51:14}
\crossref{2Sam}{11}{18}{11:18}
\crossref{2Sam}{11}{19}{11:19}
\crossref{2Sam}{11}{20}{}
\crossref{2Sam}{11}{21}{Jud 9:53}
\crossref{2Sam}{11}{22}{11:22}
\crossref{2Sam}{11}{23}{11:23}
\crossref{2Sam}{11}{24}{}
\crossref{2Sam}{11}{25}{Jos 7:8,\allowbreak9 1Sa 6:9 Ec 9:1-\allowbreak3,\allowbreak11,\allowbreak12}
\crossref{2Sam}{11}{26}{2Sa 3:31; 14:2 Ge 27:41}
\crossref{2Sam}{11}{27}{2Sa 3:2-\allowbreak5; 5:13-\allowbreak16; 12:9 De 22:29}
\crossref{2Sam}{12}{1}{2Sa 7:1-\allowbreak5; 24:11-\allowbreak13 1Ki 13:1; 18:1 2Ki 1:3}
\crossref{2Sam}{12}{2}{12:8; 3:2-\allowbreak5; 5:13-\allowbreak16; 15:16 Job 1:3}
\crossref{2Sam}{12}{3}{2Sa 11:3 Pr 5:18,\allowbreak19}
\crossref{2Sam}{12}{4}{Ge 18:2-\allowbreak7 Jas 1:14}
\crossref{2Sam}{12}{5}{Ge 38:24 1Sa 25:21,\allowbreak22 Lu 6:41,\allowbreak42; 9:55 Ro 2:1}
\crossref{2Sam}{12}{6}{Ex 22:1 Pr 6:31 Lu 19:8}
\crossref{2Sam}{12}{7}{1Sa 13:13 1Ki 18:18; 21:19,\allowbreak20 Mt 14:14}
\crossref{2Sam}{12}{8}{12:11 1Ki 2:22}
\crossref{2Sam}{12}{9}{12:10; 11:4,\allowbreak14-\allowbreak17 Ge 9:5,\allowbreak6 Ex 20:13,\allowbreak14 Nu 15:30,\allowbreak31 1Sa 15:19,\allowbreak23}
\crossref{2Sam}{12}{10}{2Sa 13:28,\allowbreak29; 18:14,\allowbreak15,\allowbreak33 1Ki 2:23-\allowbreak25 Am 7:9 Mt 26:52}
\crossref{2Sam}{12}{11}{2Sa 13:1-\allowbreak14,\allowbreak28,\allowbreak29; 15:6,\allowbreak10}
\crossref{2Sam}{12}{12}{2Sa 11:4,\allowbreak8,\allowbreak13,\allowbreak15 Ec 12:14 Lu 12:1,\allowbreak2 1Co 4:5}
\crossref{2Sam}{12}{13}{1Sa 15:20,\allowbreak24 1Ki 13:4; 21:20; 22:8 2Ki 1:9 2Ch 16:10; 24:20-\allowbreak22}
\crossref{2Sam}{12}{14}{Ne 5:9 Ps 74:10 Isa 52:5 Eze 36:20-\allowbreak23 Mt 18:7 Ro 2:24}
\crossref{2Sam}{12}{15}{De 32:39 1Sa 25:38; 26:10 2Ki 15:5 2Ch 13:20 Ps 104:29}
\crossref{2Sam}{12}{16}{12:22 Ps 50:15 Isa 26:16 Joe 2:12-\allowbreak14 Jon 3:9}
\crossref{2Sam}{12}{17}{2Sa 3:35 1Sa 28:23}
\crossref{2Sam}{12}{18}{12:18 Nu 20:15}
\crossref{2Sam}{12}{19}{}
\crossref{2Sam}{12}{20}{Job 1:20; 2:10 Ps 39:9 La 3:39-\allowbreak41}
\crossref{2Sam}{12}{21}{1Co 2:15}
\crossref{2Sam}{12}{22}{Isa 38:1-\allowbreak3,\allowbreak5 Joe 1:14; 2:14 Am 5:15 Jon 1:6; 3:9,\allowbreak10 Jas 4:9,\allowbreak10}
\crossref{2Sam}{12}{23}{Ge 37:35 Job 30:23 Lu 23:43}
\crossref{2Sam}{12}{24}{2Sa 7:12 1Ch 3:5; 22:9,\allowbreak10; 28:5,\allowbreak6; 29:1 Mt 1:6}
\crossref{2Sam}{12}{25}{12:1-\allowbreak14; 7:4 1Ki 1:11,\allowbreak23}
\crossref{2Sam}{12}{26}{2Sa 11:25 1Ch 20:1}
\crossref{2Sam}{12}{27}{2Sa 11:1 De 3:11 Eze 21:20}
\crossref{2Sam}{12}{28}{Joh 7:18}
\crossref{2Sam}{12}{29}{}
\crossref{2Sam}{12}{30}{1Ch 20:2}
\crossref{2Sam}{12}{31}{1Ch 20:3}
\crossref{2Sam}{13}{1}{2Sa 3:2,\allowbreak3 1Ch 3:2}
\crossref{2Sam}{13}{2}{1Ki 21:4 So 5:8 2Co 7:10}
\crossref{2Sam}{13}{3}{Ge 38:1,\allowbreak20 Jud 14:20 Es 5:10,\allowbreak14; 6:13 Pr 19:6}
\crossref{2Sam}{13}{4}{1Ki 21:7 Es 5:13,\allowbreak14 Lu 12:32}
\crossref{2Sam}{13}{5}{2Sa 16:21-\allowbreak23; 17:1-\allowbreak4 Ps 50:18,\allowbreak19 Pr 19:27 Mr 6:24,\allowbreak25 Ac 23:15}
\crossref{2Sam}{13}{6}{Ge 18:6 Mt 13:33}
\crossref{2Sam}{13}{7}{}
\crossref{2Sam}{13}{8}{}
\crossref{2Sam}{13}{9}{Ge 45:1 Jud 3:19 Joh 3:20}
\crossref{2Sam}{13}{10}{}
\crossref{2Sam}{13}{11}{Ge 39:11,\allowbreak12}
\crossref{2Sam}{13}{12}{Ge 34:2 De 22:29}
\crossref{2Sam}{13}{13}{Ge 19:8 Jud 19:24}
\crossref{2Sam}{13}{14}{2Sa 12:11 De 22:25-\allowbreak27 Jud 20:5 Es 7:8}
\crossref{2Sam}{13}{15}{Eze 23:17}
\crossref{2Sam}{13}{16}{13:16}
\crossref{2Sam}{13}{17}{}
\crossref{2Sam}{13}{18}{Ge 37:3,\allowbreak32 Jud 5:30 Ps 45:13,\allowbreak14}
\crossref{2Sam}{13}{19}{2Sa 1:2 Jos 7:6 Job 2:12; 42:6}
\crossref{2Sam}{13}{20}{Pr 26:24 Ro 12:19}
\crossref{2Sam}{13}{21}{}
\crossref{2Sam}{13}{22}{Le 19:17,\allowbreak18 Pr 25:9 Mt 18:15}
\crossref{2Sam}{13}{23}{Ge 38:12,\allowbreak13 1Sa 25:2,\allowbreak4,\allowbreak36 2Ki 3:4 2Ch 26:10}
\crossref{2Sam}{13}{24}{2Sa 11:8-\allowbreak15 Ps 12:2; 55:21 Jer 41:6,\allowbreak7}
\crossref{2Sam}{13}{25}{Ge 19:2,\allowbreak3 Jud 19:7-\allowbreak10 Lu 14:23; 24:29 Ac 16:15}
\crossref{2Sam}{13}{26}{}
\crossref{2Sam}{13}{27}{Pr 26:24-\allowbreak26}
\crossref{2Sam}{13}{28}{2Sa 11:15 Ex 1:16,\allowbreak17 1Sa 22:17,\allowbreak18 Ac 5:29}
\crossref{2Sam}{13}{29}{1Sa 22:18,\allowbreak19 1Ki 21:11-\allowbreak13 2Ki 1:9-\allowbreak12 Pr 29:12 Mic 7:3}
\crossref{2Sam}{13}{30}{}
\crossref{2Sam}{13}{31}{2Sa 12:16 Ge 37:29,\allowbreak34 Jos 7:6 Job 1:20}
\crossref{2Sam}{13}{32}{13:3-\allowbreak5}
\crossref{2Sam}{13}{33}{2Sa 19:19}
\crossref{2Sam}{13}{34}{13:38 Ge 4:8-\allowbreak14 Pr 28:17 Am 5:19}
\crossref{2Sam}{13}{35}{}
\crossref{2Sam}{13}{36}{13:15}
\crossref{2Sam}{13}{37}{2Sa 3:3 1Ch 3:2}
\crossref{2Sam}{13}{38}{}
\crossref{2Sam}{13}{39}{Ge 31:30 De 28:32 Php 2:26}
\crossref{2Sam}{14}{1}{2Sa 2:18 1Ch 2:16}
\crossref{2Sam}{14}{2}{2Sa 11:26 Ru 3:3 Ps 104:15 Ec 9:8 Mt 6:17}
\crossref{2Sam}{14}{3}{14:19 Ex 4:15 Nu 23:5 De 18:18 Isa 51:16; 59:21 Jer 1:9}
\crossref{2Sam}{14}{4}{2Sa 1:2 1Sa 20:41; 25:23}
\crossref{2Sam}{14}{5}{}
\crossref{2Sam}{14}{6}{Ge 4:8 Ex 2:13 De 22:26,\allowbreak27}
\crossref{2Sam}{14}{7}{Ge 4:14 Nu 35:19 De 19:12}
\crossref{2Sam}{14}{8}{2Sa 12:5,\allowbreak6; 16:4 Job 29:16 Pr 18:13 Isa 11:3,\allowbreak4}
\crossref{2Sam}{14}{9}{Ge 27:13 1Sa 25:24 Mt 27:25}
\crossref{2Sam}{14}{10}{}
\crossref{2Sam}{14}{11}{Ge 14:22; 24:2,\allowbreak3; 31:50 1Sa 20:42}
\crossref{2Sam}{14}{12}{1Sa 25:24}
\crossref{2Sam}{14}{13}{2Sa 12:7 1Ki 20:40-\allowbreak42 Lu 7:42-\allowbreak44}
\crossref{2Sam}{14}{14}{2Sa 11:25 Job 30:23; 34:15 Ps 90:3,\allowbreak10 Ec 3:19,\allowbreak20; 9:5 Heb 9:27}
\crossref{2Sam}{14}{15}{14:15}
\crossref{2Sam}{14}{16}{}
\crossref{2Sam}{14}{17}{1Ki 3:9,\allowbreak28 Job 6:30 1Co 2:14,\allowbreak15}
\crossref{2Sam}{14}{18}{1Sa 3:17,\allowbreak18 Jer 38:14,\allowbreak25}
\crossref{2Sam}{14}{19}{2Sa 3:27,\allowbreak29,\allowbreak34; 11:14,\allowbreak15 1Ki 2:5,\allowbreak6}
\crossref{2Sam}{14}{20}{2Sa 5:23}
\crossref{2Sam}{14}{21}{14:11 1Sa 14:39 Mr 6:26}
\crossref{2Sam}{14}{22}{2Sa 19:39 Ne 11:2 Job 29:11; 31:20 Pr 31:28}
\crossref{2Sam}{14}{23}{2Sa 3:3; 13:37}
\crossref{2Sam}{14}{24}{14:28; 3:13 Ge 43:3 Ex 10:28 Re 22:4}
\crossref{2Sam}{14}{25}{De 28:35 Job 2:7 Isa 1:6 Eph 5:27}
\crossref{2Sam}{14}{26}{2Sa 18:9 Isa 3:24 1Co 11:14}
\crossref{2Sam}{14}{27}{2Sa 18:18 Job 18:16-\allowbreak19 Isa 14:22 Jer 22:30}
\crossref{2Sam}{14}{28}{}
\crossref{2Sam}{14}{29}{14:30,\allowbreak31 Es 1:12 Mt 22:3}
\crossref{2Sam}{14}{30}{2Sa 13:28,\allowbreak29 Jud 15:4,\allowbreak5}
\crossref{2Sam}{14}{31}{}
\crossref{2Sam}{14}{32}{Ex 14:12; 16:3; 17:3}
\crossref{2Sam}{14}{33}{Ge 27:26; 33:4; 45:15 Lu 15:20}
\crossref{2Sam}{15}{1}{2Sa 12:11 De 17:16 1Sa 8:11 1Ki 1:5,\allowbreak33; 10:26-\allowbreak29 Ps 20:7 Pr 11:2}
\crossref{2Sam}{15}{2}{Job 24:14 Pr 4:16 Mt 27:1}
\crossref{2Sam}{15}{3}{Nu 16:3,\allowbreak13,\allowbreak14 Ps 12:2 Da 11:21 2Pe 2:10}
\crossref{2Sam}{15}{4}{Jud 9:1-\allowbreak5,\allowbreak29 Pr 25:6 Lu 14:8-\allowbreak11}
\crossref{2Sam}{15}{5}{Ps 10:9,\allowbreak10; 55:21 Pr 26:25}
\crossref{2Sam}{15}{6}{Pr 11:9 Ro 16:18 2Pe 2:3}
\crossref{2Sam}{15}{7}{2Sa 13:24-\allowbreak27}
\crossref{2Sam}{15}{8}{Ge 28:20,\allowbreak21 1Sa 1:11; 16:2 Ps 56:12 Ec 5:4}
\crossref{2Sam}{15}{9}{}
\crossref{2Sam}{15}{10}{2Sa 13:28; 14:30}
\crossref{2Sam}{15}{11}{1Sa 9:13; 16:3-\allowbreak5}
\crossref{2Sam}{15}{12}{15:31; 16:20-\allowbreak23; 17:14,\allowbreak23}
\crossref{2Sam}{15}{13}{15:6; 3:36 Jud 9:3 Ps 62:9 Mt 21:9; 27:22}
\crossref{2Sam}{15}{14}{2Sa 19:9 Ps 3:1}
\crossref{2Sam}{15}{15}{Pr 18:24 Lu 22:28,\allowbreak29 Joh 6:66-\allowbreak69; 15:14}
\crossref{2Sam}{15}{16}{Ps 3:1}
\crossref{2Sam}{15}{17}{Ps 3:1}
\crossref{2Sam}{15}{18}{2Sa 8:18; 20:7,\allowbreak23 1Sa 30:14 1Ki 1:38 1Ch 18:17}
\crossref{2Sam}{15}{19}{2Sa 18:2 Ru 1:11-\allowbreak13}
\crossref{2Sam}{15}{20}{Ps 56:8; 59:15 Am 8:12 Heb 11:37,\allowbreak38}
\crossref{2Sam}{15}{21}{1Sa 20:3; 25:26 2Ki 2:2,\allowbreak4,\allowbreak6; 4:30}
\crossref{2Sam}{15}{22}{}
\crossref{2Sam}{15}{23}{Ro 12:15}
\crossref{2Sam}{15}{24}{15:27,\allowbreak35; 8:17; 20:25 1Ki 1:8; 2:35; 4:2-\allowbreak4 1Ch 6:8-\allowbreak12 Eze 48:11}
\crossref{2Sam}{15}{25}{2Sa 12:10,\allowbreak11 1Sa 4:3-\allowbreak11 Jer 7:4}
\crossref{2Sam}{15}{26}{2Sa 22:20 Nu 14:8 1Ki 10:9 2Ch 9:8 Isa 42:1; 62:4 Jer 22:28; 32:41}
\crossref{2Sam}{15}{27}{2Sa 24:11 1Sa 9:9 1Ch 25:5}
\crossref{2Sam}{15}{28}{15:23; 16:2; 17:1,\allowbreak16}
\crossref{2Sam}{15}{29}{}
\crossref{2Sam}{15}{30}{Zec 14:4 Lu 19:29,\allowbreak37; 21:37; 22:39 Ac 1:12}
\crossref{2Sam}{15}{31}{15:12 Ps 3:1; 41:9; 55:12,\allowbreak14 Mt 26:14,\allowbreak15 Joh 13:18}
\crossref{2Sam}{15}{32}{15:30 1Ki 11:7 Lu 19:29}
\crossref{2Sam}{15}{33}{2Sa 19:35}
\crossref{2Sam}{15}{34}{15:20 Jos 8:2 Mt 10:16}
\crossref{2Sam}{15}{35}{2Sa 17:15,\allowbreak16}
\crossref{2Sam}{15}{36}{15:27; 17:17; 18:19-\allowbreak33}
\crossref{2Sam}{15}{37}{2Sa 16:16 1Ch 27:33}
\crossref{2Sam}{16}{1}{2Sa 15:30,\allowbreak32}
\crossref{2Sam}{16}{2}{Ge 21:29; 33:8 Eze 37:18}
\crossref{2Sam}{16}{3}{2Sa 9:9,\allowbreak10 Ps 88:18 Mic 7:5}
\crossref{2Sam}{16}{4}{2Sa 14:10,\allowbreak11 Ex 23:8 De 19:15 Pr 18:13,\allowbreak17; 19:2}
\crossref{2Sam}{16}{5}{2Sa 19:16-\allowbreak18 1Ki 2:8,\allowbreak9,\allowbreak36-\allowbreak44,\allowbreak45,\allowbreak46}
\crossref{2Sam}{16}{6}{}
\crossref{2Sam}{16}{7}{2Sa 3:37; 11:15-\allowbreak17; 12:9 Ps 5:6; 51:14}
\crossref{2Sam}{16}{8}{Jud 9:24,\allowbreak56,\allowbreak57 1Ki 2:32,\allowbreak33 Ac 28:4,\allowbreak5 Re 16:6}
\crossref{2Sam}{16}{9}{2Sa 3:30 1Sa 26:6-\allowbreak8}
\crossref{2Sam}{16}{10}{2Sa 3:39; 19:22 1Ki 2:5 Mt 16:23 Lu 9:54-\allowbreak56 1Pe 2:23}
\crossref{2Sam}{16}{11}{2Sa 12:11,\allowbreak12}
\crossref{2Sam}{16}{12}{Ge 29:32,\allowbreak33 Ex 2:24,\allowbreak25; 3:7,\allowbreak8 1Sa 1:11 Ps 25:18}
\crossref{2Sam}{16}{13}{16:5,\allowbreak6}
\crossref{2Sam}{16}{14}{16:5}
\crossref{2Sam}{16}{15}{2Sa 15:37}
\crossref{2Sam}{16}{16}{1Sa 10:24 1Ki 1:25,\allowbreak34 2Ki 11:12 Da 2:4; 5:10; 6:6,\allowbreak21 Mt 21:9}
\crossref{2Sam}{16}{17}{De 32:6}
\crossref{2Sam}{16}{18}{2Sa 5:1-\allowbreak3 1Sa 16:13}
\crossref{2Sam}{16}{19}{2Sa 15:34 1Sa 28:2; 29:8 Ps 55:21 Ga 2:13}
\crossref{2Sam}{16}{20}{Ex 1:10 Ps 2:2; 37:12,\allowbreak13 Pr 21:30 Isa 8:10; 29:15 Mt 27:1}
\crossref{2Sam}{16}{21}{Ge 6:4; 38:16}
\crossref{2Sam}{16}{22}{2Sa 11:2}
\crossref{2Sam}{16}{23}{Nu 27:21 1Sa 30:8 Ps 28:2 1Pe 4:11}
\crossref{2Sam}{17}{1}{Pr 1:16; 4:16 Isa 59:7,\allowbreak8}
\crossref{2Sam}{17}{2}{2Sa 16:14 De 25:18}
\crossref{2Sam}{17}{3}{2Sa 3:21}
\crossref{2Sam}{17}{4}{1Sa 18:20,\allowbreak21; 23:21 Es 5:14 Ro 1:32}
\crossref{2Sam}{17}{5}{2Sa 15:32-\allowbreak37; 16:16-\allowbreak19}
\crossref{2Sam}{17}{6}{17:6}
\crossref{2Sam}{17}{7}{Pr 31:8}
\crossref{2Sam}{17}{8}{2Sa 15:18; 21:18-\allowbreak22; 23:8,\allowbreak9,\allowbreak16,\allowbreak18,\allowbreak20-\allowbreak22 1Sa 16:18; 17:34-\allowbreak36,\allowbreak50}
\crossref{2Sam}{17}{9}{Jud 20:33 1Sa 22:1; 24:3}
\crossref{2Sam}{17}{10}{2Sa 1:23; 23:20 Ge 49:9 Nu 24:8,\allowbreak9 Pr 28:1}
\crossref{2Sam}{17}{11}{2Sa 24:2 Jud 20:1}
\crossref{2Sam}{17}{12}{1Sa 23:23}
\crossref{2Sam}{17}{13}{Mt 24:2}
\crossref{2Sam}{17}{14}{2Sa 15:31 Ge 32:28 Ex 9:16 De 2:30 2Ch 25:16,\allowbreak20}
\crossref{2Sam}{17}{15}{2Sa 15:35}
\crossref{2Sam}{17}{16}{2Sa 15:28}
\crossref{2Sam}{17}{17}{2Sa 15:27,\allowbreak36}
\crossref{2Sam}{17}{18}{2Sa 3:16; 16:5; 19:16}
\crossref{2Sam}{17}{19}{Jos 2:4-\allowbreak6,\allowbreak5-\allowbreak24}
\crossref{2Sam}{17}{20}{2Sa 15:34 Ex 1:19 Jos 2:4,\allowbreak5 1Sa 19:14-\allowbreak17; 21:2; 27:11,\allowbreak12}
\crossref{2Sam}{17}{21}{17:15,\allowbreak16}
\crossref{2Sam}{17}{22}{17:24 Pr 27:12 Mt 10:16}
\crossref{2Sam}{17}{23}{Pr 16:18; 19:3}
\crossref{2Sam}{17}{24}{2Sa 2:8 Ge 32:2 Jos 13:26}
\crossref{2Sam}{17}{25}{2Sa 19:13; 20:4,\allowbreak9-\allowbreak12}
\crossref{2Sam}{17}{26}{Nu 32:1-\allowbreak42 De 3:15 Jos 17:1}
\crossref{2Sam}{17}{27}{2Sa 10:1,\allowbreak2; 12:29,\allowbreak30 1Sa 11:1}
\crossref{2Sam}{17}{28}{}
\crossref{2Sam}{17}{29}{1Sa 17:18}
\crossref{2Sam}{18}{1}{Ex 17:9 Jos 8:10}
\crossref{2Sam}{18}{2}{Jud 7:16,\allowbreak19; 9:43}
\crossref{2Sam}{18}{3}{2Sa 21:17}
\crossref{2Sam}{18}{4}{18:24 Isa 28:6}
\crossref{2Sam}{18}{5}{2Sa 16:11; 17:1-\allowbreak4,\allowbreak14 De 21:18-\allowbreak21 Ps 103:13 Lu 23:34}
\crossref{2Sam}{18}{6}{}
\crossref{2Sam}{18}{7}{2Sa 2:17; 15:6; 19:41-\allowbreak43}
\crossref{2Sam}{18}{8}{}
\crossref{2Sam}{18}{9}{De 21:23; 27:16,\allowbreak20 Job 18:8-\allowbreak10; 31:3 Ps 63:9,\allowbreak10 Pr 20:20; 30:17}
\crossref{2Sam}{18}{10}{18:10}
\crossref{2Sam}{18}{11}{}
\crossref{2Sam}{18}{12}{18:5}
\crossref{2Sam}{18}{13}{2Sa 1:15,\allowbreak16; 4:10-\allowbreak12}
\crossref{2Sam}{18}{14}{18:5 Jud 4:21; 5:26,\allowbreak31 Ps 45:5 1Th 5:3}
\crossref{2Sam}{18}{15}{}
\crossref{2Sam}{18}{16}{2Sa 2:28; 20:22 Nu 10:2-\allowbreak10 1Co 14:8}
\crossref{2Sam}{18}{17}{}
\crossref{2Sam}{18}{18}{1Sa 15:12}
\crossref{2Sam}{18}{19}{18:23,\allowbreak27-\allowbreak29; 15:36; 17:17}
\crossref{2Sam}{18}{20}{2Sa 17:16-\allowbreak21}
\crossref{2Sam}{18}{21}{}
\crossref{2Sam}{18}{22}{Ro 1:28 Eph 5:4}
\crossref{2Sam}{18}{23}{Joh 20:4}
\crossref{2Sam}{18}{24}{18:4 1Sa 4:13}
\crossref{2Sam}{18}{25}{18:25}
\crossref{2Sam}{18}{26}{}
\crossref{2Sam}{18}{27}{2Ki 9:20}
\crossref{2Sam}{18}{28}{2Sa 22:27 Ge 14:20; 24:27 2Ch 20:26 Ps 115:1; 124:6; 144:1,\allowbreak2}
\crossref{2Sam}{18}{29}{}
\crossref{2Sam}{18}{30}{}
\crossref{2Sam}{18}{31}{18:19,\allowbreak28; 22:48,\allowbreak49 De 32:35,\allowbreak36 Ps 58:10; 94:1-\allowbreak4; 124:2,\allowbreak3 Lu 18:7,\allowbreak8}
\crossref{2Sam}{18}{32}{}
\crossref{2Sam}{18}{33}{2Sa 19:4}
\crossref{2Sam}{19}{1}{2Sa 18:5,\allowbreak12,\allowbreak14,\allowbreak20,\allowbreak33 Pr 17:25}
\crossref{2Sam}{19}{2}{Pr 16:15; 19:12}
\crossref{2Sam}{19}{3}{19:32; 17:24}
\crossref{2Sam}{19}{4}{2Sa 15:30}
\crossref{2Sam}{19}{5}{Ne 9:27 Ps 3:8; 18:47,\allowbreak48}
\crossref{2Sam}{19}{6}{}
\crossref{2Sam}{19}{7}{Ge 34:3 Pr 19:15 Isa 40:1 Ho 2:14}
\crossref{2Sam}{19}{8}{19:3; 18:6-\allowbreak8 1Ki 22:36 2Ki 14:12}
\crossref{2Sam}{19}{9}{Ge 3:12,\allowbreak13 Ex 32:24 Jas 3:14-\allowbreak16}
\crossref{2Sam}{19}{10}{2Sa 15:12,\allowbreak13 Ho 8:4}
\crossref{2Sam}{19}{11}{2Sa 15:29,\allowbreak35,\allowbreak36 1Ki 2:25,\allowbreak26,\allowbreak35}
\crossref{2Sam}{19}{12}{2Sa 5:1 Ge 2:23 Jud 9:2 Eph 5:30}
\crossref{2Sam}{19}{13}{2Sa 17:25 1Ch 2:16,\allowbreak17; 12:18}
\crossref{2Sam}{19}{14}{Jud 20:1 Ps 110:2,\allowbreak3 Ac 4:32}
\crossref{2Sam}{19}{15}{Jos 5:9 1Sa 11:14,\allowbreak15}
\crossref{2Sam}{19}{16}{Job 2:4 Pr 6:4,\allowbreak5 Mt 5:25}
\crossref{2Sam}{19}{17}{19:26,\allowbreak27; 9:2,\allowbreak10; 16:1-\allowbreak4}
\crossref{2Sam}{19}{18}{Ps 66:3; 81:15 Re 3:9}
\crossref{2Sam}{19}{19}{Ec 10:4}
\crossref{2Sam}{19}{20}{Ps 78:34-\allowbreak37 Jer 22:23 Ho 5:15}
\crossref{2Sam}{19}{21}{Ex 22:28 1Ki 21:10,\allowbreak11}
\crossref{2Sam}{19}{22}{2Sa 3:39; 16:10 1Sa 26:8 Mt 8:29}
\crossref{2Sam}{19}{23}{1Ki 2:8,\allowbreak9,\allowbreak37,\allowbreak46}
\crossref{2Sam}{19}{24}{2Sa 9:6; 16:3}
\crossref{2Sam}{19}{25}{2Sa 16:17}
\crossref{2Sam}{19}{26}{2Sa 16:2,\allowbreak3}
\crossref{2Sam}{19}{27}{2Sa 16:3 Ex 20:16 Ps 15:3; 101:5 Jer 9:4}
\crossref{2Sam}{19}{28}{Ge 32:10}
\crossref{2Sam}{19}{29}{Job 19:16,\allowbreak17 Pr 18:13 Ac 18:15}
\crossref{2Sam}{19}{30}{2Sa 1:26 Ac 20:24 Php 1:20}
\crossref{2Sam}{19}{31}{1Ki 2:7 Ezr 2:61 Ne 7:63}
\crossref{2Sam}{19}{32}{Ge 5:27; 9:29; 25:7; 47:28; 50:26 De 34:7 Ps 90:3-\allowbreak10 Pr 16:31}
\crossref{2Sam}{19}{33}{2Sa 9:11 Mt 25:34-\allowbreak40 Lu 22:28-\allowbreak30 2Th 1:7}
\crossref{2Sam}{19}{34}{}
\crossref{2Sam}{19}{35}{Job 6:30; 12:11 Heb 5:14 1Pe 2:3}
\crossref{2Sam}{19}{36}{Lu 6:38}
\crossref{2Sam}{19}{37}{Ge 48:21 Jos 23:14 Lu 2:29,\allowbreak30 2Ti 4:6 2Pe 1:14}
\crossref{2Sam}{19}{38}{19:38}
\crossref{2Sam}{19}{39}{Ge 31:55; 45:15 Ru 1:14 1Ki 19:20 Ac 20:37 1Th 5:26}
\crossref{2Sam}{19}{40}{19:11-\allowbreak15 Ge 49:10 Mt 21:9}
\crossref{2Sam}{19}{41}{Jud 8:1; 12:1 Joh 7:5,\allowbreak6}
\crossref{2Sam}{19}{42}{19:12; 5:1 1Ch 2:3-\allowbreak17}
\crossref{2Sam}{19}{43}{2Sa 20:1,\allowbreak6 1Ki 12:16}
\crossref{2Sam}{20}{1}{2Sa 19:41-\allowbreak43 Ps 34:19}
\crossref{2Sam}{20}{2}{2Sa 19:41 Ps 62:9; 118:8-\allowbreak10 Pr 17:14}
\crossref{2Sam}{20}{3}{2Sa 15:16; 16:21,\allowbreak22}
\crossref{2Sam}{20}{4}{2Sa 17:25; 19:13 1Ch 2:17}
\crossref{2Sam}{20}{5}{2Sa 19:13}
\crossref{2Sam}{20}{6}{2Sa 2:18; 3:30,\allowbreak39; 10:9,\allowbreak10,\allowbreak14; 18:2,\allowbreak12; 21:17; 23:18 1Sa 26:6}
\crossref{2Sam}{20}{7}{20:23; 8:16,\allowbreak18; 15:18; 23:22,\allowbreak23 1Ki 1:38,\allowbreak44}
\crossref{2Sam}{20}{8}{2Sa 2:13; 3:30}
\crossref{2Sam}{20}{9}{Ps 55:21 Pr 26:24-\allowbreak26 Mic 7:2}
\crossref{2Sam}{20}{10}{20:9 Jud 3:21 1Ch 12:2}
\crossref{2Sam}{20}{11}{20:6,\allowbreak7,\allowbreak13,\allowbreak21}
\crossref{2Sam}{20}{12}{2Sa 17:25 Ps 9:16; 55:23 Pr 24:21,\allowbreak22}
\crossref{2Sam}{20}{13}{20:12,\allowbreak13 Nu 20:19 Jud 21:19 1Sa 6:12 2Ki 18:17 Pr 16:17 Isa 7:3}
\crossref{2Sam}{20}{14}{Jos 18:25}
\crossref{2Sam}{20}{15}{2Ki 19:32 Jer 32:24; 33:4 Lu 19:43}
\crossref{2Sam}{20}{16}{2Sa 14:2 1Sa 25:3,\allowbreak32,\allowbreak33 Ec 9:14-\allowbreak18}
\crossref{2Sam}{20}{17}{2Sa 14:12 1Sa 25:24}
\crossref{2Sam}{20}{18}{}
\crossref{2Sam}{20}{19}{Ge 18:23 Ro 13:3,\allowbreak4 1Ti 2:2}
\crossref{2Sam}{20}{20}{2Sa 23:17 Job 21:16; 22:18}
\crossref{2Sam}{20}{21}{20:1 Jud 2:9; 7:24 2Ki 5:22 Jer 4:15; 50:19}
\crossref{2Sam}{20}{22}{Ec 7:19; 9:14-\allowbreak18}
\crossref{2Sam}{20}{23}{2Sa 8:16-\allowbreak18 1Ch 18:15-\allowbreak17}
\crossref{2Sam}{20}{24}{1Ki 4:6; 12:18}
\crossref{2Sam}{20}{25}{2Sa 8:17 1Ki 4:4 1Ch 18:16}
\crossref{2Sam}{20}{26}{2Sa 23:38 1Ch 11:40}
\crossref{2Sam}{21}{1}{Ge 12:10; 26:1; 41:57; 42:1; 43:1 Le 26:19,\allowbreak20,\allowbreak26 1Ki 17:1; 18:2}
\crossref{2Sam}{21}{2}{Jos 9:3-\allowbreak21}
\crossref{2Sam}{21}{3}{Ex 32:30 Le 1:4 1Sa 2:25 Mic 6:6,\allowbreak7 Heb 9:22; 10:4-\allowbreak12}
\crossref{2Sam}{21}{4}{}
\crossref{2Sam}{21}{5}{21:1 Es 9:24,\allowbreak25 Mt 7:2}
\crossref{2Sam}{21}{6}{2Sa 17:23; 18:10 Ge 40:19,\allowbreak22 Nu 25:4,\allowbreak5 De 21:22 Jos 8:29; 10:26}
\crossref{2Sam}{21}{7}{2Sa 4:4; 9:10; 16:4; 19:25}
\crossref{2Sam}{21}{8}{2Sa 3:7}
\crossref{2Sam}{21}{9}{21:6; 6:17,\allowbreak21 Ex 20:5 Nu 35:31-\allowbreak34 De 21:1-\allowbreak9 1Sa 15:33 2Ki 24:3,\allowbreak4}
\crossref{2Sam}{21}{10}{21:8; 3:7}
\crossref{2Sam}{21}{11}{2Sa 2:4 Ru 2:11,\allowbreak12}
\crossref{2Sam}{21}{12}{2Sa 2:5-\allowbreak7 1Sa 31:11-\allowbreak13}
\crossref{2Sam}{21}{13}{}
\crossref{2Sam}{21}{14}{2Sa 3:32; 4:12}
\crossref{2Sam}{21}{15}{2Sa 5:17,\allowbreak22 1Ch 20:4}
\crossref{2Sam}{21}{16}{Ge 6:4 Nu 13:32,\allowbreak33 De 1:28; 2:10,\allowbreak21; 3:11; 9:2 1Sa 17:4,\allowbreak5}
\crossref{2Sam}{21}{17}{2Sa 20:6-\allowbreak10}
\crossref{2Sam}{21}{18}{1Ch 11:29; 20:4}
\crossref{2Sam}{21}{19}{1Ch 20:5}
\crossref{2Sam}{21}{20}{1Ch 20:6}
\crossref{2Sam}{21}{21}{1Sa 17:10,\allowbreak25,\allowbreak26,\allowbreak36 2Ki 19:13}
\crossref{2Sam}{21}{22}{1Ch 20:8}
\crossref{2Sam}{22}{1}{Ps 50:14; 103:1-\allowbreak6; 116:1-\allowbreak19}
\crossref{2Sam}{22}{2}{De 32:4 1Sa 2:2 Ps 18:2-\allowbreak50; 31:3; 42:9; 71:3; 91:2; 144:2 Mt 16:18}
\crossref{2Sam}{22}{3}{Heb 2:13}
\crossref{2Sam}{22}{4}{Ps 116:2,\allowbreak4,\allowbreak13,\allowbreak17}
\crossref{2Sam}{22}{5}{1Th 5:3}
\crossref{2Sam}{22}{6}{Job 36:8 Ps 18:5; 116:3; 140:5 Pr 5:22 Jon 2:2 Ac 2:24}
\crossref{2Sam}{22}{7}{Ps 116:4; 120:1 Mt 26:38,\allowbreak39 Lu 22:44 Heb 5:7}
\crossref{2Sam}{22}{8}{Jud 5:4 Ps 18:7; 77:18; 97:4 Hab 3:6-\allowbreak11 Mt 27:51; 28:2 Ac 4:31}
\crossref{2Sam}{22}{9}{22:16 Ex 15:7,\allowbreak8; 19:18; 24:17 De 32:22 Job 4:9; 41:20,\allowbreak21 Ps 18:8,\allowbreak15}
\crossref{2Sam}{22}{10}{Ps 144:5 Isa 64:1-\allowbreak3}
\crossref{2Sam}{22}{11}{Ge 3:24 Ex 25:19 1Sa 4:4 Ps 18:10; 68:17; 80:1; 99:1 Eze 9:3}
\crossref{2Sam}{22}{12}{22:10 Ps 18:11,\allowbreak12; 27:5; 97:2}
\crossref{2Sam}{22}{13}{22:9}
\crossref{2Sam}{22}{14}{Ex 19:6 Jud 5:20 1Sa 2:10; 7:10; 12:17,\allowbreak18 Job 37:2-\allowbreak5; 40:9}
\crossref{2Sam}{22}{15}{De 32:23 Jos 10:10 Ps 7:12,\allowbreak13; 18:14; 45:5; 144:6,\allowbreak7 Hab 3:11}
\crossref{2Sam}{22}{16}{Ex 14:21-\allowbreak27; 15:8-\allowbreak10 Ps 18:15-\allowbreak17; 114:3-\allowbreak7}
\crossref{2Sam}{22}{17}{Ps 18:16; 144:7}
\crossref{2Sam}{22}{18}{22:1 Ps 3:7; 56:9 2Co 1:10 2Ti 4:17}
\crossref{2Sam}{22}{19}{2Sa 15:10-\allowbreak13 1Sa 19:11-\allowbreak17; 23:26,\allowbreak27 Ps 18:18,\allowbreak19; 118:10-\allowbreak13}
\crossref{2Sam}{22}{20}{Ge 26:22 1Ch 4:10 Ps 31:8; 118:5 Ho 4:16}
\crossref{2Sam}{22}{21}{22:25 1Sa 26:23 1Ki 8:32 Ps 7:3,\allowbreak4,\allowbreak8; 18:20-\allowbreak25; 19:11 1Co 15:58}
\crossref{2Sam}{22}{22}{Nu 16:15 1Sa 12:3 Job 23:10-\allowbreak12 2Co 1:12}
\crossref{2Sam}{22}{23}{Ps 119:6,\allowbreak86,\allowbreak128 Lu 1:6 Joh 15:14}
\crossref{2Sam}{22}{24}{Ge 6:9; 17:1 Job 1:1 Ps 51:6; 84:11 Joh 1:47 2Co 5:11}
\crossref{2Sam}{22}{25}{22:21 Isa 3:10 Ro 2:7,\allowbreak8 2Co 5:10}
\crossref{2Sam}{22}{26}{Mt 5:7 Jas 2:13}
\crossref{2Sam}{22}{27}{Mt 5:8}
\crossref{2Sam}{22}{28}{Ex 3:7,\allowbreak8 Ps 12:5; 72:12,\allowbreak13; 140:12 Isa 61:1-\allowbreak3; 63:9 Mt 5:3}
\crossref{2Sam}{22}{29}{Job 29:3 Ps 27:1; 84:11 Joh 8:12 Re 21:23}
\crossref{2Sam}{22}{30}{Ps 18:29; 118:10-\allowbreak12 Ro 8:37 Php 4:13}
\crossref{2Sam}{22}{31}{De 32:4 Da 4:37 Mt 5:48 Re 15:3}
\crossref{2Sam}{22}{32}{De 32:31,\allowbreak39 1Sa 2:2 Isa 44:6,\allowbreak8; 45:5,\allowbreak6,\allowbreak21 Jer 10:6,\allowbreak7,\allowbreak16}
\crossref{2Sam}{22}{33}{Ex 15:2 Ps 18:32; 27:1; 28:7,\allowbreak8; 31:4; 46:1 Isa 41:10 Zec 10:12}
\crossref{2Sam}{22}{34}{2Sa 2:18 De 33:25 Hab 3:19}
\crossref{2Sam}{22}{35}{Ps 18:33,\allowbreak34; 144:1}
\crossref{2Sam}{22}{36}{Ge 15:1 Ps 84:11 Eph 6:16}
\crossref{2Sam}{22}{37}{Ps 4:1; 18:36 Pr 4:12}
\crossref{2Sam}{22}{38}{2Sa 5:18-\allowbreak25; 8:1,\allowbreak2,\allowbreak13,\allowbreak14; 10:14 Ps 21:8,\allowbreak9 Ro 8:37}
\crossref{2Sam}{22}{39}{Ps 18:37; 110:1,\allowbreak5,\allowbreak6; 118:10-\allowbreak12 Mal 4:1,\allowbreak3}
\crossref{2Sam}{22}{40}{1Sa 17:49-\allowbreak51; 23:5 Ps 18:32,\allowbreak39 Isa 45:5 Col 1:11}
\crossref{2Sam}{22}{41}{Ge 49:8 Ex 23:27 Jos 10:24 Ps 18:40,\allowbreak41}
\crossref{2Sam}{22}{42}{1Sa 28:6 Job 27:9 Pr 1:28 Isa 1:15 Eze 20:3 Mic 3:4}
\crossref{2Sam}{22}{43}{2Ki 13:7 Ps 35:5 Da 2:35 Mal 4:1}
\crossref{2Sam}{22}{44}{2Sa 3:1; 5:1; 18:6-\allowbreak8; 19:9,\allowbreak14; 20:1,\allowbreak2,\allowbreak22 Ps 2:1-\allowbreak6; 18:43 Ac 4:25-\allowbreak28}
\crossref{2Sam}{22}{45}{Isa 56:3,\allowbreak6}
\crossref{2Sam}{22}{46}{Isa 64:6 Jas 1:11}
\crossref{2Sam}{22}{47}{De 32:39,\allowbreak40 Job 19:25}
\crossref{2Sam}{22}{48}{2Sa 18:19,\allowbreak31 1Sa 25:30 Ps 94:1}
\crossref{2Sam}{22}{49}{2Sa 5:12; 7:8,\allowbreak9 Nu 24:7,\allowbreak17-\allowbreak19 1Sa 2:8 Ps 18:48}
\crossref{2Sam}{22}{50}{Ro 15:9}
\crossref{2Sam}{22}{51}{22:2 Ps 3:3; 21:1; 48:3; 89:26; 91:2; 144:10}
\crossref{2Sam}{23}{1}{Ge 49:1 De 33:1 Jos 23:1-\allowbreak24:32 Ps 72:20 2Pe 1:13-\allowbreak15}
\crossref{2Sam}{23}{2}{Mt 22:43 Mr 12:36 Ac 2:25-\allowbreak31 Heb 3:7,\allowbreak8 2Pe 1:21}
\crossref{2Sam}{23}{3}{Ge 33:20 Ex 3:15; 19:5,\allowbreak6; 20:2}
\crossref{2Sam}{23}{4}{Jud 5:31 Ps 89:36; 110:3 Pr 4:18 Isa 60:1,\allowbreak3,\allowbreak18-\allowbreak20 Ho 6:5}
\crossref{2Sam}{23}{5}{2Sa 7:18; 12:10; 13:14,\allowbreak28; 18:14 1Ki 1:5; 2:24,\allowbreak25; 11:6-\allowbreak8; 12:14}
\crossref{2Sam}{23}{6}{2Sa 20:1 De 13:13 1Sa 2:12}
\crossref{2Sam}{23}{7}{2Sa 22:8-\allowbreak10 Isa 27:4 Mt 3:10-\allowbreak12; 13:42 Lu 19:14,\allowbreak27 Joh 15:6}
\crossref{2Sam}{23}{8}{}
\crossref{2Sam}{23}{9}{1Ch 11:12-\allowbreak14; 27:4}
\crossref{2Sam}{23}{10}{Jos 10:10,\allowbreak42; 11:8 Jud 15:14,\allowbreak18 1Sa 11:13; 14:6,\allowbreak23; 19:5 2Ki 5:1}
\crossref{2Sam}{23}{11}{1Ch 11:27}
\crossref{2Sam}{23}{12}{23:10 Ps 3:8; 44:2 Pr 21:31}
\crossref{2Sam}{23}{13}{1Ch 11:15-\allowbreak19}
\crossref{2Sam}{23}{14}{1Sa 22:1,\allowbreak4,\allowbreak5; 24:22 1Ch 12:16}
\crossref{2Sam}{23}{15}{Nu 11:4,\allowbreak5 Ps 42:1,\allowbreak2; 63:1; 119:81 Isa 41:17; 44:3 Joh 4:10,\allowbreak14}
\crossref{2Sam}{23}{16}{23:9 1Sa 19:5 Ac 20:24 Ro 5:7 2Co 5:14}
\crossref{2Sam}{23}{17}{2Sa 20:20 Ge 44:17 1Sa 2:30; 26:11 1Ki 21:3 1Ch 11:19}
\crossref{2Sam}{23}{18}{2Sa 2:18; 3:30; 10:10,\allowbreak14; 18:2; 20:10 1Sa 26:6-\allowbreak8 1Ch 2:16; 11:20,\allowbreak21}
\crossref{2Sam}{23}{19}{23:9,\allowbreak13,\allowbreak16 1Ch 11:25 Mt 13:8,\allowbreak23 1Co 15:41}
\crossref{2Sam}{23}{20}{2Sa 8:18; 20:23 1Ki 1:8,\allowbreak26,\allowbreak38; 2:29-\allowbreak35,\allowbreak46 1Ch 18:17; 27:5,\allowbreak6}
\crossref{2Sam}{23}{21}{1Ch 11:23}
\crossref{2Sam}{23}{22}{}
\crossref{2Sam}{23}{23}{1Ch 27:6}
\crossref{2Sam}{23}{24}{2Sa 2:18 1Ch 11:26; 27:7}
\crossref{2Sam}{23}{25}{1Ch 11:27,\allowbreak28}
\crossref{2Sam}{23}{26}{1Ch 11:27; 27:10}
\crossref{2Sam}{23}{27}{1Ch 11:28}
\crossref{2Sam}{23}{28}{1Ch 11:30; 27:13}
\crossref{2Sam}{23}{29}{1Ch 11:30}
\crossref{2Sam}{23}{30}{1Ch 11:31; 27:14}
\crossref{2Sam}{23}{31}{1Ch 11:32}
\crossref{2Sam}{23}{32}{1Ch 11:34}
\crossref{2Sam}{23}{33}{1Ch 11:27}
\crossref{2Sam}{23}{34}{2Sa 11:3; 15:31; 17:23 1Ch 27:33,\allowbreak34}
\crossref{2Sam}{23}{35}{1Ch 11:37}
\crossref{2Sam}{23}{36}{1Ch 11:38}
\crossref{2Sam}{23}{37}{1Ch 11:39}
\crossref{2Sam}{23}{38}{2Sa 20:26 1Ch 2:53; 11:40}
\crossref{2Sam}{23}{39}{2Sa 11:3,\allowbreak6-\allowbreak27; 12:9 1Ki 15:5 1Ch 11:41 Mt 1:6}
\crossref{2Sam}{24}{1}{2Sa 21:1-\allowbreak14}
\crossref{2Sam}{24}{2}{2Sa 2:13; 8:16; 20:23; 23:37}
\crossref{2Sam}{24}{3}{2Sa 10:12 1Ch 21:3,\allowbreak4 Ps 115:14 Pr 14:28 Isa 60:5}
\crossref{2Sam}{24}{4}{1Ch 21:4 Ec 8:4}
\crossref{2Sam}{24}{5}{De 2:36 Jos 13:9,\allowbreak16 1Sa 30:28 Isa 17:2}
\crossref{2Sam}{24}{6}{Ge 31:21,\allowbreak47,\allowbreak48 Nu 32:1,\allowbreak39}
\crossref{2Sam}{24}{7}{Jos 19:29}
\crossref{2Sam}{24}{8}{}
\crossref{2Sam}{24}{9}{1Ch 21:5,\allowbreak6; 27:23,\allowbreak24}
\crossref{2Sam}{24}{10}{1Sa 24:5 Joh 8:9 1Jo 3:20,\allowbreak21}
\crossref{2Sam}{24}{11}{1Sa 22:5 1Ch 2:19; 29:29}
\crossref{2Sam}{24}{12}{1Ch 21:10,\allowbreak11}
\crossref{2Sam}{24}{13}{2Sa 21:1 Le 26:20 1Ki 17:1-\allowbreak7 1Ch 21:12 Eze 14:13,\allowbreak21 Lu 4:25}
\crossref{2Sam}{24}{14}{1Sa 13:6 2Ki 6:15 Joh 12:27 Php 1:23}
\crossref{2Sam}{24}{15}{Nu 16:46-\allowbreak49; 25:9 1Sa 6:19 1Ch 21:14; 27:4 Mt 24:7 Re 6:8}
\crossref{2Sam}{24}{16}{Ex 12:23 2Ki 19:35 1Ch 21:15,\allowbreak16 2Ch 32:21 Ps 35:6 Ac 12:23}
\crossref{2Sam}{24}{17}{1Ch 21:16,\allowbreak17}
\crossref{2Sam}{24}{18}{24:11 1Ch 21:18-\allowbreak30}
\crossref{2Sam}{24}{19}{Ge 6:22 1Ch 21:19 2Ch 20:20; 36:16 Ne 9:26 Heb 11:8}
\crossref{2Sam}{24}{20}{2Sa 9:8 Ge 18:2 Ru 2:10 1Ch 21:20,\allowbreak21}
\crossref{2Sam}{24}{21}{24:3,\allowbreak18}
\crossref{2Sam}{24}{22}{Ge 23:11 1Ch 21:22}
\crossref{2Sam}{24}{23}{Ps 45:16 Isa 32:8}
\crossref{2Sam}{24}{24}{Ge 23:13 1Ch 21:24 Mal 1:12-\allowbreak14 Ro 12:17}
\crossref{2Sam}{24}{25}{Ge 8:20; 22:9 1Sa 7:9,\allowbreak17}

% 1King
\crossref{1King}{1}{1}{Ge 18:11; 24:1 Jos 23:1,\allowbreak2 Lu 1:7}
\crossref{1King}{1}{2}{Ge 16:5 De 13:6 2Sa 12:3 Mic 7:5}
\crossref{1King}{1}{3}{Es 2:2,\allowbreak4}
\crossref{1King}{1}{4}{Mt 1:25}
\crossref{1King}{1}{5}{2Sa 3:4 1Ch 3:2}
\crossref{1King}{1}{6}{1Sa 3:13 Pr 22:15; 23:13,\allowbreak14; 29:15 Heb 12:5,\allowbreak6}
\crossref{1King}{1}{7}{2Sa 15:12 Ps 2:2}
\crossref{1King}{1}{8}{1Ki 2:35 2Sa 8:17,\allowbreak18; 20:25 1Ch 27:5,\allowbreak6 Eze 44:15}
\crossref{1King}{1}{9}{2Sa 15:12 Pr 15:8}
\crossref{1King}{1}{10}{1:8,\allowbreak19 2Sa 12:1-\allowbreak15}
\crossref{1King}{1}{11}{2Sa 7:12-\allowbreak17; 12:24,\allowbreak25 1Ch 22:9,\allowbreak10; 28:4,\allowbreak5; 29:1}
\crossref{1King}{1}{12}{Pr 11:14; 20:18; 27:9 Jer 38:15}
\crossref{1King}{1}{13}{1:11,\allowbreak17,\allowbreak30 1Ch 22:6-\allowbreak13}
\crossref{1King}{1}{14}{1:17-\allowbreak27 2Co 13:1}
\crossref{1King}{1}{15}{1:2-\allowbreak4}
\crossref{1King}{1}{16}{1:23 1Sa 20:41; 24:8; 25:23}
\crossref{1King}{1}{17}{Ge 18:12 1Pe 3:6}
\crossref{1King}{1}{18}{1:5,\allowbreak24 2Sa 15:10}
\crossref{1King}{1}{19}{1:7-\allowbreak10,\allowbreak25}
\crossref{1King}{1}{20}{2Ch 20:12 Ps 25:15; 123:2 Zec 3:9}
\crossref{1King}{1}{21}{1Ki 2:10 Ge 15:15 De 31:16}
\crossref{1King}{1}{22}{Ge 24:15 Job 1:16-\allowbreak18 Da 9:20}
\crossref{1King}{1}{23}{1:16 Ro 13:7 1Pe 2:17}
\crossref{1King}{1}{24}{1:14,\allowbreak18}
\crossref{1King}{1}{25}{1:9,\allowbreak19 1Sa 11:14,\allowbreak15 1Ch 29:21-\allowbreak13}
\crossref{1King}{1}{26}{1:8,\allowbreak19 2Sa 7:2,\allowbreak12-\allowbreak17; 12:25}
\crossref{1King}{1}{27}{1:24 2Ki 4:27 Joh 15:15}
\crossref{1King}{1}{28}{1:28}
\crossref{1King}{1}{29}{1Ki 2:24; 17:1; 18:10 Jud 8:19 1Sa 14:39,\allowbreak45; 19:6; 20:21 2Sa 12:5}
\crossref{1King}{1}{30}{1:13,\allowbreak17}
\crossref{1King}{1}{31}{2Sa 9:6 Es 3:2 Mt 21:37 Eph 5:33 Heb 12:9}
\crossref{1King}{1}{32}{1:8,\allowbreak26,\allowbreak38}
\crossref{1King}{1}{33}{2Sa 20:6}
\crossref{1King}{1}{34}{1Ki 19:16 1Sa 10:1; 16:3,\allowbreak12,\allowbreak13 2Sa 2:4; 5:3 2Ki 9:3,\allowbreak6; 11:12}
\crossref{1King}{1}{35}{1:13,\allowbreak17; 2:12}
\crossref{1King}{1}{36}{De 27:15-\allowbreak26 Ps 72:19 Jer 11:5; 28:6 Mt 6:13; 28:20 1Co 14:16}
\crossref{1King}{1}{37}{1Ki 3:7-\allowbreak9 Ex 3:12 Jos 1:5,\allowbreak17 1Sa 20:13 1Ch 28:20 2Ch 1:1 Ps 46:7}
\crossref{1King}{1}{38}{1:8,\allowbreak26}
\crossref{1King}{1}{39}{1Sa 16:13}
\crossref{1King}{1}{40}{Da 3:5}
\crossref{1King}{1}{41}{Job 20:5 Pr 14:13 Ec 7:4-\allowbreak6 Mt 24:38,\allowbreak39 Lu 17:26-\allowbreak29}
\crossref{1King}{1}{42}{2Sa 15:36; 17:17}
\crossref{1King}{1}{43}{1:32-\allowbreak40}
\crossref{1King}{1}{44}{}
\crossref{1King}{1}{45}{1:40 1Sa 4:5 Ezr 3:13}
\crossref{1King}{1}{46}{1:13 1Ch 29:23 Ps 132:11 Hag 2:22}
\crossref{1King}{1}{47}{Ex 12:32 2Sa 8:10; 21:3 Ezr 6:10 Ps 20:1-\allowbreak4}
\crossref{1King}{1}{48}{Ge 14:20 1Ch 29:10,\allowbreak20 Ne 9:5 Ps 34:1; 41:13; 72:17-\allowbreak19; 103:1,\allowbreak2}
\crossref{1King}{1}{49}{Pr 28:1 Isa 21:4,\allowbreak5 Da 5:4-\allowbreak6}
\crossref{1King}{1}{50}{1Ki 2:28 Ex 21:14; 38:2 Ps 118:27}
\crossref{1King}{1}{51}{}
\crossref{1King}{1}{52}{1Sa 14:45 2Sa 14:11 Mt 10:30 Lu 21:18 Ac 27:34}
\crossref{1King}{1}{53}{1:16,\allowbreak31 2Sa 1:2}
\crossref{1King}{2}{1}{Ge 47:29 De 31:14; 33:1 2Ti 4:6 2Pe 1:13-\allowbreak15}
\crossref{1King}{2}{2}{Jos 23:14 Job 16:22; 30:23 Ps 89:48 Heb 9:27}
\crossref{1King}{2}{3}{De 29:9 Jos 1:7; 22:5 1Ch 22:12,\allowbreak13; 28:8,\allowbreak9; 29:19}
\crossref{1King}{2}{4}{Ge 18:19 De 7:12 1Ch 28:9 Joh 15:9,\allowbreak10 Jude 1:20,\allowbreak21,\allowbreak24}
\crossref{1King}{2}{5}{1Ki 1:7,\allowbreak18,\allowbreak19 2Sa 3:39; 18:5,\allowbreak12,\allowbreak14; 19:5-\allowbreak7}
\crossref{1King}{2}{6}{2:9 Pr 20:26}
\crossref{1King}{2}{7}{2Sa 17:27-\allowbreak29; 19:31-\allowbreak40 Pr 27:10}
\crossref{1King}{2}{8}{2:36-\allowbreak46 2Sa 16:5-\allowbreak8}
\crossref{1King}{2}{9}{1Ki 3:12,\allowbreak28}
\crossref{1King}{2}{10}{1Ki 1:21 1Ch 29:28 Ac 2:29; 13:36}
\crossref{1King}{2}{11}{2Sa 5:4 1Ch 29:26,\allowbreak27}
\crossref{1King}{2}{12}{1Ki 1:46 1Ch 29:23-\allowbreak25 2Ch 1:1 Ps 132:12}
\crossref{1King}{2}{13}{1Ki 1:5-\allowbreak10,\allowbreak50-\allowbreak53}
\crossref{1King}{2}{14}{2Sa 14:12 Lu 7:40}
\crossref{1King}{2}{15}{1Ki 1:5,\allowbreak25 2Sa 15:6,\allowbreak13; 16:18}
\crossref{1King}{2}{16}{Ps 132:10 Pr 30:7}
\crossref{1King}{2}{17}{1Ki 1:2-\allowbreak4 2Sa 3:7; 12:8}
\crossref{1King}{2}{18}{Pr 14:15}
\crossref{1King}{2}{19}{Ex 20:12 Le 19:3,\allowbreak32}
\crossref{1King}{2}{20}{Mt 20:20,\allowbreak21 Joh 2:3,\allowbreak4}
\crossref{1King}{2}{21}{2Sa 16:21,\allowbreak22}
\crossref{1King}{2}{22}{Mt 20:22 Mr 10:38 Jas 4:3}
\crossref{1King}{2}{23}{1Ki 20:10 Ru 1:17 1Sa 14:44 2Sa 3:9,\allowbreak35; 19:13 2Ki 6:31}
\crossref{1King}{2}{24}{1Ki 1:29}
\crossref{1King}{2}{25}{2:31,\allowbreak34,\allowbreak46 Jud 8:20,\allowbreak21 1Sa 15:33 2Sa 1:15; 4:12}
\crossref{1King}{2}{26}{2:35; 1:7,\allowbreak25}
\crossref{1King}{2}{27}{1Sa 2:30-\allowbreak36; 3:12-\allowbreak14 Mt 26:56 Joh 12:38; 19:24,\allowbreak28,\allowbreak36,\allowbreak37}
\crossref{1King}{2}{28}{1Ki 1:7 De 32:35 2Sa 18:2,\allowbreak14,\allowbreak15}
\crossref{1King}{2}{29}{Ex 21:14 Eze 9:6 1Pe 4:17}
\crossref{1King}{2}{30}{}
\crossref{1King}{2}{31}{Ex 21:14}
\crossref{1King}{2}{32}{2:44 Ge 4:11 Jud 9:24,\allowbreak57 Ps 7:16}
\crossref{1King}{2}{33}{2:32 2Sa 3:29 2Ki 5:27 Ps 101:8; 109:6-\allowbreak15 Mt 27:25}
\crossref{1King}{2}{34}{2:25,\allowbreak31,\allowbreak46}
\crossref{1King}{2}{35}{Job 34:24}
\crossref{1King}{2}{36}{2:8,\allowbreak9 2Sa 16:5-\allowbreak9 Pr 20:8,\allowbreak26}
\crossref{1King}{2}{37}{1Ki 15:13 2Sa 15:23 2Ki 23:6 2Ch 29:16 Jer 31:40 Joh 18:1}
\crossref{1King}{2}{38}{1Ki 20:4 2Ki 20:19}
\crossref{1King}{2}{39}{1Sa 21:10; 27:2,\allowbreak3}
\crossref{1King}{2}{40}{Pr 15:27 Lu 12:15 1Ti 6:10}
\crossref{1King}{2}{41}{}
\crossref{1King}{2}{42}{2:36-\allowbreak38 Ps 15:4 Lu 19:22}
\crossref{1King}{2}{43}{2Sa 21:2 Eze 17:18,\allowbreak19}
\crossref{1King}{2}{44}{2Sa 16:5-\allowbreak13 Joh 8:9 Ro 2:15 1Jo 3:20}
\crossref{1King}{2}{45}{Ps 21:6; 72:17}
\crossref{1King}{2}{46}{2:12,\allowbreak45 2Ch 1:1 Pr 29:4}
\crossref{1King}{3}{1}{2Ch 18:1 Ezr 9:14}
\crossref{1King}{3}{2}{1Ki 5:3 1Ch 17:4-\allowbreak6; 28:3-\allowbreak6 Ac 7:47-\allowbreak49}
\crossref{1King}{3}{3}{De 6:5; 10:12; 30:6,\allowbreak16,\allowbreak20 2Sa 12:24,\allowbreak25 Ps 31:23 Mt 22:37}
\crossref{1King}{3}{4}{1Ki 9:2 Jos 9:3; 10:2 1Ch 16:39; 21:29 2Ch 1:3,\allowbreak7-\allowbreak12}
\crossref{1King}{3}{5}{1Ki 9:2}
\crossref{1King}{3}{6}{Nu 12:7 2Sa 7:5}
\crossref{1King}{3}{7}{Da 2:21; 4:25,\allowbreak32; 5:18,\allowbreak21}
\crossref{1King}{3}{8}{Ex 19:5,\allowbreak6 De 7:6-\allowbreak8 1Sa 12:22 Ps 78:71}
\crossref{1King}{3}{9}{1Ch 22:12; 29:19 2Ch 1:10 Ps 119:34,\allowbreak73,\allowbreak144 Pr 2:3-\allowbreak9; 3:13-\allowbreak18}
\crossref{1King}{3}{10}{Pr 15:8}
\crossref{1King}{3}{11}{Ps 4:6 Pr 16:31 Mt 20:21,\allowbreak22 Ro 8:26 Jas 4:2,\allowbreak3}
\crossref{1King}{3}{12}{Ps 10:17 Isa 65:24 Ro 8:26,\allowbreak27 1Jo 5:14,\allowbreak15}
\crossref{1King}{3}{13}{Ps 84:11,\allowbreak12 Mt 6:33 Ro 8:32 1Co 3:22,\allowbreak23 Eph 3:20}
\crossref{1King}{3}{14}{1Ki 2:3,\allowbreak4 1Ch 22:12,\allowbreak13; 28:9 2Ch 7:17-\allowbreak19 Ps 132:12 Zec 3:7}
\crossref{1King}{3}{15}{Ge 41:7 Jer 31:26}
\crossref{1King}{3}{16}{Le 19:29 De 23:17 Jos 2:1}
\crossref{1King}{3}{17}{Ge 43:20 Ro 13:7}
\crossref{1King}{3}{18}{3:18}
\crossref{1King}{3}{19}{}
\crossref{1King}{3}{20}{Job 24:13-\allowbreak17 Ps 139:11 Mt 13:25 Joh 3:20}
\crossref{1King}{3}{21}{Ge 21:7 1Sa 1:23 La 4:3,\allowbreak4}
\crossref{1King}{3}{22}{3:23,\allowbreak24}
\crossref{1King}{3}{23}{3:23}
\crossref{1King}{3}{24}{}
\crossref{1King}{3}{25}{}
\crossref{1King}{3}{26}{Ge 43:30 Isa 49:15 Jer 31:20 Ho 11:8 Php 1:8; 2:1 1Jo 3:17}
\crossref{1King}{3}{27}{}
\crossref{1King}{3}{28}{Ex 14:31 Jos 4:14 1Sa 12:18 1Ch 29:24 Pr 24:21}
\crossref{1King}{4}{1}{1Ki 11:13,\allowbreak35,\allowbreak36; 12:19,\allowbreak20 2Sa 5:5 1Ch 12:38 2Ch 9:30 Ec 1:12}
\crossref{1King}{4}{2}{1Ch 6:8-\allowbreak10; 27:17}
\crossref{1King}{4}{3}{2Sa 20:25}
\crossref{1King}{4}{4}{1Ki 2:35}
\crossref{1King}{4}{5}{1Ki 1:10-\allowbreak53 2Sa 7:2; 12:1-\allowbreak15,\allowbreak25}
\crossref{1King}{4}{6}{1Ki 12:18 2Sa 20:24}
\crossref{1King}{4}{7}{1Ch 27:1-\allowbreak15}
\crossref{1King}{4}{8}{Jud 17:1; 19:1}
\crossref{1King}{4}{9}{Jos 19:42}
\crossref{1King}{4}{10}{Jos 15:35}
\crossref{1King}{4}{11}{Jos 12:23; 17:11 Jud 1:27}
\crossref{1King}{4}{12}{Jos 17:11 Jud 5:19}
\crossref{1King}{4}{13}{1Ki 22:3 De 4:43 Jos 20:8; 21:38 2Ki 9:1,\allowbreak14}
\crossref{1King}{4}{14}{Ge 32:2 2Sa 2:8; 17:24,\allowbreak27}
\crossref{1King}{4}{15}{Jos 19:32-\allowbreak39}
\crossref{1King}{4}{16}{Jos 19:24-\allowbreak31}
\crossref{1King}{4}{17}{Jos 19:17-\allowbreak23}
\crossref{1King}{4}{18}{1Ki 1:8 Zec 12:13}
\crossref{1King}{4}{19}{Nu 21:21-\allowbreak35 De 2:26-\allowbreak37; 3:1-\allowbreak17 Jos 13:9-\allowbreak12}
\crossref{1King}{4}{20}{1Ki 3:8 Ge 13:16; 15:5; 22:17 Pr 14:28}
\crossref{1King}{4}{21}{4:24 Ge 15:18 Ex 23:31 De 11:24 Jos 1:4 2Ch 9:26-\allowbreak31 Ezr 4:20}
\crossref{1King}{4}{22}{4:22}
\crossref{1King}{4}{23}{Ne 5:17,\allowbreak18}
\crossref{1King}{4}{24}{Ge 10:19 Jud 16:1}
\crossref{1King}{4}{25}{Isa 60:18 Jer 23:5,\allowbreak6; 33:15,\allowbreak16 Eze 38:11}
\crossref{1King}{4}{26}{1Ki 10:25,\allowbreak26 De 17:16 2Sa 8:4 2Ch 1:14; 9:25 Ps 20:7}
\crossref{1King}{4}{27}{4:7-\allowbreak19}
\crossref{1King}{4}{28}{Es 8:10,\allowbreak14 Mic 1:13}
\crossref{1King}{4}{29}{1Ki 3:12,\allowbreak28; 10:23,\allowbreak24 2Ch 1:10-\allowbreak12 Ps 119:34 Pr 2:6 Ec 1:16; 2:26}
\crossref{1King}{4}{30}{Ge 25:6 Job 1:3 Da 1:20; 4:7; 5:11,\allowbreak12 Mt 2:1,\allowbreak16}
\crossref{1King}{4}{31}{1Ki 3:12 Mt 12:42 Lu 11:31 Col 2:3}
\crossref{1King}{4}{32}{Pr 1:1-\allowbreak31:30 Ec 12:9 Mt 13:35}
\crossref{1King}{4}{33}{Ex 12:22 Nu 19:18 Ps 51:7 Heb 9:19}
\crossref{1King}{4}{34}{1Ki 10:1 2Ch 9:1,\allowbreak23 Isa 2:2 Zec 8:23}
\crossref{1King}{5}{1}{5:10,\allowbreak13; 9:12-\allowbreak14 2Ch 2:3}
\crossref{1King}{5}{2}{2Ch 2:3}
\crossref{1King}{5}{3}{2Sa 7:5-\allowbreak11 1Ch 22:4-\allowbreak6 2Ch 6:6-\allowbreak8}
\crossref{1King}{5}{4}{1Ki 4:24 1Ch 22:9 Ps 72:7 Isa 9:7 Ac 9:31}
\crossref{1King}{5}{5}{2Ch 2:1-\allowbreak4,\allowbreak5-\allowbreak8}
\crossref{1King}{5}{6}{1Ki 6:9,\allowbreak10,\allowbreak16,\allowbreak20 2Ch 2:8,\allowbreak10 Ps 29:5}
\crossref{1King}{5}{7}{1Ki 10:9 2Ch 2:11,\allowbreak12; 9:7,\allowbreak8 Ps 122:6,\allowbreak7; 137:6}
\crossref{1King}{5}{8}{1Ki 6:15,\allowbreak34 2Sa 6:5 2Ch 3:5}
\crossref{1King}{5}{9}{De 3:25}
\crossref{1King}{5}{10}{}
\crossref{1King}{5}{11}{1Ki 4:22}
\crossref{1King}{5}{12}{1Ki 3:12; 4:29 2Ch 1:12 Jas 1:5}
\crossref{1King}{5}{13}{1Ki 4:6}
\crossref{1King}{5}{14}{1Ki 4:7-\allowbreak19 1Ch 27:1-\allowbreak15}
\crossref{1King}{5}{15}{}
\crossref{1King}{5}{16}{}
\crossref{1King}{5}{17}{1Ki 6:7; 7:9 1Ch 22:2 Isa 28:16 1Co 3:11,\allowbreak12 1Pe 2:6,\allowbreak7 Re 21:14-\allowbreak21}
\crossref{1King}{5}{18}{Jos 13:5 Ps 83:7 Eze 27:9}
\crossref{1King}{6}{1}{Jud 11:26 2Ch 3:1,\allowbreak2}
\crossref{1King}{6}{2}{Eze 40:1-\allowbreak41:26}
\crossref{1King}{6}{3}{1Ch 28:11 2Ch 3:3,\allowbreak4 Eze 41:15 Mt 4:5 Joh 10:23 Ac 3:10,\allowbreak11}
\crossref{1King}{6}{4}{}
\crossref{1King}{6}{5}{1Ch 9:26; 23:28; 28:11 2Ch 31:11 Ne 10:37; 12:44; 13:5-\allowbreak9 So 1:4}
\crossref{1King}{6}{6}{6:6}
\crossref{1King}{6}{7}{1Ki 5:17,\allowbreak18 De 27:5,\allowbreak6 Pr 24:27 Ro 9:23 2Co 5:5 Col 1:12 1Pe 2:5}
\crossref{1King}{6}{8}{Eze 41:6,\allowbreak7}
\crossref{1King}{6}{9}{6:14,\allowbreak38}
\crossref{1King}{6}{10}{6:10}
\crossref{1King}{6}{11}{}
\crossref{1King}{6}{12}{1Ki 2:3,\allowbreak4; 3:14; 8:25; 9:3-\allowbreak6 1Sa 12:14,\allowbreak15 1Ch 28:9 2Ch 7:17,\allowbreak18}
\crossref{1King}{6}{13}{1Ki 8:27 Ex 25:8 Le 26:11 Ps 68:18; 132:12,\allowbreak13 Isa 57:15}
\crossref{1King}{6}{14}{6:9,\allowbreak38 Ac 7:47,\allowbreak48}
\crossref{1King}{6}{15}{}
\crossref{1King}{6}{16}{6:5,\allowbreak19,\allowbreak20; 8:6 Ex 25:21,\allowbreak22; 26:23 Le 16:2 2Ch 3:8 Eze 45:3}
\crossref{1King}{6}{17}{}
\crossref{1King}{6}{18}{6:18}
\crossref{1King}{6}{19}{6:5,\allowbreak16 2Ch 4:20 Ps 28:2}
\crossref{1King}{6}{20}{6:2,\allowbreak3}
\crossref{1King}{6}{21}{Ex 26:29,\allowbreak32; 36:34 2Ch 3:7-\allowbreak9}
\crossref{1King}{6}{22}{6:20 Ex 30:1,\allowbreak3,\allowbreak5,\allowbreak6 2Ch 3:7-\allowbreak14}
\crossref{1King}{6}{23}{}
\crossref{1King}{6}{24}{6:24}
\crossref{1King}{6}{25}{6:25}
\crossref{1King}{6}{26}{}
\crossref{1King}{6}{27}{}
\crossref{1King}{6}{28}{}
\crossref{1King}{6}{29}{Ex 36:8 2Ch 3:14; 4:2-\allowbreak5 Ps 103:20; 148:2 Lu 2:13,\allowbreak14 Eph 3:10}
\crossref{1King}{6}{30}{Isa 54:11,\allowbreak12; 60:17 Re 21:18-\allowbreak21}
\crossref{1King}{6}{31}{Joh 10:9; 14:6 Eph 3:18 Heb 10:19,\allowbreak20}
\crossref{1King}{6}{32}{}
\crossref{1King}{6}{33}{6:33}
\crossref{1King}{6}{34}{1Ki 5:8}
\crossref{1King}{6}{35}{}
\crossref{1King}{6}{36}{Ex 27:9-\allowbreak19; 38:9-\allowbreak20 2Ch 4:9; 7:7 Re 11:2}
\crossref{1King}{6}{37}{6:1 2Ch 3:2}
\crossref{1King}{6}{38}{Ezr 6:14,\allowbreak15 Zec 4:9; 6:13-\allowbreak15}
\crossref{1King}{7}{1}{1Ki 9:10 2Ch 8:1 Ec 2:4,\allowbreak5 Mt 6:33}
\crossref{1King}{7}{2}{1Ki 9:19; 10:17 2Ch 9:16 So 7:4}
\crossref{1King}{7}{3}{1Ki 6:5}
\crossref{1King}{7}{4}{7:5; 6:4 Isa 54:12 Eze 40:16,\allowbreak22,\allowbreak25,\allowbreak29,\allowbreak33,\allowbreak36; 41:26}
\crossref{1King}{7}{5}{}
\crossref{1King}{7}{6}{}
\crossref{1King}{7}{7}{1Ki 6:3}
\crossref{1King}{7}{8}{2Ki 20:4}
\crossref{1King}{7}{9}{7:10,\allowbreak11; 5:17}
\crossref{1King}{7}{10}{Isa 28:16; 54:11 1Co 3:10,\allowbreak11 Re 21:19,\allowbreak20}
\crossref{1King}{7}{11}{Eph 2:20-\allowbreak22 1Pe 2:5}
\crossref{1King}{7}{12}{1Ki 6:36}
\crossref{1King}{7}{13}{7:40 2Ch 2:13; 4:11}
\crossref{1King}{7}{14}{2Ch 2:14}
\crossref{1King}{7}{15}{7:21 2Ki 25:16,\allowbreak17 2Ch 3:15-\allowbreak17; 4:12-\allowbreak22 Jer 52:21-\allowbreak13}
\crossref{1King}{7}{16}{Ex 36:38; 38:17,\allowbreak19,\allowbreak28 2Ch 4:12,\allowbreak13}
\crossref{1King}{7}{17}{Ex 28:14,\allowbreak22,\allowbreak24,\allowbreak25; 39:15-\allowbreak18 2Ki 25:17}
\crossref{1King}{7}{18}{Ex 28:14,\allowbreak22,\allowbreak24,\allowbreak25; 39:15-\allowbreak18 2Ki 25:17}
\crossref{1King}{7}{19}{7:22; 6:18,\allowbreak32-\allowbreak35}
\crossref{1King}{7}{20}{2Ki 25:17 2Ch 3:16; 4:13 Jer 52:22,\allowbreak23}
\crossref{1King}{7}{21}{2Ch 3:17 Ga 2:9 Re 3:12}
\crossref{1King}{7}{22}{}
\crossref{1King}{7}{23}{Ex 30:18-\allowbreak21; 38:8}
\crossref{1King}{7}{24}{1Ki 6:18 Ex 25:31-\allowbreak36; 37:17-\allowbreak22}
\crossref{1King}{7}{25}{2Ch 4:4,\allowbreak5 Jer 52:20 Eze 1:10 Mt 28:19 Mr 16:15,\allowbreak16 Lu 24:47}
\crossref{1King}{7}{26}{Jer 52:21}
\crossref{1King}{7}{27}{}
\crossref{1King}{7}{28}{}
\crossref{1King}{7}{29}{7:25; 6:27 Eze 1:10; 10:14; 41:18,\allowbreak19 Ho 5:14 Re 4:6,\allowbreak7; 5:5}
\crossref{1King}{7}{30}{Eze 1:15-\allowbreak21; 3:13; 10:10-\allowbreak13}
\crossref{1King}{7}{31}{}
\crossref{1King}{7}{32}{7:32}
\crossref{1King}{7}{33}{Eze 1:16,\allowbreak18}
\crossref{1King}{7}{34}{7:34}
\crossref{1King}{7}{35}{}
\crossref{1King}{7}{36}{7:29; 6:29,\allowbreak32,\allowbreak35 Eze 40:31,\allowbreak37; 41:18-\allowbreak20,\allowbreak25,\allowbreak26}
\crossref{1King}{7}{37}{}
\crossref{1King}{7}{38}{Ex 30:17-\allowbreak21,\allowbreak28; 38:8; 40:11,\allowbreak12 2Ch 4:6-\allowbreak22 Zec 13:1 Heb 9:10}
\crossref{1King}{7}{39}{2Ch 4:6,\allowbreak10}
\crossref{1King}{7}{40}{7:13}
\crossref{1King}{7}{41}{7:15-\allowbreak22 2Ch 4:12}
\crossref{1King}{7}{42}{7:42}
\crossref{1King}{7}{43}{7:27-\allowbreak39}
\crossref{1King}{7}{44}{7:23-\allowbreak26}
\crossref{1King}{7}{45}{Ex 27:3; 38:3 Le 8:31 1Sa 2:13,\allowbreak14 2Ch 4:16 Eze 46:20-\allowbreak24}
\crossref{1King}{7}{46}{Ge 33:17}
\crossref{1King}{7}{47}{1Ch 22:14,\allowbreak16}
\crossref{1King}{7}{48}{Ex 30:1-\allowbreak5; 37:25-\allowbreak28; 39:38; 40:26 2Ch 4:19}
\crossref{1King}{7}{49}{Ex 25:31-\allowbreak40; 37:17-\allowbreak24; 39:37; 40:24,\allowbreak25 2Ch 4:7 Zec 4:1-\allowbreak3,\allowbreak11-\allowbreak14}
\crossref{1King}{7}{50}{Ex 25:29 Nu 7:86}
\crossref{1King}{7}{51}{Ex 40:33 Ezr 6:15 Zec 4:9}
\crossref{1King}{8}{1}{Jos 23:2; 24:1 1Ch 28:1 2Ch 30:1 Ezr 3:1}
\crossref{1King}{8}{2}{Le 23:34 Nu 29:12-\allowbreak40 De 16:13 2Ch 5:3; 7:8-\allowbreak10 Ezr 3:4}
\crossref{1King}{8}{3}{Nu 4:15 De 31:9 Jos 3:6,\allowbreak14,\allowbreak15; 4:9; 6:6 1Ch 15:2,\allowbreak11-\allowbreak15}
\crossref{1King}{8}{4}{1Ki 3:4 2Ch 1:3}
\crossref{1King}{8}{5}{8:62,\allowbreak63 2Sa 6:13 1Ch 16:1}
\crossref{1King}{8}{6}{8:4 2Sa 6:17 2Ch 5:7}
\crossref{1King}{8}{7}{}
\crossref{1King}{8}{8}{Ex 25:14,\allowbreak15; 37:4,\allowbreak5; 40:20}
\crossref{1King}{8}{9}{Ex 25:21 De 10:2 2Ch 5:10}
\crossref{1King}{8}{10}{}
\crossref{1King}{8}{11}{Le 9:6,\allowbreak23 Eze 43:2,\allowbreak4,\allowbreak5; 44:4 Joh 1:14 Ac 7:55 2Co 3:18; 4:6}
\crossref{1King}{8}{12}{De 4:11 2Ch 6:1,\allowbreak2-\allowbreak11 Ps 18:8-\allowbreak11; 97:2}
\crossref{1King}{8}{13}{2Sa 7:13 1Ch 17:12; 22:10,\allowbreak11; 28:6,\allowbreak10,\allowbreak20 2Ch 6:2}
\crossref{1King}{8}{14}{8:55,\allowbreak56 Jos 22:6 2Sa 6:18 1Ch 16:2 2Ch 6:3; 30:18-\allowbreak20 Ps 118:26}
\crossref{1King}{8}{15}{1Ch 29:10,\allowbreak20 2Ch 6:4; 20:26 Ne 9:5 Ps 41:13; 72:18,\allowbreak19; 115:18}
\crossref{1King}{8}{16}{2Sa 7:6,\allowbreak7 2Ch 6:5-\allowbreak11}
\crossref{1King}{8}{17}{2Sa 7:2,\allowbreak3 1Ch 17:1,\allowbreak2-\allowbreak15; 22:7; 28:2}
\crossref{1King}{8}{18}{2Ch 6:7-\allowbreak9 2Co 8:12}
\crossref{1King}{8}{19}{1Ki 5:3-\allowbreak5 2Sa 7:5,\allowbreak12,\allowbreak13 1Ch 17:4,\allowbreak11,\allowbreak12; 22:8-\allowbreak10; 28:6}
\crossref{1King}{8}{20}{8:15 Ne 9:8 Isa 9:7 Jer 29:10,\allowbreak11 Eze 12:25; 37:14 Mic 7:20}
\crossref{1King}{8}{21}{8:5,\allowbreak6}
\crossref{1King}{8}{22}{8:54 2Ki 11:14; 23:3 2Ch 6:12,\allowbreak13-\allowbreak42}
\crossref{1King}{8}{23}{Ge 33:20 Ex 3:15}
\crossref{1King}{8}{24}{8:15 2Sa 7:12,\allowbreak16 2Ch 6:14,\allowbreak15}
\crossref{1King}{8}{25}{1Ki 2:4 2Sa 7:27-\allowbreak29 1Ch 17:23-\allowbreak27 Lu 1:68-\allowbreak72}
\crossref{1King}{8}{26}{8:23 Ex 24:10 1Sa 1:17 Ps 41:13 Isa 41:17; 45:3}
\crossref{1King}{8}{27}{2Ch 6:18 Isa 66:1 Joh 1:14 Ac 7:48,\allowbreak49; 17:24 2Co 6:16 1Jo 3:1}
\crossref{1King}{8}{28}{2Ch 6:19 Ps 141:2 Da 9:17-\allowbreak19 Lu 18:1,\allowbreak7}
\crossref{1King}{8}{29}{8:52 2Ki 19:16 2Ch 6:20,\allowbreak40; 7:15; 16:9 Ne 1:6 Ps 34:15 Da 9:18}
\crossref{1King}{8}{30}{2Ch 20:8,\allowbreak9 Ne 1:5,\allowbreak6}
\crossref{1King}{8}{31}{2Ch 6:22,\allowbreak23}
\crossref{1King}{8}{32}{8:30}
\crossref{1King}{8}{33}{Le 26:17,\allowbreak25 De 28:25,\allowbreak48 Jos 7:8 2Ch 6:24,\allowbreak25 Ps 44:10}
\crossref{1King}{8}{34}{8:30 Ezr 1:1-\allowbreak6 Ps 106:47 Jer 31:4-\allowbreak9,\allowbreak27; 32:37; 33:10-\allowbreak13}
\crossref{1King}{8}{35}{1Ki 17:1 Le 26:19 De 11:17; 28:12,\allowbreak23,\allowbreak24 2Sa 24:13 Jer 14:1-\allowbreak7}
\crossref{1King}{8}{36}{Ps 25:4,\allowbreak5,\allowbreak8,\allowbreak12; 27:11; 32:8; 94:12; 119:33; 143:8 Isa 35:8 Mic 4:2}
\crossref{1King}{8}{37}{Le 26:16,\allowbreak25,\allowbreak26-\allowbreak46 De 28:21,\allowbreak22,\allowbreak25,\allowbreak38-\allowbreak42,\allowbreak52-\allowbreak61 2Ki 6:25-\allowbreak29}
\crossref{1King}{8}{38}{2Ch 20:5-\allowbreak13 Ps 50:15; 91:15 Isa 37:4,\allowbreak15-\allowbreak21 Joe 2:17 Am 7:1-\allowbreak6}
\crossref{1King}{8}{39}{8:32,\allowbreak36}
\crossref{1King}{8}{40}{Ge 22:12 Ex 20:20 De 6:2,\allowbreak13 1Sa 12:24 Ps 115:13; 130:4}
\crossref{1King}{8}{41}{1Ki 10:1,\allowbreak2 Ru 1:16; 2:11 2Ch 6:32 Isa 56:3-\allowbreak7 Mt 8:5,\allowbreak10,\allowbreak11; 15:22-\allowbreak28}
\crossref{1King}{8}{42}{Ex 15:14 De 4:6 Jos 2:10,\allowbreak11; 9:9,\allowbreak10 2Ch 32:31 Da 2:47; 3:28}
\crossref{1King}{8}{43}{1Sa 17:46 2Ki 19:19 2Ch 6:33 Ps 22:27; 67:2; 72:10,\allowbreak11; 86:9}
\crossref{1King}{8}{44}{De 20:1-\allowbreak4; 31:3-\allowbreak6 Jos 1:2-\allowbreak5 2Ch 6:34}
\crossref{1King}{8}{45}{Ge 18:25 Ps 9:4 Jer 5:28}
\crossref{1King}{8}{46}{2Ch 6:36 Job 14:4; 15:14-\allowbreak16 Ps 19:12; 130:3; 143:2 Pr 20:9}
\crossref{1King}{8}{47}{Le 26:40-\allowbreak45 De 4:29-\allowbreak31; 30:1,\allowbreak2 2Ch 6:37; 33:12,\allowbreak13 Eze 16:61,\allowbreak63}
\crossref{1King}{8}{48}{De 4:29; 6:5,\allowbreak6 Jud 10:15,\allowbreak16 1Sa 7:3,\allowbreak4 Ne 1:9 Ps 119:2,\allowbreak10,\allowbreak145}
\crossref{1King}{8}{49}{8:30}
\crossref{1King}{8}{50}{2Ch 30:9 Ezr 7:6,\allowbreak27,\allowbreak28 Ne 1:11; 2:4-\allowbreak8 Ps 106:46 Pr 16:7}
\crossref{1King}{8}{51}{8:53 Ex 32:11,\allowbreak12 Nu 14:13-\allowbreak19 De 9:26-\allowbreak29 2Ch 6:39 Ne 1:10}
\crossref{1King}{8}{52}{8:29 2Ch 6:40}
\crossref{1King}{8}{53}{Ex 19:5,\allowbreak6; 33:16 Nu 23:9 De 4:34; 7:6-\allowbreak8; 9:26,\allowbreak29; 14:2; 32:9}
\crossref{1King}{8}{54}{Lu 11:1; 22:45}
\crossref{1King}{8}{55}{8:14 Nu 6:23-\allowbreak26 2Sa 6:18 1Ch 16:2}
\crossref{1King}{8}{56}{8:15}
\crossref{1King}{8}{57}{De 31:6,\allowbreak8 Jos 1:5,\allowbreak9 1Ch 28:9 2Ch 32:7,\allowbreak8 Ps 46:7,\allowbreak11 Isa 8:10}
\crossref{1King}{8}{58}{Ps 110:3; 119:36 So 1:4 Jer 31:33 Eze 36:26,\allowbreak27 Php 2:13}
\crossref{1King}{8}{59}{}
\crossref{1King}{8}{60}{8:43 Jos 4:24 1Sa 17:46 2Ki 19:19}
\crossref{1King}{8}{61}{1Ki 11:4; 15:3,\allowbreak14 Ge 17:1 De 18:13 2Ki 20:3 1Ch 28:9 Job 1:1,\allowbreak8}
\crossref{1King}{8}{62}{2Sa 6:17-\allowbreak19 2Ch 7:4-\allowbreak7}
\crossref{1King}{8}{63}{Le 3:1-\allowbreak17 1Ch 29:21 2Ch 15:11; 29:32-\allowbreak35; 30:24; 35:7-\allowbreak9}
\crossref{1King}{8}{64}{2Ch 7:7}
\crossref{1King}{8}{65}{8:2 Le 23:34-\allowbreak43 2Ch 7:8,\allowbreak9}
\crossref{1King}{8}{66}{}
\crossref{1King}{9}{1}{1Ki 6:37,\allowbreak38; 7:1,\allowbreak51 2Ch 7:11-\allowbreak22}
\crossref{1King}{9}{2}{1Ki 3:5; 11:9 2Ch 1:7-\allowbreak12; 7:12}
\crossref{1King}{9}{3}{2Ki 20:5 Ps 10:17; 66:19; 116:1 Da 9:23 Joh 11:42 Ac 10:31}
\crossref{1King}{9}{4}{1Ki 3:14; 8:25; 11:4,\allowbreak6,\allowbreak38; 14:8; 15:5 Ge 17:1 De 28:1 2Ch 7:17,\allowbreak18}
\crossref{1King}{9}{5}{1Ki 2:4; 6:12; 8:15,\allowbreak20 2Sa 7:12,\allowbreak16 1Ch 22:9,\allowbreak10 Ps 89:28-\allowbreak39}
\crossref{1King}{9}{6}{1Sa 2:30 2Sa 7:14-\allowbreak16 1Ch 28:9 2Ch 7:19-\allowbreak22; 15:2}
\crossref{1King}{9}{7}{Le 18:24-\allowbreak28 De 4:26; 29:26-\allowbreak28 2Ki 17:20-\allowbreak23; 25:9,\allowbreak21 Jer 7:15}
\crossref{1King}{9}{8}{2Ch 7:21 Isa 64:11 Jer 19:8; 49:17; 50:13 Da 9:12}
\crossref{1King}{9}{9}{De 29:25-\allowbreak28 2Ch 7:22 Jer 2:10-\allowbreak13,\allowbreak19; 5:19; 16:10-\allowbreak13; 50:7}
\crossref{1King}{9}{10}{9:1; 6:37,\allowbreak38; 7:1 2Ch 8:1-\allowbreak18}
\crossref{1King}{9}{11}{1Ki 5:6-\allowbreak10 2Ch 2:8-\allowbreak10,\allowbreak16}
\crossref{1King}{9}{12}{Nu 22:34 Jud 14:3}
\crossref{1King}{9}{13}{1Ki 5:1,\allowbreak2 Am 1:9}
\crossref{1King}{9}{14}{9:11,\allowbreak28; 10:10,\allowbreak14,\allowbreak21}
\crossref{1King}{9}{15}{9:21}
\crossref{1King}{9}{16}{9:24; 3:1}
\crossref{1King}{9}{17}{Jos 16:3; 19:44; 21:22 2Ch 8:4-\allowbreak6,\allowbreak7-\allowbreak18}
\crossref{1King}{9}{18}{Jos 19:44}
\crossref{1King}{9}{19}{1Ki 4:26-\allowbreak28 Ex 1:11}
\crossref{1King}{9}{20}{2Ch 8:7,\allowbreak8-\allowbreak18}
\crossref{1King}{9}{21}{Jud 1:21,\allowbreak27-\allowbreak35; 2:20-\allowbreak23; 3:1-\allowbreak4 Ps 106:34-\allowbreak36}
\crossref{1King}{9}{22}{Le 25:39}
\crossref{1King}{9}{23}{1Ki 5:16 2Ch 2:18; 8:10}
\crossref{1King}{9}{24}{9:16; 3:1; 7:8 2Ch 8:11}
\crossref{1King}{9}{25}{Ex 23:14-\allowbreak17; 34:23 De 16:16 2Ch 8:12,\allowbreak13}
\crossref{1King}{9}{26}{2Ch 8:12,\allowbreak17,\allowbreak18-\allowbreak11:4}
\crossref{1King}{9}{27}{1Ki 5:6,\allowbreak9; 22:49 2Ch 20:36,\allowbreak37}
\crossref{1King}{9}{28}{1Ki 10:11 Ge 10:29 1Ch 29:4 2Ch 8:18; 9:10 Job 22:24; 28:16 Ps 45:9}
\crossref{1King}{10}{1}{2Ch 9:1-\allowbreak12 Mt 12:42 Lu 11:31}
\crossref{1King}{10}{2}{2Ki 5:5,\allowbreak9 Isa 60:6-\allowbreak9 Ac 25:23}
\crossref{1King}{10}{3}{2Ch 9:2 Pr 1:5,\allowbreak6; 13:20 Isa 42:16 Mt 13:11 Joh 7:17 1Co 1:30}
\crossref{1King}{10}{4}{1Ki 3:28; 4:29-\allowbreak31 2Ch 9:3,\allowbreak4 Ec 12:9 Mt 12:42}
\crossref{1King}{10}{5}{1Ki 4:22,\allowbreak23}
\crossref{1King}{10}{6}{2Ch 9:5,\allowbreak6}
\crossref{1King}{10}{7}{Isa 64:4 Zec 9:17 Mr 16:11 Joh 20:25-\allowbreak29 1Co 2:9 1Jo 3:2}
\crossref{1King}{10}{8}{2Ch 9:7,\allowbreak8 Pr 3:13,\allowbreak14; 8:34; 10:21; 13:20 Lu 10:39-\allowbreak42; 11:28,\allowbreak31}
\crossref{1King}{10}{9}{1Ki 5:7 Ps 72:17-\allowbreak19}
\crossref{1King}{10}{10}{10:2; 9:14 Ps 72:10,\allowbreak15 Mt 2:11}
\crossref{1King}{10}{11}{1Ki 9:27,\allowbreak28 2Ch 8:18 Ps 45:9}
\crossref{1King}{10}{12}{1Ch 23:5; 25:1-\allowbreak31 Ps 92:1-\allowbreak3; 150:3-\allowbreak5 Re 14:2,\allowbreak3}
\crossref{1King}{10}{13}{10:2; 9:1 Ps 20:4; 37:4 Mt 15:28 Joh 14:13,\allowbreak14 Eph 3:20}
\crossref{1King}{10}{14}{1Ki 9:28}
\crossref{1King}{10}{15}{1Ch 9:24 2Ch 9:13,\allowbreak14 Ps 72:10 Isa 21:13 Ga 4:25}
\crossref{1King}{10}{16}{}
\crossref{1King}{10}{17}{1Ki 7:2}
\crossref{1King}{10}{18}{2Ch 9:17-\allowbreak19 Ps 45:6; 110:1; 122:5 Heb 1:3,\allowbreak8 Re 20:11}
\crossref{1King}{10}{19}{}
\crossref{1King}{10}{20}{Ge 49:9 Nu 23:24; 24:9 Re 5:5}
\crossref{1King}{10}{21}{2Ch 9:20-\allowbreak22}
\crossref{1King}{10}{22}{1Ki 22:48 Ge 10:4 2Ch 9:21; 20:36,\allowbreak37 Ps 48:7; 72:10 Isa 2:16}
\crossref{1King}{10}{23}{}
\crossref{1King}{10}{24}{1Ki 3:9,\allowbreak12,\allowbreak28 Pr 2:6 Da 1:17; 2:21,\allowbreak23; 5:11 Jas 1:5}
\crossref{1King}{10}{25}{10:10 Jud 3:15 1Sa 10:27 2Sa 8:2,\allowbreak10 2Ch 26:8 Job 42:11}
\crossref{1King}{10}{26}{1Ki 4:26 De 17:16 2Ch 1:14; 9:25 Isa 2:7}
\crossref{1King}{10}{27}{2Ch 1:15-\allowbreak17; 9:27 Job 22:24,\allowbreak25}
\crossref{1King}{10}{28}{Ge 41:42 Pr 7:16 Isa 19:9 Eze 27:7}
\crossref{1King}{10}{29}{Jos 1:4 2Ki 7:6}
\crossref{1King}{11}{1}{11:8 Ge 6:2-\allowbreak5 De 17:17 Ne 13:23-\allowbreak27 Pr 2:16; 5:8-\allowbreak20; 6:24; 7:5}
\crossref{1King}{11}{2}{Ex 23:32,\allowbreak33; 34:16 De 7:3,\allowbreak4 Jos 23:12,\allowbreak13 Ezr 9:12; 10:2-\allowbreak17}
\crossref{1King}{11}{3}{Jud 8:30,\allowbreak31; 9:5 2Sa 3:2-\allowbreak5; 5:13-\allowbreak16 2Ch 11:21 Ec 7:28}
\crossref{1King}{11}{4}{11:42; 6:1; 9:10; 14:21}
\crossref{1King}{11}{5}{11:33 Jud 2:13; 10:6 1Sa 7:3,\allowbreak4; 12:10 2Ki 23:13 Jer 2:10-\allowbreak13}
\crossref{1King}{11}{6}{Nu 14:24 Jos 14:8,\allowbreak14}
\crossref{1King}{11}{7}{Le 26:30 Nu 33:52 2Ki 21:2,\allowbreak3; 23:13,\allowbreak14 Ps 78:58 Eze 20:28,\allowbreak29}
\crossref{1King}{11}{8}{11:1 Eze 16:22-\allowbreak29 Ho 4:11,\allowbreak12 1Co 10:11,\allowbreak12,\allowbreak20-\allowbreak22}
\crossref{1King}{11}{9}{Ex 4:14 Nu 12:9 De 3:26; 9:8,\allowbreak20 2Sa 6:7; 11:27 1Ch 21:7}
\crossref{1King}{11}{10}{1Ki 6:12,\allowbreak13; 9:4-\allowbreak7 2Ch 7:17-\allowbreak22}
\crossref{1King}{11}{11}{Isa 29:13,\allowbreak14}
\crossref{1King}{11}{12}{1Ki 21:29 2Ki 20:17,\allowbreak19; 22:19,\allowbreak20}
\crossref{1King}{11}{13}{11:39 2Sa 7:15,\allowbreak16 1Ch 17:13,\allowbreak14 Ps 89:33-\allowbreak37}
\crossref{1King}{11}{14}{1Ki 12:15 1Sa 26:19 2Sa 24:1 1Ch 5:26 Isa 10:5,\allowbreak26; 13:17}
\crossref{1King}{11}{15}{2Sa 8:14 1Ch 18:12,\allowbreak13 Ps 60:1}
\crossref{1King}{11}{16}{}
\crossref{1King}{11}{17}{Ex 2:1-\allowbreak10 2Sa 4:4 2Ki 11:2 Mt 2:13,\allowbreak14}
\crossref{1King}{11}{18}{}
\crossref{1King}{11}{19}{Ge 39:4,\allowbreak21 Ac 7:10,\allowbreak21}
\crossref{1King}{11}{20}{Ge 21:7 1Sa 1:24}
\crossref{1King}{11}{21}{1Ki 2:10,\allowbreak34 Ex 4:19 Mt 2:20}
\crossref{1King}{11}{22}{Jer 2:31 Lu 22:35}
\crossref{1King}{11}{23}{11:14 2Sa 16:11 Ezr 1:1 Isa 13:17; 37:26; 45:5 Eze 38:16}
\crossref{1King}{11}{24}{1Ki 19:15; 20:34 Ge 14:15 Ac 9:2}
\crossref{1King}{11}{25}{1Ki 5:4 2Ch 15:2}
\crossref{1King}{11}{26}{11:11,\allowbreak28; 12:2,\allowbreak20-\allowbreak24; 13:1-\allowbreak10; 14:16; 15:30; 16:3; 21:22}
\crossref{1King}{11}{27}{2Sa 20:21 Pr 30:32 Isa 26:11}
\crossref{1King}{11}{28}{Pr 22:29}
\crossref{1King}{11}{29}{1Ki 12:15; 14:2 2Ch 9:29}
\crossref{1King}{11}{30}{1Sa 15:27,\allowbreak28; 24:4,\allowbreak5}
\crossref{1King}{11}{31}{11:11,\allowbreak12}
\crossref{1King}{11}{32}{1Ki 12:20}
\crossref{1King}{11}{33}{11:9; 3:14; 6:12,\allowbreak13; 9:5-\allowbreak7 1Ch 28:9 2Ch 15:2 Jer 2:13 Ho 4:17}
\crossref{1King}{11}{34}{11:12,\allowbreak13,\allowbreak31 Job 11:6 Ps 103:10 Hab 3:2}
\crossref{1King}{11}{35}{Ex 20:5,\allowbreak6}
\crossref{1King}{11}{36}{1Ki 15:4 2Sa 7:16,\allowbreak29; 21:17 2Ki 8:19 2Ch 21:7 Ps 132:17}
\crossref{1King}{11}{37}{11:26 De 14:26 2Sa 3:21}
\crossref{1King}{11}{38}{1Ki 3:14; 6:12; 9:4,\allowbreak5 Ex 19:5 De 15:5 Zec 3:7}
\crossref{1King}{11}{39}{1Ki 12:16; 14:8,\allowbreak25,\allowbreak26 Ps 89:38-\allowbreak45,\allowbreak49-\allowbreak51}
\crossref{1King}{11}{40}{2Ch 16:10 Pr 21:30 Isa 14:24-\allowbreak27; 46:10 La 3:37}
\crossref{1King}{11}{41}{2Ch 9:29-\allowbreak31}
\crossref{1King}{11}{42}{}
\crossref{1King}{11}{43}{1Ki 1:21; 14:20; 15:8,\allowbreak24; 16:6 De 31:16 2Ki 16:20; 20:21; 21:18}
\crossref{1King}{12}{1}{1Ki 11:43 2Ch 10:1-\allowbreak19}
\crossref{1King}{12}{2}{1Ki 11:26-\allowbreak31,\allowbreak40 2Ch 10:2,\allowbreak3}
\crossref{1King}{12}{3}{}
\crossref{1King}{12}{4}{1Ki 4:7,\allowbreak20,\allowbreak22,\allowbreak23,\allowbreak25; 9:15,\allowbreak22,\allowbreak23 1Sa 8:11-\allowbreak18 2Ch 10:4,\allowbreak5 Mt 11:29,\allowbreak30}
\crossref{1King}{12}{5}{}
\crossref{1King}{12}{6}{2Sa 16:20; 17:5 Job 12:12; 32:7 Pr 27:10 Jer 42:2-\allowbreak5; 43:2}
\crossref{1King}{12}{7}{2Ch 10:6,\allowbreak7 Pr 15:1 Mr 10:43,\allowbreak44 Php 2:7-\allowbreak11}
\crossref{1King}{12}{8}{2Ch 10:8; 25:15,\allowbreak16 Pr 1:2-\allowbreak5,\allowbreak25,\allowbreak30; 19:20; 25:12 Ec 10:2,\allowbreak3}
\crossref{1King}{12}{9}{1Ki 22:6-\allowbreak8 2Sa 17:5,\allowbreak6 2Ch 10:9; 18:5-\allowbreak7}
\crossref{1King}{12}{10}{2Sa 17:7-\allowbreak13}
\crossref{1King}{12}{11}{Ex 1:13,\allowbreak14; 5:5-\allowbreak9,\allowbreak18 1Sa 8:18 2Ch 16:10 Isa 58:6 Jer 27:11}
\crossref{1King}{12}{12}{12:5 2Ch 10:12-\allowbreak14}
\crossref{1King}{12}{13}{1Ki 20:6-\allowbreak11 Ge 42:7,\allowbreak30 Ex 5:2; 10:28 Jud 12:1-\allowbreak6 1Sa 20:10,\allowbreak30,\allowbreak31}
\crossref{1King}{12}{14}{2Ch 22:4,\allowbreak5 Es 1:16-\allowbreak21; 2:2-\allowbreak4 Pr 12:5 Isa 19:11-\allowbreak13 Da 6:7}
\crossref{1King}{12}{15}{1Ki 11:11,\allowbreak29-\allowbreak38 1Sa 15:29 2Sa 17:14 2Ki 9:36; 10:10 Isa 14:13-\allowbreak17}
\crossref{1King}{12}{16}{2Sa 20:1 2Ch 10:16}
\crossref{1King}{12}{17}{1Ki 11:13,\allowbreak36 2Ch 10:17; 11:13-\allowbreak17}
\crossref{1King}{12}{18}{1Ki 4:6; 5:14}
\crossref{1King}{12}{19}{1Sa 10:19 2Ki 17:21 2Ch 10:19; 13:5-\allowbreak7,\allowbreak17 Isa 7:17}
\crossref{1King}{12}{20}{1Sa 10:24 Ho 8:4}
\crossref{1King}{12}{21}{2Ch 11:1-\allowbreak3}
\crossref{1King}{12}{22}{2Ch 11:2; 12:5,\allowbreak7}
\crossref{1King}{12}{23}{}
\crossref{1King}{12}{24}{Nu 14:42 2Ch 11:4; 25:7,\allowbreak8; 28:9-\allowbreak13}
\crossref{1King}{12}{25}{1Ki 9:15,\allowbreak17,\allowbreak18; 15:17; 16:24 2Ch 11:5-\allowbreak12}
\crossref{1King}{12}{26}{Ps 14:1 Mr 2:6-\allowbreak8 Lu 7:39}
\crossref{1King}{12}{27}{1Ki 8:29,\allowbreak30,\allowbreak44; 11:32 De 12:5-\allowbreak7,\allowbreak14; 16:2,\allowbreak6}
\crossref{1King}{12}{28}{12:8,\allowbreak9 Ex 1:10 Isa 30:1}
\crossref{1King}{12}{29}{Ge 12:8; 28:19; 35:1 Ho 4:15}
\crossref{1King}{12}{30}{1Ki 13:34 2Ki 10:31; 17:21}
\crossref{1King}{12}{31}{1Ki 13:24,\allowbreak32 De 24:15 Eze 16:25 Ho 12:11}
\crossref{1King}{12}{32}{1Ki 8:2,\allowbreak5 Le 23:33,\allowbreak34-\allowbreak44 Nu 29:12-\allowbreak40 Eze 43:8 Mt 15:8,\allowbreak9}
\crossref{1King}{12}{33}{12:32}
\crossref{1King}{13}{1}{1Ki 12:22 2Ki 23:17 2Ch 9:29}
\crossref{1King}{13}{2}{De 32:1 Isa 1:2; 58:1 Jer 22:29 Eze 36:1,\allowbreak4; 38:4 Lu 19:40}
\crossref{1King}{13}{3}{Ex 4:3-\allowbreak5,\allowbreak8,\allowbreak9; 7:10 De 13:1-\allowbreak3 1Sa 2:34 2Ki 20:8 Isa 7:11-\allowbreak14}
\crossref{1King}{13}{4}{2Ch 16:10; 18:25-\allowbreak27; 25:15,\allowbreak16 Ps 105:15 Jer 20:2-\allowbreak4}
\crossref{1King}{13}{5}{13:3; 22:28,\allowbreak35 Ex 9:18-\allowbreak25 Nu 16:23-\allowbreak35 De 18:22 Jer 28:16,\allowbreak17}
\crossref{1King}{13}{6}{Ex 8:8,\allowbreak28; 9:28; 10:17; 12:32 Nu 21:7 1Sa 12:19 Jer 37:3; 42:2-\allowbreak4}
\crossref{1King}{13}{7}{Ge 18:5 Jud 13:15; 19:21}
\crossref{1King}{13}{8}{Nu 22:18; 24:13 Ex 5:3,\allowbreak6; 7:2 Mr 6:23}
\crossref{1King}{13}{9}{13:1,\allowbreak21,\allowbreak22 1Sa 15:22 Job 23:12 Joh 13:17; 15:9,\allowbreak10,\allowbreak14}
\crossref{1King}{13}{10}{}
\crossref{1King}{13}{11}{13:20,\allowbreak21 Nu 23:4,\allowbreak5; 24:2 1Sa 10:11 2Ki 23:18 Eze 13:2,\allowbreak16}
\crossref{1King}{13}{12}{13:12}
\crossref{1King}{13}{13}{13:27 Nu 22:21 Jud 5:10; 10:4 2Sa 19:26}
\crossref{1King}{13}{14}{1Ki 19:4 Joh 4:6,\allowbreak34 1Co 4:11,\allowbreak12 2Co 11:27 Php 4:12,\allowbreak13}
\crossref{1King}{13}{15}{13:15}
\crossref{1King}{13}{16}{13:8,\allowbreak9 Ge 3:1-\allowbreak3 Nu 22:13,\allowbreak19 Mt 4:10; 16:23}
\crossref{1King}{13}{17}{13:1; 20:35 1Th 4:15}
\crossref{1King}{13}{18}{Nu 22:35 Jud 6:11,\allowbreak12; 13:3}
\crossref{1King}{13}{19}{13:9 Ge 3:6 De 13:1,\allowbreak3,\allowbreak5; 18:20 Ac 4:19 2Pe 2:18,\allowbreak19}
\crossref{1King}{13}{20}{}
\crossref{1King}{13}{21}{13:17 Ge 3:7 Es 6:13 Jer 2:19 Ga 1:8,\allowbreak9}
\crossref{1King}{13}{22}{13:19}
\crossref{1King}{13}{23}{}
\crossref{1King}{13}{24}{1Ki 20:36 2Ki 2:24 Pr 22:13; 26:13 Am 5:19 1Co 11:31,\allowbreak32}
\crossref{1King}{13}{25}{}
\crossref{1King}{13}{26}{Le 10:3 2Sa 12:10,\allowbreak14 Ps 119:120 Pr 11:31 Eze 9:6 1Co 11:30}
\crossref{1King}{13}{27}{}
\crossref{1King}{13}{28}{}
\crossref{1King}{13}{29}{}
\crossref{1King}{13}{30}{1Ki 14:13 Jer 22:18 Ac 8:2}
\crossref{1King}{13}{31}{Nu 23:10 Ps 26:9 Ec 8:10 Lu 16:22,\allowbreak23}
\crossref{1King}{13}{32}{13:2 2Ki 23:16-\allowbreak19}
\crossref{1King}{13}{33}{1Ki 12:31-\allowbreak33 2Ch 11:15; 13:9 Am 6:11}
\crossref{1King}{13}{34}{1Ki 12:30 2Ki 10:31; 17:21}
\crossref{1King}{14}{1}{1Ki 13:33,\allowbreak34}
\crossref{1King}{14}{2}{14:5,\allowbreak6; 22:30 1Sa 28:8 2Sa 14:2 2Ch 18:29 Lu 12:2}
\crossref{1King}{14}{3}{1Ki 13:7 1Sa 9:7,\allowbreak8 2Ki 4:42; 5:5,\allowbreak15; 8:7-\allowbreak9}
\crossref{1King}{14}{4}{1Ki 11:29 Jos 18:1 1Sa 4:3,\allowbreak4 Jer 7:12-\allowbreak14}
\crossref{1King}{14}{5}{2Ki 4:27; 6:8-\allowbreak12 Ps 139:1-\allowbreak4 Pr 21:30 Am 3:7 Ac 10:19,\allowbreak20}
\crossref{1King}{14}{6}{Job 5:13 Ps 33:10}
\crossref{1King}{14}{7}{1Ki 12:24; 16:2 1Sa 2:27-\allowbreak30; 15:16 2Sa 12:7,\allowbreak8}
\crossref{1King}{14}{8}{1Ki 11:30,\allowbreak31}
\crossref{1King}{14}{9}{14:16; 12:28; 13:33,\allowbreak34; 15:34; 16:31}
\crossref{1King}{14}{10}{1Ki 15:25-\allowbreak30 Am 3:6}
\crossref{1King}{14}{11}{1Ki 16:4; 21:19,\allowbreak23,\allowbreak24 Isa 66:24 Jer 15:3 Eze 39:17-\allowbreak19 Re 19:17,\allowbreak18}
\crossref{1King}{14}{12}{14:3,\allowbreak16,\allowbreak17 2Ki 1:6,\allowbreak16 Joh 4:50-\allowbreak52}
\crossref{1King}{14}{13}{Nu 20:29 Jer 22:10,\allowbreak18}
\crossref{1King}{14}{14}{1Ki 15:27-\allowbreak29}
\crossref{1King}{14}{15}{1Sa 12:25 2Ki 17:6,\allowbreak7}
\crossref{1King}{14}{16}{Ps 81:12 Isa 40:24 Ho 9:11,\allowbreak12,\allowbreak16,\allowbreak17}
\crossref{1King}{14}{17}{14:12,\allowbreak13 1Sa 2:20-\allowbreak34; 4:18-\allowbreak20}
\crossref{1King}{14}{18}{}
\crossref{1King}{14}{19}{14:30 2Ch 13:2-\allowbreak20}
\crossref{1King}{14}{20}{1Ki 2:10; 11:43 Job 14:12 Ps 3:5; 4:8}
\crossref{1King}{14}{21}{1Ki 11:43 2Ch 12:13; 13:7}
\crossref{1King}{14}{22}{Jud 3:7,\allowbreak12; 4:1 2Ki 17:19 2Ch 12:1 Jer 3:7-\allowbreak11}
\crossref{1King}{14}{23}{1Ki 3:2 De 12:2 Isa 57:5 Eze 16:24,\allowbreak25; 20:28,\allowbreak29}
\crossref{1King}{14}{24}{1Ki 15:12; 22:46 Ge 19:5 De 23:17 Jud 19:22 2Ki 23:7 Ro 1:24-\allowbreak27}
\crossref{1King}{14}{25}{1Ki 11:40 2Ch 12:2-\allowbreak4}
\crossref{1King}{14}{26}{1Ki 7:51; 15:18 2Ki 24:13 2Ch 12:9-\allowbreak11 Ps 39:6; 89:35-\allowbreak45}
\crossref{1King}{14}{27}{La 4:1,\allowbreak2}
\crossref{1King}{14}{28}{2Ch 12:11}
\crossref{1King}{14}{29}{}
\crossref{1King}{14}{30}{1Ki 12:24; 15:6,\allowbreak7 2Ch 12:15}
\crossref{1King}{14}{31}{14:20; 11:43; 15:3,\allowbreak24; 22:50 2Ch 12:16}
\crossref{1King}{15}{1}{1Ki 14:31 2Ch 13:1,\allowbreak2-\allowbreak22}
\crossref{1King}{15}{2}{15:13 2Ch 11:20-\allowbreak22}
\crossref{1King}{15}{3}{1Ki 14:21,\allowbreak22}
\crossref{1King}{15}{4}{1Ki 11:12,\allowbreak32 Ge 12:2; 19:29; 26:5 De 4:37 2Sa 7:12-\allowbreak16 Isa 37:35}
\crossref{1King}{15}{5}{15:3; 14:8 2Ki 22:2 2Ch 34:2 Ps 119:6 Lu 1:6 Ac 13:22,\allowbreak36}
\crossref{1King}{15}{6}{1Ki 14:30}
\crossref{1King}{15}{7}{1Ki 14:29 2Ch 13:2,\allowbreak21,\allowbreak22}
\crossref{1King}{15}{8}{1Ki 14:1,\allowbreak31 2Ch 14:1}
\crossref{1King}{15}{9}{}
\crossref{1King}{15}{10}{}
\crossref{1King}{15}{11}{15:3 2Ch 14:2,\allowbreak11; 15:17; 16:7-\allowbreak10}
\crossref{1King}{15}{12}{1Ki 14:24; 22:46 Ro 1:26,\allowbreak27 Jude 1:7}
\crossref{1King}{15}{13}{15:2,\allowbreak10 2Ch 15:15,\allowbreak16-\allowbreak19}
\crossref{1King}{15}{14}{1Ki 22:43 2Ki 12:3; 14:4; 15:4 2Ch 14:3,\allowbreak5}
\crossref{1King}{15}{15}{1Ki 7:51 1Ch 26:26-\allowbreak28 2Ch 14:13; 15:18}
\crossref{1King}{15}{16}{15:6,\allowbreak7,\allowbreak32; 14:30 2Ch 16:1-\allowbreak6}
\crossref{1King}{15}{17}{15:27 2Ch 16:1-\allowbreak6}
\crossref{1King}{15}{18}{15:15; 14:26 2Ki 12:18; 18:15,\allowbreak16 2Ch 15:18; 16:2-\allowbreak6}
\crossref{1King}{15}{19}{2Ch 19:2 Isa 31:1}
\crossref{1King}{15}{20}{1Ki 12:29 Ge 14:14 Jud 18:29}
\crossref{1King}{15}{21}{2Ch 16:5}
\crossref{1King}{15}{22}{2Ch 16:6}
\crossref{1King}{15}{23}{15:7,\allowbreak8; 14:29-\allowbreak31}
\crossref{1King}{15}{24}{1Ki 22:41-\allowbreak43 2Ch 17:1-\allowbreak9 Mt 1:8}
\crossref{1King}{15}{25}{1Ki 14:12,\allowbreak20}
\crossref{1King}{15}{26}{1Ki 16:7,\allowbreak25,\allowbreak30}
\crossref{1King}{15}{27}{15:16,\allowbreak17; 14:14}
\crossref{1King}{15}{28}{De 32:35}
\crossref{1King}{15}{29}{1Ki 14:9-\allowbreak16 2Ki 9:7-\allowbreak10,\allowbreak36,\allowbreak37; 10:10,\allowbreak11,\allowbreak31; 19:25}
\crossref{1King}{15}{30}{15:26; 14:9-\allowbreak16}
\crossref{1King}{15}{31}{1Ki 14:19; 16:5,\allowbreak14,\allowbreak20,\allowbreak27}
\crossref{1King}{15}{32}{}
\crossref{1King}{15}{33}{1Ki 16:8}
\crossref{1King}{15}{34}{15:26}
\crossref{1King}{16}{1}{16:7 2Ch 19:2; 20:34}
\crossref{1King}{16}{2}{1Ki 14:7 1Sa 2:8,\allowbreak27,\allowbreak28; 15:17-\allowbreak19 2Sa 12:7-\allowbreak11 Ps 113:7,\allowbreak8 Lu 1:52}
\crossref{1King}{16}{3}{16:11,\allowbreak12; 14:10; 15:29,\allowbreak30; 21:21-\allowbreak24 Isa 66:24 Jer 22:19}
\crossref{1King}{16}{4}{1Ki 14:11}
\crossref{1King}{16}{5}{1Ki 14:19; 15:31 2Ch 16:1-\allowbreak6}
\crossref{1King}{16}{6}{1Ki 14:20; 15:25}
\crossref{1King}{16}{7}{16:1,\allowbreak2}
\crossref{1King}{16}{8}{}
\crossref{1King}{16}{9}{2Ki 9:31}
\crossref{1King}{16}{10}{2Ki 9:31}
\crossref{1King}{16}{11}{1Ki 15:29 Jud 1:7}
\crossref{1King}{16}{12}{16:1-\allowbreak4}
\crossref{1King}{16}{13}{1Ki 15:30}
\crossref{1King}{16}{14}{16:5}
\crossref{1King}{16}{15}{16:8 2Ki 9:31 Job 20:5 Ps 37:35}
\crossref{1King}{16}{16}{16:30 2Ki 8:26 2Ch 22:2 Mic 6:16}
\crossref{1King}{16}{17}{Jud 9:45,\allowbreak50,\allowbreak56,\allowbreak57 2Ki 6:24,\allowbreak25; 18:9-\allowbreak12; 25:1-\allowbreak4 Lu 19:43,\allowbreak44}
\crossref{1King}{16}{18}{Jud 9:54 1Sa 31:4,\allowbreak5 2Sa 17:23 Job 2:9,\allowbreak10 Mt 27:5}
\crossref{1King}{16}{19}{16:7,\allowbreak13; 15:30 Ps 9:16; 58:9-\allowbreak11}
\crossref{1King}{16}{20}{16:5,\allowbreak14,\allowbreak27; 14:19; 15:31; 22:39}
\crossref{1King}{16}{21}{16:8,\allowbreak29; 15:25,\allowbreak28 Pr 28:2 Isa 9:18-\allowbreak21; 19:2 Mt 12:25}
\crossref{1King}{16}{22}{}
\crossref{1King}{16}{23}{16:8,\allowbreak29}
\crossref{1King}{16}{24}{1Ki 13:32; 18:2; 20:1; 22:37 2Ki 17:1,\allowbreak6,\allowbreak24 Joh 4:4,\allowbreak5 Ac 8:5-\allowbreak8}
\crossref{1King}{16}{25}{16:30,\allowbreak31,\allowbreak33; 14:9 Mic 6:16}
\crossref{1King}{16}{26}{16:2,\allowbreak7,\allowbreak19; 12:26-\allowbreak33; 13:33,\allowbreak34}
\crossref{1King}{16}{27}{16:5,\allowbreak14,\allowbreak20; 15:31}
\crossref{1King}{16}{28}{16:6}
\crossref{1King}{16}{29}{16:24}
\crossref{1King}{16}{30}{16:25,\allowbreak31,\allowbreak33; 14:9; 21:25 2Ki 3:2}
\crossref{1King}{16}{31}{Ge 30:15 Nu 16:9 Isa 7:13 Eze 8:17; 16:20,\allowbreak47; 34:18}
\crossref{1King}{16}{32}{2Ki 10:21,\allowbreak26,\allowbreak27}
\crossref{1King}{16}{33}{Ex 34:13 2Ki 13:6; 17:16; 21:3 Jer 17:1,\allowbreak2}
\crossref{1King}{16}{34}{Jos 6:26; 23:14,\allowbreak15 Zec 1:5 Mt 24:35}
\crossref{1King}{17}{1}{Mt 11:14; 16:14; 27:47,\allowbreak49 Lu 1:17; 4:25,\allowbreak26; 9:30,\allowbreak33,\allowbreak54}
\crossref{1King}{17}{2}{1Ki 12:22 1Ch 17:3 Jer 7:1; 11:1; 18:1 Ho 1:1,\allowbreak2}
\crossref{1King}{17}{3}{1Ki 22:25 Ps 31:20; 83:3 Jer 36:19,\allowbreak26 Joh 8:59 Ac 17:14 Heb 11:38}
\crossref{1King}{17}{4}{17:9; 19:5-\allowbreak8 Nu 20:8 Job 34:29; 38:8-\allowbreak13,\allowbreak41 Ps 33:8,\allowbreak9; 147:9}
\crossref{1King}{17}{5}{1Ki 19:9 Pr 3:5 Mt 16:24 Joh 15:14}
\crossref{1King}{17}{6}{Ex 16:35 Nu 11:23 Jud 14:14; 15:18,\allowbreak19 Ps 34:9,\allowbreak10; 37:3,\allowbreak19}
\crossref{1King}{17}{7}{}
\crossref{1King}{17}{8}{17:2 Ge 22:14 Isa 41:17 Heb 13:6}
\crossref{1King}{17}{9}{Ob 1:20 Lu 4:26}
\crossref{1King}{17}{10}{Ge 21:15; 24:17 Joh 4:7 2Co 11:27 Heb 11:37}
\crossref{1King}{17}{11}{Ge 24:18,\allowbreak19 Mt 10:42; 25:35-\allowbreak40 Heb 13:2}
\crossref{1King}{17}{12}{17:1 1Sa 14:39,\allowbreak45; 20:3,\allowbreak21; 25:26; 26:10 2Sa 15:21 Jer 4:2; 5:2}
\crossref{1King}{17}{13}{Ex 14:13 2Ki 6:16 2Ch 20:17 Isa 41:10,\allowbreak13 Mt 28:5 Ac 27:24}
\crossref{1King}{17}{14}{2Ki 3:16; 7:1; 9:6}
\crossref{1King}{17}{15}{Ge 6:22; 12:4; 22:3 2Ch 20:20 Mt 15:28 Mr 12:43 Joh 11:40}
\crossref{1King}{17}{16}{Mt 9:28-\allowbreak30; 19:26 Lu 1:37,\allowbreak45 Joh 4:50,\allowbreak51}
\crossref{1King}{17}{17}{Ge 22:1,\allowbreak2 2Ki 4:18-\allowbreak20 Zec 12:10 Joh 11:3,\allowbreak4,\allowbreak14 Jas 1:2-\allowbreak4,\allowbreak12}
\crossref{1King}{17}{18}{2Sa 16:10; 19:22 2Ki 3:13 2Ch 35:21 Lu 4:34; 5:8; 8:28 Joh 2:4}
\crossref{1King}{17}{19}{2Ki 4:10,\allowbreak21,\allowbreak32 Ac 9:37}
\crossref{1King}{17}{20}{1Ki 18:36,\allowbreak37 Ex 17:4 1Sa 7:8,\allowbreak9 2Ki 19:4,\allowbreak15 Ps 99:6 Mt 21:22}
\crossref{1King}{17}{21}{2Ki 4:33-\allowbreak35 Ac 10:10}
\crossref{1King}{17}{22}{De 32:39 1Sa 2:6 2Ki 13:21 Lu 8:54 Joh 5:28,\allowbreak29; 11:43 Ac 20:12}
\crossref{1King}{17}{23}{2Ki 4:36,\allowbreak37 Lu 7:15 Ac 9:41 Heb 11:35}
\crossref{1King}{17}{24}{Joh 2:11; 3:2; 4:42-\allowbreak48; 11:15,\allowbreak42; 15:24; 16:30}
\crossref{1King}{18}{1}{Lu 4:25 Jas 5:17 Re 11:2,\allowbreak6}
\crossref{1King}{18}{2}{Ps 27:1; 51:4 Pr 28:1 Isa 51:12 Heb 13:5,\allowbreak6}
\crossref{1King}{18}{3}{18:12 Ge 22:12; 42:18 2Ki 4:1 Ne 5:15; 7:2 Pr 14:26 Mal 3:16}
\crossref{1King}{18}{4}{Ne 9:26 Mt 21:35 Re 17:4-\allowbreak6}
\crossref{1King}{18}{5}{Ps 104:14 Jer 14:5,\allowbreak6 Joe 1:18; 2:22 Hab 3:17 Ro 8:20-\allowbreak22}
\crossref{1King}{18}{6}{Jer 14:3}
\crossref{1King}{18}{7}{1Ki 11:29}
\crossref{1King}{18}{8}{18:3 Ro 13:7 1Pe 2:17,\allowbreak18}
\crossref{1King}{18}{9}{18:12}
\crossref{1King}{18}{10}{18:15; 1:29; 2:24; 17:1,\allowbreak12 1Sa 29:6}
\crossref{1King}{18}{11}{18:8,\allowbreak14}
\crossref{1King}{18}{12}{2Ki 2:11,\allowbreak16 Eze 3:12-\allowbreak14; 8:3; 11:24; 37:1; 40:1,\allowbreak2 Mt 4:1 Ac 8:39}
\crossref{1King}{18}{13}{18:4 Ge 20:4,\allowbreak5 Ps 18:21-\allowbreak24 Ac 20:34 1Th 2:9,\allowbreak10}
\crossref{1King}{18}{14}{Mt 10:28}
\crossref{1King}{18}{15}{18:10 Heb 6:16,\allowbreak17}
\crossref{1King}{18}{16}{}
\crossref{1King}{18}{17}{1Ki 21:20 Jos 7:25 Jer 26:8,\allowbreak9; 38:4 Am 7:10 Ac 16:20; 17:6; 24:5}
\crossref{1King}{18}{18}{Eze 3:8 Mt 14:4 Ac 24:13,\allowbreak20}
\crossref{1King}{18}{19}{1Ki 22:6 2Pe 2:1 Re 19:20}
\crossref{1King}{18}{20}{1Ki 22:9}
\crossref{1King}{18}{21}{De 4:35 2Ki 17:41 Zep 1:5 Mt 6:24 Lu 6:13 Ro 6:16-\allowbreak22}
\crossref{1King}{18}{22}{1Ki 19:10,\allowbreak14; 20:13,\allowbreak32,\allowbreak35,\allowbreak38; 22:6-\allowbreak8 Ro 11:3}
\crossref{1King}{18}{23}{}
\crossref{1King}{18}{24}{18:38 Le 9:24 Jud 6:21 1Ch 21:26 2Ch 7:1,\allowbreak3}
\crossref{1King}{18}{25}{}
\crossref{1King}{18}{26}{Mt 6:7}
\crossref{1King}{18}{27}{1Ki 22:15 2Ch 25:8 Ec 11:9 Isa 8:9,\allowbreak10; 44:15-\allowbreak17 Eze 20:39}
\crossref{1King}{18}{28}{Le 19:28 De 14:1 Mic 6:7 Mr 5:5; 9:22}
\crossref{1King}{18}{29}{1Ki 22:10,\allowbreak12 1Sa 18:10 Jer 28:6-\allowbreak9 Ac 16:16,\allowbreak17 1Co 11:4,\allowbreak5}
\crossref{1King}{18}{30}{1Ki 19:10,\allowbreak14 2Ch 33:16 Ro 11:3}
\crossref{1King}{18}{31}{Ex 24:4 Jos 4:3,\allowbreak4,\allowbreak20 Ezr 6:17 Jer 31:1 Eze 37:16-\allowbreak22; 47:13}
\crossref{1King}{18}{32}{Ex 20:24,\allowbreak25 Jud 6:26; 21:4 1Sa 7:9,\allowbreak17}
\crossref{1King}{18}{33}{Ge 22:9 Le 1:6-\allowbreak8}
\crossref{1King}{18}{34}{2Co 4:2; 8:21}
\crossref{1King}{18}{35}{18:32,\allowbreak38}
\crossref{1King}{18}{36}{18:29 Ex 29:39-\allowbreak41 Ezr 9:4,\allowbreak5 Ps 141:2 Da 8:13; 9:21; 12:11 Ac 3:1}
\crossref{1King}{18}{37}{18:24,\allowbreak29,\allowbreak36 Ge 32:24,\allowbreak26,\allowbreak28 2Ch 14:11; 32:19,\allowbreak20 Isa 37:17-\allowbreak20}
\crossref{1King}{18}{38}{Ge 15:17 Le 9:24 Jud 6:21 1Ch 21:26 2Ch 7:1}
\crossref{1King}{18}{39}{Jud 13:20 1Ch 21:16 2Ch 7:3}
\crossref{1King}{18}{40}{2Ki 10:25}
\crossref{1King}{18}{41}{Ec 9:7 Ac 27:34}
\crossref{1King}{18}{42}{18:19 Mt 14:23 Lu 6:12 Ac 10:9}
\crossref{1King}{18}{43}{Ps 5:3 Lu 18:1}
\crossref{1King}{18}{44}{1Sa 6:7,\allowbreak10 Mic 1:13}
\crossref{1King}{18}{45}{18:39,\allowbreak40 Nu 25:8 2Sa 21:14}
\crossref{1King}{18}{46}{2Ki 3:15 Isa 8:11 Eze 1:3; 3:14}
\crossref{1King}{19}{1}{1Ki 16:31; 21:5-\allowbreak7,\allowbreak25}
\crossref{1King}{19}{2}{1Ki 2:28; 20:10 Ru 1:17 2Ki 6:31}
\crossref{1King}{19}{3}{Ge 12:12,\allowbreak13 Ex 2:15 1Sa 27:1 Isa 51:12,\allowbreak13 Mt 26:56,\allowbreak70-\allowbreak74}
\crossref{1King}{19}{4}{1Ki 13:14 Ge 21:15,\allowbreak16 Joh 4:6}
\crossref{1King}{19}{5}{Ge 28:11-\allowbreak15}
\crossref{1King}{19}{6}{1Ki 17:6,\allowbreak9-\allowbreak15 Ps 37:3 Isa 33:16 Mt 4:11; 6:32 Mr 8:2,\allowbreak3 Joh 21:5,\allowbreak9}
\crossref{1King}{19}{7}{19:5}
\crossref{1King}{19}{8}{Da 1:15 2Co 12:9}
\crossref{1King}{19}{9}{Ex 33:21,\allowbreak22 Jer 9:2 Heb 11:38}
\crossref{1King}{19}{10}{Ex 20:5; 34:14 Nu 25:11,\allowbreak13 Ps 69:9; 119:139 Joh 2:17}
\crossref{1King}{19}{11}{Ex 19:20; 24:12,\allowbreak18; 34:2 Mt 17:1-\allowbreak3 2Pe 1:17,\allowbreak18}
\crossref{1King}{19}{12}{1Ki 18:38 Ge 15:17 Ex 3:2 De 4:11,\allowbreak12,\allowbreak33 2Ki 1:10; 2:11 Heb 12:29}
\crossref{1King}{19}{13}{19:9 Ge 16:8 Joh 21:15-\allowbreak17}
\crossref{1King}{19}{14}{19:9,\allowbreak10 Isa 62:1,\allowbreak6,\allowbreak7}
\crossref{1King}{19}{15}{Isa 45:1 Jer 1:10; 27:2-\allowbreak22}
\crossref{1King}{19}{16}{2Ki 9:1-\allowbreak3,\allowbreak6-\allowbreak14}
\crossref{1King}{19}{17}{Isa 24:17,\allowbreak18 Am 2:14; 5:19}
\crossref{1King}{19}{18}{Isa 1:9; 10:20-\allowbreak22 Ro 11:4,\allowbreak5}
\crossref{1King}{19}{19}{19:16}
\crossref{1King}{19}{20}{Mt 4:20,\allowbreak22; 9:9; 19:27}
\crossref{1King}{19}{21}{2Sa 24:22}
\crossref{1King}{20}{1}{1Ki 15:18,\allowbreak20 2Ki 8:7-\allowbreak10 2Ch 16:2-\allowbreak4 Jer 49:27 Am 1:4}
\crossref{1King}{20}{2}{2Ki 19:9 Isa 36:2-\allowbreak22; 37:9,\allowbreak10}
\crossref{1King}{20}{3}{Ex 15:9 Isa 10:13,\allowbreak14}
\crossref{1King}{20}{4}{Le 26:36 De 28:48 Jud 15:11-\allowbreak13 1Sa 13:6,\allowbreak7 2Ki 18:14-\allowbreak16}
\crossref{1King}{20}{5}{}
\crossref{1King}{20}{6}{1Sa 13:19-\allowbreak21 2Sa 24:14 2Ki 18:31,\allowbreak32}
\crossref{1King}{20}{7}{1Ki 8:1 2Ki 5:7 1Ch 13:1; 28:1 Pr 11:14}
\crossref{1King}{20}{8}{20:8}
\crossref{1King}{20}{9}{}
\crossref{1King}{20}{10}{1Ki 19:2 Ac 23:12}
\crossref{1King}{20}{11}{}
\crossref{1King}{20}{12}{20:16; 16:9 1Sa 25:36 2Sa 13:28 Pr 31:4,\allowbreak5 Da 5:2,\allowbreak30 Lu 21:34}
\crossref{1King}{20}{13}{2Ki 6:8-\allowbreak12; 7:1; 13:23 Isa 7:1-\allowbreak9 Eze 20:14,\allowbreak22}
\crossref{1King}{20}{14}{Ge 14:14-\allowbreak16 Jud 7:16-\allowbreak20 1Sa 17:50 1Co 1:27-\allowbreak29}
\crossref{1King}{20}{15}{Jud 7:7,\allowbreak16 1Sa 14:6 2Ch 14:11}
\crossref{1King}{20}{16}{20:11,\allowbreak12; 16:7 Pr 23:29-\allowbreak32 Ec 10:16,\allowbreak17 Ho 4:11}
\crossref{1King}{20}{17}{20:14,\allowbreak15,\allowbreak19}
\crossref{1King}{20}{18}{1Sa 2:3,\allowbreak4; 14:11,\allowbreak12; 17:44 2Ki 14:8-\allowbreak12 Pr 18:12}
\crossref{1King}{20}{19}{}
\crossref{1King}{20}{20}{2Sa 2:16 Ec 9:11}
\crossref{1King}{20}{21}{Jud 3:28; 7:23-\allowbreak25 1Sa 14:20-\allowbreak22; 17:52 2Ki 3:18,\allowbreak24}
\crossref{1King}{20}{22}{20:13,\allowbreak38; 19:10; 22:8 2Ki 6:12}
\crossref{1King}{20}{23}{}
\crossref{1King}{20}{24}{20:1,\allowbreak16; 22:31 Pr 21:30}
\crossref{1King}{20}{25}{Ps 10:3}
\crossref{1King}{20}{26}{1Sa 4:1; 29:1 2Ki 13:17}
\crossref{1King}{20}{27}{Jos 1:11 Jud 7:8}
\crossref{1King}{20}{28}{20:13,\allowbreak22; 13:1; 17:18 2Ch 20:14-\allowbreak20}
\crossref{1King}{20}{29}{Jos 6:15 1Sa 17:16 Ps 10:16}
\crossref{1King}{20}{30}{Ps 18:25}
\crossref{1King}{20}{31}{20:23 2Ki 5:13}
\crossref{1King}{20}{32}{20:3-\allowbreak6 Job 12:17,\allowbreak18; 40:11,\allowbreak12 Isa 2:11,\allowbreak12; 10:12 Da 5:20-\allowbreak23 Ob 1:3,\allowbreak4}
\crossref{1King}{20}{33}{Pr 25:13 Lu 16:8}
\crossref{1King}{20}{34}{1Ki 15:20 2Ch 16:4}
\crossref{1King}{20}{35}{20:38 1Sa 10:12 2Ki 2:3,\allowbreak5,\allowbreak7,\allowbreak15; 4:1,\allowbreak38 Am 7:14}
\crossref{1King}{20}{36}{1Ki 13:21-\allowbreak24,\allowbreak26 1Sa 15:22,\allowbreak23}
\crossref{1King}{20}{37}{20:35 Ex 21:12}
\crossref{1King}{20}{38}{1Ki 14:2; 22:30 2Sa 14:2 Mt 6:16}
\crossref{1King}{20}{39}{Jud 9:7-\allowbreak20 2Sa 12:1-\allowbreak7; 14:5-\allowbreak7 Mr 12:1-\allowbreak12}
\crossref{1King}{20}{40}{2Sa 12:5-\allowbreak7 Job 15:6 Mt 21:41-\allowbreak43; 25:24-\allowbreak27 Lu 19:22}
\crossref{1King}{20}{41}{20:38 2Sa 13:19 Job 2:8 Jer 6:26}
\crossref{1King}{20}{42}{20:34; 22:31-\allowbreak37 1Sa 15:9-\allowbreak11}
\crossref{1King}{20}{43}{1Ki 21:4; 22:8 Es 5:13; 6:12,\allowbreak13 Job 5:2 Pr 19:3}
\crossref{1King}{21}{1}{1Ki 20:35-\allowbreak43 2Ch 28:22 Ezr 9:13,\allowbreak14 Isa 9:13 Jer 5:3}
\crossref{1King}{21}{2}{2Ki 9:27 De 11:10 Ec 2:5 So 4:15}
\crossref{1King}{21}{3}{Ge 44:7,\allowbreak17 Jos 22:29; 24:16 1Sa 12:23; 24:6; 26:9-\allowbreak11 1Ch 11:19}
\crossref{1King}{21}{4}{1Ki 20:43 Job 5:2 Isa 57:20,\allowbreak21 Jon 4:1,\allowbreak9 Hab 2:9-\allowbreak12}
\crossref{1King}{21}{5}{21:25; 16:31; 18:4; 19:2 Ge 3:6}
\crossref{1King}{21}{6}{21:2 Es 5:9-\allowbreak14; 6:12 Pr 14:30 1Ti 6:9,\allowbreak10 Jas 4:2-\allowbreak7}
\crossref{1King}{21}{7}{1Sa 8:4 2Sa 13:4 Pr 30:31 Ec 4:1; 8:4 Da 5:19-\allowbreak21}
\crossref{1King}{21}{8}{2Sa 11:14,\allowbreak15 2Ch 32:17 Ezr 4:7,\allowbreak8,\allowbreak11 Ne 6:5 Es 3:12-\allowbreak15; 8:8-\allowbreak13}
\crossref{1King}{21}{9}{Ge 34:13-\allowbreak17 Isa 58:4 Mt 2:8; 23:14 Lu 20:47 Joh 18:28}
\crossref{1King}{21}{10}{De 19:15 Mt 26:59,\allowbreak60 Ac 6:11}
\crossref{1King}{21}{11}{Ex 1:17,\allowbreak21; 23:1,\allowbreak2 Le 19:15 1Sa 22:17; 23:20 2Ki 10:6,\allowbreak7}
\crossref{1King}{21}{12}{21:8-\allowbreak10 Isa 58:4}
\crossref{1King}{21}{13}{Ex 20:16 De 5:20; 19:16-\allowbreak21 Ps 27:12; 35:11 Pr 6:19; 19:5,\allowbreak9; 25:18}
\crossref{1King}{21}{14}{2Sa 11:14-\allowbreak24 Ec 5:8; 8:14}
\crossref{1King}{21}{15}{21:7 Pr 1:10-\allowbreak16; 4:17}
\crossref{1King}{21}{16}{2Sa 1:13-\allowbreak16; 4:9-\allowbreak12; 11:25-\allowbreak27; 23:15-\allowbreak17 Ps 50:18 Isa 33:15}
\crossref{1King}{21}{17}{2Ki 1:15,\allowbreak16; 5:26 Ps 9:12 Isa 26:21}
\crossref{1King}{21}{18}{1Ki 13:32 2Ch 22:9}
\crossref{1King}{21}{19}{Ge 3:11; 4:9,\allowbreak10 2Sa 12:9 Mic 3:1-\allowbreak4 Hab 2:9,\allowbreak12}
\crossref{1King}{21}{20}{1Ki 18:17; 22:8 2Ch 18:7,\allowbreak17 Am 5:10 Mr 12:12 Ga 4:16 Re 11:10}
\crossref{1King}{21}{21}{1Ki 14:10 Ex 20:5,\allowbreak6 2Ki 9:7-\allowbreak9; 10:1-\allowbreak7,\allowbreak11-\allowbreak14,\allowbreak17,\allowbreak30}
\crossref{1King}{21}{22}{1Ki 15:29; 16:3,\allowbreak11}
\crossref{1King}{21}{23}{21:25 2Ki 9:10,\allowbreak30-\allowbreak37}
\crossref{1King}{21}{24}{1Ki 14:11; 16:4 Isa 14:19 Jer 15:3 Eze 32:4,\allowbreak5; 39:18-\allowbreak20 Re 19:18}
\crossref{1King}{21}{25}{21:20; 16:30-\allowbreak33 2Ki 23:25}
\crossref{1King}{21}{26}{2Ch 15:8 Isa 65:4 Jer 16:18; 44:4 Eze 18:12 1Pe 4:3 Re 21:8}
\crossref{1King}{21}{27}{Ge 37:34 2Ki 6:30; 18:37 Jon 3:6}
\crossref{1King}{21}{28}{}
\crossref{1King}{21}{29}{Jer 7:17 Lu 7:44}
\crossref{1King}{22}{1}{1Ki 20:34}
\crossref{1King}{22}{2}{22:1 Mt 12:40; 16:21}
\crossref{1King}{22}{3}{1Ki 4:13 De 4:43 Jos 20:8}
\crossref{1King}{22}{4}{2Ki 3:7 2Ch 18:3}
\crossref{1King}{22}{5}{Nu 27:21 Jos 9:14 Jud 1:1; 20:18,\allowbreak23,\allowbreak29 1Sa 14:18; 23:2,\allowbreak4,\allowbreak9-\allowbreak12}
\crossref{1King}{22}{6}{1Ki 18:19 2Ti 4:3}
\crossref{1King}{22}{7}{2Ki 3:11-\allowbreak13 2Ch 18:6,\allowbreak7}
\crossref{1King}{22}{8}{1Ki 18:4; 19:10,\allowbreak14; 20:41,\allowbreak42}
\crossref{1King}{22}{9}{2Ki 9:32 2Ch 18:8 Isa 39:7 Da 1:18}
\crossref{1King}{22}{10}{22:30 Es 5:1; 6:8,\allowbreak9 Mt 6:20; 11:8 Ac 12:21; 25:23}
\crossref{1King}{22}{11}{Jer 27:2; 28:10-\allowbreak14 Zec 1:18-\allowbreak21 Ac 19:13-\allowbreak16 2Co 11:13-\allowbreak15}
\crossref{1King}{22}{12}{22:6-\allowbreak15,\allowbreak32-\allowbreak36 2Ch 35:22}
\crossref{1King}{22}{13}{Ps 10:11; 11:1; 14:1; 50:21 Isa 30:10,\allowbreak11 Ho 7:3 Am 7:13-\allowbreak17}
\crossref{1King}{22}{14}{Nu 22:38; 24:13 2Ch 18:12,\allowbreak13 Jer 23:28; 26:2,\allowbreak3; 42:4 Eze 2:4-\allowbreak8}
\crossref{1King}{22}{15}{22:6}
\crossref{1King}{22}{16}{Jos 6:26 1Sa 14:24 2Ch 18:15 Mt 26:63 Mr 5:7 Ac 19:13}
\crossref{1King}{22}{17}{1Sa 9:9 Jer 1:11-\allowbreak16 Eze 1:4 Ac 10:11-\allowbreak17}
\crossref{1King}{22}{18}{22:8 Pr 10:24; 27:22; 29:1 Lu 11:45}
\crossref{1King}{22}{19}{Isa 1:10; 28:14 Jer 2:4; 29:20; 42:15 Eze 13:2 Am 7:16}
\crossref{1King}{22}{20}{Job 12:16 Jer 4:10 Eze 14:9}
\crossref{1King}{22}{21}{22:23 Job 1:6,\allowbreak7; 2:1}
\crossref{1King}{22}{22}{Job 1:8-\allowbreak11; 2:4-\allowbreak6 Joh 8:44 Ac 5:3,\allowbreak4 2Th 2:9,\allowbreak10 1Ti 4:1 1Jo 4:6}
\crossref{1King}{22}{23}{Ex 4:21; 10:20 De 2:30 2Ch 25:16 Isa 6:9,\allowbreak10; 44:20 Eze 14:3-\allowbreak5,\allowbreak9}
\crossref{1King}{22}{24}{22:11}
\crossref{1King}{22}{25}{Nu 31:8 Isa 9:14-\allowbreak16 Jer 23:15; 28:16,\allowbreak17; 29:21,\allowbreak22,\allowbreak32 Am 7:17}
\crossref{1King}{22}{26}{22:9}
\crossref{1King}{22}{27}{2Ch 16:10; 18:25-\allowbreak27 Jer 20:2; 29:26; 37:15; 38:6 La 3:53-\allowbreak55}
\crossref{1King}{22}{28}{Nu 16:29 De 18:20-\allowbreak22 2Ki 1:10,\allowbreak12 Isa 44:26 Jer 28:8,\allowbreak9}
\crossref{1King}{22}{29}{22:2-\allowbreak9 2Ch 18:28}
\crossref{1King}{22}{30}{1Ki 14:2; 20:38 1Sa 28:8 2Sa 14:2 2Ch 18:29; 35:22 Pr 21:30}
\crossref{1King}{22}{31}{1Ki 20:24 2Ch 18:30}
\crossref{1King}{22}{32}{Pr 13:20}
\crossref{1King}{22}{33}{22:31 Ps 76:10}
\crossref{1King}{22}{34}{2Sa 15:11}
\crossref{1King}{22}{35}{22:28; 20:42}
\crossref{1King}{22}{36}{22:17,\allowbreak31; 12:16 2Ki 14:12}
\crossref{1King}{22}{37}{22:37}
\crossref{1King}{22}{38}{1Ki 21:19 Jos 23:14,\allowbreak15 Isa 44:25,\allowbreak26; 48:3-\allowbreak5 Jer 44:21-\allowbreak23 Zec 1:4-\allowbreak6}
\crossref{1King}{22}{39}{1Ki 14:19; 15:23,\allowbreak31; 16:5,\allowbreak20,\allowbreak27}
\crossref{1King}{22}{40}{1Ki 2:10; 11:21; 14:31 De 31:16 2Sa 7:12}
\crossref{1King}{22}{41}{22:2 1Ch 3:10 2Ch 17:1; 20:31}
\crossref{1King}{22}{42}{2Ki 1:17; 8:16}
\crossref{1King}{22}{43}{1Ki 15:11,\allowbreak14 2Ch 14:2-\allowbreak5,\allowbreak11; 15:8,\allowbreak17; 17:3}
\crossref{1King}{22}{44}{22:2 2Ki 8:18 2Ch 19:2; 21:6 2Co 6:14}
\crossref{1King}{22}{45}{22:39}
\crossref{1King}{22}{46}{1Ki 14:24; 15:12 Ge 19:5 De 23:17 Jud 19:22 Ro 1:26,\allowbreak27 1Co 6:9}
\crossref{1King}{22}{47}{Ge 25:23; 27:40; 36:31-\allowbreak43 2Sa 8:14 2Ki 3:9; 8:20 Ps 108:9,\allowbreak10}
\crossref{1King}{22}{48}{2Ch 20:35,\allowbreak36-\allowbreak21:1}
\crossref{1King}{22}{49}{}
\crossref{1King}{22}{50}{22:40; 2:10 2Ch 21:1}
\crossref{1King}{22}{51}{1Ki 15:25 2Ki 1:17}
\crossref{1King}{22}{52}{1Ki 15:26; 16:30-\allowbreak33 2Ki 1:2-\allowbreak7}
\crossref{1King}{22}{53}{1Ki 16:31 Jud 2:1-\allowbreak11 2Ki 1:2; 3:2}

% 2King
\crossref{2King}{1}{1}{Nu 24:7 2Sa 8:2 1Ch 18:2 Ps 60:8}
\crossref{2King}{1}{2}{1Ki 22:34}
\crossref{2King}{1}{3}{1:15 1Ki 19:5,\allowbreak7 Ac 8:26; 12:7-\allowbreak11}
\crossref{2King}{1}{4}{}
\crossref{2King}{1}{5}{}
\crossref{2King}{1}{6}{Isa 41:22,\allowbreak23}
\crossref{2King}{1}{7}{}
\crossref{2King}{1}{8}{}
\crossref{2King}{1}{9}{2Ki 6:13,\allowbreak14 1Ki 18:4,\allowbreak10; 19:2; 22:8,\allowbreak26,\allowbreak27 Mt 14:3}
\crossref{2King}{1}{10}{2Ki 2:23,\allowbreak24 Nu 16:28-\allowbreak30 1Ki 18:36-\allowbreak38; 22:28 2Ch 36:16 Ps 105:15}
\crossref{2King}{1}{11}{Nu 16:41 1Sa 6:9 Isa 26:11 Jer 5:3 Joh 18:5-\allowbreak12 Ac 4:16,\allowbreak17}
\crossref{2King}{1}{12}{1:9,\allowbreak10}
\crossref{2King}{1}{13}{Job 15:25,\allowbreak26 Pr 27:22 Ec 9:3 Isa 1:5}
\crossref{2King}{1}{14}{1:10,\allowbreak11}
\crossref{2King}{1}{15}{Ge 15:1 1Ki 18:15 Ps 27:1 Isa 51:12 Jer 1:17; 15:20 Eze 2:6}
\crossref{2King}{1}{16}{1:3,\allowbreak4,\allowbreak6 Ex 4:22,\allowbreak23 1Ki 14:6-\allowbreak13; 21:18-\allowbreak24; 22:28}
\crossref{2King}{1}{17}{2Ki 3:1; 8:16,\allowbreak17 1Ki 22:51}
\crossref{2King}{1}{18}{1Ki 14:19; 22:39}
\crossref{2King}{2}{1}{Ge 5:24 1Ki 19:4 Lu 9:51 Ac 1:9 Heb 11:5 Re 11:12}
\crossref{2King}{2}{2}{Ru 1:15,\allowbreak16 2Sa 15:19,\allowbreak20 Joh 6:67,\allowbreak68}
\crossref{2King}{2}{3}{2:5,\allowbreak7,\allowbreak15; 4:1,\allowbreak38; 9:1 1Sa 10:10-\allowbreak12; 19:20 1Ki 18:4; 20:35 Isa 8:18}
\crossref{2King}{2}{4}{Jos 6:26 1Ki 16:34 Lu 19:1}
\crossref{2King}{2}{5}{Ge 48:19 Ec 3:7 Isa 41:1 Hab 2:20}
\crossref{2King}{2}{6}{}
\crossref{2King}{2}{7}{2:17 1Ki 18:4,\allowbreak13}
\crossref{2King}{2}{8}{2:14 Ex 14:21,\allowbreak22 Jos 3:14-\allowbreak17 Ps 114:5-\allowbreak7 Isa 11:15 Heb 11:29}
\crossref{2King}{2}{9}{2Ki 13:14-\allowbreak19 Nu 27:16-\allowbreak23 De 34:9 1Ch 29:18,\allowbreak19 Ps 72:1,\allowbreak20}
\crossref{2King}{2}{10}{Mr 11:22-\allowbreak24 Joh 16:24}
\crossref{2King}{2}{11}{2Ki 6:17 Ps 68:17; 104:3,\allowbreak4 Eze 1:4-\allowbreak28; 10:9-\allowbreak22 Hab 3:8 Zec 3:8}
\crossref{2King}{2}{12}{2:10}
\crossref{2King}{2}{13}{2:8 1Ki 19:19}
\crossref{2King}{2}{14}{2:8-\allowbreak10 Jos 1:1-\allowbreak9 Mr 16:20 Joh 14:12 Ac 2:33; 3:12,\allowbreak13}
\crossref{2King}{2}{15}{2:7}
\crossref{2King}{2}{16}{1Ki 18:12 Eze 3:14; 8:3; 11:24; 40:2 Ac 8:39 2Co 12:2,\allowbreak3}
\crossref{2King}{2}{17}{2Sa 18:22,\allowbreak23 Lu 11:8 Ro 10:2}
\crossref{2King}{2}{18}{}
\crossref{2King}{2}{19}{Nu 12:11 1Ki 18:7,\allowbreak13 1Ti 5:17}
\crossref{2King}{2}{20}{Jud 9:45 Eze 47:11 Zep 2:9}
\crossref{2King}{2}{21}{2Ki 4:41; 6:6 Ex 15:25,\allowbreak26 Le 2:13 Mt 5:11 Mr 9:50 Joh 9:6}
\crossref{2King}{2}{22}{}
\crossref{2King}{2}{23}{1Ki 12:28-\allowbreak32 Ho 4:15; 10:5,\allowbreak15 Am 3:14; 4:4; 5:5; 7:13}
\crossref{2King}{2}{24}{2Ki 1:10-\allowbreak12 Ge 9:25 De 28:15-\allowbreak26 Jud 9:20,\allowbreak57 Jer 28:16; 29:21-\allowbreak23}
\crossref{2King}{2}{25}{2Ki 4:25 1Ki 18:19,\allowbreak42}
\crossref{2King}{3}{1}{2Ki 1:17; 8:16}
\crossref{2King}{3}{2}{2Ki 6:31,\allowbreak32; 21:6,\allowbreak20}
\crossref{2King}{3}{3}{2Ki 10:20-\allowbreak31}
\crossref{2King}{3}{4}{Ge 13:2; 26:13,\allowbreak14 2Ch 26:10 Job 1:3; 42:12}
\crossref{2King}{3}{5}{2Ki 1:1; 8:20 2Ch 21:8-\allowbreak10}
\crossref{2King}{3}{6}{1Sa 11:8; 15:4 2Sa 24:1-\allowbreak25 1Ki 20:27}
\crossref{2King}{3}{7}{1Ki 22:4,\allowbreak32,\allowbreak33 2Ch 18:3,\allowbreak29-\allowbreak32; 19:2; 21:4-\allowbreak7; 22:3,\allowbreak4,\allowbreak10-\allowbreak12}
\crossref{2King}{3}{8}{}
\crossref{2King}{3}{9}{1Ki 22:27}
\crossref{2King}{3}{10}{2Ki 6:33 Ge 4:13 Ps 78:34-\allowbreak36 Pr 19:3 Isa 8:21; 51:20}
\crossref{2King}{3}{11}{1Ki 22:7 Ps 74:9 Am 3:7}
\crossref{2King}{3}{12}{2Ki 2:14,\allowbreak15,\allowbreak21,\allowbreak24 1Sa 3:19-\allowbreak21}
\crossref{2King}{3}{13}{Eze 14:3-\allowbreak5 Mt 8:29 Joh 2:4 2Co 5:16; 6:15}
\crossref{2King}{3}{14}{2Ki 5:16 1Ki 17:1; 18:15}
\crossref{2King}{3}{15}{1Ki 18:46 Eze 1:3; 3:14,\allowbreak22; 8:1 Ac 11:21}
\crossref{2King}{3}{16}{2Ki 4:3 Nu 2:18,\allowbreak16-\allowbreak18}
\crossref{2King}{3}{17}{1Ki 18:36-\allowbreak39 Ps 84:6; 107:35 Isa 41:17,\allowbreak18; 43:19,\allowbreak20; 48:21}
\crossref{2King}{3}{18}{1Ki 3:13 Jer 32:17,\allowbreak27 Lu 1:37 Eph 3:20}
\crossref{2King}{3}{19}{2Ki 13:17 Nu 24:17 Jud 6:16 1Sa 15:3; 23:2}
\crossref{2King}{3}{20}{Ex 29:39,\allowbreak40 1Ki 18:36 Da 9:21}
\crossref{2King}{3}{21}{}
\crossref{2King}{3}{22}{}
\crossref{2King}{3}{23}{2Ki 6:18-\allowbreak20; 7:6}
\crossref{2King}{3}{24}{Jos 8:20-\allowbreak22 Jud 20:40-\allowbreak46 1Th 5:3,\allowbreak4}
\crossref{2King}{3}{25}{3:19 Jud 9:45 2Sa 8:2 Isa 37:26,\allowbreak27}
\crossref{2King}{3}{26}{3:9 Am 2:1}
\crossref{2King}{3}{27}{1Sa 14:36-\allowbreak46 1Ki 20:13,\allowbreak28,\allowbreak43}
\crossref{2King}{4}{1}{4:38}
\crossref{2King}{4}{2}{2Ki 2:9; 6:26,\allowbreak27 Mt 15:34 Joh 6:5-\allowbreak7 Ac 3:6 2Co 6:10}
\crossref{2King}{4}{3}{2Ki 3:16 Joh 2:7}
\crossref{2King}{4}{4}{4:32,\allowbreak33 1Ki 17:19,\allowbreak20 Isa 26:20 Mt 6:6 Mr 5:40 Ac 9:40}
\crossref{2King}{4}{5}{2Ki 5:11 1Ki 17:15,\allowbreak16 Lu 1:45 Heb 11:7,\allowbreak8}
\crossref{2King}{4}{6}{4:43,\allowbreak44 Mt 9:29; 13:58; 14:20; 15:37 Lu 6:19 2Ch 6:12,\allowbreak13}
\crossref{2King}{4}{7}{Ps 37:21 Ro 12:17 Php 4:8 1Th 2:9,\allowbreak10; 4:12 2Th 3:7-\allowbreak12}
\crossref{2King}{4}{8}{4:11,\allowbreak18}
\crossref{2King}{4}{9}{Pr 31:10,\allowbreak11 1Pe 3:1}
\crossref{2King}{4}{10}{Isa 32:8 Mt 10:41,\allowbreak42; 25:40 Mr 9:41 Lu 8:3 Ro 12:13 Heb 10:24}
\crossref{2King}{4}{11}{}
\crossref{2King}{4}{12}{4:29-\allowbreak31; 5:20-\allowbreak27; 8:4,\allowbreak5}
\crossref{2King}{4}{13}{Mt 10:40-\allowbreak42 Lu 9:3-\allowbreak5 Ro 16:2,\allowbreak6 Php 4:18,\allowbreak19 1Th 5:12,\allowbreak13}
\crossref{2King}{4}{14}{Ge 15:2,\allowbreak3; 17:17; 18:10-\allowbreak14; 25:21; 30:1 Jud 13:2 1Sa 1:2,\allowbreak8 Lu 1:7}
\crossref{2King}{4}{15}{}
\crossref{2King}{4}{16}{Ge 17:21; 18:10,\allowbreak14}
\crossref{2King}{4}{17}{Ge 21:1 1Sa 1:19,\allowbreak20 Ps 113:9 Lu 1:24,\allowbreak25,\allowbreak36 Heb 11:11}
\crossref{2King}{4}{18}{Ru 2:4}
\crossref{2King}{4}{19}{}
\crossref{2King}{4}{20}{Isa 49:15; 66:13 Lu 7:12}
\crossref{2King}{4}{21}{4:10 1Ki 17:19}
\crossref{2King}{4}{22}{4:24,\allowbreak26 Joh 11:3 Ac 9:38}
\crossref{2King}{4}{23}{Nu 10:10; 28:11 1Ch 23:31 Isa 1:13-\allowbreak15}
\crossref{2King}{4}{24}{Ex 4:20 1Sa 25:20 1Ki 13:13,\allowbreak23}
\crossref{2King}{4}{25}{2Ki 2:25 1Ki 18:19,\allowbreak42 Isa 35:2}
\crossref{2King}{4}{26}{Zec 2:4}
\crossref{2King}{4}{27}{Mt 28:9 Lu 7:38}
\crossref{2King}{4}{28}{Ge 30:1}
\crossref{2King}{4}{29}{2Ki 9:1 1Ki 18:46 Eph 6:14 1Pe 1:13}
\crossref{2King}{4}{30}{2Ki 2:2,\allowbreak4}
\crossref{2King}{4}{31}{1Sa 14:37; 28:6 Eze 14:3 Mt 17:16-\allowbreak21 Mr 9:19-\allowbreak29 Ac 19:13-\allowbreak17}
\crossref{2King}{4}{32}{1Ki 17:17 Lu 8:52,\allowbreak53 Joh 11:17}
\crossref{2King}{4}{33}{4:4 Mt 6:6}
\crossref{2King}{4}{34}{1Ki 17:21 Ac 20:10}
\crossref{2King}{4}{35}{}
\crossref{2King}{4}{36}{4:12}
\crossref{2King}{4}{37}{4:27; 2:15 1Ki 17:24}
\crossref{2King}{4}{38}{2Ki 2:1 1Sa 7:16,\allowbreak17 Ac 10:38; 15:36}
\crossref{2King}{4}{39}{Isa 5:4 Jer 2:21 Mt 15:13 Heb 12:15}
\crossref{2King}{4}{40}{4:9; 1:9,\allowbreak11,\allowbreak13 De 33:1 1Ki 17:18}
\crossref{2King}{4}{41}{2Ki 2:21; 5:10; 6:6 Ex 15:25 Joh 9:6 1Co 1:25}
\crossref{2King}{4}{42}{1Sa 9:4,\allowbreak7}
\crossref{2King}{4}{43}{4:12}
\crossref{2King}{4}{44}{4:44}
\crossref{2King}{5}{1}{Lu 4:27}
\crossref{2King}{5}{2}{2Ki 6:23; 13:20 Jud 9:34 1Sa 13:17,\allowbreak18}
\crossref{2King}{5}{3}{Nu 11:29 Ac 26:29 1Co 4:8}
\crossref{2King}{5}{4}{2Ki 7:9-\allowbreak11 Mr 5:19; 16:9,\allowbreak10 Joh 1:42-\allowbreak46; 4:28,\allowbreak29 1Co 1:26,\allowbreak27}
\crossref{2King}{5}{5}{Ge 11:3,\allowbreak7 Ec 2:1 Isa 5:5 Jas 4:13; 5:1}
\crossref{2King}{5}{6}{}
\crossref{2King}{5}{7}{2Ki 11:14; 18:37; 19:1 Nu 14:6 Jer 36:24 Mt 26:65 Ac 14:14}
\crossref{2King}{5}{8}{5:7 2Sa 3:31}
\crossref{2King}{5}{9}{2Ki 3:12; 6:32 Isa 60:14 Ac 16:29,\allowbreak30,\allowbreak37-\allowbreak39}
\crossref{2King}{5}{10}{Mt 15:23-\allowbreak26}
\crossref{2King}{5}{11}{Pr 13:10 Mt 8:8; 15:27 Lu 14:11}
\crossref{2King}{5}{12}{5:17; 2:8,\allowbreak14 Jos 3:15-\allowbreak17 Eze 47:1-\allowbreak8 Zec 13:1; 14:8 Mr 1:9}
\crossref{2King}{5}{13}{5:3 1Sa 25:14-\allowbreak17 1Ki 20:23,\allowbreak31 Job 32:8,\allowbreak9 Jer 38:7-\allowbreak10}
\crossref{2King}{5}{14}{Job 31:13 Pr 9:9; 25:11,\allowbreak12 Eze 47:1-\allowbreak9 Zec 13:1; 14:8}
\crossref{2King}{5}{15}{Lu 17:15-\allowbreak18}
\crossref{2King}{5}{16}{2Ki 3:14 1Ki 17:1; 18:15}
\crossref{2King}{5}{17}{5:12 Ro 14:1}
\crossref{2King}{5}{18}{2Ki 17:35 Ex 20:5 1Ki 19:18}
\crossref{2King}{5}{19}{Mt 9:16,\allowbreak17 Joh 16:12 1Co 3:2 Heb 5:13,\allowbreak14}
\crossref{2King}{5}{20}{2Ki 4:12,\allowbreak31,\allowbreak36 Mt 10:4 Joh 6:70; 12:6; 13:2 Ac 8:18,\allowbreak19}
\crossref{2King}{5}{21}{Lu 7:6,\allowbreak7 Ac 8:31; 10:25,\allowbreak26}
\crossref{2King}{5}{22}{1Ki 13:18 Isa 59:3 Jer 9:3,\allowbreak5 Joh 8:44 Ac 5:3,\allowbreak4 Re 21:8}
\crossref{2King}{5}{23}{1Ki 20:7 Lu 11:54}
\crossref{2King}{5}{24}{Jos 7:1,\allowbreak11,\allowbreak12,\allowbreak21 1Ki 21:16 Isa 29:15 Hab 2:6 Zec 5:3,\allowbreak4}
\crossref{2King}{5}{25}{Pr 30:20 Eze 33:31 Mt 26:15,\allowbreak16,\allowbreak21-\allowbreak15 Joh 13:2,\allowbreak26-\allowbreak30}
\crossref{2King}{5}{26}{Ps 63:11 Pr 12:19,\allowbreak22 Ac 5:9}
\crossref{2King}{5}{27}{5:1 Jos 7:25 Isa 59:2,\allowbreak3 Ho 10:13 Mal 2:3,\allowbreak4,\allowbreak8,\allowbreak9 Mt 27:3-\allowbreak5}
\crossref{2King}{6}{1}{2Ki 2:3; 4:1 1Ki 20:35}
\crossref{2King}{6}{2}{Joh 21:3 Ac 18:3; 20:34,\allowbreak35 1Co 9:6 1Th 2:9 2Th 3:8 1Ti 6:6}
\crossref{2King}{6}{3}{2Ki 5:23 Jud 19:6 Job 6:28}
\crossref{2King}{6}{4}{De 19:5; 29:11}
\crossref{2King}{6}{5}{Ec 10:10 Isa 10:34}
\crossref{2King}{6}{6}{}
\crossref{2King}{6}{7}{2Ki 4:7,\allowbreak36 Lu 7:15 Ac 9:41}
\crossref{2King}{6}{8}{6:24 1Ki 20:1,\allowbreak34; 22:31}
\crossref{2King}{6}{9}{2Ki 3:17-\allowbreak19 1Ki 20:13,\allowbreak28}
\crossref{2King}{6}{10}{Eze 3:18-\allowbreak21 Mt 2:12; 3:7 Heb 11:7}
\crossref{2King}{6}{11}{1Sa 28:21 Job 18:7-\allowbreak11 Ps 48:4,\allowbreak5 Isa 57:20,\allowbreak21 Mt 2:3-\allowbreak12}
\crossref{2King}{6}{12}{2Ki 5:3,\allowbreak8,\allowbreak13-\allowbreak15 Am 3:7}
\crossref{2King}{6}{13}{1Sa 23:22,\allowbreak23 Ps 10:8-\allowbreak10; 37:12-\allowbreak14,\allowbreak32,\allowbreak33 Jer 36:26 Mt 2:4-\allowbreak8}
\crossref{2King}{6}{14}{2Ki 18:17}
\crossref{2King}{6}{15}{2Ki 3:11; 5:20,\allowbreak27 Ex 24:13 1Ki 19:21 Mt 20:26-\allowbreak28 Ac 13:5}
\crossref{2King}{6}{16}{Ex 14:13 Ps 3:6; 11:1; 27:3; 118:11,\allowbreak12 Isa 8:12,\allowbreak13; 41:10-\allowbreak14}
\crossref{2King}{6}{17}{Ps 91:15 Jas 5:16-\allowbreak18}
\crossref{2King}{6}{18}{}
\crossref{2King}{6}{19}{Mt 16:24 Mr 8:34 Lu 9:23}
\crossref{2King}{6}{20}{6:17 Lu 24:31}
\crossref{2King}{6}{21}{1Sa 24:4,\allowbreak19; 26:8 Lu 9:54-\allowbreak56; 22:49}
\crossref{2King}{6}{22}{De 20:11-\allowbreak16 2Ch 28:8-\allowbreak13}
\crossref{2King}{6}{23}{1Sa 24:17,\allowbreak18 2Ch 28:15 Pr 25:21,\allowbreak22 Mt 5:47 Lu 6:35; 10:29-\allowbreak37}
\crossref{2King}{6}{24}{2Ki 17:5; 18:9; 25:1 De 28:52 1Ki 20:1; 22:31 Ec 9:14}
\crossref{2King}{6}{25}{6:28,\allowbreak29; 7:4; 25:3 1Ki 18:2 Jer 14:13-\allowbreak15,\allowbreak18; 32:24; 52:6}
\crossref{2King}{6}{26}{2Sa 14:4 Isa 10:3 Lu 18:3 Ac 21:28}
\crossref{2King}{6}{27}{Ps 60:11; 62:8; 118:8,\allowbreak9; 124:1-\allowbreak3; 127:1; 146:3 Isa 2:2 Jer 17:5}
\crossref{2King}{6}{28}{Ge 21:17 Jud 18:23 1Sa 1:8 2Sa 14:5 Ps 114:5 Isa 22:1}
\crossref{2King}{6}{29}{1Ki 3:26 Isa 49:15; 66:13}
\crossref{2King}{6}{30}{2Ki 5:7; 19:1 1Ki 21:27 Isa 58:5-\allowbreak7}
\crossref{2King}{6}{31}{Ru 1:17 1Sa 3:17; 14:44; 25:22 2Sa 3:9,\allowbreak35; 19:13 1Ki 2:23}
\crossref{2King}{6}{32}{Eze 8:1; 14:1; 20:1; 33:31}
\crossref{2King}{6}{33}{Ge 4:13 Ex 16:6-\allowbreak8 1Sa 28:6-\allowbreak8; 31:4 Job 1:11,\allowbreak21; 2:5,\allowbreak9 Pr 19:3}
\crossref{2King}{7}{1}{2Ki 6:33; 20:16 1Ki 22:19 Isa 1:10 Eze 37:4}
\crossref{2King}{7}{2}{Ge 18:12-\allowbreak14 Nu 11:21-\allowbreak23 Ps 78:19-\allowbreak21,\allowbreak41}
\crossref{2King}{7}{3}{2Ki 5:1; 8:4 Le 13:46 Nu 5:2-\allowbreak4; 12:14}
\crossref{2King}{7}{4}{Jer 14:18}
\crossref{2King}{7}{5}{1Sa 30:17 Eze 12:6,\allowbreak7,\allowbreak12}
\crossref{2King}{7}{6}{2Ki 3:22,\allowbreak23-\allowbreak27; 19:7 2Sa 5:24 Job 15:21 Ps 14:5 Jer 20:3,\allowbreak4}
\crossref{2King}{7}{7}{Job 18:11 Ps 48:4-\allowbreak6; 68:12 Pr 21:1; 28:1 Jer 48:8,\allowbreak9}
\crossref{2King}{7}{8}{2Ki 5:24 Jos 7:21 Jer 41:8 Mt 13:44; 25:18}
\crossref{2King}{7}{9}{7:3 Hag 1:4,\allowbreak5}
\crossref{2King}{7}{10}{7:11 2Sa 18:26 Ps 127:1 Mr 13:34,\allowbreak35}
\crossref{2King}{7}{11}{2Ki 6:8 Ge 20:8; 41:38 1Ki 20:7,\allowbreak23}
\crossref{2King}{7}{12}{}
\crossref{2King}{7}{13}{2Ki 5:13}
\crossref{2King}{7}{14}{}
\crossref{2King}{7}{15}{Es 1:7 Isa 22:24}
\crossref{2King}{7}{16}{1Sa 17:53 2Ch 14:12-\allowbreak15; 20:25 Job 27:16,\allowbreak17 Ps 68:12}
\crossref{2King}{7}{17}{7:2}
\crossref{2King}{7}{18}{7:1,\allowbreak2; 6:32 Ge 18:14}
\crossref{2King}{7}{19}{7:19}
\crossref{2King}{7}{20}{Nu 20:12 2Ch 20:20 Job 20:23 Isa 7:9 Jer 17:5,\allowbreak6 Heb 3:18,\allowbreak19}
\crossref{2King}{8}{1}{2Ki 4:18,\allowbreak31-\allowbreak35}
\crossref{2King}{8}{2}{1Ti 5:8}
\crossref{2King}{8}{3}{8:6; 4:13; 6:26 2Sa 14:4 Ps 82:3,\allowbreak4 Jer 22:16 Lu 18:3-\allowbreak5}
\crossref{2King}{8}{4}{2Ki 5:20-\allowbreak27; 7:3,\allowbreak10}
\crossref{2King}{8}{5}{2Ki 4:35}
\crossref{2King}{8}{6}{2Ki 9:32 Ge 37:36 1Ch 28:1}
\crossref{2King}{8}{7}{Ge 14:15 1Ki 11:24 Isa 7:8}
\crossref{2King}{8}{8}{1Ki 19:15}
\crossref{2King}{8}{9}{1Ki 19:15}
\crossref{2King}{8}{10}{1Ki 22:15}
\crossref{2King}{8}{11}{Ge 45:2 Ps 119:136 Jer 4:19; 9:1,\allowbreak18; 13:17; 14:17 Lu 19:41}
\crossref{2King}{8}{12}{2Ki 4:28 1Ki 18:13}
\crossref{2King}{8}{13}{1Sa 17:43 2Sa 9:8 Ps 22:16,\allowbreak20 Isa 56:10,\allowbreak11 Mt 7:6 Php 3:2}
\crossref{2King}{8}{14}{8:10; 5:25 Mt 26:16}
\crossref{2King}{8}{15}{8:13 1Sa 16:12,\allowbreak13; 24:4-\allowbreak7,\allowbreak13; 26:9-\allowbreak11 1Ki 11:26-\allowbreak37}
\crossref{2King}{8}{16}{2Ki 1:17 1Ki 22:50 2Ch 21:1-\allowbreak20}
\crossref{2King}{8}{17}{8:17}
\crossref{2King}{8}{18}{2Ki 3:2,\allowbreak3 1Ki 22:52,\allowbreak53}
\crossref{2King}{8}{19}{2Ki 19:34 2Sa 7:12,\allowbreak13,\allowbreak15 1Ki 11:36; 15:4,\allowbreak5 2Ch 21:7 Isa 7:14; 37:35}
\crossref{2King}{8}{20}{8:22; 3:9,\allowbreak27 Ge 27:40 2Ch 21:8-\allowbreak10}
\crossref{2King}{8}{21}{}
\crossref{2King}{8}{22}{8:20}
\crossref{2King}{8}{23}{2Ki 15:6,\allowbreak36}
\crossref{2King}{8}{24}{1Ki 2:10; 11:43; 14:20,\allowbreak31}
\crossref{2King}{8}{25}{8:16,\allowbreak17; 9:29 2Ch 21:20}
\crossref{2King}{8}{26}{2Ki 9:21-\allowbreak27 2Ch 22:5-\allowbreak8}
\crossref{2King}{8}{27}{8:18}
\crossref{2King}{8}{28}{2Ki 3:7; 9:15 1Ki 22:4 2Ch 18:2,\allowbreak3,\allowbreak31; 19:2; 22:5}
\crossref{2King}{8}{29}{2Ki 9:15}
\crossref{2King}{9}{1}{2Ki 4:1; 6:1-\allowbreak3}
\crossref{2King}{9}{2}{9:5,\allowbreak11}
\crossref{2King}{9}{3}{Ex 29:7 Le 8:12 1Sa 16:13 1Ki 19:16}
\crossref{2King}{9}{4}{}
\crossref{2King}{9}{5}{Jud 3:19}
\crossref{2King}{9}{6}{Ac 23:18,\allowbreak19}
\crossref{2King}{9}{7}{De 32:35,\allowbreak43 Ps 94:1-\allowbreak7 Mt 23:35 Lu 18:7,\allowbreak8 Ro 12:19; 13:4}
\crossref{2King}{9}{8}{1Ki 14:10,\allowbreak11; 21:21,\allowbreak22}
\crossref{2King}{9}{9}{1Ki 14:10,\allowbreak11; 15:29; 21:22}
\crossref{2King}{9}{10}{9:35,\allowbreak36 1Ki 21:23 Jer 22:19}
\crossref{2King}{9}{11}{9:17,\allowbreak19,\allowbreak22; 4:26; 5:21}
\crossref{2King}{9}{12}{9:6-\allowbreak10}
\crossref{2King}{9}{13}{}
\crossref{2King}{9}{14}{9:31; 8:12-\allowbreak15; 10:9; 15:30 1Ki 15:27; 16:7,\allowbreak9,\allowbreak16}
\crossref{2King}{9}{15}{2Ki 8:29 2Ch 22:6}
\crossref{2King}{9}{16}{2Ki 8:28,\allowbreak29 2Ch 22:6,\allowbreak7}
\crossref{2King}{9}{17}{2Sa 13:34; 18:24 Isa 21:6-\allowbreak9,\allowbreak11,\allowbreak12; 56:10; 62:6 Eze 33:2-\allowbreak9}
\crossref{2King}{9}{18}{9:19,\allowbreak22 Isa 48:22; 59:8 Jer 16:5 Ro 3:17}
\crossref{2King}{9}{19}{}
\crossref{2King}{9}{20}{Hab 1:6; 3:12}
\crossref{2King}{9}{21}{1Ki 20:14}
\crossref{2King}{9}{22}{9:17}
\crossref{2King}{9}{23}{2Ki 11:14 2Ch 23:13}
\crossref{2King}{9}{24}{}
\crossref{2King}{9}{25}{1Ki 21:19,\allowbreak24-\allowbreak29 Isa 13:1 Jer 23:33-\allowbreak38 Na 1:1 Mal 1:1 Mt 11:30}
\crossref{2King}{9}{26}{De 24:16 2Ch 24:25; 25:4}
\crossref{2King}{9}{27}{2Ki 8:29 Nu 16:26 2Ch 22:7-\allowbreak9 Pr 13:20 2Co 6:17}
\crossref{2King}{9}{28}{2Ki 12:21; 14:19,\allowbreak20; 23:20 2Ch 25:28; 35:24}
\crossref{2King}{9}{29}{2Ki 8:16,\allowbreak24 2Ch 21:18,\allowbreak19; 22:1,\allowbreak2}
\crossref{2King}{9}{30}{1Ki 19:1,\allowbreak2}
\crossref{2King}{9}{31}{1Ki 16:9-\allowbreak20}
\crossref{2King}{9}{32}{Ex 32:26 1Ch 12:18 2Ch 11:12 Ps 118:6; 124:1,\allowbreak2}
\crossref{2King}{9}{33}{1Ki 21:11}
\crossref{2King}{9}{34}{1Ki 18:41 Es 3:15 Am 6:4}
\crossref{2King}{9}{35}{Job 31:3 Ec 6:3 Isa 14:18-\allowbreak20 Jer 22:19; 36:30 Ac 12:23}
\crossref{2King}{9}{36}{1Ki 21:23}
\crossref{2King}{9}{37}{Ps 83:10 Ec 6:3 Isa 14:18-\allowbreak20 Jer 8:2; 16:4; 22:19; 36:20}
\crossref{2King}{10}{1}{Jud 8:30; 10:4; 12:14}
\crossref{2King}{10}{2}{2Ki 5:6}
\crossref{2King}{10}{3}{De 17:14,\allowbreak15 1Sa 10:24; 11:15 2Sa 2:8,\allowbreak9 1Ki 1:24,\allowbreak25; 12:20}
\crossref{2King}{10}{4}{2Ki 9:24,\allowbreak27}
\crossref{2King}{10}{5}{2Ki 18:14 Jos 9:11,\allowbreak24,\allowbreak25 1Ki 20:4,\allowbreak32 Jer 27:7,\allowbreak8,\allowbreak17 Joh 12:26}
\crossref{2King}{10}{6}{2Ki 9:32 Mt 12:30 Lu 9:50}
\crossref{2King}{10}{7}{10:9; 11:1 Jud 9:5-\allowbreak57 1Ki 21:21 2Ch 21:4 Mt 14:8-\allowbreak11}
\crossref{2King}{10}{8}{2Sa 11:18-\allowbreak21 1Ki 21:14 Mr 6:28}
\crossref{2King}{10}{9}{1Sa 12:3 Isa 5:3}
\crossref{2King}{10}{10}{1Sa 3:19; 15:29 Jer 44:28,\allowbreak29 Zec 1:6 Mr 13:31}
\crossref{2King}{10}{11}{Ps 125:5 Pr 13:20}
\crossref{2King}{10}{12}{10:12}
\crossref{2King}{10}{13}{2Ki 8:24,\allowbreak29; 9:21-\allowbreak27 2Ch 21:17; 22:1-\allowbreak10}
\crossref{2King}{10}{14}{10:6,\allowbreak10,\allowbreak11 1Ki 20:18}
\crossref{2King}{10}{15}{10:13; 9:21}
\crossref{2King}{10}{16}{10:31; 9:7-\allowbreak9 Nu 23:4; 24:13-\allowbreak16 1Ki 19:10,\allowbreak14,\allowbreak17 Pr 27:2 Eze 33:31}
\crossref{2King}{10}{17}{10:11; 9:8 2Ch 22:8 Ps 109:8,\allowbreak9 Mal 4:1}
\crossref{2King}{10}{18}{2Ki 3:2 1Ki 16:31,\allowbreak32; 18:19,\allowbreak22,\allowbreak40}
\crossref{2King}{10}{19}{2Ki 3:13 1Ki 22:6}
\crossref{2King}{10}{20}{1Ki 18:19,\allowbreak20; 21:12 Joe 1:14}
\crossref{2King}{10}{21}{Joe 3:2,\allowbreak11-\allowbreak14 Re 16:16}
\crossref{2King}{10}{22}{Ex 28:2 Mt 22:11,\allowbreak12}
\crossref{2King}{10}{23}{10:15}
\crossref{2King}{10}{24}{1Ki 20:30-\allowbreak42}
\crossref{2King}{10}{25}{Ex 32:27 De 13:6-\allowbreak11 Eze 9:5-\allowbreak7}
\crossref{2King}{10}{26}{1Ki 14:23}
\crossref{2King}{10}{27}{2Ki 18:4; 23:7-\allowbreak14 Le 26:30 De 7:5,\allowbreak25 1Ki 16:32 2Ch 34:3-\allowbreak7}
\crossref{2King}{10}{28}{}
\crossref{2King}{10}{29}{2Ki 13:2,\allowbreak11; 14:24; 15:9,\allowbreak18,\allowbreak24,\allowbreak28; 17:22 1Ki 12:28-\allowbreak30; 13:33,\allowbreak34; 14:16}
\crossref{2King}{10}{30}{1Ki 21:29 Eze 29:18-\allowbreak20 Ho 1:4}
\crossref{2King}{10}{31}{De 4:15,\allowbreak23 1Ki 2:4 Ps 39:1; 119:9 Pr 4:23 Heb 2:1; 12:15}
\crossref{2King}{10}{32}{2Ki 8:12; 13:22 1Ki 19:17}
\crossref{2King}{10}{33}{Am 1:3,\allowbreak4}
\crossref{2King}{10}{34}{2Ki 12:19; 13:8}
\crossref{2King}{10}{35}{2Sa 7:12 1Ki 1:21; 2:10; 14:20,\allowbreak31}
\crossref{2King}{10}{36}{10:36}
\crossref{2King}{11}{1}{2Ch 22:10; 24:7}
\crossref{2King}{11}{2}{2Ch 22:11}
\crossref{2King}{11}{3}{2Ch 22:12 Ps 12:8 Mal 3:15}
\crossref{2King}{11}{4}{2Ch 23:1-\allowbreak15}
\crossref{2King}{11}{5}{1Ch 9:25; 23:3-\allowbreak6,\allowbreak32; 24:3-\allowbreak6 Lu 1:8,\allowbreak9}
\crossref{2King}{11}{6}{1Ch 26:13-\allowbreak19 2Ch 23:4,\allowbreak5}
\crossref{2King}{11}{7}{11:5 2Ch 23:6}
\crossref{2King}{11}{8}{11:15 Ex 21:14 1Ki 2:28-\allowbreak31 2Ch 23:7}
\crossref{2King}{11}{9}{11:4 1Ch 26:26 2Ch 23:8}
\crossref{2King}{11}{10}{}
\crossref{2King}{11}{11}{11:8,\allowbreak10}
\crossref{2King}{11}{12}{11:2,\allowbreak4 2Ch 23:11}
\crossref{2King}{11}{13}{2Ch 23:12-\allowbreak15}
\crossref{2King}{11}{14}{2Ki 23:3 2Ch 34:31}
\crossref{2King}{11}{15}{11:4,\allowbreak9,\allowbreak10 2Ch 23:9,\allowbreak14}
\crossref{2King}{11}{16}{2Ch 23:15}
\crossref{2King}{11}{17}{11:4 De 5:2,\allowbreak3; 29:1-\allowbreak15 Jos 24:25 2Ch 15:12-\allowbreak14; 29:10; 34:31}
\crossref{2King}{11}{18}{2Ki 9:25-\allowbreak28; 10:26; 18:4; 23:4-\allowbreak6,\allowbreak10,\allowbreak14 2Ch 23:17; 34:4,\allowbreak7}
\crossref{2King}{11}{19}{11:4-\allowbreak11}
\crossref{2King}{11}{20}{11:14 2Ch 23:21 Pr 11:10; 29:2}
\crossref{2King}{11}{21}{11:4; 22:1 2Ch 24:1-\allowbreak14}
\crossref{2King}{12}{1}{2Ki 9:27; 11:1,\allowbreak3,\allowbreak4,\allowbreak21 2Ch 24:1-\allowbreak14}
\crossref{2King}{12}{2}{2Ki 14:3 2Ch 24:2,\allowbreak17-\allowbreak22; 25:2; 26:4}
\crossref{2King}{12}{3}{2Ki 14:4; 18:4 1Ki 15:14; 22:43 2Ch 31:4 Jer 2:20}
\crossref{2King}{12}{4}{2Ki 22:4 2Ch 29:4-\allowbreak11; 35:2}
\crossref{2King}{12}{5}{2Ch 24:5}
\crossref{2King}{12}{6}{}
\crossref{2King}{12}{7}{2Ch 24:5,\allowbreak6-\allowbreak14}
\crossref{2King}{12}{8}{}
\crossref{2King}{12}{9}{2Ch 24:8-\allowbreak14 Mr 12:41}
\crossref{2King}{12}{10}{2Ki 19:2; 22:3,\allowbreak12 2Sa 8:17; 20:25}
\crossref{2King}{12}{11}{2Ki 22:5,\allowbreak6 2Ch 24:11,\allowbreak12; 34:9-\allowbreak11}
\crossref{2King}{12}{12}{1Ki 5:17,\allowbreak18 Ezr 3:7; 5:8 Lu 21:5}
\crossref{2King}{12}{13}{Nu 7:13,\allowbreak14 1Ki 7:48-\allowbreak50 Ezr 1:9-\allowbreak11}
\crossref{2King}{12}{14}{}
\crossref{2King}{12}{15}{2Ki 22:7}
\crossref{2King}{12}{16}{Le 5:15-\allowbreak18; 7:7 Nu 5:8-\allowbreak10; 18:8,\allowbreak9 Ho 4:8}
\crossref{2King}{12}{17}{2Ki 8:12-\allowbreak15}
\crossref{2King}{12}{18}{}
\crossref{2King}{12}{19}{2Ki 8:23 1Ki 11:41; 14:19,\allowbreak29}
\crossref{2King}{12}{20}{2Ki 14:5 2Ch 24:24,\allowbreak25; 25:27; 33:24}
\crossref{2King}{12}{21}{2Ch 24:26}
\crossref{2King}{13}{1}{2Ki 10:35}
\crossref{2King}{13}{2}{13:11}
\crossref{2King}{13}{3}{Le 26:17 De 4:24-\allowbreak27; 28:25 Jud 2:14; 3:8; 10:7-\allowbreak14 Isa 10:5,\allowbreak6}
\crossref{2King}{13}{4}{Nu 21:7 Jud 6:6,\allowbreak7; 10:10 Ps 78:34 Isa 26:16 Jer 2:27}
\crossref{2King}{13}{5}{Ex 4:10 De 19:4 1Sa 19:7 1Ch 11:2}
\crossref{2King}{13}{6}{13:2; 10:29; 17:20-\allowbreak23 De 32:15-\allowbreak18}
\crossref{2King}{13}{7}{1Sa 13:6,\allowbreak7,\allowbreak15,\allowbreak19-\allowbreak23 1Ki 20:15,\allowbreak27 Isa 36:8}
\crossref{2King}{13}{8}{2Ki 10:34,\allowbreak35}
\crossref{2King}{13}{9}{13:13; 10:35 1Ki 14:13}
\crossref{2King}{13}{10}{}
\crossref{2King}{13}{11}{13:2,\allowbreak6; 3:3; 10:29}
\crossref{2King}{13}{12}{13:14-\allowbreak25; 14:15,\allowbreak25}
\crossref{2King}{13}{13}{2Sa 7:12 1Ki 1:21; 2:10; 11:31}
\crossref{2King}{13}{14}{2Ki 20:1 Ge 48:1 Joh 11:3 Php 2:26}
\crossref{2King}{13}{15}{}
\crossref{2King}{13}{16}{2Ki 4:34 Ge 49:24 Ps 144:1}
\crossref{2King}{13}{17}{2Ki 5:10-\allowbreak14 Joh 2:5-\allowbreak8; 11:39-\allowbreak41}
\crossref{2King}{13}{18}{Isa 20:2-\allowbreak4 Eze 4:1-\allowbreak10; 5:1-\allowbreak4; 12:1-\allowbreak7}
\crossref{2King}{13}{19}{2Ki 1:9-\allowbreak15; 4:16,\allowbreak40; 6:9}
\crossref{2King}{13}{20}{2Ch 24:16 Ac 8:2}
\crossref{2King}{13}{21}{2Ki 4:35 Isa 26:19 Eze 37:1-\allowbreak10 Mt 27:52,\allowbreak53 Joh 5:25,\allowbreak28,\allowbreak29; 11:44}
\crossref{2King}{13}{22}{13:3-\allowbreak7; 8:12 Ps 106:40-\allowbreak42}
\crossref{2King}{13}{23}{2Ki 14:27 Ex 33:19; 34:6,\allowbreak7 Jud 10:16 Ne 9:31 Ps 86:15 Isa 30:18,\allowbreak19}
\crossref{2King}{13}{24}{Ps 125:3 Lu 18:7}
\crossref{2King}{13}{25}{}
\crossref{2King}{14}{1}{14:15; 13:10}
\crossref{2King}{14}{2}{}
\crossref{2King}{14}{3}{2Ki 12:2 1Ki 11:4; 15:3 2Ch 25:2,\allowbreak3}
\crossref{2King}{14}{4}{2Ki 12:3; 15:4,\allowbreak35}
\crossref{2King}{14}{5}{Ge 9:6 Ex 21:12-\allowbreak14 Nu 35:33}
\crossref{2King}{14}{6}{De 24:16 Eze 18:4,\allowbreak20}
\crossref{2King}{14}{7}{2Ki 8:20-\allowbreak22 2Ch 25:11,\allowbreak12}
\crossref{2King}{14}{8}{2Ch 25:17-\allowbreak24}
\crossref{2King}{14}{9}{Jud 9:8-\allowbreak15 2Sa 12:1-\allowbreak4 1Ki 4:33 Eze 20:49}
\crossref{2King}{14}{10}{De 8:14 2Ch 26:16; 32:25 Pr 16:18 Eze 38:2,\allowbreak5,\allowbreak17 Da 5:20-\allowbreak23}
\crossref{2King}{14}{11}{2Ch 25:16,\allowbreak20}
\crossref{2King}{14}{12}{1Sa 4:10 2Sa 18:17 1Ki 22:36}
\crossref{2King}{14}{13}{2Ki 25:6 2Ch 33:11; 36:6,\allowbreak10 Job 40:11,\allowbreak12 Pr 16:18; 29:23}
\crossref{2King}{14}{14}{2Ki 24:13; 25:15 1Ki 7:51; 14:26; 15:18}
\crossref{2King}{14}{15}{2Ki 10:34,\allowbreak35; 13:12 1Ki 14:19,\allowbreak20}
\crossref{2King}{14}{16}{2Sa 7:12 1Ki 1:21}
\crossref{2King}{14}{17}{14:1,\allowbreak2,\allowbreak23; 13:10 2Ch 25:25-\allowbreak28}
\crossref{2King}{14}{18}{2Ki 13:8,\allowbreak12 1Ki 11:41; 14:29}
\crossref{2King}{14}{19}{2Ki 12:20,\allowbreak21; 15:10,\allowbreak14,\allowbreak25,\allowbreak30; 21:23 2Ch 25:27,\allowbreak28}
\crossref{2King}{14}{20}{2Ki 8:24; 9:28; 12:21 1Ki 2:10; 11:43 2Ch 21:20; 26:23; 33:20}
\crossref{2King}{14}{21}{2Ki 15:13 2Ch 26:1}
\crossref{2King}{14}{22}{}
\crossref{2King}{14}{23}{14:17}
\crossref{2King}{14}{24}{2Ki 21:6 Ge 38:7 De 9:18 1Ki 21:25}
\crossref{2King}{14}{25}{Nu 13:21; 34:7,\allowbreak8 Eze 47:16-\allowbreak18 Am 6:14}
\crossref{2King}{14}{26}{2Ki 13:4 Ex 3:7,\allowbreak9 Jud 10:16 Ps 106:43-\allowbreak45 Isa 63:9}
\crossref{2King}{14}{27}{2Ki 13:23 Ho 1:6}
\crossref{2King}{14}{28}{14:15}
\crossref{2King}{14}{29}{2Ki 15:8}
\crossref{2King}{15}{1}{15:8; 14:16,\allowbreak17}
\crossref{2King}{15}{2}{15:2}
\crossref{2King}{15}{3}{2Ki 12:2,\allowbreak3; 14:3,\allowbreak4 2Ch 26:4}
\crossref{2King}{15}{4}{15:35; 14:4; 18:4 1Ki 15:14; 22:43 2Ch 17:6; 32:12; 34:3}
\crossref{2King}{15}{5}{2Sa 3:29 2Ch 26:16-\allowbreak20 Job 34:19}
\crossref{2King}{15}{6}{2Ki 14:18 2Ch 26:5-\allowbreak15}
\crossref{2King}{15}{7}{2Ch 26:23 Isa 6:1}
\crossref{2King}{15}{8}{2Ki 14:29}
\crossref{2King}{15}{9}{2Ki 10:29,\allowbreak31; 13:2,\allowbreak11; 14:24}
\crossref{2King}{15}{10}{15:14,\allowbreak25,\allowbreak30; 9:24,\allowbreak31 1Ki 15:28; 16:9,\allowbreak10 Ho 1:4,\allowbreak5}
\crossref{2King}{15}{11}{2Ki 14:28}
\crossref{2King}{15}{12}{2Ki 10:30}
\crossref{2King}{15}{13}{15:1}
\crossref{2King}{15}{14}{1Ki 14:17; 15:21,\allowbreak33; 16:8,\allowbreak9,\allowbreak15,\allowbreak17}
\crossref{2King}{15}{15}{15:11 1Ki 14:19,\allowbreak29; 22:39}
\crossref{2King}{15}{16}{1Ki 4:24}
\crossref{2King}{15}{17}{15:13}
\crossref{2King}{15}{18}{15:9}
\crossref{2King}{15}{19}{2Ki 12:18; 16:8; 17:3,\allowbreak4; 18:16 Ho 5:13; 8:9,\allowbreak10; 10:6}
\crossref{2King}{15}{20}{2Ki 23:35}
\crossref{2King}{15}{21}{15:15}
\crossref{2King}{15}{22}{}
\crossref{2King}{15}{23}{2Ki 21:19 1Ki 15:25; 16:8; 22:51 Job 20:5}
\crossref{2King}{15}{24}{15:9,\allowbreak18}
\crossref{2King}{15}{25}{15:27 2Ch 28:6}
\crossref{2King}{15}{26}{15:15}
\crossref{2King}{15}{27}{15:2,\allowbreak8,\allowbreak13,\allowbreak23}
\crossref{2King}{15}{28}{15:9,\allowbreak18; 13:2,\allowbreak6; 21:2}
\crossref{2King}{15}{29}{Isa 9:1}
\crossref{2King}{15}{30}{15:10,\allowbreak25}
\crossref{2King}{15}{31}{}
\crossref{2King}{15}{32}{15:7 1Ch 3:12 2Ch 27:1-\allowbreak9 Mt 1:9}
\crossref{2King}{15}{33}{2Ch 27:1}
\crossref{2King}{15}{34}{15:3,\allowbreak4 2Ch 26:4,\allowbreak5; 27:2}
\crossref{2King}{15}{35}{15:4; 18:4 2Ch 32:12}
\crossref{2King}{15}{36}{15:6,\allowbreak7 2Ch 27:4-\allowbreak9}
\crossref{2King}{15}{37}{2Ki 10:32 1Sa 3:12 Jer 25:29 Lu 21:28}
\crossref{2King}{15}{38}{2Sa 7:12 1Ki 1:2; 14:20,\allowbreak31}
\crossref{2King}{16}{1}{2Ki 15:27-\allowbreak30,\allowbreak32,\allowbreak33}
\crossref{2King}{16}{2}{2Ki 14:3; 15:3,\allowbreak34; 18:3; 22:2 1Ki 3:14; 9:4; 11:4-\allowbreak8; 15:3 2Ch 17:3}
\crossref{2King}{16}{3}{2Ki 8:18 1Ki 12:28-\allowbreak30; 16:31-\allowbreak33; 21:25,\allowbreak26; 22:52,\allowbreak53 2Ch 22:3; 28:2-\allowbreak4}
\crossref{2King}{16}{4}{De 12:2 1Ki 14:23 Isa 57:5-\allowbreak7; 65:4; 66:17 Jer 17:2 Eze 20:28,\allowbreak29}
\crossref{2King}{16}{5}{2Ki 15:37 2Ch 28:5-\allowbreak15 Isa 7:1,\allowbreak2-\allowbreak9}
\crossref{2King}{16}{6}{2Ki 14:22 De 2:8}
\crossref{2King}{16}{7}{2Ki 15:29 1Ch 5:26 2Ch 28:20}
\crossref{2King}{16}{8}{16:17,\allowbreak18; 12:17,\allowbreak18; 18:15,\allowbreak16 2Ch 16:2; 28:20,\allowbreak21}
\crossref{2King}{16}{9}{2Ch 28:5}
\crossref{2King}{16}{10}{De 12:30 2Ch 28:23-\allowbreak25 Jer 10:2 Eze 23:16,\allowbreak17 Ro 12:2 1Pe 1:18}
\crossref{2King}{16}{11}{1Ki 21:11-\allowbreak13 2Ch 26:17,\allowbreak18 Jer 23:11 Eze 22:26 Da 3:7 Ho 4:6}
\crossref{2King}{16}{12}{1Ki 13:1 2Ch 26:16-\allowbreak19; 28:23,\allowbreak25}
\crossref{2King}{16}{13}{Le 1:1-\allowbreak3:16}
\crossref{2King}{16}{14}{Ex 40:6,\allowbreak29 2Ch 1:5; 4:1 Mt 23:35}
\crossref{2King}{16}{15}{2Ki 3:20 Ex 29:39-\allowbreak41 Nu 28:2-\allowbreak10 Da 9:21,\allowbreak27; 11:31; 12:11}
\crossref{2King}{16}{16}{16:11 Ac 4:19; 5:29 1Th 2:4 Jude 1:11}
\crossref{2King}{16}{17}{2Ch 28:24; 29:19}
\crossref{2King}{16}{18}{}
\crossref{2King}{16}{19}{2Ki 15:6,\allowbreak7,\allowbreak36,\allowbreak38; 20:20,\allowbreak21 1Ki 14:29}
\crossref{2King}{16}{20}{2Ki 21:18,\allowbreak26 2Ch 28:27}
\crossref{2King}{17}{1}{}
\crossref{2King}{17}{2}{2Ki 3:2; 10:31; 13:2,\allowbreak11; 15:9,\allowbreak18,\allowbreak24 2Ch 30:5-\allowbreak11}
\crossref{2King}{17}{3}{2Ki 15:19,\allowbreak29; 16:7; 18:13; 19:36,\allowbreak37 Isa 7:7,\allowbreak8; 10:5,\allowbreak6,\allowbreak11,\allowbreak12}
\crossref{2King}{17}{4}{2Ki 24:1,\allowbreak20 Eze 17:13-\allowbreak19}
\crossref{2King}{17}{5}{2Ki 18:9}
\crossref{2King}{17}{6}{2Ki 18:10,\allowbreak11 Ho 1:6,\allowbreak9; 13:16}
\crossref{2King}{17}{7}{De 31:16,\allowbreak17,\allowbreak29; 32:15-\allowbreak52 Jos 23:16 Jud 2:14-\allowbreak17 2Ch 36:14-\allowbreak16}
\crossref{2King}{17}{8}{2Ki 16:3,\allowbreak10; 21:2 Le 18:3,\allowbreak27-\allowbreak30 De 12:30,\allowbreak31; 18:9 1Ki 12:28}
\crossref{2King}{17}{9}{De 13:6; 27:15 Job 31:27 Eze 8:12}
\crossref{2King}{17}{10}{2Ki 16:4 Ex 34:13 Le 26:1 1Ki 14:23 Isa 57:5}
\crossref{2King}{17}{11}{1Ki 13:1 2Ch 28:25 Jer 44:17}
\crossref{2King}{17}{12}{Ex 20:3-\allowbreak5; 34:14 Le 26:1 De 4:19; 5:7-\allowbreak9}
\crossref{2King}{17}{13}{De 8:19; 31:21 Ne 9:29,\allowbreak30 Ps 50:7; 81:8,\allowbreak9 Jer 42:19 Ac 20:21}
\crossref{2King}{17}{14}{De 31:27 2Ch 36:13 Pr 29:1 Isa 48:4 Jer 7:26 Ro 2:4,\allowbreak5}
\crossref{2King}{17}{15}{Jer 8:9}
\crossref{2King}{17}{16}{Ex 32:4,\allowbreak8 1Ki 12:28 Ps 106:18-\allowbreak20 Isa 44:9,\allowbreak10}
\crossref{2King}{17}{17}{2Ki 16:3; 21:6 Le 18:21 2Ch 28:3 Ps 106:37,\allowbreak38 Eze 20:26,\allowbreak31}
\crossref{2King}{17}{18}{2Ki 13:23; 23:27 De 29:20-\allowbreak28; 32:21-\allowbreak26 Jos 23:13,\allowbreak15 Jer 15:1 Ho 9:3}
\crossref{2King}{17}{19}{1Ki 14:22,\allowbreak23 2Ch 21:11,\allowbreak13 Jer 2:28; 3:8-\allowbreak11 Eze 16:51,\allowbreak52}
\crossref{2King}{17}{20}{17:15 1Sa 15:23,\allowbreak26; 16:1 Jer 6:30 Ro 11:1,\allowbreak2}
\crossref{2King}{17}{21}{1Ki 11:11,\allowbreak31; 14:8 Isa 7:17}
\crossref{2King}{17}{22}{2Ki 3:3; 10:29,\allowbreak31; 13:2,\allowbreak6,\allowbreak11; 15:9}
\crossref{2King}{17}{23}{17:18,\allowbreak20}
\crossref{2King}{17}{24}{Ezr 4:2-\allowbreak10}
\crossref{2King}{17}{25}{17:28,\allowbreak32,\allowbreak34,\allowbreak41 Jos 22:25 Jer 10:7 Da 6:26 Jon 1:9}
\crossref{2King}{17}{26}{17:24}
\crossref{2King}{17}{27}{Jud 17:13 1Ki 12:31; 13:2 2Ch 11:15}
\crossref{2King}{17}{28}{1Ki 12:29-\allowbreak32}
\crossref{2King}{17}{29}{Ps 115:4-\allowbreak8; 135:15-\allowbreak18 Isa 44:9-\allowbreak20 Jer 10:3-\allowbreak5 Ho 8:5,\allowbreak6 Mic 4:5}
\crossref{2King}{17}{30}{17:24}
\crossref{2King}{17}{31}{17:24 Ezr 4:9}
\crossref{2King}{17}{32}{1Ki 12:31; 13:33}
\crossref{2King}{17}{33}{17:41 1Ki 18:21 Ho 10:2 Zep 1:5 Mt 6:24 Lu 16:13}
\crossref{2King}{17}{34}{17:25,\allowbreak27,\allowbreak28,\allowbreak33}
\crossref{2King}{17}{35}{17:15 Ex 19:5,\allowbreak6; 24:6-\allowbreak8 De 29:10-\allowbreak15 Jer 31:31-\allowbreak34 Heb 8:6-\allowbreak13}
\crossref{2King}{17}{36}{Ex 6:6; 9:15 De 5:15 Jer 32:21 Ac 4:30}
\crossref{2King}{17}{37}{Le 19:37 De 4:44,\allowbreak45; 5:31-\allowbreak33; 6:1,\allowbreak2; 12:32 1Ch 29:19 Ps 19:8-\allowbreak11}
\crossref{2King}{17}{38}{De 4:23; 6:12; 8:14-\allowbreak18}
\crossref{2King}{17}{39}{17:36 1Sa 12:24 Isa 8:12-\allowbreak14 Jer 10:7 Mt 10:28 Lu 1:50}
\crossref{2King}{17}{40}{Jer 13:23}
\crossref{2King}{17}{41}{17:32,\allowbreak33 Jos 24:14-\allowbreak20 1Ki 18:21 Zep 1:5 Mt 6:24 Re 3:15,\allowbreak16}
\crossref{2King}{18}{1}{18:9; 15:30; 17:1}
\crossref{2King}{18}{2}{2Ch 29:1}
\crossref{2King}{18}{3}{2Ki 20:3 Ex 15:26 De 6:18 2Ch 31:20,\allowbreak21 Job 33:27 Ps 119:128}
\crossref{2King}{18}{4}{2Ki 12:3; 14:4; 15:4,\allowbreak35 Le 26:30 1Ki 3:2,\allowbreak3; 15:14; 22:43 Ps 78:58}
\crossref{2King}{18}{5}{2Ki 19:10 2Ch 32:7,\allowbreak8 Job 13:15 Ps 13:5; 27:1,\allowbreak2; 46:1,\allowbreak2; 84:12}
\crossref{2King}{18}{6}{2Ki 17:13,\allowbreak16,\allowbreak19 Jer 11:4 Joh 14:15,\allowbreak21; 15:10,\allowbreak14 1Jo 5:3}
\crossref{2King}{18}{7}{Ge 21:22; 39:2,\allowbreak3 1Sa 18:14 2Ch 15:2 Ps 46:11; 60:12 Mt 1:23}
\crossref{2King}{18}{8}{1Ch 4:41 2Ch 28:18 Isa 14:29}
\crossref{2King}{18}{9}{18:1; 17:4-\allowbreak6}
\crossref{2King}{18}{10}{Ho 13:16 Am 3:11-\allowbreak15; 4:1-\allowbreak3; 6:7; 9:1-\allowbreak4 Mic 1:6-\allowbreak9; 6:16; 7:13}
\crossref{2King}{18}{11}{2Ki 17:6; 19:11 1Ch 5:26 Isa 7:8; 8:4; 9:9-\allowbreak21; 10:5,\allowbreak11; 37:12 Ho 8:8,\allowbreak9}
\crossref{2King}{18}{12}{2Ki 17:7-\allowbreak23 De 8:20; 11:28; 29:24-\allowbreak28; 31:17 Ne 9:17,\allowbreak26,\allowbreak27 Ps 107:17}
\crossref{2King}{18}{13}{2Ch 32:1-\allowbreak23 Isa 36:1-\allowbreak22}
\crossref{2King}{18}{14}{18:7 1Ki 20:4 Pr 29:25 Lu 14:32}
\crossref{2King}{18}{15}{2Ki 12:18; 16:8 1Ki 15:15,\allowbreak18,\allowbreak19 2Ch 16:2}
\crossref{2King}{18}{16}{1Ki 6:31-\allowbreak35 2Ch 29:3}
\crossref{2King}{18}{17}{2Ch 32:9 Isa 20:1; 36:2}
\crossref{2King}{18}{18}{2Ki 19:2 Isa 22:20-\allowbreak24; 36:3,\allowbreak22; 37:2}
\crossref{2King}{18}{19}{2Ch 32:10 Isa 10:8-\allowbreak14; 36:4; 37:13 Da 4:30}
\crossref{2King}{18}{20}{18:14}
\crossref{2King}{18}{21}{Isa 36:6 Eze 29:6,\allowbreak7}
\crossref{2King}{18}{22}{18:5 Da 3:15 Mt 27:43}
\crossref{2King}{18}{23}{1Sa 17:42,\allowbreak44 1Ki 20:10,\allowbreak18 Ne 4:2-\allowbreak5 Ps 123:3,\allowbreak4 Isa 10:13,\allowbreak14}
\crossref{2King}{18}{24}{Isa 10:8 Da 2:37,\allowbreak38; 4:22,\allowbreak37}
\crossref{2King}{18}{25}{2Ki 19:6,\allowbreak22-\allowbreak37 1Ki 13:18 2Ch 35:21 Isa 10:5,\allowbreak6 Am 3:6 Joh 19:10,\allowbreak11}
\crossref{2King}{18}{26}{Ezr 4:7 Isa 36:11,\allowbreak12 Da 2:4}
\crossref{2King}{18}{27}{2Ki 6:25 De 28:53-\allowbreak57 Ps 73:8 La 4:5 Eze 4:13,\allowbreak15}
\crossref{2King}{18}{28}{2Ch 32:18 Isa 36:13-\allowbreak18}
\crossref{2King}{18}{29}{Ps 73:8,\allowbreak9}
\crossref{2King}{18}{30}{18:22}
\crossref{2King}{18}{31}{1Ki 4:20,\allowbreak25 Zec 3:10}
\crossref{2King}{18}{32}{18:11; 17:6,\allowbreak23; 24:14-\allowbreak16; 25:11}
\crossref{2King}{18}{33}{2Ki 19:12,\allowbreak13,\allowbreak17,\allowbreak18 2Ch 32:14-\allowbreak17,\allowbreak19 Isa 10:10; 36:18-\allowbreak20}
\crossref{2King}{18}{34}{2Ki 19:13 Nu 13:21 2Sa 8:9 Jer 49:23}
\crossref{2King}{18}{35}{2Ki 19:17 Da 3:15}
\crossref{2King}{18}{36}{Ps 38:13,\allowbreak14; 39:1 Pr 9:7; 26:4 Am 5:13 Mt 7:6}
\crossref{2King}{18}{37}{2Ki 5:7; 22:11,\allowbreak19 Ge 37:29,\allowbreak34 Job 1:20 Isa 33:7; 36:21,\allowbreak22 Jer 36:24}
\crossref{2King}{19}{1}{Isa 37:1-\allowbreak7}
\crossref{2King}{19}{2}{2Ki 18:18; 22:13,\allowbreak14 Isa 37:2-\allowbreak5}
\crossref{2King}{19}{3}{2Ki 18:29 Ps 39:11; 123:3,\allowbreak4 Jer 30:5-\allowbreak7 Ho 5:15; 6:1}
\crossref{2King}{19}{4}{Ge 22:14 De 32:36 Jos 14:12 1Sa 14:6 2Sa 16:12}
\crossref{2King}{19}{5}{}
\crossref{2King}{19}{6}{Isa 37:6,\allowbreak7-\allowbreak38}
\crossref{2King}{19}{7}{19:35-\allowbreak37 Job 4:9 Ps 11:6; 18:14,\allowbreak15; 50:3 Isa 10:16-\allowbreak18; 11:4}
\crossref{2King}{19}{8}{2Ki 8:22 Jos 10:29; 12:15; 15:42}
\crossref{2King}{19}{9}{2Ki 18:17}
\crossref{2King}{19}{10}{2Ki 18:5,\allowbreak29,\allowbreak30 2Ch 32:15-\allowbreak19 Isa 37:10-\allowbreak14}
\crossref{2King}{19}{11}{19:17,\allowbreak18; 17:5-\allowbreak11 2Ch 32:13,\allowbreak14 Isa 10:8-\allowbreak11}
\crossref{2King}{19}{12}{2Ki 18:33,\allowbreak34}
\crossref{2King}{19}{13}{2Ki 17:24 Nu 13:21; 34:8 Isa 11:11 Jer 39:5; 49:23 Zec 9:2}
\crossref{2King}{19}{14}{Isa 37:14}
\crossref{2King}{19}{15}{2Sa 7:18-\allowbreak28 2Ch 14:11; 20:6; 32:20 Da 9:3,\allowbreak4}
\crossref{2King}{19}{16}{Ps 31:2 Isa 37:17}
\crossref{2King}{19}{17}{Job 9:2 Isa 5:9 Jer 26:15 Da 2:47 Mt 14:33 Lu 22:59 Ac 4:27}
\crossref{2King}{19}{18}{2Sa 5:21 Isa 46:1,\allowbreak2}
\crossref{2King}{19}{19}{Ex 9:15,\allowbreak16 Jos 7:9 1Sa 17:45-\allowbreak47 1Ki 8:43; 18:36,\allowbreak37; 20:28}
\crossref{2King}{19}{20}{2Sa 15:31; 17:23}
\crossref{2King}{19}{21}{Isa 23:12; 37:21,\allowbreak22-\allowbreak35; 47:1 Jer 14:17; 18:13; 31:4 La 1:15; 2:13}
\crossref{2King}{19}{22}{2Ki 18:28-\allowbreak35 Ex 5:2 Ps 73:9; 74:22,\allowbreak23}
\crossref{2King}{19}{23}{2Ki 18:17 2Ch 32:17}
\crossref{2King}{19}{24}{}
\crossref{2King}{19}{25}{}
\crossref{2King}{19}{26}{Nu 11:23; 14:9 Ps 48:4-\allowbreak7; 127:1 Jer 37:10; 50:36,\allowbreak37; 51:30,\allowbreak32}
\crossref{2King}{19}{27}{Ps 139:1-\allowbreak11 Jer 23:23,\allowbreak24}
\crossref{2King}{19}{28}{Ps 2:1-\allowbreak5; 7:6; 10:13,\allowbreak14; 46:6; 93:3,\allowbreak4 Lu 6:11 Joh 15:18,\allowbreak23,\allowbreak24}
\crossref{2King}{19}{29}{19:21,\allowbreak31-\allowbreak34; 20:8,\allowbreak9 Ex 3:12 1Sa 2:34 Isa 7:11-\allowbreak14 Lu 2:12}
\crossref{2King}{19}{30}{Ps 80:9 Isa 27:6; 37:31,\allowbreak32}
\crossref{2King}{19}{31}{19:4 Jer 44:14 Ro 9:27; 11:5}
\crossref{2King}{19}{32}{Isa 8:7-\allowbreak10; 10:24,\allowbreak25,\allowbreak28-\allowbreak32; 37:33-\allowbreak35}
\crossref{2King}{19}{33}{19:28,\allowbreak36}
\crossref{2King}{19}{34}{2Ki 20:6 Ps 46:5,\allowbreak6; 48:2-\allowbreak8 Isa 31:5; 38:6}
\crossref{2King}{19}{35}{Ex 12:29 Da 5:30 1Th 5:2,\allowbreak3}
\crossref{2King}{19}{36}{19:7,\allowbreak28,\allowbreak33}
\crossref{2King}{19}{37}{19:10; 18:5,\allowbreak30 De 32:31 2Ch 32:14,\allowbreak19 Isa 37:37,\allowbreak38}
\crossref{2King}{20}{1}{2Ch 32:24-\allowbreak26 Isa 38:1-\allowbreak20 Joh 11:1-\allowbreak5 Php 2:27,\allowbreak30}
\crossref{2King}{20}{2}{1Ki 8:30 Ps 50:15 Isa 38:2,\allowbreak3 Mt 6:6}
\crossref{2King}{20}{3}{Ge 8:1 Ne 5:19; 13:14,\allowbreak22,\allowbreak31 Ps 25:7; 89:47,\allowbreak50; 119:49 Isa 63:11}
\crossref{2King}{20}{4}{2Ki 22:14 1Ki 7:8}
\crossref{2King}{20}{5}{2Sa 7:3-\allowbreak5 1Ch 17:2-\allowbreak4}
\crossref{2King}{20}{6}{Ps 116:15 Ac 27:24}
\crossref{2King}{20}{7}{2Ki 2:20-\allowbreak22; 4:41 Isa 38:21}
\crossref{2King}{20}{8}{20:5; 19:29 Jud 6:17,\allowbreak37-\allowbreak40 Isa 7:11,\allowbreak14; 38:22 Ho 6:2}
\crossref{2King}{20}{9}{Isa 38:7,\allowbreak8 Mt 16:1-\allowbreak4 Mr 8:11,\allowbreak12 Lu 11:29,\allowbreak30}
\crossref{2King}{20}{10}{2Ki 2:10; 3:18 Isa 49:6 Mr 9:28,\allowbreak29 Joh 14:12}
\crossref{2King}{20}{11}{Ex 14:15 1Ki 17:20,\allowbreak21; 18:36-\allowbreak38 Ac 9:40}
\crossref{2King}{20}{12}{Isa 39:1-\allowbreak8}
\crossref{2King}{20}{13}{2Ch 32:27 Isa 39:2}
\crossref{2King}{20}{14}{Isa 39:3-\allowbreak8}
\crossref{2King}{20}{15}{20:13 Jos 7:19 Job 31:33 Pr 28:13 1Jo 1:8-\allowbreak10}
\crossref{2King}{20}{16}{2Ki 7:1 1Ki 22:19 Isa 1:10 Am 7:16}
\crossref{2King}{20}{17}{2Ki 24:13; 25:13-\allowbreak15 Le 26:19 2Ch 36:10,\allowbreak18 Jer 27:21,\allowbreak22; 52:17-\allowbreak19}
\crossref{2King}{20}{18}{2Ki 24:12; 25:6 2Ch 33:11}
\crossref{2King}{20}{19}{Le 10:3 1Sa 3:18 Job 1:21 Ps 39:9 La 3:22,\allowbreak39}
\crossref{2King}{20}{20}{2Ch 32:4,\allowbreak30,\allowbreak32 Ne 3:16 Isa 22:9-\allowbreak11}
\crossref{2King}{20}{21}{2Ki 21:18 1Ki 2:10; 11:43; 14:31 2Ch 26:23; 32:33}
\crossref{2King}{21}{1}{2Ki 20:21 1Ch 3:13 2Ch 32:33; 33:1-\allowbreak9 Mt 1:10}
\crossref{2King}{21}{2}{21:7,\allowbreak16; 16:2-\allowbreak4; 22:17 2Ch 33:2-\allowbreak4}
\crossref{2King}{21}{3}{2Ki 18:4,\allowbreak22 2Ch 32:12; 34:3}
\crossref{2King}{21}{4}{2Ki 16:10-\allowbreak16 Jer 32:34}
\crossref{2King}{21}{5}{2Ki 23:4,\allowbreak6 1Ki 6:36; 7:12 2Ch 33:5,\allowbreak15 Eze 40:28,\allowbreak32,\allowbreak37,\allowbreak47; 42:3; 43:5}
\crossref{2King}{21}{6}{2Ki 16:3; 17:17 Le 18:21; 20:2,\allowbreak3 2Ch 28:3; 33:6 Mic 6:7}
\crossref{2King}{21}{7}{2Ki 23:6 2Ch 33:7,\allowbreak15}
\crossref{2King}{21}{8}{2Ki 18:11 2Sa 7:10 1Ch 17:9 2Ch 33:8}
\crossref{2King}{21}{9}{2Ch 36:16 Ezr 9:10,\allowbreak11 Ne 9:26,\allowbreak29,\allowbreak30 Ps 81:10 Da 9:6,\allowbreak10,\allowbreak11}
\crossref{2King}{21}{10}{2Ch 33:10; 36:15 Ne 9:26,\allowbreak30 Mt 23:34-\allowbreak37}
\crossref{2King}{21}{11}{2Ki 23:26,\allowbreak27; 24:3,\allowbreak4 Jer 15:4}
\crossref{2King}{21}{12}{2Ki 22:16 Da 9:12 Mic 3:12}
\crossref{2King}{21}{13}{2Ki 10:11 1Ki 21:21-\allowbreak24}
\crossref{2King}{21}{14}{De 31:17 2Ch 15:2 Ps 37:28; 89:38-\allowbreak45 Jer 12:7; 23:33 La 5:20}
\crossref{2King}{21}{15}{De 9:21; 31:27,\allowbreak29 Jud 2:11-\allowbreak13 Ps 106:34-\allowbreak40 Eze 16:15-\allowbreak22}
\crossref{2King}{21}{16}{2Ki 24:3,\allowbreak4 Nu 35:33 De 21:8,\allowbreak9 Jer 2:34; 7:6; 15:4; 19:4 Mt 23:30,\allowbreak31}
\crossref{2King}{21}{17}{2Ki 20:20,\allowbreak21 2Ch 33:1-\allowbreak20}
\crossref{2King}{21}{18}{2Ch 21:20; 24:16,\allowbreak25; 28:27; 32:33; 33:20 Jer 22:19}
\crossref{2King}{21}{19}{1Ch 3:14 2Ch 33:21-\allowbreak23 Mt 1:10}
\crossref{2King}{21}{20}{21:2-\allowbreak7 Nu 32:14 2Ch 33:22,\allowbreak23 Mt 23:32 Ac 7:51}
\crossref{2King}{21}{21}{21:21}
\crossref{2King}{21}{22}{2Ki 22:17 De 32:15 1Ki 11:33 1Ch 28:9 Jer 2:13 Jon 2:8}
\crossref{2King}{21}{23}{2Ki 12:20; 14:19; 15:25,\allowbreak30 1Ki 15:27; 16:9 2Ch 33:24,\allowbreak25}
\crossref{2King}{21}{24}{2Ki 14:5}
\crossref{2King}{21}{25}{21:17; 20:20}
\crossref{2King}{21}{26}{21:18}
\crossref{2King}{22}{1}{2Ki 11:21; 21:1 Ps 8:2 Ec 10:16 Isa 3:4}
\crossref{2King}{22}{2}{2Ki 16:2; 18:3 2Ch 17:3; 29:2 Pr 20:11}
\crossref{2King}{22}{3}{2Ch 34:3-\allowbreak8,\allowbreak9-\allowbreak33}
\crossref{2King}{22}{4}{1Ch 6:13; 9:11 2Ch 34:9-\allowbreak18}
\crossref{2King}{22}{5}{2Ki 12:11-\allowbreak14}
\crossref{2King}{22}{6}{Ex 28:11; 35:35; 38:23}
\crossref{2King}{22}{7}{2Ki 12:15 2Ch 24:14}
\crossref{2King}{22}{8}{}
\crossref{2King}{22}{9}{22:3,\allowbreak12; 25:22 Jer 26:24; 29:3; 36:10-\allowbreak12; 39:14; 40:11; 41:2 Eze 8:11}
\crossref{2King}{22}{10}{De 31:9-\allowbreak13 2Ch 34:18 Ne 8:1-\allowbreak7,\allowbreak14,\allowbreak15,\allowbreak18; 13:1 Jer 36:6,\allowbreak15,\allowbreak21}
\crossref{2King}{22}{11}{22:19 2Ch 34:19 Jer 36:24 Joe 2:13 Jon 3:6,\allowbreak7}
\crossref{2King}{22}{12}{2Ki 19:2,\allowbreak3 2Ch 34:19-\allowbreak21 Isa 37:1-\allowbreak4}
\crossref{2King}{22}{13}{2Ki 3:11 1Ki 22:7,\allowbreak8 1Ch 10:13,\allowbreak14 Ps 25:14 Pr 3:6 Jer 21:1,\allowbreak2; 37:17}
\crossref{2King}{22}{14}{Ex 15:20 Jud 4:4 Mic 6:4 Lu 1:41-\allowbreak56; 2:36 Ac 21:9 1Co 11:5}
\crossref{2King}{22}{15}{2Ki 1:6,\allowbreak16 Jer 23:28}
\crossref{2King}{22}{16}{2Ki 20:17; 21:12,\allowbreak13 2Ch 34:24,\allowbreak25}
\crossref{2King}{22}{17}{Ex 32:34 De 29:24-\allowbreak28; 32:15-\allowbreak19 Jud 2:12-\allowbreak14; 3:7,\allowbreak8; 10:6,\allowbreak7,\allowbreak10-\allowbreak14}
\crossref{2King}{22}{18}{2Ch 34:26-\allowbreak28}
\crossref{2King}{22}{19}{1Sa 24:5 Ps 51:17; 119:120 Isa 46:12; 57:15; 66:2,\allowbreak5 Jer 36:24}
\crossref{2King}{22}{20}{Ge 25:8 De 31:16 1Ch 17:11 2Ch 34:28}
\crossref{2King}{23}{1}{De 31:28 2Sa 6:1 2Ch 29:20; 30:2; 34:29,\allowbreak30-\allowbreak33}
\crossref{2King}{23}{2}{Ge 19:11 1Sa 5:9; 30:2 2Ch 15:13 Es 1:5 Job 3:19 Ps 115:13}
\crossref{2King}{23}{3}{2Ki 11:14,\allowbreak17 2Ch 23:13; 34:31,\allowbreak32}
\crossref{2King}{23}{4}{2Ki 22:4 1Ch 26:1-\allowbreak19}
\crossref{2King}{23}{5}{}
\crossref{2King}{23}{6}{Ex 32:20 De 7:25; 9:21}
\crossref{2King}{23}{7}{Ge 19:4,\allowbreak5 1Ki 14:24; 15:12; 22:46 2Ch 34:33 Ro 1:26,\allowbreak27}
\crossref{2King}{23}{8}{Jos 21:17 1Ki 15:22 1Ch 6:60 Isa 10:29 Zec 14:10}
\crossref{2King}{23}{9}{Eze 44:10-\allowbreak14 Mal 2:8,\allowbreak9}
\crossref{2King}{23}{10}{Isa 30:33 Jer 7:31,\allowbreak32; 19:6,\allowbreak11-\allowbreak13}
\crossref{2King}{23}{11}{23:5 2Ch 14:5; 34:4 Eze 8:16}
\crossref{2King}{23}{12}{De 22:8 Jer 19:13 Zep 1:5}
\crossref{2King}{23}{13}{1Ki 11:7 Ne 13:26}
\crossref{2King}{23}{14}{Ex 23:24 Nu 33:52 De 7:5,\allowbreak25,\allowbreak26 2Ch 34:3,\allowbreak4 Mic 1:7}
\crossref{2King}{23}{15}{2Ki 10:31 1Ki 12:28-\allowbreak33; 14:16; 15:30; 21:22}
\crossref{2King}{23}{16}{1Ki 13:1,\allowbreak2,\allowbreak32 Mt 24:35 Joh 10:35}
\crossref{2King}{23}{17}{1Ki 13:1,\allowbreak30,\allowbreak31}
\crossref{2King}{23}{18}{1Ki 13:1-\allowbreak22,\allowbreak31}
\crossref{2King}{23}{19}{2Ki 17:9 1Ki 12:31; 13:32}
\crossref{2King}{23}{20}{2Ki 10:25; 11:18 Ex 22:20 De 13:5 1Ki 13:2; 18:40 Isa 34:6}
\crossref{2King}{23}{21}{2Ch 35:1-\allowbreak19}
\crossref{2King}{23}{22}{2Ch 35:18,\allowbreak19}
\crossref{2King}{23}{23}{}
\crossref{2King}{23}{24}{2Ki 21:3,\allowbreak6 1Sa 28:3-\allowbreak7 Isa 8:19; 19:3 Ac 16:16-\allowbreak18 Re 22:15}
\crossref{2King}{23}{25}{2Ki 18:5}
\crossref{2King}{23}{26}{2Ki 21:11-\allowbreak13; 22:16,\allowbreak17; 24:2,\allowbreak4 2Ch 36:16 Jer 3:7-\allowbreak10; 15:1-\allowbreak4}
\crossref{2King}{23}{27}{2Ki 17:18,\allowbreak20; 18:11; 21:13; 24:3; 25:11 De 29:27,\allowbreak28 Eze 23:32-\allowbreak35}
\crossref{2King}{23}{28}{2Ki 20:20}
\crossref{2King}{23}{29}{2Ki 24:7 2Ch 35:20 Jer 46:2}
\crossref{2King}{23}{30}{2Ki 9:28 1Ki 22:33-\allowbreak38 2Ch 35:24}
\crossref{2King}{23}{31}{1Ch 3:15 Jer 22:11}
\crossref{2King}{23}{32}{2Ki 21:2-\allowbreak7,\allowbreak21,\allowbreak22}
\crossref{2King}{23}{33}{2Ch 36:3,\allowbreak4 Eze 19:3,\allowbreak4}
\crossref{2King}{23}{34}{Jos 18:18 2Ch 36:3,\allowbreak4}
\crossref{2King}{23}{35}{23:33}
\crossref{2King}{23}{36}{1Ch 3:15 2Ch 36:5 Jer 1:3}
\crossref{2King}{23}{37}{Jer 22:13-\allowbreak17; 26:20-\allowbreak23; 36:23-\allowbreak26,\allowbreak31 Eze 19:5-\allowbreak9}
\crossref{2King}{24}{1}{2Ki 17:5 2Ch 36:6-\allowbreak21 Jer 25:1,\allowbreak9; 46:2 Da 1:1}
\crossref{2King}{24}{2}{2Ki 6:23; 13:20,\allowbreak21 De 28:49,\allowbreak50 2Ch 33:11 Job 1:17 Isa 7:17; 13:5}
\crossref{2King}{24}{3}{2Ki 18:25 Ge 50:20 2Ch 24:24; 25:16 Isa 10:5,\allowbreak6; 45:7; 46:10,\allowbreak11}
\crossref{2King}{24}{4}{2Ki 21:16 Nu 35:33 De 19:10 Jer 2:34; 19:4}
\crossref{2King}{24}{5}{2Ch 36:8 Jer 22:13-\allowbreak17; 26:1-\allowbreak36:32}
\crossref{2King}{24}{6}{}
\crossref{2King}{24}{7}{Jer 37:5-\allowbreak7; 46:2}
\crossref{2King}{24}{8}{1Ch 3:16 Jer 24:1}
\crossref{2King}{24}{9}{24:19 2Ch 36:12}
\crossref{2King}{24}{10}{Da 1:1,\allowbreak2}
\crossref{2King}{24}{11}{}
\crossref{2King}{24}{12}{2Ch 36:10 Jer 24:1; 29:1,\allowbreak2; 38:17,\allowbreak18 Eze 17:12}
\crossref{2King}{24}{13}{2Ki 20:17 Isa 39:6 Jer 20:5}
\crossref{2King}{24}{14}{2Ch 36:9,\allowbreak10 Jer 24:1-\allowbreak5; 52:28 Eze 1:1,\allowbreak2}
\crossref{2King}{24}{15}{24:8 2Ch 36:10 Es 2:6 Jer 22:24-\allowbreak28}
\crossref{2King}{24}{16}{Jer 29:2; 52:28}
\crossref{2King}{24}{17}{2Ch 36:10,\allowbreak11 Jer 37:1; 52:1}
\crossref{2King}{24}{18}{2Ch 36:11 Jer 37:1; 52:1-\allowbreak11}
\crossref{2King}{24}{19}{2Ki 23:37 2Ch 36:12 Jer 24:8; 37:1-\allowbreak38:28 Eze 21:25}
\crossref{2King}{24}{20}{2Ki 22:17 Ex 9:14-\allowbreak17 De 2:30 Isa 19:11-\allowbreak14 1Co 1:20 2Th 2:9-\allowbreak11}
\crossref{2King}{25}{1}{2Ki 24:1,\allowbreak10 1Ch 6:15 Jer 27:8; 32:28; 43:10; 51:34 Eze 26:7}
\crossref{2King}{25}{2}{}
\crossref{2King}{25}{3}{Jer 39:2; 52:6 Zec 8:19}
\crossref{2King}{25}{4}{Le 26:17,\allowbreak36 De 28:25; 32:24,\allowbreak25,\allowbreak30 Jer 39:4-\allowbreak7}
\crossref{2King}{25}{5}{Isa 30:16 Jer 24:8; 39:5; 52:8 Am 2:14-\allowbreak16}
\crossref{2King}{25}{6}{2Ch 33:11 Jer 21:7; 34:21,\allowbreak22; 38:23 La 4:19,\allowbreak20 Eze 17:20,\allowbreak21}
\crossref{2King}{25}{7}{Ge 21:16; 44:34 De 28:34 Jer 22:30; 39:6,\allowbreak7; 52:10,\allowbreak11}
\crossref{2King}{25}{8}{25:27; 24:12}
\crossref{2King}{25}{9}{1Ki 9:8 2Ch 36:19 Ps 74:3-\allowbreak7; 79:1 Isa 64:10,\allowbreak11 Jer 7:14; 26:9}
\crossref{2King}{25}{10}{Ne 1:3 Jer 5:10; 39:8; 52:14-\allowbreak23}
\crossref{2King}{25}{11}{Jer 15:1,\allowbreak2; 39:9; 52:12 Eze 5:2; 12:15,\allowbreak16; 22:15,\allowbreak16}
\crossref{2King}{25}{12}{2Ki 24:14 Jer 39:10; 40:7; 52:16 Eze 33:24}
\crossref{2King}{25}{13}{2Ki 20:17 2Ch 36:18 Jer 27:19-\allowbreak22; 52:17-\allowbreak20 La 1:10}
\crossref{2King}{25}{14}{Ex 27:3; 38:3 1Ki 7:47-\allowbreak50 2Ch 4:20-\allowbreak22; 24:14}
\crossref{2King}{25}{15}{Ex 37:23 Nu 7:13,\allowbreak14 1Ki 7:48-\allowbreak51 2Ch 24:14 Ezr 1:9-\allowbreak11 Da 5:2,\allowbreak3}
\crossref{2King}{25}{16}{1Ki 7:47}
\crossref{2King}{25}{17}{1Ki 7:15,\allowbreak16 Jer 52:21-\allowbreak23}
\crossref{2King}{25}{18}{25:24,\allowbreak25,\allowbreak26}
\crossref{2King}{25}{19}{}
\crossref{2King}{25}{20}{Jer 52:26,\allowbreak27 La 4:16}
\crossref{2King}{25}{21}{2Ki 17:20; 23:27 Le 26:33-\allowbreak35 De 4:26; 28:36,\allowbreak64 Jer 24:9,\allowbreak10; 25:9-\allowbreak11}
\crossref{2King}{25}{22}{Jer 40:5,\allowbreak6-\allowbreak12}
\crossref{2King}{25}{23}{Jer 40:7-\allowbreak9,\allowbreak11,\allowbreak12}
\crossref{2King}{25}{24}{2Sa 14:11; 19:23 Jer 40:9,\allowbreak10 Eze 33:24-\allowbreak29}
\crossref{2King}{25}{25}{Zec 7:5; 8:19}
\crossref{2King}{25}{26}{Jer 41:16-\allowbreak18; 42:14-\allowbreak22; 43:4-\allowbreak7}
\crossref{2King}{25}{27}{Jer 24:5,\allowbreak6; 52:31-\allowbreak34}
\crossref{2King}{25}{28}{Jer 27:6-\allowbreak11 Da 2:37; 5:18,\allowbreak19}
\crossref{2King}{25}{29}{2Ki 24:12 Ge 41:14,\allowbreak42 Es 4:4; 8:15 Isa 61:3 Zec 3:4 Lu 15:22}
\crossref{2King}{25}{30}{Ne 11:23; 12:47 Da 1:5 Mt 6:11 Lu 11:3 Ac 6:1}

% 1Chr
\crossref{1Chr}{1}{1}{Ge 4:25,\allowbreak26; 5:3,\allowbreak8 Lu 3:38}
\crossref{1Chr}{1}{2}{Ge 5:12-\allowbreak14 Lu 3:37}
\crossref{1Chr}{1}{3}{Ge 5:21-\allowbreak24 Heb 11:5 Jude 1:14}
\crossref{1Chr}{1}{4}{Ge 5:32; 6:8,\allowbreak9; 7:1; 9:29 Isa 54:9,\allowbreak10 Eze 14:14 Mt 24:37,\allowbreak38}
\crossref{1Chr}{1}{5}{Ge 10:1-\allowbreak5 Eze 27:13; 38:2,\allowbreak3,\allowbreak6; 39:1}
\crossref{1Chr}{1}{6}{Ge 10:3}
\crossref{1Chr}{1}{7}{Ps 72:10 Isa 66:19}
\crossref{1Chr}{1}{8}{Ge 10:6,\allowbreak7}
\crossref{1Chr}{1}{9}{Ge 10:7}
\crossref{1Chr}{1}{10}{Ge 10:8-\allowbreak12 Mic 5:6}
\crossref{1Chr}{1}{11}{Ge 10:13,\allowbreak14}
\crossref{1Chr}{1}{12}{De 2:23 Jer 47:4 Am 9:7}
\crossref{1Chr}{1}{13}{Ge 9:22,\allowbreak25,\allowbreak26; 10:15-\allowbreak19}
\crossref{1Chr}{1}{14}{Ge 15:21 Ex 33:2; 34:11 Jud 1:21; 19:11 2Sa 24:16 Zec 9:7}
\crossref{1Chr}{1}{15}{Ex 3:8,\allowbreak17; 13:5 1Ki 9:20}
\crossref{1Chr}{1}{16}{Nu 34:8 1Ki 8:65}
\crossref{1Chr}{1}{17}{Ge 10:22-\allowbreak32; 11:10}
\crossref{1Chr}{1}{18}{Ge 10:24; 11:12-\allowbreak15}
\crossref{1Chr}{1}{19}{Ge 10:21,\allowbreak25; 11:16,\allowbreak17 Nu 24:24}
\crossref{1Chr}{1}{20}{Ge 10:26,\allowbreak27}
\crossref{1Chr}{1}{21}{Ge 10:27}
\crossref{1Chr}{1}{22}{Ge 10:28}
\crossref{1Chr}{1}{23}{Ge 10:29 1Ki 9:28; 10:11 1Ch 29:4 Job 22:24 Ps 45:9 Isa 13:12}
\crossref{1Chr}{1}{24}{Ge 11:10-\allowbreak26}
\crossref{1Chr}{1}{25}{Lu 3:35}
\crossref{1Chr}{1}{26}{Lu 3:35}
\crossref{1Chr}{1}{27}{Ge 11:27-\allowbreak32; 17:5 Jos 24:2 Ne 9:7}
\crossref{1Chr}{1}{28}{Ge 17:19-\allowbreak21; 21:2-\allowbreak5,\allowbreak12}
\crossref{1Chr}{1}{29}{Ge 25:12-\allowbreak16}
\crossref{1Chr}{1}{30}{Isa 21:11}
\crossref{1Chr}{1}{31}{Ge 25:15}
\crossref{1Chr}{1}{32}{Ge 25:1-\allowbreak4}
\crossref{1Chr}{1}{33}{Isa 60:6}
\crossref{1Chr}{1}{34}{Ge 21:2,\allowbreak3 Mt 1:2 Lu 3:34 Ac 7:8}
\crossref{1Chr}{1}{35}{Ge 36:4,\allowbreak5,\allowbreak9,\allowbreak10}
\crossref{1Chr}{1}{36}{1:53 Ge 36:11-\allowbreak15 Jer 49:7,\allowbreak20 Am 1:12 Ob 1:9 Hab 3:3}
\crossref{1Chr}{1}{37}{Ge 36:4}
\crossref{1Chr}{1}{38}{Ge 36:20,\allowbreak29,\allowbreak30}
\crossref{1Chr}{1}{39}{Ge 36:22 De 2:12,\allowbreak22}
\crossref{1Chr}{1}{40}{}
\crossref{1Chr}{1}{41}{Ge 46:7}
\crossref{1Chr}{1}{42}{Ge 36:27}
\crossref{1Chr}{1}{43}{Ge 36:31-\allowbreak39; 49:10 Nu 24:17-\allowbreak19}
\crossref{1Chr}{1}{44}{Isa 34:6; 63:1 Jer 49:13 Am 1:12 Mic 2:12}
\crossref{1Chr}{1}{45}{Ge 10:29}
\crossref{1Chr}{1}{46}{Ge 36:35}
\crossref{1Chr}{1}{47}{Ge 36:36}
\crossref{1Chr}{1}{48}{}
\crossref{1Chr}{1}{49}{Ge 36:38}
\crossref{1Chr}{1}{50}{}
\crossref{1Chr}{1}{51}{}
\crossref{1Chr}{1}{52}{Ge 36:41}
\crossref{1Chr}{1}{53}{Ge 36:11}
\crossref{1Chr}{1}{54}{Ge 36:41-\allowbreak43}
\crossref{1Chr}{2}{1}{Ge 32:28; 49:2}
\crossref{1Chr}{2}{2}{Ge 30:6}
\crossref{1Chr}{2}{3}{1Ch 9:5 Ge 38:2-\allowbreak10; 46:12 Nu 26:19}
\crossref{1Chr}{2}{4}{Ge 38:13-\allowbreak30 Ru 4:12 Mt 1:3}
\crossref{1Chr}{2}{5}{Ge 46:12 Nu 26:21 Ru 4:18 Mt 1:3 Lu 3:33}
\crossref{1Chr}{2}{6}{1Ch 2:6; 8:36; 9:42 Nu 25:14 1Ki 16:9,\allowbreak10,\allowbreak12,\allowbreak15,\allowbreak16,\allowbreak18,\allowbreak20 2Ki 9:31}
\crossref{1Chr}{2}{7}{1Ch 4:1}
\crossref{1Chr}{2}{8}{2:6}
\crossref{1Chr}{2}{9}{2:25-\allowbreak33}
\crossref{1Chr}{2}{10}{Ru 4:19,\allowbreak20 Mt 1:4 Lu 3:33}
\crossref{1Chr}{2}{11}{Ru 4:21 Mt 1:4,\allowbreak5 Lu 3:32}
\crossref{1Chr}{2}{12}{1Ch 10:14 Ru 4:22 1Sa 16:1 Isa 11:1,\allowbreak10 Mt 1:5 Lu 3:32 Ac 13:22}
\crossref{1Chr}{2}{13}{1Sa 16:6-\allowbreak13; 17:13,\allowbreak28}
\crossref{1Chr}{2}{14}{}
\crossref{1Chr}{2}{15}{}
\crossref{1Chr}{2}{16}{1Sa 26:6 2Sa 2:18-\allowbreak23; 3:39; 16:9-\allowbreak11; 19:22}
\crossref{1Chr}{2}{17}{2Sa 17:25; 19:13; 20:4-\allowbreak12 1Ki 2:5,\allowbreak32}
\crossref{1Chr}{2}{18}{2:42}
\crossref{1Chr}{2}{19}{2:24,\allowbreak50}
\crossref{1Chr}{2}{20}{Ex 31:2; 36:1,\allowbreak2; 37:1; 38:22 2Ch 1:5}
\crossref{1Chr}{2}{21}{Ge 50:23 Nu 26:29; 27:1; 32:39,\allowbreak40 De 3:15}
\crossref{1Chr}{2}{22}{Nu 32:41 De 3:14 Jos 13:30}
\crossref{1Chr}{2}{23}{Jos 13:13 2Sa 13:38}
\crossref{1Chr}{2}{24}{2:9,\allowbreak18,\allowbreak19 1Sa 30:14}
\crossref{1Chr}{2}{25}{}
\crossref{1Chr}{2}{26}{}
\crossref{1Chr}{2}{27}{2:25}
\crossref{1Chr}{2}{28}{2:26}
\crossref{1Chr}{2}{29}{}
\crossref{1Chr}{2}{30}{2:28}
\crossref{1Chr}{2}{31}{2:34,\allowbreak35}
\crossref{1Chr}{2}{32}{2:28}
\crossref{1Chr}{2}{33}{Nu 16:1}
\crossref{1Chr}{2}{34}{Nu 27:3,\allowbreak4,\allowbreak8}
\crossref{1Chr}{2}{35}{2:31}
\crossref{1Chr}{2}{36}{1Ch 11:41}
\crossref{1Chr}{2}{37}{Ru 4:17}
\crossref{1Chr}{2}{38}{2Ki 9:2}
\crossref{1Chr}{2}{39}{2Sa 23:26}
\crossref{1Chr}{2}{40}{2Ki 15:10}
\crossref{1Chr}{2}{41}{1Ch 3:18}
\crossref{1Chr}{2}{42}{2:18,\allowbreak19,\allowbreak24}
\crossref{1Chr}{2}{43}{}
\crossref{1Chr}{2}{44}{}
\crossref{1Chr}{2}{45}{}
\crossref{1Chr}{2}{46}{2:18,\allowbreak19,\allowbreak48}
\crossref{1Chr}{2}{47}{}
\crossref{1Chr}{2}{48}{2:46 Ge 25:5,\allowbreak6}
\crossref{1Chr}{2}{49}{2:42}
\crossref{1Chr}{2}{50}{2:19,\allowbreak20}
\crossref{1Chr}{2}{51}{1Ch 4:4}
\crossref{1Chr}{2}{52}{1Ch 4:2}
\crossref{1Chr}{2}{53}{1Ch 11:40 2Sa 23:38}
\crossref{1Chr}{2}{54}{2:51}
\crossref{1Chr}{2}{55}{Ezr 7:6 Jer 8:8}
\crossref{1Chr}{3}{1}{2Sa 3:2-\allowbreak5}
\crossref{1Chr}{3}{2}{2Sa 13:1,\allowbreak20-\allowbreak28,\allowbreak38; 18:14,\allowbreak18,\allowbreak33; 19:4-\allowbreak10}
\crossref{1Chr}{3}{3}{}
\crossref{1Chr}{3}{4}{2Sa 2:11; 5:4,\allowbreak5 1Ki 2:11}
\crossref{1Chr}{3}{5}{1Ch 14:4 2Sa 5:14}
\crossref{1Chr}{3}{6}{1Ch 14:5 2Sa 5:15}
\crossref{1Chr}{3}{7}{2Sa 5:15,\allowbreak16}
\crossref{1Chr}{3}{8}{1Ch 14:7}
\crossref{1Chr}{3}{9}{2Sa 5:13}
\crossref{1Chr}{3}{10}{1Ki 11:43; 14:31; 15:6 Mt 1:7}
\crossref{1Chr}{3}{11}{1Ki 22:50 2Ch 21:1}
\crossref{1Chr}{3}{12}{2Ki 14:1 2Ch 25:1}
\crossref{1Chr}{3}{13}{2Ki 16:1 2Ch 28:1-\allowbreak8 Mt 1:9}
\crossref{1Chr}{3}{14}{2Ki 21:19 2Ch 33:20,\allowbreak21}
\crossref{1Chr}{3}{15}{}
\crossref{1Chr}{3}{16}{2Ki 24:6,\allowbreak8; 25:27 2Ch 36:9}
\crossref{1Chr}{3}{17}{Ezr 3:2,\allowbreak8; 5:2}
\crossref{1Chr}{3}{18}{}
\crossref{1Chr}{3}{19}{Ezr 2:2; 3:2 Hag 1:12-\allowbreak14; 2:2,\allowbreak4 Zec 4:6-\allowbreak9 Mt 1:12}
\crossref{1Chr}{3}{20}{3:20}
\crossref{1Chr}{3}{21}{Ne 10:22}
\crossref{1Chr}{3}{22}{Ezr 8:2}
\crossref{1Chr}{3}{23}{3:23}
\crossref{1Chr}{3}{24}{3:24; 24:12 Ezr 10:6,\allowbreak24,\allowbreak27,\allowbreak36 Ne 3:1,\allowbreak20,\allowbreak21; 12:10,\allowbreak22,\allowbreak23; 13:4,\allowbreak7,\allowbreak28}
\crossref{1Chr}{4}{1}{1Ch 2:5 Ge 38:29; 46:12 Nu 26:20,\allowbreak21 Ru 4:18 Mt 1:3 Lu 3:33}
\crossref{1Chr}{4}{2}{1Ch 2:52}
\crossref{1Chr}{4}{3}{Jud 15:11 2Ch 11:6}
\crossref{1Chr}{4}{4}{4:18,\allowbreak39 Jos 15:36}
\crossref{1Chr}{4}{5}{1Ch 2:24}
\crossref{1Chr}{4}{6}{4:6}
\crossref{1Chr}{4}{7}{4:7}
\crossref{1Chr}{4}{8}{Ge 34:19 Isa 43:4 Ac 17:11}
\crossref{1Chr}{4}{9}{1Ch 16:8 Ge 12:8 Job 12:4 Ps 55:16; 99:6; 116:2-\allowbreak4 Jer 33:3}
\crossref{1Chr}{4}{10}{}
\crossref{1Chr}{4}{11}{4:12}
\crossref{1Chr}{4}{12}{Jos 15:17 Jud 1:13; 3:9-\allowbreak11}
\crossref{1Chr}{4}{13}{}
\crossref{1Chr}{4}{14}{Nu 13:6,\allowbreak30; 14:6-\allowbreak10,\allowbreak24,\allowbreak30 Jos 14:6-\allowbreak14; 15:13-\allowbreak20 Jud 1:12-\allowbreak14}
\crossref{1Chr}{4}{15}{}
\crossref{1Chr}{4}{16}{}
\crossref{1Chr}{4}{17}{4:19}
\crossref{1Chr}{4}{18}{4:18}
\crossref{1Chr}{4}{19}{}
\crossref{1Chr}{4}{20}{}
\crossref{1Chr}{4}{21}{1Ch 2:3; 9:5 Ge 38:5; 46:12 Nu 26:20 Ne 11:5}
\crossref{1Chr}{4}{22}{4:22}
\crossref{1Chr}{4}{23}{4:14 Ps 81:6}
\crossref{1Chr}{4}{24}{Ge 46:10 Ex 6:15}
\crossref{1Chr}{4}{25}{4:25}
\crossref{1Chr}{4}{26}{}
\crossref{1Chr}{4}{27}{Nu 2:4,\allowbreak13; 26:14,\allowbreak22}
\crossref{1Chr}{4}{28}{}
\crossref{1Chr}{4}{29}{Jos 19:3}
\crossref{1Chr}{4}{30}{Jos 19:4}
\crossref{1Chr}{4}{31}{Jos 19:5,\allowbreak6}
\crossref{1Chr}{4}{32}{}
\crossref{1Chr}{4}{33}{Jos 19:8}
\crossref{1Chr}{4}{34}{4:34}
\crossref{1Chr}{4}{35}{4:35}
\crossref{1Chr}{4}{36}{4:36}
\crossref{1Chr}{4}{37}{}
\crossref{1Chr}{4}{38}{1Ch 5:24 Ge 6:4}
\crossref{1Chr}{4}{39}{4:4,\allowbreak18 Jos 12:13}
\crossref{1Chr}{4}{40}{Jud 18:7-\allowbreak10}
\crossref{1Chr}{4}{41}{4:33-\allowbreak38}
\crossref{1Chr}{4}{42}{Ge 36:8,\allowbreak9 De 1:2}
\crossref{1Chr}{4}{43}{De 34:6 Jud 1:26 2Ch 5:9 Jer 44:6 Mt 27:8; 28:15}
\crossref{1Chr}{5}{1}{1Ch 2:1 Ge 29:32; 46:8; 49:3 Ex 6:14 Nu 1:5; 16:1; 26:5}
\crossref{1Chr}{5}{2}{Ge 35:23; 49:8-\allowbreak10 Nu 2:3; 7:12 Jos 14:6 Jud 1:2 Ps 60:7; 108:8}
\crossref{1Chr}{5}{3}{Ge 46:9 Ex 6:14 Nu 26:5-\allowbreak9}
\crossref{1Chr}{5}{4}{1Sa 8:2}
\crossref{1Chr}{5}{5}{Mic 1:1}
\crossref{1Chr}{5}{6}{5:26 2Ki 15:29; 16:7}
\crossref{1Chr}{5}{7}{5:17}
\crossref{1Chr}{5}{8}{Ge 14:2}
\crossref{1Chr}{5}{9}{1Ch 18:3 Ge 2:14 2Ki 24:7}
\crossref{1Chr}{5}{10}{5:19,\allowbreak20 Ge 21:9; 25:12 2Sa 24:6 Ps 83:6}
\crossref{1Chr}{5}{11}{Ge 30:11}
\crossref{1Chr}{5}{12}{1Sa 8:2}
\crossref{1Chr}{5}{13}{Da 12:1}
\crossref{1Chr}{5}{14}{Nu 3:35}
\crossref{1Chr}{5}{15}{5:15; 7:34}
\crossref{1Chr}{5}{16}{Ge 31:23}
\crossref{1Chr}{5}{17}{5:7}
\crossref{1Chr}{5}{18}{1Ch 12:37 Ge 29:32 2Ki 10:33}
\crossref{1Chr}{5}{19}{5:10}
\crossref{1Chr}{5}{20}{5:22 Ex 17:11 Jos 10:14,\allowbreak42 1Sa 7:12; 19:15 Ps 46:1; 146:5,\allowbreak6}
\crossref{1Chr}{5}{21}{Nu 31:35 Eze 27:13 Re 18:13}
\crossref{1Chr}{5}{22}{Ex 14:14 Jos 23:10 Jud 3:2 2Ch 32:8 Ne 4:20 Ps 24:8 Pr 22:3}
\crossref{1Chr}{5}{23}{Ge 41:51}
\crossref{1Chr}{5}{24}{Ge 25:4}
\crossref{1Chr}{5}{25}{Jud 2:17; 8:33 2Ki 17:7-\allowbreak18 Ho 1:2; 9:1 Re 17:5}
\crossref{1Chr}{5}{26}{2Sa 24:1 2Ch 33:11 Ezr 1:5 Isa 10:5,\allowbreak6; 13:2-\allowbreak5}
\crossref{1Chr}{6}{1}{1Ch 23:6 Ge 46:11 Ex 6:16 Nu 3:17; 26:57}
\crossref{1Chr}{6}{2}{1Ch 23:12 Ex 6:18,\allowbreak21-\allowbreak24}
\crossref{1Chr}{6}{3}{1Ch 23:13 Ex 6:20}
\crossref{1Chr}{6}{4}{6:50; 9:20 Ex 6:25 Nu 25:6-\allowbreak11,\allowbreak13; 31:6 Jos 22:13,\allowbreak30-\allowbreak32; 24:33}
\crossref{1Chr}{6}{5}{}
\crossref{1Chr}{6}{6}{Ezr 7:4}
\crossref{1Chr}{6}{7}{Ezr 7:3}
\crossref{1Chr}{6}{8}{2Sa 8:17}
\crossref{1Chr}{6}{9}{6:36}
\crossref{1Chr}{6}{10}{2Ch 26:17-\allowbreak20}
\crossref{1Chr}{6}{11}{Ezr 7:3}
\crossref{1Chr}{6}{12}{1Ch 9:11 Ne 11:11}
\crossref{1Chr}{6}{13}{2Ki 22:12-\allowbreak14 2Ch 34:14-\allowbreak20; 35:8}
\crossref{1Chr}{6}{14}{}
\crossref{1Chr}{6}{15}{Ezr 5:2}
\crossref{1Chr}{6}{16}{6:1 Ex 6:16}
\crossref{1Chr}{6}{17}{1Ch 23:7 Nu 3:18,\allowbreak21}
\crossref{1Chr}{6}{18}{6:2,\allowbreak3; 23:12}
\crossref{1Chr}{6}{19}{1Ch 23:21; 24:26 Ex 6:19}
\crossref{1Chr}{6}{20}{6:17}
\crossref{1Chr}{6}{21}{6:42}
\crossref{1Chr}{6}{22}{6:2,\allowbreak18 Ex 6:21,\allowbreak24}
\crossref{1Chr}{6}{23}{}
\crossref{1Chr}{6}{24}{6:36}
\crossref{1Chr}{6}{25}{6:35,\allowbreak36 Ex 6:24}
\crossref{1Chr}{6}{26}{6:35 1Sa 1:1}
\crossref{1Chr}{6}{27}{6:34}
\crossref{1Chr}{6}{28}{1Ch 5:1 Le 27:26 Ge 14:20}
\crossref{1Chr}{6}{29}{6:19 Nu 3:33}
\crossref{1Chr}{6}{30}{2Ch 34:20}
\crossref{1Chr}{6}{31}{1Ch 15:16-\allowbreak22,\allowbreak27; 25:1-\allowbreak31}
\crossref{1Chr}{6}{32}{1Ch 16:4-\allowbreak6,\allowbreak37-\allowbreak42 Ps 68:24,\allowbreak25}
\crossref{1Chr}{6}{33}{1Ch 15:17,\allowbreak19; 16:41,\allowbreak42; 25:1-\allowbreak5 2Ch 5:12; 29:14 Ps 88:1}
\crossref{1Chr}{6}{34}{Ex 6:24}
\crossref{1Chr}{6}{35}{6:26}
\crossref{1Chr}{6}{36}{6:24}
\crossref{1Chr}{6}{37}{Nu 33:26}
\crossref{1Chr}{6}{38}{Ex 6:18 Nu 3:19}
\crossref{1Chr}{6}{39}{1Ch 15:17-\allowbreak19; 16:7; 25:2; 26:1 2Ki 18:18 2Ch 5:12; 20:14; 29:13,\allowbreak30}
\crossref{1Chr}{6}{40}{Da 12:1}
\crossref{1Chr}{6}{41}{6:21}
\crossref{1Chr}{6}{42}{6:21}
\crossref{1Chr}{6}{43}{6:20}
\crossref{1Chr}{6}{44}{6:32,\allowbreak39}
\crossref{1Chr}{6}{45}{1Ch 9:14}
\crossref{1Chr}{6}{46}{Ne 11:12}
\crossref{1Chr}{6}{47}{1Ch 23:23; 24:30 Nu 3:20}
\crossref{1Chr}{6}{48}{1Ch 23:2-\allowbreak32; 25:1-\allowbreak26:32 Nu 3:1-\allowbreak4:49 8:5-\allowbreak26 16:9,\allowbreak10 18:1-\allowbreak32}
\crossref{1Chr}{6}{49}{Ex 27:1-\allowbreak8; 30:1-\allowbreak7 Le 1:5,\allowbreak7-\allowbreak9; 8:1-\allowbreak10:20; 21:1-\allowbreak22:33 Nu 16:16-\allowbreak50}
\crossref{1Chr}{6}{50}{6:3-\allowbreak9; 24:1 Ex 6:23; 28:1 Le 10:16 Nu 3:4,\allowbreak32; 20:26-\allowbreak28; 27:22}
\crossref{1Chr}{6}{51}{6:5}
\crossref{1Chr}{6}{52}{6:6,\allowbreak7}
\crossref{1Chr}{6}{53}{6:8; 12:28; 23:16; 24:3,\allowbreak31 1Sa 2:35 2Sa 8:17; 15:24-\allowbreak27,\allowbreak35,\allowbreak36}
\crossref{1Chr}{6}{54}{Nu 35:1-\allowbreak8 Jos 21:3-\allowbreak8}
\crossref{1Chr}{6}{55}{Jos 14:13; 15:13; 21:11-\allowbreak13 Jud 1:20}
\crossref{1Chr}{6}{56}{Jos 14:13; 15:13}
\crossref{1Chr}{6}{57}{1Sa 22:10 2Ch 31:15}
\crossref{1Chr}{6}{58}{Jos 15:51; 21:15}
\crossref{1Chr}{6}{59}{Jos 15:10; 21:16 1Sa 6:12-\allowbreak19 Jer 43:13}
\crossref{1Chr}{6}{60}{1Ch 8:6 Jos 18:24; 21:17 1Sa 13:3}
\crossref{1Chr}{6}{61}{6:1,\allowbreak2,\allowbreak18,\allowbreak33}
\crossref{1Chr}{6}{62}{6:71-\allowbreak76 Ex 2:22 Jos 21:27-\allowbreak33}
\crossref{1Chr}{6}{63}{6:77-\allowbreak81 Ge 46:11 Nu 3:20 Jos 21:7,\allowbreak34-\allowbreak40}
\crossref{1Chr}{6}{64}{Jos 21:41,\allowbreak42}
\crossref{1Chr}{6}{65}{6:57-\allowbreak60}
\crossref{1Chr}{6}{66}{6:61 Jos 21:20-\allowbreak26}
\crossref{1Chr}{6}{67}{Ge 33:18; 35:4 Jos 20:7; 21:21}
\crossref{1Chr}{6}{68}{Jos 21:22}
\crossref{1Chr}{6}{69}{Jos 10:12}
\crossref{1Chr}{6}{70}{}
\crossref{1Chr}{6}{71}{De 4:43 Jos 20:8; 21:27}
\crossref{1Chr}{6}{72}{Jos 15:23; 19:37; 21:32 Jud 4:9}
\crossref{1Chr}{6}{73}{De 4:43 Jos 20:8 1Ch 6:73,\allowbreak80}
\crossref{1Chr}{6}{74}{Jos 19:26; 21:30}
\crossref{1Chr}{6}{75}{Jos 19:25,\allowbreak26; 21:31}
\crossref{1Chr}{6}{76}{Jos 12:22; 19:37; 20:7; 21:32 Jud 4:6}
\crossref{1Chr}{6}{77}{Jos 21:34-\allowbreak39}
\crossref{1Chr}{6}{78}{De 4:41-\allowbreak43 Jos 20:8; 21:36}
\crossref{1Chr}{6}{79}{Jos 13:18}
\crossref{1Chr}{6}{80}{6:73 Jos 21:38,\allowbreak39 1Ki 22:3-\allowbreak28 2Ki 9:1}
\crossref{1Chr}{6}{81}{Nu 21:25; 32:37 De 2:24 Jos 13:26 Ne 9:22 So 7:4}
\crossref{1Chr}{7}{1}{Ge 46:13}
\crossref{1Chr}{7}{2}{}
\crossref{1Chr}{7}{3}{7:3}
\crossref{1Chr}{7}{4}{1Ch 12:32}
\crossref{1Chr}{7}{5}{}
\crossref{1Chr}{7}{6}{7:10,\allowbreak11}
\crossref{1Chr}{7}{7}{1Ch 21:1-\allowbreak5 2Ch 17:17,\allowbreak18}
\crossref{1Chr}{7}{8}{Ge 46:21}
\crossref{1Chr}{7}{9}{7:7}
\crossref{1Chr}{7}{10}{Jud 3:15-\allowbreak30}
\crossref{1Chr}{7}{11}{2Ch 17:13-\allowbreak19}
\crossref{1Chr}{7}{12}{7:15 Ge 46:21}
\crossref{1Chr}{7}{13}{Ge 46:24 Nu 26:48}
\crossref{1Chr}{7}{14}{1Ch 2:21-\allowbreak23 Ge 50:23 Nu 26:29-\allowbreak34; 27:1; 32:30-\allowbreak42 De 3:13-\allowbreak15}
\crossref{1Chr}{7}{15}{7:12}
\crossref{1Chr}{7}{16}{7:16}
\crossref{1Chr}{7}{17}{1Sa 12:11}
\crossref{1Chr}{7}{18}{Nu 26:30}
\crossref{1Chr}{7}{19}{Jos 17:2}
\crossref{1Chr}{7}{20}{Nu 26:35,\allowbreak36}
\crossref{1Chr}{7}{21}{1Ch 2:36}
\crossref{1Chr}{7}{22}{Ge 37:34}
\crossref{1Chr}{7}{23}{}
\crossref{1Chr}{7}{24}{Jos 16:3,\allowbreak5 1Ki 9:17 2Ch 8:5}
\crossref{1Chr}{7}{25}{7:25 Nu 26:35}
\crossref{1Chr}{7}{26}{1Ch 23:7}
\crossref{1Chr}{7}{27}{Nu 13:8,\allowbreak16}
\crossref{1Chr}{7}{28}{Ge 28:19 Jos 16:2 Jud 1:22}
\crossref{1Chr}{7}{29}{Jos 17:7-\allowbreak11}
\crossref{1Chr}{7}{30}{}
\crossref{1Chr}{7}{31}{Ge 46:17}
\crossref{1Chr}{7}{32}{7:34}
\crossref{1Chr}{7}{33}{}
\crossref{1Chr}{7}{34}{7:32}
\crossref{1Chr}{7}{35}{7:35}
\crossref{1Chr}{7}{36}{7:36}
\crossref{1Chr}{7}{37}{}
\crossref{1Chr}{7}{38}{1Ch 2:32}
\crossref{1Chr}{7}{39}{7:39 Ezr 2:5 Ne 6:18; 7:10}
\crossref{1Chr}{7}{40}{1Ch 21:1-\allowbreak5 2Sa 24:1-\allowbreak9}
\crossref{1Chr}{8}{1}{1Ch 7:6-\allowbreak12 Ge 46:21}
\crossref{1Chr}{8}{2}{8:37 2Sa 21:16,\allowbreak18,\allowbreak20,\allowbreak22}
\crossref{1Chr}{8}{3}{}
\crossref{1Chr}{8}{4}{1Ch 6:4}
\crossref{1Chr}{8}{5}{Jud 3:15}
\crossref{1Chr}{8}{6}{1Ch 7:10 Jud 3:20-\allowbreak30; 4:1}
\crossref{1Chr}{8}{7}{8:4}
\crossref{1Chr}{8}{8}{Ge 46:23}
\crossref{1Chr}{8}{9}{Ge 10:29}
\crossref{1Chr}{8}{10}{8:10}
\crossref{1Chr}{8}{11}{Ge 46:23}
\crossref{1Chr}{8}{12}{}
\crossref{1Chr}{8}{13}{8:21}
\crossref{1Chr}{8}{14}{2Sa 6:3}
\crossref{1Chr}{8}{15}{1Ch 26:2}
\crossref{1Chr}{8}{16}{8:13}
\crossref{1Chr}{8}{17}{1Ch 26:2}
\crossref{1Chr}{8}{18}{Ge 10:29}
\crossref{1Chr}{8}{19}{8:19; 24:12}
\crossref{1Chr}{8}{20}{8:20}
\crossref{1Chr}{8}{21}{8:13}
\crossref{1Chr}{8}{22}{Ge 46:17}
\crossref{1Chr}{8}{23}{Jud 12:13}
\crossref{1Chr}{8}{24}{1Ch 25:23}
\crossref{1Chr}{8}{25}{1Ch 4:4}
\crossref{1Chr}{8}{26}{2Ki 8:26; 11:2,\allowbreak20 2Ch 22:2,\allowbreak10,\allowbreak11; 23:12,\allowbreak13,\allowbreak21; 24:7}
\crossref{1Chr}{8}{27}{8:27 2Ki 1:3,\allowbreak4,\allowbreak8,\allowbreak12 Ezr 10:21,\allowbreak26 Mal 4:5}
\crossref{1Chr}{8}{28}{Jos 15:63; 18:28 Jud 1:21 Ne 11:1,\allowbreak7-\allowbreak9}
\crossref{1Chr}{8}{29}{1Ch 9:35,\allowbreak36}
\crossref{1Chr}{8}{30}{1Ch 9:36,\allowbreak37}
\crossref{1Chr}{8}{31}{1Ch 9:37}
\crossref{1Chr}{8}{32}{1Ch 9:38}
\crossref{1Chr}{8}{33}{1Ch 9:39 1Sa 9:1; 14:50,\allowbreak51}
\crossref{1Chr}{8}{34}{2Sa 4:4; 9:6,\allowbreak10; 19:24-\allowbreak30}
\crossref{1Chr}{8}{35}{1Ch 9:41}
\crossref{1Chr}{8}{36}{1Ch 9:42}
\crossref{1Chr}{8}{37}{1Ch 9:43}
\crossref{1Chr}{8}{38}{1Ch 3:23}
\crossref{1Chr}{8}{39}{1Ch 7:16}
\crossref{1Chr}{8}{40}{1Ch 12:2 2Ch 14:8}
\crossref{1Chr}{9}{1}{Ezr 2:59,\allowbreak62,\allowbreak63 Ne 7:5,\allowbreak64 Mt 1:1-\allowbreak16 Lu 3:28-\allowbreak38}
\crossref{1Chr}{9}{2}{Ezr 2:70 Ne 7:73; 11:3}
\crossref{1Chr}{9}{3}{Ne 11:1,\allowbreak4-\allowbreak9}
\crossref{1Chr}{9}{4}{Ne 8:7; 10:13}
\crossref{1Chr}{9}{5}{Nu 26:20}
\crossref{1Chr}{9}{6}{1Ch 2:4,\allowbreak6 Ge 38:30}
\crossref{1Chr}{9}{7}{Ne 8:4; 10:20; 11:7}
\crossref{1Chr}{9}{8}{Ge 9:3}
\crossref{1Chr}{9}{9}{Ne 11:8}
\crossref{1Chr}{9}{10}{Ne 11:10-\allowbreak24; 12:19}
\crossref{1Chr}{9}{11}{1Ch 6:8-\allowbreak15 Ne 10:2; 11:11}
\crossref{1Chr}{9}{12}{Ne 11:12,\allowbreak13}
\crossref{1Chr}{9}{13}{1Ch 26:6,\allowbreak30,\allowbreak32 Ne 11:14}
\crossref{1Chr}{9}{14}{Ne 11:15}
\crossref{1Chr}{9}{15}{Ne 11:17,\allowbreak22}
\crossref{1Chr}{9}{16}{Ne 11:17}
\crossref{1Chr}{9}{17}{1Ch 23:5; 26:1-\allowbreak32 Ne 11:19}
\crossref{1Chr}{9}{18}{1Ki 10:5 2Ki 11:19 Eze 44:2,\allowbreak3; 46:1,\allowbreak2 Ac 3:11}
\crossref{1Chr}{9}{19}{1Ch 6:22,\allowbreak23}
\crossref{1Chr}{9}{20}{Nu 3:32; 4:16,\allowbreak28,\allowbreak33; 31:6}
\crossref{1Chr}{9}{21}{1Ch 26:14}
\crossref{1Chr}{9}{22}{9:16,\allowbreak25 Ne 11:25-\allowbreak30,\allowbreak36; 12:28,\allowbreak29,\allowbreak44}
\crossref{1Chr}{9}{23}{1Ch 23:32 2Ch 23:19 Ne 12:45 Eze 44:10,\allowbreak11,\allowbreak14}
\crossref{1Chr}{9}{24}{1Ch 26:14-\allowbreak18}
\crossref{1Chr}{9}{25}{2Ki 11:5,\allowbreak7 2Ch 23:8}
\crossref{1Chr}{9}{26}{}
\crossref{1Chr}{9}{27}{1Ch 23:32 Ro 12:7}
\crossref{1Chr}{9}{28}{1Ch 26:22-\allowbreak26 Nu 23:25-\allowbreak27 Ezr 8:25-\allowbreak30 Ne 12:44; 13:4,\allowbreak5}
\crossref{1Chr}{9}{29}{Ex 27:20}
\crossref{1Chr}{9}{30}{Ex 30:25,\allowbreak33,\allowbreak35-\allowbreak38; 37:29}
\crossref{1Chr}{9}{31}{9:17,\allowbreak19}
\crossref{1Chr}{9}{32}{1Ch 6:33-\allowbreak48}
\crossref{1Chr}{9}{33}{1Ch 6:31-\allowbreak33; 15:16-\allowbreak22; 16:4-\allowbreak6; 25:1-\allowbreak31 Ezr 7:24}
\crossref{1Chr}{9}{34}{9:13 Ne 11:1-\allowbreak15}
\crossref{1Chr}{9}{35}{1Ch 8:29-\allowbreak40}
\crossref{1Chr}{9}{36}{9:39}
\crossref{1Chr}{9}{37}{}
\crossref{1Chr}{9}{38}{}
\crossref{1Chr}{9}{39}{1Ch 8:33 1Sa 14:50,\allowbreak51}
\crossref{1Chr}{9}{40}{1Ch 8:34-\allowbreak36}
\crossref{1Chr}{9}{41}{1Ch 8:35}
\crossref{1Chr}{9}{42}{}
\crossref{1Chr}{9}{43}{}
\crossref{1Chr}{9}{44}{1Ch 8:38}
\crossref{1Chr}{10}{1}{1Sa 28:1; 29:1,\allowbreak2; 31:1,\allowbreak2-\allowbreak13}
\crossref{1Chr}{10}{2}{1Ch 8:33; 9:39 1Sa 14:6,\allowbreak39,\allowbreak40 2Ki 23:29 Isa 57:1,\allowbreak2}
\crossref{1Chr}{10}{3}{1Sa 31:3-\allowbreak6 2Sa 1:4-\allowbreak10 Am 2:14}
\crossref{1Chr}{10}{4}{Jud 9:54}
\crossref{1Chr}{10}{5}{}
\crossref{1Chr}{10}{6}{1Sa 4:10,\allowbreak11,\allowbreak18; 12:25 Ec 9:1,\allowbreak2 Ho 13:10,\allowbreak11}
\crossref{1Chr}{10}{7}{Le 26:31,\allowbreak36 De 28:33,\allowbreak43 Jud 6:2 1Sa 13:6; 31:7}
\crossref{1Chr}{10}{8}{1Sa 31:8 2Ki 3:23 2Ch 20:25}
\crossref{1Chr}{10}{9}{10:4 1Sa 31:9,\allowbreak10 2Sa 1:20 Mt 14:11}
\crossref{1Chr}{10}{10}{1Sa 31:10}
\crossref{1Chr}{10}{11}{1Sa 11:1-\allowbreak11; 31:11-\allowbreak13 2Sa 2:4-\allowbreak7}
\crossref{1Chr}{10}{12}{Ge 35:8 2Sa 21:12-\allowbreak14}
\crossref{1Chr}{10}{13}{1Sa 13:13; 15:2,\allowbreak23}
\crossref{1Chr}{10}{14}{Jud 10:11-\allowbreak16 1Sa 28:6 Eze 14:3-\allowbreak6}
\crossref{1Chr}{11}{1}{1Ch 12:23-\allowbreak40 2Sa 5:1-\allowbreak16}
\crossref{1Chr}{11}{2}{1Sa 16:1,\allowbreak13 2Sa 7:7 Ps 78:71 Isa 40:11 Jer 3:15 Mic 5:2,\allowbreak4}
\crossref{1Chr}{11}{3}{2Sa 5:3}
\crossref{1Chr}{11}{4}{2Sa 5:6-\allowbreak10}
\crossref{1Chr}{11}{5}{1Sa 17:9,\allowbreak10,\allowbreak26,\allowbreak36}
\crossref{1Chr}{11}{6}{Jos 15:16,\allowbreak17 1Sa 17:25}
\crossref{1Chr}{11}{7}{Ps 2:6}
\crossref{1Chr}{11}{8}{Jud 9:6,\allowbreak20 1Ki 9:15; 11:27 2Ki 12:20}
\crossref{1Chr}{11}{9}{1Ch 9:20 Ps 46:7,\allowbreak11 Isa 8:9,\allowbreak10; 41:10,\allowbreak14 Ro 8:31}
\crossref{1Chr}{11}{10}{1Sa 16:1,\allowbreak12-\allowbreak14}
\crossref{1Chr}{11}{11}{1Ch 27:2 2Sa 23:8}
\crossref{1Chr}{11}{12}{1Ch 27:4}
\crossref{1Chr}{11}{13}{}
\crossref{1Chr}{11}{14}{1Sa 14:23; 19:5 2Sa 23:10 2Ki 5:1 Ps 18:50}
\crossref{1Chr}{11}{15}{2Sa 23:13-\allowbreak39}
\crossref{1Chr}{11}{16}{1Sa 22:1; 23:25 Ps 142:1}
\crossref{1Chr}{11}{17}{Nu 11:4,\allowbreak5 2Sa 23:15,\allowbreak16 Ps 143:6}
\crossref{1Chr}{11}{18}{1Sa 19:5 So 8:6 Ac 20:24; 21:13 2Co 5:14,\allowbreak15}
\crossref{1Chr}{11}{19}{2Sa 23:17 1Ki 21:3 Ro 6:1,\allowbreak2}
\crossref{1Chr}{11}{20}{1Ch 2:16 1Sa 26:6 2Sa 2:18; 3:30; 18:2; 20:6; 21:17; 23:18,\allowbreak19-\allowbreak39}
\crossref{1Chr}{11}{21}{Mt 13:8 1Co 15:41}
\crossref{1Chr}{11}{22}{1Ch 27:5,\allowbreak6 2Sa 8:18; 20:23; 23:20-\allowbreak23 1Ki 1:8,\allowbreak38; 2:30,\allowbreak34,\allowbreak35}
\crossref{1Chr}{11}{23}{De 3:11 1Sa 17:4}
\crossref{1Chr}{11}{24}{2Sa 23:18}
\crossref{1Chr}{11}{25}{11:21}
\crossref{1Chr}{11}{26}{1Ch 27:7 2Sa 2:18-\allowbreak23; 3:30; 23:24}
\crossref{1Chr}{11}{27}{}
\crossref{1Chr}{11}{28}{1Ch 27:9}
\crossref{1Chr}{11}{29}{11:12}
\crossref{1Chr}{11}{30}{1Ch 27:13}
\crossref{1Chr}{11}{31}{}
\crossref{1Chr}{11}{32}{2Sa 23:30}
\crossref{1Chr}{11}{33}{}
\crossref{1Chr}{11}{34}{2Sa 23:32}
\crossref{1Chr}{11}{35}{2Sa 23:33}
\crossref{1Chr}{11}{36}{Nu 26:32}
\crossref{1Chr}{11}{37}{2Sa 23:35}
\crossref{1Chr}{11}{38}{2Sa 23:36}
\crossref{1Chr}{11}{39}{2Sa 23:37}
\crossref{1Chr}{11}{40}{2Sa 20:26}
\crossref{1Chr}{11}{41}{2Sa 11:6-\allowbreak27; 23:39}
\crossref{1Chr}{11}{42}{}
\crossref{1Chr}{11}{43}{1Ch 9:44}
\crossref{1Chr}{11}{44}{1Ch 9:35}
\crossref{1Chr}{11}{45}{1Ch 7:6}
\crossref{1Chr}{11}{46}{1Ch 8:20}
\crossref{1Chr}{11}{47}{11:46}
\crossref{1Chr}{12}{1}{1Sa 27:2,\allowbreak6 2Sa 1:1; 4:10}
\crossref{1Chr}{12}{2}{Jud 3:15; 20:16}
\crossref{1Chr}{12}{3}{1Sa 11:4 2Sa 21:6}
\crossref{1Chr}{12}{4}{Jos 9:3,\allowbreak17-\allowbreak23}
\crossref{1Chr}{12}{5}{1Ch 7:7}
\crossref{1Chr}{12}{6}{Ex 6:24}
\crossref{1Chr}{12}{7}{1Ch 4:18,\allowbreak39 Jos 15:58}
\crossref{1Chr}{12}{8}{12:16; 11:16 1Sa 23:14,\allowbreak29; 24:22}
\crossref{1Chr}{12}{9}{Ge 36:21}
\crossref{1Chr}{12}{10}{12:4,\allowbreak13}
\crossref{1Chr}{12}{11}{1Ch 2:35}
\crossref{1Chr}{12}{12}{2Ki 25:23}
\crossref{1Chr}{12}{13}{12:4,\allowbreak10}
\crossref{1Chr}{12}{14}{}
\crossref{1Chr}{12}{15}{Jos 3:15; 4:18 Jer 12:5; 49:19}
\crossref{1Chr}{12}{16}{12:2}
\crossref{1Chr}{12}{17}{1Sa 16:4 2Sa 3:20-\allowbreak25 1Ki 2:13 2Ki 9:22 Ps 12:1,\allowbreak2}
\crossref{1Chr}{12}{18}{Jud 6:34; 13:25 Isa 59:17}
\crossref{1Chr}{12}{19}{1Sa 29:2-\allowbreak4}
\crossref{1Chr}{12}{20}{Ex 18:21 De 1:15; 33:17}
\crossref{1Chr}{12}{21}{1Sa 30:1-\allowbreak17}
\crossref{1Chr}{12}{22}{2Sa 2:2-\allowbreak4; 3:1 Job 17:9}
\crossref{1Chr}{12}{23}{1Ch 11:1-\allowbreak3 2Sa 2:3,\allowbreak4; 5:1-\allowbreak3}
\crossref{1Chr}{12}{24}{2Ch 25:5}
\crossref{1Chr}{12}{25}{1Ch 4:39-\allowbreak43 Ge 49:5 Nu 2:12; 10:19; 13:5; 34:20 De 27:12 Jos 19:1}
\crossref{1Chr}{12}{26}{12:6 Ge 29:34 Nu 1:47}
\crossref{1Chr}{12}{27}{1Ch 9:20 2Ki 11:4,\allowbreak9; 25:18}
\crossref{1Chr}{12}{28}{1Ch 6:8,\allowbreak53 2Sa 8:17 1Ki 1:8; 2:35 Eze 44:15}
\crossref{1Chr}{12}{29}{12:2 Ge 31:23}
\crossref{1Chr}{12}{30}{Ge 6:4}
\crossref{1Chr}{12}{31}{Jos 17:1-\allowbreak18}
\crossref{1Chr}{12}{32}{Pr 14:8 Eph 5:17}
\crossref{1Chr}{12}{33}{}
\crossref{1Chr}{12}{34}{Ge 30:8}
\crossref{1Chr}{12}{35}{Jud 13:2}
\crossref{1Chr}{12}{36}{12:33 Joe 2:7}
\crossref{1Chr}{12}{37}{1Ch 5:1-\allowbreak10 Nu 32:33-\allowbreak42 De 3:12-\allowbreak16 Jos 13:7-\allowbreak32; 14:3; 22:1-\allowbreak10}
\crossref{1Chr}{12}{38}{12:17,\allowbreak18 Ge 49:8-\allowbreak10 2Ch 30:12 Ps 110:3 Eze 11:19}
\crossref{1Chr}{12}{39}{Ge 26:30; 31:54 2Sa 6:19; 19:42}
\crossref{1Chr}{12}{40}{1Sa 25:18}
\crossref{1Chr}{13}{1}{1Ch 12:14,\allowbreak20,\allowbreak32 2Sa 6:1 2Ki 23:1 2Ch 29:20; 34:29,\allowbreak30}
\crossref{1Chr}{13}{2}{1Ki 12:7 2Ki 9:15 Pr 15:22 Phm 1:8,\allowbreak9}
\crossref{1Chr}{13}{3}{1Sa 7:1,\allowbreak2 Ps 132:6}
\crossref{1Chr}{13}{4}{1Sa 18:20 2Sa 3:36 2Ch 30:4}
\crossref{1Chr}{13}{5}{1Sa 7:1 2Sa 6:1}
\crossref{1Chr}{13}{6}{Jos 15:9,\allowbreak60 2Sa 6:2}
\crossref{1Chr}{13}{7}{1Ch 15:2,\allowbreak13 Nu 4:15 1Sa 6:7 2Sa 6:3}
\crossref{1Chr}{13}{8}{1Ch 15:10-\allowbreak24 1Sa 10:5 2Sa 6:5-\allowbreak23 2Ki 3:15 Ps 47:5; 68:25-\allowbreak27}
\crossref{1Chr}{13}{9}{2Sa 6:6}
\crossref{1Chr}{13}{10}{1Ch 15:13,\allowbreak15 Nu 4:15 Jos 6:6}
\crossref{1Chr}{13}{11}{2Sa 6:7,\allowbreak9 Jon 4:4,\allowbreak9}
\crossref{1Chr}{13}{12}{Nu 17:12,\allowbreak13 1Sa 5:10,\allowbreak11; 6:20 Ps 119:120 Isa 6:5 Lu 5:8,\allowbreak9}
\crossref{1Chr}{13}{13}{1Ch 15:18; 16:5; 26:4,\allowbreak8 2Sa 6:10,\allowbreak11}
\crossref{1Chr}{13}{14}{1Ch 26:5 Ge 30:27; 39:5 Pr 3:9,\allowbreak10; 10:22 Mal 3:10,\allowbreak11}
\crossref{1Chr}{14}{1}{2Sa 5:11,\allowbreak12-\allowbreak16 1Ki 5:1,\allowbreak8-\allowbreak12 2Ch 2:11,\allowbreak12}
\crossref{1Chr}{14}{2}{1Ch 17:17 2Sa 7:16 Ps 89:20-\allowbreak37}
\crossref{1Chr}{14}{3}{1Ch 3:1-\allowbreak4 De 17:17 2Sa 5:13 1Ki 11:3 Pr 5:18,\allowbreak19 Ec 7:26-\allowbreak29; 9:9}
\crossref{1Chr}{14}{4}{1Ch 3:5-\allowbreak9}
\crossref{1Chr}{14}{5}{1Ch 3:6}
\crossref{1Chr}{14}{6}{1Ch 3:7}
\crossref{1Chr}{14}{7}{1Ch 3:8}
\crossref{1Chr}{14}{8}{1Sa 21:11 2Sa 5:17-\allowbreak25}
\crossref{1Chr}{14}{9}{1Ch 11:15 2Sa 5:18; 23:13 Isa 17:5}
\crossref{1Chr}{14}{10}{14:14; 13:3 1Sa 23:2-\allowbreak4,\allowbreak9-\allowbreak12 2Sa 2:1; 5:19,\allowbreak23}
\crossref{1Chr}{14}{11}{2Sa 5:20 Isa 28:21}
\crossref{1Chr}{14}{12}{Ex 12:12; 32:20 De 7:5,\allowbreak25 1Sa 5:2-\allowbreak6 2Ki 19:18}
\crossref{1Chr}{14}{13}{14:9 2Sa 5:22-\allowbreak25 1Ki 20:22}
\crossref{1Chr}{14}{14}{14:10 Ps 27:4}
\crossref{1Chr}{14}{15}{Jud 4:14; 7:9,\allowbreak15 1Sa 14:9-\allowbreak22 Php 2:12,\allowbreak13}
\crossref{1Chr}{14}{16}{Ge 6:22 Ex 39:42,\allowbreak43 Joh 2:5; 13:17; 15:14}
\crossref{1Chr}{14}{17}{Jos 6:27 2Ch 26:8 Ps 18:44}
\crossref{1Chr}{15}{1}{2Sa 5:9; 13:7,\allowbreak8; 14:24}
\crossref{1Chr}{15}{2}{Nu 8:13,\allowbreak14,\allowbreak24-\allowbreak26; 18:1-\allowbreak8 Isa 66:21 Jer 33:17-\allowbreak22}
\crossref{1Chr}{15}{3}{1Ch 13:5 1Ki 8:1}
\crossref{1Chr}{15}{4}{1Ch 6:16-\allowbreak20,\allowbreak49,\allowbreak50; 12:26-\allowbreak28 Ex 6:16-\allowbreak22 Nu 3:4}
\crossref{1Chr}{15}{5}{1Ch 6:22-\allowbreak24}
\crossref{1Chr}{15}{6}{1Ch 6:29,\allowbreak30}
\crossref{1Chr}{15}{7}{15:11; 23:8}
\crossref{1Chr}{15}{8}{Ex 6:22}
\crossref{1Chr}{15}{9}{1Ch 6:2; 23:12,\allowbreak19; 26:23,\allowbreak30,\allowbreak31 Ex 6:18 Nu 26:58}
\crossref{1Chr}{15}{10}{1Ch 6:18; 23:12 Ex 6:18,\allowbreak22}
\crossref{1Chr}{15}{11}{1Ch 12:28; 18:16 1Sa 22:20-\allowbreak23 2Sa 8:17; 15:24-\allowbreak29,\allowbreak35; 20:25 1Ki 2:35}
\crossref{1Chr}{15}{12}{1Ch 9:34; 24:31}
\crossref{1Chr}{15}{13}{1Ch 13:7-\allowbreak9 2Sa 6:3}
\crossref{1Chr}{15}{14}{Le 10:3 2Ch 29:15,\allowbreak34 Joe 2:16,\allowbreak17}
\crossref{1Chr}{15}{15}{Ex 25:12,\allowbreak12-\allowbreak15; 37:3-\allowbreak5; 40:20 Nu 4:6,\allowbreak15; 7:9 1Ki 8:8 2Ch 5:9}
\crossref{1Chr}{15}{16}{2Ch 30:12 Ezr 7:24-\allowbreak28 Isa 49:23}
\crossref{1Chr}{15}{17}{1Ch 6:33; 25:1-\allowbreak5 1Sa 8:2}
\crossref{1Chr}{15}{18}{1Ch 25:2-\allowbreak6,\allowbreak9-\allowbreak31}
\crossref{1Chr}{15}{19}{15:16; 13:8; 16:5,\allowbreak42; 25:1,\allowbreak6 Ps 150:5}
\crossref{1Chr}{15}{20}{15:18}
\crossref{1Chr}{15}{21}{15:18; 16:5}
\crossref{1Chr}{15}{22}{1Ch 25:7,\allowbreak8}
\crossref{1Chr}{15}{23}{1Ch 9:21-\allowbreak23 2Ki 22:4; 25:18 Ps 84:10}
\crossref{1Chr}{15}{24}{1Ch 16:6 Nu 10:8 2Ch 5:12,\allowbreak13 Ps 81:13 Joe 2:1,\allowbreak15}
\crossref{1Chr}{15}{25}{2Sa 6:12,\allowbreak13-\allowbreak23 1Ki 8:1}
\crossref{1Chr}{15}{26}{1Ch 29:14 1Sa 7:12 Ac 26:22 2Co 2:16; 3:5}
\crossref{1Chr}{15}{27}{1Sa 2:18 2Sa 6:14}
\crossref{1Chr}{15}{28}{2Sa 6:15}
\crossref{1Chr}{15}{29}{1Ch 17:1 Nu 10:33 De 31:26 Jos 4:7 Jud 20:27 1Sa 4:3 Jer 3:16}
\crossref{1Chr}{16}{1}{2Sa 6:17-\allowbreak19 1Ki 8:6 2Ch 5:7}
\crossref{1Chr}{16}{2}{Le 1:3}
\crossref{1Chr}{16}{3}{2Ch 30:24; 35:7,\allowbreak8 Ne 8:10 Eze 45:17 1Pe 4:9}
\crossref{1Chr}{16}{4}{1Ch 15:16; 23:2-\allowbreak6; 24:3}
\crossref{1Chr}{16}{5}{1Ch 6:39; 15:16-\allowbreak24; 25:1-\allowbreak6}
\crossref{1Chr}{16}{6}{Nu 10:8 2Ch 5:12,\allowbreak13; 13:12; 29:26-\allowbreak28}
\crossref{1Chr}{16}{7}{2Sa 22:1; 23:1,\allowbreak2 2Ch 29:30 Ne 12:24}
\crossref{1Chr}{16}{8}{Isa 12:4 Ac 9:14 1Co 1:2}
\crossref{1Chr}{16}{9}{Ps 95:1,\allowbreak2; 96:1,\allowbreak2; 98:1-\allowbreak4 Mal 3:16}
\crossref{1Chr}{16}{10}{Ps 34:2 Isa 45:25 Jer 9:23,\allowbreak24 1Co 1:30,\allowbreak31}
\crossref{1Chr}{16}{11}{Am 5:6 Zep 2:2,\allowbreak3}
\crossref{1Chr}{16}{12}{16:8,\allowbreak9 Ps 103:2; 111:4}
\crossref{1Chr}{16}{13}{Ge 17:7; 28:13,\allowbreak14; 35:10-\allowbreak12}
\crossref{1Chr}{16}{14}{Ex 15:2 Ps 63:1; 95:7; 100:3; 118:28}
\crossref{1Chr}{16}{15}{Ps 25:10; 44:17; 105:8 Mal 4:4}
\crossref{1Chr}{16}{16}{Ge 15:18; 17:2; 26:3; 28:13,\allowbreak14; 35:11 Ex 3:15 Ne 9:8 Lu 1:72,\allowbreak73}
\crossref{1Chr}{16}{17}{Ps 78:10}
\crossref{1Chr}{16}{18}{Ge 12:7; 13:15; 17:8; 28:13,\allowbreak14; 35:11,\allowbreak12}
\crossref{1Chr}{16}{19}{Ge 34:30 Ac 7:5 Heb 11:13}
\crossref{1Chr}{16}{20}{Ge 12:10; 20:1; 46:3,\allowbreak6}
\crossref{1Chr}{16}{21}{Ge 31:24,\allowbreak29,\allowbreak42}
\crossref{1Chr}{16}{22}{1Ki 19:16 Ps 105:15 1Jo 2:27}
\crossref{1Chr}{16}{23}{16:9 Ps 96:1-\allowbreak13 Ex 15:21 Ps 30:4 Isa 12:5}
\crossref{1Chr}{16}{24}{2Ki 19:19 Ps 22:27 Isa 12:2-\allowbreak6 Da 4:1-\allowbreak3}
\crossref{1Chr}{16}{25}{Ps 89:7; 144:3-\allowbreak6 Isa 40:12-\allowbreak17 Re 15:3,\allowbreak4}
\crossref{1Chr}{16}{26}{Le 19:4 Ps 115:4-\allowbreak8 Isa 44:9-\allowbreak20 Jer 10:10-\allowbreak14 Ac 19:26 1Co 8:4}
\crossref{1Chr}{16}{27}{Ps 8:1; 16:11; 63:2,\allowbreak3 Joh 17:24}
\crossref{1Chr}{16}{28}{Ps 29:1,\allowbreak2; 68:34}
\crossref{1Chr}{16}{29}{Ps 89:5-\allowbreak8; 108:3-\allowbreak5; 148:13,\allowbreak14 Isa 6:3 Re 4:9-\allowbreak11; 5:12-\allowbreak14; 7:12}
\crossref{1Chr}{16}{30}{16:23,\allowbreak25 Ps 96:9 Re 11:15}
\crossref{1Chr}{16}{31}{Ps 19:1; 89:5; 148:1-\allowbreak4 Lu 2:13,\allowbreak14; 15:10}
\crossref{1Chr}{16}{32}{Ps 93:4; 98:7}
\crossref{1Chr}{16}{33}{Ps 96:12,\allowbreak13 Eze 17:22-\allowbreak24}
\crossref{1Chr}{16}{34}{2Ch 5:13; 7:3 Ezr 3:11 Ps 106:1; 107:1; 118:1; 136:1-\allowbreak26 Jer 33:11}
\crossref{1Chr}{16}{35}{Ps 14:7; 53:6; 79:9,\allowbreak10; 106:47,\allowbreak48}
\crossref{1Chr}{16}{36}{1Ki 8:15,\allowbreak56 Ps 72:18,\allowbreak19; 106:48 Eph 1:3 1Pe 1:3}
\crossref{1Chr}{16}{37}{16:4-\allowbreak6; 15:17-\allowbreak24; 25:1-\allowbreak6}
\crossref{1Chr}{16}{38}{1Ch 13:14; 26:4-\allowbreak8}
\crossref{1Chr}{16}{39}{1Ch 12:28}
\crossref{1Chr}{16}{40}{Ex 29:38-\allowbreak42 Nu 28:3-\allowbreak8 1Ki 18:29 2Ch 2:4; 31:3 Ezr 3:3}
\crossref{1Chr}{16}{41}{16:37; 6:39-\allowbreak47; 25:1-\allowbreak6}
\crossref{1Chr}{16}{42}{2Ch 29:25-\allowbreak28 Ps 150:3-\allowbreak6}
\crossref{1Chr}{16}{43}{2Sa 6:19,\allowbreak20 1Ki 8:66}
\crossref{1Chr}{17}{1}{2Sa 7:1,\allowbreak2-\allowbreak17 2Ch 6:7-\allowbreak9 Da 4:4,\allowbreak29,\allowbreak30}
\crossref{1Chr}{17}{2}{1Ch 22:7; 28:2 Jos 9:14 1Sa 16:7 Ps 20:4 1Co 13:9}
\crossref{1Chr}{17}{3}{Nu 12:6 2Ki 20:1-\allowbreak5 Isa 30:21 Am 3:7}
\crossref{1Chr}{17}{4}{Isa 55:8,\allowbreak9 Ro 11:33,\allowbreak34}
\crossref{1Chr}{17}{5}{2Sa 7:6 1Ki 8:27 2Ch 2:6; 6:18 Isa 66:1,\allowbreak2 Ac 7:44-\allowbreak50}
\crossref{1Chr}{17}{6}{Ex 33:14,\allowbreak14,\allowbreak15; 40:35-\allowbreak38 Le 26:11,\allowbreak12 Nu 10:33-\allowbreak36 De 23:14}
\crossref{1Chr}{17}{7}{Ex 3:1-\allowbreak10 1Sa 16:11,\allowbreak12; 17:15 2Sa 7:8 Ps 78:70,\allowbreak71 Am 7:14,\allowbreak15}
\crossref{1Chr}{17}{8}{17:2 Ge 28:15 1Sa 18:14,\allowbreak28 2Sa 7:9; 8:6,\allowbreak8,\allowbreak14 Ps 46:7,\allowbreak11}
\crossref{1Chr}{17}{9}{Jer 31:3-\allowbreak12 Eze 34:13}
\crossref{1Chr}{17}{10}{Jud 2:14-\allowbreak18; 3:8; 4:3; 6:3-\allowbreak6 1Sa 13:5,\allowbreak6,\allowbreak19,\allowbreak20}
\crossref{1Chr}{17}{11}{1Ch 29:15,\allowbreak28 Ac 13:36}
\crossref{1Chr}{17}{12}{1Ch 22:9,\allowbreak10; 28:6-\allowbreak10 1Ki 5:5 2Ch 3:1-\allowbreak4:22 Ezr 5:11 Zec 6:12,\allowbreak13}
\crossref{1Chr}{17}{13}{2Sa 7:14 Ps 89:26-\allowbreak28,\allowbreak29-\allowbreak37 Isa 55:3 Heb 1:5}
\crossref{1Chr}{17}{14}{}
\crossref{1Chr}{17}{15}{2Sa 7:17 Jer 23:28 Ac 20:27}
\crossref{1Chr}{17}{16}{2Sa 7:18 2Ki 19:14}
\crossref{1Chr}{17}{17}{17:7,\allowbreak8 2Sa 7:19; 12:8 2Ki 3:18 Isa 49:6}
\crossref{1Chr}{17}{18}{1Sa 2:30 2Sa 7:20-\allowbreak24}
\crossref{1Chr}{17}{19}{Isa 37:35; 42:1; 49:3,\allowbreak5,\allowbreak6 Da 9:17}
\crossref{1Chr}{17}{20}{Ex 15:11; 18:11 De 3:24; 33:26 Ps 86:8; 89:6,\allowbreak8 Isa 40:18,\allowbreak25}
\crossref{1Chr}{17}{21}{De 4:7,\allowbreak32-\allowbreak34; 33:26-\allowbreak29 Ps 147:20}
\crossref{1Chr}{17}{22}{Ge 17:7 Ex 19:5,\allowbreak6 De 7:6-\allowbreak8; 26:18,\allowbreak19 1Sa 12:22 Jer 31:31-\allowbreak34}
\crossref{1Chr}{17}{23}{Ge 32:12 2Sa 7:25-\allowbreak29 Ps 119:49 Jer 11:5 Lu 1:38}
\crossref{1Chr}{17}{24}{2Ch 6:33 Ps 21:13; 72:19 Mt 6:9,\allowbreak13 Joh 12:28; 17:1 Php 2:11}
\crossref{1Chr}{17}{25}{1Sa 9:15}
\crossref{1Chr}{17}{26}{Ex 34:6,\allowbreak7 Tit 1:2 Heb 6:18}
\crossref{1Chr}{17}{27}{Ge 27:33 Ps 72:17 Ro 11:29 Eph 1:3}
\crossref{1Chr}{18}{1}{2Sa 8:1,\allowbreak2-\allowbreak18}
\crossref{1Chr}{18}{2}{Nu 24:17 Jud 3:29,\allowbreak30 2Sa 8:2 Ps 60:8 Isa 11:14}
\crossref{1Chr}{18}{3}{1Sa 14:47 2Sa 10:6 Ps 60:1}
\crossref{1Chr}{18}{4}{2Sa 8:4}
\crossref{1Chr}{18}{5}{2Sa 8:5,\allowbreak6 1Ki 11:23,\allowbreak24}
\crossref{1Chr}{18}{6}{18:2 Ps 18:43,\allowbreak44}
\crossref{1Chr}{18}{7}{1Ki 10:16,\allowbreak17; 14:26-\allowbreak28 2Ch 9:15,\allowbreak16; 12:9,\allowbreak10}
\crossref{1Chr}{18}{8}{2Sa 8:8}
\crossref{1Chr}{18}{9}{2Sa 8:9}
\crossref{1Chr}{18}{10}{2Sa 8:10}
\crossref{1Chr}{18}{11}{1Ch 22:14; 26:20,\allowbreak26,\allowbreak27; 29:14 Ex 35:5,\allowbreak21-\allowbreak24 Jos 6:19 2Sa 8:11,\allowbreak12}
\crossref{1Chr}{18}{12}{1Ch 2:16; 11:20 1Sa 26:6,\allowbreak8 2Sa 3:30; 10:10,\allowbreak14; 16:9-\allowbreak11; 19:21,\allowbreak22; 20:6}
\crossref{1Chr}{18}{13}{18:6 1Sa 10:5; 13:3; 14:1 2Sa 7:14-\allowbreak17; 23:14 2Co 11:32}
\crossref{1Chr}{18}{14}{1Ch 12:38}
\crossref{1Chr}{18}{15}{1Ch 11:6 2Sa 8:16}
\crossref{1Chr}{18}{16}{2Sa 8:17}
\crossref{1Chr}{18}{17}{2Sa 8:18; 15:18; 20:7,\allowbreak23; 23:19-\allowbreak23 1Ki 1:38,\allowbreak44; 2:34,\allowbreak35}
\crossref{1Chr}{19}{1}{1Sa 11:1,\allowbreak2; 12:12 2Sa 10:1-\allowbreak3}
\crossref{1Chr}{19}{2}{1Sa 30:26 2Sa 9:1,\allowbreak7 2Ki 4:13 Es 6:3 Ec 9:15}
\crossref{1Chr}{19}{3}{1Sa 29:4,\allowbreak9 1Ki 12:8-\allowbreak11}
\crossref{1Chr}{19}{4}{Ps 35:12; 109:4,\allowbreak5}
\crossref{1Chr}{19}{5}{Mt 18:31}
\crossref{1Chr}{19}{6}{Lu 10:16 1Th 4:8}
\crossref{1Chr}{19}{7}{1Ch 18:4 Ex 14:9 Jud 4:3 1Sa 13:5 2Ch 14:9 Ps 20:7-\allowbreak9}
\crossref{1Chr}{19}{8}{1Ch 11:6,\allowbreak10-\allowbreak47 2Sa 23:8-\allowbreak39}
\crossref{1Chr}{19}{9}{1Sa 17:2 2Sa 18:4 2Ch 13:3; 14:10 Isa 28:6 Jer 50:42 Joe 2:5}
\crossref{1Chr}{19}{10}{2Sa 10:9-\allowbreak14}
\crossref{1Chr}{19}{11}{1Ch 11:20; 18:12}
\crossref{1Chr}{19}{12}{Ne 4:20 Ec 4:9-\allowbreak12 Ga 6:2 Php 1:27,\allowbreak28}
\crossref{1Chr}{19}{13}{De 31:6,\allowbreak7 Jos 1:7; 10:25 1Sa 4:9; 14:6-\allowbreak12; 17:32 2Sa 10:12}
\crossref{1Chr}{19}{14}{1Ki 20:13,\allowbreak19-\allowbreak21,\allowbreak28-\allowbreak30 2Ch 13:5-\allowbreak16 Jer 46:15,\allowbreak16}
\crossref{1Chr}{19}{15}{Le 26:7 Ro 8:31}
\crossref{1Chr}{19}{16}{Ps 2:1 Isa 8:9 Mic 4:11,\allowbreak12 Zec 14:1-\allowbreak3}
\crossref{1Chr}{19}{17}{19:9 Isa 22:6,\allowbreak7}
\crossref{1Chr}{19}{18}{19:13,\allowbreak14 Ps 18:32; 33:16; 46:11}
\crossref{1Chr}{19}{19}{Ge 14:4,\allowbreak5 Jos 9:9-\allowbreak11 2Sa 10:19 1Ki 20:1,\allowbreak12 Ps 18:39,\allowbreak44}
\crossref{1Chr}{20}{1}{2Sa 11:1}
\crossref{1Chr}{20}{2}{1Ch 18:11 2Sa 8:11,\allowbreak12}
\crossref{1Chr}{20}{3}{1Ch 19:2-\allowbreak5 Ps 21:8,\allowbreak9}
\crossref{1Chr}{20}{4}{Jos 12:12; 16:3 2Sa 21:18-\allowbreak22}
\crossref{1Chr}{20}{5}{2Sa 21:19}
\crossref{1Chr}{20}{6}{2Sa 21:20}
\crossref{1Chr}{20}{7}{1Sa 17:10,\allowbreak26,\allowbreak36 Isa 37:23}
\crossref{1Chr}{20}{8}{Jos 14:12 Ec 9:11 Jer 9:23 Ro 8:31}
\crossref{1Chr}{21}{1}{2Sa 24:1 1Ki 22:20-\allowbreak22 Job 1:6-\allowbreak12; 2:1,\allowbreak4-\allowbreak6 Zec 3:1 Mt 4:3}
\crossref{1Chr}{21}{2}{2Sa 24:2-\allowbreak4}
\crossref{1Chr}{21}{3}{1Ch 19:13 Ps 115:14 Pr 14:28 Isa 26:15; 48:19}
\crossref{1Chr}{21}{4}{Ec 8:4}
\crossref{1Chr}{21}{5}{}
\crossref{1Chr}{21}{6}{Nu 1:47-\allowbreak49}
\crossref{1Chr}{21}{7}{21:14 Jos 7:1,\allowbreak5,\allowbreak13; 22:16-\allowbreak26 2Sa 21:1,\allowbreak14; 24:1}
\crossref{1Chr}{21}{8}{2Sa 12:13; 24:10 Ps 25:11; 32:5 Jer 3:13 Lu 15:18,\allowbreak19 1Jo 1:9}
\crossref{1Chr}{21}{9}{1Ch 29:29 1Sa 9:9 2Sa 24:11}
\crossref{1Chr}{21}{10}{Jos 24:15 Pr 1:29-\allowbreak31}
\crossref{1Chr}{21}{11}{Pr 19:20}
\crossref{1Chr}{21}{12}{Le 26:17,\allowbreak36,\allowbreak37 De 28:15,\allowbreak25,\allowbreak51,\allowbreak52 Jer 42:16}
\crossref{1Chr}{21}{13}{2Ki 6:15; 7:4 Es 4:11,\allowbreak16 Joh 12:27 Php 1:23}
\crossref{1Chr}{21}{14}{Nu 16:46-\allowbreak49 2Sa 24:15}
\crossref{1Chr}{21}{15}{2Sa 24:16 Jer 7:12; 26:9,\allowbreak18 Mt 23:37,\allowbreak38}
\crossref{1Chr}{21}{16}{Ge 3:24 Ex 14:19,\allowbreak20 Nu 22:31 Jos 5:13,\allowbreak14 2Ki 6:17}
\crossref{1Chr}{21}{17}{21:8 2Sa 24:17 Ps 51:4 Eze 16:63}
\crossref{1Chr}{21}{18}{21:11 Ac 8:26-\allowbreak40}
\crossref{1Chr}{21}{19}{2Ki 5:10-\allowbreak14 Joh 2:5 Ac 9:6}
\crossref{1Chr}{21}{20}{}
\crossref{1Chr}{21}{21}{1Sa 25:23 2Sa 24:18-\allowbreak20}
\crossref{1Chr}{21}{22}{1Ki 21:2}
\crossref{1Chr}{21}{23}{Ge 23:4-\allowbreak6 2Sa 24:22,\allowbreak23 Jer 32:8}
\crossref{1Chr}{21}{24}{Ge 14:23; 23:13 De 16:16,\allowbreak17 Mal 1:12-\allowbreak14 Ro 12:17}
\crossref{1Chr}{21}{25}{2Sa 24:24,\allowbreak25}
\crossref{1Chr}{21}{26}{Ex 20:24,\allowbreak25; 24:4,\allowbreak5}
\crossref{1Chr}{21}{27}{21:15,\allowbreak16 2Sa 24:16 Ps 103:20 Heb 1:14}
\crossref{1Chr}{21}{28}{Ge 50:10 De 16:13}
\crossref{1Chr}{21}{29}{Ex 40:1-\allowbreak38}
\crossref{1Chr}{21}{30}{21:16; 13:12 De 10:12 2Sa 6:9 Job 13:21; 21:6; 23:15 Ps 90:11}
\crossref{1Chr}{22}{1}{2Ki 18:22 2Ch 32:12}
\crossref{1Chr}{22}{2}{1Ki 9:20,\allowbreak21 2Ch 2:17; 8:7,\allowbreak8 Isa 61:5,\allowbreak6 Eph 2:12,\allowbreak19-\allowbreak22}
\crossref{1Chr}{22}{3}{1Ch 29:2,\allowbreak7}
\crossref{1Chr}{22}{4}{2Sa 5:11 1Ki 5:6-\allowbreak10 2Ch 2:3 Ezr 3:7}
\crossref{1Chr}{22}{5}{1Ch 29:1 1Ki 3:7 2Ch 13:7}
\crossref{1Chr}{22}{6}{Nu 27:18,\allowbreak19,\allowbreak23 De 31:14,\allowbreak23 Mt 28:18-\allowbreak20 Ac 1:2; 20:25-\allowbreak31}
\crossref{1Chr}{22}{7}{1Ch 17:1-\allowbreak15; 28:2-\allowbreak21; 29:3 2Sa 7:2 1Ki 8:17-\allowbreak19 2Ch 6:7-\allowbreak9 Ps 132:5}
\crossref{1Chr}{22}{8}{1Ch 28:3 Nu 31:20,\allowbreak24 1Ki 5:3}
\crossref{1Chr}{22}{9}{1Ch 17:11; 28:5-\allowbreak7 2Sa 7:12,\allowbreak13}
\crossref{1Chr}{22}{10}{1Ch 17:12,\allowbreak13; 28:6 2Sa 7:13 1Ki 5:5; 8:19,\allowbreak20 Zec 6:12,\allowbreak13}
\crossref{1Chr}{22}{11}{22:16; 28:20 Isa 26:12 Mt 1:23; 28:20 Ro 15:33 2Ti 4:22}
\crossref{1Chr}{22}{12}{1Ki 3:9-\allowbreak12 2Ch 1:10 Ps 72:1 Pr 2:6,\allowbreak7 Lu 21:15 Jas 1:5}
\crossref{1Chr}{22}{13}{1Ch 28:7 Jos 1:7,\allowbreak8}
\crossref{1Chr}{22}{14}{2Co 8:2}
\crossref{1Chr}{22}{15}{Ex 28:6; 31:3-\allowbreak5; 35:32,\allowbreak35 1Ki 7:14}
\crossref{1Chr}{22}{16}{22:3,\allowbreak14}
\crossref{1Chr}{22}{17}{1Ch 28:21; 29:6 Ro 16:2,\allowbreak3 Php 4:3 3Jo 1:8}
\crossref{1Chr}{22}{18}{Jud 6:12-\allowbreak14 Ro 8:31}
\crossref{1Chr}{22}{19}{1Ch 16:11; 28:9 De 4:29; 32:46,\allowbreak47 Ps 27:4 2Ch 20:3 Da 9:3}
\crossref{1Chr}{23}{1}{1Ch 29:28 Ge 25:8; 35:29 1Ki 1:1 Job 5:26}
\crossref{1Chr}{23}{2}{1Ch 13:1; 28:1 Jos 23:2; 24:1 2Ch 34:29,\allowbreak30}
\crossref{1Chr}{23}{3}{Nu 4:2,\allowbreak3,\allowbreak23,\allowbreak30,\allowbreak35,\allowbreak43,\allowbreak47}
\crossref{1Chr}{23}{4}{23:28-\allowbreak32; 6:48; 9:28-\allowbreak32; 26:20-\allowbreak27}
\crossref{1Chr}{23}{5}{1Ch 9:17-\allowbreak27; 15:23,\allowbreak24; 16:38; 26:1-\allowbreak12 2Ch 8:14; 35:15 Ezr 7:7 Ne 7:73}
\crossref{1Chr}{23}{6}{2Ch 8:14; 29:25; 31:2; 35:10 Ezr 6:18}
\crossref{1Chr}{23}{7}{1Ch 6:17-\allowbreak20; 15:7; 26:21}
\crossref{1Chr}{23}{8}{1Ch 15:18,\allowbreak20,\allowbreak21}
\crossref{1Chr}{23}{9}{Le 24:11}
\crossref{1Chr}{23}{10}{}
\crossref{1Chr}{23}{11}{23:10}
\crossref{1Chr}{23}{12}{1Ch 6:2 Ex 6:18 Nu 3:27; 26:58}
\crossref{1Chr}{23}{13}{1Ch 6:3 Ex 6:20 Nu 3:27; 26:59}
\crossref{1Chr}{23}{14}{De 33:1 Ps 90:1}
\crossref{1Chr}{23}{15}{Ex 2:22; 4:20; 18:3,\allowbreak4}
\crossref{1Chr}{23}{16}{1Ch 24:20; 25:20}
\crossref{1Chr}{23}{17}{1Ch 26:25}
\crossref{1Chr}{23}{18}{1Ch 24:22}
\crossref{1Chr}{23}{19}{23:12; 15:9; 24:23}
\crossref{1Chr}{23}{20}{}
\crossref{1Chr}{23}{21}{23:6}
\crossref{1Chr}{23}{22}{1Ch 24:28}
\crossref{1Chr}{23}{23}{1Ch 24:30}
\crossref{1Chr}{23}{24}{Nu 10:17,\allowbreak21}
\crossref{1Chr}{23}{25}{1Ch 22:18 2Sa 7:1,\allowbreak11}
\crossref{1Chr}{23}{26}{Nu 4:5,\allowbreak49; 7:9}
\crossref{1Chr}{23}{27}{23:3,\allowbreak24 2Sa 23:1 Ps 72:20}
\crossref{1Chr}{23}{28}{23:4; 28:13 Nu 3:6-\allowbreak9 1Ki 6:5 2Ch 31:11 Ezr 8:29 Ne 13:4,\allowbreak5,\allowbreak9}
\crossref{1Chr}{23}{29}{1Ch 9:29-\allowbreak34 Le 6:20-\allowbreak23}
\crossref{1Chr}{23}{30}{1Ch 6:31-\allowbreak33; 9:33; 16:37-\allowbreak42; 25:1-\allowbreak7 2Ch 29:25-\allowbreak28; 31:2 Ezr 3:10,\allowbreak11}
\crossref{1Chr}{23}{31}{Le 23:24,\allowbreak39 Nu 10:10 Ps 81:1-\allowbreak4 Isa 1:13,\allowbreak14}
\crossref{1Chr}{23}{32}{1Ch 9:27 Nu 1:53 1Ki 8:4}
\crossref{1Chr}{24}{1}{1Ch 23:6}
\crossref{1Chr}{24}{2}{Ex 24:1,\allowbreak9}
\crossref{1Chr}{24}{3}{24:6,\allowbreak31; 6:4-\allowbreak8,\allowbreak50-\allowbreak53; 12:27,\allowbreak28; 15:11; 16:39 2Sa 20:25 1Ki 2:35}
\crossref{1Chr}{24}{4}{1Ch 15:6-\allowbreak12}
\crossref{1Chr}{24}{5}{Jos 18:10 Pr 16:33 Jon 1:7 Ac 1:26}
\crossref{1Chr}{24}{6}{1Ki 4:3 2Ch 34:13 Ezr 7:6 Ne 8:1 Mt 8:19; 13:52; 23:1,\allowbreak2}
\crossref{1Chr}{24}{7}{1Ch 9:10 Ne 12:19}
\crossref{1Chr}{24}{8}{Ezr 2:39; 10:21 Ne 7:35; 12:15}
\crossref{1Chr}{24}{9}{Ne 12:17}
\crossref{1Chr}{24}{10}{}
\crossref{1Chr}{24}{11}{Ezr 2:36 Ne 7:39; 12:10}
\crossref{1Chr}{24}{12}{Ne 12:10}
\crossref{1Chr}{24}{13}{}
\crossref{1Chr}{24}{14}{Ezr 2:37; 10:20 Ne 7:40}
\crossref{1Chr}{24}{15}{24:15 Ne 10:20}
\crossref{1Chr}{24}{16}{Ne 11:24}
\crossref{1Chr}{24}{17}{Ge 46:10}
\crossref{1Chr}{24}{18}{Jer 36:12}
\crossref{1Chr}{24}{19}{1Ch 9:25 2Ch 23:4,\allowbreak8 1Co 14:40}
\crossref{1Chr}{24}{20}{1Ch 6:18; 23:12-\allowbreak14}
\crossref{1Chr}{24}{21}{1Ch 23:17}
\crossref{1Chr}{24}{22}{}
\crossref{1Chr}{24}{23}{}
\crossref{1Chr}{24}{24}{1Ch 23:20}
\crossref{1Chr}{24}{25}{1Ch 23:20}
\crossref{1Chr}{24}{26}{24:27}
\crossref{1Chr}{24}{27}{1Ch 6:19; 23:21 Ex 6:19 Nu 3:20}
\crossref{1Chr}{24}{28}{1Ch 23:22}
\crossref{1Chr}{24}{29}{1Ch 2:9}
\crossref{1Chr}{24}{30}{1Ch 6:47; 23:23}
\crossref{1Chr}{24}{31}{24:5,\allowbreak6 Nu 26:56}
\crossref{1Chr}{25}{1}{1Ch 6:33,\allowbreak39,\allowbreak44; 15:16-\allowbreak19}
\crossref{1Chr}{25}{2}{25:1; 6:39; 15:17; 16:5 Ps 73:1; 74:1; 75:1; 76:1; 77:1; 78:1}
\crossref{1Chr}{25}{3}{1Ch 9:16; 16:41,\allowbreak42 2Ch 29:14}
\crossref{1Chr}{25}{4}{1Ch 6:33; 15:17,\allowbreak19; 16:41,\allowbreak42 Ps 88:1}
\crossref{1Chr}{25}{5}{1Ch 21:9 1Sa 9:9}
\crossref{1Chr}{25}{6}{25:2,\allowbreak3}
\crossref{1Chr}{25}{7}{So 3:8 Isa 29:13 Ho 10:11}
\crossref{1Chr}{25}{8}{1Ch 24:5 Le 16:8 1Sa 14:41,\allowbreak42 Pr 16:33 Ac 1:26}
\crossref{1Chr}{25}{9}{25:2}
\crossref{1Chr}{25}{10}{25:2}
\crossref{1Chr}{25}{11}{}
\crossref{1Chr}{25}{12}{25:2}
\crossref{1Chr}{25}{13}{}
\crossref{1Chr}{25}{14}{}
\crossref{1Chr}{25}{15}{25:3}
\crossref{1Chr}{25}{16}{25:4}
\crossref{1Chr}{25}{17}{2Sa 16:5}
\crossref{1Chr}{25}{18}{}
\crossref{1Chr}{25}{19}{25:3}
\crossref{1Chr}{25}{20}{25:4}
\crossref{1Chr}{25}{21}{25:3}
\crossref{1Chr}{25}{22}{25:4; 23:23; 24:30}
\crossref{1Chr}{25}{23}{25:4; 3:19,\allowbreak21; 8:24 Ezr 10:28 Ne 3:8,\allowbreak30; 7:2; 10:23; 12:12,\allowbreak41 Jer 28:1}
\crossref{1Chr}{25}{24}{}
\crossref{1Chr}{25}{25}{25:4}
\crossref{1Chr}{25}{26}{25:4}
\crossref{1Chr}{25}{27}{25:4}
\crossref{1Chr}{25}{28}{25:4}
\crossref{1Chr}{25}{29}{25:4}
\crossref{1Chr}{25}{30}{25:4}
\crossref{1Chr}{25}{31}{25:4}
\crossref{1Chr}{26}{1}{1Ch 9:17-\allowbreak27; 15:18,\allowbreak23,\allowbreak24 2Ch 23:19}
\crossref{1Chr}{26}{2}{Zec 1:1}
\crossref{1Chr}{26}{3}{Ge 10:22}
\crossref{1Chr}{26}{4}{1Ch 15:18,\allowbreak21,\allowbreak24; 16:5,\allowbreak38}
\crossref{1Chr}{26}{5}{}
\crossref{1Chr}{26}{6}{}
\crossref{1Chr}{26}{7}{Ru 4:17}
\crossref{1Chr}{26}{8}{Mt 25:15 1Co 12:4-\allowbreak11 2Co 3:6 1Pe 4:11}
\crossref{1Chr}{26}{9}{26:1,\allowbreak14}
\crossref{1Chr}{26}{10}{1Ch 16:38}
\crossref{1Chr}{26}{11}{2Ki 18:18}
\crossref{1Chr}{26}{12}{1Ch 25:8}
\crossref{1Chr}{26}{13}{}
\crossref{1Chr}{26}{14}{26:1}
\crossref{1Chr}{26}{15}{26:17 2Ch 25:24 Ne 12:25 Ec 12:11}
\crossref{1Chr}{26}{16}{26:10,\allowbreak11}
\crossref{1Chr}{26}{17}{1Ch 9:24 2Ch 8:14}
\crossref{1Chr}{26}{18}{2Ki 23:11}
\crossref{1Chr}{26}{19}{}
\crossref{1Chr}{26}{20}{26:22; 9:26-\allowbreak30; 22:3,\allowbreak4,\allowbreak14-\allowbreak16; 28:12-\allowbreak19; 29:2-\allowbreak8 1Ki 14:26; 15:18}
\crossref{1Chr}{26}{21}{1Ch 6:17}
\crossref{1Chr}{26}{22}{26:20 Ne 10:38}
\crossref{1Chr}{26}{23}{1Ch 23:12 Nu 3:19,\allowbreak27}
\crossref{1Chr}{26}{24}{}
\crossref{1Chr}{26}{25}{1Ch 23:15 Ex 18:4}
\crossref{1Chr}{26}{26}{1Ch 18:11; 22:14; 29:2-\allowbreak9 Nu 31:30-\allowbreak52}
\crossref{1Chr}{26}{27}{Jos 6:19}
\crossref{1Chr}{26}{28}{1Sa 9:9}
\crossref{1Chr}{26}{29}{26:23; 23:12}
\crossref{1Chr}{26}{30}{1Ch 23:12,\allowbreak19}
\crossref{1Chr}{26}{31}{1Ch 23:19}
\crossref{1Chr}{26}{32}{26:6-\allowbreak9}
\crossref{1Chr}{27}{1}{1Ch 13:1 Ex 18:25 De 1:15 1Sa 8:12 Mic 5:2}
\crossref{1Chr}{27}{2}{1Ch 11:11 2Sa 23:8}
\crossref{1Chr}{27}{3}{Ge 38:29 Nu 26:20}
\crossref{1Chr}{27}{4}{1Ch 11:12 2Sa 23:9}
\crossref{1Chr}{27}{5}{1Ki 4:5}
\crossref{1Chr}{27}{6}{1Ch 11:22-\allowbreak25 2Sa 22:20-\allowbreak23; 23:20-\allowbreak23}
\crossref{1Chr}{27}{7}{1Ch 11:26 2Sa 2:18-\allowbreak23; 23:24}
\crossref{1Chr}{27}{8}{1Ch 26:29 2Sa 23:25}
\crossref{1Chr}{27}{9}{1Ch 11:28 2Sa 23:26}
\crossref{1Chr}{27}{10}{1Ch 11:27}
\crossref{1Chr}{27}{11}{1Ch 11:29 2Sa 21:18}
\crossref{1Chr}{27}{12}{1Ch 11:28}
\crossref{1Chr}{27}{13}{1Ch 11:30 2Sa 23:28}
\crossref{1Chr}{27}{14}{1Ch 11:31 2Sa 23:30}
\crossref{1Chr}{27}{15}{1Ch 11:30}
\crossref{1Chr}{27}{16}{Ge 15:2}
\crossref{1Chr}{27}{17}{1Ch 26:30}
\crossref{1Chr}{27}{18}{}
\crossref{1Chr}{27}{19}{1Ch 7:7}
\crossref{1Chr}{27}{20}{1Ch 15:21}
\crossref{1Chr}{27}{21}{1Ki 4:14}
\crossref{1Chr}{27}{22}{}
\crossref{1Chr}{27}{23}{Nu 1:18}
\crossref{1Chr}{27}{24}{1Ch 21:1-\allowbreak17 2Sa 24:1-\allowbreak15}
\crossref{1Chr}{27}{25}{2Ki 18:15 2Ch 16:2}
\crossref{1Chr}{27}{26}{}
\crossref{1Chr}{27}{27}{2Sa 16:5}
\crossref{1Chr}{27}{28}{1Ki 4:7}
\crossref{1Chr}{27}{29}{1Ch 5:16 Isa 65:10}
\crossref{1Chr}{27}{30}{Job 1:3}
\crossref{1Chr}{27}{31}{1Ch 11:38}
\crossref{1Chr}{27}{32}{2Sa 13:3; 21:21}
\crossref{1Chr}{27}{33}{2Sa 15:12; 16:23; 17:23}
\crossref{1Chr}{27}{34}{1Ki 1:7}
\crossref{1Chr}{28}{1}{1Ch 23:2 Jos 23:2; 24:1}
\crossref{1Chr}{28}{2}{Ge 48:2 1Ki 1:47}
\crossref{1Chr}{28}{3}{1Ch 17:4; 22:8 2Sa 7:5-\allowbreak13 1Ki 5:3 2Ch 6:8,\allowbreak9}
\crossref{1Chr}{28}{4}{1Sa 16:6-\allowbreak13 2Sa 7:8-\allowbreak16 Ps 78:68-\allowbreak72; 89:16-\allowbreak27}
\crossref{1Chr}{28}{5}{1Ch 3:1-\allowbreak9; 14:4-\allowbreak7}
\crossref{1Chr}{28}{6}{1Ch 17:11-\allowbreak14; 22:9,\allowbreak10 2Sa 7:13,\allowbreak14 2Ch 1:9 Zec 6:12,\allowbreak13 Heb 3:3,\allowbreak6}
\crossref{1Chr}{28}{7}{Ps 89:28-\allowbreak37; 132:12 Da 2:44}
\crossref{1Chr}{28}{8}{De 4:6 Mt 5:14-\allowbreak16 Php 2:15,\allowbreak16 Heb 12:1,\allowbreak2}
\crossref{1Chr}{28}{9}{De 4:35 1Ki 8:43 Ps 9:10 Jer 9:24; 22:16; 24:7; 31:34 Ho 4:1,\allowbreak6}
\crossref{1Chr}{28}{10}{28:6; 22:16-\allowbreak19 1Ti 4:16}
\crossref{1Chr}{28}{11}{28:19 Ex 25:40; 26:30; 39:42,\allowbreak43 2Ch 3:3 Eze 43:10,\allowbreak11 Heb 8:5}
\crossref{1Chr}{28}{12}{Ex 31:2}
\crossref{1Chr}{28}{13}{1Ch 24:1-\allowbreak19; 25:1-\allowbreak31}
\crossref{1Chr}{28}{14}{}
\crossref{1Chr}{28}{15}{Ex 25:31-\allowbreak39 1Ki 7:19 2Ch 4:7 Zec 4:2,\allowbreak3,\allowbreak11-\allowbreak14 Re 1:12,\allowbreak13,\allowbreak20}
\crossref{1Chr}{28}{16}{Ex 25:23-\allowbreak30 1Ki 7:48 2Ch 4:8,\allowbreak19}
\crossref{1Chr}{28}{17}{1Sa 2:13,\allowbreak14 2Ch 4:20-\allowbreak22}
\crossref{1Chr}{28}{18}{Ex 30:1-\allowbreak10 1Ki 7:48}
\crossref{1Chr}{28}{19}{28:11,\allowbreak12 Ex 25:40; 26:30}
\crossref{1Chr}{28}{20}{28:10; 22:13 De 31:7,\allowbreak8 Jos 1:6-\allowbreak9 1Co 16:13}
\crossref{1Chr}{28}{21}{1Ch 24:1-\allowbreak26:32}
\crossref{1Chr}{29}{1}{1Ch 28:1,\allowbreak8}
\crossref{1Chr}{29}{2}{1Ch 22:3-\allowbreak5,\allowbreak14-\allowbreak16}
\crossref{1Chr}{29}{3}{Ps 26:8; 27:4; 84:1,\allowbreak10; 122:1-\allowbreak9}
\crossref{1Chr}{29}{4}{1Ki 9:28 Job 28:16}
\crossref{1Chr}{29}{5}{Ex 25:2-\allowbreak9; 35:5-\allowbreak9 Nu 7:2,\allowbreak3,\allowbreak10-\allowbreak14,\allowbreak15-\allowbreak89 Ezr 1:4-\allowbreak6; 2:68,\allowbreak69}
\crossref{1Chr}{29}{6}{1Ch 27:1-\allowbreak15 Isa 60:3-\allowbreak10}
\crossref{1Chr}{29}{7}{Ezr 2:69; 8:7 Ne 7:70-\allowbreak72}
\crossref{1Chr}{29}{8}{1Ch 26:21,\allowbreak22}
\crossref{1Chr}{29}{9}{De 16:10,\allowbreak11 Jud 5:9 Ps 110:3 2Co 8:3,\allowbreak12; 9:7,\allowbreak8}
\crossref{1Chr}{29}{10}{29:20 2Ch 20:26-\allowbreak28 Ps 103:1,\allowbreak2; 138:1; 146:2}
\crossref{1Chr}{29}{11}{Da 4:30,\allowbreak34,\allowbreak35 Mt 6:13 1Ti 1:17; 6:15,\allowbreak16 Jude 1:25 Re 4:10,\allowbreak11}
\crossref{1Chr}{29}{12}{De 8:18 1Sa 2:7,\allowbreak8 Job 42:10 Ps 75:6,\allowbreak7; 113:7,\allowbreak8 Pr 8:18; 10:22}
\crossref{1Chr}{29}{13}{Ps 105:1; 106:1 Da 2:23 2Co 2:14; 8:16; 9:15 1Th 2:13}
\crossref{1Chr}{29}{14}{Ge 32:10 2Sa 7:18 Da 4:30 1Co 15:9,\allowbreak10 2Co 3:5; 12:9-\allowbreak11}
\crossref{1Chr}{29}{15}{Ge 47:9 Ps 39:12; 119:19 Heb 11:13-\allowbreak16 1Pe 2:11}
\crossref{1Chr}{29}{16}{29:14 2Ch 31:10 Ps 24:1 Ho 2:8 Lu 19:16}
\crossref{1Chr}{29}{17}{1Ch 28:9 De 8:2 1Sa 16:7 Ps 7:9; 51:6 Pr 16:2; 21:2 Jer 17:10}
\crossref{1Chr}{29}{18}{Ex 3:6,\allowbreak15; 4:5 Mt 22:32 Ac 3:13}
\crossref{1Chr}{29}{19}{1Ch 28:9 Ps 72:1; 119:80 Jas 1:17}
\crossref{1Chr}{29}{20}{1Ch 16:36 2Ch 20:21 Ps 134:2; 135:19-\allowbreak21; 145:1-\allowbreak146:2}
\crossref{1Chr}{29}{21}{1Ki 8:62-\allowbreak65 2Ch 7:4-\allowbreak9 Ezr 6:17}
\crossref{1Chr}{29}{22}{Ex 24:11 De 12:7,\allowbreak11,\allowbreak12; 16:14-\allowbreak17 2Ch 7:10 Ne 8:12 Ec 2:24}
\crossref{1Chr}{29}{23}{1Ch 17:11,\allowbreak12; 28:5 Ps 132:11 Isa 9:6,\allowbreak7}
\crossref{1Chr}{29}{24}{1Ch 22:17; 28:21}
\crossref{1Chr}{29}{25}{Jos 3:7; 4:14 2Ch 1:1 Job 7:17 Ac 19:17}
\crossref{1Chr}{29}{26}{1Ch 18:14 Ps 78:71,\allowbreak72}
\crossref{1Chr}{29}{27}{1Ch 3:4 2Sa 5:4,\allowbreak5 1Ki 2:11}
\crossref{1Chr}{29}{28}{Ge 15:15; 25:8 Job 5:26 Pr 16:31 Ac 13:36}
\crossref{1Chr}{29}{29}{1Ki 11:41; 14:29 Heb 11:32,\allowbreak33}
\crossref{1Chr}{29}{30}{2Ki 10:34; 14:28}

% 2Chr
\crossref{2Chr}{1}{1}{1Ki 2:12,\allowbreak46}
\crossref{2Chr}{1}{2}{2Ch 29:20; 30:2; 34:29,\allowbreak30 1Ch 13:1; 15:3; 27:1; 28:1; 29:1}
\crossref{2Chr}{1}{3}{1Ki 3:4-\allowbreak15 1Ch 16:39; 21:29}
\crossref{2Chr}{1}{4}{1Ch 16:1 Ps 132:5,\allowbreak6}
\crossref{2Chr}{1}{5}{Ex 27:1-\allowbreak8; 38:1-\allowbreak7}
\crossref{2Chr}{1}{6}{1Ki 3:4; 8:63 1Ch 29:21 Isa 40:16}
\crossref{2Chr}{1}{7}{Mt 7:7,\allowbreak8 Mr 10:36,\allowbreak37,\allowbreak51 Joh 16:23 1Jo 5:14,\allowbreak15}
\crossref{2Chr}{1}{8}{2Sa 7:8,\allowbreak9; 12:7,\allowbreak8; 22:51; 23:1 Ps 86:13; 89:20-\allowbreak28,\allowbreak49 Isa 55:3}
\crossref{2Chr}{1}{9}{2Sa 7:12-\allowbreak16,\allowbreak25-\allowbreak29 1Ch 17:11-\allowbreak14,\allowbreak23-\allowbreak27; 28:6,\allowbreak7 Ps 89:35-\allowbreak37}
\crossref{2Chr}{1}{10}{1Ki 3:9 Ps 119:34,\allowbreak73 Pr 2:2-\allowbreak6; 3:13-\allowbreak18; 4:7 Jas 1:5}
\crossref{2Chr}{1}{11}{1Sa 16:7 1Ki 3:11-\allowbreak13; 8:18 1Ch 28:2; 29:17,\allowbreak18 Pr 23:7 Ac 5:4}
\crossref{2Chr}{1}{12}{Mt 6:33 Eph 3:20}
\crossref{2Chr}{1}{13}{1:3}
\crossref{2Chr}{1}{14}{2Ch 9:25 De 17:16 1Ki 4:26; 10:16,\allowbreak26-\allowbreak29}
\crossref{2Chr}{1}{15}{2Ch 9:27 Isa 9:10 Am 7:14}
\crossref{2Chr}{1}{16}{}
\crossref{2Chr}{1}{17}{2Ki 10:29}
\crossref{2Chr}{2}{1}{1Ki 5:5}
\crossref{2Chr}{2}{2}{2:18 1Ki 5:15,\allowbreak16}
\crossref{2Chr}{2}{3}{1Ki 5:1}
\crossref{2Chr}{2}{4}{2:1 1Ki 8:18}
\crossref{2Chr}{2}{5}{2:9 1Ki 9:8 1Ch 29:1 Eze 7:20}
\crossref{2Chr}{2}{6}{2Ch 6:18 1Ki 8:27 Isa 66:1 Ac 7:48,\allowbreak49}
\crossref{2Chr}{2}{7}{Ex 31:3-\allowbreak5 1Ki 7:14 Isa 28:26,\allowbreak29; 60:10}
\crossref{2Chr}{2}{8}{1Ki 5:6}
\crossref{2Chr}{2}{9}{2:5; 7:21 1Ki 9:8}
\crossref{2Chr}{2}{10}{1Ki 5:11 Lu 10:7 Ro 13:7,\allowbreak8}
\crossref{2Chr}{2}{11}{2Ch 9:8 De 7:7,\allowbreak8 1Ki 10:9 Ps 72:17}
\crossref{2Chr}{2}{12}{1Ki 5:7 1Ch 29:20 Ps 72:18,\allowbreak19 Lu 1:68 1Pe 1:3}
\crossref{2Chr}{2}{13}{2Ch 4:16}
\crossref{2Chr}{2}{14}{1Ki 7:13,\allowbreak14}
\crossref{2Chr}{2}{15}{2:10 1Ki 5:11}
\crossref{2Chr}{2}{16}{1Ki 5:8,\allowbreak9}
\crossref{2Chr}{2}{17}{2:2; 8:7,\allowbreak8 1Ki 5:13-\allowbreak16; 9:20,\allowbreak21}
\crossref{2Chr}{2}{18}{}
\crossref{2Chr}{3}{1}{1Ki 6:1-\allowbreak14}
\crossref{2Chr}{3}{2}{1Ki 6:1}
\crossref{2Chr}{3}{3}{1Ch 28:11-\allowbreak19}
\crossref{2Chr}{3}{4}{Joh 10:23 Ac 3:11; 5:12}
\crossref{2Chr}{3}{5}{1Ki 6:15-\allowbreak17,\allowbreak21,\allowbreak22}
\crossref{2Chr}{3}{6}{1Ch 29:2,\allowbreak8 Isa 54:11,\allowbreak12 Re 21:18-\allowbreak21}
\crossref{2Chr}{3}{7}{Ex 26:29 1Ki 6:20-\allowbreak22,\allowbreak30 Eze 7:20}
\crossref{2Chr}{3}{8}{Ex 26:33 1Ki 6:19,\allowbreak20 Heb 9:3,\allowbreak9; 10:19}
\crossref{2Chr}{3}{9}{3:5; 7:21 Ge 1:26 1Ki 9:8 Joe 2:26}
\crossref{2Chr}{3}{10}{1Ki 6:23-\allowbreak28}
\crossref{2Chr}{3}{11}{2Ch 9:8 De 7:7,\allowbreak8 1Ki 10:9 Ps 72:17}
\crossref{2Chr}{3}{12}{1Ki 5:7 1Ch 29:20 Ps 72:18,\allowbreak19 Lu 1:68 1Pe 1:3}
\crossref{2Chr}{3}{13}{Ex 25:20}
\crossref{2Chr}{3}{14}{Ex 26:31-\allowbreak35 Mt 27:51 Heb 9:3; 10:20}
\crossref{2Chr}{3}{15}{1Ki 7:15-\allowbreak24 Jer 52:20-\allowbreak23}
\crossref{2Chr}{3}{16}{1Ki 6:21}
\crossref{2Chr}{3}{17}{1Ki 7:21}
\crossref{2Chr}{4}{1}{2Ch 1:5 Ex 27:1-\allowbreak8 1Ki 8:22,\allowbreak64; 9:25 2Ki 16:14,\allowbreak15 Eze 43:13-\allowbreak17}
\crossref{2Chr}{4}{2}{Ex 30:18-\allowbreak21 1Ki 7:23 Zec 13:1 Tit 3:5 Re 7:14}
\crossref{2Chr}{4}{3}{1Ki 7:24-\allowbreak26 Eze 1:10; 10:14 1Co 9:9,\allowbreak10 Re 4:7}
\crossref{2Chr}{4}{4}{Mt 16:18 Eph 2:20 Re 21:14}
\crossref{2Chr}{4}{5}{}
\crossref{2Chr}{4}{6}{Ex 30:18-\allowbreak21 1Ki 7:38,\allowbreak40 Ps 51:2 1Co 6:11 1Jo 1:7}
\crossref{2Chr}{4}{7}{1Ki 7:49 1Ch 28:15 Zec 4:2,\allowbreak3,\allowbreak11-\allowbreak14 Mt 5:14-\allowbreak16 Joh 8:12}
\crossref{2Chr}{4}{8}{Ex 25:23-\allowbreak30; 37:10-\allowbreak16 1Ki 7:48 Isa 25:6 Eze 44:16 Mal 1:12}
\crossref{2Chr}{4}{9}{1Ki 6:36; 7:12}
\crossref{2Chr}{4}{10}{1Ki 7:39}
\crossref{2Chr}{4}{11}{1Ki 7:40,\allowbreak45}
\crossref{2Chr}{4}{12}{2Ch 3:15-\allowbreak17}
\crossref{2Chr}{4}{13}{Ex 28:33,\allowbreak34 1Ki 7:20,\allowbreak42 So 4:13 Jer 52:23}
\crossref{2Chr}{4}{14}{1Ki 7:27-\allowbreak43}
\crossref{2Chr}{4}{15}{4:2-\allowbreak5}
\crossref{2Chr}{4}{16}{4:11 Ex 27:3; 38:3 Zec 14:20,\allowbreak21}
\crossref{2Chr}{4}{17}{1Ki 7:46}
\crossref{2Chr}{4}{18}{1Ki 7:47 1Ch 22:3,\allowbreak14 Jer 52:20}
\crossref{2Chr}{4}{19}{2Ch 36:10,\allowbreak18 1Ki 7:48-\allowbreak50 2Ki 24:13; 25:13-\allowbreak15 Ezr 1:7-\allowbreak11 Jer 28:3}
\crossref{2Chr}{4}{20}{4:7 Ex 25:31-\allowbreak37}
\crossref{2Chr}{4}{21}{}
\crossref{2Chr}{4}{22}{Ex 37:23 1Ki 7:50 2Ki 12:13; 25:14 Jer 52:18}
\crossref{2Chr}{5}{1}{1Ki 7:51 1Ch 22:14; 26:26-\allowbreak28}
\crossref{2Chr}{5}{2}{5:1,\allowbreak12 1Ki 8:1-\allowbreak11 1Ch 28:1}
\crossref{2Chr}{5}{3}{1Ki 8:2}
\crossref{2Chr}{5}{4}{}
\crossref{2Chr}{5}{5}{2Ch 1:3 1Ki 8:4,\allowbreak6}
\crossref{2Chr}{5}{6}{2Sa 6:13 1Ki 8:5 1Ch 16:1,\allowbreak2; 29:21}
\crossref{2Chr}{5}{7}{Ps 132:8}
\crossref{2Chr}{5}{8}{Ex 25:12-\allowbreak15; 37:3-\allowbreak5 Nu 4:6}
\crossref{2Chr}{5}{9}{1Ki 8:8,\allowbreak9}
\crossref{2Chr}{5}{10}{2Ch 6:11 Ex 31:18; 32:15,\allowbreak16,\allowbreak19; 34:1; 40:20 De 10:2-\allowbreak5 Heb 9:4}
\crossref{2Chr}{5}{11}{2Ch 29:5,\allowbreak15,\allowbreak34; 30:15,\allowbreak17-\allowbreak20 Ex 19:10,\allowbreak14,\allowbreak15 Job 1:5}
\crossref{2Chr}{5}{12}{2Ch 29:25 1Ch 15:16-\allowbreak22; 16:4-\allowbreak6,\allowbreak41,\allowbreak42; 23:5,\allowbreak30; 25:1-\allowbreak7 Ezr 3:10,\allowbreak11}
\crossref{2Chr}{5}{13}{Ps 95:1,\allowbreak2; 100:1,\allowbreak2 Isa 52:8 Jer 32:39 Ac 4:32 Ro 15:6}
\crossref{2Chr}{5}{14}{2Ch 7:2 1Ti 6:16}
\crossref{2Chr}{6}{1}{Ex 20:21; 24:15-\allowbreak18 Le 16:2 De 4:11 1Ki 8:12-\allowbreak21 Ps 18:8-\allowbreak11; 97:2}
\crossref{2Chr}{6}{2}{2Ch 2:4-\allowbreak6 2Sa 7:13 1Ki 8:13 1Ch 17:12; 22:10,\allowbreak11; 28:6,\allowbreak20 Ps 132:5}
\crossref{2Chr}{6}{3}{1Ki 8:14}
\crossref{2Chr}{6}{4}{1Ki 8:15 1Ch 29:10,\allowbreak20 Ps 41:13; 68:4,\allowbreak32-\allowbreak35; 72:18,\allowbreak19 Lu 1:68,\allowbreak69}
\crossref{2Chr}{6}{5}{2Sa 7:6,\allowbreak7 1Ki 8:16}
\crossref{2Chr}{6}{6}{1Sa 16:1 1Ch 28:4 Ps 89:19,\allowbreak20}
\crossref{2Chr}{6}{7}{2Sa 7:2,\allowbreak3 1Ki 5:3; 8:17 1Ch 17:1; 22:7; 28:2-\allowbreak4}
\crossref{2Chr}{6}{8}{1Ki 8:18-\allowbreak21 Mr 14:8 2Co 8:12}
\crossref{2Chr}{6}{9}{2Sa 7:12,\allowbreak13 1Ch 17:4,\allowbreak11,\allowbreak12}
\crossref{2Chr}{6}{10}{6:4}
\crossref{2Chr}{6}{11}{2Ch 5:7,\allowbreak10 Ex 40:20 1Ki 8:9,\allowbreak21 Heb 9:4}
\crossref{2Chr}{6}{12}{1Ki 8:22-\allowbreak53 2Ki 11:14; 23:3 Ps 29:1,\allowbreak2}
\crossref{2Chr}{6}{13}{Ne 8:4}
\crossref{2Chr}{6}{14}{Ge 33:20; 35:10 Ex 3:15 1Ki 8:23; 18:36 1Ch 29:10,\allowbreak20}
\crossref{2Chr}{6}{15}{2Sa 7:12 1Ki 8:24 1Ch 22:9,\allowbreak10}
\crossref{2Chr}{6}{16}{Eze 36:37 Joh 15:14,\allowbreak15}
\crossref{2Chr}{6}{17}{6:4,\allowbreak14 Ex 24:10 Isa 41:17; 45:3}
\crossref{2Chr}{6}{18}{Ex 29:45,\allowbreak46 1Ki 8:27 Ps 68:18; 113:5,\allowbreak6 Isa 57:15; 66:1}
\crossref{2Chr}{6}{19}{1Ki 8:28 Ps 74:20; 130:2 Da 9:17-\allowbreak19 Lu 18:1-\allowbreak7}
\crossref{2Chr}{6}{20}{2Ch 16:9 1Ki 8:29,\allowbreak30 2Ki 19:16 Ne 1:6 Ps 34:15; 121:5}
\crossref{2Chr}{6}{21}{6:39; 30:27 Job 22:12-\allowbreak14 Ps 123:1 Ec 5:2 Isa 57:15 Mt 6:9}
\crossref{2Chr}{6}{22}{1Ki 8:31,\allowbreak32}
\crossref{2Chr}{6}{23}{6:21}
\crossref{2Chr}{6}{24}{Le 26:17,\allowbreak37 De 28:25,\allowbreak48 Jos 7:8 1Ki 8:33,\allowbreak34 Ps 44:10 Jos 7:11,\allowbreak12 Jud 2:11,\allowbreak14,\allowbreak15 2Ki 17:7-\allowbreak18}
\crossref{2Chr}{6}{25}{Ezr 1:1-\allowbreak6 Ps 106:40-\allowbreak47 Jer 33:6-\allowbreak13}
\crossref{2Chr}{6}{26}{Le 26:19 De 11:17; 28:23 1Ki 17:1-\allowbreak18:46 Lu 4:25}
\crossref{2Chr}{6}{27}{1Ki 8:35,\allowbreak36 Ps 25:4,\allowbreak5,\allowbreak8,\allowbreak12; 94:12; 119:33 Mic 4:2 Joh 6:45}
\crossref{2Chr}{6}{28}{Ex 10:12-\allowbreak15 Joe 1:4-\allowbreak7,\allowbreak11; 2:25 Re 9:3-\allowbreak11}
\crossref{2Chr}{6}{29}{Ps 33:12,\allowbreak13; 50:15; 91:15}
\crossref{2Chr}{6}{30}{Ps 18:20-\allowbreak26; 62:12 Jer 17:10 Eze 18:30 Mt 16:27}
\crossref{2Chr}{6}{31}{Ex 20:20 1Sa 12:24 Job 28:28 Ps 128:1; 130:4 Ac 9:31}
\crossref{2Chr}{6}{32}{Ex 12:48,\allowbreak49 Ru 1:16; 2:11,\allowbreak12 1Ki 8:41-\allowbreak43; 10:1,\allowbreak2 Isa 56:3-\allowbreak7}
\crossref{2Chr}{6}{33}{1Sa 17:46 2Ki 19:19 Ps 22:27; 46:10; 67:2; 138:4,\allowbreak5 Isa 11:10}
\crossref{2Chr}{6}{34}{2Ch 14:11,\allowbreak12; 20:4 De 20:1-\allowbreak4 Jos 1:2-\allowbreak5 1Ki 8:44,\allowbreak45}
\crossref{2Chr}{6}{35}{Da 9:17-\allowbreak19}
\crossref{2Chr}{6}{36}{1Ki 8:46,\allowbreak50}
\crossref{2Chr}{6}{37}{Le 26:40-\allowbreak45 De 4:29,\allowbreak30; 30:1-\allowbreak3 Lu 15:17}
\crossref{2Chr}{6}{38}{De 30:2-\allowbreak6 Jer 29:12-\allowbreak14 Ho 14:1-\allowbreak4 Joe 2:12,\allowbreak13}
\crossref{2Chr}{6}{39}{6:35 Zec 1:15,\allowbreak16}
\crossref{2Chr}{6}{40}{Ps 7:3; 13:3; 22:1,\allowbreak2; 88:1}
\crossref{2Chr}{6}{41}{Ps 132:8-\allowbreak10,\allowbreak16}
\crossref{2Chr}{6}{42}{1Ki 2:16}
\crossref{2Chr}{7}{1}{1Ki 8:54-\allowbreak61 Isa 65:24 Da 9:20 Ac 4:31; 16:25,\allowbreak26}
\crossref{2Chr}{7}{2}{2Ch 5:14 Ex 24:17 Isa 6:5 Re 15:8}
\crossref{2Chr}{7}{3}{Ex 4:31 Le 9:24 Nu 14:5; 16:22 1Ki 18:39 1Ch 29:20 Ps 95:6}
\crossref{2Chr}{7}{4}{}
\crossref{2Chr}{7}{5}{2Ch 1:6; 5:6; 15:11; 29:32,\allowbreak33; 30:24; 35:7-\allowbreak9 1Ki 8:62,\allowbreak63 1Ch 29:21}
\crossref{2Chr}{7}{6}{1Ch 16:39,\allowbreak40; 24:1-\allowbreak3}
\crossref{2Chr}{7}{7}{2Ch 36:14 Nu 16:37,\allowbreak38 1Ki 8:64 Heb 13:10-\allowbreak12}
\crossref{2Chr}{7}{8}{Le 23:34-\allowbreak43 Nu 29:12-\allowbreak38 De 16:13-\allowbreak15 1Ki 8:65 Ne 8:13-\allowbreak18}
\crossref{2Chr}{7}{9}{Le 23:36 De 16:8 Ne 8:18 Joe 1:14}
\crossref{2Chr}{7}{10}{1Ki 8:66}
\crossref{2Chr}{7}{11}{2Ch 2:1 1Ki 9:1-\allowbreak9}
\crossref{2Chr}{7}{12}{2Ch 1:7 Ge 17:1 1Ki 9:2}
\crossref{2Chr}{7}{13}{2Ch 6:26-\allowbreak28 De 11:17 Job 11:10; 12:14 Ps 107:34 Lu 4:25 Re 3:7}
\crossref{2Chr}{7}{14}{Isa 63:19}
\crossref{2Chr}{7}{15}{2Ch 6:20,\allowbreak40 De 11:12 Ne 1:6 Ps 65:2; 130:2 1Pe 3:12}
\crossref{2Chr}{7}{16}{De 12:21; 16:11 1Ki 8:16,\allowbreak44,\allowbreak48 Ps 132:14 Zec 3:2}
\crossref{2Chr}{7}{17}{De 28:1-\allowbreak14 1Ki 2:3; 3:14; 8:25; 9:4-\allowbreak9; 11:38 1Ch 28:9 Zec 3:7}
\crossref{2Chr}{7}{18}{2Sa 7:13-\allowbreak16}
\crossref{2Chr}{7}{19}{Le 26:14,\allowbreak33-\allowbreak46 De 28:15,\allowbreak36,\allowbreak37-\allowbreak68 1Sa 12:25 1Ch 28:9}
\crossref{2Chr}{7}{20}{2Ki 17:20 Ps 52:5 Jer 12:17; 18:7; 31:28; 45:4 Jude 1:12}
\crossref{2Chr}{7}{21}{1Ki 9:8}
\crossref{2Chr}{7}{22}{Jud 2:12,\allowbreak13 Jer 1:16 La 2:16,\allowbreak17; 4:13-\allowbreak15 Eze 14:23; 36:17-\allowbreak20}
\crossref{2Chr}{8}{1}{1Ki 9:10}
\crossref{2Chr}{8}{2}{1Ki 9:11-\allowbreak18}
\crossref{2Chr}{8}{3}{Nu 13:21; 34:8 2Sa 8:3 1Ki 11:23-\allowbreak25 1Ch 18:3}
\crossref{2Chr}{8}{4}{1Ki 9:17-\allowbreak19}
\crossref{2Chr}{8}{5}{Jos 16:3,\allowbreak5 1Ch 7:24}
\crossref{2Chr}{8}{6}{Jos 19:44 1Ki 9:18}
\crossref{2Chr}{8}{7}{1Ki 9:20-\allowbreak22}
\crossref{2Chr}{8}{8}{Jud 1:21-\allowbreak36 Ps 106:34}
\crossref{2Chr}{8}{9}{Ex 19:5,\allowbreak6 Le 25:39-\allowbreak46 Ga 4:26,\allowbreak31}
\crossref{2Chr}{8}{10}{2Ch 2:18 1Ki 5:16; 9:23}
\crossref{2Chr}{8}{11}{1Ki 3:1; 7:8; 9:24}
\crossref{2Chr}{8}{12}{2Ch 4:1 1Ch 28:17 Eze 8:16 Joe 2:17}
\crossref{2Chr}{8}{13}{Ex 29:38-\allowbreak42 Le 23:1-\allowbreak17 Nu 28:1-\allowbreak29:40 Eze 45:17; 46:3-\allowbreak15}
\crossref{2Chr}{8}{14}{2Ch 5:11; 23:4; 31:2 1Ch 24:1-\allowbreak19 Lu 1:5,\allowbreak8}
\crossref{2Chr}{8}{15}{2Ch 30:12 Ex 39:42,\allowbreak43}
\crossref{2Chr}{8}{16}{1Ki 5:18; 6:7}
\crossref{2Chr}{8}{17}{2Ch 20:36 Nu 33:35 1Ki 9:26,\allowbreak27; 22:48}
\crossref{2Chr}{8}{18}{2Ch 9:10,\allowbreak13}
\crossref{2Chr}{9}{1}{1Ki 10:1,\allowbreak2-\allowbreak13 Mt 12:42 Lu 11:31}
\crossref{2Chr}{9}{2}{Pr 13:20; 18:4 Mr 4:11,\allowbreak34 Joh 15:15 Jas 1:5}
\crossref{2Chr}{9}{3}{1Ki 10:3 Ac 11:23}
\crossref{2Chr}{9}{4}{1Ki 4:22,\allowbreak23 Pr 9:5 Joh 6:53-\allowbreak57}
\crossref{2Chr}{9}{5}{1Ki 10:6}
\crossref{2Chr}{9}{6}{Joh 20:25-\allowbreak29}
\crossref{2Chr}{9}{7}{De 33:9 1Ki 10:8 Ps 27:4; 84:10-\allowbreak12 Pr 3:3,\allowbreak14; 8:34; 10:21; 13:20}
\crossref{2Chr}{9}{8}{1Ch 29:10,\allowbreak20 Ps 72:18,\allowbreak19 2Co 9:12-\allowbreak15}
\crossref{2Chr}{9}{9}{9:24 1Ki 9:14; 10:10 Ps 72:10,\allowbreak15}
\crossref{2Chr}{9}{10}{2Ch 8:18 1Ki 9:27,\allowbreak28; 10:22}
\crossref{2Chr}{9}{11}{1Ki 10:12 1Ch 23:5; 25:1 Ps 92:1-\allowbreak3; 150:3-\allowbreak5 Re 5:8}
\crossref{2Chr}{9}{12}{1Ki 10:13 Ps 20:4 Eph 3:20}
\crossref{2Chr}{9}{13}{1Ki 10:14,\allowbreak15 Ps 68:29; 72:10,\allowbreak15}
\crossref{2Chr}{9}{14}{Nu 14:6}
\crossref{2Chr}{9}{15}{2Ch 12:9,\allowbreak10}
\crossref{2Chr}{9}{16}{1Ki 7:2}
\crossref{2Chr}{9}{17}{1Ki 10:18-\allowbreak20 Ps 45:8 Re 20:11}
\crossref{2Chr}{9}{18}{Ge 49:9,\allowbreak10 Nu 23:24; 24:9 Re 5:5}
\crossref{2Chr}{9}{19}{Mt 19:28 Re 21:12}
\crossref{2Chr}{9}{20}{1Ki 10:21 Es 1:7 Da 5:2,\allowbreak3}
\crossref{2Chr}{9}{21}{1Ki 10:22; 22:48}
\crossref{2Chr}{9}{22}{2Ch 1:12 1Ki 3:12,\allowbreak13; 4:30,\allowbreak31; 10:23,\allowbreak24 Ps 89:27 Mt 12:42 Col 2:2,\allowbreak3}
\crossref{2Chr}{9}{23}{9:6,\allowbreak7 1Ki 4:34 Isa 11:2,\allowbreak10}
\crossref{2Chr}{9}{24}{9:9 1Sa 10:27 1Ki 9:14; 10:10,\allowbreak25 Job 42:11}
\crossref{2Chr}{9}{25}{2Ch 1:14 De 17:16 1Ki 4:26; 10:26}
\crossref{2Chr}{9}{26}{1Ki 4:21,\allowbreak24 Ps 72:8-\allowbreak11 Da 7:14 Re 19:16}
\crossref{2Chr}{9}{27}{9:20; 1:15-\allowbreak17 1Ki 10:27-\allowbreak29 Job 22:24,\allowbreak25}
\crossref{2Chr}{9}{28}{}
\crossref{2Chr}{9}{29}{1Ki 11:41-\allowbreak43}
\crossref{2Chr}{9}{30}{1Ki 11:42,\allowbreak43}
\crossref{2Chr}{9}{31}{2Sa 7:12 1Ki 1:21; 2:10}
\crossref{2Chr}{10}{1}{1Ki 12:1 1Ch 3:10 Mt 1:7}
\crossref{2Chr}{10}{2}{1Ki 11:26,\allowbreak28,\allowbreak40; 12:2}
\crossref{2Chr}{10}{3}{1Ki 12:3}
\crossref{2Chr}{10}{4}{1Sa 8:11-\allowbreak18 1Ki 12:4 Isa 47:6 Mt 11:29,\allowbreak30; 23:4 1Jo 5:3}
\crossref{2Chr}{10}{5}{1Ki 12:5 Pr 3:28}
\crossref{2Chr}{10}{6}{Job 12:12,\allowbreak13; 32:7 Pr 12:15; 19:20; 27:10 Jer 42:2-\allowbreak5,\allowbreak20}
\crossref{2Chr}{10}{7}{1Ki 12:7 Pr 15:1}
\crossref{2Chr}{10}{8}{2Ch 25:15,\allowbreak16 2Sa 17:14 Pr 1:25; 9:9; 19:20; 25:12 Ec 10:2,\allowbreak3,\allowbreak16}
\crossref{2Chr}{10}{9}{10:6 2Sa 17:5,\allowbreak6 1Ki 22:6-\allowbreak8}
\crossref{2Chr}{10}{10}{2Sa 17:7-\allowbreak13 Pr 21:30 Isa 19:11-\allowbreak13}
\crossref{2Chr}{10}{11}{10:4}
\crossref{2Chr}{10}{12}{10:5 1Ki 12:12-\allowbreak15}
\crossref{2Chr}{10}{13}{Ge 42:7,\allowbreak30 Ex 10:28 1Sa 25:10,\allowbreak11 1Ki 20:6-\allowbreak11 Pr 15:1}
\crossref{2Chr}{10}{14}{2Ch 22:4,\allowbreak5 Pr 12:5 Da 6:7}
\crossref{2Chr}{10}{15}{Isa 30:12,\allowbreak13}
\crossref{2Chr}{10}{16}{2Sa 20:1 1Ki 12:16,\allowbreak17}
\crossref{2Chr}{10}{17}{2Ch 11:1 1Ki 11:36; 12:17}
\crossref{2Chr}{10}{18}{1Ki 4:6; 5:14}
\crossref{2Chr}{10}{19}{10:16; 13:5-\allowbreak7 1Ki 12:19,\allowbreak20 2Ki 17:21-\allowbreak23 Ps 89:30}
\crossref{2Chr}{11}{1}{1Ki 12:21}
\crossref{2Chr}{11}{2}{2Ch 12:5,\allowbreak7,\allowbreak15 1Ki 12:22-\allowbreak24}
\crossref{2Chr}{11}{3}{Ge 49:28 Ex 24:4 2Ki 17:34 Php 3:5 Re 7:4-\allowbreak8}
\crossref{2Chr}{11}{4}{Ge 13:8 2Sa 2:26 Ac 7:26 1Co 6:5-\allowbreak8 Heb 13:1 1Pe 3:8}
\crossref{2Chr}{11}{5}{2Ch 8:2-\allowbreak6; 14:6,\allowbreak7; 16:6; 17:12; 26:6; 27:4 Isa 22:8-\allowbreak11}
\crossref{2Chr}{11}{6}{Jud 15:8 1Ch 4:32}
\crossref{2Chr}{11}{7}{Jos 15:58}
\crossref{2Chr}{11}{8}{1Ch 18:1}
\crossref{2Chr}{11}{9}{2Ch 32:9 Jos 10:5,\allowbreak11; 15:35,\allowbreak39}
\crossref{2Chr}{11}{10}{Jos 15:33}
\crossref{2Chr}{11}{11}{Isa 22:10,\allowbreak11}
\crossref{2Chr}{11}{12}{2Ch 26:14,\allowbreak15; 32:5 2Sa 13:19,\allowbreak22}
\crossref{2Chr}{11}{13}{}
\crossref{2Chr}{11}{14}{Le 27:30-\allowbreak34 Nu 18:21-\allowbreak28}
\crossref{2Chr}{11}{15}{Ex 32:4-\allowbreak8,\allowbreak31 1Ki 12:28; 14:9 Ps 106:19,\allowbreak20 Ho 8:5,\allowbreak6; 13:2}
\crossref{2Chr}{11}{16}{2Ch 15:9; 30:11,\allowbreak18,\allowbreak19 Jos 22:19 Ps 84:5-\allowbreak7}
\crossref{2Chr}{11}{17}{2Ch 12:1}
\crossref{2Chr}{11}{18}{1Sa 16:6; 17:13,\allowbreak28 1Ch 2:13; 27:18}
\crossref{2Chr}{11}{19}{Ge 36:5}
\crossref{2Chr}{11}{20}{11:21; 13:2}
\crossref{2Chr}{11}{21}{11:23 De 17:17 Jud 8:30 2Sa 3:2-\allowbreak5; 5:13 1Ki 11:3 1Ch 3:1-\allowbreak9}
\crossref{2Chr}{11}{22}{De 21:15-\allowbreak17 1Ch 5:1,\allowbreak2; 29:1}
\crossref{2Chr}{11}{23}{2Ch 10:8-\allowbreak15 Lu 16:8}
\crossref{2Chr}{12}{1}{12:13; 11:17}
\crossref{2Chr}{12}{2}{1Ki 11:40; 14:24-\allowbreak26}
\crossref{2Chr}{12}{3}{Jud 4:13 1Sa 13:5 2Sa 10:18}
\crossref{2Chr}{12}{4}{2Ch 11:5-\allowbreak12 Isa 36:1 Jer 5:10}
\crossref{2Chr}{12}{5}{2Ch 11:2 1Ki 12:22}
\crossref{2Chr}{12}{6}{2Ch 32:26; 33:12,\allowbreak19,\allowbreak23 Ex 10:3 Le 26:40,\allowbreak41 1Ki 8:37-\allowbreak39 Ps 78:34,\allowbreak35}
\crossref{2Chr}{12}{7}{Jud 10:15,\allowbreak16 1Ki 21:28,\allowbreak29 Jer 3:13 Lu 15:18-\allowbreak21}
\crossref{2Chr}{12}{8}{Ne 9:36 Isa 26:13}
\crossref{2Chr}{12}{9}{1Ki 14:25,\allowbreak26}
\crossref{2Chr}{12}{10}{1Ki 14:27 La 4:1}
\crossref{2Chr}{12}{11}{1Ki 14:28 Eze 40:7,\allowbreak10,\allowbreak12,\allowbreak13,\allowbreak16,\allowbreak21,\allowbreak29,\allowbreak33,\allowbreak36}
\crossref{2Chr}{12}{12}{12:6,\allowbreak7; 33:12,\allowbreak13 Isa 57:15 La 3:22,\allowbreak33,\allowbreak42 1Pe 5:6}
\crossref{2Chr}{12}{13}{2Ch 13:7 1Ki 14:21}
\crossref{2Chr}{12}{14}{2Ch 11:16; 19:3; 30:19 1Sa 7:3 1Ch 29:18 Ps 57:7; 78:8,\allowbreak37 1Co 15:58}
\crossref{2Chr}{12}{15}{2Ch 9:29}
\crossref{2Chr}{12}{16}{1Ki 14:29-\allowbreak31}
\crossref{2Chr}{13}{1}{}
\crossref{2Chr}{13}{2}{2Ch 11:20}
\crossref{2Chr}{13}{3}{1Sa 17:1-\allowbreak3}
\crossref{2Chr}{13}{4}{2Ch 15:2 Jud 9:7}
\crossref{2Chr}{13}{5}{Ne 5:9 Pr 1:29 2Pe 3:5}
\crossref{2Chr}{13}{6}{2Ch 10:19 1Ki 11:26; 12:20,\allowbreak27}
\crossref{2Chr}{13}{7}{Jud 9:4; 11:3 1Sa 22:2 Job 30:8 Ps 26:4 Pr 12:11; 28:19 Ac 17:5}
\crossref{2Chr}{13}{8}{2Ch 9:8 Ps 2:1-\allowbreak6 Isa 7:6,\allowbreak7; 9:6,\allowbreak7 Lu 19:14,\allowbreak27}
\crossref{2Chr}{13}{9}{2Ch 11:14,\allowbreak15}
\crossref{2Chr}{13}{10}{Ex 29:1-\allowbreak37 Nu 16:40; 18:1-\allowbreak7}
\crossref{2Chr}{13}{11}{2Ch 2:4 Ex 29:38}
\crossref{2Chr}{13}{12}{Nu 23:21 1Sa 4:5-\allowbreak7 Isa 8:10 Zec 10:5 Ro 8:31}
\crossref{2Chr}{13}{13}{2Ch 20:22 Jos 8:4 Pr 21:30 Jer 4:22}
\crossref{2Chr}{13}{14}{Ex 14:10 Jos 8:20 Jud 20:33-\allowbreak43 2Sa 10:8-\allowbreak14}
\crossref{2Chr}{13}{15}{2Ch 20:21 Jos 6:16,\allowbreak20 Jud 7:18-\allowbreak22 Ps 47:1,\allowbreak5}
\crossref{2Chr}{13}{16}{Ge 14:20 De 2:36; 3:3 Jos 10:12; 21:44 Jud 1:4; 11:21 1Sa 23:7}
\crossref{2Chr}{13}{17}{13:3,\allowbreak12; 28:6 Isa 10:16-\allowbreak19; 37:36 Na 1:5 1Co 10:22}
\crossref{2Chr}{13}{18}{2Ch 16:8,\allowbreak9; 20:20 2Ki 18:5 1Ch 5:20 Ps 22:4,\allowbreak5; 146:5 Da 3:28 Na 1:7}
\crossref{2Chr}{13}{19}{Jos 10:19,\allowbreak39; 11:12 1Sa 31:7}
\crossref{2Chr}{13}{20}{Ps 18:37,\allowbreak38}
\crossref{2Chr}{13}{21}{2Sa 5:12,\allowbreak13}
\crossref{2Chr}{13}{22}{2Ch 9:29; 12:15}
\crossref{2Chr}{14}{1}{2Ch 9:31}
\crossref{2Chr}{14}{2}{2Ch 31:20 1Ki 15:11,\allowbreak14 Lu 1:75}
\crossref{2Chr}{14}{3}{De 7:5 1Ki 11:7,\allowbreak8; 14:22-\allowbreak24}
\crossref{2Chr}{14}{4}{2Ch 29:21,\allowbreak27,\allowbreak30; 30:12; 33:16; 34:32,\allowbreak33 Ge 18:19 Jos 24:15 1Sa 3:13}
\crossref{2Chr}{14}{5}{2Ch 34:4}
\crossref{2Chr}{14}{6}{2Ch 8:2-\allowbreak6; 11:5-\allowbreak12}
\crossref{2Chr}{14}{7}{2Ch 32:5 Ac 9:31}
\crossref{2Chr}{14}{8}{2Ch 11:1; 13:3; 17:14-\allowbreak19; 25:5}
\crossref{2Chr}{14}{9}{2Ch 12:2,\allowbreak3; 16:8 2Ki 19:9 Isa 8:9,\allowbreak10 Eze 30:5 Re 16:14}
\crossref{2Chr}{14}{10}{Jos 19:4 Jud 1:17}
\crossref{2Chr}{14}{11}{2Ch 13:14; 18:31; 32:20 Ex 14:10 1Ch 5:20 Ps 18:6; 22:5; 34:6; 50:15}
\crossref{2Chr}{14}{12}{2Ch 13:15; 20:22 Ex 14:25 De 28:7; 32:39 Jos 10:10 Ps 60:12}
\crossref{2Chr}{14}{13}{14:14 Ge 10:1,\allowbreak19; 20:1; 26:1}
\crossref{2Chr}{14}{14}{2Ch 17:10; 20:29 Ge 35:5 De 2:25 Jos 2:9-\allowbreak11,\allowbreak24; 5:1 1Sa 14:15}
\crossref{2Chr}{14}{15}{1Ch 4:41}
\crossref{2Chr}{15}{1}{2Ch 20:14; 24:20 Nu 24:2 Jud 3:10 2Sa 23:2 2Pe 1:21}
\crossref{2Chr}{15}{2}{2Ch 13:4; 20:15,\allowbreak20 Jud 9:7 Ps 49:1,\allowbreak2 Isa 7:13 Mt 13:9 Re 2:7,\allowbreak11,\allowbreak17}
\crossref{2Chr}{15}{3}{1Ki 12:28-\allowbreak33 Ho 3:4}
\crossref{2Chr}{15}{4}{De 4:29,\allowbreak30 Jud 3:9,\allowbreak10; 10:10-\allowbreak16 Ps 106:44 Ho 6:1; 14:1-\allowbreak3}
\crossref{2Chr}{15}{5}{Jud 5:6 1Sa 13:6 Ps 121:8}
\crossref{2Chr}{15}{6}{2Ch 12:15; 13:17 Mr 13:8 Lu 21:9,\allowbreak10}
\crossref{2Chr}{15}{7}{Jos 1:7,\allowbreak9 1Ch 28:20 Ps 27:14 Isa 35:3,\allowbreak4 Da 10:19 1Co 16:13}
\crossref{2Chr}{15}{8}{15:1}
\crossref{2Chr}{15}{9}{2Ch 11:16; 30:1-\allowbreak11,\allowbreak25}
\crossref{2Chr}{15}{10}{Es 8:9}
\crossref{2Chr}{15}{11}{2Ch 14:13-\allowbreak15 Nu 31:28,\allowbreak29,\allowbreak50 1Sa 15:15,\allowbreak21 1Ch 26:26,\allowbreak27}
\crossref{2Chr}{15}{12}{2Ch 23:16; 29:10; 34:31,\allowbreak32 De 29:1,\allowbreak12 2Ki 23:3 Ne 9:38; 10:29}
\crossref{2Chr}{15}{13}{Ex 22:20 De 13:5-\allowbreak15; 17:2-\allowbreak5 1Ki 18:40}
\crossref{2Chr}{15}{14}{Ne 5:13; 10:29}
\crossref{2Chr}{15}{15}{2Ch 23:16-\allowbreak21; 29:10,\allowbreak36 De 26:11 Ne 8:9 Ps 32:11; 119:111 Pr 3:17}
\crossref{2Chr}{15}{16}{1Ki 15:13-\allowbreak24}
\crossref{2Chr}{15}{17}{2Ch 14:3-\allowbreak5 De 12:13,\allowbreak14 1Ki 3:2-\allowbreak4; 22:43 2Ki 12:3; 14:4}
\crossref{2Chr}{15}{18}{1Ki 7:51; 15:14,\allowbreak15 1Ch 26:20-\allowbreak26}
\crossref{2Chr}{15}{19}{2Ch 16:1 1Ki 15:16,\allowbreak17,\allowbreak31,\allowbreak33}
\crossref{2Chr}{16}{1}{1Ki 15:16-\allowbreak22}
\crossref{2Chr}{16}{2}{2Ch 28:21 2Ki 12:18; 16:8; 18:15}
\crossref{2Chr}{16}{3}{2Ch 18:3; 19:2 Jud 2:2 Isa 31:1-\allowbreak3 2Co 6:16}
\crossref{2Chr}{16}{4}{1Ti 6:10 2Pe 2:15}
\crossref{2Chr}{16}{5}{16:1}
\crossref{2Chr}{16}{6}{1Ki 15:22}
\crossref{2Chr}{16}{7}{2Ch 19:2; 20:34 1Ki 16:1}
\crossref{2Chr}{16}{8}{2Ch 12:3; 14:9-\allowbreak12}
\crossref{2Chr}{16}{9}{2Ch 6:20 Job 34:21 Ps 34:15; 113:6 Pr 5:21; 15:3 Jer 16:17; 32:19}
\crossref{2Chr}{16}{10}{2Ch 25:16; 26:19 2Sa 12:13; 24:10-\allowbreak14 Ps 141:5 Pr 9:7-\allowbreak9}
\crossref{2Chr}{16}{11}{2Ch 9:29; 12:15; 20:34; 26:22}
\crossref{2Chr}{16}{12}{Mt 7:2 Lu 6:37,\allowbreak38 Re 3:19}
\crossref{2Chr}{16}{13}{1Ki 15:24}
\crossref{2Chr}{16}{14}{2Ch 35:24 Isa 22:16 Joh 19:41,\allowbreak42}
\crossref{2Chr}{17}{1}{1Ki 15:24; 22:41 1Ch 11:10 Mt 1:8}
\crossref{2Chr}{17}{2}{2Ch 11:11,\allowbreak12}
\crossref{2Chr}{17}{3}{2Ch 15:2,\allowbreak9 Ge 39:2,\allowbreak3,\allowbreak21 Ex 3:12; 4:12 Jos 1:5,\allowbreak9 Jud 2:18; 6:12}
\crossref{2Chr}{17}{4}{Lu 1:6 1Th 2:12; 4:1}
\crossref{2Chr}{17}{5}{2Sa 7:25,\allowbreak26 1Ki 9:4,\allowbreak5 Ps 127:1; 132:12 1Pe 5:10}
\crossref{2Chr}{17}{6}{De 28:47 Job 22:26}
\crossref{2Chr}{17}{7}{2Ch 15:3; 30:22; 35:3 De 33:10 Ne 8:7,\allowbreak8,\allowbreak13,\allowbreak14; 9:3 Mt 4:23 Mr 4:2}
\crossref{2Chr}{17}{8}{Ezr 7:1-\allowbreak6 Mal 2:7}
\crossref{2Chr}{17}{9}{2Ch 35:3 Ne 8:7}
\crossref{2Chr}{17}{10}{2Ch 14:14 Ge 35:5 Ex 15:14-\allowbreak16 Jos 2:9-\allowbreak11}
\crossref{2Chr}{17}{11}{17:5; 9:14; 26:8 2Sa 8:2 2Ki 3:4}
\crossref{2Chr}{17}{12}{2Ch 18:1 1Ch 29:25}
\crossref{2Chr}{17}{13}{2Ch 26:10-\allowbreak15 1Ch 27:25-\allowbreak31}
\crossref{2Chr}{17}{14}{Ge 12:2; 13:16; 15:5}
\crossref{2Chr}{17}{15}{17:15}
\crossref{2Chr}{17}{16}{Jud 5:2,\allowbreak9 1Ch 29:9,\allowbreak14,\allowbreak17 Ps 110:3 2Co 8:3-\allowbreak5,\allowbreak12}
\crossref{2Chr}{17}{17}{2Ch 14:8 2Sa 1:21,\allowbreak22}
\crossref{2Chr}{17}{18}{1Ch 26:4}
\crossref{2Chr}{17}{19}{17:2,\allowbreak12; 11:12,\allowbreak23}
\crossref{2Chr}{18}{1}{2Ch 1:11-\allowbreak15; 17:5,\allowbreak12 Mt 6:33}
\crossref{2Chr}{18}{2}{1Ki 17:7 Ne 13:6}
\crossref{2Chr}{18}{3}{1Ki 22:4 2Ki 3:7 Ps 139:21 Eph 5:11 2Jo 1:10,\allowbreak11}
\crossref{2Chr}{18}{4}{2Ch 34:26 1Sa 23:2; 23:2,\allowbreak4,\allowbreak9-\allowbreak12 2Sa 2:1; 5:19,\allowbreak23 1Ki 22:5,\allowbreak6}
\crossref{2Chr}{18}{5}{1Ki 18:19 2Ki 3:13 2Ti 4:3}
\crossref{2Chr}{18}{6}{1Ki 22:7-\allowbreak9 2Ki 3:11-\allowbreak13}
\crossref{2Chr}{18}{7}{1Ki 18:4; 19:10}
\crossref{2Chr}{18}{8}{1Sa 8:15 1Ch 28:1}
\crossref{2Chr}{18}{9}{1Ki 22:10-\allowbreak12 Isa 14:9 Eze 26:16 Da 7:9 Mt 19:28}
\crossref{2Chr}{18}{10}{Jer 23:17,\allowbreak21,\allowbreak25,\allowbreak31; 28:2,\allowbreak3; 29:21 Eze 13:7; 22:28}
\crossref{2Chr}{18}{11}{18:5,\allowbreak12,\allowbreak33,\allowbreak34 Pr 24:24,\allowbreak25 Mic 3:5 2Pe 2:1-\allowbreak3 Jude 1:16 Re 16:13,\allowbreak14}
\crossref{2Chr}{18}{12}{Job 22:13 Ps 10:11 Isa 30:10 Ho 7:3 Am 7:13 Mic 2:6,\allowbreak11}
\crossref{2Chr}{18}{13}{Nu 22:18-\allowbreak20,\allowbreak35; 23:12,\allowbreak26; 24:13 1Ki 22:14 Jer 23:28; 42:4}
\crossref{2Chr}{18}{14}{1Ki 18:27; 22:15 Ec 11:1 La 4:21 Am 4:4,\allowbreak5 Mt 26:45}
\crossref{2Chr}{18}{15}{1Sa 14:24 1Ki 22:16 Mt 26:63 Mr 5:7 Ac 19:13}
\crossref{2Chr}{18}{16}{Mt 26:64}
\crossref{2Chr}{18}{17}{18:7 1Ki 22:18 Pr 29:1 Jer 43:2,\allowbreak3}
\crossref{2Chr}{18}{18}{Isa 1:10; 28:14; 39:5 Jer 2:4; 19:3; 34:4 Am 7:16}
\crossref{2Chr}{18}{19}{1Ki 22:20 Job 12:16 Isa 6:9,\allowbreak10; 54:16 Eze 14:9 2Th 2:11,\allowbreak12}
\crossref{2Chr}{18}{20}{Job 1:6; 2:1 2Co 11:3,\allowbreak13-\allowbreak15}
\crossref{2Chr}{18}{21}{18:22 Ge 3:4,\allowbreak5}
\crossref{2Chr}{18}{22}{Ex 4:21 Job 12:16 Isa 19:14 Eze 14:3-\allowbreak5,\allowbreak9 Mt 24:24,\allowbreak25}
\crossref{2Chr}{18}{23}{18:10 1Ki 22:23-\allowbreak25 Isa 50:5,\allowbreak6 Jer 20:2 La 3:30 Mic 5:1 Mt 26:67}
\crossref{2Chr}{18}{24}{Isa 26:11 Jer 28:16,\allowbreak17; 29:21,\allowbreak22,\allowbreak32}
\crossref{2Chr}{18}{25}{18:8 Jer 37:15-\allowbreak21; 38:6,\allowbreak7 Ac 24:25-\allowbreak27}
\crossref{2Chr}{18}{26}{18:15; 16:10 1Ki 22:26-\allowbreak28 Jer 20:2,\allowbreak3 Mt 5:12 Lu 3:19,\allowbreak20 Ac 5:18}
\crossref{2Chr}{18}{27}{Nu 16:29 Am 9:10 Ac 13:10,\allowbreak11}
\crossref{2Chr}{18}{28}{1Ki 22:29-\allowbreak33}
\crossref{2Chr}{18}{29}{1Sa 28:8 1Ki 14:2-\allowbreak6; 20:38 Job 24:15 Jer 23:24}
\crossref{2Chr}{18}{30}{1Ki 20:33,\allowbreak34,\allowbreak42}
\crossref{2Chr}{18}{31}{2Ch 13:14; 14:11 Ex 14:10 Ps 116:1,\allowbreak2 2Co 1:9,\allowbreak10}
\crossref{2Chr}{18}{32}{1Ki 22:33}
\crossref{2Chr}{18}{33}{1Ki 22:34}
\crossref{2Chr}{18}{34}{18:16,\allowbreak19,\allowbreak27 Nu 32:23 Pr 13:21; 28:17}
\crossref{2Chr}{19}{1}{2Ch 18:31,\allowbreak32}
\crossref{2Chr}{19}{2}{2Ch 20:34 1Ki 16:1,\allowbreak7,\allowbreak12}
\crossref{2Chr}{19}{3}{2Ch 12:12; 17:3-\allowbreak6 1Ki 14:13 Ro 7:18}
\crossref{2Chr}{19}{4}{1Sa 7:15-\allowbreak17}
\crossref{2Chr}{19}{5}{19:8 De 16:18-\allowbreak20 Ro 13:1-\allowbreak5 1Pe 2:13,\allowbreak14}
\crossref{2Chr}{19}{6}{Jos 22:5 1Ch 28:10 Lu 12:15; 21:8 Ac 5:35; 22:26}
\crossref{2Chr}{19}{7}{Ge 42:18 Ex 18:21,\allowbreak22,\allowbreak25,\allowbreak26 Ne 5:15 Isa 1:23-\allowbreak26}
\crossref{2Chr}{19}{8}{2Ch 17:8 De 17:8-\allowbreak13 1Ch 23:4; 26:29}
\crossref{2Chr}{19}{9}{19:7 De 1:16,\allowbreak17 2Sa 23:3 Isa 11:3-\allowbreak5; 32:1}
\crossref{2Chr}{19}{10}{De 17:8-\allowbreak13}
\crossref{2Chr}{19}{11}{1Ch 6:11}
\crossref{2Chr}{20}{1}{2Ch 19:5,\allowbreak11; 32:1}
\crossref{2Chr}{20}{2}{Ge 14:7}
\crossref{2Chr}{20}{3}{Ge 32:7-\allowbreak11,\allowbreak24-\allowbreak28 Ps 56:3,\allowbreak4 Isa 37:3-\allowbreak6 Jon 1:16 Mt 10:28}
\crossref{2Chr}{20}{4}{Ps 34:5,\allowbreak6; 50:15; 60:10-\allowbreak12}
\crossref{2Chr}{20}{5}{2Ch 6:12,\allowbreak13; 34:31 2Ki 19:15-\allowbreak19}
\crossref{2Chr}{20}{6}{Ex 3:6,\allowbreak15,\allowbreak16 1Ch 29:18}
\crossref{2Chr}{20}{7}{2Ch 14:11 Ge 17:7 Ex 6:7; 19:5-\allowbreak7; 20:2 1Ch 17:21-\allowbreak24}
\crossref{2Chr}{20}{8}{2Ch 2:4; 6:10}
\crossref{2Chr}{20}{9}{2Ch 6:28-\allowbreak30 1Ki 8:33,\allowbreak37}
\crossref{2Chr}{20}{10}{Nu 20:17-\allowbreak21 De 2:4,\allowbreak5,\allowbreak9,\allowbreak19 Jud 11:15-\allowbreak18}
\crossref{2Chr}{20}{11}{Jud 11:23,\allowbreak24 Ps 83:3-\allowbreak12}
\crossref{2Chr}{20}{12}{De 32:36 Jud 11:27 1Sa 3:13 Ps 7:6,\allowbreak8; 9:19; 43:1 Isa 2:4; 42:4}
\crossref{2Chr}{20}{13}{De 29:10 Ezr 10:1 Jon 3:5 Ac 21:5}
\crossref{2Chr}{20}{14}{Isa 58:9; 65:24 Da 9:20,\allowbreak21 Ac 10:4,\allowbreak31}
\crossref{2Chr}{20}{15}{2Ch 32:7,\allowbreak8 Ex 14:13,\allowbreak14 De 1:29,\allowbreak30; 20:1,\allowbreak4; 31:6,\allowbreak8 Jos 11:6 Ne 4:14}
\crossref{2Chr}{20}{16}{}
\crossref{2Chr}{20}{17}{20:22,\allowbreak23 Ex 14:13,\allowbreak14,\allowbreak25}
\crossref{2Chr}{20}{18}{2Ch 7:3 Ge 24:26 Ex 4:31}
\crossref{2Chr}{20}{19}{1Ch 15:16-\allowbreak22; 16:5,\allowbreak42; 23:5; 25:1-\allowbreak7}
\crossref{2Chr}{20}{20}{2Ch 11:6 2Sa 14:2 1Ch 4:5 Jer 6:1}
\crossref{2Chr}{20}{21}{1Ch 13:1,\allowbreak2 Pr 11:14}
\crossref{2Chr}{20}{22}{}
\crossref{2Chr}{20}{23}{Ge 14:6; 36:8,\allowbreak9 De 2:5 Jos 24:4 Eze 35:2,\allowbreak3}
\crossref{2Chr}{20}{24}{Ex 14:30 1Ch 5:22 Ps 110:6 Isa 37:36 Jer 33:5}
\crossref{2Chr}{20}{25}{Ex 12:35,\allowbreak36 1Sa 30:19,\allowbreak20 2Ki 7:9-\allowbreak16 Ps 68:12 Ro 8:37}
\crossref{2Chr}{20}{26}{Ex 15:1-\allowbreak19 2Sa 22:1 Ps 103:1,\allowbreak2; 107:21,\allowbreak22 Lu 1:68 Re 19:1-\allowbreak6}
\crossref{2Chr}{20}{27}{2Sa 6:14,\allowbreak15 Mic 2:13 Heb 6:20}
\crossref{2Chr}{20}{28}{}
\crossref{2Chr}{20}{29}{2Ch 17:10 Ge 35:5 Ex 23:27 Jos 5:1 2Ki 7:6}
\crossref{2Chr}{20}{30}{2Ch 14:6,\allowbreak7; 15:15 Jos 23:1 2Sa 7:1 Job 34:29 Pr 16:7 Joh 14:27}
\crossref{2Chr}{20}{31}{1Ki 22:41-\allowbreak44}
\crossref{2Chr}{20}{32}{2Ch 17:3-\allowbreak6}
\crossref{2Chr}{20}{33}{2Ch 14:3; 17:6}
\crossref{2Chr}{20}{34}{2Ch 12:15; 13:22; 16:11}
\crossref{2Chr}{20}{35}{1Ki 22:48,\allowbreak49}
\crossref{2Chr}{20}{36}{}
\crossref{2Chr}{20}{37}{2Ch 19:2 Jos 7:11,\allowbreak12 Pr 13:20}
\crossref{2Chr}{21}{1}{1Ki 22:50}
\crossref{2Chr}{21}{2}{}
\crossref{2Chr}{21}{3}{2Ch 11:23 Ge 25:6 De 21:15-\allowbreak17}
\crossref{2Chr}{21}{4}{21:17; 22:8,\allowbreak10 Ge 4:8 Jud 9:5,\allowbreak56,\allowbreak57 1Jo 3:12}
\crossref{2Chr}{21}{5}{}
\crossref{2Chr}{21}{6}{1Ki 16:25-\allowbreak33}
\crossref{2Chr}{21}{7}{2Ch 22:11 Isa 7:6,\allowbreak7}
\crossref{2Chr}{21}{8}{Ge 27:40 2Ki 8:20-\allowbreak22}
\crossref{2Chr}{21}{9}{2Ki 8:21}
\crossref{2Chr}{21}{10}{Jos 21:13 2Ki 19:8}
\crossref{2Chr}{21}{11}{De 12:2-\allowbreak4 1Ki 11:7 Ps 78:58 Eze 20:28}
\crossref{2Chr}{21}{12}{}
\crossref{2Chr}{21}{13}{1Ki 16:25,\allowbreak30-\allowbreak33}
\crossref{2Chr}{21}{14}{Le 26:21}
\crossref{2Chr}{21}{15}{Ps 109:18 Ac 1:18}
\crossref{2Chr}{21}{16}{2Ch 33:11 1Sa 26:19 2Sa 24:1 1Ki 11:11,\allowbreak14,\allowbreak23 Ezr 1:1,\allowbreak5 Isa 10:5,\allowbreak6}
\crossref{2Chr}{21}{17}{Job 5:3,\allowbreak4}
\crossref{2Chr}{21}{18}{21:15 2Ki 9:29 Ac 12:23}
\crossref{2Chr}{21}{19}{2Ch 16:14 Jer 34:5}
\crossref{2Chr}{21}{20}{21:5}
\crossref{2Chr}{22}{1}{2Ch 23:3; 26:1; 33:25; 36:1}
\crossref{2Chr}{22}{2}{2Ch 21:6 1Ki 16:28}
\crossref{2Chr}{22}{3}{Ge 6:4,\allowbreak5 De 7:3,\allowbreak4; 13:6-\allowbreak10 Jud 17:4,\allowbreak5 Ne 13:23-\allowbreak27 Mal 2:15}
\crossref{2Chr}{22}{4}{2Ch 24:17,\allowbreak18 Pr 1:10; 12:5; 13:20; 19:27}
\crossref{2Chr}{22}{5}{Ps 1:1 Mic 6:16}
\crossref{2Chr}{22}{6}{2Ki 9:15}
\crossref{2Chr}{22}{7}{Mal 4:3}
\crossref{2Chr}{22}{8}{2Ki 10:10-\allowbreak14}
\crossref{2Chr}{22}{9}{1Ki 13:21}
\crossref{2Chr}{22}{10}{22:2-\allowbreak4 2Ki 11:1}
\crossref{2Chr}{22}{11}{2Ki 11:2}
\crossref{2Chr}{22}{12}{Ps 27:5}
\crossref{2Chr}{23}{1}{2Ki 11:4-\allowbreak16}
\crossref{2Chr}{23}{2}{Ps 112:5 Mt 10:16 Eph 5:15}
\crossref{2Chr}{23}{3}{23:16 2Sa 5:3 2Ki 11:17 1Ch 11:3}
\crossref{2Chr}{23}{4}{1Ch 23:3-\allowbreak6; 24:3-\allowbreak6 Lu 1:8,\allowbreak9}
\crossref{2Chr}{23}{5}{2Ki 11:5,\allowbreak6 Eze 44:2,\allowbreak3; 46:2,\allowbreak3}
\crossref{2Chr}{23}{6}{2Ki 11:6,\allowbreak7 1Ch 23:28-\allowbreak32}
\crossref{2Chr}{23}{7}{2Ki 11:8,\allowbreak9}
\crossref{2Chr}{23}{8}{2Ki 11:9}
\crossref{2Chr}{23}{9}{1Sa 21:9 2Sa 8:7}
\crossref{2Chr}{23}{10}{2Ki 11:11}
\crossref{2Chr}{23}{11}{2Ch 22:11 2Ki 11:12}
\crossref{2Chr}{23}{12}{2Ki 11:13-\allowbreak16}
\crossref{2Chr}{23}{13}{Ps 14:5}
\crossref{2Chr}{23}{14}{}
\crossref{2Chr}{23}{15}{2Ch 22:10 Jud 1:7 Ps 5:6; 55:23 Mt 7:2 Jas 2:13 Re 16:5-\allowbreak7}
\crossref{2Chr}{23}{16}{2Ch 15:12,\allowbreak14; 29:10; 34:31,\allowbreak32 De 5:2,\allowbreak3; 29:1-\allowbreak15 2Ki 11:17 Ezr 10:3}
\crossref{2Chr}{23}{17}{2Ch 34:4,\allowbreak7 2Ki 10:25-\allowbreak28; 11:18; 18:4}
\crossref{2Chr}{23}{18}{1Ch 23:1-\allowbreak24:31}
\crossref{2Chr}{23}{19}{1Ch 9:23,\allowbreak24; 26:1-\allowbreak32}
\crossref{2Chr}{23}{20}{2Ki 11:9,\allowbreak10,\allowbreak19}
\crossref{2Chr}{23}{21}{2Ki 11:20 Ps 58:10,\allowbreak11 Pr 11:10 Re 18:20; 19:1-\allowbreak4}
\crossref{2Chr}{24}{1}{1Ch 3:11}
\crossref{2Chr}{24}{2}{2Ch 25:2; 26:4,\allowbreak5 2Ki 12:2 Ps 78:36,\allowbreak37; 106:12,\allowbreak13 Mr 4:16,\allowbreak17}
\crossref{2Chr}{24}{3}{24:15 Ge 21:21; 24:4}
\crossref{2Chr}{24}{4}{24:5-\allowbreak7}
\crossref{2Chr}{24}{5}{2Ch 29:3; 34:8,\allowbreak9 2Ki 12:4,\allowbreak5}
\crossref{2Chr}{24}{6}{2Sa 24:3}
\crossref{2Chr}{24}{7}{2Ch 28:22-\allowbreak24 Es 7:6 Pr 10:7 2Th 2:8 Re 2:20}
\crossref{2Chr}{24}{8}{2Ki 12:8,\allowbreak9 Mr 12:41}
\crossref{2Chr}{24}{9}{24:6 Mt 17:24-\allowbreak27}
\crossref{2Chr}{24}{10}{1Ch 29:9 Isa 64:5 Ac 2:45-\allowbreak47 2Co 8:2; 9:7}
\crossref{2Chr}{24}{11}{2Ki 12:10-\allowbreak12}
\crossref{2Chr}{24}{12}{2Ch 34:9-\allowbreak11}
\crossref{2Chr}{24}{13}{1Ch 22:5 Hag 2:3 Mr 13:1,\allowbreak2}
\crossref{2Chr}{24}{14}{2Ki 12:13,\allowbreak14}
\crossref{2Chr}{24}{15}{Ge 47:9 Ps 90:10}
\crossref{2Chr}{24}{16}{1Sa 2:30 1Ki 2:10 Ac 2:29}
\crossref{2Chr}{24}{17}{De 31:27 Ac 20:29,\allowbreak30 2Pe 1:15}
\crossref{2Chr}{24}{18}{24:4; 21:13; 33:3-\allowbreak7 1Ki 11:4,\allowbreak5; 14:9,\allowbreak23}
\crossref{2Chr}{24}{19}{2Ch 36:15,\allowbreak16 2Ki 17:13-\allowbreak15 Ne 9:26 Jer 7:25,\allowbreak26; 25:4,\allowbreak5; 26:5; 44:4,\allowbreak5}
\crossref{2Chr}{24}{20}{2Ch 15:1; 20:14}
\crossref{2Chr}{24}{21}{Jer 11:19; 18:18; 38:4-\allowbreak6}
\crossref{2Chr}{24}{22}{Ps 109:4 Lu 17:15-\allowbreak18 Joh 10:32}
\crossref{2Chr}{24}{23}{1Ki 20:22,\allowbreak26}
\crossref{2Chr}{24}{24}{Le 26:8,\allowbreak37 De 32:30 Isa 30:17 Jer 37:10}
\crossref{2Chr}{24}{25}{2Ch 21:16,\allowbreak18,\allowbreak19; 22:6}
\crossref{2Chr}{24}{26}{2Ki 12:21}
\crossref{2Chr}{24}{27}{2Ki 12:18}
\crossref{2Chr}{25}{1}{2Ki 14:1-\allowbreak3}
\crossref{2Chr}{25}{2}{25:14; 24:2; 26:4 1Sa 16:7 2Ki 14:4 Ps 78:37 Isa 29:13 Ho 10:2}
\crossref{2Chr}{25}{3}{2Ki 14:5-\allowbreak22}
\crossref{2Chr}{25}{4}{De 24:16 2Ki 14:5,\allowbreak6 Jer 31:29,\allowbreak30 Eze 18:4,\allowbreak20}
\crossref{2Chr}{25}{5}{Ex 18:25 1Sa 8:12 1Ch 13:1; 27:1}
\crossref{2Chr}{25}{6}{2Ch 13:3; 17:13 Jud 6:12}
\crossref{2Chr}{25}{7}{2Sa 12:1 1Ki 13:1 1Ti 6:11 2Ti 3:17}
\crossref{2Chr}{25}{8}{2Ch 18:14 Ec 11:9 Isa 8:9,\allowbreak10 Joe 3:9-\allowbreak14 Mt 26:45}
\crossref{2Chr}{25}{9}{25:10,\allowbreak13 2Ki 6:23}
\crossref{2Chr}{25}{10}{1Ki 12:24}
\crossref{2Chr}{25}{11}{2Sa 8:13 2Ki 14:7 Ps 60:1}
\crossref{2Chr}{25}{12}{2Sa 12:31 1Ch 20:3}
\crossref{2Chr}{25}{13}{25:9}
\crossref{2Chr}{25}{14}{2Ch 28:23 Isa 44:19}
\crossref{2Chr}{25}{15}{25:7; 16:7-\allowbreak9; 19:2; 20:37 2Sa 12:1-\allowbreak6}
\crossref{2Chr}{25}{16}{2Ch 16:10; 18:25; 24:21 Am 7:10-\allowbreak13 Mt 21:23}
\crossref{2Chr}{25}{17}{25:13 2Ki 14:8-\allowbreak14}
\crossref{2Chr}{25}{18}{Jud 9:8-\allowbreak15 1Ki 4:33}
\crossref{2Chr}{25}{19}{2Ch 26:16; 32:25 De 8:14 Pr 13:10; 16:18; 28:25 Da 5:20-\allowbreak23 Hab 2:4}
\crossref{2Chr}{25}{20}{25:16; 22:7 1Ki 12:15 Ps 81:11,\allowbreak12 Ac 28:25-\allowbreak27 2Th 2:9-\allowbreak11}
\crossref{2Chr}{25}{21}{Jos 21:16 1Sa 6:9,\allowbreak19,\allowbreak20}
\crossref{2Chr}{25}{22}{2Ch 28:5,\allowbreak6}
\crossref{2Chr}{25}{23}{2Ch 33:11; 36:6,\allowbreak10 Pr 16:18; 29:23 Da 4:37 Ob 1:3 Lu 14:11}
\crossref{2Chr}{25}{24}{2Ch 12:9 2Ki 14:14}
\crossref{2Chr}{25}{25}{2Ki 14:17-\allowbreak22}
\crossref{2Chr}{25}{26}{2Ch 20:34 2Ki 14:15}
\crossref{2Chr}{25}{27}{2Ch 15:2}
\crossref{2Chr}{25}{28}{2Ki 14:20}
\crossref{2Chr}{26}{1}{2Ch 22:1; 33:25}
\crossref{2Chr}{26}{2}{2Ch 8:17 2Ki 14:22; 16:6}
\crossref{2Chr}{26}{3}{Isa 1:1; 6:1 Ho 1:1 Am 1:1 Zec 14:5}
\crossref{2Chr}{26}{4}{2Ch 25:2}
\crossref{2Chr}{26}{5}{2Ch 24:2 Jud 2:7 Ho 6:4 Mr 4:16,\allowbreak17 Ac 20:30}
\crossref{2Chr}{26}{6}{2Ch 21:16 Isa 14:29}
\crossref{2Chr}{26}{7}{2Ch 14:11 1Ch 5:20; 12:18 Ps 18:29,\allowbreak34,\allowbreak35 Isa 14:29 Ac 26:22}
\crossref{2Chr}{26}{8}{2Ch 20:1 Ge 19:38 De 2:19 Jud 11:15-\allowbreak18 1Sa 11:1 2Sa 8:2}
\crossref{2Chr}{26}{9}{2Ch 25:23 2Ki 14:13 Jer 31:38 Zec 14:10}
\crossref{2Chr}{26}{10}{Ge 26:18-\allowbreak21}
\crossref{2Chr}{26}{11}{2Ki 5:2}
\crossref{2Chr}{26}{12}{}
\crossref{2Chr}{26}{13}{2Ch 11:1; 13:3; 14:8; 17:14-\allowbreak19}
\crossref{2Chr}{26}{14}{Jud 20:16 1Sa 17:49}
\crossref{2Chr}{26}{15}{2Ch 2:7,\allowbreak14 Ex 31:4}
\crossref{2Chr}{26}{16}{2Ch 25:19; 32:25 De 8:14,\allowbreak17; 32:13-\allowbreak15 Pr 16:18 Hab 2:4 Col 2:18}
\crossref{2Chr}{26}{17}{1Ch 6:10}
\crossref{2Chr}{26}{18}{2Ch 16:7-\allowbreak9; 19:2 Jer 13:18 Mt 10:18,\allowbreak28; 14:4 2Co 5:16 Ga 2:11}
\crossref{2Chr}{26}{19}{2Ch 16:10; 25:16}
\crossref{2Chr}{26}{20}{Es 6:12}
\crossref{2Chr}{26}{21}{2Ki 15:5}
\crossref{2Chr}{26}{22}{2Ch 9:29; 12:15}
\crossref{2Chr}{26}{23}{2Ki 15:6,\allowbreak7}
\crossref{2Chr}{27}{1}{2Ki 15:32,\allowbreak33-\allowbreak38 1Ch 3:12 Isa 1:1 Ho 1:1 Mic 1:1 Mt 1:9}
\crossref{2Chr}{27}{2}{2Ch 26:4 2Ki 15:34}
\crossref{2Chr}{27}{3}{2Ch 23:20 Jer 20:2}
\crossref{2Chr}{27}{4}{2Ch 11:5-\allowbreak10; 14:7; 26:9,\allowbreak10}
\crossref{2Chr}{27}{5}{}
\crossref{2Chr}{27}{6}{2Ch 26:5}
\crossref{2Chr}{27}{7}{2Ch 20:34; 26:22,\allowbreak23; 32:32,\allowbreak33}
\crossref{2Chr}{27}{8}{27:1 2Ki 15:33}
\crossref{2Chr}{27}{9}{2Ki 15:38}
\crossref{2Chr}{28}{1}{2Ki 16:1,\allowbreak2-\allowbreak20 1Ch 3:13 Isa 1:1; 7:1-\allowbreak12 Ho 1:1 Mic 1:1 Mt 1:9}
\crossref{2Chr}{28}{2}{2Ch 21:6; 22:3,\allowbreak4 1Ki 16:31-\allowbreak33 2Ki 10:26-\allowbreak28}
\crossref{2Chr}{28}{3}{2Ki 23:10 Jer 7:31,\allowbreak32; 19:2-\allowbreak6,\allowbreak13}
\crossref{2Chr}{28}{4}{Le 26:30 De 12:2,\allowbreak3 2Ki 16:4}
\crossref{2Chr}{28}{5}{2Ch 36:5 Ex 20:2,\allowbreak3}
\crossref{2Chr}{28}{6}{2Ki 15:27,\allowbreak37 Isa 7:4,\allowbreak5,\allowbreak9; 9:21}
\crossref{2Chr}{28}{7}{Ge 41:43; 43:12,\allowbreak15 Es 10:3}
\crossref{2Chr}{28}{8}{De 28:25,\allowbreak41}
\crossref{2Chr}{28}{9}{2Ch 19:1,\allowbreak2; 25:15,\allowbreak16 1Ki 20:13,\allowbreak22,\allowbreak42 2Ki 20:14,\allowbreak15}
\crossref{2Chr}{28}{10}{Le 25:39-\allowbreak46}
\crossref{2Chr}{28}{11}{Isa 58:6 Jer 34:14,\allowbreak15 Heb 13:1-\allowbreak3}
\crossref{2Chr}{28}{12}{1Ch 28:1}
\crossref{2Chr}{28}{13}{Nu 32:14 Jos 22:17,\allowbreak18 Mt 23:32,\allowbreak35 Ro 2:5}
\crossref{2Chr}{28}{14}{2Ch 20:21 1Ch 12:23}
\crossref{2Chr}{28}{15}{28:12}
\crossref{2Chr}{28}{16}{2Ki 16:5-\allowbreak7 Isa 7:1-\allowbreak9,\allowbreak17}
\crossref{2Chr}{28}{17}{2Ch 25:11,\allowbreak12 Le 26:18 Ob 1:10,\allowbreak13,\allowbreak14}
\crossref{2Chr}{28}{18}{Eze 16:27,\allowbreak57}
\crossref{2Chr}{28}{19}{De 28:43 1Sa 2:7 Job 40:12 Ps 106:41-\allowbreak43 Pr 29:23}
\crossref{2Chr}{28}{20}{2Ki 15:29; 16:7-\allowbreak10}
\crossref{2Chr}{28}{21}{2Ch 12:9 2Ki 18:15,\allowbreak16 Pr 20:25}
\crossref{2Chr}{28}{22}{2Ch 33:12 Ps 50:15 Isa 1:5 Eze 21:13 Ho 5:15 Re 16:9-\allowbreak11}
\crossref{2Chr}{28}{23}{Hab 1:11}
\crossref{2Chr}{28}{24}{2Ki 16:17,\allowbreak18; 25:13-\allowbreak17}
\crossref{2Chr}{28}{25}{28:3}
\crossref{2Chr}{28}{26}{2Ch 20:34; 27:7-\allowbreak9 2Ki 16:19,\allowbreak20}
\crossref{2Chr}{28}{27}{2Ch 21:20; 26:23; 33:20 1Sa 2:30 Pr 10:7}
\crossref{2Chr}{29}{1}{2Ki 18:1-\allowbreak3 1Ch 3:13 Isa 1:1 Ho 1:1 Mic 1:1 Mt 1:9,\allowbreak10}
\crossref{2Chr}{29}{2}{}
\crossref{2Chr}{29}{3}{2Ch 34:3 Ps 101:3 Ec 9:10 Mt 6:33 Ga 1:16}
\crossref{2Chr}{29}{4}{2Ch 32:6 Ne 3:29 Jer 19:2}
\crossref{2Chr}{29}{5}{2Ch 35:6 Ex 19:10,\allowbreak15 1Ch 15:12}
\crossref{2Chr}{29}{6}{2Ch 28:2-\allowbreak4,\allowbreak23-\allowbreak25; 34:21 Ezr 5:12; 9:7 Ne 9:16,\allowbreak32 Jer 16:19; 44:21}
\crossref{2Chr}{29}{7}{}
\crossref{2Chr}{29}{8}{2Ch 24:18; 34:24,\allowbreak25; 36:14-\allowbreak16 De 28:15-\allowbreak20}
\crossref{2Chr}{29}{9}{2Ch 28:5-\allowbreak8,\allowbreak17 Le 26:17 La 5:7}
\crossref{2Chr}{29}{10}{2Ch 6:7,\allowbreak8}
\crossref{2Chr}{29}{11}{Ga 6:7,\allowbreak8}
\crossref{2Chr}{29}{12}{Ex 6:16-\allowbreak25 Nu 4:2-\allowbreak20 1Ch 6:16-\allowbreak18; 15:5; 23:12-\allowbreak20}
\crossref{2Chr}{29}{13}{Le 10:4}
\crossref{2Chr}{29}{14}{1Ch 6:33; 15:19}
\crossref{2Chr}{29}{15}{29:5}
\crossref{2Chr}{29}{16}{2Ch 3:8; 5:7 Ex 26:33,\allowbreak34 1Ki 6:19,\allowbreak20 Heb 9:2-\allowbreak8,\allowbreak23,\allowbreak24}
\crossref{2Chr}{29}{17}{29:7; 3:4 1Ki 6:3 1Ch 28:11}
\crossref{2Chr}{29}{18}{2Ch 4:1,\allowbreak7}
\crossref{2Chr}{29}{19}{2Ch 28:24}
\crossref{2Chr}{29}{20}{Ge 22:3 Ex 24:4 Jos 6:12 Jer 25:4}
\crossref{2Chr}{29}{21}{Nu 23:1,\allowbreak14,\allowbreak29 1Ch 15:26 Ezr 8:35 Job 42:8 Eze 45:23}
\crossref{2Chr}{29}{22}{Le 1:5; 4:7,\allowbreak18,\allowbreak34; 8:14,\allowbreak15,\allowbreak19,\allowbreak24 Heb 9:21,\allowbreak22}
\crossref{2Chr}{29}{23}{Le 1:4; 4:15,\allowbreak24}
\crossref{2Chr}{29}{24}{Le 6:30; 8:15 Eze 45:15,\allowbreak17 Da 9:24 Ro 5:10,\allowbreak11 2Co 5:18-\allowbreak21}
\crossref{2Chr}{29}{25}{1Ch 9:33; 15:16-\allowbreak22; 16:4,\allowbreak5,\allowbreak42; 25:1-\allowbreak7}
\crossref{2Chr}{29}{26}{1Ch 23:5 Ps 87:7; 150:3-\allowbreak5 Isa 38:20 Am 6:5}
\crossref{2Chr}{29}{27}{2Ch 7:3; 20:21; 23:18 Ps 136:1; 137:3,\allowbreak4}
\crossref{2Chr}{29}{28}{Ps 68:24-\allowbreak26 Re 5:8-\allowbreak14}
\crossref{2Chr}{29}{29}{2Ch 20:18 1Ch 29:20 Ps 72:11 Ro 14:11 Php 2:10,\allowbreak11}
\crossref{2Chr}{29}{30}{2Sa 23:1,\allowbreak2 1Ch 16:7-\allowbreak36}
\crossref{2Chr}{29}{31}{2Ch 13:9}
\crossref{2Chr}{29}{32}{}
\crossref{2Chr}{29}{33}{}
\crossref{2Chr}{29}{34}{2Ch 35:11 Nu 8:15,\allowbreak19; 18:3,\allowbreak6,\allowbreak7}
\crossref{2Chr}{29}{35}{29:32}
\crossref{2Chr}{29}{36}{2Ch 30:12 1Ch 29:18 Ps 10:17 Pr 16:1}
\crossref{2Chr}{30}{1}{2Ch 11:13,\allowbreak16}
\crossref{2Chr}{30}{2}{1Ch 13:1-\allowbreak3 Pr 11:14; 15:22 Ec 4:13}
\crossref{2Chr}{30}{3}{Ex 12:6,\allowbreak18}
\crossref{2Chr}{30}{4}{1Ch 13:4}
\crossref{2Chr}{30}{5}{Ezr 6:8-\allowbreak12 Es 3:12-\allowbreak15; 8:8-\allowbreak10; 9:20,\allowbreak21 Da 6:8}
\crossref{2Chr}{30}{6}{Isa 55:6,\allowbreak7 Jer 4:1 La 5:21 Eze 33:11 Ho 14:1 Joe 2:12-\allowbreak14}
\crossref{2Chr}{30}{7}{Eze 20:13-\allowbreak18 Zec 1:3,\allowbreak4}
\crossref{2Chr}{30}{8}{2Ch 36:13 Ex 32:9 De 10:16 Ro 10:21}
\crossref{2Chr}{30}{9}{2Ch 7:14 Le 26:40-\allowbreak42 De 30:2-\allowbreak4 1Ki 8:50 Ps 106:46}
\crossref{2Chr}{30}{10}{30:6 Es 3:13,\allowbreak15; 8:10,\allowbreak14 Job 9:25}
\crossref{2Chr}{30}{11}{2Ch 12:6,\allowbreak7,\allowbreak12; 33:12,\allowbreak19,\allowbreak23; 34:27 Ex 10:3 Le 26:41 Da 5:22 Lu 14:11}
\crossref{2Chr}{30}{12}{2Ch 29:36 1Ch 29:18,\allowbreak19 Ezr 7:27 Ps 110:3 Jer 24:7; 32:39 Eze 36:26}
\crossref{2Chr}{30}{13}{Ps 84:7}
\crossref{2Chr}{30}{14}{2Ch 28:24; 34:4,\allowbreak7 2Ki 18:22; 23:12,\allowbreak13 Isa 2:18-\allowbreak20}
\crossref{2Chr}{30}{15}{2Ch 29:34 Eze 16:61-\allowbreak63; 43:10,\allowbreak11}
\crossref{2Chr}{30}{16}{2Ch 35:10,\allowbreak15}
\crossref{2Chr}{30}{17}{2Ch 29:34; 35:3-\allowbreak6}
\crossref{2Chr}{30}{18}{30:11}
\crossref{2Chr}{30}{19}{2Ch 19:3; 20:33 1Sa 7:3 1Ch 29:18 Ezr 7:10 Job 11:13 Ps 10:17}
\crossref{2Chr}{30}{20}{Ex 15:26 Ps 103:3 Jas 5:15,\allowbreak16}
\crossref{2Chr}{30}{21}{Ex 12:15; 13:6 Le 23:6 Lu 22:1,\allowbreak7 1Co 5:7,\allowbreak8}
\crossref{2Chr}{30}{22}{2Ch 32:6 Isa 40:1,\allowbreak2 Ho 2:14}
\crossref{2Chr}{30}{23}{30:2}
\crossref{2Chr}{30}{24}{2Ch 35:7,\allowbreak8 1Ch 29:3-\allowbreak9 Eze 45:17 Eph 4:8}
\crossref{2Chr}{30}{25}{30:11,\allowbreak18 Ex 12:43-\allowbreak49}
\crossref{2Chr}{30}{26}{2Ch 7:9,\allowbreak10}
\crossref{2Chr}{30}{27}{Nu 6:23-\allowbreak26 De 10:8}
\crossref{2Chr}{31}{1}{2Ch 30:1-\allowbreak27}
\crossref{2Chr}{31}{2}{2Ch 5:11; 8:14; 23:8 1Ch 16:37,\allowbreak40; 23:1-\allowbreak26:32 Ezr 6:18 Lu 1:5}
\crossref{2Chr}{31}{3}{2Ch 30:24 1Ch 26:26 Eze 45:17; 46:4-\allowbreak7,\allowbreak12-\allowbreak18}
\crossref{2Chr}{31}{4}{31:16 Le 27:30-\allowbreak33 Nu 18:8-\allowbreak21,\allowbreak26-\allowbreak28 Mal 3:8-\allowbreak10}
\crossref{2Chr}{31}{5}{2Ch 24:10,\allowbreak11 Ex 35:5,\allowbreak20-\allowbreak29; 36:5,\allowbreak6 2Co 8:2-\allowbreak5}
\crossref{2Chr}{31}{6}{2Ch 11:16,\allowbreak17}
\crossref{2Chr}{31}{7}{Le 23:16-\allowbreak24}
\crossref{2Chr}{31}{8}{Ge 14:20 Jud 5:9 1Ki 8:14,\allowbreak15 1Ch 29:10-\allowbreak20 Ezr 7:27 2Co 8:16}
\crossref{2Chr}{31}{9}{}
\crossref{2Chr}{31}{10}{2Ch 26:17 1Ki 2:35 1Ch 6:8,\allowbreak14 Eze 44:15}
\crossref{2Chr}{31}{11}{Ne 10:38,\allowbreak39; 13:5,\allowbreak12,\allowbreak13}
\crossref{2Chr}{31}{12}{2Ki 12:15}
\crossref{2Chr}{31}{13}{31:4,\allowbreak11}
\crossref{2Chr}{31}{14}{1Ch 26:12,\allowbreak14,\allowbreak17}
\crossref{2Chr}{31}{15}{31:13}
\crossref{2Chr}{31}{16}{Le 21:22,\allowbreak23}
\crossref{2Chr}{31}{17}{Nu 3:15,\allowbreak20; 4:38,\allowbreak42,\allowbreak46; 17:2,\allowbreak3 Ezr 2:59}
\crossref{2Chr}{31}{18}{31:15 1Ch 9:22}
\crossref{2Chr}{31}{19}{31:15 Le 25:34 Nu 35:2-\allowbreak5 1Ch 6:54,\allowbreak60}
\crossref{2Chr}{31}{20}{1Ki 15:5 2Ki 20:3; 22:2 Joh 1:47 Ac 24:16 1Th 2:10 3Jo 1:5}
\crossref{2Chr}{31}{21}{Ps 1:2,\allowbreak3}
\crossref{2Chr}{32}{1}{2Ch 20:1,\allowbreak2 2Ki 18:13-\allowbreak37 Isa 36:1-\allowbreak22}
\crossref{2Chr}{32}{2}{2Ki 12:17 Lu 9:51,\allowbreak53}
\crossref{2Chr}{32}{3}{2Ch 30:2 2Ki 18:20 Pr 15:22; 20:18; 24:6 Isa 40:13 Ro 11:34}
\crossref{2Chr}{32}{4}{32:30; 30:14}
\crossref{2Chr}{32}{5}{2Ch 12:1; 14:5-\allowbreak7; 17:1,\allowbreak2; 23:1; 26:8 Isa 22:9,\allowbreak10}
\crossref{2Chr}{32}{6}{2Ch 17:14-\allowbreak19 1Ch 27:3,\allowbreak4-\allowbreak34}
\crossref{2Chr}{32}{7}{De 31:6,\allowbreak7,\allowbreak23 Jos 1:6-\allowbreak9 1Ch 28:10,\allowbreak20 Isa 35:4 Da 10:19}
\crossref{2Chr}{32}{8}{Job 40:9 Jer 17:5 1Jo 4:4}
\crossref{2Chr}{32}{9}{2Ki 18:17 Isa 36:2}
\crossref{2Chr}{32}{10}{2Ki 18:19 Isa 36:4}
\crossref{2Chr}{32}{11}{2Ki 18:27 Isa 36:12,\allowbreak18}
\crossref{2Chr}{32}{12}{2Ch 31:1 2Ki 18:4,\allowbreak22 Isa 36:7}
\crossref{2Chr}{32}{13}{2Ki 15:29; 17:5,\allowbreak6; 19:11-\allowbreak13,\allowbreak17,\allowbreak18 Isa 10:9,\allowbreak10,\allowbreak14; 37:12,\allowbreak13,\allowbreak18-\allowbreak20}
\crossref{2Chr}{32}{14}{Isa 10:11,\allowbreak12}
\crossref{2Chr}{32}{15}{2Ki 18:29; 19:10}
\crossref{2Chr}{32}{16}{Job 15:25,\allowbreak26 Ps 73:9}
\crossref{2Chr}{32}{17}{2Ki 19:9,\allowbreak14 Ne 6:5 Isa 37:14}
\crossref{2Chr}{32}{18}{2Ki 18:26-\allowbreak28 Isa 36:13}
\crossref{2Chr}{32}{19}{32:13-\allowbreak17 1Sa 17:36 Job 15:25,\allowbreak26 Ps 10:13,\allowbreak14; 73:8-\allowbreak11; 139:19,\allowbreak20}
\crossref{2Chr}{32}{20}{2Ki 19:14-\allowbreak19 Isa 37:1,\allowbreak14-\allowbreak20}
\crossref{2Chr}{32}{21}{2Ki 19:20,\allowbreak35-\allowbreak37 Isa 10:16-\allowbreak18; 37:21,\allowbreak36,\allowbreak37; 42:8}
\crossref{2Chr}{32}{22}{Ps 18:48-\allowbreak50; 37:39,\allowbreak40; 144:10 Isa 10:24,\allowbreak25; 31:4,\allowbreak5; 33:22 Ho 1:7}
\crossref{2Chr}{32}{23}{2Sa 8:10,\allowbreak11 Ezr 7:15-\allowbreak22,\allowbreak27 Ps 68:29; 72:10 Isa 60:7-\allowbreak9 Mt 2:11}
\crossref{2Chr}{32}{24}{2Ki 20:1-\allowbreak3 Isa 38:1-\allowbreak3}
\crossref{2Chr}{32}{25}{De 32:6 Ps 116:12,\allowbreak13 Ho 14:2 Lu 17:17,\allowbreak18}
\crossref{2Chr}{32}{26}{2Ch 33:12,\allowbreak19,\allowbreak23; 34:27 Le 26:40,\allowbreak41 2Ki 20:19 Jer 26:18,\allowbreak19 Jas 4:10}
\crossref{2Chr}{32}{27}{2Ch 1:12; 9:27; 17:5 Pr 10:22}
\crossref{2Chr}{32}{28}{2Ch 26:10}
\crossref{2Chr}{32}{29}{2Ch 26:10 Ge 13:2-\allowbreak6 1Ch 27:29-\allowbreak31 Job 1:3,\allowbreak9; 42:12}
\crossref{2Chr}{32}{30}{32:4 Isa 22:9-\allowbreak11}
\crossref{2Chr}{32}{31}{2Ki 20:12,\allowbreak13 Isa 39:1,\allowbreak2-\allowbreak8}
\crossref{2Chr}{32}{32}{2Ch 31:20,\allowbreak21}
\crossref{2Chr}{32}{33}{1Ki 1:21; 2:10; 11:43}
\crossref{2Chr}{33}{1}{2Ch 32:33 2Ki 21:1-\allowbreak18 1Ch 3:13 Mt 1:10}
\crossref{2Chr}{33}{2}{2Ch 28:3; 36:14 Le 18:24-\allowbreak30; 20:22,\allowbreak23 De 12:31; 18:9,\allowbreak14}
\crossref{2Chr}{33}{3}{Ec 2:19; 9:18}
\crossref{2Chr}{33}{4}{33:15; 34:3,\allowbreak4 2Ki 21:4,\allowbreak5 Jer 7:30}
\crossref{2Chr}{33}{5}{2Ch 4:9 Jer 32:34,\allowbreak35 Eze 8:7-\allowbreak18}
\crossref{2Chr}{33}{6}{2Ch 28:3 Le 18:21; 20:2 De 12:31; 18:10 2Ki 21:6; 23:10 Jer 7:31,\allowbreak32}
\crossref{2Chr}{33}{7}{2Ki 21:7,\allowbreak8; 23:6}
\crossref{2Chr}{33}{8}{2Sa 7:10 1Ch 17:9}
\crossref{2Chr}{33}{9}{1Ki 14:16; 15:26 2Ki 21:16; 23:26; 24:3,\allowbreak4 Pr 29:12 Mic 6:16}
\crossref{2Chr}{33}{10}{2Ch 36:15,\allowbreak16 Ne 9:29,\allowbreak30 Jer 25:4-\allowbreak7; 44:4,\allowbreak5 Zec 1:4 Ac 7:51,\allowbreak52}
\crossref{2Chr}{33}{11}{De 28:36 Job 36:8}
\crossref{2Chr}{33}{12}{2Ch 28:22 Le 26:39-\allowbreak42 De 4:30,\allowbreak31 Jer 31:18-\allowbreak20 Ho 5:15 Mic 6:9}
\crossref{2Chr}{33}{13}{1Ch 5:20 Ezr 8:23 Job 22:23,\allowbreak27; 33:16-\allowbreak30 Ps 32:3-\allowbreak5; 86:5}
\crossref{2Chr}{33}{14}{2Ch 32:5}
\crossref{2Chr}{33}{15}{33:3-\allowbreak7 2Ki 21:7 Isa 2:17-\allowbreak21 Eze 18:20-\allowbreak22 Ho 14:1-\allowbreak3 Mt 3:8}
\crossref{2Chr}{33}{16}{2Ch 29:18 1Ki 18:30}
\crossref{2Chr}{33}{17}{2Ch 15:17; 32:12 1Ki 22:43 2Ki 15:4}
\crossref{2Chr}{33}{18}{2Ch 20:34; 32:32}
\crossref{2Chr}{33}{19}{33:11,\allowbreak12,\allowbreak19 Pr 15:8 Ac 9:11 1Jo 1:9}
\crossref{2Chr}{33}{20}{2Ch 32:33 2Ki 21:18}
\crossref{2Chr}{33}{21}{33:1 Lu 12:19,\allowbreak20 Jas 4:13-\allowbreak15}
\crossref{2Chr}{33}{22}{33:1-\allowbreak10 2Ki 21:1-\allowbreak11,\allowbreak20 Eze 20:18}
\crossref{2Chr}{33}{23}{33:1,\allowbreak12,\allowbreak19 Jer 8:12}
\crossref{2Chr}{33}{24}{2Ch 24:25,\allowbreak26; 25:27,\allowbreak28 2Sa 4:5-\allowbreak12 2Ki 21:23-\allowbreak26 Ps 55:23 Ro 11:22}
\crossref{2Chr}{33}{25}{Ge 9:5,\allowbreak6 Nu 35:31,\allowbreak33}
\crossref{2Chr}{34}{1}{2Ch 33:25 1Ki 13:2 2Ki 22:1-\allowbreak20 1Ch 3:14,\allowbreak15 Jer 1:2 Zep 1:1}
\crossref{2Chr}{34}{2}{2Ch 14:2; 17:3; 29:2 1Ki 14:8; 15:5 2Ki 22:2}
\crossref{2Chr}{34}{3}{1Ch 22:5; 29:1 Ps 119:9 Ec 12:1 2Ti 3:15}
\crossref{2Chr}{34}{4}{2Ch 33:3 Ex 23:24 Le 26:30 De 7:5,\allowbreak25}
\crossref{2Chr}{34}{5}{1Ki 13:2 2Ki 23:16 Jer 8:1,\allowbreak2}
\crossref{2Chr}{34}{6}{2Ch 30:1,\allowbreak10,\allowbreak11; 31:1 2Ki 23:15-\allowbreak20}
\crossref{2Chr}{34}{7}{34:1 De 9:21}
\crossref{2Chr}{34}{8}{Jer 1:2,\allowbreak3}
\crossref{2Chr}{34}{9}{34:14,\allowbreak15,\allowbreak18,\allowbreak20,\allowbreak22 2Ki 22:4; 23:4}
\crossref{2Chr}{34}{10}{2Ki 12:11,\allowbreak12,\allowbreak14; 22:5,\allowbreak6 Ezr 3:7}
\crossref{2Chr}{34}{11}{2Ch 33:4-\allowbreak7,\allowbreak22}
\crossref{2Chr}{34}{12}{2Ch 31:12 2Ki 12:15; 22:7 Ne 7:2 Pr 28:20 1Co 4:2}
\crossref{2Chr}{34}{13}{2Ch 2:10,\allowbreak18; 8:10 Ne 4:10}
\crossref{2Chr}{34}{14}{2Ki 22:8-\allowbreak20 De 31:24-\allowbreak26}
\crossref{2Chr}{34}{15}{Jud 18:14 1Sa 9:17}
\crossref{2Chr}{34}{16}{2Ki 22:9,\allowbreak10 Jer 36:20,\allowbreak21}
\crossref{2Chr}{34}{17}{34:8-\allowbreak10}
\crossref{2Chr}{34}{18}{De 17:19 Jos 1:8 Ps 119:46,\allowbreak97-\allowbreak99 Jer 36:20,\allowbreak21}
\crossref{2Chr}{34}{19}{Ro 3:20; 7:7-\allowbreak11 Ga 2:19; 3:10-\allowbreak13}
\crossref{2Chr}{34}{20}{2Ki 25:22 Jer 26:24; 40:6,\allowbreak9,\allowbreak14}
\crossref{2Chr}{34}{21}{Ex 18:15 1Sa 9:9 1Ki 22:5-\allowbreak7 Jer 21:2 Eze 14:1-\allowbreak11; 20:1-\allowbreak7}
\crossref{2Chr}{34}{22}{Ex 15:20 Jud 4:4 Lu 1:41-\allowbreak45; 2:36 Ac 21:9}
\crossref{2Chr}{34}{23}{2Ki 22:15-\allowbreak20 Jer 21:3-\allowbreak7; 37:7-\allowbreak10}
\crossref{2Chr}{34}{24}{2Ch 36:14-\allowbreak20 Jos 23:16 2Ki 21:12; 23:26,\allowbreak27 Isa 5:4-\allowbreak6 Jer 6:19}
\crossref{2Chr}{34}{25}{2Ch 12:2; 15:2; 33:3-\allowbreak9 2Ki 24:3,\allowbreak4 Isa 2:8,\allowbreak9 Jer 15:1-\allowbreak4}
\crossref{2Chr}{34}{26}{34:21,\allowbreak23}
\crossref{2Chr}{34}{27}{2Ch 32:12,\allowbreak13 2Ki 22:18,\allowbreak19 Ps 34:18; 51:17 Isa 57:15; 66:2 Eze 9:4}
\crossref{2Chr}{34}{28}{2Ch 35:24}
\crossref{2Chr}{34}{29}{1Sa 12:23 1Ch 29:2-\allowbreak9 Mr 14:8}
\crossref{2Chr}{34}{30}{2Ch 15:12,\allowbreak13; 18:30 De 1:17 Job 3:19}
\crossref{2Chr}{34}{31}{}
\crossref{2Chr}{34}{32}{2Ch 14:4; 30:12; 33:16 Ge 18:19 Ec 8:2}
\crossref{2Chr}{34}{33}{34:3-\allowbreak7 2Ki 23:4-\allowbreak20}
\crossref{2Chr}{35}{1}{Ex 12:6 Nu 9:3 De 16:1-\allowbreak8 Ezr 6:19 Eze 45:21}
\crossref{2Chr}{35}{2}{2Ch 23:8,\allowbreak18; 31:2 Nu 18:5-\allowbreak7 1Ch 24:1-\allowbreak31 Ezr 6:18}
\crossref{2Chr}{35}{3}{2Ch 17:8,\allowbreak9; 30:22 De 33:10 Ne 8:7,\allowbreak8 Mal 2:7}
\crossref{2Chr}{35}{4}{1Ch 9:10-\allowbreak34 Ne 11:10-\allowbreak20}
\crossref{2Chr}{35}{5}{Ps 134:1; 135:2}
\crossref{2Chr}{35}{6}{2Ch 30:15-\allowbreak17 Ex 12:6,\allowbreak21,\allowbreak22 Ezr 6:20,\allowbreak21}
\crossref{2Chr}{35}{7}{2Ch 7:8-\allowbreak10; 30:24 Isa 32:8 Eze 45:17}
\crossref{2Chr}{35}{8}{2Ch 29:31-\allowbreak33 1Ch 29:6-\allowbreak9,\allowbreak17 Ezr 1:6; 2:68,\allowbreak69; 7:16; 8:25-\allowbreak35}
\crossref{2Chr}{35}{9}{Isa 1:10-\allowbreak15 Jer 3:10; 7:21-\allowbreak28 Mic 6:6-\allowbreak8}
\crossref{2Chr}{35}{10}{35:4,\allowbreak5; 30:16 Ezr 6:18}
\crossref{2Chr}{35}{11}{2Ch 29:22-\allowbreak24; 30:16 Le 1:5,\allowbreak6 Nu 18:3,\allowbreak7 Heb 9:21,\allowbreak22}
\crossref{2Chr}{35}{12}{Le 3:3,\allowbreak5,\allowbreak9-\allowbreak11,\allowbreak14-\allowbreak16}
\crossref{2Chr}{35}{13}{Ex 12:8,\allowbreak9 De 16:7 Ps 22:14 La 1:12,\allowbreak13}
\crossref{2Chr}{35}{14}{Ac 6:2-\allowbreak4}
\crossref{2Chr}{35}{15}{2Ch 29:25,\allowbreak26 1Ch 16:41,\allowbreak42; 23:5; 25:1-\allowbreak7 Ps 77:1; 78:1; 79:1}
\crossref{2Chr}{35}{16}{}
\crossref{2Chr}{35}{17}{2Ch 30:21-\allowbreak23 Ex 12:15-\allowbreak20; 13:6,\allowbreak7; 23:15; 34:18 Le 23:5-\allowbreak8 Nu 28:16-\allowbreak25}
\crossref{2Chr}{35}{18}{2Ch 30:26,\allowbreak27}
\crossref{2Chr}{35}{19}{}
\crossref{2Chr}{35}{20}{Jer 46:2-\allowbreak12}
\crossref{2Chr}{35}{21}{2Sa 16:10 Mt 8:29 Joh 2:4}
\crossref{2Chr}{35}{22}{2Ch 18:29 1Ki 14:2; 22:30,\allowbreak34}
\crossref{2Chr}{35}{23}{2Ch 18:33 Ge 49:23 2Ki 9:24 La 3:13}
\crossref{2Chr}{35}{24}{Ge 41:43}
\crossref{2Chr}{35}{25}{Jer 22:10 La 4:20}
\crossref{2Chr}{35}{26}{2Ch 31:20; 32:32}
\crossref{2Chr}{35}{27}{2Ch 20:34; 24:27; 25:26; 26:22; 32:32; 33:19 2Ki 10:34; 16:19; 20:20}
\crossref{2Chr}{36}{1}{2Ch 26:1; 33:25 2Ki 23:30-\allowbreak37}
\crossref{2Chr}{36}{2}{}
\crossref{2Chr}{36}{3}{2Ki 23:33}
\crossref{2Chr}{36}{4}{2Ki 23:34,\allowbreak35 1Ch 3:15}
\crossref{2Chr}{36}{5}{2Ki 23:36,\allowbreak37 Jer 22:13-\allowbreak19; 26:21-\allowbreak23; 36:1,\allowbreak27-\allowbreak32}
\crossref{2Chr}{36}{6}{2Ki 24:1,\allowbreak2,\allowbreak5,\allowbreak6,\allowbreak13-\allowbreak20 Eze 19:5-\allowbreak9 Da 1:1,\allowbreak2 Hab 1:5-\allowbreak10}
\crossref{2Chr}{36}{7}{2Ki 24:13 Ezr 1:7-\allowbreak11 Jer 27:16-\allowbreak18; 28:3 Da 5:2-\allowbreak4}
\crossref{2Chr}{36}{8}{2Ki 24:5,\allowbreak6}
\crossref{2Chr}{36}{9}{}
\crossref{2Chr}{36}{10}{36:7 Jer 27:18-\allowbreak22 Da 1:1,\allowbreak2; 5:2,\allowbreak23}
\crossref{2Chr}{36}{11}{2Ki 24:18-\allowbreak20 Jer 52:1-\allowbreak3}
\crossref{2Chr}{36}{12}{2Ch 32:26; 33:12,\allowbreak19,\allowbreak23 Ex 10:3 Da 5:22,\allowbreak23 Jas 4:10 1Pe 5:6}
\crossref{2Chr}{36}{13}{2Ki 24:20 Jer 52:2,\allowbreak3 Eze 17:11-\allowbreak20}
\crossref{2Chr}{36}{14}{2Ki 16:10-\allowbreak16 Ezr 9:7 Jer 5:5; 37:13-\allowbreak15; 38:4 Eze 22:6,\allowbreak26-\allowbreak28}
\crossref{2Chr}{36}{15}{2Ch 24:18-\allowbreak21; 33:10 2Ki 17:13 Jer 25:3,\allowbreak4; 26:5; 35:15; 44:4,\allowbreak5}
\crossref{2Chr}{36}{16}{2Ch 30:10 Ps 35:16 Isa 28:22 Jer 5:12,\allowbreak13; 20:7 Lu 18:32; 22:63,\allowbreak64}
\crossref{2Chr}{36}{17}{2Ch 33:11 De 28:49 2Ki 24:2,\allowbreak3 Ezr 9:7 Jer 15:8; 32:42; 40:3 Da 9:14}
\crossref{2Chr}{36}{18}{36:7,\allowbreak10 2Ki 25:13-\allowbreak17 Jer 27:18-\allowbreak22; 52:17-\allowbreak23 Da 5:3}
\crossref{2Chr}{36}{19}{2Ki 25:9 Ps 74:4-\allowbreak8; 79:1,\allowbreak7 Isa 64:10,\allowbreak11 Jer 7:4,\allowbreak14; 52:13}
\crossref{2Chr}{36}{20}{36:22 Ezr 1:1-\allowbreak11}
\crossref{2Chr}{36}{21}{Jer 25:9,\allowbreak12; 26:6,\allowbreak7; 27:12,\allowbreak13; 29:10 Da 9:2 Zec 1:4-\allowbreak6}
\crossref{2Chr}{36}{22}{Ezr 1:1-\allowbreak3}
\crossref{2Chr}{36}{23}{Ps 75:5-\allowbreak7 Da 2:21,\allowbreak37; 4:35; 5:18,\allowbreak23}

% Ezra
\crossref{Ezra}{1}{1}{2Ch 36:22,\allowbreak23}
\crossref{Ezra}{1}{2}{1Ki 8:27 2Ch 2:12 Isa 66:1 Jer 10:11 Da 2:21,\allowbreak28; 5:23}
\crossref{Ezra}{1}{3}{Jos 1:9 1Ch 28:20 Mt 28:20}
\crossref{Ezra}{1}{4}{Ezr 7:16-\allowbreak18 Ac 24:17 3Jo 1:6-\allowbreak8}
\crossref{Ezra}{1}{5}{1:1 2Ch 36:22 Ne 2:12 Pr 16:1 2Co 8:16 Php 2:13 Jas 1:16,\allowbreak17}
\crossref{Ezra}{1}{6}{Ezr 7:15,\allowbreak16; 8:25-\allowbreak28,\allowbreak33}
\crossref{Ezra}{1}{7}{Ezr 5:14; 6:5}
\crossref{Ezra}{1}{8}{1:11; 5:14,\allowbreak16 Hag 1:1,\allowbreak14; 2:2-\allowbreak4 Zec 4:6-\allowbreak10}
\crossref{Ezra}{1}{9}{Nu 7:13,\allowbreak19-\allowbreak89 1Ki 7:50 2Ch 4:8,\allowbreak11,\allowbreak21,\allowbreak22; 24:14 Mt 14:8}
\crossref{Ezra}{1}{10}{}
\crossref{Ezra}{1}{11}{Ro 9:23 2Ti 2:19-\allowbreak21}
\crossref{Ezra}{2}{1}{Ezr 5:8; 6:2 Ne 7:6-\allowbreak73 Es 1:1,\allowbreak3,\allowbreak8,\allowbreak11; 8:9 Ac 23:34}
\crossref{Ezra}{2}{2}{Ezr 1:11}
\crossref{Ezra}{2}{3}{Ezr 8:3}
\crossref{Ezra}{2}{4}{Ezr 8:8 Ne 7:9}
\crossref{Ezra}{2}{5}{Ne 6:18; 7:10}
\crossref{Ezra}{2}{6}{Ezr 8:4; 10:30 Ne 7:11}
\crossref{Ezra}{2}{7}{2:31; 8:7; 10:26 Ne 7:12}
\crossref{Ezra}{2}{8}{Ezr 10:27 Ne 7:13}
\crossref{Ezra}{2}{9}{Ne 7:14}
\crossref{Ezra}{2}{10}{}
\crossref{Ezra}{2}{11}{Ezr 8:11; 10:28 Ne 7:16}
\crossref{Ezra}{2}{12}{Ezr 8:12 Ne 7:17}
\crossref{Ezra}{2}{13}{Ezr 8:13 Ne 7:18}
\crossref{Ezra}{2}{14}{Ezr 8:14 Ne 7:19}
\crossref{Ezra}{2}{15}{Ezr 8:6 Ne 7:20}
\crossref{Ezra}{2}{16}{Ne 7:21}
\crossref{Ezra}{2}{17}{Ne 7:23}
\crossref{Ezra}{2}{18}{Ne 7:24}
\crossref{Ezra}{2}{19}{Ezr 10:33 Ne 7:22}
\crossref{Ezra}{2}{20}{Ne 7:25}
\crossref{Ezra}{2}{21}{1Ch 2:50-\allowbreak52}
\crossref{Ezra}{2}{22}{2Sa 23:28 1Ch 2:54 Ne 7:26}
\crossref{Ezra}{2}{23}{Jos 21:18 Ne 7:27 Isa 10:30 Jer 1:1; 11:21}
\crossref{Ezra}{2}{24}{Ne 7:28}
\crossref{Ezra}{2}{25}{Jos 9:17 Ne 7:29}
\crossref{Ezra}{2}{26}{Jos 18:24,\allowbreak25 Ne 7:30}
\crossref{Ezra}{2}{27}{1Sa 13:5,\allowbreak23 Isa 10:28}
\crossref{Ezra}{2}{28}{Ge 12:8}
\crossref{Ezra}{2}{29}{}
\crossref{Ezra}{2}{30}{}
\crossref{Ezra}{2}{31}{2:7 Ne 7:34}
\crossref{Ezra}{2}{32}{Ezr 10:31 Ne 7:35}
\crossref{Ezra}{2}{33}{1Ch 8:12 Ne 6:2; 7:37; 11:34,\allowbreak35}
\crossref{Ezra}{2}{34}{1Ki 16:34 Ne 7:36}
\crossref{Ezra}{2}{35}{Ne 7:38}
\crossref{Ezra}{2}{36}{1Ch 9:10; 24:7}
\crossref{Ezra}{2}{37}{Ezr 10:20 1Ch 24:14 Ne 7:40}
\crossref{Ezra}{2}{38}{Ezr 10:22 1Ch 9:12 Ne 7:41}
\crossref{Ezra}{2}{39}{Ezr 10:21 1Ch 24:8 Ne 7:42}
\crossref{Ezra}{2}{40}{Ezr 3:9}
\crossref{Ezra}{2}{41}{1Ch 6:39; 15:17; 25:1,\allowbreak2 Ne 7:44; 11:17}
\crossref{Ezra}{2}{42}{1Ch 26:1-\allowbreak19 Ne 7:45}
\crossref{Ezra}{2}{43}{2:58 1Ch 9:2 Ne 7:46-\allowbreak56; 10:28}
\crossref{Ezra}{2}{44}{}
\crossref{Ezra}{2}{45}{}
\crossref{Ezra}{2}{46}{}
\crossref{Ezra}{2}{47}{1Ch 4:2}
\crossref{Ezra}{2}{48}{2Ki 15:37}
\crossref{Ezra}{2}{49}{Ne 7:51}
\crossref{Ezra}{2}{50}{Ne 7:52}
\crossref{Ezra}{2}{51}{2:51 Ne 7:53}
\crossref{Ezra}{2}{52}{Ne 7:54}
\crossref{Ezra}{2}{53}{Ne 7:55}
\crossref{Ezra}{2}{54}{}
\crossref{Ezra}{2}{55}{1Ki 9:21}
\crossref{Ezra}{2}{56}{Ne 7:58}
\crossref{Ezra}{2}{57}{Ne 7:59}
\crossref{Ezra}{2}{58}{Ezr 7:7 Jos 9:21,\allowbreak23,\allowbreak27 1Ch 9:2 Ne 3:26; 7:60}
\crossref{Ezra}{2}{59}{Ne 7:61}
\crossref{Ezra}{2}{60}{Ne 7:62}
\crossref{Ezra}{2}{61}{Ne 7:63,\allowbreak64}
\crossref{Ezra}{2}{62}{Le 21:21-\allowbreak23 Nu 3:10; 16:40; 18:7}
\crossref{Ezra}{2}{63}{Le 2:3,\allowbreak10; 6:17,\allowbreak29; 7:16; 10:17,\allowbreak18; 22:2,\allowbreak3,\allowbreak10,\allowbreak14-\allowbreak16 Nu 18:9-\allowbreak11,\allowbreak19}
\crossref{Ezra}{2}{64}{}
\crossref{Ezra}{2}{65}{Isa 14:1,\allowbreak2}
\crossref{Ezra}{2}{66}{2:66}
\crossref{Ezra}{2}{67}{}
\crossref{Ezra}{2}{68}{Ex 35:5-\allowbreak19,\allowbreak29; 36:3 Nu 7:3-\allowbreak89 1Ch 29:5-\allowbreak17 Ne 7:70-\allowbreak73 Ps 110:3}
\crossref{Ezra}{2}{69}{Ezr 8:25-\allowbreak34 1Ki 7:51 1Ch 22:14; 26:20-\allowbreak28 Ne 7:71,\allowbreak72}
\crossref{Ezra}{2}{70}{Ezr 6:16,\allowbreak17 1Ch 11:2 Ne 7:73; 11:3-\allowbreak36}
\crossref{Ezra}{3}{1}{Ex 23:14-\allowbreak17 Le 16:29; 23:24,\allowbreak27-\allowbreak44 Nu 29:1-\allowbreak40 Ne 8:2,\allowbreak14}
\crossref{Ezra}{3}{2}{Hag 1:1,\allowbreak12,\allowbreak14; 2:2-\allowbreak4 Zec 3:1,\allowbreak8; 6:11}
\crossref{Ezra}{3}{3}{2Ch 4:1}
\crossref{Ezra}{3}{4}{Ex 23:16 Le 23:34-\allowbreak36 Ne 8:14-\allowbreak17 Zec 14:16-\allowbreak19 Joh 7:2,\allowbreak37}
\crossref{Ezra}{3}{5}{Ex 29:38-\allowbreak42 Nu 28:3-\allowbreak10,\allowbreak11,\allowbreak19,\allowbreak27; 29:2,\allowbreak8,\allowbreak13}
\crossref{Ezra}{3}{6}{Le 23:23-\allowbreak25 Ne 8:18}
\crossref{Ezra}{3}{7}{2Ki 12:11,\allowbreak12; 22:5,\allowbreak6 2Ch 24:12,\allowbreak13}
\crossref{Ezra}{3}{8}{3:2}
\crossref{Ezra}{3}{9}{}
\crossref{Ezra}{3}{10}{Zec 4:10}
\crossref{Ezra}{3}{11}{Ex 15:21 Ne 12:24,\allowbreak40 Ps 24:7-\allowbreak10 Isa 6:3}
\crossref{Ezra}{3}{12}{Hag 2:3}
\crossref{Ezra}{3}{13}{Jud 2:5}
\crossref{Ezra}{4}{1}{Ezr 1:11}
\crossref{Ezra}{4}{2}{Ezr 1:5; 2:2; 3:2,\allowbreak12}
\crossref{Ezra}{4}{3}{Ne 2:20 Joh 4:22,\allowbreak23 Ac 8:21 Ro 9:4,\allowbreak5 3Jo 1:9,\allowbreak10}
\crossref{Ezra}{4}{4}{Ezr 3:3 Ne 6:9 Isa 35:3,\allowbreak4 Jer 38:4}
\crossref{Ezra}{4}{5}{Ps 2:1,\allowbreak2 Na 1:11 Ac 24:1-\allowbreak27}
\crossref{Ezra}{4}{6}{Mt 27:37 Ac 24:5-\allowbreak9,\allowbreak13; 25:7 Re 12:10}
\crossref{Ezra}{4}{7}{4:9,\allowbreak17; 5:6}
\crossref{Ezra}{4}{8}{4:9 2Sa 8:17; 20:25 2Ki 18:18}
\crossref{Ezra}{4}{9}{2Ki 17:24,\allowbreak30,\allowbreak31}
\crossref{Ezra}{4}{10}{4:1 2Ki 17:24-\allowbreak41}
\crossref{Ezra}{4}{11}{4:15,\allowbreak19 2Ki 18:20; 24:1 2Ch 36:13 Jer 52:3 Eze 17:12-\allowbreak21}
\crossref{Ezra}{4}{12}{4:15,\allowbreak19 2Ki 18:20; 24:1 2Ch 36:13 Jer 52:3 Eze 17:12-\allowbreak21}
\crossref{Ezra}{4}{13}{Ne 5:4 Ps 52:2; 119:69}
\crossref{Ezra}{4}{14}{Eze 33:31 Joh 12:5,\allowbreak6; 19:12-\allowbreak15}
\crossref{Ezra}{4}{15}{4:12 Ne 2:19; 6:6 Es 3:5-\allowbreak8 Da 6:4-\allowbreak13 Ac 17:6,\allowbreak7}
\crossref{Ezra}{4}{16}{4:20 2Sa 8:3 1Ki 4:24}
\crossref{Ezra}{4}{17}{4:7,\allowbreak9}
\crossref{Ezra}{4}{18}{Le 24:12}
\crossref{Ezra}{4}{19}{4:15; 5:17; 6:1,\allowbreak2 De 13:14 Pr 25:2}
\crossref{Ezra}{4}{20}{1Ki 4:21,\allowbreak24 1Ch 18:3 Ps 72:8}
\crossref{Ezra}{4}{21}{4:19}
\crossref{Ezra}{4}{22}{4:13 Es 3:8,\allowbreak9; 7:3,\allowbreak4}
\crossref{Ezra}{4}{23}{4:8,\allowbreak9,\allowbreak17}
\crossref{Ezra}{4}{24}{Ne 6:3,\allowbreak9 Job 20:5 1Th 2:18}
\crossref{Ezra}{5}{1}{Zec 1:1-\allowbreak21}
\crossref{Ezra}{5}{2}{Ezr 3:2 Hag 1:12-\allowbreak15}
\crossref{Ezra}{5}{3}{5:9; 1:3 Mt 21:23 Ac 4:7}
\crossref{Ezra}{5}{4}{5:10}
\crossref{Ezra}{5}{5}{Ezr 7:6,\allowbreak28; 8:22 2Ch 16:9 Ps 32:8; 33:18; 34:15; 76:10 Php 1:28}
\crossref{Ezra}{5}{6}{Ezr 4:11,\allowbreak23}
\crossref{Ezra}{5}{7}{Ezr 4:17 Da 3:9; 4:1; 6:21 Joh 14:27 2Th 3:16}
\crossref{Ezra}{5}{8}{Ezr 2:1 Ne 7:6; 11:3 Es 1:1,\allowbreak22}
\crossref{Ezra}{5}{9}{5:3,\allowbreak4}
\crossref{Ezra}{5}{10}{5:4}
\crossref{Ezra}{5}{11}{Jos 24:15 Ps 119:46 Da 3:26 Jon 1:9 Mt 10:32 Lu 12:8 Ac 27:23}
\crossref{Ezra}{5}{12}{2Ki 21:12-\allowbreak15 2Ch 34:24,\allowbreak25; 36:16,\allowbreak17 Ne 9:26,\allowbreak27 Isa 59:1,\allowbreak2}
\crossref{Ezra}{5}{13}{Ezr 1:1-\allowbreak8; 6:3-\allowbreak5 Isa 44:28; 45:1}
\crossref{Ezra}{5}{14}{Ezr 1:7-\allowbreak10; 6:5 2Ch 36:7,\allowbreak18 Jer 52:19 Da 5:2,\allowbreak3}
\crossref{Ezra}{5}{15}{Ezr 1:2; 3:3; 6:3}
\crossref{Ezra}{5}{16}{5:14}
\crossref{Ezra}{5}{17}{Ezr 4:15,\allowbreak19; 6:1,\allowbreak2 Pr 25:2}
\crossref{Ezra}{6}{1}{Ezr 4:15,\allowbreak19; 5:17 Job 29:16 Pr 25:2}
\crossref{Ezra}{6}{2}{}
\crossref{Ezra}{6}{3}{Ezr 1:1-\allowbreak4; 5:13-\allowbreak15 2Ch 36:22,\allowbreak23}
\crossref{Ezra}{6}{4}{1Ki 6:36}
\crossref{Ezra}{6}{5}{Ezr 1:7,\allowbreak8; 5:14 Jer 27:16,\allowbreak18-\allowbreak22 Da 1:2; 5:2}
\crossref{Ezra}{6}{6}{Ezr 5:3}
\crossref{Ezra}{6}{7}{Ac 5:38,\allowbreak39}
\crossref{Ezra}{6}{8}{6:4; 4:16,\allowbreak20; 7:15-\allowbreak22 Ps 68:29-\allowbreak31 Hag 2:8}
\crossref{Ezra}{6}{9}{Le 1:3-\allowbreak5,\allowbreak10; 9:2 Ps 50:9-\allowbreak13}
\crossref{Ezra}{6}{10}{Ge 8:21 Le 1:9,\allowbreak13 Eph 5:2}
\crossref{Ezra}{6}{11}{Ezr 7:26}
\crossref{Ezra}{6}{12}{Ex 20:24 De 12:5,\allowbreak11; 16:2 1Ki 9:3 2Ch 7:16 Ps 132:13,\allowbreak14}
\crossref{Ezra}{6}{13}{Ezr 4:9,\allowbreak23; 5:6}
\crossref{Ezra}{6}{14}{Ezr 3:8; 4:3}
\crossref{Ezra}{6}{15}{Es 3:7,\allowbreak13; 8:12; 9:1,\allowbreak15,\allowbreak17,\allowbreak19,\allowbreak21}
\crossref{Ezra}{6}{16}{1Ch 9:2 Ne 7:73}
\crossref{Ezra}{6}{17}{Ezr 8:35 Nu 7:2-\allowbreak89 1Ki 8:63,\allowbreak64 1Ch 16:1-\allowbreak3 2Ch 7:5; 29:31-\allowbreak35}
\crossref{Ezra}{6}{18}{1Ch 23:1-\allowbreak26:32 2Ch 35:4,\allowbreak5}
\crossref{Ezra}{6}{19}{6:16}
\crossref{Ezra}{6}{20}{2Ch 29:34; 30:15-\allowbreak17}
\crossref{Ezra}{6}{21}{Ezr 9:11 Nu 9:6,\allowbreak7,\allowbreak10-\allowbreak14 Isa 52:11 Eze 36:25 2Co 6:17; 7:1}
\crossref{Ezra}{6}{22}{Ex 12:15-\allowbreak20; 13:6,\allowbreak7 2Ch 30:21; 35:17 Mt 26:17 1Co 5:7,\allowbreak8}
\crossref{Ezra}{7}{1}{7:12,\allowbreak21; 6:14 Ne 2:1}
\crossref{Ezra}{7}{2}{2Sa 8:17 1Ki 2:35}
\crossref{Ezra}{7}{3}{7:1}
\crossref{Ezra}{7}{4}{1Ch 6:5}
\crossref{Ezra}{7}{5}{Ex 6:25 Nu 25:7-\allowbreak13; 31:6 Jos 22:13,\allowbreak31 Jud 20:28 1Ch 6:4,\allowbreak50-\allowbreak52}
\crossref{Ezra}{7}{6}{Ne 8:4,\allowbreak9,\allowbreak13; 12:26,\allowbreak36 Jer 8:8 1Co 1:20}
\crossref{Ezra}{7}{7}{Ezr 8:1-\allowbreak14}
\crossref{Ezra}{7}{8}{}
\crossref{Ezra}{7}{9}{}
\crossref{Ezra}{7}{10}{1Sa 7:3 1Ch 29:18 2Ch 12:14; 19:3 Job 11:13 Ps 10:17; 57:7}
\crossref{Ezra}{7}{11}{Ezr 4:11; 5:6}
\crossref{Ezra}{7}{12}{}
\crossref{Ezra}{7}{13}{Ezr 5:13; 6:1 2Ch 30:5 Es 3:15; 9:14 Ps 148:6}
\crossref{Ezra}{7}{14}{7:25,\allowbreak26 De 17:18,\allowbreak19 Isa 8:20}
\crossref{Ezra}{7}{15}{Ezr 6:4,\allowbreak8-\allowbreak10 Ps 68:29,\allowbreak30; 72:10; 76:11 Isa 60:6-\allowbreak9 Re 21:24-\allowbreak26}
\crossref{Ezra}{7}{16}{Ezr 8:25-\allowbreak28}
\crossref{Ezra}{7}{17}{Ezr 6:9,\allowbreak10 De 14:24-\allowbreak26 Mt 21:12,\allowbreak13 Joh 2:14}
\crossref{Ezra}{7}{18}{2Ki 12:15; 22:7}
\crossref{Ezra}{7}{19}{Ezr 8:27-\allowbreak30,\allowbreak33,\allowbreak34}
\crossref{Ezra}{7}{20}{Ezr 6:4,\allowbreak8-\allowbreak18}
\crossref{Ezra}{7}{21}{7:12,\allowbreak13}
\crossref{Ezra}{7}{22}{Lu 16:7}
\crossref{Ezra}{7}{23}{7:13,\allowbreak18}
\crossref{Ezra}{7}{24}{7:7; 2:36-\allowbreak55}
\crossref{Ezra}{7}{25}{7:14 1Ki 3:28 1Ch 22:12 Ps 19:7; 119:98-\allowbreak100 Pr 2:6; 6:23 Jas 1:5}
\crossref{Ezra}{7}{26}{Ezr 6:11 Da 3:28,\allowbreak29; 6:26}
\crossref{Ezra}{7}{27}{Ezr 6:22 Ne 2:12; 7:5 2Co 8:16 Heb 8:10; 10:16 Jas 1:17 Re 17:17}
\crossref{Ezra}{7}{28}{Ezr 9:9 Ge 32:28; 43:14 Ne 1:11}
\crossref{Ezra}{8}{1}{Ezr 1:5 1Ch 9:34; 24:31; 26:32 2Ch 26:12 Ne 7:70,\allowbreak71}
\crossref{Ezra}{8}{2}{1Ch 6:3,\allowbreak4-\allowbreak15; 24:1-\allowbreak6}
\crossref{Ezra}{8}{3}{}
\crossref{Ezra}{8}{4}{Ezr 2:6 Ne 7:11; 10:14}
\crossref{Ezra}{8}{5}{8:3}
\crossref{Ezra}{8}{6}{Ezr 2:15 Ne 7:20; 10:16}
\crossref{Ezra}{8}{7}{Ezr 2:7,\allowbreak31 Ne 7:12,\allowbreak34}
\crossref{Ezra}{8}{8}{Ezr 2:4 Ne 7:9; 11:4}
\crossref{Ezra}{8}{9}{Ezr 2:6 Ne 7:11}
\crossref{Ezra}{8}{10}{Le 24:11}
\crossref{Ezra}{8}{11}{Ezr 2:11; 10:28 Ne 7:16}
\crossref{Ezra}{8}{12}{Ezr 2:12 Ne 7:17}
\crossref{Ezra}{8}{13}{Ezr 2:13 Ne 7:18}
\crossref{Ezra}{8}{14}{Ezr 2:14 Ne 7:19}
\crossref{Ezra}{8}{15}{8:21,\allowbreak31}
\crossref{Ezra}{8}{16}{8:13; 10:21}
\crossref{Ezra}{8}{17}{Ex 4:15 De 18:18 2Sa 14:3,\allowbreak19 Jer 1:9; 15:19}
\crossref{Ezra}{8}{18}{8:22; 7:28 Ne 2:8 Pr 3:6}
\crossref{Ezra}{8}{19}{Ne 3:17; 10:11}
\crossref{Ezra}{8}{20}{8:17; 2:43; 7:7 1Ch 9:2}
\crossref{Ezra}{8}{21}{Jud 20:26 1Sa 7:6 2Ch 20:3 Joe 1:14; 2:12-\allowbreak18 Jon 3:5}
\crossref{Ezra}{8}{22}{1Co 9:15 2Co 7:14}
\crossref{Ezra}{8}{23}{Ne 9:1 Es 4:16 Da 9:3 Lu 2:37 Ac 10:30}
\crossref{Ezra}{8}{24}{8:18,\allowbreak19}
\crossref{Ezra}{8}{25}{8:33; 1:8 2Co 8:20,\allowbreak21 Php 4:8}
\crossref{Ezra}{8}{26}{}
\crossref{Ezra}{8}{27}{La 4:2}
\crossref{Ezra}{8}{28}{Le 21:6-\allowbreak8 De 33:8 Isa 52:11}
\crossref{Ezra}{8}{29}{1Ch 26:20-\allowbreak26 Mr 13:34,\allowbreak35 Ac 20:31 2Ti 4:5}
\crossref{Ezra}{8}{30}{8:22 1Ch 29:2,\allowbreak3 Ps 122:9 Isa 60:13}
\crossref{Ezra}{8}{31}{8:15,\allowbreak21}
\crossref{Ezra}{8}{32}{Ezr 7:8,\allowbreak9 Ne 2:11}
\crossref{Ezra}{8}{33}{8:26,\allowbreak30 1Ch 28:14-\allowbreak18 2Co 8:20,\allowbreak21}
\crossref{Ezra}{8}{34}{}
\crossref{Ezra}{8}{35}{Le 1:1-\allowbreak7:38 2Ch 29:31,\allowbreak32 Ps 66:10-\allowbreak15; 116:12-\allowbreak19 Lu 1:74,\allowbreak75}
\crossref{Ezra}{8}{36}{Ezr 7:21-\allowbreak24}
\crossref{Ezra}{9}{1}{Ezr 10:8 Jer 26:10,\allowbreak16}
\crossref{Ezra}{9}{2}{Ezr 10:18-\allowbreak44 Ex 34:16 De 7:1-\allowbreak4 Ne 13:23,\allowbreak24 Mal 2:11}
\crossref{Ezra}{9}{3}{Jos 7:6 2Ki 18:37; 19:1 Job 1:20 Jer 36:24}
\crossref{Ezra}{9}{4}{Ezr 10:3 2Ch 34:27 Ps 119:136 Isa 66:2 Eze 9:4}
\crossref{Ezra}{9}{5}{2Ch 6:13 Ps 95:6 Lu 22:41 Ac 21:5 Eph 3:14}
\crossref{Ezra}{9}{6}{Job 40:4; 42:6 Jer 3:3,\allowbreak24,\allowbreak25; 6:15; 8:12; 31:19 Eze 16:63}
\crossref{Ezra}{9}{7}{Nu 32:14 2Ch 29:6; 30:7 Ne 9:32-\allowbreak34 Ps 106:6,\allowbreak7 La 5:7 Da 9:5-\allowbreak8}
\crossref{Ezra}{9}{8}{9:9 Ne 1:11; 9:31 Hab 3:2}
\crossref{Ezra}{9}{9}{Ne 9:36,\allowbreak37}
\crossref{Ezra}{9}{10}{Ge 44:16 Jos 7:8 La 3:22 Da 9:4-\allowbreak16 Ro 3:19}
\crossref{Ezra}{9}{11}{Ezr 6:21 Eze 36:25-\allowbreak27 2Co 7:1}
\crossref{Ezra}{9}{12}{Ex 23:32; 34:16 De 7:3 Jos 23:12,\allowbreak13}
\crossref{Ezra}{9}{13}{Ne 9:32 Eze 24:13,\allowbreak14 Ga 3:4}
\crossref{Ezra}{9}{14}{Joh 5:14 Ro 6:1 2Pe 2:20,\allowbreak21}
\crossref{Ezra}{9}{15}{Ne 9:33,\allowbreak34 Da 9:7-\allowbreak11,\allowbreak14 Ro 10:3}
\crossref{Ezra}{10}{1}{Da 9:3,\allowbreak4,\allowbreak20 Ac 10:30}
\crossref{Ezra}{10}{2}{10:26 Ne 3:29}
\crossref{Ezra}{10}{3}{2Ch 30:12}
\crossref{Ezra}{10}{4}{Jos 7:10-\allowbreak26 1Ch 22:16,\allowbreak19 Ec 9:10}
\crossref{Ezra}{10}{5}{Pr 1:5; 9:9; 15:23; 25:11,\allowbreak12; 27:9}
\crossref{Ezra}{10}{6}{Ne 13:5}
\crossref{Ezra}{10}{7}{Ezr 1:1 2Ch 30:5}
\crossref{Ezra}{10}{8}{Ezr 7:26 Jud 21:5 1Sa 11:7}
\crossref{Ezra}{10}{9}{1Sa 12:17,\allowbreak18 Jer 10:10,\allowbreak13}
\crossref{Ezra}{10}{10}{}
\crossref{Ezra}{10}{11}{Le 26:40-\allowbreak42 Jos 7:19 Ps 32:5 Pr 28:13 Jer 3:13 1Jo 1:7-\allowbreak9}
\crossref{Ezra}{10}{12}{10:3,\allowbreak4 Ne 13:23 Ps 78:37,\allowbreak57}
\crossref{Ezra}{10}{13}{10:18-\allowbreak44 Mt 7:13,\allowbreak14}
\crossref{Ezra}{10}{14}{De 17:9,\allowbreak18,\allowbreak19 2Ch 19:5-\allowbreak7}
\crossref{Ezra}{10}{15}{Ne 3:6; 10:20; 12:33}
\crossref{Ezra}{10}{16}{De 13:14 Job 29:16 Joh 7:51}
\crossref{Ezra}{10}{17}{}
\crossref{Ezra}{10}{18}{Ezr 9:1 Le 21:7,\allowbreak13-\allowbreak15 1Sa 2:22-\allowbreak24 Ne 13:28 Jer 23:11,\allowbreak14 Eze 44:22}
\crossref{Ezra}{10}{19}{Le 5:15,\allowbreak16; 6:4,\allowbreak6}
\crossref{Ezra}{10}{20}{Ezr 2:37 1Ch 24:14 Ne 7:40}
\crossref{Ezra}{10}{21}{Ezr 2:39 1Ch 24:8 Ne 7:42}
\crossref{Ezra}{10}{22}{Ezr 2:38 1Ch 9:12 Ne 7:41}
\crossref{Ezra}{10}{23}{Ezr 8:33 Ne 11:16}
\crossref{Ezra}{10}{24}{2Ch 35:15}
\crossref{Ezra}{10}{25}{Ezr 2:3 Ne 7:8}
\crossref{Ezra}{10}{26}{10:2; 2:7,\allowbreak31; 8:7 Ne 7:12,\allowbreak34}
\crossref{Ezra}{10}{27}{Ezr 2:8 Ne 7:13}
\crossref{Ezra}{10}{28}{Ezr 2:11; 8:11 Ne 7:16}
\crossref{Ezra}{10}{29}{Ezr 2:10 Ne 7:15}
\crossref{Ezra}{10}{30}{Ezr 2:6; 8:4 Ne 7:11}
\crossref{Ezra}{10}{31}{Ezr 2:32 Ne 7:35}
\crossref{Ezra}{10}{32}{}
\crossref{Ezra}{10}{33}{Ezr 2:19 Ne 7:22}
\crossref{Ezra}{10}{34}{10:29}
\crossref{Ezra}{10}{35}{1Ch 15:24}
\crossref{Ezra}{10}{36}{Ezr 8:33}
\crossref{Ezra}{10}{37}{10:30}
\crossref{Ezra}{10}{38}{10:38}
\crossref{Ezra}{10}{39}{}
\crossref{Ezra}{10}{40}{}
\crossref{Ezra}{10}{41}{1Ch 12:6}
\crossref{Ezra}{10}{42}{}
\crossref{Ezra}{10}{43}{Ezr 2:29 Ne 7:33}
\crossref{Ezra}{10}{44}{Pr 2:16; 5:3,\allowbreak20}

% Neh
\crossref{Neh}{1}{1}{Ne 10:1}
\crossref{Neh}{1}{2}{Ne 7:2}
\crossref{Neh}{1}{3}{Ne 7:6; 11:3 Ezr 2:1; 5:8 Es 1:1}
\crossref{Neh}{1}{4}{1Sa 4:17-\allowbreak22 Ezr 10:1 Ps 69:9,\allowbreak10; 102:13,\allowbreak14; 137:1 Da 9:3}
\crossref{Neh}{1}{5}{Ne 4:14 De 7:21 1Ch 17:21 Ps 47:2 Da 9:4-\allowbreak19}
\crossref{Neh}{1}{6}{1Ki 8:28,\allowbreak29 2Ch 6:40 Ps 34:15; 130:2 Da 9:17,\allowbreak18}
\crossref{Neh}{1}{7}{Ne 9:29-\allowbreak35 Ps 106:6 Da 9:5,\allowbreak6}
\crossref{Neh}{1}{8}{Ps 119:49 Lu 1:72}
\crossref{Neh}{1}{9}{Le 26:39-\allowbreak42 De 4:29-\allowbreak31; 30:2-\allowbreak5 Jer 29:11-\allowbreak14}
\crossref{Neh}{1}{10}{Ex 32:11 De 9:29 Isa 63:16-\allowbreak19; 64:9 Da 9:15-\allowbreak27}
\crossref{Neh}{1}{11}{1:6 Ps 86:6; 130:2}
\crossref{Neh}{2}{1}{Es 3:7}
\crossref{Neh}{2}{2}{Ge 40:7}
\crossref{Neh}{2}{3}{Ne 1:3 Ps 102:14; 137:6 La 2:9}
\crossref{Neh}{2}{4}{1Ki 3:5 Es 5:3,\allowbreak6; 7:2 Mr 10:51}
\crossref{Neh}{2}{5}{Ezr 5:17 Es 1:19; 5:8; 7:3; 8:5}
\crossref{Neh}{2}{6}{2:4; 1:11 Isa 58:12; 61:4; 65:24}
\crossref{Neh}{2}{7}{2:9 Ezr 6:6; 7:21}
\crossref{Neh}{2}{8}{2:17; 3:1-\allowbreak32}
\crossref{Neh}{2}{9}{2:7}
\crossref{Neh}{2}{10}{2:19; 4:1-\allowbreak3,\allowbreak7; 6:1}
\crossref{Neh}{2}{11}{Ezr 8:32}
\crossref{Neh}{2}{12}{Ge 32:22-\allowbreak24 Jos 10:9 Jud 6:27; 9:32 Mt 2:14}
\crossref{Neh}{2}{13}{2:15; 3:13 2Ch 26:9}
\crossref{Neh}{2}{14}{}
\crossref{Neh}{2}{15}{2Sa 15:23 Jer 31:38-\allowbreak40 Joh 18:1}
\crossref{Neh}{2}{16}{2:12}
\crossref{Neh}{2}{17}{La 2:2,\allowbreak8,\allowbreak9; 3:51}
\crossref{Neh}{2}{18}{2:8}
\crossref{Neh}{2}{19}{2:10; 6:1,\allowbreak2}
\crossref{Neh}{2}{20}{2:4 2Ch 26:5 Ps 20:5; 35:27; 102:13,\allowbreak14; 122:6 Ec 7:18}
\crossref{Neh}{3}{1}{Ne 12:10; 13:28}
\crossref{Neh}{3}{2}{Ne 7:36 Ezr 2:34}
\crossref{Neh}{3}{3}{Ne 12:39 2Ch 33:14 Zep 1:10}
\crossref{Neh}{3}{4}{3:21; 10:15}
\crossref{Neh}{3}{5}{3:27 2Sa 14:2 Am 1:1}
\crossref{Neh}{3}{6}{Ne 12:39}
\crossref{Neh}{3}{7}{Jos 9:3-\allowbreak27 2Sa 21:2}
\crossref{Neh}{3}{8}{3:31,\allowbreak32 Isa 46:6}
\crossref{Neh}{3}{9}{3:12,\allowbreak17}
\crossref{Neh}{3}{10}{3:23,\allowbreak28-\allowbreak30}
\crossref{Neh}{3}{11}{Ne 10:5}
\crossref{Neh}{3}{12}{3:9,\allowbreak14-\allowbreak18}
\crossref{Neh}{3}{13}{Ne 2:13}
\crossref{Neh}{3}{14}{Ne 2:13; 12:31}
\crossref{Neh}{3}{15}{Ne 2:14; 12:37 2Ch 32:30}
\crossref{Neh}{3}{16}{3:9,\allowbreak12,\allowbreak14}
\crossref{Neh}{3}{17}{3:16 1Ch 23:4}
\crossref{Neh}{3}{18}{Ezr 3:9}
\crossref{Neh}{3}{19}{Ne 10:9; 12:8}
\crossref{Neh}{3}{20}{Ec 9:10 Ro 12:11}
\crossref{Neh}{3}{21}{3:4}
\crossref{Neh}{3}{22}{Ne 6:2; 12:28}
\crossref{Neh}{3}{23}{3:10,\allowbreak29,\allowbreak30}
\crossref{Neh}{3}{24}{Ne 10:9}
\crossref{Neh}{3}{25}{Jer 22:14; 39:8}
\crossref{Neh}{3}{26}{Ne 7:46-\allowbreak56; 10:28 1Ch 9:2 Ezr 2:43-\allowbreak58}
\crossref{Neh}{3}{27}{3:5}
\crossref{Neh}{3}{28}{2Ki 11:16 2Ch 23:15 Jer 31:40}
\crossref{Neh}{3}{29}{Ne 7:40 Ezr 2:37}
\crossref{Neh}{3}{30}{3:21}
\crossref{Neh}{3}{31}{3:8,\allowbreak32}
\crossref{Neh}{3}{32}{}
\crossref{Neh}{4}{1}{Ne 2:10,\allowbreak19 Ezr 4:1-\allowbreak5 Ac 5:17}
\crossref{Neh}{4}{2}{Ezr 4:9,\allowbreak10}
\crossref{Neh}{4}{3}{Ne 2:10,\allowbreak19; 6:1 1Ki 20:10,\allowbreak18 2Ki 18:23}
\crossref{Neh}{4}{4}{Ps 123:3,\allowbreak4}
\crossref{Neh}{4}{5}{Ps 59:5-\allowbreak13; 69:27; 109:14 Jer 18:23 2Ti 4:14}
\crossref{Neh}{4}{6}{}
\crossref{Neh}{4}{7}{4:1; 2:10,\allowbreak19}
\crossref{Neh}{4}{8}{Ps 2:1-\allowbreak3; 83:3-\allowbreak11 Isa 8:9,\allowbreak10 Ac 23:12,\allowbreak13}
\crossref{Neh}{4}{9}{4:11 Ge 32:9-\allowbreak12,\allowbreak28 2Ki 19:14-\allowbreak19 Ps 50:15; 55:16-\allowbreak22 Lu 6:11,\allowbreak12}
\crossref{Neh}{4}{10}{Nu 13:31; 32:9 Ps 11:1,\allowbreak2 Hag 1:2}
\crossref{Neh}{4}{11}{Jud 20:29-\allowbreak48 2Sa 17:2 Ps 56:6 Isa 47:11 Ac 23:12,\allowbreak21 1Th 5:2}
\crossref{Neh}{4}{12}{Ge 31:7,\allowbreak41 Nu 14:22 Job 19:3}
\crossref{Neh}{4}{13}{Ge 32:13-\allowbreak20 2Ch 32:2-\allowbreak8 Ps 112:5 Mt 10:16 1Co 14:20}
\crossref{Neh}{4}{14}{Nu 14:9 De 1:21,\allowbreak29,\allowbreak30; 20:3,\allowbreak4 Jos 1:9 2Ch 20:15-\allowbreak17; 32:7}
\crossref{Neh}{4}{15}{2Sa 15:31; 17:14 Job 5:12,\allowbreak13 Ps 33:10,\allowbreak11 Pr 21:30 Isa 8:10}
\crossref{Neh}{4}{16}{4:23; 5:15,\allowbreak16 Ps 101:6}
\crossref{Neh}{4}{17}{4:10}
\crossref{Neh}{4}{18}{Nu 10:9 2Ch 13:12-\allowbreak17}
\crossref{Neh}{4}{19}{4:14; 5:7; 7:5}
\crossref{Neh}{4}{20}{Ex 14:14,\allowbreak25 De 1:30; 3:22; 20:4 Jos 23:10 Zec 14:3}
\crossref{Neh}{4}{21}{1Co 15:10,\allowbreak58 Ga 6:9 Col 1:29}
\crossref{Neh}{4}{22}{Ne 11:1,\allowbreak2}
\crossref{Neh}{4}{23}{Ne 5:16; 7:2 Jud 9:48 1Co 15:10}
\crossref{Neh}{5}{1}{Ex 3:7; 22:25-\allowbreak27 Job 31:38,\allowbreak39; 34:28 Isa 5:7 Lu 18:7 Jas 5:4}
\crossref{Neh}{5}{2}{Ps 127:3-\allowbreak5; 128:2-\allowbreak4 Mal 2:2}
\crossref{Neh}{5}{3}{Ge 47:15-\allowbreak25 Le 25:35-\allowbreak39 De 15:7}
\crossref{Neh}{5}{4}{Ne 9:37 De 28:47,\allowbreak48 Jos 16:10 1Ki 9:21 Ezr 4:13,\allowbreak20}
\crossref{Neh}{5}{5}{Ge 37:27 Isa 58:7 Jas 2:5,\allowbreak6}
\crossref{Neh}{5}{6}{Ne 13:8,\allowbreak25 Ex 11:8 Nu 16:15 Mr 3:5 Eph 4:26}
\crossref{Neh}{5}{7}{Ps 4:4; 27:8}
\crossref{Neh}{5}{8}{Mt 25:15,\allowbreak29 2Co 8:12 Ga 6:10}
\crossref{Neh}{5}{9}{1Sa 2:24 Pr 16:29; 17:26; 18:5; 19:2; 24:23}
\crossref{Neh}{5}{10}{Mic 2:1 Lu 3:13,\allowbreak14 1Co 9:12-\allowbreak18}
\crossref{Neh}{5}{11}{Le 6:4,\allowbreak5 1Sa 12:3 2Sa 12:6 Isa 58:6 Lu 3:8}
\crossref{Neh}{5}{12}{2Ch 28:14,\allowbreak15 Ezr 10:12 Mt 19:21,\allowbreak22 Lu 19:8}
\crossref{Neh}{5}{13}{1Sa 15:28 1Ki 11:29-\allowbreak31 Zec 5:3,\allowbreak4}
\crossref{Neh}{5}{14}{Ne 2:1; 13:6}
\crossref{Neh}{5}{15}{1Sa 2:15-\allowbreak17; 8:15 Pr 29:12}
\crossref{Neh}{5}{16}{Lu 8:15 Ro 2:7 1Co 15:58 Ga 6:9}
\crossref{Neh}{5}{17}{2Sa 9:7,\allowbreak13 1Ki 18:19}
\crossref{Neh}{5}{18}{1Ki 4:22,\allowbreak23}
\crossref{Neh}{5}{19}{Ne 13:14,\allowbreak22,\allowbreak31 Ge 40:14 Ps 25:6,\allowbreak7; 40:17; 106:4 Jer 29:11}
\crossref{Neh}{6}{1}{Ne 2:10,\allowbreak19; 4:1,\allowbreak7}
\crossref{Neh}{6}{2}{Ne 11:35 1Ch 8:12}
\crossref{Neh}{6}{3}{Pr 14:15 Mt 10:16}
\crossref{Neh}{6}{4}{Jud 16:6,\allowbreak10,\allowbreak15-\allowbreak20 Pr 7:21 Lu 18:5 1Co 15:58 Ga 2:5}
\crossref{Neh}{6}{5}{}
\crossref{Neh}{6}{6}{Jer 9:3-\allowbreak6; 20:10 Mt 5:11 Ro 3:8 2Co 6:8 1Pe 2:12,\allowbreak13; 3:16}
\crossref{Neh}{6}{7}{6:12,\allowbreak13}
\crossref{Neh}{6}{8}{Ac 24:12,\allowbreak13; 25:7,\allowbreak10}
\crossref{Neh}{6}{9}{6:14; 4:10-\allowbreak14 2Ch 32:18}
\crossref{Neh}{6}{10}{6:12 Ezr 8:16; 10:31 Pr 11:9 Mt 7:15}
\crossref{Neh}{6}{11}{6:3 1Sa 19:5 Job 4:3-\allowbreak6 Ps 11:1,\allowbreak2; 112:6,\allowbreak8 Pr 28:1 Isa 10:18}
\crossref{Neh}{6}{12}{Eze 13:22 1Co 2:15; 12:10}
\crossref{Neh}{6}{13}{Pr 29:5 Isa 51:7,\allowbreak12,\allowbreak13; 57:11 Jer 1:17 Eze 2:6; 13:17-\allowbreak23}
\crossref{Neh}{6}{14}{Ne 5:19 Ps 22:1; 63:1}
\crossref{Neh}{6}{15}{Ezr 6:15 Ps 1:3 Da 9:25}
\crossref{Neh}{6}{16}{Ne 2:10; 4:1,\allowbreak7; 6:1,\allowbreak2}
\crossref{Neh}{6}{17}{Ne 3:5; 5:7; 13:28 Mic 7:1-\allowbreak6 Mt 24:10-\allowbreak12}
\crossref{Neh}{6}{18}{Ne 7:10 Ezr 2:5}
\crossref{Neh}{6}{19}{Pr 28:4 Joh 7:7; 15:19 1Jo 4:5}
\crossref{Neh}{7}{1}{Ne 3:1-\allowbreak32; 6:15}
\crossref{Neh}{7}{2}{Ne 1:2}
\crossref{Neh}{7}{3}{Ne 3:23,\allowbreak28-\allowbreak30}
\crossref{Neh}{7}{4}{Isa 58:12 Hag 1:4-\allowbreak6 Mt 6:33}
\crossref{Neh}{7}{5}{Ne 5:19; 6:14}
\crossref{Neh}{7}{6}{Ezr 2:1-\allowbreak70; 5:8; 6:2}
\crossref{Neh}{7}{7}{Ne 12:1,\allowbreak7,\allowbreak10 Eze 1:11}
\crossref{Neh}{7}{8}{Ne 10:14 Ezr 2:3; 8:3}
\crossref{Neh}{7}{9}{Ezr 2:4; 8:8}
\crossref{Neh}{7}{10}{Ne 6:18 Ezr 2:5}
\crossref{Neh}{7}{11}{Ne 10:14 Ezr 2:6}
\crossref{Neh}{7}{12}{Ezr 2:7; 8:7; 10:26}
\crossref{Neh}{7}{13}{Ezr 2:8}
\crossref{Neh}{7}{14}{Ezr 2:9}
\crossref{Neh}{7}{15}{Ezr 2:10}
\crossref{Neh}{7}{16}{Ezr 2:11}
\crossref{Neh}{7}{17}{Ezr 2:12}
\crossref{Neh}{7}{18}{Ezr 2:13}
\crossref{Neh}{7}{19}{Ezr 2:14}
\crossref{Neh}{7}{20}{Ezr 2:15}
\crossref{Neh}{7}{21}{Ezr 2:16}
\crossref{Neh}{7}{22}{Ezr 2:19}
\crossref{Neh}{7}{23}{Ezr 2:17}
\crossref{Neh}{7}{24}{}
\crossref{Neh}{7}{25}{}
\crossref{Neh}{7}{26}{Ezr 2:21,\allowbreak22}
\crossref{Neh}{7}{27}{Ezr 2:23 Isa 10:30 Jer 1:1; 11:21}
\crossref{Neh}{7}{28}{Ezr 2:24}
\crossref{Neh}{7}{29}{}
\crossref{Neh}{7}{30}{Jos 18:24,\allowbreak25 Ezr 2:26}
\crossref{Neh}{7}{31}{}
\crossref{Neh}{7}{32}{Jos 8:9,\allowbreak17 Ezr 2:28}
\crossref{Neh}{7}{33}{Ezr 2:29}
\crossref{Neh}{7}{34}{7:12 Ezr 2:31}
\crossref{Neh}{7}{35}{Ezr 2:32; 10:31}
\crossref{Neh}{7}{36}{Ezr 2:34}
\crossref{Neh}{7}{37}{Ne 6:2; 11:34,\allowbreak35 1Ch 8:12 Ezr 2:33}
\crossref{Neh}{7}{38}{Ezr 2:35}
\crossref{Neh}{7}{39}{1Ch 24:7-\allowbreak19 Ezr 2:36}
\crossref{Neh}{7}{40}{1Ch 24:14 Ezr 2:37}
\crossref{Neh}{7}{41}{1Ch 9:12; 24:9 Ezr 2:38; 10:22}
\crossref{Neh}{7}{42}{1Ch 24:8 Ezr 2:39; 10:31}
\crossref{Neh}{7}{43}{Ezr 2:40}
\crossref{Neh}{7}{44}{1Ch 25:2 Ezr 2:41}
\crossref{Neh}{7}{45}{1Ch 26:1-\allowbreak32 Ezr 2:42}
\crossref{Neh}{7}{46}{Le 27:2-\allowbreak8 Jos 9:23-\allowbreak27 1Ch 9:2}
\crossref{Neh}{7}{47}{Ezr 2:44}
\crossref{Neh}{7}{48}{Ezr 2:45,\allowbreak46}
\crossref{Neh}{7}{49}{7:49}
\crossref{Neh}{7}{50}{}
\crossref{Neh}{7}{51}{}
\crossref{Neh}{7}{52}{}
\crossref{Neh}{7}{53}{}
\crossref{Neh}{7}{54}{}
\crossref{Neh}{7}{55}{}
\crossref{Neh}{7}{56}{}
\crossref{Neh}{7}{57}{Ne 11:3}
\crossref{Neh}{7}{58}{}
\crossref{Neh}{7}{59}{}
\crossref{Neh}{7}{60}{Ezr 2:58}
\crossref{Neh}{7}{61}{}
\crossref{Neh}{7}{62}{Ezr 2:60}
\crossref{Neh}{7}{63}{Ezr 2:61-\allowbreak63}
\crossref{Neh}{7}{64}{Mt 22:11-\allowbreak13}
\crossref{Neh}{7}{65}{Ne 8:9; 10:1 Ezr 2:63}
\crossref{Neh}{7}{66}{Ezr 2:64}
\crossref{Neh}{7}{67}{Isa 45:1,\allowbreak2 Jer 27:7}
\crossref{Neh}{7}{68}{Ezr 2:66,\allowbreak67}
\crossref{Neh}{7}{69}{}
\crossref{Neh}{7}{70}{Ezr 2:68-\allowbreak70}
\crossref{Neh}{7}{71}{Job 34:19 Lu 21:1-\allowbreak4 2Co 8:12}
\crossref{Neh}{7}{72}{}
\crossref{Neh}{7}{73}{Ezr 2:70; 3:1}
\crossref{Neh}{8}{1}{Ezr 3:1-\allowbreak13}
\crossref{Neh}{8}{2}{De 17:18; 31:9,\allowbreak10 Mal 2:7}
\crossref{Neh}{8}{3}{Lu 4:16-\allowbreak20 Ac 13:15,\allowbreak27; 15:21}
\crossref{Neh}{8}{4}{Ne 10:25; 11:5}
\crossref{Neh}{8}{5}{Lu 4:16,\allowbreak17}
\crossref{Neh}{8}{6}{1Ch 29:20 2Ch 6:4 Ps 41:13; 72:18,\allowbreak19 Eph 1:3 1Pe 1:3}
\crossref{Neh}{8}{7}{Ne 3:19; 9:4; 10:9; 12:24}
\crossref{Neh}{8}{8}{Hab 2:2 Mt 5:21,\allowbreak22,\allowbreak27,\allowbreak28 Lu 24:27,\allowbreak32,\allowbreak45 Ac 8:30-\allowbreak35; 17:2,\allowbreak3}
\crossref{Neh}{8}{9}{Ne 7:65,\allowbreak70; 10:1 Ezr 2:63}
\crossref{Neh}{8}{10}{Ec 2:24; 3:13; 5:18; 9:7 1Ti 6:17,\allowbreak18}
\crossref{Neh}{8}{11}{Nu 13:30}
\crossref{Neh}{8}{12}{8:10}
\crossref{Neh}{8}{13}{2Ch 30:23 Pr 2:1-\allowbreak6; 8:33,\allowbreak34; 12:1 Mr 6:33,\allowbreak34 Lu 19:47,\allowbreak48 Ac 4:1}
\crossref{Neh}{8}{14}{Le 23:34,\allowbreak40-\allowbreak43 De 16:13-\allowbreak15 Zec 14:16-\allowbreak19 Joh 7:2}
\crossref{Neh}{8}{15}{Le 23:4}
\crossref{Neh}{8}{16}{De 22:8 2Sa 11:2 Jer 19:13; 32:29}
\crossref{Neh}{8}{17}{Joh 1:14 Heb 11:9,\allowbreak13}
\crossref{Neh}{8}{18}{De 31:10-\allowbreak13}
\crossref{Neh}{9}{1}{Le 23:34,\allowbreak39 2Ch 7:10}
\crossref{Neh}{9}{2}{Ne 13:3,\allowbreak30 Ezr 9:2; 10:11}
\crossref{Neh}{9}{3}{Ne 8:4,\allowbreak7,\allowbreak8}
\crossref{Neh}{9}{4}{9:5}
\crossref{Neh}{9}{5}{1Ki 8:14,\allowbreak22 2Ch 20:13,\allowbreak19 Ps 134:1-\allowbreak3; 135:1-\allowbreak3}
\crossref{Neh}{9}{6}{De 6:4 2Ki 19:15,\allowbreak19 Ps 86:10 Isa 37:16,\allowbreak20; 43:10; 44:6,\allowbreak8}
\crossref{Neh}{9}{7}{Ge 12:1,\allowbreak2 De 10:15 Jos 24:2,\allowbreak3 Isa 41:8,\allowbreak9; 51:2}
\crossref{Neh}{9}{8}{Ge 12:1-\allowbreak3; 15:6,\allowbreak18; 22:12 Ac 13:22 1Ti 1:12,\allowbreak13 Heb 11:17}
\crossref{Neh}{9}{9}{Ex 2:25; 3:7-\allowbreak9,\allowbreak16 Ac 7:34}
\crossref{Neh}{9}{10}{Ex 7:1-\allowbreak25; 14:1-\allowbreak31 De 4:34; 11:3,\allowbreak4 Ps 78:12,\allowbreak13,\allowbreak43-\allowbreak53; 105:27-\allowbreak37}
\crossref{Neh}{9}{11}{Ex 14:21,\allowbreak22,\allowbreak27,\allowbreak28 Ps 66:6; 78:13; 114:3-\allowbreak5; 136:13-\allowbreak15}
\crossref{Neh}{9}{12}{9:19 Ex 13:21,\allowbreak22; 14:19,\allowbreak20 Ps 78:14; 105:39}
\crossref{Neh}{9}{13}{Ex 19:11,\allowbreak16-\allowbreak20 De 33:2 Isa 64:1,\allowbreak3 Hab 3:3}
\crossref{Neh}{9}{14}{Ge 2:3 Ex 16:29; 20:8-\allowbreak11 Eze 20:12,\allowbreak20}
\crossref{Neh}{9}{15}{Ex 16:4,\allowbreak14,\allowbreak15 De 8:3,\allowbreak16 Ps 78:24,\allowbreak25; 105:40 Joh 6:31-\allowbreak35}
\crossref{Neh}{9}{16}{9:10,\allowbreak29 Ex 32:9 De 9:6,\allowbreak13,\allowbreak23,\allowbreak24,\allowbreak27; 32:15 Ps 78:8-\allowbreak72; 106:6}
\crossref{Neh}{9}{17}{Nu 14:3,\allowbreak4,\allowbreak11,\allowbreak41; 16:14 Ps 106:24,\allowbreak25 Pr 1:24 Heb 12:25}
\crossref{Neh}{9}{18}{Ex 32:4-\allowbreak8,\allowbreak31,\allowbreak32 De 9:12-\allowbreak16 Ps 106:19-\allowbreak23 Eze 20:7-\allowbreak44}
\crossref{Neh}{9}{19}{9:27 1Sa 12:22 Ps 106:7,\allowbreak8,\allowbreak45 Isa 44:21 La 3:22 Eze 20:14,\allowbreak22}
\crossref{Neh}{9}{20}{9:30 Nu 11:17,\allowbreak25-\allowbreak29 Isa 63:11-\allowbreak14}
\crossref{Neh}{9}{21}{Ex 16:35 Nu 14:33,\allowbreak34 De 2:7; 8:2 Am 5:25 Ac 13:18}
\crossref{Neh}{9}{22}{Jos 10:11 Ps 78:65; 105:44}
\crossref{Neh}{9}{23}{Ge 15:5; 22:17 1Ch 27:23}
\crossref{Neh}{9}{24}{Nu 14:31 Jos 21:43,\allowbreak45}
\crossref{Neh}{9}{25}{Nu 13:27,\allowbreak28 De 3:5; 6:10-\allowbreak12; 9:1-\allowbreak3}
\crossref{Neh}{9}{26}{Jud 2:11,\allowbreak12; 3:6,\allowbreak7; 10:6,\allowbreak13,\allowbreak14 Ps 78:56,\allowbreak57; 106:34-\allowbreak40}
\crossref{Neh}{9}{27}{De 31:16-\allowbreak18 Jud 2:14,\allowbreak15; 3:8-\allowbreak30 2Ch 36:17 Ps 106:41,\allowbreak42}
\crossref{Neh}{9}{28}{Jud 3:11,\allowbreak12,\allowbreak30; 4:1; 5:31; 6:1}
\crossref{Neh}{9}{29}{9:26 De 4:26; 31:21 2Ki 17:13 2Ch 24:19; 36:15 Jer 25:3-\allowbreak7 Ho 6:5}
\crossref{Neh}{9}{30}{Ps 86:15 Ro 2:4 2Pe 3:9}
\crossref{Neh}{9}{31}{Jer 4:27; 5:10,\allowbreak18 La 3:22 Eze 14:22,\allowbreak23 Da 9:9}
\crossref{Neh}{9}{32}{Ne 1:5 De 7:21 Ps 47:2; 66:3,\allowbreak5}
\crossref{Neh}{9}{33}{Ge 18:25 Job 34:23 Ps 119:137; 145:17 Jer 12:1 La 1:18}
\crossref{Neh}{9}{34}{Jer 29:19}
\crossref{Neh}{9}{35}{De 28:47 Jer 5:19 Ro 3:4,\allowbreak5}
\crossref{Neh}{9}{36}{De 28:48 2Ch 12:8 Ezr 9:9}
\crossref{Neh}{9}{37}{De 28:33,\allowbreak39,\allowbreak51 Ezr 4:13; 6:8; 7:24}
\crossref{Neh}{9}{38}{Ne 10:29 2Ki 23:3 2Ch 15:12,\allowbreak13; 23:16; 29:10; 34:31 Ezr 10:3}
\crossref{Neh}{10}{1}{Ne 9:38}
\crossref{Neh}{10}{2}{Ne 3:23; 11:11; 12:1,\allowbreak33,\allowbreak34}
\crossref{Neh}{10}{3}{Ne 11:12}
\crossref{Neh}{10}{4}{Ne 3:10}
\crossref{Neh}{10}{5}{Ne 3:11}
\crossref{Neh}{10}{6}{Ne 12:4}
\crossref{Neh}{10}{7}{Ne 3:6; 8:4; 11:11; 12:13,\allowbreak25-\allowbreak33}
\crossref{Neh}{10}{8}{Ne 12:5}
\crossref{Neh}{10}{9}{Ne 3:19; 7:43; 8:7; 9:4}
\crossref{Neh}{10}{10}{Ne 8:7; 9:4,\allowbreak5 Ezr 10:23}
\crossref{Neh}{10}{11}{Ne 11:15,\allowbreak22; 12:24 Ezr 8:19,\allowbreak24}
\crossref{Neh}{10}{12}{Ne 8:7; 9:4; 12:8}
\crossref{Neh}{10}{13}{10:13}
\crossref{Neh}{10}{14}{Ne 3:11; 7:8,\allowbreak11-\allowbreak13}
\crossref{Neh}{10}{15}{Ne 7:16,\allowbreak17 Ezr 2:11,\allowbreak12; 8:11,\allowbreak12; 10:28}
\crossref{Neh}{10}{16}{Ne 7:19-\allowbreak21 Ezr 2:14-\allowbreak16; 8:14}
\crossref{Neh}{10}{17}{Ezr 2:16}
\crossref{Neh}{10}{18}{Ne 7:22-\allowbreak73}
\crossref{Neh}{10}{19}{Ne 7:24}
\crossref{Neh}{10}{20}{}
\crossref{Neh}{10}{21}{Ne 3:4}
\crossref{Neh}{10}{22}{1Ch 3:21}
\crossref{Neh}{10}{23}{2Ki 15:30}
\crossref{Neh}{10}{24}{}
\crossref{Neh}{10}{25}{Ne 3:17-\allowbreak32}
\crossref{Neh}{10}{26}{1Ki 14:6}
\crossref{Neh}{10}{27}{}
\crossref{Neh}{10}{28}{Ne 7:72,\allowbreak73 Ezr 2:36-\allowbreak43,\allowbreak70}
\crossref{Neh}{10}{29}{Isa 14:1 Ac 11:23; 17:34 Ro 12:9}
\crossref{Neh}{10}{30}{Ex 34:16 De 7:3 Ezr 9:1-\allowbreak3,\allowbreak12-\allowbreak14; 10:10-\allowbreak12}
\crossref{Neh}{10}{31}{Ne 13:15-\allowbreak22 Ex 20:10 Le 23:3 De 5:12-\allowbreak14 Isa 58:13,\allowbreak14}
\crossref{Neh}{10}{32}{Ge 28:22 Pr 3:9}
\crossref{Neh}{10}{33}{Le 24:5 2Ch 2:4}
\crossref{Neh}{10}{34}{1Ch 24:5,\allowbreak7; 25:8,\allowbreak9 Pr 18:18}
\crossref{Neh}{10}{35}{Ex 23:19; 34:26 Le 19:23-\allowbreak26 Nu 18:2,\allowbreak12 De 26:2 2Ch 31:3-\allowbreak10}
\crossref{Neh}{10}{36}{Ex 13:2,\allowbreak12-\allowbreak15; 34:19 Le 27:26,\allowbreak27 Nu 18:15,\allowbreak16 De 12:6}
\crossref{Neh}{10}{37}{Le 23:17 Nu 15:19-\allowbreak21; 18:12,\allowbreak13 De 18:4; 26:2}
\crossref{Neh}{10}{38}{Nu 18:26-\allowbreak28}
\crossref{Neh}{10}{39}{De 12:6-\allowbreak11,\allowbreak17; 14:23-\allowbreak27 2Ch 31:12}
\crossref{Neh}{11}{1}{Ne 7:4,\allowbreak5 De 17:8,\allowbreak9 Ps 122:5}
\crossref{Neh}{11}{2}{De 24:13 Job 29:13; 31:20}
\crossref{Neh}{11}{3}{Ne 7:6 Ezr 2:1}
\crossref{Neh}{11}{4}{1Ch 9:3,\allowbreak4-\allowbreak9}
\crossref{Neh}{11}{5}{Ne 3:15}
\crossref{Neh}{11}{6}{}
\crossref{Neh}{11}{7}{1Ch 9:7-\allowbreak9}
\crossref{Neh}{11}{8}{}
\crossref{Neh}{11}{9}{1Ch 9:7}
\crossref{Neh}{11}{10}{Ezr 2:36; 8:16}
\crossref{Neh}{11}{11}{Ezr 7:1-\allowbreak5}
\crossref{Neh}{11}{12}{1Ch 9:12,\allowbreak13}
\crossref{Neh}{11}{13}{11:13}
\crossref{Neh}{11}{14}{11:14}
\crossref{Neh}{11}{15}{1Ch 9:14,\allowbreak19}
\crossref{Neh}{11}{16}{Ne 8:7}
\crossref{Neh}{11}{17}{Ne 12:8,\allowbreak31 1Ch 16:4,\allowbreak41; 25:1-\allowbreak6}
\crossref{Neh}{11}{18}{11:1 1Ki 11:13 Da 9:24 Mt 24:15; 27:53 Re 11:2; 21:2}
\crossref{Neh}{11}{19}{Ne 7:45; 12:25 1Ch 9:17-\allowbreak22}
\crossref{Neh}{11}{20}{}
\crossref{Neh}{11}{21}{Ne 3:26,\allowbreak31 2Ch 27:3}
\crossref{Neh}{11}{22}{11:9,\allowbreak14; 12:42 Ac 20:28}
\crossref{Neh}{11}{23}{1Ch 9:33 Ezr 6:8,\allowbreak9; 7:20-\allowbreak24}
\crossref{Neh}{11}{24}{Ne 10:21}
\crossref{Neh}{11}{25}{Jos 14:15}
\crossref{Neh}{11}{26}{Jos 15:26; 19:2}
\crossref{Neh}{11}{27}{Jos 15:28; 19:3}
\crossref{Neh}{11}{28}{Jos 15:31 1Sa 27:6}
\crossref{Neh}{11}{29}{}
\crossref{Neh}{11}{30}{Ne 3:13 Jos 15:34}
\crossref{Neh}{11}{31}{Ne 7:30}
\crossref{Neh}{11}{32}{Ne 7:27 Jos 21:18 Isa 10:30 Jer 1:1}
\crossref{Neh}{11}{33}{Jos 18:25 1Sa 7:17 Mt 2:18}
\crossref{Neh}{11}{34}{1Sa 13:18}
\crossref{Neh}{11}{35}{Ne 7:37 1Ch 8:12}
\crossref{Neh}{11}{36}{Jos 21:1-\allowbreak45 1Ch 6:54-\allowbreak81}
\crossref{Neh}{12}{1}{Ne 7:7 Ezr 2:1,\allowbreak2}
\crossref{Neh}{12}{2}{12:14}
\crossref{Neh}{12}{3}{12:14}
\crossref{Neh}{12}{4}{Lu 1:5}
\crossref{Neh}{12}{5}{12:17}
\crossref{Neh}{12}{6}{Ne 11:10 1Ch 9:10}
\crossref{Neh}{12}{7}{}
\crossref{Neh}{12}{8}{Ne 7:48; 9:4; 10:9-\allowbreak13}
\crossref{Neh}{12}{9}{Ps 134:1-\allowbreak3}
\crossref{Neh}{12}{10}{12:26 1Ch 6:3-\allowbreak15}
\crossref{Neh}{12}{11}{}
\crossref{Neh}{12}{12}{12:22 1Ch 9:33,\allowbreak34; 15:12; 24:6-\allowbreak31}
\crossref{Neh}{12}{13}{}
\crossref{Neh}{12}{14}{12:2}
\crossref{Neh}{12}{15}{12:3}
\crossref{Neh}{12}{16}{12:4}
\crossref{Neh}{12}{17}{12:5}
\crossref{Neh}{12}{18}{12:6}
\crossref{Neh}{12}{19}{Ezr 8:16}
\crossref{Neh}{12}{20}{12:7}
\crossref{Neh}{12}{21}{}
\crossref{Neh}{12}{22}{12:10,\allowbreak11}
\crossref{Neh}{12}{23}{1Ch 9:14-\allowbreak44}
\crossref{Neh}{12}{24}{12:8; 8:7; 9:4; 10:9-\allowbreak13}
\crossref{Neh}{12}{25}{12:8,\allowbreak9; 11:17-\allowbreak19 1Ch 9:14-\allowbreak17}
\crossref{Neh}{12}{26}{12:10}
\crossref{Neh}{12}{27}{Ne 11:20 1Ch 15:4,\allowbreak12; 25:6; 26:31 2Ch 5:13; 29:4-\allowbreak11,\allowbreak30 Ezr 8:15-\allowbreak20}
\crossref{Neh}{12}{28}{Ne 6:2}
\crossref{Neh}{12}{29}{Ne 11:31 Jos 21:17 1Ch 6:60}
\crossref{Neh}{12}{30}{Ge 35:2 Ex 19:10,\allowbreak15 Nu 19:2-\allowbreak20 2Ch 29:5,\allowbreak34 Ezr 6:21 Job 1:5}
\crossref{Neh}{12}{31}{1Ch 13:1; 28:1 2Ch 5:2}
\crossref{Neh}{12}{32}{}
\crossref{Neh}{12}{33}{Ne 10:2-\allowbreak7}
\crossref{Neh}{12}{34}{}
\crossref{Neh}{12}{35}{Nu 10:2-\allowbreak10 Jos 6:4 2Ch 5:12; 13:12}
\crossref{Neh}{12}{36}{12:24 1Ch 23:5 2Ch 8:14 Am 6:5}
\crossref{Neh}{12}{37}{Ne 2:14; 3:15,\allowbreak16}
\crossref{Neh}{12}{38}{12:31}
\crossref{Neh}{12}{39}{Ne 8:16 2Ki 14:13}
\crossref{Neh}{12}{40}{12:31,\allowbreak32 Ps 42:4; 47:6-\allowbreak9; 134:1-\allowbreak3}
\crossref{Neh}{12}{41}{12:35}
\crossref{Neh}{12}{42}{Ps 81:1; 95:1; 98:4-\allowbreak9; 100:1,\allowbreak2 Isa 12:5,\allowbreak6}
\crossref{Neh}{12}{43}{Nu 10:10 De 12:11,\allowbreak12 1Ch 29:21,\allowbreak22 2Ch 7:5-\allowbreak7,\allowbreak10; 29:35,\allowbreak36}
\crossref{Neh}{12}{44}{Ne 10:37-\allowbreak39; 13:5,\allowbreak12,\allowbreak13 2Ch 13:11,\allowbreak12; 31:11-\allowbreak13}
\crossref{Neh}{12}{45}{1Ch 25:1-\allowbreak26:32}
\crossref{Neh}{12}{46}{1Ch 25:1-\allowbreak31 2Ch 29:30 Ps 73:1}
\crossref{Neh}{12}{47}{12:1,\allowbreak12,\allowbreak26}
\crossref{Neh}{13}{1}{Ne 8:3-\allowbreak8; 9:3 De 31:11,\allowbreak12 2Ki 23:2 Isa 34:16 Lu 4:16-\allowbreak19; 10:26}
\crossref{Neh}{13}{2}{Mt 25:40}
\crossref{Neh}{13}{3}{Ps 19:7-\allowbreak11; 119:9,\allowbreak11 Pr 6:23 Ro 3:20}
\crossref{Neh}{13}{4}{13:7}
\crossref{Neh}{13}{5}{Ne 10:38; 12:44 2Ch 34:11}
\crossref{Neh}{13}{6}{Ex 32:1 2Ch 24:17,\allowbreak18 Mt 13:25}
\crossref{Neh}{13}{7}{Ezr 9:1 1Co 1:11}
\crossref{Neh}{13}{8}{Ezr 9:3,\allowbreak4; 10:1 Ps 69:9}
\crossref{Neh}{13}{9}{Ne 12:45 2Ch 29:5,\allowbreak15-\allowbreak19}
\crossref{Neh}{13}{10}{Ne 10:37; 12:47 Mal 1:6-\allowbreak14; 3:8 1Ti 5:17,\allowbreak18}
\crossref{Neh}{13}{11}{13:17,\allowbreak25; 5:6-\allowbreak13 Job 31:34 Pr 28:4}
\crossref{Neh}{13}{12}{Ne 10:37-\allowbreak39; 12:44 Le 27:30 Nu 18:20-\allowbreak26 De 14:22}
\crossref{Neh}{13}{13}{Ne 12:44 2Ch 31:12-\allowbreak15}
\crossref{Neh}{13}{14}{13:22,\allowbreak31; 5:19 Ps 122:6-\allowbreak9 Heb 6:10 Re 3:5}
\crossref{Neh}{13}{15}{Ex 20:8-\allowbreak11; 34:21; 35:2 Isa 58:13 Eze 20:13}
\crossref{Neh}{13}{16}{Ex 23:12 De 5:14}
\crossref{Neh}{13}{17}{13:11,\allowbreak25; 5:7 Ps 82:1,\allowbreak2 Pr 28:4 Isa 1:10 Jer 5:5; 13:18; 22:2-\allowbreak23}
\crossref{Neh}{13}{18}{Ezr 9:13-\allowbreak15 Jer 17:21-\allowbreak23,\allowbreak27; 44:9,\allowbreak22 Eze 23:8,\allowbreak26 Zec 1:4-\allowbreak6}
\crossref{Neh}{13}{19}{Le 23:22}
\crossref{Neh}{13}{20}{}
\crossref{Neh}{13}{21}{13:15}
\crossref{Neh}{13}{22}{Ne 7:64,\allowbreak65; 12:30 2Ki 23:4 1Ch 15:12-\allowbreak14 2Ch 29:4,\allowbreak5,\allowbreak24,\allowbreak27,\allowbreak30}
\crossref{Neh}{13}{23}{Ne 10:30 Ezr 9:2,\allowbreak11,\allowbreak12; 10:10,\allowbreak44 2Co 6:14}
\crossref{Neh}{13}{24}{}
\crossref{Neh}{13}{25}{13:11,\allowbreak17 Pr 28:4}
\crossref{Neh}{13}{26}{1Ki 11:1-\allowbreak8 Ec 7:26}
\crossref{Neh}{13}{27}{1Sa 30:24}
\crossref{Neh}{13}{28}{Ne 12:10,\allowbreak22}
\crossref{Neh}{13}{29}{Ne 6:14 Ps 59:5-\allowbreak13 2Ti 4:14}
\crossref{Neh}{13}{30}{Ne 10:30}
\crossref{Neh}{13}{31}{Ne 10:34}

% Esth
\crossref{Esth}{1}{1}{Es 8:9 Isa 18:1; 37:9}
\crossref{Esth}{1}{2}{2Sa 7:1 1Ki 1:46 Da 4:4}
\crossref{Esth}{1}{3}{Es 2:18 Ge 40:20 1Ki 3:15 Da 5:1 Mr 6:21}
\crossref{Esth}{1}{4}{Isa 39:2 Eze 28:5 Da 4:30}
\crossref{Esth}{1}{5}{2Ch 7:8,\allowbreak9; 30:21-\allowbreak25}
\crossref{Esth}{1}{6}{Ex 26:1,\allowbreak31,\allowbreak32,\allowbreak36,\allowbreak37}
\crossref{Esth}{1}{7}{1Ki 10:21 2Ch 9:20 Da 5:2-\allowbreak4}
\crossref{Esth}{1}{8}{Joh 2:8}
\crossref{Esth}{1}{9}{Es 5:4,\allowbreak8}
\crossref{Esth}{1}{10}{Ge 43:34 Jud 16:25 1Sa 25:36,\allowbreak37 2Sa 13:28 Pr 20:1 Ec 7:2-\allowbreak4}
\crossref{Esth}{1}{11}{Pr 16:9; 23:29-\allowbreak33 Mr 6:21,\allowbreak22}
\crossref{Esth}{1}{12}{Ge 3:16 Eph 5:22,\allowbreak24 1Pe 3:1}
\crossref{Esth}{1}{13}{Jer 10:7 Da 2:2,\allowbreak12,\allowbreak27; 4:6,\allowbreak7; 5:7 Mt 2:1}
\crossref{Esth}{1}{14}{Ezr 7:14}
\crossref{Esth}{1}{15}{Es 6:6}
\crossref{Esth}{1}{16}{Ac 18:14; 25:10 1Co 6:7,\allowbreak8}
\crossref{Esth}{1}{17}{2Sa 6:16 Eph 5:33}
\crossref{Esth}{1}{18}{}
\crossref{Esth}{1}{19}{1:21; 3:9; 8:5}
\crossref{Esth}{1}{20}{De 17:13; 21:21}
\crossref{Esth}{1}{21}{1:19; 2:4 Ge 41:37}
\crossref{Esth}{1}{22}{Es 3:12; 8:9 Da 3:29; 4:1}
\crossref{Esth}{2}{1}{Da 6:14-\allowbreak18}
\crossref{Esth}{2}{2}{Es 1:10,\allowbreak14; 6:14}
\crossref{Esth}{2}{3}{Es 1:1,\allowbreak2}
\crossref{Esth}{2}{4}{Mt 20:16; 22:14}
\crossref{Esth}{2}{5}{2:3; 1:2; 5:1}
\crossref{Esth}{2}{6}{2Ki 24:6,\allowbreak14,\allowbreak15 2Ch 36:9,\allowbreak10,\allowbreak20}
\crossref{Esth}{2}{7}{Eph 6:4}
\crossref{Esth}{2}{8}{}
\crossref{Esth}{2}{9}{Ge 39:21 1Ki 8:50 Ezr 7:6 Ne 2:8 Ps 106:46 Pr 16:7 Da 1:9}
\crossref{Esth}{2}{10}{Es 3:8; 4:13,\allowbreak14; 7:4 Mt 10:16}
\crossref{Esth}{2}{11}{2:13,\allowbreak14}
\crossref{Esth}{2}{12}{1Th 4:4,\allowbreak5}
\crossref{Esth}{2}{13}{}
\crossref{Esth}{2}{14}{Es 4:11 Ge 34:19 De 21:14 Isa 62:4,\allowbreak5}
\crossref{Esth}{2}{15}{2:7}
\crossref{Esth}{2}{16}{Es 8:9}
\crossref{Esth}{2}{17}{}
\crossref{Esth}{2}{18}{Es 1:3-\allowbreak5 Ge 29:22 Jud 14:10-\allowbreak17 So 3:11; 5:1 Mt 22:2 Lu 14:8}
\crossref{Esth}{2}{19}{2:3,\allowbreak4}
\crossref{Esth}{2}{20}{2:10}
\crossref{Esth}{2}{21}{Es 6:2}
\crossref{Esth}{2}{22}{Ec 10:20 Ac 23:12-\allowbreak22}
\crossref{Esth}{2}{23}{Es 5:14; 7:10 Ge 40:19,\allowbreak22 De 21:22,\allowbreak23 Jos 8:29}
\crossref{Esth}{3}{1}{Es 7:6 Ps 12:8 Pr 29:2}
\crossref{Esth}{3}{2}{Ge 41:43 Php 2:10}
\crossref{Esth}{3}{3}{3:2 Ex 1:17 Mt 15:2,\allowbreak3}
\crossref{Esth}{3}{4}{Ge 39:10}
\crossref{Esth}{3}{5}{3:2; 5:9}
\crossref{Esth}{3}{6}{Ps 83:4 Re 12:12}
\crossref{Esth}{3}{7}{Ne 2:1}
\crossref{Esth}{3}{8}{Le 26:33 De 4:27; 30:3; 32:26 Ne 1:8 Jer 50:17 Eze 6:8; 11:16}
\crossref{Esth}{3}{9}{Mt 18:24}
\crossref{Esth}{3}{10}{Es 8:2,\allowbreak8 Ge 41:42}
\crossref{Esth}{3}{11}{Ps 73:7 Jer 26:14; 40:4 Lu 23:25}
\crossref{Esth}{3}{12}{Es 8:9-\allowbreak17}
\crossref{Esth}{3}{13}{Es 8:10,\allowbreak14 2Ch 30:6 Job 9:25 Jer 51:31 Ro 3:15}
\crossref{Esth}{3}{14}{Es 8:13,\allowbreak14}
\crossref{Esth}{3}{15}{Pr 1:16; 4:16}
\crossref{Esth}{4}{1}{Es 3:8-\allowbreak13}
\crossref{Esth}{4}{2}{Ge 32:28 Ne 1:11 Ps 116:1 Pr 21:1 Ac 7:10; 10:4}
\crossref{Esth}{4}{3}{Es 1:1; 3:12}
\crossref{Esth}{4}{4}{Es 1:12 1Sa 8:15}
\crossref{Esth}{4}{5}{Es 1:10,\allowbreak12}
\crossref{Esth}{4}{6}{4:3; 7:2; 9:12}
\crossref{Esth}{4}{7}{Es 3:2-\allowbreak15}
\crossref{Esth}{4}{8}{Es 3:14,\allowbreak15}
\crossref{Esth}{4}{9}{Job 20:5 Am 6:12,\allowbreak13 Lu 6:25 Joh 16:20 Jas 4:9}
\crossref{Esth}{4}{10}{Ge 43:30,\allowbreak31; 45:1 2Sa 13:22,\allowbreak23 Ps 55:21 Ec 7:9}
\crossref{Esth}{4}{11}{Es 5:1}
\crossref{Esth}{4}{12}{}
\crossref{Esth}{4}{13}{Pr 24:10-\allowbreak12 Mt 16:24,\allowbreak25 Joh 12:25 Php 2:30 Heb 12:3}
\crossref{Esth}{4}{14}{Ge 22:14 Nu 23:22-\allowbreak24 De 32:26,\allowbreak27,\allowbreak36 1Sa 12:22 Isa 54:17}
\crossref{Esth}{4}{15}{}
\crossref{Esth}{4}{16}{2Ch 20:3 Isa 22:12 Joe 1:14,\allowbreak15; 2:12-\allowbreak17 Jon 3:4-\allowbreak9}
\crossref{Esth}{4}{17}{4:17}
\crossref{Esth}{5}{1}{Es 4:16 Mt 27:64}
\crossref{Esth}{5}{2}{Ge 32:28 Ne 1:11 Ps 116:1 Pr 21:1 Ac 7:10; 10:4}
\crossref{Esth}{5}{3}{5:6}
\crossref{Esth}{5}{4}{5:8 Pr 29:11}
\crossref{Esth}{5}{5}{Es 6:14}
\crossref{Esth}{5}{6}{5:3; 7:2; 9:12}
\crossref{Esth}{5}{7}{5:6,\allowbreak8; 7:2,\allowbreak3; 9:12}
\crossref{Esth}{5}{8}{Es 6:1-\allowbreak13 Pr 16:9}
\crossref{Esth}{5}{9}{Job 20:5 Am 6:12,\allowbreak13 Lu 6:25 Joh 16:20 Jas 4:9}
\crossref{Esth}{5}{10}{Ge 43:30,\allowbreak31; 45:1 2Sa 13:22,\allowbreak23 Ec 7:9}
\crossref{Esth}{5}{11}{Es 1:4 Ge 31:1 Job 31:24,\allowbreak25 Ps 49:6,\allowbreak16,\allowbreak17 Isa 10:8 Jer 9:23,\allowbreak24}
\crossref{Esth}{5}{12}{Job 8:12,\allowbreak13; 20:5-\allowbreak8 Ps 37:35,\allowbreak36 Pr 7:22,\allowbreak23; 27:1 Lu 21:34,\allowbreak35}
\crossref{Esth}{5}{13}{}
\crossref{Esth}{5}{14}{2Sa 13:3-\allowbreak5 1Ki 21:7,\allowbreak25 2Ch 22:3,\allowbreak4 Mr 6:19-\allowbreak24}
\crossref{Esth}{6}{1}{Es 5:8 Ge 22:14 1Sa 23:26,\allowbreak27 Isa 41:17 Ro 11:33}
\crossref{Esth}{6}{2}{Es 2:21}
\crossref{Esth}{6}{3}{Jud 1:12,\allowbreak13 1Sa 17:25,\allowbreak26 1Ch 11:6 Da 5:7,\allowbreak16,\allowbreak29 Ac 28:8-\allowbreak10}
\crossref{Esth}{6}{4}{Pr 3:27,\allowbreak28 Ec 9:10}
\crossref{Esth}{6}{5}{}
\crossref{Esth}{6}{6}{Es 3:2,\allowbreak3; 5:11 Pr 1:32; 16:18; 18:12; 30:13 Ob 1:3}
\crossref{Esth}{6}{7}{}
\crossref{Esth}{6}{8}{}
\crossref{Esth}{6}{9}{Ge 41:43 1Ki 1:33,\allowbreak34 Zec 9:9}
\crossref{Esth}{6}{10}{Da 4:37 Lu 14:11 Re 18:7}
\crossref{Esth}{6}{11}{Ezr 6:13 Isa 60:14 Lu 1:52 Re 3:9}
\crossref{Esth}{6}{12}{Es 2:19 1Sa 3:15 Ps 131:1,\allowbreak2}
\crossref{Esth}{6}{13}{Es 5:10-\allowbreak14}
\crossref{Esth}{6}{14}{Es 5:8,\allowbreak14 De 32:35,\allowbreak36}
\crossref{Esth}{7}{1}{Es 3:15; 5:8}
\crossref{Esth}{7}{2}{Es 5:6 Joh 16:24}
\crossref{Esth}{7}{3}{7:7 1Ki 20:31 2Ki 1:13 Job 2:4 Jer 38:26}
\crossref{Esth}{7}{4}{Es 3:9; 4:7,\allowbreak8 De 28:68 1Sa 22:23}
\crossref{Esth}{7}{5}{Ge 27:33 Job 9:24}
\crossref{Esth}{7}{6}{1Sa 24:13 Ps 27:2; 139:19-\allowbreak22 Pr 24:24,\allowbreak25 Ec 5:8 1Co 5:13}
\crossref{Esth}{7}{7}{Es 1:12}
\crossref{Esth}{7}{8}{Es 1:6 Isa 49:23}
\crossref{Esth}{7}{9}{Es 1:10}
\crossref{Esth}{7}{10}{Jud 15:7 Eze 5:13 Zec 6:8}
\crossref{Esth}{8}{1}{Job 27:16,\allowbreak17 Ps 39:6; 49:6-\allowbreak13 Pr 13:22; 28:8 Ec 2:18,\allowbreak19}
\crossref{Esth}{8}{2}{Es 3:10 Ge 41:42 Isa 22:19-\allowbreak22 Lu 15:22}
\crossref{Esth}{8}{3}{1Sa 25:24 2Ki 4:27}
\crossref{Esth}{8}{4}{Es 4:11; 5:2}
\crossref{Esth}{8}{5}{Es 7:3 Ex 33:13,\allowbreak16 1Sa 20:29}
\crossref{Esth}{8}{6}{Ge 44:34 Jer 4:19; 9:1 Lu 19:41,\allowbreak42 Ro 9:2,\allowbreak3; 10:1}
\crossref{Esth}{8}{7}{8:1 Pr 13:22}
\crossref{Esth}{8}{8}{Es 3:12 1Ki 21:8}
\crossref{Esth}{8}{9}{Es 3:12}
\crossref{Esth}{8}{10}{1Ki 21:8 Ec 8:4 Da 4:1}
\crossref{Esth}{8}{11}{Es 9:2-\allowbreak16}
\crossref{Esth}{8}{12}{Es 9:1 Ex 15:9,\allowbreak10 Jud 1:6,\allowbreak7}
\crossref{Esth}{8}{13}{Jud 16:28 Ps 37:14,\allowbreak15; 68:23; 92:10,\allowbreak11; 149:6-\allowbreak9 Lu 18:7 Re 6:10}
\crossref{Esth}{8}{14}{1Sa 21:8 Ec 9:10}
\crossref{Esth}{8}{15}{Es 5:1; 6:8,\allowbreak11 Ge 41:42 Mt 6:29; 11:8 Lu 16:19}
\crossref{Esth}{8}{16}{Es 4:1-\allowbreak3,\allowbreak16 Ps 30:5-\allowbreak11}
\crossref{Esth}{8}{17}{Es 9:17,\allowbreak19,\allowbreak22 1Sa 25:8 Ne 8:10}
\crossref{Esth}{9}{1}{Es 3:7,\allowbreak13; 8:12}
\crossref{Esth}{9}{2}{9:10,\allowbreak16; 8:11}
\crossref{Esth}{9}{3}{Es 3:12; 8:9 Ezr 8:36 Da 3:2; 6:1,\allowbreak2}
\crossref{Esth}{9}{4}{Ps 18:43}
\crossref{Esth}{9}{5}{Ps 18:34-\allowbreak40,\allowbreak47,\allowbreak48; 20:7,\allowbreak8; 149:6-\allowbreak9 2Th 1:6}
\crossref{Esth}{9}{6}{Es 3:15}
\crossref{Esth}{9}{7}{}
\crossref{Esth}{9}{8}{}
\crossref{Esth}{9}{9}{}
\crossref{Esth}{9}{10}{Es 5:11 Ex 20:5 Job 18:18,\allowbreak19; 27:13-\allowbreak15 Ps 21:10; 109:12,\allowbreak13}
\crossref{Esth}{9}{11}{9:11}
\crossref{Esth}{9}{12}{Es 5:6; 7:2}
\crossref{Esth}{9}{13}{Es 8:11}
\crossref{Esth}{9}{14}{}
\crossref{Esth}{9}{15}{9:2,\allowbreak13; 8:11 Ps 118:7-\allowbreak12}
\crossref{Esth}{9}{16}{9:2}
\crossref{Esth}{9}{17}{9:1,\allowbreak18,\allowbreak21; 3:12; 8:9}
\crossref{Esth}{9}{18}{9:1,\allowbreak11,\allowbreak13,\allowbreak15}
\crossref{Esth}{9}{19}{9:22; 8:17 De 16:11,\allowbreak14 Ne 8:10-\allowbreak12 Ps 118:11-\allowbreak16 Lu 11:41 Re 11:10}
\crossref{Esth}{9}{20}{Ex 17:14 De 31:19-\allowbreak22 1Ch 16:12 Ps 124:1-\allowbreak3; 145:4-\allowbreak12}
\crossref{Esth}{9}{21}{}
\crossref{Esth}{9}{22}{Es 3:12,\allowbreak13 Ex 13:3-\allowbreak8 Ps 103:2 Isa 12:1,\allowbreak2; 14:3}
\crossref{Esth}{9}{23}{}
\crossref{Esth}{9}{24}{9:10; 3:5-\allowbreak13}
\crossref{Esth}{9}{25}{9:13,\allowbreak14; 7:5-\allowbreak10; 8:1-\allowbreak14}
\crossref{Esth}{9}{26}{Nu 16:40 Eze 39:11}
\crossref{Esth}{9}{27}{De 5:3; 29:14,\allowbreak15 Jos 9:15 1Sa 30:25 2Sa 21:1,\allowbreak2}
\crossref{Esth}{9}{28}{Ex 12:17 Ps 78:5-\allowbreak7; 103:2}
\crossref{Esth}{9}{29}{Es 3:15}
\crossref{Esth}{9}{30}{Es 1:1; 8:9}
\crossref{Esth}{9}{31}{9:27}
\crossref{Esth}{9}{32}{Es 1:15; 2:20 2Ch 34:27}
\crossref{Esth}{10}{1}{Es 1:1; 8:9 Lu 2:1}
\crossref{Esth}{10}{2}{1Ki 11:41; 22:39}
\crossref{Esth}{10}{3}{Ge 41:44 1Sa 23:17 2Ch 28:7 Da 5:16,\allowbreak29}

% Job
\crossref{Job}{1}{1}{Ge 10:23; 22:20,\allowbreak21}
\crossref{Job}{1}{2}{Job 13:13 Es 5:11 Ps 107:38; 127:3-\allowbreak5; 128:3}
\crossref{Job}{1}{3}{Ge 12:5; 13:6; 34:23 2Ch 32:29}
\crossref{Job}{1}{4}{Ps 133:1 Heb 13:1}
\crossref{Job}{1}{5}{Job 41:25 Ge 35:2,\allowbreak3 Ex 19:10 1Sa 16:5 Ne 12:30 Joh 11:55}
\crossref{Job}{1}{6}{Job 2:1}
\crossref{Job}{1}{7}{Job 2:2 2Ki 5:25}
\crossref{Job}{1}{8}{Job 2:3; 34:14 Eze 40:4}
\crossref{Job}{1}{9}{1:21; 2:10; 21:14,\allowbreak15 Mal 1:10 Mt 16:26 1Ti 4:8; 6:6}
\crossref{Job}{1}{10}{Ge 15:1 De 33:27 1Sa 25:16 Ps 5:12; 34:7; 80:12 Isa 5:2,\allowbreak5}
\crossref{Job}{1}{11}{1:12; 2:5 Isa 5:25}
\crossref{Job}{1}{12}{1Ki 22:23 Lu 8:32; 22:31,\allowbreak32 Joh 19:11 2Co 12:7}
\crossref{Job}{1}{13}{1:4 Pr 27:1 Ec 9:12 Lu 12:19,\allowbreak20; 17:27-\allowbreak29; 21:34}
\crossref{Job}{1}{14}{1Sa 4:17 2Sa 15:13 Jer 51:31}
\crossref{Job}{1}{15}{Ge 10:7,\allowbreak28; 25:3 Ps 72:10 Isa 45:14 Eze 23:42 Joe 3:8}
\crossref{Job}{1}{16}{Ge 19:24 Le 9:24 1Ki 18:38 2Ki 1:10,\allowbreak12,\allowbreak14 Am 7:4 Re 13:13}
\crossref{Job}{1}{17}{Ge 11:28 Isa 23:13 Hab 1:6}
\crossref{Job}{1}{18}{Job 6:2,\allowbreak3; 16:14; 19:9,\allowbreak10; 23:2 Isa 28:19 Jer 51:31 La 1:12}
\crossref{Job}{1}{19}{Jer 4:11,\allowbreak12 Eph 2:2}
\crossref{Job}{1}{20}{Ge 37:29,\allowbreak34 Ezr 9:3}
\crossref{Job}{1}{21}{Ge 3:19 Ps 49:17 Ec 5:15; 12:7 1Ti 6:7}
\crossref{Job}{1}{22}{Job 2:10 Jas 1:4,\allowbreak12 1Pe 1:7}
\crossref{Job}{2}{1}{Job 1:6 Isa 6:1,\allowbreak2 Lu 1:19 Heb 1:14}
\crossref{Job}{2}{2}{Ge 16:8}
\crossref{Job}{2}{3}{Job 1:1,\allowbreak8; 9:20 Ge 6:9 Ps 37:37 Php 3:12 1Pe 5:10}
\crossref{Job}{2}{4}{Es 7:3,\allowbreak4 Isa 2:20,\allowbreak21 Jer 41:8 Mt 6:25; 16:26 Ac 27:18,\allowbreak19}
\crossref{Job}{2}{5}{Job 1:11; 19:20,\allowbreak21 1Ch 21:17 Ps 32:3,\allowbreak4; 38:2-\allowbreak7; 39:10}
\crossref{Job}{2}{6}{Job 1:12}
\crossref{Job}{2}{7}{1Ki 22:22}
\crossref{Job}{2}{8}{Job 19:14-\allowbreak17 Ps 38:5,\allowbreak7 Lu 16:20,\allowbreak21}
\crossref{Job}{2}{9}{Ge 3:6,\allowbreak12 1Ki 11:4}
\crossref{Job}{2}{10}{Ge 3:17 2Sa 19:22 Mt 16:23}
\crossref{Job}{2}{11}{Job 6:14; 16:20; 19:19,\allowbreak21; 42:7 Pr 17:17; 18:24; 27:10}
\crossref{Job}{2}{12}{Job 19:14 Ru 1:19-\allowbreak21 La 4:7,\allowbreak8}
\crossref{Job}{2}{13}{Ezr 9:3 Ne 1:4 Isa 3:26; 47:1}
\crossref{Job}{3}{1}{Job 1:22; 2:10}
\crossref{Job}{3}{2}{Jud 18:14}
\crossref{Job}{3}{3}{}
\crossref{Job}{3}{4}{Ex 10:22,\allowbreak23 Joe 2:2 Am 5:18 Mt 27:45 Ac 27:20 Re 16:10}
\crossref{Job}{3}{5}{Job 10:21,\allowbreak22; 16:16; 24:17; 28:3; 38:17 Ps 23:4; 44:19; 107:10,\allowbreak14}
\crossref{Job}{3}{6}{Ge 49:6 Ps 86:11 Isa 14:20}
\crossref{Job}{3}{7}{Isa 13:20-\allowbreak22; 24:8 Jer 7:34 Re 18:22,\allowbreak23}
\crossref{Job}{3}{8}{2Ch 35:25 Jer 9:17,\allowbreak18 Am 5:16 Mt 11:17 Mr 5:38}
\crossref{Job}{3}{9}{Job 30:26 Jer 8:15; 13:16}
\crossref{Job}{3}{10}{Job 10:18,\allowbreak19 Ge 20:18; 29:31 1Sa 1:5 Ec 6:3-\allowbreak5 Jer 20:17}
\crossref{Job}{3}{11}{Ps 58:8 Jer 15:10 Ho 9:14}
\crossref{Job}{3}{12}{Ge 30:3; 50:23 Isa 66:12 Eze 16:4,\allowbreak5}
\crossref{Job}{3}{13}{Ec 6:3-\allowbreak5; 9:10}
\crossref{Job}{3}{14}{Job 30:23 1Ki 2:10; 11:43 Ps 49:6-\allowbreak10,\allowbreak14; 89:48 Ec 8:8 Isa 14:10-\allowbreak16}
\crossref{Job}{3}{15}{}
\crossref{Job}{3}{16}{Ps 58:8 1Co 15:8}
\crossref{Job}{3}{17}{Job 14:13 Ps 55:5-\allowbreak8 Mt 10:28 Lu 12:4 2Th 1:6,\allowbreak7 2Pe 2:8}
\crossref{Job}{3}{18}{Job 39:7 Ex 5:6-\allowbreak8,\allowbreak15-\allowbreak19 Jud 4:3 Isa 14:3,\allowbreak4}
\crossref{Job}{3}{19}{Job 30:23 Ps 49:2,\allowbreak6-\allowbreak10 Ec 8:8; 12:5,\allowbreak7 Lu 16:22,\allowbreak23 Heb 9:27}
\crossref{Job}{3}{20}{Job 6:9; 7:15,\allowbreak16 Jer 20:18}
\crossref{Job}{3}{21}{Nu 11:15 1Ki 19:4 Jon 4:3,\allowbreak8 Re 9:6}
\crossref{Job}{3}{22}{Ps 43:4; 45:15; 65:12 Pr 23:24 Isa 16:10 Jer 48:33 Da 1:10}
\crossref{Job}{3}{23}{Isa 40:27}
\crossref{Job}{3}{24}{Job 7:19 Ps 80:5; 102:9}
\crossref{Job}{3}{25}{}
\crossref{Job}{3}{26}{Job 27:9 Ps 143:11}
\crossref{Job}{4}{1}{Job 2:11; 15:1; 22:1; 42:9}
\crossref{Job}{4}{2}{2Co 2:4-\allowbreak6; 7:8-\allowbreak10}
\crossref{Job}{4}{3}{Ge 18:19 Pr 10:21; 15:7; 16:21 Isa 50:4 Eph 4:29 Col 4:6}
\crossref{Job}{4}{4}{Ps 145:14 Pr 12:18; 16:23,\allowbreak24 2Co 2:7; 7:6 1Th 5:14}
\crossref{Job}{4}{5}{Job 3:25,\allowbreak26}
\crossref{Job}{4}{6}{Job 1:1,\allowbreak9,\allowbreak10 2Ki 20:3}
\crossref{Job}{4}{7}{Job 9:22,\allowbreak23 Ps 37:25 Ec 7:15; 9:1,\allowbreak2 Ac 28:4 2Pe 2:9}
\crossref{Job}{4}{8}{Ps 7:14-\allowbreak16 Pr 22:8 Jer 4:18 Ho 8:7; 10:12,\allowbreak13 2Co 9:6 Ga 6:7,\allowbreak8}
\crossref{Job}{4}{9}{Ex 15:8,\allowbreak10 2Ki 19:7 Ps 18:15}
\crossref{Job}{4}{10}{Job 29:17 Ps 3:7; 57:4; 58:6 Pr 30:14}
\crossref{Job}{4}{11}{Job 38:39 Ge 49:9 Nu 23:24; 24:9 Ps 7:2 Jer 4:7 Ho 11:10 2Ti 4:17}
\crossref{Job}{4}{12}{Ps 62:11}
\crossref{Job}{4}{13}{Job 33:14-\allowbreak16 Ge 20:3; 28:12; 31:24; 46:2 Nu 12:6; 22:19,\allowbreak20}
\crossref{Job}{4}{14}{Job 7:14 Ps 119:120 Isa 6:5 Da 10:11 Hab 3:16 Lu 1:12,\allowbreak29 Re 1:17}
\crossref{Job}{4}{15}{Ps 104:4 Mt 14:26 Lu 24:37-\allowbreak39 Heb 1:7,\allowbreak14}
\crossref{Job}{4}{16}{1Ki 19:12}
\crossref{Job}{4}{17}{Job 8:3; 9:2; 35:2; 40:8 Ge 18:25 Ps 143:2; 145:17 Ec 7:20 Jer 12:1}
\crossref{Job}{4}{18}{Job 15:15,\allowbreak16; 25:5,\allowbreak6 Ps 103:20,\allowbreak21; 104:4 Isa 6:2,\allowbreak3}
\crossref{Job}{4}{19}{Job 10:9; 13:12; 33:6 Ge 2:7; 3:19; 18:27 Ec 12:7 2Co 4:7; 5:1}
\crossref{Job}{4}{20}{2Ch 15:6}
\crossref{Job}{4}{21}{Ps 39:5,\allowbreak11; 49:14; 146:3,\allowbreak4 Isa 14:16 Lu 16:22,\allowbreak23 Jas 1:11}
\crossref{Job}{5}{1}{Job 15:8-\allowbreak10,\allowbreak15 Isa 41:1,\allowbreak21-\allowbreak23 Heb 12:1}
\crossref{Job}{5}{2}{Job 18:4 Jon 4:9}
\crossref{Job}{5}{3}{Job 27:8 Ps 37:35,\allowbreak36; 73:3-\allowbreak9,\allowbreak18-\allowbreak20; 92:7 Jer 12:1-\allowbreak3}
\crossref{Job}{5}{4}{Job 4:10,\allowbreak11; 8:4; 18:16-\allowbreak19; 27:14 Ex 20:5 Ps 109:9-\allowbreak15; 119:155; 127:5}
\crossref{Job}{5}{5}{De 28:33,\allowbreak51 Jud 6:3-\allowbreak6 Isa 62:8}
\crossref{Job}{5}{6}{Job 34:29 De 32:27 1Sa 6:9 Ps 90:7 Isa 45:7 La 3:38 Am 3:6}
\crossref{Job}{5}{7}{Job 14:1 Ge 3:17-\allowbreak19 Ps 90:8,\allowbreak9 1Co 10:13}
\crossref{Job}{5}{8}{Job 8:5; 22:21,\allowbreak27 Ge 32:7-\allowbreak12 2Ch 33:12,\allowbreak13 Ps 50:15; 77:1,\allowbreak2}
\crossref{Job}{5}{9}{Job 9:10; 11:7-\allowbreak9; 37:5 Ps 40:5; 72:18; 86:10 Ro 11:33}
\crossref{Job}{5}{10}{Job 28:26 Ps 65:9-\allowbreak11; 147:8 Jer 5:24; 10:13; 14:22 Am 4:7 Ac 14:17}
\crossref{Job}{5}{11}{1Sa 2:7,\allowbreak8 Ps 91:14; 107:41 Eze 17:24 Lu 1:52,\allowbreak53}
\crossref{Job}{5}{12}{Job 12:16,\allowbreak17 Ne 4:15 Ps 33:10,\allowbreak11; 37:17 Pr 21:30 Isa 8:10; 19:3}
\crossref{Job}{5}{13}{2Sa 15:31,\allowbreak34; 17:23 Es 6:4-\allowbreak11; 7:10; 9:25 Ps 7:15,\allowbreak16; 9:15,\allowbreak16}
\crossref{Job}{5}{14}{Job 12:25 De 28:29 Pr 4:19 Isa 59:10 Am 8:9}
\crossref{Job}{5}{15}{Ps 10:14,\allowbreak17; 35:10; 72:4,\allowbreak12,\allowbreak13; 107:41; 109:31; 140:12}
\crossref{Job}{5}{16}{1Sa 2:8,\allowbreak9 Ps 9:18 Isa 14:32 Zec 9:12}
\crossref{Job}{5}{17}{Ps 94:12 Pr 3:11,\allowbreak12 Jer 31:18 Heb 12:5-\allowbreak11 Jas 1:12; 5:11}
\crossref{Job}{5}{18}{De 32:39 1Sa 2:6 Ps 147:3 Isa 30:26}
\crossref{Job}{5}{19}{Ps 34:19; 91:3-\allowbreak7 Pr 24:16 1Co 10:13 2Co 1:8 2Pe 2:9}
\crossref{Job}{5}{20}{Ge 45:7 1Ki 17:6 Ps 33:19 Pr 10:3 Isa 33:16 Hab 3:17}
\crossref{Job}{5}{21}{Ps 31:20; 55:21; 57:4 Pr 12:18 Isa 54:17 Jer 18:18 Jas 3:5-\allowbreak8}
\crossref{Job}{5}{22}{2Ki 19:21}
\crossref{Job}{5}{23}{Ps 91:12,\allowbreak13 Ho 2:18 Ro 8:38,\allowbreak39}
\crossref{Job}{5}{24}{Job 18:6,\allowbreak15,\allowbreak21; 21:7-\allowbreak9 1Sa 30:3 Isa 4:5,\allowbreak6}
\crossref{Job}{5}{25}{Job 42:13-\allowbreak16 Ge 15:5 Le 26:9 De 28:4 Ps 112:2; 127:3-\allowbreak5; 128:3-\allowbreak6}
\crossref{Job}{5}{26}{Job 42:16,\allowbreak17 Ge 15:15; 25:8 Ps 91:16 Pr 9:11; 10:27}
\crossref{Job}{5}{27}{Job 8:8-\allowbreak10; 12:2; 15:9,\allowbreak10,\allowbreak17; 32:11,\allowbreak12 Ps 111:2 Pr 2:3-\allowbreak5}
\crossref{Job}{6}{1}{Job 4:1}
\crossref{Job}{6}{2}{Job 4:5; 23:2}
\crossref{Job}{6}{3}{Pr 27:3 Mt 11:28}
\crossref{Job}{6}{4}{Job 16:12-\allowbreak14 De 32:23,\allowbreak42 Ps 7:13; 18:14; 21:12; 38:2; 45:5 La 3:12,\allowbreak13}
\crossref{Job}{6}{5}{Ps 104:14}
\crossref{Job}{6}{6}{6:25; 16:2 Le 2:13 Lu 14:34 Col 4:6}
\crossref{Job}{6}{7}{1Ki 17:12; 22:27 Ps 102:9 Eze 4:14,\allowbreak16; 12:18,\allowbreak19 Da 10:3}
\crossref{Job}{6}{8}{6:11-\allowbreak13; 17:14-\allowbreak16 Ps 119:81}
\crossref{Job}{6}{9}{Job 3:20-\allowbreak22; 7:15,\allowbreak16; 14:13 Nu 11:14,\allowbreak15 1Ki 19:4 Jon 4:3,\allowbreak8 Re 9:6}
\crossref{Job}{6}{10}{Job 3:22; 21:33}
\crossref{Job}{6}{11}{Job 7:5-\allowbreak7; 10:20; 13:25,\allowbreak28; 17:1,\allowbreak14-\allowbreak16 Ps 39:5; 90:5-\allowbreak10; 102:23}
\crossref{Job}{6}{12}{Job 40:18; 41:24}
\crossref{Job}{6}{13}{Job 19:28 2Co 1:12 Ga 6:4}
\crossref{Job}{6}{14}{Job 4:3,\allowbreak4; 16:5; 19:21 Pr 17:17 Ro 12:15 1Co 12:26 2Co 11:29 Ga 6:2}
\crossref{Job}{6}{15}{Job 19:19 Ps 38:11; 41:9; 55:12-\allowbreak14; 88:18 Jer 9:4,\allowbreak5; 30:14 Mic 7:5,\allowbreak6}
\crossref{Job}{6}{16}{Job 5:15; 30:28 Ps 35:14}
\crossref{Job}{6}{17}{}
\crossref{Job}{6}{18}{Isa 3:12}
\crossref{Job}{6}{19}{Ge 25:15 Isa 21:14 Jer 25:23}
\crossref{Job}{6}{20}{Jer 14:3,\allowbreak4; 17:13 Ro 5:5; 9:33}
\crossref{Job}{6}{21}{6:15; 13:4 Ps 62:9 Isa 2:22 Jer 17:5,\allowbreak6}
\crossref{Job}{6}{22}{Job 42:11 1Sa 12:3 Ac 20:33}
\crossref{Job}{6}{23}{Job 5:20 Le 25:48 Ne 5:8 Ps 49:7,\allowbreak8,\allowbreak15; 107:2 Jer 15:21}
\crossref{Job}{6}{24}{Job 5:27; 32:11,\allowbreak15,\allowbreak16; 33:1,\allowbreak31-\allowbreak33; 34:32 Ps 32:8 Pr 9:9; 25:12}
\crossref{Job}{6}{25}{Job 4:4; 16:5 Pr 12:18; 16:21-\allowbreak24; 18:21; 25:11 Ec 12:10,\allowbreak11}
\crossref{Job}{6}{26}{Job 2:10; 3:3-\allowbreak26; 4:3,\allowbreak4; 34:3-\allowbreak9; 38:2; 40:5,\allowbreak8; 42:3,\allowbreak7 Mt 12:37}
\crossref{Job}{6}{27}{Job 22:9; 24:3,\allowbreak9; 29:12; 31:17,\allowbreak21 Ex 22:22-\allowbreak24 Ps 82:3 Pr 23:10,\allowbreak11}
\crossref{Job}{6}{28}{Job 11:3; 13:4}
\crossref{Job}{6}{29}{Job 17:10 Mal 3:18}
\crossref{Job}{6}{30}{Job 33:8-\allowbreak12; 42:3-\allowbreak6}
\crossref{Job}{7}{1}{Job 14:5,\allowbreak13,\allowbreak14 Ps 39:4 Isa 38:5 Joh 11:9,\allowbreak10}
\crossref{Job}{7}{2}{Ps 119:131; 143:6}
\crossref{Job}{7}{3}{Job 29:2 Ps 6:6; 39:5 Ec 1:14}
\crossref{Job}{7}{4}{7:13,\allowbreak14; 17:12; 30:17 De 28:67 Ps 6:6; 77:4; 130:6}
\crossref{Job}{7}{5}{Job 2:7,\allowbreak8; 17:14; 19:26; 24:20; 30:18,\allowbreak19 Ps 38:5-\allowbreak7 Isa 1:6; 14:11}
\crossref{Job}{7}{6}{Job 9:25; 16:22; 17:11 Ps 90:5,\allowbreak6; 102:11; 103:15,\allowbreak16; 144:4}
\crossref{Job}{7}{7}{Job 10:9 Ge 42:36 Ne 1:8 Ps 74:18,\allowbreak22; 89:47,\allowbreak50 Jer 15:15}
\crossref{Job}{7}{8}{Job 20:9 Ps 37:36}
\crossref{Job}{7}{9}{Job 37:11}
\crossref{Job}{7}{10}{Job 8:18; 20:9 Ps 103:16}
\crossref{Job}{7}{11}{Job 6:26; 10:1; 13:13; 16:6; 21:3 Ps 39:3; 40:9}
\crossref{Job}{7}{12}{7:17; 38:6-\allowbreak11 La 3:7}
\crossref{Job}{7}{13}{7:3,\allowbreak4; 9:27,\allowbreak28 Ps 6:6; 77:4}
\crossref{Job}{7}{14}{Ge 40:5-\allowbreak7; 41:8 Jud 7:13,\allowbreak14 Da 2:1 Mt 27:19}
\crossref{Job}{7}{15}{2Sa 17:23 Mt 27:5}
\crossref{Job}{7}{16}{Job 3:20-\allowbreak22; 6:9; 10:1 Ge 27:46 1Ki 19:4 Jon 4:3,\allowbreak8}
\crossref{Job}{7}{17}{Ps 8:4; 144:3 Heb 2:6}
\crossref{Job}{7}{18}{Ex 20:5; 32:34 Isa 26:14; 38:12,\allowbreak13}
\crossref{Job}{7}{19}{Job 9:18 Ps 6:3; 13:1-\allowbreak3; 94:3 Re 6:10}
\crossref{Job}{7}{20}{Job 9:29-\allowbreak31; 13:26; 14:16; 22:5; 31:33; 33:9,\allowbreak27 Ps 80:4}
\crossref{Job}{7}{21}{Job 10:14; 13:23,\allowbreak24 Isa 64:9 La 3:42-\allowbreak44; 5:20-\allowbreak22}
\crossref{Job}{8}{1}{Job 2:11}
\crossref{Job}{8}{2}{Job 11:2,\allowbreak3; 16:3; 18:2; 19:2,\allowbreak3 Ex 10:3,\allowbreak7 Pr 1:22}
\crossref{Job}{8}{3}{Job 4:17; 9:2; 10:3; 19:7; 34:5,\allowbreak12,\allowbreak17-\allowbreak19; 40:8 Ge 18:25 De 32:4}
\crossref{Job}{8}{4}{Job 1:5,\allowbreak18,\allowbreak19; 5:4; 18:16-\allowbreak19 Ge 13:13; 19:13-\allowbreak25}
\crossref{Job}{8}{5}{Job 5:8; 11:13; 22:21-\allowbreak23,\allowbreak24-\allowbreak30 2Ch 33:12,\allowbreak13 Isa 55:6,\allowbreak7 Mt 7:7,\allowbreak8}
\crossref{Job}{8}{6}{Job 1:8; 4:6,\allowbreak7; 21:14,\allowbreak15; 16:17 Ps 26:5,\allowbreak6 Pr 15:8 Isa 1:15 1Ti 2:8}
\crossref{Job}{8}{7}{Job 42:12,\allowbreak13 Pr 4:18 Zec 4:10 Mt 13:12,\allowbreak31,\allowbreak32}
\crossref{Job}{8}{8}{Job 12:12; 15:10,\allowbreak18; 32:6,\allowbreak7 De 4:32; 32:7 Ps 44:1; 78:3,\allowbreak4 Isa 38:19}
\crossref{Job}{8}{9}{Job 7:6 Ge 47:9 1Ch 29:15 Ps 39:5; 90:4; 102:11; 144:4}
\crossref{Job}{8}{10}{Job 12:7,\allowbreak8; 32:7 De 6:7; 11:19 Ps 145:4 Heb 11:4; 12:1}
\crossref{Job}{8}{11}{Ex 2:3 Isa 19:5-\allowbreak7}
\crossref{Job}{8}{12}{Ps 129:6,\allowbreak7 Jer 17:6 Mt 13:20 Jas 1:10,\allowbreak11 1Pe 1:24}
\crossref{Job}{8}{13}{De 6:12; 8:11,\allowbreak14,\allowbreak19 Ps 9:17; 10:4; 50:22 Isa 51:13}
\crossref{Job}{8}{14}{Isa 59:5,\allowbreak6}
\crossref{Job}{8}{15}{Job 18:14; 27:18 Ps 52:5-\allowbreak7; 112:10 Pr 10:28 Mt 7:24-\allowbreak27 Lu 6:47-\allowbreak49}
\crossref{Job}{8}{16}{Job 21:7-\allowbreak15 Ps 37:35,\allowbreak36; 73:3-\allowbreak12}
\crossref{Job}{8}{17}{Job 18:16; 29:19 Isa 5:24; 40:24 Jer 12:1,\allowbreak2 Mr 11:20 Jude 1:12}
\crossref{Job}{8}{18}{Job 7:10; 20:9 Ps 37:10,\allowbreak36; 73:18,\allowbreak19; 92:7}
\crossref{Job}{8}{19}{Job 20:5 Mt 13:20,\allowbreak21}
\crossref{Job}{8}{20}{Job 4:7; 9:22 Ps 37:24,\allowbreak37; 94:14}
\crossref{Job}{8}{21}{Ge 21:6 Ps 126:2,\allowbreak6 Lu 6:21}
\crossref{Job}{8}{22}{Ps 35:26; 109:29; 132:18 1Pe 5:5}
\crossref{Job}{9}{1}{9:1}
\crossref{Job}{9}{2}{Job 4:17; 14:3,\allowbreak4; 25:4; 32:2; 33:9; 34:5 1Ki 8:46 Ps 130:3; 143:2}
\crossref{Job}{9}{3}{9:20,\allowbreak32,\allowbreak33; 10:2; 23:3-\allowbreak7; 31:35-\allowbreak37; 33:13; 34:14,\allowbreak15; 40:2}
\crossref{Job}{9}{4}{9:19; 36:5 Ps 104:24; 136:5 Da 2:20; 4:34-\allowbreak37 Ro 11:33 Eph 1:8,\allowbreak19}
\crossref{Job}{9}{5}{Job 28:9 Ps 46:2; 68:8; 114:6 Isa 40:12 Hab 3:6,\allowbreak10 Zec 4:7 Mt 21:21}
\crossref{Job}{9}{6}{Isa 2:19,\allowbreak21; 13:13,\allowbreak14; 24:1,\allowbreak19,\allowbreak20 Hag 2:6,\allowbreak21 Heb 12:26 Re 20:11}
\crossref{Job}{9}{7}{Ex 10:21,\allowbreak22 Jos 10:12 Da 4:35 Am 4:13; 8:9 Mt 24:29}
\crossref{Job}{9}{8}{Job 37:18 Ge 1:6,\allowbreak7 Ps 33:6; 104:2,\allowbreak3 Isa 40:22; 42:5; 44:24 Jer 10:11}
\crossref{Job}{9}{9}{Job 38:31,\allowbreak32-\allowbreak41 Ge 1:16 Ps 147:4 Am 5:8}
\crossref{Job}{9}{10}{Job 5:9; 26:12-\allowbreak14; 37:23 Ps 71:15; 72:18 Ec 3:11 Isa 40:26-\allowbreak28}
\crossref{Job}{9}{11}{Job 23:8,\allowbreak9; 35:14 Ps 77:19 1Ti 6:16}
\crossref{Job}{9}{12}{Job 23:13; 34:29 Da 4:35 Eph 1:11}
\crossref{Job}{9}{13}{Job 26:12; 40:9-\allowbreak11 Isa 30:7; 31:2,\allowbreak3 Jas 4:6,\allowbreak7}
\crossref{Job}{9}{14}{Job 4:19; 25:6 1Ki 8:27}
\crossref{Job}{9}{15}{Job 10:15 1Co 4:4}
\crossref{Job}{9}{16}{Ps 18:6; 66:18-\allowbreak20; 116:1,\allowbreak2}
\crossref{Job}{9}{17}{Job 16:14 Ps 29:5; 42:7; 83:15 Isa 28:17 Jer 23:19 Eze 13:13}
\crossref{Job}{9}{18}{Job 7:19 Ps 39:13; 88:7,\allowbreak15-\allowbreak18 La 3:3,\allowbreak18}
\crossref{Job}{9}{19}{9:4; 36:17-\allowbreak19; 40:9,\allowbreak10 Ps 62:11 Mt 6:13 1Co 1:25; 10:22}
\crossref{Job}{9}{20}{9:2; 4:17; 32:1,\allowbreak2 Ps 130:3; 143:2 Lu 10:29; 16:15}
\crossref{Job}{9}{21}{Ps 139:23,\allowbreak24 Pr 28:26 Jer 17:9,\allowbreak10 1Co 4:4 1Jo 3:20}
\crossref{Job}{9}{22}{Ec 9:1-\allowbreak3 Eze 21:3,\allowbreak4 Lu 13:2-\allowbreak4}
\crossref{Job}{9}{23}{Job 1:13-\allowbreak19; 2:7}
\crossref{Job}{9}{24}{Job 12:6-\allowbreak10; 21:7-\allowbreak15 Ps 17:14; 73:3-\allowbreak7 Jer 12:1,\allowbreak2 Da 4:17; 5:18-\allowbreak21}
\crossref{Job}{9}{25}{Job 7:6,\allowbreak7 Es 8:14}
\crossref{Job}{9}{26}{}
\crossref{Job}{9}{27}{Job 7:13 Ps 77:2,\allowbreak3 Jer 8:18}
\crossref{Job}{9}{28}{Job 21:6 Ps 88:15,\allowbreak16; 119:120}
\crossref{Job}{9}{29}{9:22; 10:7,\allowbreak14-\allowbreak17; 21:16,\allowbreak17,\allowbreak27; 22:5-\allowbreak30 Ps 73:13 Jer 2:35}
\crossref{Job}{9}{30}{Ps 26:6 Pr 28:13 Isa 1:16-\allowbreak18 Jer 2:22; 4:14 Ro 10:3 1Jo 1:8}
\crossref{Job}{9}{31}{9:20; 15:6}
\crossref{Job}{9}{32}{Job 33:12; 35:5-\allowbreak7 Nu 23:19 1Sa 16:7 Ec 6:10 Isa 45:9 Jer 49:19}
\crossref{Job}{9}{33}{9:19 1Sa 2:25 Ps 106:23 1Jo 2:1,\allowbreak2}
\crossref{Job}{9}{34}{Job 13:11,\allowbreak20-\allowbreak22; 23:15; 31:23; 33:7; 37:1 Ps 39:10; 90:11}
\crossref{Job}{9}{35}{Job 13:22}
\crossref{Job}{10}{1}{Job 3:20-\allowbreak23; 6:8,\allowbreak9; 5:15,\allowbreak16,\allowbreak20; 9:21; 14:13 Nu 11:15 1Ki 19:4}
\crossref{Job}{10}{2}{Ps 6:1-\allowbreak4; 25:7; 38:1-\allowbreak8; 109:21; 143:2 Ro 8:1}
\crossref{Job}{10}{3}{Job 34:5-\allowbreak7,\allowbreak18,\allowbreak19; 36:7-\allowbreak9,\allowbreak17,\allowbreak18; 40:2,\allowbreak8 La 3:2-\allowbreak18}
\crossref{Job}{10}{4}{Job 9:32 1Sa 16:7 Lu 16:15 Re 1:14}
\crossref{Job}{10}{5}{Ps 90:2-\allowbreak4; 102:12,\allowbreak24-\allowbreak27 Heb 1:12 2Pe 3:8}
\crossref{Job}{10}{6}{10:14-\allowbreak17 Ps 10:15; 44:21 Jer 2:34 Zep 1:12 Joh 2:24,\allowbreak25 1Co 4:5}
\crossref{Job}{10}{7}{Job 23:10; 31:6,\allowbreak14,\allowbreak35; 42:7 Ps 1:6; 7:3,\allowbreak8,\allowbreak9; 17:3; 26:1-\allowbreak5}
\crossref{Job}{10}{8}{Ps 119:73 Isa 43:7}
\crossref{Job}{10}{9}{Job 7:7 Ps 25:6,\allowbreak7,\allowbreak18; 89:47; 106:4}
\crossref{Job}{10}{10}{Ps 139:14-\allowbreak16}
\crossref{Job}{10}{11}{2Co 5:2,\allowbreak3}
\crossref{Job}{10}{12}{Ge 19:19 Mt 6:25 Ac 17:25,\allowbreak28}
\crossref{Job}{10}{13}{Job 23:9 Ec 8:6,\allowbreak7 Isa 45:15 Ro 11:33}
\crossref{Job}{10}{14}{Job 13:26,\allowbreak27; 14:16 Ps 130:3; 139:1}
\crossref{Job}{10}{15}{10:7; 9:29; 27:7 Ps 9:17 Isa 3:11; 6:5 Mal 3:18 Ro 2:8,\allowbreak9}
\crossref{Job}{10}{16}{Isa 38:13 La 3:10 Ho 13:7,\allowbreak8 Am 3:8}
\crossref{Job}{10}{17}{Job 16:8 Ru 1:21}
\crossref{Job}{10}{18}{Job 3:10,\allowbreak11 Jer 15:10; 20:14-\allowbreak18 Mt 26:24}
\crossref{Job}{10}{19}{Ps 58:8}
\crossref{Job}{10}{20}{Job 7:6,\allowbreak7,\allowbreak16; 8:9; 9:25,\allowbreak26; 14:1 Ps 39:5; 103:15,\allowbreak16}
\crossref{Job}{10}{21}{Job 7:8-\allowbreak10; 14:10-\allowbreak14 2Sa 12:23; 14:14 Isa 38:11}
\crossref{Job}{10}{22}{}
\crossref{Job}{11}{1}{Job 2:11; 20:1}
\crossref{Job}{11}{2}{Job 16:3; 18:2 Ps 140:11 Pr 10:19 Ac 17:18 Jas 1:19}
\crossref{Job}{11}{3}{Job 13:4; 15:2,\allowbreak3; 24:25}
\crossref{Job}{11}{4}{Job 6:10; 10:7 1Pe 3:15}
\crossref{Job}{11}{5}{Job 23:3-\allowbreak7; 31:35; 33:6-\allowbreak18; 38:1,\allowbreak2; 40:1-\allowbreak5,\allowbreak8; 42:7}
\crossref{Job}{11}{6}{Job 15:8,\allowbreak11; 28:28 De 29:29 Ps 25:14 Da 2:28,\allowbreak47 Mt 13:35}
\crossref{Job}{11}{7}{Job 5:9; 26:14; 37:23 Ps 77:19; 145:3 Ec 3:11 Isa 40:28 Mt 11:27}
\crossref{Job}{11}{8}{Job 22:12; 35:5 2Ch 6:18 Ps 103:11; 148:13 Pr 25:2,\allowbreak3 Isa 55:9}
\crossref{Job}{11}{9}{Job 28:24,\allowbreak25 Ps 65:5-\allowbreak8; 139:9,\allowbreak10}
\crossref{Job}{11}{10}{Job 5:18; 9:4,\allowbreak12,\allowbreak13; 12:14; 34:29 Isa 41:27 Da 4:35}
\crossref{Job}{11}{11}{Ps 94:11 Jer 17:9,\allowbreak10 Joh 2:24,\allowbreak25 Heb 4:13 Re 2:23}
\crossref{Job}{11}{12}{Ps 62:9,\allowbreak10; 73:22; 92:6 Ec 3:18 Ro 1:22 Jas 2:20}
\crossref{Job}{11}{13}{Job 5:8; 8:5,\allowbreak6; 22:21,\allowbreak22 1Sa 7:3 2Ch 12:14; 19:3 Ps 78:8 Lu 12:47}
\crossref{Job}{11}{14}{Job 4:7; 22:5 Isa 1:15}
\crossref{Job}{11}{15}{Job 10:15; 22:26 Ge 4:5,\allowbreak6 Ps 119:6,\allowbreak7 2Co 1:12 1Ti 2:8 1Jo 2:28}
\crossref{Job}{11}{16}{Ge 41:51 Pr 31:7 Ec 5:20 Isa 54:4; 65:16 Joh 16:21 Re 7:14-\allowbreak17}
\crossref{Job}{11}{17}{Job 42:11-\allowbreak17 Ps 37:6; 92:14; 112:4 Pr 4:18 Isa 58:8-\allowbreak10 Mic 7:8,\allowbreak9}
\crossref{Job}{11}{18}{Job 6:11; 7:6; 22:27-\allowbreak29 Ps 43:5 Pr 14:32 Ro 5:3-\allowbreak5 Col 1:27}
\crossref{Job}{11}{19}{Job 42:8,\allowbreak9 Ge 26:26-\allowbreak31 Ps 45:12 Pr 19:6 Isa 60:14 Re 3:9}
\crossref{Job}{11}{20}{Job 31:16 Le 26:16 De 28:65 Ps 69:3 La 4:17}
\crossref{Job}{12}{1}{12:1}
\crossref{Job}{12}{2}{Job 6:24,\allowbreak25; 8:8-\allowbreak10; 11:2,\allowbreak6,\allowbreak12; 15:2; 17:4; 20:3; 32:7-\allowbreak13 Pr 28:11}
\crossref{Job}{12}{3}{Job 13:2-\allowbreak5 Pr 26:4 2Co 11:5,\allowbreak21-\allowbreak23}
\crossref{Job}{12}{4}{Job 11:3; 16:10; 17:2,\allowbreak6; 21:3; 30:1 Ps 22:7,\allowbreak8; 35:16 Mt 27:29}
\crossref{Job}{12}{5}{De 32:35 Ps 17:5; 94:18 Jer 13:16}
\crossref{Job}{12}{6}{Job 9:24; 21:7-\allowbreak15 Ps 17:14; 37:1,\allowbreak35; 73:11,\allowbreak12 Jer 5:27}
\crossref{Job}{12}{7}{Job 21:29,\allowbreak30 Pr 6:6 Isa 1:3 Jer 8:7}
\crossref{Job}{12}{8}{Job 7:11 Jud 5:10 1Ch 16:9}
\crossref{Job}{12}{9}{12:3 Ac 19:35}
\crossref{Job}{12}{10}{Nu 16:22 Da 5:23 Ac 17:25,\allowbreak28}
\crossref{Job}{12}{11}{Job 34:3 1Co 10:15 Php 1:10}
\crossref{Job}{12}{12}{Job 8:8; 15:10; 32:7}
\crossref{Job}{12}{13}{Job 32:6-\allowbreak9}
\crossref{Job}{12}{14}{Job 9:12,\allowbreak13; 11:10 Isa 14:23 Jer 51:58,\allowbreak64 Mal 1:4}
\crossref{Job}{12}{15}{12:10 Ge 8:1,\allowbreak2 1Ki 8:35,\allowbreak36; 17:1 Jer 14:22 Na 1:4 Lu 4:25}
\crossref{Job}{12}{16}{12:13 Mt 6:13}
\crossref{Job}{12}{17}{2Sa 15:31; 17:14,\allowbreak23 Isa 19:12-\allowbreak14; 29:14 1Co 1:19,\allowbreak20}
\crossref{Job}{12}{18}{2Ch 33:11-\allowbreak14 Jer 52:31-\allowbreak34 Da 2:21 Re 19:16}
\crossref{Job}{12}{19}{Jos 10:24,\allowbreak42 1Sa 17:45,\allowbreak46 Isa 37:36-\allowbreak38; 45:1 Re 17:14; 19:19-\allowbreak21}
\crossref{Job}{12}{20}{Pr 10:21; 12:19,\allowbreak22}
\crossref{Job}{12}{21}{Ex 8:2; 16:24 1Ki 21:23,\allowbreak24 2Ki 9:26,\allowbreak34-\allowbreak37 Ps 107:40 Isa 23:9}
\crossref{Job}{12}{22}{Job 11:6; 28:20-\allowbreak23 2Ki 6:12 Ps 44:21; 139:12 Da 2:22 Mt 10:26}
\crossref{Job}{12}{23}{Ex 1:7,\allowbreak20 Ps 107:38 Isa 9:3; 26:15; 27:6; 51:2; 60:22 Jer 30:19}
\crossref{Job}{12}{24}{12:20; 17:4 Isa 6:9,\allowbreak10; 19:1 Da 4:16,\allowbreak33 Ho 7:11}
\crossref{Job}{12}{25}{Job 5:14 Ge 19:11 De 28:29 Isa 59:10 Ac 13:11 1Jo 2:11}
\crossref{Job}{13}{1}{Job 5:9-\allowbreak16; 12:9-\allowbreak25; 42:3-\allowbreak6}
\crossref{Job}{13}{2}{Job 12:3; 15:8,\allowbreak9; 34:35; 35:16; 37:2; 40:4,\allowbreak5; 42:7 1Co 8:1,\allowbreak2}
\crossref{Job}{13}{3}{13:22; 9:34,\allowbreak35; 11:5; 23:3-\allowbreak7; 31:35}
\crossref{Job}{13}{4}{Job 4:7-\allowbreak11; 5:1-\allowbreak5; 8:3,\allowbreak4; 18:5-\allowbreak21; 21:27-\allowbreak34; 22:6-\allowbreak30 Ex 20:16}
\crossref{Job}{13}{5}{13:13; 11:3; 16:3; 18:2; 19:2; 21:2,\allowbreak3; 32:1}
\crossref{Job}{13}{6}{Job 21:2,\allowbreak3; 33:1-\allowbreak3; 34:2 Jud 9:7 Pr 8:6,\allowbreak7}
\crossref{Job}{13}{7}{Job 4:7; 11:2-\allowbreak4; 17:5; 32:21,\allowbreak22; 36:4 Joh 16:2 Ro 3:5-\allowbreak8 2Co 4:2}
\crossref{Job}{13}{8}{Job 32:21; 34:19 Ex 23:2,\allowbreak3 Pr 24:23 Mal 2:9}
\crossref{Job}{13}{9}{Job 34:36 Ps 44:21; 139:23 Jer 17:10}
\crossref{Job}{13}{10}{Job 42:7,\allowbreak8 Ps 50:21,\allowbreak22; 82:2 Jas 2:9}
\crossref{Job}{13}{11}{Ps 119:120 Jer 5:22; 10:10 Mt 10:28 Re 15:3,\allowbreak4}
\crossref{Job}{13}{12}{Job 18:17 Ex 17:14 Ps 34:16; 102:12; 109:15 Pr 10:7 Isa 26:14}
\crossref{Job}{13}{13}{Job 6:9,\allowbreak10; 7:15,\allowbreak16}
\crossref{Job}{13}{14}{Job 18:4 Ec 4:5 Isa 9:20; 49:26}
\crossref{Job}{13}{15}{13:18; 19:25-\allowbreak28; 23:10 Ps 23:4 Pr 14:32 Ro 8:38,\allowbreak39}
\crossref{Job}{13}{16}{Ex 15:2 Ps 27:1; 62:6,\allowbreak7; 118:14,\allowbreak21 Isa 12:2 Jer 3:23 Ac 13:47}
\crossref{Job}{13}{17}{13:6; 33:1}
\crossref{Job}{13}{18}{Job 16:21; 23:4; 40:7}
\crossref{Job}{13}{19}{Job 19:5; 33:5-\allowbreak7,\allowbreak32 Isa 50:7,\allowbreak8 Ro 8:33}
\crossref{Job}{13}{20}{Job 9:34,\allowbreak35}
\crossref{Job}{13}{21}{Job 10:20; 22:15-\allowbreak17}
\crossref{Job}{13}{22}{Job 9:32; 38:3; 40:4,\allowbreak5; 42:3-\allowbreak6}
\crossref{Job}{13}{23}{Job 22:5 Ps 44:20,\allowbreak21}
\crossref{Job}{13}{24}{Job 10:2; 29:2,\allowbreak3 De 32:20 Ps 10:1; 13:1; 44:24; 77:6-\allowbreak9; 88:14 Isa 8:17}
\crossref{Job}{13}{25}{Job 14:3 1Sa 24:14 Isa 17:13 Mt 12:20}
\crossref{Job}{13}{26}{Job 3:20 Ru 1:20 Ps 88:3-\allowbreak18}
\crossref{Job}{13}{27}{Job 33:11 2Ch 16:10-\allowbreak12 Pr 7:22 Ac 16:24}
\crossref{Job}{13}{28}{Job 30:17-\allowbreak19,\allowbreak29,\allowbreak30 Nu 12:12}
\crossref{Job}{14}{1}{Job 15:14; 25:4 Ps 51:5 Mt 11:11}
\crossref{Job}{14}{2}{Ps 90:5-\allowbreak9; 92:7,\allowbreak12; 103:15,\allowbreak16 Isa 40:6-\allowbreak8 Jas 1:10,\allowbreak11; 4:14}
\crossref{Job}{14}{3}{Job 7:17,\allowbreak18; 13:25 Ps 144:3}
\crossref{Job}{14}{4}{Job 15:14; 25:4-\allowbreak6 Ge 5:3 Ps 51:5; 90:5 Joh 3:6 Ro 5:12; 8:8,\allowbreak9}
\crossref{Job}{14}{5}{14:14; 7:1; 12:10 Ps 39:4 Da 5:26,\allowbreak30; 9:24; 11:36 Lu 12:20 Ac 17:26}
\crossref{Job}{14}{6}{Job 7:16,\allowbreak19; 10:20 Ps 39:13}
\crossref{Job}{14}{7}{14:14; 19:10 Isa 11:1; 27:6 Da 4:15,\allowbreak23-\allowbreak25}
\crossref{Job}{14}{8}{Isa 26:19 Joh 12:24 1Co 15:36}
\crossref{Job}{14}{9}{Eze 17:3-\allowbreak10,\allowbreak22-\allowbreak24; 19:10 Ro 11:17-\allowbreak24}
\crossref{Job}{14}{10}{Job 3:11; 10:18; 11:20; 17:13-\allowbreak16 Ge 49:33 Mt 27:50 Ac 5:10}
\crossref{Job}{14}{11}{Job 6:15-\allowbreak18 Jer 15:18}
\crossref{Job}{14}{12}{Job 10:21,\allowbreak22; 30:23 Ec 3:19-\allowbreak21; 12:5}
\crossref{Job}{14}{13}{Job 3:17-\allowbreak19 Isa 57:1,\allowbreak2}
\crossref{Job}{14}{14}{Job 19:25,\allowbreak26 Eze 37:1-\allowbreak14 Mt 22:29-\allowbreak32 Joh 5:28,\allowbreak29 Ac 26:8}
\crossref{Job}{14}{15}{Job 13:22 Ps 50:4,\allowbreak5 1Th 4:17 1Jo 2:28}
\crossref{Job}{14}{16}{Job 10:6,\allowbreak14; 13:27; 31:4; 33:11; 34:21 Ps 56:6; 139:1-\allowbreak4 Pr 5:21}
\crossref{Job}{14}{17}{Job 21:19 De 32:34 Ho 13:12}
\crossref{Job}{14}{18}{Ps 102:25,\allowbreak26 Isa 40:12; 41:15,\allowbreak16; 54:10; 64:1 Jer 4:24 Re 6:14}
\crossref{Job}{14}{19}{Ge 6:17; 7:21-\allowbreak23}
\crossref{Job}{14}{20}{Ec 8:8}
\crossref{Job}{14}{21}{1Sa 4:20 Ps 39:6 Ec 2:18,\allowbreak19; 9:5 Isa 39:7,\allowbreak8; 63:16}
\crossref{Job}{14}{22}{Job 19:20,\allowbreak22,\allowbreak26; 33:19-\allowbreak21}
\crossref{Job}{15}{1}{Job 2:11; 4:1; 22:1; 42:7,\allowbreak9}
\crossref{Job}{15}{2}{Job 11:2,\allowbreak3; 13:2 Jas 3:13}
\crossref{Job}{15}{3}{Job 13:4,\allowbreak5; 16:2,\allowbreak3; 26:1-\allowbreak3 Mal 3:13-\allowbreak15 Mt 12:36,\allowbreak37 Col 4:6}
\crossref{Job}{15}{4}{Job 4:5,\allowbreak6; 6:14 Ps 36:1-\allowbreak3; 119:126 Zep 1:6 Ro 3:31 Ga 2:21}
\crossref{Job}{15}{5}{Job 9:22-\allowbreak24; 12:6 Mr 7:21,\allowbreak22 Lu 6:45 Jas 1:26}
\crossref{Job}{15}{6}{Job 9:20 Ps 64:8 Mt 12:37; 26:65 Lu 19:22}
\crossref{Job}{15}{7}{15:10; 12:12 Ge 4:1}
\crossref{Job}{15}{8}{Job 11:6 De 29:29 Ps 25:14 Pr 3:32 Jer 23:18 Am 3:7 Mt 11:25}
\crossref{Job}{15}{9}{Job 13:2; 26:3,\allowbreak4 2Co 10:7; 11:5,\allowbreak21-\allowbreak30}
\crossref{Job}{15}{10}{Job 8:8-\allowbreak10; 12:20; 32:6,\allowbreak7 De 32:7 Pr 16:31}
\crossref{Job}{15}{11}{Job 5:8-\allowbreak26; 11:13-\allowbreak19 2Co 1:3-\allowbreak5; 7:6}
\crossref{Job}{15}{12}{Ec 11:9 Mr 7:21,\allowbreak22 Ac 5:3,\allowbreak4; 8:22 Jas 1:14,\allowbreak15}
\crossref{Job}{15}{13}{15:25-\allowbreak27; 9:4 Ro 8:7,\allowbreak8}
\crossref{Job}{15}{14}{Job 9:2; 14:4; 25:4-\allowbreak6 1Ki 8:46 2Ch 6:36 Ps 14:3; 51:5 Pr 20:9}
\crossref{Job}{15}{15}{Job 4:18; 25:5 Isa 6:2-\allowbreak5}
\crossref{Job}{15}{16}{Job 4:19; 42:6 Ps 14:1-\allowbreak3; 53:3 Ro 1:28-\allowbreak30; 3:9-\allowbreak19 Tit 3:3}
\crossref{Job}{15}{17}{Job 5:27; 13:5,\allowbreak6; 33:1; 34:2; 36:2}
\crossref{Job}{15}{18}{15:10; 8:8 Ps 71:18; 78:3-\allowbreak6 Isa 38:19}
\crossref{Job}{15}{19}{Ge 10:25,\allowbreak32 De 32:8 Joe 3:17}
\crossref{Job}{15}{20}{Ro 8:22 Ec 9:3}
\crossref{Job}{15}{21}{Job 18:11 Ge 3:9,\allowbreak10 Le 26:36 2Ki 7:6 Pr 1:26,\allowbreak27}
\crossref{Job}{15}{22}{Job 6:11; 9:16 2Ki 6:33 Isa 8:21,\allowbreak22 Mt 27:5}
\crossref{Job}{15}{23}{Job 30:3,\allowbreak4 Ge 4:12 Ps 59:15; 109:10 La 5:6,\allowbreak9 Heb 11:37,\allowbreak38}
\crossref{Job}{15}{24}{Job 6:2-\allowbreak4 Ps 119:143 Pr 1:27 Isa 13:3 Mt 26:37,\allowbreak38 Ro 2:9}
\crossref{Job}{15}{25}{Le 26:23 Ps 73:9,\allowbreak11 Isa 27:4 Da 5:23 Mal 3:13 Ac 9:5; 12:1,\allowbreak23}
\crossref{Job}{15}{26}{2Ch 28:22; 32:13-\allowbreak17}
\crossref{Job}{15}{27}{Job 17:10 De 32:15 Ps 17:10; 73:7; 78:31 Isa 6:10 Jer 5:28}
\crossref{Job}{15}{28}{Job 3:14; 18:15 Isa 5:8-\allowbreak10 Mic 7:18}
\crossref{Job}{15}{29}{Job 20:22-\allowbreak28; 22:15-\allowbreak20; 27:16,\allowbreak17 Ps 49:16,\allowbreak17 Lu 12:19-\allowbreak21; 16:2,\allowbreak19-\allowbreak22}
\crossref{Job}{15}{30}{15:22; 10:21,\allowbreak22; 18:5,\allowbreak6,\allowbreak18 Mt 8:12; 22:13 2Pe 2:17 Jude 1:13}
\crossref{Job}{15}{31}{Job 12:16 Isa 44:20 Ga 6:3,\allowbreak7 Eph 5:6}
\crossref{Job}{15}{32}{Job 22:16 Ps 55:23 Ec 7:17}
\crossref{Job}{15}{33}{Isa 33:9 Re 6:13}
\crossref{Job}{15}{34}{Job 8:13; 20:1; 27:8; 36:13 Isa 33:14,\allowbreak15 Mt 24:51}
\crossref{Job}{15}{35}{Ps 7:14 Isa 59:4,\allowbreak5 Ho 10:13 Ga 6:7,\allowbreak8 Jas 1:15}
\crossref{Job}{16}{1}{16:1}
\crossref{Job}{16}{2}{Job 6:6,\allowbreak25; 11:2,\allowbreak3; 13:5; 19:2,\allowbreak3; 26:2,\allowbreak3 Jas 1:19}
\crossref{Job}{16}{3}{Job 6:26; 8:2; 15:2}
\crossref{Job}{16}{4}{Job 6:2-\allowbreak5,\allowbreak14 Mt 7:12 Ro 12:15 1Co 12:26}
\crossref{Job}{16}{5}{Job 4:3,\allowbreak4; 6:14; 29:25 Ps 27:14 Pr 27:9,\allowbreak17 Isa 35:3,\allowbreak4 Ga 6:1}
\crossref{Job}{16}{6}{Job 10:1 Ps 77:1-\allowbreak9; 88:15-\allowbreak18}
\crossref{Job}{16}{7}{Job 3:17; 7:3,\allowbreak16; 10:1 Ps 6:6,\allowbreak7 Pr 3:11,\allowbreak12 Isa 50:4 Mic 6:13}
\crossref{Job}{16}{8}{Job 10:17 Ru 1:21 Eph 5:27}
\crossref{Job}{16}{9}{Job 10:16,\allowbreak17; 18:4 Ps 50:22 La 3:10 Ho 5:14}
\crossref{Job}{16}{10}{Ps 22:13,\allowbreak16,\allowbreak17; 35:21 Lu 23:35,\allowbreak36}
\crossref{Job}{16}{11}{1Sa 24:18}
\crossref{Job}{16}{12}{Job 1:2,\allowbreak3; 3:26; 29:3,\allowbreak18,\allowbreak19}
\crossref{Job}{16}{13}{Job 6:4 Ge 49:23 Ps 7:12,\allowbreak13}
\crossref{Job}{16}{14}{La 3:3-\allowbreak5}
\crossref{Job}{16}{15}{1Ki 21:27 Isa 22:12}
\crossref{Job}{16}{16}{Ps 6:6,\allowbreak7; 31:9; 32:3; 69:3; 102:3-\allowbreak5,\allowbreak9 Isa 52:14 La 1:16}
\crossref{Job}{16}{17}{Job 11:14; 15:20,\allowbreak34; 21:27,\allowbreak28; 22:5-\allowbreak9; 27:6,\allowbreak7; 29:12-\allowbreak17; 31:1-\allowbreak40}
\crossref{Job}{16}{18}{Jer 22:29}
\crossref{Job}{16}{19}{1Sa 12:5 Ro 1:9; 9:1 2Co 1:23; 11:31 1Th 2:10}
\crossref{Job}{16}{20}{16:4; 12:4,\allowbreak5; 17:2}
\crossref{Job}{16}{21}{Job 9:34,\allowbreak35; 13:3,\allowbreak22; 23:3-\allowbreak7; 31:35; 40:1-\allowbreak5 Ec 6:10 Isa 45:9 Ro 9:20}
\crossref{Job}{16}{22}{Job 14:5,\allowbreak14}
\crossref{Job}{17}{1}{Job 19:17}
\crossref{Job}{17}{2}{Job 12:4; 13:9; 16:20; 21:3 Ps 35:14-\allowbreak16 Mt 27:39-\allowbreak44}
\crossref{Job}{17}{3}{Job 9:33 Ge 43:9; 44:32 Pr 11:15; 20:16 Heb 7:22}
\crossref{Job}{17}{4}{2Sa 15:31; 17:14 2Ch 25:16 Isa 19:14 Mt 11:25; 13:11 Ro 11:8}
\crossref{Job}{17}{5}{Job 32:21,\allowbreak22 Ps 12:2,\allowbreak3 Pr 20:19; 29:5 1Th 2:5}
\crossref{Job}{17}{6}{Job 30:9 1Ki 9:7 Ps 44:14}
\crossref{Job}{17}{7}{Job 16:16 Ps 6:7; 31:9,\allowbreak10 La 5:17}
\crossref{Job}{17}{8}{Ps 73:12-\allowbreak15 Ec 5:8 Hab 1:13 Ro 11:33}
\crossref{Job}{17}{9}{Ps 84:7,\allowbreak11 Pr 4:18; 14:16 Isa 35:8-\allowbreak10 1Pe 1:5 1Jo 2:19}
\crossref{Job}{17}{10}{Job 6:29 Mal 3:18}
\crossref{Job}{17}{11}{Job 7:6; 9:25,\allowbreak26 Isa 38:10}
\crossref{Job}{17}{12}{Job 7:3,\allowbreak4,\allowbreak13,\allowbreak14; 24:14-\allowbreak16 De 28:67}
\crossref{Job}{17}{13}{Job 14:14 Ps 27:14 La 3:25,\allowbreak26}
\crossref{Job}{17}{14}{Job 21:32,\allowbreak33 Ps 16:10; 49:9 Ac 2:27-\allowbreak31; 13:34-\allowbreak37 1Co 15:42,\allowbreak53,\allowbreak54}
\crossref{Job}{17}{15}{Job 4:6; 6:11; 13:15; 19:10}
\crossref{Job}{17}{16}{Job 18:13,\allowbreak14; 33:18-\allowbreak28 Ps 88:4-\allowbreak8; 143:7 Isa 38:17,\allowbreak18 Jon 2:6}
\crossref{Job}{18}{1}{Job 2:11; 8:1; 25:1; 42:7-\allowbreak9}
\crossref{Job}{18}{2}{Job 8:2; 11:2; 13:5,\allowbreak6; 16:2,\allowbreak3}
\crossref{Job}{18}{3}{Job 12:7,\allowbreak8; 17:4,\allowbreak10 Ps 73:22 Ec 3:18 Ro 12:10}
\crossref{Job}{18}{4}{Job 5:2; 13:14; 16:9 Jon 4:9 Mr 9:18 Lu 9:39}
\crossref{Job}{18}{5}{Job 20:5 Pr 4:19; 13:9; 20:20; 24:20}
\crossref{Job}{18}{6}{Job 21:17 Ps 18:28 Re 18:23}
\crossref{Job}{18}{7}{Job 20:22; 36:16 Ps 18:36 Pr 4:12}
\crossref{Job}{18}{8}{Job 22:10 Es 3:9; 6:13; 7:5,\allowbreak10 Ps 9:15; 35:8 Pr 5:22; 29:6 Eze 32:3}
\crossref{Job}{18}{9}{Isa 8:14,\allowbreak15}
\crossref{Job}{18}{10}{Ps 11:6 Eze 12:13 Ro 11:9}
\crossref{Job}{18}{11}{Job 6:4; 15:21; 20:25 Ps 73:19 Jer 6:25; 20:3,\allowbreak4; 46:5; 49:29 2Co 5:11}
\crossref{Job}{18}{12}{Job 15:23,\allowbreak24 1Sa 2:5,\allowbreak36 Ps 34:10; 109:10}
\crossref{Job}{18}{13}{Job 17:16 Jon 2:6}
\crossref{Job}{18}{14}{Job 8:14; 11:20 Ps 112:10 Pr 10:28 Mt 7:26,\allowbreak27}
\crossref{Job}{18}{15}{18:12,\allowbreak13 Zec 5:4}
\crossref{Job}{18}{16}{Job 29:19 Isa 5:24 Ho 9:16 Am 2:9 Mal 4:1}
\crossref{Job}{18}{17}{Job 13:12 Ps 34:16; 83:4; 109:13 Pr 2:22; 10:7}
\crossref{Job}{18}{18}{Job 3:20; 10:22; 11:14 Isa 8:21,\allowbreak22 Jude 1:13}
\crossref{Job}{18}{19}{Job 1:19; 8:4; 42:13-\allowbreak16 Ps 109:13 Isa 14:21,\allowbreak22 Jer 22:30}
\crossref{Job}{18}{20}{De 29:23,\allowbreak24 1Ki 9:8 Jer 18:16}
\crossref{Job}{18}{21}{18:14-\allowbreak16}
\crossref{Job}{19}{1}{19:1}
\crossref{Job}{19}{2}{Job 8:2; 18:2 Ps 13:1 Re 6:10}
\crossref{Job}{19}{3}{Ge 31:7 Le 26:26 Nu 14:22 Ne 4:12 Da 1:20}
\crossref{Job}{19}{4}{Job 11:3-\allowbreak6}
\crossref{Job}{19}{5}{Ps 35:26; 38:16; 41:11; 55:12 Mic 7:8 Zep 2:10 Zec 12:7}
\crossref{Job}{19}{6}{Job 7:20; 16:11-\allowbreak14 Ps 44:9-\allowbreak14; 66:10-\allowbreak12}
\crossref{Job}{19}{7}{Job 10:3,\allowbreak15-\allowbreak17; 16:17-\allowbreak19; 21:27 Ps 22:2 Jer 20:8 La 3:8 Hab 1:2,\allowbreak3}
\crossref{Job}{19}{8}{Job 3:23 Ps 88:8 La 3:7,\allowbreak9 Ho 2:6}
\crossref{Job}{19}{9}{Job 29:7-\allowbreak14,\allowbreak20,\allowbreak21; 30:1 Ps 49:16,\allowbreak17; 89:44 Isa 61:6 Ho 9:11}
\crossref{Job}{19}{10}{Job 1:13-\allowbreak19; 2:7 Ps 88:13-\allowbreak18 La 2:5,\allowbreak6 2Co 4:8,\allowbreak9}
\crossref{Job}{19}{11}{De 32:22 Ps 89:46; 90:7}
\crossref{Job}{19}{12}{Job 16:11 Isa 10:5,\allowbreak6; 51:23}
\crossref{Job}{19}{13}{Ps 31:11; 38:11; 69:8,\allowbreak20; 88:8,\allowbreak18 Mt 26:56 2Ti 4:16}
\crossref{Job}{19}{14}{Ps 38:11 Pr 18:24 Mic 7:5,\allowbreak6 Mt 10:21}
\crossref{Job}{19}{15}{19:16-\allowbreak19}
\crossref{Job}{19}{16}{Job 1:15,\allowbreak16,\allowbreak17,\allowbreak19}
\crossref{Job}{19}{17}{Job 2:9,\allowbreak10; 17:1}
\crossref{Job}{19}{18}{Job 30:1,\allowbreak12 2Ki 2:23 Isa 3:5}
\crossref{Job}{19}{19}{Ps 41:9; 55:12-\allowbreak14,\allowbreak20}
\crossref{Job}{19}{20}{Job 30:30; 33:19-\allowbreak22 Ps 22:14-\allowbreak17; 32:3,\allowbreak4; 38:3; 102:3,\allowbreak5 La 4:8}
\crossref{Job}{19}{21}{Job 6:14 Ro 12:15 1Co 12:26 Heb 13:3}
\crossref{Job}{19}{22}{Job 10:16; 16:13,\allowbreak14 Ps 69:26}
\crossref{Job}{19}{23}{Job 31:35 Isa 8:1; 30:8}
\crossref{Job}{19}{24}{Ex 28:11,\allowbreak12,\allowbreak21; 32:16 De 27:2,\allowbreak3,\allowbreak8 Jer 17:1}
\crossref{Job}{19}{25}{Job 33:23,\allowbreak24 Ps 19:14 Isa 54:5; 59:20,\allowbreak21 Eph 1:7}
\crossref{Job}{19}{26}{Ps 16:9,\allowbreak11 Mt 5:8 1Co 13:12; 15:53 Php 3:21 1Jo 3:2 Re 1:7}
\crossref{Job}{19}{27}{Nu 24:17 Isa 26:19}
\crossref{Job}{19}{28}{19:22 Ps 69:26}
\crossref{Job}{19}{29}{Job 13:7-\allowbreak11 Ro 13:1-\allowbreak4}
\crossref{Job}{20}{1}{Job 2:11; 11:1; 42:9}
\crossref{Job}{20}{2}{20:3; 4:2; 13:19; 32:13-\allowbreak20 Ps 39:2,\allowbreak3 Jer 20:9 Ro 10:2}
\crossref{Job}{20}{3}{Job 19:29}
\crossref{Job}{20}{4}{Job 8:8,\allowbreak9; 15:10; 32:7}
\crossref{Job}{20}{5}{Job 5:3; 15:29-\allowbreak34; 18:5,\allowbreak6; 27:13-\allowbreak23 Ex 15:9,\allowbreak10 Jud 16:21-\allowbreak30}
\crossref{Job}{20}{6}{Ge 11:4 Isa 14:13,\allowbreak14 Da 4:11,\allowbreak22 Am 9:2 Ob 1:3,\allowbreak4 Mt 11:23}
\crossref{Job}{20}{7}{1Ki 14:10 2Ki 9:37 Ps 83:10 Jer 8:2}
\crossref{Job}{20}{8}{Ps 73:20; 18:10; 90:5 Isa 29:7,\allowbreak8}
\crossref{Job}{20}{9}{20:7; 7:8,\allowbreak10; 8:18; 27:3 Ps 37:10,\allowbreak36; 103:15,\allowbreak16}
\crossref{Job}{20}{10}{Pr 28:3}
\crossref{Job}{20}{11}{Job 13:26; 19:20 Ps 25:7 Pr 5:11-\allowbreak13,\allowbreak22,\allowbreak23 Eze 32:27}
\crossref{Job}{20}{12}{Job 15:16 Ge 3:6 Pr 9:17,\allowbreak18; 20:17 Ec 11:9}
\crossref{Job}{20}{13}{Mt 5:29,\allowbreak30 Mr 9:43-\allowbreak49 Ro 8:13}
\crossref{Job}{20}{14}{2Sa 11:2-\allowbreak5; 12:10,\allowbreak11 Ps 32:3,\allowbreak4; 38:1-\allowbreak8; 51:8,\allowbreak9 Pr 1:31}
\crossref{Job}{20}{15}{Pr 23:8 Mt 27:3,\allowbreak4}
\crossref{Job}{20}{16}{Ro 3:13}
\crossref{Job}{20}{17}{Nu 14:23 2Ki 7:2 Jer 17:6-\allowbreak8 Lu 16:24}
\crossref{Job}{20}{18}{20:10,\allowbreak15}
\crossref{Job}{20}{19}{Job 21:27,\allowbreak28; 22:6; 24:2-\allowbreak12; 31:13-\allowbreak22,\allowbreak38,\allowbreak39; 35:9 1Sa 12:3,\allowbreak4 Ps 10:18}
\crossref{Job}{20}{20}{Ec 5:13,\allowbreak14 Isa 57:20,\allowbreak21}
\crossref{Job}{20}{21}{Job 18:19 Jer 17:11 Lu 16:24,\allowbreak25}
\crossref{Job}{20}{22}{Job 15:29; 18:7 Ps 39:5 Ec 2:18-\allowbreak20 Re 18:7}
\crossref{Job}{20}{23}{Nu 11:33 Ps 78:30,\allowbreak31 Mal 2:2 Lu 12:17-\allowbreak20}
\crossref{Job}{20}{24}{1Ki 20:30 Isa 24:18 Jer 48:43,\allowbreak44 Am 5:19; 9:1-\allowbreak3}
\crossref{Job}{20}{25}{Job 16:13 De 32:41 2Sa 18:14 Ps 7:12}
\crossref{Job}{20}{26}{Job 18:5,\allowbreak6 Isa 8:22 Mt 8:12 Jude 1:13}
\crossref{Job}{20}{27}{Ps 44:20,\allowbreak21 Jer 29:23 Mal 3:5 Lu 12:2,\allowbreak3 Ro 2:16 1Co 4:5}
\crossref{Job}{20}{28}{20:10,\allowbreak18-\allowbreak22; 5:5; 27:14-\allowbreak19 2Ki 20:17 Re 18:17}
\crossref{Job}{20}{29}{Job 18:21; 27:13; 31:2,\allowbreak3 De 29:20-\allowbreak28 Ps 11:5,\allowbreak6 Mt 24:51}
\crossref{Job}{21}{1}{21:1}
\crossref{Job}{21}{2}{Job 13:3,\allowbreak4; 18:2; 33:1,\allowbreak31-\allowbreak33; 34:2 Jud 9:7 Isa 55:2 Heb 2:1}
\crossref{Job}{21}{3}{Job 13:13; 33:31-\allowbreak33}
\crossref{Job}{21}{4}{Job 7:11-\allowbreak21; 10:1,\allowbreak2 1Sa 1:16 Ps 22:1-\allowbreak3; 77:3-\allowbreak9; 102:1}
\crossref{Job}{21}{5}{Job 2:12; 17:8; 19:20,\allowbreak21}
\crossref{Job}{21}{6}{Ps 77:3; 88:15; 119:120 La 3:19,\allowbreak20 Hab 3:16}
\crossref{Job}{21}{7}{Job 12:6 Ps 17:10; 73:3-\allowbreak12 Jer 12:1-\allowbreak3 Hab 1:15,\allowbreak16}
\crossref{Job}{21}{8}{Job 5:3,\allowbreak4; 18:19; 20:10,\allowbreak28 Pr 17:6}
\crossref{Job}{21}{9}{Job 15:21; 18:11 Ps 73:19 Isa 57:19-\allowbreak21}
\crossref{Job}{21}{10}{Ex 23:26 De 7:13,\allowbreak14; 28:11 Ps 144:13,\allowbreak14 Ec 9:1,\allowbreak2 Lu 12:16-\allowbreak21}
\crossref{Job}{21}{11}{Ps 107:41; 127:3-\allowbreak5}
\crossref{Job}{21}{12}{Ge 4:21; 31:27 Isa 5:12; 22:13 Am 6:4-\allowbreak6}
\crossref{Job}{21}{13}{Job 36:11 Ps 73:4 Mt 24:38,\allowbreak39 Lu 12:19,\allowbreak20; 17:28,\allowbreak29}
\crossref{Job}{21}{14}{Job 22:17 Ps 10:4,\allowbreak11 Lu 8:28,\allowbreak37 Hab 1:15 Joh 15:23,\allowbreak24 Ro 8:7}
\crossref{Job}{21}{15}{Ex 5:2 Ps 12:4 Pr 30:9 Ho 13:6}
\crossref{Job}{21}{16}{Job 1:21; 12:9,\allowbreak10 Ps 49:6,\allowbreak7; 52:5-\allowbreak7 Ec 8:8 Lu 16:2,\allowbreak25}
\crossref{Job}{21}{17}{Job 18:5,\allowbreak6,\allowbreak18 Pr 13:9; 20:20; 24:20 Mt 25:8}
\crossref{Job}{21}{18}{Job 13:25 Ex 15:7 Ps 1:4; 35:5; 83:13 Isa 5:24; 17:13; 29:5; 40:24}
\crossref{Job}{21}{19}{Job 22:24 De 32:34 Mt 6:19,\allowbreak20 Ro 2:5}
\crossref{Job}{21}{20}{Job 27:19 Lu 16:23}
\crossref{Job}{21}{21}{Job 14:21 Ec 2:18,\allowbreak19 Lu 16:27,\allowbreak28}
\crossref{Job}{21}{22}{Job 40:2 Isa 40:13,\allowbreak14; 45:9 Ro 11:34 1Co 2:16}
\crossref{Job}{21}{23}{}
\crossref{Job}{21}{24}{Job 15:27 Ps 17:10}
\crossref{Job}{21}{25}{Job 3:20; 7:11; 9:18; 10:1 2Sa 17:8}
\crossref{Job}{21}{26}{Job 3:18,\allowbreak19; 20:11 Ec 9:2}
\crossref{Job}{21}{27}{Job 4:8-\allowbreak11; 5:3-\allowbreak5; 8:3-\allowbreak6; 15:20-\allowbreak35; 20:5,\allowbreak29 Lu 5:22}
\crossref{Job}{21}{28}{Job 20:7 Ps 37:36; 52:5,\allowbreak6 Hab 2:9-\allowbreak11 Zec 5:4}
\crossref{Job}{21}{29}{Ps 129:8}
\crossref{Job}{21}{30}{Pr 16:4 Na 1:2 2Pe 2:9-\allowbreak17; 3:7 Jude 1:13}
\crossref{Job}{21}{31}{2Sa 12:7-\allowbreak12 1Ki 21:19-\allowbreak24 Ps 50:21 Jer 2:33-\allowbreak35 Mr 6:18}
\crossref{Job}{21}{32}{Ps 49:14 Eze 32:21-\allowbreak32 Lu 16:22}
\crossref{Job}{21}{33}{Job 3:17,\allowbreak18}
\crossref{Job}{21}{34}{Job 16:2}
\crossref{Job}{22}{1}{22:1}
\crossref{Job}{22}{2}{Job 35:6-\allowbreak8 Ps 16:2 Lu 17:10}
\crossref{Job}{22}{3}{1Ch 29:17 Ps 147:10,\allowbreak11 Pr 11:1,\allowbreak20; 12:22; 15:8 Mal 2:17}
\crossref{Job}{22}{4}{Ps 39:11; 76:6; 80:16 Re 3:19}
\crossref{Job}{22}{5}{Job 4:7-\allowbreak11; 11:14; 15:5,\allowbreak6,\allowbreak31-\allowbreak34; 21:27; 32:3}
\crossref{Job}{22}{6}{Job 24:3,\allowbreak9 Ex 22:26 De 24:10-\allowbreak18 Eze 18:7,\allowbreak12,\allowbreak16 Am 2:8}
\crossref{Job}{22}{7}{Job 31:17 De 15:7-\allowbreak11 Ps 112:9 Pr 11:24,\allowbreak25; 19:17 Isa 58:7,\allowbreak10}
\crossref{Job}{22}{8}{Job 29:7-\allowbreak17; 31:34 1Ki 21:11-\allowbreak15 Ps 12:8 Mic 7:3}
\crossref{Job}{22}{9}{Job 24:3,\allowbreak21; 29:12,\allowbreak13; 31:16-\allowbreak18,\allowbreak21 Ex 22:21-\allowbreak24 De 27:19 Ps 94:6}
\crossref{Job}{22}{10}{Job 18:8-\allowbreak10; 19:6 Ps 11:6}
\crossref{Job}{22}{11}{Job 18:6,\allowbreak18; 19:8 Pr 4:19 Isa 8:22 La 3:2 Joe 2:2,\allowbreak3 Mt 8:12}
\crossref{Job}{22}{12}{Ps 115:3,\allowbreak16 Ec 5:2 Isa 57:15; 66:1}
\crossref{Job}{22}{13}{Ps 10:11; 59:7; 73:11; 94:7-\allowbreak9 Eze 8:12; 9:9 Zep 1:12}
\crossref{Job}{22}{14}{Job 34:22 Ps 33:14; 97:2; 139:1,\allowbreak2,\allowbreak11,\allowbreak12 Jer 23:24 Lu 12:2,\allowbreak3}
\crossref{Job}{22}{15}{Ge 6:5,\allowbreak11-\allowbreak13 Lu 17:26,\allowbreak27}
\crossref{Job}{22}{16}{Job 15:32 Ps 55:23; 102:24 Ec 7:17}
\crossref{Job}{22}{17}{Job 21:10,\allowbreak14,\allowbreak15 Isa 30:11 Mt 8:29,\allowbreak34 Ro 1:28}
\crossref{Job}{22}{18}{Job 12:6 1Sa 2:7 Ps 17:14 Jer 12:2 Ac 14:17; 15:16}
\crossref{Job}{22}{19}{Ps 48:11; 58:10; 97:8; 107:42 Pr 11:10 Re 18:20; 19:1-\allowbreak3}
\crossref{Job}{22}{20}{Job 4:7; 8:3,\allowbreak4; 15:5,\allowbreak6; 20:18,\allowbreak19; 21:27,\allowbreak28 Lu 13:1-\allowbreak5}
\crossref{Job}{22}{21}{1Ch 28:9 Joh 17:3 2Co 4:6}
\crossref{Job}{22}{22}{De 4:1,\allowbreak2 Pr 2:1-\allowbreak9 1Th 4:1,\allowbreak2}
\crossref{Job}{22}{23}{Job 8:5,\allowbreak6; 11:13,\allowbreak14 Isa 55:6,\allowbreak7 Ho 14:1,\allowbreak2 Zec 1:3 Ac 26:20}
\crossref{Job}{22}{24}{1Ki 10:21 2Ch 1:5; 9:10,\allowbreak27}
\crossref{Job}{22}{25}{Ge 15:1 Ps 18:2; 84:11 Isa 41:10 Ro 8:31}
\crossref{Job}{22}{26}{Job 27:10; 34:9 Ps 37:4 So 2:3 Isa 58:14 Ro 7:22}
\crossref{Job}{22}{27}{Ps 50:14,\allowbreak15; 66:17,\allowbreak18-\allowbreak20; 91:15; 116:1 Isa 58:9 1Jo 5:14,\allowbreak15}
\crossref{Job}{22}{28}{Ps 20:4; 90:17 La 3:37 Mt 21:22 Jas 4:15}
\crossref{Job}{22}{29}{Job 5:19-\allowbreak27 Ps 9:2,\allowbreak3; 91:14-\allowbreak16; 92:9-\allowbreak11}
\crossref{Job}{22}{30}{Isa 1:15 Mal 1:9 Mt 17:19,\allowbreak20 Ac 19:15,\allowbreak16 1Ti 2:8 Jas 5:15,\allowbreak16}
\crossref{Job}{23}{1}{23:1}
\crossref{Job}{23}{2}{Job 6:2; 10:1 La 3:19,\allowbreak20 Ps 77:2-\allowbreak9}
\crossref{Job}{23}{3}{Job 13:3; 16:21; 40:1-\allowbreak5 Isa 26:8 Jer 14:7}
\crossref{Job}{23}{4}{Job 13:18; 37:19 Ps 43:1 Isa 43:26}
\crossref{Job}{23}{5}{Job 10:2; 13:22,\allowbreak23; 42:2-\allowbreak6 1Co 4:3,\allowbreak4}
\crossref{Job}{23}{6}{Job 9:19,\allowbreak33,\allowbreak34; 13:21 Isa 27:4,\allowbreak8 Eze 20:33,\allowbreak35}
\crossref{Job}{23}{7}{Isa 1:18 Jer 3:5; 12:1}
\crossref{Job}{23}{8}{Job 9:11 Ps 10:1; 13:1-\allowbreak3 Isa 45:15 1Ti 6:16}
\crossref{Job}{23}{9}{Ps 89:46 Isa 8:17}
\crossref{Job}{23}{10}{Ge 18:19 2Ki 20:3 Ps 1:6; 139:1-\allowbreak3 Joh 21:17 2Ti 2:19}
\crossref{Job}{23}{11}{1Sa 12:2-\allowbreak5 Ps 18:20-\allowbreak24; 44:18 Ac 20:18,\allowbreak19,\allowbreak33,\allowbreak34 2Co 1:12}
\crossref{Job}{23}{12}{Joh 6:66-\allowbreak69; 8:31 Ac 14:22 Heb 10:38,\allowbreak39 1Jo 2:19}
\crossref{Job}{23}{13}{Job 9:12,\allowbreak13; 11:10; 12:14; 34:29 Nu 23:19,\allowbreak20 Ec 1:15; 3:14 Ro 9:19}
\crossref{Job}{23}{14}{Job 7:3 Mic 6:9 1Th 3:3; 5:9 1Pe 2:8}
\crossref{Job}{23}{15}{23:3; 10:15; 31:23 Ps 77:3; 119:120 Hab 3:16}
\crossref{Job}{23}{16}{Ps 22:14 Isa 6:5; 57:16}
\crossref{Job}{23}{17}{Job 6:9 2Ki 22:20 Isa 57:1}
\crossref{Job}{24}{1}{Ps 31:15 Ec 3:17; 8:6,\allowbreak7; 9:11,\allowbreak12 Isa 60:22 Da 2:21 Lu 21:22-\allowbreak24}
\crossref{Job}{24}{2}{De 19:14; 27:17 Pr 22:28; 23:10 Ho 5:10}
\crossref{Job}{24}{3}{Job 22:6-\allowbreak9; 31:16,\allowbreak17 De 24:6,\allowbreak10-\allowbreak13,\allowbreak17-\allowbreak21 1Sa 12:3}
\crossref{Job}{24}{4}{24:14; 31:16 Ps 109:16 Pr 22:16; 30:14 Isa 10:2 Eze 18:12,\allowbreak18; 22:29}
\crossref{Job}{24}{5}{Job 39:5-\allowbreak7 Jer 2:24 Ho 8:9}
\crossref{Job}{24}{6}{De 28:33,\allowbreak51 Jud 6:3-\allowbreak6 Mic 6:15}
\crossref{Job}{24}{7}{24:10; 22:6; 31:19,\allowbreak20 Ex 22:26,\allowbreak27 De 24:11-\allowbreak13 Isa 58:7 Ac 9:31}
\crossref{Job}{24}{8}{So 5:2}
\crossref{Job}{24}{9}{2Ki 4:1 Ne 5:5}
\crossref{Job}{24}{10}{De 24:19 Am 2:7,\allowbreak8; 5:11,\allowbreak12}
\crossref{Job}{24}{11}{De 25:4 Jer 22:13 Jas 5:4}
\crossref{Job}{24}{12}{Ex 1:13,\allowbreak14; 2:23,\allowbreak24; 22:27 Jud 10:16 Ps 12:5 Ec 4:1 Isa 52:5}
\crossref{Job}{24}{13}{Lu 12:47,\allowbreak48 Joh 3:19,\allowbreak20; 9:39-\allowbreak41; 15:22-\allowbreak24 Ro 1:32; 2:17-\allowbreak24}
\crossref{Job}{24}{14}{2Sa 11:14-\allowbreak17 Ps 10:8-\allowbreak10 Mic 2:1,\allowbreak2 Eph 5:7-\allowbreak11}
\crossref{Job}{24}{15}{Ex 20:14 2Sa 11:4-\allowbreak13; 12:12 Ps 50:18 Pr 6:32-\allowbreak35; 7:9,\allowbreak10}
\crossref{Job}{24}{16}{Ex 22:2,\allowbreak3 Eze 12:5-\allowbreak7,\allowbreak12 Mt 24:43}
\crossref{Job}{24}{17}{Job 3:5 Ps 73:18,\allowbreak19 Jer 2:26 2Co 5:10,\allowbreak11 Re 6:16,\allowbreak17}
\crossref{Job}{24}{18}{Ps 58:7; 73:18-\allowbreak20 Isa 23:10}
\crossref{Job}{24}{19}{Job 6:15-\allowbreak17}
\crossref{Job}{24}{20}{Job 17:14; 19:26}
\crossref{Job}{24}{21}{1Sa 1:6,\allowbreak7}
\crossref{Job}{24}{22}{Es 3:8-\allowbreak10 Da 6:4-\allowbreak9 Joh 19:12-\allowbreak16 Re 16:13,\allowbreak14; 17:2}
\crossref{Job}{24}{23}{Ps 73:3-\allowbreak12 Jer 12:1-\allowbreak3}
\crossref{Job}{24}{24}{Job 20:5 Ps 37:10,\allowbreak35,\allowbreak36; 73:19; 92:7 Jas 1:11; 5:1-\allowbreak3}
\crossref{Job}{24}{25}{Job 9:24; 11:2,\allowbreak3; 15:2}
\crossref{Job}{25}{1}{Job 2:11}
\crossref{Job}{25}{2}{Job 9:2-\allowbreak10; 26:5-\allowbreak14; 40:9-\allowbreak14 1Ch 29:11,\allowbreak12 Ps 99:1-\allowbreak3 Jer 10:6,\allowbreak7}
\crossref{Job}{25}{3}{Ps 103:20,\allowbreak21; 148:2-\allowbreak4 Isa 40:26 Da 7:10 Mt 26:53 Re 5:11}
\crossref{Job}{25}{4}{Job 4:17-\allowbreak19; 9:2; 15:14-\allowbreak16 Ps 130:3; 143:2 Ro 3:19,\allowbreak20; 5:1}
\crossref{Job}{25}{5}{Isa 24:23; 60:19,\allowbreak20 2Co 3:10}
\crossref{Job}{25}{6}{}
\crossref{Job}{26}{1}{26:1}
\crossref{Job}{26}{2}{Job 4:3,\allowbreak4; 6:25; 16:4,\allowbreak5 Isa 35:3,\allowbreak4; 40:14; 41:5-\allowbreak7}
\crossref{Job}{26}{3}{Job 6:13; 12:3; 13:5; 15:8-\allowbreak10; 17:10; 32:11-\allowbreak13}
\crossref{Job}{26}{4}{Job 20:3; 32:18 1Ki 22:23,\allowbreak24 Ec 12:7 1Co 12:3 1Jo 4:1-\allowbreak3}
\crossref{Job}{26}{5}{}
\crossref{Job}{26}{6}{Job 11:8 Ps 139:8,\allowbreak11 Pr 15:11 Isa 14:9 Am 9:2 Heb 4:13}
\crossref{Job}{26}{7}{Job 9:8 Ge 1:1,\allowbreak2 Ps 24:2; 104:2-\allowbreak5 Pr 8:23-\allowbreak27 Isa 40:22,\allowbreak26; 42:5}
\crossref{Job}{26}{8}{Job 36:29; 38:9,\allowbreak37 Ge 1:6,\allowbreak7 Ps 135:7 Pr 30:4 Jer 10:13}
\crossref{Job}{26}{9}{Ex 20:21; 33:20-\allowbreak23; 34:3 1Ki 8:12 Ps 97:2 Hab 3:3-\allowbreak5 1Ti 6:16}
\crossref{Job}{26}{10}{Job 38:8-\allowbreak11 Ps 33:7; 104:6-\allowbreak9 Pr 8:29 Jer 5:22}
\crossref{Job}{26}{11}{1Sa 2:8 Ps 18:7 Hag 2:21 Heb 12:26,\allowbreak27 2Pe 3:10 Re 20:11}
\crossref{Job}{26}{12}{Ex 14:21-\allowbreak31 Ps 29:10; 74:13; 93:3,\allowbreak4; 114:2-\allowbreak7 Isa 51:15 Jer 31:35}
\crossref{Job}{26}{13}{Ge 1:2 Ps 33:6,\allowbreak7; 104:30}
\crossref{Job}{26}{14}{Job 11:7-\allowbreak9 Ps 139:6; 145:3 Isa 40:26-\allowbreak29 Ro 11:33 1Co 13:9-\allowbreak12}
\crossref{Job}{27}{1}{Nu 23:7; 24:3,\allowbreak15 Ps 49:4; 78:2 Pr 26:7}
\crossref{Job}{27}{2}{Nu 14:21 Ru 3:13 1Sa 14:39,\allowbreak45; 20:21; 25:26,\allowbreak34 2Sa 2:27}
\crossref{Job}{27}{3}{Ge 2:7 Isa 2:22 Ac 17:25}
\crossref{Job}{27}{4}{Job 13:7; 34:6 Joh 8:55 2Co 11:10}
\crossref{Job}{27}{5}{Job 32:3; 42:7 De 25:1 Pr 17:15 Ga 2:11}
\crossref{Job}{27}{6}{Job 2:3 Ps 18:20-\allowbreak23 Pr 4:13}
\crossref{Job}{27}{7}{1Sa 25:26 2Sa 18:32 Da 4:19}
\crossref{Job}{27}{8}{Job 11:20; 13:16; 15:34; 20:5; 31:3 Isa 33:14,\allowbreak15 Mt 16:26; 23:14}
\crossref{Job}{27}{9}{Job 35:12,\allowbreak13 Ps 18:41; 66:18; 109:7 Pr 1:28; 28:9 Isa 1:15}
\crossref{Job}{27}{10}{Job 22:26,\allowbreak27 Ps 37:4; 43:4 Hab 3:18}
\crossref{Job}{27}{11}{Job 4:3,\allowbreak4; 6:10 Isa 8:11}
\crossref{Job}{27}{12}{Job 21:28-\allowbreak30 Ec 8:14; 9:1-\allowbreak3}
\crossref{Job}{27}{13}{Job 20:29; 31:3 Ps 11:6 Ec 8:13 Isa 3:11 2Pe 2:9}
\crossref{Job}{27}{14}{Job 21:11,\allowbreak12 De 28:32,\allowbreak41 2Ki 9:7,\allowbreak8; 10:6-\allowbreak10 Es 5:11; 9:5-\allowbreak10}
\crossref{Job}{27}{15}{1Ki 14:10,\allowbreak11; 16:3,\allowbreak4; 21:21-\allowbreak24}
\crossref{Job}{27}{16}{Job 22:24 1Ki 10:27 Hab 2:6 Zec 9:3}
\crossref{Job}{27}{17}{Pr 13:22; 28:8 Ec 2:26}
\crossref{Job}{27}{18}{Job 8:14,\allowbreak15 Isa 51:8}
\crossref{Job}{27}{19}{Job 14:13-\allowbreak15; 21:23-\allowbreak26,\allowbreak30; 30:23}
\crossref{Job}{27}{20}{Job 15:21; 18:11; 22:16 Ps 18:4; 42:7; 69:14,\allowbreak15 Jon 2:3}
\crossref{Job}{27}{21}{Jer 18:17 Ho 13:15}
\crossref{Job}{27}{22}{Ex 9:14 De 32:23 Jos 10:11}
\crossref{Job}{27}{23}{Es 9:22-\allowbreak25 Pr 11:10 La 2:15 Re 18:20}
\crossref{Job}{28}{1}{Ge 2:11,\allowbreak12; 23:15; 24:22 1Ki 7:48-\allowbreak50; 10:21 1Ch 29:2-\allowbreak5}
\crossref{Job}{28}{2}{Ge 4:22 Nu 31:22 De 8:9 1Ch 22:14}
\crossref{Job}{28}{3}{Pr 2:4 Ec 1:13 Hab 2:13 Mt 6:33 Lu 16:8}
\crossref{Job}{28}{4}{Job 19:15 Ex 3:22; 12:49}
\crossref{Job}{28}{5}{Ge 1:11,\allowbreak12,\allowbreak29 Ps 104:14,\allowbreak15 Isa 28:25-\allowbreak29}
\crossref{Job}{28}{6}{28:16 Ex 24:10 So 5:14 Isa 54:11 Re 21:19}
\crossref{Job}{28}{7}{28:21-\allowbreak23; 11:6; 38:19,\allowbreak24 Ro 11:33}
\crossref{Job}{28}{8}{Jud 20:43 Ps 25:5,\allowbreak9}
\crossref{Job}{28}{9}{Na 1:4-\allowbreak6}
\crossref{Job}{28}{10}{Pr 14:23; 24:4 Hab 3:9}
\crossref{Job}{28}{11}{Job 26:8 Isa 37:25; 44:27}
\crossref{Job}{28}{12}{28:20,\allowbreak28 1Ki 3:9 Ps 51:6 Pr 2:4-\allowbreak6; 3:19 Ec 7:23-\allowbreak25 1Co 1:19,\allowbreak20}
\crossref{Job}{28}{13}{28:15-\allowbreak19 Ps 19:10; 119:72 Pr 3:14,\allowbreak15; 8:11,\allowbreak18,\allowbreak19; 16:16; 23:23}
\crossref{Job}{28}{14}{Ro 11:33,\allowbreak34}
\crossref{Job}{28}{15}{}
\crossref{Job}{28}{16}{1Ch 29:4 Ps 45:9 Isa 13:12}
\crossref{Job}{28}{17}{Eze 1:22 Re 4:6; 21:11; 22:1}
\crossref{Job}{28}{18}{Eze 27:16}
\crossref{Job}{28}{19}{Ex 28:17; 39:10 Eze 28:13 Re 21:20}
\crossref{Job}{28}{20}{28:12 Pr 2:6 Ec 7:23,\allowbreak24 1Co 2:6-\allowbreak15 Jas 1:5,\allowbreak17}
\crossref{Job}{28}{21}{Ps 49:3,\allowbreak4 Mt 11:25; 13:17,\allowbreak35 1Co 2:7-\allowbreak10 Col 2:3}
\crossref{Job}{28}{22}{28:14 Ps 83:10-\allowbreak12}
\crossref{Job}{28}{23}{Ps 19:7; 147:5 Pr 2:6; 8:14 Mt 11:27 Lu 10:21,\allowbreak22 Ac 15:18}
\crossref{Job}{28}{24}{2Ch 16:9 Pr 15:3 Zec 4:10 Re 5:6}
\crossref{Job}{28}{25}{}
\crossref{Job}{28}{26}{Job 36:26,\allowbreak32; 38:25 Ps 148:8 Jer 14:22 Am 4:7 Zec 10:1}
\crossref{Job}{28}{27}{Ps 19:1 Pr 8:22-\allowbreak29}
\crossref{Job}{28}{28}{De 29:29 Pr 8:4,\allowbreak5,\allowbreak26-\allowbreak32}
\crossref{Job}{29}{1}{Job 27:1}
\crossref{Job}{29}{2}{Job 1:1-\allowbreak5; 7:3}
\crossref{Job}{29}{3}{Job 18:6; 21:17 Ps 18:28 Pr 13:9; 20:20; 24:20}
\crossref{Job}{29}{4}{Job 1:10; 15:8 Ps 25:14; 27:5; 91:1 Pr 3:32 Col 3:3}
\crossref{Job}{29}{5}{Job 23:3,\allowbreak8-\allowbreak10 De 33:27-\allowbreak29 Jos 1:9 Jud 6:12,\allowbreak13 Ps 30:7; 43:2; 44:8,\allowbreak9}
\crossref{Job}{29}{6}{Job 20:17 Ge 49:11 De 32:13; 33:24 Ps 81:16}
\crossref{Job}{29}{7}{De 16:18; 21:19 Ru 4:1,\allowbreak2,\allowbreak11 Zec 8:16}
\crossref{Job}{29}{8}{Le 19:32 Pr 16:31; 20:8 Ro 13:3,\allowbreak4 Tit 3:1 1Pe 5:5}
\crossref{Job}{29}{9}{Job 4:2; 7:11 Pr 10:19 Jas 1:19}
\crossref{Job}{29}{10}{}
\crossref{Job}{29}{11}{Job 31:20 Pr 29:2 Lu 4:22; 11:27}
\crossref{Job}{29}{12}{Job 22:5-\allowbreak9 Ne 5:2-\allowbreak13 Ps 72:12; 82:2-\allowbreak4 Pr 21:13; 24:11,\allowbreak12 Jer 22:16}
\crossref{Job}{29}{13}{De 24:13 Ac 9:39-\allowbreak41 2Co 9:12-\allowbreak14 2Ti 1:16-\allowbreak18}
\crossref{Job}{29}{14}{De 24:13 Ps 132:9 Isa 59:17; 61:10 Ro 13:14 2Co 6:7 Eph 6:14}
\crossref{Job}{29}{15}{Nu 10:31 Mt 11:5 1Co 12:12-\allowbreak31}
\crossref{Job}{29}{16}{Job 31:18 Es 2:7 Ps 68:5 Eph 5:1 Jas 1:27}
\crossref{Job}{29}{17}{Ps 3:7; 58:8 Pr 30:14}
\crossref{Job}{29}{18}{Ps 30:6,\allowbreak7 Jer 22:23; 49:16 Ob 1:4 Hab 2:9}
\crossref{Job}{29}{19}{Job 18:16 Ps 1:3 Jer 17:8 Ho 14:5-\allowbreak7}
\crossref{Job}{29}{20}{29:14; 19:9 Ge 45:13 Ps 3:3}
\crossref{Job}{29}{21}{29:9,\allowbreak10; 32:11,\allowbreak12}
\crossref{Job}{29}{22}{Job 32:15,\allowbreak16; 33:31-\allowbreak33 Isa 52:15 Mt 22:46}
\crossref{Job}{29}{23}{Ps 72:6}
\crossref{Job}{29}{24}{Ge 45:26 Ps 126:1 Lu 24:41}
\crossref{Job}{29}{25}{Ge 41:40 Jud 11:8 2Sa 5:2 1Ch 13:1-\allowbreak4}
\crossref{Job}{30}{1}{Job 19:13-\allowbreak19; 29:8-\allowbreak10 2Ki 2:23 Isa 3:5}
\crossref{Job}{30}{2}{Isa 10:13}
\crossref{Job}{30}{3}{Job 24:13-\allowbreak16}
\crossref{Job}{30}{4}{}
\crossref{Job}{30}{5}{Ge 4:12-\allowbreak14 Ps 109:10 Da 4:25,\allowbreak32,\allowbreak33}
\crossref{Job}{30}{6}{Jud 6:2 1Sa 22:1,\allowbreak2 Isa 2:19 Re 6:15}
\crossref{Job}{30}{7}{Job 6:5; 11:12 Ge 16:12}
\crossref{Job}{30}{8}{2Ki 8:18,\allowbreak27 2Ch 22:3 Ps 49:10-\allowbreak13 Jer 7:18 Mr 6:24}
\crossref{Job}{30}{9}{Job 17:6 Ps 35:15,\allowbreak16; 44:14; 69:12 La 3:14,\allowbreak63}
\crossref{Job}{30}{10}{Job 19:19; 42:6 Ps 88:8 Zec 11:8}
\crossref{Job}{30}{11}{Job 12:18,\allowbreak21 2Sa 16:5-\allowbreak8}
\crossref{Job}{30}{12}{Job 19:18 Isa 3:5}
\crossref{Job}{30}{13}{Ps 69:26 Zec 1:15}
\crossref{Job}{30}{14}{Job 22:16 Ps 18:4; 69:14,\allowbreak15 Isa 8:7,\allowbreak8}
\crossref{Job}{30}{15}{Job 6:4; 7:14; 9:27,\allowbreak28; 10:16 Ps 88:15}
\crossref{Job}{30}{16}{Ps 22:14; 42:4 Isa 53:12}
\crossref{Job}{30}{17}{Job 33:19-\allowbreak21 Ps 6:2-\allowbreak6; 38:2-\allowbreak8}
\crossref{Job}{30}{18}{Job 2:7; 7:5; 19:20 Ps 38:5 Isa 1:5,\allowbreak6}
\crossref{Job}{30}{19}{Job 9:31 Ps 69:1,\allowbreak2 Jer 38:6}
\crossref{Job}{30}{20}{Job 19:7; 27:9 Ps 22:2; 80:4,\allowbreak5 La 3:8,\allowbreak44 Mt 15:23}
\crossref{Job}{30}{21}{Job 7:20,\allowbreak21; 10:14-\allowbreak17; 13:25-\allowbreak28; 16:9-\allowbreak14; 19:6-\allowbreak9 Ps 77:7-\allowbreak9 Jer 30:14}
\crossref{Job}{30}{22}{Job 21:18 Ps 1:4 Isa 17:13 Jer 4:11,\allowbreak12 Eze 5:2 Ho 4:19; 13:3}
\crossref{Job}{30}{23}{Job 14:5; 21:33 Ge 3:19 2Sa 14:14 Ec 8:8; 9:5; 12:5-\allowbreak7 Heb 9:27}
\crossref{Job}{30}{24}{Jud 5:31 Ps 35:25 Mt 27:39-\allowbreak44}
\crossref{Job}{30}{25}{Ps 35:13,\allowbreak14 Jer 13:17; 18:20 Lu 19:41 Joh 11:35 Ro 12:15}
\crossref{Job}{30}{26}{Job 3:25,\allowbreak26; 29:18 Jer 8:15; 14:19; 15:18 Mic 1:12}
\crossref{Job}{30}{27}{Ps 22:4 Jer 4:19; 31:20 La 1:20; 2:11}
\crossref{Job}{30}{28}{Ps 38:6; 42:9; 43:2 Isa 53:3,\allowbreak4 La 3:1-\allowbreak3}
\crossref{Job}{30}{29}{Job 17:14 Ps 102:6 Isa 13:21,\allowbreak22; 38:14 Mic 1:8 Mal 1:3}
\crossref{Job}{30}{30}{Ps 119:83 La 3:4; 4:8; 5:10}
\crossref{Job}{30}{31}{Ps 137:1-\allowbreak4 Ec 3:4 Isa 21:4; 22:12; 24:7-\allowbreak9 La 5:15 Da 6:18}
\crossref{Job}{31}{1}{Ge 6:2 2Sa 11:2-\allowbreak4 Ps 119:37 Pr 4:25; 23:31-\allowbreak33 Mt 5:28,\allowbreak29}
\crossref{Job}{31}{2}{Job 20:29; 27:13 Heb 13:4}
\crossref{Job}{31}{3}{Job 21:30 Ps 55:23; 73:18 Pr 1:27; 10:29; 21:15 Mt 7:13 Ro 9:22}
\crossref{Job}{31}{4}{Job 14:16; 34:21 Ge 16:13 2Ch 16:9 Ps 44:21; 139:1-\allowbreak3 Pr 5:21; 15:3}
\crossref{Job}{31}{5}{Ps 7:3-\allowbreak5}
\crossref{Job}{31}{6}{Jos 22:22 Ps 1:6; 139:23 Mt 7:23 2Ti 2:19}
\crossref{Job}{31}{7}{Ps 44:20,\allowbreak21}
\crossref{Job}{31}{8}{Job 5:5; 24:6 Le 26:16 De 28:30-\allowbreak33,\allowbreak38,\allowbreak51 Jud 6:3-\allowbreak6 Mic 6:15}
\crossref{Job}{31}{9}{Jud 16:5 1Ki 11:4 Ne 13:26 Pr 2:16-\allowbreak19; 5:3-\allowbreak23; 6:25; 7:21; 22:14}
\crossref{Job}{31}{10}{Ex 11:5 Isa 47:2 Mt 24:41}
\crossref{Job}{31}{11}{Ge 20:9; 26:10; 39:9 Ex 20:14 Pr 6:29-\allowbreak33}
\crossref{Job}{31}{12}{Pr 3:33; 6:27 Jer 5:7-\allowbreak9 Mal 3:5 Heb 13:4}
\crossref{Job}{31}{13}{Ex 21:20,\allowbreak21,\allowbreak26,\allowbreak27 Le 25:43,\allowbreak46 De 15:12-\allowbreak15 Jer 34:14-\allowbreak17}
\crossref{Job}{31}{14}{Job 9:32; 10:2 Ps 7:6; 9:12,\allowbreak19; 10:12-\allowbreak15; 44:21; 76:9; 143:2 Isa 10:3}
\crossref{Job}{31}{15}{Job 34:19 Ne 5:5 Pr 14:31; 22:2 Isa 58:7 Mal 2:10}
\crossref{Job}{31}{16}{Job 22:7-\allowbreak9 De 15:7-\allowbreak10 Ps 112:9 Lu 16:21 Ac 11:29 Ga 2:10}
\crossref{Job}{31}{17}{De 15:11,\allowbreak14 Ne 8:10 Lu 11:41 Joh 13:29 Ac 4:32}
\crossref{Job}{31}{18}{Job 13:26 Ge 8:21}
\crossref{Job}{31}{19}{Job 22:6 2Ch 28:15 Isa 58:7 Mt 25:36,\allowbreak43 Lu 3:11 Ac 9:39 Jas 2:16}
\crossref{Job}{31}{20}{Job 29:11 De 24:13}
\crossref{Job}{31}{21}{Job 6:27; 22:9; 24:9; 29:12 Pr 23:10,\allowbreak11 Jer 5:28 Eze 22:7}
\crossref{Job}{31}{22}{31:10,\allowbreak40 Jos 22:22,\allowbreak23 Ps 7:4,\allowbreak5; 137:6}
\crossref{Job}{31}{23}{Job 20:23; 21:20 Ge 39:9 Ps 119:120 Isa 13:6 Joe 1:15 2Co 5:11}
\crossref{Job}{31}{24}{Ge 31:1 De 8:12-\allowbreak14 Ps 49:6,\allowbreak7,\allowbreak17; 52:7; 62:10 Pr 10:15; 11:28}
\crossref{Job}{31}{25}{Es 5:11 Pr 23:5 Jer 9:23 Eze 28:5 Lu 12:19; 16:19,\allowbreak25}
\crossref{Job}{31}{26}{Ge 1:16-\allowbreak18 De 4:19; 11:16; 17:3 2Ki 23:5,\allowbreak11 Jer 8:2 Eze 8:16}
\crossref{Job}{31}{27}{De 11:16; 13:6 Isa 44:20 Ro 1:21,\allowbreak28}
\crossref{Job}{31}{28}{31:11; 9:15; 23:7 Ge 18:25 De 17:2-\allowbreak7,\allowbreak9 Jud 11:27 Ps 50:6 Heb 12:23}
\crossref{Job}{31}{29}{2Sa 1:12; 4:10,\allowbreak11; 16:5-\allowbreak8 Ps 35:13,\allowbreak14,\allowbreak25,\allowbreak26 Pr 17:5; 24:17,\allowbreak18}
\crossref{Job}{31}{30}{Ex 23:4,\allowbreak5 Mt 5:43,\allowbreak44 Ro 12:14 1Pe 2:22,\allowbreak23; 3:9}
\crossref{Job}{31}{31}{1Sa 24:4,\allowbreak10; 26:8 2Sa 16:9,\allowbreak10; 19:21,\allowbreak22 Jer 40:15,\allowbreak16 Lu 9:54,\allowbreak55}
\crossref{Job}{31}{32}{31:17,\allowbreak18 Ge 19:2,\allowbreak3 Jud 19:15,\allowbreak20,\allowbreak21 Isa 58:7 Mt 25:35,\allowbreak40,\allowbreak44,\allowbreak45}
\crossref{Job}{31}{33}{Ge 3:7,\allowbreak8,\allowbreak12 Jos 7:11 Pr 28:13 Ho 6:7 Ac 5:8 1Jo 1:8-\allowbreak10}
\crossref{Job}{31}{34}{Ex 23:2 Pr 29:25 Jer 38:4,\allowbreak5,\allowbreak16,\allowbreak19 Mt 27:20-\allowbreak26}
\crossref{Job}{31}{35}{Job 13:3; 17:3; 23:3-\allowbreak7; 33:6; 38:1-\allowbreak3; 40:4,\allowbreak5}
\crossref{Job}{31}{36}{Ex 28:12 Isa 22:22}
\crossref{Job}{31}{37}{Job 9:3; 13:15; 14:16; 42:3-\allowbreak6 Ps 19:12}
\crossref{Job}{31}{38}{Job 20:27 Hab 2:11 Jas 5:4}
\crossref{Job}{31}{39}{Ge 4:12}
\crossref{Job}{31}{40}{Ps 72:20}
\crossref{Job}{32}{1}{Job 6:29; 10:2,\allowbreak7; 13:15; 23:7; 27:4-\allowbreak6; 29:11-\allowbreak17; 31:1-\allowbreak40; 33:9}
\crossref{Job}{32}{2}{Ps 69:9 Mr 3:5 Eph 4:26}
\crossref{Job}{32}{3}{32:1; 24:25; 25:2-\allowbreak6; 26:2-\allowbreak4}
\crossref{Job}{32}{4}{32:11,\allowbreak12 Pr 18:13}
\crossref{Job}{32}{5}{32:2 Ex 32:19}
\crossref{Job}{32}{6}{Le 19:32 Ro 13:7 1Ti 5:1 Tit 2:6 1Pe 5:5}
\crossref{Job}{32}{7}{Job 8:8-\allowbreak10; 12:12 1Ki 12:6-\allowbreak8 Ps 34:11,\allowbreak12 Pr 1:1-\allowbreak4; 16:31 Heb 5:12}
\crossref{Job}{32}{8}{Job 4:12-\allowbreak21; 33:16; 35:11; 38:36 Ge 41:39 1Ki 3:12,\allowbreak28; 4:29 Pr 2:6}
\crossref{Job}{32}{9}{Jer 5:5 Mt 11:25 Joh 7:48 1Co 1:26,\allowbreak27; 2:7,\allowbreak8 Jas 2:6,\allowbreak7}
\crossref{Job}{32}{10}{1Co 7:25,\allowbreak40}
\crossref{Job}{32}{11}{32:4; 29:21,\allowbreak23}
\crossref{Job}{32}{12}{32:3 1Ti 1:7}
\crossref{Job}{32}{13}{Ge 14:23 Jud 7:2 Isa 48:5,\allowbreak7 Zec 12:7}
\crossref{Job}{32}{14}{Job 6:4; 13:18; 23:4; 33:5}
\crossref{Job}{32}{15}{Job 6:24,\allowbreak25; 29:22 Mt 7:23; 22:22,\allowbreak26,\allowbreak34,\allowbreak46}
\crossref{Job}{32}{16}{Job 13:5 Pr 17:28 Am 5:13 Jas 1:19}
\crossref{Job}{32}{17}{32:10; 33:12; 35:3,\allowbreak4}
\crossref{Job}{32}{18}{Ps 39:3 Jer 20:9 Eze 3:14-\allowbreak27 Ac 4:20 2Co 5:13,\allowbreak14}
\crossref{Job}{32}{19}{Mt 9:17}
\crossref{Job}{32}{20}{Job 13:13,\allowbreak19; 20:2; 21:3}
\crossref{Job}{32}{21}{Job 13:8; 34:19 Le 19:15 De 1:17; 16:19 Pr 24:23 Mt 22:16}
\crossref{Job}{32}{22}{Job 17:5 Ps 12:2,\allowbreak3 Pr 29:5 1Th 2:5 Ga 1:10}
\crossref{Job}{33}{1}{Job 13:6; 34:2 Ps 49:1-\allowbreak3 Mr 4:9}
\crossref{Job}{33}{2}{Job 3:1 Ps 78:2 Mt 5:2}
\crossref{Job}{33}{3}{Job 27:4 Pr 8:7,\allowbreak8 1Th 2:3,\allowbreak4}
\crossref{Job}{33}{4}{Job 10:12; 32:8 Ge 2:7 Ps 33:6 Ro 8:2 1Co 15:45}
\crossref{Job}{33}{5}{33:32,\allowbreak33; 32:1,\allowbreak12}
\crossref{Job}{33}{6}{Job 9:32-\allowbreak35; 13:3; 20:22; 23:3,\allowbreak4; 31:35}
\crossref{Job}{33}{7}{Job 9:34; 13:21 Ps 88:16}
\crossref{Job}{33}{8}{De 13:14 Jer 29:23}
\crossref{Job}{33}{9}{Job 9:17; 10:7; 11:4; 16:17; 23:11,\allowbreak12; 27:5,\allowbreak6; 29:14}
\crossref{Job}{33}{10}{Job 9:30,\allowbreak31; 10:15-\allowbreak17; 13:25; 14:16; 34:5}
\crossref{Job}{33}{11}{Job 13:27 Ps 105:18 Jer 20:2 Ac 16:24}
\crossref{Job}{33}{12}{Job 1:22; 34:10-\allowbreak12,\allowbreak17-\allowbreak19,\allowbreak23; 35:2; 36:22,\allowbreak23 Eze 18:25 Ro 9:19-\allowbreak21}
\crossref{Job}{33}{13}{Job 9:14; 15:25,\allowbreak26 Isa 45:9 Jer 50:24 Eze 22:14 Ac 5:39; 9:4,\allowbreak5}
\crossref{Job}{33}{14}{Job 40:5 Ps 62:11}
\crossref{Job}{33}{15}{Job 4:13 Ge 20:3; 31:24 Nu 12:6 Jer 23:28 Da 4:5 Heb 1:1}
\crossref{Job}{33}{16}{Job 36:10,\allowbreak15 2Sa 7:27 Ps 40:6 Isa 6:10; 48:8; 50:5 Lu 24:45}
\crossref{Job}{33}{17}{Job 17:11 Ge 20:6 Isa 23:9 Ho 2:6 Mt 27:19 Ac 9:2-\allowbreak6}
\crossref{Job}{33}{18}{Ac 16:27-\allowbreak33 Ro 2:4 2Pe 3:9,\allowbreak15}
\crossref{Job}{33}{19}{Job 5:17,\allowbreak18 De 8:5 Ps 94:12; 119:67,\allowbreak71 Isa 27:9 1Co 11:32 Re 3:19}
\crossref{Job}{33}{20}{Ps 107:17,\allowbreak18}
\crossref{Job}{33}{21}{Job 7:5; 13:28; 14:20,\allowbreak22; 19:20 Ps 32:3,\allowbreak4; 39:11; 102:3-\allowbreak5 Pr 5:11}
\crossref{Job}{33}{22}{Job 7:7; 17:1,\allowbreak13-\allowbreak16 1Sa 2:6 Ps 30:3; 88:3-\allowbreak5 Isa 38:10}
\crossref{Job}{33}{23}{Jud 2:1 2Ch 36:15,\allowbreak16 Hag 1:13 Mal 2:7; 3:1 2Co 5:20}
\crossref{Job}{33}{24}{33:18; 22:21 Ex 33:19; 34:6,\allowbreak7 Ps 86:5,\allowbreak15 Ho 14:2,\allowbreak4 Mic 7:18-\allowbreak20}
\crossref{Job}{33}{25}{2Ki 5:14}
\crossref{Job}{33}{26}{2Ki 20:2-\allowbreak5 2Ch 33:12,\allowbreak13,\allowbreak19 Ps 6:1-\allowbreak9; 28:1,\allowbreak2,\allowbreak6; 30:7-\allowbreak11; 41:8-\allowbreak11}
\crossref{Job}{33}{27}{Job 7:20 Nu 12:11 2Sa 12:13 Pr 28:13 Jer 3:13; 31:18,\allowbreak19}
\crossref{Job}{33}{28}{33:20,\allowbreak22; 3:9,\allowbreak16,\allowbreak20 Ps 49:19 Isa 9:2 Joh 11:9}
\crossref{Job}{33}{29}{33:14-\allowbreak17 1Co 12:6 2Co 5:5 Eph 1:11 Php 2:13 Col 1:29 Heb 13:21}
\crossref{Job}{33}{30}{33:24,\allowbreak28 Ps 40:1,\allowbreak2; 118:17,\allowbreak18}
\crossref{Job}{33}{31}{Job 13:6; 18:2; 21:2; 32:11}
\crossref{Job}{33}{32}{Job 15:4,\allowbreak5; 21:27; 22:5-\allowbreak9; 27:5}
\crossref{Job}{33}{33}{Ps 34:11 Pr 4:1,\allowbreak2; 5:1,\allowbreak2}
\crossref{Job}{34}{1}{Job 18:14}
\crossref{Job}{34}{2}{Pr 1:5 1Co 10:15; 14:20}
\crossref{Job}{34}{3}{Job 6:30; 12:11 1Co 2:15 Heb 5:14}
\crossref{Job}{34}{4}{34:36 Jud 19:30; 20:7 1Co 6:2-\allowbreak5 Ga 2:11-\allowbreak14 1Th 5:21}
\crossref{Job}{34}{5}{Job 10:7; 11:4; 16:17; 29:14; 32:1; 33:9}
\crossref{Job}{34}{6}{Job 27:4-\allowbreak6}
\crossref{Job}{34}{7}{Job 15:16 De 29:19 Pr 1:22; 4:17}
\crossref{Job}{34}{8}{Job 2:10; 11:3; 15:5 Ps 1:1; 26:4; 50:18; 73:12-\allowbreak15 Pr 1:15; 2:12; 4:14}
\crossref{Job}{34}{9}{Job 9:22,\allowbreak23,\allowbreak30,\allowbreak31; 21:14-\allowbreak16,\allowbreak30; 22:17; 35:3 Mal 3:14}
\crossref{Job}{34}{10}{34:2,\allowbreak3,\allowbreak34 Pr 6:32; 15:32}
\crossref{Job}{34}{11}{Job 33:26 Ps 62:12 Pr 24:12 Jer 32:19 Eze 33:17-\allowbreak20 Mt 16:27}
\crossref{Job}{34}{12}{Ps 11:7; 145:17 Hab 1:12,\allowbreak13}
\crossref{Job}{34}{13}{Job 36:23; 38:4-\allowbreak41; 40:8-\allowbreak11 1Ch 29:11 Pr 8:23-\allowbreak30 Isa 40:13,\allowbreak14}
\crossref{Job}{34}{14}{Job 7:17; 9:4}
\crossref{Job}{34}{15}{Job 30:23 Ge 3:19 Ps 90:3-\allowbreak10 Ec 12:7 Isa 27:4; 57:16}
\crossref{Job}{34}{16}{Job 12:3; 13:2-\allowbreak6}
\crossref{Job}{34}{17}{Ge 18:25 2Sa 23:3 Ro 3:5-\allowbreak7}
\crossref{Job}{34}{18}{Ex 22:28 Pr 17:26 Ac 23:3,\allowbreak5 Ro 13:7 1Pe 2:17 2Pe 2:10 Jude 1:8}
\crossref{Job}{34}{19}{Job 13:8 De 10:17 2Ch 19:7 Ac 10:34 Ro 2:11 Ga 2:6 Eph 6:9}
\crossref{Job}{34}{20}{Ps 73:19 Isa 30:13; 37:38 Da 5:30 Lu 12:20 Ac 12:23 1Th 5:2}
\crossref{Job}{34}{21}{Job 31:4 Ge 16:13 2Ch 16:9 Ps 34:15; 139:23 Pr 5:21; 15:3 Jer 16:17}
\crossref{Job}{34}{22}{Ps 139:11,\allowbreak12 Isa 29:15 Jer 23:24 Am 9:2,\allowbreak3 1Co 4:5 Heb 4:13}
\crossref{Job}{34}{23}{34:10-\allowbreak12; 11:6 Ezr 9:13 Ps 119:137 Isa 42:3 Da 9:7-\allowbreak9}
\crossref{Job}{34}{24}{Job 19:2 Ps 2:9; 72:4; 94:5 Jer 51:20-\allowbreak23 Da 2:21,\allowbreak34,\allowbreak35,\allowbreak44,\allowbreak45}
\crossref{Job}{34}{25}{Ps 33:15 Isa 66:18 Ho 7:2 Am 8:7 Re 20:12}
\crossref{Job}{34}{26}{Ex 14:30 De 13:9-\allowbreak11; 21:21 2Sa 12:11,\allowbreak12 Ps 58:10,\allowbreak11 Isa 66:24}
\crossref{Job}{34}{27}{1Sa 15:11 Ps 125:5 Zep 1:6 Lu 17:31,\allowbreak32 Ac 15:38 2Ti 4:10}
\crossref{Job}{34}{28}{Job 22:9,\allowbreak10; 24:12; 29:12,\allowbreak13; 31:19,\allowbreak20; 35:9 Ex 2:23,\allowbreak24; 3:7,\allowbreak9 Ps 12:5}
\crossref{Job}{34}{29}{Job 29:1-\allowbreak3 2Sa 7:1 Isa 14:3-\allowbreak8; 26:3; 32:17 Joh 14:27 Ro 8:31-\allowbreak34}
\crossref{Job}{34}{30}{34:21 1Ki 12:28-\allowbreak30 2Ki 21:9 Ps 12:8 Ec 9:18 Ho 5:11; 13:11}
\crossref{Job}{34}{31}{Job 33:27; 40:3-\allowbreak5; 42:6 Le 26:41 Ezr 9:13,\allowbreak14 Ne 9:33-\allowbreak38}
\crossref{Job}{34}{32}{Job 10:2 Ps 19:12; 25:4,\allowbreak5; 32:8; 139:23,\allowbreak24; 143:8-\allowbreak10}
\crossref{Job}{34}{33}{Job 9:12; 18:4 Isa 45:9 Ro 9:20; 11:35}
\crossref{Job}{34}{34}{34:2,\allowbreak4,\allowbreak10,\allowbreak16 1Co 10:15}
\crossref{Job}{34}{35}{Job 13:2; 15:2; 35:16; 38:2; 42:3}
\crossref{Job}{34}{36}{34:8,\allowbreak9; 12:6; 21:7; 24:1}
\crossref{Job}{34}{37}{1Sa 15:23 Isa 1:19,\allowbreak20}
\crossref{Job}{35}{1}{35:1}
\crossref{Job}{35}{2}{Mt 12:36,\allowbreak37 Lu 19:22}
\crossref{Job}{35}{3}{Job 9:21,\allowbreak22; 10:15; 21:15; 31:2; 34:9 Ps 73:13 Mal 3:14}
\crossref{Job}{35}{4}{Job 34:8 Pr 13:20}
\crossref{Job}{35}{5}{Job 22:12; 25:5,\allowbreak6; 36:26-\allowbreak33; 37:1-\allowbreak5,\allowbreak22,\allowbreak23 1Ki 8:27 Ps 8:3,\allowbreak4}
\crossref{Job}{35}{6}{Pr 8:36; 9:12 Jer 7:19}
\crossref{Job}{35}{7}{Job 22:2,\allowbreak3 1Ch 29:14 Ps 16:2 Pr 9:12 Ro 11:35}
\crossref{Job}{35}{8}{Jos 7:1-\allowbreak5; 22:20 Ec 9:18 Jon 1:12}
\crossref{Job}{35}{9}{Job 24:12; 34:28 Ex 2:23; 3:7,\allowbreak9 Ne 5:1-\allowbreak5 Ps 12:5; 43:2; 55:2,\allowbreak3; 56:1,\allowbreak2}
\crossref{Job}{35}{10}{Job 36:13 1Ch 10:13,\allowbreak14 2Ch 28:22,\allowbreak23 Isa 8:21}
\crossref{Job}{35}{11}{Job 32:8 Ge 1:26; 2:7 Ps 94:12}
\crossref{Job}{35}{12}{Ps 18:41 Pr 1:28 Joh 9:31}
\crossref{Job}{35}{13}{Job 22:22-\allowbreak27; 27:8,\allowbreak9 Pr 15:8,\allowbreak29; 28:9 Ec 5:1-\allowbreak3 Isa 1:15 Jer 11:11}
\crossref{Job}{35}{14}{Job 9:11; 23:3,\allowbreak8-\allowbreak10}
\crossref{Job}{35}{15}{Job 9:14; 13:15 Nu 20:12 Lu 1:20}
\crossref{Job}{35}{16}{Job 3:1; 33:2,\allowbreak8-\allowbreak12; 34:35-\allowbreak37; 38:2}
\crossref{Job}{36}{1}{36:1}
\crossref{Job}{36}{2}{Job 21:3; 33:31-\allowbreak33 Heb 13:22}
\crossref{Job}{36}{3}{Job 28:12,\allowbreak13,\allowbreak20-\allowbreak24; 32:8 Pr 2:4,\allowbreak5 Mt 2:1,\allowbreak2; 12:42 Ac 8:27-\allowbreak40}
\crossref{Job}{36}{4}{Job 13:4,\allowbreak7; 21:27,\allowbreak34; 22:6-\allowbreak30}
\crossref{Job}{36}{5}{Job 10:3; 31:13 Ps 22:24; 138:6}
\crossref{Job}{36}{6}{Job 21:7-\allowbreak9,\allowbreak30 Ps 55:23 Jer 12:1,\allowbreak2 2Pe 2:9}
\crossref{Job}{36}{7}{2Ch 16:9 Ps 33:18; 34:15 Zep 3:17 1Pe 3:12}
\crossref{Job}{36}{8}{Job 13:27; 19:6; 33:18,\allowbreak19 Ps 18:5; 107:10; 116:3 La 3:9}
\crossref{Job}{36}{9}{Job 10:2 De 4:21,\allowbreak22 2Ch 33:11-\allowbreak13 Ps 94:12; 119:67,\allowbreak71 La 3:39,\allowbreak40}
\crossref{Job}{36}{10}{36:15; 33:16-\allowbreak23 Ps 40:6 Isa 48:8,\allowbreak17; 50:5 Ac 16:14}
\crossref{Job}{36}{11}{Job 22:21 De 4:30 Jer 7:23; 26:13 Ro 6:17 Heb 11:8}
\crossref{Job}{36}{12}{De 18:15-\allowbreak22; 29:15-\allowbreak20 Isa 1:20; 3:11 Ro 2:8,\allowbreak9}
\crossref{Job}{36}{13}{Nu 32:14 2Ch 28:13,\allowbreak22 Ro 2:5}
\crossref{Job}{36}{14}{Job 15:32; 21:23-\allowbreak25; 22:16 Ge 38:7-\allowbreak10 Le 10:1,\allowbreak2 Ps 55:23}
\crossref{Job}{36}{15}{36:6}
\crossref{Job}{36}{16}{Job 19:8; 42:10-\allowbreak17 Ps 18:19; 31:8; 40:1-\allowbreak3; 118:5}
\crossref{Job}{36}{17}{Job 16:5; 34:8,\allowbreak36 Ro 1:32 Re 18:4}
\crossref{Job}{36}{18}{Ps 2:5,\allowbreak12; 110:5 Mt 3:7 Ro 1:18; 2:5 Eph 5:6}
\crossref{Job}{36}{19}{Pr 10:2; 11:4 Isa 2:20 Zep 1:18 Jas 5:3}
\crossref{Job}{36}{20}{Job 3:20,\allowbreak21; 6:9; 7:15; 14:13; 17:13,\allowbreak14}
\crossref{Job}{36}{21}{Ps 66:18 Eze 14:4 Mt 5:29,\allowbreak30}
\crossref{Job}{36}{22}{1Sa 2:7,\allowbreak8 Ps 75:7 Isa 14:5 Jer 27:5-\allowbreak8 Da 4:25,\allowbreak32; 5:18 Lu 1:52}
\crossref{Job}{36}{23}{Job 34:13-\allowbreak33 Isa 40:13,\allowbreak14 Ro 11:34 1Co 2:16 Eph 1:11}
\crossref{Job}{36}{24}{Job 12:13-\allowbreak25; 26:5-\allowbreak14 Ps 28:5; 34:3; 72:18; 86:8-\allowbreak10; 92:4,\allowbreak5; 104:24}
\crossref{Job}{36}{25}{Job 6:19; 28:24; 35:5; 39:29}
\crossref{Job}{36}{26}{Job 37:5 Ps 145:3}
\crossref{Job}{36}{27}{Job 5:9; 38:25-\allowbreak28,\allowbreak34 Ge 2:5,\allowbreak6 Ps 65:9-\allowbreak13 Isa 5:6 Jer 14:22}
\crossref{Job}{36}{28}{Job 37:11-\allowbreak13 Ge 7:11,\allowbreak12 Pr 3:20}
\crossref{Job}{36}{29}{Job 37:16; 38:9,\allowbreak37 1Ki 18:44,\allowbreak45 Ps 104:3}
\crossref{Job}{36}{30}{Job 38:25,\allowbreak34,\allowbreak35 Lu 17:24}
\crossref{Job}{36}{31}{Job 37:13; 38:22,\allowbreak23 Ge 6:17; 7:17-\allowbreak24; 19:24 Ex 9:23-\allowbreak25 De 8:2,\allowbreak15}
\crossref{Job}{36}{32}{}
\crossref{Job}{36}{33}{36:29; 37:2 2Sa 22:14 1Ki 18:41-\allowbreak45}
\crossref{Job}{37}{1}{Job 4:14; 21:6; 38:1 Ex 19:16 Ps 89:7; 119:120 Jer 5:22 Da 10:7,\allowbreak8}
\crossref{Job}{37}{2}{37:5; 36:29,\allowbreak33; 38:1 Ex 19:16-\allowbreak19 Ps 104:7}
\crossref{Job}{37}{3}{Ps 77:13; 97:4 Mt 24:27 Re 11:19}
\crossref{Job}{37}{4}{Ps 29:3-\allowbreak9; 68:33}
\crossref{Job}{37}{5}{2Sa 22:14,\allowbreak15}
\crossref{Job}{37}{6}{Job 38:22 Ps 147:16-\allowbreak18; 148:8}
\crossref{Job}{37}{7}{Job 5:12; 9:7}
\crossref{Job}{37}{8}{Ps 104:22}
\crossref{Job}{37}{9}{Job 9:9 Ps 104:3}
\crossref{Job}{37}{10}{Job 38:29,\allowbreak30 Ps 78:47; 147:16-\allowbreak18}
\crossref{Job}{37}{11}{Job 36:27,\allowbreak28}
\crossref{Job}{37}{12}{Ps 65:9,\allowbreak10; 104:24 Jer 14:22 Joe 2:23 Am 4:7}
\crossref{Job}{37}{13}{37:6; 36:31; 38:37,\allowbreak38 Ex 9:18-\allowbreak25 1Sa 12:18,\allowbreak19 Ezr 10:9}
\crossref{Job}{37}{14}{Ex 14:13 Ps 46:10 Hab 2:20}
\crossref{Job}{37}{15}{Job 28:24-\allowbreak27; 34:13; 38:4-\allowbreak41 Ps 119:90,\allowbreak91 Isa 40:26}
\crossref{Job}{37}{16}{Job 26:8; 36:29 Ps 104:2,\allowbreak3 Isa 40:22 Jer 10:13}
\crossref{Job}{37}{17}{Job 6:17; 38:31 Ps 147:18 Lu 12:55}
\crossref{Job}{37}{18}{Job 9:8,\allowbreak9 Ge 1:6-\allowbreak8 Ps 104:2; 148:4-\allowbreak6; 150:1 Pr 8:27 Isa 40:12,\allowbreak22}
\crossref{Job}{37}{19}{Job 12:3; 13:3,\allowbreak6}
\crossref{Job}{37}{20}{Ps 139:4 Mt 12:36,\allowbreak37}
\crossref{Job}{37}{21}{Job 26:9; 36:32; 38:25}
\crossref{Job}{37}{22}{Pr 25:23}
\crossref{Job}{37}{23}{37:19; 11:7; 26:14; 36:26 Pr 30:3,\allowbreak4 Ec 3:11 Lu 10:22 Ro 11:33}
\crossref{Job}{37}{24}{Ps 130:4 Jer 32:39; 33:9 Ho 3:5 Mt 10:28 Lu 12:4,\allowbreak5 Ro 2:4}
\crossref{Job}{38}{1}{Job 37:1,\allowbreak2,\allowbreak9,\allowbreak14 Ex 19:16-\allowbreak19 De 4:11,\allowbreak12; 5:22-\allowbreak24 1Ki 19:11}
\crossref{Job}{38}{2}{Job 12:3; 23:4,\allowbreak5; 24:25; 26:3; 27:11; 34:35; 35:16; 42:3 1Ti 1:7}
\crossref{Job}{38}{3}{Job 40:7 Ex 12:11 1Ki 18:46 Jer 1:17 1Pe 1:13}
\crossref{Job}{38}{4}{Pr 8:22,\allowbreak29,\allowbreak30; 30:4}
\crossref{Job}{38}{5}{Job 11:9; 28:25 Pr 8:27 Isa 40:12,\allowbreak22}
\crossref{Job}{38}{6}{Job 26:7 1Sa 2:8 Ps 24:2; 93:1; 104:5 Zec 12:1 2Pe 3:5}
\crossref{Job}{38}{7}{Re 2:28; 22:16}
\crossref{Job}{38}{8}{38:10 Ge 1:9 Ps 33:7; 104:9 Pr 8:29 Jer 5:22}
\crossref{Job}{38}{9}{Ge 1:2}
\crossref{Job}{38}{10}{}
\crossref{Job}{38}{11}{Job 1:22; 2:6 Ps 76:10; 89:9 Isa 27:8 Lu 8:32,\allowbreak33 Re 20:2,\allowbreak3,\allowbreak7,\allowbreak8}
\crossref{Job}{38}{12}{Ge 1:5 Ps 74:16; 136:7,\allowbreak8; 148:3-\allowbreak5}
\crossref{Job}{38}{13}{Ps 19:4-\allowbreak6; 139:9-\allowbreak12}
\crossref{Job}{38}{14}{Ps 104:2,\allowbreak6}
\crossref{Job}{38}{15}{Job 5:14; 18:5,\allowbreak18 Ex 10:21-\allowbreak23 2Ki 6:18 Pr 4:19 Isa 8:21,\allowbreak22}
\crossref{Job}{38}{16}{Ps 77:19 Pr 8:24 Jer 51:36}
\crossref{Job}{38}{17}{Ps 9:13; 107:18; 116:3}
\crossref{Job}{38}{18}{Ps 74:17; 89:11,\allowbreak12 Isa 40:28 Jer 31:37 Re 20:9}
\crossref{Job}{38}{19}{38:12,\allowbreak13 Ge 1:3,\allowbreak4,\allowbreak14-\allowbreak18 De 4:19 Isa 45:7 Joh 1:9; 8:12}
\crossref{Job}{38}{20}{Ge 10:19; 23:17}
\crossref{Job}{38}{21}{38:4,\allowbreak12; 15:7}
\crossref{Job}{38}{22}{Job 6:16; 37:6 Ps 33:7; 135:7}
\crossref{Job}{38}{23}{Job 36:31; 36:13 Ex 9:18,\allowbreak24 Jos 10:11 Isa 30:30 Eze 13:11-\allowbreak13}
\crossref{Job}{38}{24}{38:12,\allowbreak13 Jon 4:8 Mt 24:27}
\crossref{Job}{38}{25}{Job 28:26; 36:27,\allowbreak28; 37:3-\allowbreak6 Ps 29:3-\allowbreak10}
\crossref{Job}{38}{26}{Ps 104:10-\allowbreak14; 107:35; 147:8,\allowbreak9 Isa 35:1,\allowbreak2; 41:18,\allowbreak19; 43:19,\allowbreak20}
\crossref{Job}{38}{27}{}
\crossref{Job}{38}{28}{38:8; 5:9,\allowbreak10 1Sa 12:17,\allowbreak18 Ps 65:9,\allowbreak10 Jer 5:24; 10:13; 14:22}
\crossref{Job}{38}{29}{38:8; 6:16; 37:10 Ps 147:16,\allowbreak17}
\crossref{Job}{38}{30}{Job 37:10}
\crossref{Job}{38}{31}{Job 9:9}
\crossref{Job}{38}{32}{Job 9:9}
\crossref{Job}{38}{33}{Ge 1:16; 8:22 Ps 119:90,\allowbreak91 Jer 31:35,\allowbreak36; 33:25}
\crossref{Job}{38}{34}{1Sa 12:18 Am 5:8 Zec 10:1 Jas 5:18}
\crossref{Job}{38}{35}{Ex 9:23-\allowbreak25,\allowbreak29 Le 10:2 Nu 11:1; 16:35 2Ki 1:10,\allowbreak14 Re 11:5,\allowbreak6}
\crossref{Job}{38}{36}{Job 32:8 Ps 51:6 Pr 2:6 Ec 2:26 Jas 1:5,\allowbreak17}
\crossref{Job}{38}{37}{Ge 15:5 Ps 147:4}
\crossref{Job}{38}{38}{Job 21:33}
\crossref{Job}{38}{39}{Job 4:10,\allowbreak11 Ps 34:10; 104:21; 145:15,\allowbreak16}
\crossref{Job}{38}{40}{Ge 49:9 Nu 23:24; 24:9}
\crossref{Job}{38}{41}{Ps 104:27,\allowbreak28; 147:9 Mt 6:26 Lu 12:24}
\crossref{Job}{39}{1}{1Sa 24:2 Ps 104:18}
\crossref{Job}{39}{2}{Jer 2:24}
\crossref{Job}{39}{3}{Job 31:10 Ge 49:9}
\crossref{Job}{39}{4}{}
\crossref{Job}{39}{5}{Job 6:5; 11:12; 24:5 Ge 16:12 Ps 104:11 Isa 32:14 Jer 2:24; 14:6}
\crossref{Job}{39}{6}{De 29:23 Ps 107:34 Jer 17:6 Eze 47:11}
\crossref{Job}{39}{7}{39:18; 3:18 Isa 31:4}
\crossref{Job}{39}{8}{Job 40:15,\allowbreak20-\allowbreak22 Ge 1:29,\allowbreak30 Ps 104:27,\allowbreak28; 145:15,\allowbreak16}
\crossref{Job}{39}{9}{Nu 23:22 De 33:17 Ps 22:21; 92:10}
\crossref{Job}{39}{10}{39:5,\allowbreak7; 1:14; 41:5 Ps 129:3 Ho 10:10,\allowbreak11 Mic 1:13}
\crossref{Job}{39}{11}{Ps 20:7; 33:16,\allowbreak17; 147:10 Isa 30:16; 31:1-\allowbreak3}
\crossref{Job}{39}{12}{Ne 13:15 Am 2:13}
\crossref{Job}{39}{13}{1Ki 10:22 2Ch 9:21}
\crossref{Job}{39}{14}{De 22:6 Isa 10:14; 59:5}
\crossref{Job}{39}{15}{Isa 59:5}
\crossref{Job}{39}{16}{La 4:3}
\crossref{Job}{39}{17}{Job 17:4; 35:11 De 2:30 2Ch 32:31 Isa 19:11-\allowbreak14; 57:17 Jas 1:17}
\crossref{Job}{39}{18}{39:7,\allowbreak22; 5:22; 41:29 2Ki 19:21}
\crossref{Job}{39}{19}{Ex 15:1 Ps 147:10}
\crossref{Job}{39}{20}{Job 41:20,\allowbreak21 Jer 8:16}
\crossref{Job}{39}{21}{Jud 5:22}
\crossref{Job}{39}{22}{39:16,\allowbreak18; 41:33}
\crossref{Job}{39}{23}{Job 41:26-\allowbreak29}
\crossref{Job}{39}{24}{Job 37:20 Hab 1:8,\allowbreak9}
\crossref{Job}{39}{25}{Ps 70:3 Eze 26:2; 36:2}
\crossref{Job}{39}{26}{}
\crossref{Job}{39}{27}{Ex 19:4 Le 11:13 Ps 103:5 Pr 23:5 Isa 40:31 Ho 8:1}
\crossref{Job}{39}{28}{1Sa 14:4}
\crossref{Job}{39}{29}{Job 9:26}
\crossref{Job}{39}{30}{Eze 39:17-\allowbreak19 Mt 24:28 Lu 17:37}
\crossref{Job}{40}{1}{40:6; 38:1}
\crossref{Job}{40}{2}{Job 9:3; 33:13 Ec 6:10 Isa 45:9-\allowbreak11; 50:8 1Co 10:22}
\crossref{Job}{40}{3}{}
\crossref{Job}{40}{4}{Job 42:6 Ge 18:27; 32:10 2Sa 24:10 1Ki 19:4 Ezr 9:6,\allowbreak15 Ne 9:33}
\crossref{Job}{40}{5}{Job 34:31,\allowbreak32 Ro 3:19}
\crossref{Job}{40}{6}{Job 38:1 Ps 50:3,\allowbreak4 Heb 12:18-\allowbreak20 2Pe 3:10-\allowbreak12}
\crossref{Job}{40}{7}{Job 13:22; 23:3,\allowbreak4; 38:3}
\crossref{Job}{40}{8}{Ps 51:4 Ro 3:4}
\crossref{Job}{40}{9}{Job 9:4; 23:6; 33:12,\allowbreak13 Ex 15:6 Ps 89:10,\allowbreak13 Isa 45:9 1Co 10:22}
\crossref{Job}{40}{10}{Job 39:19 Ps 93:1; 104:1,\allowbreak2 Isa 59:17}
\crossref{Job}{40}{11}{Job 20:23; 27:22 De 32:22 Ps 78:49,\allowbreak50; 144:6 Ro 2:8,\allowbreak9}
\crossref{Job}{40}{12}{Ps 60:12 Pr 15:25 Isa 10:6 Zec 10:5 Mal 4:3 Ro 16:20}
\crossref{Job}{40}{13}{Job 14:13 Ps 49:14 Isa 2:10}
\crossref{Job}{40}{14}{Ps 44:3,\allowbreak6 Isa 40:29 Ro 5:6 Eph 2:4-\allowbreak9}
\crossref{Job}{40}{15}{Ge 1:24-\allowbreak26}
\crossref{Job}{40}{16}{}
\crossref{Job}{40}{17}{Job 41:23}
\crossref{Job}{40}{18}{Job 7:12 Isa 48:4}
\crossref{Job}{40}{19}{Job 26:13 Ps 104:24}
\crossref{Job}{40}{20}{40:15 Ps 147:8,\allowbreak9}
\crossref{Job}{40}{21}{Isa 19:6,\allowbreak7; 35:7}
\crossref{Job}{40}{22}{Le 23:40 Isa 15:7 Eze 17:5}
\crossref{Job}{40}{23}{Isa 37:25}
\crossref{Job}{40}{24}{}
\crossref{Job}{41}{1}{Job 3:8}
\crossref{Job}{41}{2}{Isa 27:1; 37:29 Eze 29:4,\allowbreak5}
\crossref{Job}{41}{3}{Ps 55:21 Pr 15:1; 18:23; 25:15 Isa 30:10}
\crossref{Job}{41}{4}{1Ki 20:31-\allowbreak34}
\crossref{Job}{41}{5}{Jud 16:25-\allowbreak30}
\crossref{Job}{41}{6}{Jud 14:11}
\crossref{Job}{41}{7}{41:26-\allowbreak29}
\crossref{Job}{41}{8}{1Ki 20:11 2Ki 10:4 Lu 14:31,\allowbreak32}
\crossref{Job}{41}{9}{De 28:34 1Sa 3:11 Isa 28:19 Lu 21:11}
\crossref{Job}{41}{10}{Ge 49:9 Nu 24:9 Ps 2:11,\allowbreak12 Eze 8:17,\allowbreak18}
\crossref{Job}{41}{11}{Job 22:2,\allowbreak3; 35:7 Ps 21:3 Ro 11:35}
\crossref{Job}{41}{12}{Ge 1:25}
\crossref{Job}{41}{13}{2Ki 19:28 Ps 32:9 Jas 3:3}
\crossref{Job}{41}{14}{Job 38:10 Ec 12:4}
\crossref{Job}{41}{15}{Jer 9:23}
\crossref{Job}{41}{16}{Ge 18:23; 19:9}
\crossref{Job}{41}{17}{41:23; 19:20; 29:10; 31:7}
\crossref{Job}{41}{18}{Job 3:9}
\crossref{Job}{41}{19}{Ps 18:8}
\crossref{Job}{41}{20}{Jer 1:13,\allowbreak14}
\crossref{Job}{41}{21}{Ps 18:8,\allowbreak12 Isa 30:33 Hab 3:5}
\crossref{Job}{41}{22}{Job 39:19; 40:16}
\crossref{Job}{41}{23}{41:17}
\crossref{Job}{41}{24}{Isa 48:4 Jer 5:3 Zec 7:12}
\crossref{Job}{41}{25}{Ps 107:28 Jon 1:4-\allowbreak6}
\crossref{Job}{41}{26}{Job 39:21-\allowbreak24}
\crossref{Job}{41}{27}{Job 6:26; 13:24; 19:11}
\crossref{Job}{41}{28}{Job 39:7 Hab 1:10}
\crossref{Job}{41}{29}{2Ch 26:14}
\crossref{Job}{41}{30}{Job 17:13}
\crossref{Job}{41}{31}{Job 30:27 Eze 24:5}
\crossref{Job}{41}{32}{Ge 1:15}
\crossref{Job}{41}{33}{41:24}
\crossref{Job}{41}{34}{Job 26:12 Ex 5:2 Ps 73:6,\allowbreak10 Isa 28:1 Eze 29:3 Re 12:1-\allowbreak3; 13:2}
\crossref{Job}{42}{1}{42:1}
\crossref{Job}{42}{2}{Ge 18:14 Isa 43:13 Jer 32:17 Mt 19:26 Mr 10:27; 14:36 Lu 18:27}
\crossref{Job}{42}{3}{Job 38:2}
\crossref{Job}{42}{4}{Ge 18:27,\allowbreak30-\allowbreak32}
\crossref{Job}{42}{5}{Job 4:12; 28:22; 33:16 Ro 10:17}
\crossref{Job}{42}{6}{Job 9:31; 40:3,\allowbreak4 Ezr 9:6 Ps 51:17 Isa 5:5 Jer 31:19 Eze 16:63}
\crossref{Job}{42}{7}{Job 2:11; 4:1; 8:1; 11:1}
\crossref{Job}{42}{8}{Nu 23:1,\allowbreak14,\allowbreak29 1Ch 15:26 2Ch 29:21 Eze 45:23 Heb 10:4,\allowbreak10-\allowbreak14}
\crossref{Job}{42}{9}{Job 34:31,\allowbreak32 Isa 60:14 Mt 7:24 Joh 2:5 Ac 9:6; 10:33 Heb 11:8}
\crossref{Job}{42}{10}{Job 5:18-\allowbreak20 De 30:3 Ps 14:7; 53:6; 126:1,\allowbreak4}
\crossref{Job}{42}{11}{Job 19:13,\allowbreak14 Pr 16:7}
\crossref{Job}{42}{12}{Job 8:7 De 8:16 Pr 10:22 Ec 7:8 1Ti 6:17 Jas 5:11}
\crossref{Job}{42}{13}{Job 1:2 Ps 107:41; 127:3 Isa 49:20}
\crossref{Job}{42}{14}{Ps 45:9}
\crossref{Job}{42}{15}{Ps 144:12 Ac 7:20}
\crossref{Job}{42}{16}{Ge 11:32; 25:7; 35:28; 47:28; 50:26 De 34:7 Jos 24:29 Ps 90:10}
\crossref{Job}{42}{17}{Job 5:26 Ge 15:15; 25:8 De 6:2 Ps 91:16 Pr 3:16}

% Ps
\crossref{Ps}{1}{1}{Ps 2:12; 32:1,\allowbreak2; 34:8; 84:12; 106:3; 112:1; 115:12-\allowbreak15; 119:1,\allowbreak2; 144:15}
\crossref{Ps}{1}{2}{Ps 40:8; 112:1; 119:11,\allowbreak35,\allowbreak47,\allowbreak48,\allowbreak72,\allowbreak92 Job 23:12 Jer 15:16 Ro 7:22}
\crossref{Ps}{1}{3}{Job 14:9 Isa 44:4 Jer 17:8 Eze 17:8; 19:10; 47:12 Re 22:2}
\crossref{Ps}{1}{4}{Ps 35:5 Job 21:18 Isa 17:13; 29:5 Ho 13:3 Mt 3:12}
\crossref{Ps}{1}{5}{Ps 5:5; 24:3 Lu 21:36 Jude 1:15}
\crossref{Ps}{1}{6}{Ps 37:18-\allowbreak24; 139:1,\allowbreak2; 142:3 Job 23:10 Na 1:7 Joh 10:14,\allowbreak27 2Ti 2:19}
\crossref{Ps}{2}{1}{Ps 18:42; 46:6; 83:4-\allowbreak8 Isa 8:9 Lu 18:32 Ac 4:25}
\crossref{Ps}{2}{2}{2:10; 48:4; 110:5 Mt 2:16 Lu 13:31; 23:11,\allowbreak12 Ac 12:1-\allowbreak6 Re 17:12-\allowbreak14}
\crossref{Ps}{2}{3}{Jer 5:5 Lu 19:4 1Pe 2:7,\allowbreak8}
\crossref{Ps}{2}{4}{Ps 11:4; 68:33; 115:3 Isa 40:22; 57:15; 66:1}
\crossref{Ps}{2}{5}{Ps 50:16-\allowbreak22 Isa 11:4; 66:6 Mt 22:7; 23:33-\allowbreak36 Lu 19:27,\allowbreak43,\allowbreak44}
\crossref{Ps}{2}{6}{Ps 45:6; 89:27,\allowbreak36,\allowbreak37; 110:1,\allowbreak2 Isa 9:6,\allowbreak7 Da 7:13,\allowbreak14 Mt 28:18}
\crossref{Ps}{2}{7}{2:148:6 Job 23:13 Isa 46:10}
\crossref{Ps}{2}{8}{Joh 17:4,\allowbreak5}
\crossref{Ps}{2}{9}{Ps 21:8,\allowbreak9; 89:23; 110:5,\allowbreak6 Isa 30:14; 60:12 Jer 19:11 Da 2:44}
\crossref{Ps}{2}{10}{Jer 6:8 Ho 14:9}
\crossref{Ps}{2}{11}{2:89:7 Heb 12:28,\allowbreak29}
\crossref{Ps}{2}{12}{Ge 41:40,\allowbreak43,\allowbreak44 1Sa 10:1 1Ki 19:18 Ho 13:2 Joh 5:23}
\crossref{Ps}{3}{1}{2Sa 15:1-\allowbreak18:33}
\crossref{Ps}{3}{2}{Ps 22:7; 42:3,\allowbreak10; 71:11 2Sa 16:7,\allowbreak8 Mt 27:42,\allowbreak43}
\crossref{Ps}{3}{3}{Ps 18:2; 28:7; 84:11; 119:114 Ge 15:1 De 33:29}
\crossref{Ps}{3}{4}{Ps 22:2-\allowbreak5; 34:6; 50:15; 66:17-\allowbreak19; 86:3,\allowbreak4; 91:15; 116:1-\allowbreak4; 130:1,\allowbreak2; 138:3}
\crossref{Ps}{3}{5}{Ps 4:8; 127:2 Le 26:6 Job 11:18,\allowbreak19 Pr 3:24 Ac 12:6}
\crossref{Ps}{3}{6}{Ps 27:1-\allowbreak3; 46:2,\allowbreak7; 118:10-\allowbreak12 2Ki 6:15-\allowbreak17 Ro 8:31}
\crossref{Ps}{3}{7}{Ps 10:12; 12:5; 35:23; 44:23; 59:5; 74:11; 76:9 Isa 51:9 Hab 2:19}
\crossref{Ps}{3}{8}{Ps 37:39,\allowbreak40 Pr 21:31 Isa 43:11; 45:21,\allowbreak22 Jer 3:23 Ho 13:4 Jon 2:9}
\crossref{Ps}{4}{1}{Ps 22:1; 42:1; 45:1}
\crossref{Ps}{4}{2}{Ps 57:4; 58:1 Ec 8:11; 9:3}
\crossref{Ps}{4}{3}{Ex 33:16 Eph 2:10 2Th 2:13,\allowbreak14 2Ti 2:19 1Pe 2:9 2Pe 2:9}
\crossref{Ps}{4}{4}{Ps 2:11; 33:8; 119:161 Jer 5:22}
\crossref{Ps}{4}{5}{Ps 50:14; 51:19 De 33:19 2Sa 15:12 Isa 1:11-\allowbreak18; 61:8 Mal 1:8,\allowbreak11-\allowbreak14}
\crossref{Ps}{4}{6}{Ps 39:6; 49:16-\allowbreak20 Ec 2:3-\allowbreak26 Isa 55:2 Lu 12:19; 16:19 Jas 4:13}
\crossref{Ps}{4}{7}{Ps 37:4; 43:4; 63:2-\allowbreak5; 92:4 So 1:4 1Pe 1:8}
\crossref{Ps}{4}{8}{Ps 3:5; 16:8 Job 11:18,\allowbreak19 Pr 3:24 1Th 4:13,\allowbreak14; 5:10 Re 14:13}
\crossref{Ps}{5}{1}{Ps 17:1; 54:2; 55:1,\allowbreak2; 64:1; 80:1; 86:1 1Pe 3:12 1Jo 5:14,\allowbreak15}
\crossref{Ps}{5}{2}{Ps 3:4}
\crossref{Ps}{5}{3}{Ps 22:2; 55:17; 69:16; 88:13; 119:147; 130:6 Isa 26:9 Mr 1:35}
\crossref{Ps}{5}{4}{Ps 50:21 1Ch 29:17 Hab 1:13 Mal 2:17}
\crossref{Ps}{5}{5}{Ps 14:1; 92:6; 94:8 Pr 1:7,\allowbreak22; 8:5 Ec 5:4 Hab 1:13}
\crossref{Ps}{5}{6}{Ps 4:2 Re 21:8; 22:15}
\crossref{Ps}{5}{7}{Ps 55:16 Jos 24:15 Lu 6:11,\allowbreak12}
\crossref{Ps}{5}{8}{Ps 25:4,\allowbreak5; 86:11; 119:10,\allowbreak64; 143:8-\allowbreak10 Pr 3:5,\allowbreak6}
\crossref{Ps}{5}{9}{Ps 36:1-\allowbreak4; 52:2; 58:3; 62:4,\allowbreak9; 111:1-\allowbreak3 Jer 9:3-\allowbreak6 Mic 6:12 Ro 1:29-\allowbreak31}
\crossref{Ps}{5}{10}{Ro 3:19,\allowbreak20}
\crossref{Ps}{5}{11}{Ps 35:27; 40:16; 58:10; 68:3; 70:1-\allowbreak4 Jud 5:31 Isa 65:13-\allowbreak16 Re 18:20}
\crossref{Ps}{5}{12}{Ps 1:1-\allowbreak3; 3:8; 29:11; 112:1; 115:13}
\crossref{Ps}{6}{1}{Ps 4:1}
\crossref{Ps}{6}{2}{Ps 38:7; 41:3; 103:13-\allowbreak17}
\crossref{Ps}{6}{3}{Ps 22:14; 31:9,\allowbreak10; 38:8; 42:5,\allowbreak11; 77:2,\allowbreak3 Pr 18:14 Mt 26:38}
\crossref{Ps}{6}{4}{6:80:14; 90:13 Mal 3:7}
\crossref{Ps}{6}{5}{Ps 30:9; 88:10-\allowbreak12; 115:17; 118:17 Isa 38:18,\allowbreak19}
\crossref{Ps}{6}{6}{Ps 38:9; 69:3; 77:2-\allowbreak9; 88:9; 102:3-\allowbreak5; 143:4-\allowbreak7 Job 7:3; 10:1; 23:2}
\crossref{Ps}{6}{7}{Ps 31:9,\allowbreak10; 38:10; 88:9 Job 17:7 La 5:17}
\crossref{Ps}{6}{8}{6:119:115; 139:19 Mt 7:23; 25:41 Lu 13:27}
\crossref{Ps}{6}{9}{Ps 3:4; 31:22; 40:1,\allowbreak2; 66:19,\allowbreak20; 118:5; 120:1; 138:3 Jon 2:2,\allowbreak7}
\crossref{Ps}{6}{10}{Ps 5:10; 7:6; 25:3; 35:26; 40:14,\allowbreak15; 71:13; 83:16,\allowbreak17; 86:17; 109:28,\allowbreak29}
\crossref{Ps}{7}{1}{2Sa 16:1-\allowbreak23}
\crossref{Ps}{7}{2}{Ps 35:15 Isa 38:13}
\crossref{Ps}{7}{3}{Ps 59:3 Jos 22:22 1Sa 20:8; 22:8,\allowbreak13; 24:9; 26:18,\allowbreak19 2Sa 16:7,\allowbreak8}
\crossref{Ps}{7}{4}{Ps 55:20; 109:5 Ge 44:4 Pr 17:3 Jer 18:20,\allowbreak21}
\crossref{Ps}{7}{5}{Job 31:5-\allowbreak10,\allowbreak38-\allowbreak40}
\crossref{Ps}{7}{6}{Ps 3:7; 12:5; 35:1,\allowbreak23; 44:26; 68:1,\allowbreak2 Isa 3:13}
\crossref{Ps}{7}{7}{Ps 48:11; 58:10,\allowbreak11 Re 11:17,\allowbreak18; 16:5-\allowbreak7; 18:20; 19:2}
\crossref{Ps}{7}{8}{Ps 9:8; 11:4; 82:1; 96:13; 98:9 Ge 18:25 Ac 17:31 Ro 14:10-\allowbreak12}
\crossref{Ps}{7}{9}{Ps 9:5,\allowbreak6; 10:15,\allowbreak18; 58:6; 74:10,\allowbreak11,\allowbreak22,\allowbreak23 Isa 37:36-\allowbreak38 Da 11:45}
\crossref{Ps}{7}{10}{Ps 3:3; 18:1,\allowbreak2; 84:11; 89:18 Ge 15:1}
\crossref{Ps}{7}{11}{7:8; 94:15; 140:12,\allowbreak13}
\crossref{Ps}{7}{12}{7:85:4 Isa 55:6,\allowbreak7 Jer 31:18,\allowbreak19 Eze 18:30; 33:11 Mt 3:10 Ac 3:19}
\crossref{Ps}{7}{13}{Ps 11:2; 45:5; 64:3,\allowbreak7; 144:6 De 32:23,\allowbreak42 Job 6:4 La 3:12,\allowbreak13}
\crossref{Ps}{7}{14}{Job 15:20,\allowbreak35 Isa 33:11; 59:4,\allowbreak5 Jas 1:15}
\crossref{Ps}{7}{15}{Ps 35:7; 119:85 Job 6:27 Jer 18:20}
\crossref{Ps}{7}{16}{Ps 36:4,\allowbreak12; 37:12,\allowbreak13 1Sa 23:9; 24:12,\allowbreak13; 26:10; 28:19; 31:3,\allowbreak4}
\crossref{Ps}{7}{17}{Ps 35:28; 51:14; 71:15,\allowbreak16; 98:2; 111:3; 145:7}
\crossref{Ps}{8}{1}{8:81:1 84:1}
\crossref{Ps}{8}{2}{8:81:1 84:1}
\crossref{Ps}{8}{3}{Mt 11:25; 21:16 Lu 10:21 1Co 1:27}
\crossref{Ps}{8}{4}{Ps 19:1; 111:2 Job 22:12; 36:24 Ro 1:20}
\crossref{Ps}{8}{5}{8:144:3 2Ch 6:18 Job 7:17; 25:6 Isa 40:17 Heb 2:6-\allowbreak9}
\crossref{Ps}{8}{6}{8:103:20 Ge 1:26,\allowbreak27; 2:7 2Sa 14:29 Job 4:18-\allowbreak20 Php 2:7,\allowbreak8}
\crossref{Ps}{8}{7}{Ge 1:26,\allowbreak28; 9:2 Mt 28:18 Heb 1:2}
\crossref{Ps}{8}{8}{Ge 2:20}
\crossref{Ps}{8}{9}{8:148:10 Ge 1:20-\allowbreak25 Job 38:39-\allowbreak41; 39:1-\allowbreak30; 40:15-\allowbreak24; 41:1-\allowbreak34}
\crossref{Ps}{9}{1}{Ps 7:17; 34:1-\allowbreak4; 103:1,\allowbreak2; 145:1-\allowbreak3; 146:1,\allowbreak2 1Ch 29:10-\allowbreak13 Isa 12:1}
\crossref{Ps}{9}{2}{Ps 7:17; 34:1-\allowbreak4; 103:1,\allowbreak2; 145:1-\allowbreak3; 146:1,\allowbreak2 1Ch 29:10-\allowbreak13 Isa 12:1}
\crossref{Ps}{9}{3}{Ps 5:11; 27:6; 28:7; 43:4; 92:4; 97:12 Hab 3:17,\allowbreak18 Php 4:4}
\crossref{Ps}{9}{4}{9:68:1,\allowbreak2 76:7 80:16 Isa 64:3 2Th 1:9 Re 6:12-\allowbreak17; 20:11}
\crossref{Ps}{9}{5}{Ps 16:5; 140:12}
\crossref{Ps}{9}{6}{Ps 2:1,\allowbreak8,\allowbreak9; 78:55; 79:10; 149:7 1Sa 17:45-\allowbreak51 2Sa 5:6-\allowbreak16; 8:1-\allowbreak15}
\crossref{Ps}{9}{7}{Ps 46:9 Ex 14:13 Isa 10:24,\allowbreak25; 14:6-\allowbreak8 Na 1:9-\allowbreak13 1Co 15:26,\allowbreak54-\allowbreak57}
\crossref{Ps}{9}{8}{9:90:2 102:12,\allowbreak24-\allowbreak27 Heb 1:11,\allowbreak12; 13:8 2Pe 3:8}
\crossref{Ps}{9}{9}{Ps 50:6; 94:15; 96:13; 98:9; 99:4 Ge 18:25 Isa 11:4,\allowbreak5 Ac 17:31}
\crossref{Ps}{9}{10}{Ps 18:2; 32:7; 37:39; 46:1; 48:3; 62:8; 91:1,\allowbreak2; 142:4 De 33:27 Pr 18:10}
\crossref{Ps}{9}{11}{9:91:14 Ex 34:5-\allowbreak7 1Ch 28:9 Pr 18:10 Joh 17:3 2Co 4:6 2Ti 1:12}
\crossref{Ps}{9}{12}{Ps 33:1-\allowbreak3; 47:6,\allowbreak7; 96:1,\allowbreak2; 148:1-\allowbreak5,\allowbreak13,\allowbreak14}
\crossref{Ps}{9}{13}{Ge 9:5 2Ki 24:4 Isa 26:21 Mt 23:35 Lu 11:50,\allowbreak51 Re 6:9,\allowbreak10; 16:6}
\crossref{Ps}{9}{14}{Ps 51:1; 119:132}
\crossref{Ps}{9}{15}{Ps 51:15; 79:13; 106:2}
\crossref{Ps}{9}{16}{Ps 7:15,\allowbreak16; 35:8; 37:15; 57:6; 94:23 Pr 5:22; 22:8}
\crossref{Ps}{9}{17}{Ps 48:11; 58:10,\allowbreak11; 83:17,\allowbreak18 Ex 7:5; 14:4,\allowbreak10,\allowbreak31 De 29:22-\allowbreak28}
\crossref{Ps}{9}{18}{Pr 14:32 Isa 3:11; 5:14 Mt 25:41-\allowbreak46 Ro 2:8,\allowbreak9 2Th 1:7-\allowbreak9}
\crossref{Ps}{9}{19}{9:12; 12:5; 72:4,\allowbreak12-\allowbreak14; 102:17,\allowbreak20; 109:31 Lu 1:53; 6:20 Jas 2:5}
\crossref{Ps}{9}{20}{Ps 3:7; 7:6; 10:12; 44:23,\allowbreak26; 68:1,\allowbreak2; 74:22,\allowbreak23; 76:8,\allowbreak9; 80:2}
\crossref{Ps}{10}{1}{Ps 7:1; 9:10; 16:1; 25:2; 31:14; 56:11 2Ch 14:11; 16:8 Isa 26:3,\allowbreak4}
\crossref{Ps}{10}{2}{Ps 7:1; 9:10; 16:1; 25:2; 31:14; 56:11 2Ch 14:11; 16:8 Isa 26:3,\allowbreak4}
\crossref{Ps}{10}{3}{Ps 10:2; 37:14; 64:3,\allowbreak4 Jer 9:3}
\crossref{Ps}{10}{4}{10:75:3 82:5 Isa 58:12 2Ti 2:19}
\crossref{Ps}{10}{5}{Ps 9:11; 18:6 Ex 40:34,\allowbreak35 1Ch 17:5 Hab 2:20 Zec 2:13 2Th 2:4}
\crossref{Ps}{10}{6}{Ps 7:9; 17:3; 26:2; 139:1,\allowbreak23,\allowbreak24 Ge 22:1 Zec 13:9 Mal 3:3 Jas 1:12}
\crossref{Ps}{10}{7}{10:105:32 Ge 19:24 Ex 9:23,\allowbreak24 Job 18:15; 20:23 Isa 24:17,\allowbreak18}
\crossref{Ps}{10}{8}{}
\crossref{Ps}{10}{9}{}
\crossref{Ps}{10}{10}{}
\crossref{Ps}{10}{11}{}
\crossref{Ps}{10}{12}{}
\crossref{Ps}{10}{13}{}
\crossref{Ps}{10}{14}{}
\crossref{Ps}{10}{15}{}
\crossref{Ps}{10}{16}{}
\crossref{Ps}{10}{17}{}
\crossref{Ps}{10}{18}{}
\crossref{Ps}{11}{1}{Ps 45:7; 99:4; 146:8 Isa 61:8}
\crossref{Ps}{11}{2}{Ps 6:1}
\crossref{Ps}{11}{3}{Ps 10:7; 36:3,\allowbreak4; 38:12; 41:6; 52:1-\allowbreak4; 59:12; 144:8,\allowbreak11 Jer 9:2-\allowbreak6,\allowbreak8}
\crossref{Ps}{11}{4}{Job 32:22}
\crossref{Ps}{11}{5}{Jer 18:18 Jas 3:5,\allowbreak6}
\crossref{Ps}{11}{6}{Ps 10:12; 74:21,\allowbreak22; 79:10,\allowbreak11; 146:7,\allowbreak8 Ex 2:23,\allowbreak24; 3:7-\allowbreak9 Jud 10:16}
\crossref{Ps}{11}{7}{Ps 18:30; 19:8; 119:140 2Sa 22:31 Pr 30:5}
\crossref{Ps}{12}{1}{Ps 6:3; 35:17; 74:1; 80:4; 85:5; 89:46; 90:14; 94:3,\allowbreak4}
\crossref{Ps}{12}{2}{12:77:2-\allowbreak12 94:18,\allowbreak19 142:4-\allowbreak7 Job 7:12-\allowbreak15; 9:19-\allowbreak21,\allowbreak27,\allowbreak28; 10:15}
\crossref{Ps}{12}{3}{Ps 9:13; 25:19; 31:7; 119:153 La 5:1}
\crossref{Ps}{12}{4}{Ps 10:11; 25:2; 35:19,\allowbreak25; 38:16 Jos 7:9 Eze 35:12-\allowbreak15}
\crossref{Ps}{12}{5}{Ps 32:10; 33:18,\allowbreak21,\allowbreak22; 36:7; 52:8; 147:11 Isa 12:2 Jude 1:21}
\crossref{Ps}{12}{6}{Ps 21:13}
\crossref{Ps}{12}{7}{}
\crossref{Ps}{12}{8}{}
\crossref{Ps}{13}{1}{13:73:3 92:6 107:17 1Sa 25:25 Pr 1:7,\allowbreak22; 13:19; 27:22 Lu 12:20}
\crossref{Ps}{13}{2}{Ps 33:13,\allowbreak14; 102:19,\allowbreak20 Ge 6:12; 11:5; 18:21 Isa 63:15; 64:1 La 3:50}
\crossref{Ps}{13}{3}{13:119:176 Ec 7:29 Isa 53:6; 59:7,\allowbreak8,\allowbreak13-\allowbreak15 Jer 2:13 Ro 3:10-\allowbreak12,\allowbreak23}
\crossref{Ps}{13}{4}{13:94:8,\allowbreak9 Isa 5:13; 27:11; 29:14; 44:19,\allowbreak20; 45:20 Ro 1:21,\allowbreak22,\allowbreak28}
\crossref{Ps}{13}{5}{Ps 53:5 Ex 15:16 Es 8:7 Pr 1:26,\allowbreak27; 28:1}
\crossref{Ps}{13}{6}{Ps 3:2; 4:2; 22:7,\allowbreak8; 42:10 Ne 4:2-\allowbreak4 Isa 37:10,\allowbreak11 Eze 35:10 Da 3:15}
\crossref{Ps}{14}{1}{Ps 1:1-\allowbreak4; 23:6; 21:3-\allowbreak5; 27:4; 61:4; 84:4; 92:13 Joh 3:3-\allowbreak5; 14:3; 17:24}
\crossref{Ps}{14}{2}{14:84:11 Pr 2:7,\allowbreak8; 28:18 Isa 33:15 Mic 2:7 Lu 1:6 Ga 2:14 1Jo 2:6}
\crossref{Ps}{14}{3}{14:101:5-\allowbreak8 Ex 23:1-\allowbreak33 Le 19:16 Jer 9:4-\allowbreak9 Ro 1:30 Tit 3:2}
\crossref{Ps}{14}{4}{14:101:4 2Ki 3:13,\allowbreak14 Es 3:2 Job 32:21,\allowbreak22 Isa 32:5,\allowbreak6 Da 5:17-\allowbreak31}
\crossref{Ps}{14}{5}{Ex 22:25 Le 25:35-\allowbreak37 De 23:19,\allowbreak20 Ne 5:2-\allowbreak5,\allowbreak7-\allowbreak13 Eze 18:8,\allowbreak17}
\crossref{Ps}{14}{6}{}
\crossref{Ps}{14}{7}{}
\crossref{Ps}{15}{1}{Ps 17:5,\allowbreak8; 31:23; 37:28; 97:10; 116:6 Pr 2:8}
\crossref{Ps}{15}{2}{Ps 8:1; 27:8; 31:14; 89:26; 91:2 Isa 26:13; 44:5 Zec 13:9 Joh 20:28}
\crossref{Ps}{15}{3}{Ga 6:10 Tit 3:8 Heb 6:10}
\crossref{Ps}{15}{4}{Ps 32:10; 97:7 Jon 2:8 Re 14:9-\allowbreak11; 18:4,\allowbreak5}
\crossref{Ps}{15}{5}{15:73:26; 119:57 142:5 De 32:9 Jer 10:16 La 3:24}
\crossref{Ps}{16}{1}{16:86:1 142:1}
\crossref{Ps}{16}{2}{Ps 37:6,\allowbreak33 2Th 1:6-\allowbreak9 Jude 1:24}
\crossref{Ps}{16}{3}{Ps 11:5; 26:2; 66:10; 139:1 Job 23:10 Zec 13:9 Mal 3:2 1Co 4:4}
\crossref{Ps}{16}{4}{Ps 14:1-\allowbreak3 Ge 6:5,\allowbreak11 Job 15:16; 31:33 1Co 3:3 1Pe 4:2,\allowbreak3}
\crossref{Ps}{16}{5}{16:119:116,\allowbreak117,\allowbreak133; 121:3,\allowbreak7 1Sa 2:9 Jer 10:23}
\crossref{Ps}{16}{6}{Ps 55:16; 66:19,\allowbreak20; 116:2}
\crossref{Ps}{16}{7}{Ps 31:21; 78:12 Ro 5:20,\allowbreak21 Re 15:3}
\crossref{Ps}{16}{8}{De 32:10 Pr 7:2 Zec 2:8}
\crossref{Ps}{16}{9}{1Ch 17:9}
\crossref{Ps}{16}{10}{16:73:7-\allowbreak9 119:70 De 32:15 Job 15:27 Isa 6:10 Mt 13:15 Ac 28:27}
\crossref{Ps}{16}{11}{1Sa 23:26; 24:2,\allowbreak3; 26:2,\allowbreak3}
\crossref{Ps}{17}{1}{Ps 36:1}
\crossref{Ps}{17}{2}{Ps 36:1}
\crossref{Ps}{17}{3}{Ps 28:1; 62:2,\allowbreak7 Isa 32:2}
\crossref{Ps}{17}{4}{Ps 5:2,\allowbreak3; 28:1,\allowbreak2; 55:16; 62:8 2Sa 22:4 Php 4:6,\allowbreak7}
\crossref{Ps}{17}{5}{17:116:3 2Sa 22:5,\allowbreak6 Isa 13:8; 53:3,\allowbreak4 Mt 26:38,\allowbreak39 Mr 14:33,\allowbreak34}
\crossref{Ps}{17}{6}{17:86:13; 88:3-\allowbreak8,\allowbreak15-\allowbreak17 Ac 2:24}
\crossref{Ps}{17}{7}{17:3,\allowbreak4; 50:15; 130:1,\allowbreak2 Mr 14:36 Ac 12:5}
\crossref{Ps}{17}{8}{17:114:4-\allowbreak7 Mt 28:2 Ac 4:31; 16:25,\allowbreak26}
\crossref{Ps}{17}{9}{Ps 11:6; 21:9; 74:1; 104:32; 144:5,\allowbreak6 Ge 19:28 Le 10:2 Nu 11:1; 16:35}
\crossref{Ps}{17}{10}{De 5:22,\allowbreak23 Mr 15:33 Joh 13:7}
\crossref{Ps}{17}{11}{17:99:1 2Sa 22:11,\allowbreak12 Eze 1:5-\allowbreak14; 10:20-\allowbreak22}
\crossref{Ps}{17}{12}{Ps 27:5; 81:7; 91:1}
\crossref{Ps}{17}{13}{17:97:3,\allowbreak4 Hab 3:4,\allowbreak5 Mt 17:2,\allowbreak5}
\crossref{Ps}{17}{14}{17:78:48; 104:7 Ex 20:18 1Sa 7:10 Job 40:9 Joh 12:29 Re 4:5; 8:5}
\crossref{Ps}{17}{15}{Ps 21:12; 77:17 Nu 24:8 De 32:23,\allowbreak42 Jos 10:10 Job 6:4 Isa 30:30}
\crossref{Ps}{18}{1}{Ps 8:3; 33:6; 115:16; 148:3,\allowbreak4 Isa 40:22-\allowbreak26 Jer 10:11,\allowbreak12 Ro 1:19,\allowbreak20}
\crossref{Ps}{18}{2}{Ps 8:3; 33:6; 115:16; 148:3,\allowbreak4 Isa 40:22-\allowbreak26 Jer 10:11,\allowbreak12 Ro 1:19,\allowbreak20}
\crossref{Ps}{18}{3}{Ps 24:7-\allowbreak10; 78:3-\allowbreak6; 134:1-\allowbreak3; 148:12 Ex 15:20,\allowbreak21 Isa 38:19}
\crossref{Ps}{18}{4}{}
\crossref{Ps}{18}{5}{18:98:3 Isa 49:6 Ro 10:18 2Co 10:13-\allowbreak16}
\crossref{Ps}{18}{6}{Isa 61:10; 62:5 Joh 3:29}
\crossref{Ps}{18}{7}{18:139:9 Job 25:3 Ec 1:5 Col 1:23}
\crossref{Ps}{18}{8}{18:78:1-\allowbreak7 119:72,\allowbreak96-\allowbreak100,\allowbreak105,\allowbreak127,\allowbreak128 147:19,\allowbreak20 De 6:6-\allowbreak9; 17:18-\allowbreak20}
\crossref{Ps}{18}{9}{18:105:45; 119:12,\allowbreak16,\allowbreak80,\allowbreak171 Ge 26:5 Ex 18:16 De 4:5,\allowbreak6 Eze 36:27}
\crossref{Ps}{18}{10}{Ps 34:11-\allowbreak14; 36:1; 115:13 Ge 22:12; 42:18 1Sa 12:24 1Ki 18:3,\allowbreak4,\allowbreak12}
\crossref{Ps}{18}{11}{18:119:72,\allowbreak127 Job 28:15-\allowbreak17 Pr 3:13-\allowbreak15; 8:10,\allowbreak11,\allowbreak19; 16:16}
\crossref{Ps}{18}{12}{18:119:11 2Ch 19:10 Pr 6:22,\allowbreak23 Eze 3:17-\allowbreak21; 33:3-\allowbreak9 Mt 3:7}
\crossref{Ps}{18}{13}{Ps 40:12 Job 6:24 Isa 64:6 1Co 4:4 Heb 9:7}
\crossref{Ps}{18}{14}{Ge 20:6 1Sa 25:32-\allowbreak34,\allowbreak39}
\crossref{Ps}{18}{15}{Ps 5:1,\allowbreak2; 51:15; 66:18-\allowbreak20; 119:108 Ge 4:4,\allowbreak5 Pr 15:8 Ro 15:16}
\crossref{Ps}{18}{16}{}
\crossref{Ps}{18}{17}{}
\crossref{Ps}{18}{18}{}
\crossref{Ps}{18}{19}{}
\crossref{Ps}{18}{20}{}
\crossref{Ps}{18}{21}{}
\crossref{Ps}{18}{22}{}
\crossref{Ps}{18}{23}{}
\crossref{Ps}{18}{24}{}
\crossref{Ps}{18}{25}{}
\crossref{Ps}{18}{26}{}
\crossref{Ps}{18}{27}{}
\crossref{Ps}{18}{28}{}
\crossref{Ps}{18}{29}{}
\crossref{Ps}{18}{30}{}
\crossref{Ps}{18}{31}{}
\crossref{Ps}{18}{32}{}
\crossref{Ps}{18}{33}{}
\crossref{Ps}{18}{34}{}
\crossref{Ps}{18}{35}{}
\crossref{Ps}{18}{36}{}
\crossref{Ps}{18}{37}{}
\crossref{Ps}{18}{38}{}
\crossref{Ps}{18}{39}{}
\crossref{Ps}{18}{40}{}
\crossref{Ps}{18}{41}{}
\crossref{Ps}{18}{42}{}
\crossref{Ps}{18}{43}{}
\crossref{Ps}{18}{44}{}
\crossref{Ps}{18}{45}{}
\crossref{Ps}{18}{46}{}
\crossref{Ps}{18}{47}{}
\crossref{Ps}{18}{48}{}
\crossref{Ps}{18}{49}{}
\crossref{Ps}{18}{50}{}
\crossref{Ps}{19}{1}{Ps 41:1; 46:1; 50:5; 60:11; 91:15; 138:7 Jer 30:7 Mt 26:38,\allowbreak39 Heb 5:7}
\crossref{Ps}{19}{2}{Ps 41:1; 46:1; 50:5; 60:11; 91:15; 138:7 Jer 30:7 Mt 26:38,\allowbreak39 Heb 5:7}
\crossref{Ps}{19}{3}{19:73:17 1Ki 6:16; 8:44,\allowbreak45 2Ch 20:8,\allowbreak9}
\crossref{Ps}{19}{4}{Ge 4:4 Isa 60:7 Eph 5:2 1Pe 2:5}
\crossref{Ps}{19}{5}{Ps 21:2; 37:4; 145:19 Pr 11:23 Mt 21:22 Joh 11:42; 16:23 Ro 8:27,\allowbreak28}
\crossref{Ps}{19}{6}{Ps 13:5; 19:4; 21:1; 35:9; 118:15 Isa 12:1-\allowbreak3; 25:9; 61:10 Hab 3:18}
\crossref{Ps}{19}{7}{Ps 2:2; 18:50; 28:8; 89:20-\allowbreak23 Ac 2:36; 4:10}
\crossref{Ps}{19}{8}{Ps 33:16,\allowbreak17 1Sa 13:5 2Sa 8:4; 10:18 Pr 21:31 Isa 30:16; 31:1}
\crossref{Ps}{19}{9}{Ps 34:21,\allowbreak22 Jud 5:31}
\crossref{Ps}{19}{10}{Ps 2:6-\allowbreak10; 5:2; 24:7; 44:4; 74:12}
\crossref{Ps}{19}{11}{}
\crossref{Ps}{19}{12}{}
\crossref{Ps}{19}{13}{}
\crossref{Ps}{19}{14}{}
\crossref{Ps}{20}{1}{Ps 2:6; 20:6,\allowbreak9; 63:11; 72:1,\allowbreak2 Isa 9:6,\allowbreak7 Mt 2:2}
\crossref{Ps}{20}{2}{Ps 2:6; 20:6,\allowbreak9; 63:11; 72:1,\allowbreak2 Isa 9:6,\allowbreak7 Mt 2:2}
\crossref{Ps}{20}{3}{Ps 2:8,\allowbreak9; 20:4,\allowbreak5; 92:11 Isa 49:6-\allowbreak12 Heb 7:25}
\crossref{Ps}{20}{4}{Ps 18:18 1Sa 16:13 2Sa 2:4; 5:3 Job 41:11 Ro 11:35}
\crossref{Ps}{20}{5}{Ps 13:3; 16:10,\allowbreak11; 61:5,\allowbreak6; 119:77,\allowbreak175}
\crossref{Ps}{20}{6}{Ps 3:3; 62:7 2Sa 7:8,\allowbreak9,\allowbreak19 Isa 49:5-\allowbreak7; 63:1 Joh 13:31,\allowbreak32; 17:1,\allowbreak5,\allowbreak22}
\crossref{Ps}{20}{7}{20:72:17-\allowbreak19 Ge 12:2 Lu 2:10,\allowbreak11,\allowbreak30-\allowbreak32 Ac 3:26 Ga 3:9,\allowbreak14 Eph 1:3}
\crossref{Ps}{20}{8}{Ps 13:5; 18:2; 20:7,\allowbreak8; 26:1; 61:4,\allowbreak6,\allowbreak7; 91:2,\allowbreak9,\allowbreak10 1Sa 30:6 Mt 27:43}
\crossref{Ps}{20}{9}{Ps 2:9; 18:1}
\crossref{Ps}{21}{1}{Ps 31:14-\allowbreak16; 43:1-\allowbreak5 Mt 27:46 Mr 15:34 Lu 24:44}
\crossref{Ps}{21}{2}{Ps 31:14-\allowbreak16; 43:1-\allowbreak5 Mt 27:46 Mr 15:34 Lu 24:44}
\crossref{Ps}{21}{3}{Ps 42:3; 55:16,\allowbreak17; 88:1 Lu 18:7 1Th 3:10 2Ti 1:3}
\crossref{Ps}{21}{4}{21:145:17 Isa 6:3 Re 4:8}
\crossref{Ps}{21}{5}{Ps 44:1-\allowbreak7 Ge 15:6; 32:9-\allowbreak12,\allowbreak28 Ex 14:13,\allowbreak14,\allowbreak31 1Sa 7:9-\allowbreak12}
\crossref{Ps}{21}{6}{21:99:6,\allowbreak7 106:44 Jud 4:3; 6:6; 10:10-\allowbreak16}
\crossref{Ps}{21}{7}{Job 25:6 Isa 41:14}
\crossref{Ps}{21}{8}{Ps 35:15,\allowbreak16 Mt 9:24; 27:29,\allowbreak39 Mr 15:20,\allowbreak29 Lu 16:14; 23:11,\allowbreak35-\allowbreak39}
\crossref{Ps}{21}{9}{Ps 37:5; 55:22 Pr 16:3}
\crossref{Ps}{21}{10}{21:71:6 139:15,\allowbreak16 Isa 49:1,\allowbreak2}
\crossref{Ps}{21}{11}{Isa 46:3,\allowbreak4; 49:1 Lu 2:40,\allowbreak52}
\crossref{Ps}{21}{12}{Ps 10:1; 13:1-\allowbreak3; 35:22; 38:21; 69:1,\allowbreak2,\allowbreak18; 71:12 Joh 16:32 Heb 5:7}
\crossref{Ps}{21}{13}{21:68:30 Jer 50:11}
\crossref{Ps}{22}{1}{22:79:13; 80:1 Isa 40:11 Jer 23:3,\allowbreak4 Eze 34:11,\allowbreak12,\allowbreak23,\allowbreak24 Mic 5:2,\allowbreak4}
\crossref{Ps}{22}{2}{Isa 30:23 Eze 34:13,\allowbreak14}
\crossref{Ps}{22}{3}{Ps 19:7}
\crossref{Ps}{22}{4}{Ps 44:19 Job 3:5; 10:21,\allowbreak22; 24:17 Jer 2:6 Lu 1:79}
\crossref{Ps}{22}{5}{Ps 22:26,\allowbreak29; 31:19,\allowbreak20; 104:15 Job 36:16 Isa 25:6 Joh 6:53-\allowbreak56}
\crossref{Ps}{22}{6}{Ps 30:11,\allowbreak12; 36:7-\allowbreak10; 103:17 2Co 1:10 2Ti 4:18}
\crossref{Ps}{22}{7}{}
\crossref{Ps}{22}{8}{}
\crossref{Ps}{22}{9}{}
\crossref{Ps}{22}{10}{}
\crossref{Ps}{22}{11}{}
\crossref{Ps}{22}{12}{}
\crossref{Ps}{22}{13}{}
\crossref{Ps}{22}{14}{}
\crossref{Ps}{22}{15}{}
\crossref{Ps}{22}{16}{}
\crossref{Ps}{22}{17}{}
\crossref{Ps}{22}{18}{}
\crossref{Ps}{22}{19}{}
\crossref{Ps}{22}{20}{}
\crossref{Ps}{22}{21}{}
\crossref{Ps}{22}{22}{}
\crossref{Ps}{22}{23}{}
\crossref{Ps}{22}{24}{}
\crossref{Ps}{22}{25}{}
\crossref{Ps}{22}{26}{}
\crossref{Ps}{22}{27}{}
\crossref{Ps}{22}{28}{}
\crossref{Ps}{22}{29}{}
\crossref{Ps}{22}{30}{}
\crossref{Ps}{22}{31}{}
\crossref{Ps}{23}{1}{Ps 50:12 Ex 9:29; 19:5 De 10:14 1Ch 29:11 Job 41:11 Da 4:25}
\crossref{Ps}{23}{2}{Ps 33:6; 95:4; 104:5,\allowbreak6; 136:6 Ge 1:9,\allowbreak10 Job 38:4 Jer 10:11-\allowbreak16}
\crossref{Ps}{23}{3}{Ps 15:1; 68:18 Joh 13:36; 20:17 Eph 4:8-\allowbreak10}
\crossref{Ps}{23}{4}{Ps 18:20; 26:6 Job 9:30; 17:9 Isa 1:15,\allowbreak16; 33:15,\allowbreak16 1Ti 2:8 Jas 4:8}
\crossref{Ps}{23}{5}{Ps 50:23; 67:6,\allowbreak7; 72:17; 115:12,\allowbreak13; 128:1-\allowbreak5 Nu 6:24-\allowbreak27 Isa 33:15-\allowbreak17}
\crossref{Ps}{23}{6}{Ps 22:30; 73:15 Isa 53:10 Ro 4:16 1Pe 2:9}
\crossref{Ps}{24}{1}{Ps 24:4; 86:4; 143:8 1Sa 1:15 La 3:41}
\crossref{Ps}{24}{2}{Ps 7:1; 18:2; 22:1,\allowbreak5,\allowbreak8; 31:1; 34:8; 37:40; 71:1 Isa 26:3; 28:16; 41:16}
\crossref{Ps}{24}{3}{Ps 27:14; 33:20; 37:34; 40:1-\allowbreak3; 62:1,\allowbreak5; 123:2 Ge 49:13 Isa 25:9}
\crossref{Ps}{24}{4}{Ps 5:1,\allowbreak8; 27:11; 86:11; 119:27; 143:8 Ex 33:13 Pr 8:20 Isa 2:3}
\crossref{Ps}{24}{5}{24:8,\allowbreak10; 43:3,\allowbreak4; 107:7 Isa 35:8; 42:16; 49:10 Jer 31:9 Joh 8:31,\allowbreak32}
\crossref{Ps}{24}{6}{24:98:3 106:45 136:23 2Ch 6:42 Lu 1:54,\allowbreak71,\allowbreak72}
\crossref{Ps}{24}{7}{24:79:8 109:14,\allowbreak16 Isa 38:17; 43:25; 64:9 Heb 8:12; 10:16-\allowbreak18}
\crossref{Ps}{24}{8}{24:119:68}
\crossref{Ps}{24}{9}{Ps 22:26; 76:9; 147:6; 149:4 Isa 11:4; 61:1 Zep 2:3 Mt 5:5 Ga 5:23}
\crossref{Ps}{24}{10}{Ps 18:25,\allowbreak26; 28:4-\allowbreak6; 37:23,\allowbreak24; 91:14; 119:75,\allowbreak76; 138:7 Ge 5:24; 17:1}
\crossref{Ps}{25}{1}{Ps 7:8; 35:24; 43:1; 54:1 1Sa 24:15}
\crossref{Ps}{25}{2}{Ps 7:9; 17:3; 66:10; 139:23,\allowbreak24 Job 13:23; 31:4-\allowbreak6 Jer 20:12 Zec 13:9}
\crossref{Ps}{25}{3}{Ps 52:1; 85:10-\allowbreak13 Mt 5:44-\allowbreak48 Lu 6:36 2Co 3:18; 5:14,\allowbreak15; 8:9}
\crossref{Ps}{25}{4}{Ps 1:1; 119:63,\allowbreak115 Pr 9:6; 12:11; 13:20 Jer 15:17 1Co 15:33}
\crossref{Ps}{25}{5}{Ps 31:6; 101:3-\allowbreak8; 139:21,\allowbreak22}
\crossref{Ps}{25}{6}{Ps 24:4; 73:13 Ex 30:19,\allowbreak20 Isa 1:16-\allowbreak18 Tit 3:5 Heb 10:19-\allowbreak22}
\crossref{Ps}{25}{7}{Ps 9:14; 66:13-\allowbreak15; 95:2; 100:4,\allowbreak5; 116:12-\allowbreak14,\allowbreak18,\allowbreak19; 118:19,\allowbreak27; 134:2}
\crossref{Ps}{25}{8}{Ps 27:4-\allowbreak6; 42:4; 84:1,\allowbreak2,\allowbreak10; 122:1-\allowbreak4,\allowbreak9 2Sa 15:25 1Ch 29:3}
\crossref{Ps}{25}{9}{Ps 28:1-\allowbreak3 1Sa 25:29 Mal 3:18 Mt 24:51; 25:32,\allowbreak44,\allowbreak46 Re 22:14,\allowbreak15}
\crossref{Ps}{25}{10}{Ps 10:14; 11:2; 36:4; 52:2; 55:9-\allowbreak11 Pr 1:16; 4:16 Mic 2:1-\allowbreak3 Mt 26:3,\allowbreak4}
\crossref{Ps}{25}{11}{25:1 1Sa 12:2-\allowbreak5 2Ch 31:20,\allowbreak21 Ne 5:15 Job 1:1 Isa 38:3 Lu 1:6}
\crossref{Ps}{25}{12}{Ps 27:11; 40:2 1Sa 2:9 Pr 10:9}
\crossref{Ps}{25}{13}{}
\crossref{Ps}{25}{14}{}
\crossref{Ps}{25}{15}{}
\crossref{Ps}{25}{16}{}
\crossref{Ps}{25}{17}{}
\crossref{Ps}{25}{18}{}
\crossref{Ps}{25}{19}{}
\crossref{Ps}{25}{20}{}
\crossref{Ps}{25}{21}{}
\crossref{Ps}{25}{22}{}
\crossref{Ps}{26}{1}{Ps 18:28; 84:11 Job 29:3 Isa 2:5; 60:1-\allowbreak3,\allowbreak19,\allowbreak20 Mic 7:7,\allowbreak8 Mal 4:2}
\crossref{Ps}{26}{2}{Ps 3:7; 18:4; 22:16; 62:3,\allowbreak4}
\crossref{Ps}{26}{3}{Ps 3:6; 52:6 2Ki 6:15-\allowbreak17 2Ch 20:15 Php 1:28 1Pe 3:14}
\crossref{Ps}{26}{4}{Ps 26:8 Lu 10:42 Php 3:13}
\crossref{Ps}{26}{5}{Ps 10:1; 32:6,\allowbreak7; 46:1; 50:15; 77:2; 91:15; 138:7 Pr 1:24-\allowbreak28 Isa 26:16}
\crossref{Ps}{26}{6}{Ps 3:3; 110:7 Ge 40:13,\allowbreak20 2Ki 25:27}
\crossref{Ps}{26}{7}{Ps 4:1; 5:2; 130:2-\allowbreak4; 143:1,\allowbreak2}
\crossref{Ps}{26}{8}{Ps 63:1,\allowbreak2; 119:58}
\crossref{Ps}{26}{9}{Ps 13:1; 44:24; 69:17; 102:2; 143:7 Isa 59:2}
\crossref{Ps}{26}{10}{26:69:8 2Sa 16:11 Isa 49:15 Mt 10:21,\allowbreak22,\allowbreak36}
\crossref{Ps}{26}{11}{Ps 25:4,\allowbreak5,\allowbreak9,\allowbreak12; 86:11; 119:10; 143:8-\allowbreak10 Pr 2:6-\allowbreak9 Isa 30:20,\allowbreak21}
\crossref{Ps}{26}{12}{Ps 31:8; 35:25; 38:16; 41:11; 140:8}
\crossref{Ps}{27}{1}{Ps 3:4; 5:2; 22:2; 77:1; 142:1}
\crossref{Ps}{27}{2}{Ps 63:4; 125:5; 134:2; 141:2; 143:6 2Ch 6:13 1Ti 2:8}
\crossref{Ps}{27}{3}{Ps 26:9 Nu 16:26 Mt 25:41,\allowbreak46}
\crossref{Ps}{27}{4}{Ps 5:10; 59:12,\allowbreak13; 69:22-\allowbreak24 Jer 18:21-\allowbreak23 2Ti 4:14 Re 18:6}
\crossref{Ps}{27}{5}{Ps 10:5; 92:4-\allowbreak6; 104:24; 111:2-\allowbreak4 Job 34:26,\allowbreak27 Isa 5:12; 22:11}
\crossref{Ps}{27}{6}{Ps 31:21,\allowbreak22; 66:19,\allowbreak20; 69:33,\allowbreak34; 107:19-\allowbreak22; 116:1,\allowbreak2; 118:5}
\crossref{Ps}{27}{7}{27:8; 18:1,\allowbreak2; 19:14; 46:1 Isa 12:2; 45:24 Eph 6:10}
\crossref{Ps}{27}{8}{}
\crossref{Ps}{27}{9}{Ps 14:7; 25:22; 80:14-\allowbreak19 Jer 31:7}
\crossref{Ps}{27}{10}{}
\crossref{Ps}{27}{11}{}
\crossref{Ps}{27}{12}{}
\crossref{Ps}{27}{13}{}
\crossref{Ps}{27}{14}{}
\crossref{Ps}{28}{1}{Ps 2:10-\allowbreak12; 68:31-\allowbreak34; 96:7-\allowbreak9 Isa 60:12 Jer 13:16-\allowbreak18 Re 5:11-\allowbreak14}
\crossref{Ps}{28}{2}{Ps 2:10-\allowbreak12; 68:31-\allowbreak34; 96:7-\allowbreak9 Isa 60:12 Jer 13:16-\allowbreak18 Re 5:11-\allowbreak14}
\crossref{Ps}{28}{3}{1Ch 16:28,\allowbreak29}
\crossref{Ps}{28}{4}{Ps 18:13-\allowbreak15; 77:16-\allowbreak19 Mt 8:26,\allowbreak27 Re 17:14,\allowbreak15}
\crossref{Ps}{28}{5}{Ps 33:9 Job 26:11-\allowbreak14 Jer 51:15,\allowbreak16 Lu 4:36; 8:25}
\crossref{Ps}{28}{6}{Isa 2:13}
\crossref{Ps}{28}{7}{28:114:4-\allowbreak7}
\crossref{Ps}{28}{8}{28:77:18; 144:5,\allowbreak6 Ex 9:23 Le 10:2 Nu 16:35 2Ki 1:10-\allowbreak12 Job 37:3}
\crossref{Ps}{28}{9}{Ps 18:7; 46:3 Job 9:6 Isa 13:13 Joe 3:16 Hag 2:6,\allowbreak21 Heb 12:26}
\crossref{Ps}{29}{1}{Ps 28:8,\allowbreak9; 68:35; 84:7; 85:8,\allowbreak10; 138:3 Isa 40:29,\allowbreak31; 41:10}
\crossref{Ps}{29}{2}{De 20:5 2Sa 5:11; 6:20; 7:2; 20:3}
\crossref{Ps}{29}{3}{Ps 6:2; 51:8; 103:3,\allowbreak4; 107:17-\allowbreak22; 118:18; 147:3 Ge 20:17 Ex 15:26}
\crossref{Ps}{29}{4}{Ps 16:10; 40:1,\allowbreak2; 56:13; 71:20; 86:13}
\crossref{Ps}{29}{5}{Ps 32:11; 33:1-\allowbreak3; 97:12; 103:20-\allowbreak22; 132:9; 135:19-\allowbreak21; 148:14; 149:1}
\crossref{Ps}{29}{6}{29:103:9,\allowbreak17 Isa 26:20; 54:7,\allowbreak8; 57:15,\allowbreak16 2Co 4:17}
\crossref{Ps}{29}{7}{Job 29:18-\allowbreak20 Isa 47:7; 56:12 Da 4:30 Lu 12:19 2Co 12:7}
\crossref{Ps}{29}{8}{29:5; 5:12; 18:35,\allowbreak36; 44:3; 89:17 Job 10:12}
\crossref{Ps}{29}{9}{Ps 34:6; 77:1,\allowbreak2; 130:1,\allowbreak2 1Co 12:8,\allowbreak9 Php 4:6,\allowbreak7}
\crossref{Ps}{29}{10}{Ps 6:5; 88:10-\allowbreak12; 115:17,\allowbreak18; 118:17 Ec 9:10 Isa 38:18}
\crossref{Ps}{29}{11}{Ps 51:1,\allowbreak2; 143:1,\allowbreak7-\allowbreak9}
\crossref{Ps}{30}{1}{Ps 22:4,\allowbreak5; 25:2; 71:1,\allowbreak2 Isa 49:23 Ro 5:5; 10:11}
\crossref{Ps}{30}{2}{Ps 22:4,\allowbreak5; 25:2; 71:1,\allowbreak2 Isa 49:23 Ro 5:5; 10:11}
\crossref{Ps}{30}{3}{30:71:2 86:1 130:2 Pr 22:17}
\crossref{Ps}{30}{4}{Ps 23:2,\allowbreak3; 25:11; 79:9 Jos 7:9 Jer 14:7 Eze 36:21,\allowbreak22 Eph 1:12}
\crossref{Ps}{30}{5}{Ps 25:15; 35:7; 57:6; 124:7; 140:5 Pr 29:5 2Ti 2:26}
\crossref{Ps}{30}{6}{Lu 23:46 Ac 7:59 2Ti 1:12}
\crossref{Ps}{30}{7}{Ps 26:5; 139:2}
\crossref{Ps}{30}{8}{Ps 13:5 Isa 49:13 Jer 33:11}
\crossref{Ps}{30}{9}{30:88:8 De 32:30 1Sa 17:46; 24:18; 26:8 Job 16:11 Isa 19:4}
\crossref{Ps}{30}{10}{Ps 6:7; 88:9 Job 17:7 La 4:17; 5:17}
\crossref{Ps}{30}{11}{30:78:33; 88:15 102:3-\allowbreak28 Job 3:24 Ro 9:2}
\crossref{Ps}{30}{12}{Ps 22:6; 69:19,\allowbreak20; 89:50,\allowbreak51 Isa 49:7; 53:4,\allowbreak5 Mt 27:39-\allowbreak44 Ro 15:3}
\crossref{Ps}{31}{1}{Ps 1:1,\allowbreak2; 40:4; 84:12; 89:15; 106:3; 119:1,\allowbreak2; 128:1 Jer 17:7,\allowbreak8}
\crossref{Ps}{31}{2}{Le 17:4 Ro 5:13 2Co 5:19-\allowbreak21}
\crossref{Ps}{31}{3}{Ge 3:8-\allowbreak19 1Sa 31:13 2Sa 11:27; 12:1-\allowbreak12; 21:12-\allowbreak14 Pr 28:13}
\crossref{Ps}{31}{4}{Ps 38:2-\allowbreak8; 39:10,\allowbreak11 1Sa 5:6,\allowbreak7,\allowbreak9,\allowbreak11; 6:9 Job 16:21; 33:7}
\crossref{Ps}{31}{5}{Ps 38:18; 51:3-\allowbreak5 Le 26:39,\allowbreak40 Jos 7:19 2Sa 12:13; 24:10 Job 33:27}
\crossref{Ps}{31}{6}{Ps 34:2-\allowbreak5; 40:3; 51:12,\allowbreak13 2Co 1:4 1Ti 1:16}
\crossref{Ps}{31}{7}{Ps 9:9; 27:5; 31:20; 119:114; 143:9 Jer 36:26 Col 3:3}
\crossref{Ps}{31}{8}{Ps 34:11 Pr 3:1; 4:1-\allowbreak13; 8:10,\allowbreak11 Mt 11:29}
\crossref{Ps}{31}{9}{Pr 26:3 Jer 31:18 Jas 3:3; 4:7-\allowbreak10}
\crossref{Ps}{31}{10}{Ps 16:4; 34:19-\allowbreak21; 140:11 Pr 13:21 Ec 8:12 Isa 3:11; 57:21 Ro 2:8,\allowbreak9}
\crossref{Ps}{31}{11}{Ps 33:1; 64:10; 68:3; 97:12 De 12:12 1Sa 2:1 Ro 5:11 Php 3:1,\allowbreak3; 4:4}
\crossref{Ps}{31}{12}{}
\crossref{Ps}{31}{13}{}
\crossref{Ps}{31}{14}{}
\crossref{Ps}{31}{15}{}
\crossref{Ps}{31}{16}{}
\crossref{Ps}{31}{17}{}
\crossref{Ps}{31}{18}{}
\crossref{Ps}{31}{19}{}
\crossref{Ps}{31}{20}{}
\crossref{Ps}{31}{21}{}
\crossref{Ps}{31}{22}{}
\crossref{Ps}{31}{23}{}
\crossref{Ps}{31}{24}{}
\crossref{Ps}{32}{1}{Ps 32:11; 97:12 1Co 1:30,\allowbreak31 Php 4:4}
\crossref{Ps}{32}{2}{32:81:2,\allowbreak3 92:3 98:4,\allowbreak5 144:9 149:3 150:3-\allowbreak6 Ex 15:20 2Sa 6:5}
\crossref{Ps}{32}{3}{32:96:1 98:1 144:9 149:1 Isa 42:10 Eph 5:19 Col 3:16 Re 5:9; 14:3}
\crossref{Ps}{32}{4}{Ps 12:6; 19:8; 119:75,\allowbreak128 Pr 30:5 Mic 2:7 Ro 7:12}
\crossref{Ps}{32}{5}{Ps 11:7; 45:7; 99:4 Heb 1:9 Re 15:3,\allowbreak4}
\crossref{Ps}{32}{6}{32:9; 148:1-\allowbreak5 Ge 1:1; 6:7 Joh 1:1-\allowbreak3 Heb 11:3 2Pe 3:5}
\crossref{Ps}{32}{7}{32:104:6-\allowbreak9 Ge 1:9,\allowbreak10 Job 26:10; 38:8-\allowbreak11 Pr 8:29 Jer 5:22}
\crossref{Ps}{32}{8}{Ps 22:27; 96:9,\allowbreak10 Jer 10:7-\allowbreak12 Da 6:25,\allowbreak26 Re 14:6,\allowbreak7; 15:4}
\crossref{Ps}{32}{9}{32:6; 148:5,\allowbreak6 Ge 1:3 Heb 11:3}
\crossref{Ps}{32}{10}{Ps 2:1-\allowbreak4; 9:15 Ex 1:10-\allowbreak12 2Sa 15:31,\allowbreak34; 17:14,\allowbreak23 Job 5:12,\allowbreak13}
\crossref{Ps}{32}{11}{Job 23:13 Pr 19:21 Isa 14:24,\allowbreak27; 46:10 La 3:37 Eze 38:10-\allowbreak23}
\crossref{Ps}{33}{1}{}
\crossref{Ps}{33}{2}{Ps 44:8; 105:3 Isa 45:25 Jer 9:24 1Co 1:31 2Co 10:17}
\crossref{Ps}{33}{3}{Ps 44:8; 105:3 Isa 45:25 Jer 9:24 1Co 1:31 2Co 10:17}
\crossref{Ps}{33}{4}{Ps 35:27; 40:16; 69:30 Lu 1:46 Ac 19:17 Php 1:20}
\crossref{Ps}{33}{5}{Ps 18:6; 22:24; 31:22; 77:1,\allowbreak2; 116:1-\allowbreak6 Jon 2:2 Mt 7:7 Lu 11:9}
\crossref{Ps}{33}{6}{33:123:1,\allowbreak2 Isa 45:22 Heb 12:2}
\crossref{Ps}{33}{7}{Ps 3:4; 10:17; 40:17; 66:16-\allowbreak20}
\crossref{Ps}{33}{8}{33:91:11 2Ki 6:17; 19:35 Da 6:22 Mt 18:10 Lu 16:22 Heb 1:14}
\crossref{Ps}{33}{9}{Ps 63:5; 119:103 So 2:3; 5:1 Heb 6:4,\allowbreak5 1Pe 2:2,\allowbreak3 1Jo 1:1-\allowbreak3}
\crossref{Ps}{33}{10}{Ps 22:23; 31:23; 89:7 Ge 22:12 Isa 8:13,\allowbreak14 Ho 3:5 Re 15:3,\allowbreak4}
\crossref{Ps}{33}{11}{33:104:21 Job 4:10,\allowbreak11 Lu 1:51-\allowbreak53}
\crossref{Ps}{33}{12}{Pr 4:1; 7:24; 8:17,\allowbreak32; 22:6 Ec 11:9,\allowbreak10; 12:1 Isa 28:9 Mt 18:2-\allowbreak4}
\crossref{Ps}{33}{13}{Ps 21:4; 91:16 De 6:2; 30:20 1Pe 3:10,\allowbreak11}
\crossref{Ps}{33}{14}{Ps 39:1 Pr 18:21 Mt 12:35-\allowbreak37 Jas 1:19,\allowbreak26; 3:2,\allowbreak5-\allowbreak10}
\crossref{Ps}{33}{15}{Ps 37:27 Job 28:28 Pr 3:7; 8:13; 13:14; 16:16,\allowbreak17 Isa 1:16,\allowbreak17}
\crossref{Ps}{33}{16}{Ps 33:18 Job 36:7 1Pe 3:12}
\crossref{Ps}{33}{17}{Le 17:10; 26:17 Jer 44:11 Eze 14:7,\allowbreak8 Am 9:4}
\crossref{Ps}{33}{18}{33:6,\allowbreak15,\allowbreak19; 91:15; 145:18-\allowbreak20 2Ch 32:20,\allowbreak21,\allowbreak24 Isa 65:24 Ac 12:5-\allowbreak11}
\crossref{Ps}{33}{19}{33:75:1 85:9 119:151 145:18 Isa 55:6}
\crossref{Ps}{33}{20}{33:71:20 Job 5:19; 30:9-\allowbreak31; 42:12 Pr 24:16 Joh 16:33 Ac 14:22}
\crossref{Ps}{33}{21}{Ps 35:10; 91:12 Da 6:22-\allowbreak24 Joh 19:36}
\crossref{Ps}{33}{22}{Ps 37:30-\allowbreak40; 94:23 Isa 3:11}
\crossref{Ps}{34}{1}{Ps 43:1; 119:154 1Sa 24:15 Pr 22:23; 23:11 Jer 51:36 La 3:58}
\crossref{Ps}{34}{2}{Ps 7:12,\allowbreak13 Ex 15:3 De 32:41,\allowbreak42 Isa 13:5; 42:13}
\crossref{Ps}{34}{3}{Ps 27:2; 76:10 1Sa 23:26,\allowbreak27 Job 1:10 Isa 8:9,\allowbreak10; 10:12 Ac 4:28}
\crossref{Ps}{34}{4}{34:26; 31:17,\allowbreak18; 40:14,\allowbreak15; 70:2,\allowbreak3; 71:24}
\crossref{Ps}{34}{5}{Ps 1:4; 83:13-\allowbreak17 Job 21:18 Isa 17:13; 29:5 Ho 13:3}
\crossref{Ps}{34}{6}{34:73:18 Pr 4:19 Jer 13:16; 23:12}
\crossref{Ps}{34}{7}{Ps 7:3-\allowbreak5; 25:3; 64:4 Joh 15:25}
\crossref{Ps}{34}{8}{Ps 7:15,\allowbreak16; 57:6; 141:9,\allowbreak10 Pr 5:22}
\crossref{Ps}{34}{9}{Ps 13:5; 21:1; 33:21; 48:11; 58:10,\allowbreak11; 68:1-\allowbreak3 1Sa 2:1 Isa 61:10}
\crossref{Ps}{34}{10}{Ps 22:14; 32:3; 34:20; 38:3; 51:8; 102:3 Job 33:19-\allowbreak25}
\crossref{Ps}{34}{11}{Ps 27:12 1Sa 24:9; 25:10 Mt 26:59,\allowbreak60 Ac 6:13; 24:5,\allowbreak6,\allowbreak12,\allowbreak13}
\crossref{Ps}{34}{12}{Ps 38:20; 109:3-\allowbreak5 1Sa 19:4,\allowbreak5,\allowbreak15; 22:13,\allowbreak14 Pr 17:13 Jer 18:20}
\crossref{Ps}{34}{13}{34:69:10,\allowbreak11 Job 30:25 Mt 5:44 Ro 12:14,\allowbreak15}
\crossref{Ps}{34}{14}{}
\crossref{Ps}{34}{15}{34:25,\allowbreak26; 41:8; 71:10,\allowbreak11 Job 31:29 Pr 17:5; 24:17,\allowbreak18}
\crossref{Ps}{34}{16}{1Sa 20:24-\allowbreak42 Isa 1:14,\allowbreak15 Joh 18:28 1Co 5:8}
\crossref{Ps}{34}{17}{Ps 6:3; 13:1,\allowbreak2; 74:9,\allowbreak10; 89:46; 94:3,\allowbreak4}
\crossref{Ps}{34}{18}{Ps 22:22-\allowbreak25,\allowbreak31; 40:9,\allowbreak10; 69:30-\allowbreak34; 111:1; 116:14,\allowbreak18 Heb 2:12}
\crossref{Ps}{34}{19}{34:15; 13:4; 25:2; 38:16 Joh 16:20-\allowbreak22 Re 11:7-\allowbreak10}
\crossref{Ps}{34}{20}{34:120:5-\allowbreak7}
\crossref{Ps}{34}{21}{Ps 22:13 Isa 9:12 Lu 11:53,\allowbreak54}
\crossref{Ps}{34}{22}{Ex 3:7 Ac 7:34}
\crossref{Ps}{35}{1}{Ps 18:1; 90:1}
\crossref{Ps}{35}{2}{Ps 10:3; 49:18 De 29:19 Jer 2:23,\allowbreak34,\allowbreak35; 17:9 Ho 12:7,\allowbreak8 Lu 10:29}
\crossref{Ps}{35}{3}{Ps 10:3; 49:18 De 29:19 Jer 2:23,\allowbreak34,\allowbreak35; 17:9 Ho 12:7,\allowbreak8 Lu 10:29}
\crossref{Ps}{35}{4}{Ps 5:9; 12:2,\allowbreak3; 55:21; 58:3; 140:3 1Sa 18:21; 19:6,\allowbreak7; 26:21}
\crossref{Ps}{35}{5}{Ps 38:12 1Sa 19:11 Es 5:14; 6:4 Pr 4:16 Ho 7:6,\allowbreak7 Mic 2:1 Mt 27:1}
\crossref{Ps}{35}{6}{Ps 52:1; 57:10; 103:11; 108:4 Isa 55:7-\allowbreak9}
\crossref{Ps}{35}{7}{35:71:19; 97:2 145:17 Ge 18:25 De 32:4 Isa 45:19,\allowbreak21-\allowbreak24 Ro 3:25}
\crossref{Ps}{35}{8}{Ps 31:19; 86:5,\allowbreak15; 145:7,\allowbreak8 Ex 34:6 Joh 3:16 1Jo 3:1; 4:9,\allowbreak10}
\crossref{Ps}{35}{9}{Ps 16:11; 17:15; 63:5; 65:4 So 5:1 Isa 25:6; 55:1,\allowbreak2 Jer 31:12-\allowbreak14}
\crossref{Ps}{35}{10}{Isa 12:3 Jer 2:13 Joh 4:10,\allowbreak14; 7:37-\allowbreak39 Re 21:6; 22:17}
\crossref{Ps}{35}{11}{35:103:17 Jer 31:3 Joh 15:9,\allowbreak10 1Pe 1:5}
\crossref{Ps}{35}{12}{Ps 10:2; 12:3-\allowbreak5; 119:51,\allowbreak69,\allowbreak85,\allowbreak122; 123:3,\allowbreak4 Job 40:11,\allowbreak12 Isa 51:23}
\crossref{Ps}{35}{13}{Ps 9:16; 55:23; 58:10,\allowbreak11; 64:7-\allowbreak9 Jud 5:31 2Th 1:8,\allowbreak9 Re 15:4; 19:1-\allowbreak6}
\crossref{Ps}{35}{14}{}
\crossref{Ps}{35}{15}{}
\crossref{Ps}{35}{16}{}
\crossref{Ps}{35}{17}{}
\crossref{Ps}{35}{18}{}
\crossref{Ps}{35}{19}{}
\crossref{Ps}{35}{20}{}
\crossref{Ps}{35}{21}{}
\crossref{Ps}{35}{22}{}
\crossref{Ps}{35}{23}{}
\crossref{Ps}{35}{24}{}
\crossref{Ps}{35}{25}{}
\crossref{Ps}{35}{26}{}
\crossref{Ps}{35}{27}{}
\crossref{Ps}{35}{28}{}
\crossref{Ps}{36}{1}{36:7 1Sa 1:6-\allowbreak8 Pr 19:3; 24:1,\allowbreak19}
\crossref{Ps}{36}{2}{36:35,\allowbreak36; 73:17-\allowbreak20; 90:5,\allowbreak6; 92:7; 129:5-\allowbreak7 Job 20:5-\allowbreak9 Jas 1:10,\allowbreak11}
\crossref{Ps}{36}{3}{Ps 4:5; 26:1 Isa 1:16-\allowbreak19; 50:10 Jer 17:7,\allowbreak8 1Co 15:57,\allowbreak58}
\crossref{Ps}{36}{4}{Ps 43:4; 104:34 Job 27:10; 34:9 So 2:3 Isa 58:14 1Pe 1:8}
\crossref{Ps}{36}{5}{Ps 22:8; 55:22 Pr 16:3}
\crossref{Ps}{36}{6}{Ps 31:20 Isa 54:17 Mic 7:8,\allowbreak9 1Co 4:5}
\crossref{Ps}{36}{7}{Ps 62:1 Jos 10:12 Jon 1:11}
\crossref{Ps}{36}{8}{Job 5:2; 18:4 Pr 14:29; 16:32 Eph 4:26,\allowbreak31 Jas 1:19,\allowbreak20; 3:14-\allowbreak18}
\crossref{Ps}{36}{9}{36:35,\allowbreak36; 55:23 Job 20:23-\allowbreak29; 27:13,\allowbreak14-\allowbreak23}
\crossref{Ps}{36}{10}{36:73:18-\allowbreak20 Job 24:24 Heb 10:36,\allowbreak37 1Pe 4:7 Re 6:10,\allowbreak11}
\crossref{Ps}{36}{11}{Mt 5:5 Ga 5:22,\allowbreak23 1Ti 6:11 Jas 1:21; 3:13}
\crossref{Ps}{36}{12}{36:32 1Sa 18:21; 23:7-\allowbreak9 2Sa 15:10-\allowbreak12 Es 3:6 Mt 26:4,\allowbreak16}
\crossref{Ps}{37}{1}{37:70:1}
\crossref{Ps}{37}{2}{37:70:1}
\crossref{Ps}{37}{3}{Ps 21:12; 64:7 Job 6:4 La 3:12}
\crossref{Ps}{37}{4}{Ps 31:9 2Ch 26:19 Job 2:7,\allowbreak8; 33:19-\allowbreak22 Isa 1:5,\allowbreak6}
\crossref{Ps}{37}{5}{Ps 40:12 Ezr 9:6}
\crossref{Ps}{37}{6}{}
\crossref{Ps}{37}{7}{Ps 35:14; 42:5}
\crossref{Ps}{37}{8}{Ps 41:8 2Ch 21:18,\allowbreak19 Job 7:5; 30:18 Ac 12:23}
\crossref{Ps}{37}{9}{Ps 22:1,\allowbreak2; 32:3 Job 3:24; 30:28 Isa 59:11}
\crossref{Ps}{37}{10}{37:102:5,\allowbreak20 Joh 1:48 Ro 8:22,\allowbreak23,\allowbreak26,\allowbreak27 2Co 5:2}
\crossref{Ps}{37}{11}{Ps 42:1; 119:81-\allowbreak83; 143:4-\allowbreak7 Isa 21:4}
\crossref{Ps}{37}{12}{Ps 31:11 Job 6:21-\allowbreak23; 19:13-\allowbreak17 Mt 26:56 Joh 16:32}
\crossref{Ps}{37}{13}{Ps 10:9; 64:2-\allowbreak5; 119:110; 140:5; 141:9 2Sa 17:1-\allowbreak3 Lu 20:19,\allowbreak20}
\crossref{Ps}{37}{14}{Ps 39:2,\allowbreak9 2Sa 16:10-\allowbreak12 Isa 53:7 1Pe 2:23}
\crossref{Ps}{37}{15}{Am 5:13 Mic 7:5 Mr 15:3-\allowbreak5 Joh 8:6}
\crossref{Ps}{37}{16}{Ps 39:7; 123:1-\allowbreak3}
\crossref{Ps}{37}{17}{Ps 13:3,\allowbreak4; 35:24-\allowbreak26}
\crossref{Ps}{37}{18}{Ps 35:15}
\crossref{Ps}{37}{19}{Ps 32:5; 51:3 Job 31:33; 33:27 Pr 28:13}
\crossref{Ps}{37}{20}{Ps 3:1; 25:19; 56:1,\allowbreak2; 59:1-\allowbreak3}
\crossref{Ps}{37}{21}{Ps 7:4; 35:12; 109:3-\allowbreak5 1Sa 19:4-\allowbreak6; 23:5,\allowbreak12; 25:16,\allowbreak21 Jer 18:20}
\crossref{Ps}{37}{22}{Ps 22:1,\allowbreak11,\allowbreak19,\allowbreak24; 35:21,\allowbreak22}
\crossref{Ps}{37}{23}{Ps 40:13,\allowbreak17; 70:1,\allowbreak5; 71:12; 141:1}
\crossref{Ps}{37}{24}{}
\crossref{Ps}{37}{25}{}
\crossref{Ps}{37}{26}{}
\crossref{Ps}{37}{27}{}
\crossref{Ps}{37}{28}{}
\crossref{Ps}{37}{29}{}
\crossref{Ps}{37}{30}{}
\crossref{Ps}{37}{31}{}
\crossref{Ps}{37}{32}{}
\crossref{Ps}{37}{33}{}
\crossref{Ps}{37}{34}{}
\crossref{Ps}{37}{35}{}
\crossref{Ps}{37}{36}{}
\crossref{Ps}{37}{37}{}
\crossref{Ps}{37}{38}{}
\crossref{Ps}{37}{39}{}
\crossref{Ps}{37}{40}{}
\crossref{Ps}{38}{1}{}
\crossref{Ps}{38}{2}{Ps 38:13,\allowbreak14 Isa 53:7 Mt 27:12-\allowbreak14}
\crossref{Ps}{38}{3}{Ps 38:13,\allowbreak14 Isa 53:7 Mt 27:12-\allowbreak14}
\crossref{Ps}{38}{4}{Jer 20:9 Eze 3:14 Lu 24:32}
\crossref{Ps}{38}{5}{38:90:12; 119:84 Job 14:13}
\crossref{Ps}{38}{6}{38:90:4,\allowbreak5,\allowbreak9,\allowbreak10 Ge 47:9 Job 7:6; 9:25,\allowbreak26; 14:1,\allowbreak2 Jas 4:14}
\crossref{Ps}{38}{7}{Ec 1:14; 2:17,\allowbreak18,\allowbreak20,\allowbreak21; 4:7,\allowbreak8; 6:11,\allowbreak12; 12:8,\allowbreak13 Isa 55:2}
\crossref{Ps}{38}{8}{38:130:5,\allowbreak6 Ge 49:18 Lu 2:25}
\crossref{Ps}{38}{9}{Ps 25:11,\allowbreak18; 51:7-\allowbreak10,\allowbreak14; 65:3; 130:8 Mic 7:19 Mt 1:21 Tit 2:14}
\crossref{Ps}{38}{10}{Ps 38:13 Le 10:3 1Sa 3:18 2Sa 16:10 Job 1:21; 2:10; 40:4,\allowbreak5 Da 4:35}
\crossref{Ps}{38}{11}{Ps 25:16,\allowbreak17 1Sa 6:5 Job 9:34; 13:21}
\crossref{Ps}{38}{12}{Ps 38:1-\allowbreak8; 90:7-\allowbreak10 1Co 5:5; 11:30-\allowbreak32 Heb 12:6 Re 3:19}
\crossref{Ps}{38}{13}{Ps 56:8; 116:3 2Sa 16:12}
\crossref{Ps}{38}{14}{Job 10:20,\allowbreak21; 14:5,\allowbreak6}
\crossref{Ps}{38}{15}{}
\crossref{Ps}{38}{16}{}
\crossref{Ps}{38}{17}{}
\crossref{Ps}{38}{18}{}
\crossref{Ps}{38}{19}{}
\crossref{Ps}{38}{20}{}
\crossref{Ps}{38}{21}{}
\crossref{Ps}{38}{22}{}
\crossref{Ps}{39}{1}{Ps 27:13,\allowbreak14; 37:7 Jas 5:7-\allowbreak11}
\crossref{Ps}{39}{2}{Ps 27:13,\allowbreak14; 37:7 Jas 5:7-\allowbreak11}
\crossref{Ps}{39}{3}{Ps 18:16,\allowbreak17; 71:20; 86:13; 116:3; 142:6,\allowbreak7; 143:3 Isa 24:22 Jon 2:5,\allowbreak6}
\crossref{Ps}{39}{4}{Ps 33:3; 144:9 Re 5:9,\allowbreak10; 14:3}
\crossref{Ps}{39}{5}{Ps 2:12; 34:8; 84:11,\allowbreak12; 118:8,\allowbreak9 Jer 17:7,\allowbreak8 Ro 15:12,\allowbreak13}
\crossref{Ps}{39}{6}{39:136:4 Ex 11:8; 15:11 Job 5:9; 9:10; 26:14}
\crossref{Ps}{39}{7}{Ps 50:8; 51:16 1Sa 15:22 Isa 1:11; 66:3 Jer 7:21-\allowbreak23 Ho 6:6 Mt 9:13}
\crossref{Ps}{39}{8}{Heb 10:7-\allowbreak9}
\crossref{Ps}{39}{9}{39:112:1; 119:16,\allowbreak24,\allowbreak47,\allowbreak92 Job 23:12 Jer 15:16 Joh 4:34 Ro 7:22}
\crossref{Ps}{39}{10}{Ps 22:22,\allowbreak25; 35:18; 71:15-\allowbreak18 Mr 16:15,\allowbreak16 Lu 4:16-\allowbreak22 Heb 2:12}
\crossref{Ps}{39}{11}{Eze 2:7; 3:17,\allowbreak18 Ac 20:20,\allowbreak21,\allowbreak26,\allowbreak27 Ro 10:9,\allowbreak10 1Th 1:8 Re 22:17}
\crossref{Ps}{39}{12}{Ps 23:6; 43:3; 57:3; 61:7; 85:10 Heb 5:7}
\crossref{Ps}{39}{13}{Ps 22:11-\allowbreak19 Heb 4:15}
\crossref{Ps}{40}{1}{40:112:9 De 15:7-\allowbreak11 Job 29:12-\allowbreak16; 31:16-\allowbreak20 Pr 14:21; 19:17}
\crossref{Ps}{40}{2}{40:112:9 De 15:7-\allowbreak11 Job 29:12-\allowbreak16; 31:16-\allowbreak20 Pr 14:21; 19:17}
\crossref{Ps}{40}{3}{Ps 33:19; 91:3-\allowbreak7 Jer 45:4,\allowbreak5}
\crossref{Ps}{40}{4}{40:73:26 2Ki 1:6,\allowbreak16; 20:5,\allowbreak6 2Co 4:16,\allowbreak17 Php 2:26,\allowbreak27}
\crossref{Ps}{40}{5}{Ps 32:5; 51:1-\allowbreak3}
\crossref{Ps}{40}{6}{Ps 22:6-\allowbreak8; 102:8}
\crossref{Ps}{40}{7}{Ps 12:2 Pr 26:24,\allowbreak25 Ne 6:1-\allowbreak14 Pr 26:24-\allowbreak26 Da 11:27 Mic 7:5-\allowbreak7}
\crossref{Ps}{40}{8}{Pr 16:28; 26:20}
\crossref{Ps}{40}{9}{Ps 38:3-\allowbreak7 Job 2:7,\allowbreak8 Lu 13:16}
\crossref{Ps}{40}{10}{Ps 55:12-\allowbreak14,\allowbreak20-\allowbreak22 2Sa 15:12 Job 19:19 Jer 20:10}
\crossref{Ps}{40}{11}{Ps 57:1; 109:21}
\crossref{Ps}{40}{12}{Ps 13:4; 31:8; 35:25; 86:17; 124:6 Jer 20:13 Col 2:15}
\crossref{Ps}{40}{13}{Ps 25:21; 94:18}
\crossref{Ps}{40}{14}{40:72:18,\allowbreak19 89:52 106:48 1Ch 29:10 Eph 1:3 Re 4:8; 5:9-\allowbreak14; 7:12}
\crossref{Ps}{40}{15}{}
\crossref{Ps}{40}{16}{}
\crossref{Ps}{40}{17}{}
\crossref{Ps}{41}{1}{Ps 44:1; 45:1; 46:1; 47:1; 48:1; 49:1; 84:1; 85:1}
\crossref{Ps}{41}{2}{Ps 44:1; 45:1; 46:1; 47:1; 48:1; 49:1; 84:1; 85:1}
\crossref{Ps}{41}{3}{Ps 36:8,\allowbreak9; 63:1 Joh 7:37 Re 22:1}
\crossref{Ps}{41}{4}{41:80:5 102:9 2Sa 16:12}
\crossref{Ps}{41}{5}{Ru 1:21 Job 29:2-\allowbreak25; 30:1-\allowbreak31 La 4:1 Lu 16:25}
\crossref{Ps}{41}{6}{41:11; 35:14; 43:5; 55:4,\allowbreak5; 61:2; 142:2,\allowbreak3; 143:3,\allowbreak4 1Sa 30:6}
\crossref{Ps}{41}{7}{Ps 22:1; 43:4; 88:1-\allowbreak3 Mt 26:39; 27:46}
\crossref{Ps}{41}{8}{Job 1:14-\allowbreak19; 10:17 Jer 4:20 Eze 7:26}
\crossref{Ps}{41}{9}{Ps 44:4; 133:3 Le 25:21 De 28:8 Mt 8:8}
\crossref{Ps}{41}{10}{Ps 18:2; 28:1; 62:2,\allowbreak6,\allowbreak7; 78:35}
\crossref{Ps}{41}{11}{41:3 Pr 12:18 Lu 2:35}
\crossref{Ps}{41}{12}{41:5; 43:5}
\crossref{Ps}{41}{13}{}
\crossref{Ps}{42}{1}{Ps 7:8; 26:1; 35:24; 75:7 1Co 4:4 1Pe 2:23}
\crossref{Ps}{42}{2}{Ps 28:7; 140:7 Ex 15:2 Isa 40:31; 45:24 Zec 10:12 Eph 6:10}
\crossref{Ps}{42}{3}{Ps 40:11; 57:3; 97:11; 119:105 2Sa 15:20 Mic 7:8,\allowbreak20 Joh 1:4,\allowbreak17}
\crossref{Ps}{42}{4}{Ps 66:13-\allowbreak15; 116:12-\allowbreak19}
\crossref{Ps}{42}{5}{Ps 42:5,\allowbreak11}
\crossref{Ps}{42}{6}{}
\crossref{Ps}{42}{7}{}
\crossref{Ps}{42}{8}{}
\crossref{Ps}{42}{9}{}
\crossref{Ps}{42}{10}{}
\crossref{Ps}{42}{11}{}
\crossref{Ps}{43}{1}{Ps 42:1}
\crossref{Ps}{43}{2}{Ps 42:1}
\crossref{Ps}{43}{3}{43:78:55; 80:8 105:44 135:10-\allowbreak12 136:17-\allowbreak22 Ex 15:17,\allowbreak19; 34:11}
\crossref{Ps}{43}{4}{De 4:37,\allowbreak38; 8:17,\allowbreak18 Jos 24:12 Zec 4:6 2Co 4:7}
\crossref{Ps}{43}{5}{43:74:12; 89:18 149:2 Isa 33:22}
\crossref{Ps}{44}{1}{Job 32:18-\allowbreak20 Pr 16:23 Mt 12:35}
\crossref{Ps}{44}{2}{Job 32:18-\allowbreak20 Pr 16:23 Mt 12:35}
\crossref{Ps}{44}{3}{So 2:3; 5:10-\allowbreak16 Zec 9:17 Mt 17:2 Joh 1:14 Col 1:15-\allowbreak18}
\crossref{Ps}{44}{4}{Isa 49:2; 63:1-\allowbreak6 Heb 4:12 Re 1:16; 19:15,\allowbreak21}
\crossref{Ps}{44}{5}{Re 6:2; 19:11}
\crossref{Ps}{44}{6}{Ps 21:12; 38:2 Nu 24:8 Zec 9:13,\allowbreak14}
\crossref{Ps}{44}{7}{44:89:29,\allowbreak36,\allowbreak37 93:2 145:13 Da 2:44 Lu 1:32,\allowbreak33 Heb 1:8}
\crossref{Ps}{44}{8}{Ps 33:5; 99:4 Mt 3:15 Heb 1:9; 7:26}
\crossref{Ps}{44}{9}{So 1:3,\allowbreak13; 3:6; 4:6,\allowbreak13,\allowbreak14; 5:1,\allowbreak5,\allowbreak13 Mt 2:11 Joh 19:39}
\crossref{Ps}{44}{10}{44:13; 72:10 So 6:8,\allowbreak9; 7:1 Isa 49:23; 60:10,\allowbreak11 Re 21:24}
\crossref{Ps}{44}{11}{So 2:10-\allowbreak13 Isa 55:1-\allowbreak3 2Co 6:17,\allowbreak18; 7:1}
\crossref{Ps}{44}{12}{So 1:8,\allowbreak12-\allowbreak16; 2:2,\allowbreak14; 4:1-\allowbreak5,\allowbreak7,\allowbreak9,\allowbreak10; 6:4; 7:1-\allowbreak10 Isa 62:4,\allowbreak5}
\crossref{Ps}{44}{13}{Isa 23:17,\allowbreak18 Ac 21:3-\allowbreak6}
\crossref{Ps}{44}{14}{44:9,\allowbreak10 So 7:1 Isa 61:10 1Pe 2:9 Re 19:7,\allowbreak8}
\crossref{Ps}{44}{15}{So 1:4 Joh 17:24 2Co 11:2}
\crossref{Ps}{44}{16}{Isa 35:10; 51:11; 55:12,\allowbreak13; 60:19,\allowbreak20; 61:10 Jude 1:24 Re 7:15-\allowbreak17}
\crossref{Ps}{44}{17}{Ps 22:30 Mt 19:29 Mr 10:29,\allowbreak30 Php 3:7,\allowbreak8}
\crossref{Ps}{44}{18}{Ps 22:30,\allowbreak31; 72:17-\allowbreak19; 145:4-\allowbreak7 Isa 59:21 Mal 1:11 Mt 26:13}
\crossref{Ps}{44}{19}{}
\crossref{Ps}{44}{20}{}
\crossref{Ps}{44}{21}{}
\crossref{Ps}{44}{22}{}
\crossref{Ps}{44}{23}{}
\crossref{Ps}{44}{24}{}
\crossref{Ps}{44}{25}{}
\crossref{Ps}{44}{26}{}
\crossref{Ps}{45}{1}{45:84:1 85:1 87:1}
\crossref{Ps}{45}{2}{45:84:1 85:1 87:1}
\crossref{Ps}{45}{3}{Ps 23:4; 27:3 Mt 8:24-\allowbreak26 Heb 13:6}
\crossref{Ps}{45}{4}{Ps 18:4; 93:3,\allowbreak4 Job 38:11 Isa 5:3; 17:12,\allowbreak13 Jer 5:22 Mt 7:25}
\crossref{Ps}{45}{5}{Ps 23:2; 36:8,\allowbreak9 Isa 8:6,\allowbreak7; 48:18 Eze 47:1-\allowbreak12 Re 22:1-\allowbreak3}
\crossref{Ps}{45}{6}{45:68:18 De 23:14 Isa 12:6 Eze 43:7,\allowbreak9 Ho 11:9 Joe 2:27 Zep 3:15}
\crossref{Ps}{45}{7}{Ps 2:1-\allowbreak4; 83:2-\allowbreak8 2Ch 14:9-\allowbreak13; 20:1,\allowbreak20-\allowbreak24 Isa 8:9,\allowbreak10; 37:21-\allowbreak36}
\crossref{Ps}{45}{8}{45:11 Nu 14:9 2Ch 13:12 Isa 8:10 Mt 28:20 Ro 8:31 2Ti 4:22}
\crossref{Ps}{45}{9}{Ps 66:5; 92:4-\allowbreak6; 111:2,\allowbreak3 Nu 23:23}
\crossref{Ps}{45}{10}{Isa 2:4; 11:9; 60:18 Mic 4:3,\allowbreak4}
\crossref{Ps}{45}{11}{Hab 2:20 Zec 2:13}
\crossref{Ps}{45}{12}{45:1,\allowbreak7; 48:3 De 33:27 Jer 16:19}
\crossref{Ps}{45}{13}{}
\crossref{Ps}{45}{14}{}
\crossref{Ps}{45}{15}{}
\crossref{Ps}{45}{16}{}
\crossref{Ps}{45}{17}{}
\crossref{Ps}{46}{1}{Ps 46:1}
\crossref{Ps}{46}{2}{Ps 46:1}
\crossref{Ps}{46}{3}{Ps 65:5; 66:3-\allowbreak5; 68:35; 76:12; 99:3; 145:6 De 7:21; 28:58 Ne 1:5}
\crossref{Ps}{46}{4}{Ps 18:47; 81:14 De 33:29}
\crossref{Ps}{46}{5}{De 11:12 Jer 3:19 Eze 20:6 Mt 25:34 1Co 3:22,\allowbreak23 Eph 1:18}
\crossref{Ps}{46}{6}{Ps 24:7-\allowbreak10; 68:17-\allowbreak19,\allowbreak24,\allowbreak25,\allowbreak33 Lu 24:51-\allowbreak53 Ac 1:5-\allowbreak11 Eph 4:8-\allowbreak10}
\crossref{Ps}{46}{7}{46:96:1,\allowbreak2 117:1,\allowbreak2 149:1-\allowbreak3 Ex 15:21 1Ch 16:9; 29:20 Isa 12:4-\allowbreak6}
\crossref{Ps}{46}{8}{46:2,\allowbreak8 Zec 14:9 Re 11:15}
\crossref{Ps}{46}{9}{Ps 22:27-\allowbreak29; 93:1; 96:10; 97:1; 99:1; 110:6 1Ch 16:31 Re 19:6}
\crossref{Ps}{46}{10}{Ge 17:7,\allowbreak8 Ex 3:6,\allowbreak15 Isa 41:8-\allowbreak10 Mt 22:32 Ro 4:11,\allowbreak12 Ga 3:29}
\crossref{Ps}{46}{11}{}
\crossref{Ps}{47}{1}{Ps 46:1}
\crossref{Ps}{47}{2}{Ps 46:1}
\crossref{Ps}{47}{3}{Ps 50:2 Jer 3:19 La 2:15 Da 8:9; 11:16}
\crossref{Ps}{47}{4}{47:76:1-\allowbreak5 125:1 2Ch 12:7; 14:9-\allowbreak15; 20:1-\allowbreak37 Isa 4:5,\allowbreak6; 37:33-\allowbreak36}
\crossref{Ps}{47}{5}{47:83:2-\allowbreak8 2Sa 10:6,\allowbreak14,\allowbreak16-\allowbreak19 Isa 7:1; 8:8-\allowbreak10; 10:8; 29:5-\allowbreak8}
\crossref{Ps}{47}{6}{Ex 14:25 2Ki 7:6,\allowbreak7; 19:35-\allowbreak37}
\crossref{Ps}{47}{7}{Ex 15:15,\allowbreak16 Isa 13:6-\allowbreak8 Da 5:6}
\crossref{Ps}{47}{8}{Eze 27:25,\allowbreak26}
\crossref{Ps}{47}{9}{Ps 44:1,\allowbreak2; 78:3-\allowbreak6 Isa 38:19}
\crossref{Ps}{48}{1}{Ps 46:1; 48:1}
\crossref{Ps}{48}{2}{Ps 46:1; 48:1}
\crossref{Ps}{48}{3}{Ps 62:9 1Sa 2:7,\allowbreak8 Job 34:19 Pr 22:2 Jer 5:4,\allowbreak5 Jas 1:9-\allowbreak11; 2:1-\allowbreak7}
\crossref{Ps}{48}{4}{De 32:2 Job 33:3,\allowbreak33 Pr 4:1,\allowbreak2; 8:6-\allowbreak11; 22:17,\allowbreak20,\allowbreak21 2Ti 3:15-\allowbreak17}
\crossref{Ps}{48}{5}{48:78:2 Mt 13:35}
\crossref{Ps}{48}{6}{Ps 27:1,\allowbreak2; 46:1,\allowbreak2 Isa 41:10,\allowbreak11 Ac 27:24 Ro 8:33,\allowbreak34 Php 1:28}
\crossref{Ps}{48}{7}{Ps 52:7; 62:10 Job 31:24,\allowbreak25 Pr 10:15; 23:5 Mr 10:24 1Ti 6:17}
\crossref{Ps}{48}{8}{Mt 16:26; 20:28 1Ti 2:6 1Pe 1:18}
\crossref{Ps}{48}{9}{Job 36:18,\allowbreak19}
\crossref{Ps}{48}{10}{48:89:48 Pr 10:2; 11:4 Ec 8:8 Zec 1:5 Lu 16:22,\allowbreak23}
\crossref{Ps}{48}{11}{Ec 2:16-\allowbreak21; 9:1,\allowbreak2 Ro 5:12-\allowbreak14 Heb 9:27}
\crossref{Ps}{48}{12}{}
\crossref{Ps}{48}{13}{48:20; 39:5; 82:7 Jas 1:10,\allowbreak11 1Pe 1:24}
\crossref{Ps}{48}{14}{Lu 12:20 1Co 3:19}
\crossref{Ps}{49}{1}{49:73:1 83:1}
\crossref{Ps}{49}{2}{49:68:24 Isa 12:6; 26:21 Ho 5:15 Hab 2:20 Heb 12:22-\allowbreak26}
\crossref{Ps}{49}{3}{Ps 48:14; 68:20 Re 22:20}
\crossref{Ps}{49}{4}{49:6 De 4:36; 30:19; 31:28; 32:1 Isa 1:2 Mic 6:1,\allowbreak2}
\crossref{Ps}{49}{5}{Mt 24:31 1Th 4:16,\allowbreak17 2Th 2:1}
\crossref{Ps}{49}{6}{49:97:6 Ro 2:5 Re 16:5-\allowbreak7; 19:2}
\crossref{Ps}{49}{7}{49:81:8 Isa 1:18 Jer 2:4,\allowbreak5,\allowbreak9 Mic 6:1-\allowbreak8}
\crossref{Ps}{49}{8}{Ps 40:6-\allowbreak8; 51:16 Isa 1:11-\allowbreak31 Jer 7:21-\allowbreak23 Ho 6:6 Heb 10:4-\allowbreak10}
\crossref{Ps}{49}{9}{Isa 43:23,\allowbreak24 Mic 6:6-\allowbreak8 Ac 17:25 Heb 10:4-\allowbreak6}
\crossref{Ps}{49}{10}{Ps 8:6-\allowbreak8; 104:24,\allowbreak25 Ge 1:24,\allowbreak25; 2:19; 8:17; 9:2,\allowbreak3 1Ch 29:14-\allowbreak16}
\crossref{Ps}{49}{11}{49:104:12; 147:9 Ge 1:20-\allowbreak22 Job 38:41; 39:13-\allowbreak18,\allowbreak26-\allowbreak30 Mt 6:26}
\crossref{Ps}{49}{12}{Ps 24:1,\allowbreak2; 115:15,\allowbreak16 Ex 19:5 De 10:14 Job 41:11 1Co 10:26-\allowbreak28}
\crossref{Ps}{49}{13}{49:3}
\crossref{Ps}{49}{14}{49:23; 69:30,\allowbreak31; 107:21,\allowbreak22; 147:1 Ho 14:2 1Th 5:18 Heb 13:15}
\crossref{Ps}{49}{15}{49:77:2 91:15 107:6-\allowbreak13,\allowbreak19,\allowbreak28 2Ch 33:12,\allowbreak13 Job 22:27 Zec 13:9}
\crossref{Ps}{49}{16}{Isa 48:22; 55:6,\allowbreak7 Eze 18:27}
\crossref{Ps}{49}{17}{Pr 1:7,\allowbreak28,\allowbreak29; 5:12,\allowbreak13; 8:36; 12:1 Joh 3:20 Ro 1:28; 2:21,\allowbreak23}
\crossref{Ps}{49}{18}{Pr 1:10-\allowbreak19 Isa 5:23 Mic 7:3 Ro 1:32 Eph 5:11-\allowbreak13}
\crossref{Ps}{49}{19}{Ps 52:3-\allowbreak4 Jer 9:5}
\crossref{Ps}{49}{20}{Ps 31:18 Mt 5:11 Lu 22:65}
\crossref{Ps}{50}{1}{2Sa 12:1-\allowbreak13}
\crossref{Ps}{50}{2}{2Sa 12:1-\allowbreak13}
\crossref{Ps}{50}{3}{50:7 Eze 36:25 Zec 13:1 1Co 6:11 Heb 9:13,\allowbreak14; 10:21,\allowbreak22 1Jo 1:7-\allowbreak9}
\crossref{Ps}{50}{4}{50:7 Eze 36:25 Zec 13:1 1Co 6:11 Heb 9:13,\allowbreak14; 10:21,\allowbreak22 1Jo 1:7-\allowbreak9}
\crossref{Ps}{50}{5}{Ps 32:5; 38:18 Le 26:40,\allowbreak41 Ne 9:2 Job 33:27 Pr 28:13 Lu 15:18-\allowbreak21}
\crossref{Ps}{50}{6}{Ge 9:6; 20:6; 39:9 Le 5:19; 6:2-\allowbreak7 2Sa 12:9,\allowbreak10,\allowbreak13,\allowbreak14 Jas 2:9,\allowbreak11}
\crossref{Ps}{50}{7}{Ps 58:3 Ge 5:3; 8:21 Job 14:4; 15:14-\allowbreak16 Joh 3:6 Ro 5:12 Eph 2:3}
\crossref{Ps}{50}{8}{Ps 26:2; 125:4 Ge 20:5,\allowbreak6 2Ki 20:3 1Ch 29:17 2Ch 31:20,\allowbreak21 Pr 2:21}
\crossref{Ps}{50}{9}{Le 14:4-\allowbreak7,\allowbreak49-\allowbreak52 Nu 19:18-\allowbreak20 Heb 9:19}
\crossref{Ps}{50}{10}{Ps 13:5; 30:11; 119:81,\allowbreak82; 126:5,\allowbreak6 Mt 5:4}
\crossref{Ps}{50}{11}{Isa 38:17 Jer 16:17 Mic 7:18,\allowbreak19}
\crossref{Ps}{50}{12}{2Co 5:17 Eph 2:10}
\crossref{Ps}{50}{13}{Ps 43:2; 71:9,\allowbreak18 Ge 4:14 2Ki 13:23; 17:18-\allowbreak23; 23:27 2Th 1:9}
\crossref{Ps}{50}{14}{50:85:6-\allowbreak8 Job 29:2,\allowbreak3 Isa 57:17,\allowbreak18 Jer 31:9-\allowbreak14}
\crossref{Ps}{50}{15}{Ps 32:5,\allowbreak8-\allowbreak10 Zec 3:1-\allowbreak8 Lu 22:32 Joh 21:15-\allowbreak17 Ac 2:38-\allowbreak41; 9:19-\allowbreak22}
\crossref{Ps}{50}{16}{Ps 26:9; 55:23 Ge 9:6; 42:22 2Sa 3:28; 11:15-\allowbreak17; 12:9; 21:1}
\crossref{Ps}{50}{17}{Ge 44:16 1Sa 2:9 Eze 16:63 Mt 22:12 Ro 3:19}
\crossref{Ps}{50}{18}{50:6 Ex 21:14 Nu 15:27,\allowbreak30,\allowbreak31; 35:31 De 22:22 Ho 6:6}
\crossref{Ps}{50}{19}{50:107:22 Mr 12:33 Ro 12:1 Php 4:18 Heb 13:16 1Pe 2:5}
\crossref{Ps}{50}{20}{Ps 25:22; 102:16; 122:6-\allowbreak9; 137:5,\allowbreak6 Isa 62:1,\allowbreak6,\allowbreak7 Jer 51:50}
\crossref{Ps}{50}{21}{Ps 66:13-\allowbreak15; 118:27 Eph 5:2}
\crossref{Ps}{50}{22}{}
\crossref{Ps}{50}{23}{}
\crossref{Ps}{51}{1}{Ps 54:3 1Sa 21:7; 22:9-\allowbreak19}
\crossref{Ps}{51}{2}{Ps 54:3 1Sa 21:7; 22:9-\allowbreak19}
\crossref{Ps}{51}{3}{Ps 50:19; 64:2-\allowbreak6; 140:2,\allowbreak3 Pr 6:16-\allowbreak19; 30:14 Jer 9:3,\allowbreak4; 18:18}
\crossref{Ps}{51}{4}{Ps 50:19; 64:2-\allowbreak6; 140:2,\allowbreak3 Pr 6:16-\allowbreak19; 30:14 Jer 9:3,\allowbreak4; 18:18}
\crossref{Ps}{51}{5}{Jer 4:22 Mic 3:2 Ro 1:25 2Ti 3:4}
\crossref{Ps}{51}{6}{1Sa 22:18,\allowbreak19 Jas 3:6-\allowbreak9}
\crossref{Ps}{51}{7}{Ps 7:14-\allowbreak16; 55:23; 64:7-\allowbreak10; 120:2-\allowbreak4; 140:9-\allowbreak11 Pr 12:19; 19:5,\allowbreak9}
\crossref{Ps}{51}{8}{Ps 37:34; 64:9; 97:8 Job 22:19 Mal 1:5 Re 15:4; 16:5-\allowbreak7; 18:20; 19:1,\allowbreak2}
\crossref{Ps}{51}{9}{Isa 14:16,\allowbreak17 Joh 19:5}
\crossref{Ps}{51}{10}{Ps 1:3; 92:12-\allowbreak14 Jer 11:16 Ho 14:6-\allowbreak8 Ro 11:24}
\crossref{Ps}{51}{11}{51:145:1,\allowbreak2; 146:2 Eph 3:20,\allowbreak21}
\crossref{Ps}{51}{12}{}
\crossref{Ps}{51}{13}{}
\crossref{Ps}{51}{14}{}
\crossref{Ps}{51}{15}{}
\crossref{Ps}{51}{16}{}
\crossref{Ps}{51}{17}{}
\crossref{Ps}{51}{18}{}
\crossref{Ps}{51}{19}{}
\crossref{Ps}{52}{1}{Ps 14:1-\allowbreak7; 92:6 Mt 5:22 Lu 12:20}
\crossref{Ps}{52}{2}{Ps 14:1-\allowbreak7; 92:6 Mt 5:22 Lu 12:20}
\crossref{Ps}{52}{3}{Ps 11:4; 33:13,\allowbreak14; 102:19 Jer 16:17; 23:24}
\crossref{Ps}{52}{4}{Ps 14:3 2Sa 20:2 Isa 53:6; 64:6 Jer 8:5,\allowbreak6 Zep 1:6}
\crossref{Ps}{52}{5}{Ps 27:2 Jer 10:25 Re 17:16}
\crossref{Ps}{52}{6}{Le 26:17,\allowbreak36 De 28:65-\allowbreak67 1Sa 14:15 2Ki 7:6,\allowbreak7 Job 15:21 Pr 28:1}
\crossref{Ps}{52}{7}{Ps 14:7}
\crossref{Ps}{52}{8}{}
\crossref{Ps}{52}{9}{}
\crossref{Ps}{53}{1}{1Sa 23:19,\allowbreak20; 26:1 Mic 7:5,\allowbreak6 Mt 10:21}
\crossref{Ps}{53}{2}{1Sa 23:19,\allowbreak20; 26:1 Mic 7:5,\allowbreak6 Mt 10:21}
\crossref{Ps}{53}{3}{1Sa 23:19,\allowbreak20; 26:1 Mic 7:5,\allowbreak6 Mt 10:21}
\crossref{Ps}{53}{4}{Ps 5:1-\allowbreak3; 13:3; 55:1,\allowbreak2; 130:2; 143:7}
\crossref{Ps}{53}{5}{53:69:8 86:14 Job 19:13-\allowbreak15}
\crossref{Ps}{53}{6}{53:118:6,\allowbreak7,\allowbreak13 1Ch 12:18 Isa 41:10; 42:1; 50:7-\allowbreak9 Ro 8:31 Heb 13:6}
\crossref{Ps}{54}{1}{Ps 6:1; 54:1}
\crossref{Ps}{54}{2}{Ps 6:1; 54:1}
\crossref{Ps}{54}{3}{Ps 13:1,\allowbreak2; 32:3; 38:6; 43:2; 102:9,\allowbreak10 Isa 38:14}
\crossref{Ps}{54}{4}{Ps 12:5; 54:3; 73:8 La 3:34-\allowbreak36}
\crossref{Ps}{54}{5}{Ps 6:3; 69:20; 88:3; 102:3-\allowbreak5 Mt 26:37,\allowbreak38 Mr 14:33,\allowbreak34 Joh 12:27}
\crossref{Ps}{54}{6}{54:119:120 2Sa 15:14 Job 6:4; 23:15,\allowbreak16}
\crossref{Ps}{54}{7}{Ps 11:1; 139:9 Re 12:14}
\crossref{Ps}{55}{1}{Ps 16:1; 57:1; 58:1; 59:1; 60:1}
\crossref{Ps}{55}{2}{Ps 16:1; 57:1; 58:1; 59:1; 60:1}
\crossref{Ps}{55}{3}{Ps 54:5}
\crossref{Ps}{55}{4}{Ps 34:4; 55:4,\allowbreak5 1Sa 21:10,\allowbreak12; 30:6 2Ch 20:3 2Co 1:8-\allowbreak10; 7:5,\allowbreak6}
\crossref{Ps}{55}{5}{55:10,\allowbreak11; 12:6; 19:7,\allowbreak8; 119:89,\allowbreak90,\allowbreak160; 138:2 Joh 10:35}
\crossref{Ps}{55}{6}{Isa 29:20,\allowbreak21 Mt 22:15; 26:61 Lu 11:54 Joh 2:19 2Pe 3:16}
\crossref{Ps}{55}{7}{Ps 2:1-\allowbreak3; 59:3; 71:10; 140:2 Mt 26:3,\allowbreak4,\allowbreak57; 27:1 Ac 4:5,\allowbreak6; 23:12-\allowbreak14}
\crossref{Ps}{55}{8}{55:94:20,\allowbreak21 Ec 8:8 Isa 28:15 Jer 7:10 Hab 1:13}
\crossref{Ps}{55}{9}{55:105:13,\allowbreak14; 121:8 Nu 33:2-\allowbreak56 1Sa 19:18; 22:1-\allowbreak5; 27:1 Isa 63:9}
\crossref{Ps}{55}{10}{55:118:11-\allowbreak13 Ex 17:9-\allowbreak11 Jer 33:3}
\crossref{Ps}{55}{11}{55:4; 60:6 Ge 32:11 Mt 24:35 Heb 6:18 2Pe 1:4}
\crossref{Ps}{55}{12}{Ps 27:1; 112:7,\allowbreak8 Isa 51:7,\allowbreak8,\allowbreak12,\allowbreak13}
\crossref{Ps}{55}{13}{Ps 66:13,\allowbreak14; 76:11; 116:14-\allowbreak19; 119:106 Ge 28:20-\allowbreak22; 35:1-\allowbreak3}
\crossref{Ps}{55}{14}{55:86:12,\allowbreak13 116:8 2Co 1:10 1Th 1:10 Heb 2:15 Jas 5:20}
\crossref{Ps}{55}{15}{}
\crossref{Ps}{55}{16}{}
\crossref{Ps}{55}{17}{}
\crossref{Ps}{55}{18}{}
\crossref{Ps}{55}{19}{}
\crossref{Ps}{55}{20}{}
\crossref{Ps}{55}{21}{}
\crossref{Ps}{55}{22}{}
\crossref{Ps}{55}{23}{}
\crossref{Ps}{56}{1}{56:142:1}
\crossref{Ps}{56}{2}{56:142:1}
\crossref{Ps}{56}{3}{Ps 56:2; 136:2,\allowbreak3 Isa 57:15}
\crossref{Ps}{56}{4}{Ps 18:6-\allowbreak50; 144:5-\allowbreak7 Mt 28:2-\allowbreak6 Ac 12:11}
\crossref{Ps}{56}{5}{Ps 10:9; 17:12,\allowbreak13; 22:13-\allowbreak16; 35:17; 58:6 Pr 28:15 Da 6:22-\allowbreak24}
\crossref{Ps}{56}{6}{56:11; 21:13; 108:4,\allowbreak5 1Ch 29:1 Isa 2:11,\allowbreak17; 12:4; 37:20 Mt 6:9,\allowbreak10}
\crossref{Ps}{56}{7}{Ps 7:15,\allowbreak16; 9:15,\allowbreak16; 35:7,\allowbreak8; 140:5 1Sa 23:22-\allowbreak26 Pr 29:5 Mic 7:2}
\crossref{Ps}{56}{8}{56:108:1,\allowbreak2; 112:7}
\crossref{Ps}{56}{9}{Jud 5:12 Isa 52:1,\allowbreak9}
\crossref{Ps}{56}{10}{Ps 2:1; 18:49; 22:22,\allowbreak23; 96:3; 138:1,\allowbreak4,\allowbreak5; 145:10-\allowbreak12 Ro 15:9}
\crossref{Ps}{56}{11}{Ps 36:5; 71:19; 85:10,\allowbreak11; 89:1,\allowbreak2; 103:11; 108:4}
\crossref{Ps}{56}{12}{56:5; 8:1,\allowbreak9 Re 15:3,\allowbreak4}
\crossref{Ps}{56}{13}{}
\crossref{Ps}{57}{1}{Ps 57:1; 59:1}
\crossref{Ps}{57}{2}{Ps 21:11 Ec 3:16 Isa 59:4-\allowbreak6 Jer 22:16,\allowbreak17 Eze 22:12,\allowbreak27}
\crossref{Ps}{57}{3}{Ps 21:11 Ec 3:16 Isa 59:4-\allowbreak6 Jer 22:16,\allowbreak17 Eze 22:12,\allowbreak27}
\crossref{Ps}{57}{4}{Ps 51:5 Job 15:14 Pr 22:15 Isa 48:8 Eph 2:3}
\crossref{Ps}{57}{5}{57:140:3 Ec 10:11 Ro 3:13 Jas 3:8}
\crossref{Ps}{57}{6}{}
\crossref{Ps}{57}{7}{Ps 3:7; 10:15 Job 4:10,\allowbreak11; 29:17 Eze 30:21-\allowbreak26}
\crossref{Ps}{57}{8}{Ps 22:14; 64:7,\allowbreak8; 112:10 Ex 15:15 Jos 2:9-\allowbreak11; 7:5 2Sa 17:10}
\crossref{Ps}{57}{9}{Ps 37:35,\allowbreak36 Mt 24:35 Jas 1:10}
\crossref{Ps}{57}{10}{57:118:12 Ec 7:6}
\crossref{Ps}{57}{11}{Ps 52:6; 64:10; 68:1-\allowbreak3; 107:42 Jud 5:31 Pr 11:10 Re 11:17,\allowbreak18; 18:20}
\crossref{Ps}{58}{1}{}
\crossref{Ps}{58}{2}{Ps 26:9; 27:2; 55:23; 139:19}
\crossref{Ps}{58}{3}{Ps 26:9; 27:2; 55:23; 139:19}
\crossref{Ps}{58}{4}{Ps 10:9,\allowbreak10; 37:32,\allowbreak33; 38:12; 56:6 1Sa 19:1 Pr 12:6 Mic 7:2 Ac 23:21}
\crossref{Ps}{58}{5}{1Sa 19:12-\allowbreak24 Pr 1:16 Isa 59:7 Ac 23:15 Ro 3:15}
\crossref{Ps}{58}{6}{Ge 33:20 Ex 3:15}
\crossref{Ps}{58}{7}{58:14 1Sa 19:11}
\crossref{Ps}{58}{8}{Pr 15:2}
\crossref{Ps}{58}{9}{Ps 2:4; 37:13 1Sa 19:15,\allowbreak16 Pr 1:26}
\crossref{Ps}{58}{10}{58:17; 62:2}
\crossref{Ps}{58}{11}{58:17 2Co 1:3 Eph 2:4,\allowbreak5 1Pe 5:10}
\crossref{Ps}{59}{1}{Ps 59:1}
\crossref{Ps}{59}{2}{Ps 59:1}
\crossref{Ps}{59}{3}{59:104:32; 114:7 2Sa 22:8 Job 9:6 Isa 5:25 Jer 4:24; 10:10 Am 8:8}
\crossref{Ps}{59}{4}{59:104:32; 114:7 2Sa 22:8 Job 9:6 Isa 5:25 Jer 4:24; 10:10 Am 8:8}
\crossref{Ps}{59}{5}{59:71:20 Ne 9:32 Da 9:12}
\crossref{Ps}{59}{6}{Ps 20:5 Ex 17:15 So 2:4 Isa 11:12; 49:22; 59:19}
\crossref{Ps}{59}{7}{59:12; 22:8; 108:6-\allowbreak13 De 7:7,\allowbreak8; 33:3 Mt 3:17; 17:5}
\crossref{Ps}{59}{8}{59:89:19,\allowbreak35 108:7-\allowbreak13 132:11 2Sa 3:18; 5:2 Jer 23:9 Am 4:2}
\crossref{Ps}{59}{9}{Jos 17:1,\allowbreak5,\allowbreak6 1Ch 12:19,\allowbreak37}
\crossref{Ps}{59}{10}{2Sa 8:2 1Ch 18:1,\allowbreak2}
\crossref{Ps}{59}{11}{Jud 1:12,\allowbreak24,\allowbreak25 1Ch 11:6,\allowbreak17-\allowbreak19}
\crossref{Ps}{59}{12}{Ps 20:7; 44:5-\allowbreak9; 118:9,\allowbreak10 Isa 8:17; 12:1,\allowbreak2}
\crossref{Ps}{59}{13}{Ps 25:22; 130:8}
\crossref{Ps}{59}{14}{Ps 18:32-\allowbreak42; 144:1 Nu 24:18,\allowbreak19 Jos 1:9; 14:12 2Sa 10:12 1Ch 19:13}
\crossref{Ps}{59}{15}{}
\crossref{Ps}{59}{16}{}
\crossref{Ps}{59}{17}{}
\crossref{Ps}{60}{1}{}
\crossref{Ps}{60}{2}{Ps 42:6; 139:9,\allowbreak10 De 4:29 Jon 2:2-\allowbreak4}
\crossref{Ps}{60}{3}{Ps 42:6; 139:9,\allowbreak10 De 4:29 Jon 2:2-\allowbreak4}
\crossref{Ps}{60}{4}{Ps 4:6,\allowbreak7; 116:2; 140:7 Isa 46:3,\allowbreak4 2Co 1:10}
\crossref{Ps}{60}{5}{60:7; 15:1; 23:6; 27:4; 90:1; 91:1; 92:13 Re 3:12}
\crossref{Ps}{60}{6}{Ps 56:12; 65:1; 66:19}
\crossref{Ps}{60}{7}{60:89:36,\allowbreak37}
\crossref{Ps}{60}{8}{Ps 41:12 Isa 9:6,\allowbreak7 Lu 1:33 Heb 7:21-\allowbreak25; 9:24}
\crossref{Ps}{60}{9}{Ps 30:12; 79:13; 145:1,\allowbreak2; 146:2}
\crossref{Ps}{60}{10}{}
\crossref{Ps}{60}{11}{}
\crossref{Ps}{60}{12}{}
\crossref{Ps}{61}{1}{Ps 39:1; 77:1}
\crossref{Ps}{61}{2}{Ps 39:1; 77:1}
\crossref{Ps}{61}{3}{61:6; 18:2; 21:1; 27:1; 73:25,\allowbreak26 De 32:30,\allowbreak31 Isa 26:4; 32:2}
\crossref{Ps}{61}{4}{Ps 4:2; 82:2 Ex 10:3; 16:28 Pr 1:22; 6:9 Jer 4:14 Mt 17:17}
\crossref{Ps}{61}{5}{Ps 2:1-\allowbreak3 Mt 2:3,\allowbreak4,\allowbreak16; 22:15,\allowbreak23,\allowbreak34,\allowbreak35; 26:3,\allowbreak4; 27:1 Joh 11:47-\allowbreak50}
\crossref{Ps}{61}{6}{Ps 42:5,\allowbreak11; 43:5; 103:1,\allowbreak2; 104:1,\allowbreak35; 146:1}
\crossref{Ps}{61}{7}{61:2; 18:31,\allowbreak32 Isa 45:17 Ho 1:7}
\crossref{Ps}{61}{8}{Isa 45:25 Jer 3:23; 9:23,\allowbreak24 1Co 1:30,\allowbreak31 Ga 6:14}
\crossref{Ps}{62}{1}{1Sa 22:5; 23:14-\allowbreak16,\allowbreak23-\allowbreak25; 26:1-\allowbreak3 2Sa 15:28}
\crossref{Ps}{62}{2}{1Sa 22:5; 23:14-\allowbreak16,\allowbreak23-\allowbreak25; 26:1-\allowbreak3 2Sa 15:28}
\crossref{Ps}{62}{3}{Ps 27:4; 78:61; 105:4; 145:11 Ex 33:18,\allowbreak19 1Sa 4:21,\allowbreak22 1Ch 16:11}
\crossref{Ps}{62}{4}{Ps 4:6; 21:6; 30:5 Php 1:23 1Jo 3:2}
\crossref{Ps}{62}{5}{62:104:33; 145:1-\allowbreak3; 146:1,\allowbreak2}
\crossref{Ps}{62}{6}{Ps 17:15; 36:7-\allowbreak9; 65:4; 104:34 So 1:4 Isa 25:6 Jer 31:4}
\crossref{Ps}{62}{7}{Ps 42:8; 77:4-\allowbreak6; 119:55,\allowbreak147,\allowbreak148; 139:17,\allowbreak18; 149:5 So 3:1,\allowbreak2; 5:2}
\crossref{Ps}{62}{8}{Ps 54:3,\allowbreak4}
\crossref{Ps}{62}{9}{62:73:25; 143:6,\allowbreak7 Ge 32:26-\allowbreak28 2Ch 31:21 So 3:2 Isa 26:9 Mt 11:12}
\crossref{Ps}{62}{10}{Ps 35:4,\allowbreak26; 38:12; 40:14; 70:2 1Sa 25:29}
\crossref{Ps}{62}{11}{So 2:15 Eze 39:4,\allowbreak17-\allowbreak20 Re 19:17,\allowbreak18}
\crossref{Ps}{62}{12}{Ps 2:6; 21:1 1Sa 23:17; 24:20}
\crossref{Ps}{63}{1}{Ps 27:7; 55:1,\allowbreak2; 130:1,\allowbreak2; 141:1; 143:1-\allowbreak3 La 3:55,\allowbreak56}
\crossref{Ps}{63}{2}{Ps 27:7; 55:1,\allowbreak2; 130:1,\allowbreak2; 141:1; 143:1-\allowbreak3 La 3:55,\allowbreak56}
\crossref{Ps}{63}{3}{Ps 27:5; 31:20; 143:9 Isa 32:2}
\crossref{Ps}{63}{4}{Ps 57:4 Pr 12:18; 30:14 Isa 54:17 Jer 9:3 Jas 3:6-\allowbreak8}
\crossref{Ps}{63}{5}{Ps 10:8,\allowbreak9 Ne 4:11 Hab 3:14}
\crossref{Ps}{63}{6}{Ex 15:9 Nu 22:6 Pr 1:11-\allowbreak14 Isa 41:6 Re 11:10}
\crossref{Ps}{63}{7}{Ps 35:11 1Sa 22:9; 24:9; 25:10 Da 6:4,\allowbreak5 Mt 26:59 Joh 18:29,\allowbreak30; 19:7}
\crossref{Ps}{63}{8}{Ps 7:12,\allowbreak13; 18:14 De 32:23,\allowbreak42 Job 6:4 La 3:12,\allowbreak13}
\crossref{Ps}{63}{9}{Ps 59:12; 140:9 Job 15:6 Pr 12:13; 18:7 Mt 21:41 Lu 19:22}
\crossref{Ps}{63}{10}{Ps 40:3; 53:5; 119:20 Jer 50:28; 51:10 Re 11:13}
\crossref{Ps}{63}{11}{Ps 32:11; 33:1; 40:3; 58:10; 68:2,\allowbreak3 Php 4:4}
\crossref{Ps}{64}{1}{Ps 21:13; 115:1,\allowbreak2}
\crossref{Ps}{64}{2}{Ps 21:13; 115:1,\allowbreak2}
\crossref{Ps}{64}{3}{Ps 66:19; 102:17; 145:18,\allowbreak19 1Ki 18:29,\allowbreak37 2Ch 33:13 Isa 65:24}
\crossref{Ps}{64}{4}{Ps 38:4; 40:12 2Sa 12:7-\allowbreak13 Mic 7:8,\allowbreak9 Ro 7:23-\allowbreak25 Ga 5:17}
\crossref{Ps}{64}{5}{Ps 33:12; 84:4}
\crossref{Ps}{64}{6}{Ps 45:4; 47:2,\allowbreak3; 66:3; 76:3-\allowbreak9 De 4:34; 10:21 Isa 37:36}
\crossref{Ps}{64}{7}{Ps 24:2; 119:90 Mic 6:2 Hab 3:6}
\crossref{Ps}{64}{8}{64:89:9 107:29 Jon 1:4,\allowbreak15 Mt 8:26,\allowbreak27}
\crossref{Ps}{64}{9}{Ps 2:8}
\crossref{Ps}{64}{10}{64:104:13,\allowbreak14 De 11:11,\allowbreak12 Ru 1:6 Job 37:6-\allowbreak13 Jer 14:22 Ac 14:17}
\crossref{Ps}{65}{1}{65:81:1 95:1,\allowbreak2 98:4 100:1 1Ch 15:28}
\crossref{Ps}{65}{2}{Ps 47:6,\allowbreak7; 72:18; 96:3-\allowbreak10; 105:2,\allowbreak3; 106:2; 107:15,\allowbreak22 1Ch 29:10-\allowbreak13}
\crossref{Ps}{65}{3}{Ps 47:2; 65:5; 76:12 Ex 15:1-\allowbreak16,\allowbreak21 Jud 5:2-\allowbreak4,\allowbreak20-\allowbreak22 Isa 2:19; 64:3}
\crossref{Ps}{65}{4}{Ps 22:27; 65:5; 67:2,\allowbreak3; 96:1,\allowbreak2; 117:1 Isa 2:2-\allowbreak4; 11:9; 42:10-\allowbreak12}
\crossref{Ps}{65}{5}{65:16; 46:8; 111:2; 126:1-\allowbreak3 Nu 23:23}
\crossref{Ps}{65}{6}{65:78:13; 106:8-\allowbreak10 104:5-\allowbreak7 136:13,\allowbreak14 Ex 14:21,\allowbreak22 Isa 63:13,\allowbreak14}
\crossref{Ps}{65}{7}{Ps 62:11 Da 4:35; 6:26,\allowbreak27 Mt 6:13; 28:18}
\crossref{Ps}{65}{8}{De 32:43 Ro 15:10,\allowbreak11}
\crossref{Ps}{65}{9}{Ps 22:29 1Sa 25:29 Ac 17:28 Col 3:3,\allowbreak4}
\crossref{Ps}{65}{10}{Ps 17:3 De 8:2,\allowbreak16; 13:3}
\crossref{Ps}{65}{11}{Job 19:6 La 1:13; 3:2-\allowbreak66 Ho 7:12 Mt 6:13}
\crossref{Ps}{65}{12}{65:129:1-\allowbreak3 Isa 51:23}
\crossref{Ps}{65}{13}{Ps 51:18,\allowbreak19; 100:4; 118:19,\allowbreak27 De 12:11,\allowbreak12 Heb 13:15}
\crossref{Ps}{66}{1}{Ps 4:1; 6:1; 76:1}
\crossref{Ps}{66}{2}{Ps 4:1; 6:1; 76:1}
\crossref{Ps}{66}{3}{66:98:2,\allowbreak3 Es 8:15-\allowbreak17 Zec 8:20-\allowbreak23 Ac 9:31}
\crossref{Ps}{66}{4}{66:5; 45:17; 74:21; 119:175; 142:7 Isa 38:18,\allowbreak19}
\crossref{Ps}{66}{5}{66:97:1 138:4,\allowbreak5 De 32:43 Isa 24:14-\allowbreak16; 42:10-\allowbreak12; 54:1 Ro 15:10,\allowbreak11}
\crossref{Ps}{66}{6}{66:3 Mt 6:9,\allowbreak10}
\crossref{Ps}{66}{7}{66:85:9-\allowbreak12 Le 26:4 Isa 1:19; 30:23,\allowbreak24 Eze 34:26,\allowbreak27 Ho 2:21,\allowbreak22}
\crossref{Ps}{66}{8}{Ps 29:11; 72:17 Ge 12:2,\allowbreak3 Ac 2:28 Ga 3:9,\allowbreak14}
\crossref{Ps}{66}{9}{}
\crossref{Ps}{66}{10}{}
\crossref{Ps}{66}{11}{}
\crossref{Ps}{66}{12}{}
\crossref{Ps}{66}{13}{}
\crossref{Ps}{66}{14}{}
\crossref{Ps}{66}{15}{}
\crossref{Ps}{66}{16}{}
\crossref{Ps}{66}{17}{}
\crossref{Ps}{66}{18}{}
\crossref{Ps}{66}{19}{}
\crossref{Ps}{66}{20}{}
\crossref{Ps}{67}{1}{Ps 7:6,\allowbreak7; 44:26; 78:65-\allowbreak68; 132:8,\allowbreak9 Nu 10:35 2Ch 6:41 Isa 33:3}
\crossref{Ps}{67}{2}{Ps 7:6,\allowbreak7; 44:26; 78:65-\allowbreak68; 132:8,\allowbreak9 Nu 10:35 2Ch 6:41 Isa 33:3}
\crossref{Ps}{67}{3}{Ps 37:20 Isa 9:18 Ho 13:3}
\crossref{Ps}{67}{4}{Ps 32:11; 33:1; 58:10; 64:10; 97:12 Re 18:20; 19:7}
\crossref{Ps}{67}{5}{Ps 66:4; 67:4 Isa 12:4-\allowbreak6}
\crossref{Ps}{67}{6}{Ps 10:14,\allowbreak18; 82:3,\allowbreak4; 146:9 Job 31:16,\allowbreak17 Jer 49:11 Ho 14:3}
\crossref{Ps}{67}{7}{67:107:10,\allowbreak41; 113:9 1Sa 2:5 Ga 4:27}
\crossref{Ps}{68}{1}{Ps 45:1; 60:1; 80:1}
\crossref{Ps}{68}{2}{Ps 45:1; 60:1; 80:1}
\crossref{Ps}{68}{3}{Ps 40:2 Jer 38:6,\allowbreak22}
\crossref{Ps}{68}{4}{Ps 6:6; 13:1-\allowbreak3; 22:2 Heb 5:7}
\crossref{Ps}{68}{5}{Joh 15:25 1Pe 2:22}
\crossref{Ps}{68}{6}{Ps 17:3; 19:12; 44:20,\allowbreak21}
\crossref{Ps}{68}{7}{Ps 7:7; 25:3; 35:26 Isa 49:23 Lu 24:19-\allowbreak21 Ac 4:7}
\crossref{Ps}{68}{8}{Ps 22:6-\allowbreak8; 44:22 Jer 15:15 Joh 15:21-\allowbreak24}
\crossref{Ps}{68}{9}{Ps 31:11 Job 19:13-\allowbreak19 Mt 26:48-\allowbreak50,\allowbreak56,\allowbreak70-\allowbreak74 Joh 1:11; 7:5}
\crossref{Ps}{68}{10}{68:119:139 1Ki 19:10 1Ch 15:27-\allowbreak29; 29:3 Mr 11:15-\allowbreak17 Joh 2:14-\allowbreak17}
\crossref{Ps}{68}{11}{68:102:8,\allowbreak9; 109:24,\allowbreak25 Lu 7:33,\allowbreak34}
\crossref{Ps}{68}{12}{Ps 35:13,\allowbreak14 Isa 20:2; 22:12 Joe 1:8,\allowbreak13}
\crossref{Ps}{68}{13}{De 16:18 Mt 27:12,\allowbreak13,\allowbreak20,\allowbreak41,\allowbreak42,\allowbreak62,\allowbreak63 Lu 23:2 Ac 4:26,\allowbreak27}
\crossref{Ps}{68}{14}{Ps 55:16,\allowbreak17; 91:15 Mt 26:36-\allowbreak46 Lu 22:44 Joh 17:1-\allowbreak26 Heb 5:7}
\crossref{Ps}{68}{15}{Ps 40:1-\allowbreak3 Jer 38:6-\allowbreak13 La 3:55}
\crossref{Ps}{68}{16}{Isa 43:1,\allowbreak2 Jon 2:2-\allowbreak7 Mt 12:40 Re 12:15,\allowbreak16}
\crossref{Ps}{68}{17}{Ps 36:7; 63:3; 109:21}
\crossref{Ps}{68}{18}{Ps 13:1; 22:24; 27:9; 44:24; 102:2; 143:9 Mt 27:46}
\crossref{Ps}{68}{19}{Ps 10:1; 22:1,\allowbreak19 Jer 14:8}
\crossref{Ps}{68}{20}{68:7-\allowbreak9; 22:6,\allowbreak7 Isa 53:3 Heb 12:2 1Pe 2:23}
\crossref{Ps}{68}{21}{Ps 42:10; 123:4 Heb 11:36}
\crossref{Ps}{68}{22}{Mr 15:23,\allowbreak36 Lu 23:36 Joh 19:29,\allowbreak30}
\crossref{Ps}{68}{23}{Isa 8:14,\allowbreak15 1Pe 2:8}
\crossref{Ps}{68}{24}{Isa 6:9,\allowbreak10; 29:9,\allowbreak10 Mt 13:14,\allowbreak15 Joh 12:39,\allowbreak40 Ac 28:26,\allowbreak27}
\crossref{Ps}{68}{25}{68:79:6 Le 26:14-\allowbreak46 De 28:15-\allowbreak68; 29:18-\allowbreak28; 31:17; 32:20-\allowbreak26 Ho 5:10}
\crossref{Ps}{68}{26}{1Ki 9:8 Jer 7:12-\allowbreak14 Mt 23:38; 24:1,\allowbreak2 Ac 1:20}
\crossref{Ps}{68}{27}{68:109:16 2Ch 28:9 Job 19:21,\allowbreak22 Zec 1:15 1Th 2:15}
\crossref{Ps}{68}{28}{68:81:12 Ex 8:15,\allowbreak32; 9:12 Le 26:39 Isa 5:6 Mt 21:19; 23:31,\allowbreak32}
\crossref{Ps}{68}{29}{Ex 32:32,\allowbreak33 Isa 65:16 Ho 1:9 Re 3:5; 22:19}
\crossref{Ps}{68}{30}{Ps 40:17; 109:22,\allowbreak31 Isa 53:2,\allowbreak3 Mt 8:20 2Co 8:9}
\crossref{Ps}{68}{31}{Ps 28:7; 40:1-\allowbreak3; 118:21,\allowbreak28,\allowbreak29}
\crossref{Ps}{68}{32}{Ps 50:13,\allowbreak14,\allowbreak23 Ho 14:2 Eph 5:19,\allowbreak20 Heb 13:15 1Pe 2:5}
\crossref{Ps}{68}{33}{Ps 25:9; 34:2 Isa 61:1-\allowbreak3 Joh 16:22; 20:20}
\crossref{Ps}{68}{34}{Ps 10:17; 34:6; 72:12-\allowbreak14; 102:17,\allowbreak20 Isa 66:2 Lu 4:18}
\crossref{Ps}{68}{35}{68:96:11; 98:7,\allowbreak8 148:1-\allowbreak14 150:6 Isa 44:22,\allowbreak23; 49:13; 55:12}
\crossref{Ps}{69}{1}{Ps 38:1}
\crossref{Ps}{69}{2}{Ps 38:1}
\crossref{Ps}{69}{3}{Ps 6:10; 35:4,\allowbreak26; 71:13; 109:29 Isa 41:11}
\crossref{Ps}{69}{4}{Ps 40:15 Ac 1:18}
\crossref{Ps}{69}{5}{Ps 5:11; 35:27; 40:16; 97:12 Isa 61:10; 65:13,\allowbreak14 La 3:25 Joh 16:20}
\crossref{Ps}{69}{6}{Ps 40:17; 69:29; 109:22}
\crossref{Ps}{69}{7}{}
\crossref{Ps}{69}{8}{}
\crossref{Ps}{69}{9}{}
\crossref{Ps}{69}{10}{}
\crossref{Ps}{69}{11}{}
\crossref{Ps}{69}{12}{}
\crossref{Ps}{69}{13}{}
\crossref{Ps}{69}{14}{}
\crossref{Ps}{69}{15}{}
\crossref{Ps}{69}{16}{}
\crossref{Ps}{69}{17}{}
\crossref{Ps}{69}{18}{}
\crossref{Ps}{69}{19}{}
\crossref{Ps}{69}{20}{}
\crossref{Ps}{69}{21}{}
\crossref{Ps}{69}{22}{}
\crossref{Ps}{69}{23}{}
\crossref{Ps}{69}{24}{}
\crossref{Ps}{69}{25}{}
\crossref{Ps}{69}{26}{}
\crossref{Ps}{69}{27}{}
\crossref{Ps}{69}{28}{}
\crossref{Ps}{69}{29}{}
\crossref{Ps}{69}{30}{}
\crossref{Ps}{69}{31}{}
\crossref{Ps}{69}{32}{}
\crossref{Ps}{69}{33}{}
\crossref{Ps}{69}{34}{}
\crossref{Ps}{69}{35}{}
\crossref{Ps}{69}{36}{}
\crossref{Ps}{70}{1}{Ps 22:5; 25:2,\allowbreak3; 31:1-\allowbreak3; 125:1; 146:5 2Ki 18:5 1Ch 5:20 Ro 9:33}
\crossref{Ps}{70}{2}{Ps 17:2; 31:1; 34:15; 43:1; 143:1,\allowbreak11 Da 9:16}
\crossref{Ps}{70}{3}{Ps 31:2,\allowbreak3; 91:1,\allowbreak2 Pr 18:10 Isa 33:16}
\crossref{Ps}{70}{4}{Ps 17:8,\allowbreak9,\allowbreak13; 59:1,\allowbreak2; 140:1-\allowbreak4 2Sa 16:21,\allowbreak22; 17:1,\allowbreak2,\allowbreak12-\allowbreak14,\allowbreak21}
\crossref{Ps}{70}{5}{Ps 13:5; 39:7; 42:11; 119:81,\allowbreak166 Jer 17:7,\allowbreak13,\allowbreak17 Ro 15:13}
\crossref{Ps}{71}{1}{71:127:1}
\crossref{Ps}{71}{2}{71:12-\allowbreak14; 45:6,\allowbreak7 1Ki 3:5-\allowbreak10 Isa 11:3-\allowbreak5; 32:1,\allowbreak17 Jer 33:15 Re 19:11}
\crossref{Ps}{71}{3}{71:16 Isa 32:16,\allowbreak17; 52:7 Eze 34:13,\allowbreak14 Joe 3:18}
\crossref{Ps}{71}{4}{71:12-\allowbreak14; 109:31 Isa 11:4 Eze 34:15,\allowbreak16 Zec 11:7,\allowbreak11 Mt 11:5}
\crossref{Ps}{71}{5}{1Sa 12:18 1Ki 3:28}
\crossref{Ps}{71}{6}{De 32:2 2Sa 23:4 Pr 16:15; 19:12 Isa 5:6; 14:3-\allowbreak5 Eze 34:23-\allowbreak26}
\crossref{Ps}{71}{7}{71:132:15-\allowbreak18 Isa 11:6-\allowbreak9; 32:3-\allowbreak8,\allowbreak15-\allowbreak20; 35:1-\allowbreak10; 54:11-\allowbreak17; 55:10-\allowbreak13}
\crossref{Ps}{71}{8}{Ps 2:8; 80:11; 89:25,\allowbreak36 Ex 23:31 1Ki 4:21-\allowbreak24 Zec 9:10 Re 11:15}
\crossref{Ps}{71}{9}{1Ki 9:18,\allowbreak20,\allowbreak21 Isa 35:1,\allowbreak2}
\crossref{Ps}{71}{10}{Ps 45:12; 68:29 1Ki 10:1,\allowbreak10,\allowbreak25 2Ch 9:21 Isa 43:6; 49:7; 60:3,\allowbreak6,\allowbreak9}
\crossref{Ps}{71}{11}{Ps 2:10-\allowbreak12; 138:4,\allowbreak5 Isa 49:22,\allowbreak23 Re 11:15; 17:14; 21:24,\allowbreak26}
\crossref{Ps}{71}{12}{71:4; 10:17; 82:3,\allowbreak4; 102:17,\allowbreak20 Job 29:12 Isa 41:17 Lu 4:18; 7:22}
\crossref{Ps}{71}{13}{71:109:31 Job 5:15,\allowbreak16 Eze 34:16 Mt 5:3; 18:11 Jas 2:5,\allowbreak6}
\crossref{Ps}{71}{14}{Ps 25:22; 130:8 Ge 48:16 2Sa 4:9 Lu 1:68-\allowbreak75 Tit 2:14}
\crossref{Ps}{71}{15}{Ps 21:4 Joh 11:25; 14:19 1Jo 1:2 Re 1:18}
\crossref{Ps}{71}{16}{Job 8:7 Isa 30:23; 32:15,\allowbreak20 Mt 13:31-\allowbreak33 Mr 16:15,\allowbreak16 Ac 1:15}
\crossref{Ps}{71}{17}{Ps 45:17; 89:36 Isa 7:14 Mt 1:21,\allowbreak23 Lu 1:31-\allowbreak33 Php 2:10}
\crossref{Ps}{71}{18}{Ps 41:13; 68:35; 106:48 1Ch 29:10,\allowbreak20}
\crossref{Ps}{71}{19}{Ne 9:5 Re 5:13}
\crossref{Ps}{71}{20}{}
\crossref{Ps}{71}{21}{}
\crossref{Ps}{71}{22}{}
\crossref{Ps}{71}{23}{}
\crossref{Ps}{71}{24}{}
\crossref{Ps}{72}{1}{Ps 50:1; 74:1; 83:1}
\crossref{Ps}{72}{2}{Ps 5:7; 17:15; 35:13 Jos 24:15 1Sa 12:23 1Ch 22:7 Job 21:4}
\crossref{Ps}{72}{3}{Ps 37:1,\allowbreak7 Job 21:7 Pr 3:31; 24:1 Jer 12:1 Jas 4:5}
\crossref{Ps}{72}{4}{Ps 17:14 Job 21:23,\allowbreak24; 24:20 Ec 2:16; 7:15 Lu 16:22}
\crossref{Ps}{72}{5}{72:12 Job 21:6 Pr 3:11,\allowbreak12 Jer 12:1,\allowbreak2 1Co 11:32 Heb 12:8 Re 3:19}
\crossref{Ps}{72}{6}{De 8:13,\allowbreak14; 32:15 Es 3:1,\allowbreak5,\allowbreak6; 5:9-\allowbreak11 Job 21:7-\allowbreak15 Ec 8:11}
\crossref{Ps}{72}{7}{Ps 17:10; 119:70 Job 15:27 Isa 3:9 Jer 5:28 Eze 16:49}
\crossref{Ps}{72}{8}{Ps 53:1-\allowbreak4 Pr 30:13,\allowbreak14}
\crossref{Ps}{72}{9}{Ex 5:2 2Ch 32:15 Job 21:14 Da 3:15; 7:25 Re 13:6}
\crossref{Ps}{72}{10}{72:75:8}
\crossref{Ps}{72}{11}{72:9; 10:11; 94:7 Job 22:13,\allowbreak14 Eze 8:12 Zep 1:12}
\crossref{Ps}{72}{12}{Ps 37:35; 52:7 Jer 12:1,\allowbreak2 Lu 16:19 Jas 5:1-\allowbreak3}
\crossref{Ps}{72}{13}{}
\crossref{Ps}{72}{14}{Ps 34:19; 94:12 Job 7:3,\allowbreak4,\allowbreak18; 10:3,\allowbreak17 Jer 15:18 Am 3:2 Heb 12:5}
\crossref{Ps}{72}{15}{1Sa 2:24 Mal 2:8 Mt 18:6,\allowbreak7 Ro 14:15,\allowbreak21 1Co 8:11-\allowbreak13}
\crossref{Ps}{72}{16}{Ps 36:6; 77:19; 97:2 Pr 30:2,\allowbreak3 Ec 8:17 Ro 11:33}
\crossref{Ps}{72}{17}{Ps 27:4; 63:2; 77:13; 119:24,\allowbreak130}
\crossref{Ps}{72}{18}{Ps 35:6 De 32:35 Jer 23:12}
\crossref{Ps}{72}{19}{Ps 58:9 Job 20:5 Isa 30:13 Ac 2:23 1Th 5:3 Re 18:10}
\crossref{Ps}{72}{20}{72:90:5 Job 20:8 Isa 29:7,\allowbreak8}
\crossref{Ps}{73}{1}{Ps 10:1; 42:9,\allowbreak11; 44:9; 60:1,\allowbreak10; 77:7 Jer 31:37; 33:24-\allowbreak26 Ro 11:1,\allowbreak2}
\crossref{Ps}{73}{2}{Ex 15:16 De 9:29 Ac 20:28}
\crossref{Ps}{73}{3}{Ps 44:23,\allowbreak26 Jos 10:24 2Sa 22:39-\allowbreak43 Isa 10:6; 25:10; 63:3-\allowbreak6 Mic 1:3}
\crossref{Ps}{73}{4}{2Ch 36:17 La 2:7 Lu 13:1 Re 13:6}
\crossref{Ps}{73}{5}{1Ki 5:6 2Ch 2:14 Jer 46:22,\allowbreak23}
\crossref{Ps}{73}{6}{1Ki 6:18,\allowbreak29,\allowbreak32,\allowbreak35}
\crossref{Ps}{73}{7}{73:89:39 Eze 24:21}
\crossref{Ps}{73}{8}{73:83:4 137:7 Es 3:8,\allowbreak9}
\crossref{Ps}{73}{9}{Ex 12:13; 13:9,\allowbreak10 Jud 6:17 Eze 20:12 Heb 2:4}
\crossref{Ps}{73}{10}{Ps 13:1,\allowbreak2; 79:4,\allowbreak5; 89:46,\allowbreak50,\allowbreak51 Da 12:6 Re 6:10}
\crossref{Ps}{73}{11}{Isa 64:12 La 2:3}
\crossref{Ps}{73}{12}{Ps 44:4 Ex 19:5,\allowbreak6 Nu 23:21,\allowbreak22 Isa 33:22}
\crossref{Ps}{73}{13}{Ps 66:6; 78:13; 106:8,\allowbreak9; 136:13-\allowbreak18 Ex 14:21 Ne 9:11 Isa 11:15,\allowbreak16}
\crossref{Ps}{73}{14}{73:104:25,\allowbreak26 Job 3:8}
\crossref{Ps}{73}{15}{73:105:41 Ex 17:5,\allowbreak6 Nu 20:11 Isa 48:21}
\crossref{Ps}{73}{16}{73:136:7-\allowbreak9 Ge 1:3-\allowbreak5}
\crossref{Ps}{73}{17}{Ps 24:1,\allowbreak2 De 32:8 Ac 17:26}
\crossref{Ps}{73}{18}{73:22; 89:50,\allowbreak51; 137:7 Isa 62:6,\allowbreak7}
\crossref{Ps}{73}{19}{73:68:13 So 2:14; 4:1; 6:9 Isa 60:8 Mt 10:16}
\crossref{Ps}{73}{20}{73:89:28,\allowbreak34-\allowbreak36,\allowbreak39 105:8 106:45 Ge 17:7,\allowbreak8 Ex 24:6-\allowbreak8 Le 26:40-\allowbreak45}
\crossref{Ps}{73}{21}{Ps 9:18; 12:5; 102:19-\allowbreak21; 109:22 Isa 45:17}
\crossref{Ps}{73}{22}{Ps 9:19,\allowbreak20; 79:9,\allowbreak10}
\crossref{Ps}{73}{23}{Ps 10:11,\allowbreak12; 13:1}
\crossref{Ps}{73}{24}{}
\crossref{Ps}{73}{25}{}
\crossref{Ps}{73}{26}{}
\crossref{Ps}{73}{27}{}
\crossref{Ps}{73}{28}{}
\crossref{Ps}{74}{1}{}
\crossref{Ps}{74}{2}{74:78:70-\allowbreak72 101:2 2Sa 2:4; 5:3; 8:15; 23:3,\allowbreak4}
\crossref{Ps}{74}{3}{74:78:70-\allowbreak72 101:2 2Sa 2:4; 5:3; 8:15; 23:3,\allowbreak4}
\crossref{Ps}{74}{4}{Ps 60:1-\allowbreak3; 78:60-\allowbreak72 1Sa 31:1-\allowbreak7 Isa 24:1-\allowbreak12}
\crossref{Ps}{74}{5}{74:82:2-\allowbreak8 94:8 Pr 1:22; 8:5; 9:6}
\crossref{Ps}{74}{6}{Ex 32:9 De 31:27 2Ch 30:8 Isa 48:4 Eze 2:4 Ac 7:51}
\crossref{Ps}{74}{7}{1Sa 16:7 Mt 7:21,\allowbreak23}
\crossref{Ps}{74}{8}{Ps 50:6; 58:11}
\crossref{Ps}{74}{9}{Ps 11:6; 60:3 Job 21:20 Isa 51:17,\allowbreak22 Jer 25:15,\allowbreak17,\allowbreak27,\allowbreak28}
\crossref{Ps}{74}{10}{Ps 9:14; 104:33; 145:1,\allowbreak2}
\crossref{Ps}{74}{11}{74:101:8 Jer 48:25 Zec 1:20,\allowbreak21}
\crossref{Ps}{74}{12}{}
\crossref{Ps}{74}{13}{}
\crossref{Ps}{74}{14}{}
\crossref{Ps}{74}{15}{}
\crossref{Ps}{74}{16}{}
\crossref{Ps}{74}{17}{}
\crossref{Ps}{74}{18}{}
\crossref{Ps}{74}{19}{}
\crossref{Ps}{74}{20}{}
\crossref{Ps}{74}{21}{}
\crossref{Ps}{74}{22}{}
\crossref{Ps}{74}{23}{}
\crossref{Ps}{75}{1}{Ps 4:1; 54:1; 61:1; 67:1}
\crossref{Ps}{75}{2}{Ps 4:1; 54:1; 61:1; 67:1}
\crossref{Ps}{75}{3}{Ge 14:18 Heb 7:1,\allowbreak2}
\crossref{Ps}{75}{4}{Ps 46:9 2Ch 14:12,\allowbreak13; 20:25; 32:21 Isa 37:35,\allowbreak36 Eze 39:3,\allowbreak4,\allowbreak9,\allowbreak10}
\crossref{Ps}{75}{5}{Jer 4:7 Eze 19:1-\allowbreak4,\allowbreak6; 38:12,\allowbreak13 Da 7:4-\allowbreak8,\allowbreak17-\allowbreak28}
\crossref{Ps}{75}{6}{Job 40:10-\allowbreak12 Isa 46:12 Da 4:37 Lu 1:51,\allowbreak52}
\crossref{Ps}{75}{7}{Ps 18:15; 80:16; 104:7 Ex 15:1,\allowbreak21}
\crossref{Ps}{75}{8}{75:89:7 Jer 10:7-\allowbreak10 Mt 10:28 Re 14:7; 15:4}
\crossref{Ps}{75}{9}{Ex 19:10 Jud 5:20 2Ch 32:20-\allowbreak22 Eze 38:20-\allowbreak23}
\crossref{Ps}{75}{10}{Ps 9:7-\allowbreak9; 72:4; 82:2-\allowbreak5 Isa 11:4 Jer 5:28}
\crossref{Ps}{76}{1}{Ps 39:1; 62:1}
\crossref{Ps}{76}{2}{Ps 39:1; 62:1}
\crossref{Ps}{76}{3}{Ps 18:6; 50:15; 88:1-\allowbreak3; 102:1,\allowbreak2; 130:1,\allowbreak2 Ge 32:7-\allowbreak12,\allowbreak28}
\crossref{Ps}{76}{4}{Job 6:4; 23:15,\allowbreak16; 31:23 Jer 17:17}
\crossref{Ps}{76}{5}{Ps 6:6 Es 6:1 Job 7:13-\allowbreak15}
\crossref{Ps}{76}{6}{76:74:12-\allowbreak18 143:5 De 32:7 Isa 51:9; 63:9-\allowbreak15 Mic 7:14,\allowbreak15}
\crossref{Ps}{76}{7}{Ps 42:8 Job 35:10 Hab 3:17,\allowbreak18 Jon 1:2 Ac 16:25}
\crossref{Ps}{76}{8}{Ps 13:1,\allowbreak2; 37:24; 74:1; 89:38,\allowbreak46 Jer 23:24-\allowbreak26 La 3:31,\allowbreak32 Ro 11:1,\allowbreak2}
\crossref{Ps}{76}{9}{Isa 27:11 Lu 16:25,\allowbreak26}
\crossref{Ps}{76}{10}{Isa 40:27; 49:14,\allowbreak15; 63:15}
\crossref{Ps}{76}{11}{76:5 Ex 15:6 Nu 23:21,\allowbreak22 De 4:34 Hab 3:2-\allowbreak13}
\crossref{Ps}{76}{12}{76:10; 28:5; 78:11; 111:4 1Ch 16:12 Isa 5:12}
\crossref{Ps}{77}{1}{}
\crossref{Ps}{77}{2}{Ps 49:4 Mt 13:13,\allowbreak34,\allowbreak35}
\crossref{Ps}{77}{3}{Ps 44:1; 48:8 Ex 12:26,\allowbreak27; 13:8,\allowbreak14,\allowbreak15}
\crossref{Ps}{77}{4}{77:145:4-\allowbreak6 De 4:9; 6:7 Joe 1:3}
\crossref{Ps}{77}{5}{77:81:5 119:152 147:19 De 4:45; 6:7; 11:19 Isa 8:20 Ro 3:2}
\crossref{Ps}{77}{6}{Ps 48:13; 71:18; 102:18; 145:4 Es 9:28}
\crossref{Ps}{77}{7}{Ps 40:4; 62:5; 91:14; 130:6,\allowbreak7; 146:5 Jer 17:7,\allowbreak8 1Pe 1:21}
\crossref{Ps}{77}{8}{77:68:6 106:7 Ex 32:9; 33:3,\allowbreak5; 34:9 De 9:6,\allowbreak13; 31:27 2Ki 17:14}
\crossref{Ps}{77}{9}{Jud 9:28,\allowbreak38-\allowbreak40 Lu 22:33}
\crossref{Ps}{77}{10}{De 31:16,\allowbreak20 Jud 2:10-\allowbreak12 2Ki 17:14,\allowbreak15 Ne 9:26-\allowbreak29 Jer 31:32}
\crossref{Ps}{77}{11}{77:7; 106:13,\allowbreak21,\allowbreak22 De 32:18 Jer 2:32}
\crossref{Ps}{77}{12}{77:42-\allowbreak50; 105:27-\allowbreak38; 135:9 Ex 7:1-\allowbreak12:51 De 4:34; 6:22 Ne 9:10}
\crossref{Ps}{77}{13}{Ps 66:6; 106:9,\allowbreak10; 136:13-\allowbreak15 Ex 14:1-\allowbreak15:27 Isa 63:13 1Co 10:2,\allowbreak3}
\crossref{Ps}{77}{14}{77:105:39 Ex 13:21,\allowbreak22; 14:24; 40:35-\allowbreak38 Ne 9:12,\allowbreak19}
\crossref{Ps}{77}{15}{77:105:41; 114:8 Ex 17:6 Nu 20:11 Isa 41:18; 43:20 Joh 7:37,\allowbreak38}
\crossref{Ps}{77}{16}{77:105:41 De 8:15; 9:21}
\crossref{Ps}{77}{17}{77:32; 95:8-\allowbreak10; 106:13-\allowbreak32 De 9:8,\allowbreak12-\allowbreak22 Heb 3:16-\allowbreak19}
\crossref{Ps}{77}{18}{77:106:14,\allowbreak15 Ex 16:2,\allowbreak3 Nu 11:4 1Co 10:6 Jas 4:2,\allowbreak3}
\crossref{Ps}{77}{19}{Ex 16:8-\allowbreak10 Nu 21:5 2Ch 32:19 Job 34:37 Ro 9:20 Re 13:6}
\crossref{Ps}{77}{20}{Ex 17:6,\allowbreak7 Nu 20:11}
\crossref{Ps}{78}{1}{78:74:1}
\crossref{Ps}{78}{2}{Jer 7:33; 15:3; 16:4; 34:20}
\crossref{Ps}{78}{3}{78:10 Mt 23:35 Ro 8:36 Re 16:6; 17:6; 18:24}
\crossref{Ps}{78}{4}{Ps 44:13,\allowbreak14; 80:6; 89:41 De 28:37 Jer 24:9; 25:18; 42:18 La 2:15,\allowbreak16}
\crossref{Ps}{78}{5}{Ps 13:1,\allowbreak2; 74:1,\allowbreak9,\allowbreak10; 80:4; 89:46 Re 6:10}
\crossref{Ps}{78}{6}{78:69:24 Isa 42:25 Re 16:1-\allowbreak21}
\crossref{Ps}{78}{7}{78:80:13 Isa 9:12 Jer 50:7; 51:34,\allowbreak35 Zec 1:15}
\crossref{Ps}{78}{8}{Ps 25:7; 130:3 Ex 32:34 1Ki 17:18 Isa 64:9 Ho 8:13; 9:9 Re 18:5}
\crossref{Ps}{78}{9}{78:115:1 2Ch 14:11 Mal 2:2 Eph 1:6}
\crossref{Ps}{78}{10}{Ps 42:3,\allowbreak10; 115:2 Joe 2:17 Mic 7:10}
\crossref{Ps}{78}{11}{Ps 12:5; 69:33; 102:20 Ex 2:23,\allowbreak24 Isa 42:7}
\crossref{Ps}{78}{12}{Ge 4:15 Le 26:21,\allowbreak28 Isa 65:5-\allowbreak7 Jer 32:18 Lu 6:38}
\crossref{Ps}{78}{13}{78:74:1 95:7 100:3}
\crossref{Ps}{78}{14}{}
\crossref{Ps}{78}{15}{}
\crossref{Ps}{78}{16}{}
\crossref{Ps}{78}{17}{}
\crossref{Ps}{78}{18}{}
\crossref{Ps}{78}{19}{}
\crossref{Ps}{78}{20}{}
\crossref{Ps}{78}{21}{}
\crossref{Ps}{78}{22}{}
\crossref{Ps}{78}{23}{}
\crossref{Ps}{78}{24}{}
\crossref{Ps}{78}{25}{}
\crossref{Ps}{78}{26}{}
\crossref{Ps}{78}{27}{}
\crossref{Ps}{78}{28}{}
\crossref{Ps}{78}{29}{}
\crossref{Ps}{78}{30}{}
\crossref{Ps}{78}{31}{}
\crossref{Ps}{78}{32}{}
\crossref{Ps}{78}{33}{}
\crossref{Ps}{78}{34}{}
\crossref{Ps}{78}{35}{}
\crossref{Ps}{78}{36}{}
\crossref{Ps}{78}{37}{}
\crossref{Ps}{78}{38}{}
\crossref{Ps}{78}{39}{}
\crossref{Ps}{78}{40}{}
\crossref{Ps}{78}{41}{}
\crossref{Ps}{78}{42}{}
\crossref{Ps}{78}{43}{}
\crossref{Ps}{78}{44}{}
\crossref{Ps}{78}{45}{}
\crossref{Ps}{78}{46}{}
\crossref{Ps}{78}{47}{}
\crossref{Ps}{78}{48}{}
\crossref{Ps}{78}{49}{}
\crossref{Ps}{78}{50}{}
\crossref{Ps}{78}{51}{}
\crossref{Ps}{78}{52}{}
\crossref{Ps}{78}{53}{}
\crossref{Ps}{78}{54}{}
\crossref{Ps}{78}{55}{}
\crossref{Ps}{78}{56}{}
\crossref{Ps}{78}{57}{}
\crossref{Ps}{78}{58}{}
\crossref{Ps}{78}{59}{}
\crossref{Ps}{78}{60}{}
\crossref{Ps}{78}{61}{}
\crossref{Ps}{78}{62}{}
\crossref{Ps}{78}{63}{}
\crossref{Ps}{78}{64}{}
\crossref{Ps}{78}{65}{}
\crossref{Ps}{78}{66}{}
\crossref{Ps}{78}{67}{}
\crossref{Ps}{78}{68}{}
\crossref{Ps}{78}{69}{}
\crossref{Ps}{78}{70}{}
\crossref{Ps}{78}{71}{}
\crossref{Ps}{78}{72}{}
\crossref{Ps}{79}{1}{Ps 45:1; 60:1; 69:1}
\crossref{Ps}{79}{2}{Ps 45:1; 60:1; 69:1}
\crossref{Ps}{79}{3}{Ps 35:23; 44:23-\allowbreak26; 78:38 Isa 42:13,\allowbreak14}
\crossref{Ps}{79}{4}{79:7,\allowbreak19; 85:4 1Ki 18:37 Jer 31:18,\allowbreak19 La 5:21}
\crossref{Ps}{79}{5}{79:85:5 Isa 58:2,\allowbreak3,\allowbreak6-\allowbreak9 La 3:44 Mt 15:22-\allowbreak28 Lu 18:1-\allowbreak8}
\crossref{Ps}{79}{6}{Ps 42:3; 102:9 Job 6:7 Isa 30:20 Eze 4:16,\allowbreak17}
\crossref{Ps}{79}{7}{Jer 15:10}
\crossref{Ps}{79}{8}{79:3,\allowbreak19; 51:10 Lu 1:16}
\crossref{Ps}{79}{9}{Isa 5:1-\allowbreak7; 27:2,\allowbreak3 Jer 2:21 Eze 15:6; 17:6; 19:10 Mt 21:33-\allowbreak41}
\crossref{Ps}{79}{10}{79:105:44 Ex 23:28-\allowbreak30 Jos 23:13-\allowbreak15; 24:12 Ne 9:22-\allowbreak25}
\crossref{Ps}{79}{11}{79:104:16}
\crossref{Ps}{79}{12}{79:72:8 Ge 15:18 Ex 23:31 1Ki 4:21,\allowbreak24 1Ch 18:3}
\crossref{Ps}{79}{13}{79:89:40,\allowbreak41 Isa 5:5; 18:5,\allowbreak6 Na 2:2 Lu 20:16}
\crossref{Ps}{80}{1}{Ps 8:1}
\crossref{Ps}{80}{2}{Ps 8:1}
\crossref{Ps}{80}{3}{80:92:3 95:1,\allowbreak2 149:1-\allowbreak3 Mr 14:26 Eph 5:19 Col 3:16 Jas 5:13}
\crossref{Ps}{80}{4}{80:98:6 Nu 10:1-\allowbreak9 1Ch 15:24; 16:6,\allowbreak42 2Ch 5:12; 13:12,\allowbreak14}
\crossref{Ps}{80}{5}{Ps 2:7; 50:16; 94:20}
\crossref{Ps}{80}{6}{80:77:15; 80:1,\allowbreak2 Am 6:6}
\crossref{Ps}{80}{7}{Ex 1:14; 6:6 Isa 9:4; 10:27 Mt 11:29}
\crossref{Ps}{80}{8}{Ps 50:15; 91:14,\allowbreak15 Ex 2:23; 14:10,\allowbreak30,\allowbreak31; 17:2-\allowbreak7}
\crossref{Ps}{80}{9}{Ps 50:7 De 32:46 Isa 55:3,\allowbreak4 Joh 3:11,\allowbreak32,\allowbreak33 Ac 20:21 1Jo 5:9}
\crossref{Ps}{80}{10}{Ex 20:3-\allowbreak5 1Co 8:5,\allowbreak6}
\crossref{Ps}{80}{11}{Ex 20:2 Jer 11:4; 31:31-\allowbreak33}
\crossref{Ps}{80}{12}{80:106:12,\allowbreak13 Jer 2:11-\allowbreak13; 7:23,\allowbreak24 Zec 7:11}
\crossref{Ps}{80}{13}{Ge 6:3 Ac 7:42; 14:16 Ro 1:24,\allowbreak26,\allowbreak27 2Th 2:9-\allowbreak11}
\crossref{Ps}{80}{14}{De 5:29; 10:12,\allowbreak13; 32:29 Isa 48:18 Mt 23:37 Lu 19:41,\allowbreak42}
\crossref{Ps}{80}{15}{Nu 14:9,\allowbreak45 Jos 23:13 Jud 2:20-\allowbreak23}
\crossref{Ps}{80}{16}{Ps 18:45; 83:2-\allowbreak18 Ex 20:5 De 7:10 Joh 15:22,\allowbreak23 Ro 1:30; 8:7}
\crossref{Ps}{80}{17}{80:147:14 De 32:13,\allowbreak14 Joe 2:24}
\crossref{Ps}{80}{18}{}
\crossref{Ps}{80}{19}{}
\crossref{Ps}{81}{1}{}
\crossref{Ps}{81}{2}{Ps 62:3 Ex 10:3 1Ki 18:21 Mt 17:17}
\crossref{Ps}{81}{3}{Ps 10:18 De 10:18 Isa 1:17,\allowbreak23}
\crossref{Ps}{81}{4}{81:72:12-\allowbreak14 Job 29:12,\allowbreak16,\allowbreak17 Pr 24:11,\allowbreak12}
\crossref{Ps}{81}{5}{Ps 53:4 Pr 1:29 Mic 3:1 Ro 1:28}
\crossref{Ps}{81}{6}{81:1 Ex 22:9,\allowbreak28 Joh 10:34-\allowbreak36}
\crossref{Ps}{81}{7}{Ps 49:12 Job 21:32 Eze 31:14}
\crossref{Ps}{81}{8}{Ps 7:6; 44:26; 96:13; 102:13 Isa 51:9 Mic 7:2,\allowbreak7 Zep 3:8}
\crossref{Ps}{81}{9}{}
\crossref{Ps}{81}{10}{}
\crossref{Ps}{81}{11}{}
\crossref{Ps}{81}{12}{}
\crossref{Ps}{81}{13}{}
\crossref{Ps}{81}{14}{}
\crossref{Ps}{81}{15}{}
\crossref{Ps}{81}{16}{}
\crossref{Ps}{82}{1}{}
\crossref{Ps}{82}{2}{Ps 2:1,\allowbreak2; 74:4,\allowbreak23 2Ki 19:28 Isa 37:29 Jer 1:19 Mt 27:24}
\crossref{Ps}{82}{3}{Ps 2:1,\allowbreak2; 74:4,\allowbreak23 2Ki 19:28 Isa 37:29 Jer 1:19 Mt 27:24}
\crossref{Ps}{82}{4}{Ps 10:9; 56:6; 64:2 1Sa 13:19 Isa 7:6,\allowbreak7 Lu 20:20-\allowbreak23}
\crossref{Ps}{82}{5}{Ex 1:10 Es 3:6-\allowbreak9 Pr 1:12 Jer 11:19; 31:36 Da 7:25 Mt 27:62-\allowbreak66}
\crossref{Ps}{82}{6}{Ps 2:2 Pr 21:30 Isa 7:5-\allowbreak7; 8:9,\allowbreak10 Joh 11:47-\allowbreak53 Ac 23:12,\allowbreak13}
\crossref{Ps}{82}{7}{2Ch 20:1,\allowbreak10,\allowbreak11}
\crossref{Ps}{82}{8}{Jos 13:5 Eze 27:9}
\crossref{Ps}{83}{1}{Ps 8:1; 81:1}
\crossref{Ps}{83}{2}{Ps 8:1; 81:1}
\crossref{Ps}{83}{3}{Ps 42:1,\allowbreak2; 63:1,\allowbreak2; 73:26; 119:20,\allowbreak81; 143:6 So 2:4,\allowbreak5; 5:8}
\crossref{Ps}{83}{4}{83:90:1 91:1 116:7 Mt 8:20; 23:37}
\crossref{Ps}{83}{5}{Ps 23:6; 27:4; 65:4; 134:1-\allowbreak3}
\crossref{Ps}{83}{6}{Ps 28:7,\allowbreak8 Isa 45:24 Zec 10:12 2Co 12:9 Php 4:13}
\crossref{Ps}{83}{7}{Ps 66:10-\allowbreak12 Joh 16:33 Ac 14:22 Ro 5:3-\allowbreak5; 8:37 2Co 4:17 Re 7:14}
\crossref{Ps}{83}{8}{Job 17:9 Pr 4:18 Isa 40:31 Joh 15:2 2Co 3:18 2Pe 3:18}
\crossref{Ps}{83}{9}{83:1,\allowbreak2,\allowbreak3,\allowbreak11,\allowbreak12}
\crossref{Ps}{83}{10}{83:11; 98:1 Ge 15:1 De 33:29}
\crossref{Ps}{83}{11}{83:1,\allowbreak2; 27:4; 43:3,\allowbreak4; 63:2 Lu 2:46 Ro 8:5,\allowbreak6 Php 3:20}
\crossref{Ps}{83}{12}{Ps 27:1 Isa 60:19,\allowbreak20 Mal 4:2 Joh 1:9; 8:12 Re 21:23}
\crossref{Ps}{83}{13}{Ps 2:12; 34:8; 62:8; 146:5,\allowbreak6 Isa 30:18; 50:10 Jer 17:7,\allowbreak8}
\crossref{Ps}{83}{14}{}
\crossref{Ps}{83}{15}{}
\crossref{Ps}{83}{16}{}
\crossref{Ps}{83}{17}{}
\crossref{Ps}{83}{18}{}
\crossref{Ps}{84}{1}{Ps 42:1}
\crossref{Ps}{84}{2}{Ps 42:1}
\crossref{Ps}{84}{3}{Ps 32:1; 79:8,\allowbreak9 Jer 50:20 Mic 7:18 Ac 13:39 Col 2:13}
\crossref{Ps}{84}{4}{Isa 6:7; 12:1; 54:7-\allowbreak10 Joh 1:29}
\crossref{Ps}{84}{5}{Ps 25:2; 27:1 Mic 7:7,\allowbreak18-\allowbreak20 Joh 4:22}
\crossref{Ps}{84}{6}{84:74:1 77:9 79:5 80:4 89:46 Isa 64:9-\allowbreak12 Mic 7:18}
\crossref{Ps}{84}{7}{84:80:18; 138:7 Ezr 9:8,\allowbreak9 Isa 57:15 Ho 6:2 Hab 3:2}
\crossref{Ps}{84}{8}{Ps 50:23; 91:16 Jer 42:12}
\crossref{Ps}{84}{9}{Hab 2:1 Heb 12:25}
\crossref{Ps}{84}{10}{Ps 24:4,\allowbreak5; 50:23; 119:155 Isa 46:13 Mr 12:32-\allowbreak34 Joh 7:17 Ac 10:2-\allowbreak4}
\crossref{Ps}{84}{11}{84:89:14; 100:5 Ex 34:6,\allowbreak7 Mic 7:20 Lu 1:54,\allowbreak55 Joh 1:17}
\crossref{Ps}{84}{12}{Isa 4:2; 45:8; 53:2 Joh 14:6 1Jo 5:20,\allowbreak21}
\crossref{Ps}{85}{1}{}
\crossref{Ps}{85}{2}{Ps 4:3; 37:28; 119:94 1Sa 2:9 Joh 10:27-\allowbreak29; 17:11 1Pe 5:3-\allowbreak5}
\crossref{Ps}{85}{3}{Ps 56:1; 57:1}
\crossref{Ps}{85}{4}{Ps 51:12 Isa 61:3; 65:18; 66:13,\allowbreak14}
\crossref{Ps}{85}{5}{85:15; 25:8; 36:7; 52:1; 69:16; 119:68; 130:7; 145:8,\allowbreak9 Ex 34:6 Joe 2:13}
\crossref{Ps}{85}{6}{Ps 5:1,\allowbreak2; 17:1; 130:2}
\crossref{Ps}{85}{7}{Ps 18:6; 34:4-\allowbreak6; 50:15; 55:16-\allowbreak18; 77:1,\allowbreak2; 91:15; 142:1,\allowbreak3 Isa 26:16}
\crossref{Ps}{85}{8}{85:89:6,\allowbreak8 Ex 15:11 Isa 40:18,\allowbreak25 Jer 10:6,\allowbreak7,\allowbreak16 Da 3:29}
\crossref{Ps}{85}{9}{Ps 22:27-\allowbreak31; 66:4; 67:7; 72:8,\allowbreak19; 102:15,\allowbreak18 Isa 2:2-\allowbreak4; 11:9; 43:7}
\crossref{Ps}{85}{10}{85:8; 72:18; 77:14,\allowbreak15; 145:3-\allowbreak5 Ex 15:11 Job 11:7 Da 6:26,\allowbreak27}
\crossref{Ps}{85}{11}{Ps 5:8; 25:4,\allowbreak12; 27:11; 119:33,\allowbreak73; 143:8-\allowbreak10 Job 34:32 Joh 6:45,\allowbreak46}
\crossref{Ps}{85}{12}{Ps 34:1; 103:1-\allowbreak3; 104:33; 145:1-\allowbreak5; 146:1,\allowbreak2 1Ch 29:13,\allowbreak20 Isa 12:1}
\crossref{Ps}{85}{13}{Ps 57:10; 103:8-\allowbreak12; 108:4 Lu 1:58}
\crossref{Ps}{86}{1}{}
\crossref{Ps}{86}{2}{86:78:67-\allowbreak69 132:13,\allowbreak14 De 12:5 2Ch 6:6 Isa 14:32 Joe 2:32}
\crossref{Ps}{86}{3}{Ps 48:2,\allowbreak3,\allowbreak11-\allowbreak13; 125:1,\allowbreak2 Isa 12:6; 49:14-\allowbreak26; 54:2-\allowbreak10; 59:20,\allowbreak21}
\crossref{Ps}{86}{4}{86:89:10 Isa 51:9}
\crossref{Ps}{86}{5}{Isa 44:4,\allowbreak5; 60:1-\allowbreak9 Joh 1:12-\allowbreak14; 3:3-\allowbreak5 Ga 3:26-\allowbreak28 Heb 11:32-\allowbreak40}
\crossref{Ps}{86}{6}{Ps 22:30 Isa 4:3 Eze 9:4; 13:9 Lu 10:20 Php 4:3 Re 13:8}
\crossref{Ps}{86}{7}{86:68:24,\allowbreak25 1Ch 15:16-\allowbreak29; 23:5; 25:1-\allowbreak6 Re 14:1-\allowbreak3}
\crossref{Ps}{86}{8}{}
\crossref{Ps}{86}{9}{}
\crossref{Ps}{86}{10}{}
\crossref{Ps}{86}{11}{}
\crossref{Ps}{86}{12}{}
\crossref{Ps}{86}{13}{}
\crossref{Ps}{86}{14}{}
\crossref{Ps}{86}{15}{}
\crossref{Ps}{86}{16}{}
\crossref{Ps}{86}{17}{}
\crossref{Ps}{87}{1}{}
\crossref{Ps}{87}{2}{87:79:11; 141:1,\allowbreak2 1Ki 8:31 La 3:8}
\crossref{Ps}{87}{3}{87:79:11; 141:1,\allowbreak2 1Ki 8:31 La 3:8}
\crossref{Ps}{87}{4}{87:14,\allowbreak15; 22:11-\allowbreak21; 69:17-\allowbreak21; 77:2; 143:3,\allowbreak4 Job 6:2-\allowbreak4 Isa 53:3,\allowbreak10,\allowbreak11}
\crossref{Ps}{87}{5}{Ps 28:1; 30:9; 143:7 Job 17:1 Isa 38:17,\allowbreak18 Eze 26:20 Jon 2:6}
\crossref{Ps}{87}{6}{Isa 14:9-\allowbreak12; 38:10-\allowbreak12 Eze 32:18-\allowbreak32}
\crossref{Ps}{87}{7}{Ps 40:2; 86:13 De 32:22}
\crossref{Ps}{88}{1}{1Ki 4:31 1Ch 2:6}
\crossref{Ps}{88}{2}{1Ki 4:31 1Ch 2:6}
\crossref{Ps}{88}{3}{Ps 36:5; 103:17 Ne 1:5; 9:17,\allowbreak31 Lu 1:50 Eph 1:6,\allowbreak7}
\crossref{Ps}{88}{4}{88:28,\allowbreak34,\allowbreak39 2Sa 7:10-\allowbreak16; 23:5 1Ki 8:16 Isa 55:3 Jer 30:9; 33:20,\allowbreak21}
\crossref{Ps}{88}{5}{88:1,\allowbreak29,\allowbreak36; 72:17; 132:12 2Sa 7:12-\allowbreak16,\allowbreak29 1Ki 9:5 1Ch 17:10-\allowbreak14}
\crossref{Ps}{88}{6}{Ps 19:1; 50:6; 97:6 Isa 44:23 Lu 2:10-\allowbreak15 Eph 3:10 1Pe 1:12}
\crossref{Ps}{88}{7}{88:8; 40:5; 71:19; 73:25; 86:8; 113:5 Ex 15:11 Jer 10:6}
\crossref{Ps}{88}{8}{88:76:7-\allowbreak11 Le 10:3 Isa 6:2-\allowbreak7; 66:2 Jer 10:7,\allowbreak10 Mt 10:28}
\crossref{Ps}{88}{9}{88:84:12 Jos 22:22 Isa 28:22}
\crossref{Ps}{88}{10}{Ps 29:10; 65:7; 66:5,\allowbreak6; 93:3,\allowbreak4; 107:25-\allowbreak29 Job 38:8-\allowbreak11 Na 1:4}
\crossref{Ps}{88}{11}{88:78:43-\allowbreak72 105:27-\allowbreak45 Ex 7:1-\allowbreak15:27}
\crossref{Ps}{88}{12}{Ps 24:1,\allowbreak2; 50:12; 115:16 Ge 1:1; 2:1 1Ch 29:11 Job 41:11}
\crossref{Ps}{88}{13}{Job 26:7}
\crossref{Ps}{88}{14}{88:10; 62:11 Da 4:34,\allowbreak35 Mt 6:13}
\crossref{Ps}{88}{15}{Ps 45:6,\allowbreak7; 97:2; 99:4; 145:17 De 32:4 Re 15:3}
\crossref{Ps}{88}{16}{88:90:6 98:4-\allowbreak6 100:1 Le 25:9 Nu 10:10; 23:21 Isa 52:7,\allowbreak8 Na 1:15}
\crossref{Ps}{88}{17}{88:12; 29:5,\allowbreak7; 33:21; 44:8 Lu 1:47 Php 4:4}
\crossref{Ps}{88}{18}{Ps 28:7 1Co 1:30,\allowbreak31 2Co 12:9,\allowbreak10 Php 4:13}
\crossref{Ps}{89}{1}{Ex 33:14-\allowbreak19 De 33:1 1Ki 13:1 1Ti 6:11}
\crossref{Ps}{89}{2}{Job 38:4-\allowbreak6,\allowbreak28,\allowbreak29 Pr 8:25,\allowbreak26}
\crossref{Ps}{89}{3}{89:104:29; 146:4 Ge 3:19; 6:6,\allowbreak7 Nu 14:35 Job 12:10; 34:14,\allowbreak15}
\crossref{Ps}{89}{4}{2Pe 3:8}
\crossref{Ps}{89}{5}{Job 9:26; 22:16; 27:20,\allowbreak21 Isa 8:7,\allowbreak8 Jer 46:7,\allowbreak8}
\crossref{Ps}{89}{6}{89:92:7 Job 14:2 Mt 6:30}
\crossref{Ps}{89}{7}{89:9,\allowbreak11; 39:11; 59:13 Nu 17:12,\allowbreak13 De 2:14-\allowbreak16 Heb 3:10,\allowbreak11,\allowbreak17-\allowbreak19}
\crossref{Ps}{89}{8}{Ps 10:11; 50:21; 139:1-\allowbreak4 Job 34:21 Jer 9:13-\allowbreak16; 16:17; 23:24}
\crossref{Ps}{89}{9}{89:78:33}
\crossref{Ps}{89}{10}{2Sa 19:35 1Ki 1:1 Ec 12:2-\allowbreak7}
\crossref{Ps}{89}{11}{Le 26:18,\allowbreak21,\allowbreak24,\allowbreak28 De 28:59; 29:20-\allowbreak29 Isa 33:14 Na 1:6 Lu 12:5}
\crossref{Ps}{89}{12}{Ps 39:4 De 32:29 Ec 9:10 Lu 12:35-\allowbreak40 Joh 9:4 Eph 5:16,\allowbreak17}
\crossref{Ps}{89}{13}{Ps 6:4; 80:14 Jer 12:15 Joe 2:13,\allowbreak14 Zec 1:16}
\crossref{Ps}{89}{14}{Ps 36:7,\allowbreak8; 63:3-\allowbreak5; 65:4; 103:3-\allowbreak5 Jer 31:15 Zec 9:17}
\crossref{Ps}{89}{15}{Ps 30:5; 126:5,\allowbreak6 Isa 12:1; 40:1,\allowbreak2; 61:3; 65:18,\allowbreak19 Jer 31:12,\allowbreak13}
\crossref{Ps}{89}{16}{Ps 44:1 Nu 14:15-\allowbreak24 Hab 3:2}
\crossref{Ps}{89}{17}{Ps 27:4; 50:2; 80:3,\allowbreak7; 110:3 2Co 3:18 1Jo 3:2}
\crossref{Ps}{89}{18}{}
\crossref{Ps}{89}{19}{}
\crossref{Ps}{89}{20}{}
\crossref{Ps}{89}{21}{}
\crossref{Ps}{89}{22}{}
\crossref{Ps}{89}{23}{}
\crossref{Ps}{89}{24}{}
\crossref{Ps}{89}{25}{}
\crossref{Ps}{89}{26}{}
\crossref{Ps}{89}{27}{}
\crossref{Ps}{89}{28}{}
\crossref{Ps}{89}{29}{}
\crossref{Ps}{89}{30}{}
\crossref{Ps}{89}{31}{}
\crossref{Ps}{89}{32}{}
\crossref{Ps}{89}{33}{}
\crossref{Ps}{89}{34}{}
\crossref{Ps}{89}{35}{}
\crossref{Ps}{89}{36}{}
\crossref{Ps}{89}{37}{}
\crossref{Ps}{89}{38}{}
\crossref{Ps}{89}{39}{}
\crossref{Ps}{89}{40}{}
\crossref{Ps}{89}{41}{}
\crossref{Ps}{89}{42}{}
\crossref{Ps}{89}{43}{}
\crossref{Ps}{89}{44}{}
\crossref{Ps}{89}{45}{}
\crossref{Ps}{89}{46}{}
\crossref{Ps}{89}{47}{}
\crossref{Ps}{89}{48}{}
\crossref{Ps}{89}{49}{}
\crossref{Ps}{89}{50}{}
\crossref{Ps}{89}{51}{}
\crossref{Ps}{89}{52}{}
\crossref{Ps}{90}{1}{}
\crossref{Ps}{90}{2}{90:9; 18:2; 46:1; 71:3; 142:5 De 32:30,\allowbreak31; 33:27-\allowbreak29 Pr 18:10}
\crossref{Ps}{90}{3}{90:124:7; 141:9 Pr 7:23 Ec 9:12 Ho 9:8 Am 3:5 1Ti 6:9 2Ti 2:26}
\crossref{Ps}{90}{4}{Ps 17:8; 57:1; 61:4 De 32:11 Ru 2:12 Mt 23:37}
\crossref{Ps}{90}{5}{Ps 3:6; 27:1-\allowbreak3; 46:2; 112:7 Job 5:19-\allowbreak27 Pr 28:1 Isa 43:2 Mt 8:26}
\crossref{Ps}{90}{6}{90:121:5,\allowbreak6 Ex 12:29,\allowbreak30 2Ki 19:35}
\crossref{Ps}{90}{7}{Ps 32:6 Ge 7:23 Ex 12:12,\allowbreak13 Nu 14:37,\allowbreak38 Jos 14:10}
\crossref{Ps}{90}{8}{Ps 37:34; 58:10,\allowbreak11; 92:11 Pr 3:25,\allowbreak26 Mal 1:5}
\crossref{Ps}{90}{9}{90:2; 142:4,\allowbreak5; 146:5,\allowbreak6}
\crossref{Ps}{90}{10}{90:121:7 Pr 12:21 Ro 8:25}
\crossref{Ps}{90}{11}{Ps 34:7; 71:3 2Ki 6:16,\allowbreak17 Mt 4:6 Lu 4:10,\allowbreak11 Heb 1:14}
\crossref{Ps}{90}{12}{Isa 46:3; 63:9}
\crossref{Ps}{90}{13}{Jud 14:5,\allowbreak6 Job 5:23 1Sa 17:37 Da 6:22 2Ti 4:17}
\crossref{Ps}{90}{14}{90:9 1Ch 29:3 Joh 14:23; 16:27 Ro 8:28 Jas 1:12; 2:5}
\crossref{Ps}{90}{15}{Ps 10:17; 18:3,\allowbreak4,\allowbreak15 Isa 58:9; 65:24 Jer 29:12,\allowbreak13; 33:3 Ro 10:12,\allowbreak13}
\crossref{Ps}{90}{16}{Ps 21:4 Ge 25:8 Job 5:26 Pr 3:2,\allowbreak16; 22:4 Isa 65:20-\allowbreak22}
\crossref{Ps}{90}{17}{}
\crossref{Ps}{91}{1}{Isa 58:13,\allowbreak14 Heb 4:9}
\crossref{Ps}{91}{2}{Isa 58:13,\allowbreak14 Heb 4:9}
\crossref{Ps}{91}{3}{91:71:15; 89:1,\allowbreak2 145:2 Isa 63:7 La 3:22,\allowbreak23 Joh 1:17}
\crossref{Ps}{91}{4}{Ps 33:2; 57:8; 68:25; 81:2,\allowbreak3; 149:3; 150:3-\allowbreak5 1Ch 15:16; 25:6 2Ch 23:5}
\crossref{Ps}{91}{5}{Ps 64:10; 104:31,\allowbreak34; 106:47,\allowbreak48; 126:3; 145:6,\allowbreak7 Isa 61:2-\allowbreak11; 65:13,\allowbreak14}
\crossref{Ps}{91}{6}{Ps 40:5; 66:3; 104:24; 111:2; 145:3,\allowbreak4 Re 15:3}
\crossref{Ps}{91}{7}{Ps 32:9; 73:22; 94:8 Pr 30:2 Isa 1:3 Jer 10:14 1Co 2:14}
\crossref{Ps}{91}{8}{Ps 37:1,\allowbreak2,\allowbreak35,\allowbreak38; 90:5,\allowbreak6; 103:15,\allowbreak16 Isa 37:27; 40:6,\allowbreak7 Jas 1:10,\allowbreak11}
\crossref{Ps}{91}{9}{Ps 56:2; 83:18; 102:26,\allowbreak27 Ex 18:11 Ec 5:8 Da 4:34,\allowbreak35 Ac 12:1,\allowbreak22-\allowbreak24}
\crossref{Ps}{91}{10}{Ps 21:8,\allowbreak9; 37:20; 68:1,\allowbreak2; 73:27; 89:10 Jud 5:31 Lu 19:27 2Th 1:7-\allowbreak9}
\crossref{Ps}{91}{11}{91:89:17,\allowbreak24 112:9 132:17 148:14 1Sa 2:1,\allowbreak10 Lu 1:69}
\crossref{Ps}{91}{12}{Ps 37:34; 54:7; 59:10; 91:8; 112:8}
\crossref{Ps}{91}{13}{91:7; 52:8 Isa 55:13; 65:22 Ho 14:5,\allowbreak6}
\crossref{Ps}{91}{14}{Isa 60:21 Ro 6:5; 11:17 Eph 3:17}
\crossref{Ps}{91}{15}{Ps 1:3 Mt 3:10 Joh 15:2-\allowbreak5 Ga 5:22,\allowbreak23 Php 1:11 Jude 1:12}
\crossref{Ps}{91}{16}{Joh 10:27-\allowbreak29; 15:1-\allowbreak3 1Co 1:8,\allowbreak9 1Th 5:23,\allowbreak24 Tit 1:2 1Pe 1:4,\allowbreak5}
\crossref{Ps}{92}{1}{Ps 59:13; 96:10; 97:1; 99:1; 103:19; 145:13 1Ch 29:12 Isa 52:7}
\crossref{Ps}{92}{2}{Ps 45:6; 145:13 Pr 8:22,\allowbreak23 Da 4:34 Mic 5:2}
\crossref{Ps}{92}{3}{Ps 18:4; 69:1,\allowbreak2,\allowbreak14-\allowbreak16 Isa 17:12,\allowbreak13 Jer 46:7,\allowbreak8 Jon 2:3 Re 12:15}
\crossref{Ps}{92}{4}{Ps 65:7; 89:6,\allowbreak9; 114:3-\allowbreak5 Job 38:11 Jer 5:22 Mr 4:37-\allowbreak39}
\crossref{Ps}{92}{5}{Ps 19:7,\allowbreak8; 119:111,\allowbreak129,\allowbreak138,\allowbreak144 Isa 8:20 Mt 24:35 Heb 6:17,\allowbreak18}
\crossref{Ps}{92}{6}{}
\crossref{Ps}{92}{7}{}
\crossref{Ps}{92}{8}{}
\crossref{Ps}{92}{9}{}
\crossref{Ps}{92}{10}{}
\crossref{Ps}{92}{11}{}
\crossref{Ps}{92}{12}{}
\crossref{Ps}{92}{13}{}
\crossref{Ps}{92}{14}{}
\crossref{Ps}{92}{15}{}
\crossref{Ps}{93}{1}{93:80:1}
\crossref{Ps}{93}{2}{Ps 7:6; 68:1; 74:22 Mic 5:9}
\crossref{Ps}{93}{3}{Ps 43:2; 73:8; 74:9,\allowbreak10; 79:5; 80:4; 89:46 Jer 12:1,\allowbreak2; 47:6 Re 6:10}
\crossref{Ps}{93}{4}{Ps 31:18; 59:7,\allowbreak12; 64:3,\allowbreak4; 73:8,\allowbreak9; 140:3 Pr 30:14 Jer 18:18}
\crossref{Ps}{93}{5}{Ps 7:2; 14:4; 44:22; 74:8,\allowbreak19,\allowbreak20; 79:2,\allowbreak3,\allowbreak7; 129:2,\allowbreak3 Isa 3:15; 52:5}
\crossref{Ps}{94}{1}{Ps 34:3; 66:8; 107:8,\allowbreak15,\allowbreak21; 117:1; 118:1; 136:1-\allowbreak3; 148:11-\allowbreak13; 150:6}
\crossref{Ps}{94}{2}{Ps 7:7; 100:2,\allowbreak4 Jer 31:12,\allowbreak13}
\crossref{Ps}{94}{3}{94:86:8-\allowbreak10 96:4 97:9 145:3 Jer 10:6,\allowbreak7}
\crossref{Ps}{94}{4}{Ps 21:2 Job 11:10}
\crossref{Ps}{94}{5}{Ps 33:7 Ge 1:9,\allowbreak10 Job 38:10,\allowbreak11 Pr 8:29 Jer 5:22}
\crossref{Ps}{94}{6}{94:1 Ho 6:1 Mt 4:2 Re 22:17}
\crossref{Ps}{94}{7}{Ps 48:14; 67:6; 115:3 Ex 15:2; 20:2 Jer 31:33 Heb 11:16}
\crossref{Ps}{94}{8}{Ex 8:15 1Sa 6:6 Da 5:20 Ac 19:9 Ro 2:5 Heb 3:13; 12:25}
\crossref{Ps}{94}{9}{94:78:17,\allowbreak18,\allowbreak40,\allowbreak41,\allowbreak56 1Co 10:9}
\crossref{Ps}{94}{10}{Nu 14:33,\allowbreak34; 32:13 De 1:3; 2:14-\allowbreak16 Heb 3:9,\allowbreak10,\allowbreak17}
\crossref{Ps}{94}{11}{Nu 14:23,\allowbreak28-\allowbreak30 De 1:34,\allowbreak35 Heb 3:11,\allowbreak18; 4:3,\allowbreak5}
\crossref{Ps}{94}{12}{}
\crossref{Ps}{94}{13}{}
\crossref{Ps}{94}{14}{}
\crossref{Ps}{94}{15}{}
\crossref{Ps}{94}{16}{}
\crossref{Ps}{94}{17}{}
\crossref{Ps}{94}{18}{}
\crossref{Ps}{94}{19}{}
\crossref{Ps}{94}{20}{}
\crossref{Ps}{94}{21}{}
\crossref{Ps}{94}{22}{}
\crossref{Ps}{94}{23}{}
\crossref{Ps}{95}{1}{Ps 33:3; 98:1; 149:1 1Ch 16:23-\allowbreak33 Re 5:9; 14:3}
\crossref{Ps}{95}{2}{95:72:17,\allowbreak18 103:1,\allowbreak2,\allowbreak20-\allowbreak22 104:1 145:1,\allowbreak10 1Ch 29:20 Eph 1:3}
\crossref{Ps}{95}{3}{Ps 22:27; 72:18,\allowbreak19; 117:1,\allowbreak2 Isa 19:23-\allowbreak25; 49:6 Da 4:1-\allowbreak3; 6:26,\allowbreak27}
\crossref{Ps}{95}{4}{Ps 18:3; 86:10; 89:7; 145:3 Ex 18:11 1Sa 4:8 Ne 9:5}
\crossref{Ps}{95}{5}{95:115:3-\allowbreak8; 135:15,\allowbreak18 Isa 44:8-\allowbreak28; 46:1,\allowbreak2 Jer 10:3-\allowbreak5,\allowbreak11,\allowbreak12,\allowbreak14,\allowbreak15}
\crossref{Ps}{95}{6}{Ps 8:1; 19:1; 63:2,\allowbreak3; 93:1; 104:1 Heb 1:3 2Pe 1:16,\allowbreak17}
\crossref{Ps}{95}{7}{Ps 29:1,\allowbreak2; 68:32-\allowbreak34 Lu 2:14 Jude 1:25}
\crossref{Ps}{95}{8}{95:108:3-\allowbreak5; 111:9; 148:13,\allowbreak14 Ex 34:5-\allowbreak9 Re 15:4}
\crossref{Ps}{95}{9}{Ps 29:2; 110:3 Ezr 7:27 Eze 7:20 Da 11:45 Lu 21:5,\allowbreak6}
\crossref{Ps}{95}{10}{Ps 18:49; 46:6,\allowbreak10; 126:2 Mal 1:11,\allowbreak14 Ga 1:16}
\crossref{Ps}{95}{11}{95:69:34; 148:1-\allowbreak4 Isa 44:23; 49:13 Lu 2:10,\allowbreak13,\allowbreak14; 15:10 Re 12:12}
\crossref{Ps}{96}{1}{96:93:1 96:10,\allowbreak11 99:1 Ob 1:21 Mt 3:3; 6:10,\allowbreak13 Mr 11:10 Col 1:13}
\crossref{Ps}{96}{2}{Ps 18:11,\allowbreak12; 77:19 Ex 20:21; 24:16-\allowbreak18 De 4:11,\allowbreak12 1Ki 8:10-\allowbreak12}
\crossref{Ps}{96}{3}{Ps 18:8; 21:8,\allowbreak9; 50:3 De 4:11,\allowbreak36; 5:4,\allowbreak23,\allowbreak24; 32:22 Da 7:10 Na 1:5,\allowbreak6}
\crossref{Ps}{96}{4}{96:77:18; 144:5,\allowbreak6 Ex 19:16-\allowbreak18}
\crossref{Ps}{96}{5}{Jud 5:4,\allowbreak5 Isa 24:19,\allowbreak20; 64:1,\allowbreak2 Mic 1:3,\allowbreak4 Na 1:5 Hab 3:6}
\crossref{Ps}{96}{6}{Ps 19:1; 36:5,\allowbreak6; 50:6; 89:2,\allowbreak5 Isa 1:2 Re 19:2}
\crossref{Ps}{96}{7}{Ex 20:4 Le 26:1 De 5:8; 27:15 Isa 37:18,\allowbreak19; 41:29; 42:17; 44:9-\allowbreak11}
\crossref{Ps}{96}{8}{Ps 48:11 Isa 51:3; 52:7-\allowbreak10; 62:11 Zep 3:14-\allowbreak17 Zec 9:9 Mt 21:4-\allowbreak9}
\crossref{Ps}{96}{9}{96:83:18 Eph 1:21 Php 2:9-\allowbreak11}
\crossref{Ps}{96}{10}{96:91:14 Ro 8:28 1Co 8:3 Jas 1:12; 2:5 1Pe 1:8 1Jo 4:19; 5:2,\allowbreak3}
\crossref{Ps}{96}{11}{Ps 18:28; 112:4 Es 8:16 Job 22:28 Pr 4:18 Isa 60:1,\allowbreak2; 62:1 Mic 7:9}
\crossref{Ps}{96}{12}{Ps 32:11; 33:1 Hab 3:17,\allowbreak18 Zep 3:14-\allowbreak17 Php 4:4}
\crossref{Ps}{96}{13}{}
\crossref{Ps}{97}{1}{Ps 33:3; 96:1; 149:1 Isa 42:10 Re 5:9; 14:3}
\crossref{Ps}{97}{2}{Isa 45:21-\allowbreak23; 49:6; 52:10 Mt 28:19 Mr 16:15 Lu 2:30-\allowbreak32; 3:6}
\crossref{Ps}{97}{3}{97:106:45 Le 26:42 De 4:31 Mic 7:20 Lu 1:54,\allowbreak55,\allowbreak72 Ro 15:8,\allowbreak9}
\crossref{Ps}{97}{4}{Ps 47:1-\allowbreak5; 66:1,\allowbreak4; 67:4; 95:1; 100:1 Isa 12:6; 42:11; 44:23 Jer 33:11}
\crossref{Ps}{97}{5}{Ps 33:2; 92:3,\allowbreak4 1Ch 15:16; 25:1-\allowbreak6 2Ch 29:25 Re 5:8; 14:2,\allowbreak3}
\crossref{Ps}{97}{6}{Ps 47:5; 81:2-\allowbreak4 Nu 10:1-\allowbreak10 1Ch 15:28 2Ch 5:12,\allowbreak13; 29:27}
\crossref{Ps}{97}{7}{97:96:11-\allowbreak99:9}
\crossref{Ps}{97}{8}{Ps 47:1 2Ki 11:12 Isa 55:12}
\crossref{Ps}{97}{9}{97:96:10,\allowbreak13 Re 1:7}
\crossref{Ps}{97}{10}{}
\crossref{Ps}{97}{11}{}
\crossref{Ps}{97}{12}{}
\crossref{Ps}{98}{1}{Ps 2:6; 93:1; 96:10; 97:1 Lu 19:12,\allowbreak14 Re 11:17}
\crossref{Ps}{98}{2}{Ps 48:1-\allowbreak3; 50:2; 76:1,\allowbreak2 Isa 12:6; 14:32 Heb 12:22-\allowbreak24 Re 14:1-\allowbreak5}
\crossref{Ps}{98}{3}{Ps 66:3; 76:12 De 7:21; 28:58 Ne 1:5; 4:14; 9:32 Jer 20:11}
\crossref{Ps}{98}{4}{Ps 45:6,\allowbreak7; 72:1,\allowbreak2 De 32:3,\allowbreak4 2Sa 23:3,\allowbreak4 Job 36:5-\allowbreak7; 37:23}
\crossref{Ps}{98}{5}{98:9; 21:13; 34:3; 108:5 Ex 15:2 Isa 12:4; 25:1 Ho 11:7}
\crossref{Ps}{98}{6}{Ex 24:6-\allowbreak8; 29:11-\allowbreak37; 40:23-\allowbreak29 Nu 16:47,\allowbreak48}
\crossref{Ps}{98}{7}{Ex 19:9; 33:9 Nu 12:5}
\crossref{Ps}{98}{8}{98:89:33 Nu 14:20 De 9:19 Jer 46:28 Zep 3:7}
\crossref{Ps}{98}{9}{98:5}
\crossref{Ps}{99}{1}{99:145:1}
\crossref{Ps}{99}{2}{Ps 63:4,\allowbreak5; 71:23; 107:21,\allowbreak22 De 12:12; 16:11,\allowbreak14; 28:47 1Ki 8:66}
\crossref{Ps}{99}{3}{Ps 46:10; 95:3,\allowbreak6,\allowbreak7 De 4:35,\allowbreak39; 7:9 1Sa 17:46,\allowbreak47 1Ki 18:36-\allowbreak39}
\crossref{Ps}{99}{4}{Ps 65:1; 66:13; 116:17-\allowbreak19 Isa 35:10}
\crossref{Ps}{99}{5}{Ps 52:1; 86:5; 106:1; 107:1,\allowbreak8,\allowbreak15,\allowbreak22; 119:68 Jer 33:11}
\crossref{Ps}{99}{6}{}
\crossref{Ps}{99}{7}{}
\crossref{Ps}{99}{8}{}
\crossref{Ps}{99}{9}{}
\crossref{Ps}{100}{1}{100:89:1 97:8 103:6-\allowbreak8 136:10-\allowbreak22 Ro 9:15-\allowbreak18,\allowbreak22,\allowbreak23; 11:22 Re 15:3,\allowbreak4}
\crossref{Ps}{100}{2}{100:6; 75:1,\allowbreak2; 119:106,\allowbreak115 1Sa 18:14,\allowbreak15; 22:14 2Sa 8:15 2Ch 30:12}
\crossref{Ps}{100}{3}{Ps 18:20-\allowbreak23; 26:4,\allowbreak5; 39:1; 119:37,\allowbreak113 Ex 20:17 2Sa 11:2,\allowbreak3}
\crossref{Ps}{100}{4}{Pr 2:12-\allowbreak15; 3:32; 8:13; 11:20}
\crossref{Ps}{100}{5}{Ps 15:3; 50:20 Ex 20:16; 23:1 Le 19:16 Pr 10:18; 20:19; 25:23}
\crossref{Ps}{101}{1}{Ps 5:2; 55:1-\allowbreak5; 57:1-\allowbreak3; 130:1,\allowbreak2; 41:1,\allowbreak2; 143:7; 145:19}
\crossref{Ps}{101}{2}{Ps 5:2; 55:1-\allowbreak5; 57:1-\allowbreak3; 130:1,\allowbreak2; 41:1,\allowbreak2; 143:7; 145:19}
\crossref{Ps}{101}{3}{Ps 13:1; 27:9; 69:17; 88:14; 104:29; 143:7 Job 34:29 Isa 8:17; 43:2}
\crossref{Ps}{101}{4}{Ps 37:20; 119:83 Jas 4:14}
\crossref{Ps}{101}{5}{Ps 6:2,\allowbreak3; 42:6; 55:4,\allowbreak5; 69:20; 77:3; 143:3,\allowbreak4 Job 6:4; 10:1 La 3:13,\allowbreak20}
\crossref{Ps}{101}{6}{Ps 6:6,\allowbreak8; 32:3,\allowbreak4; 38:8-\allowbreak10 Job 19:20 Pr 17:22 La 4:8}
\crossref{Ps}{101}{7}{Job 30:29,\allowbreak30 Isa 38:14 Mic 1:8}
\crossref{Ps}{101}{8}{Ps 22:2; 77:4; 130:6 De 28:66,\allowbreak67 Job 7:13-\allowbreak16 Mr 14:33-\allowbreak37}
\crossref{Ps}{102}{1}{102:22; 104:1; 146:1,\allowbreak2 Lu 1:46,\allowbreak47}
\crossref{Ps}{102}{2}{102:105:5; 106:7,\allowbreak21; 116:12 De 8:2-\allowbreak4,\allowbreak10-\allowbreak14; 32:6,\allowbreak18 2Ch 32:25}
\crossref{Ps}{102}{3}{Ps 32:1-\allowbreak5; 51:1-\allowbreak3; 130:8 2Sa 12:13 Isa 43:25 Mt 9:2-\allowbreak6 Mr 2:5,\allowbreak10,\allowbreak11}
\crossref{Ps}{102}{4}{Ps 34:22; 56:13; 71:23 Ge 48:16 Job 33:19-\allowbreak30 Re 5:9}
\crossref{Ps}{102}{5}{Ps 23:5; 63:5; 65:4; 104:28; 107:9; 115:15,\allowbreak16 1Ti 6:17}
\crossref{Ps}{102}{6}{Ps 9:9; 10:14-\allowbreak18; 12:5; 72:4,\allowbreak12; 109:31; 146:7 De 24:14,\allowbreak15}
\crossref{Ps}{102}{7}{102:77:20; 105:26-\allowbreak45 Ex 19:8,\allowbreak20; 20:21; 24:2-\allowbreak4 Nu 12:7 De 34:10}
\crossref{Ps}{102}{8}{102:86:5,\allowbreak15 130:7 145:8 Ex 34:6,\allowbreak7 Nu 14:18 De 5:10 Ne 9:17}
\crossref{Ps}{102}{9}{Ps 30:5 Isa 57:16 Jer 3:5 Mic 7:18,\allowbreak19}
\crossref{Ps}{102}{10}{102:130:3 Ezr 9:13 Ne 9:31 Job 11:6 La 3:22 Da 9:18,\allowbreak19 Hab 3:2}
\crossref{Ps}{102}{11}{Ps 36:5; 57:10; 89:2 Job 22:12 Pr 25:3 Isa 55:9 Eph 2:4-\allowbreak7; 3:18,\allowbreak19}
\crossref{Ps}{102}{12}{Ps 50:1; 113:3 Isa 45:6}
\crossref{Ps}{102}{13}{Nu 11:12 De 3:5 Pr 3:12 Isa 63:15,\allowbreak16 Jer 31:9,\allowbreak20 Mt 6:9,\allowbreak32}
\crossref{Ps}{102}{14}{102:78:38,\allowbreak39 89:47}
\crossref{Ps}{102}{15}{102:90:5,\allowbreak6 Isa 40:6-\allowbreak8; 51:12 Jas 1:10,\allowbreak11 1Pe 1:24}
\crossref{Ps}{102}{16}{Job 27:20,\allowbreak21 Isa 40:7}
\crossref{Ps}{102}{17}{102:89:1,\allowbreak2 100:5 118:1 136:1-\allowbreak26 Jer 31:3 Ro 8:28-\allowbreak30 Eph 1:4-\allowbreak8}
\crossref{Ps}{102}{18}{Ps 25:10; 132:12 Ge 17:9,\allowbreak10 Ex 19:5; 24:8 De 7:9 2Ch 34:31}
\crossref{Ps}{102}{19}{Ps 2:4; 9:7; 11:4; 115:3 Isa 66:1 Heb 8:1}
\crossref{Ps}{102}{20}{102:148:2 Lu 2:13,\allowbreak14 Re 19:5,\allowbreak6}
\crossref{Ps}{102}{21}{Ps 33:6 Ge 32:2 Jos 5:14 1Ki 22:19 2Ch 18:18 Lu 2:13}
\crossref{Ps}{102}{22}{102:145:10; 148:3-\allowbreak12; 150:6 Isa 42:10-\allowbreak12; 43:20; 44:23; 49:13}
\crossref{Ps}{102}{23}{}
\crossref{Ps}{102}{24}{}
\crossref{Ps}{102}{25}{}
\crossref{Ps}{102}{26}{}
\crossref{Ps}{102}{27}{}
\crossref{Ps}{102}{28}{}
\crossref{Ps}{103}{1}{Ps 7:1 Da 9:4 Hab 1:12}
\crossref{Ps}{103}{2}{Da 7:9 Mt 17:2 1Ti 6:16 1Jo 1:5}
\crossref{Ps}{103}{3}{Ps 18:10,\allowbreak11 Am 9:6}
\crossref{Ps}{103}{4}{Ac 23:8 Heb 1:7,\allowbreak14}
\crossref{Ps}{103}{5}{103:93:1 96:10 Ec 1:4 2Pe 3:10 Re 6:14; 20:11}
\crossref{Ps}{103}{6}{Ge 1:2-\allowbreak10; 7:19 2Pe 3:5}
\crossref{Ps}{103}{7}{Ge 8:1 Pr 8:28 Mr 4:39}
\crossref{Ps}{103}{8}{}
\crossref{Ps}{103}{9}{Ps 33:7 Ge 9:11-\allowbreak15 Job 26:10; 38:10,\allowbreak11 Isa 54:9 Jer 5:22}
\crossref{Ps}{103}{10}{103:107:35}
\crossref{Ps}{103}{11}{103:145:16}
\crossref{Ps}{103}{12}{103:16,\allowbreak17; 50:11; 84:3; 148:10 Mt 6:26}
\crossref{Ps}{103}{13}{103:147:8 De 11:11 Job 38:25-\allowbreak28,\allowbreak37 Jer 10:13; 14:22 Mt 5:45}
\crossref{Ps}{103}{14}{103:145:15,\allowbreak16; 147:8,\allowbreak9 Ge 1:11,\allowbreak12,\allowbreak29; 2:5 1Ki 18:5 Jer 14:5,\allowbreak6}
\crossref{Ps}{103}{15}{Ps 23:5 Jud 9:13 Pr 31:6 Ec 10:19 Jer 31:12 Zec 9:15-\allowbreak17 Mr 14:23}
\crossref{Ps}{103}{16}{Ps 29:5; 92:2 Nu 24:6 Eze 17:23}
\crossref{Ps}{103}{17}{103:12 Jer 22:23 Eze 31:6 Da 4:21 Ob 1:4 Mt 13:32}
\crossref{Ps}{103}{18}{De 14:7 Pr 30:26}
\crossref{Ps}{103}{19}{Ps 8:3; 136:7-\allowbreak9 Ge 1:14-\allowbreak18 De 4:19 Job 31:26-\allowbreak28; 38:12 Jer 31:35}
\crossref{Ps}{103}{20}{103:74:16; 139:10-\allowbreak12 Ge 1:4,\allowbreak5; 8:22 Isa 45:7 Am 1:13}
\crossref{Ps}{103}{21}{Ps 34:10 Job 38:39 Isa 31:4 Eze 19:2-\allowbreak14 Am 3:4}
\crossref{Ps}{103}{22}{Job 24:13-\allowbreak17 Na 3:17 Joh 3:20}
\crossref{Ps}{104}{1}{104:136:1-\allowbreak3 1Ch 16:7-\allowbreak22; 25:3; 29:13,\allowbreak20}
\crossref{Ps}{104}{2}{Ps 47:6,\allowbreak7; 96:1,\allowbreak2; 98:1,\allowbreak5 Jud 5:3 Isa 12:5,\allowbreak6; 42:10-\allowbreak12 Eph 5:19}
\crossref{Ps}{104}{3}{Ps 34:2 Isa 45:25 Jer 9:23,\allowbreak24 1Co 1:29,\allowbreak31 Ga 6:14}
\crossref{Ps}{104}{4}{Am 5:4-\allowbreak6 Zep 2:2,\allowbreak3}
\crossref{Ps}{104}{5}{104:77:11; 103:2 De 7:18,\allowbreak19; 8:2; 32:7 Isa 43:18,\allowbreak19 Lu 22:19}
\crossref{Ps}{104}{6}{Ex 3:6 Isa 41:8,\allowbreak14; 44:1,\allowbreak2 Ro 9:4-\allowbreak29}
\crossref{Ps}{104}{7}{104:95:7 100:3 Ge 17:7 Ex 20:2 De 26:17,\allowbreak18; 29:10-\allowbreak15 Jos 24:15-\allowbreak24}
\crossref{Ps}{104}{8}{104:42; 111:5,\allowbreak9 1Ch 16:15 Ne 1:5 Da 9:4 Lu 1:72-\allowbreak74}
\crossref{Ps}{104}{9}{Ge 17:2; 22:16,\allowbreak17; 26:3; 28:13; 35:11 Ne 9:8 Ac 7:8 Heb 6:17}
\crossref{Ps}{104}{10}{Ge 17:7,\allowbreak8 2Sa 23:5 Heb 13:20}
\crossref{Ps}{104}{11}{Ge 12:7; 13:15; 15:18; 26:3,\allowbreak4; 28:13}
\crossref{Ps}{104}{12}{Ge 34:30 De 7:7; 26:5 Isa 51:2 Eze 33:24-\allowbreak33}
\crossref{Ps}{104}{13}{Ps 12:8; 39:6 Ge 13:17}
\crossref{Ps}{104}{14}{Ge 12:14-\allowbreak17; 20:1-\allowbreak7; 26:14-\allowbreak33; 31:24-\allowbreak29; 35:5 Ex 7:16,\allowbreak17}
\crossref{Ps}{104}{15}{Ge 26:11 Zec 2:8}
\crossref{Ps}{104}{16}{Ge 41:25-\allowbreak32,\allowbreak54; 42:5,\allowbreak6 2Ki 8:1 Am 3:6; 7:1-\allowbreak4 Hag 1:10,\allowbreak11; 2:17}
\crossref{Ps}{104}{17}{Ge 45:5,\allowbreak7,\allowbreak8; 50:20}
\crossref{Ps}{104}{18}{Ge 39:20; 40:15 Ac 16:24}
\crossref{Ps}{104}{19}{Ps 44:4 Ge 41:11-\allowbreak16,\allowbreak25 Pr 21:1 Da 2:30 Ac 7:10}
\crossref{Ps}{104}{20}{Ge 41:14}
\crossref{Ps}{104}{21}{Ge 41:40-\allowbreak44,\allowbreak55; 45:8,\allowbreak26}
\crossref{Ps}{104}{22}{Ge 41:33,\allowbreak38 Isa 19:11}
\crossref{Ps}{104}{23}{Ge 45:9-\allowbreak11; 46:2-\allowbreak7 Jos 24:4 Ac 7:11-\allowbreak15}
\crossref{Ps}{104}{24}{Ge 13:16; 46:3 Ex 1:7 De 26:5 Ac 7:17 Heb 11:12}
\crossref{Ps}{104}{25}{Ge 15:13 Ex 9:16; 10:1 De 2:30 Ro 9:17-\allowbreak19}
\crossref{Ps}{104}{26}{104:77:20 Ex 3:10; 4:12-\allowbreak14; 6:11,\allowbreak26,\allowbreak27 Jos 24:5 Mic 6:4 Ac 7:34,\allowbreak35}
\crossref{Ps}{104}{27}{104:78:43-\allowbreak51 135:8,\allowbreak9 Ex 7:1-\allowbreak11:10 De 4:34 Ne 9:10,\allowbreak11 Isa 63:11,\allowbreak12}
\crossref{Ps}{104}{28}{Ex 10:21-\allowbreak23 Joe 2:2,\allowbreak31 Lu 23:44,\allowbreak45 2Pe 2:4,\allowbreak17}
\crossref{Ps}{104}{29}{104:78:44 Ex 7:20,\allowbreak21 Isa 50:2 Eze 29:4,\allowbreak5 Re 16:3}
\crossref{Ps}{104}{30}{104:78:45 Ex 8:3-\allowbreak14 Re 16:13,\allowbreak14}
\crossref{Ps}{104}{31}{104:78:45 Ex 8:21-\allowbreak24 Isa 7:18}
\crossref{Ps}{104}{32}{104:78:47,\allowbreak48 Ex 9:18-\allowbreak28 Re 8:7; 11:19; 16:21}
\crossref{Ps}{104}{33}{Re 9:4}
\crossref{Ps}{104}{34}{104:78:46 Ex 10:12-\allowbreak15 Joe 1:4-\allowbreak7; 2:25 Re 9:3-\allowbreak10}
\crossref{Ps}{104}{35}{Ps 18:8; 21:9; 22:26 Ex 10:15}
\crossref{Ps}{105}{1}{105:105:45}
\crossref{Ps}{105}{2}{Ps 40:5; 139:17,\allowbreak18; 145:3-\allowbreak12 Job 5:9; 26:14 Ro 11:33 Eph 1:19; 3:18}
\crossref{Ps}{105}{3}{Ps 1:1-\allowbreak3; 84:11,\allowbreak12; 119:1-\allowbreak3 Mr 3:35 Lu 6:47-\allowbreak49; 11:28 Joh 13:17}
\crossref{Ps}{105}{4}{Ps 25:7; 119:132 Ne 5:19; 13:14,\allowbreak22,\allowbreak31 Lu 23:42}
\crossref{Ps}{105}{5}{105:105:6,\allowbreak43 De 7:6 Joh 15:16 Ac 9:15 Eph 1:4 2Th 2:13 Jas 2:5}
\crossref{Ps}{105}{6}{105:78:8 Le 26:40 Nu 32:14 1Ki 8:47 Ezr 9:6,\allowbreak7 Ne 9:16,\allowbreak32-\allowbreak34}
\crossref{Ps}{105}{7}{De 29:4; 32:28,\allowbreak29 Pr 1:22 Isa 44:18 Mr 4:12; 8:17-\allowbreak21}
\crossref{Ps}{105}{8}{105:143:11 Nu 14:13-\allowbreak16 De 32:26,\allowbreak27 Jos 7:9 Jer 14:7,\allowbreak21}
\crossref{Ps}{105}{9}{105:77:19,\allowbreak20 Isa 63:11-\allowbreak14}
\crossref{Ps}{105}{10}{Ex 14:30; 15:9,\allowbreak10 De 11:4 Ne 9:11}
\crossref{Ps}{105}{11}{105:78:53 Ex 14:13,\allowbreak27,\allowbreak28; 15:5,\allowbreak10,\allowbreak19}
\crossref{Ps}{105}{12}{Ex 14:31; 15:1-\allowbreak21 Lu 8:13 Joh 8:30,\allowbreak31}
\crossref{Ps}{105}{13}{Pr 1:25,\allowbreak30 Isa 48:17,\allowbreak18}
\crossref{Ps}{105}{14}{105:78:18,\allowbreak30 Nu 11:4,\allowbreak33,\allowbreak34 De 9:22 1Co 10:6}
\crossref{Ps}{105}{15}{105:78:29-\allowbreak31 Nu 11:31-\allowbreak34 Isa 10:16; 24:16}
\crossref{Ps}{105}{16}{Nu 16:1,\allowbreak3-\allowbreak50}
\crossref{Ps}{105}{17}{Nu 16:29-\allowbreak33; 26:10 De 11:6}
\crossref{Ps}{105}{18}{Nu 16:35-\allowbreak40,\allowbreak46 Heb 12:29}
\crossref{Ps}{105}{19}{Ex 32:4-\allowbreak8,\allowbreak35 De 9:12-\allowbreak16,\allowbreak21 Ne 9:18 1Co 10:7}
\crossref{Ps}{105}{20}{105:89:17 Jer 2:11 Ro 1:22,\allowbreak23}
\crossref{Ps}{105}{21}{105:13; 78:11,\allowbreak12,\allowbreak42-\allowbreak51 De 32:17,\allowbreak18 Jer 2:32}
\crossref{Ps}{105}{22}{Ex 14:25-\allowbreak28; 15:10}
\crossref{Ps}{105}{23}{Ex 32:10,\allowbreak11,\allowbreak32 De 9:13,\allowbreak14,\allowbreak19,\allowbreak25; 10:10 Eze 20:13,\allowbreak14}
\crossref{Ps}{105}{24}{Ge 25:34 Nu 13:32; 14:31 Mt 22:5 Heb 12:16}
\crossref{Ps}{105}{25}{Nu 14:1-\allowbreak4,\allowbreak27-\allowbreak29 De 1:26,\allowbreak27}
\crossref{Ps}{105}{26}{105:95:11 Nu 14:28-\allowbreak35 De 1:34,\allowbreak35 Heb 3:11,\allowbreak18}
\crossref{Ps}{105}{27}{Ps 44:11 Le 26:33 De 4:26,\allowbreak27; 28:37,\allowbreak64,\allowbreak65; 32:26,\allowbreak27 Eze 20:23}
\crossref{Ps}{105}{28}{Nu 25:1-\allowbreak3,\allowbreak5; 31:16 De 4:3; 32:17 Jos 22:17 Ho 9:10 Re 2:14}
\crossref{Ps}{105}{29}{105:39; 99:8 De 32:16-\allowbreak21 Ec 7:29 Ro 1:21-\allowbreak24}
\crossref{Ps}{105}{30}{Nu 25:6-\allowbreak8,\allowbreak14,\allowbreak15 De 13:9-\allowbreak11,\allowbreak15-\allowbreak17 Jos 7:12 1Ki 18:40,\allowbreak41}
\crossref{Ps}{105}{31}{Nu 25:11-\allowbreak13 De 24:13 Mr 14:3-\allowbreak9}
\crossref{Ps}{105}{32}{105:78:40; 81:7 Nu 20:2,\allowbreak6,\allowbreak13}
\crossref{Ps}{105}{33}{Nu 20:10,\allowbreak11}
\crossref{Ps}{105}{34}{Jos 15:63; 16:10; 17:12-\allowbreak16; 23:12,\allowbreak13 Jud 1:19,\allowbreak21,\allowbreak27-\allowbreak35}
\crossref{Ps}{105}{35}{Jos 15:63 Jud 1:27-\allowbreak36; 2:2,\allowbreak3}
\crossref{Ps}{105}{36}{105:78:58 Ex 34:15,\allowbreak16 Jud 2:12,\allowbreak13,\allowbreak17,\allowbreak19; 3:5-\allowbreak7; 10:6}
\crossref{Ps}{105}{37}{Le 17:7 De 32:17 2Ch 11:15 1Co 10:20 Ro 9:20}
\crossref{Ps}{105}{38}{De 21:9 2Ki 21:16; 24:4 Jer 2:34}
\crossref{Ps}{105}{39}{Isa 24:5,\allowbreak6; 59:3 Eze 20:18,\allowbreak30,\allowbreak31,\allowbreak43}
\crossref{Ps}{105}{40}{105:78:59-\allowbreak62 Jud 2:14,\allowbreak20; 3:8 Ne 9:27-\allowbreak38}
\crossref{Ps}{105}{41}{De 32:30 Jud 2:14; 3:8,\allowbreak12; 4:1,\allowbreak2; 6:1-\allowbreak6; 10:7-\allowbreak18 Ne 9:27-\allowbreak38}
\crossref{Ps}{105}{42}{105:10; 3:7; 6:10; 7:5; 8:2; 9:3}
\crossref{Ps}{105}{43}{Jud 2:16-\allowbreak18 1Sa 12:9-\allowbreak11}
\crossref{Ps}{105}{44}{Jud 2:18; 3:9; 4:3; 6:6-\allowbreak10; 10:10-\allowbreak16 1Sa 7:8-\allowbreak12 2Ki 14:26,\allowbreak27}
\crossref{Ps}{105}{45}{105:105:8 Le 26:40-\allowbreak42 2Ki 13:23 Lu 1:71,\allowbreak72}
\crossref{Ps}{106}{1}{106:106:1; 118:1; 136:1-\allowbreak26 1Ch 16:34,\allowbreak41 2Ch 5:13; 7:3,\allowbreak6; 20:21}
\crossref{Ps}{106}{2}{Ps 31:5; 130:8 Ex 15:16 De 15:15 Isa 35:9; 43:1; 44:22 Lu 1:68}
\crossref{Ps}{106}{3}{106:106:47 Isa 11:11-\allowbreak16; 43:5,\allowbreak6; 49:12 Jer 29:14; 31:8,\allowbreak10 Eze 36:24}
\crossref{Ps}{106}{4}{106:40 Ge 21:14-\allowbreak16 Nu 14:33 De 8:15; 32:10 Job 12:24 Eze 34:6,\allowbreak12}
\crossref{Ps}{106}{5}{Jud 15:18,\allowbreak19 1Sa 30:11,\allowbreak12 Isa 44:12 Jer 14:18 La 2:19}
\crossref{Ps}{106}{6}{106:13,\allowbreak19,\allowbreak28; 50:15; 91:15 Isa 41:17,\allowbreak18 Jer 29:12-\allowbreak14 Ho 5:15}
\crossref{Ps}{106}{7}{106:77:20; 78:52 136:16 Ezr 8:21-\allowbreak23 Isa 30:21; 35:8-\allowbreak10; 48:17}
\crossref{Ps}{106}{8}{106:15,\allowbreak21,\allowbreak31; 81:13-\allowbreak16 De 5:29; 32:29 Isa 48:18}
\crossref{Ps}{106}{9}{Ps 34:10; 132:15; 146:7 Isa 55:1-\allowbreak3 Jer 31:14,\allowbreak25 Mt 5:6 Lu 1:53}
\crossref{Ps}{106}{10}{Job 3:5 Isa 9:2 Mt 4:16; 22:13 Lu 1:79}
\crossref{Ps}{106}{11}{106:68:6,\allowbreak18 106:43 Isa 63:10,\allowbreak11 La 3:39-\allowbreak42; 5:15-\allowbreak17}
\crossref{Ps}{106}{12}{Ex 2:23; 5:18,\allowbreak19 Jud 10:16-\allowbreak18; 16:21,\allowbreak30 Ne 9:37 Isa 51:19,\allowbreak20,\allowbreak23}
\crossref{Ps}{106}{13}{106:6,\allowbreak19,\allowbreak28; 18:6; 116:3-\allowbreak6 Ex 3:7,\allowbreak8 Jud 4:3; 6:6-\allowbreak10; 10:10-\allowbreak18}
\crossref{Ps}{106}{14}{106:10; 68:6 Job 3:5; 10:21,\allowbreak22; 15:22,\allowbreak30; 19:8; 33:30; 42:10-\allowbreak12}
\crossref{Ps}{106}{15}{106:8,\allowbreak21,\allowbreak31; 116:17-\allowbreak19}
\crossref{Ps}{106}{16}{Jud 16:3 Isa 45:1,\allowbreak2 Mic 2:13}
\crossref{Ps}{106}{17}{Ps 38:1-\allowbreak8 Nu 11:33,\allowbreak34; 12:10-\allowbreak13; 21:5-\allowbreak9 Isa 57:17,\allowbreak18 Jer 2:19}
\crossref{Ps}{106}{18}{Job 33:19-\allowbreak22}
\crossref{Ps}{106}{19}{106:6,\allowbreak13,\allowbreak28; 30:8-\allowbreak12; 34:4-\allowbreak6; 78:34,\allowbreak35; 116:4-\allowbreak8 Jer 33:3}
\crossref{Ps}{106}{20}{106:147:15,\allowbreak19 2Ki 20:4,\allowbreak5 Mt 8:8}
\crossref{Ps}{106}{21}{106:8,\allowbreak15,\allowbreak31; 66:5 2Ch 32:25 Lu 17:18}
\crossref{Ps}{106}{22}{Ps 50:14; 116:12,\allowbreak17 Le 7:12 Heb 13:15 1Pe 2:5,\allowbreak9}
\crossref{Ps}{106}{23}{Ps 48:7 Eze 27:26 Ac 27:9-\allowbreak28:31 Re 18:17}
\crossref{Ps}{106}{24}{106:95:5 104:24-\allowbreak27 Job 38:8-\allowbreak11}
\crossref{Ps}{106}{25}{106:135:7; 148:8 Jon 1:4}
\crossref{Ps}{106}{26}{Ps 22:14; 119:28 2Sa 17:10 Isa 13:7 Na 2:10}
\crossref{Ps}{106}{27}{Job 12:25 Isa 19:14; 29:9}
\crossref{Ps}{106}{28}{106:6,\allowbreak13,\allowbreak19 Jon 1:5,\allowbreak6,\allowbreak14 Mt 8:25 Ac 27:23-\allowbreak25}
\crossref{Ps}{106}{29}{Ps 65:7; 89:9 Jon 1:15 Mt 8:26 Mr 4:39-\allowbreak41 Lu 8:23-\allowbreak25}
\crossref{Ps}{106}{30}{Joh 6:21}
\crossref{Ps}{106}{31}{106:8,\allowbreak15,\allowbreak21; 103:2; 105:1 Ho 2:8 Jon 1:16; 2:9 Mic 6:4,\allowbreak5 Ro 1:20,\allowbreak21}
\crossref{Ps}{106}{32}{Ps 18:46; 46:10; 99:5,\allowbreak9 Ex 15:2 Isa 12:4; 25:1}
\crossref{Ps}{106}{33}{1Ki 17:1-\allowbreak7 Isa 13:19-\allowbreak21; 19:5-\allowbreak10; 34:9,\allowbreak10; 42:15; 44:27; 50:2}
\crossref{Ps}{106}{34}{Ge 13:10,\allowbreak13; 19:25 De 29:23-\allowbreak28 Isa 32:13-\allowbreak15}
\crossref{Ps}{106}{35}{106:114:8 Nu 21:16-\allowbreak18 2Ki 3:16-\allowbreak20 Isa 35:6,\allowbreak7; 41:17-\allowbreak19; 44:3-\allowbreak5}
\crossref{Ps}{106}{36}{106:146:7 Lu 1:53}
\crossref{Ps}{106}{37}{Isa 37:30 Jer 29:5; 31:5 Eze 28:26 Am 9:13-\allowbreak15}
\crossref{Ps}{106}{38}{106:128:1-\allowbreak6 Ge 1:28; 9:1; 12:2; 17:16,\allowbreak20 Ex 1:7 De 28:4,\allowbreak11; 30:9}
\crossref{Ps}{106}{39}{Ps 30:6,\allowbreak7 Ge 45:11 Ru 1:20,\allowbreak21 1Sa 2:5-\allowbreak7 2Ki 4:8; 8:3 Job 1:10-\allowbreak17}
\crossref{Ps}{106}{40}{Job 12:21,\allowbreak24 Isa 23:8,\allowbreak9}
\crossref{Ps}{106}{41}{106:113:7,\allowbreak8 Ru 4:14-\allowbreak17 1Sa 2:8 Es 8:15-\allowbreak17 Job 5:11; 8:7; 11:15-\allowbreak19}
\crossref{Ps}{106}{42}{Ps 52:6; 58:10,\allowbreak11 Job 22:19 Isa 66:10,\allowbreak11,\allowbreak14}
\crossref{Ps}{106}{43}{Ps 28:5; 64:9 Isa 5:12 Jer 9:12 Da 10:12 Ho 14:9}
\crossref{Ps}{106}{44}{}
\crossref{Ps}{106}{45}{}
\crossref{Ps}{106}{46}{}
\crossref{Ps}{106}{47}{}
\crossref{Ps}{106}{48}{}
\crossref{Ps}{107}{1}{}
\crossref{Ps}{107}{2}{Ps 33:2; 69:30; 81:2; 92:1-\allowbreak4 Jud 5:12}
\crossref{Ps}{107}{3}{Ps 33:2; 69:30; 81:2; 92:1-\allowbreak4 Jud 5:12}
\crossref{Ps}{107}{4}{Ps 22:22,\allowbreak27; 96:10; 117:1; 138:4,\allowbreak5 Zep 3:14,\allowbreak20}
\crossref{Ps}{107}{5}{Ps 36:5; 85:10; 89:2,\allowbreak5; 103:11 Isa 55:9 Mic 7:18-\allowbreak20 Eph 2:4-\allowbreak7}
\crossref{Ps}{107}{6}{Ps 8:1; 21:13; 57:5,\allowbreak11; 148:13 1Ch 29:10-\allowbreak13}
\crossref{Ps}{107}{7}{Ps 60:5-\allowbreak12 De 33:12 2Sa 12:25 Mt 3:17; 17:5 Ro 1:7 Eph 1:6}
\crossref{Ps}{107}{8}{107:89:35,\allowbreak36 Am 4:2}
\crossref{Ps}{107}{9}{Jos 13:8-\allowbreak11 2Sa 2:8; 5:5}
\crossref{Ps}{107}{10}{Ps 60:8-\allowbreak10 2Sa 8:1,\allowbreak2 Joh 13:8,\allowbreak14}
\crossref{Ps}{107}{11}{Ps 20:6-\allowbreak8; 60:1}
\crossref{Ps}{107}{12}{Ps 44:9 1Sa 29:1-\allowbreak31:13}
\crossref{Ps}{107}{13}{Ps 20:1-\allowbreak9}
\crossref{Ps}{107}{14}{Ps 18:29-\allowbreak34; 118:6-\allowbreak13; 144:1 2Ch 20:12 1Co 15:10 Eph 6:10-\allowbreak18}
\crossref{Ps}{107}{15}{}
\crossref{Ps}{107}{16}{}
\crossref{Ps}{107}{17}{}
\crossref{Ps}{107}{18}{}
\crossref{Ps}{107}{19}{}
\crossref{Ps}{107}{20}{}
\crossref{Ps}{107}{21}{}
\crossref{Ps}{107}{22}{}
\crossref{Ps}{107}{23}{}
\crossref{Ps}{107}{24}{}
\crossref{Ps}{107}{25}{}
\crossref{Ps}{107}{26}{}
\crossref{Ps}{107}{27}{}
\crossref{Ps}{107}{28}{}
\crossref{Ps}{107}{29}{}
\crossref{Ps}{107}{30}{}
\crossref{Ps}{107}{31}{}
\crossref{Ps}{107}{32}{}
\crossref{Ps}{107}{33}{}
\crossref{Ps}{107}{34}{}
\crossref{Ps}{107}{35}{}
\crossref{Ps}{107}{36}{}
\crossref{Ps}{107}{37}{}
\crossref{Ps}{107}{38}{}
\crossref{Ps}{107}{39}{}
\crossref{Ps}{107}{40}{}
\crossref{Ps}{107}{41}{}
\crossref{Ps}{107}{42}{}
\crossref{Ps}{107}{43}{}
\crossref{Ps}{108}{1}{Ps 28:1; 35:22,\allowbreak23; 83:1 Isa 42:14}
\crossref{Ps}{108}{2}{Ps 31:13,\allowbreak18; 64:3,\allowbreak4; 140:3 2Sa 15:3-\allowbreak8; 17:1 Pr 15:28 Mt 26:59-\allowbreak62}
\crossref{Ps}{108}{3}{Ps 17:11; 22:12; 88:17 2Sa 16:7,\allowbreak8 Ho 11:12}
\crossref{Ps}{108}{4}{Ps 35:7,\allowbreak12; 38:20 2Sa 13:39 Joh 10:32 2Co 12:15}
\crossref{Ps}{108}{5}{Ps 35:7-\allowbreak12 Ge 44:4 Pr 17:13}
\crossref{Ps}{108}{6}{Zec 3:1 Joh 13:2,\allowbreak27}
\crossref{Ps}{108}{7}{Ro 3:19 Ga 3:10}
\crossref{Ps}{108}{8}{Ps 55:23 Mt 27:5}
\crossref{Ps}{108}{9}{Ex 22:24 Jer 18:21 La 5:3}
\crossref{Ps}{108}{10}{Ps 37:25 Ge 4:12-\allowbreak14 2Sa 3:29 2Ki 5:27 Job 24:8-\allowbreak12; 30:3-\allowbreak9}
\crossref{Ps}{108}{11}{Job 5:5; 18:9-\allowbreak19; 20:18}
\crossref{Ps}{108}{12}{Isa 27:11 Lu 6:38 Jas 2:13}
\crossref{Ps}{108}{13}{Ps 37:28 1Sa 2:31-\allowbreak33; 3:13 2Ki 10:10,\allowbreak11 Job 18:19 Isa 14:20-\allowbreak22}
\crossref{Ps}{109}{1}{Ps 8:1 Mt 22:42-\allowbreak46 Mr 12:35-\allowbreak37 Lu 22:41}
\crossref{Ps}{109}{2}{Ex 7:19; 8:5 Mic 7:14 Mt 28:18-\allowbreak20 Ac 2:34-\allowbreak37 Ro 1:16}
\crossref{Ps}{109}{3}{Ps 22:27,\allowbreak28 Jud 5:2 Ac 2:41 Ro 11:2-\allowbreak6 2Co 8:1-\allowbreak3,\allowbreak12,\allowbreak16 Php 2:13}
\crossref{Ps}{109}{4}{109:89:34-\allowbreak36 Heb 5:6; 6:13-\allowbreak18; 7:28}
\crossref{Ps}{109}{5}{109:1; 16:8 Mr 16:19 Ac 2:34-\allowbreak36; 7:55,\allowbreak56}
\crossref{Ps}{109}{6}{1Sa 2:10 Isa 2:4; 11:3; 42:1,\allowbreak4; 51:5 Joe 3:12-\allowbreak16 Mic 4:3}
\crossref{Ps}{109}{7}{109:102:9 Jud 7:5,\allowbreak6 Job 21:20 Isa 53:12 Jer 23:15 Mt 20:22; 26:42}
\crossref{Ps}{109}{8}{}
\crossref{Ps}{109}{9}{}
\crossref{Ps}{109}{10}{}
\crossref{Ps}{109}{11}{}
\crossref{Ps}{109}{12}{}
\crossref{Ps}{109}{13}{}
\crossref{Ps}{109}{14}{}
\crossref{Ps}{109}{15}{}
\crossref{Ps}{109}{16}{}
\crossref{Ps}{109}{17}{}
\crossref{Ps}{109}{18}{}
\crossref{Ps}{109}{19}{}
\crossref{Ps}{109}{20}{}
\crossref{Ps}{109}{21}{}
\crossref{Ps}{109}{22}{}
\crossref{Ps}{109}{23}{}
\crossref{Ps}{109}{24}{}
\crossref{Ps}{109}{25}{}
\crossref{Ps}{109}{26}{}
\crossref{Ps}{109}{27}{}
\crossref{Ps}{109}{28}{}
\crossref{Ps}{109}{29}{}
\crossref{Ps}{109}{30}{}
\crossref{Ps}{109}{31}{}
\crossref{Ps}{110}{1}{110:106:1,\allowbreak48}
\crossref{Ps}{110}{2}{110:92:5 104:24 139:14 Job 5:9; 9:10; 26:12-\allowbreak14; 38:1-\allowbreak41; 41:1-\allowbreak34}
\crossref{Ps}{110}{3}{Ps 19:1; 145:4,\allowbreak5,\allowbreak10-\allowbreak12,\allowbreak17 Ex 15:6,\allowbreak7,\allowbreak11 Eph 1:6-\allowbreak8; 3:10 Re 5:12-\allowbreak14}
\crossref{Ps}{110}{4}{110:78:4-\allowbreak8 Ex 12:26,\allowbreak27; 13:14,\allowbreak15 De 4:9; 31:19-\allowbreak30 Jos 4:6,\allowbreak7,\allowbreak21-\allowbreak24}
\crossref{Ps}{110}{5}{Ps 34:9,\allowbreak10; 37:3 Isa 33:16 Mt 6:26-\allowbreak33 Lu 12:30}
\crossref{Ps}{110}{6}{110:78:12-\allowbreak72 105:27-\allowbreak45 De 4:32-\allowbreak38 Jos 3:14-\allowbreak17; 6:20; 10:13,\allowbreak14}
\crossref{Ps}{110}{7}{110:85:10; 89:14 98:3 De 32:4 2Ti 2:13 Re 15:3,\allowbreak4}
\crossref{Ps}{111}{1}{111:111:1; 147:1; 148:11-\allowbreak14; 150:1}
\crossref{Ps}{111}{2}{Ps 25:13; 37:26; 102:28 Ge 17:7; 22:17,\allowbreak18 Pr 20:7 Jer 32:39 Ac 2:39}
\crossref{Ps}{111}{3}{Pr 3:16; 15:6 Isa 33:6 Mt 6:33 2Co 6:10 Php 4:18,\allowbreak19 1Ti 6:6-\allowbreak8}
\crossref{Ps}{111}{4}{Ps 37:6; 97:11 Job 11:17 Isa 50:10; 58:10 Mic 7:8,\allowbreak9 Mal 4:2}
\crossref{Ps}{111}{5}{Pr 2:20; 12:2 Lu 23:50 Ac 11:24 Ro 5:7}
\crossref{Ps}{111}{6}{Ps 15:5; 62:2,\allowbreak6; 125:1 2Pe 1:5-\allowbreak11}
\crossref{Ps}{111}{7}{Ps 27:1-\allowbreak3; 34:4; 56:3,\allowbreak4 Pr 1:33; 3:25,\allowbreak26 Lu 21:9,\allowbreak19}
\crossref{Ps}{111}{8}{Ps 27:14; 31:24 Heb 13:9}
\crossref{Ps}{111}{9}{2Co 9:9}
\crossref{Ps}{111}{10}{Es 6:11,\allowbreak12 Isa 65:13,\allowbreak14 Lu 13:28; 16:23}
\crossref{Ps}{112}{1}{112:112:1}
\crossref{Ps}{112}{2}{Ps 41:13; 106:48 1Ch 16:36; 29:10-\allowbreak13 Da 2:20 Eph 3:21 Re 5:13}
\crossref{Ps}{112}{3}{112:72:11,\allowbreak17-\allowbreak19 86:9 Isa 24:16; 42:10-\allowbreak12; 49:13; 59:19 Hab 2:14}
\crossref{Ps}{112}{4}{112:97:9 99:2 Isa 40:15,\allowbreak17,\allowbreak22}
\crossref{Ps}{112}{5}{112:89:6,\allowbreak8 Ex 15:11 De 33:26 Isa 40:18,\allowbreak25; 16:5 Jer 10:6}
\crossref{Ps}{112}{6}{Ps 11:4 Job 4:18; 15:15 Isa 6:2}
\crossref{Ps}{112}{7}{112:75:6,\allowbreak7 107:41 Job 5:11,\allowbreak15,\allowbreak16 Eze 17:24; 21:26,\allowbreak27 Lu 1:52,\allowbreak53}
\crossref{Ps}{112}{8}{Ps 45:16; 68:13 Ge 41:41 Php 2:8-\allowbreak11 Re 5:9,\allowbreak10}
\crossref{Ps}{112}{9}{112:68:6 Ge 21:5-\allowbreak7; 25:21; 30:22,\allowbreak23 1Sa 2:5 Isa 54:1 Lu 1:13-\allowbreak15}
\crossref{Ps}{112}{10}{}
\crossref{Ps}{113}{1}{Ex 12:41,\allowbreak42; 13:3; 20:2 De 16:1; 26:8 Isa 11:16}
\crossref{Ps}{113}{2}{Ex 6:7; 19:5,\allowbreak6; 25:8; 29:45,\allowbreak46 Le 11:45 De 23:14; 27:9,\allowbreak12}
\crossref{Ps}{113}{3}{113:77:16; 104:7 106:9 Ex 14:21; 15:8 Isa 63:12 Hab 3:8,\allowbreak15}
\crossref{Ps}{113}{4}{Ps 39:6; 68:16 Ex 19:18; 20:18 Jud 5:4,\allowbreak5 Jer 4:23,\allowbreak24 Mic 1:3,\allowbreak4}
\crossref{Ps}{113}{5}{Jer 47:6,\allowbreak7 Hab 3:8}
\crossref{Ps}{113}{6}{113:4; 29:6}
\crossref{Ps}{113}{7}{113:77:18; 97:4,\allowbreak5 104:32 Job 9:6; 26:11 Isa 64:1-\allowbreak3 Jer 5:22}
\crossref{Ps}{113}{8}{113:78:15,\allowbreak16 105:41 107:35 Ex 17:6 Nu 20:11 De 8:15 Ne 9:15}
\crossref{Ps}{113}{9}{113:74:22; 79:9,\allowbreak10 Jos 7:9 Isa 48:11 Eze 20:14; 36:32 Da 9:19}
\crossref{Ps}{114}{1}{Ps 18:1-\allowbreak6; 119:132 Mr 12:33 Joh 21:17 1Jo 4:19; 5:2,\allowbreak3}
\crossref{Ps}{114}{2}{Ps 55:16,\allowbreak17; 86:6,\allowbreak7; 88:1; 145:18,\allowbreak19 Job 27:10 Lu 18:1 Php 4:6}
\crossref{Ps}{114}{3}{Ps 18:4-\allowbreak6; 88:6,\allowbreak7 Jon 2:2,\allowbreak3 Mr 14:33-\allowbreak36 Lu 22:44 Heb 5:7}
\crossref{Ps}{114}{4}{Ps 22:1-\allowbreak3; 30:7,\allowbreak8; 34:6; 50:15; 118:5; 130:1,\allowbreak2 2Ch 33:12,\allowbreak13}
\crossref{Ps}{114}{5}{114:86:5,\allowbreak15 103:8 112:4 115:1 145:8 Ex 34:6,\allowbreak7 Ne 9:17,\allowbreak31 Da 9:9}
\crossref{Ps}{114}{6}{Ps 19:7; 25:21 Isa 35:8 Mt 11:25 Ro 16:19 2Co 1:12; 11:3 Col 3:22}
\crossref{Ps}{114}{7}{114:95:11 Jer 6:16; 30:10 Mt 11:28,\allowbreak29 Heb 4:8-\allowbreak10}
\crossref{Ps}{114}{8}{Ps 56:13; 86:13}
\crossref{Ps}{115}{1}{}
\crossref{Ps}{115}{2}{}
\crossref{Ps}{115}{3}{}
\crossref{Ps}{115}{4}{}
\crossref{Ps}{115}{5}{}
\crossref{Ps}{115}{6}{}
\crossref{Ps}{115}{7}{}
\crossref{Ps}{115}{8}{}
\crossref{Ps}{115}{9}{}
\crossref{Ps}{115}{10}{2Co 4:13 Heb 11:1}
\crossref{Ps}{115}{11}{Ps 31:22 1Sa 27:1}
\crossref{Ps}{115}{12}{Ps 51:12-\allowbreak14; 103:2 Isa 6:5-\allowbreak8 Ro 12:1 1Co 6:20 2Co 5:14,\allowbreak15}
\crossref{Ps}{115}{13}{115:17 Lu 22:17,\allowbreak18,\allowbreak20 1Co 10:16,\allowbreak21; 11:25-\allowbreak27}
\crossref{Ps}{115}{14}{115:18; 22:25; 56:12; 66:13-\allowbreak15 Jon 1:16; 2:9 Na 1:15 Mt 5:33}
\crossref{Ps}{115}{15}{Ps 37:32,\allowbreak33; 72:14 1Sa 25:29 Job 5:26 Lu 16:22 Re 1:18; 14:3}
\crossref{Ps}{115}{16}{115:86:16; 119:125 143:12 Joh 12:26 Ac 27:23 Jas 1:1}
\crossref{Ps}{115}{17}{Ps 50:14; 107:22 Le 7:12 Heb 13:15}
\crossref{Ps}{115}{18}{115:14; 22:25; 76:11 Ec 5:5}
\crossref{Ps}{116}{1}{Ps 66:1,\allowbreak4; 67:3; 86:9 Isa 24:15,\allowbreak16; 42:10-\allowbreak12 Ro 15:11 Re 15:4}
\crossref{Ps}{116}{2}{116:85:10; 89:1 100:4,\allowbreak5 Isa 25:1 Mic 7:20 Lu 1:54,\allowbreak55 Joh 14:6}
\crossref{Ps}{116}{3}{}
\crossref{Ps}{116}{4}{}
\crossref{Ps}{116}{5}{}
\crossref{Ps}{116}{6}{}
\crossref{Ps}{116}{7}{}
\crossref{Ps}{116}{8}{}
\crossref{Ps}{116}{9}{}
\crossref{Ps}{116}{10}{}
\crossref{Ps}{116}{11}{}
\crossref{Ps}{116}{12}{}
\crossref{Ps}{116}{13}{}
\crossref{Ps}{116}{14}{}
\crossref{Ps}{116}{15}{}
\crossref{Ps}{116}{16}{}
\crossref{Ps}{116}{17}{}
\crossref{Ps}{116}{18}{}
\crossref{Ps}{116}{19}{}
\crossref{Ps}{117}{1}{}
\crossref{Ps}{117}{2}{117:115:9-\allowbreak11; 135:19,\allowbreak20; 145:10; 147:19,\allowbreak20 Ga 6:16 Heb 13:15}
\crossref{Ps}{118}{1}{Ps 1:1-\allowbreak3; 32:1,\allowbreak2; 112:1; 128:1 Mt 5:3-\allowbreak12 Lu 11:28 Joh 13:17}
\crossref{Ps}{118}{2}{118:22,\allowbreak146; 25:10; 105:45 De 6:17 1Ki 2:3 Pr 23:26 Eze 36:27}
\crossref{Ps}{118}{3}{1Jo 3:9; 5:18}
\crossref{Ps}{118}{4}{De 4:1,\allowbreak9; 5:29-\allowbreak33; 6:17; 11:13,\allowbreak22; 12:32; 28:1-\allowbreak14; 30:16 Jos 1:7}
\crossref{Ps}{118}{5}{118:32,\allowbreak36,\allowbreak44,\allowbreak45,\allowbreak131,\allowbreak159,\allowbreak173; 51:10 Jer 31:33 Ro 7:22-\allowbreak24 2Th 3:5}
\crossref{Ps}{118}{6}{118:31,\allowbreak80 Job 22:26 Da 12:2,\allowbreak3 1Jo 2:28; 3:20,\allowbreak21}
\crossref{Ps}{118}{7}{118:171; 9:1; 86:12,\allowbreak13 1Ch 29:13-\allowbreak17}
\crossref{Ps}{118}{8}{118:16,\allowbreak106,\allowbreak115 Jos 24:15}
\crossref{Ps}{118}{9}{Ps 25:7; 34:11 Job 1:5; 13:26 Pr 1:4,\allowbreak10; 4:1,\allowbreak10-\allowbreak17; 5:7-\allowbreak23; 6:20-\allowbreak35}
\crossref{Ps}{118}{10}{118:2,\allowbreak34,\allowbreak58,\allowbreak69; 78:37 1Sa 7:3 2Ch 15:15 Jer 3:10 Ho 10:2}
\crossref{Ps}{118}{11}{118:97; 12 37:31 40:8 Job 22:22 Pr 2:1,\allowbreak10,\allowbreak11 Isa 51:7 Jer 15:16}
\crossref{Ps}{118}{12}{1Ti 1:11; 6:15}
\crossref{Ps}{118}{13}{118:46,\allowbreak172; 34:11; 37:30; 40:9,\allowbreak10; 71:15-\allowbreak18; 118:17 Mt 10:27; 12:34}
\crossref{Ps}{118}{14}{118:47,\allowbreak72,\allowbreak77,\allowbreak111,\allowbreak127,\allowbreak162; 19:9,\allowbreak10; 112:1 Job 23:12 Jer 15:16}
\crossref{Ps}{118}{15}{118:23,\allowbreak48,\allowbreak78,\allowbreak97,\allowbreak131,\allowbreak148; 12 Jas 1:25}
\crossref{Ps}{118}{16}{118:14,\allowbreak24,\allowbreak35,\allowbreak47,\allowbreak70,\allowbreak77,\allowbreak92; 40:8 Ro 7:22 Heb 10:16,\allowbreak17}
\crossref{Ps}{118}{17}{Ro 8:2-\allowbreak4 Eph 2:4,\allowbreak5,\allowbreak10 Tit 2:11,\allowbreak12 1Jo 2:29; 5:3,\allowbreak4}
\crossref{Ps}{118}{18}{Isa 29:10-\allowbreak12,\allowbreak18; 32:3; 35:5 Mt 13:13; 16:17 Joh 9:39 Ac 26:18}
\crossref{Ps}{118}{19}{Ps 39:12 Ge 47:9 1Ch 29:15 2Co 5:6 Heb 11:13-\allowbreak16 1Pe 2:11}
\crossref{Ps}{118}{20}{118:40,\allowbreak131,\allowbreak174; 42:1; 63:1; 84:2 Pr 13:12 So 5:8 Re 3:15,\allowbreak16}
\crossref{Ps}{118}{21}{118:78; 138:6 Ex 10:3; 18:11 Job 40:11,\allowbreak12 Isa 2:11,\allowbreak12; 10:12}
\crossref{Ps}{118}{22}{118:39,\allowbreak42; 39:8; 42:10; 68:9-\allowbreak11,\allowbreak19,\allowbreak20; 123:3,\allowbreak4 1Sa 25:10,\allowbreak39}
\crossref{Ps}{118}{23}{Ps 2:1,\allowbreak2 1Sa 20:31; 22:7-\allowbreak13 Lu 22:66; 23:1,\allowbreak2,\allowbreak10,\allowbreak11}
\crossref{Ps}{118}{24}{118:16,\allowbreak77,\allowbreak92,\allowbreak143,\allowbreak162 Job 27:10 Jer 6:10}
\crossref{Ps}{118}{25}{Ps 22:15; 44:25 Isa 65:25 Mt 16:23 Ro 7:22-\allowbreak24 Php 3:19 Col 3:2}
\crossref{Ps}{118}{26}{118:106; 32:5; 38:18; 51:1-\allowbreak19 Pr 28:13}
\crossref{Ps}{118}{27}{118:71:17; 78:4 105:2 111:4 145:5,\allowbreak6 Ex 13:14,\allowbreak15 Jos 4:6,\allowbreak7 Ac 2:11}
\crossref{Ps}{118}{28}{Ps 22:14; 107:26 Jos 2:11,\allowbreak24}
\crossref{Ps}{118}{29}{118:37,\allowbreak104,\allowbreak128,\allowbreak163; 141:3,\allowbreak4 Pr 30:8 Isa 44:20 Jer 16:19 Jon 2:8}
\crossref{Ps}{119}{1}{}
\crossref{Ps}{119}{2}{Ps 35:11; 52:2-\allowbreak4; 109:1,\allowbreak2; 140:1-\allowbreak3 Mt 26:59-\allowbreak62}
\crossref{Ps}{119}{3}{}
\crossref{Ps}{119}{4}{Ps 7:13; 52:5; 140:9,\allowbreak11 De 32:23,\allowbreak24 Pr 12:22; 19:5,\allowbreak9 Re 21:8}
\crossref{Ps}{119}{5}{Jer 9:2,\allowbreak3,\allowbreak6; 15:10 Mic 7:1,\allowbreak2 2Pe 2:7,\allowbreak8 Re 2:13}
\crossref{Ps}{119}{6}{Ps 57:4 1Sa 20:30-\allowbreak33 Eze 2:6 Mt 10:16,\allowbreak36 Tit 3:3}
\crossref{Ps}{119}{7}{Ps 34:14; 35:20; 55:20 2Sa 20:19 Mt 5:9 Ro 12:18 Eph 2:14-\allowbreak17}
\crossref{Ps}{119}{8}{}
\crossref{Ps}{119}{9}{}
\crossref{Ps}{119}{10}{}
\crossref{Ps}{119}{11}{}
\crossref{Ps}{119}{12}{}
\crossref{Ps}{119}{13}{}
\crossref{Ps}{119}{14}{}
\crossref{Ps}{119}{15}{}
\crossref{Ps}{119}{16}{}
\crossref{Ps}{119}{17}{}
\crossref{Ps}{119}{18}{}
\crossref{Ps}{119}{19}{}
\crossref{Ps}{119}{20}{}
\crossref{Ps}{119}{21}{}
\crossref{Ps}{119}{22}{}
\crossref{Ps}{119}{23}{}
\crossref{Ps}{119}{24}{}
\crossref{Ps}{119}{25}{}
\crossref{Ps}{119}{26}{}
\crossref{Ps}{119}{27}{}
\crossref{Ps}{119}{28}{}
\crossref{Ps}{119}{29}{}
\crossref{Ps}{119}{30}{}
\crossref{Ps}{119}{31}{}
\crossref{Ps}{119}{32}{}
\crossref{Ps}{119}{33}{}
\crossref{Ps}{119}{34}{}
\crossref{Ps}{119}{35}{}
\crossref{Ps}{119}{36}{}
\crossref{Ps}{119}{37}{}
\crossref{Ps}{119}{38}{}
\crossref{Ps}{119}{39}{}
\crossref{Ps}{119}{40}{}
\crossref{Ps}{119}{41}{}
\crossref{Ps}{119}{42}{}
\crossref{Ps}{119}{43}{}
\crossref{Ps}{119}{44}{}
\crossref{Ps}{119}{45}{}
\crossref{Ps}{119}{46}{}
\crossref{Ps}{119}{47}{}
\crossref{Ps}{119}{48}{}
\crossref{Ps}{119}{49}{}
\crossref{Ps}{119}{50}{}
\crossref{Ps}{119}{51}{}
\crossref{Ps}{119}{52}{}
\crossref{Ps}{119}{53}{}
\crossref{Ps}{119}{54}{}
\crossref{Ps}{119}{55}{}
\crossref{Ps}{119}{56}{}
\crossref{Ps}{119}{57}{}
\crossref{Ps}{119}{58}{}
\crossref{Ps}{119}{59}{}
\crossref{Ps}{119}{60}{}
\crossref{Ps}{119}{61}{}
\crossref{Ps}{119}{62}{}
\crossref{Ps}{119}{63}{}
\crossref{Ps}{119}{64}{}
\crossref{Ps}{119}{65}{}
\crossref{Ps}{119}{66}{}
\crossref{Ps}{119}{67}{}
\crossref{Ps}{119}{68}{}
\crossref{Ps}{119}{69}{}
\crossref{Ps}{119}{70}{}
\crossref{Ps}{119}{71}{}
\crossref{Ps}{119}{72}{}
\crossref{Ps}{119}{73}{}
\crossref{Ps}{119}{74}{}
\crossref{Ps}{119}{75}{}
\crossref{Ps}{119}{76}{}
\crossref{Ps}{119}{77}{}
\crossref{Ps}{119}{78}{}
\crossref{Ps}{119}{79}{}
\crossref{Ps}{119}{80}{}
\crossref{Ps}{119}{81}{}
\crossref{Ps}{119}{82}{}
\crossref{Ps}{119}{83}{}
\crossref{Ps}{119}{84}{}
\crossref{Ps}{119}{85}{}
\crossref{Ps}{119}{86}{}
\crossref{Ps}{119}{87}{}
\crossref{Ps}{119}{88}{}
\crossref{Ps}{119}{89}{}
\crossref{Ps}{119}{90}{}
\crossref{Ps}{119}{91}{}
\crossref{Ps}{119}{92}{}
\crossref{Ps}{119}{93}{}
\crossref{Ps}{119}{94}{}
\crossref{Ps}{119}{95}{}
\crossref{Ps}{119}{96}{}
\crossref{Ps}{119}{97}{}
\crossref{Ps}{119}{98}{}
\crossref{Ps}{119}{99}{}
\crossref{Ps}{119}{100}{}
\crossref{Ps}{119}{101}{}
\crossref{Ps}{119}{102}{}
\crossref{Ps}{119}{103}{}
\crossref{Ps}{119}{104}{}
\crossref{Ps}{119}{105}{}
\crossref{Ps}{119}{106}{}
\crossref{Ps}{119}{107}{}
\crossref{Ps}{119}{108}{}
\crossref{Ps}{119}{109}{}
\crossref{Ps}{119}{110}{}
\crossref{Ps}{119}{111}{}
\crossref{Ps}{119}{112}{}
\crossref{Ps}{119}{113}{}
\crossref{Ps}{119}{114}{}
\crossref{Ps}{119}{115}{}
\crossref{Ps}{119}{116}{}
\crossref{Ps}{119}{117}{}
\crossref{Ps}{119}{118}{}
\crossref{Ps}{119}{119}{}
\crossref{Ps}{119}{120}{}
\crossref{Ps}{119}{121}{}
\crossref{Ps}{119}{122}{}
\crossref{Ps}{119}{123}{}
\crossref{Ps}{119}{124}{}
\crossref{Ps}{119}{125}{}
\crossref{Ps}{119}{126}{}
\crossref{Ps}{119}{127}{}
\crossref{Ps}{119}{128}{}
\crossref{Ps}{119}{129}{}
\crossref{Ps}{119}{130}{}
\crossref{Ps}{119}{131}{}
\crossref{Ps}{119}{132}{}
\crossref{Ps}{119}{133}{}
\crossref{Ps}{119}{134}{}
\crossref{Ps}{119}{135}{}
\crossref{Ps}{119}{136}{}
\crossref{Ps}{119}{137}{}
\crossref{Ps}{119}{138}{}
\crossref{Ps}{119}{139}{}
\crossref{Ps}{119}{140}{}
\crossref{Ps}{119}{141}{}
\crossref{Ps}{119}{142}{}
\crossref{Ps}{119}{143}{}
\crossref{Ps}{119}{144}{}
\crossref{Ps}{119}{145}{}
\crossref{Ps}{119}{146}{}
\crossref{Ps}{119}{147}{}
\crossref{Ps}{119}{148}{}
\crossref{Ps}{119}{149}{}
\crossref{Ps}{119}{150}{}
\crossref{Ps}{119}{151}{}
\crossref{Ps}{119}{152}{}
\crossref{Ps}{119}{153}{}
\crossref{Ps}{119}{154}{}
\crossref{Ps}{119}{155}{}
\crossref{Ps}{119}{156}{}
\crossref{Ps}{119}{157}{}
\crossref{Ps}{119}{158}{}
\crossref{Ps}{119}{159}{}
\crossref{Ps}{119}{160}{}
\crossref{Ps}{119}{161}{}
\crossref{Ps}{119}{162}{}
\crossref{Ps}{119}{163}{}
\crossref{Ps}{119}{164}{}
\crossref{Ps}{119}{165}{}
\crossref{Ps}{119}{166}{}
\crossref{Ps}{119}{167}{}
\crossref{Ps}{119}{168}{}
\crossref{Ps}{119}{169}{}
\crossref{Ps}{119}{170}{}
\crossref{Ps}{119}{171}{}
\crossref{Ps}{119}{172}{}
\crossref{Ps}{119}{173}{}
\crossref{Ps}{119}{174}{}
\crossref{Ps}{119}{175}{}
\crossref{Ps}{119}{176}{}
\crossref{Ps}{120}{1}{}
\crossref{Ps}{120}{2}{Ps 46:1; 124:8; 146:5,\allowbreak6 Isa 40:28,\allowbreak29; 41:13 Jer 20:11 Ho 13:9}
\crossref{Ps}{120}{3}{120:91:12 1Sa 2:9 Pr 2:8; 3:23,\allowbreak26 1Pe 1:5}
\crossref{Ps}{120}{4}{Ps 27:1; 32:7,\allowbreak8; 127:1 Isa 27:3}
\crossref{Ps}{120}{5}{Ex 13:21 Isa 4:5,\allowbreak6; 25:4; 32:2 Mt 23:37}
\crossref{Ps}{120}{6}{120:91:5-\allowbreak10 Isa 49:10 Re 7:16}
\crossref{Ps}{120}{7}{120:91:9,\allowbreak10 Job 5:19-\allowbreak27 Pr 12:21 Mt 6:13 Ro 8:28,\allowbreak35-\allowbreak39 2Ti 4:18}
\crossref{Ps}{121}{1}{}
\crossref{Ps}{121}{2}{121:84:7 87:1-\allowbreak3 100:4 Ex 20:24 2Ch 6:6}
\crossref{Ps}{121}{3}{2Sa 5:9 Eph 2:20,\allowbreak21; 4:4-\allowbreak7 Re 21:10-\allowbreak27}
\crossref{Ps}{121}{4}{121:78:68; 132:13 Ex 23:17; 34:23,\allowbreak24 De 12:5,\allowbreak11; 16:16}
\crossref{Ps}{121}{5}{De 17:8,\allowbreak18 2Ch 19:8}
\crossref{Ps}{121}{6}{Ps 51:18; 137:6,\allowbreak7 Jer 51:50 Joh 17:21 Eph 4:3 2Th 3:16}
\crossref{Ps}{121}{7}{1Ch 12:18 Isa 9:7; 54:13 Joh 14:27 Jas 3:18}
\crossref{Ps}{121}{8}{Ps 16:3; 42:4; 119:63 Eph 4:4-\allowbreak6 Php 2:2-\allowbreak5 Jas 3:13-\allowbreak18}
\crossref{Ps}{122}{1}{}
\crossref{Ps}{122}{2}{Jos 9:23,\allowbreak27; 10:6}
\crossref{Ps}{122}{3}{Ps 56:1,\allowbreak2; 57:1; 69:13-\allowbreak16 Lu 18:11-\allowbreak13}
\crossref{Ps}{122}{4}{122:73:5-\allowbreak9 119:51 Job 12:5; 16:4 Jer 48:11,\allowbreak27,\allowbreak29 Ac 17:21,\allowbreak32}
\crossref{Ps}{122}{5}{}
\crossref{Ps}{122}{6}{}
\crossref{Ps}{122}{7}{}
\crossref{Ps}{122}{8}{}
\crossref{Ps}{122}{9}{}
\crossref{Ps}{123}{1}{}
\crossref{Ps}{123}{2}{Ps 21:1,\allowbreak2; 3:1; 22:12,\allowbreak13,\allowbreak16; 37:32 Nu 16:2,\allowbreak3}
\crossref{Ps}{123}{3}{Ps 27:2; 35:25; 56:1,\allowbreak2; 57:3; 74:8; 83:4 Es 3:6,\allowbreak12,\allowbreak13}
\crossref{Ps}{123}{4}{Ps 18:4; 42:7; 69:15 Isa 8:7,\allowbreak8; 28:2; 59:19 Jer 46:7,\allowbreak8 Da 9:26}
\crossref{Ps}{124}{1}{}
\crossref{Ps}{124}{2}{La 4:12}
\crossref{Ps}{124}{3}{124:103:9,\allowbreak14 Pr 22:8 Isa 10:5; 14:5,\allowbreak6; 27:8 1Co 10:13 Re 2:10}
\crossref{Ps}{124}{4}{Ps 41:1-\allowbreak3; 51:18; 73:1 Isa 58:10,\allowbreak11 Heb 6:10 1Jo 3:17-\allowbreak24}
\crossref{Ps}{124}{5}{Ps 40:4; 101:3 1Ch 10:13,\allowbreak14 Pr 14:14 Jer 2:19 Zep 1:6 Heb 10:38}
\crossref{Ps}{124}{6}{}
\crossref{Ps}{124}{7}{}
\crossref{Ps}{124}{8}{}
\crossref{Ps}{125}{1}{125:120:1; 121:1; 122:1; 123:1; 124:1; 125:1}
\crossref{Ps}{125}{2}{Ps 14:7; 53:6; 106:47,\allowbreak48 Ezr 3:11 Job 8:21 Isa 35:10; 49:9-\allowbreak13}
\crossref{Ps}{125}{3}{Ps 18:50; 31:19; 66:5,\allowbreak6; 68:7,\allowbreak8,\allowbreak22 Ezr 7:27,\allowbreak28 Isa 11:11-\allowbreak16; 12:4-\allowbreak6}
\crossref{Ps}{125}{4}{125:1; 85:4 Ho 1:11}
\crossref{Ps}{125}{5}{125:137:1 Isa 12:1-\allowbreak3 Jer 31:9-\allowbreak13 Joe 2:17,\allowbreak23 Mt 5:4 Joh 16:20-\allowbreak22}
\crossref{Ps}{126}{1}{}
\crossref{Ps}{126}{2}{Ps 39:5,\allowbreak6 Ec 1:14; 2:1-\allowbreak11,\allowbreak20-\allowbreak23; 4:8}
\crossref{Ps}{126}{3}{126:128:3,\allowbreak4 Ge 1:28; 15:4,\allowbreak5; 24:60; 30:1,\allowbreak2; 33:5; 41:51,\allowbreak52; 48:4}
\crossref{Ps}{126}{4}{Jer 50:9}
\crossref{Ps}{126}{5}{Ge 50:23 Job 1:2; 42:12-\allowbreak16}
\crossref{Ps}{126}{6}{}
\crossref{Ps}{127}{1}{}
\crossref{Ps}{127}{2}{Ge 3:19 De 28:4,\allowbreak11,\allowbreak39,\allowbreak51 Jud 6:3-\allowbreak6 Ec 5:18,\allowbreak19 Isa 62:8}
\crossref{Ps}{127}{3}{Ge 49:22 Pr 5:15-\allowbreak18 Eze 19:10}
\crossref{Ps}{127}{4}{127:112:2 Jud 5:24}
\crossref{Ps}{127}{5}{Ps 20:2; 118:26; 134:3 Isa 2:3 Eph 1:3}
\crossref{Ps}{128}{1}{}
\crossref{Ps}{128}{2}{Ps 34:19; 118:13; 125:1 Job 5:19 Mt 16:18 Ro 8:35-\allowbreak39 Joh 16:33}
\crossref{Ps}{128}{3}{128:141:7 Isa 51:23}
\crossref{Ps}{128}{4}{Ezr 9:15 Ne 9:33 La 1:18; 3:22 Da 9:7}
\crossref{Ps}{128}{5}{128:83:4-\allowbreak11 122:6 Es 6:13; 9:5 Isa 10:12; 37:22,\allowbreak28,\allowbreak29,\allowbreak35}
\crossref{Ps}{128}{6}{Ps 37:2; 92:7 Jer 17:5,\allowbreak6 Mt 13:6}
\crossref{Ps}{129}{1}{}
\crossref{Ps}{129}{2}{Ps 5:1,\allowbreak2; 17:1; 55:1,\allowbreak2; 61:1,\allowbreak2 2Ch 6:40 Ne 1:6,\allowbreak11 Isa 37:17}
\crossref{Ps}{129}{3}{129:143:2 Job 9:2,\allowbreak3,\allowbreak20; 10:14; 15:14 Isa 53:6 Joh 8:7-\allowbreak9 Ro 3:20-\allowbreak24}
\crossref{Ps}{129}{4}{Ps 25:11; 86:5; 103:2,\allowbreak3 Ex 34:5-\allowbreak7 Isa 1:18; 55:7 Jer 31:34 Da 9:9}
\crossref{Ps}{129}{5}{Ps 27:14; 33:20; 40:1; 62:1,\allowbreak5 Ge 49:18 Isa 8:17; 26:8; 30:18}
\crossref{Ps}{129}{6}{Ps 63:6; 119:147 Ac 27:29}
\crossref{Ps}{129}{7}{Ps 40:3; 71:5; 115:9-\allowbreak13; 131:1,\allowbreak3 Zep 3:12}
\crossref{Ps}{129}{8}{129:103:3,\allowbreak4 Mt 1:21 Ro 6:14 Tit 2:14 1Jo 3:5-\allowbreak8}
\crossref{Ps}{130}{1}{}
\crossref{Ps}{130}{2}{Ps 42:5,\allowbreak11; 43:5; 62:1}
\crossref{Ps}{130}{3}{130:115:9-\allowbreak11; 130:7; 146:5 Jer 17:7,\allowbreak8}
\crossref{Ps}{130}{4}{}
\crossref{Ps}{130}{5}{}
\crossref{Ps}{130}{6}{}
\crossref{Ps}{130}{7}{}
\crossref{Ps}{130}{8}{}
\crossref{Ps}{131}{1}{}
\crossref{Ps}{131}{2}{Ps 56:12; 65:1; 66:13,\allowbreak14; 116:14-\allowbreak18; 119:106 2Sa 7:1}
\crossref{Ps}{131}{3}{Ec 9:10 Hag 1:4 Mt 6:33}
\crossref{Ps}{132}{1}{132:122:1; 124:1; 131:1}
\crossref{Ps}{132}{2}{132:141:5 Pr 27:9 So 1:3 Joh 12:3}
\crossref{Ps}{132}{3}{Ps 42:8 Le 25:21 De 28:8}
\crossref{Ps}{132}{4}{}
\crossref{Ps}{132}{5}{}
\crossref{Ps}{132}{6}{}
\crossref{Ps}{132}{7}{}
\crossref{Ps}{132}{8}{}
\crossref{Ps}{132}{9}{}
\crossref{Ps}{132}{10}{}
\crossref{Ps}{132}{11}{}
\crossref{Ps}{132}{12}{}
\crossref{Ps}{132}{13}{}
\crossref{Ps}{132}{14}{}
\crossref{Ps}{132}{15}{}
\crossref{Ps}{132}{16}{}
\crossref{Ps}{132}{17}{}
\crossref{Ps}{132}{18}{}
\crossref{Ps}{133}{1}{}
\crossref{Ps}{133}{2}{Ps 28:2; 63:4; 141:2 La 2:19; 3:41}
\crossref{Ps}{133}{3}{133:124:8; 146:5,\allowbreak6}
\crossref{Ps}{134}{1}{Ps 33:1,\allowbreak2; 96:1-\allowbreak4; 106:1; 107:8,\allowbreak15; 111:1; 112:1; 113:1; 117:1,\allowbreak2; 150:6}
\crossref{Ps}{134}{2}{1Ch 16:37-\allowbreak42; 23:30 Ne 9:5 Lu 2:37}
\crossref{Ps}{134}{3}{134:106:1; 107:1; 118:1; 119:68; 136:1; 145:7,\allowbreak8 Mt 19:17}
\crossref{Ps}{135}{1}{135:105:1; 106:1; 107:1; 118:1; 119:68 2Ch 7:3,\allowbreak6 Ezr 3:11 Jer 33:11}
\crossref{Ps}{135}{2}{135:82:1 97:7,\allowbreak9 Ex 18:11 De 10:17 Jos 22:22 2Ch 2:5 Da 2:47}
\crossref{Ps}{135}{3}{1Ti 6:15 Re 17:14; 19:16}
\crossref{Ps}{135}{4}{135:72:18; 86:10 Ex 15:11 Job 5:9 Re 15:3}
\crossref{Ps}{135}{5}{}
\crossref{Ps}{135}{6}{Ps 24:2; 104:2,\allowbreak3 Ge 1:9 Job 26:7; 37:18 Isa 40:22; 44:24 Jer 10:12}
\crossref{Ps}{135}{7}{135:74:16,\allowbreak17 104:19 Ge 1:14-\allowbreak19 De 4:19}
\crossref{Ps}{135}{8}{135:148:3 Jer 31:35 Mt 5:45}
\crossref{Ps}{135}{9}{}
\crossref{Ps}{135}{10}{135:78:51; 105:36 135:8 Ex 11:5,\allowbreak6; 12:12,\allowbreak29 Heb 11:28}
\crossref{Ps}{135}{11}{135:78:52; 105:37 Ex 12:51; 13:3,\allowbreak17 1Sa 12:6-\allowbreak8}
\crossref{Ps}{135}{12}{Ex 6:6; 13:14; 15:6 De 11:2-\allowbreak4 Isa 51:9,\allowbreak10 Jer 32:21 Ac 7:36}
\crossref{Ps}{135}{13}{Ps 66:5,\allowbreak6; 74:13; 78:13; 106:9-\allowbreak11 Ex 14:21,\allowbreak22,\allowbreak29 Isa 63:12,\allowbreak13}
\crossref{Ps}{135}{14}{135:78:13; 106:9 Ex 14:22}
\crossref{Ps}{135}{15}{135:78:53; 135:9 Ex 14:27,\allowbreak28; 15:4,\allowbreak5,\allowbreak10,\allowbreak11 Ne 9:10,\allowbreak11}
\crossref{Ps}{135}{16}{135:77:20 Ex 13:18; 15:22 Nu 9:17-\allowbreak22 De 8:2,\allowbreak15 Ne 9:12,\allowbreak19}
\crossref{Ps}{135}{17}{135:135:10,\allowbreak11 Jos 12:1-\allowbreak24}
\crossref{Ps}{135}{18}{Ps 10:8; 59:11; 78:31; 94:6}
\crossref{Ps}{135}{19}{Nu 21:21,\allowbreak23 De 2:30-\allowbreak36; 29:7}
\crossref{Ps}{135}{20}{Nu 21:33 De 3:1-\allowbreak29}
\crossref{Ps}{135}{21}{Ps 44:2,\allowbreak3; 78:55; 105:44; 135:12 Nu 32:33-\allowbreak42 De 3:12-\allowbreak17 Jos 12:1-\allowbreak7}
\crossref{Ps}{136}{1}{Ge 2:10-\allowbreak14 Ezr 8:21,\allowbreak31 Eze 1:1}
\crossref{Ps}{136}{2}{}
\crossref{Ps}{136}{3}{136:123:3,\allowbreak4 La 2:15,\allowbreak16}
\crossref{Ps}{136}{4}{Ec 3:4 Isa 22:12 La 5:14,\allowbreak15 Ho 9:4 Am 8:3}
\crossref{Ps}{136}{5}{136:84:1,\allowbreak2,\allowbreak10 102:13,\allowbreak14 122:5-\allowbreak9 Ne 1:2-\allowbreak4; 2:2,\allowbreak3 Isa 62:1,\allowbreak6,\allowbreak7}
\crossref{Ps}{136}{6}{Ps 22:15 Isa 41:17 La 4:4 Eze 3:26}
\crossref{Ps}{136}{7}{136:74:18; 79:8-\allowbreak12 Ex 17:14 1Sa 15:2 Ho 7:2}
\crossref{Ps}{136}{8}{Isa 47:1-\allowbreak5 Jer 50:42; 51:33 Zec 2:7}
\crossref{Ps}{136}{9}{Isa 13:16 Ho 10:14; 13:16}
\crossref{Ps}{136}{10}{}
\crossref{Ps}{136}{11}{}
\crossref{Ps}{136}{12}{}
\crossref{Ps}{136}{13}{}
\crossref{Ps}{136}{14}{}
\crossref{Ps}{136}{15}{}
\crossref{Ps}{136}{16}{}
\crossref{Ps}{136}{17}{}
\crossref{Ps}{136}{18}{}
\crossref{Ps}{136}{19}{}
\crossref{Ps}{136}{20}{}
\crossref{Ps}{136}{21}{}
\crossref{Ps}{136}{22}{}
\crossref{Ps}{136}{23}{}
\crossref{Ps}{136}{24}{}
\crossref{Ps}{136}{25}{}
\crossref{Ps}{136}{26}{}
\crossref{Ps}{137}{1}{Ps 9:1; 86:12,\allowbreak13; 103:1,\allowbreak2; 111:1 1Co 14:15 Eph 5:19}
\crossref{Ps}{137}{2}{Ps 5:7; 28:2; 99:5,\allowbreak9 1Ki 8:29,\allowbreak30 Da 6:10}
\crossref{Ps}{137}{3}{Ps 18:6; 34:4-\allowbreak6; 77:1,\allowbreak2 Isa 65:24}
\crossref{Ps}{137}{4}{137:72:11; 102:15,\allowbreak22 Isa 49:23; 60:3-\allowbreak5,\allowbreak16 Re 11:15; 21:24}
\crossref{Ps}{137}{5}{Isa 52:7-\allowbreak10; 65:14; 66:10-\allowbreak14 Jer 31:11,\allowbreak12 Zep 3:14,\allowbreak15 Mt 21:5-\allowbreak9}
\crossref{Ps}{137}{6}{Ps 51:17; 113:5,\allowbreak6 1Sa 2:7,\allowbreak8 Pr 3:34 Isa 57:15; 66:2 Lu 1:51-\allowbreak53}
\crossref{Ps}{137}{7}{Ps 23:3,\allowbreak4; 41:7,\allowbreak8; 66:10-\allowbreak12 Job 13:15; 19:25,\allowbreak26 Isa 57:16}
\crossref{Ps}{137}{8}{Ps 57:2 Isa 26:12 Jer 32:39,\allowbreak40 Joh 15:2 Ro 5:10; 8:28-\allowbreak30 Php 1:6}
\crossref{Ps}{137}{9}{}
\crossref{Ps}{138}{1}{138:23; 11:4,\allowbreak5; 17:3; 44:21 1Ki 8:39 1Ch 28:9 Jer 12:3; 17:9,\allowbreak10}
\crossref{Ps}{138}{2}{Ps 56:8 Ge 16:13 2Ki 6:12; 19:27 Pr 15:3 Isa 37:28 Zec 4:10}
\crossref{Ps}{138}{3}{Job 13:26,\allowbreak27; 14:16,\allowbreak17; 31:4 Mt 3:12}
\crossref{Ps}{138}{4}{Ps 19:14 Job 8:2; 38:2; 42:3,\allowbreak6-\allowbreak8 Zep 1:12 Mal 3:13-\allowbreak16 Mt 12:35-\allowbreak37}
\crossref{Ps}{138}{5}{De 33:27 Job 23:8,\allowbreak9}
\crossref{Ps}{138}{6}{Ps 40:5; 13:1 Job 11:7-\allowbreak9; 26:14; 42:3 Pr 30:2-\allowbreak4 Ro 11:33}
\crossref{Ps}{138}{7}{Jer 23:23,\allowbreak24 Jon 1:3,\allowbreak10 Ac 5:9}
\crossref{Ps}{138}{8}{Eze 28:12-\allowbreak17 Am 9:2-\allowbreak4 Ob 1:4}
\crossref{Ps}{139}{1}{Ps 43:1; 59:1-\allowbreak3; 71:4}
\crossref{Ps}{139}{2}{Ps 43:1; 59:1-\allowbreak3; 71:4}
\crossref{Ps}{139}{3}{Ps 2:1,\allowbreak2; 21:11; 36:4; 38:12; 62:3; 64:5,\allowbreak6 Pr 12:20 Ho 7:6 Mic 2:1-\allowbreak3}
\crossref{Ps}{139}{4}{Ps 52:2,\allowbreak3; 57:4; 59:7; 64:3,\allowbreak4 Pr 12:18 Isa 59:3-\allowbreak5,\allowbreak13 Jer 9:3,\allowbreak5}
\crossref{Ps}{139}{5}{Ps 17:8,\allowbreak9; 36:11; 37:32,\allowbreak33-\allowbreak40; 55:1-\allowbreak3; 71:4}
\crossref{Ps}{139}{6}{Ps 10:4-\allowbreak12; 17:8-\allowbreak13; 35:7; 36:11; 57:6; 119:69,\allowbreak85,\allowbreak110; 123:3,\allowbreak4}
\crossref{Ps}{139}{7}{Ps 16:2,\allowbreak5,\allowbreak6; 31:14; 91:2; 119:57; 142:5 La 3:24 Zec 13:9}
\crossref{Ps}{139}{8}{Ps 18:1,\allowbreak2,\allowbreak35; 27:1; 28:7,\allowbreak8; 59:17; 62:2,\allowbreak7; 89:26; 95:1 De 33:27-\allowbreak29}
\crossref{Ps}{139}{9}{Ps 27:12; 94:20,\allowbreak21 2Sa 15:31 Job 5:12,\allowbreak13}
\crossref{Ps}{139}{10}{Ps 7:16; 64:8; 94:23 Es 5:14; 7:10 Pr 10:6,\allowbreak11; 12:13; 18:7 Mt 27:25}
\crossref{Ps}{139}{11}{Ps 11:6; 18:13,\allowbreak14; 21:9; 120:4 Ge 19:24 Ex 9:23,\allowbreak24 Re 16:8,\allowbreak9}
\crossref{Ps}{139}{12}{Ps 12:3,\allowbreak4 Pr 6:17; 12:13; 17:20; 18:21}
\crossref{Ps}{139}{13}{Ps 9:4; 10:17,\allowbreak18; 22:24; 72:4,\allowbreak12-\allowbreak14; 102:17 1Ki 8:45,\allowbreak49 Pr 22:22}
\crossref{Ps}{139}{14}{Ps 32:11; 33:1 Isa 3:10}
\crossref{Ps}{139}{15}{}
\crossref{Ps}{139}{16}{}
\crossref{Ps}{139}{17}{}
\crossref{Ps}{139}{18}{}
\crossref{Ps}{139}{19}{}
\crossref{Ps}{139}{20}{}
\crossref{Ps}{139}{21}{}
\crossref{Ps}{139}{22}{}
\crossref{Ps}{139}{23}{}
\crossref{Ps}{139}{24}{}
\crossref{Ps}{140}{1}{Ps 40:13; 69:17,\allowbreak18; 70:5; 71:12; 143:7 Job 7:21}
\crossref{Ps}{140}{2}{Ps 5:3}
\crossref{Ps}{140}{3}{Ps 17:3-\allowbreak5; 39:1; 71:8 Mic 7:5 Jas 1:26; 3:2}
\crossref{Ps}{140}{4}{140:119:36 De 2:30; 29:4 1Ki 8:58; 22:22 Isa 63:17 Mt 6:13 Jas 1:13}
\crossref{Ps}{140}{5}{1Sa 25:31-\allowbreak34 2Sa 12:7-\allowbreak13 2Ch 16:7-\allowbreak10; 25:16 Pr 6:23; 9:8,\allowbreak9}
\crossref{Ps}{140}{6}{1Sa 31:1-\allowbreak8 2Sa 1:17-\allowbreak27 1Ch 10:1-\allowbreak7}
\crossref{Ps}{140}{7}{Ps 44:22 1Sa 22:18,\allowbreak19 Ro 8:36 2Co 1:9 Heb 11:37 Re 11:8,\allowbreak9}
\crossref{Ps}{140}{8}{Ps 25:15; 123:1,\allowbreak2 2Ch 20:12}
\crossref{Ps}{140}{9}{140:119:110; 140:5; 142:3 Pr 13:14 Jer 18:22 Lu 20:20}
\crossref{Ps}{140}{10}{Ps 7:15,\allowbreak16; 35:8; 37:14,\allowbreak15; 64:7,\allowbreak8; 140:9 Es 7:10 Pr 11:8}
\crossref{Ps}{140}{11}{}
\crossref{Ps}{140}{12}{}
\crossref{Ps}{140}{13}{}
\crossref{Ps}{141}{1}{}
\crossref{Ps}{141}{2}{Ps 42:4; 62:8; 102:1}
\crossref{Ps}{141}{3}{Ps 42:4; 62:8; 102:1}
\crossref{Ps}{141}{4}{Ps 42:4; 62:8; 102:1}
\crossref{Ps}{141}{5}{Ps 22:14; 61:2; 102:4; 143:4 Mr 14:33-\allowbreak36}
\crossref{Ps}{141}{6}{1Sa 23:11-\allowbreak13,\allowbreak19,\allowbreak20; 27:1}
\crossref{Ps}{141}{7}{Ps 46:1,\allowbreak7,\allowbreak11; 62:6,\allowbreak7; 91:2,\allowbreak9,\allowbreak10 Joh 16:32 2Ti 4:17}
\crossref{Ps}{141}{8}{Ps 44:24-\allowbreak26; 79:8; 116:6; 136:23; 143:3,\allowbreak7}
\crossref{Ps}{141}{9}{}
\crossref{Ps}{141}{10}{}
\crossref{Ps}{142}{1}{142:142:1}
\crossref{Ps}{142}{2}{142:130:3 Job 14:3}
\crossref{Ps}{142}{3}{Ps 7:1,\allowbreak2; 17:9-\allowbreak13; 35:4; 54:3; 142:6}
\crossref{Ps}{142}{4}{Ps 55:5; 61:2; 77:3; 102:1}
\crossref{Ps}{142}{5}{Ps 42:6; 77:5,\allowbreak6,\allowbreak10-\allowbreak12; 111:4 De 8:2,\allowbreak3 1Sa 17:34-\allowbreak37,\allowbreak45-\allowbreak50}
\crossref{Ps}{142}{6}{Ps 44:20; 88:9 Job 11:13}
\crossref{Ps}{142}{7}{Ps 13:1-\allowbreak4; 40:13,\allowbreak17; 70:5; 71:12}
\crossref{Ps}{143}{1}{Ps 18:2,\allowbreak31; 71:3; 95:1 De 32:30,\allowbreak31 Isa 26:4}
\crossref{Ps}{143}{2}{2Sa 22:2,\allowbreak3,\allowbreak40-\allowbreak48 Jer 16:19}
\crossref{Ps}{143}{3}{Ps 8:4 Job 7:17; 15:14 Heb 2:6}
\crossref{Ps}{143}{4}{Ps 39:5,\allowbreak6; 62:9; 89:47 Job 4:19; 14:1-\allowbreak3 Ec 1:2,\allowbreak14; 12:8}
\crossref{Ps}{143}{5}{Ps 18:9 Isa 64:1,\allowbreak2}
\crossref{Ps}{143}{6}{Ps 18:13,\allowbreak14; 77:17,\allowbreak18 2Sa 22:12-\allowbreak15}
\crossref{Ps}{143}{7}{Ps 18:16 2Sa 22:17 Mt 27:43}
\crossref{Ps}{143}{8}{Ps 10:7; 12:2; 41:6; 58:3; 62:4; 109:2,\allowbreak3 Isa 59:5-\allowbreak7}
\crossref{Ps}{143}{9}{Ps 33:2,\allowbreak3; 40:3; 98:1; 149:1 Re 5:9,\allowbreak10; 14:3}
\crossref{Ps}{143}{10}{Ps 18:50; 33:16-\allowbreak18 2Sa 5:19-\allowbreak25; 8:6-\allowbreak14 2Ki 5:1}
\crossref{Ps}{143}{11}{143:7,\allowbreak8 2Sa 10:6-\allowbreak19; 16:5-\allowbreak14; 17:1-\allowbreak14}
\crossref{Ps}{143}{12}{143:115:14,\allowbreak15; 127:4,\allowbreak5; 128:3 Isa 44:3-\allowbreak5 La 4:2}
\crossref{Ps}{144}{1}{144:100:1}
\crossref{Ps}{144}{2}{144:72:15; 119:164 Re 7:15}
\crossref{Ps}{144}{3}{Ps 48:1; 96:4; 147:5 Job 5:9; 9:10 Re 15:3}
\crossref{Ps}{144}{4}{Ps 44:1,\allowbreak2; 71:18; 78:3-\allowbreak7 Ex 12:26,\allowbreak27; 13:14,\allowbreak15 De 6:7 Jos 4:21-\allowbreak24}
\crossref{Ps}{144}{5}{Ps 40:9,\allowbreak10; 66:3,\allowbreak4; 71:17-\allowbreak19,\allowbreak24; 96:3; 104:1,\allowbreak2; 105:2 Isa 12:4}
\crossref{Ps}{144}{6}{Ps 22:22,\allowbreak23,\allowbreak27,\allowbreak31; 98:2,\allowbreak3; 113:3; 126:2,\allowbreak3 Jos 2:9-\allowbreak11; 9:9,\allowbreak10}
\crossref{Ps}{144}{7}{Ps 36:5-\allowbreak8 Isa 63:7 Mt 12:34,\allowbreak35 2Co 9:11,\allowbreak12 1Pe 2:9,\allowbreak10}
\crossref{Ps}{144}{8}{144:86:5,\allowbreak15 100:5 103:8 116:5 Ex 34:6,\allowbreak7 Nu 14:18 Da 9:9 Jon 4:2}
\crossref{Ps}{144}{9}{Ps 25:8; 36:6,\allowbreak7; 65:9-\allowbreak13; 104:27 Jon 4:11 Na 1:7 Mt 5:45 Ac 14:17}
\crossref{Ps}{144}{10}{Ps 19:1; 96:11-\allowbreak13; 98:3-\allowbreak9; 103:22; 104:24; 148:1-\allowbreak13 Isa 43:20}
\crossref{Ps}{144}{11}{Ps 2:6-\allowbreak8; 45:6,\allowbreak7; 72:1-\allowbreak20; 93:1,\allowbreak2; 96:10-\allowbreak13; 97:1-\allowbreak12; 99:1-\allowbreak4}
\crossref{Ps}{144}{12}{144:98:1 105:5 106:2 110:2,\allowbreak3 145:6-\allowbreak12 136:4-\allowbreak26 Da 4:34,\allowbreak35}
\crossref{Ps}{144}{13}{144:146:10 Isa 9:7 Da 2:44; 7:14,\allowbreak27 1Ti 1:17 Re 11:15}
\crossref{Ps}{144}{14}{Ps 37:24; 94:18; 119:117 Lu 22:31,\allowbreak32}
\crossref{Ps}{144}{15}{144:9; 104:21,\allowbreak27; 136:25; 147:8,\allowbreak9 Ge 1:30 Job 38:39-\allowbreak41 Joe 2:22}
\crossref{Ps}{145}{1}{145:105:45}
\crossref{Ps}{145}{2}{Ps 63:4; 71:14,\allowbreak15; 104:33; 145:1,\allowbreak2 Re 7:9-\allowbreak17}
\crossref{Ps}{145}{3}{Ps 62:9; 118:8,\allowbreak9 Isa 2:22; 31:3; 37:6 Jer 17:5,\allowbreak6}
\crossref{Ps}{145}{4}{145:104:29 Ge 2:7; 6:17 Job 14:10; 17:1; 27:3 Da 5:23}
\crossref{Ps}{145}{5}{Ps 33:12; 84:12; 144:15 De 33:29}
\crossref{Ps}{145}{6}{Ps 33:6; 136:5,\allowbreak6; 148:5,\allowbreak6 Ge 1:1 Jer 10:11,\allowbreak12; 32:17 Joh 1:3}
\crossref{Ps}{145}{7}{Ps 9:16; 10:14,\allowbreak15,\allowbreak18; 12:5; 72:4; 103:6 Pr 22:22,\allowbreak23; 23:10,\allowbreak11 Isa 9:4}
\crossref{Ps}{145}{8}{Isa 35:5; 42:16,\allowbreak18 Mt 9:30; 11:5 Lu 18:41,\allowbreak42 Joh 9:7-\allowbreak33}
\crossref{Ps}{145}{9}{145:68:5 De 10:18,\allowbreak19; 16:11 Pr 15:25 Jer 49:11 Ho 14:3 Mal 3:5}
\crossref{Ps}{145}{10}{Ps 10:16; 145:13 Ex 15:18 Isa 9:7 Da 2:44; 6:26; 7:14 Re 11:15}
\crossref{Ps}{145}{11}{}
\crossref{Ps}{145}{12}{}
\crossref{Ps}{145}{13}{}
\crossref{Ps}{145}{14}{}
\crossref{Ps}{145}{15}{}
\crossref{Ps}{145}{16}{}
\crossref{Ps}{145}{17}{}
\crossref{Ps}{145}{18}{}
\crossref{Ps}{145}{19}{}
\crossref{Ps}{145}{20}{}
\crossref{Ps}{145}{21}{}
\crossref{Ps}{146}{1}{Ps 63:3-\allowbreak5; 92:1; 135:3}
\crossref{Ps}{146}{2}{Ps 51:18; 102:13-\allowbreak16 Ne 3:1-\allowbreak16; 7:4 Isa 14:32; 62:7 Jer 31:4 Da 9:25}
\crossref{Ps}{146}{3}{Ps 51:17 Job 5:18 Isa 57:15; 61:1 Jer 33:6 Ho 6:1,\allowbreak2 Mal 4:2}
\crossref{Ps}{146}{4}{Ps 8:3; 148:3 Ge 15:5 Isa 40:26}
\crossref{Ps}{146}{5}{Ps 48:1; 96:4; 99:2; 135:5; 145:3 1Ch 16:25 Jer 10:6; 32:17-\allowbreak19}
\crossref{Ps}{146}{6}{Ps 25:9; 37:11; 145:14; 146:8,\allowbreak9; 149:4 1Sa 2:8 Zep 2:3 Mt 5:5}
\crossref{Ps}{146}{7}{Ps 47:6,\allowbreak7; 68:32; 92:1-\allowbreak3; 95:1,\allowbreak2; 107:21,\allowbreak22 Ex 15:20,\allowbreak21 Re 5:8-\allowbreak10}
\crossref{Ps}{146}{8}{146:135:1 Ge 9:14 1Ki 18:44,\allowbreak45 Job 26:8,\allowbreak9; 36:27-\allowbreak33; 38:24-\allowbreak27}
\crossref{Ps}{146}{9}{146:104:27,\allowbreak28; 136:25; 145:15,\allowbreak16 Job 38:41 Mt 6:26 Lu 12:24}
\crossref{Ps}{146}{10}{Ps 20:7; 33:16-\allowbreak18 Job 39:19-\allowbreak25 Pr 21:31 Isa 31:1 Ho 1:7}
\crossref{Ps}{147}{1}{}
\crossref{Ps}{147}{2}{}
\crossref{Ps}{147}{3}{}
\crossref{Ps}{147}{4}{}
\crossref{Ps}{147}{5}{}
\crossref{Ps}{147}{6}{}
\crossref{Ps}{147}{7}{}
\crossref{Ps}{147}{8}{}
\crossref{Ps}{147}{9}{}
\crossref{Ps}{147}{10}{}
\crossref{Ps}{147}{11}{}
\crossref{Ps}{147}{12}{147:135:19-\allowbreak21; 146:10; 149:2 Isa 12:6; 52:7 Joe 2:23}
\crossref{Ps}{147}{13}{Ps 48:11-\allowbreak14; 51:18; 125:2 Ne 3:1-\allowbreak16; 6:1; 7:1; 12:30 La 2:8,\allowbreak9; 4:12}
\crossref{Ps}{147}{14}{Ps 29:11; 122:6 Le 26:6 1Ch 22:9 Isa 9:6,\allowbreak7; 60:17,\allowbreak18; 66:12 Zec 9:8}
\crossref{Ps}{147}{15}{Ps 33:9; 107:20,\allowbreak25 Job 34:29; 37:12 Jon 1:4 Mt 8:8,\allowbreak9,\allowbreak13}
\crossref{Ps}{147}{16}{147:148:8 Job 37:6 Isa 55:10}
\crossref{Ps}{147}{17}{147:78:47,\allowbreak48 Ex 9:23-\allowbreak25 Jos 10:11 Job 38:22,\allowbreak23}
\crossref{Ps}{147}{18}{147:15 Job 6:16,\allowbreak17; 37:10,\allowbreak17}
\crossref{Ps}{147}{19}{147:76:1 78:5 103:7 De 33:2-\allowbreak4 Mal 4:4 Ro 3:2; 9:4 2Ti 3:15-\allowbreak17}
\crossref{Ps}{147}{20}{De 4:32-\allowbreak34 Pr 29:18 Isa 5:1-\allowbreak7 Mt 21:33-\allowbreak41 Ac 14:16; 26:27,\allowbreak18}
\crossref{Ps}{148}{1}{148:89:5 146:1 Isa 49:13 Lu 2:13,\allowbreak14 Re 19:1-\allowbreak6}
\crossref{Ps}{148}{2}{148:103:20,\allowbreak21 Job 38:7 Isa 6:2-\allowbreak4 Eze 3:12 Re 5:11-\allowbreak13}
\crossref{Ps}{148}{3}{Ps 8:1-\allowbreak3; 19:1-\allowbreak6; 89:36,\allowbreak37; 136:7-\allowbreak9 Ge 1:14-\allowbreak16; 8:22 De 4:19}
\crossref{Ps}{148}{4}{148:113:6 1Ki 8:27 2Co 12:2}
\crossref{Ps}{148}{5}{Ps 33:6-\allowbreak9; 95:5 Ge 1:1,\allowbreak2,\allowbreak6 Jer 10:11-\allowbreak13 Am 9:6 Re 4:11}
\crossref{Ps}{148}{6}{148:89:37; 93:1 119:90,\allowbreak91 Job 38:10,\allowbreak11,\allowbreak33 Pr 8:27-\allowbreak29 Isa 54:9}
\crossref{Ps}{148}{7}{148:1}
\crossref{Ps}{148}{8}{148:147:15-\allowbreak18 Ge 19:24 Ex 9:23-\allowbreak25 Le 10:2 Nu 16:35 Jos 10:11}
\crossref{Ps}{148}{9}{Ps 65:12,\allowbreak13; 96:11-\allowbreak13; 97:4,\allowbreak5; 98:7-\allowbreak9; 114:3-\allowbreak7 Isa 42:11; 44:23}
\crossref{Ps}{148}{10}{Ps 50:10,\allowbreak11; 103:22; 150:6 Ge 1:20-\allowbreak25}
\crossref{Ps}{148}{11}{Ps 2:10-\allowbreak12; 22:27-\allowbreak29; 66:1-\allowbreak4; 68:31,\allowbreak32; 72:10,\allowbreak11; 86:9; 102:15; 138:4,\allowbreak5}
\crossref{Ps}{148}{12}{Ps 8:2; 68:25 Jer 31:13 Zec 9:17 Mt 21:15,\allowbreak16 Lu 19:37 Tit 2:4-\allowbreak6}
\crossref{Ps}{148}{13}{Ps 8:1,\allowbreak9; 99:3,\allowbreak4,\allowbreak9 So 5:9,\allowbreak16 Isa 6:3 Zec 9:17 Php 3:8}
\crossref{Ps}{148}{14}{148:75:10; 89:17 92:10 112:9 1Sa 2:1 Lu 1:52}
\crossref{Ps}{149}{1}{149:148:1}
\crossref{Ps}{149}{2}{149:100:1-\allowbreak3; 135:3,\allowbreak4 De 7:6,\allowbreak7; 12:7 1Sa 12:22 Job 35:10 Isa 54:5}
\crossref{Ps}{149}{3}{149:150:4}
\crossref{Ps}{149}{4}{Ps 22:8; 35:27; 117:2; 147:11 Pr 11:20 Isa 62:4,\allowbreak5 Jer 32:41}
\crossref{Ps}{149}{5}{Ps 23:1; 118:15; 145:10 Ro 5:2 1Pe 1:8}
\crossref{Ps}{149}{6}{149:96:4 Ne 9:5 Da 4:37 Lu 2:14 Re 19:6}
\crossref{Ps}{149}{7}{149:137:8,\allowbreak9 Nu 31:2,\allowbreak3 Jud 5:23 1Sa 15:2,\allowbreak3,\allowbreak18-\allowbreak23 Zec 9:13-\allowbreak16}
\crossref{Ps}{149}{8}{Jos 10:23,\allowbreak24; 12:7 Jud 1:6,\allowbreak7}
\crossref{Ps}{149}{9}{149:137:8 De 7:1,\allowbreak2; 32:42,\allowbreak43 Isa 14:22,\allowbreak23 Re 17:14-\allowbreak16}
\crossref{Ps}{150}{1}{150:149:1}
\crossref{Ps}{150}{2}{150:145:5,\allowbreak6 Re 15:3,\allowbreak4}
\crossref{Ps}{150}{3}{150:81:2,\allowbreak3 98:5,\allowbreak6 Nu 10:10 1Ch 15:24,\allowbreak28; 16:42 Da 3:5}
\crossref{Ps}{150}{4}{Ex 15:20}
\crossref{Ps}{150}{5}{1Ch 15:16,\allowbreak19,\allowbreak28; 16:5; 25:1,\allowbreak6}
\crossref{Ps}{150}{6}{150:103:22; 145:10; 148:7-\allowbreak11 Re 5:13}

% Prov
\crossref{Prov}{1}{1}{Pr 10:1; 25:1 1Ki 4:31,\allowbreak32 Ec 12:9 Joh 16:25}
\crossref{Prov}{1}{2}{Pr 4:5-\allowbreak7; 7:4; 8:5; 16:16; 17:16 De 4:5,\allowbreak6 1Ki 3:9-\allowbreak12 2Ti 3:15-\allowbreak17}
\crossref{Prov}{1}{3}{Pr 2:1-\allowbreak9; 8:10,\allowbreak11 Job 22:22}
\crossref{Prov}{1}{4}{1:22,\allowbreak23; 8:5; 9:4-\allowbreak6 Ps 19:7; 119:130 Isa 35:8}
\crossref{Prov}{1}{5}{Pr 9:9; 12:1 Job 34:10,\allowbreak16,\allowbreak34 Ps 119:98-\allowbreak100 1Co 10:15}
\crossref{Prov}{1}{6}{Mt 13:10-\allowbreak17,\allowbreak51,\allowbreak52 Mr 4:11,\allowbreak34 Ac 8:30,\allowbreak31}
\crossref{Prov}{1}{7}{Pr 9:10 Job 28:28 Ps 111:10; 112:1 Ec 12:13}
\crossref{Prov}{1}{8}{1:10,\allowbreak15; 2:1; 3:1; 7:1 Mt 9:2,\allowbreak22}
\crossref{Prov}{1}{9}{Pr 3:22; 4:9; 6:20,\allowbreak21 1Ti 2:9,\allowbreak10 1Pe 3:3,\allowbreak4}
\crossref{Prov}{1}{10}{Pr 7:21-\allowbreak23; 13:20; 20:19 Ge 39:7-\allowbreak13 Jud 16:16-\allowbreak21 Ps 1:1; 50:18}
\crossref{Prov}{1}{11}{1:16; 12:6; 30:14 Ps 56:6; 64:5,\allowbreak6 Jer 5:26 Mic 7:2 Ac 23:15; 25:3}
\crossref{Prov}{1}{12}{Ps 35:25; 56:1,\allowbreak2; 57:3; 124:3 Jer 51:34 La 2:5,\allowbreak16 Mic 3:2,\allowbreak3}
\crossref{Prov}{1}{13}{1:19 Job 24:2,\allowbreak3 Isa 10:13,\allowbreak14 Jer 22:16,\allowbreak17 Na 2:12 Hag 2:9}
\crossref{Prov}{1}{14}{Pr 16:33; 18:18 Le 16:8,\allowbreak9}
\crossref{Prov}{1}{15}{Pr 4:14,\allowbreak15; 9:6; 13:20 Ps 1:1; 26:4,\allowbreak5 2Co 6:17}
\crossref{Prov}{1}{16}{Pr 4:16; 6:18 Isa 59:7 Ro 3:5}
\crossref{Prov}{1}{17}{Pr 7:23 Job 35:11 Isa 1:3 Jer 8:7}
\crossref{Prov}{1}{18}{Pr 5:22,\allowbreak23; 9:17,\allowbreak18; 28:17 Es 7:10 Ps 7:14-\allowbreak16; 9:16; 55:23 Mt 27:4,\allowbreak5}
\crossref{Prov}{1}{19}{Pr 15:27; 23:3,\allowbreak4 2Sa 18:11-\allowbreak13 2Ki 5:20-\allowbreak27 Jer 22:17-\allowbreak19 Mic 2:1-\allowbreak3}
\crossref{Prov}{1}{20}{Mt 13:54 Lu 11:49 1Co 1:24,\allowbreak30 Col 2:3}
\crossref{Prov}{1}{21}{Pr 9:3 Mt 10:27; 13:2 Joh 18:20 Ac 5:20}
\crossref{Prov}{1}{22}{Pr 6:9 Ex 10:3; 16:28 Nu 14:27 Mt 17:17}
\crossref{Prov}{1}{23}{Isa 55:1-\allowbreak3,\allowbreak6,\allowbreak7 Jer 3:14 Eze 18:27-\allowbreak30; 33:11 Ho 14:1 Ac 3:19}
\crossref{Prov}{1}{24}{Isa 50:2; 65:12; 66:4 Jer 7:13 Eze 8:18 Zec 7:11,\allowbreak12 Mt 22:5,\allowbreak6}
\crossref{Prov}{1}{25}{1:30 2Ch 36:16 Ps 107:11 Lu 7:30}
\crossref{Prov}{1}{26}{Jud 10:14 Ps 2:4; 37:13 Lu 14:24}
\crossref{Prov}{1}{27}{Pr 3:25,\allowbreak26; 10:24,\allowbreak25 Ps 69:22-\allowbreak28 Lu 21:26,\allowbreak34,\allowbreak35 1Th 5:3}
\crossref{Prov}{1}{28}{Ge 6:3 Job 27:9; 35:12 Ps 18:41 Isa 1:15 Jer 11:11; 14:12}
\crossref{Prov}{1}{29}{1:22; 5:12; 6:23 Job 21:14,\allowbreak15 Ps 50:16,\allowbreak17 Isa 27:11; 30:9-\allowbreak12}
\crossref{Prov}{1}{30}{1:25 Ps 81:11; 119:111,\allowbreak173 Jer 8:9 Lu 14:18-\allowbreak20}
\crossref{Prov}{1}{31}{Pr 14:14; 22:8 Job 4:8 Isa 3:10,\allowbreak11 Jer 2:19; 6:19 Ga 6:7,\allowbreak8}
\crossref{Prov}{1}{32}{Pr 8:36 Joh 3:36 Heb 10:38,\allowbreak39; 12:25}
\crossref{Prov}{1}{33}{Pr 8:32-\allowbreak35; 9:11 Ps 25:12,\allowbreak13; 81:13 Isa 48:18; 55:3 Mt 17:5}
\crossref{Prov}{2}{1}{Pr 1:3; 4:1; 7:1 Joh 12:47,\allowbreak48 1Ti 1:15}
\crossref{Prov}{2}{2}{Pr 18:1 Ps 119:111,\allowbreak112 Isa 55:3 Mt 13:9}
\crossref{Prov}{2}{3}{Pr 3:6; 8:17 1Ki 3:9-\allowbreak12 1Ch 22:12 Ps 25:4,\allowbreak5; 119:34,\allowbreak73,\allowbreak125,\allowbreak169}
\crossref{Prov}{2}{4}{Pr 3:14,\allowbreak15; 8:18,\allowbreak19; 16:16; 23:23 Ps 19:10; 119:14,\allowbreak72,\allowbreak127 Mt 6:19-\allowbreak21}
\crossref{Prov}{2}{5}{2Ch 1:10-\allowbreak12 Ho 6:3 Mt 7:7,\allowbreak8 Lu 11:9-\allowbreak13}
\crossref{Prov}{2}{6}{Ex 31:3 1Ki 3:9,\allowbreak12; 4:29 1Ch 22:12 Job 32:8 Isa 54:13 Da 1:17}
\crossref{Prov}{2}{7}{Pr 8:14; 14:8 Job 28:8 1Co 1:19,\allowbreak24,\allowbreak30; 2:6,\allowbreak7; 3:18,\allowbreak19 Col 2:3}
\crossref{Prov}{2}{8}{Pr 8:20 Ps 1:6; 23:3; 121:5-\allowbreak8 Isa 35:9; 49:9,\allowbreak10 Joh 10:28,\allowbreak29}
\crossref{Prov}{2}{9}{Pr 1:2-\allowbreak6 Ps 25:8,\allowbreak9; 32:8; 119:99,\allowbreak105; 143:8-\allowbreak10 Isa 35:8; 48:17}
\crossref{Prov}{2}{10}{Pr 18:1,\allowbreak2; 24:13,\allowbreak14 Job 23:12 Ps 19:10; 104:34; 119:97,\allowbreak103,\allowbreak111,\allowbreak162}
\crossref{Prov}{2}{11}{Pr 4:6; 6:22-\allowbreak24 Ps 25:21; 119:9-\allowbreak11 Ec 9:15-\allowbreak18; 10:10 Eph 5:15}
\crossref{Prov}{2}{12}{Pr 1:10-\allowbreak19; 4:14-\allowbreak17; 9:6; 13:20 Ps 17:4,\allowbreak5; 26:4,\allowbreak5; 141:4 2Co 6:17}
\crossref{Prov}{2}{13}{Pr 21:16 Ps 14:3; 36:3 Eze 18:26; 33:12,\allowbreak13 Zep 1:6 Mt 12:43-\allowbreak45}
\crossref{Prov}{2}{14}{Pr 10:23 Jer 11:15 Hab 1:15 Zep 3:11 1Co 13:6}
\crossref{Prov}{2}{15}{De 32:5 Ps 125:5 Isa 30:8-\allowbreak13; 59:8 Php 2:15}
\crossref{Prov}{2}{16}{Pr 5:3-\allowbreak20; 6:24; 7:5-\allowbreak23; 22:14; 23:27 Ge 39:3-\allowbreak12 Ne 13:26,\allowbreak27 Ec 7:26}
\crossref{Prov}{2}{17}{Pr 5:18 Jer 3:4}
\crossref{Prov}{2}{18}{Pr 5:4-\allowbreak14; 6:26-\allowbreak35; 7:22-\allowbreak27; 9:18 1Co 6:9-\allowbreak11 Ga 5:19-\allowbreak21 Eph 5:5}
\crossref{Prov}{2}{19}{Ps 81:12 Ec 7:26 Jer 13:23 Ho 4:14 Mt 19:24-\allowbreak26}
\crossref{Prov}{2}{20}{Pr 13:20 Ps 119:63,\allowbreak115 So 1:7,\allowbreak8 Jer 6:16 Heb 6:12 3Jo 1:11}
\crossref{Prov}{2}{21}{Job 1:1; 42:12 Ps 37:3,\allowbreak9,\allowbreak11,\allowbreak22,\allowbreak29; 84:11; 112:4-\allowbreak6}
\crossref{Prov}{2}{22}{Pr 5:22,\allowbreak23 Job 18:16-\allowbreak18; 21:30 Ps 37:20,\allowbreak22,\allowbreak28,\allowbreak37,\allowbreak38; 52:5; 104:35}
\crossref{Prov}{3}{1}{Pr 1:8; 4:5; 31:5 De 4:23 Ps 119:93,\allowbreak153,\allowbreak176 Ho 4:6}
\crossref{Prov}{3}{2}{3:16; 4:10; 9:11 Job 5:26 Ps 34:11-\allowbreak14; 91:16; 128:6 Eph 6:1-\allowbreak3}
\crossref{Prov}{3}{3}{Pr 16:6; 20:28 2Sa 15:20 Ps 25:10 Ho 4:1 Mic 7:18-\allowbreak20 Mal 2:6}
\crossref{Prov}{3}{4}{Ge 39:2-\allowbreak4,\allowbreak21 1Sa 2:26 Ps 111:10 Da 1:9 Lu 2:52 Ac 2:47}
\crossref{Prov}{3}{5}{Pr 22:19 Job 13:15 Ps 37:3,\allowbreak5,\allowbreak7; 62:8; 115:9-\allowbreak11; 125:1; 146:3-\allowbreak5}
\crossref{Prov}{3}{6}{Pr 16:3; 23:17 1Sa 4:11,\allowbreak12; 30:8 1Ch 28:9 Ezr 7:27; 8:22,\allowbreak23}
\crossref{Prov}{3}{7}{Pr 26:12 Isa 5:21 Ro 11:25; 12:16}
\crossref{Prov}{3}{8}{Pr 4:22; 16:24 Ps 147:3 Isa 1:6 Jer 30:12,\allowbreak13}
\crossref{Prov}{3}{9}{Pr 14:31 Ge 14:18-\allowbreak21; 28:22 Ex 22:29; 23:19; 34:26; 35:20-\allowbreak29}
\crossref{Prov}{3}{10}{Pr 11:24,\allowbreak25; 19:17; 22:9 Le 26:2-\allowbreak5 De 28:8 Ec 11:1,\allowbreak2 Hag 2:19}
\crossref{Prov}{3}{11}{Job 5:17 Ps 94:12 1Co 11:32 Heb 12:5,\allowbreak6 Re 3:19}
\crossref{Prov}{3}{12}{Pr 29:17 De 8:5 Ps 103:13}
\crossref{Prov}{3}{13}{Pr 4:5-\allowbreak9; 8:32-\allowbreak35 1Ki 10:1-\allowbreak9,\allowbreak23,\allowbreak24 Ec 9:15-\allowbreak18}
\crossref{Prov}{3}{14}{Pr 2:4; 8:10,\allowbreak11,\allowbreak19; 16:16 2Ch 1:11,\allowbreak12 Job 28:13-\allowbreak19}
\crossref{Prov}{3}{15}{Pr 8:11; 20:15; 31:10 Mt 13:44-\allowbreak46}
\crossref{Prov}{3}{16}{3:2; 4:10 Ps 21:4; 71:9 1Ti 4:8}
\crossref{Prov}{3}{17}{Pr 2:10; 22:18 Ps 19:10,\allowbreak11; 63:3-\allowbreak5; 112:1; 119:14,\allowbreak47,\allowbreak103,\allowbreak174}
\crossref{Prov}{3}{18}{Pr 11:30; 13:12 Ge 2:9; 3:22 Re 22:2}
\crossref{Prov}{3}{19}{Pr 8:27-\allowbreak29 Ps 104:24; 136:5 Jer 10:12; 51:15 Joh 1:3}
\crossref{Prov}{3}{20}{Ge 1:9; 7:11 Job 38:8-\allowbreak11 Ps 104:8,\allowbreak9}
\crossref{Prov}{3}{21}{3:1-\allowbreak3 De 4:9; 6:6-\allowbreak9 Jos 1:8 Joh 8:31; 15:6,\allowbreak7 Heb 2:1-\allowbreak3}
\crossref{Prov}{3}{22}{Pr 4:22 Isa 38:16 Joh 12:49,\allowbreak50}
\crossref{Prov}{3}{23}{Pr 2:8; 4:12; 10:9 Ps 37:23,\allowbreak24,\allowbreak31; 91:11; 121:3,\allowbreak8 Zec 10:12}
\crossref{Prov}{3}{24}{Pr 6:22 Le 26:6 Ps 3:5; 4:8; 121:4-\allowbreak7 Eze 34:15}
\crossref{Prov}{3}{25}{Job 5:21,\allowbreak22; 11:13-\allowbreak15 Ps 27:1,\allowbreak2; 46:1-\allowbreak3; 91:5; 112:7 Isa 8:12,\allowbreak13}
\crossref{Prov}{3}{26}{Pr 14:26 Ps 91:3,\allowbreak9,\allowbreak10 Hab 3:17,\allowbreak18}
\crossref{Prov}{3}{27}{Ro 13:7 Ga 6:10 Tit 2:14 Jas 2:15,\allowbreak16; 5:4}
\crossref{Prov}{3}{28}{Pr 27:1 Le 19:13 De 24:12-\allowbreak15 Ec 9:10; 11:6 2Co 8:11; 9:3 1Ti 6:18}
\crossref{Prov}{3}{29}{Pr 6:14,\allowbreak18; 16:29,\allowbreak30 Ps 35:20; 55:20; 59:3 Jer 18:18-\allowbreak20 Mic 2:1,\allowbreak2}
\crossref{Prov}{3}{30}{Pr 17:14; 18:6; 25:8,\allowbreak9; 29:22 Mt 5:39-\allowbreak41 Ro 12:18-\allowbreak21 1Co 6:6-\allowbreak8}
\crossref{Prov}{3}{31}{Pr 23:17; 24:1,\allowbreak19,\allowbreak20 Ps 37:1,\allowbreak7-\allowbreak9; 73:3 Ga 5:21}
\crossref{Prov}{3}{32}{Pr 6:6-\allowbreak19; 8:13; 11:20; 17:15 Ps 18:26 Lu 16:15}
\crossref{Prov}{3}{33}{Pr 21:12 Le 26:14-\allowbreak46 De 7:26; 28:15-\allowbreak68; 29:19-\allowbreak29 Jos 6:18; 7:13}
\crossref{Prov}{3}{34}{Pr 9:7,\allowbreak8,\allowbreak12; 19:29; 21:24 Ps 138:6}
\crossref{Prov}{3}{35}{Pr 4:8 1Sa 2:30 Ps 73:24}
\crossref{Prov}{4}{1}{Pr 1:8; 6:20-\allowbreak23 Ps 34:11 1Th 2:11,\allowbreak12}
\crossref{Prov}{4}{2}{Pr 8:6-\allowbreak9; 22:20,\allowbreak21 De 32:2 Job 33:3 Ps 49:1-\allowbreak3 Joh 7:16,\allowbreak17}
\crossref{Prov}{4}{3}{2Sa 12:24,\allowbreak25 1Ki 1:13-\allowbreak17 1Ch 3:5; 22:5; 29:1 Jer 10:23 Ro 12:16}
\crossref{Prov}{4}{4}{Pr 22:6 Ge 18:19 1Ch 22:11-\allowbreak16; 28:9 Eph 6:4 2Ti 1:5; 3:15}
\crossref{Prov}{4}{5}{Pr 1:22,\allowbreak23; 2:2-\allowbreak4; 3:13-\allowbreak18; 8:5; 17:16; 18:1; 19:8; 23:23 Jas 1:5}
\crossref{Prov}{4}{6}{4:21,\allowbreak22; 2:10-\allowbreak12 Eph 3:17 2Th 2:10}
\crossref{Prov}{4}{7}{Ec 7:12; 9:16-\allowbreak18 Mt 13:44-\allowbreak46 Lu 10:42 Php 3:8}
\crossref{Prov}{4}{8}{Pr 3:35; 22:4 1Sa 2:30 1Ki 3:5-\allowbreak13 Da 12:3}
\crossref{Prov}{4}{9}{Pr 1:9; 3:22 1Ti 2:9,\allowbreak10 1Pe 3:4}
\crossref{Prov}{4}{10}{Pr 8:10; 19:20 Job 22:22 Jer 9:20 Joh 3:32,\allowbreak33 1Th 2:13 1Ti 1:15}
\crossref{Prov}{4}{11}{4:4 De 4:5 1Sa 12:24 Ec 12:9}
\crossref{Prov}{4}{12}{Pr 6:22 2Sa 22:37 Job 18:7,\allowbreak8 Ps 18:36}
\crossref{Prov}{4}{13}{Pr 3:18; 23:23 Ac 2:42; 11:23 1Th 5:21 Heb 2:1 Re 2:13; 12:11}
\crossref{Prov}{4}{14}{Pr 1:10,\allowbreak15; 2:11,\allowbreak12; 9:6; 13:20 Ps 1:1; 26:4,\allowbreak5 1Co 15:33}
\crossref{Prov}{4}{15}{Pr 5:8; 6:5 Ex 23:7 Job 11:14; 22:23 Isa 33:15 Eph 5:11 1Th 5:22}
\crossref{Prov}{4}{16}{Pr 1:16 Ps 36:4 Isa 57:20 Mic 2:1 Lu 22:66 Joh 18:28 2Pe 2:14}
\crossref{Prov}{4}{17}{Pr 9:17; 20:17 Job 24:5,\allowbreak6 Ps 14:4 Jer 5:26-\allowbreak28 Eze 22:25-\allowbreak29}
\crossref{Prov}{4}{18}{2Sa 23:4 Job 11:17; 23:10 Ps 84:7 Ho 6:3 Zec 14:6,\allowbreak7}
\crossref{Prov}{4}{19}{1Sa 2:9 Job 5:14; 12:25; 18:5,\allowbreak6,\allowbreak18 Isa 59:9,\allowbreak10 Jer 13:16; 23:12}
\crossref{Prov}{4}{20}{Pr 5:1; 6:20,\allowbreak21; 7:1 Ps 78:1; 90:12 Isa 55:3 Mt 17:5}
\crossref{Prov}{4}{21}{Pr 3:3,\allowbreak21}
\crossref{Prov}{4}{22}{4:4,\allowbreak10}
\crossref{Prov}{4}{23}{Pr 22:5; 23:19; 28:26 De 4:9 Ps 139:23,\allowbreak24 Jer 17:9 Mr 14:38}
\crossref{Prov}{4}{24}{Job 11:14 Eze 18:31 Eph 4:25-\allowbreak31 Col 3:8 Jas 1:21,\allowbreak26 1Pe 2:1}
\crossref{Prov}{4}{25}{Pr 23:5,\allowbreak33 Job 31:1 Ps 119:37 Mt 6:22}
\crossref{Prov}{4}{26}{Pr 5:6 Ps 119:59 Eze 18:28 Hag 1:5,\allowbreak7 Eph 5:15,\allowbreak17}
\crossref{Prov}{4}{27}{De 5:32; 12:32; 28:14 Jos 1:7}
\crossref{Prov}{5}{1}{Pr 2:1; 4:1,\allowbreak20 Mt 3:9 Mr 4:23 Re 2:7,\allowbreak11,\allowbreak17,\allowbreak29; 3:6,\allowbreak13,\allowbreak22}
\crossref{Prov}{5}{2}{Pr 10:21; 15:2,\allowbreak7; 16:23; 20:15 Ps 45:2; 71:15; 119:13 So 4:11}
\crossref{Prov}{5}{3}{Pr 2:16; 6:24; 7:21 Re 17:2-\allowbreak6}
\crossref{Prov}{5}{4}{Pr 6:24-\allowbreak35; 7:22,\allowbreak23; 9:18; 23:27,\allowbreak28 Ec 7:26 Heb 12:15,\allowbreak16}
\crossref{Prov}{5}{5}{Pr 2:18,\allowbreak19; 7:27}
\crossref{Prov}{5}{6}{Pr 4:26 Ps 119:59}
\crossref{Prov}{5}{7}{Pr 4:1; 8:32-\allowbreak36; 22:17-\allowbreak21 Heb 12:25}
\crossref{Prov}{5}{8}{Pr 4:15; 6:27,\allowbreak28 Mt 6:13 Eph 5:11}
\crossref{Prov}{5}{9}{Pr 6:29-\allowbreak35 Ge 38:23-\allowbreak26 Jud 16:19-\allowbreak21 Ne 13:26 Ho 4:13,\allowbreak14}
\crossref{Prov}{5}{10}{Pr 6:35 Ho 7:9 Lu 15:30}
\crossref{Prov}{5}{11}{Pr 7:23 De 32:29 Jer 5:31 Ro 6:21 Heb 13:4 Re 21:8; 22:15}
\crossref{Prov}{5}{12}{Pr 1:7,\allowbreak22,\allowbreak29,\allowbreak30; 15:5 Ps 50:17; 73:22 Zec 7:11-\allowbreak14 Joh 3:19,\allowbreak20}
\crossref{Prov}{5}{13}{Lu 15:18 1Th 4:8; 5:12,\allowbreak13 Heb 13:7}
\crossref{Prov}{5}{14}{Pr 13:20 Nu 25:1-\allowbreak6 Ho 4:11-\allowbreak14 1Co 10:6-\allowbreak8 2Pe 2:10-\allowbreak18 Jude 1:7-\allowbreak13}
\crossref{Prov}{5}{15}{5:18,\allowbreak19 1Co 7:2-\allowbreak5 Heb 13:4}
\crossref{Prov}{5}{16}{De 33:28 Ps 68:26 Isa 48:21}
\crossref{Prov}{5}{17}{5:3,\allowbreak10,\allowbreak20 Ge 34:27 1Ki 11:1}
\crossref{Prov}{5}{18}{Ec 9:9 Mal 2:14,\allowbreak15}
\crossref{Prov}{5}{19}{So 2:9; 4:5; 7:3; 8:14}
\crossref{Prov}{5}{20}{Pr 2:16-\allowbreak19; 6:24; 7:5; 22:14; 23:27,\allowbreak28,\allowbreak33 1Ki 11:1}
\crossref{Prov}{5}{21}{Pr 15:3 2Ch 16:9 Job 31:4; 34:21 Ps 11:4; 17:3; 139:1-\allowbreak12 Jer 16:17}
\crossref{Prov}{5}{22}{Pr 1:18,\allowbreak31; 11:3,\allowbreak5 Ps 7:15,\allowbreak16; 9:15 Jer 2:19 Ho 4:11-\allowbreak14 Ga 6:7,\allowbreak8}
\crossref{Prov}{5}{23}{Pr 10:21; 14:32 Job 4:21; 36:12}
\crossref{Prov}{6}{1}{Pr 11:15; 17:18; 20:16; 22:26; 27:13 Ge 43:9; 44:32,\allowbreak33 Job 17:3}
\crossref{Prov}{6}{2}{Pr 12:13; 18:7}
\crossref{Prov}{6}{3}{2Sa 24:14 2Ch 12:5 Ps 31:8}
\crossref{Prov}{6}{4}{6:10,\allowbreak11 Ps 132:4 Ec 9:10 Mt 24:17,\allowbreak18 Mr 13:35,\allowbreak36}
\crossref{Prov}{6}{5}{Pr 1:17 Ps 11:1; 124:7}
\crossref{Prov}{6}{6}{6:9; 10:26; 13:4; 15:19; 18:9; 19:15,\allowbreak24; 20:4; 21:25; 22:13; 24:30-\allowbreak34}
\crossref{Prov}{6}{7}{Job 38:39-\allowbreak41; 39:1-\allowbreak12,\allowbreak26-\allowbreak30; 41:4-\allowbreak34}
\crossref{Prov}{6}{8}{Pr 30:25 1Ti 6:19}
\crossref{Prov}{6}{9}{Pr 1:22; 24:33,\allowbreak34 Jer 4:14}
\crossref{Prov}{6}{10}{6:6; 23:33,\allowbreak34; 24:33,\allowbreak34}
\crossref{Prov}{6}{11}{Pr 10:4; 13:4; 20:4}
\crossref{Prov}{6}{12}{Pr 11:6; 17:4 1Sa 17:28 Jer 24:2,\allowbreak8-\allowbreak10 Jas 1:21}
\crossref{Prov}{6}{13}{Pr 5:6; 10:10 Job 15:12 Ps 35:19}
\crossref{Prov}{6}{14}{Pr 2:14; 16:28-\allowbreak30; 21:8}
\crossref{Prov}{6}{15}{Pr 1:27; 29:1 Ps 73:18-\allowbreak20 Isa 30:13 1Th 5:3}
\crossref{Prov}{6}{16}{Pr 8:13; 30:18,\allowbreak21,\allowbreak24,\allowbreak29 Am 1:3,\allowbreak6,\allowbreak9,\allowbreak11; 2:1,\allowbreak4,\allowbreak6}
\crossref{Prov}{6}{17}{Pr 30:13 Ps 10:4; 18:27; 73:6-\allowbreak8; 101:5; 131:1 Isa 2:11; 3:9,\allowbreak16}
\crossref{Prov}{6}{18}{Pr 24:8 Ge 6:5 Ps 36:4 Jer 4:14 Mic 2:1 Zec 8:17}
\crossref{Prov}{6}{19}{Pr 12:17; 19:5,\allowbreak9; 25:18 Ex 20:16; 23:1 De 19:16-\allowbreak20 1Ki 21:10-\allowbreak15}
\crossref{Prov}{6}{20}{Pr 1:8,\allowbreak9; 7:1-\allowbreak4; 23:22; 30:11 De 21:18; 27:16 Eph 6:1}
\crossref{Prov}{6}{21}{Pr 3:3; 4:6,\allowbreak21; 7:3,\allowbreak4 Ex 13:16 De 6:8 2Co 3:3}
\crossref{Prov}{6}{22}{Pr 2:11; 3:23,\allowbreak24 Ps 17:4; 43:3; 119:9,\allowbreak11,\allowbreak24,\allowbreak54,\allowbreak97,\allowbreak148 Da 11:18-\allowbreak21}
\crossref{Prov}{6}{23}{Ps 19:8; 119:98-\allowbreak100,\allowbreak105 Isa 8:20 2Pe 1:19}
\crossref{Prov}{6}{24}{Pr 2:16; 5:3; 7:5 Ec 7:26}
\crossref{Prov}{6}{25}{2Sa 11:2-\allowbreak5 Mt 5:28 Jas 1:14,\allowbreak15}
\crossref{Prov}{6}{26}{Pr 5:10; 29:3,\allowbreak8 Lu 15:13-\allowbreak15,\allowbreak30}
\crossref{Prov}{6}{27}{Job 31:9-\allowbreak12 Ho 7:4-\allowbreak7 Jas 3:5}
\crossref{Prov}{6}{28}{Isa 43:2}
\crossref{Prov}{6}{29}{Ge 12:18,\allowbreak19 Le 20:10 2Sa 11:3,\allowbreak4; 12:9,\allowbreak10; 16:21 Jer 5:8,\allowbreak9}
\crossref{Prov}{6}{30}{Pr 23:9,\allowbreak22; 30:17}
\crossref{Prov}{6}{31}{Ex 22:1,\allowbreak3,\allowbreak4 2Sa 12:6 Job 20:18 Lu 19:8}
\crossref{Prov}{6}{32}{Pr 7:7 Ge 39:9,\allowbreak10; 41:39 Ec 7:25,\allowbreak26 Jer 5:8,\allowbreak21 Ro 1:22-\allowbreak24}
\crossref{Prov}{6}{33}{Pr 5:9-\allowbreak11 Jud 16:19-\allowbreak21 Ps 38:1-\allowbreak8; 51:8}
\crossref{Prov}{6}{34}{Pr 27:4 Nu 5:14; 25:11 Jud 19:29,\allowbreak30 So 8:6 1Co 10:22}
\crossref{Prov}{6}{35}{Pr 4:3; 7:13; 8:25 2Ki 5:1}
\crossref{Prov}{7}{1}{Pr 1:8; 3:1}
\crossref{Prov}{7}{2}{Pr 4:13 Le 18:5 Isa 55:3 Joh 12:49,\allowbreak50; 14:21; 15:14 1Jo 2:3,\allowbreak4}
\crossref{Prov}{7}{3}{Pr 3:3; 6:21 De 6:8,\allowbreak9; 11:18-\allowbreak20 Isa 30:8 Jer 17:1; 31:33 2Co 3:3}
\crossref{Prov}{7}{4}{Pr 2:2-\allowbreak4; 4:6-\allowbreak8}
\crossref{Prov}{7}{5}{Pr 2:16; 5:3; 6:24}
\crossref{Prov}{7}{6}{Ge 26:8 2Sa 6:16}
\crossref{Prov}{7}{7}{Pr 1:4,\allowbreak22,\allowbreak32; 8:5; 14:15,\allowbreak18; 19:25; 22:3; 27:12 Ps 19:7; 119:130}
\crossref{Prov}{7}{8}{Pr 4:14,\allowbreak15; 5:8 Jud 16:1 2Sa 11:2,\allowbreak3 1Co 6:18 2Ti 2:22 Jude 1:23}
\crossref{Prov}{7}{9}{Ge 39:11 Job 24:13-\allowbreak15 Ro 13:12-\allowbreak14 Eph 5:11}
\crossref{Prov}{7}{10}{Ge 38:14,\allowbreak15 2Ki 9:22,\allowbreak30 Isa 3:16-\allowbreak24; 23:16 Jer 4:30 1Ti 2:9}
\crossref{Prov}{7}{11}{Pr 9:13; 25:24; 27:14,\allowbreak15; 31:10-\allowbreak31}
\crossref{Prov}{7}{12}{Pr 9:14; 23:28 Jer 2:20,\allowbreak33,\allowbreak36; 3:2 Eze 16:24,\allowbreak25,\allowbreak31 Re 18:3,\allowbreak23}
\crossref{Prov}{7}{13}{Ge 39:7,\allowbreak12 Nu 25:1,\allowbreak6-\allowbreak8; 31:16 Eze 16:33 Re 2:20}
\crossref{Prov}{7}{14}{2Sa 15:7-\allowbreak9 1Ki 21:9,\allowbreak10 Joh 18:28}
\crossref{Prov}{7}{15}{1Ti 5:13-\allowbreak15 Tit 2:5}
\crossref{Prov}{7}{16}{So 1:16; 3:7-\allowbreak10 Re 2:22}
\crossref{Prov}{7}{17}{So 3:6 Isa 57:7-\allowbreak9}
\crossref{Prov}{7}{18}{So 1:2; 2:3; 4:10}
\crossref{Prov}{7}{19}{Mt 20:11; 24:43 Lu 12:39}
\crossref{Prov}{7}{20}{}
\crossref{Prov}{7}{21}{7:5; 5:3 Jud 16:15-\allowbreak17 Ps 12:2}
\crossref{Prov}{7}{22}{Ac 14:13}
\crossref{Prov}{7}{23}{Nu 25:8,\allowbreak9}
\crossref{Prov}{7}{24}{Pr 4:1; 5:7; 8:32,\allowbreak33 1Co 4:14,\allowbreak15 Ga 4:19 1Jo 2:1}
\crossref{Prov}{7}{25}{Pr 4:14,\allowbreak15; 5:8; 6:25; 23:31-\allowbreak33 Mt 5:28}
\crossref{Prov}{7}{26}{Pr 6:33 Jud 16:21 2Sa 3:6-\allowbreak8,\allowbreak27; 12:9-\allowbreak11 1Ki 11:1,\allowbreak2 Ne 13:26}
\crossref{Prov}{7}{27}{Pr 2:18,\allowbreak19; 5:5; 9:18 Ec 7:26}
\crossref{Prov}{8}{1}{Pr 1:20,\allowbreak21; 9:1-\allowbreak3 Isa 49:1-\allowbreak6; 55:1-\allowbreak3 Mt 3:3; 4:17; 28:19,\allowbreak20}
\crossref{Prov}{8}{2}{Ge 37:7 Ex 7:15; 15:8 Re 3:20}
\crossref{Prov}{8}{3}{Mt 22:9 Lu 14:21-\allowbreak23 Joh 18:20 Ac 5:20}
\crossref{Prov}{8}{4}{Ps 49:1-\allowbreak3; 50:1 Mt 11:15 Joh 3:16 2Co 5:19,\allowbreak20 Col 1:23,\allowbreak28}
\crossref{Prov}{8}{5}{Pr 1:22; 9:4 Ps 19:7; 94:8 Isa 42:13; 55:1-\allowbreak3 Ac 26:18 1Co 1:28}
\crossref{Prov}{8}{6}{Pr 2:6,\allowbreak7; 4:2,\allowbreak20-\allowbreak22; 22:20,\allowbreak21 Ps 19:7-\allowbreak11; 49:3 1Co 2:6,\allowbreak7 Col 1:26}
\crossref{Prov}{8}{7}{Job 36:4 Joh 1:17; 8:14,\allowbreak45,\allowbreak46; 14:6; 17:17; 18:37 Re 3:14}
\crossref{Prov}{8}{8}{Ps 12:6 Isa 45:23; 63:1}
\crossref{Prov}{8}{9}{Pr 14:6; 15:14,\allowbreak24; 17:24; 18:1,\allowbreak2,\allowbreak15 Ps 19:7,\allowbreak8; 25:12-\allowbreak14; 119:98-\allowbreak100}
\crossref{Prov}{8}{10}{Pr 2:4,\allowbreak5; 3:13,\allowbreak14; 10:20; 16:16; 23:23 Ps 119:72,\allowbreak127,\allowbreak162 Ec 7:11}
\crossref{Prov}{8}{11}{Pr 3:14; 4:5-\allowbreak7; 16:16; 20:15 Job 28:15-\allowbreak19 Ps 19:10; 119:127}
\crossref{Prov}{8}{12}{Ps 104:24 Isa 55:8,\allowbreak9 Ro 11:33 Eph 1:8,\allowbreak11; 3:10 Col 2:3}
\crossref{Prov}{8}{13}{Pr 16:6 Ps 97:10; 101:3; 119:104,\allowbreak128 Am 5:15 Ro 12:9 1Th 5:22}
\crossref{Prov}{8}{14}{Isa 9:6; 40:14 Joh 1:9 Ro 11:33,\allowbreak34 1Co 1:24,\allowbreak30 Col 2:3}
\crossref{Prov}{8}{15}{1Sa 9:17; 16:1 1Ch 28:5 Jer 27:5-\allowbreak7 Da 2:21; 4:25,\allowbreak32; 5:18-\allowbreak31}
\crossref{Prov}{8}{16}{Pr 19:10; 28:2 Ge 12:15}
\crossref{Prov}{8}{17}{1Sa 2:30 Ps 91:14 Joh 14:21,\allowbreak23; 16:27 1Jo 4:19}
\crossref{Prov}{8}{18}{Pr 3:16; 4:7-\allowbreak9 Jas 2:5}
\crossref{Prov}{8}{19}{8:10; 3:14 Ec 7:12}
\crossref{Prov}{8}{20}{Pr 3:6; 4:11,\allowbreak12; 6:22 Ps 23:3; 25:4,\allowbreak5; 32:8 Isa 2:3; 49:10; 55:4}
\crossref{Prov}{8}{21}{8:18; 1:13; 6:31 Ge 15:14 1Sa 2:8 Mt 25:46 Joh 1:1-\allowbreak18 Ro 8:17}
\crossref{Prov}{8}{22}{Pr 3:19 Joh 1:1,\allowbreak2 Col 1:17}
\crossref{Prov}{8}{23}{Ge 1:26 Ps 2:6 Mic 5:2 Joh 17:24 Eph 1:10,\allowbreak11 1Jo 1:1,\allowbreak2}
\crossref{Prov}{8}{24}{Ps 2:7 Joh 1:14; 3:16; 5:20 Heb 1:5 1Jo 4:9}
\crossref{Prov}{8}{25}{Job 15:7,\allowbreak8; 38:4-\allowbreak11 Ps 90:2; 102:25-\allowbreak28 Heb 1:10}
\crossref{Prov}{8}{26}{Ge 1:1-\allowbreak31}
\crossref{Prov}{8}{27}{Ps 33:6; 103:19; 136:5 Jer 10:12 Col 1:16 Heb 1:2}
\crossref{Prov}{8}{28}{Pr 31:17 De 15:7}
\crossref{Prov}{8}{29}{Ge 1:9,\allowbreak10 Job 38:8-\allowbreak11 Ps 33:7; 104:9 Jer 5:22}
\crossref{Prov}{8}{30}{Joh 1:1-\allowbreak3,\allowbreak18; 16:28}
\crossref{Prov}{8}{31}{Ps 16:3; 40:6-\allowbreak8 Joh 4:34; 13:1 2Co 8:9}
\crossref{Prov}{8}{32}{Ps 1:1-\allowbreak4; 119:1,\allowbreak2; 128:1 Lu 11:28}
\crossref{Prov}{8}{33}{Pr 1:2,\allowbreak3,\allowbreak8; 4:1; 5:1 Isa 55:1-\allowbreak3 Ro 10:16,\allowbreak17}
\crossref{Prov}{8}{34}{Pr 1:21; 2:3,\allowbreak4 Ps 27:4; 84:10; 92:13 Mt 7:24 Lu 1:6; 10:39; 11:28}
\crossref{Prov}{8}{35}{Pr 1:33; 3:13-\allowbreak18 Joh 3:16,\allowbreak36; 14:6 Php 3:8 Col 3:3 1Jo 5:11,\allowbreak12}
\crossref{Prov}{8}{36}{Pr 1:31; 20:2 Joh 3:19,\allowbreak20 Ac 13:46 Heb 2:3; 10:29}
\crossref{Prov}{9}{1}{Mt 16:18 1Co 3:9-\allowbreak15 Eph 2:20-\allowbreak22 1Ti 3:15 Heb 3:3-\allowbreak6 1Pe 2:5,\allowbreak6}
\crossref{Prov}{9}{2}{Isa 25:6 Mt 22:3,\allowbreak4-\allowbreak14 1Co 5:7,\allowbreak8}
\crossref{Prov}{9}{3}{Mt 22:3,\allowbreak4,\allowbreak9 Lu 11:49; 14:17,\allowbreak21-\allowbreak23 Ro 10:15 2Co 5:20,\allowbreak21}
\crossref{Prov}{9}{4}{9:16; 1:22; 6:32; 8:5 Ps 19:7; 119:130 Mt 11:25 Re 3:17,\allowbreak18; 22:17}
\crossref{Prov}{9}{5}{9:2,\allowbreak17 Ps 22:26,\allowbreak29 So 5:1 Isa 55:1-\allowbreak3 Jer 31:12-\allowbreak14 Mt 26:26-\allowbreak28}
\crossref{Prov}{9}{6}{Pr 4:14,\allowbreak15; 13:20 Ps 26:4-\allowbreak6; 45:10; 119:115 Ac 2:40 2Co 6:17}
\crossref{Prov}{9}{7}{Pr 15:12 Ge 19:8,\allowbreak9 1Ki 18:17; 21:20; 22:24,\allowbreak27 2Ch 24:20-\allowbreak22}
\crossref{Prov}{9}{8}{Pr 23:9; 29:1 Nu 14:6-\allowbreak10 1Ki 22:8 Mt 7:6; 15:14 Heb 6:4-\allowbreak8}
\crossref{Prov}{9}{9}{Pr 1:5; 25:12 Ho 6:3 Mt 13:11,\allowbreak12 2Pe 3:18 1Jo 2:20,\allowbreak21; 5:13}
\crossref{Prov}{9}{10}{Pr 1:7 Job 28:28 Ps 111:10 Ec 12:13}
\crossref{Prov}{9}{11}{Pr 3:2,\allowbreak16; 10:27 De 6:2}
\crossref{Prov}{9}{12}{Pr 16:26 Job 22:2,\allowbreak3,\allowbreak21; 35:6,\allowbreak7 Isa 28:22 Eze 18:20 2Pe 3:3,\allowbreak4,\allowbreak16}
\crossref{Prov}{9}{13}{Pr 7:11; 21:9,\allowbreak19 1Ti 6:4}
\crossref{Prov}{9}{14}{Pr 7:10-\allowbreak12}
\crossref{Prov}{9}{15}{Pr 7:13-\allowbreak15,\allowbreak25-\allowbreak27; 23:27,\allowbreak28}
\crossref{Prov}{9}{16}{}
\crossref{Prov}{9}{17}{Pr 20:17; 23:31,\allowbreak32 Ge 3:6 Ro 7:8 Jas 1:14,\allowbreak15}
\crossref{Prov}{9}{18}{Pr 1:7 Ps 82:5 2Pe 3:5}
\crossref{Prov}{10}{1}{Pr 1:1; 25:1 1Ki 4:32 Ec 12:9}
\crossref{Prov}{10}{2}{Pr 11:4 Ps 49:6-\allowbreak10 Isa 10:2,\allowbreak3 Zep 1:18 Lu 12:15-\allowbreak21; 16:22,\allowbreak23}
\crossref{Prov}{10}{3}{Job 5:20 Ps 10:14; 33:19; 34:9,\allowbreak10; 37:3,\allowbreak19,\allowbreak25 Isa 33:16}
\crossref{Prov}{10}{4}{Pr 6:6-\allowbreak11; 11:24; 12:24; 19:15,\allowbreak24; 20:4,\allowbreak13; 24:30-\allowbreak34 Ec 10:18}
\crossref{Prov}{10}{5}{Pr 6:6,\allowbreak8; 30:25 Isa 55:6,\allowbreak7}
\crossref{Prov}{10}{6}{Pr 11:26; 24:25; 28:20 De 28:2 Job 29:13 2Ti 1:16-\allowbreak18}
\crossref{Prov}{10}{7}{1Ki 11:36 2Ki 19:34 2Ch 24:16 Ps 112:6 Mr 14:9 Lu 1:48}
\crossref{Prov}{10}{8}{Pr 1:5; 9:9; 12:1; 14:8 Ps 119:34 Jas 3:13}
\crossref{Prov}{10}{9}{Pr 28:18 Ps 23:4; 25:21; 26:11,\allowbreak12; 84:11 Isa 33:15,\allowbreak16 Ga 2:13,\allowbreak14}
\crossref{Prov}{10}{10}{Pr 6:13 Job 15:12 Ps 35:19}
\crossref{Prov}{10}{11}{10:20,\allowbreak21,\allowbreak32; 13:14; 15:7; 16:22-\allowbreak24; 18:4; 20:15 Ps 37:30,\allowbreak31 Eph 4:29}
\crossref{Prov}{10}{12}{Pr 15:18; 16:27; 28:25; 29:22 Jas 4:1}
\crossref{Prov}{10}{13}{10:11,\allowbreak21; 15:7,\allowbreak23; 20:15; 26:3 Ex 10:12 Isa 50:4 Lu 4:22}
\crossref{Prov}{10}{14}{Pr 1:5; 9:9; 18:1,\allowbreak15; 19:8 Mt 12:35; 13:44,\allowbreak52 2Co 4:6,\allowbreak7}
\crossref{Prov}{10}{15}{Pr 18:11 Job 31:24,\allowbreak25 Ps 49:6; 52:7 Ec 7:12 Jer 9:23 Mr 10:24}
\crossref{Prov}{10}{16}{Pr 11:30 Isa 3:10,\allowbreak11 Joh 6:27 1Co 15:58 Ga 6:7-\allowbreak9 Heb 6:10}
\crossref{Prov}{10}{17}{Pr 3:1,\allowbreak2,\allowbreak18; 4:4,\allowbreak13; 12:1; 22:17-\allowbreak19 Mt 7:24-\allowbreak27 Lu 11:28 Heb 2:1}
\crossref{Prov}{10}{18}{Pr 26:24-\allowbreak26 1Sa 18:21,\allowbreak22,\allowbreak29 2Sa 3:27; 11:8-\allowbreak15; 13:23-\allowbreak29; 20:9,\allowbreak10}
\crossref{Prov}{10}{19}{Ec 5:3; 10:13,\allowbreak14 Jas 3:2}
\crossref{Prov}{10}{20}{Pr 12:18; 15:4; 16:13; 25:11,\allowbreak12 Mt 12:35}
\crossref{Prov}{10}{21}{Pr 12:18; 15:4 Job 4:3,\allowbreak4; 23:12; 29:21,\allowbreak22 Ps 37:30 Ec 12:9,\allowbreak10}
\crossref{Prov}{10}{22}{Ge 12:2; 13:2; 14:23; 24:35; 26:12 De 8:17,\allowbreak18 1Sa 2:7,\allowbreak8 Ps 37:22}
\crossref{Prov}{10}{23}{}
\crossref{Prov}{10}{24}{Job 3:25; 15:21 Heb 10:27}
\crossref{Prov}{10}{25}{Pr 1:27 Job 27:19-\allowbreak21 Ps 37:9,\allowbreak10; 58:9; 73:18-\allowbreak20 Isa 40:24}
\crossref{Prov}{10}{26}{Pr 25:13,\allowbreak20}
\crossref{Prov}{10}{27}{Pr 3:2,\allowbreak16; 9:11 Ps 21:4; 34:11-\allowbreak13; 91:16}
\crossref{Prov}{10}{28}{Ps 16:9; 73:24-\allowbreak26 Ro 5:2; 12:12; 15:13 2Th 2:16}
\crossref{Prov}{10}{29}{Ps 84:7 Isa 40:31 Zec 10:12 Php 4:13}
\crossref{Prov}{10}{30}{10:25 Ps 16:8; 37:22,\allowbreak28,\allowbreak29; 112:6; 125:1 Ro 8:35-\allowbreak39 2Pe 1:10,\allowbreak11}
\crossref{Prov}{10}{31}{10:11,\allowbreak13,\allowbreak20,\allowbreak21 Ps 37:30}
\crossref{Prov}{10}{32}{Ec 12:10 Da 4:27 Tit 2:8}
\crossref{Prov}{11}{1}{Pr 16:11; 20:10,\allowbreak23 Le 19:35,\allowbreak36 De 25:13-\allowbreak16 Ho 12:7 Am 8:5,\allowbreak6}
\crossref{Prov}{11}{2}{Pr 3:34,\allowbreak35; 16:18,\allowbreak19 Da 4:30-\allowbreak32 Lu 14:8-\allowbreak11; 18:14}
\crossref{Prov}{11}{3}{11:5; 13:6 Ps 25:21; 26:1 Joh 7:17}
\crossref{Prov}{11}{4}{Pr 10:2 Job 36:18,\allowbreak19 Ps 49:6-\allowbreak8 Eze 7:19 Zep 1:18 Mt 16:26}
\crossref{Prov}{11}{5}{11:3; 1:31,\allowbreak32; 5:22 2Sa 17:23 Es 7:3-\allowbreak10 Ps 9:15,\allowbreak16 Mt 27:4,\allowbreak5}
\crossref{Prov}{11}{6}{Ge 30:33; 31:37 1Sa 12:3,\allowbreak4}
\crossref{Prov}{11}{7}{Pr 10:28; 14:32 Ex 15:9,\allowbreak10 Job 8:13,\allowbreak14; 11:20 Ps 146:4 Eze 28:9}
\crossref{Prov}{11}{8}{Pr 21:18 Es 7:9,\allowbreak10 Isa 43:3,\allowbreak4 Da 6:23,\allowbreak24}
\crossref{Prov}{11}{9}{Pr 2:10-\allowbreak16; 4:5,\allowbreak6; 6:23,\allowbreak24 Mr 13:14,\allowbreak22,\allowbreak23 Eph 4:13,\allowbreak14 2Pe 3:16-\allowbreak18}
\crossref{Prov}{11}{10}{Pr 28:12,\allowbreak28 Es 8:15,\allowbreak16}
\crossref{Prov}{11}{11}{Pr 14:34; 29:8 Ge 41:38-\allowbreak42; 45:8 2Ch 32:20-\allowbreak22 Job 22:30 Ec 9:15}
\crossref{Prov}{11}{12}{Jud 9:27-\allowbreak29,\allowbreak38 Ne 4:2-\allowbreak4 Ps 123:3,\allowbreak4 Lu 16:14; 18:9 Joh 7:48-\allowbreak52}
\crossref{Prov}{11}{13}{Pr 20:19 Le 19:16}
\crossref{Prov}{11}{14}{Pr 15:22; 16:22; 24:6 1Ki 12:1-\allowbreak19 Isa 19:11-\allowbreak14 Ac 15:6-\allowbreak21}
\crossref{Prov}{11}{15}{Pr 6:1-\allowbreak5; 17:18; 20:16; 22:26,\allowbreak27}
\crossref{Prov}{11}{16}{Pr 31:30,\allowbreak31 1Sa 25:32,\allowbreak33 2Sa 20:16-\allowbreak22 Es 9:25 Mt 26:13 Lu 8:3}
\crossref{Prov}{11}{17}{Ps 41:1-\allowbreak4; 112:4-\allowbreak9 Isa 32:7,\allowbreak8; 57:1; 58:7-\allowbreak12 Da 4:27 Mt 5:7}
\crossref{Prov}{11}{18}{Pr 1:18; 5:22 Job 27:13-\allowbreak23 Ec 10:8 Isa 59:5-\allowbreak8 Eph 4:22}
\crossref{Prov}{11}{19}{11:4; 10:16; 12:28; 19:23 Ac 10:35 1Jo 3:7,\allowbreak10}
\crossref{Prov}{11}{20}{Pr 6:14,\allowbreak16-\allowbreak19; 8:13 Ps 18:25,\allowbreak26}
\crossref{Prov}{11}{21}{Pr 16:5 Ex 23:2}
\crossref{Prov}{11}{22}{Pr 31:30 Eze 16:15-\allowbreak22 Na 3:4-\allowbreak6 1Pe 3:3,\allowbreak4 2Pe 2:22}
\crossref{Prov}{11}{23}{Ps 10:17; 27:4; 37:4; 39:7,\allowbreak8; 119:5,\allowbreak10 Isa 26:9 Jer 17:16 Mt 5:6}
\crossref{Prov}{11}{24}{11:18; 19:17; 28:8 De 15:10 Ps 112:9 Ec 11:1,\allowbreak2,\allowbreak6 Lu 6:38}
\crossref{Prov}{11}{25}{Pr 28:27 Job 29:13-\allowbreak18; 31:16-\allowbreak20 Isa 32:8; 58:7-\allowbreak11 Mt 5:7; 25:34,\allowbreak35}
\crossref{Prov}{11}{26}{Am 8:4-\allowbreak6}
\crossref{Prov}{11}{27}{Pr 17:11 Es 7:10 Ps 7:15,\allowbreak16; 9:15,\allowbreak16; 10:2; 57:6}
\crossref{Prov}{11}{28}{Pr 10:15 De 8:12-\allowbreak14 Job 31:24,\allowbreak25 Ps 52:7; 62:10 Mr 10:24 Lu 12:20}
\crossref{Prov}{11}{29}{Ge 34:30 Jos 7:24,\allowbreak25 1Sa 25:3,\allowbreak17,\allowbreak38 Hab 2:9,\allowbreak10}
\crossref{Prov}{11}{30}{Pr 3:18; 15:4}
\crossref{Prov}{11}{31}{2Sa 7:14,\allowbreak15; 12:9-\allowbreak12 1Ki 13:24 Jer 25:29 1Co 11:30-\allowbreak32}
\crossref{Prov}{12}{1}{Pr 2:10,\allowbreak11; 8:17,\allowbreak32; 18:1 Ps 119:27,\allowbreak97-\allowbreak100 2Th 2:10}
\crossref{Prov}{12}{2}{Pr 8:35 Ps 112:5 Ec 8:8 Ac 11:24 Ro 5:7}
\crossref{Prov}{12}{3}{Pr 10:25 Job 5:3-\allowbreak5; 15:29; 20:5-\allowbreak9; 27:16-\allowbreak18}
\crossref{Prov}{12}{4}{Pr 14:1; 19:13,\allowbreak14; 31:10-\allowbreak25 1Co 11:7,\allowbreak11}
\crossref{Prov}{12}{5}{Pr 11:23; 24:9 Ps 119:15; 139:23 Isa 55:7 Jer 4:14}
\crossref{Prov}{12}{6}{Pr 1:11-\allowbreak19 2Sa 17:1-\allowbreak4 Isa 59:7 Jer 5:26 Mic 7:1,\allowbreak2 Ac 23:12,\allowbreak15}
\crossref{Prov}{12}{7}{Pr 11:21; 14:11; 15:25 Es 9:6-\allowbreak10,\allowbreak14 Job 5:3,\allowbreak4; 11:20; 18:15-\allowbreak20}
\crossref{Prov}{12}{8}{Ge 41:39 1Sa 16:18; 18:30 Ec 8:1 Lu 12:42-\allowbreak44; 16:8 1Co 3:10-\allowbreak15}
\crossref{Prov}{12}{9}{Pr 13:7 Lu 14:11}
\crossref{Prov}{12}{10}{Ge 33:13,\allowbreak14 Nu 22:28-\allowbreak32 De 25:4 Joh 4:11}
\crossref{Prov}{12}{11}{Pr 13:23; 14:4,\allowbreak23; 27:27; 28:19 Ge 3:19 Ps 128:2 Eph 4:28}
\crossref{Prov}{12}{12}{Pr 1:17-\allowbreak19; 29:5,\allowbreak6 Ps 9:15; 10:9 Jer 5:26-\allowbreak28 Mic 7:2 Hab 1:15-\allowbreak17}
\crossref{Prov}{12}{13}{Pr 11:8 Ge 48:16 2Sa 4:9 Ps 34:19 Ec 7:18 Ro 8:35-\allowbreak37 2Pe 2:9}
\crossref{Prov}{12}{14}{Pr 13:2; 18:20,\allowbreak21 Ps 63:5}
\crossref{Prov}{12}{15}{Pr 3:7; 14:16; 16:2,\allowbreak25; 26:12,\allowbreak16; 28:11; 30:12 Lu 18:11 Ga 6:3}
\crossref{Prov}{12}{16}{Pr 25:28; 29:11 1Sa 20:30-\allowbreak34 1Ki 19:1,\allowbreak2}
\crossref{Prov}{12}{17}{Pr 14:5,\allowbreak25 1Sa 22:14,\allowbreak15}
\crossref{Prov}{12}{18}{Pr 25:18 Ps 52:2; 57:4; 59:7; 64:3 Jas 3:6-\allowbreak8}
\crossref{Prov}{12}{19}{Zec 1:4-\allowbreak6 Mt 24:35}
\crossref{Prov}{12}{20}{12:12; 26:24-\allowbreak26 Jer 17:16 Mr 7:21,\allowbreak22; 12:14-\allowbreak17 Ro 1:29}
\crossref{Prov}{12}{21}{Ro 8:28 1Co 3:22,\allowbreak23 2Co 4:17}
\crossref{Prov}{12}{22}{Pr 6:16,\allowbreak17 Ps 5:6 Isa 9:15 Eze 13:19,\allowbreak22 Re 21:8; 22:15}
\crossref{Prov}{12}{23}{Pr 10:19; 11:13; 13:16}
\crossref{Prov}{12}{24}{Pr 10:4; 13:4; 17:2; 22:29 1Ki 11:28; 12:20}
\crossref{Prov}{12}{25}{Pr 14:10; 15:13,\allowbreak15,\allowbreak23; 17:22; 18:14 Ne 2:1,\allowbreak2 Ps 38:6; 42:11}
\crossref{Prov}{12}{26}{12:13; 17:27 Ps 16:3 Mt 5:46-\allowbreak48 Lu 6:32-\allowbreak36 1Pe 2:18-\allowbreak21}
\crossref{Prov}{12}{27}{Pr 13:4; 23:2; 26:15}
\crossref{Prov}{12}{28}{Pr 8:35; 9:11; 10:16; 11:19 Eze 18:9,\allowbreak20-\allowbreak24 Ro 5:21; 6:22,\allowbreak23}
\crossref{Prov}{13}{1}{Pr 4:1-\allowbreak14,\allowbreak20-\allowbreak22; 10:1; 15:5,\allowbreak20}
\crossref{Prov}{13}{2}{Pr 12:14; 18:20}
\crossref{Prov}{13}{3}{Pr 10:19; 12:13; 21:23 Ps 39:1 Mt 12:36,\allowbreak37 Jas 1:26; 3:2-\allowbreak12}
\crossref{Prov}{13}{4}{Pr 10:4; 12:11,\allowbreak24; 26:13 Nu 23:10}
\crossref{Prov}{13}{5}{Pr 6:17; 30:8 Ps 119:163 Eph 4:25 Col 3:9}
\crossref{Prov}{13}{6}{Pr 11:3,\allowbreak5,\allowbreak6 Ps 15:2; 25:21; 26:1}
\crossref{Prov}{13}{7}{13:11; 12:9 Lu 18:11-\allowbreak14 1Co 4:8 2Pe 2:19 Re 3:17}
\crossref{Prov}{13}{8}{Pr 6:35 Ex 21:30 Job 2:4 Ps 49:6-\allowbreak10 Jer 41:8 Mt 16:26}
\crossref{Prov}{13}{9}{Pr 4:18 1Ki 11:36 Ps 97:11; 112:4}
\crossref{Prov}{13}{10}{Pr 21:24 Jud 12:1-\allowbreak6 1Ki 12:10,\allowbreak11,\allowbreak16 2Ki 14:10 Lu 22:24 1Ti 6:4}
\crossref{Prov}{13}{11}{Pr 10:2; 20:21; 28:8 Job 15:28,\allowbreak29; 20:15,\allowbreak19-\allowbreak22; 27:16,\allowbreak17 Ec 5:14}
\crossref{Prov}{13}{12}{Ps 42:1-\allowbreak3; 69:3; 119:81-\allowbreak83; 143:7 So 5:8}
\crossref{Prov}{13}{13}{Pr 1:25,\allowbreak30,\allowbreak31 2Sa 12:9,\allowbreak10 2Ch 36:16 Jer 43:2; 44:16,\allowbreak17}
\crossref{Prov}{13}{14}{Pr 9:11; 10:11; 14:27; 16:22}
\crossref{Prov}{13}{15}{Pr 3:4; 14:35 1Sa 18:14-\allowbreak16 Lu 2:52 Ac 7:10}
\crossref{Prov}{13}{16}{Pr 12:22,\allowbreak23; 15:2; 21:24 Ps 112:5 Isa 52:13 Mt 10:16 Ro 16:19}
\crossref{Prov}{13}{17}{Pr 10:26; 26:6 Jer 23:13-\allowbreak16,\allowbreak28 Eze 3:18; 33:7,\allowbreak8 2Co 2:17}
\crossref{Prov}{13}{18}{13:13; 5:9-\allowbreak14; 12:1; 15:5,\allowbreak31,\allowbreak32; 19:6 Jer 5:3-\allowbreak9 Heb 12:25}
\crossref{Prov}{13}{19}{13:12 1Ki 1:48 Ps 21:1,\allowbreak2 So 3:4 2Ti 4:7,\allowbreak8 Re 7:14-\allowbreak17}
\crossref{Prov}{13}{20}{Pr 2:20 Ps 119:63 So 1:7,\allowbreak8 Mal 3:16 Ac 2:42 Heb 10:24}
\crossref{Prov}{13}{21}{Ge 4:7 Nu 32:23 Ps 32:10; 140:11 Ac 28:4}
\crossref{Prov}{13}{22}{Ge 17:7,\allowbreak8 Ps 25:12,\allowbreak13; 102:28; 112:2; 128:6}
\crossref{Prov}{13}{23}{Pr 12:11,\allowbreak14; 27:18,\allowbreak23-\allowbreak27; 28:19 Ec 5:9}
\crossref{Prov}{13}{24}{Pr 3:12; 8:36; 19:18; 22:15; 23:13,\allowbreak14; 29:15,\allowbreak17 Lu 14:26 Heb 12:6-\allowbreak8}
\crossref{Prov}{13}{25}{Ps 34:10; 37:3,\allowbreak16,\allowbreak18,\allowbreak19 1Ti 4:8 Heb 13:5}
\crossref{Prov}{14}{1}{Pr 24:3,\allowbreak4; 31:10-\allowbreak31 Ru 4:11}
\crossref{Prov}{14}{2}{Pr 16:17; 28:6 1Ki 3:6 Job 1:1; 28:28 Ps 25:21; 112:1 Ec 12:13}
\crossref{Prov}{14}{3}{Pr 18:6; 21:24; 22:8; 28:25 1Sa 2:3 Job 5:21 Ps 12:3; 31:18; 52:1,\allowbreak2}
\crossref{Prov}{14}{4}{Am 4:6}
\crossref{Prov}{14}{5}{14:25; 6:19; 12:17; 13:5; 19:5,\allowbreak9 Ex 20:16; 23:1 1Ki 21:13; 22:12-\allowbreak14}
\crossref{Prov}{14}{6}{Pr 18:2; 26:12 Isa 8:20 Jer 8:9 Mt 6:22,\allowbreak23; 11:25-\allowbreak27 Ro 1:21-\allowbreak28}
\crossref{Prov}{14}{7}{Pr 9:6; 13:20; 19:27 1Co 5:11 Eph 5:11}
\crossref{Prov}{14}{8}{Pr 2:9; 8:20 Ps 111:10; 119:5,\allowbreak34,\allowbreak35,\allowbreak73; 143:8 Eph 5:17 Col 1:9,\allowbreak10}
\crossref{Prov}{14}{9}{Pr 1:22; 10:23; 26:18,\allowbreak19; 30:20 Job 15:16; 34:7-\allowbreak9 Jude 1:18}
\crossref{Prov}{14}{10}{Pr 15:13; 18:14 1Sa 1:10 2Ki 4:27 Job 6:2-\allowbreak4; 7:11; 9:18; 10:1}
\crossref{Prov}{14}{11}{Pr 3:33; 12:7; 21:12 Job 8:15; 15:34; 18:14,\allowbreak15,\allowbreak21; 20:26-\allowbreak28; 21:28}
\crossref{Prov}{14}{12}{Pr 12:15; 16:25; 30:12 Mt 7:13,\allowbreak14 Lu 13:24 Ro 6:21 Ga 6:3 Eph 5:6}
\crossref{Prov}{14}{13}{Pr 5:4 Ec 2:2,\allowbreak10,\allowbreak11; 7:5,\allowbreak6; 11:9 Lu 16:25 Jas 4:9 Re 18:7,\allowbreak8}
\crossref{Prov}{14}{14}{Pr 1:32 Jer 2:19; 8:5; 17:5 Ho 4:16 Zep 1:6 Heb 3:12 2Pe 2:20-\allowbreak22}
\crossref{Prov}{14}{15}{Pr 4:26; 22:3; 27:12 Ro 16:18,\allowbreak19 Eph 4:14; 5:17 1Jo 4:1}
\crossref{Prov}{14}{16}{Pr 3:7; 16:6,\allowbreak17; 22:3 Ge 33:9; 42:18 Ne 5:15 Job 31:21-\allowbreak23}
\crossref{Prov}{14}{17}{14:29; 12:16; 15:18; 16:32; 22:24; 29:22 Ec 7:9 Jas 1:19}
\crossref{Prov}{14}{18}{Pr 3:35; 11:29 Jer 16:19; 44:17 Mt 23:29-\allowbreak32 1Pe 1:18}
\crossref{Prov}{14}{19}{Ge 42:6; 43:28 Ex 8:8; 9:27,\allowbreak28; 11:8 2Ki 3:12 Es 7:7,\allowbreak8 Ps 49:14}
\crossref{Prov}{14}{20}{Pr 10:15; 19:7 Job 6:21-\allowbreak23; 19:13,\allowbreak14; 30:10}
\crossref{Prov}{14}{21}{Pr 11:12; 17:5; 18:3 Job 31:13-\allowbreak15; 35:5,\allowbreak6 Ps 22:24 Lu 18:9}
\crossref{Prov}{14}{22}{14:17; 12:2 Isa 32:7,\allowbreak8}
\crossref{Prov}{14}{23}{Pr 12:24; 28:19 Joh 6:27 Heb 6:10,\allowbreak11}
\crossref{Prov}{14}{24}{Ps 112:9 Ec 7:11,\allowbreak12 Isa 33:6 Lu 16:9}
\crossref{Prov}{14}{25}{14:5 Ac 20:21,\allowbreak26,\allowbreak27; 26:16-\allowbreak20 1Ti 4:1-\allowbreak3 2Pe 3:3}
\crossref{Prov}{14}{26}{Pr 3:7,\allowbreak8,\allowbreak25,\allowbreak26; 19:23 Ge 31:42 Ps 34:7-\allowbreak11; 112:1,\allowbreak6-\allowbreak8; 115:13,\allowbreak14}
\crossref{Prov}{14}{27}{Pr 13:14 Isa 33:6 Re 21:6}
\crossref{Prov}{14}{28}{Ex 1:12,\allowbreak22 1Ki 4:20,\allowbreak21; 20:27 2Ki 10:32,\allowbreak33; 13:7}
\crossref{Prov}{14}{29}{14:17; 15:18; 16:32 Nu 12:3 Mt 11:29 1Co 13:4,\allowbreak5 Jas 1:19; 3:17,\allowbreak18}
\crossref{Prov}{14}{30}{Pr 4:23 Ps 119:80 2Ti 1:7}
\crossref{Prov}{14}{31}{Pr 17:5; 22:2,\allowbreak16,\allowbreak22,\allowbreak23 Job 31:13-\allowbreak16 Ps 12:5 Ec 5:8 Mt 25:40-\allowbreak46}
\crossref{Prov}{14}{32}{Job 18:18; 27:20-\allowbreak22 Ps 58:9 Da 5:26-\allowbreak30 Joh 8:21,\allowbreak24 Ro 9:22}
\crossref{Prov}{14}{33}{Pr 12:16,\allowbreak23; 13:16; 15:2,\allowbreak28; 29:11 Ec 10:3}
\crossref{Prov}{14}{34}{De 4:6-\allowbreak8; 28:1-\allowbreak14 Jud 2:6-\allowbreak14 Jer 2:2-\allowbreak25 Ho 13:1}
\crossref{Prov}{14}{35}{Pr 19:12,\allowbreak13; 20:8,\allowbreak26; 22:11; 25:5; 29:12 Ps 101:4-\allowbreak8 Mt 24:45-\allowbreak51}
\crossref{Prov}{15}{1}{Pr 25:15 Jud 8:1-\allowbreak3 1Sa 25:21-\allowbreak33}
\crossref{Prov}{15}{2}{15:23,\allowbreak28; 12:23; 13:16; 16:23; 25:11,\allowbreak12 Ps 45:1 Ec 10:12,\allowbreak13 Isa 50:4}
\crossref{Prov}{15}{3}{Pr 5:21 2Ch 16:9 Job 34:21,\allowbreak22 Jer 16:17; 23:24; 32:19 Heb 4:13}
\crossref{Prov}{15}{4}{Pr 12:18; 16:24 Mal 4:2}
\crossref{Prov}{15}{5}{Pr 10:1; 13:1,\allowbreak18 1Sa 2:23-\allowbreak25 2Sa 15:1-\allowbreak6 1Ch 22:11-\allowbreak13; 28:9,\allowbreak20}
\crossref{Prov}{15}{6}{15:16; 8:21; 13:22; 21:20 Ps 112:3 Heb 11:26}
\crossref{Prov}{15}{7}{Ps 37:30; 45:2; 51:13-\allowbreak15; 71:15-\allowbreak18; 78:2-\allowbreak6; 119:13 Ec 12:9,\allowbreak10}
\crossref{Prov}{15}{8}{Pr 21:27; 28:9 Isa 1:10-\allowbreak15; 61:8; 66:3 Jer 6:20; 7:21-\allowbreak23 Am 5:21,\allowbreak22}
\crossref{Prov}{15}{9}{Pr 4:19; 21:4,\allowbreak8 Ps 1:6; 146:8,\allowbreak9 Mt 7:13 Jer 44:4 Hab 1:13}
\crossref{Prov}{15}{10}{Pr 12:1; 13:1; 23:35 1Ki 18:17; 21:20; 22:8 Joh 3:20; 7:7}
\crossref{Prov}{15}{11}{Pr 27:20 Job 26:6 Ps 139:8 Re 1:18}
\crossref{Prov}{15}{12}{15:10; 9:7,\allowbreak8 Am 5:10 Joh 3:18-\allowbreak21; 7:7 2Ti 4:3}
\crossref{Prov}{15}{13}{15:15; 17:22 2Co 1:12}
\crossref{Prov}{15}{14}{Pr 1:5; 9:9 1Ki 3:6-\allowbreak12 Ps 119:97,\allowbreak100 Ac 17:11 2Pe 3:18}
\crossref{Prov}{15}{15}{Ge 37:35; 47:9 Ps 90:7-\allowbreak9}
\crossref{Prov}{15}{16}{Pr 16:8; 28:6 Ps 37:16 1Ti 6:6}
\crossref{Prov}{15}{17}{Pr 17:1; 21:19 Ps 133:1-\allowbreak3 Php 2:1 1Jo 4:16}
\crossref{Prov}{15}{18}{Pr 10:12; 26:21; 28:25; 29:22 2Sa 19:43; 20:1 Jas 3:14-\allowbreak16}
\crossref{Prov}{15}{19}{Pr 22:5,\allowbreak13; 26:13 Nu 14:1-\allowbreak3,\allowbreak7-\allowbreak9}
\crossref{Prov}{15}{20}{Pr 10:1; 23:15,\allowbreak16; 29:3 1Ki 1:48; 2:9; 5:7 Php 2:22}
\crossref{Prov}{15}{21}{Pr 10:23; 14:9; 26:18,\allowbreak19}
\crossref{Prov}{15}{22}{Pr 11:14; 20:18 Ec 8:6}
\crossref{Prov}{15}{23}{Pr 12:14; 16:13; 24:26; 25:11,\allowbreak12 Eph 4:29}
\crossref{Prov}{15}{24}{Pr 6:23 Ps 16:11; 139:24 Jer 21:8 Mt 7:14 Joh 14:6}
\crossref{Prov}{15}{25}{Pr 12:7; 14:11 Job 40:11-\allowbreak13 Ps 52:1,\allowbreak5; 138:6 Isa 2:12 Da 5:20}
\crossref{Prov}{15}{26}{Pr 6:16-\allowbreak19; 24:9 Jer 4:14 Mt 15:19}
\crossref{Prov}{15}{27}{Pr 1:19; 11:19,\allowbreak29; 20:21 De 7:26 Jos 6:18; 7:11,\allowbreak12,\allowbreak24,\allowbreak25 1Sa 8:3-\allowbreak5}
\crossref{Prov}{15}{28}{15:2; 16:23 1Ki 3:23-\allowbreak28 Ec 5:2,\allowbreak6 1Pe 3:15}
\crossref{Prov}{15}{29}{Ps 10:1; 34:16; 73:27; 138:6 Mt 25:46 Eph 2:12,\allowbreak13}
\crossref{Prov}{15}{30}{Pr 13:9 Ezr 9:8 Ec 11:7 Re 21:23; 22:5}
\crossref{Prov}{15}{31}{15:5; 1:23; 9:8,\allowbreak9; 13:20; 19:20; 25:12 Isa 55:3}
\crossref{Prov}{15}{32}{Pr 1:24-\allowbreak33; 5:11,\allowbreak12; 8:33-\allowbreak36 Ps 50:17 Heb 12:15}
\crossref{Prov}{15}{33}{Pr 1:7; 8:13 Job 28:28 Ps 34:11; 111:10}
\crossref{Prov}{16}{1}{16:9; 19:21; 20:24; 21:1 2Ch 18:31 Ezr 7:27 Ne 1:11 Ps 10:17; 119:36}
\crossref{Prov}{16}{2}{16:25; 21:2; 30:12 1Sa 15:13,\allowbreak14 Ps 36:2 Jer 2:22,\allowbreak23 Lu 18:9-\allowbreak11}
\crossref{Prov}{16}{3}{Job 5:8 Ps 37:4,\allowbreak5; 55:22 Mt 6:25-\allowbreak34 Lu 12:22 Php 4:6 1Pe 5:7}
\crossref{Prov}{16}{4}{Isa 43:7,\allowbreak21 Ro 11:36 Re 4:11}
\crossref{Prov}{16}{5}{Pr 6:16,\allowbreak17; 8:13 Job 40:12 Jas 4:6}
\crossref{Prov}{16}{6}{Pr 20:28 Ps 85:10 Da 4:27 Mic 7:18-\allowbreak20 Lu 11:41 Joh 15:2 Ac 15:9}
\crossref{Prov}{16}{7}{Ps 69:31 Ro 8:31 Php 4:18 Col 1:10; 3:20 Heb 13:21 1Jo 3:22}
\crossref{Prov}{16}{8}{Pr 15:16 Ps 37:16 1Ti 6:6-\allowbreak9}
\crossref{Prov}{16}{9}{16:1; 19:21; 20:24; 21:30 Ps 37:23 Isa 46:10 Jer 10:23}
\crossref{Prov}{16}{10}{16:12,\allowbreak13 Ge 44:5,\allowbreak15 De 17:18-\allowbreak20 2Sa 23:3,\allowbreak4 Ps 45:6,\allowbreak7; 72:1-\allowbreak4}
\crossref{Prov}{16}{11}{Pr 11:1; 20:10,\allowbreak23 Le 19:35,\allowbreak36 De 25:13-\allowbreak15 Eze 45:10 Ho 12:7}
\crossref{Prov}{16}{12}{Pr 28:9 De 25:16 Lu 12:48}
\crossref{Prov}{16}{13}{Pr 14:35; 22:11 Ps 101:5-\allowbreak7}
\crossref{Prov}{16}{14}{Pr 19:12; 20:2 Da 3:13-\allowbreak25 Lu 12:4,\allowbreak5}
\crossref{Prov}{16}{15}{Pr 19:12 Job 29:23,\allowbreak24 Ps 4:6; 21:6 Ac 2:28}
\crossref{Prov}{16}{16}{Pr 3:15-\allowbreak18; 4:7; 8:10,\allowbreak11,\allowbreak19 Job 28:13-\allowbreak28 Ps 119:127 Ec 7:12}
\crossref{Prov}{16}{17}{Pr 4:24-\allowbreak27 Isa 35:8 Ac 10:35; 24:16 Tit 2:10-\allowbreak14}
\crossref{Prov}{16}{18}{Pr 11:2; 17:19; 18:12; 29:23 Es 3:5; 6:6; 7:10 Isa 2:11,\allowbreak12}
\crossref{Prov}{16}{19}{Ps 34:18; 138:6 Isa 57:15 Mt 5:3 Lu 1:51-\allowbreak53; 18:13,\allowbreak14}
\crossref{Prov}{16}{20}{Pr 8:35; 13:15; 17:2; 19:8; 24:3-\allowbreak5 Ge 41:38-\allowbreak40 Da 1:19-\allowbreak21 Mt 10:16}
\crossref{Prov}{16}{21}{16:23; 10:8; 23:15 1Ki 3:12 Ro 16:19 Jas 3:17}
\crossref{Prov}{16}{22}{Pr 10:11; 13:14; 14:27; 18:4 Joh 5:24; 6:63,\allowbreak68}
\crossref{Prov}{16}{23}{Pr 15:28; 22:17,\allowbreak18 Ps 37:30,\allowbreak31; 45:1 Mt 12:34,\allowbreak35 Col 3:16}
\crossref{Prov}{16}{24}{Pr 12:18; 15:23,\allowbreak26; 23:16; 25:11,\allowbreak12; 27:9 De 32:2 So 4:11}
\crossref{Prov}{16}{25}{Pr 12:26; 14:12 Isa 28:15-\allowbreak19 Joh 7:47-\allowbreak49; 9:40 Ac 26:9 2Co 13:5}
\crossref{Prov}{16}{26}{Pr 9:12; 14:23 Ec 6:7 1Th 4:11,\allowbreak12 2Th 3:8-\allowbreak12}
\crossref{Prov}{16}{27}{1Sa 25:17 2Sa 20:1}
\crossref{Prov}{16}{28}{Pr 6:14,\allowbreak19; 15:18; 18:8; 26:20-\allowbreak22; 29:22; 30:33 1Ti 6:3-\allowbreak5 Jas 3:14-\allowbreak16}
\crossref{Prov}{16}{29}{Pr 1:10-\allowbreak14; 2:12-\allowbreak15; 3:31 1Sa 19:11,\allowbreak17; 22:7-\allowbreak9; 23:19-\allowbreak21 Ne 6:13}
\crossref{Prov}{16}{30}{Pr 6:12-\allowbreak14; 10:10 Isa 6:10 Mt 13:15 Joh 3:20}
\crossref{Prov}{16}{31}{Pr 20:29 Le 19:32 Job 32:6,\allowbreak7}
\crossref{Prov}{16}{32}{Pr 14:29; 15:18; 19:11 Ps 103:8 Eph 5:1 Jas 1:19}
\crossref{Prov}{16}{33}{Nu 26:55-\allowbreak65 Jos 7:14; 18:5,\allowbreak10 1Sa 14:41,\allowbreak42 Ne 11:1 Jon 1:7}
\crossref{Prov}{17}{1}{Pr 15:17 Ps 37:16}
\crossref{Prov}{17}{2}{Pr 11:29; 14:35 Ge 24:4-\allowbreak67 Ec 4:13}
\crossref{Prov}{17}{3}{Pr 27:21 Ps 26:2; 66:10 Isa 48:10 Jer 17:10 Zec 13:9 Mal 3:2,\allowbreak3}
\crossref{Prov}{17}{4}{Pr 28:4 1Sa 22:7-\allowbreak11 1Ki 22:6-\allowbreak28 Isa 30:10 Jer 5:31 2Ti 4:3,\allowbreak4}
\crossref{Prov}{17}{5}{Pr 14:21,\allowbreak31 Ps 69:9 1Jo 3:17}
\crossref{Prov}{17}{6}{Ge 50:23 Job 42:16,\allowbreak17 Ps 127:3-\allowbreak5; 128:3-\allowbreak6}
\crossref{Prov}{17}{7}{Pr 26:7 Ps 50:16,\allowbreak17 Mt 7:5}
\crossref{Prov}{17}{8}{17:23 Ps 18:16; 19:6; 21:13; 29:4 Ex 23:8 De 16:19}
\crossref{Prov}{17}{9}{Pr 10:12 Ps 32:1 1Pe 4:8}
\crossref{Prov}{17}{10}{Pr 9:8,\allowbreak9; 13:1; 15:5; 19:25; 27:22; 29:19 Ps 141:5 Re 3:19}
\crossref{Prov}{17}{11}{2Sa 15:12; 16:5-\allowbreak9; 18:15,\allowbreak19; 20:1,\allowbreak22 1Ki 2:24,\allowbreak25,\allowbreak31,\allowbreak46 Mt 21:41}
\crossref{Prov}{17}{12}{Pr 28:15 2Sa 17:8 2Ki 2:24 Ho 13:8}
\crossref{Prov}{17}{13}{1Sa 24:17; 31:2,\allowbreak3 2Sa 21:1-\allowbreak14 Ps 35:12; 38:20; 55:12-\allowbreak15; 109:4-\allowbreak13}
\crossref{Prov}{17}{14}{17:19; 26:21; 29:22 Jud 12:1-\allowbreak6 2Sa 2:14-\allowbreak17; 19:41-\allowbreak43; 20:1-\allowbreak22}
\crossref{Prov}{17}{15}{Pr 24:23,\allowbreak24 Ex 23:7 1Ki 21:13 Isa 5:23; 55:8,\allowbreak9 Eze 22:27-\allowbreak29}
\crossref{Prov}{17}{16}{Pr 1:22,\allowbreak23; 8:4,\allowbreak5; 9:4-\allowbreak6 Isa 55:1-\allowbreak3 Ac 13:46 2Co 6:1}
\crossref{Prov}{17}{17}{Pr 18:24; 19:7 Ru 1:16 1Sa 18:3; 19:2; 20:17; 23:16 2Sa 1:26; 9:1-\allowbreak13}
\crossref{Prov}{17}{18}{Pr 6:1-\allowbreak5; 11:15; 20:16; 22:16,\allowbreak27}
\crossref{Prov}{17}{19}{17:14; 29:9,\allowbreak22 2Co 12:20 Jas 1:20; 3:14-\allowbreak16}
\crossref{Prov}{17}{20}{Pr 3:32; 6:12-\allowbreak15; 8:13 Ps 18:26}
\crossref{Prov}{17}{21}{17:25; 10:1; 15:20; 19:13 Ge 26:34 1Sa 2:32-\allowbreak35; 8:3 2Sa 18:33}
\crossref{Prov}{17}{22}{Pr 12:25; 15:13; 18:14 Ec 9:7-\allowbreak9 Ro 5:2-\allowbreak5}
\crossref{Prov}{17}{23}{17:8; 18:16; 21:14 Ex 23:8 De 16:19 1Sa 8:3; 12:3 Isa 1:23}
\crossref{Prov}{17}{24}{Pr 14:6; 15:14 Ec 2:14; 8:1 Joh 7:17}
\crossref{Prov}{17}{25}{Pr 10:1; 15:20; 19:13 2Sa 13:1-\allowbreak22 Ec 2:18,\allowbreak19}
\crossref{Prov}{17}{26}{17:15; 18:5 Ge 18:25}
\crossref{Prov}{17}{27}{Pr 10:19; 15:28 Jas 1:19; 3:2}
\crossref{Prov}{17}{28}{Pr 15:2 Job 13:5 Ec 5:3; 10:3,\allowbreak14}
\crossref{Prov}{18}{1}{Pr 2:1-\allowbreak6 Mt 13:11,\allowbreak44 Mr 4:11 Eph 5:15-\allowbreak17}
\crossref{Prov}{18}{2}{Pr 1:7,\allowbreak22; 17:16 Ps 1:1,\allowbreak2 Mt 8:34 1Co 8:1}
\crossref{Prov}{18}{3}{Pr 11:2; 22:10; 29:16 1Sa 20:30 Ne 4:4 Ps 69:9,\allowbreak20; 123:3,\allowbreak4}
\crossref{Prov}{18}{4}{Pr 10:11; 13:14; 16:22; 20:5 Mt 12:34 Joh 4:14; 7:38,\allowbreak39 Col 3:16; 4:6}
\crossref{Prov}{18}{5}{Pr 24:23; 28:21 Le 19:15 De 1:16,\allowbreak17; 16:19 Job 13:7,\allowbreak8; 34:19}
\crossref{Prov}{18}{6}{Pr 12:16; 13:10; 14:16; 16:27,\allowbreak28; 17:14; 20:3; 27:3}
\crossref{Prov}{18}{7}{Pr 10:8,\allowbreak14; 12:13; 13:3 Ec 10:11-\allowbreak14}
\crossref{Prov}{18}{8}{Pr 12:18; 16:28; 26:20-\allowbreak22 Le 19:16 Ps 52:2; 64:3,\allowbreak4}
\crossref{Prov}{18}{9}{Pr 10:4; 23:20,\allowbreak21; 24:30-\allowbreak34 Mt 25:26 Ro 12:11 Heb 6:12}
\crossref{Prov}{18}{10}{Ge 17:1 Ex 3:13-\allowbreak15; 6:3; 34:5-\allowbreak7 Isa 9:6; 57:15 Jer 23:6 Mt 1:23}
\crossref{Prov}{18}{11}{Pr 10:15; 11:4 De 32:31 Job 31:24,\allowbreak25 Ps 49:6-\allowbreak9; 52:5-\allowbreak7; 62:10,\allowbreak11}
\crossref{Prov}{18}{12}{Pr 11:2; 16:18; 29:23 Eze 16:49,\allowbreak50; 28:2,\allowbreak9 Da 5:23,\allowbreak24 Ac 12:21-\allowbreak23}
\crossref{Prov}{18}{13}{De 13:14 2Sa 16:4; 19:24-\allowbreak30 Es 3:10-\allowbreak15; 8:5-\allowbreak17 Job 29:16}
\crossref{Prov}{18}{14}{Job 1:20,\allowbreak21; 2:7-\allowbreak10 Ps 147:3 Ro 5:3-\allowbreak5; 8:35-\allowbreak37 2Co 1:12; 12:9,\allowbreak10}
\crossref{Prov}{18}{15}{Pr 1:5; 4:5,\allowbreak7; 9:9; 10:14; 15:14; 23:23 1Ki 3:9 Ps 119:97-\allowbreak104}
\crossref{Prov}{18}{16}{Pr 17:8; 19:6; 21:14 Ge 32:20; 33:10; 43:11 1Sa 25:27}
\crossref{Prov}{18}{17}{18:13 2Sa 16:1-\allowbreak3; 19:24-\allowbreak27 Ac 24:5,\allowbreak6,\allowbreak12,\allowbreak13}
\crossref{Prov}{18}{18}{Pr 16:33 Jos 14:2 1Sa 10:21-\allowbreak27; 14:42 1Ch 6:63; 24:31 Ne 11:1}
\crossref{Prov}{18}{19}{Pr 6:19 Ge 4:5-\allowbreak8; 27:41-\allowbreak45; 32:6-\allowbreak11; 37:3-\allowbreak5,\allowbreak11,\allowbreak18-\allowbreak27 2Sa 13:22,\allowbreak28}
\crossref{Prov}{18}{20}{Pr 12:13,\allowbreak14; 13:2; 22:18,\allowbreak21; 25:11,\allowbreak12}
\crossref{Prov}{18}{21}{18:4-\allowbreak7; 10:20,\allowbreak21,\allowbreak31; 11:30 Mt 12:35-\allowbreak37 Ro 10:14,\allowbreak15 2Co 2:16; 11:15}
\crossref{Prov}{18}{22}{Pr 5:15-\allowbreak23; 12:4; 19:14; 31:10-\allowbreak31 Ge 24:67; 29:20,\allowbreak21,\allowbreak28 Ec 9:9}
\crossref{Prov}{18}{23}{Ru 2:7 1Sa 2:36 2Ki 4:1,\allowbreak2 Isa 66:2 Mt 5:3 Jas 1:9-\allowbreak11}
\crossref{Prov}{18}{24}{Pr 17:17; 27:9 1Sa 19:4,\allowbreak5; 30:26-\allowbreak31 2Sa 9:1-\allowbreak13; 16:17; 17:27-\allowbreak29}
\crossref{Prov}{19}{1}{19:22; 12:26; 15:16; 16:8; 28:6 Ps 37:26 Mt 16:26 Jas 2:5,\allowbreak6}
\crossref{Prov}{19}{2}{Pr 10:21 Ec 12:9 Isa 27:11 Ho 4:6 Joh 16:3 Ro 10:2 Php 1:9}
\crossref{Prov}{19}{3}{Ge 3:6-\allowbreak12; 4:5-\allowbreak14 Nu 16:19-\allowbreak41; 17:12,\allowbreak13 1Sa 13:13; 15:23}
\crossref{Prov}{19}{4}{19:6,\allowbreak7; 14:20 Lu 15:13-\allowbreak15}
\crossref{Prov}{19}{5}{19:9; 6:19; 21:28 Ex 23:1 De 19:16-\allowbreak21 Ps 120:3,\allowbreak4 Da 6:24}
\crossref{Prov}{19}{6}{19:12; 16:15; 29:26 Ge 42:6 2Sa 19:19-\allowbreak39 Job 29:24,\allowbreak25 Ps 45:12}
\crossref{Prov}{19}{7}{19:4; 14:20 Ps 38:11; 88:8,\allowbreak18 Ec 9:15,\allowbreak16 Jas 2:6}
\crossref{Prov}{19}{8}{Pr 17:16 Eze 36:26}
\crossref{Prov}{19}{9}{19:5}
\crossref{Prov}{19}{10}{Pr 30:21,\allowbreak22 1Sa 25:36 Es 3:15 Isa 5:11,\allowbreak12; 22:12-\allowbreak14 Ho 7:3-\allowbreak5; 9:1}
\crossref{Prov}{19}{11}{Pr 12:16; 14:29; 15:18; 16:32; 17:14 Col 3:12,\allowbreak13 Jas 1:19}
\crossref{Prov}{19}{12}{Pr 16:14,\allowbreak15; 20:2; 28:15 Es 7:8 Ec 8:4 Da 2:12,\allowbreak13; 3:19-\allowbreak23; 5:19}
\crossref{Prov}{19}{13}{Pr 10:1; 15:20; 17:21,\allowbreak25 2Sa 13:1-\allowbreak18:33 Ec 2:18,\allowbreak19}
\crossref{Prov}{19}{14}{Pr 13:22 De 21:16 Jos 11:23 2Co 12:14}
\crossref{Prov}{19}{15}{19:24; 6:9,\allowbreak10; 20:13; 23:21; 24:33 Isa 56:10 Ro 13:11,\allowbreak12 Eph 5:14}
\crossref{Prov}{19}{16}{Pr 3:1; 29:18 Ps 103:18 Ec 8:5; 12:13 Jer 7:23 Lu 10:28; 11:28}
\crossref{Prov}{19}{17}{Pr 14:21; 28:8,\allowbreak27 2Sa 12:6 Ec 11:1}
\crossref{Prov}{19}{18}{Pr 13:24; 22:15; 23:13,\allowbreak14; 29:15,\allowbreak17 Heb 12:7-\allowbreak10}
\crossref{Prov}{19}{19}{Pr 22:24,\allowbreak25; 25:28; 29:22 1Sa 20:30,\allowbreak31; 22:7-\allowbreak23; 24:17-\allowbreak22; 26:21-\allowbreak25}
\crossref{Prov}{19}{20}{Pr 1:8; 2:1-\allowbreak9; 8:34,\allowbreak35}
\crossref{Prov}{19}{21}{Pr 12:2 Ge 37:19,\allowbreak20 Es 9:25 Ps 21:11; 33:10,\allowbreak11; 83:4 Ec 7:29}
\crossref{Prov}{19}{22}{1Ch 29:2,\allowbreak3,\allowbreak17 2Ch 6:8 Mr 12:41-\allowbreak44; 14:6-\allowbreak8 2Co 8:2,\allowbreak3,\allowbreak12}
\crossref{Prov}{19}{23}{Pr 10:27; 14:26,\allowbreak27 Ps 19:9; 33:18,\allowbreak19; 34:9-\allowbreak11; 85:9; 103:17; 145:18-\allowbreak20}
\crossref{Prov}{19}{24}{19:15; 6:9,\allowbreak10; 12:27; 15:19; 24:30-\allowbreak34; 26:13-\allowbreak16 Ps 74:11}
\crossref{Prov}{19}{25}{Pr 21:11 De 13:11; 21:21}
\crossref{Prov}{19}{26}{Pr 10:1; 17:25; 23:22-\allowbreak25; 28:14; 30:11,\allowbreak17 De 21:18-\allowbreak21 Lu 15:12-\allowbreak16,\allowbreak30}
\crossref{Prov}{19}{27}{Pr 14:7 De 13:1-\allowbreak4 1Ki 22:22-\allowbreak28 Mt 7:15; 16:6,\allowbreak12 Mr 4:24; 7:6-\allowbreak14}
\crossref{Prov}{19}{28}{1Ki 21:10,\allowbreak13 Ac 6:11-\allowbreak13}
\crossref{Prov}{19}{29}{Pr 3:34; 9:12 Isa 28:22; 29:20 Ac 13:40,\allowbreak41 2Pe 3:3-\allowbreak7}
\crossref{Prov}{20}{1}{Pr 23:29-\allowbreak35; 31:4 Ge 9:21-\allowbreak23; 19:31-\allowbreak36 1Sa 25:36-\allowbreak38 2Sa 11:13}
\crossref{Prov}{20}{2}{Pr 16:14,\allowbreak15; 19:12 Ec 10:4 Ho 11:10 Am 3:8}
\crossref{Prov}{20}{3}{Pr 14:29; 16:32; 17:14; 19:11; 25:8-\allowbreak10 Eph 1:6-\allowbreak8; 4:32; 5:1}
\crossref{Prov}{20}{4}{Pr 10:4; 19:15,\allowbreak24; 26:13-\allowbreak16}
\crossref{Prov}{20}{5}{Pr 18:4 Ps 64:6 1Co 2:11}
\crossref{Prov}{20}{6}{Pr 25:14; 27:2 Mt 6:2 Lu 18:8,\allowbreak11,\allowbreak28; 22:33 2Co 12:11}
\crossref{Prov}{20}{7}{Pr 14:2; 19:1 Ps 15:2; 26:1,\allowbreak11 Isa 33:15 Lu 1:6 2Co 1:12}
\crossref{Prov}{20}{8}{20:26; 16:12; 29:14 1Sa 23:3,\allowbreak4 2Sa 23:4 Ps 72:4; 92:9; 99:4; 101:6-\allowbreak8}
\crossref{Prov}{20}{9}{1Ki 8:46 2Ch 6:36 Job 14:4; 15:14; 25:4 Ps 51:5 Ec 7:20 1Co 4:4}
\crossref{Prov}{20}{10}{20:23; 11:1; 16:11 Le 19:35 De 25:13-\allowbreak15 Am 8:4-\allowbreak7}
\crossref{Prov}{20}{11}{Pr 21:8; 22:15 Ps 51:5; 58:3 Mt 7:16 Lu 1:15,\allowbreak66; 2:46,\allowbreak47; 6:43,\allowbreak44}
\crossref{Prov}{20}{12}{Ex 4:11 Ps 94:9; 119:18 Mt 13:13-\allowbreak16 Ac 26:18 Eph 1:17,\allowbreak18}
\crossref{Prov}{20}{13}{Pr 6:9-\allowbreak11; 10:4; 12:11; 13:4; 19:15; 24:30-\allowbreak34 Ro 12:11 2Th 3:10}
\crossref{Prov}{20}{14}{Ec 1:10 Ho 12:7,\allowbreak8 1Th 4:6}
\crossref{Prov}{20}{15}{Pr 3:15; 8:11; 10:20,\allowbreak21; 15:7,\allowbreak23; 16:16,\allowbreak21,\allowbreak24; 25:12 Job 28:12-\allowbreak19}
\crossref{Prov}{20}{16}{Pr 11:15; 22:26,\allowbreak27; 27:13 Ex 22:26,\allowbreak27}
\crossref{Prov}{20}{17}{Pr 4:17}
\crossref{Prov}{20}{18}{Pr 15:22; 24:6}
\crossref{Prov}{20}{19}{Pr 11:13; 18:8; 26:20-\allowbreak22 Le 19:16}
\crossref{Prov}{20}{20}{Pr 30:11,\allowbreak17 Ex 20:12; 21:17 Le 20:9 De 27:16 Mt 15:4 Mr 7:10-\allowbreak13}
\crossref{Prov}{20}{21}{Pr 23:4; 28:20,\allowbreak22 1Ti 6:9}
\crossref{Prov}{20}{22}{Pr 17:13; 24:29 De 32:35 Ro 12:17-\allowbreak19 1Th 5:15 1Pe 3:9}
\crossref{Prov}{20}{23}{20:10 Eze 45:10}
\crossref{Prov}{20}{24}{Ps 37:23 Jer 10:23 Da 5:23 Ac 17:28}
\crossref{Prov}{20}{25}{Pr 18:7 Le 5:15; 22:10-\allowbreak15; 27:30 Mal 3:8-\allowbreak10}
\crossref{Prov}{20}{26}{20:8 2Sa 4:9-\allowbreak12 Ps 101:5-\allowbreak8}
\crossref{Prov}{20}{27}{Ge 2:7 Job 32:8 Ro 2:15 1Co 2:11 2Co 4:2-\allowbreak6 1Jo 3:19-\allowbreak21}
\crossref{Prov}{20}{28}{Pr 16:6}
\crossref{Prov}{20}{29}{Jer 9:23,\allowbreak24 1Jo 2:14}
\crossref{Prov}{20}{30}{}
\crossref{Prov}{21}{1}{Pr 16:1,\allowbreak9; 20:24 Ezr 7:27,\allowbreak28 Ne 1:11; 2:4 Ps 105:25; 106:46 Da 4:35}
\crossref{Prov}{21}{2}{Pr 16:2,\allowbreak25; 20:6; 30:12 Ps 36:2 Lu 18:11,\allowbreak12 Ga 6:3 Jas 1:22}
\crossref{Prov}{21}{3}{Pr 15:8 1Sa 15:22 Ps 50:8 Isa 1:11-\allowbreak17 Jer 7:21-\allowbreak23 Ho 6:6}
\crossref{Prov}{21}{4}{Pr 6:17; 8:13; 30:13 Ps 10:4 Isa 2:11,\allowbreak17; 3:16 Lu 18:14 1Pe 5:5}
\crossref{Prov}{21}{5}{Pr 10:4; 13:4; 27:23-\allowbreak27 Eph 4:28 1Th 4:11,\allowbreak12}
\crossref{Prov}{21}{6}{Pr 10:2; 13:11; 20:14,\allowbreak21; 22:8; 30:8 Jer 17:11 1Ti 6:9,\allowbreak10 Tit 1:11}
\crossref{Prov}{21}{7}{Pr 1:18,\allowbreak19; 10:6; 22:22,\allowbreak23 Ps 7:16; 9:16 Isa 1:23,\allowbreak24 Jer 7:9-\allowbreak11,\allowbreak15}
\crossref{Prov}{21}{8}{Ge 6:5,\allowbreak6,\allowbreak12 Job 15:14-\allowbreak16 Ps 14:2,\allowbreak3 Ec 7:29; 9:3 1Co 3:3}
\crossref{Prov}{21}{9}{21:19; 12:4; 19:13; 25:24; 27:15,\allowbreak16}
\crossref{Prov}{21}{10}{Pr 3:29; 12:12 Ps 36:4; 52:2,\allowbreak3 Mr 7:21,\allowbreak22 1Co 10:6 Jas 4:1-\allowbreak5}
\crossref{Prov}{21}{11}{Pr 19:25 Nu 16:34 De 13:11; 21:21 Ps 64:7-\allowbreak9 Ac 5:5,\allowbreak11-\allowbreak14}
\crossref{Prov}{21}{12}{Job 5:3; 8:15; 18:14-\allowbreak21; 21:28-\allowbreak30; 27:13-\allowbreak23 Ps 37:35,\allowbreak36; 52:5}
\crossref{Prov}{21}{13}{Ps 58:4 Zec 7:11 Ac 7:57}
\crossref{Prov}{21}{14}{Pr 17:8,\allowbreak23; 18:16; 19:6 Ge 32:20; 43:11 1Sa 25:35}
\crossref{Prov}{21}{15}{Job 29:12-\allowbreak17 Ps 40:8; 112:1; 119:16,\allowbreak92 Ec 3:12 Isa 64:5}
\crossref{Prov}{21}{16}{Pr 13:20 Ps 125:5 Zep 1:6 Joh 3:19,\allowbreak20 Heb 6:4-\allowbreak6; 10:26,\allowbreak27,\allowbreak38}
\crossref{Prov}{21}{17}{21:20; 5:10,\allowbreak11; 23:21 Lu 15:13-\allowbreak16; 16:24,\allowbreak25 1Ti 5:6 2Ti 3:4}
\crossref{Prov}{21}{18}{Pr 11:8 Isa 43:3,\allowbreak4; 53:4,\allowbreak5; 55:8,\allowbreak9 1Pe 3:18}
\crossref{Prov}{21}{19}{21:9 Ps 55:6,\allowbreak7; 120:5,\allowbreak6 Jer 9:2}
\crossref{Prov}{21}{20}{Pr 10:22; 15:6 Ps 112:3 Ec 5:19; 7:11; 10:19 Mt 6:19,\allowbreak20 Lu 6:45}
\crossref{Prov}{21}{21}{Pr 15:9 Isa 51:1 Ho 6:3 Mt 5:6 Ro 14:19 Php 3:12 1Th 5:21}
\crossref{Prov}{21}{22}{2Sa 20:16-\allowbreak22 Ec 7:19; 9:13-\allowbreak18}
\crossref{Prov}{21}{23}{Pr 10:19; 12:13; 13:3; 17:27,\allowbreak28; 18:21 Jas 1:26; 3:2-\allowbreak13}
\crossref{Prov}{21}{24}{Pr 6:17; 16:18; 18:12; 19:29 Es 3:5,\allowbreak6 Ec 7:8,\allowbreak9 Mt 2:16}
\crossref{Prov}{21}{25}{Pr 6:6-\allowbreak11; 12:24,\allowbreak27; 13:4; 15:19; 19:24; 20:4; 22:13; 24:30-\allowbreak34; 26:13,\allowbreak16}
\crossref{Prov}{21}{26}{Ac 20:33-\allowbreak35 1Th 2:5-\allowbreak9}
\crossref{Prov}{21}{27}{Pr 15:8; 28:9 1Sa 13:12,\allowbreak13; 15:21-\allowbreak23 Ps 50:8-\allowbreak13 Isa 1:11-\allowbreak16; 66:3}
\crossref{Prov}{21}{28}{Pr 6:19; 19:5,\allowbreak9; 25:18 Ex 23:1 De 19:16-\allowbreak19}
\crossref{Prov}{21}{29}{Pr 28:14; 29:1 Jer 3:2,\allowbreak3; 5:3; 8:12; 44:16,\allowbreak17}
\crossref{Prov}{21}{30}{Pr 19:21 Isa 7:5-\allowbreak7; 8:9,\allowbreak10; 14:27; 46:10,\allowbreak11 Jer 9:23 Jon 1:13}
\crossref{Prov}{21}{31}{Ps 20:7; 33:17,\allowbreak18; 147:10 Ec 9:11 Isa 31:1}
\crossref{Prov}{22}{1}{1Ki 1:47 Ec 7:1 Lu 10:20 Php 4:3 Heb 11:39}
\crossref{Prov}{22}{2}{Pr 29:13 1Sa 2:7 Ps 49:1,\allowbreak2 Lu 16:19,\allowbreak20 1Co 12:21 Jas 2:2-\allowbreak5}
\crossref{Prov}{22}{3}{Pr 14:16; 27:12 Ex 9:20,\allowbreak21 Isa 26:20,\allowbreak21 Mt 24:15-\allowbreak18 1Th 5:2-\allowbreak6}
\crossref{Prov}{22}{4}{Pr 3:16; 21:21 Ps 34:9,\allowbreak10; 112:1-\allowbreak3 Isa 33:6; 57:15 Mt 6:33 1Ti 4:8}
\crossref{Prov}{22}{5}{Pr 13:15; 15:19 Jos 23:13 Job 18:8 Ps 11:6; 18:26,\allowbreak27}
\crossref{Prov}{22}{6}{Ge 18:19 De 4:9; 6:7 Ps 78:3-\allowbreak6 Eph 6:4 2Ti 3:15}
\crossref{Prov}{22}{7}{22:16,\allowbreak22; 14:31; 18:23 Am 2:6; 4:1; 5:11,\allowbreak12; 8:4,\allowbreak6 Jas 2:6; 5:1,\allowbreak4}
\crossref{Prov}{22}{8}{Job 4:8 Ho 8:7; 10:13 Ga 6:7,\allowbreak8}
\crossref{Prov}{22}{9}{Pr 11:25; 19:17; 21:13 De 15:7-\allowbreak11; 28:56 Job 31:16-\allowbreak20 Ps 41:1-\allowbreak3}
\crossref{Prov}{22}{10}{Pr 21:24; 26:20,\allowbreak21 Ge 21:9,\allowbreak10 Ne 4:1-\allowbreak3; 13:28 Ps 101:5 Mt 18:17}
\crossref{Prov}{22}{11}{Pr 16:13 Ps 101:6 Mt 5:8}
\crossref{Prov}{22}{12}{2Ch 16:9 Isa 59:19-\allowbreak21 Mt 16:16-\allowbreak18 Ac 5:39; 12:23,\allowbreak24 Re 11:3-\allowbreak11}
\crossref{Prov}{22}{13}{}
\crossref{Prov}{22}{14}{Pr 2:16-\allowbreak19; 5:3-\allowbreak23; 6:24-\allowbreak29; 7:5-\allowbreak27; 23:27 Jud 16:20,\allowbreak21 Ne 13:26}
\crossref{Prov}{22}{15}{Job 14:4 Ps 51:5 Joh 3:6 Eph 2:3}
\crossref{Prov}{22}{16}{22:22,\allowbreak23; 14:31; 28:3 Job 20:19-\allowbreak29 Ps 12:5 Mic 2:2,\allowbreak3 Zec 7:9-\allowbreak14}
\crossref{Prov}{22}{17}{Pr 2:2-\allowbreak5; 5:1,\allowbreak2}
\crossref{Prov}{22}{18}{Pr 2:10; 3:17; 24:13,\allowbreak14 Ps 19:10; 119:103,\allowbreak111,\allowbreak162 Jer 15:16}
\crossref{Prov}{22}{19}{Pr 3:5 Ps 62:8 Isa 12:2; 26:4 Jer 17:7 1Pe 1:21}
\crossref{Prov}{22}{20}{Pr 8:6 Ps 12:6 Ho 8:12 2Ti 3:15-\allowbreak17 2Pe 1:19-\allowbreak21}
\crossref{Prov}{22}{21}{Lu 1:3,\allowbreak4 Joh 20:31 1Jo 5:13}
\crossref{Prov}{22}{22}{Pr 23:10,\allowbreak11 Eze 22:29}
\crossref{Prov}{22}{23}{Pr 23:11 1Sa 24:12,\allowbreak15; 25:39 Ps 12:5; 35:1,\allowbreak10; 43:1; 68:5; 140:12}
\crossref{Prov}{22}{24}{Pr 21:24; 29:22 2Co 6:14-\allowbreak17}
\crossref{Prov}{22}{25}{Pr 13:20 Ps 106:35 1Co 15:33}
\crossref{Prov}{22}{26}{Pr 6:1-\allowbreak5; 11:15; 17:18; 27:13}
\crossref{Prov}{22}{27}{Pr 20:16 Ex 22:26,\allowbreak27 2Ki 4:1}
\crossref{Prov}{22}{28}{Pr 23:10 De 19:14; 27:17 Job 24:2}
\crossref{Prov}{22}{29}{Pr 10:4; 12:24 1Ki 11:28 Ec 9:10 Mt 25:21,\allowbreak23 Ro 12:11 2Ti 4:2}
\crossref{Prov}{23}{1}{Ge 43:32-\allowbreak34 Jude 1:12}
\crossref{Prov}{23}{2}{Mt 18:8,\allowbreak9 1Co 9:27 Php 3:19}
\crossref{Prov}{23}{3}{23:6 Ps 141:4 Da 1:8 Lu 21:34 Eph 4:22}
\crossref{Prov}{23}{4}{Pr 28:20 Joh 6:27 1Ti 6:8-\allowbreak10}
\crossref{Prov}{23}{5}{Ps 119:36,\allowbreak37 Jer 22:17 1Jo 2:16}
\crossref{Prov}{23}{6}{Pr 22:9 De 15:9; 28:56 Mt 20:15 Mr 7:22}
\crossref{Prov}{23}{7}{Pr 19:22 Mt 9:3,\allowbreak4 Lu 7:39}
\crossref{Prov}{23}{8}{Pr 17:1; 28:21 Ge 18:5}
\crossref{Prov}{23}{9}{Pr 9:7,\allowbreak8; 26:4,\allowbreak5 Isa 36:21 Mt 7:6 Ac 13:45,\allowbreak46; 28:25-\allowbreak28}
\crossref{Prov}{23}{10}{Pr 22:28 De 19:14; 27:17 Job 24:2}
\crossref{Prov}{23}{11}{Pr 22:23 Ex 22:22-\allowbreak24 De 27:19 Ps 12:5 Jer 50:33,\allowbreak34; 51:36}
\crossref{Prov}{23}{12}{23:19; 2:2-\allowbreak6; 5:1,\allowbreak2; 22:17 Eze 33:31 Mt 13:52 Jas 1:21-\allowbreak25}
\crossref{Prov}{23}{13}{Pr 13:24; 19:18; 29:15,\allowbreak17}
\crossref{Prov}{23}{14}{Pr 22:15 1Co 5:5; 11:32}
\crossref{Prov}{23}{15}{Pr 1:10; 2:1; 4:1 Mt 9:2 Joh 21:5 1Jo 2:1}
\crossref{Prov}{23}{16}{Pr 8:6 Eph 4:29; 5:4 Col 4:4 Jas 3:2}
\crossref{Prov}{23}{17}{Pr 3:31; 24:1 Ps 37:1-\allowbreak3; 73:3-\allowbreak7}
\crossref{Prov}{23}{18}{Ps 37:37 Jer 29:11 Lu 16:25 Ro 6:21,\allowbreak22}
\crossref{Prov}{23}{19}{23:12,\allowbreak26; 4:10-\allowbreak23}
\crossref{Prov}{23}{20}{23:29-\allowbreak35; 20:1; 28:7; 31:6,\allowbreak7 Isa 5:11,\allowbreak22; 22:13 Mt 24:49 Lu 15:13}
\crossref{Prov}{23}{21}{Pr 21:17 De 21:20 Isa 28:1-\allowbreak3 Joe 1:5 1Co 5:11; 6:10 Ga 5:21}
\crossref{Prov}{23}{22}{Pr 1:8; 6:20 De 21:18-\allowbreak21; 27:16 Mr 7:10 Eph 6:1,\allowbreak2}
\crossref{Prov}{23}{23}{Pr 2:2-\allowbreak4; 4:5-\allowbreak7; 10:1; 16:16; 17:16 Job 28:12-\allowbreak19 Ps 119:72,\allowbreak127}
\crossref{Prov}{23}{24}{23:15,\allowbreak16; 10:1; 15:20 1Ki 1:48; 2:1-\allowbreak3,\allowbreak9 Ec 2:19}
\crossref{Prov}{23}{25}{Pr 17:25 1Ch 4:9,\allowbreak10 Lu 1:31-\allowbreak33,\allowbreak40-\allowbreak47,\allowbreak58; 11:27,\allowbreak28}
\crossref{Prov}{23}{26}{23:15}
\crossref{Prov}{23}{27}{Pr 22:14}
\crossref{Prov}{23}{28}{Pr 2:16-\allowbreak19; 7:12,\allowbreak22-\allowbreak27; 9:18; 22:14 Jud 16:4-\allowbreak22 Ec 7:26 Jer 3:2}
\crossref{Prov}{23}{29}{23:21; 20:1 1Sa 25:36,\allowbreak37 2Sa 13:28 1Ki 20:16-\allowbreak22 Isa 5:11,\allowbreak22}
\crossref{Prov}{23}{30}{Pr 20:1 Ge 9:21 Isa 5:11 Am 6:6 Eph 5:18}
\crossref{Prov}{23}{31}{Pr 6:25 2Sa 11:2 Job 33:1 Ps 119:37 Mt 5:28-\allowbreak30 Mr 9:47 1Jo 2:16}
\crossref{Prov}{23}{32}{Pr 5:11 Isa 28:3,\allowbreak7,\allowbreak8 Jer 5:31 Ex 7:5,\allowbreak6,\allowbreak12 Lu 16:25,\allowbreak26 Ro 6:21}
\crossref{Prov}{23}{33}{Ge 19:32-\allowbreak38}
\crossref{Prov}{23}{34}{1Sa 25:33-\allowbreak38; 30:16,\allowbreak17 2Sa 13:28 1Ki 16:9; 20:16-\allowbreak22 Joe 1:5}
\crossref{Prov}{23}{35}{Pr 27:22 Jer 5:3; 31:18}
\crossref{Prov}{24}{1}{24:19; 3:31; 23:17 Ps 37:1,\allowbreak7; 73:3 Ga 5:19-\allowbreak21 Jas 4:5,\allowbreak6}
\crossref{Prov}{24}{2}{24:8; 6:14 1Sa 23:9 Es 3:6,\allowbreak7 Job 15:35 Ps 7:14; 10:7; 28:3; 36:4}
\crossref{Prov}{24}{3}{Pr 9:1; 14:1 1Co 3:9}
\crossref{Prov}{24}{4}{Pr 15:6; 20:15; 21:20; 27:23-\allowbreak27 1Ki 4:22-\allowbreak28 1Ch 27:25-\allowbreak31; 29:2-\allowbreak9}
\crossref{Prov}{24}{5}{Pr 8:14; 10:29; 21:22 Ec 7:19; 9:14-\allowbreak18}
\crossref{Prov}{24}{6}{Pr 20:18 Lu 14:31 1Co 9:25-\allowbreak27 Eph 6:10-\allowbreak20 1Ti 6:11,\allowbreak12 2Ti 4:7}
\crossref{Prov}{24}{7}{Pr 14:6; 15:24; 17:24 Ps 10:5; 92:5,\allowbreak6 1Co 2:14}
\crossref{Prov}{24}{8}{24:2,\allowbreak9; 6:14,\allowbreak18; 14:22 1Ki 2:44 Ps 21:11 Isa 10:7-\allowbreak13; 32:7}
\crossref{Prov}{24}{9}{24:8; 23:7 Ge 6:5; 8:21 Ps 119:113 Isa 55:7 Jer 4:14 Mt 5:28; 9:4}
\crossref{Prov}{24}{10}{1Sa 27:1 Job 4:5 Isa 40:28-\allowbreak31 Joh 4:8 2Co 4:1 Eph 3:13}
\crossref{Prov}{24}{11}{1Sa 26:8,\allowbreak9 Job 29:17 Ps 82:4 Isa 58:6,\allowbreak7 Lu 10:31,\allowbreak32; 23:23-\allowbreak25}
\crossref{Prov}{24}{12}{Pr 5:21; 21:2 1Sa 16:7 Ps 7:9; 17:3; 44:21 Ec 5:8 Jer 17:10 Ro 2:16}
\crossref{Prov}{24}{13}{Pr 25:16,\allowbreak27 So 5:1 Isa 7:15 Mt 3:4}
\crossref{Prov}{24}{14}{Pr 22:18 Ps 19:10,\allowbreak11; 119:103,\allowbreak111 Jer 15:16}
\crossref{Prov}{24}{15}{Pr 1:11 1Sa 9:11; 22:18,\allowbreak19; 23:20-\allowbreak23 Ps 10:8-\allowbreak10; 37:32; 56:6; 59:3}
\crossref{Prov}{24}{16}{Job 5:19 Ps 34:19; 37:24 Mic 7:8-\allowbreak10 2Co 1:8-\allowbreak10; 4:8-\allowbreak12; 11:23-\allowbreak27}
\crossref{Prov}{24}{17}{Pr 17:5 Jud 16:25 2Sa 16:5-\allowbreak14 Job 31:29 Ps 35:15,\allowbreak19; 42:10 Ob 1:12}
\crossref{Prov}{24}{18}{La 4:21,\allowbreak22 Zec 1:15,\allowbreak16}
\crossref{Prov}{24}{19}{24:1; 23:17 Ps 37:1; 73:3}
\crossref{Prov}{24}{20}{Ps 9:17; 11:6 Isa 3:11}
\crossref{Prov}{24}{21}{Ex 14:31 1Sa 24:6 Ec 8:2-\allowbreak5 Mt 22:21 Ro 13:1-\allowbreak7 Tit 3:1}
\crossref{Prov}{24}{22}{Nu 16:31-\allowbreak35 1Sa 31:1-\allowbreak7 2Sa 18:7,\allowbreak8 2Ch 13:16,\allowbreak17 Ho 5:11}
\crossref{Prov}{24}{23}{Ps 107:43 Ec 8:1-\allowbreak5 Ho 14:9 Jas 3:17}
\crossref{Prov}{24}{24}{Pr 17:15 Ex 23:6,\allowbreak7 Isa 5:20,\allowbreak23 Jer 6:13,\allowbreak14; 8:10,\allowbreak11 Eze 13:22}
\crossref{Prov}{24}{25}{Le 19:17 1Sa 3:13 1Ki 21:19,\allowbreak20 Ne 5:7-\allowbreak9; 13:8-\allowbreak11,\allowbreak17,\allowbreak25,\allowbreak28}
\crossref{Prov}{24}{26}{Pr 15:23; 16:1; 25:11,\allowbreak12 Ge 41:38-\allowbreak57 Da 2:46-\allowbreak48 Mr 12:17,\allowbreak18,\allowbreak32-\allowbreak34}
\crossref{Prov}{24}{27}{1Ki 5:17,\allowbreak18; 6:7 Lu 14:28-\allowbreak30}
\crossref{Prov}{24}{28}{Pr 14:5; 19:5,\allowbreak9; 21:28 Ex 20:16; 23:1 1Sa 22:9,\allowbreak10 1Ki 21:9-\allowbreak13}
\crossref{Prov}{24}{29}{Pr 20:22; 25:21,\allowbreak22 Mt 5:39-\allowbreak44 Ro 12:17-\allowbreak21 1Th 5:15}
\crossref{Prov}{24}{30}{Pr 6:6-\allowbreak19 Job 4:8; 5:27; 15:17 Ps 37:25; 107:42 Ec 4:1-\allowbreak8; 7:15}
\crossref{Prov}{24}{31}{Ge 3:17-\allowbreak19 Job 31:40 Jer 4:3 Mt 13:7,\allowbreak22 Heb 6:8}
\crossref{Prov}{24}{32}{Job 7:17 Ps 4:4 Lu 2:19,\allowbreak51}
\crossref{Prov}{24}{33}{Pr 6:4-\allowbreak11 Ro 13:11 Eph 5:14 1Th 5:6-\allowbreak8}
\crossref{Prov}{24}{34}{Pr 10:4; 13:4}
\crossref{Prov}{25}{1}{Pr 1:1; 10:1 1Ki 4:32 Ec 12:9}
\crossref{Prov}{25}{2}{De 29:29 Job 11:7,\allowbreak8; 38:4-\allowbreak41; 39:1-\allowbreak30; 40:2; 42:3 Ro 11:33,\allowbreak34}
\crossref{Prov}{25}{3}{Ps 103:11 Isa 7:11; 55:9 Ro 8:39}
\crossref{Prov}{25}{4}{Pr 17:3 Isa 1:25-\allowbreak27 Mal 3:3 2Ti 2:20,\allowbreak21 1Pe 1:7}
\crossref{Prov}{25}{5}{Pr 20:8 1Ki 2:33,\allowbreak46 Es 7:10; 8:11-\allowbreak17 Ps 101:7,\allowbreak8}
\crossref{Prov}{25}{6}{25:27; 27:2}
\crossref{Prov}{25}{7}{Pr 16:19 Lu 14:8-\allowbreak10}
\crossref{Prov}{25}{8}{Pr 17:14; 18:6; 30:33 2Sa 2:14-\allowbreak16,\allowbreak26 2Ki 14:8-\allowbreak12 Lu 14:31,\allowbreak32}
\crossref{Prov}{25}{9}{Mt 18:5-\allowbreak17}
\crossref{Prov}{25}{10}{Ps 119:39}
\crossref{Prov}{25}{11}{Pr 15:23; 24:26 Ec 12:10 Isa 50:4}
\crossref{Prov}{25}{12}{Job 42:11}
\crossref{Prov}{25}{13}{25:25; 13:17; 26:6 Php 2:25-\allowbreak30}
\crossref{Prov}{25}{14}{Pr 20:6 1Ki 22:11 Lu 14:11; 18:10-\allowbreak14 2Co 11:13-\allowbreak18,\allowbreak31 2Pe 2:15-\allowbreak19}
\crossref{Prov}{25}{15}{Pr 15:1; 16:14 Ge 32:4-\allowbreak21 1Sa 25:14,\allowbreak24-\allowbreak44 Ec 10:4}
\crossref{Prov}{25}{16}{Pr 24:13,\allowbreak14 Jud 14:8,\allowbreak9 1Sa 14:25-\allowbreak27 Isa 7:15,\allowbreak22}
\crossref{Prov}{25}{17}{Ro 15:24}
\crossref{Prov}{25}{18}{Pr 12:18 Ps 52:2; 55:21; 57:4; 120:3,\allowbreak4; 140:3 Jer 9:3,\allowbreak8 Jas 3:6}
\crossref{Prov}{25}{19}{2Ch 28:20,\allowbreak21 Job 6:14-\allowbreak20 Isa 30:1-\allowbreak3; 36:6 Eze 29:6,\allowbreak7 2Ti 4:16}
\crossref{Prov}{25}{20}{De 24:12-\allowbreak17 Job 24:7-\allowbreak10 Isa 58:7 Jas 2:15,\allowbreak16}
\crossref{Prov}{25}{21}{Pr 24:17 Ex 23:4,\allowbreak5 Mt 5:44 Lu 10:33-\allowbreak36 Ro 12:20,\allowbreak21}
\crossref{Prov}{25}{22}{2Sa 16:12 Mt 10:13 1Co 15:18}
\crossref{Prov}{25}{23}{Job 37:22}
\crossref{Prov}{25}{24}{Pr 19:13; 21:9,\allowbreak19; 27:15,\allowbreak16}
\crossref{Prov}{25}{25}{Ge 21:16-\allowbreak19 Ex 17:2,\allowbreak3,\allowbreak6 Jud 15:18,\allowbreak19 2Sa 23:15 Ps 42:1,\allowbreak2}
\crossref{Prov}{25}{26}{Ge 4:8 1Sa 22:14-\allowbreak18 2Ch 24:21,\allowbreak22 Mt 23:34-\allowbreak37; 26:69-\allowbreak74 Ac 7:52}
\crossref{Prov}{25}{27}{25:16}
\crossref{Prov}{25}{28}{Pr 16:32; 22:24 1Sa 20:30; 25:17}
\crossref{Prov}{26}{1}{1Sa 12:17,\allowbreak18}
\crossref{Prov}{26}{2}{Nu 23:8 De 23:4,\allowbreak5 1Sa 14:28,\allowbreak29; 17:43 2Sa 16:12 Ne 13:2}
\crossref{Prov}{26}{3}{}
\crossref{Prov}{26}{4}{Pr 17:14 Jud 12:1-\allowbreak6 2Sa 19:41-\allowbreak43 1Ki 12:14,\allowbreak16 2Ki 14:8-\allowbreak10}
\crossref{Prov}{26}{5}{1Ki 22:24-\allowbreak28 Jer 36:17,\allowbreak18 Mt 15:1-\allowbreak3; 16:1-\allowbreak4; 21:23-\allowbreak27; 22:15-\allowbreak32}
\crossref{Prov}{26}{6}{Pr 10:26; 13:17; 25:13 Nu 13:31}
\crossref{Prov}{26}{7}{26:9; 17:7 Ps 50:16-\allowbreak21; 64:8 Mt 7:4,\allowbreak5 Lu 4:23}
\crossref{Prov}{26}{8}{}
\crossref{Prov}{26}{9}{Pr 23:35}
\crossref{Prov}{26}{10}{Pr 11:31 Ro 2:6}
\crossref{Prov}{26}{11}{Ex 8:15 Mt 12:45 2Pe 2:22}
\crossref{Prov}{26}{12}{Pr 22:29; 29:20 Mt 21:31 Lu 7:44}
\crossref{Prov}{26}{13}{Pr 15:19; 19:15; 22:13}
\crossref{Prov}{26}{14}{Pr 6:9,\allowbreak10; 12:24,\allowbreak27; 24:33 Heb 6:12}
\crossref{Prov}{26}{15}{Pr 19:24}
\crossref{Prov}{26}{16}{26:12; 12:15 1Pe 3:15}
\crossref{Prov}{26}{17}{Pr 17:11; 18:6; 20:3 Lu 12:14 2Ti 2:23,\allowbreak24}
\crossref{Prov}{26}{18}{Pr 7:23; 25:18 Ge 49:23}
\crossref{Prov}{26}{19}{Pr 10:23; 14:9; 15:21 Eph 5:4 2Pe 2:13}
\crossref{Prov}{26}{20}{26:22; 16:28; 22:10 Jas 3:6}
\crossref{Prov}{26}{21}{Pr 10:12; 15:18; 29:22; 30:33 2Sa 20:1 1Ki 12:2,\allowbreak3,\allowbreak20 Ps 120:4}
\crossref{Prov}{26}{22}{Pr 18:8; 20:19 Eze 22:9}
\crossref{Prov}{26}{23}{}
\crossref{Prov}{26}{24}{Pr 11:1; 12:5,\allowbreak17,\allowbreak20; 14:8}
\crossref{Prov}{26}{25}{Ps 12:2; 28:3 Jer 9:2-\allowbreak8 Mic 7:5}
\crossref{Prov}{26}{26}{}
\crossref{Prov}{26}{27}{Pr 28:10 Es 7:10 Ps 7:15,\allowbreak16; 9:15; 10:2; 57:6 Ec 10:8}
\crossref{Prov}{26}{28}{Pr 6:24; 7:5,\allowbreak21-\allowbreak23; 29:5 Lu 20:20,\allowbreak21}
\crossref{Prov}{27}{1}{Ps 95:7 Isa 56:12 Lu 12:19,\allowbreak20 2Co 6:2 Jas 4:13-\allowbreak16}
\crossref{Prov}{27}{2}{Pr 25:27 2Co 10:12,\allowbreak18; 12:11}
\crossref{Prov}{27}{3}{Pr 17:12 Ge 34:25,\allowbreak26; 49:7 1Sa 22:18,\allowbreak19 Es 3:5,\allowbreak6 Da 3:19 1Jo 3:12}
\crossref{Prov}{27}{4}{Pr 14:30 Ge 26:14; 37:11 Job 5:2 Mt 27:18 Ac 5:17}
\crossref{Prov}{27}{5}{Pr 28:23 Le 19:17 Mt 18:15 Ga 2:14 1Ti 5:20}
\crossref{Prov}{27}{6}{2Sa 12:7-\allowbreak15 Job 5:17,\allowbreak18 Ps 141:5 Heb 12:10 Re 3:19}
\crossref{Prov}{27}{7}{Nu 11:4-\allowbreak9,\allowbreak18-\allowbreak20; 21:5}
\crossref{Prov}{27}{8}{Job 39:14-\allowbreak16 Isa 16:2}
\crossref{Prov}{27}{9}{Pr 7:17 Jud 9:9 Ps 45:7,\allowbreak8; 104:15; 133:2 So 1:3; 3:6; 4:10 Joh 12:3}
\crossref{Prov}{27}{10}{2Sa 19:24,\allowbreak28; 21:7 1Ki 12:6-\allowbreak8 2Ch 24:22 Isa 41:8-\allowbreak10 Jer 2:5}
\crossref{Prov}{27}{11}{Pr 10:1; 15:20; 23:15,\allowbreak16,\allowbreak24,\allowbreak25 Ec 2:18-\allowbreak21 Phm 1:7,\allowbreak19,\allowbreak20 2Jo 1:4}
\crossref{Prov}{27}{12}{Pr 18:10; 22:3 Ex 9:20,\allowbreak21 Ps 57:1-\allowbreak3 Isa 26:20,\allowbreak21 Mt 3:7 Heb 6:18}
\crossref{Prov}{27}{13}{Pr 6:1-\allowbreak4; 20:16; 22:26,\allowbreak27 Ex 22:26}
\crossref{Prov}{27}{14}{2Sa 15:2-\allowbreak7; 16:16-\allowbreak19; 17:7-\allowbreak13 1Ki 22:6,\allowbreak13 Jer 28:2-\allowbreak4}
\crossref{Prov}{27}{15}{Pr 19:13; 21:9,\allowbreak19; 25:24 Job 14:19}
\crossref{Prov}{27}{16}{Joh 12:3}
\crossref{Prov}{27}{17}{1Sa 13:20,\allowbreak21}
\crossref{Prov}{27}{18}{So 8:12 1Co 9:7,\allowbreak13}
\crossref{Prov}{27}{19}{Jas 1:22-\allowbreak25}
\crossref{Prov}{27}{20}{Pr 30:15,\allowbreak16 Hab 2:5}
\crossref{Prov}{27}{21}{Pr 17:3 Ps 12:6; 66:10 Zec 13:9 Mal 3:3 1Pe 1:7; 4:12}
\crossref{Prov}{27}{22}{Pr 23:35 Ex 12:30; 14:5; 15:9 2Ch 28:22,\allowbreak23 Isa 1:5 Jer 5:3}
\crossref{Prov}{27}{23}{Ge 31:38-\allowbreak40; 33:13 1Sa 17:28 1Ch 27:29-\allowbreak31 2Ch 26:10}
\crossref{Prov}{27}{24}{Pr 23:5 Zep 1:18 1Ti 6:17,\allowbreak18}
\crossref{Prov}{27}{25}{}
\crossref{Prov}{27}{26}{Job 31:20}
\crossref{Prov}{27}{27}{Pr 30:8,\allowbreak9 Mt 6:33}
\crossref{Prov}{28}{1}{Le 26:17,\allowbreak36 De 28:7,\allowbreak25 2Ki 7:6,\allowbreak7,\allowbreak15 Ps 53:5 Isa 7:2 Jer 20:4}
\crossref{Prov}{28}{2}{1Ki 15:25,\allowbreak28; 16:8-\allowbreak29 2Ki 15:8-\allowbreak31 2Ch 36:1-\allowbreak12 Isa 3:1-\allowbreak7}
\crossref{Prov}{28}{3}{Mt 18:28-\allowbreak30}
\crossref{Prov}{28}{4}{1Sa 23:19-\allowbreak21 Ps 10:3; 49:18 Jer 5:30 Mt 3:15 Ac 12:22; 24:2-\allowbreak4}
\crossref{Prov}{28}{5}{Pr 15:24; 24:7 Ps 25:14; 92:6 Jer 4:22 Mr 4:10-\allowbreak13 Joh 7:17}
\crossref{Prov}{28}{6}{28:18; 16:8; 19:1,\allowbreak22 Lu 16:19-\allowbreak23 Ac 24:24-\allowbreak27}
\crossref{Prov}{28}{7}{Pr 2:1-\allowbreak6; 3:1-\allowbreak35}
\crossref{Prov}{28}{8}{Pr 13:22 Job 27:16,\allowbreak17 Ec 2:26}
\crossref{Prov}{28}{9}{Pr 21:13 Isa 1:15,\allowbreak16; 58:7-\allowbreak11 Zec 7:11-\allowbreak13 2Ti 4:3,\allowbreak4}
\crossref{Prov}{28}{10}{Nu 31:15,\allowbreak16 1Sa 26:19 Ac 13:8-\allowbreak10 Ro 16:17,\allowbreak18 2Co 11:3,\allowbreak4,\allowbreak13-\allowbreak15}
\crossref{Prov}{28}{11}{Pr 18:11; 23:4 Isa 10:13,\allowbreak14 Eze 28:3-\allowbreak5 Lu 16:13,\allowbreak14 1Co 3:18,\allowbreak19}
\crossref{Prov}{28}{12}{28:28; 11:10; 29:2 1Ch 15:25-\allowbreak28; 16:7-\allowbreak36; 29:20-\allowbreak22 2Ch 7:10; 30:22-\allowbreak27}
\crossref{Prov}{28}{13}{Pr 10:12; 17:9 Ge 3:12,\allowbreak13; 4:9 1Sa 15:13,\allowbreak24 Job 31:33 Ps 32:3-\allowbreak5}
\crossref{Prov}{28}{14}{Pr 23:17 Ps 2:11; 16:8; 112:1 Isa 66:2 Jer 32:40 Ro 11:20 Heb 4:1}
\crossref{Prov}{28}{15}{Pr 20:2 Ho 5:11 1Pe 5:8}
\crossref{Prov}{28}{16}{1Ki 12:10,\allowbreak14 Ne 5:15 Ec 4:1 Isa 3:12 Am 4:1}
\crossref{Prov}{28}{17}{Ge 9:6 Ex 21:14 Nu 35:14-\allowbreak34 1Ki 21:19,\allowbreak23 2Ki 9:26}
\crossref{Prov}{28}{18}{Pr 10:9,\allowbreak25; 11:3-\allowbreak6 Ps 25:21; 26:11; 84:11 Ga 2:14}
\crossref{Prov}{28}{19}{Pr 12:11; 14:4; 27:23-\allowbreak27}
\crossref{Prov}{28}{20}{Pr 20:6 1Sa 22:14 Ne 7:2 Ps 101:6; 112:4-\allowbreak9 Lu 12:42; 16:1,\allowbreak10-\allowbreak12}
\crossref{Prov}{28}{21}{Pr 18:5; 24:23 Ex 23:2,\allowbreak8}
\crossref{Prov}{28}{22}{Pr 23:6 Mt 20:15 Mr 7:22}
\crossref{Prov}{28}{23}{Pr 27:5,\allowbreak6 2Sa 12:7 1Ki 1:23,\allowbreak32-\allowbreak40 Ps 141:5 Mt 18:15 Ga 2:11}
\crossref{Prov}{28}{24}{Pr 19:26 Jud 17:2 Mt 15:4-\allowbreak6}
\crossref{Prov}{28}{25}{Pr 10:12; 13:10; 15:18; 21:24; 22:10; 29:22}
\crossref{Prov}{28}{26}{Pr 3:5 2Ki 8:13 Jer 17:9 Mr 7:21-\allowbreak23; 14:27-\allowbreak31 Ro 8:7}
\crossref{Prov}{28}{27}{Pr 19:17; 22:9 De 15:7,\allowbreak10 Ps 41:1-\allowbreak3; 112:5-\allowbreak9 2Co 9:6-\allowbreak11 Heb 13:16}
\crossref{Prov}{28}{28}{28:12; 29:2}
\crossref{Prov}{29}{1}{Pr 1:24-\allowbreak31 1Sa 2:25,\allowbreak34 1Ki 17:1; 18:18; 20:42; 21:20-\allowbreak23}
\crossref{Prov}{29}{2}{Pr 11:10; 28:12,\allowbreak28 Es 8:15 Ps 72:1-\allowbreak7 Isa 32:1,\allowbreak2 Jer 23:5,\allowbreak6}
\crossref{Prov}{29}{3}{Pr 10:1; 15:20; 23:15,\allowbreak24,\allowbreak25; 27:11 Lu 1:13-\allowbreak17}
\crossref{Prov}{29}{4}{29:14; 16:12; 20:8 1Sa 13:13 2Sa 8:15 1Ki 2:12 Ps 89:14; 99:4}
\crossref{Prov}{29}{5}{Pr 7:5,\allowbreak21; 20:19; 26:24,\allowbreak25,\allowbreak28 2Sa 14:17-\allowbreak24 Job 17:5 Ps 5:9; 12:2}
\crossref{Prov}{29}{6}{Pr 5:22; 11:5,\allowbreak6; 12:13 Job 18:7-\allowbreak10 Ps 11:6 Isa 8:14,\allowbreak15 2Ti 2:26}
\crossref{Prov}{29}{7}{Job 29:16; 31:13,\allowbreak21 Ps 31:7; 41:1 Ga 6:1}
\crossref{Prov}{29}{8}{Pr 11:11 Isa 28:14-\allowbreak22 Mt 27:39-\allowbreak43 Joh 9:40,\allowbreak41; 11:47-\allowbreak50}
\crossref{Prov}{29}{9}{Pr 26:4 Ec 10:13 Mt 7:6; 11:17-\allowbreak19}
\crossref{Prov}{29}{10}{Ge 4:5-\allowbreak8 1Sa 20:31-\allowbreak33; 22:11-\allowbreak23 1Ki 21:20; 22:8 2Ch 18:7}
\crossref{Prov}{29}{11}{Pr 12:16,\allowbreak23; 14:33 Jud 16:17 Am 5:13 Mic 7:5}
\crossref{Prov}{29}{12}{Pr 20:8; 25:23 1Sa 22:8-\allowbreak23; 23:19-\allowbreak23 2Sa 3:7-\allowbreak11; 4:5-\allowbreak12}
\crossref{Prov}{29}{13}{Mt 9:9 1Co 6:10}
\crossref{Prov}{29}{14}{29:4; 16:12; 20:28; 25:5; 28:16 Job 29:11-\allowbreak18 Ps 72:2-\allowbreak4,\allowbreak12-\allowbreak14; 82:2,\allowbreak3}
\crossref{Prov}{29}{15}{29:17,\allowbreak21; 22:6,\allowbreak15; 23:13,\allowbreak14 Heb 12:10,\allowbreak11}
\crossref{Prov}{29}{16}{29:2}
\crossref{Prov}{29}{17}{29:15; 13:24; 19:18; 22:15; 23:13,\allowbreak14}
\crossref{Prov}{29}{18}{1Sa 3:1 Ho 4:6 Am 8:11,\allowbreak12 Mt 9:36 Ro 10:13-\allowbreak15}
\crossref{Prov}{29}{19}{Pr 26:3; 30:22}
\crossref{Prov}{29}{20}{29:11 Ec 5:2 Jas 1:19}
\crossref{Prov}{29}{21}{29:19 La 3:27 Eph 6:9 Col 4:1}
\crossref{Prov}{29}{22}{Pr 10:12; 15:18; 17:19; 26:21; 30:33}
\crossref{Prov}{29}{23}{Pr 18:12 2Ch 32:25,\allowbreak26; 33:10-\allowbreak12,\allowbreak23,\allowbreak24 Job 22:29; 40:12 Isa 2:11,\allowbreak12}
\crossref{Prov}{29}{24}{Pr 1:11-\allowbreak19 Ps 50:18-\allowbreak22 Isa 1:23 Mr 11:17}
\crossref{Prov}{29}{25}{Ge 12:11-\allowbreak13; 20:2,\allowbreak11; 26:7 Ex 32:22-\allowbreak24 1Sa 15:24; 27:1,\allowbreak11}
\crossref{Prov}{29}{26}{Pr 19:6 Ps 20:9}
\crossref{Prov}{29}{27}{Pr 24:9 Ps 119:115; 139:21 Zec 11:8 Joh 7:7; 15:17-\allowbreak19,\allowbreak23}
\crossref{Prov}{30}{1}{Pr 31:1 2Pe 1:19-\allowbreak21}
\crossref{Prov}{30}{2}{Job 42:3-\allowbreak6 Ps 73:22 Isa 6:5 Ro 11:25 1Co 3:18; 8:2 Jas 1:5}
\crossref{Prov}{30}{3}{Am 7:14,\allowbreak15 Mt 16:17}
\crossref{Prov}{30}{4}{De 30:12 Joh 3:13 Ro 10:6 Eph 4:9,\allowbreak10}
\crossref{Prov}{30}{5}{Ps 12:6; 18:30; 19:8; 119:140 Ro 7:12 Jas 3:17}
\crossref{Prov}{30}{6}{De 4:2; 12:32 Re 22:18,\allowbreak19}
\crossref{Prov}{30}{7}{1Ki 3:5-\allowbreak9 2Ki 2:9 Ps 27:4 Lu 10:42}
\crossref{Prov}{30}{8}{Pr 21:6; 22:8; 23:5 Ps 62:9,\allowbreak10; 119:29,\allowbreak37 Ec 1:2 Isa 5:18; 59:4}
\crossref{Prov}{30}{9}{De 6:10-\allowbreak12; 8:10-\allowbreak14; 31:20; 32:15 Ne 9:25,\allowbreak26 Job 31:24-\allowbreak28}
\crossref{Prov}{30}{10}{Pr 24:23 De 23:15 1Sa 22:9,\allowbreak10; 24:9; 26:19; 30:15 2Sa 16:1-\allowbreak4}
\crossref{Prov}{30}{11}{30:12-\allowbreak14 Mt 3:7 1Pe 2:9}
\crossref{Prov}{30}{12}{Pr 21:2 Jud 17:5,\allowbreak13 1Sa 15:13,\allowbreak14 Job 33:9 Ps 36:2 Isa 65:5}
\crossref{Prov}{30}{13}{Pr 6:17; 21:4 Ps 101:5; 131:1 Isa 2:11; 3:16 Eze 28:2-\allowbreak5,\allowbreak9}
\crossref{Prov}{30}{14}{Pr 12:18 Job 29:17 Ps 3:7; 52:2; 57:4; 58:6 Da 7:5-\allowbreak7 Re 9:8}
\crossref{Prov}{30}{15}{Isa 57:3 Eze 16:44-\allowbreak46 Mt 23:32 Joh 8:39,\allowbreak44}
\crossref{Prov}{30}{16}{Pr 27:20 Hab 2:5}
\crossref{Prov}{30}{17}{30:11; 20:20; 23:22 Ge 9:21-\allowbreak27 Le 20:9 De 21:18-\allowbreak21}
\crossref{Prov}{30}{18}{Job 42:3 Ps 139:6}
\crossref{Prov}{30}{19}{Job 39:27 Isa 40:31}
\crossref{Prov}{30}{20}{Pr 7:13-\allowbreak23 Nu 5:11-\allowbreak30}
\crossref{Prov}{30}{21}{Job 40:5}
\crossref{Prov}{30}{22}{Pr 19:10; 28:3 Ec 10:7 Isa 3:4,\allowbreak5}
\crossref{Prov}{30}{23}{Pr 19:13; 21:9,\allowbreak19; 27:15}
\crossref{Prov}{30}{24}{Job 12:7}
\crossref{Prov}{30}{25}{Pr 6:6-\allowbreak8}
\crossref{Prov}{30}{26}{Le 11:5 Ps 104:18}
\crossref{Prov}{30}{27}{Ex 10:4-\allowbreak6,\allowbreak13-\allowbreak15 Ps 105:34 Joe 1:4,\allowbreak6,\allowbreak7; 2:7-\allowbreak11,\allowbreak25 Re 9:3-\allowbreak11}
\crossref{Prov}{30}{28}{Ps 144:12 Isa 13:22 Da 1:4 Eph 2:6}
\crossref{Prov}{30}{29}{30:21 Job 40:5}
\crossref{Prov}{30}{30}{Nu 23:24 Jud 14:18}
\crossref{Prov}{30}{31}{Pr 16:14; 20:2 Da 3:15-\allowbreak18}
\crossref{Prov}{30}{32}{Pr 26:12 Ec 8:3}
\crossref{Prov}{30}{33}{Pr 15:18; 16:28; 17:14; 26:21; 28:25; 29:22}
\crossref{Prov}{31}{1}{Pr 30:1}
\crossref{Prov}{31}{2}{Isa 49:15}
\crossref{Prov}{31}{3}{Pr 5:9-\allowbreak11; 7:26,\allowbreak27 Ho 4:11}
\crossref{Prov}{31}{4}{Le 10:9,\allowbreak10 1Ki 20:12,\allowbreak16-\allowbreak20 Es 3:15 Ec 10:17 Isa 28:7,\allowbreak8}
\crossref{Prov}{31}{5}{}
\crossref{Prov}{31}{6}{Ps 104:15 1Ti 5:23}
\crossref{Prov}{31}{7}{Eph 5:18}
\crossref{Prov}{31}{8}{Pr 24:7,\allowbreak11,\allowbreak12 1Sa 19:4-\allowbreak7; 20:32; 22:14,\allowbreak15 Es 4:13-\allowbreak16 Job 29:9,\allowbreak17}
\crossref{Prov}{31}{9}{Pr 16:12; 20:8 Le 19:15 De 1:16; 16:18-\allowbreak20 2Sa 8:15 Ps 58:1,\allowbreak2}
\crossref{Prov}{31}{10}{Pr 12:4; 18:22; 19:14 Ru 3:11 Ec 7:28 So 6:8,\allowbreak9 Eph 5:25-\allowbreak33}
\crossref{Prov}{31}{11}{2Ki 4:9,\allowbreak10,\allowbreak22,\allowbreak23 1Pe 3:1-\allowbreak7}
\crossref{Prov}{31}{12}{1Sa 25:18-\allowbreak22,\allowbreak26,\allowbreak27}
\crossref{Prov}{31}{13}{Ge 18:6-\allowbreak8; 24:13,\allowbreak14,\allowbreak18-\allowbreak20; 29:9,\allowbreak10 Ex 2:16 Ru 2:2,\allowbreak3,\allowbreak23}
\crossref{Prov}{31}{14}{31:24 1Ki 9:26-\allowbreak28 2Ch 9:10 Eze 27:3-\allowbreak36}
\crossref{Prov}{31}{15}{Jos 3:1 2Ch 36:15 Ps 119:147,\allowbreak148 Ec 9:10 Mr 1:35 Ro 12:11}
\crossref{Prov}{31}{16}{Jos 15:18 So 8:12 Mt 13:44}
\crossref{Prov}{31}{17}{1Ki 18:46 2Ki 4:29 Job 38:3 Lu 12:35 Eph 6:10,\allowbreak14 1Pe 1:13}
\crossref{Prov}{31}{18}{Ge 31:40 Ps 127:2 Mt 25:3-\allowbreak10 1Th 2:9 2Th 3:7-\allowbreak9}
\crossref{Prov}{31}{19}{}
\crossref{Prov}{31}{20}{Pr 1:24 Ro 10:21}
\crossref{Prov}{31}{21}{Pr 25:20}
\crossref{Prov}{31}{22}{Pr 7:16}
\crossref{Prov}{31}{23}{Pr 12:4}
\crossref{Prov}{31}{24}{31:13,\allowbreak19 1Ki 10:28 Eze 27:16 Lu 16:19}
\crossref{Prov}{31}{25}{Job 29:14; 40:10 Ps 132:9,\allowbreak16 Isa 61:10 Ro 13:14 Eph 4:24}
\crossref{Prov}{31}{26}{31:8,\allowbreak9 Jud 13:23 1Sa 25:24-\allowbreak31 2Sa 20:16-\allowbreak22 2Ki 22:15-\allowbreak20 Es 4:4}
\crossref{Prov}{31}{27}{Pr 14:1 1Th 4:11 2Th 3:6 1Ti 5:10 Tit 2:4}
\crossref{Prov}{31}{28}{31:1 1Ki 2:19 Ps 116:16 2Ti 1:5; 3:15-\allowbreak17}
\crossref{Prov}{31}{29}{So 6:8,\allowbreak9 Eph 5:27}
\crossref{Prov}{31}{30}{Pr 6:25; 11:22 2Sa 14:25 Es 1:11,\allowbreak12 Eze 16:15 Jas 1:11 1Pe 1:24}
\crossref{Prov}{31}{31}{31:16; 11:30 Ps 128:2 Mt 7:16,\allowbreak20 Ro 6:21,\allowbreak22 Php 4:17}

% Eccl
\crossref{Eccl}{1}{1}{1:12; 7:27; 12:8-\allowbreak10 Ne 6:7 Ps 40:9 Isa 61:1 Jon 3:2 2Pe 2:5}
\crossref{Eccl}{1}{2}{Ec 2:11,\allowbreak15,\allowbreak17,\allowbreak19,\allowbreak21,\allowbreak23,\allowbreak26; 3:19; 4:4,\allowbreak8,\allowbreak16; 5:10; 6:11; 11:8,\allowbreak10; 12:8}
\crossref{Eccl}{1}{3}{Ec 2:22; 3:9; 5:16 Pr 23:4,\allowbreak5 Isa 55:2 Hab 2:13,\allowbreak18 Mt 16:26}
\crossref{Eccl}{1}{4}{Ec 6:12 Ge 5:3-\allowbreak31; 11:20-\allowbreak32; 36:9-\allowbreak19; 47:9 Ex 1:6,\allowbreak7; 6:16-\allowbreak27}
\crossref{Eccl}{1}{5}{Ge 8:22 Ps 19:4-\allowbreak6; 89:36,\allowbreak37; 104:19-\allowbreak23 Jer 33:20}
\crossref{Eccl}{1}{6}{}
\crossref{Eccl}{1}{7}{Job 38:10,\allowbreak11 Ps 104:6-\allowbreak9}
\crossref{Eccl}{1}{8}{Ec 2:11,\allowbreak26 Mt 11:28 Ro 8:22,\allowbreak23}
\crossref{Eccl}{1}{9}{Ec 3:15; 7:10 2Pe 2:1}
\crossref{Eccl}{1}{10}{Mt 5:12; 23:30-\allowbreak32 Lu 17:26-\allowbreak30 Ac 7:51 1Th 2:14-\allowbreak16 2Ti 3:8}
\crossref{Eccl}{1}{11}{Ec 2:16 Ps 9:6 Isa 41:22-\allowbreak26; 42:9}
\crossref{Eccl}{1}{12}{1:1 1Ki 4:1-\allowbreak19}
\crossref{Eccl}{1}{13}{1:17; 7:25; 8:9,\allowbreak16,\allowbreak17 Ps 111:2 Pr 2:2-\allowbreak4; 4:7; 18:1,\allowbreak15; 23:26}
\crossref{Eccl}{1}{14}{1:17,\allowbreak18; 2:11,\allowbreak17,\allowbreak26 1Ki 4:30-\allowbreak32 Ps 39:5,\allowbreak6}
\crossref{Eccl}{1}{15}{Ec 3:14; 7:12,\allowbreak13 Job 11:6; 34:29 Isa 40:4 La 3:37 Da 4:35 Mt 6:27}
\crossref{Eccl}{1}{16}{2Ki 5:20 Ps 4:4; 77:6 Isa 10:7-\allowbreak14 Jer 22:14 Eze 38:10,\allowbreak11}
\crossref{Eccl}{1}{17}{1:13; 2:3,\allowbreak12; 7:23-\allowbreak25 1Th 5:21}
\crossref{Eccl}{1}{18}{Ec 2:15; 7:16; 12:12,\allowbreak13 Job 28:28 1Co 3:18-\allowbreak20 Jas 3:13-\allowbreak17}
\crossref{Eccl}{2}{1}{2:15; 1:16,\allowbreak17; 3:17,\allowbreak18 Ps 10:6; 14:1; 27:8; 30:6,\allowbreak7 Lu 12:19}
\crossref{Eccl}{2}{2}{Ec 7:2-\allowbreak6 Pr 14:13 Isa 22:12,\allowbreak13 Am 6:3-\allowbreak6 1Pe 4:2-\allowbreak4}
\crossref{Eccl}{2}{3}{Ec 1:17 1Sa 25:36}
\crossref{Eccl}{2}{4}{Ge 11:4 2Sa 18:18 Da 4:30}
\crossref{Eccl}{2}{5}{So 4:12-\allowbreak16; 5:1; 6:2 Jer 39:4}
\crossref{Eccl}{2}{6}{Ne 2:14 So 7:4}
\crossref{Eccl}{2}{7}{1Ki 9:20-\allowbreak22 Ezr 2:58 Ne 7:57}
\crossref{Eccl}{2}{8}{1Ki 9:14,\allowbreak28; 10:10; 14:21,\allowbreak22,\allowbreak27 2Ch 9:11,\allowbreak15-\allowbreak21}
\crossref{Eccl}{2}{9}{Ec 1:16 1Ki 3:12; 10:7,\allowbreak23 1Ch 29:25 2Ch 1:1; 9:22,\allowbreak23}
\crossref{Eccl}{2}{10}{Ec 3:22; 6:9; 11:9 Ge 3:6; 6:2 Jud 14:2 Job 31:1 Ps 119:37 Pr 23:5}
\crossref{Eccl}{2}{11}{Ec 1:14 Ge 1:31 Ex 39:43 1Jo 2:16,\allowbreak17}
\crossref{Eccl}{2}{12}{Ec 1:17; 7:25}
\crossref{Eccl}{2}{13}{Ec 7:11,\allowbreak12; 9:16 Pr 4:5-\allowbreak7; 16:16 Mal 3:18; 4:1,\allowbreak2}
\crossref{Eccl}{2}{14}{Ec 8:1; 10:2,\allowbreak3 Pr 14:8; 17:24 1Jo 2:11}
\crossref{Eccl}{2}{15}{Ec 1:16,\allowbreak18 1Ki 3:12}
\crossref{Eccl}{2}{16}{Ec 1:11 Ex 1:6,\allowbreak8 Ps 88:12; 103:16 Mal 3:16}
\crossref{Eccl}{2}{17}{Nu 11:15 1Ki 19:4 Job 3:20-\allowbreak22; 7:15,\allowbreak16; 14:13 Jer 20:14-\allowbreak18}
\crossref{Eccl}{2}{18}{2:4-\allowbreak9; 1:13; 4:3; 5:18; 9:9}
\crossref{Eccl}{2}{19}{Ec 3:22 1Ki 12:14-\allowbreak20; 14:25-\allowbreak28 2Ch 10:13-\allowbreak16; 12:9,\allowbreak10}
\crossref{Eccl}{2}{20}{Ge 43:14 Job 17:11-\allowbreak15 Ps 39:6,\allowbreak7 1Co 15:19 2Co 1:8-\allowbreak10}
\crossref{Eccl}{2}{21}{2:17,\allowbreak18; 9:18 2Ch 31:20,\allowbreak21; 33:2-\allowbreak9; 34:2; 35:18; 36:5-\allowbreak10}
\crossref{Eccl}{2}{22}{Ec 1:3; 3:9; 5:10,\allowbreak11,\allowbreak17; 6:7,\allowbreak8; 8:15 Pr 16:26 1Ti 6:8}
\crossref{Eccl}{2}{23}{Ge 47:9 Job 5:7; 14:1 Ps 90:7-\allowbreak10,\allowbreak15; 127:2}
\crossref{Eccl}{2}{24}{Ec 3:12,\allowbreak13,\allowbreak22; 5:18; 8:15; 9:7-\allowbreak9; 11:9,\allowbreak10 De 12:12,\allowbreak18 Ne 8:10}
\crossref{Eccl}{2}{25}{2:1-\allowbreak12 1Ki 4:21-\allowbreak24}
\crossref{Eccl}{2}{26}{Ge 7:1 Lu 1:6}
\crossref{Eccl}{3}{1}{3:17; 7:14; 8:5,\allowbreak6 2Ki 5:26 2Ch 33:12 Pr 15:23 Mt 16:3}
\crossref{Eccl}{3}{2}{Ge 17:21; 21:1,\allowbreak2 1Sa 2:5 1Ki 13:2 2Ki 4:16 Ps 113:9 Isa 54:1}
\crossref{Eccl}{3}{3}{De 32:39 1Sa 2:6,\allowbreak25 Ho 6:1,\allowbreak2}
\crossref{Eccl}{3}{4}{Ne 8:9-\allowbreak12; 9:1-\allowbreak38 Ps 30:5; 126:1,\allowbreak2,\allowbreak5,\allowbreak6 Isa 22:12,\allowbreak13 Mt 9:15}
\crossref{Eccl}{3}{5}{Jos 4:3-\allowbreak9; 10:27 2Sa 18:17,\allowbreak18 2Ki 3:25}
\crossref{Eccl}{3}{6}{Ge 30:30-\allowbreak43; 31:18 Ex 12:35,\allowbreak36 De 8:17,\allowbreak18 2Ki 5:26; 8:9}
\crossref{Eccl}{3}{7}{Ge 37:29,\allowbreak34 2Sa 1:11; 3:31 1Ki 21:27 2Ki 5:7; 6:30 Jer 36:24}
\crossref{Eccl}{3}{8}{Eze 16:8 Ps 139:21 Eph 3:19; 5:25,\allowbreak28,\allowbreak29 Tit 2:4}
\crossref{Eccl}{3}{9}{Ec 1:3; 2:11,\allowbreak22,\allowbreak23; 5:16 Pr 14:23 Mt 16:26}
\crossref{Eccl}{3}{10}{Ec 1:13,\allowbreak14; 2:26 Ge 3:19 1Th 2:9 2Th 3:8}
\crossref{Eccl}{3}{11}{Ec 7:29 Ge 1:31 De 32:4 Mr 7:37}
\crossref{Eccl}{3}{12}{3:22; 9:7-\allowbreak9 De 28:63 Ps 37:3 Isa 64:5 Lu 11:41 Ac 20:35}
\crossref{Eccl}{3}{13}{Ec 2:24; 5:18-\allowbreak20; 6:2; 9:7 De 28:30,\allowbreak31,\allowbreak47,\allowbreak48 Jud 6:3-\allowbreak6 Ps 128:2}
\crossref{Eccl}{3}{14}{Ps 33:11; 119:90,\allowbreak91 Isa 46:10 Da 4:34,\allowbreak35 Ac 2:23; 4:28 Ro 11:36}
\crossref{Eccl}{3}{15}{Ec 1:9,\allowbreak10}
\crossref{Eccl}{3}{16}{Ec 4:1; 5:8 1Ki 21:9-\allowbreak21 Ps 58:1,\allowbreak2; 82:2-\allowbreak5; 94:21,\allowbreak22 Isa 59:14}
\crossref{Eccl}{3}{17}{Ec 1:16; 2:1}
\crossref{Eccl}{3}{18}{Ge 3:17-\allowbreak19 Job 14:1-\allowbreak4; 15:16 Ps 49:14,\allowbreak19,\allowbreak20; 73:18,\allowbreak19; 90:5-\allowbreak12}
\crossref{Eccl}{3}{19}{Ec 2:16 Ps 49:12,\allowbreak20; 92:6,\allowbreak7}
\crossref{Eccl}{3}{20}{3:21; 6:6; 9:10 Ge 25:8,\allowbreak17 Nu 27:13 Job 7:9; 17:13; 30:24 Ps 49:14}
\crossref{Eccl}{3}{21}{Ec 12:7 Lu 16:22,\allowbreak23 Joh 14:3 Ac 1:25 2Co 5:1,\allowbreak8 Php 1:23}
\crossref{Eccl}{3}{22}{3:11,\allowbreak12; 2:10,\allowbreak11,\allowbreak24; 5:18-\allowbreak20; 8:15; 9:7-\allowbreak9; 11:9 De 12:7,\allowbreak18; 26:10,\allowbreak11}
\crossref{Eccl}{4}{1}{Job 6:29 Mal 3:18}
\crossref{Eccl}{4}{2}{Job 6:29 Mal 3:18}
\crossref{Eccl}{4}{3}{Ec 2:17; 9:4-\allowbreak6 Job 3:17-\allowbreak21}
\crossref{Eccl}{4}{4}{Ec 6:3-\allowbreak5 Job 3:10-\allowbreak16; 10:18,\allowbreak19 Jer 20:17,\allowbreak18 Mt 24:19 Lu 23:29}
\crossref{Eccl}{4}{5}{4:16; 1:14; 2:21,\allowbreak26; 6:9,\allowbreak11 Ge 37:4,\allowbreak11}
\crossref{Eccl}{4}{6}{Pr 6:10,\allowbreak11; 12:27; 13:4; 20:4; 24:33,\allowbreak34}
\crossref{Eccl}{4}{7}{Ps 37:16 Pr 15:16,\allowbreak17; 16:8; 17:1}
\crossref{Eccl}{4}{8}{4:1 Ps 78:33 Zec 1:6}
\crossref{Eccl}{4}{9}{4:9-\allowbreak12 Ge 2:18 Isa 56:3-\allowbreak5}
\crossref{Eccl}{4}{10}{Ge 2:18 Ex 4:14-\allowbreak16 Nu 11:14 Pr 27:17 Hag 1:14 Mr 6:7 Ac 13:2}
\crossref{Eccl}{4}{11}{Ex 32:2-\allowbreak4,\allowbreak21 De 9:19,\allowbreak20 1Sa 23:16 2Sa 11:27; 12:7-\allowbreak14 Job 4:3,\allowbreak4}
\crossref{Eccl}{4}{12}{1Ki 1:1,\allowbreak2}
\crossref{Eccl}{4}{13}{2Sa 23:9,\allowbreak16,\allowbreak18,\allowbreak19,\allowbreak23 Da 3:16,\allowbreak17 Eph 4:3}
\crossref{Eccl}{4}{14}{Ec 9:15,\allowbreak16 Ge 37:2 Pr 19:1; 28:6,\allowbreak15,\allowbreak16}
\crossref{Eccl}{4}{15}{1Ki 14:26,\allowbreak27 2Ki 23:31-\allowbreak34; 24:1,\allowbreak2,\allowbreak6,\allowbreak12; 25:7,\allowbreak27-\allowbreak30 La 4:20}
\crossref{Eccl}{4}{16}{2Sa 15:6}
\crossref{Eccl}{5}{1}{Ge 28:16,\allowbreak17 Ex 3:5 Le 10:3 Jos 5:15 2Ch 26:16 Ps 89:7}
\crossref{Eccl}{5}{2}{Ge 18:27,\allowbreak30,\allowbreak32; 28:20,\allowbreak22 Nu 30:2-\allowbreak5 Jud 11:30 1Sa 14:24-\allowbreak45}
\crossref{Eccl}{5}{3}{Ec 10:12-\allowbreak14 Pr 10:19; 15:2}
\crossref{Eccl}{5}{4}{Ge 28:20; 35:1,\allowbreak3 Nu 30:2 De 23:21-\allowbreak23 Ps 50:14; 76:11; 119:106}
\crossref{Eccl}{5}{5}{De 23:22 Pr 20:25 Ac 5:4}
\crossref{Eccl}{5}{6}{5:1,\allowbreak2 Jas 1:26; 3:2}
\crossref{Eccl}{5}{7}{5:3 Mt 12:36}
\crossref{Eccl}{5}{8}{Ec 3:16; 4:1 Ps 12:5; 55:9; 58:11 Pr 8:17 Hab 1:2,\allowbreak3,\allowbreak13}
\crossref{Eccl}{5}{9}{Ge 1:29,\allowbreak30; 3:17-\allowbreak19 Ps 104:14,\allowbreak15; 115:16 Pr 13:23; 27:23-\allowbreak27}
\crossref{Eccl}{5}{10}{Ec 1:17; 2:11,\allowbreak17,\allowbreak18,\allowbreak26; 3:19; 4:4,\allowbreak8,\allowbreak16}
\crossref{Eccl}{5}{11}{Ge 12:16; 13:2,\allowbreak5-\allowbreak7 1Ki 4:22,\allowbreak23; 5:13-\allowbreak16 Ne 5:17,\allowbreak18 Ps 119:36,\allowbreak37}
\crossref{Eccl}{5}{12}{Ps 4:8; 127:2 Pr 3:24 Jer 31:26}
\crossref{Eccl}{5}{13}{Ec 4:8; 6:1,\allowbreak2}
\crossref{Eccl}{5}{14}{Ec 2:26 Job 5:5; 20:15-\allowbreak29; 27:16,\allowbreak17 Ps 39:6 Pr 23:5 Hag 1:9}
\crossref{Eccl}{5}{15}{Job 1:21 Ps 49:17 Lu 12:20 1Ti 6:7}
\crossref{Eccl}{5}{16}{5:13; 2:22,\allowbreak23}
\crossref{Eccl}{5}{17}{Ge 3:17 1Ki 17:12 Job 21:25 Ps 78:33; 102:9; 127:2 Eze 4:16,\allowbreak17}
\crossref{Eccl}{5}{18}{Ec 2:10; 3:22 Jer 52:34}
\crossref{Eccl}{5}{19}{Ec 2:24; 3:13; 6:2 De 8:18 1Ki 3:13}
\crossref{Eccl}{5}{20}{De 28:8-\allowbreak12,\allowbreak47 Ps 4:6,\allowbreak7 Isa 64:5; 65:13,\allowbreak14,\allowbreak21-\allowbreak24 Ro 5:1,\allowbreak5-\allowbreak11}
\crossref{Eccl}{6}{1}{Ec 5:13}
\crossref{Eccl}{6}{2}{Ec 5:19 1Ki 3:13 1Ch 29:25,\allowbreak28 2Ch 1:11 Da 5:18}
\crossref{Eccl}{6}{3}{Ge 33:5 1Sa 2:20,\allowbreak21 2Ki 10:1 1Ch 28:5 2Ch 11:21 Es 5:11}
\crossref{Eccl}{6}{4}{Ps 109:13}
\crossref{Eccl}{6}{5}{Job 3:10-\allowbreak13; 14:1 Ps 58:8; 90:7-\allowbreak9}
\crossref{Eccl}{6}{6}{Ge 5:5,\allowbreak23,\allowbreak24 Isa 65:22}
\crossref{Eccl}{6}{7}{Ge 3:17-\allowbreak19 Pr 16:26 Mt 6:25 Joh 6:27 1Ti 6:6-\allowbreak8}
\crossref{Eccl}{6}{8}{Ec 2:14-\allowbreak16; 5:11}
\crossref{Eccl}{6}{9}{Ec 2:24; 3:12,\allowbreak13; 5:18}
\crossref{Eccl}{6}{10}{Ec 1:9-\allowbreak11; 3:15}
\crossref{Eccl}{6}{11}{Ec 1:6-\allowbreak9,\allowbreak17,\allowbreak18; 2:3-\allowbreak11; 3:19; 4:1-\allowbreak4,\allowbreak8,\allowbreak16; 5:7 Ps 73:6 Ho 12:1}
\crossref{Eccl}{6}{12}{Ec 2:3; 12:13 Ps 4:6; 16:5; 17:15; 47:4 La 3:24-\allowbreak27 Mic 6:8}
\crossref{Eccl}{7}{1}{Pr 15:30; 22:1 Isa 56:5 Lu 10:20 Heb 11:2,\allowbreak39}
\crossref{Eccl}{7}{2}{Ge 48:1-\allowbreak22; 49:2-\allowbreak33; 50:15-\allowbreak17 Job 1:4,\allowbreak5 Isa 5:11,\allowbreak12; 22:12-\allowbreak14}
\crossref{Eccl}{7}{3}{Ps 119:67,\allowbreak71; 126:5,\allowbreak6 Jer 31:8,\allowbreak9,\allowbreak15-\allowbreak20; 50:4,\allowbreak5 Da 9:3-\allowbreak19}
\crossref{Eccl}{7}{4}{Ne 2:2-\allowbreak5 Isa 53:3,\allowbreak4 Mt 8:14-\allowbreak16 Mr 5:38-\allowbreak43 Lu 7:12,\allowbreak13}
\crossref{Eccl}{7}{5}{Ps 141:5 Pr 9:8; 13:13; 15:31,\allowbreak32; 17:10; 27:6 Re 3:19}
\crossref{Eccl}{7}{6}{Ec 2:2 Ps 58:9; 118:12 Isa 65:13-\allowbreak15 Am 8:10 Lu 6:25; 16:25}
\crossref{Eccl}{7}{7}{De 28:33,\allowbreak34,\allowbreak65}
\crossref{Eccl}{7}{8}{Ps 126:5,\allowbreak6 Isa 10:24,\allowbreak25,\allowbreak28-\allowbreak34 Lu 16:25 Jas 5:11 1Pe 1:13}
\crossref{Eccl}{7}{9}{1Sa 25:21,\allowbreak22 2Sa 19:43 Es 3:5,\allowbreak6 Pr 14:17; 16:32 Jon 4:9}
\crossref{Eccl}{7}{10}{Jud 6:13 Jer 44:17-\allowbreak19}
\crossref{Eccl}{7}{11}{Ec 11:7}
\crossref{Eccl}{7}{12}{Job 1:10; 22:21-\allowbreak25 Pr 2:7,\allowbreak11; 14:20; 18:10,\allowbreak11 Isa 33:6}
\crossref{Eccl}{7}{13}{Job 37:14 Ps 8:3; 107:43 Isa 5:12}
\crossref{Eccl}{7}{14}{Ec 3:4 De 28:47 Ps 30:11,\allowbreak12; 40:3 Mt 9:13 Joh 16:22,\allowbreak23 Jas 5:13}
\crossref{Eccl}{7}{15}{Ec 2:23; 5:16,\allowbreak17; 6:12 Ge 47:9 Ps 39:6}
\crossref{Eccl}{7}{16}{Ec 12:12 Ge 3:6 Job 11:12; 28:28 Pr 23:4 Ro 11:25; 12:3}
\crossref{Eccl}{7}{17}{Jer 2:33,\allowbreak34 Eze 8:17; 16:20 Jas 1:21}
\crossref{Eccl}{7}{18}{Ec 11:6 Pr 4:25-\allowbreak27; 8:20 Lu 11:42}
\crossref{Eccl}{7}{19}{Ec 9:15-\allowbreak18 2Sa 20:16-\allowbreak22 Pr 21:22; 24:5 Col 1:9-\allowbreak11}
\crossref{Eccl}{7}{20}{1Ki 8:46 2Ch 6:36 Job 15:14-\allowbreak16 Ps 130:3; 143:2 Pr 20:9}
\crossref{Eccl}{7}{21}{2Sa 19:19}
\crossref{Eccl}{7}{22}{1Ki 2:44 Mt 15:19; 18:32-\allowbreak35 Joh 8:7-\allowbreak9 Jas 3:9}
\crossref{Eccl}{7}{23}{Ge 3:5 1Ki 3:11,\allowbreak12; 11:1-\allowbreak8 Ro 1:22 1Co 1:20}
\crossref{Eccl}{7}{24}{De 30:11-\allowbreak14 Job 11:7,\allowbreak8; 28:12-\allowbreak23,\allowbreak28 Ps 36:6; 139:6 Isa 55:8,\allowbreak9}
\crossref{Eccl}{7}{25}{Ec 1:13-\allowbreak17; 2:1-\allowbreak3,\allowbreak12,\allowbreak20}
\crossref{Eccl}{7}{26}{Jud 16:18-\allowbreak21 Pr 2:18,\allowbreak19; 5:3-\allowbreak5; 7:21-\allowbreak27; 9:18; 22:14; 23:27,\allowbreak28}
\crossref{Eccl}{7}{27}{Ec 1:1,\allowbreak2; 12:8-\allowbreak10}
\crossref{Eccl}{7}{28}{7:23,\allowbreak24 Isa 26:9}
\crossref{Eccl}{7}{29}{Ge 1:26,\allowbreak27; 5:1}
\crossref{Eccl}{8}{1}{Ec 2:13,\allowbreak14 1Co 2:13-\allowbreak16}
\crossref{Eccl}{8}{2}{Pr 24:21 Ro 13:1-\allowbreak4 Tit 3:1 1Pe 2:13-\allowbreak17}
\crossref{Eccl}{8}{3}{Ec 10:4 Pr 14:29}
\crossref{Eccl}{8}{4}{1Ki 2:25,\allowbreak29-\allowbreak34,\allowbreak46 Pr 19:12; 20:2; 30:31 Da 3:15 Lu 12:4,\allowbreak5}
\crossref{Eccl}{8}{5}{8:2 Ex 1:17,\allowbreak20,\allowbreak21 Ps 119:6 Ho 5:11 Lu 20:25 Ac 4:19; 5:29}
\crossref{Eccl}{8}{6}{Ec 3:1,\allowbreak11,\allowbreak17; 7:13,\allowbreak14}
\crossref{Eccl}{8}{7}{Ec 6:12; 9:12; 10:14 Pr 24:22; 29:1 Mt 24:44,\allowbreak50; 25:6-\allowbreak13 1Th 5:1-\allowbreak3}
\crossref{Eccl}{8}{8}{Ec 3:21 2Sa 14:14 Job 14:5; 34:14 Ps 49:6-\allowbreak9; 89:48 Heb 9:27}
\crossref{Eccl}{8}{9}{Ec 1:14; 3:10; 4:7,\allowbreak8; 7:25}
\crossref{Eccl}{8}{10}{2Ki 9:34,\allowbreak35 Job 21:18,\allowbreak32,\allowbreak33 Lu 16:22}
\crossref{Eccl}{8}{11}{Ex 8:15,\allowbreak32 Job 21:11-\allowbreak15 Ps 10:6; 50:21,\allowbreak22 Isa 5:18,\allowbreak19; 26:10}
\crossref{Eccl}{8}{12}{Ec 5:16; 7:15 1Ki 2:5-\allowbreak9; 21:25; 22:34,\allowbreak35 Pr 13:21 Isa 65:20 Ro 2:5}
\crossref{Eccl}{8}{13}{Job 18:5; 20:5; 21:30 Ps 11:5 Isa 57:21 Mal 3:18 Mt 13:49,\allowbreak50}
\crossref{Eccl}{8}{14}{Ec 4:4,\allowbreak8; 9:3; 10:5}
\crossref{Eccl}{8}{15}{}
\crossref{Eccl}{8}{16}{Ec 2:23; 4:8; 5:12 Ge 31:40 Ps 127:2}
\crossref{Eccl}{8}{17}{Ec 3:11; 7:23,\allowbreak24; 11:5 Job 5:9; 11:7-\allowbreak9 Ps 40:5; 73:16; 104:24}
\crossref{Eccl}{9}{1}{Ec 1:17; 7:25; 8:16; 12:9,\allowbreak10}
\crossref{Eccl}{9}{2}{Ec 2:14-\allowbreak16 Job 21:7-\allowbreak34 Ps 73:3 Mal 3:15}
\crossref{Eccl}{9}{3}{Ec 8:11 Ge 6:5; 8:21 Job 15:16 Ps 51:5 Jer 17:9 Mt 15:19,\allowbreak20}
\crossref{Eccl}{9}{4}{Job 14:7-\allowbreak12; 27:8 Isa 38:18 La 3:21,\allowbreak22 Lu 16:26-\allowbreak29}
\crossref{Eccl}{9}{5}{Ec 7:2 Job 30:23 Heb 9:27}
\crossref{Eccl}{9}{6}{Ex 1:8 Job 3:17,\allowbreak18 Ps 146:3,\allowbreak4 Pr 10:28 Mt 2:20}
\crossref{Eccl}{9}{7}{Ge 12:19 Mr 7:29 Joh 4:50}
\crossref{Eccl}{9}{8}{2Sa 19:24 Es 8:15 Re 3:4,\allowbreak5; 7:9,\allowbreak13,\allowbreak14; 16:15; 19:8,\allowbreak14}
\crossref{Eccl}{9}{9}{Pr 5:18,\allowbreak19; 18:22; 19:14 Mal 2:15}
\crossref{Eccl}{9}{10}{Nu 13:30 1Ch 22:19; 28:20; 29:2,\allowbreak3 2Ch 31:20,\allowbreak21 Ezr 6:14,\allowbreak15}
\crossref{Eccl}{9}{11}{Ec 2:12; 4:1,\allowbreak4 Mal 3:18}
\crossref{Eccl}{9}{12}{Ec 8:5-\allowbreak7,\allowbreak11 Lu 19:42-\allowbreak44 2Co 6:2 1Pe 2:12}
\crossref{Eccl}{9}{13}{9:11; 6:1; 7:15; 8:16}
\crossref{Eccl}{9}{14}{2Sa 20:15-\allowbreak22 2Ki 6:24-\allowbreak33; 7:1-\allowbreak20}
\crossref{Eccl}{9}{15}{Ge 40:23 Es 6:2,\allowbreak3}
\crossref{Eccl}{9}{16}{9:18; 7:19 Pr 21:22; 24:5}
\crossref{Eccl}{9}{17}{Ge 41:33-\allowbreak40 1Sa 7:3-\allowbreak6 Pr 28:23 Isa 42:2,\allowbreak4 Jas 1:20; 3:17,\allowbreak18}
\crossref{Eccl}{9}{18}{9:16}
\crossref{Eccl}{10}{1}{Ex 30:34,\allowbreak35}
\crossref{Eccl}{10}{2}{Ec 9:10 Pr 14:8 Lu 14:28-\allowbreak32}
\crossref{Eccl}{10}{3}{Ec 5:3 Pr 13:16; 18:2,\allowbreak6 1Pe 4:4}
\crossref{Eccl}{10}{4}{Ec 8:3}
\crossref{Eccl}{10}{5}{Ec 4:7; 5:13; 6:1; 9:3}
\crossref{Eccl}{10}{6}{Jud 9:14-\allowbreak20 1Ki 12:13,\allowbreak14 Es 3:1 Ps 12:8 Pr 28:12,\allowbreak28}
\crossref{Eccl}{10}{7}{Pr 19:10; 30:22}
\crossref{Eccl}{10}{8}{Jud 9:5,\allowbreak53-\allowbreak57 2Sa 17:23; 18:15 Es 7:10 Ps 7:15,\allowbreak16; 9:15,\allowbreak16}
\crossref{Eccl}{10}{9}{Ec 3:5 Ge 2:12; 11:3}
\crossref{Eccl}{10}{10}{10:15; 9:15-\allowbreak17 Ge 41:33-\allowbreak39 Ex 18:19-\allowbreak23 1Ki 3:9 1Ch 12:32}
\crossref{Eccl}{10}{11}{Ps 58:4,\allowbreak5 Jer 8:17}
\crossref{Eccl}{10}{12}{Job 4:3,\allowbreak4; 16:5 Ps 37:30; 40:9,\allowbreak10; 71:15-\allowbreak18 Pr 10:13,\allowbreak20,\allowbreak21,\allowbreak31,\allowbreak32}
\crossref{Eccl}{10}{13}{Jud 14:15 1Sa 20:26-\allowbreak33; 22:7,\allowbreak8,\allowbreak16-\allowbreak18; 25:10,\allowbreak11 2Sa 19:41-\allowbreak43}
\crossref{Eccl}{10}{14}{Ec 5:3 Pr 10:19; 15:2}
\crossref{Eccl}{10}{15}{10:3,\allowbreak10 Isa 44:12-\allowbreak17; 47:12,\allowbreak13; 55:2; 57:1 Hab 2:6 Mt 11:28-\allowbreak30}
\crossref{Eccl}{10}{16}{2Ch 13:7; 33:1-\allowbreak20; 36:2,\allowbreak5,\allowbreak9,\allowbreak11 Isa 3:4,\allowbreak5,\allowbreak12}
\crossref{Eccl}{10}{17}{10:6,\allowbreak7 Pr 28:2,\allowbreak3 Jer 30:21}
\crossref{Eccl}{10}{18}{Pr 12:24; 14:1; 20:4; 21:25; 23:21; 24:30,\allowbreak31 Heb 6:11 2Pe 1:5-\allowbreak10}
\crossref{Eccl}{10}{19}{Ec 2:1,\allowbreak2; 7:2-\allowbreak6 Ge 43:34 Da 5:1-\allowbreak12 1Pe 4:3}
\crossref{Eccl}{10}{20}{Ex 22:28 Isa 8:21 Ac 23:5}
\crossref{Eccl}{11}{1}{}
\crossref{Eccl}{11}{2}{Ne 8:10 Es 9:19,\allowbreak22 Ps 112:9 Lu 6:30-\allowbreak35 1Ti 6:18,\allowbreak19}
\crossref{Eccl}{11}{3}{1Ki 18:45 Ps 65:9-\allowbreak13 Isa 55:10,\allowbreak11 1Jo 3:17}
\crossref{Eccl}{11}{4}{Pr 3:27; 20:4; 22:13}
\crossref{Eccl}{11}{5}{Joh 3:8}
\crossref{Eccl}{11}{6}{Ec 9:10 Isa 55:10 Ho 10:12 Mr 4:26-\allowbreak29 Joh 4:36-\allowbreak38 2Co 9:6}
\crossref{Eccl}{11}{7}{Job 33:28,\allowbreak30 Ps 56:13 Pr 15:30; 29:13}
\crossref{Eccl}{11}{8}{Ec 6:6; 8:12}
\crossref{Eccl}{11}{9}{1Ki 18:27; 22:15 Lu 15:12,\allowbreak13}
\crossref{Eccl}{11}{10}{Ec 12:1 Job 13:26 Ps 25:7 2Pe 3:11-\allowbreak14}
\crossref{Eccl}{12}{1}{Ec 11:10 Ge 39:2,\allowbreak8,\allowbreak9,\allowbreak23 1Sa 1:28; 2:18,\allowbreak26; 3:19-\allowbreak21; 16:7,\allowbreak12,\allowbreak13}
\crossref{Eccl}{12}{2}{Ec 11:7,\allowbreak8 Ge 27:1; 48:10 1Sa 3:2; 4:15,\allowbreak18}
\crossref{Eccl}{12}{3}{2Sa 21:15-\allowbreak17 Ps 90:9,\allowbreak10; 102:23 Zec 8:4}
\crossref{Eccl}{12}{4}{2Sa 19:35}
\crossref{Eccl}{12}{5}{Ge 42:38; 44:29,\allowbreak31 Le 19:32 Job 15:10 Ps 71:18 Pr 16:31; 20:29}
\crossref{Eccl}{12}{6}{12:1,\allowbreak2}
\crossref{Eccl}{12}{7}{Ec 3:20 Ge 3:19; 18:27 Job 4:19,\allowbreak20; 7:21; 20:11; 34:14,\allowbreak15 Ps 90:3}
\crossref{Eccl}{12}{8}{Ec 1:2,\allowbreak14; 2:17; 4:4; 6:12; 8:8 Ps 62:9}
\crossref{Eccl}{12}{9}{1Ki 4:32 Pr 1:1; 10:1; 25:1}
\crossref{Eccl}{12}{10}{Ec 1:1,\allowbreak12}
\crossref{Eccl}{12}{11}{Jer 23:29 Mt 3:7 Ac 2:37 2Co 10:4 Heb 4:12}
\crossref{Eccl}{12}{12}{Lu 16:29-\allowbreak31 Joh 5:39; 20:31; 21:25 2Pe 1:19-\allowbreak21}
\crossref{Eccl}{12}{13}{Ec 2:3; 6:12 Job 28:28 Ps 115:13-\allowbreak15 Pr 19:23 Lu 1:50}
\crossref{Eccl}{12}{14}{Ec 11:9 Ps 96:13 Mt 12:36; 25:31-\allowbreak46 Lu 12:1,\allowbreak2 Joh 5:29}

% Song
\crossref{Song}{1}{1}{Ps 14:1}
\crossref{Song}{1}{2}{So 5:16; 8:1 Ge 27:26,\allowbreak27; 29:11; 45:15 Ps 2:12 Lu 15:20 Ac 21:7}
\crossref{Song}{1}{3}{So 3:6; 4:10; 5:5,\allowbreak13 Ex 30:23-\allowbreak28 Ps 45:7,\allowbreak8; 133:2 Pr 27:9 Ec 7:1}
\crossref{Song}{1}{4}{Jer 31:3 Ho 11:4 Joh 6:44; 12:32 Php 2:12,\allowbreak13}
\crossref{Song}{1}{5}{Isa 53:2 Mt 10:25 1Co 4:10-\allowbreak13 1Jo 3:1}
\crossref{Song}{1}{6}{Ru 1:19-\allowbreak21}
\crossref{Song}{1}{7}{So 2:3; 3:1-\allowbreak4; 5:8,\allowbreak10,\allowbreak16 Ps 18:1; 116:1 Isa 5:1; 26:9 Mt 10:37}
\crossref{Song}{1}{8}{1:15; 2:10; 4:1,\allowbreak7,\allowbreak10; 5:9; 6:1,\allowbreak4-\allowbreak10; 7:1-\allowbreak13 Ps 16:3; 45:11,\allowbreak13}
\crossref{Song}{1}{9}{So 2:2,\allowbreak10,\allowbreak13; 4:1,\allowbreak7; 5:2; 6:4 Joh 15:14,\allowbreak15}
\crossref{Song}{1}{10}{Ge 24:22,\allowbreak47 Isa 3:18-\allowbreak21 Eze 16:11-\allowbreak13 2Pe 1:3,\allowbreak4}
\crossref{Song}{1}{11}{So 8:9 Ge 1:26 Ps 149:4 Eph 5:25-\allowbreak27 Php 3:21}
\crossref{Song}{1}{12}{So 7:5 Ps 45:1 Mt 22:11; 25:34}
\crossref{Song}{1}{13}{So 4:6,\allowbreak14; 5:1,\allowbreak5,\allowbreak13 Ge 43:11 Ps 45:8 Joh 19:39}
\crossref{Song}{1}{14}{1:13; 2:3}
\crossref{Song}{1}{15}{1:8; 4:1,\allowbreak7,\allowbreak10; 5:12; 7:6}
\crossref{Song}{1}{16}{So 2:3; 5:10-\allowbreak16 Ps 45:2 Zec 9:17 Php 3:8,\allowbreak9 Re 5:11-\allowbreak13}
\crossref{Song}{1}{17}{So 8:9 2Ch 2:8,\allowbreak9 Ps 92:12 1Ti 3:15,\allowbreak16 Heb 11:10 1Pe 2:4,\allowbreak5}
\crossref{Song}{2}{1}{Ps 85:11 Isa 35:1,\allowbreak2}
\crossref{Song}{2}{2}{Isa 55:13 Mt 6:28,\allowbreak29; 10:16 Php 2:15,\allowbreak16 1Pe 2:12}
\crossref{Song}{2}{3}{So 8:5 Isa 4:2 Eze 17:23,\allowbreak24 Joh 15:1-\allowbreak8}
\crossref{Song}{2}{4}{So 1:4; 5:1 Ps 63:2-\allowbreak5; 84:10 Joh 14:21-\allowbreak23 Re 3:20}
\crossref{Song}{2}{5}{Ps 4:6,\allowbreak7; 42:1,\allowbreak2; 63:1-\allowbreak3,\allowbreak8 Isa 26:8,\allowbreak9 Lu 24:32 Php 1:23}
\crossref{Song}{2}{6}{So 8:3-\allowbreak5 Isa 54:5-\allowbreak10; 62:4,\allowbreak5 Jer 32:41 Zep 3:17 Joh 3:29}
\crossref{Song}{2}{7}{Mt 26:63}
\crossref{Song}{2}{8}{So 5:2 Joh 3:29; 10:4,\allowbreak5,\allowbreak27 Re 3:20}
\crossref{Song}{2}{9}{2:17; 8:14}
\crossref{Song}{2}{10}{2:8 2Sa 23:3 Ps 85:8 Jer 31:3}
\crossref{Song}{2}{11}{Ec 3:4,\allowbreak11 Isa 12:1,\allowbreak2; 40:2; 54:6-\allowbreak8; 60:1,\allowbreak2 Mt 5:4 Eph 5:8}
\crossref{Song}{2}{12}{So 6:2,\allowbreak11 Isa 35:1,\allowbreak2 Ho 14:5-\allowbreak7}
\crossref{Song}{2}{13}{So 6:11; 7:8,\allowbreak11-\allowbreak13 Isa 18:5; 55:10,\allowbreak11; 61:11 Ho 14:6 Hag 2:19}
\crossref{Song}{2}{14}{So 5:2; 6:9 Ps 68:13; 74:19 Isa 60:8 Eze 7:16 Mt 3:16; 10:16}
\crossref{Song}{2}{15}{Ps 80:13 Eze 13:4-\allowbreak16 Lu 13:32 2Pe 2:1-\allowbreak3 Re 2:2}
\crossref{Song}{2}{16}{So 6:3; 7:10,\allowbreak13 Ps 48:14; 63:1 Jer 31:33 1Co 3:21-\allowbreak23 Ga 2:20}
\crossref{Song}{2}{17}{So 4:6 Lu 1:78 Ro 13:12 2Pe 1:19}
\crossref{Song}{3}{1}{Ps 4:4; 6:6; 22:2; 63:6-\allowbreak8; 77:2-\allowbreak4 Isa 26:9}
\crossref{Song}{3}{2}{So 5:5 Isa 64:7 Joh 1:6 Mt 26:40,\allowbreak41 Ro 13:11 1Co 15:34 Eph 5:14}
\crossref{Song}{3}{3}{So 5:7 Isa 21:6-\allowbreak8,\allowbreak11,\allowbreak12; 56:10; 62:6 Eze 3:17; 33:2-\allowbreak9 Heb 13:17}
\crossref{Song}{3}{4}{So 6:12 Pr 8:17 Isa 45:19; 55:6,\allowbreak7 Jer 29:13 La 3:25 Mt 7:7}
\crossref{Song}{3}{5}{So 2:7; 8:4 Mic 4:8}
\crossref{Song}{3}{6}{So 8:5 De 8:2 Isa 43:19 Jer 2:2; 31:2 Re 12:6,\allowbreak14}
\crossref{Song}{3}{7}{3:9}
\crossref{Song}{3}{8}{Ps 45:3; 149:5-\allowbreak9 Isa 27:3 Eph 6:16-\allowbreak18}
\crossref{Song}{3}{9}{}
\crossref{Song}{3}{10}{Ps 87:3 1Ti 3:15,\allowbreak16 Re 3:12}
\crossref{Song}{3}{11}{So 7:11 Heb 13:13}
\crossref{Song}{4}{1}{4:9,\allowbreak10; 1:15; 2:10,\allowbreak14 Ps 45:11 Eze 16:14 2Co 3:18}
\crossref{Song}{4}{2}{So 6:6 Jer 15:16 Joh 15:7 Col 1:4-\allowbreak6 1Th 2:13 2Pe 1:5-\allowbreak8}
\crossref{Song}{4}{3}{4:11; 5:13,\allowbreak16; 7:9 Ps 37:30; 45:2; 119:13 Pr 10:13,\allowbreak20,\allowbreak21; 16:21-\allowbreak24}
\crossref{Song}{4}{4}{So 1:10; 7:4 2Sa 22:51 Eph 4:15,\allowbreak16 Col 2:19 1Pe 1:5}
\crossref{Song}{4}{5}{So 1:13; 7:3,\allowbreak7; 8:1,\allowbreak10 Pr 5:19 Isa 66:10-\allowbreak12 1Pe 2:2}
\crossref{Song}{4}{6}{So 2:17 Mal 4:2 Lu 1:78 2Pe 1:19 1Jo 2:8 Re 22:16}
\crossref{Song}{4}{7}{4:1; 5:16 Nu 24:5 Ps 45:11,\allowbreak13 Eph 5:25-\allowbreak27 Col 1:22 2Pe 3:14}
\crossref{Song}{4}{8}{So 2:13; 7:11 Ps 45:10 Pr 9:6 Joh 12:26 Col 3:1,\allowbreak2}
\crossref{Song}{4}{9}{4:10,\allowbreak12; 5:1,\allowbreak2 Ge 20:12 Mt 12:50 1Co 9:5 Heb 2:11-\allowbreak14}
\crossref{Song}{4}{10}{So 1:2}
\crossref{Song}{4}{11}{4:3; 5:13; 7:9 Ps 71:14,\allowbreak15,\allowbreak23,\allowbreak24 Pr 16:24 Ho 14:2 Heb 13:15}
\crossref{Song}{4}{12}{So 6:2,\allowbreak11 Pr 5:15-\allowbreak18 Isa 58:11; 61:10,\allowbreak11 Jer 31:12 Ho 6:3}
\crossref{Song}{4}{13}{So 6:11; 7:12; 8:2 Ps 92:14 Ec 2:5 Isa 60:21; 61:11 Joh 15:1-\allowbreak3}
\crossref{Song}{4}{14}{Ex 30:23 Eze 27:19}
\crossref{Song}{4}{15}{4:12 Ec 2:6}
\crossref{Song}{4}{16}{So 1:4 Ec 1:6 Isa 51:9-\allowbreak11; 64:1 Eze 37:9 Joh 3:8 Ac 2:1,\allowbreak2; 4:31}
\crossref{Song}{5}{1}{So 4:16; 6:2,\allowbreak11; 8:13 Isa 5:1; 51:3; 58:11; 61:11 Joh 14:21-\allowbreak23}
\crossref{Song}{5}{2}{So 3:1; 7:9 Da 8:18 Zec 4:1 Mt 25:4,\allowbreak5; 26:40,\allowbreak41 Lu 9:32 Eph 5:14}
\crossref{Song}{5}{3}{Pr 3:28; 13:4; 22:13 Mt 25:5; 26:38-\allowbreak43 Lu 11:7 Ro 7:22,\allowbreak23}
\crossref{Song}{5}{4}{So 1:4 Ps 110:3 Ac 16:14 2Co 8:1,\allowbreak2,\allowbreak16 Php 2:13}
\crossref{Song}{5}{5}{5:2 Lu 12:36 Eph 3:17 Re 3:20}
\crossref{Song}{5}{6}{Ps 30:7 Isa 8:17; 12:1; 50:2; 54:6-\allowbreak8 Ho 5:6,\allowbreak15 Mt 15:22-\allowbreak28}
\crossref{Song}{5}{7}{So 3:3 Isa 6:10,\allowbreak11 Ho 9:7,\allowbreak8 Ac 20:29,\allowbreak30 2Co 11:13}
\crossref{Song}{5}{8}{So 2:7; 8:4}
\crossref{Song}{5}{9}{Isa 53:2 Mt 16:13-\allowbreak17; 21:10 Joh 1:14 2Co 4:3-\allowbreak6}
\crossref{Song}{5}{10}{So 2:1 De 32:31 Ps 45:17 Isa 66:19 Heb 7:26}
\crossref{Song}{5}{11}{Da 2:37,\allowbreak38 Eph 1:21,\allowbreak22}
\crossref{Song}{5}{12}{}
\crossref{Song}{5}{13}{So 1:10 Isa 50:6}
\crossref{Song}{5}{14}{Ex 15:6 Ps 44:4-\allowbreak7; 99:4 Isa 9:7; 52:13}
\crossref{Song}{5}{15}{Re 1:15}
\crossref{Song}{5}{16}{So 1:2 Ps 19:10; 119:103 Jer 15:16}
\crossref{Song}{6}{1}{6:4,\allowbreak9,\allowbreak10; 1:8; 2:2; 5:9}
\crossref{Song}{6}{2}{6:11; 4:12-\allowbreak16; 5:1 Isa 58:11; 61:11 Mt 18:20; 28:20}
\crossref{Song}{6}{3}{So 2:16; 7:10 Heb 8:10 Re 21:2-\allowbreak4}
\crossref{Song}{6}{4}{6:10; 2:14; 4:7; 5:2 Eze 16:13,\allowbreak14 Eph 5:27}
\crossref{Song}{6}{5}{Ge 32:26-\allowbreak28 Ex 32:10 Jer 15:1 Mt 15:27,\allowbreak28}
\crossref{Song}{6}{6}{So 4:2 Mt 21:19; 25:30}
\crossref{Song}{6}{7}{So 4:3}
\crossref{Song}{6}{8}{1Ki 11:1 2Ch 11:21 Ps 45:14 Re 7:9}
\crossref{Song}{6}{9}{So 2:14; 5:2}
\crossref{Song}{6}{10}{So 3:6; 8:5 Isa 63:1 Re 21:10,\allowbreak11}
\crossref{Song}{6}{11}{6:2; 4:12-\allowbreak15; 5:1 Ge 2:9 Ps 92:12-\allowbreak15 Joh 15:16}
\crossref{Song}{6}{12}{Jer 31:18-\allowbreak20 Ho 11:8,\allowbreak9 Lu 15:20}
\crossref{Song}{6}{13}{So 2:14 Jer 3:12-\allowbreak14,\allowbreak22 Ho 14:1-\allowbreak4}
\crossref{Song}{7}{1}{Lu 15:22 Eph 6:15 Php 1:27}
\crossref{Song}{7}{2}{Lu 15:22 Eph 6:15 Php 1:27}
\crossref{Song}{7}{3}{Pr 3:8}
\crossref{Song}{7}{4}{So 4:5; 6:6}
\crossref{Song}{7}{5}{So 1:10; 4:4}
\crossref{Song}{7}{6}{Isa 35:2 Eph 1:22; 4:15,\allowbreak16 Col 1:18; 2:19}
\crossref{Song}{7}{7}{7:10; 1:15,\allowbreak16; 2:14; 4:7,\allowbreak10 Ps 45:11 Isa 62:4,\allowbreak5 Zep 3:17}
\crossref{Song}{7}{8}{Ps 92:12 Jer 10:5 Eph 4:13}
\crossref{Song}{7}{9}{So 4:16; 5:1 Jer 32:41 Joh 14:21-\allowbreak23}
\crossref{Song}{7}{10}{So 2:14; 5:16 Pr 16:24 Eph 4:29 Col 3:16,\allowbreak17; 4:6 Heb 13:15}
\crossref{Song}{7}{11}{So 2:16; 6:3 Ac 27:23 1Co 6:19,\allowbreak20 Ga 2:20}
\crossref{Song}{7}{12}{So 1:4; 2:10-\allowbreak13; 4:8}
\crossref{Song}{7}{13}{Pr 8:17 Ec 9:10}
\crossref{Song}{8}{1}{Isa 7:14; 9:6 Hag 2:7 Zec 9:9 Mal 3:1 Mt 13:16,\allowbreak17}
\crossref{Song}{8}{2}{So 3:4 Ga 4:26}
\crossref{Song}{8}{3}{So 2:6 De 33:27 Isa 62:4,\allowbreak5 2Co 12:9}
\crossref{Song}{8}{4}{So 2:7; 3:5}
\crossref{Song}{8}{5}{So 3:6; 6:10}
\crossref{Song}{8}{6}{Ex 28:9-\allowbreak12,\allowbreak21,\allowbreak29,\allowbreak30 Isa 49:16 Jer 22:24 Hag 2:23 Zec 3:9}
\crossref{Song}{8}{7}{Isa 43:2 Mt 7:24,\allowbreak25 Ro 8:28-\allowbreak39}
\crossref{Song}{8}{8}{Eze 16:46,\allowbreak55,\allowbreak56,\allowbreak61; 23:33 Joh 10:16 Ac 15:14-\allowbreak17 Ro 15:9-\allowbreak12}
\crossref{Song}{8}{9}{So 2:9 Re 21:12-\allowbreak19}
\crossref{Song}{8}{10}{8:9}
\crossref{Song}{8}{11}{So 7:12 Ec 2:4 Isa 5:1-\allowbreak7 Mt 21:33-\allowbreak43 Mr 12:1}
\crossref{Song}{8}{12}{So 1:6 Pr 4:23 Ac 20:28 1Ti 4:15,\allowbreak16}
\crossref{Song}{8}{13}{So 2:13; 4:16; 6:2,\allowbreak11; 7:11,\allowbreak12 Mt 18:20; 28:20 Joh 14:21-\allowbreak23}
\crossref{Song}{8}{14}{So 2:17 Lu 19:12 Php 1:23 Re 22:17,\allowbreak20}

% Isa
\crossref{Isa}{1}{1}{Isa 21:2 Nu 12:6; 24:4,\allowbreak16 2Ch 32:32 Ps 89:19 Jer 23:16 Na 1:1}
\crossref{Isa}{1}{2}{De 4:26; 30:19; 32:1 Ps 50:4 Jer 2:12; 6:19; 22:29 Eze 36:4}
\crossref{Isa}{1}{3}{Pr 6:6 Jer 8:7}
\crossref{Isa}{1}{4}{1:23; 10:6; 30:9 Ge 13:13 Mt 11:28 Ac 7:51,\allowbreak52 Re 18:5}
\crossref{Isa}{1}{5}{Isa 9:13,\allowbreak21 Jer 2:30; 5:3; 6:28-\allowbreak30 Eze 24:13 Heb 12:5-\allowbreak8}
\crossref{Isa}{1}{6}{Job 2:7,\allowbreak8 Lu 16:20,\allowbreak21}
\crossref{Isa}{1}{7}{Isa 5:5,\allowbreak6,\allowbreak9; 6:11; 24:10-\allowbreak12 Le 26:34 De 28:51 2Ch 28:5,\allowbreak16-\allowbreak21}
\crossref{Isa}{1}{8}{Isa 4:4; 10:32; 37:22; 62:11 Ps 9:14 La 2:1 Zec 2:10; 9:9 Joh 12:15}
\crossref{Isa}{1}{9}{La 3:22 Hab 3:2 Ro 9:29}
\crossref{Isa}{1}{10}{1Ki 22:19-\allowbreak23 Am 3:1,\allowbreak8 Mic 3:8-\allowbreak12}
\crossref{Isa}{1}{11}{Isa 66:3 1Sa 15:22 Ps 50:8; 51:16 Pr 15:8; 21:27 Jer 6:20; 7:21}
\crossref{Isa}{1}{12}{Isa 58:1,\allowbreak2 Ex 23:17; 34:23 De 16:16 Ec 5:1 Mt 23:5}
\crossref{Isa}{1}{13}{Eze 20:39 Mal 1:10 Mt 15:9 Lu 11:42}
\crossref{Isa}{1}{14}{Isa 61:8 Am 5:21}
\crossref{Isa}{1}{15}{Isa 59:2 1Ki 8:22,\allowbreak54 Ezr 9:5 Job 27:8,\allowbreak9,\allowbreak20 Ps 66:18; 134:2}
\crossref{Isa}{1}{16}{Job 11:13,\allowbreak14 Ps 26:6 Jer 4:14 Ac 22:16 2Co 7:1 Jas 4:8}
\crossref{Isa}{1}{17}{1:23 Ps 82:3,\allowbreak4 Pr 31:9 Jer 22:3,\allowbreak15,\allowbreak16 Da 4:27 Mic 6:8 Zep 2:3}
\crossref{Isa}{1}{18}{Isa 41:21; 43:24-\allowbreak26 1Sa 12:7 Jer 2:5 Mic 6:2 Ac 17:2; 18:4; 24:25}
\crossref{Isa}{1}{19}{Isa 3:10; 55:1-\allowbreak3,\allowbreak6,\allowbreak7 Jer 3:12-\allowbreak14; 31:18-\allowbreak20 Ho 14:1-\allowbreak4 Joe 2:26}
\crossref{Isa}{1}{20}{Isa 3:11 1Sa 12:25 2Ch 36:14-\allowbreak16 Heb 2:1-\allowbreak3}
\crossref{Isa}{1}{21}{Isa 48:2 Ne 11:1 Ps 46:4; 48:1,\allowbreak8 Ho 11:12 Zec 8:3 Heb 12:22}
\crossref{Isa}{1}{22}{Jer 6:28-\allowbreak30 La 4:1,\allowbreak2 Eze 22:18-\allowbreak22 Ho 6:4}
\crossref{Isa}{1}{23}{Isa 3:14 2Ch 24:17-\allowbreak21; 36:14 Jer 5:5 Eze 22:6-\allowbreak12 Da 9:5,\allowbreak6}
\crossref{Isa}{1}{24}{Isa 30:29; 49:26; 60:16 Jer 50:34 Re 18:8}
\crossref{Isa}{1}{25}{Zec 13:7-\allowbreak9 Re 3:19}
\crossref{Isa}{1}{26}{Isa 32:1,\allowbreak2; 60:17,\allowbreak18 Nu 12:3; 16:15 1Sa 12:2-\allowbreak5 Jer 33:7,\allowbreak15-\allowbreak17}
\crossref{Isa}{1}{27}{Isa 5:16; 45:21-\allowbreak25 Ro 3:24-\allowbreak26; 11:26,\allowbreak27 2Co 5:21 Eph 1:7,\allowbreak8}
\crossref{Isa}{1}{28}{Job 31:3 Ps 1:6; 5:6; 37:38; 73:27; 92:9; 104:35; 125:5 Pr 29:1}
\crossref{Isa}{1}{29}{Isa 30:22; 31:7; 45:16 Eze 16:63; 36:31 Ho 14:3,\allowbreak8 Ro 6:21}
\crossref{Isa}{1}{30}{Isa 5:6 Jer 17:5,\allowbreak6 Eze 17:9,\allowbreak10,\allowbreak24 Mt 21:19}
\crossref{Isa}{1}{31}{Eze 32:21}
\crossref{Isa}{2}{1}{Isa 1:1; 13:1 Am 1:1 Mic 1:1; 6:9 Hab 1:1}
\crossref{Isa}{2}{2}{Mic 4:1-\allowbreak3}
\crossref{Isa}{2}{3}{Jer 31:6; 50:4,\allowbreak5 Zec 8:20-\allowbreak23}
\crossref{Isa}{2}{4}{Isa 11:3,\allowbreak4 1Sa 2:10 Ps 82:8; 96:13; 110:6 Joh 16:8-\allowbreak11 Ac 17:31}
\crossref{Isa}{2}{5}{2:3; 50:10,\allowbreak11; 60:1,\allowbreak19 Ps 89:15 Lu 1:79 Joh 12:35,\allowbreak36 Ro 13:12-\allowbreak14}
\crossref{Isa}{2}{6}{De 31:16,\allowbreak17 2Ch 15:2; 24:20 La 5:20 Ro 11:1,\allowbreak2,\allowbreak20}
\crossref{Isa}{2}{7}{De 17:16,\allowbreak17 1Ki 10:21-\allowbreak27 2Ch 9:20-\allowbreak25 Jer 5:27,\allowbreak28 Jas 5:1-\allowbreak3}
\crossref{Isa}{2}{8}{Isa 57:5 2Ch 27:2; 28:2-\allowbreak4,\allowbreak23-\allowbreak25; 33:3-\allowbreak7 Jer 2:28; 11:13 Eze 16:23-\allowbreak25}
\crossref{Isa}{2}{9}{Isa 5:15 Ps 49:2 Jer 5:4,\allowbreak5 Ro 3:23 Re 6:15-\allowbreak17}
\crossref{Isa}{2}{10}{2:19-\allowbreak21; 10:3; 42:22 Jud 6:1,\allowbreak2 Job 30:5,\allowbreak6 Ho 10:8 Lu 23:30}
\crossref{Isa}{2}{11}{2:17; 5:15,\allowbreak16; 13:11; 24:21 Job 40:10-\allowbreak12 Ps 18:27 Jer 50:31,\allowbreak32}
\crossref{Isa}{2}{12}{Isa 13:6,\allowbreak9 Jer 46:10 Eze 13:5 Am 5:18 Mal 4:5 1Co 5:5 1Th 5:2}
\crossref{Isa}{2}{13}{Isa 10:33,\allowbreak34; 14:8; 37:24 Eze 31:3-\allowbreak12 Am 2:5 Zec 11:1,\allowbreak2}
\crossref{Isa}{2}{14}{Isa 30:25; 40:4 Ps 68:16; 110:5,\allowbreak6 2Co 10:5}
\crossref{Isa}{2}{15}{Ge 8:22}
\crossref{Isa}{2}{16}{Isa 23:1 1Ki 10:22; 22:48,\allowbreak49 Ps 47:7 Re 18:17-\allowbreak19}
\crossref{Isa}{2}{17}{2:11; 13:11 Jer 48:29,\allowbreak30 Eze 28:2-\allowbreak7}
\crossref{Isa}{2}{18}{Isa 27:9 Eze 36:25; 37:23 Ho 14:8 Zep 1:3 Zec 13:2}
\crossref{Isa}{2}{19}{2:10,\allowbreak21 1Sa 13:6; 14:11 Jer 16:16 Ho 10:8 Mic 7:17 Lu 23:30}
\crossref{Isa}{2}{20}{Isa 30:22; 31:7; 46:1 Ho 14:8 Php 3:7,\allowbreak8}
\crossref{Isa}{2}{21}{2:10,\allowbreak19 Ex 33:22 Job 30:6 So 2:14}
\crossref{Isa}{2}{22}{Ps 62:9; 146:3 Jer 17:5}
\crossref{Isa}{3}{1}{Isa 2:22}
\crossref{Isa}{3}{2}{Isa 2:13-\allowbreak15 2Ki 24:14-\allowbreak16 Ps 74:9 La 5:12-\allowbreak14 Am 2:3}
\crossref{Isa}{3}{3}{Ex 18:21 De 1:15 1Sa 8:12}
\crossref{Isa}{3}{4}{1Ki 3:7-\allowbreak9 2Ch 33:1; 34:1; 36:2,\allowbreak5,\allowbreak9,\allowbreak11 Ec 10:16}
\crossref{Isa}{3}{5}{Isa 9:19-\allowbreak21; 11:13 Jer 9:3-\allowbreak8; 22:17 Eze 22:6,\allowbreak7,\allowbreak12 Am 4:1}
\crossref{Isa}{3}{6}{Isa 4:1 Jud 11:6-\allowbreak8 Joh 6:15}
\crossref{Isa}{3}{7}{Ge 14:22 De 32:40 Re 10:5,\allowbreak6}
\crossref{Isa}{3}{8}{2Ch 28:5-\allowbreak7,\allowbreak18; 33:11; 36:17-\allowbreak19 Jer 26:6,\allowbreak18 La 5:16,\allowbreak17 Mic 3:12}
\crossref{Isa}{3}{9}{3:16 1Sa 15:32 2Ki 9:30 Ps 10:4; 73:6,\allowbreak7 Pr 30:13 Jer 3:3; 6:15}
\crossref{Isa}{3}{10}{Isa 26:20,\allowbreak21 Ec 8:12 Jer 15:11 Eze 9:4; 18:9-\allowbreak19 Zep 2:3 Mal 3:18}
\crossref{Isa}{3}{11}{Isa 48:22; 57:20,\allowbreak21; 65:13-\allowbreak15,\allowbreak20 Ps 1:3-\allowbreak5; 11:5,\allowbreak6 Ec 8:13}
\crossref{Isa}{3}{12}{3:4 2Ki 11:1 Na 3:13}
\crossref{Isa}{3}{13}{Ps 12:5 Pr 22:22,\allowbreak23; 23:10,\allowbreak11 Ho 4:1,\allowbreak2 Mic 6:2}
\crossref{Isa}{3}{14}{Job 22:4; 34:23 Ps 143:2}
\crossref{Isa}{3}{15}{Eze 18:2 Jon 1:6}
\crossref{Isa}{3}{16}{Isa 1:8; 4:4 Mt 21:5 Lu 23:28}
\crossref{Isa}{3}{17}{Le 13:29,\allowbreak30,\allowbreak43,\allowbreak44 De 28:27 Re 16:2}
\crossref{Isa}{3}{18}{3:16}
\crossref{Isa}{3}{19}{Ge 24:22,\allowbreak30,\allowbreak53; 38:18,\allowbreak25 Ex 35:22 Nu 31:50 Eze 16:11}
\crossref{Isa}{3}{20}{}
\crossref{Isa}{3}{21}{Ge 41:42 Es 8:12 So 5:14 Lu 15:22 Jas 2:2}
\crossref{Isa}{3}{22}{Ge 8:22}
\crossref{Isa}{3}{23}{Ex 38:8}
\crossref{Isa}{3}{24}{Isa 57:9 Pr 7:17}
\crossref{Isa}{3}{25}{2Ch 29:9 Jer 11:22; 14:18; 18:21; 19:7; 21:9 La 2:21 Am 9:10}
\crossref{Isa}{3}{26}{Jer 14:2 La 1:4}
\crossref{Isa}{4}{1}{Isa 2:11,\allowbreak17; 10:20; 17:7 Lu 21:22}
\crossref{Isa}{4}{2}{Isa 11:1; 60:21 Jer 23:5; 33:15 Eze 17:22,\allowbreak23 Zec 3:8; 6:12}
\crossref{Isa}{4}{3}{Isa 1:27; 52:1; 60:21 Eze 36:24-\allowbreak28; 43:12 Zec 14:20,\allowbreak21 Eph 1:4}
\crossref{Isa}{4}{4}{Isa 3:16-\allowbreak26 La 1:9 Eze 16:6-\allowbreak9; 22:15; 36:25,\allowbreak29 Joe 3:21 Zep 3:1}
\crossref{Isa}{4}{5}{Isa 32:18; 33:20 Ps 87:2,\allowbreak3; 89:7; 111:1 Mt 18:20; 28:20}
\crossref{Isa}{4}{6}{Isa 8:14; 25:4 Ps 27:5; 91:1; 121:5,\allowbreak6 Pr 18:10 Eze 11:16 Heb 6:18}
\crossref{Isa}{5}{1}{De 31:19-\allowbreak22 Jud 5:1-\allowbreak31 Ps 45:1; 101:1}
\crossref{Isa}{5}{2}{Ex 33:16 Nu 23:9 De 32:8,\allowbreak9 Ps 44:1-\allowbreak3 Ro 9:4}
\crossref{Isa}{5}{3}{Ps 50:4-\allowbreak6; 51:4 Jer 2:4,\allowbreak5 Mic 6:2,\allowbreak3 Mt 21:40,\allowbreak41 Mr 12:9-\allowbreak12}
\crossref{Isa}{5}{4}{Isa 1:5 2Ch 36:14-\allowbreak16 Jer 2:30,\allowbreak31; 6:29,\allowbreak30 Eze 24:13 Mt 23:37}
\crossref{Isa}{5}{5}{Ge 11:4,\allowbreak7}
\crossref{Isa}{5}{6}{5:9,\allowbreak10; 6:11,\allowbreak12; 24:1-\allowbreak3,\allowbreak12; 32:13,\allowbreak14 Le 26:33-\allowbreak35 De 29:23}
\crossref{Isa}{5}{7}{Ps 80:8-\allowbreak11,\allowbreak15 Jer 12:10}
\crossref{Isa}{5}{8}{Jer 22:13-\allowbreak17 Mic 2:2 Hab 2:9-\allowbreak12 Mt 23:14 Lu 12:16-\allowbreak24}
\crossref{Isa}{5}{9}{Isa 22:14 Am 3:7}
\crossref{Isa}{5}{10}{Le 27:16 Eze 45:10,\allowbreak11 Joe 1:17 Hag 1:9-\allowbreak11}
\crossref{Isa}{5}{11}{5:22; 28:1 Pr 23:29,\allowbreak30 Ec 10:16,\allowbreak17 Ho 7:5,\allowbreak6 Hab 2:15 Lu 21:34}
\crossref{Isa}{5}{12}{Isa 22:13 Ge 31:27 Job 21:11-\allowbreak14 Da 5:1-\allowbreak4,\allowbreak23 Am 6:4-\allowbreak6 Lu 16:19}
\crossref{Isa}{5}{13}{Isa 1:7; 42:22-\allowbreak25 2Ki 17:6 2Ch 28:5-\allowbreak8}
\crossref{Isa}{5}{14}{Isa 14:9; 30:33 Ps 49:14 Pr 27:20 Eze 32:18-\allowbreak30 Hab 2:5 Mt 7:13}
\crossref{Isa}{5}{15}{Isa 2:9,\allowbreak11,\allowbreak17; 9:14-\allowbreak17; 24:2-\allowbreak4 Ps 62:9 Jer 5:4,\allowbreak5,\allowbreak9 Jas 1:9-\allowbreak11}
\crossref{Isa}{5}{16}{Isa 12:4 1Ch 29:11 Ps 9:16; 21:13; 46:10 Eze 28:22; 38:23 Ro 2:5}
\crossref{Isa}{5}{17}{Isa 7:21,\allowbreak22,\allowbreak25; 17:2; 32:14; 40:11; 65:10 Zep 2:6,\allowbreak14}
\crossref{Isa}{5}{18}{Isa 28:15 Jud 17:5,\allowbreak13 2Sa 16:20-\allowbreak23 Ps 10:11; 14:1; 36:2; 94:5-\allowbreak11}
\crossref{Isa}{5}{19}{Isa 66:5 Jer 5:12,\allowbreak13; 17:15 Eze 12:22,\allowbreak27 Am 5:18,\allowbreak19 2Pe 3:3,\allowbreak4}
\crossref{Isa}{5}{20}{Pr 17:15 Mal 2:17; 3:15 Mt 6:23; 15:3-\allowbreak6; 23:16-\allowbreak23 Lu 11:35; 16:15}
\crossref{Isa}{5}{21}{Job 11:12 Pr 3:7; 26:12,\allowbreak16 Joh 9:41 Ro 1:22; 11:25; 12:16}
\crossref{Isa}{5}{22}{5:11; 28:1-\allowbreak3,\allowbreak7 Pr 23:19,\allowbreak20 Hab 2:15}
\crossref{Isa}{5}{23}{Ex 23:6-\allowbreak9 Pr 17:15; 24:24; 31:4,\allowbreak5}
\crossref{Isa}{5}{24}{Isa 47:14 Ex 15:7 Joe 2:5 Na 1:10 Mal 4:1 1Co 3:12,\allowbreak13}
\crossref{Isa}{5}{25}{De 31:17; 32:19-\allowbreak22 2Ki 13:3; 22:13-\allowbreak17 2Ch 36:16 Ps 106:40}
\crossref{Isa}{5}{26}{Isa 11:12; 18:3 Jer 51:27}
\crossref{Isa}{5}{27}{Joe 2:7,\allowbreak8}
\crossref{Isa}{5}{28}{Ps 45:5; 120:4 Jer 5:16 Eze 21:9-\allowbreak11}
\crossref{Isa}{5}{29}{Isa 31:4 Ge 49:9 Nu 24:9 Jer 4:7; 49:19; 50:17 Ho 11:10 Am 3:8}
\crossref{Isa}{5}{30}{Ps 93:3,\allowbreak4 Jer 6:23; 50:42 Lu 21:25}
\crossref{Isa}{6}{1}{2Ki 15:7}
\crossref{Isa}{6}{2}{1Ki 22:19 Job 1:6 Da 7:10 Zec 3:4 Lu 1:10 Re 7:11}
\crossref{Isa}{6}{3}{Ex 15:20,\allowbreak21 Ezr 3:11 Ps 24:7-\allowbreak10}
\crossref{Isa}{6}{4}{Eze 1:24; 10:5 Am 9:1}
\crossref{Isa}{6}{5}{Ex 33:20 Jud 6:22; 13:22 Job 42:5,\allowbreak6 Da 10:6-\allowbreak8 Hab 3:16}
\crossref{Isa}{6}{6}{6:2 Da 9:21-\allowbreak23 Heb 1:7,\allowbreak14}
\crossref{Isa}{6}{7}{Jer 1:9 Da 10:16}
\crossref{Isa}{6}{8}{Ge 3:8-\allowbreak10 De 4:33-\allowbreak36 Eze 1:24; 10:5 Ac 28:25-\allowbreak28}
\crossref{Isa}{6}{9}{Isa 29:13; 30:8-\allowbreak11 Ex 32:7-\allowbreak10 Jer 15:1,\allowbreak2 Ho 1:9}
\crossref{Isa}{6}{10}{Isa 29:10; 63:17 Ex 7:3; 10:27; 11:10; 14:17 De 2:30 Eze 3:6-\allowbreak11}
\crossref{Isa}{6}{11}{Ps 74:10; 90:13; 94:3}
\crossref{Isa}{6}{12}{Isa 26:15 2Ki 25:11,\allowbreak21 Jer 15:4; 52:28-\allowbreak30}
\crossref{Isa}{6}{13}{Isa 1:9; 4:3; 10:20-\allowbreak22 Mt 24:22 Mr 13:20 Ro 11:5,\allowbreak6,\allowbreak16-\allowbreak29}
\crossref{Isa}{7}{1}{2Ki 16:1 2Ch 28:1-\allowbreak6}
\crossref{Isa}{7}{2}{7:13; 6:13; 37:35 2Sa 7:16 1Ki 11:32; 12:16; 13:2 Jer 21:12}
\crossref{Isa}{7}{3}{Ex 7:15 Jer 19:2,\allowbreak3; 22:1}
\crossref{Isa}{7}{4}{Isa 30:7,\allowbreak15 Ex 14:13,\allowbreak14 2Ch 20:17 La 3:26}
\crossref{Isa}{7}{5}{Ps 2:2; 83:3,\allowbreak4 Na 1:11 Zec 1:15}
\crossref{Isa}{7}{6}{Isa 2:3; 5:24; 14:8,\allowbreak13}
\crossref{Isa}{7}{7}{Isa 8:10; 10:6-\allowbreak12; 37:29; 46:10,\allowbreak11 Ps 2:4-\allowbreak6; 33:11; 76:10 Pr 21:30}
\crossref{Isa}{7}{8}{Isa 8:4; 17:1-\allowbreak3 2Ki 17:5-\allowbreak23 Ezr 4:2}
\crossref{Isa}{7}{9}{1Ki 16:24-\allowbreak29 2Ki 15:27}
\crossref{Isa}{7}{10}{Isa 1:5,\allowbreak13; 8:5; 10:20 Ho 13:2}
\crossref{Isa}{7}{11}{Isa 37:30; 38:7,\allowbreak8,\allowbreak22 Jud 6:36-\allowbreak40 2Ki 20:8-\allowbreak11 Jer 19:1,\allowbreak10; 51:63,\allowbreak64}
\crossref{Isa}{7}{12}{2Ki 16:15 2Ch 28:22}
\crossref{Isa}{7}{13}{7:2 2Ch 21:7 Jer 21:12 Lu 1:69}
\crossref{Isa}{7}{14}{Ge 3:15 Jer 31:22 Mt 1:23 Lu 1:35}
\crossref{Isa}{7}{15}{Ps 51:5 Am 5:15 Lu 1:35; 2:40,\allowbreak52 Ro 12:9 Php 1:9,\allowbreak10}
\crossref{Isa}{7}{16}{De 1:39 Jon 4:11}
\crossref{Isa}{7}{17}{Isa 8:7,\allowbreak8; 10:5,\allowbreak6; 36:1-\allowbreak37:38 2Ki 18:1-\allowbreak19:37 2Ch 28:19-\allowbreak21; 32:1-\allowbreak33}
\crossref{Isa}{7}{18}{Isa 5:26}
\crossref{Isa}{7}{19}{Isa 2:19,\allowbreak21 2Ch 33:11 Jer 16:16 Mic 7:17}
\crossref{Isa}{7}{20}{Isa 10:6 2Ki 16:7,\allowbreak8 2Ch 28:20,\allowbreak21 Jer 27:6,\allowbreak7 Eze 5:1-\allowbreak4; 29:18,\allowbreak20}
\crossref{Isa}{7}{21}{7:25; 5:17; 17:2; 37:30 Jer 39:10}
\crossref{Isa}{7}{22}{7:15 2Sa 17:29 Mt 3:4}
\crossref{Isa}{7}{23}{So 8:11,\allowbreak12 Mt 21:33}
\crossref{Isa}{7}{24}{Ge 27:3}
\crossref{Isa}{7}{25}{7:21,\allowbreak22; 13:20-\allowbreak22; 17:2 Zep 2:6}
\crossref{Isa}{8}{1}{Jer 36:2,\allowbreak28,\allowbreak32}
\crossref{Isa}{8}{2}{Ru 4:2,\allowbreak10,\allowbreak11 2Co 13:1}
\crossref{Isa}{8}{3}{Jud 4:4 2Ki 22:14}
\crossref{Isa}{8}{4}{Isa 7:15,\allowbreak16 De 1:39 Jon 4:11 Ro 9:11}
\crossref{Isa}{8}{5}{Isa 7:10}
\crossref{Isa}{8}{6}{1Ki 7:16 2Ch 13:8-\allowbreak18}
\crossref{Isa}{8}{7}{Isa 17:12,\allowbreak13; 28:17; 59:19 Ge 6:17 De 28:49-\allowbreak52 Jer 46:7,\allowbreak8 Da 9:26}
\crossref{Isa}{8}{8}{Isa 10:28-\allowbreak32; 22:1-\allowbreak7; 28:14-\allowbreak22; 29:1-\allowbreak9; 36:1-\allowbreak37:38}
\crossref{Isa}{8}{9}{Isa 7:1,\allowbreak2; 54:15 Jer 46:9-\allowbreak11 Eze 38:9-\allowbreak23 Joe 3:9-\allowbreak14 Mic 4:11-\allowbreak13}
\crossref{Isa}{8}{10}{Isa 7:5-\allowbreak7 2Sa 15:31; 17:4,\allowbreak23 Job 5:12 Ps 2:1,\allowbreak2; 33:10,\allowbreak11; 46:1,\allowbreak7}
\crossref{Isa}{8}{11}{Jer 20:7,\allowbreak9 Eze 3:14 Ac 4:20}
\crossref{Isa}{8}{12}{Isa 7:2-\allowbreak6; 51:12,\allowbreak13 2Ki 16:5-\allowbreak7}
\crossref{Isa}{8}{13}{Isa 26:3,\allowbreak4 Le 10:3 Nu 20:12,\allowbreak13; 27:14 Ro 4:20}
\crossref{Isa}{8}{14}{Isa 26:20 Ps 46:1,\allowbreak2 Pr 18:10 Eze 11:16}
\crossref{Isa}{8}{15}{Mt 11:6; 15:14; 21:44 Lu 20:17,\allowbreak18 Joh 6:66 1Co 1:23}
\crossref{Isa}{8}{16}{Isa 29:11 Da 12:4}
\crossref{Isa}{8}{17}{Isa 25:9; 26:8; 33:2; 64:4 Ge 49:18 Ps 27:14; 33:20; 37:34; 39:7; 40:1}
\crossref{Isa}{8}{18}{8:3; 7:3,\allowbreak16; 53:10 Ps 22:30 Heb 2:13,\allowbreak14}
\crossref{Isa}{8}{19}{Isa 19:3 Le 20:6 De 18:11 1Sa 28:8 1Ch 10:13 2Ch 33:6}
\crossref{Isa}{8}{20}{8:16 Lu 10:26; 16:29-\allowbreak31 Joh 5:39,\allowbreak46,\allowbreak47 Ac 17:11 Ga 3:8-\allowbreak29}
\crossref{Isa}{8}{21}{8:7,\allowbreak8}
\crossref{Isa}{8}{22}{Isa 5:30; 9:1 2Ch 15:5,\allowbreak6 Jer 13:16; 30:6,\allowbreak7 Am 5:18-\allowbreak20 Zep 1:14,\allowbreak15}
\crossref{Isa}{9}{1}{Isa 8:22}
\crossref{Isa}{9}{2}{Isa 50:10; 60:1-\allowbreak3,\allowbreak19 Mic 7:8,\allowbreak9 Mt 4:16 Lu 1:78,\allowbreak79; 2:32 Joh 8:12}
\crossref{Isa}{9}{3}{Isa 26:15; 49:20-\allowbreak22 Ne 9:23 Ps 107:38 Ho 4:7 Zec 2:11; 8:23; 10:8}
\crossref{Isa}{9}{4}{Isa 14:25; 47:6 Ge 27:40 Le 26:13 Jer 30:8 Na 1:13}
\crossref{Isa}{9}{5}{Isa 4:4; 10:16,\allowbreak17; 30:33; 37:36; 66:15,\allowbreak16 Ps 46:9 Eze 39:8-\allowbreak10}
\crossref{Isa}{9}{6}{Isa 7:14 Lu 1:35; 2:11}
\crossref{Isa}{9}{7}{2Sa 7:16 Ps 2:8; 72:8-\allowbreak11; 89:35-\allowbreak37 Jer 33:15-\allowbreak21 Da 2:35,\allowbreak44}
\crossref{Isa}{9}{8}{Isa 7:7,\allowbreak8; 8:4-\allowbreak8 Mic 1:1-\allowbreak9 Zec 1:6; 5:1-\allowbreak4 Mt 24:35}
\crossref{Isa}{9}{9}{Isa 26:11 1Ki 22:25 Job 21:19,\allowbreak20 Jer 32:24; 44:28,\allowbreak29 Eze 7:9,\allowbreak27}
\crossref{Isa}{9}{10}{1Ki 7:9-\allowbreak12; 10:27 Mal 1:4}
\crossref{Isa}{9}{11}{Isa 8:4-\allowbreak7; 10:9-\allowbreak11; 17:1-\allowbreak5 2Ki 15:29; 16:9}
\crossref{Isa}{9}{12}{2Ki 16:6 2Ch 28:18 Jer 35:11}
\crossref{Isa}{9}{13}{Isa 1:5; 26:11; 57:17 2Ch 28:22 Job 36:13 Jer 5:3; 31:18-\allowbreak20}
\crossref{Isa}{9}{14}{Isa 3:2,\allowbreak3; 19:15 2Ki 17:6-\allowbreak20 Ho 1:4,\allowbreak6,\allowbreak9; 4:5; 5:12-\allowbreak14; 8:8; 9:11-\allowbreak17}
\crossref{Isa}{9}{15}{Isa 3:5; 5:13 1Sa 9:6}
\crossref{Isa}{9}{16}{Isa 3:12 Mt 15:14; 23:16-\allowbreak36}
\crossref{Isa}{9}{17}{Isa 10:2; 13:18; 27:11; 62:5; 65:19 Ps 147:10 Jer 18:21 Zec 9:17}
\crossref{Isa}{9}{18}{Isa 1:31; 30:30,\allowbreak33; 33:12; 34:8-\allowbreak10; 66:16,\allowbreak17 Nu 11:1-\allowbreak3 De 32:22}
\crossref{Isa}{9}{19}{Isa 5:30; 8:22; 24:11,\allowbreak12; 60:2 Jer 13:16 Joe 2:2 Am 5:18 Mt 27:45}
\crossref{Isa}{9}{20}{Isa 49:26 Le 26:26-\allowbreak29 Jer 19:9 La 4:10}
\crossref{Isa}{9}{21}{Jud 7:2 1Sa 14:20 2Ki 15:30 2Ch 28:6-\allowbreak8 Mt 24:10 Ga 5:15}
\crossref{Isa}{10}{1}{Isa 3:11; 5:8,\allowbreak11,\allowbreak18,\allowbreak20-\allowbreak22 Jer 22:13 Hab 2:6,\allowbreak9,\allowbreak12,\allowbreak15,\allowbreak19 Mt 11:21}
\crossref{Isa}{10}{2}{Isa 29:21 La 3:35 Am 2:7; 5:11,\allowbreak12 Mal 3:5}
\crossref{Isa}{10}{3}{Isa 20:6; 33:14 Job 31:14 Jer 5:31 Eze 24:13,\allowbreak14 Re 6:15,\allowbreak16}
\crossref{Isa}{10}{4}{Le 26:17,\allowbreak36,\allowbreak37 De 31:15-\allowbreak18; 32:30 Jer 37:10 Ho 9:12}
\crossref{Isa}{10}{5}{Ge 10:11}
\crossref{Isa}{10}{6}{Isa 9:17; 19:17; 29:13; 30:9-\allowbreak11; 33:14 Jer 3:10; 4:14 Mt 15:7}
\crossref{Isa}{10}{7}{Ge 50:20 Mic 4:11,\allowbreak12 Ac 2:23; 13:27-\allowbreak30}
\crossref{Isa}{10}{8}{Isa 36:8 2Ki 18:24; 19:10 Eze 26:7 Da 2:37}
\crossref{Isa}{10}{9}{Am 6:1,\allowbreak2}
\crossref{Isa}{10}{10}{10:14 2Ki 18:33-\allowbreak35; 19:12,\allowbreak13,\allowbreak17-\allowbreak19 2Ch 32:12-\allowbreak16,\allowbreak19}
\crossref{Isa}{10}{11}{Isa 36:19,\allowbreak20; 37:10-\allowbreak13}
\crossref{Isa}{10}{12}{10:5,\allowbreak6; 14:24-\allowbreak27; 27:9; 46:10,\allowbreak11 Ps 76:10 1Pe 4:17}
\crossref{Isa}{10}{13}{10:8; 37:23,\allowbreak24 De 8:17 Eze 25:3; 26:2; 28:2-\allowbreak9; 29:3 Da 4:30 Am 6:13}
\crossref{Isa}{10}{14}{Isa 5:8 Job 31:25 Pr 18:12; 21:6,\allowbreak7 Ho 12:7,\allowbreak8 Na 2:9-\allowbreak13; 3:1}
\crossref{Isa}{10}{15}{10:5 Ps 17:13,\allowbreak14 Jer 51:20-\allowbreak23 Eze 28:9 Ro 9:20,\allowbreak21}
\crossref{Isa}{10}{16}{Isa 5:17; 14:24-\allowbreak27; 29:5-\allowbreak8; 37:6,\allowbreak7,\allowbreak29,\allowbreak36 2Ch 32:21 Ps 106:15}
\crossref{Isa}{10}{17}{Isa 60:19 Ps 27:1; 84:11 Re 21:23; 22:5}
\crossref{Isa}{10}{18}{10:33,\allowbreak34; 9:18 2Ki 19:23,\allowbreak28 Jer 21:14 Eze 20:47,\allowbreak48}
\crossref{Isa}{10}{19}{Isa 37:36}
\crossref{Isa}{10}{20}{Isa 1:9; 4:2,\allowbreak3; 6:13; 37:4,\allowbreak31,\allowbreak32 Ezr 9:14 Ro 9:27-\allowbreak29}
\crossref{Isa}{10}{21}{Isa 7:3; 9:13; 19:22; 55:7; 65:8,\allowbreak9 Ho 6:1; 7:10,\allowbreak16; 14:1 Ac 26:20}
\crossref{Isa}{10}{22}{1Ki 4:20 Ho 1:10 Ro 9:27; 11:5,\allowbreak6 Re 20:8}
\crossref{Isa}{10}{23}{Isa 14:26,\allowbreak27; 24:1-\allowbreak23 Da 4:35}
\crossref{Isa}{10}{24}{Isa 4:3; 12:6; 30:19; 46:13; 61:3 Heb 12:22-\allowbreak24}
\crossref{Isa}{10}{25}{10:33,\allowbreak34; 12:1,\allowbreak2; 14:24,\allowbreak25; 17:12-\allowbreak14; 30:30-\allowbreak33; 31:4-\allowbreak9; 37:36-\allowbreak38; 54:7}
\crossref{Isa}{10}{26}{10:16-\allowbreak19 2Ki 19:35 Ps 35:23}
\crossref{Isa}{10}{27}{Isa 9:4; 14:25 2Ki 18:13,\allowbreak14 Na 1:9-\allowbreak13}
\crossref{Isa}{10}{28}{Jos 7:2 Ne 11:31}
\crossref{Isa}{10}{29}{1Sa 13:23; 14:4}
\crossref{Isa}{10}{30}{1Sa 25:44}
\crossref{Isa}{10}{31}{Jos 15:31}
\crossref{Isa}{10}{32}{1Sa 21:1; 22:19 Ne 11:32}
\crossref{Isa}{10}{33}{10:16-\allowbreak19; 37:24-\allowbreak36,\allowbreak38 2Ki 19:21-\allowbreak37 2Ch 32:21}
\crossref{Isa}{10}{34}{10:18; 37:24 Jer 22:7; 46:22,\allowbreak23; 48:2 Na 1:12}
\crossref{Isa}{11}{1}{11:10 Ru 4:17 1Sa 17:58 Mt 1:6-\allowbreak16 Lu 2:23-\allowbreak32 Ac 13:22,\allowbreak23}
\crossref{Isa}{11}{2}{Isa 42:1; 59:21; 61:1 Nu 11:25,\allowbreak26 Mt 3:16 Joh 1:32,\allowbreak33; 3:34 Ac 10:38}
\crossref{Isa}{11}{3}{Isa 33:6 Pr 2:5,\allowbreak9 Lu 2:52}
\crossref{Isa}{11}{4}{Isa 32:1 2Sa 8:15; 23:2-\allowbreak4 1Ki 10:8,\allowbreak9 Ps 45:6,\allowbreak7; 72:1-\allowbreak4,\allowbreak12-\allowbreak14; 82:2-\allowbreak4}
\crossref{Isa}{11}{5}{Isa 59:17 Ps 93:1 2Co 6:7 Eph 6:14 1Pe 4:1 Re 1:13}
\crossref{Isa}{11}{6}{Isa 65:25 Eze 34:25 Ho 2:18 Ac 9:13-\allowbreak20 Ro 14:17 1Co 6:9-\allowbreak11}
\crossref{Isa}{11}{7}{Ge 32:15; 41:2-\allowbreak4,\allowbreak18-\allowbreak20}
\crossref{Isa}{11}{8}{Isa 59:5 Ps 140:3}
\crossref{Isa}{11}{9}{11:13; 2:4; 35:9; 60:18 Job 5:23 Mic 4:2-\allowbreak4 Mt 5:44,\allowbreak45 Ac 2:41-\allowbreak47}
\crossref{Isa}{11}{10}{11:1; 2:11 Ro 15:12 Re 22:16}
\crossref{Isa}{11}{11}{Isa 60:1-\allowbreak66:24 Le 26:40-\allowbreak42 De 4:27-\allowbreak31; 30:3-\allowbreak6 Ps 68:22 Jer 23:7,\allowbreak8}
\crossref{Isa}{11}{12}{11:10; 18:3; 59:19; 62:10 Re 5:9}
\crossref{Isa}{11}{13}{Isa 7:1-\allowbreak6 Jer 3:18 Eze 37:16-\allowbreak24 Ho 1:11}
\crossref{Isa}{11}{14}{Ob 1:19 Zep 2:5 Zec 9:5-\allowbreak7}
\crossref{Isa}{11}{15}{Isa 50:2; 51:9,\allowbreak10 Zec 10:11}
\crossref{Isa}{11}{16}{11:11; 19:23; 27:13; 35:8-\allowbreak10; 40:3,\allowbreak4; 49:12; 57:14}
\crossref{Isa}{12}{1}{Isa 2:11; 11:10,\allowbreak11,\allowbreak16; 14:3; 26:1; 27:1-\allowbreak3,\allowbreak12,\allowbreak13; 35:10 Zec 14:9,\allowbreak20}
\crossref{Isa}{12}{2}{Isa 7:14; 9:6,\allowbreak7; 45:17,\allowbreak22-\allowbreak25 Ps 27:1 Jer 3:23; 23:6 Jon 2:9}
\crossref{Isa}{12}{3}{Isa 49:10; 55:1-\allowbreak3 Ps 36:9 So 2:3 Jer 2:13 Joh 1:16; 4:10-\allowbreak14}
\crossref{Isa}{12}{4}{12:1 Ps 106:47,\allowbreak48; 113:1-\allowbreak3; 117:1,\allowbreak2}
\crossref{Isa}{12}{5}{Ex 15:1,\allowbreak21 Ps 68:32-\allowbreak35; 98:1; 105:2 Re 15:3; 19:1-\allowbreak3}
\crossref{Isa}{12}{6}{Isa 40:9; 52:7-\allowbreak10; 54:1 Zep 3:14 Lu 19:37-\allowbreak40}
\crossref{Isa}{13}{1}{Isa 14:28; 15:1; 17:1; 19:1; 21:1,\allowbreak11,\allowbreak13; 22:1,\allowbreak25; 23:1 Jer 23:33-\allowbreak38}
\crossref{Isa}{13}{2}{Isa 5:26; 11:12; 18:3 Jer 50:2; 51:27,\allowbreak28}
\crossref{Isa}{13}{3}{Isa 23:11; 44:27,\allowbreak28; 45:4,\allowbreak5 Jer 50:21-\allowbreak46}
\crossref{Isa}{13}{4}{Isa 22:1-\allowbreak9 Jer 50:2,\allowbreak3,\allowbreak21-\allowbreak46; 51:11,\allowbreak27,\allowbreak28 Eze 38:3-\allowbreak23 Joe 3:14}
\crossref{Isa}{13}{5}{13:17 Jer 50:3,\allowbreak9; 51:11,\allowbreak27,\allowbreak28 Mt 24:31}
\crossref{Isa}{13}{6}{Isa 14:31; 23:1; 52:5; 65:14 Jer 25:34; 49:3; 51:8 Eze 21:12; 30:2}
\crossref{Isa}{13}{7}{Isa 10:3,\allowbreak4; 37:27; 51:20 Jer 50:43 Eze 7:17; 21:7 Na 1:6}
\crossref{Isa}{13}{8}{Isa 21:3,\allowbreak4; 26:17 Ps 48:5,\allowbreak6 Jer 30:6; 50:43 Da 5:5,\allowbreak6 1Th 5:3}
\crossref{Isa}{13}{9}{13:15-\allowbreak18; 47:10-\allowbreak15 Jer 6:22,\allowbreak23; 50:40-\allowbreak42; 51:35-\allowbreak58 Na 1:2,\allowbreak6 Mal 4:1}
\crossref{Isa}{13}{10}{Isa 5:30; 24:21,\allowbreak23 Eze 32:7,\allowbreak8 Joe 2:10,\allowbreak31; 3:15 Am 8:9,\allowbreak10}
\crossref{Isa}{13}{11}{Isa 14:21; 24:4-\allowbreak6 Jer 51:34-\allowbreak38 Re 12:9,\allowbreak10; 18:2,\allowbreak3}
\crossref{Isa}{13}{12}{13:15-\allowbreak18; 4:1; 24:6 Ps 137:9}
\crossref{Isa}{13}{13}{Joe 3:16 Hag 2:6,\allowbreak7,\allowbreak21,\allowbreak22 Mt 24:29 Heb 12:26,\allowbreak27 Re 6:13,\allowbreak14}
\crossref{Isa}{13}{14}{Isa 17:13 1Ki 22:17,\allowbreak36}
\crossref{Isa}{13}{15}{Isa 14:19-\allowbreak22; 47:9-\allowbreak14 Jer 50:27,\allowbreak35-\allowbreak42; 51:3}
\crossref{Isa}{13}{16}{Ps 137:8,\allowbreak9 Ho 10:14 Na 3:10}
\crossref{Isa}{13}{17}{13:3-\allowbreak5; 21:2; 41:25 Jer 50:9; 51:11,\allowbreak27,\allowbreak28 Da 5:28-\allowbreak31}
\crossref{Isa}{13}{18}{13:16 2Ki 8:12 Ho 13:16 Na 2:1; 3:10}
\crossref{Isa}{13}{19}{Ge 19:24 De 29:23 Jer 49:18; 50:40 Zep 2:9}
\crossref{Isa}{13}{20}{Isa 14:23 Jer 50:3,\allowbreak13,\allowbreak21,\allowbreak39,\allowbreak45; 51:25,\allowbreak29,\allowbreak43,\allowbreak62-\allowbreak64 Re 18:21-\allowbreak23}
\crossref{Isa}{13}{21}{Isa 34:11-\allowbreak15 Re 18:2}
\crossref{Isa}{13}{22}{De 32:35 Jer 51:33 Eze 7:7-\allowbreak10 Hab 2:3 2Pe 2:3; 3:9,\allowbreak10}
\crossref{Isa}{14}{1}{Isa 40:1,\allowbreak2; 44:21,\allowbreak22; 54:7,\allowbreak8 Le 26:40-\allowbreak45 De 4:29-\allowbreak31 Ne 1:8,\allowbreak9}
\crossref{Isa}{14}{2}{Isa 18:7; 60:9-\allowbreak12; 61:5 Ezr 2:65 Ro 15:27 2Co 8:4,\allowbreak5 Ga 5:13}
\crossref{Isa}{14}{3}{Isa 12:1; 32:18 De 28:48,\allowbreak65-\allowbreak68 Ezr 9:8,\allowbreak9 Jer 30:10; 46:27,\allowbreak28; 50:34}
\crossref{Isa}{14}{4}{Jer 24:9 Eze 5:15 Hab 2:6}
\crossref{Isa}{14}{5}{14:29; 9:4; 10:5 Ps 125:3 Jer 48:15-\allowbreak17}
\crossref{Isa}{14}{6}{Isa 33:1; 47:6 2Ch 36:17 Jer 25:9 Da 7:19-\allowbreak21 Jas 2:13}
\crossref{Isa}{14}{7}{Isa 49:13 Ps 96:11-\allowbreak13; 98:7-\allowbreak9; 126:1-\allowbreak3 Pr 11:10 Jer 51:48 Re 18:20}
\crossref{Isa}{14}{8}{Isa 55:12,\allowbreak13 Eze 31:16 Zec 11:2}
\crossref{Isa}{14}{9}{Pr 15:24}
\crossref{Isa}{14}{10}{Ps 49:6-\allowbreak14,\allowbreak20; 82:6,\allowbreak7 Ec 2:16 Lu 16:20-\allowbreak23}
\crossref{Isa}{14}{11}{Isa 21:4,\allowbreak5; 22:2 Job 21:11-\allowbreak15 Eze 26:13; 32:19,\allowbreak20 Da 5:1-\allowbreak4,\allowbreak25-\allowbreak30}
\crossref{Isa}{14}{12}{Isa 13:10; 34:4 Eze 28:13-\allowbreak17 Lu 10:18 2Pe 2:4 Re 12:7-\allowbreak10}
\crossref{Isa}{14}{13}{Isa 47:7-\allowbreak10 Eze 27:3; 28:2; 29:3 Da 4:30,\allowbreak31 Zep 2:15 Re 18:7,\allowbreak8}
\crossref{Isa}{14}{14}{Isa 37:23,\allowbreak24}
\crossref{Isa}{14}{15}{14:3-\allowbreak11 Eze 28:8,\allowbreak9 Mt 11:23 Ac 12:22,\allowbreak23 Re 19:20}
\crossref{Isa}{14}{16}{Ps 58:10,\allowbreak11; 64:9}
\crossref{Isa}{14}{17}{Isa 13:19-\allowbreak22; 64:10 Eze 6:14 Joe 2:3 Zep 2:13,\allowbreak14}
\crossref{Isa}{14}{18}{Isa 22:16 2Ch 24:16,\allowbreak25 Ec 6:3 Eze 32:18-\allowbreak32}
\crossref{Isa}{14}{19}{Jer 41:7,\allowbreak9 Eze 32:23}
\crossref{Isa}{14}{20}{Isa 13:15-\allowbreak19 Job 18:16,\allowbreak19 Ps 21:10; 37:28; 109:13; 137:8,\allowbreak9}
\crossref{Isa}{14}{21}{Ex 20:5 Le 26:39 Mt 23:35}
\crossref{Isa}{14}{22}{Isa 13:5; 21:9; 43:14; 47:9-\allowbreak14 Jer 50:26,\allowbreak27,\allowbreak29-\allowbreak35; 51:3,\allowbreak4,\allowbreak56,\allowbreak57}
\crossref{Isa}{14}{23}{Isa 13:21,\allowbreak22; 34:11-\allowbreak15 Jer 50:39,\allowbreak40; 51:42,\allowbreak43 Zep 2:14 Re 14:8}
\crossref{Isa}{14}{24}{Ex 17:16 Ps 110:4 Jer 44:26 Am 8:7 Heb 4:3; 6:16-\allowbreak18}
\crossref{Isa}{14}{25}{Isa 9:4; 10:16-\allowbreak19,\allowbreak32-\allowbreak34; 17:12-\allowbreak14; 30:30-\allowbreak33; 31:8,\allowbreak9; 37:36-\allowbreak38}
\crossref{Isa}{14}{26}{Isa 5:25 Zep 3:6-\allowbreak8}
\crossref{Isa}{14}{27}{Isa 23:9; 43:13; 46:11 Job 40:8 Jer 4:28; 51:59 Ro 8:28,\allowbreak31}
\crossref{Isa}{14}{28}{Isa 6:1 2Ki 16:20 2Ch 28:27}
\crossref{Isa}{14}{29}{Pr 24:17 Eze 26:2; 35:15 Ho 9:1 Ob 1:12 Mic 7:8 Zep 3:11}
\crossref{Isa}{14}{30}{Job 18:13}
\crossref{Isa}{14}{31}{Isa 13:6; 16:7}
\crossref{Isa}{14}{32}{Isa 39:1 2Sa 8:10 2Ki 20:12-\allowbreak19}
\crossref{Isa}{15}{1}{Isa 11:14; 25:10 Jer 9:26; 48:1-\allowbreak47 Eze 25:8-\allowbreak11 Am 2:1-\allowbreak3 Zep 2:8-\allowbreak11}
\crossref{Isa}{15}{2}{Isa 16:12 Jos 13:17 Jer 48:18,\allowbreak22,\allowbreak23}
\crossref{Isa}{15}{3}{2Sa 3:31 2Ki 6:30 Jon 3:6-\allowbreak8 Mt 11:21}
\crossref{Isa}{15}{4}{Isa 16:8,\allowbreak9 Nu 32:3,\allowbreak4 Jer 48:34}
\crossref{Isa}{15}{5}{Isa 16:9-\allowbreak11 Jer 8:18,\allowbreak19; 9:10,\allowbreak18,\allowbreak19; 13:17; 17:16; 48:31-\allowbreak36}
\crossref{Isa}{15}{6}{Nu 32:3,\allowbreak36}
\crossref{Isa}{15}{7}{Isa 5:29; 10:6,\allowbreak14 Na 2:12,\allowbreak13}
\crossref{Isa}{15}{8}{15:2-\allowbreak5 Jer 48:20-\allowbreak24,\allowbreak31-\allowbreak34}
\crossref{Isa}{15}{9}{Le 26:18,\allowbreak21,\allowbreak24,\allowbreak28 Jer 48:43-\allowbreak45}
\crossref{Isa}{16}{1}{2Sa 8:2 2Ki 3:4 Ezr 7:17}
\crossref{Isa}{16}{2}{Isa 13:14 Pr 27:8}
\crossref{Isa}{16}{3}{Isa 1:17 Ps 82:3,\allowbreak4 Jer 21:12; 22:3 Eze 45:9-\allowbreak12 Da 4:27 Zec 7:9}
\crossref{Isa}{16}{4}{De 23:15,\allowbreak16; 24:14 Jer 21:12}
\crossref{Isa}{16}{5}{Ps 61:6,\allowbreak7; 85:10; 89:1,\allowbreak2,\allowbreak14 Pr 20:28; 29:14 Lu 1:69-\allowbreak75}
\crossref{Isa}{16}{6}{Isa 2:11 Jer 48:26,\allowbreak29,\allowbreak30,\allowbreak42 Am 2:1 Ob 1:3,\allowbreak4 Zep 2:9,\allowbreak10 1Pe 5:5}
\crossref{Isa}{16}{7}{Isa 15:2-\allowbreak5 Jer 48:20}
\crossref{Isa}{16}{8}{Isa 15:4; 24:7 2Sa 1:21}
\crossref{Isa}{16}{9}{Isa 15:5 Jer 48:32-\allowbreak34}
\crossref{Isa}{16}{10}{Isa 24:8,\allowbreak9; 32:10 Jer 48:33 Am 5:11,\allowbreak17 Hab 3:17,\allowbreak18 Zep 1:13}
\crossref{Isa}{16}{11}{Isa 15:5; 63:15 Jer 4:19; 31:20; 48:36 Ho 11:8 Php 2:1}
\crossref{Isa}{16}{12}{Isa 15:2; 26:16 Nu 22:39,\allowbreak41; 23:1-\allowbreak3,\allowbreak14,\allowbreak28; 24:17 Pr 1:28 Jer 48:35}
\crossref{Isa}{16}{13}{Isa 44:8}
\crossref{Isa}{16}{14}{Isa 7:16; 15:5; 21:16 De 15:8}
\crossref{Isa}{17}{1}{Isa 15:1; 19:1}
\crossref{Isa}{17}{2}{Nu 32:34 De 2:36; 3:12 Jos 13:16 Jer 48:19}
\crossref{Isa}{17}{3}{Isa 7:8,\allowbreak16; 8:4; 10:9 2Ki 16:9; 17:6 Ho 1:4,\allowbreak6; 3:4; 5:13,\allowbreak14; 8:8}
\crossref{Isa}{17}{4}{Isa 9:8,\allowbreak21; 10:4}
\crossref{Isa}{17}{5}{Jer 9:22; 51:33 Ho 6:11 Joe 3:13 Mt 13:30,\allowbreak39-\allowbreak42 Re 14:15-\allowbreak20}
\crossref{Isa}{17}{6}{Isa 1:9; 10:22; 24:13 De 4:27 Jud 8:2 1Ki 19:18 Eze 36:8-\allowbreak15}
\crossref{Isa}{17}{7}{Isa 10:20,\allowbreak21; 19:22; 22:11; 24:14,\allowbreak15; 29:18,\allowbreak19,\allowbreak24 Jud 10:15,\allowbreak16}
\crossref{Isa}{17}{8}{Isa 1:29; 2:18-\allowbreak21; 27:9; 30:22 2Ch 34:6,\allowbreak7 Eze 36:25 Ho 14:8 Zep 1:3}
\crossref{Isa}{17}{9}{17:4,\allowbreak5; 6:11-\allowbreak13; 7:16-\allowbreak20; 9:9-\allowbreak12; 24:1-\allowbreak12; 27:10; 28:1-\allowbreak4 Ho 10:14}
\crossref{Isa}{17}{10}{Isa 51:13 De 6:12; 8:11,\allowbreak14,\allowbreak19 Ps 9:17; 106:13,\allowbreak21 Jer 2:32; 17:13}
\crossref{Isa}{17}{11}{Isa 18:5,\allowbreak6 Job 4:8 Jer 5:31 Ho 8:7; 9:1-\allowbreak4,\allowbreak16; 10:12-\allowbreak15 Joe 1:5-\allowbreak12}
\crossref{Isa}{17}{12}{Isa 9:5}
\crossref{Isa}{17}{13}{Isa 10:15,\allowbreak16,\allowbreak33,\allowbreak34; 14:25; 25:4,\allowbreak5; 27:1; 30:30-\allowbreak33; 31:8,\allowbreak9; 33:1-\allowbreak3,\allowbreak9-\allowbreak12}
\crossref{Isa}{17}{14}{Isa 10:28-\allowbreak32 2Ki 19:3,\allowbreak35 Ps 37:36}
\crossref{Isa}{18}{1}{Isa 20:3-\allowbreak6; 30:2,\allowbreak3; 31:1}
\crossref{Isa}{18}{2}{Isa 30:2-\allowbreak4 Eze 30:9}
\crossref{Isa}{18}{3}{Isa 1:2 Ps 49:1,\allowbreak2; 50:1 Jer 22:29 Mic 6:2}
\crossref{Isa}{18}{4}{Isa 26:21 Ps 132:13,\allowbreak14 Ho 5:15}
\crossref{Isa}{18}{5}{Isa 17:11 So 2:13,\allowbreak15 Eze 17:6-\allowbreak10}
\crossref{Isa}{18}{6}{Isa 14:19; 34:1-\allowbreak7 Jer 7:33; 15:3 Eze 32:4-\allowbreak6; 39:17-\allowbreak20 Re 19:17,\allowbreak18}
\crossref{Isa}{18}{7}{Isa 16:1; 23:17,\allowbreak18; 45:14 2Ch 32:23 Ps 68:29-\allowbreak31; 72:9-\allowbreak15 Zep 3:10}
\crossref{Isa}{19}{1}{Jer 25:19; 43:8-\allowbreak13; 44:29,\allowbreak30; 46:1-\allowbreak28 Eze 29:1-\allowbreak32:32 Joe 3:19}
\crossref{Isa}{19}{2}{19:13,\allowbreak14; 9:21 Jud 7:22; 9:23 1Sa 14:16,\allowbreak20 2Ch 20:22,\allowbreak23 Eze 38:21}
\crossref{Isa}{19}{3}{19:1,\allowbreak11-\allowbreak13; 57:16 1Sa 25:37 Ps 76:12 Jer 46:15 Eze 21:7; 22:14}
\crossref{Isa}{19}{4}{1Sa 23:7 Ps 31:8}
\crossref{Isa}{19}{5}{Jer 51:36 Eze 30:12 Zec 10:11; 14:18}
\crossref{Isa}{19}{6}{Isa 37:25 2Ki 19:24}
\crossref{Isa}{19}{7}{Isa 32:20 Jer 14:4 Eze 19:13 Joe 1:17,\allowbreak18}
\crossref{Isa}{19}{8}{Ex 7:21 Nu 11:5 Eze 47:10 Hab 1:15}
\crossref{Isa}{19}{9}{1Ki 10:28 Pr 7:16 Eze 27:7}
\crossref{Isa}{19}{10}{Ex 7:19; 8:5 De 11:10}
\crossref{Isa}{19}{11}{19:3,\allowbreak13; 29:14; 44:25 Job 5:12,\allowbreak13; 12:17 Ps 33:10 Jer 49:7 Eze 7:26}
\crossref{Isa}{19}{12}{Isa 5:21; 47:10-\allowbreak13 Jud 9:38 Jer 2:28 1Co 1:20}
\crossref{Isa}{19}{13}{19:11 Ro 1:22}
\crossref{Isa}{19}{14}{19:2; 29:10,\allowbreak14; 47:10,\allowbreak11 1Ki 22:20-\allowbreak23 Job 12:16 Eze 14:7-\allowbreak9}
\crossref{Isa}{19}{15}{Isa 9:14,\allowbreak15 Ps 128:2 Pr 14:23 Hab 3:17 Hag 1:11 1Th 4:11,\allowbreak12}
\crossref{Isa}{19}{16}{Isa 30:17 Ps 48:6 Jer 30:5-\allowbreak7; 50:37; 51:30 Na 3:13}
\crossref{Isa}{19}{17}{Isa 36:1 Jer 25:19,\allowbreak27-\allowbreak31; 43:8-\allowbreak13; 44:28-\allowbreak30 Eze 29:6,\allowbreak7}
\crossref{Isa}{19}{18}{19:19,\allowbreak21; 2:11 Zec 2:11}
\crossref{Isa}{19}{19}{Isa 66:23 Ge 12:7; 28:18 Ex 24:4 Jos 22:10,\allowbreak26 Zec 6:15 Heb 13:10}
\crossref{Isa}{19}{20}{Isa 55:13 Jos 4:20,\allowbreak21; 22:27,\allowbreak28,\allowbreak34; 24:26,\allowbreak27}
\crossref{Isa}{19}{21}{Isa 11:9; 37:20; 55:5 1Sa 17:46 1Ki 8:43 Ps 67:2; 98:2,\allowbreak3 Hab 2:14}
\crossref{Isa}{19}{22}{19:1-\allowbreak15 De 32:39 Job 5:18 Ho 5:15; 6:2 Heb 12:11}
\crossref{Isa}{19}{23}{Isa 11:16; 35:8-\allowbreak10; 40:3-\allowbreak5 Eph 2:18-\allowbreak22; 3:6-\allowbreak8}
\crossref{Isa}{19}{24}{Isa 6:13; 49:6,\allowbreak22; 65:8,\allowbreak22; 66:12,\allowbreak19-\allowbreak21 De 32:43 Ps 117:1,\allowbreak2}
\crossref{Isa}{19}{25}{Isa 61:9; 65:23 Nu 6:24,\allowbreak27; 24:1 Ps 67:6,\allowbreak7; 115:15 Eph 1:3}
\crossref{Isa}{20}{1}{1Sa 6:17 Jer 25:20 Am 1:8}
\crossref{Isa}{20}{2}{Jer 13:1-\allowbreak11; 19:1-\allowbreak15 Eze 4:5 Mt 16:24}
\crossref{Isa}{20}{3}{Nu 14:34 Eze 4:5,\allowbreak6 Re 11:2,\allowbreak3}
\crossref{Isa}{20}{4}{Isa 19:4 Jer 46:26 Eze 30:18}
\crossref{Isa}{20}{5}{Isa 30:3,\allowbreak5,\allowbreak7; 36:6 2Ki 18:21 Eze 29:6,\allowbreak7}
\crossref{Isa}{20}{6}{Job 22:30 Jer 47:4}
\crossref{Isa}{21}{1}{Isa 13:20-\allowbreak22; 14:23 Jer 51:42}
\crossref{Isa}{21}{2}{Ps 60:3 Pr 13:15}
\crossref{Isa}{21}{3}{Isa 15:5; 16:9,\allowbreak11 Hab 3:16}
\crossref{Isa}{21}{4}{Isa 5:11-\allowbreak14 1Sa 25:36-\allowbreak38 2Sa 13:28,\allowbreak29 Es 5:12; 7:6-\allowbreak10}
\crossref{Isa}{21}{5}{Isa 22:13,\allowbreak14 Da 5:1-\allowbreak5 1Co 15:32}
\crossref{Isa}{21}{6}{Isa 62:6 2Ki 9:17-\allowbreak20 Jer 51:12,\allowbreak13 Eze 3:17; 33:2-\allowbreak7 Hab 2:1,\allowbreak2}
\crossref{Isa}{21}{7}{21:9; 37:24}
\crossref{Isa}{21}{8}{Isa 5:29 Jer 4:7; 25:38; 49:19; 50:44 1Pe 5:8}
\crossref{Isa}{21}{9}{Jer 50:3,\allowbreak9,\allowbreak29,\allowbreak42; 51:27}
\crossref{Isa}{21}{10}{Isa 41:15,\allowbreak16 2Ki 13:7 Jer 51:33 Mic 4:13 Hab 3:12 Mt 3:12}
\crossref{Isa}{21}{11}{Isa 34:1-\allowbreak17; 63:1-\allowbreak6 Nu 24:18 De 2:5 Ps 137:7 Jer 49:7-\allowbreak22}
\crossref{Isa}{21}{12}{Isa 17:14 Jer 50:27 Eze 7:5-\allowbreak7,\allowbreak10,\allowbreak12}
\crossref{Isa}{21}{13}{1Ki 10:15 Jer 25:23,\allowbreak24; 49:28-\allowbreak33 Ga 4:25}
\crossref{Isa}{21}{14}{Ge 25:15 1Ch 1:30 Job 6:19}
\crossref{Isa}{21}{15}{}
\crossref{Isa}{21}{16}{Isa 16:14 Job 7:1}
\crossref{Isa}{21}{17}{Isa 10:18,\allowbreak19; 17:4,\allowbreak5 Ps 107:39}
\crossref{Isa}{22}{1}{1Sa 3:1 Ps 147:19,\allowbreak20 Pr 29:18 Mic 3:6 Ro 3:2; 9:4,\allowbreak5}
\crossref{Isa}{22}{2}{22:12,\allowbreak13; 23:7; 32:13 Am 6:3-\allowbreak6}
\crossref{Isa}{22}{3}{Isa 3:1-\allowbreak8 2Ki 25:4-\allowbreak7,\allowbreak18-\allowbreak21 Jer 39:4-\allowbreak7; 52:24-\allowbreak27}
\crossref{Isa}{22}{4}{Ru 1:20,\allowbreak21 Jer 4:19; 9:1; 13:17 Lu 1:2}
\crossref{Isa}{22}{5}{Isa 37:3 2Ki 19:3 Jer 30:7 Am 5:18-\allowbreak20}
\crossref{Isa}{22}{6}{Isa 21:2 Ge 10:22 Jer 49:35-\allowbreak39}
\crossref{Isa}{22}{7}{Isa 8:7,\allowbreak8; 10:28-\allowbreak32; 37:34 Jer 39:1-\allowbreak3}
\crossref{Isa}{22}{8}{Isa 36:1-\allowbreak3}
\crossref{Isa}{22}{9}{2Ki 20:20 2Ch 32:1-\allowbreak6,\allowbreak30}
\crossref{Isa}{22}{10}{Le 15:13,\allowbreak28; 23:15}
\crossref{Isa}{22}{11}{Ne 3:16}
\crossref{Isa}{22}{12}{2Ch 35:25 Ne 8:9-\allowbreak12; 9:9 Ec 3:4,\allowbreak11 Joe 1:13; 2:17 Jas 4:8-\allowbreak10}
\crossref{Isa}{22}{13}{Isa 5:12; 21:4,\allowbreak5; 56:12 Am 6:3-\allowbreak7 Lu 17:26-\allowbreak29}
\crossref{Isa}{22}{14}{Isa 5:9 1Sa 9:15 Am 3:7}
\crossref{Isa}{22}{15}{1Ch 27:25 Ac 8:27}
\crossref{Isa}{22}{16}{Isa 52:5 Mic 2:10}
\crossref{Isa}{22}{17}{Es 7:8 Job 9:24 Jer 14:3}
\crossref{Isa}{22}{18}{Isa 17:13 Am 7:17}
\crossref{Isa}{22}{19}{Job 40:11,\allowbreak12 Ps 75:6,\allowbreak7 Eze 17:24 Lu 1:52}
\crossref{Isa}{22}{20}{Isa 36:3,\allowbreak11,\allowbreak22; 37:2 2Ki 18:18,\allowbreak37}
\crossref{Isa}{22}{21}{Ge 41:42,\allowbreak43 1Sa 18:4 Es 8:2,\allowbreak15}
\crossref{Isa}{22}{22}{Job 12:14 Mt 18:18,\allowbreak19 Re 3:7}
\crossref{Isa}{22}{23}{Ezr 9:8 Ec 12:11 Zec 10:4}
\crossref{Isa}{22}{24}{Ge 41:44,\allowbreak45; 47:11-\allowbreak25 Da 6:1-\allowbreak3 Mt 28:18 Joh 5:22-\allowbreak27; 20:21-\allowbreak23}
\crossref{Isa}{22}{25}{22:15,\allowbreak16}
\crossref{Isa}{23}{1}{Isa 15:2,\allowbreak8 Re 18:17-\allowbreak19}
\crossref{Isa}{23}{2}{Isa 41:1; 47:5 Ps 46:10 Hab 2:20}
\crossref{Isa}{23}{3}{1Ch 13:5 Jer 2:18}
\crossref{Isa}{23}{4}{Jer 47:3,\allowbreak4 Eze 26:3-\allowbreak6 Ho 9:11-\allowbreak14 Re 18:23}
\crossref{Isa}{23}{5}{Isa 19:16 Ex 15:14-\allowbreak16 Jos 2:9-\allowbreak11}
\crossref{Isa}{23}{6}{23:10,\allowbreak12; 21:15}
\crossref{Isa}{23}{7}{Isa 22:2}
\crossref{Isa}{23}{8}{De 29:24-\allowbreak28 Jer 50:44,\allowbreak45 Re 18:8}
\crossref{Isa}{23}{9}{Isa 10:33; 14:24,\allowbreak27; 46:10,\allowbreak11 Jer 47:6,\allowbreak7; 51:62 Ac 4:28 Eph 1:11}
\crossref{Isa}{23}{10}{23:12}
\crossref{Isa}{23}{11}{Isa 2:19; 14:16,\allowbreak17 Ex 15:8-\allowbreak10 Ps 46:6 Eze 26:10,\allowbreak15-\allowbreak19; 27:34,\allowbreak35}
\crossref{Isa}{23}{12}{23:1,\allowbreak7 Eze 26:13,\allowbreak14 Re 18:22}
\crossref{Isa}{23}{13}{Isa 13:19 Ge 11:28,\allowbreak31 Job 1:17 Hab 1:6 Ac 7:4}
\crossref{Isa}{23}{14}{23:1,\allowbreak6 Eze 27:25-\allowbreak30 Re 18:11-\allowbreak19}
\crossref{Isa}{23}{15}{Jer 25:9-\allowbreak11,\allowbreak22; 27:3-\allowbreak7; 29:10 Eze 29:11}
\crossref{Isa}{23}{16}{Pr 7:10-\allowbreak12 Jer 30:14}
\crossref{Isa}{23}{17}{Jer 29:10 Zep 2:7 Ac 15:14}
\crossref{Isa}{23}{18}{Isa 60:6,\allowbreak7 2Ch 2:7-\allowbreak9,\allowbreak11-\allowbreak16 Ps 45:12; 72:10 Zec 14:20,\allowbreak21 Mr 3:8}
\crossref{Isa}{24}{1}{Isa 1:7-\allowbreak9; 5:6; 6:11,\allowbreak12; 7:17-\allowbreak25; 27:10; 32:13,\allowbreak14; 42:15 Jer 4:7}
\crossref{Isa}{24}{2}{Isa 2:9; 3:2-\allowbreak8; 5:15; 9:14-\allowbreak17 2Ch 36:14-\allowbreak17,\allowbreak20 Jer 5:3-\allowbreak6; 23:11-\allowbreak13}
\crossref{Isa}{24}{3}{24:1; 6:11 Le 26:30-\allowbreak35 De 29:23,\allowbreak28 2Ch 36:21 Eze 36:4}
\crossref{Isa}{24}{4}{Isa 3:26; 28:1; 33:9; 64:6 Jer 4:28; 12:4 Ho 4:3}
\crossref{Isa}{24}{5}{Ge 3:17,\allowbreak18; 6:11-\allowbreak13 Le 18:24-\allowbreak28; 20:22 Nu 35:33,\allowbreak34 2Ch 33:9}
\crossref{Isa}{24}{6}{Isa 42:24,\allowbreak25 De 28:15-\allowbreak20; 29:22-\allowbreak28; 30:18,\allowbreak19 Jos 23:15,\allowbreak16 Zec 5:3,\allowbreak4}
\crossref{Isa}{24}{7}{Isa 16:8,\allowbreak10; 32:9-\allowbreak13 Ho 9:1,\allowbreak2 Joe 1:10-\allowbreak12}
\crossref{Isa}{24}{8}{Isa 23:15,\allowbreak16 Jer 7:34; 16:9; 25:10 Eze 26:13 Ho 2:11 Re 18:22}
\crossref{Isa}{24}{9}{Isa 5:11,\allowbreak12 Ps 69:12 Ec 9:7 Am 6:5-\allowbreak7; 8:3,\allowbreak10 Zec 9:15 Eph 5:18,\allowbreak19}
\crossref{Isa}{24}{10}{24:12; 25:2; 27:10; 32:14; 34:13-\allowbreak15 2Ki 25:4,\allowbreak9,\allowbreak10 Jer 39:4,\allowbreak8}
\crossref{Isa}{24}{11}{Pr 31:6 Ho 7:14 Joe 1:15}
\crossref{Isa}{24}{12}{Isa 32:14 Jer 9:11 La 1:1,\allowbreak4; 2:9; 5:18 Mic 1:9,\allowbreak12 Mt 22:7}
\crossref{Isa}{24}{13}{Isa 1:9; 6:13; 10:20-\allowbreak22; 17:5,\allowbreak6 Jer 44:28 Eze 6:8-\allowbreak11; 7:16; 9:4-\allowbreak6}
\crossref{Isa}{24}{14}{Isa 12:1-\allowbreak6; 25:1; 26:1; 27:2; 35:2,\allowbreak10; 40:9; 42:10-\allowbreak12; 44:23; 51:11}
\crossref{Isa}{24}{15}{Job 35:9,\allowbreak10 Hab 3:17,\allowbreak18 Zec 13:8,\allowbreak9 Ac 16:25 1Pe 1:7; 3:15}
\crossref{Isa}{24}{16}{Isa 26:15; 45:22-\allowbreak25; 52:10; 66:19,\allowbreak20 Ps 2:8; 22:27-\allowbreak31; 67:7; 72:8-\allowbreak11}
\crossref{Isa}{24}{17}{Le 26:21,\allowbreak22 1Ki 19:17 Jer 8:3; 48:43,\allowbreak44 Eze 14:21}
\crossref{Isa}{24}{18}{De 32:23-\allowbreak26 Jos 10:10,\allowbreak11 1Ki 20:29,\allowbreak30 Job 18:8-\allowbreak16; 20:24}
\crossref{Isa}{24}{19}{24:1-\allowbreak5; 34:4-\allowbreak10 Jer 4:23-\allowbreak28 Na 1:5 Hab 3:6 Mt 24:3 Re 20:11}
\crossref{Isa}{24}{20}{Isa 19:14; 29:9 Ps 107:27}
\crossref{Isa}{24}{21}{Isa 10:25-\allowbreak27; 14:1,\allowbreak2; 25:10-\allowbreak12; 34:2-\allowbreak17 Ps 76:12; 149:6-\allowbreak9}
\crossref{Isa}{24}{22}{24:17; 2:19 Jos 10:16,\allowbreak17,\allowbreak22-\allowbreak26}
\crossref{Isa}{24}{23}{Isa 13:10; 30:26; 60:19 Eze 32:7,\allowbreak8 Joe 2:31; 3:15 Mr 13:24-\allowbreak26}
\crossref{Isa}{25}{1}{Isa 26:13; 61:10 Ex 15:2 1Ch 29:10-\allowbreak20 Ps 99:5; 118:28; 145:1; 146:2}
\crossref{Isa}{25}{2}{25:12; 14:23; 17:1; 21:9; 23:13 De 13:16 Jer 51:26 Na 3:12-\allowbreak15}
\crossref{Isa}{25}{3}{Isa 49:23-\allowbreak26; 60:10-\allowbreak14; 66:18-\allowbreak20 Ps 46:10,\allowbreak11; 66:3; 72:8-\allowbreak11 Eze 38:23}
\crossref{Isa}{25}{4}{Isa 11:4; 14:32; 29:19; 33:2; 66:2 Job 5:15,\allowbreak16 Ps 12:5; 35:10}
\crossref{Isa}{25}{5}{Isa 10:8-\allowbreak15,\allowbreak32-\allowbreak34; 13:11; 14:10-\allowbreak16; 17:12-\allowbreak14; 30:30-\allowbreak33; 49:25,\allowbreak26}
\crossref{Isa}{25}{6}{25:10; 2:2,\allowbreak3 Ps 72:14-\allowbreak16; 78:68 Mic 4:1,\allowbreak2 Zec 8:3 Heb 12:22}
\crossref{Isa}{25}{7}{Isa 60:1-\allowbreak3 Mt 27:51 Lu 2:32 Ac 17:30 2Co 3:13-\allowbreak18 Eph 3:5,\allowbreak6; 4:18}
\crossref{Isa}{25}{8}{Ho 13:14 1Co 15:26,\allowbreak54 2Ti 1:10 Heb 2:14,\allowbreak15 Re 20:14; 21:4}
\crossref{Isa}{25}{9}{Isa 12:1 Zep 3:14-\allowbreak20 Re 1:7; 19:1-\allowbreak7}
\crossref{Isa}{25}{10}{25:6; 11:10; 12:6; 18:4 Ps 132:13,\allowbreak14 Eze 48:35 Zep 3:15-\allowbreak17}
\crossref{Isa}{25}{11}{Isa 5:25; 14:26; 65:2 Col 2:15}
\crossref{Isa}{25}{12}{Isa 26:5 Jer 51:58,\allowbreak64 2Co 10:4,\allowbreak5 Heb 11:30 Re 18:21}
\crossref{Isa}{26}{1}{Isa 2:11,\allowbreak20; 12:1; 24:21-\allowbreak23; 25:9}
\crossref{Isa}{26}{2}{Isa 60:11; 62:10 Ps 118:20 Eze 48:31-\allowbreak34 Zec 8:20 Ac 2:47}
\crossref{Isa}{26}{3}{Isa 9:6,\allowbreak7; 57:19-\allowbreak21 Ps 85:7,\allowbreak8 Mic 5:5 Joh 14:27; 16:33 Ro 5:1}
\crossref{Isa}{26}{4}{Isa 12:2; 50:10 2Ch 20:20; 32:8 Ps 55:22; 62:8 Pr 3:5,\allowbreak6}
\crossref{Isa}{26}{5}{Isa 2:12; 13:11; 14:13; 25:11 Job 40:11-\allowbreak13}
\crossref{Isa}{26}{6}{Isa 25:10; 37:25; 60:14 Jos 10:24 Jer 50:45 Da 7:27 Zep 3:11}
\crossref{Isa}{26}{7}{Isa 35:8 1Ch 29:17 Job 27:5,\allowbreak6 Ps 18:23-\allowbreak26 Pr 20:7 2Co 1:12}
\crossref{Isa}{26}{8}{Isa 64:4,\allowbreak5 Nu 36:13 Job 23:10-\allowbreak12 Ps 18:23; 44:17,\allowbreak18; 65:6; 106:3}
\crossref{Isa}{26}{9}{Ps 63:6,\allowbreak7; 77:2,\allowbreak3; 119:62; 130:6 So 3:1-\allowbreak4; 5:2-\allowbreak8 Lu 6:12}
\crossref{Isa}{26}{10}{Isa 63:9,\allowbreak10 Ex 8:15,\allowbreak31,\allowbreak32; 9:34 De 32:15 1Sa 15:17 Ps 106:43}
\crossref{Isa}{26}{11}{Ps 10:12 Mic 5:9}
\crossref{Isa}{26}{12}{Isa 57:10 Ps 29:11 Jer 33:6 Joh 14:27}
\crossref{Isa}{26}{13}{Isa 51:22 2Ch 12:8 Joh 8:32 Ro 6:22}
\crossref{Isa}{26}{14}{26:19; 8:19; 51:12,\allowbreak13 Ex 14:30 Ps 106:28 Hab 2:18-\allowbreak20 Mt 2:20}
\crossref{Isa}{26}{15}{Isa 9:3; 10:22 Ge 12:2; 13:16 Nu 23:10 De 10:22 Ne 9:23 Jer 30:19}
\crossref{Isa}{26}{16}{De 4:29,\allowbreak30 Jud 10:9,\allowbreak10 2Ch 6:37,\allowbreak38; 33:12,\allowbreak13 Ps 50:15; 77:1,\allowbreak2}
\crossref{Isa}{26}{17}{Isa 13:8; 21:3 Ps 48:6 Jer 4:31; 6:24; 30:6 Joh 16:21 1Th 5:3}
\crossref{Isa}{26}{18}{Isa 37:3 2Ki 19:3 Ho 13:13}
\crossref{Isa}{26}{19}{Isa 25:8 Eze 37:1-\allowbreak14 Ho 6:2; 13:14 Joh 5:28,\allowbreak29 Ac 24:15}
\crossref{Isa}{26}{20}{Isa 51:4,\allowbreak16 Jer 7:23; 31:14}
\crossref{Isa}{26}{21}{Isa 18:4 Ps 50:2,\allowbreak3 Eze 8:6; 9:3-\allowbreak6; 10:3-\allowbreak5,\allowbreak18,\allowbreak19 Ho 5:14,\allowbreak15}
\crossref{Isa}{27}{1}{Isa 26:21}
\crossref{Isa}{27}{2}{Isa 5:1-\allowbreak7 Nu 21:17}
\crossref{Isa}{27}{3}{Isa 46:4,\allowbreak9; 60:16 Ge 6:17; 9:9 Eze 34:11,\allowbreak24; 37:14,\allowbreak28}
\crossref{Isa}{27}{4}{Isa 12:1; 26:20,\allowbreak21; 54:6-\allowbreak10 Ps 85:3; 103:9 Eze 16:63 Na 1:3-\allowbreak7}
\crossref{Isa}{27}{5}{Isa 25:4; 26:3,\allowbreak4; 45:24; 56:2; 64:7 Jos 9:24,\allowbreak25; 10:6 Job 22:21}
\crossref{Isa}{27}{6}{Isa 6:13; 37:31; 49:20-\allowbreak23; 54:1-\allowbreak3; 60:22 Ps 92:13-\allowbreak15 Jer 30:19}
\crossref{Isa}{27}{7}{Isa 10:20-\allowbreak25; 14:22,\allowbreak23; 17:3,\allowbreak14 Jer 30:11-\allowbreak16; 50:33,\allowbreak34,\allowbreak40; 51:24}
\crossref{Isa}{27}{8}{Isa 57:16 Job 23:6 Ps 6:1; 38:1; 103:14 Jer 10:24; 30:11; 46:28}
\crossref{Isa}{27}{9}{Isa 1:24,\allowbreak25; 4:4; 48:10 Ps 119:67,\allowbreak71 Pr 20:30}
\crossref{Isa}{27}{10}{Isa 5:9,\allowbreak10; 6:11,\allowbreak12; 17:9; 25:2; 64:10 Jer 26:6,\allowbreak18 La 1:4; 2:5-\allowbreak9}
\crossref{Isa}{27}{11}{Ps 80:15,\allowbreak16 Eze 15:2-\allowbreak8; 20:47 Mt 3:10 Joh 15:6}
\crossref{Isa}{27}{12}{Isa 11:11-\allowbreak16; 24:13-\allowbreak16; 56:8 Ge 15:18 Ps 68:22; 72:8}
\crossref{Isa}{27}{13}{Isa 2:11}
\crossref{Isa}{28}{1}{28:7; 5:11,\allowbreak22 Pr 23:29 Ho 4:11; 7:5 Am 2:8,\allowbreak12; 6:6}
\crossref{Isa}{28}{2}{Isa 9:9-\allowbreak12; 27:1 Eze 30:10,\allowbreak11}
\crossref{Isa}{28}{3}{28:1}
\crossref{Isa}{28}{4}{28:1 Ps 73:19,\allowbreak20 Ho 6:4; 9:10,\allowbreak11,\allowbreak16; 13:1,\allowbreak15 Jas 1:10,\allowbreak11}
\crossref{Isa}{28}{5}{Isa 41:16; 45:25; 60:1-\allowbreak3,\allowbreak19; 62:3 Jer 9:23,\allowbreak24 Zec 6:13-\allowbreak15 Lu 2:32}
\crossref{Isa}{28}{6}{Isa 11:2-\allowbreak4; 32:15,\allowbreak16 Ge 41:38,\allowbreak39 Nu 11:16,\allowbreak17; 27:16-\allowbreak18 1Ki 3:28}
\crossref{Isa}{28}{7}{Isa 19:14; 56:10-\allowbreak12 Le 10:9,\allowbreak10 Pr 20:1; 31:4,\allowbreak5 Ec 10:17 Eze 44:21}
\crossref{Isa}{28}{8}{Pr 26:11 Jer 48:26 Hab 2:15,\allowbreak16}
\crossref{Isa}{28}{9}{Isa 30:10-\allowbreak12 Ps 50:17 Pr 1:29 Jer 5:31; 6:10 Joh 3:19; 12:38,\allowbreak47,\allowbreak48}
\crossref{Isa}{28}{10}{28:13; 5:4 De 6:1-\allowbreak6 2Ch 36:15,\allowbreak16 Ne 9:29,\allowbreak30 Jer 11:7; 25:3-\allowbreak7}
\crossref{Isa}{28}{11}{De 28:49 Jer 5:15 1Co 14:21}
\crossref{Isa}{28}{12}{Isa 30:15 2Ch 14:11; 16:8,\allowbreak9 Jer 6:16 Mt 11:28,\allowbreak29}
\crossref{Isa}{28}{13}{28:10 Jer 23:36-\allowbreak38 Ho 6:5; 8:12}
\crossref{Isa}{28}{14}{28:22; 1:10; 5:9; 29:20 Pr 1:22; 3:34; 29:8 Ho 7:5 Ac 13:41}
\crossref{Isa}{28}{15}{Isa 8:7,\allowbreak8 Da 11:22}
\crossref{Isa}{28}{16}{Isa 8:14 Ge 49:10,\allowbreak24 Ps 118:22 Zec 3:9 Mt 21:42 Mr 12:10}
\crossref{Isa}{28}{17}{Isa 10:22 2Ki 21:13 Ps 94:15 Am 7:7-\allowbreak9 Ro 2:2,\allowbreak5; 9:28 Re 19:2}
\crossref{Isa}{28}{18}{Isa 7:7; 8:10 Jer 44:28 Eze 17:15 Zec 1:6}
\crossref{Isa}{28}{19}{Isa 10:5,\allowbreak6 2Ki 17:6; 18:13 Eze 21:19-\allowbreak23}
\crossref{Isa}{28}{20}{Isa 57:12,\allowbreak13; 59:5,\allowbreak6; 64:6; 66:3-\allowbreak6 Jer 7:8-\allowbreak10 Ro 9:30-\allowbreak32 1Co 1:18-\allowbreak31}
\crossref{Isa}{28}{21}{2Sa 5:20 1Ch 14:11}
\crossref{Isa}{28}{22}{28:15 2Ch 30:10; 36:16 Jer 15:17; 20:7 Mt 27:39,\allowbreak44 Ac 13:40,\allowbreak41}
\crossref{Isa}{28}{23}{Isa 1:2 De 32:1 Jer 22:29 Re 2:7,\allowbreak11,\allowbreak14,\allowbreak29}
\crossref{Isa}{28}{24}{Jer 4:3 Ho 10:11,\allowbreak12}
\crossref{Isa}{28}{25}{}
\crossref{Isa}{28}{26}{}
\crossref{Isa}{28}{27}{Isa 41:15 2Ki 13:7 Am 1:3}
\crossref{Isa}{28}{28}{Isa 21:10 Am 9:9 Mt 3:12; 13:37-\allowbreak43 Lu 22:31,\allowbreak32 Joh 12:24 1Co 3:9}
\crossref{Isa}{28}{29}{28:21,\allowbreak22; 9:6 Job 5:9; 37:23 Ps 40:5; 92:5 Jer 32:19 Da 4:2,\allowbreak3}
\crossref{Isa}{29}{1}{2Sa 5:9}
\crossref{Isa}{29}{2}{Isa 5:25-\allowbreak30; 10:5,\allowbreak6,\allowbreak32; 17:14; 24:1-\allowbreak12; 33:7-\allowbreak9; 36:22; 37:3}
\crossref{Isa}{29}{3}{2Ki 18:17; 19:32; 24:11,\allowbreak12; 25:1-\allowbreak4 Eze 21:22 Mt 22:7 Lu 19:43,\allowbreak44}
\crossref{Isa}{29}{4}{Isa 2:11-\allowbreak21; 3:8; 51:23 Ps 44:25 La 1:9}
\crossref{Isa}{29}{5}{Isa 10:16-\allowbreak19; 25:5; 31:3,\allowbreak8; 37:36}
\crossref{Isa}{29}{6}{Isa 5:26-\allowbreak30; 28:2; 30:30; 33:11-\allowbreak14 1Sa 2:10; 12:17,\allowbreak18 2Sa 22:14}
\crossref{Isa}{29}{7}{Isa 37:36; 41:11,\allowbreak12 Jer 25:31-\allowbreak33; 51:42-\allowbreak44 Na 1:3-\allowbreak12 Zec 12:3-\allowbreak5}
\crossref{Isa}{29}{8}{Isa 10:7-\allowbreak16 2Ch 32:21}
\crossref{Isa}{29}{9}{Isa 1:2; 33:13,\allowbreak14 Jer 2:12 Hab 1:5 Ac 13:40,\allowbreak41 Re 17:6}
\crossref{Isa}{29}{10}{29:14; 6:9,\allowbreak10 1Sa 26:12 Ps 69:23 Mic 3:6 Ac 28:26,\allowbreak27 Ro 11:8}
\crossref{Isa}{29}{11}{Isa 8:16}
\crossref{Isa}{29}{12}{29:18; 28:12,\allowbreak13 Jer 5:4 Ho 4:6 Joh 7:15,\allowbreak16}
\crossref{Isa}{29}{13}{Isa 10:6; 48:1,\allowbreak2; 58:2,\allowbreak3 Ps 17:1 Jer 3:10; 5:2; 12:2; 42:2-\allowbreak4,\allowbreak20}
\crossref{Isa}{29}{14}{29:9; 28:21 Hab 1:5 Joh 9:29-\allowbreak34}
\crossref{Isa}{29}{15}{Isa 5:18,\allowbreak19; 28:15,\allowbreak17; 30:1 Job 22:13,\allowbreak14 Ps 10:11-\allowbreak13; 64:5,\allowbreak6}
\crossref{Isa}{29}{16}{Isa 24:1 Ac 17:6}
\crossref{Isa}{29}{17}{Isa 63:18 Hab 2:3 Hag 2:6 Heb 10:37}
\crossref{Isa}{29}{18}{29:10-\allowbreak12,\allowbreak24; 35:5; 42:16-\allowbreak18 De 29:4 Ps 119:18 Pr 20:12}
\crossref{Isa}{29}{19}{Isa 61:1 Ps 25:9; 37:11; 149:4 Zep 2:3 Mt 5:5; 11:29 Ga 5:22,\allowbreak23}
\crossref{Isa}{29}{20}{29:5; 13:3; 25:4,\allowbreak5; 49:25; 51:13 Da 7:7,\allowbreak19-\allowbreak25 Hab 1:6,\allowbreak7}
\crossref{Isa}{29}{21}{Jud 12:6 Mt 22:15 Lu 11:53,\allowbreak54}
\crossref{Isa}{29}{22}{Isa 41:8,\allowbreak9,\allowbreak14; 44:21-\allowbreak23; 51:2,\allowbreak11; 54:4 Ge 48:16 Jos 24:2-\allowbreak5 Ne 9:7,\allowbreak8}
\crossref{Isa}{29}{23}{Isa 19:25; 43:21; 45:11; 60:21 Eph 2:10}
\crossref{Isa}{29}{24}{29:10,\allowbreak11; 28:7 Zec 12:10 Mt 21:28-\allowbreak32 Lu 7:47; 15:17-\allowbreak19 Ac 2:37; 6:7}
\crossref{Isa}{30}{1}{30:9; 1:2; 63:10; 65:2 De 9:7,\allowbreak24; 29:19 Jer 4:17; 5:23 Eze 2:3}
\crossref{Isa}{30}{2}{Isa 20:5,\allowbreak6; 31:1-\allowbreak3; 36:6 De 28:68 2Ki 17:4 Jer 37:5; 43:7 Eze 29:6,\allowbreak7}
\crossref{Isa}{30}{3}{30:5-\allowbreak7; 20:5 Jer 37:5-\allowbreak10}
\crossref{Isa}{30}{4}{Isa 57:9 2Ki 17:4 Ho 7:11,\allowbreak12,\allowbreak16}
\crossref{Isa}{30}{5}{30:16; 20:5,\allowbreak6; 31:1-\allowbreak3 Jer 2:36}
\crossref{Isa}{30}{6}{Isa 46:1,\allowbreak2; 57:9 Ho 8:9,\allowbreak10; 12:1}
\crossref{Isa}{30}{7}{Isa 31:1-\allowbreak5 Jer 37:7}
\crossref{Isa}{30}{8}{Isa 8:1 De 31:19,\allowbreak22 Job 19:23,\allowbreak24 Jer 36:2,\allowbreak28-\allowbreak32; 51:60 Hab 2:2}
\crossref{Isa}{30}{9}{30:1; 1:4 De 31:27-\allowbreak29; 32:20 Jer 44:2-\allowbreak17 Zep 3:2 Mt 23:31-\allowbreak33}
\crossref{Isa}{30}{10}{1Ki 21:20 2Ch 16:10; 18:7-\allowbreak27; 24:19-\allowbreak21; 25:16 Jer 5:31; 11:21}
\crossref{Isa}{30}{11}{Isa 29:21 Am 7:13}
\crossref{Isa}{30}{12}{30:1,\allowbreak7,\allowbreak15-\allowbreak17; 5:24; 31:1-\allowbreak3 2Sa 12:9,\allowbreak10 Am 2:4 Lu 10:16 1Th 4:8}
\crossref{Isa}{30}{13}{1Ki 20:30 Ps 62:3 Eze 13:10-\allowbreak15 Mt 7:27 Lu 6:49}
\crossref{Isa}{30}{14}{Ps 2:9 Jer 19:10,\allowbreak11 Re 2:27}
\crossref{Isa}{30}{15}{30:11 Jer 23:36}
\crossref{Isa}{30}{16}{Isa 5:26-\allowbreak30; 10:28-\allowbreak32; 31:1 De 28:25 2Ki 25:5 Ps 33:17; 147:10}
\crossref{Isa}{30}{17}{Le 26:8,\allowbreak36 De 28:25; 32:30 Jos 23:10 Pr 28:1 Jer 37:10}
\crossref{Isa}{30}{18}{Isa 55:8 Ex 34:6 Ho 2:14 Ro 5:20; 9:15-\allowbreak18}
\crossref{Isa}{30}{19}{Isa 10:24; 12:6; 46:13; 65:9 Jer 31:6,\allowbreak12; 50:4,\allowbreak5,\allowbreak28; 51:10 Eze 20:40}
\crossref{Isa}{30}{20}{De 16:3 1Ki 22:27 2Ch 18:26 Ps 30:5; 80:5; 102:9; 127:2}
\crossref{Isa}{30}{21}{Isa 35:8,\allowbreak9; 42:16; 48:17; 58:11 Ps 25:8,\allowbreak9; 143:8-\allowbreak10 Pr 3:5,\allowbreak6 Jer 6:16}
\crossref{Isa}{30}{22}{Isa 2:20,\allowbreak21; 17:7,\allowbreak8; 27:9; 31:7 2Ki 23:4-\allowbreak20 2Ch 31:1; 34:3-\allowbreak7}
\crossref{Isa}{30}{23}{Isa 5:6; 32:20; 44:2-\allowbreak4; 55:10,\allowbreak11; 58:11 Ps 65:9-\allowbreak13; 104:13,\allowbreak14}
\crossref{Isa}{30}{24}{De 25:4 1Co 9:9,\allowbreak10}
\crossref{Isa}{30}{25}{Isa 2:14,\allowbreak15; 35:6,\allowbreak7; 41:18,\allowbreak19; 43:19,\allowbreak20; 44:3,\allowbreak4 Eze 17:22; 34:13,\allowbreak26}
\crossref{Isa}{30}{26}{Isa 11:9; 24:23; 60:19,\allowbreak20 Zec 12:8; 14:7 Re 21:23; 22:5}
\crossref{Isa}{30}{27}{Isa 9:5; 10:16,\allowbreak17; 33:12; 34:9 De 32:22; 33:2 Ps 18:7-\allowbreak9; 79:5}
\crossref{Isa}{30}{28}{Isa 11:4 Ps 18:15 Lu 22:31 2Th 2:8 Heb 4:12 Re 1:16; 2:16}
\crossref{Isa}{30}{29}{Isa 12:1; 26:1 Ex 15:1-\allowbreak21 2Ch 20:27,\allowbreak28 Ps 32:7 Jer 33:11 Re 15:3}
\crossref{Isa}{30}{30}{Isa 29:6 Ps 2:5; 18:13,\allowbreak14; 46:6}
\crossref{Isa}{30}{31}{30:30; 37:32-\allowbreak38}
\crossref{Isa}{30}{32}{Isa 2:19; 11:15; 19:16 Job 16:12 Heb 12:26}
\crossref{Isa}{30}{33}{2Ki 23:10 Jer 7:31,\allowbreak32; 19:6,\allowbreak11-\allowbreak14 Mt 4:22; 18:8,\allowbreak9}
\crossref{Isa}{31}{1}{Isa 30:1-\allowbreak7; 36:6; 57:9 Eze 17:15 Ho 11:5}
\crossref{Isa}{31}{2}{1Sa 2:3 Job 5:13 Jer 10:7,\allowbreak12 1Co 1:21-\allowbreak29 Jude 1:25}
\crossref{Isa}{31}{3}{Isa 36:6 De 32:30,\allowbreak31 Ps 9:20; 146:3-\allowbreak5 Eze 28:9 Ac 12:22,\allowbreak23}
\crossref{Isa}{31}{4}{Nu 24:8,\allowbreak9 Jer 50:44 Ho 11:10 Am 3:8 Re 5:5}
\crossref{Isa}{31}{5}{Isa 10:14 Ex 19:4 De 32:11 Ps 46:5; 91:4}
\crossref{Isa}{31}{6}{Isa 55:7 Jer 3:10,\allowbreak14,\allowbreak22; 31:18-\allowbreak20 Ho 14:1-\allowbreak3 Joe 2:12,\allowbreak13 Ac 3:19}
\crossref{Isa}{31}{7}{Isa 2:20; 30:22 De 7:25 Eze 36:25 Ho 14:8}
\crossref{Isa}{31}{8}{Isa 10:16-\allowbreak19,\allowbreak33,\allowbreak34; 14:25; 29:5; 30:27-\allowbreak33; 37:35 2Ki 19:34-\allowbreak37}
\crossref{Isa}{31}{9}{Isa 4:4; 29:6 Le 6:13 Eze 22:18-\allowbreak22 Zec 2:5 Mal 4:1}
\crossref{Isa}{32}{1}{Isa 9:6,\allowbreak7; 40:1-\allowbreak5 2Sa 23:3 2Ch 31:20,\allowbreak21 Ps 45:1,\allowbreak6,\allowbreak7; 72:1,\allowbreak2; 99:4}
\crossref{Isa}{32}{2}{Isa 7:14; 8:10-\allowbreak14; 9:6 Ps 146:3-\allowbreak5 Mic 5:4,\allowbreak5 Zec 13:7 1Ti 3:16}
\crossref{Isa}{32}{3}{Isa 29:18,\allowbreak24; 30:26; 35:5,\allowbreak6; 54:13; 60:1,\allowbreak2 Jer 31:34 Mt 13:11}
\crossref{Isa}{32}{4}{Isa 29:24 Ne 8:8-\allowbreak12 Mt 11:25; 16:17 Ac 6:7; 26:9-\allowbreak11 Ga 1:23}
\crossref{Isa}{32}{5}{Isa 5:20 Ps 15:4 Mal 3:18}
\crossref{Isa}{32}{6}{1Sa 24:13; 25:10,\allowbreak11 Jer 13:23 Mt 12:34-\allowbreak36; 15:19 Jas 3:5,\allowbreak6}
\crossref{Isa}{32}{7}{Isa 1:23; 5:23 Jer 5:26-\allowbreak28 Mic 2:11; 7:3 Mt 26:14-\allowbreak16,\allowbreak59,\allowbreak60}
\crossref{Isa}{32}{8}{2Sa 9:1-\allowbreak13 Job 31:16-\allowbreak21 Ps 112:9 Pr 11:24 Lu 6:33-\allowbreak35 Ac 9:39}
\crossref{Isa}{32}{9}{Isa 3:16; 47:7,\allowbreak8 De 28:56 Jer 6:2-\allowbreak6; 48:11,\allowbreak12 La 4:5 Am 6:1-\allowbreak6}
\crossref{Isa}{32}{10}{Isa 3:17-\allowbreak26; 24:7-\allowbreak12 Jer 25:10,\allowbreak11 Ho 3:4}
\crossref{Isa}{32}{11}{Isa 2:19,\allowbreak21; 22:4,\allowbreak5; 33:14 Lu 23:27-\allowbreak30 Jas 5:5}
\crossref{Isa}{32}{12}{La 2:11; 4:3,\allowbreak4}
\crossref{Isa}{32}{13}{Isa 6:11; 7:23; 34:13 Ps 107:34 Ho 9:6; 10:8}
\crossref{Isa}{32}{14}{Isa 5:9; 24:1-\allowbreak3,\allowbreak10,\allowbreak12; 25:2; 27:10 2Ki 25:9 Lu 21:20,\allowbreak24}
\crossref{Isa}{32}{15}{Isa 11:2,\allowbreak3; 44:3; 45:8; 59:19-\allowbreak21; 63:11 Ps 104:30; 107:33 Pr 1:23}
\crossref{Isa}{32}{16}{Isa 35:8; 42:4; 56:6-\allowbreak8; 60:21 Ps 94:14,\allowbreak15 Ho 3:5 1Co 6:9-\allowbreak11}
\crossref{Isa}{32}{17}{Isa 26:3; 48:18; 54:13,\allowbreak14; 55:12; 57:19; 66:12 Ps 72:2,\allowbreak3; 85:8}
\crossref{Isa}{32}{18}{Isa 33:20-\allowbreak22; 35:9,\allowbreak10; 60:17,\allowbreak18 Jer 23:5,\allowbreak6; 33:16 Eze 34:25,\allowbreak26}
\crossref{Isa}{32}{19}{Isa 25:4; 28:2,\allowbreak17; 30:30; 37:24 Ex 9:18-\allowbreak26 Eze 13:11-\allowbreak13 Mt 7:25}
\crossref{Isa}{32}{20}{Isa 19:5-\allowbreak7; 30:23; 55:10,\allowbreak11 Ec 11:1 Ac 2:41; 4:4; 5:14 1Co 3:6}
\crossref{Isa}{33}{1}{Isa 10:5,\allowbreak6; 17:14; 24:16 2Ki 18:13-\allowbreak17 2Ch 28:16-\allowbreak21 Hab 2:5-\allowbreak8}
\crossref{Isa}{33}{2}{Isa 25:9; 26:8; 30:18,\allowbreak19 Ps 27:13,\allowbreak14; 62:1,\allowbreak5,\allowbreak8; 123:2; 130:4-\allowbreak8}
\crossref{Isa}{33}{3}{Isa 10:13,\allowbreak14,\allowbreak32-\allowbreak34; 17:12-\allowbreak14; 37:11-\allowbreak18,\allowbreak29-\allowbreak36 Ps 46:6}
\crossref{Isa}{33}{4}{33:23 2Ki 7:15,\allowbreak16 2Ch 14:13; 20:25}
\crossref{Isa}{33}{5}{33:10; 2:11,\allowbreak17; 12:4; 37:20 Ex 9:16,\allowbreak17; 15:1,\allowbreak6; 18:11 Job 40:9-\allowbreak14}
\crossref{Isa}{33}{6}{Isa 11:2-\allowbreak5; 38:5,\allowbreak6 2Ch 32:27-\allowbreak29 Ps 45:4 Pr 14:27; 24:3-\allowbreak7; 28:2,\allowbreak15,\allowbreak16}
\crossref{Isa}{33}{7}{Isa 36:3,\allowbreak22 2Ki 18:18,\allowbreak37; 19:1-\allowbreak3}
\crossref{Isa}{33}{8}{Isa 10:29-\allowbreak31 Jud 5:6 La 1:4}
\crossref{Isa}{33}{9}{Isa 1:7,\allowbreak8; 24:1,\allowbreak4-\allowbreak6,\allowbreak19,\allowbreak20 Jer 4:20-\allowbreak26}
\crossref{Isa}{33}{10}{Isa 10:16,\allowbreak33; 42:13,\allowbreak14; 59:16,\allowbreak17 De 32:36-\allowbreak43 Ps 12:5; 78:65}
\crossref{Isa}{33}{11}{Isa 8:9,\allowbreak10; 10:7-\allowbreak14; 17:13; 29:5-\allowbreak8; 59:4 Job 15:35 Ps 2:1; 7:14}
\crossref{Isa}{33}{12}{Am 2:1}
\crossref{Isa}{33}{13}{Isa 18:3; 37:20; 49:1; 57:19 Ex 15:14 Jos 2:9-\allowbreak11; 9:9,\allowbreak10 1Sa 17:46}
\crossref{Isa}{33}{14}{Isa 7:2; 28:14,\allowbreak15,\allowbreak17-\allowbreak22; 29:13; 30:8-\allowbreak11 Nu 17:12,\allowbreak13 Job 15:21,\allowbreak22}
\crossref{Isa}{33}{15}{Isa 56:1,\allowbreak2 Ps 1:1-\allowbreak3; 15:1,\allowbreak2; 24:4,\allowbreak5; 26:1,\allowbreak2; 106:3 Eze 18:15-\allowbreak17}
\crossref{Isa}{33}{16}{Isa 32:18 Ps 15:1; 90:1; 91:1-\allowbreak10,\allowbreak14; 107:41 Pr 1:33; 18:10 Hab 3:19}
\crossref{Isa}{33}{17}{Isa 32:1,\allowbreak2; 37:1 2Ch 32:23 Ps 45:2 So 5:10 Zec 9:17 Mt 17:2}
\crossref{Isa}{33}{18}{Isa 38:9-\allowbreak22 1Sa 25:33-\allowbreak36; 30:6 Ps 31:7,\allowbreak8,\allowbreak22; 71:20 2Co 1:8-\allowbreak10}
\crossref{Isa}{33}{19}{Ex 14:13 De 28:49,\allowbreak50 2Ki 19:32}
\crossref{Isa}{33}{20}{De 12:5 Ps 78:68,\allowbreak69}
\crossref{Isa}{33}{21}{Ps 29:3 Ac 7:2 2Co 4:4-\allowbreak6}
\crossref{Isa}{33}{22}{Ge 18:25 Ps 50:6; 75:7; 98:9 2Co 5:10}
\crossref{Isa}{33}{23}{33:1,\allowbreak4 2Ch 20:25}
\crossref{Isa}{33}{24}{Isa 58:8 Ex 15:26 De 7:15; 28:27 2Ch 30:20 Jer 33:6-\allowbreak8 Jas 5:14}
\crossref{Isa}{34}{1}{Isa 1:2 De 4:26; 32:1 Jer 22:29 Mic 6:1,\allowbreak2}
\crossref{Isa}{34}{2}{Isa 24:1-\allowbreak23 Jer 25:15-\allowbreak29 Joe 3:9-\allowbreak14 Am 1:1-\allowbreak2:16 Zep 3:8}
\crossref{Isa}{34}{3}{Isa 14:19,\allowbreak20 2Ki 9:35-\allowbreak37 Jer 8:1,\allowbreak2; 22:19 Eze 39:4,\allowbreak11 Joe 2:20}
\crossref{Isa}{34}{4}{Isa 13:10; 14:12 Ps 102:25,\allowbreak26 Jer 4:23,\allowbreak24 Eze 32:7,\allowbreak8 Joe 2:30,\allowbreak31}
\crossref{Isa}{34}{5}{De 32:14,\allowbreak42 Ps 17:13 Jer 46:10; 47:6 Eze 21:3-\allowbreak5,\allowbreak9-\allowbreak11 Zep 2:12}
\crossref{Isa}{34}{6}{Isa 63:3 Jer 49:13 Eze 21:4,\allowbreak5,\allowbreak10}
\crossref{Isa}{34}{7}{Nu 23:22; 24:8 De 33:17 Job 39:9,\allowbreak10 Ps 92:10}
\crossref{Isa}{34}{8}{Isa 26:21; 35:4; 49:26; 59:17,\allowbreak18; 61:2; 63:4 De 32:35,\allowbreak41-\allowbreak43 Ps 94:1}
\crossref{Isa}{34}{9}{Ge 19:28 De 29:23 Job 18:15 Ps 11:6 Lu 17:29 Jude 1:7 Re 19:20}
\crossref{Isa}{34}{10}{Isa 1:31; 66:24 Jer 7:20 Eze 20:47,\allowbreak48 Mr 9:43-\allowbreak48}
\crossref{Isa}{34}{11}{Isa 13:20-\allowbreak22; 14:23 Zep 2:14 Re 18:2,\allowbreak21-\allowbreak23}
\crossref{Isa}{34}{12}{Isa 3:6-\allowbreak8 Ec 10:16,\allowbreak17}
\crossref{Isa}{34}{13}{Isa 32:13,\allowbreak14 Ho 9:6 Zep 2:9}
\crossref{Isa}{34}{14}{Isa 13:21}
\crossref{Isa}{34}{15}{Ps 104:17 Jer 48:28 Eze 31:6}
\crossref{Isa}{34}{16}{Isa 30:8 De 31:21 Jos 1:8 Pr 23:12 Da 10:21 Am 3:7 Mal 3:16}
\crossref{Isa}{34}{17}{Jos 18:8 Ps 78:55 Ac 13:19; 17:26}
\crossref{Isa}{35}{1}{Isa 29:17; 32:15,\allowbreak16; 40:3; 51:3; 52:9,\allowbreak10 Eze 36:35}
\crossref{Isa}{35}{2}{Isa 42:10-\allowbreak12; 49:13; 55:12,\allowbreak13 1Ch 16:33 Ps 65:12,\allowbreak13; 89:12; 96:11-\allowbreak13}
\crossref{Isa}{35}{3}{Isa 40:1,\allowbreak2; 52:1,\allowbreak2; 57:14-\allowbreak16 Jud 7:11 Job 4:3,\allowbreak4; 16:5 Lu 22:32,\allowbreak43}
\crossref{Isa}{35}{4}{Isa 28:16; 32:4}
\crossref{Isa}{35}{5}{Isa 29:18; 32:3,\allowbreak4; 42:6,\allowbreak7,\allowbreak16; 43:8 Ps 146:8 Mt 9:27-\allowbreak30; 11:3-\allowbreak5; 12:22}
\crossref{Isa}{35}{6}{Mt 11:5; 15:30,\allowbreak31; 21:14 Joh 5:8,\allowbreak9 Ac 3:2,\allowbreak6-\allowbreak8; 8:7; 14:8-\allowbreak10}
\crossref{Isa}{35}{7}{Isa 29:17; 44:3,\allowbreak4 Mt 21:43 Lu 13:29 Joh 4:14; 7:38 1Co 6:9-\allowbreak11}
\crossref{Isa}{35}{8}{Isa 11:16; 19:23; 40:3,\allowbreak4; 42:16; 49:11,\allowbreak12; 57:14; 62:10 Jer 31:21}
\crossref{Isa}{35}{9}{Isa 11:6-\allowbreak9; 65:25 Le 26:6 Eze 34:25 Ho 2:18 Re 20:1-\allowbreak3}
\crossref{Isa}{35}{10}{Isa 51:10,\allowbreak11 Mt 20:28 1Ti 2:6}
\crossref{Isa}{36}{1}{2Ki 18:13,\allowbreak17 2Ch 32:1}
\crossref{Isa}{36}{2}{2Ki 18:17-\allowbreak37 2Ch 32:9-\allowbreak23}
\crossref{Isa}{36}{3}{Isa 22:15-\allowbreak20}
\crossref{Isa}{36}{4}{Isa 10:8-\allowbreak14; 37:11-\allowbreak15 Pr 16:18 Eze 31:3-\allowbreak18 Da 4:30 Ac 12:22,\allowbreak23}
\crossref{Isa}{36}{5}{2Ki 18:7; 24:1 Ne 2:19,\allowbreak20 Jer 52:3 Eze 17:15}
\crossref{Isa}{36}{6}{Isa 20:5,\allowbreak6; 30:1-\allowbreak7; 31:3 2Ki 17:4; 18:21 Jer 37:5-\allowbreak8 Eze 29:6,\allowbreak7}
\crossref{Isa}{36}{7}{2Ki 18:5,\allowbreak22 1Ch 5:20 2Ch 16:7-\allowbreak9; 32:7,\allowbreak8 Ps 22:4,\allowbreak5; 42:5,\allowbreak10,\allowbreak11}
\crossref{Isa}{36}{8}{2Ki 14:14}
\crossref{Isa}{36}{9}{Isa 10:8 2Ki 18:24}
\crossref{Isa}{36}{10}{Isa 10:5-\allowbreak7; 37:28 1Ki 13:18 2Ki 18:25 2Ch 35:21 Am 3:6}
\crossref{Isa}{36}{11}{2Ki 18:26,\allowbreak27 Ezr 4:7 Da 2:4}
\crossref{Isa}{36}{12}{Isa 9:20 Le 26:29 De 28:53-\allowbreak57 2Ki 6:25-\allowbreak29; 18:27 Jer 19:9}
\crossref{Isa}{36}{13}{1Sa 17:8-\allowbreak11 2Ki 18:28-\allowbreak32 2Ch 32:18 Ps 17:10-\allowbreak13; 73:8,\allowbreak9; 82:6,\allowbreak7}
\crossref{Isa}{36}{14}{Isa 37:10-\allowbreak13 2Ki 19:10-\allowbreak13,\allowbreak22 2Ch 32:11,\allowbreak13-\allowbreak19 Da 3:15-\allowbreak17; 6:20}
\crossref{Isa}{36}{15}{36:7; 37:23,\allowbreak24 Ps 4:2; 22:7,\allowbreak8; 71:9-\allowbreak11 Mt 27:43}
\crossref{Isa}{36}{16}{1Sa 11:3 2Ki 24:12-\allowbreak16}
\crossref{Isa}{36}{17}{2Ki 17:6-\allowbreak23; 18:9-\allowbreak12; 24:11 Pr 12:10}
\crossref{Isa}{36}{18}{36:7,\allowbreak10,\allowbreak15; 37:10 Ps 12:4; 92:5-\allowbreak7}
\crossref{Isa}{36}{19}{Nu 34:8 2Sa 8:9}
\crossref{Isa}{36}{20}{Isa 37:18,\allowbreak19,\allowbreak23-\allowbreak29; 45:16,\allowbreak17 Ex 5:2 2Ki 19:22-\allowbreak37 2Ch 32:15,\allowbreak19}
\crossref{Isa}{36}{21}{2Ki 18:26,\allowbreak37 Ps 38:13-\allowbreak15; 39:1 Pr 9:7; 26:4 Am 5:13 Mt 7:6}
\crossref{Isa}{36}{22}{36:3,\allowbreak11}
\crossref{Isa}{37}{1}{2Ki 19:1-\allowbreak19}
\crossref{Isa}{37}{2}{37:14; 36:3}
\crossref{Isa}{37}{3}{Isa 25:8; 33:2 2Ki 19:3 2Ch 15:4 Ps 50:15; 91:15; 116:3,\allowbreak4 Jer 30:7}
\crossref{Isa}{37}{4}{Jos 14:12 1Sa 14:6 2Sa 16:12 Am 5:15}
\crossref{Isa}{37}{5}{}
\crossref{Isa}{37}{6}{2Ki 19:5-\allowbreak7; 22:15-\allowbreak20}
\crossref{Isa}{37}{7}{Isa 10:16-\allowbreak18,\allowbreak33,\allowbreak34; 17:13,\allowbreak14; 29:5-\allowbreak8; 30:28-\allowbreak33; 31:8,\allowbreak9; 33:10-\allowbreak12}
\crossref{Isa}{37}{8}{2Ki 19:8,\allowbreak9 Nu 33:20,\allowbreak21}
\crossref{Isa}{37}{9}{1Sa 23:27,\allowbreak28}
\crossref{Isa}{37}{10}{Isa 36:4,\allowbreak15,\allowbreak20 2Ki 18:5; 19:10-\allowbreak13 2Ch 32:7,\allowbreak8,\allowbreak15-\allowbreak19 Ps 22:8}
\crossref{Isa}{37}{11}{37:18,\allowbreak19; 10:7-\allowbreak14; 14:17; 36:18-\allowbreak20 2Ki 17:4-\allowbreak6; 18:33-\allowbreak35}
\crossref{Isa}{37}{12}{Isa 36:20; 46:5-\allowbreak7}
\crossref{Isa}{37}{13}{Isa 10:9; 36:19 Jer 49:23}
\crossref{Isa}{37}{14}{2Ki 19:14}
\crossref{Isa}{37}{15}{1Sa 7:8,\allowbreak9 2Sa 7:18-\allowbreak29 2Ki 19:15-\allowbreak19 2Ch 14:11; 20:6-\allowbreak12}
\crossref{Isa}{37}{16}{Isa 6:3; 8:13 2Sa 7:26 Ps 46:7,\allowbreak11}
\crossref{Isa}{37}{17}{2Ch 6:40 Job 36:7 Ps 17:6; 71:2; 130:1,\allowbreak2 Da 9:17-\allowbreak19 1Pe 3:12}
\crossref{Isa}{37}{18}{2Ki 15:29; 16:9; 17:6,\allowbreak24 1Ch 5:26 Na 2:11,\allowbreak12}
\crossref{Isa}{37}{19}{Isa 10:9-\allowbreak11; 36:18-\allowbreak20; 46:1,\allowbreak2 Ex 32:20 2Sa 5:21}
\crossref{Isa}{37}{20}{Isa 42:8 Ex 9:15,\allowbreak16 Jos 7:8,\allowbreak9 1Sa 17:45-\allowbreak47 1Ki 8:43; 18:36,\allowbreak37}
\crossref{Isa}{37}{21}{Isa 38:3-\allowbreak6; 58:9; 65:24 2Sa 15:31; 17:23 2Ki 19:20,\allowbreak21 Job 22:27}
\crossref{Isa}{37}{22}{Isa 23:12 Jer 14:17 La 1:15; 2:13 Am 5:2}
\crossref{Isa}{37}{23}{37:10-\allowbreak13 Ex 5:2 2Ki 19:4,\allowbreak22 2Ch 32:17 Ps 44:16; 73:9; 74:18,\allowbreak23}
\crossref{Isa}{37}{24}{37:4; 36:15-\allowbreak20 2Ki 19:22,\allowbreak23}
\crossref{Isa}{37}{25}{Isa 36:12 1Ki 20:10 2Ki 19:23,\allowbreak24}
\crossref{Isa}{37}{26}{Isa 10:5,\allowbreak6,\allowbreak15; 45:7; 46:10,\allowbreak11 Ge 50:20 Ps 17:13; 76:10 Am 3:6}
\crossref{Isa}{37}{27}{Isa 19:16 Nu 14:9 2Ki 19:26 Ps 127:1,\allowbreak2 Jer 5:10; 37:10}
\crossref{Isa}{37}{28}{Ps 139:2-\allowbreak11 Pr 5:21; 15:3 Jer 23:23,\allowbreak24 Re 2:13}
\crossref{Isa}{37}{29}{37:10; 36:4,\allowbreak10 2Ki 19:27,\allowbreak28 Job 15:25,\allowbreak26 Ps 2:1-\allowbreak3; 46:6; 93:3,\allowbreak4}
\crossref{Isa}{37}{30}{Isa 7:14; 38:7 Ex 3:12 1Ki 13:3-\allowbreak5 2Ki 19:29; 20:9}
\crossref{Isa}{37}{31}{Isa 27:6; 65:9 2Ki 19:30,\allowbreak31 Ps 80:9 Jer 30:19 Ro 9:27; 11:5 Ga 3:29}
\crossref{Isa}{37}{32}{37:20; 9:7; 59:17 2Ki 19:31 Joe 2:18 Zec 1:14}
\crossref{Isa}{37}{33}{Isa 8:7-\allowbreak10; 10:32-\allowbreak34; 17:12,\allowbreak14; 33:20 2Ki 19:32-\allowbreak35}
\crossref{Isa}{37}{34}{37:29 Pr 21:30}
\crossref{Isa}{37}{35}{Isa 31:5; 38:6 2Ki 20:6}
\crossref{Isa}{37}{36}{Isa 10:12,\allowbreak16-\allowbreak19,\allowbreak33,\allowbreak34; 30:30-\allowbreak33; 31:8; 33:10-\allowbreak12 Ex 12:23 2Sa 24:16}
\crossref{Isa}{37}{37}{37:7,\allowbreak29; 31:9}
\crossref{Isa}{37}{38}{37:10; 14:9,\allowbreak12; 36:15,\allowbreak18 2Ki 19:36,\allowbreak37 2Ch 32:14,\allowbreak19,\allowbreak21}
\crossref{Isa}{38}{1}{2Ki 20:1-\allowbreak11 2Ch 32:24 Joh 11:1-\allowbreak5 Ac 9:37 Php 2:27-\allowbreak30}
\crossref{Isa}{38}{2}{}
\crossref{Isa}{38}{3}{Ne 5:19; 13:14,\allowbreak22,\allowbreak31 Ps 18:20-\allowbreak27; 20:1-\allowbreak3 Heb 6:10}
\crossref{Isa}{38}{4}{}
\crossref{Isa}{38}{5}{2Sa 7:3-\allowbreak5 1Ch 17:2-\allowbreak4}
\crossref{Isa}{38}{6}{Isa 12:6; 31:4; 37:35 2Ch 32:22 2Ti 4:17}
\crossref{Isa}{38}{7}{38:22; 7:11-\allowbreak14; 37:30 Ge 9:13 Jud 6:17-\allowbreak22,\allowbreak37-\allowbreak39 2Ki 20:8-\allowbreak21}
\crossref{Isa}{38}{8}{Jos 10:12-\allowbreak14 2Ki 20:11 2Ch 32:24,\allowbreak31 Mt 16:1}
\crossref{Isa}{38}{9}{Isa 12:1-\allowbreak6 Ex 15:1-\allowbreak21 Jud 5:1-\allowbreak31 1Sa 2:1-\allowbreak10 Ps 18:1}
\crossref{Isa}{38}{10}{38:1 Job 6:11; 7:7; 17:11-\allowbreak16 2Co 1:9}
\crossref{Isa}{38}{11}{Job 35:14,\allowbreak15 Ps 6:4,\allowbreak5; 27:13; 31:22; 116:8,\allowbreak9 Ec 9:5,\allowbreak6}
\crossref{Isa}{38}{12}{Job 7:7 Ps 89:45-\allowbreak47; 102:11,\allowbreak23,\allowbreak24}
\crossref{Isa}{38}{13}{1Ki 13:24-\allowbreak26; 20:36 Job 10:16,\allowbreak17; 16:12-\allowbreak14 Ps 39:10; 50:22; 51:8}
\crossref{Isa}{38}{14}{Job 30:29 Ps 102:4-\allowbreak7}
\crossref{Isa}{38}{15}{Jos 7:8 Ezr 9:10 Ps 39:9,\allowbreak10 Joh 12:27}
\crossref{Isa}{38}{16}{Isa 64:5 De 8:3 Job 33:19-\allowbreak28 Ps 71:20 Mt 4:4 1Co 11:32 2Co 4:17}
\crossref{Isa}{38}{17}{Isa 43:25 Ps 10:2; 85:2 Jer 31:34 Mic 7:18,\allowbreak19}
\crossref{Isa}{38}{18}{Ps 6:5; 30:9; 88:11; 115:17,\allowbreak18 Ec 9:10}
\crossref{Isa}{38}{19}{Ps 146:2 Ec 9:10 Joh 9:4}
\crossref{Isa}{38}{20}{Ps 9:13,\allowbreak14; 27:5,\allowbreak6; 30:11,\allowbreak12; 51:15; 66:13-\allowbreak15; 145:2}
\crossref{Isa}{38}{21}{2Ki 20:7 Mr 7:33 Joh 9:6}
\crossref{Isa}{38}{22}{2Ki 20:8 Ps 42:1,\allowbreak2; 84:1,\allowbreak2,\allowbreak10-\allowbreak12; 118:18,\allowbreak19; 122:1 Joh 5:14}
\crossref{Isa}{39}{1}{2Ki 20:12-\allowbreak19}
\crossref{Isa}{39}{2}{2Ch 32:25,\allowbreak31 Job 31:25 Ps 146:3,\allowbreak4 Pr 4:23 Jer 17:9}
\crossref{Isa}{39}{3}{Isa 38:1,\allowbreak5 2Sa 12:1 2Ki 20:14,\allowbreak15 2Ch 16:7; 19:2; 25:15 Jer 22:1,\allowbreak2}
\crossref{Isa}{39}{4}{Jos 7:19 Job 31:33 Pr 23:5; 28:13 1Jo 1:9}
\crossref{Isa}{39}{5}{1Sa 13:13,\allowbreak14; 15:16}
\crossref{Isa}{39}{6}{2Ki 20:17-\allowbreak19; 24:13; 25:13-\allowbreak15 2Ch 36:10,\allowbreak18 Jer 20:5; 27:21,\allowbreak22}
\crossref{Isa}{39}{7}{2Ki 24:12; 25:6,\allowbreak7 2Ch 33:11; 36:10,\allowbreak20 Jer 39:7 Eze 17:12-\allowbreak20}
\crossref{Isa}{39}{8}{Le 10:3 1Sa 3:18 2Sa 15:26 Job 1:21 Ps 39:9 La 3:22,\allowbreak39}
\crossref{Isa}{40}{1}{Isa 3:10; 35:3,\allowbreak4; 41:10-\allowbreak14,\allowbreak27; 49:13-\allowbreak16; 50:10; 51:3,\allowbreak12; 57:15-\allowbreak19}
\crossref{Isa}{40}{2}{Ge 34:3 2Ch 30:22 Ho 2:14}
\crossref{Isa}{40}{3}{Mt 3:1-\allowbreak3 Mr 1:2-\allowbreak5 Lu 3:2-\allowbreak6 Joh 1:23}
\crossref{Isa}{40}{4}{Isa 42:11,\allowbreak15,\allowbreak16 1Sa 2:8 Ps 113:7,\allowbreak8 Eze 17:24; 21:26 Lu 1:52,\allowbreak53}
\crossref{Isa}{40}{5}{Isa 6:3; 11:9; 35:2; 60:1 Ps 72:19; 96:6; 102:16 Hab 2:14 Lu 2:10-\allowbreak14}
\crossref{Isa}{40}{6}{40:3; 12:6; 58:1; 61:1,\allowbreak2 Jer 2:2; 31:6 Ho 5:8}
\crossref{Isa}{40}{7}{40:8; 15:6; 19:5}
\crossref{Isa}{40}{8}{Isa 46:10,\allowbreak11; 55:10,\allowbreak11 Ps 119:89-\allowbreak91 Zec 1:6 Mt 5:18; 24:35 Mr 13:31}
\crossref{Isa}{40}{9}{Jud 9:7 1Sa 26:13,\allowbreak14 2Ch 13:4}
\crossref{Isa}{40}{10}{Isa 9:6,\allowbreak7; 59:15-\allowbreak21; 60:1-\allowbreak22 Zec 2:8-\allowbreak11 Mal 3:1 Joh 12:13,\allowbreak15}
\crossref{Isa}{40}{11}{Isa 49:9,\allowbreak10; 63:11 Ge 49:24 Ps 23:1-\allowbreak6; 78:71,\allowbreak72; 80:1 Eze 34:12-\allowbreak14}
\crossref{Isa}{40}{12}{Isa 48:13 Job 11:7-\allowbreak9; 38:4-\allowbreak11 Ps 102:25,\allowbreak26; 104:2,\allowbreak3 Pr 8:26-\allowbreak28; 30:4}
\crossref{Isa}{40}{13}{Job 21:22; 36:22,\allowbreak23 Lu 10:22 Joh 1:13 Ro 11:34 1Co 2:16}
\crossref{Isa}{40}{14}{}
\crossref{Isa}{40}{15}{40:22 Job 34:14,\allowbreak15 Jer 10:10}
\crossref{Isa}{40}{16}{Ps 40:6; 50:10-\allowbreak12 Mic 6:6,\allowbreak7 Heb 10:5-\allowbreak10}
\crossref{Isa}{40}{17}{Job 25:6 Ps 62:9 Da 4:34,\allowbreak35 2Co 12:11}
\crossref{Isa}{40}{18}{40:25; 46:5,\allowbreak9 Ex 8:10; 9:14; 15:11; 20:4 De 33:26 1Sa 2:2 Job 40:9}
\crossref{Isa}{40}{19}{Isa 37:18,\allowbreak19; 41:6,\allowbreak7; 44:10-\allowbreak12; 46:6,\allowbreak7 Ex 32:2-\allowbreak4 Jud 17:4 Ps 115:4-\allowbreak8}
\crossref{Isa}{40}{20}{Isa 41:7; 46:7 1Sa 5:3,\allowbreak4}
\crossref{Isa}{40}{21}{Isa 27:11; 44:20; 46:8 Ps 19:1-\allowbreak5; 115:8 Jer 10:8-\allowbreak12 Ac 14:17}
\crossref{Isa}{40}{22}{Isa 19:1; 66:1 Ps 2:4; 29:10; 68:33}
\crossref{Isa}{40}{23}{Isa 19:13,\allowbreak14; 23:9; 24:21,\allowbreak22 Job 12:21; 34:19,\allowbreak20 Ps 76:12; 107:40}
\crossref{Isa}{40}{24}{Isa 14:21,\allowbreak22; 17:11 1Ki 21:21,\allowbreak22 2Ki 10:11 Job 15:30-\allowbreak33; 18:16-\allowbreak19}
\crossref{Isa}{40}{25}{40:18 De 4:15-\allowbreak18,\allowbreak33; 5:8}
\crossref{Isa}{40}{26}{Isa 51:6 De 4:19 Job 31:26-\allowbreak28 Ps 8:3,\allowbreak4; 19:1}
\crossref{Isa}{40}{27}{Isa 49:4 Job 27:2; 34:5 Mal 2:17 Lu 18:7,\allowbreak8}
\crossref{Isa}{40}{28}{Jer 4:22 Mr 8:17,\allowbreak18; 9:19; 16:14 Lu 24:25 Joh 14:9 1Co 6:3-\allowbreak5,\allowbreak9}
\crossref{Isa}{40}{29}{Isa 41:10 Ge 49:24 De 33:25 Ps 29:11 Zec 10:12 2Co 12:9,\allowbreak10}
\crossref{Isa}{40}{30}{Isa 9:17; 13:18 Ps 33:16; 34:10; 39:5 Ec 9:11 Am 2:14}
\crossref{Isa}{40}{31}{Isa 8:17; 25:9; 30:18 Ps 25:3,\allowbreak5,\allowbreak21; 27:14; 37:34; 40:1; 84:7; 92:1,\allowbreak13}
\crossref{Isa}{41}{1}{Isa 49:1 Ps 46:10 Hab 2:20 Zec 2:13}
\crossref{Isa}{41}{2}{41:25; 45:13; 46:11 Ge 11:31; 12:1-\allowbreak3; 17:1 Heb 11:8-\allowbreak10}
\crossref{Isa}{41}{3}{Isa 57:2 Job 5:24}
\crossref{Isa}{41}{4}{41:26; 40:12,\allowbreak26; 42:24}
\crossref{Isa}{41}{5}{Ge 10:5 Eze 26:15,\allowbreak16}
\crossref{Isa}{41}{6}{Isa 40:19; 44:12 1Sa 4:7-\allowbreak9; 5:3-\allowbreak5 Da 3:1-\allowbreak7 Ac 19:24-\allowbreak28}
\crossref{Isa}{41}{7}{Isa 40:19; 44:12-\allowbreak15; 46:6,\allowbreak7 Jer 10:3-\allowbreak5,\allowbreak9 Da 3:1-\allowbreak7}
\crossref{Isa}{41}{8}{Isa 43:1; 44:1,\allowbreak2,\allowbreak21; 48:12; 49:3 Ex 19:5,\allowbreak6 Le 25:42 De 7:6-\allowbreak8; 10:15}
\crossref{Isa}{41}{9}{41:2 Jos 24:2-\allowbreak4 Ne 9:7-\allowbreak38 Ps 107:2,\allowbreak3 Lu 13:29 Re 5:9}
\crossref{Isa}{41}{10}{41:13,\allowbreak14; 12:2; 43:1,\allowbreak5; 44:2; 51:12,\allowbreak13 Ge 15:1 De 20:1; 31:6-\allowbreak8}
\crossref{Isa}{41}{11}{Isa 45:24; 49:26; 54:17; 60:12-\allowbreak14 Ex 11:8; 23:22 Zec 12:3 Ac 13:8-\allowbreak11}
\crossref{Isa}{41}{12}{Job 20:7-\allowbreak9 Ps 37:35,\allowbreak36}
\crossref{Isa}{41}{13}{Isa 43:6; 45:1; 51:18 De 33:26-\allowbreak29 Ps 63:8; 73:23; 109:31 2Ti 4:17}
\crossref{Isa}{41}{14}{Job 25:6 Ps 22:6}
\crossref{Isa}{41}{15}{Isa 21:10; 28:27 Hab 3:12}
\crossref{Isa}{41}{16}{Isa 17:13 Ps 1:4 Jer 15:7; 51:2 Mt 3:12}
\crossref{Isa}{41}{17}{Isa 61:1; 66:2 Ps 68:9,\allowbreak10; 72:12,\allowbreak13; 102:16,\allowbreak17 Mt 5:3}
\crossref{Isa}{41}{18}{Isa 12:3; 30:25; 32:2; 35:6,\allowbreak7; 43:19,\allowbreak20; 44:3; 48:21; 49:9,\allowbreak10; 58:11}
\crossref{Isa}{41}{19}{Isa 27:6; 32:15; 37:31,\allowbreak32; 51:3; 55:13; 60:21; 61:3,\allowbreak11 Ps 92:13,\allowbreak14}
\crossref{Isa}{41}{20}{Isa 43:7-\allowbreak13,\allowbreak21; 44:23; 45:6-\allowbreak8; 66:18 Ex 9:16 Nu 23:23 Job 12:9}
\crossref{Isa}{41}{21}{Job 23:3,\allowbreak4; 31:37; 38:3; 40:7-\allowbreak9 Mic 6:1,\allowbreak2}
\crossref{Isa}{41}{22}{Isa 42:9; 43:9-\allowbreak12; 45:21; 48:14 Joh 13:19; 16:14}
\crossref{Isa}{41}{23}{Isa 42:9; 44:7,\allowbreak8; 45:8; 46:9,\allowbreak10 Joh 13:19 Ac 15:18}
\crossref{Isa}{41}{24}{41:29; 44:9,\allowbreak10 Ps 115:8 Jer 10:8,\allowbreak14; 51:17,\allowbreak18 1Co 8:4}
\crossref{Isa}{41}{25}{Isa 21:2; 44:28; 45:1-\allowbreak6,\allowbreak13; 46:10,\allowbreak11 Jer 51:27-\allowbreak29}
\crossref{Isa}{41}{26}{41:22; 43:9; 44:7; 45:21 Hab 2:18-\allowbreak20}
\crossref{Isa}{41}{27}{41:4; 43:10; 44:6; 48:12 Re 2:8}
\crossref{Isa}{41}{28}{Isa 63:5 Da 2:10,\allowbreak11; 4:7,\allowbreak8; 5:8}
\crossref{Isa}{41}{29}{41:24; 44:9-\allowbreak20 Ps 115:4-\allowbreak8; 135:15-\allowbreak18 Jer 10:2-\allowbreak16 Hab 2:18}
\crossref{Isa}{42}{1}{Isa 43:10; 49:3-\allowbreak6; 52:13; 53:11 Mt 12:18-\allowbreak20 Php 2:7}
\crossref{Isa}{42}{2}{Zec 9:9 Mt 11:29; 12:16-\allowbreak20 Lu 17:20 2Ti 2:24 1Pe 2:23}
\crossref{Isa}{42}{3}{Isa 35:3,\allowbreak4; 40:11,\allowbreak29-\allowbreak31; 50:4,\allowbreak10; 57:15-\allowbreak18; 61:1-\allowbreak3; 66:2 Ps 103:13,\allowbreak14}
\crossref{Isa}{42}{4}{Isa 9:7; 49:5-\allowbreak10; 52:13-\allowbreak15; 53:2-\allowbreak12 Joh 17:4,\allowbreak5 Heb 12:2-\allowbreak4}
\crossref{Isa}{42}{5}{Isa 40:12,\allowbreak22,\allowbreak28; 44:24; 45:12,\allowbreak18; 48:13 Ps 102:25,\allowbreak26; 104:2-\allowbreak35}
\crossref{Isa}{42}{6}{Isa 32:1; 43:1; 45:13; 49:1-\allowbreak3 Ps 45:6,\allowbreak7 Jer 23:5,\allowbreak6; 33:15,\allowbreak16}
\crossref{Isa}{42}{7}{42:16; 29:18; 35:5 Ps 146:8 Mt 11:5 Lu 24:45 Joh 9:39 Ac 26:18}
\crossref{Isa}{42}{8}{Ex 3:13-\allowbreak15; 4:5 Ps 83:18 Joh 8:58}
\crossref{Isa}{42}{9}{Ge 15:12-\allowbreak16 Jos 21:45; 23:14,\allowbreak15 1Ki 8:15-\allowbreak20; 11:36}
\crossref{Isa}{42}{10}{Isa 24:14-\allowbreak16; 44:23; 49:13; 65:14 Ps 33:3; 40:3; 96:1-\allowbreak3; 98:1-\allowbreak4; 117:1,\allowbreak2}
\crossref{Isa}{42}{11}{Isa 32:16; 35:1,\allowbreak6; 40:3; 41:18,\allowbreak19; 43:19 Ps 72:8-\allowbreak10}
\crossref{Isa}{42}{12}{Isa 24:15,\allowbreak16; 66:18,\allowbreak19 Ps 22:27; 96:3-\allowbreak10; 117:1,\allowbreak2 Ro 15:9-\allowbreak11}
\crossref{Isa}{42}{13}{Isa 59:16-\allowbreak19; 63:1-\allowbreak4 Ex 15:1-\allowbreak3 Ps 78:65; 110:5,\allowbreak6 Jer 25:30}
\crossref{Isa}{42}{14}{Job 32:18,\allowbreak20 Ps 50:2; 83:1,\allowbreak2 Ec 8:11,\allowbreak12 Jer 15:6; 44:22 Lu 18:7}
\crossref{Isa}{42}{15}{Isa 2:12-\allowbreak16; 11:15,\allowbreak16; 44:27; 49:11; 50:2 Ps 18:7; 107:33,\allowbreak34; 114:3-\allowbreak7}
\crossref{Isa}{42}{16}{Isa 29:18,\allowbreak24; 30:21; 32:3; 35:5,\allowbreak8; 48:17; 54:13; 60:1,\allowbreak2,\allowbreak19,\allowbreak20}
\crossref{Isa}{42}{17}{Isa 1:29; 44:11; 45:16,\allowbreak17 Ps 97:7 Jer 2:26,\allowbreak27 Hab 2:18-\allowbreak20}
\crossref{Isa}{42}{18}{Isa 29:18; 43:8 Ex 4:11 Pr 20:12 Mr 7:34-\allowbreak37 Lu 7:22 Re 3:17,\allowbreak18}
\crossref{Isa}{42}{19}{Isa 6:9; 29:9-\allowbreak14; 56:10 Jer 4:22; 5:21 Eze 12:2 Mt 13:14,\allowbreak15; 15:14-\allowbreak16}
\crossref{Isa}{42}{20}{Isa 1:3; 48:6-\allowbreak8 Nu 14:22 De 4:9; 29:2-\allowbreak4 Ne 9:10-\allowbreak17 Ps 106:7-\allowbreak13}
\crossref{Isa}{42}{21}{Isa 1:24-\allowbreak27; 46:12,\allowbreak13 Ps 71:16,\allowbreak19; 85:9-\allowbreak12 Da 9:24-\allowbreak27 Mt 3:17; 5:17}
\crossref{Isa}{42}{22}{Isa 1:7; 18:2; 36:1; 52:4,\allowbreak5; 56:9 Jer 50:17; 51:34,\allowbreak35; 52:4-\allowbreak11}
\crossref{Isa}{42}{23}{Isa 1:18-\allowbreak20; 48:18 Le 26:40-\allowbreak42 De 4:29-\allowbreak31; 32:29 Pr 1:22,\allowbreak23}
\crossref{Isa}{42}{24}{Isa 10:5,\allowbreak6; 45:7; 47:6; 50:1,\allowbreak2; 59:1,\allowbreak2; 63:10 De 28:49; 32:30 Jud 2:14}
\crossref{Isa}{42}{25}{Le 26:15-\allowbreak46 De 32:22 Ps 79:5,\allowbreak6 Eze 7:8,\allowbreak9; 20:34; 22:21,\allowbreak22}
\crossref{Isa}{43}{1}{43:7,\allowbreak15,\allowbreak21; 44:2,\allowbreak21 Ps 100:3; 102:18 Jer 31:3; 33:24,\allowbreak26 Eph 2:10}
\crossref{Isa}{43}{2}{Isa 8:7-\allowbreak10; 11:15,\allowbreak16 Ex 14:29 Jos 3:15-\allowbreak17 Ps 66:10,\allowbreak12; 91:3-\allowbreak5}
\crossref{Isa}{43}{3}{Isa 30:11; 41:14; 45:15,\allowbreak21; 49:26; 60:16 Ho 13:4 Tit 2:10-\allowbreak14}
\crossref{Isa}{43}{4}{Ex 19:5,\allowbreak6 De 7:6-\allowbreak8; 14:2; 26:18; 32:9-\allowbreak14 Ps 135:4 Mal 3:17}
\crossref{Isa}{43}{5}{43:2; 41:10,\allowbreak14; 44:2 Jer 30:10,\allowbreak11; 46:27,\allowbreak28 Ac 18:9,\allowbreak10}
\crossref{Isa}{43}{6}{Isa 18:7 Jer 3:14,\allowbreak18,\allowbreak19 Ho 1:10,\allowbreak11 Ro 9:7,\allowbreak8,\allowbreak25,\allowbreak26 2Co 6:17,\allowbreak18}
\crossref{Isa}{43}{7}{Isa 62:2-\allowbreak5; 63:19 Jer 33:16 Ac 11:26 Jas 2:7 Re 3:12}
\crossref{Isa}{43}{8}{Isa 6:9; 42:18-\allowbreak20; 44:18-\allowbreak20 De 29:2-\allowbreak4 Jer 5:21 Eze 12:2 2Co 4:4-\allowbreak6}
\crossref{Isa}{43}{9}{Isa 45:20,\allowbreak21; 48:14 Ps 49:1,\allowbreak2; 50:1 Joe 3:11}
\crossref{Isa}{43}{10}{43:12; 44:8 Joh 1:7,\allowbreak8; 15:27 Ac 1:8 1Co 15:15}
\crossref{Isa}{43}{11}{43:3; 12:2; 45:21,\allowbreak22 De 6:4 Ho 1:7; 13:4 Lu 1:47; 2:11 Joh 10:28-\allowbreak30}
\crossref{Isa}{43}{12}{Isa 37:7,\allowbreak35,\allowbreak36; 46:10; 48:4-\allowbreak7}
\crossref{Isa}{43}{13}{Isa 57:15 Ps 90:2; 93:2 Pr 8:23 Mic 5:2 Hab 1:12 Joh 1:1,\allowbreak2; 8:58}
\crossref{Isa}{43}{14}{43:1; 44:6; 54:5-\allowbreak8 Ps 19:14 Re 5:9}
\crossref{Isa}{43}{15}{43:3; 30:11; 40:25; 41:14,\allowbreak16; 45:11; 48:17 Jer 51:5 Hab 1:12 Re 3:7}
\crossref{Isa}{43}{16}{43:2; 11:15,\allowbreak16; 51:10,\allowbreak15; 63:11-\allowbreak13 Ex 14:16,\allowbreak21,\allowbreak29 Jos 3:13-\allowbreak16}
\crossref{Isa}{43}{17}{Ex 14:4-\allowbreak9,\allowbreak23-\allowbreak28; 15:4 Ps 46:8,\allowbreak9; 76:5,\allowbreak6 Eze 38:8-\allowbreak18}
\crossref{Isa}{43}{18}{Isa 46:9; 65:17 De 7:18; 8:2 1Ch 16:12 Jer 16:14,\allowbreak15; 23:7,\allowbreak8 2Co 3:10}
\crossref{Isa}{43}{19}{Isa 42:9; 48:6 Jer 31:22 Re 21:5}
\crossref{Isa}{43}{20}{Isa 11:6-\allowbreak10 Ps 104:21; 148:10}
\crossref{Isa}{43}{21}{Isa 50:7; 60:21; 61:3 Ps 4:3; 102:18 Pr 16:4 Lu 1:74,\allowbreak75 1Co 6:19,\allowbreak20}
\crossref{Isa}{43}{22}{Isa 64:7 Ps 14:4; 79:6 Jer 10:25 Da 9:13 Ho 7:10-\allowbreak14; 14:1,\allowbreak2}
\crossref{Isa}{43}{23}{Am 5:25 Mal 1:13,\allowbreak14; 3:8}
\crossref{Isa}{43}{24}{Ex 30:7,\allowbreak23,\allowbreak24,\allowbreak34 Jer 6:20}
\crossref{Isa}{43}{25}{43:11; 1:18; 44:22 Ps 51:9 Jer 50:20 Mic 7:18,\allowbreak19 Mr 2:7 Ac 3:19}
\crossref{Isa}{43}{26}{Isa 1:18 Ge 32:12 Job 16:21; 23:3-\allowbreak6; 40:4,\allowbreak5 Ps 141:2 Jer 2:21-\allowbreak35}
\crossref{Isa}{43}{27}{Nu 32:14 Ps 78:8; 106:6,\allowbreak7 Jer 3:25 Eze 16:3 Zec 1:4-\allowbreak6 Mal 3:7}
\crossref{Isa}{43}{28}{Isa 47:6 2Sa 1:21 Ps 89:39 La 2:2,\allowbreak6,\allowbreak7; 4:20}
\crossref{Isa}{44}{1}{Isa 42:23; 48:16-\allowbreak18; 55:3 Ps 81:11-\allowbreak13 Jer 4:7 Lu 13:34 Heb 3:7,\allowbreak8}
\crossref{Isa}{44}{2}{44:21; 43:1,\allowbreak7,\allowbreak21}
\crossref{Isa}{44}{3}{Isa 41:17; 59:21 Eze 34:26 Joe 3:18 Joh 7:37-\allowbreak39 Re 21:6; 22:17}
\crossref{Isa}{44}{4}{Isa 58:11; 61:11 Ps 1:3; 92:13-\allowbreak15 Ac 2:41-\allowbreak47; 4:4; 5:14}
\crossref{Isa}{44}{5}{De 26:17-\allowbreak19 Ps 116:16 Jer 50:5 Mic 4:2 Zec 8:20-\allowbreak23; 13:9}
\crossref{Isa}{44}{6}{Isa 33:22; 43:15 Mal 1:14 Mt 25:34; 27:37}
\crossref{Isa}{44}{7}{Isa 41:22,\allowbreak26; 43:9,\allowbreak12; 45:21; 46:9,\allowbreak10; 48:3-\allowbreak8}
\crossref{Isa}{44}{8}{44:2; 41:10-\allowbreak14 Pr 3:25,\allowbreak26 Jer 10:7; 30:10,\allowbreak11 Joh 6:10}
\crossref{Isa}{44}{9}{Isa 41:24,\allowbreak29 De 27:15 Ps 97:7 Jer 10:3-\allowbreak8,\allowbreak14,\allowbreak15}
\crossref{Isa}{44}{10}{1Ki 12:28 Jer 10:5 Da 3:1,\allowbreak14 Hab 2:18 Ac 19:26 1Co 8:4}
\crossref{Isa}{44}{11}{Isa 1:29; 42:17; 45:16 1Sa 5:3-\allowbreak7; 6:4,\allowbreak5 Ps 97:7 Jer 2:26,\allowbreak27; 10:14}
\crossref{Isa}{44}{12}{Isa 40:19; 41:6,\allowbreak7; 46:6,\allowbreak7 Ex 32:4,\allowbreak8 Jer 10:3-\allowbreak11}
\crossref{Isa}{44}{13}{Ex 20:4,\allowbreak5 De 4:16-\allowbreak18,\allowbreak28 Ac 17:29 Ro 1:23}
\crossref{Isa}{44}{14}{Isa 40:20 Jer 10:3-\allowbreak8 Ho 4:12 Hab 2:19}
\crossref{Isa}{44}{15}{44:10; 45:20 Jud 2:19 2Ch 25:14 Re 9:20}
\crossref{Isa}{44}{16}{44:19 Ex 24:6 Jer 10:3-\allowbreak5}
\crossref{Isa}{44}{17}{Isa 36:19,\allowbreak20; 37:38 Da 3:17,\allowbreak29; 6:16,\allowbreak20-\allowbreak22,\allowbreak27}
\crossref{Isa}{44}{18}{44:9,\allowbreak20; 45:20; 46:7,\allowbreak8 Jer 10:8,\allowbreak14 Ro 1:21-\allowbreak23}
\crossref{Isa}{44}{19}{Isa 46:8 Ex 7:23 De 32:46 Eze 40:4 Hag 1:5}
\crossref{Isa}{44}{20}{Job 15:2 Ps 102:9 Pr 15:14 Ho 12:1 Lu 15:16}
\crossref{Isa}{44}{21}{Isa 42:23; 46:8,\allowbreak9 De 4:9,\allowbreak23; 31:19-\allowbreak21; 32:18}
\crossref{Isa}{44}{22}{Isa 1:18; 43:25 Ne 4:5 Ps 51:1,\allowbreak9; 103:12; 109:14 Jer 18:23; 33:8}
\crossref{Isa}{44}{23}{Isa 42:10-\allowbreak12; 49:13; 55:12,\allowbreak13 Ps 69:34; 96:11,\allowbreak12; 98:7,\allowbreak8 Jer 51:48}
\crossref{Isa}{44}{24}{44:6; 43:14; 48:17; 49:7,\allowbreak26; 54:5,\allowbreak8; 59:20; 60:16; 63:16 Ps 78:35}
\crossref{Isa}{44}{25}{Isa 47:12-\allowbreak14 1Ki 22:11,\allowbreak12,\allowbreak22-\allowbreak25,\allowbreak37 2Ch 18:11,\allowbreak34 Jer 27:9,\allowbreak10}
\crossref{Isa}{44}{26}{Isa 42:9 Ex 11:4-\allowbreak6; 12:29,\allowbreak30 1Ki 13:3-\allowbreak5; 18:36-\allowbreak38 Eze 38:17}
\crossref{Isa}{44}{27}{Isa 11:15,\allowbreak16; 42:15; 43:16; 51:15 Ps 74:15 Jer 50:38; 51:32,\allowbreak36}
\crossref{Isa}{44}{28}{Isa 42:15; 45:1,\allowbreak3; 46:11; 48:14,\allowbreak15 Da 10:1}
\crossref{Isa}{45}{1}{Isa 13:3; 44:28 1Ki 19:15 Jer 27:6}
\crossref{Isa}{45}{2}{Isa 13:4-\allowbreak17}
\crossref{Isa}{45}{3}{Jer 27:5-\allowbreak7; 50:37; 51:53 Eze 29:19,\allowbreak20}
\crossref{Isa}{45}{4}{Isa 41:8,\allowbreak9; 43:3,\allowbreak4,\allowbreak14; 44:1 Ex 19:5,\allowbreak6 Jer 50:17-\allowbreak20 Mt 24:22}
\crossref{Isa}{45}{5}{45:14-\allowbreak18,\allowbreak21,\allowbreak22; 44:8; 46:9 De 4:35,\allowbreak39; 32:39 1Ki 8:60 Joe 2:27}
\crossref{Isa}{45}{6}{Isa 37:20 1Sa 17:46,\allowbreak47 Ps 46:10; 83:18; 102:15,\allowbreak16 Eze 38:23; 39:21}
\crossref{Isa}{45}{7}{Ge 1:3-\allowbreak5,\allowbreak17,\allowbreak18 Ps 8:3; 104:20-\allowbreak23 Jer 31:35 2Co 4:6 Jas 1:17}
\crossref{Isa}{45}{8}{Isa 32:15; 44:3 Ps 72:3,\allowbreak6; 85:9-\allowbreak12 Eze 34:26 Ho 10:12; 14:5-\allowbreak8}
\crossref{Isa}{45}{9}{Isa 64:8 Ex 9:16,\allowbreak17 Job 15:24-\allowbreak26; 40:8,\allowbreak9 Ps 2:2-\allowbreak9 Pr 21:30}
\crossref{Isa}{45}{10}{De 27:16 Mal 1:6 Heb 12:9}
\crossref{Isa}{45}{11}{Isa 43:3,\allowbreak7,\allowbreak15,\allowbreak21; 48:17}
\crossref{Isa}{45}{12}{45:18; 40:28; 42:5 Ge 1:26,\allowbreak27 Ps 102:25 Heb 11:3}
\crossref{Isa}{45}{13}{45:1-\allowbreak6; 41:2,\allowbreak25; 46:11; 48:14,\allowbreak15}
\crossref{Isa}{45}{14}{Isa 18:7; 19:23-\allowbreak25; 23:18; 49:23; 60:5-\allowbreak16; 61:5,\allowbreak6; 66:19,\allowbreak20 Ps 68:30,\allowbreak31}
\crossref{Isa}{45}{15}{Isa 8:17; 57:17 Ps 44:24; 77:19 Joh 13:7 Ro 11:33,\allowbreak34}
\crossref{Isa}{45}{16}{45:20; 41:19; 42:17; 44:9,\allowbreak11 Ps 97:7 Jer 2:26,\allowbreak27; 10:14,\allowbreak15}
\crossref{Isa}{45}{17}{45:25; 26:4 Ho 1:7 Ro 2:28,\allowbreak29; 8:1; 11:26 1Co 1:30,\allowbreak31 2Co 5:17-\allowbreak21}
\crossref{Isa}{45}{18}{Isa 42:5 Jer 10:12; 51:15}
\crossref{Isa}{45}{19}{Isa 43:9,\allowbreak10; 48:16 De 29:29; 30:11-\allowbreak14 Pr 1:21; 8:1-\allowbreak4 Joh 7:26,\allowbreak28}
\crossref{Isa}{45}{20}{Isa 41:5,\allowbreak6,\allowbreak21; 43:9}
\crossref{Isa}{45}{21}{Ps 26:7; 71:17,\allowbreak18; 96:10 Jer 50:2 Joe 3:9-\allowbreak12}
\crossref{Isa}{45}{22}{Nu 21:8,\allowbreak9 2Ch 20:12 Ps 22:17; 65:5 Mic 7:7 Zec 12:10}
\crossref{Isa}{45}{23}{Ge 22:15-\allowbreak18 Jer 22:5; 49:13 Am 6:8 Heb 6:13-\allowbreak18}
\crossref{Isa}{45}{24}{2Co 12:9,\allowbreak10 Eph 3:16 Php 4:13 Col 1:11 2Ti 4:17,\allowbreak18}
\crossref{Isa}{45}{25}{45:17,\allowbreak24 Ac 13:39 Ro 3:24,\allowbreak25; 5:1,\allowbreak18,\allowbreak19; 8:1,\allowbreak30,\allowbreak33,\allowbreak34 1Co 6:11}
\crossref{Isa}{46}{1}{Isa 2:20 Jer 10:5}
\crossref{Isa}{46}{2}{Isa 36:18,\allowbreak19; 37:12,\allowbreak19; 44:17; 45:20}
\crossref{Isa}{46}{3}{Isa 44:1,\allowbreak21; 48:1,\allowbreak17,\allowbreak18; 51:1,\allowbreak7 Ps 81:8-\allowbreak13}
\crossref{Isa}{46}{4}{Isa 41:4; 43:13,\allowbreak25 Ps 92:14; 102:26,\allowbreak27 Mal 2:16; 3:6 Ro 11:29}
\crossref{Isa}{46}{5}{Isa 40:18,\allowbreak25 Ex 15:11 Ps 86:8; 89:6,\allowbreak8; 113:5 Jer 10:6,\allowbreak7,\allowbreak16 Php 2:6}
\crossref{Isa}{46}{6}{Isa 40:19,\allowbreak20; 41:6,\allowbreak7; 44:12-\allowbreak19; 45:20 Ex 32:2-\allowbreak4 Jud 17:3,\allowbreak4}
\crossref{Isa}{46}{7}{1Sa 5:3 Jer 10:5 Da 3:1}
\crossref{Isa}{46}{8}{Isa 44:18-\allowbreak21 De 32:29 Ps 115:8; 135:18 Jer 10:8 1Co 14:20}
\crossref{Isa}{46}{9}{Isa 42:9; 65:17 De 32:7 Ne 9:7-\allowbreak37 Ps 78:1-\allowbreak72; 105:1-\allowbreak106:48; 111:4}
\crossref{Isa}{46}{10}{Isa 41:22,\allowbreak23; 44:7; 45:21 Ge 3:15; 12:2,\allowbreak3; 49:10,\allowbreak22-\allowbreak26 Nu 24:17-\allowbreak24}
\crossref{Isa}{46}{11}{Isa 13:2-\allowbreak4; 21:7-\allowbreak9; 41:2,\allowbreak25; 45:1-\allowbreak6 Jer 50:29; 51:20-\allowbreak29}
\crossref{Isa}{46}{12}{46:3; 28:23; 45:20 Ps 49:1 Pr 1:22,\allowbreak23; 8:1-\allowbreak5 Eph 5:14 Re 3:17,\allowbreak18}
\crossref{Isa}{46}{13}{Isa 51:5; 61:11 Ro 1:17; 3:21-\allowbreak26; 10:3-\allowbreak15}
\crossref{Isa}{47}{1}{Isa 3:26; 26:5; 52:2 Job 2:8,\allowbreak13 Ps 18:27 Jer 13:18; 48:18}
\crossref{Isa}{47}{2}{Ex 11:5 Jud 16:21 Job 31:10 Jer 27:7 La 5:13 Mt 24:41}
\crossref{Isa}{47}{3}{Isa 34:1-\allowbreak8; 59:17,\allowbreak18; 63:4-\allowbreak6 De 32:35,\allowbreak41-\allowbreak43 Ps 94:1,\allowbreak2; 137:8,\allowbreak9}
\crossref{Isa}{47}{4}{Isa 41:14; 43:3,\allowbreak14; 44:6; 49:26; 54:5 Jer 31:11; 50:33,\allowbreak34}
\crossref{Isa}{47}{5}{Isa 13:20; 14:23 1Sa 2:9 Ps 31:17; 46:10 Jer 25:10 La 1:1 Hab 2:20}
\crossref{Isa}{47}{6}{Isa 10:6; 42:24,\allowbreak25 2Sa 24:14 2Ch 28:9 Ps 69:26 Zec 1:15}
\crossref{Isa}{47}{7}{47:5 Eze 28:2,\allowbreak12-\allowbreak14; 29:3 Da 4:29; 5:18-\allowbreak23}
\crossref{Isa}{47}{8}{Isa 21:4,\allowbreak5; 22:12,\allowbreak13; 32:9 Jud 18:7,\allowbreak27 Jer 50:11 Da 5:1-\allowbreak4,\allowbreak30}
\crossref{Isa}{47}{9}{Isa 51:18,\allowbreak19 Ru 1:5,\allowbreak20 Lu 7:12,\allowbreak13}
\crossref{Isa}{47}{10}{Isa 28:15; 59:4 Ps 52:7; 62:9}
\crossref{Isa}{47}{11}{Isa 37:36 Ex 12:29,\allowbreak30 Ne 4:11 Re 3:3}
\crossref{Isa}{47}{12}{47:9,\allowbreak10; 8:19; 19:3; 44:25 Ex 7:11; 8:7,\allowbreak18,\allowbreak19; 9:11 Jer 2:28 Da 5:7-\allowbreak9}
\crossref{Isa}{47}{13}{Isa 57:10 Eze 24:12 Hab 2:13}
\crossref{Isa}{47}{14}{Isa 40:24; 41:2 Eze 15:7 Ps 83:13-\allowbreak15 Joe 2:5 Ob 1:18 Na 1:10 Mal 4:1}
\crossref{Isa}{47}{15}{Isa 56:11 Eze 27:12-\allowbreak25 Re 18:11-\allowbreak19}
\crossref{Isa}{48}{1}{Ge 32:28; 35:10 2Ki 17:34 Joh 1:47 Ro 2:17,\allowbreak28,\allowbreak29; 9:6,\allowbreak8 Re 2:9}
\crossref{Isa}{48}{2}{Isa 52:1; 64:10,\allowbreak11 Ne 11:1,\allowbreak18 Ps 48:1; 87:3 Da 9:24 Mt 4:5; 27:53}
\crossref{Isa}{48}{3}{Isa 41:22; 42:9; 43:9; 44:7,\allowbreak8; 45:21; 46:9,\allowbreak10}
\crossref{Isa}{48}{4}{Isa 46:12 Ps 78:8 Zec 7:11,\allowbreak12}
\crossref{Isa}{48}{5}{48:3; 44:7; 46:10 Lu 1:70 Ac 15:18}
\crossref{Isa}{48}{6}{Ps 107:43 Jer 2:31 Mic 6:9}
\crossref{Isa}{48}{7}{Ex 34:10 Ps 148:5 Eze 21:30}
\crossref{Isa}{48}{8}{Isa 6:9,\allowbreak10; 26:11; 29:10,\allowbreak11; 42:19,\allowbreak20 Jer 5:21 Mt 13:13-\allowbreak15}
\crossref{Isa}{48}{9}{48:11; 37:35; 43:25 Jos 7:9 1Sa 12:22 Ps 25:11; 79:9; 106:8; 143:11}
\crossref{Isa}{48}{10}{Isa 1:25,\allowbreak26 Job 23:10 Ps 66:10 Pr 17:3 Jer 9:7 Eze 20:38}
\crossref{Isa}{48}{11}{48:9}
\crossref{Isa}{48}{12}{Isa 34:1; 46:3; 49:1; 51:1,\allowbreak4,\allowbreak7; 55:3 Pr 7:24; 8:32}
\crossref{Isa}{48}{13}{Isa 42:5; 45:18 Ex 20:11 Ps 102:25 Heb 1:10-\allowbreak12}
\crossref{Isa}{48}{14}{Isa 41:22; 43:9; 44:7; 45:20,\allowbreak21}
\crossref{Isa}{48}{15}{Jos 1:8 Ps 45:4 Eze 1:2}
\crossref{Isa}{48}{16}{48:3-\allowbreak6; 45:19 Joh 18:20}
\crossref{Isa}{48}{17}{48:20; 43:14; 44:6-\allowbreak24; 54:5}
\crossref{Isa}{48}{18}{De 5:29; 32:29 Ps 81:13-\allowbreak16 Mt 23:37 Lu 19:41,\allowbreak42}
\crossref{Isa}{48}{19}{Isa 10:22 Ge 13:16; 22:17 Jer 33:22 Ho 1:10 Ro 9:27}
\crossref{Isa}{48}{20}{Isa 52:11 Jer 50:8; 51:6,\allowbreak45 Zec 2:6,\allowbreak7 Re 18:4}
\crossref{Isa}{48}{21}{Isa 30:25; 35:6,\allowbreak7; 41:17,\allowbreak18; 43:19,\allowbreak20; 49:10 Jer 31:9}
\crossref{Isa}{48}{22}{Isa 57:21 Job 15:20-\allowbreak24 Lu 19:42 Ro 3:17}
\crossref{Isa}{49}{1}{Isa 41:1; 42:1-\allowbreak4,\allowbreak12; 45:22; 51:5; 60:9; 66:19 Zep 2:11}
\crossref{Isa}{49}{2}{Isa 11:4 Ps 45:2-\allowbreak5 Ho 6:5 Heb 4:12 Re 1:16; 2:12; 19:15}
\crossref{Isa}{49}{3}{Isa 42:1; 43:21; 44:23; 52:13; 53:10 Zec 3:8 Mt 17:5 Lu 2:10-\allowbreak14}
\crossref{Isa}{49}{4}{Isa 65:2 Eze 3:19 Mt 17:17; 23:37 Joh 1:11 Ro 10:21 Ga 4:11}
\crossref{Isa}{49}{5}{49:1}
\crossref{Isa}{49}{6}{Isa 42:6; 60:3 Lu 2:32 Joh 1:4-\allowbreak9 Ac 13:47; 26:18}
\crossref{Isa}{49}{7}{Isa 48:7 Re 3:7}
\crossref{Isa}{49}{8}{Ps 69:13 Joh 11:41,\allowbreak42 2Co 6:2 Eph 1:6 Heb 5:7}
\crossref{Isa}{49}{9}{Isa 42:7; 61:1 Ps 69:33; 102:20; 107:10-\allowbreak16; 146:7 Zec 9:11,\allowbreak12 Lu 4:18}
\crossref{Isa}{49}{10}{Mt 5:6 Joh 6:35 Re 7:16,\allowbreak17}
\crossref{Isa}{49}{11}{Isa 11:16; 35:8-\allowbreak10; 40:3,\allowbreak4; 43:19; 57:14; 62:10 Ps 107:4,\allowbreak7 Lu 3:4,\allowbreak5}
\crossref{Isa}{49}{12}{Isa 2:2,\allowbreak3; 11:10,\allowbreak11; 43:5,\allowbreak6; 60:9-\allowbreak14; 66:19,\allowbreak20 Ps 22:27; 72:10,\allowbreak11,\allowbreak17}
\crossref{Isa}{49}{13}{Isa 42:10,\allowbreak11; 44:23; 52:9; 55:12 Ps 96:11-\allowbreak13; 98:4-\allowbreak9 Lu 2:13,\allowbreak14; 15:10}
\crossref{Isa}{49}{14}{Isa 40:27 Ps 22:1; 31:22; 77:6-\allowbreak9; 89:38-\allowbreak46 Ro 11:1-\allowbreak5}
\crossref{Isa}{49}{15}{1Ki 3:26,\allowbreak27 Ps 103:13 Mal 3:17 Mt 7:11}
\crossref{Isa}{49}{16}{Ex 13:9 So 8:6 Jer 22:24 Hag 2:23}
\crossref{Isa}{49}{17}{Isa 51:18-\allowbreak20; 62:5 Ezr 1:5 Ne 2:4-\allowbreak9,\allowbreak17 Eze 28:24}
\crossref{Isa}{49}{18}{Isa 60:4 Ge 13:14 Mt 13:41,\allowbreak42 Re 22:15}
\crossref{Isa}{49}{19}{49:8; 51:3; 54:1,\allowbreak2 Jer 30:18,\allowbreak19; 33:10,\allowbreak11 Eze 36:9-\allowbreak15 Ho 1:10,\allowbreak11}
\crossref{Isa}{49}{20}{Isa 60:4 Ho 1:10 Mt 3:9 Ga 4:26-\allowbreak28}
\crossref{Isa}{49}{21}{Jer 31:15-\allowbreak17 Ro 11:11-\allowbreak17,\allowbreak24 Ga 3:29; 4:26-\allowbreak29}
\crossref{Isa}{49}{22}{49:12; 2:2,\allowbreak3; 11:10,\allowbreak11; 42:1-\allowbreak4; 60:3-\allowbreak11; 66:20 Ps 22:27; 67:4-\allowbreak7}
\crossref{Isa}{49}{23}{Nu 11:12}
\crossref{Isa}{49}{24}{Eze 37:3,\allowbreak11}
\crossref{Isa}{49}{25}{Isa 10:27; 52:2-\allowbreak5 Jer 29:10; 50:17-\allowbreak19,\allowbreak33,\allowbreak34 Zec 9:11 Heb 2:14,\allowbreak15}
\crossref{Isa}{49}{26}{Isa 9:20 Jud 7:22}
\crossref{Isa}{50}{1}{De 24:1-\allowbreak4 Jer 3:1,\allowbreak8 Ho 2:2-\allowbreak4 Mr 10:4-\allowbreak12}
\crossref{Isa}{50}{2}{Isa 59:16; 65:12; 66:4 Pr 1:24 Jer 5:1; 7:13; 8:6; 35:15 Ho 11:2,\allowbreak7}
\crossref{Isa}{50}{3}{Ex 10:21 Ps 18:11,\allowbreak12 Mt 27:45 Re 6:12}
\crossref{Isa}{50}{4}{Ex 4:11,\allowbreak12 Ps 45:2 Jer 1:9 Mt 22:46 Lu 4:22; 21:15 Joh 7:46}
\crossref{Isa}{50}{5}{Isa 48:8 Ps 40:6-\allowbreak8 Mt 26:39 Joh 8:29; 14:31; 15:10 Php 2:8 Heb 5:8}
\crossref{Isa}{50}{6}{La 3:30 Mic 5:1 Mt 5:39; 26:67; 27:26 Mr 14:65; 15:19}
\crossref{Isa}{50}{7}{50:9; 42:1; 49:8 Ps 89:21-\allowbreak27; 110:1 Joh 16:33 Heb 13:6}
\crossref{Isa}{50}{8}{Ro 8:32-\allowbreak34 1Ti 3:16}
\crossref{Isa}{50}{9}{Isa 51:6-\allowbreak8 Job 13:28 Ps 39:11; 102:26 Heb 1:11,\allowbreak12}
\crossref{Isa}{50}{10}{Ps 25:12,\allowbreak14; 111:10; 112:1; 128:1 Ec 12:13 Mal 3:16}
\crossref{Isa}{50}{11}{Isa 28:15-\allowbreak20; 30:15,\allowbreak16; 55:2 Ps 20:7,\allowbreak8 Jer 17:5-\allowbreak7 Jon 2:8 Mt 15:6-\allowbreak8}
\crossref{Isa}{51}{1}{51:4,\allowbreak7; 46:3,\allowbreak4; 48:12; 55:2,\allowbreak3}
\crossref{Isa}{51}{2}{Ge 15:1,\allowbreak2; 18:11-\allowbreak13 Jos 24:3 Ro 4:1-\allowbreak5,\allowbreak16-\allowbreak24}
\crossref{Isa}{51}{3}{51:12; 12:1; 40:1,\allowbreak2; 49:13; 54:6-\allowbreak8; 61:1-\allowbreak3; 66:10-\allowbreak14 Ps 85:8}
\crossref{Isa}{51}{4}{Isa 26:2 Ex 19:6; 33:13 Ps 33:12; 106:5; 147:20 1Pe 2:9}
\crossref{Isa}{51}{5}{Isa 46:13; 56:1 De 30:14 Ps 85:9 Mt 3:2 Ro 1:16,\allowbreak17; 10:6-\allowbreak10}
\crossref{Isa}{51}{6}{Isa 40:26 De 4:19 Ps 8:3,\allowbreak4}
\crossref{Isa}{51}{7}{51:1}
\crossref{Isa}{51}{8}{Isa 50:9; 66:24 Job 4:19; 13:28 Ho 5:12}
\crossref{Isa}{51}{9}{51:17; 27:1 Ps 7:6; 44:23; 59:4; 78:65 Hab 2:19}
\crossref{Isa}{51}{10}{Isa 42:15; 43:16; 50:2; 63:11,\allowbreak12 Ex 14:21,\allowbreak22; 15:13 Ps 74:13}
\crossref{Isa}{51}{11}{Isa 35:10; 44:23; 48:20; 49:13 Jer 30:18,\allowbreak19; 31:11,\allowbreak12; 33:11}
\crossref{Isa}{51}{12}{51:3; 43:25; 57:15-\allowbreak18; 66:13 Joh 14:18,\allowbreak26,\allowbreak27 Ac 9:31 2Co 1:3-\allowbreak5}
\crossref{Isa}{51}{13}{Isa 17:10 De 32:18 Jer 2:32}
\crossref{Isa}{51}{14}{Isa 48:20; 52:2 Ezr 1:5 Ac 12:7,\allowbreak8}
\crossref{Isa}{51}{15}{51:10 Ne 9:11 Job 26:12 Ps 74:13; 114:3-\allowbreak5; 136:13 Jer 31:35}
\crossref{Isa}{51}{16}{Isa 50:4; 59:21 De 18:18 Joh 3:34; 8:38-\allowbreak40; 17:8 Re 1:1}
\crossref{Isa}{51}{17}{51:9; 52:1; 60:1,\allowbreak2 Jud 5:12 1Co 15:34 Eph 5:14}
\crossref{Isa}{51}{18}{Isa 3:4-\allowbreak8; 49:21 Ps 88:18; 142:4 Mt 9:36; 15:14}
\crossref{Isa}{51}{19}{Isa 47:9 Eze 14:21}
\crossref{Isa}{51}{20}{Isa 40:30 Jer 14:18 La 1:15,\allowbreak19; 2:11,\allowbreak12; 4:2; 5:13}
\crossref{Isa}{51}{21}{51:1}
\crossref{Isa}{51}{22}{1Sa 25:39 Ps 35:1 Pr 22:23 Jer 50:34; 51:36 Joe 3:2 Mic 7:9}
\crossref{Isa}{51}{23}{Isa 49:25,\allowbreak26 Pr 11:8; 21:18 Jer 25:17-\allowbreak29 Zec 12:2 Re 17:6-\allowbreak8,\allowbreak18}
\crossref{Isa}{52}{1}{Isa 51:9,\allowbreak17 Da 10:9,\allowbreak16-\allowbreak19 Hag 2:4 Eph 6:10}
\crossref{Isa}{52}{2}{Isa 3:26; 51:23 Jer 51:6,\allowbreak45,\allowbreak50 Zec 2:6 Re 18:4}
\crossref{Isa}{52}{3}{Isa 45:13; 50:1 Ps 44:12 Jer 15:13 1Pe 1:18 Ro 7:14-\allowbreak25}
\crossref{Isa}{52}{4}{Ge 46:6 Ac 7:14,\allowbreak15}
\crossref{Isa}{52}{5}{Isa 22:16 Jud 18:3}
\crossref{Isa}{52}{6}{Ex 33:19; 34:5-\allowbreak7 Ps 48:10 Eze 20:44; 37:13,\allowbreak14; 39:27-\allowbreak29}
\crossref{Isa}{52}{7}{Ps 68:11 So 2:8 Mr 13:10; 16:15 Lu 24:47 Ac 10:36-\allowbreak38 Re 14:6}
\crossref{Isa}{52}{8}{Isa 56:10; 62:6 So 3:3; 5:7 Jer 6:17; 31:6,\allowbreak7 Eze 3:17; 33:7 Heb 13:17}
\crossref{Isa}{52}{9}{Isa 14:7; 42:10,\allowbreak11; 44:23; 48:20; 49:13; 54:1-\allowbreak3; 55:12; 65:18,\allowbreak19}
\crossref{Isa}{52}{10}{Isa 51:9; 66:18,\allowbreak19 Ps 98:1-\allowbreak3 Ac 2:5-\allowbreak11 Re 11:15-\allowbreak17; 15:4}
\crossref{Isa}{52}{11}{Isa 48:20 Jer 50:8; 51:6,\allowbreak45 Zec 2:6,\allowbreak7 2Co 6:17 Re 18:4}
\crossref{Isa}{52}{12}{Isa 28:16; 51:14 Ex 12:33,\allowbreak39; 14:8}
\crossref{Isa}{52}{13}{Isa 11:2,\allowbreak3; 42:1; 49:6; 53:11 Eze 34:23 Zec 3:8 Php 2:7,\allowbreak8}
\crossref{Isa}{52}{14}{Ps 71:7 Mt 7:28; 22:22,\allowbreak23; 27:14 Mr 5:42; 6:51; 7:37; 10:26,\allowbreak32}
\crossref{Isa}{52}{15}{Nu 8:7 Eze 36:25 Mt 28:19 Ac 2:33 Tit 3:5,\allowbreak6 Heb 9:13,\allowbreak14; 10:22}
\crossref{Isa}{53}{1}{Joh 1:7,\allowbreak12; 12:38 Ro 10:16,\allowbreak17}
\crossref{Isa}{53}{2}{Isa 11:1 Jer 23:5 Eze 17:22-\allowbreak24 Zec 6:12 Mr 6:3 Lu 2:7,\allowbreak39,\allowbreak40,\allowbreak51,\allowbreak52}
\crossref{Isa}{53}{3}{Isa 49:7; 50:6 Ps 22:6-\allowbreak8; 69:10-\allowbreak12,\allowbreak19,\allowbreak20 Mic 5:1 Zec 11:8,\allowbreak12,\allowbreak13}
\crossref{Isa}{53}{4}{53:5,\allowbreak6,\allowbreak11,\allowbreak12 Mt 8:17 Ga 3:13 Heb 9:28 1Pe 2:24; 3:18 1Jo 2:2}
\crossref{Isa}{53}{5}{53:6-\allowbreak8,\allowbreak11,\allowbreak12 Da 9:24 Zec 13:7 Mt 20:28 Ro 3:24-\allowbreak26; 4:25}
\crossref{Isa}{53}{6}{Ps 119:176 Mt 18:12-\allowbreak14 Lu 15:3-\allowbreak7 Ro 3:10-\allowbreak19 1Pe 2:25}
\crossref{Isa}{53}{7}{Mt 26:63; 27:12-\allowbreak14 Mr 14:61; 15:5 Lu 23:9 Joh 19:9 1Pe 2:23}
\crossref{Isa}{53}{8}{Mt 1:1 Ac 8:33 Ro 1:4}
\crossref{Isa}{53}{9}{Mt 27:57-\allowbreak60 Mr 15:43-\allowbreak46 Lu 23:50-\allowbreak53 Joh 19:38-\allowbreak42 1Co 15:4}
\crossref{Isa}{53}{10}{Isa 42:1 Mt 3:17; 17:5}
\crossref{Isa}{53}{11}{Lu 22:44 Joh 12:24,\allowbreak27-\allowbreak32; 16:21 Ga 4:19 Heb 12:2 Re 5:9,\allowbreak10}
\crossref{Isa}{53}{12}{Isa 49:24,\allowbreak25; 52:15 Ge 3:15 Ps 2:8 Da 2:45 Mt 12:28,\allowbreak29 Ac 26:18}
\crossref{Isa}{54}{1}{Isa 62:4 So 8:8 Ga 4:27}
\crossref{Isa}{54}{2}{Isa 33:20; 49:19,\allowbreak20 Jer 10:20}
\crossref{Isa}{54}{3}{Isa 2:2-\allowbreak4; 11:9-\allowbreak12; 35:1,\allowbreak2; 42:1-\allowbreak12; 43:5,\allowbreak6; 49:12; 60:3-\allowbreak11 Ge 49:10}
\crossref{Isa}{54}{4}{Isa 41:10,\allowbreak14; 45:16,\allowbreak17; 61:7 1Pe 2:6}
\crossref{Isa}{54}{5}{Ps 45:10-\allowbreak17 Jer 3:14 Eze 16:8 Ho 2:19,\allowbreak20 Joh 3:29 2Co 11:2,\allowbreak3}
\crossref{Isa}{54}{6}{Isa 49:14; 62:4 Ho 2:1,\allowbreak2,\allowbreak14,\allowbreak15 Mt 11:28 2Co 7:6,\allowbreak9,\allowbreak10}
\crossref{Isa}{54}{7}{Isa 26:20; 60:10 Ps 30:5 2Co 4:17 2Pe 3:8}
\crossref{Isa}{54}{8}{Isa 47:6; 57:16,\allowbreak17 Zec 1:15}
\crossref{Isa}{54}{9}{Isa 12:1; 55:11 Ge 8:21; 9:11-\allowbreak16 Ps 104:9 Jer 31:35,\allowbreak36; 33:20-\allowbreak26}
\crossref{Isa}{54}{10}{Isa 51:6,\allowbreak7 Ps 46:2 Mt 5:18; 16:18; 24:35 Ro 11:29 2Pe 3:10-\allowbreak13}
\crossref{Isa}{54}{11}{54:6; 49:14; 51:17-\allowbreak19,\allowbreak23; 52:1-\allowbreak5; 60:15 Ex 2:23; 3:2,\allowbreak7 De 31:17}
\crossref{Isa}{54}{12}{Isa 14:31; 22:7; 24:12}
\crossref{Isa}{54}{13}{Isa 2:3; 11:9 Ps 25:8-\allowbreak12; 71:17 Jer 31:34 Mt 11:25-\allowbreak29; 16:17}
\crossref{Isa}{54}{14}{Isa 1:26; 45:24; 52:1; 60:21; 61:10,\allowbreak11; 62:1 Jer 31:23 Eze 36:27,\allowbreak28}
\crossref{Isa}{54}{15}{Eze 38:8-\allowbreak23 Joe 3:9-\allowbreak14 Re 16:14; 19:19-\allowbreak21; 20:8,\allowbreak9}
\crossref{Isa}{54}{16}{Isa 10:5,\allowbreak6,\allowbreak15; 37:26; 46:11 Ex 9:16 Pr 16:4 Da 4:34,\allowbreak35 Joh 19:11}
\crossref{Isa}{54}{17}{54:15 Ps 2:1-\allowbreak6 Eze 38:9,\allowbreak10 Mt 16:18 Joh 10:28-\allowbreak30 Ro 8:1,\allowbreak28-\allowbreak39}
\crossref{Isa}{55}{1}{Ru 4:1 Pr 1:21-\allowbreak23; 8:4 Zec 2:6}
\crossref{Isa}{55}{2}{Isa 44:20 Jer 2:13 Ho 8:7; 12:1 Hab 2:13 Mt 15:9 Lu 15:15,\allowbreak16}
\crossref{Isa}{55}{3}{Ps 78:1; 119:112 Pr 4:20}
\crossref{Isa}{55}{4}{Joh 3:16; 18:37 1Ti 6:13 Re 1:5; 3:14}
\crossref{Isa}{55}{5}{Isa 11:10,\allowbreak11; 52:15; 56:8 Ge 49:10 Ps 18:43 Ro 15:20 Eph 2:11; 3:5}
\crossref{Isa}{55}{6}{Isa 45:19 1Ch 28:9 2Ch 19:3 Job 8:5 Ps 14:2; 27:8; 32:6; 95:7}
\crossref{Isa}{55}{7}{Isa 1:16-\allowbreak18 2Ch 7:14 Pr 28:13 Jer 3:3; 8:4-\allowbreak6 Eze 3:18,\allowbreak19}
\crossref{Isa}{55}{8}{2Sa 7:19 Ps 25:10; 40:5; 92:5 Pr 21:8; 25:3 Jer 3:1 Eze 18:29}
\crossref{Isa}{55}{9}{Ps 36:5; 77:19; 89:2; 103:11 Mt 11:25 Ro 11:31-\allowbreak36}
\crossref{Isa}{55}{10}{Isa 5:6; 30:23; 61:11 De 32:2 1Sa 23:4 Ps 65:9-\allowbreak13; 72:6,\allowbreak7 Eze 34:26}
\crossref{Isa}{55}{11}{Isa 54:9 De 32:2 Mt 24:35 Lu 8:11-\allowbreak16 Joh 6:63 Ro 10:17 1Co 1:18}
\crossref{Isa}{55}{12}{Isa 35:10; 48:20; 49:9,\allowbreak10; 51:11; 65:13,\allowbreak14 Ps 105:43 Jer 30:19}
\crossref{Isa}{55}{13}{Isa 11:6-\allowbreak9; 41:19; 60:13,\allowbreak21; 61:3 Mic 7:4 Ro 6:19 1Co 6:9-\allowbreak11}
\crossref{Isa}{56}{1}{Isa 1:16-\allowbreak19; 26:7,\allowbreak8; 55:7 Ps 24:4-\allowbreak6; 50:23 Jer 7:3-\allowbreak11 Mal 4:4}
\crossref{Isa}{56}{2}{Ps 1:1-\allowbreak3; 15:1-\allowbreak5; 106:3; 112:1; 119:1-\allowbreak5; 128:1 Lu 11:28; 12:43}
\crossref{Isa}{56}{3}{Nu 18:4,\allowbreak7 De 23:1-\allowbreak3 Zec 8:20-\allowbreak23 Mt 8:10,\allowbreak11 Ac 8:27; 10:1,\allowbreak2,\allowbreak34}
\crossref{Isa}{56}{4}{Jos 24:15 Ps 119:111 Lu 10:42}
\crossref{Isa}{56}{5}{Mt 16:18 Eph 2:22 1Ti 3:15 Heb 3:6}
\crossref{Isa}{56}{6}{56:3; 44:5 Jer 50:5 Ac 2:41; 11:23 2Co 8:5 1Th 1:9,\allowbreak10}
\crossref{Isa}{56}{7}{Isa 2:2,\allowbreak3; 66:19,\allowbreak20 Ps 2:6 Mic 4:1,\allowbreak2 Zec 8:3 Mal 1:11 Joh 12:20-\allowbreak26}
\crossref{Isa}{56}{8}{Isa 11:11,\allowbreak12; 27:12,\allowbreak13; 54:7 Ps 106:47; 107:2,\allowbreak3; 147:2 Jer 30:17}
\crossref{Isa}{56}{9}{De 28:26 Jer 12:9 Eze 29:5; 39:17 Re 19:17,\allowbreak18}
\crossref{Isa}{56}{10}{Isa 52:8 Eze 3:17}
\crossref{Isa}{56}{11}{1Sa 2:12-\allowbreak17,\allowbreak29 Eze 13:19; 34:2,\allowbreak3 Mic 3:5,\allowbreak11 Mal 1:10}
\crossref{Isa}{56}{12}{Isa 5:22; 28:7,\allowbreak8 Pr 31:4,\allowbreak5 Ho 4:11 Am 6:3-\allowbreak6 Mt 24:49-\allowbreak51}
\crossref{Isa}{57}{1}{2Ch 32:33; 35:24}
\crossref{Isa}{57}{2}{Job 3:17 Ec 12:7 Mt 25:21 Lu 16:22 2Co 5:1,\allowbreak8 Php 1:23}
\crossref{Isa}{57}{3}{Isa 45:20 Joe 3:9-\allowbreak11}
\crossref{Isa}{57}{4}{Isa 10:15; 37:23,\allowbreak29 Ex 9:17; 16:7,\allowbreak8 Nu 16:11 Lu 10:16 Ac 9:4}
\crossref{Isa}{57}{5}{Ex 32:6 Nu 25:1,\allowbreak2,\allowbreak6 Jer 50:38; 51:7 Ho 4:11-\allowbreak13; 7:4-\allowbreak7 Am 2:7,\allowbreak8}
\crossref{Isa}{57}{6}{Jer 3:9 Hab 2:19}
\crossref{Isa}{57}{7}{Jer 2:20; 3:2 Eze 16:16,\allowbreak25; 20:28,\allowbreak29; 23:17,\allowbreak41}
\crossref{Isa}{57}{8}{Eze 8:8-\allowbreak12; 23:14,\allowbreak41}
\crossref{Isa}{57}{9}{Isa 30:1-\allowbreak6; 31:1-\allowbreak3 2Ki 16:7-\allowbreak11 Eze 16:33; 23:16 Ho 7:11; 12:1}
\crossref{Isa}{57}{10}{Isa 47:13 Jer 2:36; 9:5 Eze 24:12 Hab 2:13}
\crossref{Isa}{57}{11}{Isa 51:12,\allowbreak13 Pr 29:25 Mt 26:69-\allowbreak75 Ga 2:12,\allowbreak13}
\crossref{Isa}{57}{12}{Isa 1:11-\allowbreak15; 58:2-\allowbreak6; 59:6-\allowbreak8; 64:5; 66:3,\allowbreak4 Jer 7:4-\allowbreak11 Mic 3:2-\allowbreak4}
\crossref{Isa}{57}{13}{57:9,\allowbreak10 Jud 10:14 2Ki 3:13 Jer 22:22 Zec 7:13}
\crossref{Isa}{57}{14}{Isa 35:8; 40:3; 62:10 Lu 3:5,\allowbreak6}
\crossref{Isa}{57}{15}{Isa 6:1 Ps 83:18; 97:9; 138:6 Da 4:17,\allowbreak24,\allowbreak25,\allowbreak34}
\crossref{Isa}{57}{16}{Ps 78:38,\allowbreak39; 85:5; 103:9-\allowbreak16 Jer 10:24 Mic 7:18}
\crossref{Isa}{57}{17}{Isa 5:8,\allowbreak9; 56:11 Jer 6:13; 8:10; 22:17 Eze 33:31 Mic 2:2,\allowbreak3 Lu 12:15}
\crossref{Isa}{57}{18}{Isa 1:18; 43:24,\allowbreak25; 48:8-\allowbreak11 Jer 31:18-\allowbreak20 Eze 16:60-\allowbreak63; 36:22-\allowbreak38}
\crossref{Isa}{57}{19}{Ex 4:11,\allowbreak12 Ho 14:2 Lu 21:15 Eph 6:19 Col 4:3,\allowbreak4 Heb 13:15}
\crossref{Isa}{57}{20}{Isa 3:11 Job 15:20-\allowbreak24; 18:5-\allowbreak14; 20:11-\allowbreak29 Ps 73:18-\allowbreak20 Pr 4:16,\allowbreak17}
\crossref{Isa}{57}{21}{Isa 3:11; 48:22 2Ki 9:22 Ro 3:16,\allowbreak17}
\crossref{Isa}{58}{1}{Isa 56:10 Ps 40:9,\allowbreak10 Jer 1:7-\allowbreak10,\allowbreak17-\allowbreak19; 7:8-\allowbreak11; 15:19,\allowbreak20 Eze 2:3-\allowbreak8}
\crossref{Isa}{58}{2}{Isa 1:11-\allowbreak15; 29:13; 48:1,\allowbreak2 De 5:28,\allowbreak29 1Sa 15:21-\allowbreak25 Pr 15:8}
\crossref{Isa}{58}{3}{Nu 23:4 Mic 3:9-\allowbreak11 Zec 7:5-\allowbreak7 Mal 3:14 Mt 20:11,\allowbreak12 Lu 15:29}
\crossref{Isa}{58}{4}{1Ki 21:9-\allowbreak13 Pr 21:27 Mt 6:16; 23:14 Lu 20:47 Joh 18:28}
\crossref{Isa}{58}{5}{2Ch 20:3 Ezr 10:6 Ne 9:1,\allowbreak2 Es 4:3,\allowbreak16 Da 9:3-\allowbreak19 Zec 7:5}
\crossref{Isa}{58}{6}{Ne 5:10-\allowbreak12 Jer 34:8-\allowbreak11 Mic 3:2-\allowbreak4}
\crossref{Isa}{58}{7}{58:10 Job 22:7; 31:18-\allowbreak21 Ps 112:9 Pr 22:9; 25:21; 28:27 Ec 11:1,\allowbreak2}
\crossref{Isa}{58}{8}{58:10,\allowbreak11 Job 11:17 Ps 37:6; 97:11; 112:4 Pr 4:18 Ho 6:3 Mal 4:2}
\crossref{Isa}{58}{9}{Isa 1:15; 30:19; 65:24 Ps 34:15-\allowbreak17; 37:4; 50:15; 66:18,\allowbreak19; 91:15; 118:5}
\crossref{Isa}{58}{10}{58:7 De 15:7-\allowbreak10 Ps 41:1; 112:5-\allowbreak9 Pr 11:24,\allowbreak25; 14:31; 28:27 Lu 18:22}
\crossref{Isa}{58}{11}{Isa 49:10 Ps 25:9; 32:8; 48:14; 73:24 Joh 16:13 1Th 3:11}
\crossref{Isa}{58}{12}{Isa 61:4 Ne 2:5,\allowbreak17; 4:1-\allowbreak6 Jer 31:38 Eze 36:4,\allowbreak8-\allowbreak11,\allowbreak33 Am 9:14}
\crossref{Isa}{58}{13}{Isa 56:2-\allowbreak6 Ex 20:8-\allowbreak11; 31:13-\allowbreak17; 36:2,\allowbreak3 De 5:12-\allowbreak15 Ne 13:15-\allowbreak22}
\crossref{Isa}{58}{14}{Job 22:26; 27:10; 34:9 Ps 36:8; 37:4,\allowbreak11 Hab 3:18 Php 4:4 1Pe 1:8}
\crossref{Isa}{59}{1}{Isa 50:2 Ge 18:14 Nu 11:23 Jer 32:17}
\crossref{Isa}{59}{2}{Isa 50:1 De 32:19 Jos 7:11 Pr 15:29 Jer 5:25}
\crossref{Isa}{59}{3}{Isa 1:15,\allowbreak21 Jer 2:30,\allowbreak34; 22:17 Eze 7:23; 9:9; 22:2; 35:6 Ho 4:2}
\crossref{Isa}{59}{4}{59:16 Jer 5:1,\allowbreak4,\allowbreak5 Eze 22:29-\allowbreak31 Mic 7:2-\allowbreak5}
\crossref{Isa}{59}{5}{Isa 14:29 Pr 23:32}
\crossref{Isa}{59}{6}{Isa 28:18-\allowbreak20; 30:12-\allowbreak14 Job 8:14,\allowbreak15}
\crossref{Isa}{59}{7}{Pr 1:16; 6:17 Ro 3:15}
\crossref{Isa}{59}{8}{Pr 3:17 Lu 1:79 Ro 3:17}
\crossref{Isa}{59}{9}{La 5:16,\allowbreak17 Hab 1:13}
\crossref{Isa}{59}{10}{De 28:29 Job 5:14 Pr 4:19 Jer 13:16 La 4:14 Am 8:9}
\crossref{Isa}{59}{11}{Isa 51:20 Ps 32:3,\allowbreak4; 38:8 Ho 7:14}
\crossref{Isa}{59}{12}{Isa 1:4 Ezr 9:6 Jer 3:2; 5:3-\allowbreak9,\allowbreak25-\allowbreak29; 7:8-\allowbreak10 Eze 5:6; 7:23; 8:8-\allowbreak16}
\crossref{Isa}{59}{13}{Isa 32:6; 48:8; 57:11 Ps 78:36 Jer 3:10; 42:20 Eze 18:25 Ho 6:7; 7:13}
\crossref{Isa}{59}{14}{59:4; 5:23; 10:1,\allowbreak2 Ps 82:2-\allowbreak5 Ec 3:16 Jer 5:27,\allowbreak28,\allowbreak31 Am 5:7,\allowbreak11}
\crossref{Isa}{59}{15}{Isa 48:1 Ps 5:9; 12:1,\allowbreak2 Jer 5:1,\allowbreak2; 7:28 Ho 4:1,\allowbreak2 Mic 7:2}
\crossref{Isa}{59}{16}{Isa 50:2; 64:7 Ge 18:23-\allowbreak32 Ps 106:23 Jer 5:1 Eze 22:30 Mr 6:6}
\crossref{Isa}{59}{17}{Isa 11:5; 51:9 Job 29:14 Ro 13:12-\allowbreak14 2Co 6:7 Eph 6:14,\allowbreak17 1Th 5:8}
\crossref{Isa}{59}{18}{Isa 63:6 Job 34:11 Ps 18:24-\allowbreak26; 62:12 Jer 17:10; 50:29 Mt 16:27}
\crossref{Isa}{59}{19}{Isa 11:9-\allowbreak16; 24:14-\allowbreak16; 49:12; 66:18-\allowbreak20 Ps 22:27; 102:15,\allowbreak16; 113:3}
\crossref{Isa}{59}{20}{Ob 1:17-\allowbreak21 Ro 11:26,\allowbreak27}
\crossref{Isa}{59}{21}{Isa 49:8; 55:3 Jer 31:31-\allowbreak34; 32:38-\allowbreak41 Eze 36:25-\allowbreak27; 37:25-\allowbreak27}
\crossref{Isa}{60}{1}{Isa 52:1,\allowbreak2 Mt 5:16 Eph 5:8,\allowbreak14 Php 2:15}
\crossref{Isa}{60}{2}{Mt 15:14; 23:19,\allowbreak24 Joh 8:55 Ac 14:16; 17:23,\allowbreak30,\allowbreak31; 26:18}
\crossref{Isa}{60}{3}{Isa 2:2-\allowbreak5; 11:10; 19:23-\allowbreak25; 45:14; 49:6,\allowbreak12,\allowbreak23; 54:1-\allowbreak3; 66:12,\allowbreak19,\allowbreak20}
\crossref{Isa}{60}{4}{Isa 49:18 Joh 4:35 Ac 13:44}
\crossref{Isa}{60}{5}{Jer 33:9 Ho 1:10,\allowbreak11; 3:5 Ac 10:45; 11:17}
\crossref{Isa}{60}{6}{Isa 30:6 Jud 6:5; 7:12 1Ki 10:2 2Ki 8:9}
\crossref{Isa}{60}{7}{Isa 42:11 Ge 25:13}
\crossref{Isa}{60}{8}{60:4; 45:22 Lu 13:29 Re 7:9}
\crossref{Isa}{60}{9}{Isa 42:4,\allowbreak10; 49:1; 51:5; 66:19,\allowbreak20 Ge 9:27; 10:2-\allowbreak5 Ps 72:10 Zep 2:11}
\crossref{Isa}{60}{10}{Isa 61:5; 66:21 Zec 6:15}
\crossref{Isa}{60}{11}{Ne 13:19 Re 21:25}
\crossref{Isa}{60}{12}{Isa 41:11; 54:15 Ps 2:12 Da 2:35,\allowbreak44,\allowbreak45 Zec 12:2-\allowbreak4; 14:12-\allowbreak19}
\crossref{Isa}{60}{13}{Ezr 7:27}
\crossref{Isa}{60}{14}{Isa 14:1,\allowbreak2; 45:14; 49:23 Jer 16:19 Re 3:9}
\crossref{Isa}{60}{15}{Isa 49:14-\allowbreak23; 54:6-\allowbreak14 Ps 78:60,\allowbreak61 Jer 30:17 La 1:1,\allowbreak2 Re 11:2,\allowbreak15-\allowbreak17}
\crossref{Isa}{60}{16}{Isa 49:23; 61:6; 66:11,\allowbreak12}
\crossref{Isa}{60}{17}{Isa 30:26 1Ki 10:21-\allowbreak27 Zec 12:8 Heb 11:40 2Pe 3:13}
\crossref{Isa}{60}{18}{Isa 2:4; 11:9 Ps 72:3-\allowbreak7 Mic 4:3 Zec 9:8}
\crossref{Isa}{60}{19}{Ps 36:9 Re 21:23; 22:5}
\crossref{Isa}{60}{20}{Ps 27:1; 84:11 Am 8:9 Mal 4:2}
\crossref{Isa}{60}{21}{Isa 4:3,\allowbreak4; 51:2; 62:4 Zec 14:20,\allowbreak21 2Pe 3:13 Re 21:27}
\crossref{Isa}{60}{22}{Isa 66:8 Da 2:35,\allowbreak44 Mt 13:31,\allowbreak32 Ac 2:41; 5:14 Re 7:9}
\crossref{Isa}{61}{1}{Isa 11:2-\allowbreak5; 42:1; 59:21 Mt 3:16 Lu 4:18,\allowbreak19 Joh 1:32,\allowbreak33; 3:34}
\crossref{Isa}{61}{2}{Le 25:9-\allowbreak13 Lu 4:19 2Co 6:2}
\crossref{Isa}{61}{3}{Isa 12:1 Es 4:1-\allowbreak3; 8:15; 9:22 Ps 30:11 Eze 16:8-\allowbreak13}
\crossref{Isa}{61}{4}{Isa 49:6-\allowbreak8; 58:12 Eze 36:23-\allowbreak26,\allowbreak33-\allowbreak36 Am 9:14,\allowbreak15}
\crossref{Isa}{61}{5}{Isa 14:1,\allowbreak2; 60:10-\allowbreak14 Eph 2:12-\allowbreak20}
\crossref{Isa}{61}{6}{Isa 60:17; 66:21 Ex 19:6 Ro 12:1 1Pe 2:5,\allowbreak9 Re 1:6; 5:10; 20:6}
\crossref{Isa}{61}{7}{Isa 40:2 De 21:17 2Ki 2:9 Job 42:10 Zec 9:12 2Co 4:17}
\crossref{Isa}{61}{8}{Ps 11:7; 33:5; 37:28; 45:7; 99:4 Jer 9:24 Zec 8:16,\allowbreak17}
\crossref{Isa}{61}{9}{Isa 44:3 Ge 22:18 Zec 8:13 Ro 9:3,\allowbreak4}
\crossref{Isa}{61}{10}{Isa 35:10; 51:11 1Sa 2:1 Ne 8:10 Ps 28:7 Hab 3:18 Zec 10:7}
\crossref{Isa}{61}{11}{Isa 55:10,\allowbreak11; 58:11 So 4:16; 5:1 Mt 13:3,\allowbreak8,\allowbreak23 Mr 4:26-\allowbreak32}
\crossref{Isa}{62}{1}{62:6,\allowbreak7 Ps 51:18; 102:13-\allowbreak16; 122:6-\allowbreak9; 137:6 Zec 2:12 Lu 10:2 2Th 3:1}
\crossref{Isa}{62}{2}{Isa 49:6; 52:10; 60:1-\allowbreak3; 61:9; 66:12,\allowbreak19 Mic 5:8 Ac 9:15; 26:23}
\crossref{Isa}{62}{3}{Zec 9:16 Lu 2:14 1Th 2:19}
\crossref{Isa}{62}{4}{62:12; 32:14,\allowbreak15; 49:14; 54:1,\allowbreak6,\allowbreak7 Ho 1:9,\allowbreak10 Ro 9:25-\allowbreak27 Heb 13:5}
\crossref{Isa}{62}{5}{Isa 49:18-\allowbreak22 Ps 45:11-\allowbreak16 Jer 32:41}
\crossref{Isa}{62}{6}{Isa 52:8; 56:10 2Ch 8:14 So 3:3; 5:7 Jer 6:17 Eze 3:17-\allowbreak21; 33:2-\allowbreak9}
\crossref{Isa}{62}{7}{62:1-\allowbreak3; 61:11 Jer 33:9 Zep 3:19,\allowbreak20 Mt 6:9,\allowbreak10,\allowbreak13 Re 11:15}
\crossref{Isa}{62}{8}{De 32:40 Eze 20:5}
\crossref{Isa}{62}{9}{De 12:7,\allowbreak12; 14:23-\allowbreak29; 16:11,\allowbreak14}
\crossref{Isa}{62}{10}{Isa 18:3; 40:3; 48:20; 52:11; 57:14 Ex 17:15 Mt 22:9 Heb 12:13}
\crossref{Isa}{62}{11}{Ps 98:1-\allowbreak3 Mr 16:15 Ro 10:11-\allowbreak18}
\crossref{Isa}{62}{12}{Isa 60:21 De 7:6; 26:19; 28:9 1Pe 2:9}
\crossref{Isa}{63}{1}{Ps 24:7-\allowbreak10 So 3:6; 6:10; 8:5 Mt 21:10}
\crossref{Isa}{63}{2}{Ge 17:17}
\crossref{Isa}{63}{3}{Isa 25:10 La 1:15 Mal 4:3 Re 14:19,\allowbreak20; 19:13-\allowbreak15}
\crossref{Isa}{63}{4}{Isa 34:8; 35:4; 61:2 Jer 51:6 Zec 3:8 Lu 21:22 Re 6:9-\allowbreak17; 11:13}
\crossref{Isa}{63}{5}{63:3; 41:28; 50:2; 59:16 Joh 16:32}
\crossref{Isa}{63}{6}{63:2,\allowbreak3; 49:26; 51:21-\allowbreak23 Job 21:20 Ps 60:3; 75:8 Jer 25:16,\allowbreak17,\allowbreak26,\allowbreak27}
\crossref{Isa}{63}{7}{Isa 41:8,\allowbreak9; 51:2 Ne 9:7-\allowbreak15,\allowbreak19-\allowbreak21,\allowbreak27,\allowbreak31 Ps 63:3; 78:11-\allowbreak72; 105:5-\allowbreak45}
\crossref{Isa}{63}{8}{Isa 41:8 Ge 17:7 Ex 3:7; 4:22,\allowbreak23; 6:7; 19:5,\allowbreak6 Ro 11:1,\allowbreak2,\allowbreak28}
\crossref{Isa}{63}{9}{Ex 3:7-\allowbreak9 Jud 10:16 Zec 2:8 Mt 25:40,\allowbreak45 Ac 9:4 Heb 2:18; 4:15}
\crossref{Isa}{63}{10}{Isa 1:2; 65:2 Ex 15:24; 16:8; 32:8 Nu 14:9-\allowbreak11; 16:1-\allowbreak35 De 9:7,\allowbreak22-\allowbreak24}
\crossref{Isa}{63}{11}{Le 26:40-\allowbreak45 De 4:30,\allowbreak31 Ps 25:6; 77:5-\allowbreak11; 89:47-\allowbreak50; 143:5}
\crossref{Isa}{63}{12}{Ex 15:6,\allowbreak13,\allowbreak16 Ps 80:1}
\crossref{Isa}{63}{13}{Ps 106:9 Hab 3:15}
\crossref{Isa}{63}{14}{Jos 22:4; 23:1 Heb 4:8-\allowbreak11}
\crossref{Isa}{63}{15}{De 26:15 Ps 33:14; 80:14; 102:19,\allowbreak20 La 3:50}
\crossref{Isa}{63}{16}{Isa 64:8 Ex 4:22 De 32:6 1Ch 29:10 Jer 3:19; 31:9 Mal 1:6; 2:10}
\crossref{Isa}{63}{17}{Ps 119:10,\allowbreak36; 141:4 Eze 14:7-\allowbreak9 2Th 2:11,\allowbreak12}
\crossref{Isa}{63}{18}{Isa 62:12 Ex 19:4-\allowbreak6 De 7:6; 26:19 Da 8:24 1Pe 2:9}
\crossref{Isa}{63}{19}{Ps 79:6; 135:4 Jer 10:25 Ac 14:16 Ro 9:4 Eph 2:12}
\crossref{Isa}{64}{1}{Ps 18:7-\allowbreak15; 144:5,\allowbreak6 Mr 1:10}
\crossref{Isa}{64}{2}{Isa 37:20; 63:12 Ex 14:4 1Sa 17:46,\allowbreak47 1Ki 8:41-\allowbreak43 Ps 46:10; 67:1,\allowbreak2}
\crossref{Isa}{64}{3}{Ex 34:10 De 4:34; 10:21 Jud 5:4,\allowbreak5 2Sa 7:23 Ps 65:6; 66:3,\allowbreak5; 68:8}
\crossref{Isa}{64}{4}{Ps 31:19 1Co 2:9,\allowbreak10 Eph 3:5-\allowbreak10,\allowbreak17-\allowbreak21 Col 1:26,\allowbreak27 1Ti 3:16}
\crossref{Isa}{64}{5}{Ex 20:24; 25:22; 29:42,\allowbreak43; 30:6 Heb 4:16}
\crossref{Isa}{64}{6}{Isa 6:5; 53:6 Job 14:4; 15:14-\allowbreak16; 25:4; 40:4; 42:5,\allowbreak6 Ps 51:5}
\crossref{Isa}{64}{7}{Isa 50:2; 59:16 Ps 14:4 Eze 22:30 Ho 7:7,\allowbreak14}
\crossref{Isa}{64}{8}{Isa 63:16 Ex 4:22 De 32:6 Ga 3:26,\allowbreak29}
\crossref{Isa}{64}{9}{Ps 6:1; 38:1; 74:1,\allowbreak2; 79:5-\allowbreak9 Jer 10:24 Hab 3:2}
\crossref{Isa}{64}{10}{Isa 1:7 2Ki 25:9 2Ch 36:19-\allowbreak21 Ps 79:1-\allowbreak7 La 1:1-\allowbreak4; 2:4-\allowbreak8; 5:18}
\crossref{Isa}{64}{11}{2Ki 25:9 2Ch 36:19 Ps 74:5-\allowbreak7 Jer 52:13 La 2:7 Eze 7:20,\allowbreak21}
\crossref{Isa}{64}{12}{Isa 42:14 Ps 10:1; 74:10,\allowbreak11,\allowbreak18,\allowbreak19; 79:5; 80:3,\allowbreak4; 83:1; 89:46-\allowbreak51}
\crossref{Isa}{65}{1}{Isa 2:2,\allowbreak3; 11:10; 55:5 Ps 22:27 Ro 9:24-\allowbreak26,\allowbreak30; 10:20 Eph 2:12,\allowbreak13}
\crossref{Isa}{65}{2}{Pr 1:24 Mt 23:37 Lu 13:34; 19:41,\allowbreak42 Ro 10:21}
\crossref{Isa}{65}{3}{Isa 3:8 De 32:16-\allowbreak19,\allowbreak21 2Ki 17:14-\allowbreak17; 22:17 Ps 78:40,\allowbreak58}
\crossref{Isa}{65}{4}{Nu 19:11,\allowbreak16-\allowbreak20 De 18:11 Mt 8:28 Mr 5:2-\allowbreak5 Lu 8:27}
\crossref{Isa}{65}{5}{Mt 9:11 Lu 5:30; 7:39; 15:2,\allowbreak28-\allowbreak30; 18:9-\allowbreak12 Ac 22:21,\allowbreak22}
\crossref{Isa}{65}{6}{Ex 17:14 De 32:34 Ps 56:8 Mal 3:16 Re 20:12}
\crossref{Isa}{65}{7}{Ex 20:5 Le 26:39 Nu 32:14 Ps 106:6,\allowbreak7 Da 9:8 Mt 23:31-\allowbreak36}
\crossref{Isa}{65}{8}{Isa 6:13 Jer 30:11 Joe 2:14 Am 9:8,\allowbreak9 Mt 24:22 Mr 13:20 Ro 9:27-\allowbreak29}
\crossref{Isa}{65}{9}{Isa 10:20-\allowbreak22; 11:11-\allowbreak16; 27:6 Jer 31:36-\allowbreak40; 33:17-\allowbreak26 Eze 36:8-\allowbreak15,\allowbreak24}
\crossref{Isa}{65}{10}{Isa 33:9; 35:2 Eze 34:13,\allowbreak14}
\crossref{Isa}{65}{11}{Isa 1:28 De 29:25 1Ch 28:9 Jer 17:13}
\crossref{Isa}{65}{12}{Isa 3:25; 10:4 Le 26:25 De 32:25 Jer 18:21; 34:17 Eze 14:17-\allowbreak21}
\crossref{Isa}{65}{13}{Ps 34:10; 37:19,\allowbreak20 Mal 3:18 Lu 14:23,\allowbreak24; 16:24,\allowbreak25}
\crossref{Isa}{65}{14}{Isa 24:14; 52:8,\allowbreak9 Job 29:13 Ps 66:4 Jer 31:7 Jas 5:13}
\crossref{Isa}{65}{15}{Pr 10:7 Jer 29:22 Zec 8:13}
\crossref{Isa}{65}{16}{Ps 72:17 Jer 4:2}
\crossref{Isa}{65}{17}{Isa 51:16; 66:22 2Pe 3:13 Re 21:1-\allowbreak5}
\crossref{Isa}{65}{18}{Isa 12:4-\allowbreak6; 42:10-\allowbreak12; 44:23; 49:13; 51:11; 52:7-\allowbreak10; 66:10-\allowbreak14 Ps 67:3-\allowbreak5}
\crossref{Isa}{65}{19}{Isa 62:4,\allowbreak5 So 3:11 Jer 32:41 Zep 3:17 Lu 15:3,\allowbreak5}
\crossref{Isa}{65}{20}{De 4:40 Job 5:26 Ps 34:12}
\crossref{Isa}{65}{21}{Isa 62:8,\allowbreak9 Le 26:16 De 28:30-\allowbreak33 Jud 6:1-\allowbreak6 Jer 31:4,\allowbreak5 Am 9:14}
\crossref{Isa}{65}{22}{65:9,\allowbreak15 Ge 5:5,\allowbreak27 Le 26:16 Ps 92:12-\allowbreak14 Re 20:3-\allowbreak5}
\crossref{Isa}{65}{23}{Isa 49:4; 55:2; 61:9 Le 26:3-\allowbreak10,\allowbreak20,\allowbreak22,\allowbreak29 De 28:3-\allowbreak12,\allowbreak38-\allowbreak42}
\crossref{Isa}{65}{24}{Isa 58:9 Ps 32:5; 50:15; 91:15 Da 9:20-\allowbreak23; 10:12 Mr 11:24}
\crossref{Isa}{65}{25}{Isa 11:6-\allowbreak9; 35:9 Ac 9:1,\allowbreak19-\allowbreak21 1Co 6:9-\allowbreak11 Tit 3:3-\allowbreak7}
\crossref{Isa}{66}{1}{1Ki 8:27 1Ch 28:2 2Ch 6:18 Ps 11:4; 99:9; 132:7 Mt 5:34,\allowbreak35}
\crossref{Isa}{66}{2}{Isa 40:26 Ge 1:1-\allowbreak31 Col 1:17 Heb 1:2,\allowbreak3}
\crossref{Isa}{66}{3}{Isa 1:11-\allowbreak15 Pr 15:8; 21:27 Am 5:21,\allowbreak22}
\crossref{Isa}{66}{4}{1Ki 22:19-\allowbreak23 Ps 81:12 Pr 1:31,\allowbreak32 Mt 24:24 2Th 2:10-\allowbreak12}
\crossref{Isa}{66}{5}{66:2 Pr 13:13 Jer 36:16,\allowbreak23-\allowbreak25}
\crossref{Isa}{66}{6}{Isa 34:8; 59:18; 65:5-\allowbreak7 Joe 3:7-\allowbreak16 Am 1:2-\allowbreak2:16}
\crossref{Isa}{66}{7}{Isa 54:1 Ga 4:26 Re 12:1-\allowbreak5}
\crossref{Isa}{66}{8}{Isa 64:4 1Co 2:9}
\crossref{Isa}{66}{9}{Isa 37:3 Ge 18:14}
\crossref{Isa}{66}{10}{Isa 44:23; 65:18 De 32:43 Ro 15:9-\allowbreak12}
\crossref{Isa}{66}{11}{Isa 60:5,\allowbreak16 Ps 36:8 Joe 3:18 1Pe 2:2}
\crossref{Isa}{66}{12}{Isa 9:7; 48:18; 60:5 Ps 72:3-\allowbreak7}
\crossref{Isa}{66}{13}{Isa 51:3 1Th 2:7}
\crossref{Isa}{66}{14}{Zec 10:7 Joh 16:22}
\crossref{Isa}{66}{15}{Isa 30:27,\allowbreak28,\allowbreak33 Ps 11:6; 21:9; 50:3; 97:3 Am 7:4 Mt 22:7 2Th 1:6-\allowbreak9}
\crossref{Isa}{66}{16}{Isa 27:1; 34:5-\allowbreak10 Eze 38:21,\allowbreak22; 39:2-\allowbreak10 Re 19:11-\allowbreak21}
\crossref{Isa}{66}{17}{Isa 1:29; 65:3,\allowbreak4}
\crossref{Isa}{66}{18}{Isa 37:28 De 31:21 Am 5:12 Joh 5:42 Re 2:2,\allowbreak9,\allowbreak13}
\crossref{Isa}{66}{19}{Isa 11:10; 18:3,\allowbreak7; 62:10 Lu 2:34}
\crossref{Isa}{66}{20}{Isa 43:6; 49:12-\allowbreak26; 54:3; 60:3-\allowbreak14}
\crossref{Isa}{66}{21}{Isa 61:6 Ex 19:6 Jer 13:18-\allowbreak22 1Pe 2:5,\allowbreak9 Re 1:6; 5:10; 20:6}
\crossref{Isa}{66}{22}{Isa 65:17 Heb 12:27,\allowbreak28 2Pe 3:13 Re 21:1}
\crossref{Isa}{66}{23}{Isa 1:13,\allowbreak14 2Ki 4:23 Ps 81:3,\allowbreak4 Eze 46:1,\allowbreak6 Col 2:16,\allowbreak17}
\crossref{Isa}{66}{24}{66:16 Ps 58:10,\allowbreak11 Eze 39:9-\allowbreak16 Zec 14:12,\allowbreak18,\allowbreak19 Re 19:17-\allowbreak21}

% Jer
\crossref{Jer}{1}{1}{2Ch 36:21 Isa 1:1; 2:1 Am 1:1; 7:10}
\crossref{Jer}{1}{2}{1:4,\allowbreak11 1Ki 13:20 Ho 1:1 Jon 1:1 Mic 1:1}
\crossref{Jer}{1}{3}{Jer 25:1-\allowbreak3; 26:1-\allowbreak24; 35:1-\allowbreak36:32 2Ki 24:1-\allowbreak9 2Ch 36:5-\allowbreak8}
\crossref{Jer}{1}{4}{1:2 Eze 1:3; 3:16}
\crossref{Jer}{1}{5}{Ps 71:5,\allowbreak6 Isa 49:1,\allowbreak5 Lu 1:76 Ga 1:15,\allowbreak16}
\crossref{Jer}{1}{6}{Jer 4:10; 14:13; 32:17}
\crossref{Jer}{1}{7}{1:17,\allowbreak18 Ex 7:1,\allowbreak2 Nu 22:20,\allowbreak38 1Ki 22:14 2Ch 18:13 Eze 2:3-\allowbreak5}
\crossref{Jer}{1}{8}{1:17 Isa 51:7,\allowbreak12 Eze 2:6,\allowbreak7; 3:8,\allowbreak9 Mt 10:26 Lu 12:4,\allowbreak5 Ac 4:13,\allowbreak29}
\crossref{Jer}{1}{9}{Ex 4:11,\allowbreak12 Isa 6:6,\allowbreak7; 49:2; 50:4 Lu 21:15}
\crossref{Jer}{1}{10}{Jer 25:15-\allowbreak27; 27:2-\allowbreak7; 46:1-\allowbreak51:64 1Ki 17:1 Re 11:3-\allowbreak6}
\crossref{Jer}{1}{11}{Am 7:8; 8:2 Zec 4:2; 5:2}
\crossref{Jer}{1}{12}{De 5:28; 18:17 Lu 10:28; 20:39}
\crossref{Jer}{1}{13}{Ge 41:32 2Co 13:1,\allowbreak2}
\crossref{Jer}{1}{14}{Jer 4:6; 6:1,\allowbreak22; 10:22; 31:8; 46:20; 50:9,\allowbreak41 Isa 41:25 Eze 1:4}
\crossref{Jer}{1}{15}{Jer 5:15; 6:22; 10:22,\allowbreak25; 25:9,\allowbreak28,\allowbreak31,\allowbreak32}
\crossref{Jer}{1}{16}{Jer 4:12,\allowbreak28; 5:9,\allowbreak29 Eze 24:14 Joe 2:11 Mt 23:35,\allowbreak36}
\crossref{Jer}{1}{17}{1Ki 18:46 2Ki 4:29; 9:1 Job 38:3 Lu 12:35 1Pe 1:13}
\crossref{Jer}{1}{18}{Jer 6:27; 15:20 Isa 50:7 Eze 3:8,\allowbreak9 Mic 3:8,\allowbreak9 Joh 1:42}
\crossref{Jer}{1}{19}{Jer 11:19; 15:10-\allowbreak21; 20:1-\allowbreak6; 26:11-\allowbreak24; 29:25-\allowbreak32; 37:11-\allowbreak21; 38:6-\allowbreak13}
\crossref{Jer}{2}{1}{Jer 1:11; 7:1; 23:28 Eze 7:1 Heb 1:1 2Pe 1:21}
\crossref{Jer}{2}{2}{Jer 7:2; 11:6; 19:2 Pr 1:20; 8:1-\allowbreak4 Isa 58:1 Ho 8:1 Jon 1:2 Mt 11:15}
\crossref{Jer}{2}{3}{Ex 19:5,\allowbreak6 De 7:6; 14:2; 26:19 Zec 14:20,\allowbreak21 Eph 1:4 1Pe 2:9}
\crossref{Jer}{2}{4}{Jer 5:21; 7:2; 13:15; 19:3; 34:4; 44:24-\allowbreak26 Isa 51:1-\allowbreak4 Ho 4:1 Mic 6:1}
\crossref{Jer}{2}{5}{2:31 Isa 5:3,\allowbreak4; 43:22,\allowbreak23 Mic 6:2,\allowbreak3}
\crossref{Jer}{2}{6}{2:8; 5:2 Jud 6:13 2Ki 2:14 Job 35:10 Ps 77:5 Isa 64:7}
\crossref{Jer}{2}{7}{Nu 13:27; 14:7,\allowbreak8 De 6:10,\allowbreak11,\allowbreak18; 8:7-\allowbreak9; 11:11,\allowbreak12 Ne 9:25 Eze 20:6}
\crossref{Jer}{2}{8}{2:6; 5:31; 8:10,\allowbreak11; 23:9-\allowbreak15 1Sa 2:12 Isa 28:7; 29:10; 56:9-\allowbreak12}
\crossref{Jer}{2}{9}{2:29,\allowbreak35 Isa 3:13; 43:26 Eze 20:35,\allowbreak36 Ho 2:2 Mic 6:2}
\crossref{Jer}{2}{10}{Ge 10:4,\allowbreak5 Nu 24:24 1Ch 1:7; 23:1,\allowbreak12 Ps 120:5 Eze 27:6 Da 11:30}
\crossref{Jer}{2}{11}{2:5 Mic 4:5 1Pe 1:18}
\crossref{Jer}{2}{12}{Jer 6:19; 22:29 De 32:1 Isa 1:2 Mic 6:2 Mt 27:45,\allowbreak50-\allowbreak53}
\crossref{Jer}{2}{13}{2:31,\allowbreak32; 4:22; 5:26,\allowbreak31 Ps 81:11-\allowbreak13 Isa 1:3; 5:13; 63:8 Mic 2:8; 6:3}
\crossref{Jer}{2}{14}{Ex 4:22 Isa 50:1}
\crossref{Jer}{2}{15}{Jer 5:6; 25:30; 50:17 Jud 14:5 Job 4:10 Ps 57:4 Isa 5:29 Ho 5:14}
\crossref{Jer}{2}{16}{2Ki 18:21; 23:33 Isa 30:1-\allowbreak6; 31:1-\allowbreak3}
\crossref{Jer}{2}{17}{2:19; 4:18 Le 26:15-\allowbreak46 Nu 32:23 De 28:15-\allowbreak68 Job 4:8 Isa 1:4}
\crossref{Jer}{2}{18}{2:36; 37:5-\allowbreak10 Isa 30:1-\allowbreak7; 31:1 La 4:17 Eze 17:15 Ho 7:11}
\crossref{Jer}{2}{19}{2:17 Pr 1:31; 5:22 Isa 3:9; 5:5; 50:1 Ho 5:5}
\crossref{Jer}{2}{20}{Jer 30:8 Ex 3:8 Le 26:13 De 4:20,\allowbreak34; 15:15 Isa 9:4; 10:27; 14:25}
\crossref{Jer}{2}{21}{Ex 15:17 Ps 44:2; 80:8 Isa 5:1,\allowbreak2; 60:21; 61:3 Mt 21:33 Mr 12:1}
\crossref{Jer}{2}{22}{Job 9:30,\allowbreak31}
\crossref{Jer}{2}{23}{2:34,\allowbreak35 Ge 3:12,\allowbreak13 1Sa 15:13,\allowbreak14 Ps 36:2 Pr 28:13; 30:12,\allowbreak20}
\crossref{Jer}{2}{24}{Jer 14:6 Job 11:12; 39:5-\allowbreak8}
\crossref{Jer}{2}{25}{Jer 13:22 De 28:48 Isa 20:2-\allowbreak4 La 4:4 Ho 2:3 Lu 15:22; 16:24}
\crossref{Jer}{2}{26}{2:36; 3:24,\allowbreak25 Pr 6:30,\allowbreak31 Isa 1:29 Ro 6:21}
\crossref{Jer}{2}{27}{Jer 10:8 Ps 115:4-\allowbreak8 Isa 44:9-\allowbreak20; 46:6-\allowbreak8 Hab 2:18,\allowbreak19}
\crossref{Jer}{2}{28}{De 32:37 Jud 10:14 2Ki 3:13 Isa 45:20; 46:2,\allowbreak7}
\crossref{Jer}{2}{29}{2:23,\allowbreak35; 3:2}
\crossref{Jer}{2}{30}{Jer 5:3; 6:29,\allowbreak30; 7:28; 31:18 2Ch 28:22 Isa 1:5; 9:13 Eze 24:13}
\crossref{Jer}{2}{31}{Am 1:1 Mic 6:9}
\crossref{Jer}{2}{32}{2:11 Ge 24:22,\allowbreak30,\allowbreak53 2Sa 1:24 Ps 45:13,\allowbreak14 Isa 61:10 Eze 16:10-\allowbreak13}
\crossref{Jer}{2}{33}{2:23,\allowbreak36; 3:1,\allowbreak2 Isa 57:7-\allowbreak10 Ho 2:5-\allowbreak7,\allowbreak13}
\crossref{Jer}{2}{34}{Jer 7:31; 19:4 2Ki 21:16; 24:4 Ps 106:37,\allowbreak38 Isa 57:5; 59:7}
\crossref{Jer}{2}{35}{2:23,\allowbreak29 Job 33:9 Pr 28:13 Isa 58:3 Ro 7:9}
\crossref{Jer}{2}{36}{2:18,\allowbreak23,\allowbreak33; 31:22 Ho 5:13; 7:11; 12:1}
\crossref{Jer}{2}{37}{2Sa 13:19}
\crossref{Jer}{3}{1}{De 24:1-\allowbreak4}
\crossref{Jer}{3}{2}{Jer 2:23 Eze 8:4-\allowbreak6 Lu 16:23}
\crossref{Jer}{3}{3}{Jer 9:12; 14:4,\allowbreak22 Le 26:19 De 28:23 Isa 5:6 Joe 1:16-\allowbreak20 Am 4:7}
\crossref{Jer}{3}{4}{3:19; 31:9,\allowbreak18-\allowbreak20 Ho 14:1-\allowbreak3}
\crossref{Jer}{3}{5}{3:12 Ps 77:7-\allowbreak9; 85:5; 103:8,\allowbreak9 Isa 57:16; 64:9}
\crossref{Jer}{3}{6}{3:8,\allowbreak11-\allowbreak14}
\crossref{Jer}{3}{7}{2Ki 17:13,\allowbreak14 2Ch 30:6-\allowbreak12 Ho 6:1-\allowbreak4; 14:1}
\crossref{Jer}{3}{8}{3:1 2Ki 17:6-\allowbreak18; 18:9-\allowbreak11 Eze 23:9 Ho 2:2,\allowbreak3; 3:4; 4:15-\allowbreak17; 9:15-\allowbreak17}
\crossref{Jer}{3}{9}{Eze 23:10}
\crossref{Jer}{3}{10}{2Ch 34:33; 35:1-\allowbreak18 Ps 78:36,\allowbreak37 Isa 10:6 Ho 7:14}
\crossref{Jer}{3}{11}{3:8,\allowbreak22 Ho 4:16; 11:7}
\crossref{Jer}{3}{12}{3:18; 23:8; 31:8 2Ki 15:29; 17:6,\allowbreak23; 18:1}
\crossref{Jer}{3}{13}{3:25; 31:18-\allowbreak20 Le 26:40-\allowbreak42 De 30:1-\allowbreak3 Job 33:27,\allowbreak28 Pr 28:13}
\crossref{Jer}{3}{14}{Jer 2:19}
\crossref{Jer}{3}{15}{Jer 23:4 1Sa 13:14 Isa 30:20,\allowbreak21 Eze 34:23; 37:24 Mic 5:4,\allowbreak5}
\crossref{Jer}{3}{16}{Jer 30:19; 31:8,\allowbreak27 Isa 60:22; 61:4 Eze 36:8-\allowbreak12; 37:26 Ho 1:10,\allowbreak11}
\crossref{Jer}{3}{17}{Jer 14:21; 17:12; 31:23 Ps 87:3 Isa 6:1; 66:1 Eze 1:26; 43:7 Ga 4:26}
\crossref{Jer}{3}{18}{Jer 30:3; 50:4,\allowbreak20 Isa 11:11-\allowbreak13 Eze 37:16-\allowbreak22; 39:25-\allowbreak28 Ho 1:11; 11:12}
\crossref{Jer}{3}{19}{Jer 5:7 Ho 11:8}
\crossref{Jer}{3}{20}{Ho 3:1}
\crossref{Jer}{3}{21}{Jer 30:15-\allowbreak17; 31:9,\allowbreak18-\allowbreak20; 50:4,\allowbreak5 Isa 15:2 Eze 7:16 Zec 12:10-\allowbreak14}
\crossref{Jer}{3}{22}{Ho 6:1; 14:1,\allowbreak4}
\crossref{Jer}{3}{23}{3:6; 10:14-\allowbreak16 Ps 121:1,\allowbreak2 Isa 44:9; 45:20; 46:7,\allowbreak8 Eze 20:28}
\crossref{Jer}{3}{24}{Jer 11:13 Eze 16:61,\allowbreak63 Ho 2:8; 9:10; 10:6}
\crossref{Jer}{3}{25}{Jer 2:26; 6:26 Ezr 9:6-\allowbreak15 Ps 109:29 Isa 50:11 La 5:16 Eze 7:18}
\crossref{Jer}{4}{1}{4:4; 3:12,\allowbreak22}
\crossref{Jer}{4}{2}{Jer 5:2 De 10:20 Isa 45:23; 48:1,\allowbreak2; 65:16}
\crossref{Jer}{4}{3}{Ge 3:18 Ho 10:12 Mt 13:7,\allowbreak22 Mr 4:7,\allowbreak18,\allowbreak19 Lu 8:7,\allowbreak14 Ga 6:7,\allowbreak8}
\crossref{Jer}{4}{4}{Jer 9:26 De 10:16; 30:6 Eze 18:31 Ro 2:28,\allowbreak29 Col 2:11}
\crossref{Jer}{4}{5}{Jer 5:20; 9:12; 11:2}
\crossref{Jer}{4}{6}{4:21; 50:2; 51:12,\allowbreak27 Isa 62:10}
\crossref{Jer}{4}{7}{Jer 5:6; 25:38; 49:19; 50:17,\allowbreak44 2Ki 24:1; 25:1 Da 7:4}
\crossref{Jer}{4}{8}{Jer 6:26 Isa 15:3; 22:12; 32:11 Joe 2:12,\allowbreak13 Am 8:10}
\crossref{Jer}{4}{9}{Jer 39:4,\allowbreak5; 52:7 1Sa 25:37,\allowbreak38 2Ki 25:4 Ps 102:4 Isa 19:3,\allowbreak11,\allowbreak12,\allowbreak16}
\crossref{Jer}{4}{10}{Jer 1:6; 14:13; 32:17 Eze 11:13}
\crossref{Jer}{4}{11}{Jer 23:19; 30:23,\allowbreak24; 51:1 Isa 27:8; 64:6 Eze 17:10; 19:12 Ho 13:3,\allowbreak15}
\crossref{Jer}{4}{12}{}
\crossref{Jer}{4}{13}{Isa 13:5; 19:1 Na 1:3 Mt 24:30 Re 1:7}
\crossref{Jer}{4}{14}{Isa 1:16-\allowbreak19; 55:7 Eze 18:31 Mt 12:33; 15:19,\allowbreak20; 23:26,\allowbreak27}
\crossref{Jer}{4}{15}{Jer 6:1; 8:16 Jud 18:29; 20:1}
\crossref{Jer}{4}{16}{Jer 6:18; 31:10; 50:2 Isa 34:1}
\crossref{Jer}{4}{17}{Jer 6:2,\allowbreak3 2Ki 25:1-\allowbreak4 Isa 1:8 Lu 19:43,\allowbreak44; 21:20-\allowbreak24}
\crossref{Jer}{4}{18}{Jer 2:17,\allowbreak19; 5:19; 6:19; 26:19 Job 20:5-\allowbreak16 Ps 107:17 Pr 1:31; 5:22}
\crossref{Jer}{4}{19}{Jer 9:1,\allowbreak10; 13:17; 14:17,\allowbreak18; 23:9; 48:31,\allowbreak32 Ps 119:53,\allowbreak136 Isa 15:5}
\crossref{Jer}{4}{20}{4:6; 17:18 Le 26:18,\allowbreak21,\allowbreak24,\allowbreak28 Ps 42:7 Isa 13:6 La 3:47}
\crossref{Jer}{4}{21}{4:14}
\crossref{Jer}{4}{22}{Jer 5:4,\allowbreak21; 8:7-\allowbreak9 De 32:6,\allowbreak28 Ps 14:1-\allowbreak4 Isa 1:3; 6:9,\allowbreak10; 27:11}
\crossref{Jer}{4}{23}{Jer 9:10 Ge 1:2 Isa 24:19-\allowbreak23 Re 20:11}
\crossref{Jer}{4}{24}{Jer 8:16; 10:10 Jud 5:4,\allowbreak5 1Ki 19:11 Ps 18:7; 77:18; 97:4; 114:4-\allowbreak7}
\crossref{Jer}{4}{25}{Ho 4:3 Zep 1:2,\allowbreak3}
\crossref{Jer}{4}{26}{Jer 12:4; 14:2-\allowbreak6 De 29:23-\allowbreak28 Ps 76:7; 107:34 Isa 5:9,\allowbreak10; 7:20-\allowbreak25}
\crossref{Jer}{4}{27}{4:7; 7:34; 12:11; 18:16 2Ch 36:21 Isa 6:11,\allowbreak12; 24:1,\allowbreak3-\allowbreak12}
\crossref{Jer}{4}{28}{4:23-\allowbreak26; 12:4; 23:10 Isa 24:4; 33:8,\allowbreak9 Ho 4:3 Joe 1:10}
\crossref{Jer}{4}{29}{Jer 39:4-\allowbreak6; 52:7 2Ki 25:4-\allowbreak7 Isa 30:17 Am 9:1}
\crossref{Jer}{4}{30}{Jer 5:31; 13:21 Isa 10:3; 20:6; 33:14 Heb 2:3}
\crossref{Jer}{4}{31}{Jer 6:24; 13:21; 22:23; 30:6; 48:41; 49:22,\allowbreak24; 50:43 Isa 13:8; 21:3}
\crossref{Jer}{5}{1}{2Ch 16:9 Da 12:4 Joe 2:9 Am 8:12 Zec 2:4}
\crossref{Jer}{5}{2}{Jer 4:2; 7:9 Le 19:12 Isa 48:1 Ho 4:1,\allowbreak2,\allowbreak15; 10:4 Zec 5:3,\allowbreak4 Mal 3:5}
\crossref{Jer}{5}{3}{Jer 32:19 2Ch 16:9 Ps 11:4-\allowbreak7; 51:6 Pr 22:12 Ro 2:2}
\crossref{Jer}{5}{4}{Jer 4:22; 7:8; 8:7 Isa 27:11; 28:9-\allowbreak13 Ho 4:6 Mt 11:5 Joh 7:48,\allowbreak49}
\crossref{Jer}{5}{5}{Am 4:1 Mic 3:1 Mal 2:7}
\crossref{Jer}{5}{6}{Jer 2:15; 4:7; 25:38; 49:19 Eze 14:16-\allowbreak21 Da 7:4 Ho 5:14; 13:7,\allowbreak8}
\crossref{Jer}{5}{7}{Jer 3:19 Ho 11:8 Mt 23:37,\allowbreak38}
\crossref{Jer}{5}{8}{Jer 13:27 Ge 39:9 Ex 20:14,\allowbreak17 De 5:18,\allowbreak21 2Sa 11:2-\allowbreak4 Job 31:9}
\crossref{Jer}{5}{9}{5:29; 9:9; 23:2 La 4:22 Ho 2:13; 8:13}
\crossref{Jer}{5}{10}{Jer 6:4-\allowbreak6; 25:9; 39:8; 51:20-\allowbreak23 2Ki 24:2-\allowbreak4 2Ch 36:17 Isa 10:5-\allowbreak7}
\crossref{Jer}{5}{11}{Jer 3:6-\allowbreak11,\allowbreak20 Isa 48:8 Ho 5:7; 6:7}
\crossref{Jer}{5}{12}{5:31; 4:10; 14:13,\allowbreak14; 23:14-\allowbreak17; 28:15-\allowbreak17; 43:2,\allowbreak3 De 29:19 1Sa 6:9}
\crossref{Jer}{5}{13}{Jer 14:13,\allowbreak15; 18:18; 20:8-\allowbreak11; 28:3 Job 6:26; 8:2 Ho 9:7}
\crossref{Jer}{5}{14}{Jer 1:9; 23:29; 28:15-\allowbreak17 2Ki 1:10-\allowbreak14 Ho 6:5 Zec 1:6 Re 11:5,\allowbreak6}
\crossref{Jer}{5}{15}{Jer 1:15; 4:16; 6:22; 25:9 De 28:49 Isa 5:26; 29:3,\allowbreak6}
\crossref{Jer}{5}{16}{Ps 5:9 Isa 5:28 Ro 3:13}
\crossref{Jer}{5}{17}{Le 26:16 De 28:30,\allowbreak31,\allowbreak33 Jud 6:3,\allowbreak4 Isa 62:9; 65:22 Hab 3:17,\allowbreak18}
\crossref{Jer}{5}{18}{5:10; 4:27 Eze 9:8; 11:13 Ro 11:1-\allowbreak5}
\crossref{Jer}{5}{19}{Jer 2:35; 13:22; 16:10; 22:8,\allowbreak9 De 29:24-\allowbreak28 1Ki 9:8,\allowbreak9 2Ch 7:21,\allowbreak22}
\crossref{Jer}{5}{20}{}
\crossref{Jer}{5}{21}{5:4; 4:22; 8:7; 10:8 De 29:4; 32:6 Ps 94:8 Isa 6:9,\allowbreak10; 27:11}
\crossref{Jer}{5}{22}{Jer 10:7 De 28:58 Ps 119:120 Mt 10:28 Lu 12:5 Re 15:4}
\crossref{Jer}{5}{23}{5:5; 6:28; 17:9 Ps 95:10 Isa 1:5; 31:6 Ho 4:8; 11:7 Heb 3:12}
\crossref{Jer}{5}{24}{5:22; 50:5 Isa 64:7 Ho 3:5; 6:1}
\crossref{Jer}{5}{25}{Jer 2:17-\allowbreak19; 3:3 De 28:23,\allowbreak24 Ps 107:17,\allowbreak34 Isa 59:2 La 3:39; 4:22}
\crossref{Jer}{5}{26}{Jer 4:22 Isa 58:1 Eze 22:2-\allowbreak12}
\crossref{Jer}{5}{27}{Pr 1:11-\allowbreak13 Ho 12:7,\allowbreak8 Am 8:4-\allowbreak6 Mic 1:12; 6:10,\allowbreak11 Hab 2:9-\allowbreak11}
\crossref{Jer}{5}{28}{De 32:15 Job 15:27,\allowbreak28; 21:23,\allowbreak24 Ps 73:6,\allowbreak7,\allowbreak12; 119:70 Am 4:1}
\crossref{Jer}{5}{29}{5:9; 9:9 Mal 3:5 Jas 5:4}
\crossref{Jer}{5}{30}{}
\crossref{Jer}{5}{31}{Jer 14:14; 23:25,\allowbreak26 La 2:14 Eze 13:6 Mic 3:11 Mt 7:15-\allowbreak17}
\crossref{Jer}{6}{1}{Jos 15:63; 18:21-\allowbreak28 Jud 1:21}
\crossref{Jer}{6}{2}{Jer 4:31 Isa 1:8; 3:16,\allowbreak17 La 2:1,\allowbreak13}
\crossref{Jer}{6}{3}{Na 3:18}
\crossref{Jer}{6}{4}{Jer 5:10; 51:27,\allowbreak28 Isa 5:26-\allowbreak30; 13:2-\allowbreak5 Joe 3:9}
\crossref{Jer}{6}{5}{Jer 9:21; 17:27; 52:13 2Ch 36:19 Ps 48:3 Isa 32:14 Ho 8:14 Am 2:5}
\crossref{Jer}{6}{6}{6:9 Jer 8:3; 9:7,\allowbreak17; 10:16; 11:17,\allowbreak20,\allowbreak22; 19:11; 20:12; 23:15,\allowbreak16,\allowbreak36}
\crossref{Jer}{6}{7}{Pr 4:23 Isa 57:20 Jas 3:10-\allowbreak12}
\crossref{Jer}{6}{8}{Jer 4:14; 7:3-\allowbreak7; 17:23; 31:19; 32:33; 35:13-\allowbreak15 De 32:29 Ps 2:10; 50:17}
\crossref{Jer}{6}{9}{Jer 16:16; 49:9; 52:28-\allowbreak30 Ob 1:5,\allowbreak6 Re 14:18}
\crossref{Jer}{6}{10}{Jer 5:4,\allowbreak5 Isa 28:9-\allowbreak13; 53:1}
\crossref{Jer}{6}{11}{Jer 20:9 Job 32:18,\allowbreak19 Eze 3:14 Mic 3:8 Ac 4:20; 17:16; 18:5}
\crossref{Jer}{6}{12}{Jer 8:10 De 28:30-\allowbreak33,\allowbreak39-\allowbreak43 Isa 65:21,\allowbreak22 La 5:3,\allowbreak11 Zep 1:13}
\crossref{Jer}{6}{13}{Jer 8:10; 14:18; 22:17; 23:11 Isa 56:9-\allowbreak12; 57:17 Eze 22:12; 33:31}
\crossref{Jer}{6}{14}{Jer 8:11,\allowbreak12 Eze 13:10}
\crossref{Jer}{6}{15}{Jer 3:3; 8:12 Isa 3:9}
\crossref{Jer}{6}{16}{Jer 18:15 De 32:7 So 1:7,\allowbreak8 Isa 8:20 Mal 4:4 Lu 16:29 Joh 5:39}
\crossref{Jer}{6}{17}{Jer 25:4 Isa 21:11; 56:10 Eze 3:17-\allowbreak21; 33:2-\allowbreak9 Hab 2:1 Ac 20:27-\allowbreak31}
\crossref{Jer}{6}{18}{Jer 4:10; 31:10 De 29:24-\allowbreak28 Ps 50:4-\allowbreak6 Isa 5:3 Mic 6:5}
\crossref{Jer}{6}{19}{Jer 22:29 De 4:26; 30:19; 32:1 Isa 1:2 Mic 6:2}
\crossref{Jer}{6}{20}{Ps 40:6; 50:7-\allowbreak13,\allowbreak16,\allowbreak17; 66:3 Isa 1:11; 66:3 Eze 20:39 Am 5:21,\allowbreak22}
\crossref{Jer}{6}{21}{Jer 13:16 Isa 8:14 Eze 3:20 Ro 9:33; 11:9 1Pe 2:8}
\crossref{Jer}{6}{22}{6:1; 1:14,\allowbreak15; 5:15; 10:22; 25:9; 50:41-\allowbreak43}
\crossref{Jer}{6}{23}{Jer 5:16; 50:42 Isa 13:18 Eze 23:22-\allowbreak25 Hab 1:6-\allowbreak10}
\crossref{Jer}{6}{24}{Jer 4:6-\allowbreak9,\allowbreak19-\allowbreak21 Isa 28:19 Eze 21:6,\allowbreak7 Hab 3:16}
\crossref{Jer}{6}{25}{Jer 4:5; 8:14; 14:18 Jud 5:6,\allowbreak7}
\crossref{Jer}{6}{26}{6:14; 4:11; 8:19,\allowbreak21,\allowbreak22; 9:1; 14:17 Isa 22:4 La 2:11; 3:48; 4:3,\allowbreak6,\allowbreak10}
\crossref{Jer}{6}{27}{Jer 1:18; 15:20 Eze 3:8-\allowbreak10; 20:4; 22:2}
\crossref{Jer}{6}{28}{Jer 5:23 Isa 1:5; 31:6}
\crossref{Jer}{6}{29}{Jer 9:7 Pr 17:3 Zec 13:9 Mal 3:2,\allowbreak3 1Pe 1:7; 4:12}
\crossref{Jer}{6}{30}{Ps 119:119 Pr 25:4 Isa 1:22,\allowbreak25 Eze 22:18,\allowbreak19 Mt 5:13}
\crossref{Jer}{7}{1}{7:1}
\crossref{Jer}{7}{2}{Jer 17:19; 19:2,\allowbreak14; 22:1; 26:2; 36:6,\allowbreak10 Pr 1:20,\allowbreak21; 8:2,\allowbreak3 Joh 18:20}
\crossref{Jer}{7}{3}{7:5-\allowbreak7; 18:11; 26:13; 35:15 Pr 28:13 Isa 1:16-\allowbreak19; 55:7 Eze 18:30,\allowbreak31}
\crossref{Jer}{7}{4}{7:8; 6:14; 28:15; 29:23,\allowbreak31 Eze 13:19 Mt 3:9,\allowbreak10}
\crossref{Jer}{7}{5}{7:3; 4:1,\allowbreak2 Isa 1:19}
\crossref{Jer}{7}{6}{Jer 22:3,\allowbreak4,\allowbreak15,\allowbreak16 Ex 22:21-\allowbreak24 De 24:17; 27:19 Job 31:13-\allowbreak22}
\crossref{Jer}{7}{7}{Jer 17:20-\allowbreak27; 18:7,\allowbreak8; 25:5}
\crossref{Jer}{7}{8}{7:4; 4:10; 5:31; 8:10; 14:13,\allowbreak14; 23:14-\allowbreak16,\allowbreak26,\allowbreak32 Isa 28:15; 30:10}
\crossref{Jer}{7}{9}{Jer 9:2-\allowbreak9 Ps 50:16-\allowbreak21 Isa 59:1-\allowbreak8 Eze 18:10-\allowbreak13,\allowbreak18; 33:25,\allowbreak26}
\crossref{Jer}{7}{10}{Pr 7:14,\allowbreak15; 15:8 Isa 1:10-\allowbreak15; 48:2; 58:2-\allowbreak4 Eze 20:39; 23:29,\allowbreak37,\allowbreak39}
\crossref{Jer}{7}{11}{2Ch 6:33 Isa 56:7 Mt 21:13 Mr 11:17 Lu 19:45,\allowbreak46 Joh 2:16}
\crossref{Jer}{7}{12}{Jos 18:1 Jud 18:31 1Sa 1:3}
\crossref{Jer}{7}{13}{7:25; 11:7; 25:3; 35:15; 44:4 2Ch 36:15,\allowbreak16 Ne 9:29,\allowbreak30}
\crossref{Jer}{7}{14}{7:4,\allowbreak10 De 28:52 Mic 3:11 Ac 6:13,\allowbreak14}
\crossref{Jer}{7}{15}{Jer 3:8; 15:1; 23:39; 52:3 2Ki 17:18-\allowbreak20,\allowbreak23; 24:20 Ho 1:4; 9:9,\allowbreak16,\allowbreak17}
\crossref{Jer}{7}{16}{Jer 11:14; 14:11,\allowbreak12; 15:1; 18:20 Ex 32:10 Eze 14:14-\allowbreak20 1Jo 5:16}
\crossref{Jer}{7}{17}{Jer 6:27 Eze 8:6-\allowbreak18; 14:23}
\crossref{Jer}{7}{18}{Jer 44:17-\allowbreak19,\allowbreak25 1Co 10:22}
\crossref{Jer}{7}{19}{Jer 2:17,\allowbreak19 De 32:16,\allowbreak21,\allowbreak22 Isa 1:20,\allowbreak24 Eze 8:17,\allowbreak18 1Co 10:22}
\crossref{Jer}{7}{20}{Jer 4:23-\allowbreak26; 9:10,\allowbreak11; 12:4; 14:16; 42:18; 44:6 Isa 42:25 La 2:3-\allowbreak5}
\crossref{Jer}{7}{21}{Jer 6:20 Isa 1:11-\allowbreak15 Ho 8:13 Am 5:21-\allowbreak23}
\crossref{Jer}{7}{22}{1Sa 15:22 Ps 40:6; 50:8-\allowbreak17; 51:16,\allowbreak17 Ho 6:6 Mt 9:13 Mr 12:33}
\crossref{Jer}{7}{23}{Jer 11:4,\allowbreak7 Ex 15:26; 19:5,\allowbreak6 Le 26:3-\allowbreak12 De 5:29,\allowbreak33; 6:3; 11:27; 13:4}
\crossref{Jer}{7}{24}{7:26; 11:7,\allowbreak8 Ex 32:7,\allowbreak8 Ne 9:16-\allowbreak20 Ps 81:11,\allowbreak12; 106:7-\allowbreak48}
\crossref{Jer}{7}{25}{Jer 32:30,\allowbreak31 De 9:7,\allowbreak21-\allowbreak24 1Sa 8:7,\allowbreak8 Ezr 9:7 Ne 9:16-\allowbreak18,\allowbreak26}
\crossref{Jer}{7}{26}{7:24; 6:17; 11:8; 17:23; 25:3,\allowbreak7; 26:5; 29:19; 34:14; 44:16 2Ch 33:10}
\crossref{Jer}{7}{27}{Jer 1:7; 26:2 Eze 2:4-\allowbreak7; 3:17,\allowbreak18 Ac 20:27}
\crossref{Jer}{7}{28}{Jer 2:30; 5:3; 6:29,\allowbreak30 Isa 1:4,\allowbreak5 Zep 3:2}
\crossref{Jer}{7}{29}{Jer 16:6; 47:5; 48:37 Job 1:20 Isa 15:2,\allowbreak3 Mic 1:16}
\crossref{Jer}{7}{30}{Jer 23:11; 32:34 2Ki 21:4,\allowbreak7; 23:4-\allowbreak6,\allowbreak12 2Ch 33:4,\allowbreak5,\allowbreak7,\allowbreak15 Eze 7:20}
\crossref{Jer}{7}{31}{Jer 19:5,\allowbreak6; 32:35 2Ki 23:20 2Ch 33:6}
\crossref{Jer}{7}{32}{Jer 19:6 Le 26:30 Eze 6:5-\allowbreak7}
\crossref{Jer}{7}{33}{Jer 8:1,\allowbreak2; 9:22; 12:9; 16:4; 22:19; 25:33; 34:20 De 28:26 Ps 79:2,\allowbreak3}
\crossref{Jer}{7}{34}{Jer 16:9; 25:10; 33:10 Isa 24:7,\allowbreak8 Eze 26:13 Ho 2:11 Re 18:23}
\crossref{Jer}{8}{1}{Jer 7:32-\allowbreak34 1Ki 13:2 2Ki 23:16,\allowbreak20 2Ch 34:4,\allowbreak5 Eze 6:5; 37:1 Am 2:1}
\crossref{Jer}{8}{2}{Jer 19:13; 44:17-\allowbreak19 De 4:19; 17:3 2Ki 17:16; 21:3,\allowbreak5; 23:5 2Ch 33:3-\allowbreak5}
\crossref{Jer}{8}{3}{Jer 20:14-\allowbreak18 1Ki 19:4 Job 3:20-\allowbreak22; 7:15,\allowbreak16 Jon 4:3 Re 6:16; 9:6}
\crossref{Jer}{8}{4}{Pr 24:16 Ho 14:1 Am 5:2 Mic 7:8}
\crossref{Jer}{8}{5}{Jer 2:32; 3:11-\allowbreak14; 7:24-\allowbreak26 Ho 4:16; 11:7}
\crossref{Jer}{8}{6}{Job 33:27,\allowbreak28 Ps 14:2 Isa 30:18 Mal 3:16 2Pe 3:9}
\crossref{Jer}{8}{7}{Pr 6:6-\allowbreak8 Isa 1:3}
\crossref{Jer}{8}{8}{Job 5:12,\allowbreak13; 11:12; 12:20 Joh 9:41 Ro 1:22; 2:17-\allowbreak29 1Co 3:18-\allowbreak20}
\crossref{Jer}{8}{9}{Jer 6:15; 49:7 Job 5:12 Isa 19:11 Eze 7:26 1Co 1:26-\allowbreak29}
\crossref{Jer}{8}{10}{Jer 6:12 De 28:30-\allowbreak32 Am 5:11 Zep 1:13}
\crossref{Jer}{8}{11}{Jer 6:14; 14:14,\allowbreak15; 27:9,\allowbreak10; 28:3-\allowbreak9 1Ki 22:6,\allowbreak13 La 2:14}
\crossref{Jer}{8}{12}{Jer 3:3; 6:15 Ps 52:1,\allowbreak7 Isa 3:9 Zep 3:5 Php 3:19}
\crossref{Jer}{8}{13}{Isa 24:21,\allowbreak22 Eze 22:19-\allowbreak21; 24:3-\allowbreak11}
\crossref{Jer}{8}{14}{2Ki 7:3,\allowbreak4}
\crossref{Jer}{8}{15}{Jer 4:10; 14:19 Mic 1:12 1Th 5:3}
\crossref{Jer}{8}{16}{Jer 4:15,\allowbreak16 Jud 18:29; 20:1}
\crossref{Jer}{8}{17}{De 32:24 Isa 14:29 Am 5:19; 9:3 Re 9:19}
\crossref{Jer}{8}{18}{Jer 6:24; 10:19-\allowbreak22 Job 7:13,\allowbreak14 Isa 22:4 La 1:16,\allowbreak17 Da 10:16,\allowbreak17}
\crossref{Jer}{8}{19}{Jer 4:16,\allowbreak17,\allowbreak30,\allowbreak31 Isa 13:5; 39:3}
\crossref{Jer}{8}{20}{Pr 10:5 Lu 13:25; 19:44 Heb 3:7-\allowbreak15 Mt 25:1-\allowbreak12}
\crossref{Jer}{8}{21}{Jer 4:19; 9:1; 14:17; 17:16 Ne 2:3 Ps 137:3-\allowbreak6 Lu 19:41 Ro 9:1-\allowbreak3}
\crossref{Jer}{8}{22}{Jer 46:11; 51:8 Ge 37:25; 43:11}
\crossref{Jer}{9}{1}{Jer 4:19; 13:17; 14:17 Ps 119:136 Isa 16:9; 22:4 La 2:11,\allowbreak18,\allowbreak19}
\crossref{Jer}{9}{2}{Ps 55:6-\allowbreak8; 120:5-\allowbreak7 Mic 7:1-\allowbreak7}
\crossref{Jer}{9}{3}{9:5,\allowbreak8 Ps 52:2-\allowbreak4; 64:3,\allowbreak4; 120:2-\allowbreak4 Isa 59:3-\allowbreak5,\allowbreak13-\allowbreak15 Mic 7:3-\allowbreak5}
\crossref{Jer}{9}{4}{Jer 12:6 Ps 12:2,\allowbreak3; 55:11,\allowbreak12 Pr 26:24,\allowbreak25 Mic 7:5,\allowbreak6}
\crossref{Jer}{9}{5}{9:5,\allowbreak8 Isa 59:13-\allowbreak15 Mic 6:12 Eph 4:25}
\crossref{Jer}{9}{6}{Jer 11:19; 18:18; 20:10 Ps 120:2-\allowbreak6}
\crossref{Jer}{9}{7}{Jer 6:29,\allowbreak30 Isa 1:25; 48:10 Eze 22:18-\allowbreak22; 26:11,\allowbreak12 Zec 13:9 Mal 3:3}
\crossref{Jer}{9}{8}{9:3,\allowbreak5 Ps 12:2; 57:4; 64:3,\allowbreak4,\allowbreak8; 120:3}
\crossref{Jer}{9}{9}{Jer 5:9,\allowbreak29 Isa 1:24}
\crossref{Jer}{9}{10}{Jer 4:19-\allowbreak26; 7:29; 8:18; 13:16,\allowbreak17 La 1:16; 2:11}
\crossref{Jer}{9}{11}{Jer 26:18; 51:37 Ne 4:2 Ps 79:1 Isa 25:2 Mic 1:6; 3:12}
\crossref{Jer}{9}{12}{De 32:29 Ps 107:43 Ho 14:9 Mt 24:15 Re 1:3}
\crossref{Jer}{9}{13}{Jer 22:9 De 31:16,\allowbreak17 2Ch 7:19 Ezr 9:10 Ps 89:30; 119:53 Pr 28:4}
\crossref{Jer}{9}{14}{Jer 3:17; 7:24 Ge 6:5 Ro 1:21-\allowbreak24 Eph 2:3; 4:17-\allowbreak19}
\crossref{Jer}{9}{15}{Jer 8:14; 23:15; 25:15 Ps 60:3; 69:21; 75:8; 80:5 Isa 2:17,\allowbreak22}
\crossref{Jer}{9}{16}{Jer 13:24 Le 26:33 De 4:27; 28:25,\allowbreak36,\allowbreak64; 32:26 Ne 1:8 Ps 106:27}
\crossref{Jer}{9}{17}{2Ch 35:25 Job 3:8 Ec 12:5 Am 5:16,\allowbreak17 Mt 9:23 Mr 5:38}
\crossref{Jer}{9}{18}{9:10,\allowbreak20}
\crossref{Jer}{9}{19}{Jer 4:31 Eze 7:16-\allowbreak18 Mic 1:8,\allowbreak9}
\crossref{Jer}{9}{20}{Isa 3:16-\allowbreak4:1 32:9-\allowbreak13 Lu 23:27-\allowbreak30}
\crossref{Jer}{9}{21}{Jer 6:11; 15:7 2Ch 36:17 Eze 9:5,\allowbreak6; 21:14,\allowbreak15 Am 6:10,\allowbreak11}
\crossref{Jer}{9}{22}{Jer 7:33; 8:2; 16:4; 25:33 2Ki 9:37 Ps 83:10 Isa 5:25 Zep 1:17}
\crossref{Jer}{9}{23}{Job 5:12-\allowbreak14 Ps 49:10-\allowbreak13,\allowbreak16-\allowbreak18 Ec 2:13-\allowbreak16,\allowbreak19; 9:11 Isa 5:21}
\crossref{Jer}{9}{24}{Jer 4:2 Ps 44:8 Isa 41:16; 45:25 Ro 5:11}
\crossref{Jer}{9}{25}{Eze 28:10; 32:19-\allowbreak32 Am 3:2 Ro 2:8,\allowbreak9,\allowbreak25,\allowbreak26 Ga 5:2-\allowbreak6}
\crossref{Jer}{9}{26}{Jer 25:9-\allowbreak26; 27:3-\allowbreak7; 46:1-\allowbreak52:34 Isa 13:1-\allowbreak24:23 Eze 24:1-\allowbreak32:32}
\crossref{Jer}{10}{1}{Jer 2:4; 13:15-\allowbreak17; 22:2; 42:15 1Ki 22:19 Ps 50:7 Isa 1:10; 28:14}
\crossref{Jer}{10}{2}{Le 18:3; 20:23 De 12:30,\allowbreak31 Eze 20:32}
\crossref{Jer}{10}{3}{10:8; 2:5 Le 18:30 1Ki 18:26-\allowbreak28 Mt 6:7 Ro 1:21 1Pe 1:18}
\crossref{Jer}{10}{4}{Ps 115:4; 135:15 Isa 40:19,\allowbreak20}
\crossref{Jer}{10}{5}{Ps 115:5-\allowbreak8; 135:16-\allowbreak18 Hab 2:19 1Co 12:2 Re 13:14,\allowbreak15}
\crossref{Jer}{10}{6}{Ex 8:10; 9:14; 15:11 De 32:31; 33:26 2Sa 7:22 Ps 35:10; 86:8-\allowbreak10}
\crossref{Jer}{10}{7}{Jer 5:22 Job 37:23,\allowbreak24 Lu 12:5 Re 15:4}
\crossref{Jer}{10}{8}{10:14; 51:17,\allowbreak18 Ps 115:8; 135:18 Isa 41:29 Hab 2:18 Zec 10:2}
\crossref{Jer}{10}{9}{10:4}
\crossref{Jer}{10}{10}{1Ki 18:39 2Ch 15:3 Joh 17:3 1Th 1:9 1Jo 5:20}
\crossref{Jer}{10}{11}{10:15; 51:18 Isa 2:18 Zep 2:11 Zec 13:2 Re 20:2}
\crossref{Jer}{10}{12}{Jer 32:17; 51:15-\allowbreak19 Ge 1:1,\allowbreak6-\allowbreak9 Job 38:4-\allowbreak7 Ps 33:6; 136:5,\allowbreak6}
\crossref{Jer}{10}{13}{Job 37:2-\allowbreak5; 38:34,\allowbreak35 Ps 18:13; 29:3-\allowbreak10; 68:33}
\crossref{Jer}{10}{14}{10:8; 51:17,\allowbreak18 Ps 14:2; 92:6; 94:8 Pr 30:2 Isa 44:18-\allowbreak20; 46:7,\allowbreak8}
\crossref{Jer}{10}{15}{10:8; 8:19; 14:22; 51:18 De 32:21 1Sa 12:21 Isa 41:24,\allowbreak29 Jon 2:8}
\crossref{Jer}{10}{16}{Jer 51:19 Ps 16:5,\allowbreak6; 73:26; 119:57; 142:5 La 3:24}
\crossref{Jer}{10}{17}{Jer 6:1 Eze 12:3-\allowbreak12 Mic 2:10 Mt 24:15}
\crossref{Jer}{10}{18}{Jer 15:1,\allowbreak2; 16:13 De 28:63,\allowbreak64 1Sa 25:29}
\crossref{Jer}{10}{19}{Jer 4:19,\allowbreak31; 8:21; 9:1; 17:13 La 1:2,\allowbreak12-\allowbreak22; 2:11-\allowbreak22; 3:48}
\crossref{Jer}{10}{20}{Jer 4:20 Isa 54:2 La 2:4-\allowbreak6}
\crossref{Jer}{10}{21}{10:8,\allowbreak14; 2:8; 5:31; 8:9; 12:10; 23:9-\allowbreak32 Isa 56:10-\allowbreak12 Eze 22:25-\allowbreak30}
\crossref{Jer}{10}{22}{Jer 1:15; 4:6; 5:15; 6:1,\allowbreak22 Hab 1:6-\allowbreak9}
\crossref{Jer}{10}{23}{Ps 17:5; 37:23; 119:116,\allowbreak117 Pr 16:1; 20:24}
\crossref{Jer}{10}{24}{Jer 30:11}
\crossref{Jer}{10}{25}{Ps 79:6,\allowbreak7}
\crossref{Jer}{11}{1}{11:1}
\crossref{Jer}{11}{2}{11:6; 34:13-\allowbreak16 Ex 19:5 2Ki 11:17; 23:2,\allowbreak3 2Ch 23:16; 29:10}
\crossref{Jer}{11}{3}{De 27:26; 28:15-\allowbreak68; 29:19,\allowbreak20 Ga 3:10-\allowbreak13}
\crossref{Jer}{11}{4}{Jer 31:32 Ex 24:3-\allowbreak8 De 5:2,\allowbreak3; 29:10-\allowbreak15 Eze 20:6-\allowbreak12 Heb 8:8-\allowbreak10}
\crossref{Jer}{11}{5}{Ge 22:16-\allowbreak18; 26:3-\allowbreak5 Ps 105:9-\allowbreak11}
\crossref{Jer}{11}{6}{Jer 3:12; 7:2; 19:2 Isa 58:1 Zec 7:7}
\crossref{Jer}{11}{7}{1Sa 8:9 Eph 4:17 2Th 3:12}
\crossref{Jer}{11}{8}{Jer 3:17; 6:16,\allowbreak17; 7:24,\allowbreak26; 9:13,\allowbreak14; 35:15; 44:17 Ne 9:16,\allowbreak17,\allowbreak26,\allowbreak29}
\crossref{Jer}{11}{9}{Jer 5:31; 6:13; 8:10 Eze 22:25-\allowbreak31 Ho 6:9 Mic 3:11; 7:2,\allowbreak3 Zep 3:1-\allowbreak4}
\crossref{Jer}{11}{10}{Jer 3:10 1Sa 15:11 2Ch 34:30-\allowbreak33 Ho 6:4; 7:16 Zep 1:6}
\crossref{Jer}{11}{11}{11:17; 6:19; 19:3,\allowbreak15; 23:12; 35:17; 36:31 2Ki 22:16 2Ch 34:24 Eze 7:5}
\crossref{Jer}{11}{12}{Jer 2:28; 44:17-\allowbreak27 De 32:37 Jud 10:14 2Ch 28:22 Isa 45:20}
\crossref{Jer}{11}{13}{Jer 2:28; 3:1,\allowbreak2 De 32:16,\allowbreak17 2Ki 23:4,\allowbreak5,\allowbreak13 Isa 2:8 Ho 12:11}
\crossref{Jer}{11}{14}{Jer 7:16; 14:11; 15:1 Ex 32:10 Pr 26:24,\allowbreak25 1Jo 5:16}
\crossref{Jer}{11}{15}{Lu 8:28}
\crossref{Jer}{11}{16}{Ps 52:8 Ro 11:17-\allowbreak24}
\crossref{Jer}{11}{17}{Jer 2:21; 12:2; 24:6; 42:10; 45:4 2Sa 7:10 Ps 44:2; 80:8,\allowbreak15 Isa 5:2}
\crossref{Jer}{11}{18}{11:19 1Sa 23:11,\allowbreak12 2Ki 6:9,\allowbreak10,\allowbreak14-\allowbreak20 Eze 8:6-\allowbreak18 Mt 21:3 Ro 3:7}
\crossref{Jer}{11}{19}{Pr 7:22 Isa 53:7}
\crossref{Jer}{11}{20}{Jer 12:1 Ge 18:25 Ps 98:9 Ac 17:31}
\crossref{Jer}{11}{21}{Jer 12:5,\allowbreak6; 20:10 Mic 7:6 Mt 10:21,\allowbreak34-\allowbreak36 Lu 4:24}
\crossref{Jer}{11}{22}{Jer 9:21; 18:21 2Ch 36:17 La 2:21 1Th 2:15,\allowbreak16}
\crossref{Jer}{11}{23}{11:19; 44:27 Isa 14:20-\allowbreak22}
\crossref{Jer}{12}{1}{Jer 11:20 Ge 18:25 De 32:4 Ps 51:4; 119:75,\allowbreak137; 145:17 Da 9:7}
\crossref{Jer}{12}{2}{Jer 11:17; 45:4 Eze 17:5-\allowbreak10; 19:10-\allowbreak13}
\crossref{Jer}{12}{3}{Jer 11:20 2Ki 20:3 1Ch 29:17 Job 23:10 Ps 17:3; 26:1; 44:21}
\crossref{Jer}{12}{4}{Jer 9:10; 14:2; 23:10}
\crossref{Jer}{12}{5}{Pr 3:11; 24:10 Heb 12:3,\allowbreak4 1Pe 4:12}
\crossref{Jer}{12}{6}{Jer 9:4; 11:19,\allowbreak21; 20:10 Ge 37:4-\allowbreak11 Job 6:15 Ps 69:8 Eze 33:30,\allowbreak31}
\crossref{Jer}{12}{7}{Jer 11:15; 51:5 Isa 2:6 Ps 78:59,\allowbreak60 Ho 9:15 Joe 2:15; 3:2}
\crossref{Jer}{12}{8}{Jer 2:15; 51:38}
\crossref{Jer}{12}{9}{Jer 2:15 2Ki 24:2 Eze 16:36,\allowbreak37; 23:22-\allowbreak25 Re 17:16}
\crossref{Jer}{12}{10}{Jer 6:3; 25:9; 39:3}
\crossref{Jer}{12}{11}{Jer 6:8; 9:11; 10:22,\allowbreak25; 19:8}
\crossref{Jer}{12}{12}{Jer 4:11-\allowbreak15; 9:19-\allowbreak21}
\crossref{Jer}{12}{13}{Le 26:16 De 28:38 Mic 6:15 Hag 1:6; 2:16,\allowbreak17}
\crossref{Jer}{12}{14}{Jer 48:26,\allowbreak27; 50:9-\allowbreak17; 51:33-\allowbreak35 Eze 25:3-\allowbreak15 Am 1:2-\allowbreak15 Zep 2:8-\allowbreak10}
\crossref{Jer}{12}{15}{Jer 48:47; 49:6,\allowbreak39 De 30:3 Isa 23:17,\allowbreak18}
\crossref{Jer}{12}{16}{Jer 4:2; 5:2 De 10:20,\allowbreak21 So 1:8 Isa 9:18-\allowbreak21; 45:23; 65:16 Ro 14:11}
\crossref{Jer}{12}{17}{Ps 2:8-\allowbreak12 Isa 60:12 Zec 14:16-\allowbreak19 Lu 19:27 2Th 1:8 1Pe 2:6-\allowbreak8}
\crossref{Jer}{13}{1}{13:11; 19:1; 27:2 Eze 4:1-\allowbreak5:17 Heb 1:1}
\crossref{Jer}{13}{2}{Pr 3:5 Isa 20:2 Eze 2:8 Ho 1:2,\allowbreak3 Joh 13:6,\allowbreak7; 15:14}
\crossref{Jer}{13}{3}{13:3,\allowbreak8}
\crossref{Jer}{13}{4}{}
\crossref{Jer}{13}{5}{Ex 39:42,\allowbreak43; 40:16 Mt 22:2-\allowbreak6 Joh 2:5-\allowbreak8 Ac 26:19,\allowbreak20 2Ti 2:3}
\crossref{Jer}{13}{6}{13:2-\allowbreak5}
\crossref{Jer}{13}{7}{13:10; 24:1-\allowbreak8 Isa 64:6 Eze 15:3-\allowbreak5 Zec 3:3,\allowbreak4 Lu 14:34,\allowbreak35 Ro 3:12}
\crossref{Jer}{13}{8}{}
\crossref{Jer}{13}{9}{Jer 18:4-\allowbreak6 La 5:5-\allowbreak8}
\crossref{Jer}{13}{10}{Jer 5:23; 7:25-\allowbreak28; 8:5; 11:7,\allowbreak18; 15:1; 25:3-\allowbreak7; 34:14-\allowbreak17 Nu 14:11}
\crossref{Jer}{13}{11}{Ex 19:5,\allowbreak6 De 4:7; 26:18; 32:10-\allowbreak15 Ps 135:4; 147:20}
\crossref{Jer}{13}{12}{Eze 24:19}
\crossref{Jer}{13}{13}{Jer 25:15-\allowbreak18,\allowbreak27; 51:7 Ps 60:3; 75:8 Isa 29:9; 49:26; 51:17,\allowbreak21; 63:6}
\crossref{Jer}{13}{14}{Jer 19:9-\allowbreak11; 48:12 Jud 7:20-\allowbreak22 1Sa 14:16 2Ch 20:23 Ps 2:9}
\crossref{Jer}{13}{15}{Isa 42:23 Joe 1:2 Re 2:29}
\crossref{Jer}{13}{16}{Jos 7:19 1Sa 6:5 Ps 96:7,\allowbreak8}
\crossref{Jer}{13}{17}{Jer 22:5 Mal 2:2}
\crossref{Jer}{13}{18}{Jer 22:26 2Ki 24:12,\allowbreak15 Eze 19:2-\allowbreak14 Jon 3:6}
\crossref{Jer}{13}{19}{Jer 17:26; 33:13 Jos 18:5 Eze 20:46,\allowbreak47}
\crossref{Jer}{13}{20}{Jer 1:14; 6:22; 10:22 Hab 1:6}
\crossref{Jer}{13}{21}{Jer 5:31; 22:23 Isa 10:3 Eze 28:9}
\crossref{Jer}{13}{22}{De 7:17; 8:17; 18:21 Isa 47:8 Zep 1:12 Lu 5:21,\allowbreak22}
\crossref{Jer}{13}{23}{Jer 2:22,\allowbreak30; 5:3; 6:29,\allowbreak30; 17:9 Pr 27:22 Isa 1:5 Mt 19:24-\allowbreak28}
\crossref{Jer}{13}{24}{Le 26:33 De 4:27; 28:64; 32:26 Eze 5:2,\allowbreak12; 6:8; 17:21 Lu 21:24}
\crossref{Jer}{13}{25}{Job 20:29 Ps 11:6 Isa 17:4 Mt 24:51}
\crossref{Jer}{13}{26}{13:22 La 1:8 Eze 16:37; 23:29 Ho 2:10}
\crossref{Jer}{13}{27}{Jer 2:20-\allowbreak24; 3:1,\allowbreak2; 5:7,\allowbreak8 Eze 16:15-\allowbreak22; 23:2-\allowbreak21 Ho 1:2; 4:2 2Co 12:21}
\crossref{Jer}{14}{1}{Jer 17:8}
\crossref{Jer}{14}{2}{Jer 4:28; 12:4 Isa 3:26 Ho 4:3 Joe 1:10}
\crossref{Jer}{14}{3}{1Ki 18:5,\allowbreak6}
\crossref{Jer}{14}{4}{Le 26:19,\allowbreak20 De 28:23,\allowbreak24; 29:23 Joe 1:19,\allowbreak20}
\crossref{Jer}{14}{5}{Job 39:1-\allowbreak4 Ps 29:9}
\crossref{Jer}{14}{6}{Jer 2:24 Job 39:5,\allowbreak6}
\crossref{Jer}{14}{7}{Isa 59:12 Ho 5:5; 7:10}
\crossref{Jer}{14}{8}{Jer 17:13; 50:7 Joe 3:16 Ac 28:20 1Ti 1:1}
\crossref{Jer}{14}{9}{Nu 11:23; 14:15,\allowbreak16 Ps 44:23-\allowbreak26 Isa 50:1,\allowbreak2; 51:9; 59:1}
\crossref{Jer}{14}{10}{Jer 2:23-\allowbreak25,\allowbreak36; 3:1,\allowbreak2; 8:5 Ho 11:7,\allowbreak9}
\crossref{Jer}{14}{11}{Jer 7:16; 11:14; 15:1 Ex 32:32-\allowbreak34}
\crossref{Jer}{14}{12}{Jer 11:11 Pr 1:28; 28:9 Isa 1:15; 58:3 Eze 8:18 Mic 3:4 Zec 7:13}
\crossref{Jer}{14}{13}{Jer 1:6; 4:10}
\crossref{Jer}{14}{14}{Jer 23:25,\allowbreak26; 27:10,\allowbreak14; 28:13; 29:21; 37:19 Isa 9:15 Zec 13:3 1Ti 4:2}
\crossref{Jer}{14}{15}{Jer 5:12,\allowbreak13; 6:15; 8:12; 20:6; 23:14,\allowbreak15; 28:15-\allowbreak17; 29:20,\allowbreak21; 31,\allowbreak32}
\crossref{Jer}{14}{16}{Jer 5:31 Isa 9:16 Mt 15:14}
\crossref{Jer}{14}{17}{Jer 8:18,\allowbreak21; 9:1; 13:17 Ps 80:4,\allowbreak5; 119:136 La 1:16; 2:18; 3:48,\allowbreak49}
\crossref{Jer}{14}{18}{Jer 52:6,\allowbreak7 La 1:20; 4:9 Eze 7:15}
\crossref{Jer}{14}{19}{Jer 6:30; 15:1 2Ki 17:19,\allowbreak20 Ps 78:59; 80:12,\allowbreak13; 89:38 La 5:22}
\crossref{Jer}{14}{20}{Jer 3:13,\allowbreak25 Le 26:40-\allowbreak42 Ezr 9:6,\allowbreak7 Ne 9:2 Ps 32:5; 51:3; 106:6-\allowbreak48}
\crossref{Jer}{14}{21}{14:19 Le 26:11 De 32:19 Ps 51:11; 106:40 La 2:7 Am 6:8}
\crossref{Jer}{14}{22}{Jer 10:15; 16:19 De 32:21 Isa 41:29; 44:12-\allowbreak20}
\crossref{Jer}{15}{1}{Jer 7:16; 11:14; 14:11 Eze 14:14,\allowbreak21}
\crossref{Jer}{15}{2}{Jer 14:12; 24:9,\allowbreak10; 43:11 Isa 24:18 Eze 5:2,\allowbreak12; 14:21 Da 9:12}
\crossref{Jer}{15}{3}{Jer 7:33 Le 26:16,\allowbreak22,\allowbreak25 De 28:26 1Ki 21:23,\allowbreak24 2Ki 9:35-\allowbreak37}
\crossref{Jer}{15}{4}{2Ki 21:11-\allowbreak13; 23:26,\allowbreak27; 24:3,\allowbreak4}
\crossref{Jer}{15}{5}{Jer 16:5; 21:7 Job 19:21 Ps 69:20 Isa 51:19 La 1:12-\allowbreak16; 2:15,\allowbreak16}
\crossref{Jer}{15}{6}{Jer 1:16; 2:13,\allowbreak17,\allowbreak19}
\crossref{Jer}{15}{7}{Jer 4:11,\allowbreak12; 51:2 Ps 1:4 Isa 41:16 Mt 3:12}
\crossref{Jer}{15}{8}{Isa 3:25,\allowbreak26; 4:1}
\crossref{Jer}{15}{9}{Am 8:9,\allowbreak10}
\crossref{Jer}{15}{10}{Jer 20:14-\allowbreak18 Job 3:1-\allowbreak26}
\crossref{Jer}{15}{11}{Ps 37:3-\allowbreak11 Ec 8:12}
\crossref{Jer}{15}{12}{Jer 1:18,\allowbreak19; 21:4,\allowbreak5 Job 40:9 Isa 45:9 Hab 1:5-\allowbreak10}
\crossref{Jer}{15}{13}{15:8; 17:3; 20:5}
\crossref{Jer}{15}{14}{15:4; 14:18; 16:13; 17:4; 52:27 Le 26:38,\allowbreak39 De 28:25,\allowbreak36,\allowbreak64 Am 5:27}
\crossref{Jer}{15}{15}{Jer 12:3; 17:16 Job 10:7 Ps 7:3-\allowbreak5; 17:3 Joh 21:15-\allowbreak17 2Co 5:11}
\crossref{Jer}{15}{16}{Eze 3:1-\allowbreak3 Re 10:9}
\crossref{Jer}{15}{17}{Ps 1:1; 26:4,\allowbreak5 2Co 6:17}
\crossref{Jer}{15}{18}{Jer 14:19 Ps 6:3; 13:1-\allowbreak3 La 3:1-\allowbreak18}
\crossref{Jer}{15}{19}{15:10-\allowbreak18; 20:9 Ex 6:29,\allowbreak30 Jon 3:2}
\crossref{Jer}{15}{20}{Jer 1:18,\allowbreak19; 6:27 Eze 3:9 Ac 4:8-\allowbreak13,\allowbreak29-\allowbreak31; 5:29-\allowbreak32}
\crossref{Jer}{15}{21}{Ge 48:16 Ps 27:2; 37:40 Isa 49:24,\allowbreak25; 54:17 Mt 6:13 Ro 16:20}
\crossref{Jer}{16}{1}{Jer 1:2,\allowbreak4; 2:1}
\crossref{Jer}{16}{2}{Ge 19:14 Mt 24:19 Lu 21:23; 23:29 1Co 7:26,\allowbreak27}
\crossref{Jer}{16}{3}{16:5,\allowbreak9}
\crossref{Jer}{16}{4}{Jer 14:16; 15:2,\allowbreak3 Ps 78:64}
\crossref{Jer}{16}{5}{16:6,\allowbreak7 Eze 24:16-\allowbreak23}
\crossref{Jer}{16}{6}{Jer 13:13 Isa 9:14-\allowbreak17; 24:2 Eze 9:5,\allowbreak6 Am 6:11 Re 6:15; 20:12}
\crossref{Jer}{16}{7}{De 26:14 Job 42:11 Eze 24:17 Ho 9:4}
\crossref{Jer}{16}{8}{Jer 15:17 Ps 26:4 Ec 7:2-\allowbreak4 Isa 22:12-\allowbreak14 Am 6:4-\allowbreak6 Mt 24:38}
\crossref{Jer}{16}{9}{Jer 7:34; 25:10 Isa 24:7-\allowbreak12 Eze 26:13 Ho 2:11 Re 18:22,\allowbreak23}
\crossref{Jer}{16}{10}{Jer 2:35; 5:19; 13:22; 22:8,\allowbreak9 De 29:24,\allowbreak25 1Ki 9:8,\allowbreak9 Ho 12:8}
\crossref{Jer}{16}{11}{Jer 2:8; 5:7-\allowbreak9 Jud 2:12,\allowbreak13; 10:13,\allowbreak14 Ne 9:26-\allowbreak29 Ps 106:35-\allowbreak41}
\crossref{Jer}{16}{12}{Jer 7:26; 13:10 2Ti 3:13}
\crossref{Jer}{16}{13}{Jer 6:15; 15:4,\allowbreak14; 17:4 Le 18:27,\allowbreak28 De 4:26-\allowbreak28; 28:36,\allowbreak63-\allowbreak65; 29:28}
\crossref{Jer}{16}{14}{Jer 23:7,\allowbreak8 Isa 43:18,\allowbreak19 Ho 3:4,\allowbreak5}
\crossref{Jer}{16}{15}{Jer 3:18; 24:6; 30:3,\allowbreak10; 31:8; 32:37; 50:19 De 30:3-\allowbreak5 Ps 106:47}
\crossref{Jer}{16}{16}{Ge 10:9 1Sa 24:11; 26:20 Mic 7:2}
\crossref{Jer}{16}{17}{Jer 23:24; 32:19 2Ch 16:9 Job 34:21,\allowbreak22 Ps 90:8; 139:3 Pr 5:21}
\crossref{Jer}{16}{18}{Jer 17:18 Isa 40:2; 61:7 Re 18:6}
\crossref{Jer}{16}{19}{Jer 17:17 Ps 18:1,\allowbreak2; 19:14; 27:5; 46:1,\allowbreak7,\allowbreak11; 62:2,\allowbreak7; 91:1,\allowbreak2; 144:1,\allowbreak2}
\crossref{Jer}{16}{20}{Ps 115:4-\allowbreak8; 135:14-\allowbreak18 Isa 36:19; 37:19 Ho 8:4-\allowbreak6 Ac 19:26 Ga 1:8}
\crossref{Jer}{16}{21}{Ex 9:14-\allowbreak18; 14:4 Ps 9:16 Eze 6:7; 24:24,\allowbreak27; 25:14}
\crossref{Jer}{17}{1}{Job 19:23,\allowbreak24}
\crossref{Jer}{17}{2}{Jer 7:18 Ho 4:13,\allowbreak14}
\crossref{Jer}{17}{3}{Jer 26:18 Isa 2:2,\allowbreak3 La 5:17,\allowbreak18 Mic 3:12; 4:1,\allowbreak2}
\crossref{Jer}{17}{4}{Jer 16:13; 25:9-\allowbreak11 Le 26:31-\allowbreak34 De 4:26,\allowbreak27; 28:25 Jos 23:15,\allowbreak16}
\crossref{Jer}{17}{5}{Ps 62:9; 118:8,\allowbreak9; 146:3,\allowbreak4 Isa 2:22; 30:1-\allowbreak7; 31:1-\allowbreak9; 36:6}
\crossref{Jer}{17}{6}{Jer 48:6 Job 8:11-\allowbreak13; 15:30-\allowbreak34 Ps 1:4; 92:7; 129:6-\allowbreak8 Isa 1:30}
\crossref{Jer}{17}{7}{Ps 2:12; 34:8; 84:12; 125:1; 146:5 Pr 16:20 Isa 26:3,\allowbreak4; 30:18}
\crossref{Jer}{17}{8}{Job 8:16 Ps 1:3; 92:10-\allowbreak15 Isa 58:11 Eze 31:4-\allowbreak10; 47:12}
\crossref{Jer}{17}{9}{Jer 16:12 Ge 6:5; 8:21 Job 15:14-\allowbreak16 Ps 51:5; 53:1-\allowbreak3 Pr 28:26}
\crossref{Jer}{17}{10}{Jer 11:20; 20:12 1Sa 16:7 1Ch 28:9; 29:17 2Ch 6:30 Ps 7:9}
\crossref{Jer}{17}{11}{Ps 55:23 Pr 23:5 Ec 5:13-\allowbreak16}
\crossref{Jer}{17}{12}{Jer 3:17; 14:21 2Ch 2:5,\allowbreak6 Ps 96:6; 103:19 Isa 6:1; 66:1 Eze 1:26}
\crossref{Jer}{17}{13}{17:17; 14:8 Ps 22:4 Joe 3:16 Ac 28:20 1Ti 1:1}
\crossref{Jer}{17}{14}{Jer 31:18 De 32:39 Ps 6:2,\allowbreak4; 12:4 Isa 6:10; 57:18,\allowbreak19 Lu 4:18}
\crossref{Jer}{17}{15}{Jer 20:7,\allowbreak8 Isa 5:19 Eze 12:22,\allowbreak27,\allowbreak28 Am 5:18 2Pe 3:3,\allowbreak4}
\crossref{Jer}{17}{16}{Jer 1:4-\allowbreak10; 20:9 Eze 3:14-\allowbreak19; 33:7-\allowbreak9 Am 7:14,\allowbreak15 Jas 1:19; 3:1}
\crossref{Jer}{17}{17}{Job 31:23 Ps 77:2-\allowbreak9; 88:15,\allowbreak16}
\crossref{Jer}{17}{18}{Jer 20:11 Ps 35:4,\allowbreak26,\allowbreak27; 40:14; 70:2; 83:17,\allowbreak18}
\crossref{Jer}{17}{19}{Jer 7:2; 19:2; 26:2; 36:6,\allowbreak10 Pr 1:20-\allowbreak22; 8:1; 9:3 Ac 5:20}
\crossref{Jer}{17}{20}{Jer 13:18; 19:3; 22:2 Ps 49:1,\allowbreak2 Eze 2:7; 3:17 Ho 5:1 Am 4:1}
\crossref{Jer}{17}{21}{De 4:9,\allowbreak15,\allowbreak23; 11:16 Jos 23:11 Pr 4:23 Mr 4:24 Lu 8:18 Ac 20:28}
\crossref{Jer}{17}{22}{Ge 2:2,\allowbreak3 Ex 16:23-\allowbreak29; 20:8-\allowbreak10; 23:12; 31:13-\allowbreak17 Le 19:3; 23:3}
\crossref{Jer}{17}{23}{Jer 7:24-\allowbreak26; 11:10; 16:11,\allowbreak12; 19:15 Isa 48:4 Eze 20:13,\allowbreak16,\allowbreak21}
\crossref{Jer}{17}{24}{Ex 15:26 De 11:13,\allowbreak22 Isa 21:7; 55:2 Zec 6:15 2Pe 1:5-\allowbreak10}
\crossref{Jer}{17}{25}{Jer 22:4}
\crossref{Jer}{17}{26}{Jer 32:44; 33:13 Jos 15:21-\allowbreak63}
\crossref{Jer}{17}{27}{17:24; 6:17; 26:4-\allowbreak6; 44:16 Isa 1:20 Zec 7:11-\allowbreak14 Heb 12:25}
\crossref{Jer}{18}{1}{18:1}
\crossref{Jer}{18}{2}{Jer 13:1; 19:1,\allowbreak2 Isa 20:2 Eze 4:1-\allowbreak5:1 Am 7:7 Heb 1:1}
\crossref{Jer}{18}{3}{Jon 1:3 Joh 15:14 Ac 26:19}
\crossref{Jer}{18}{4}{}
\crossref{Jer}{18}{5}{}
\crossref{Jer}{18}{6}{18:4 Isa 64:8 Da 4:23 Mt 20:15 Ro 11:34}
\crossref{Jer}{18}{7}{Jer 1:10; 12:14-\allowbreak17; 25:9-\allowbreak14; 45:4 Am 9:8 Jon 3:4}
\crossref{Jer}{18}{8}{Jer 7:3-\allowbreak7; 36:3 Jud 10:15,\allowbreak16 1Ki 8:33,\allowbreak34 2Ch 12:6 Isa 1:16-\allowbreak19}
\crossref{Jer}{18}{9}{Jer 1:10; 11:17; 30:18; 31:4,\allowbreak28,\allowbreak38; 32:41 Ec 3:2 Am 9:11-\allowbreak15}
\crossref{Jer}{18}{10}{Jer 7:23-\allowbreak28 Ps 125:5 Eze 18:24; 33:18; 45:20 Zep 1:6}
\crossref{Jer}{18}{11}{Ge 11:3,\allowbreak4,\allowbreak7 2Ki 5:5 Isa 5:5 Jas 4:13; 5:1}
\crossref{Jer}{18}{12}{Jer 2:25 2Ki 6:33 Isa 57:10 Eze 37:11}
\crossref{Jer}{18}{13}{Jer 2:10-\allowbreak13}
\crossref{Jer}{18}{14}{Joh 6:68}
\crossref{Jer}{18}{15}{Jer 2:13,\allowbreak19,\allowbreak32; 3:21; 13:25; 17:13}
\crossref{Jer}{18}{16}{Jer 9:11; 19:8; 25:9; 49:13; 50:13 Le 26:33,\allowbreak34,\allowbreak43 De 29:23 Isa 6:11}
\crossref{Jer}{18}{17}{Jer 13:24 De 28:25,\allowbreak64 Job 27:21 Ps 48:7 Ho 13:15}
\crossref{Jer}{18}{18}{18:11; 11:19 Ps 21:11 Isa 32:7 Mic 2:1-\allowbreak3}
\crossref{Jer}{18}{19}{Jer 20:12 Ps 55:16,\allowbreak17; 64:1-\allowbreak4; 56:1-\allowbreak3; 109:4,\allowbreak28 Mic 7:8 Lu 6:11,\allowbreak12}
\crossref{Jer}{18}{20}{1Sa 24:17-\allowbreak19 Ps 35:12; 38:20; 109:4,\allowbreak5 Pr 17:13 Joh 10:32; 15:25}
\crossref{Jer}{18}{21}{Jer 11:20-\allowbreak23; 12:3; 20:1-\allowbreak6,\allowbreak11,\allowbreak12 Ps 109:9-\allowbreak20 2Ti 4:14}
\crossref{Jer}{18}{22}{Jer 4:19,\allowbreak20,\allowbreak31; 6:26; 9:20,\allowbreak21; 25:34-\allowbreak36; 47:2,\allowbreak3; 48:3-\allowbreak5 Isa 10:30}
\crossref{Jer}{18}{23}{18:18; 11:18-\allowbreak20; 15:15 Ps 37:32,\allowbreak33}
\crossref{Jer}{19}{1}{19:10,\allowbreak11; 18:2-\allowbreak4; 32:14 Isa 30:14}
\crossref{Jer}{19}{2}{Jer 7:31,\allowbreak32; 32:35 Jos 15:8 2Ki 23:10 2Ch 28:3; 33:6}
\crossref{Jer}{19}{3}{Jer 13:18; 17:20 Ps 2:10; 102:15; 110:5 Mt 10:18 Re 2:29}
\crossref{Jer}{19}{4}{Jer 2:13,\allowbreak17,\allowbreak19,\allowbreak34; 5:6; 15:6; 16:11; 17:13 De 28:20; 31:16-\allowbreak18}
\crossref{Jer}{19}{5}{Nu 22:41}
\crossref{Jer}{19}{6}{19:2,\allowbreak11; 7:32,\allowbreak33 Jos 15:8 Isa 30:33}
\crossref{Jer}{19}{7}{Job 5:12,\allowbreak13 Ps 33:10,\allowbreak11 Pr 21:30 Isa 8:10; 28:17,\allowbreak18; 30:1-\allowbreak3}
\crossref{Jer}{19}{8}{Jer 9:9-\allowbreak11; 18:16; 25:18; 49:13; 50:13 Le 26:32 1Ki 9:8 2Ch 7:20,\allowbreak21}
\crossref{Jer}{19}{9}{Le 26:29 De 28:53-\allowbreak57 2Ki 6:26-\allowbreak29 Isa 9:20 La 2:20; 4:10}
\crossref{Jer}{19}{10}{Jer 48:12; 51:63,\allowbreak64}
\crossref{Jer}{19}{11}{Jer 13:14 Ps 2:9 Isa 30:14 La 4:2 Re 2:27}
\crossref{Jer}{19}{12}{Jer 10:13; 11:5}
\crossref{Jer}{19}{13}{2Ki 23:10,\allowbreak12,\allowbreak14 Ps 74:7; 79:1 Eze 7:21,\allowbreak22}
\crossref{Jer}{19}{14}{19:2,\allowbreak3}
\crossref{Jer}{19}{15}{Jer 7:26; 17:23; 35:15-\allowbreak17 2Ch 36:16,\allowbreak17 Ne 9:17,\allowbreak29 Zec 7:11-\allowbreak14}
\crossref{Jer}{20}{1}{1Ch 24:14 Ezr 2:37,\allowbreak38 Ne 7:40,\allowbreak41}
\crossref{Jer}{20}{2}{Jer 1:19; 19:14,\allowbreak15; 26:8; 29:26; 36:26; 37:15,\allowbreak16; 38:6 1Ki 22:27}
\crossref{Jer}{20}{3}{Ac 4:5-\allowbreak7; 16:30,\allowbreak35-\allowbreak39}
\crossref{Jer}{20}{4}{De 28:65-\allowbreak67 Job 18:11-\allowbreak21; 20:23-\allowbreak26 Ps 73:19 Eze 26:17-\allowbreak21}
\crossref{Jer}{20}{5}{Jer 3:24; 4:20; 12:12; 15:13; 24:8-\allowbreak10; 27:19-\allowbreak22; 32:3-\allowbreak5; 39:2,\allowbreak8; 52:7-\allowbreak23}
\crossref{Jer}{20}{6}{Jer 28:15-\allowbreak17; 29:21,\allowbreak22,\allowbreak32 Ac 13:8-\allowbreak11}
\crossref{Jer}{20}{7}{Jer 1:6-\allowbreak8,\allowbreak18,\allowbreak19; 15:18; 17:16 Ex 5:22,\allowbreak23 Nu 11:11-\allowbreak15}
\crossref{Jer}{20}{8}{Jer 4:19-\allowbreak22; 5:1,\allowbreak6,\allowbreak15-\allowbreak17; 6:6,\allowbreak7; 7:9; 13:13,\allowbreak14; 15:1-\allowbreak4,\allowbreak13,\allowbreak14; 17:27}
\crossref{Jer}{20}{9}{1Ki 19:3,\allowbreak4 Joh 1:2,\allowbreak3; 4:2,\allowbreak3 Lu 9:62 Ac 15:37,\allowbreak38}
\crossref{Jer}{20}{10}{Ps 31:13; 57:4; 64:2-\allowbreak4 Mt 26:59,\allowbreak60}
\crossref{Jer}{20}{11}{Jer 1:8,\allowbreak19; 15:20 Isa 41:10,\allowbreak14 Ro 8:31 2Ti 4:17}
\crossref{Jer}{20}{12}{Jer 17:10 Ps 7:9; 11:5; 17:3; 26:2,\allowbreak3; 139:23 Re 2:23}
\crossref{Jer}{20}{13}{Ps 34:6; 35:9-\allowbreak11; 69:33; 72:4; 109:30,\allowbreak31 Isa 25:4 Jas 2:5,\allowbreak6}
\crossref{Jer}{20}{14}{Jer 15:10 Job 3:3-\allowbreak16}
\crossref{Jer}{20}{15}{Jer 1:5 Ge 21:5,\allowbreak6 Lu 1:14}
\crossref{Jer}{20}{16}{Ge 19:24,\allowbreak25 De 29:23 Ho 11:8 Am 4:11 Zep 2:9 Lu 17:29 2Pe 2:6}
\crossref{Jer}{20}{17}{Job 3:10,\allowbreak11,\allowbreak16; 10:18,\allowbreak19 Ec 6:3}
\crossref{Jer}{20}{18}{Job 3:20; 14:1,\allowbreak13 La 3:1}
\crossref{Jer}{21}{1}{Jer 32:1-\allowbreak3; 37:1; 52:1-\allowbreak3 2Ki 24:17,\allowbreak18 1Ch 3:15 2Ch 36:10-\allowbreak13}
\crossref{Jer}{21}{2}{Jer 37:3,\allowbreak7; 38:14-\allowbreak27; 42:4-\allowbreak6 Jud 20:27 1Sa 10:22; 28:6,\allowbreak15}
\crossref{Jer}{21}{3}{}
\crossref{Jer}{21}{4}{Jer 32:5; 33:5; 37:8-\allowbreak10; 38:2,\allowbreak3,\allowbreak17,\allowbreak18; 52:18 Isa 10:4 Ho 9:12}
\crossref{Jer}{21}{5}{Isa 63:10 La 2:4,\allowbreak5}
\crossref{Jer}{21}{6}{Jer 7:20; 12:3,\allowbreak4; 33:12; 36:29 Ge 6:7 Isa 6:11; 24:1-\allowbreak6 Eze 14:13,\allowbreak17}
\crossref{Jer}{21}{7}{Jer 24:8-\allowbreak10; 34:19-\allowbreak22; 37:17; 38:21-\allowbreak23; 39:4-\allowbreak7; 52:8-\allowbreak11,\allowbreak24-\allowbreak27}
\crossref{Jer}{21}{8}{De 11:26; 30:15,\allowbreak19 Isa 1:19,\allowbreak20}
\crossref{Jer}{21}{9}{21:7; 27:13; 38:2,\allowbreak17-\allowbreak23}
\crossref{Jer}{21}{10}{Jer 44:11,\allowbreak27 Le 17:10; 20:3-\allowbreak5; 26:17 Ps 34:16 Eze 15:7 Am 9:4}
\crossref{Jer}{21}{11}{Jer 13:18; 17:20 Mic 3:1}
\crossref{Jer}{21}{12}{Isa 7:2,\allowbreak13 Lu 1:69}
\crossref{Jer}{21}{13}{21:5; 23:30-\allowbreak32; 50:31; 51:25 Ex 13:8,\allowbreak20}
\crossref{Jer}{21}{14}{Jer 9:25; 11:22 Isa 10:12; 24:21}
\crossref{Jer}{22}{1}{Jer 21:11; 34:2 1Sa 15:16-\allowbreak23 2Sa 12:1; 24:11,\allowbreak12 1Ki 21:18-\allowbreak20}
\crossref{Jer}{22}{2}{22:29; 13:18; 17:20-\allowbreak27; 19:3; 29:20 1Ki 22:19 Isa 1:10; 28:14}
\crossref{Jer}{22}{3}{Jer 5:28; 9:24}
\crossref{Jer}{22}{4}{Jer 17:25}
\crossref{Jer}{22}{5}{Jer 17:27 2Ch 7:19,\allowbreak22 Isa 1:20}
\crossref{Jer}{22}{6}{22:24; 21:11 Ge 37:25 De 3:25 So 5:15}
\crossref{Jer}{22}{7}{Jer 4:6,\allowbreak7; 5:15; 50:20-\allowbreak23 Isa 10:3-\allowbreak7; 13:3-\allowbreak5; 54:16,\allowbreak17 Eze 9:1-\allowbreak7}
\crossref{Jer}{22}{8}{De 29:23-\allowbreak25 1Ki 9:8,\allowbreak9 2Ch 7:20-\allowbreak22 La 2:15-\allowbreak17; 4:12 Da 9:7}
\crossref{Jer}{22}{9}{Jer 2:17-\allowbreak19; 40:2,\allowbreak3; 50:7 De 29:25-\allowbreak28 2Ki 22:17 2Ch 34:25}
\crossref{Jer}{22}{10}{2Ki 22:20; 23:30 2Ch 35:23-\allowbreak25 Ec 4:2 Isa 57:1 La 4:9 Lu 23:28}
\crossref{Jer}{22}{11}{1Ch 3:15 2Ch 28:12; 34:22; 36:1-\allowbreak4}
\crossref{Jer}{22}{12}{22:18 2Ki 23:34}
\crossref{Jer}{22}{13}{22:18 2Ki 23:35-\allowbreak37 2Ch 36:4}
\crossref{Jer}{22}{14}{Pr 17:19; 24:27 Isa 5:8,\allowbreak9; 9:9 Da 4:30 Mal 1:4 Lu 14:28,\allowbreak29}
\crossref{Jer}{22}{15}{22:18 2Ki 23:25 1Ch 3:15}
\crossref{Jer}{22}{16}{Jer 5:28 Job 29:12-\allowbreak17 Ps 72:1-\allowbreak4,\allowbreak12,\allowbreak13; 82:3,\allowbreak4; 109:31 Pr 24:11,\allowbreak12}
\crossref{Jer}{22}{17}{Jos 7:21 Job 31:7 Ps 119:36,\allowbreak37 Eze 19:6; 33:31 Mr 7:21,\allowbreak22}
\crossref{Jer}{22}{18}{22:10; 16:4,\allowbreak6 2Ch 21:19,\allowbreak20; 35:25}
\crossref{Jer}{22}{19}{Jer 15:3; 36:6,\allowbreak30 1Ki 14:10; 21:23,\allowbreak24 2Ki 9:35 2Ch 36:6}
\crossref{Jer}{22}{20}{Jer 2:36,\allowbreak37; 30:13-\allowbreak15 2Ki 24:7 Isa 20:5,\allowbreak6; 30:1-\allowbreak7; 31:1-\allowbreak3}
\crossref{Jer}{22}{21}{Jer 2:31; 6:16; 35:15; 36:21-\allowbreak26 2Ch 33:10; 36:16,\allowbreak17 Pr 30:9}
\crossref{Jer}{22}{22}{Jer 4:11-\allowbreak13; 30:23,\allowbreak24 Isa 64:6 Ho 4:19; 13:15}
\crossref{Jer}{22}{23}{22:6 Zec 11:1,\allowbreak2}
\crossref{Jer}{22}{24}{22:28; 37:1 2Ki 24:6-\allowbreak8}
\crossref{Jer}{22}{25}{22:28; 21:7; 34:20,\allowbreak21; 38:16 2Ki 24:15,\allowbreak16}
\crossref{Jer}{22}{26}{Jer 15:2-\allowbreak4 2Ki 24:15 2Ch 36:9,\allowbreak10 Isa 22:17 Eze 19:9-\allowbreak14}
\crossref{Jer}{22}{27}{22:11; 44:14; 52:31-\allowbreak34 2Ki 25:27-\allowbreak30}
\crossref{Jer}{22}{28}{22:24}
\crossref{Jer}{22}{29}{Jer 6:19 De 4:26; 31:19; 32:1 Isa 1:1,\allowbreak2; 34:1 Mic 1:2; 6:1,\allowbreak2}
\crossref{Jer}{22}{30}{Jer 36:30 Ps 94:20 Lu 1:32,\allowbreak33 Mt 1:11,\allowbreak12}
\crossref{Jer}{23}{1}{Jer 2:8,\allowbreak26 Eze 13:3; 34:2 Zec 11:17 Mt 23:13-\allowbreak29 Lu 11:42-\allowbreak52}
\crossref{Jer}{23}{2}{Mt 25:36,\allowbreak43 Jas 1:27}
\crossref{Jer}{23}{3}{Jer 29:14; 30:3; 31:8; 32:37 De 30:3-\allowbreak5 Ps 106:47 Isa 11:11-\allowbreak16}
\crossref{Jer}{23}{4}{Jer 3:14,\allowbreak15; 33:26 Ps 78:70-\allowbreak72 Isa 11:11 Eze 34:23-\allowbreak31 Ho 3:3-\allowbreak5}
\crossref{Jer}{23}{5}{Jer 30:3; 31:27,\allowbreak31-\allowbreak38; 33:14 Heb 8:8}
\crossref{Jer}{23}{6}{De 33:28,\allowbreak29 Ps 130:7,\allowbreak8 Isa 12:1,\allowbreak2; 33:22; 45:17 Eze 37:24-\allowbreak28}
\crossref{Jer}{23}{7}{23:3}
\crossref{Jer}{23}{8}{23:3 Isa 14:1; 27:12,\allowbreak13; 43:5,\allowbreak6; 65:8-\allowbreak10 Eze 34:13; 36:24; 37:25}
\crossref{Jer}{23}{9}{Jer 9:1; 14:17,\allowbreak18 2Ki 22:19,\allowbreak20 Eze 9:4,\allowbreak6 Da 8:27 Hab 3:16}
\crossref{Jer}{23}{10}{Jer 5:7,\allowbreak8; 7:9; 9:2 Eze 22:9-\allowbreak11 Ho 4:2,\allowbreak3 Mal 3:5 1Co 6:9,\allowbreak10}
\crossref{Jer}{23}{11}{23:15; 5:31; 6:13; 8:10 Eze 22:25,\allowbreak26 Zep 3:4}
\crossref{Jer}{23}{12}{Jer 13:16 Ps 35:6; 73:18 Pr 4:19}
\crossref{Jer}{23}{13}{Ho 9:7,\allowbreak8}
\crossref{Jer}{23}{14}{Jer 5:30,\allowbreak31; 14:14; 23:32 Eze 13:2-\allowbreak4,\allowbreak16; 22:25 Isa 41:6,\allowbreak7 Mic 3:11}
\crossref{Jer}{23}{15}{Jer 8:14; 9:15 Ps 69:21 La 3:5,\allowbreak15,\allowbreak19 Mt 27:34 Re 8:11}
\crossref{Jer}{23}{16}{Jer 27:9,\allowbreak10,\allowbreak14-\allowbreak17; 29:8 Pr 19:27 Mt 7:15 2Co 11:13-\allowbreak15 Ga 1:8,\allowbreak9}
\crossref{Jer}{23}{17}{Nu 11:20 1Sa 2:30 2Sa 12:10 Mal 1:6 Lu 10:16 1Th 4:8}
\crossref{Jer}{23}{18}{23:22 1Ki 22:24 Job 15:8-\allowbreak10 2Ch 18:23 Isa 40:13,\allowbreak14 1Co 2:16}
\crossref{Jer}{23}{19}{Jer 4:11; 25:32; 30:23 Ps 58:9 Pr 1:27; 10:25 Isa 5:25-\allowbreak28; 21:1}
\crossref{Jer}{23}{20}{Jer 30:24 Isa 14:24 Zec 1:6; 8:14,\allowbreak15}
\crossref{Jer}{23}{21}{23:32; 14:14; 27:15; 28:15; 29:9,\allowbreak31 Isa 6:8 Joh 20:21 Ac 13:4}
\crossref{Jer}{23}{22}{23:18 Eze 2:7; 3:17 Ac 20:27}
\crossref{Jer}{23}{23}{1Ki 20:23,\allowbreak28 Ps 113:5; 139:1-\allowbreak10 Eze 20:32-\allowbreak35 Jon 1:3,\allowbreak4}
\crossref{Jer}{23}{24}{Jer 49:10 Ge 16:13 Job 22:13,\allowbreak14; 24:13-\allowbreak16 Ps 10:11; 90:8}
\crossref{Jer}{23}{25}{Jer 8:6; 13:27; 16:17; 29:23 Ps 139:2,\allowbreak4 Lu 12:3 1Co 4:5 Heb 4:13}
\crossref{Jer}{23}{26}{Jer 4:14; 13:27 Ps 4:2 Ho 8:5 Ac 13:10}
\crossref{Jer}{23}{27}{De 13:1-\allowbreak5 Ac 13:8 2Ti 2:17,\allowbreak18; 3:6-\allowbreak8}
\crossref{Jer}{23}{28}{Pr 14:5 Mt 24:45 Lu 12:42 1Co 4:2 2Co 2:17 1Ti 1:12}
\crossref{Jer}{23}{29}{Jer 5:14; 20:9 Lu 24:32 Joh 6:63 Ac 2:3,\allowbreak37 2Co 2:16; 10:4,\allowbreak5}
\crossref{Jer}{23}{30}{Jer 14:14,\allowbreak15; 44:11,\allowbreak29 Le 20:3; 26:17 De 18:20; 29:20 Ps 34:16}
\crossref{Jer}{23}{31}{Isa 30:10 Mic 2:11}
\crossref{Jer}{23}{32}{23:16; 27:14-\allowbreak22; 28:15-\allowbreak17; 29:21-\allowbreak23,\allowbreak31 De 13:1-\allowbreak18; 18:20 Isa 3:12}
\crossref{Jer}{23}{33}{Jer 17:15; 20:7,\allowbreak8 Isa 13:1; 14:28 Na 1:1 Hab 1:1 Mal 1:1}
\crossref{Jer}{23}{34}{23:2}
\crossref{Jer}{23}{35}{Jer 31:34 Heb 8:11}
\crossref{Jer}{23}{36}{Ps 12:3; 64:8; 120:3; 149:9 Pr 17:20 Isa 3:8 Mt 12:36 Lu 19:22}
\crossref{Jer}{23}{37}{23:35 Jer 33:3; 42:4}
\crossref{Jer}{23}{38}{2Ch 11:13,\allowbreak14}
\crossref{Jer}{23}{39}{Ge 6:17 Le 26:28 De 32:39 Isa 48:15; 51:12 Eze 5:8; 6:3}
\crossref{Jer}{23}{40}{Jer 20:11; 24:9; 42:18; 44:8-\allowbreak12 De 28:37 Eze 5:14,\allowbreak15 Da 9:16; 12:2}
\crossref{Jer}{24}{1}{Am 3:7; 7:1,\allowbreak4,\allowbreak7; 8:1 Zec 1:20; 3:1}
\crossref{Jer}{24}{2}{24:5-\allowbreak7 Ho 9:10 Mic 7:1}
\crossref{Jer}{24}{3}{Jer 1:11-\allowbreak14 1Sa 9:9 Am 7:8; 8:2 Zec 4:2; 5:2,\allowbreak5-\allowbreak11 Mt 25:32,\allowbreak33}
\crossref{Jer}{24}{4}{}
\crossref{Jer}{24}{5}{Na 1:7 Zec 13:9 Mt 25:12 Joh 10:27 1Co 8:3 Ga 4:9 2Ti 2:19}
\crossref{Jer}{24}{6}{Jer 21:10 De 11:12 2Ch 16:9 Ne 5:19 Job 33:27,\allowbreak28 Ps 34:15}
\crossref{Jer}{24}{7}{Jer 31:33,\allowbreak34; 32:39 De 30:6 Eze 11:19,\allowbreak20; 36:24-\allowbreak28}
\crossref{Jer}{24}{8}{24:2,\allowbreak5; 29:16-\allowbreak18}
\crossref{Jer}{24}{9}{Jer 15:4; 34:17 De 28:25,\allowbreak37,\allowbreak65-\allowbreak67 Eze 5:1,\allowbreak2,\allowbreak12,\allowbreak13}
\crossref{Jer}{24}{10}{Jer 5:12; 9:16; 14:15,\allowbreak16; 15:2; 16:4; 19:7; 34:17 Isa 51:19}
\crossref{Jer}{25}{1}{Jer 36:1; 46:2 2Ki 24:1,\allowbreak2 Da 1:1}
\crossref{Jer}{25}{2}{Jer 18:11; 19:14,\allowbreak15; 26:2; 35:13; 38:1,\allowbreak2 Ps 49:1,\allowbreak2 Mr 7:14-\allowbreak16}
\crossref{Jer}{25}{3}{Jer 1:2 1Ki 22:3 2Ch 34:3,\allowbreak8}
\crossref{Jer}{25}{4}{Jer 7:25; 11:7; 26:5; 29:19; 32:33; 35:14,\allowbreak15; 44:4,\allowbreak5 2Ch 36:15,\allowbreak16}
\crossref{Jer}{25}{5}{Jer 18:11; 35:15 2Ki 17:13,\allowbreak14 Isa 55:6,\allowbreak7 Eze 18:30; 33:11}
\crossref{Jer}{25}{6}{Jer 7:6,\allowbreak9; 35:15 Ex 20:3,\allowbreak23 De 6:14; 8:19; 13:2; 28:14 Jos 24:20}
\crossref{Jer}{25}{7}{Jer 7:18,\allowbreak19; 32:30-\allowbreak33 De 32:21 2Ki 17:17; 21:15 Ne 9:26 Pr 8:36}
\crossref{Jer}{25}{8}{Jer 6:6}
\crossref{Jer}{25}{9}{Jer 1:15; 5:15,\allowbreak16; 6:1,\allowbreak22-\allowbreak26; 8:16 Le 26:25-\allowbreak46 De 28:45-\allowbreak50 Pr 21:1}
\crossref{Jer}{25}{10}{Es 3:13; 7:4; 8:11}
\crossref{Jer}{25}{11}{}
\crossref{Jer}{25}{12}{Jer 29:10 2Ki 24:1 Ezr 1:1,\allowbreak2 Da 9:2}
\crossref{Jer}{25}{13}{Jer 1:5,\allowbreak10 Da 5:28,\allowbreak31 Re 10:11}
\crossref{Jer}{25}{14}{Jer 27:7; 50:9,\allowbreak41; 51:6,\allowbreak27,\allowbreak28 Isa 14:2; 45:1-\allowbreak3 Da 5:28 Hab 2:8-\allowbreak16}
\crossref{Jer}{25}{15}{Jer 13:12-\allowbreak14 Job 21:20 Ps 11:6; 75:8 Isa 51:17,\allowbreak22 Re 14:10,\allowbreak19}
\crossref{Jer}{25}{16}{25:27; 51:7,\allowbreak39 La 3:15; 4:21 Eze 23:32-\allowbreak34 Na 3:11 Re 14:8,\allowbreak10}
\crossref{Jer}{25}{17}{25:28; 1:10; 27:3; 46:1-\allowbreak51:64 Eze 43:3}
\crossref{Jer}{25}{18}{Jer 1:10; 19:3-\allowbreak9; 21:6-\allowbreak10 Ps 60:3 Isa 51:17-\allowbreak22 Eze 9:5-\allowbreak8 Da 9:12}
\crossref{Jer}{25}{19}{Jer 43:9-\allowbreak11; 46:2,\allowbreak13-\allowbreak26 Eze 29:1-\allowbreak32:32 Na 3:8-\allowbreak10}
\crossref{Jer}{25}{20}{25:24; 50:37 Ex 12:38 Eze 30:5}
\crossref{Jer}{25}{21}{Jer 27:3; 49:7-\allowbreak22 Ps 137:7 Isa 34:1-\allowbreak17; 63:1-\allowbreak6 La 4:21,\allowbreak22}
\crossref{Jer}{25}{22}{Jer 27:3; 47:4 Eze 26:1-\allowbreak28:1-\allowbreak19; 29:18 Am 1:9,\allowbreak10 Zec 9:2-\allowbreak4}
\crossref{Jer}{25}{23}{Jer 49:8 Ge 10:7; 22:21; 25:15 1Ch 1:30 Job 6:19 Isa 21:13,\allowbreak14}
\crossref{Jer}{25}{24}{1Ki 10:15 2Ch 9:14 Isa 21:13 Eze 27:21}
\crossref{Jer}{25}{25}{Ge 25:2}
\crossref{Jer}{25}{26}{25:9; 50:9 Eze 32:30}
\crossref{Jer}{25}{27}{Isa 51:21; 63:6 La 4:21 Hab 2:16}
\crossref{Jer}{25}{28}{Job 34:33}
\crossref{Jer}{25}{29}{Jer 49:12 Pr 11:31 Eze 9:6; 38:21 Ob 1:16 Lu 23:31 1Pe 4:17}
\crossref{Jer}{25}{30}{Isa 42:13 Ho 5:14; 13:7,\allowbreak8 Joe 2:11-\allowbreak13; 3:16 Am 1:2; 3:8}
\crossref{Jer}{25}{31}{Isa 66:16 Eze 20:35,\allowbreak36; 38:22 Joe 3:2}
\crossref{Jer}{25}{32}{2Ch 15:6 Isa 34:2; 66:18 Lu 21:10,\allowbreak25}
\crossref{Jer}{25}{33}{25:18-\allowbreak26; 13:12-\allowbreak14 Isa 34:2-\allowbreak8; 66:16 Zep 2:12 Re 14:19,\allowbreak20}
\crossref{Jer}{25}{34}{25:23,\allowbreak36; 4:8,\allowbreak9 Eze 34:16 Jas 5:1,\allowbreak2}
\crossref{Jer}{25}{35}{Jer 48:44; 52:8-\allowbreak11,\allowbreak24-\allowbreak27 Am 9:1-\allowbreak3}
\crossref{Jer}{25}{36}{25:34; 4:8}
\crossref{Jer}{25}{37}{Isa 27:10,\allowbreak11; 32:14}
\crossref{Jer}{25}{38}{Jer 4:7; 5:6; 49:19; 50:44 Ps 76:2 Ho 5:14; 11:10; 13:7,\allowbreak8 Am 8:8}
\crossref{Jer}{26}{1}{Jer 1:3; 25:1; 27:1; 35:1; 36:1 2Ki 23:34-\allowbreak36 2Ch 36:4,\allowbreak5}
\crossref{Jer}{26}{2}{Jer 7:2; 19:14; 23:28; 36:10 2Ch 24:20,\allowbreak21 Lu 19:47,\allowbreak48; 20:1; 21:37,\allowbreak38}
\crossref{Jer}{26}{3}{Jer 18:7-\allowbreak10; 36:3 Isa 1:16-\allowbreak19 Eze 18:27-\allowbreak30 Jon 3:8-\allowbreak10; 4:2}
\crossref{Jer}{26}{4}{Le 26:14-\allowbreak46 De 28:15-\allowbreak68; 29:18-\allowbreak28; 31:16-\allowbreak18,\allowbreak20; 32:15-\allowbreak25}
\crossref{Jer}{26}{5}{Jer 7:13,\allowbreak25; 11:7 2Ki 9:7; 17:13,\allowbreak23; 24:2 Ezr 9:11 Eze 38:17}
\crossref{Jer}{26}{6}{Jer 7:12-\allowbreak14 1Sa 4:10-\allowbreak12,\allowbreak19-\allowbreak22 Ps 78:60-\allowbreak64}
\crossref{Jer}{26}{7}{Jer 5:31; 23:11-\allowbreak15 Eze 22:25,\allowbreak26 Mic 3:11 Zep 3:4 Mt 21:15}
\crossref{Jer}{26}{8}{Jer 2:30; 11:19-\allowbreak21; 12:5,\allowbreak6; 18:18; 20:1,\allowbreak2,\allowbreak8-\allowbreak11 2Ch 36:16 La 4:13,\allowbreak14}
\crossref{Jer}{26}{9}{2Ch 25:16 Isa 29:21; 30:9-\allowbreak11 Am 5:10; 7:10-\allowbreak13 Mic 2:6 Mt 21:23}
\crossref{Jer}{26}{10}{26:16,\allowbreak17,\allowbreak24; 34:19; 36:12-\allowbreak19,\allowbreak25; 37:14-\allowbreak16; 38:4-\allowbreak6 Eze 22:6,\allowbreak27}
\crossref{Jer}{26}{11}{De 18:20 Mt 26:66 Lu 23:1-\allowbreak5 Joh 18:30; 19:7 Ac 22:22; 24:4-\allowbreak9}
\crossref{Jer}{26}{12}{26:2,\allowbreak15; 1:17,\allowbreak18; 19:1-\allowbreak3 Am 7:15-\allowbreak17 Ac 4:19; 5:29}
\crossref{Jer}{26}{13}{Jer 7:3-\allowbreak7; 35:15; 36:3; 38:20 Isa 1:19; 55:7 Eze 33:11 Ho 14:1-\allowbreak4}
\crossref{Jer}{26}{14}{Jer 38:5 Jos 9:25 Da 3:16}
\crossref{Jer}{26}{15}{Jer 2:30,\allowbreak34; 7:6; 22:3,\allowbreak17 Ge 4:10; 42:22 Nu 35:33 De 19:10}
\crossref{Jer}{26}{16}{Jer 36:19,\allowbreak25; 38:7-\allowbreak13 Es 4:14 Pr 16:7 Mt 27:23,\allowbreak24,\allowbreak54}
\crossref{Jer}{26}{17}{Mic 1:1 Ac 5:34}
\crossref{Jer}{26}{18}{Mic 1:1}
\crossref{Jer}{26}{19}{2Ch 29:6-\allowbreak11; 32:20,\allowbreak25,\allowbreak26; 34:21 Isa 37:1,\allowbreak4,\allowbreak15-\allowbreak20}
\crossref{Jer}{26}{20}{Jos 15:60; 18:14 1Sa 7:2}
\crossref{Jer}{26}{21}{Jer 36:26 2Ch 16:10 Ps 119:109 Mt 14:5 Mr 6:19}
\crossref{Jer}{26}{22}{Ps 12:8 Pr 29:12}
\crossref{Jer}{26}{23}{26:15; 2:30 Eze 19:6 Mt 14:10; 23:34,\allowbreak35 Ac 12:1-\allowbreak3 1Th 2:15 Re 11:7}
\crossref{Jer}{26}{24}{Jer 39:14; 40:5-\allowbreak7 2Ki 22:12-\allowbreak14; 25:22 2Ch 34:20}
\crossref{Jer}{27}{1}{27:3,\allowbreak12,\allowbreak19,\allowbreak20; 26:1; 28:1}
\crossref{Jer}{27}{2}{Am 7:1,\allowbreak4}
\crossref{Jer}{27}{3}{Jer 25:19-\allowbreak26; 47:1-\allowbreak49:39 Eze 25:1-\allowbreak28:26; 29:18 Am 1:9-\allowbreak15; 2:1-\allowbreak3}
\crossref{Jer}{27}{4}{}
\crossref{Jer}{27}{5}{Jer 10:11,\allowbreak12; 32:17; 51:15 Ge 9:6 Ex 20:11 Job 26:5-\allowbreak14; 38:4-\allowbreak41}
\crossref{Jer}{27}{6}{Jer 28:14 Da 2:37,\allowbreak38; 5:18,\allowbreak19}
\crossref{Jer}{27}{7}{Jer 25:11-\allowbreak14; 50:9,\allowbreak10; 52:31 2Ch 36:20,\allowbreak21}
\crossref{Jer}{27}{8}{Jer 25:28,\allowbreak29; 38:17-\allowbreak19; 40:9; 42:10-\allowbreak18; 52:3-\allowbreak6 Eze 17:19-\allowbreak21}
\crossref{Jer}{27}{9}{27:14-\allowbreak16; 14:14; 23:16,\allowbreak25,\allowbreak32; 29:8 Ex 7:11 De 18:10-\allowbreak12,\allowbreak14}
\crossref{Jer}{27}{10}{27:14; 28:16 Eze 14:9-\allowbreak11}
\crossref{Jer}{27}{11}{27:2,\allowbreak8,\allowbreak12}
\crossref{Jer}{27}{12}{27:3; 28:1; 38:17 2Ch 36:11-\allowbreak13 Pr 1:33 Eze 17:11-\allowbreak21}
\crossref{Jer}{27}{13}{Jer 38:20 Pr 8:36 Eze 18:24,\allowbreak31; 33:11}
\crossref{Jer}{27}{14}{27:9 Isa 28:10-\allowbreak13 2Co 11:13-\allowbreak15 Php 3:2}
\crossref{Jer}{27}{15}{27:10 2Ch 18:17-\allowbreak22; 25:16 Eze 14:3-\allowbreak10 Mt 24:24 2Th 2:9-\allowbreak12}
\crossref{Jer}{27}{16}{Jer 28:3 2Ki 24:13 2Ch 36:7-\allowbreak10 Da 1:2}
\crossref{Jer}{27}{17}{27:11,\allowbreak12}
\crossref{Jer}{27}{18}{1Ki 18:24,\allowbreak26}
\crossref{Jer}{27}{19}{Jer 52:17-\allowbreak23 1Ki 7:15-\allowbreak22 2Ki 25:13,\allowbreak17 2Ch 4:2-\allowbreak16}
\crossref{Jer}{27}{20}{Jer 22:28}
\crossref{Jer}{27}{21}{Jer 20:5}
\crossref{Jer}{27}{22}{Jer 29:10; 34:5; 52:17-\allowbreak21 2Ki 24:13-\allowbreak17 2Ch 36:17,\allowbreak18 Da 5:1-\allowbreak4,\allowbreak23}
\crossref{Jer}{28}{1}{Jer 27:1}
\crossref{Jer}{28}{2}{Jer 27:2-\allowbreak12 Eze 13:5-\allowbreak16 Mic 3:11}
\crossref{Jer}{28}{3}{Ge 47:9,\allowbreak28 Ps 90:10}
\crossref{Jer}{28}{4}{Jer 22:24,\allowbreak28}
\crossref{Jer}{28}{5}{28:1; 7:2; 19:14; 26:2}
\crossref{Jer}{28}{6}{Nu 5:22 De 27:15-\allowbreak26 1Ki 1:36 1Ch 16:36 Ps 41:13; 72:19; 89:52}
\crossref{Jer}{28}{7}{1Ki 22:28}
\crossref{Jer}{28}{8}{Le 26:14-\allowbreak46 De 4:26,\allowbreak27; 28:15-\allowbreak68; 29:18-\allowbreak28; 31:16,\allowbreak17; 32:15-\allowbreak44}
\crossref{Jer}{28}{9}{Jer 4:10; 6:14; 8:11; 14:13 Eze 13:10,\allowbreak16}
\crossref{Jer}{28}{10}{28:2,\allowbreak4; 27:2; 36:23,\allowbreak24 1Ki 22:11,\allowbreak24,\allowbreak25 Mal 3:13}
\crossref{Jer}{28}{11}{Jer 23:17; 29:9 1Ki 13:18; 22:6,\allowbreak11,\allowbreak12 2Ch 18:10,\allowbreak22,\allowbreak23 Pr 14:7}
\crossref{Jer}{28}{12}{Jer 1:2; 29:30 2Ki 20:4 1Ch 17:3 Da 9:2}
\crossref{Jer}{28}{13}{Jer 27:15 Ps 149:8 La 2:14}
\crossref{Jer}{28}{14}{Jer 27:4,\allowbreak7; 40:4 De 4:20; 28:48 Isa 14:4-\allowbreak6}
\crossref{Jer}{28}{15}{}
\crossref{Jer}{28}{16}{Ge 7:4 Ex 32:12 De 6:15 1Ki 13:34 Am 9:8}
\crossref{Jer}{28}{17}{Isa 44:25,\allowbreak26 Zec 1:6}
\crossref{Jer}{29}{1}{29:25-\allowbreak29 2Ch 30:1-\allowbreak6 Es 9:20 Ac 15:23 2Co 7:8 Ga 6:11 Heb 13:22}
\crossref{Jer}{29}{2}{Jer 22:24-\allowbreak28}
\crossref{Jer}{29}{3}{Jer 26:24; 39:14 2Ki 22:8 Eze 8:11}
\crossref{Jer}{29}{4}{Jer 24:5 Isa 5:5; 10:5,\allowbreak6; 45:7; 59:1,\allowbreak2 Am 3:6}
\crossref{Jer}{29}{5}{29:10,\allowbreak28 Eze 28:26}
\crossref{Jer}{29}{6}{Jer 16:2-\allowbreak4 Ge 1:27,\allowbreak28; 9:7 1Ti 5:14}
\crossref{Jer}{29}{7}{Da 4:27; 6:4,\allowbreak5 Ro 13:1,\allowbreak5 1Pe 2:13-\allowbreak17}
\crossref{Jer}{29}{8}{Jer 14:14; 23:21; 27:14,\allowbreak15; 28:15 Zec 13:4 Mt 24:4,\allowbreak5,\allowbreak24}
\crossref{Jer}{29}{9}{29:23,\allowbreak31; 27:15}
\crossref{Jer}{29}{10}{Jer 25:12; 27:7,\allowbreak22 2Ch 36:21-\allowbreak23 Ezr 1:1,\allowbreak2 Da 9:2 Zec 7:5}
\crossref{Jer}{29}{11}{Job 23:13 Ps 33:11; 40:5 Isa 46:10,\allowbreak11; 55:8-\allowbreak12 Mic 4:12 Zec 1:6}
\crossref{Jer}{29}{12}{Jer 31:9; 33:3 Ne 2:4-\allowbreak20 Ps 10:17; 50:15; 102:16,\allowbreak17 Isa 30:19; 65:24}
\crossref{Jer}{29}{13}{Le 26:40-\allowbreak45 De 4:29-\allowbreak31; 30:1-\allowbreak20 1Ki 8:47-\allowbreak50 2Ch 6:37-\allowbreak39}
\crossref{Jer}{29}{14}{De 4:7 1Ch 28:9 2Ch 15:12-\allowbreak15 Ps 32:6; 46:1 Isa 45:19; 55:6}
\crossref{Jer}{29}{15}{29:8,\allowbreak9; 28:1-\allowbreak17 Eze 1:1,\allowbreak3}
\crossref{Jer}{29}{16}{29:3; 24:2; 38:2,\allowbreak3,\allowbreak17-\allowbreak23 Eze 6:1-\allowbreak9:11 17:12-\allowbreak21 21:9-\allowbreak27 22:31}
\crossref{Jer}{29}{17}{29:18; 15:2,\allowbreak3; 24:8-\allowbreak10; 34:17-\allowbreak22; 43:11; 52:6 Eze 5:12-\allowbreak17; 14:12-\allowbreak21}
\crossref{Jer}{29}{18}{Jer 15:4; 24:9; 34:17 Le 26:33 De 28:25,\allowbreak64 2Ch 29:8 Ps 44:11}
\crossref{Jer}{29}{19}{Jer 6:19; 7:13,\allowbreak24-\allowbreak26; 25:3-\allowbreak7; 26:5; 32:33; 34:17; 35:14-\allowbreak16; 44:4,\allowbreak5}
\crossref{Jer}{29}{20}{Eze 3:11,\allowbreak15}
\crossref{Jer}{29}{21}{29:8,\allowbreak9; 14:14,\allowbreak15 La 2:14}
\crossref{Jer}{29}{22}{Ge 48:20 Ru 4:11 Isa 65:15 1Co 16:22}
\crossref{Jer}{29}{23}{Jer 7:9,\allowbreak10; 23:14,\allowbreak21 Ps 50:16-\allowbreak18 Zep 3:4 2Pe 2:10-\allowbreak19 Jude 1:8-\allowbreak11}
\crossref{Jer}{29}{24}{29:31,\allowbreak32}
\crossref{Jer}{29}{25}{1Ki 21:8-\allowbreak13 2Ki 10:1-\allowbreak7; 19:9,\allowbreak14 2Ch 32:17 Ezr 4:7-\allowbreak16 Ne 6:5,\allowbreak17}
\crossref{Jer}{29}{26}{Jer 20:1,\allowbreak2 2Ki 11:15,\allowbreak18 Ac 4:1; 5:24}
\crossref{Jer}{29}{27}{2Ch 25:16 Am 7:12,\allowbreak13 Joh 11:47-\allowbreak53 Ac 4:17-\allowbreak21; 5:28,\allowbreak40}
\crossref{Jer}{29}{28}{29:1-\allowbreak10}
\crossref{Jer}{29}{29}{29:25}
\crossref{Jer}{29}{30}{}
\crossref{Jer}{29}{31}{29:20}
\crossref{Jer}{29}{32}{Jer 20:6 Ex 20:5 Nu 16:27-\allowbreak33 Jos 7:24,\allowbreak25 2Ki 5:27 Ps 109:8-\allowbreak15}
\crossref{Jer}{30}{1}{Jer 1:1,\allowbreak2; 26:15}
\crossref{Jer}{30}{2}{Jer 36:2-\allowbreak4,\allowbreak32; 51:60-\allowbreak64 Ex 17:14 De 31:19,\allowbreak22-\allowbreak27 Job 19:23,\allowbreak24}
\crossref{Jer}{30}{3}{Jer 23:5,\allowbreak7; 31:27,\allowbreak31,\allowbreak38; 33:14,\allowbreak15 Lu 17:22; 19:43; 21:6 Heb 8:8}
\crossref{Jer}{30}{4}{Jer 31:6 Isa 11:13 Ho 1:11 Eze 20:40}
\crossref{Jer}{30}{5}{Jer 4:15-\allowbreak20; 6:23,\allowbreak24; 8:19; 9:19; 25:36; 31:15,\allowbreak16 Isa 5:30; 59:11}
\crossref{Jer}{30}{6}{Jer 4:31; 6:24; 13:21; 22:23; 49:24; 50:43 Ps 48:6 Isa 13:6-\allowbreak9; 21:3}
\crossref{Jer}{30}{7}{Isa 2:12-\allowbreak22 Eze 7:6-\allowbreak12 Ho 1:11 Joe 2:11,\allowbreak31 Am 5:18-\allowbreak20}
\crossref{Jer}{30}{8}{Jer 27:2; 28:4,\allowbreak10,\allowbreak13 Isa 9:4; 10:27; 14:25 Eze 34:27 Na 1:13}
\crossref{Jer}{30}{9}{Isa 55:3-\allowbreak5 Eze 34:23; 37:23-\allowbreak25 Ho 3:5 Lu 1:69 Ac 2:30; 13:34}
\crossref{Jer}{30}{10}{Jer 46:27,\allowbreak28 Ge 15:1 De 31:6-\allowbreak8 Isa 41:10-\allowbreak15; 43:5; 44:2; 54:4}
\crossref{Jer}{30}{11}{Jer 1:8,\allowbreak19; 15:20; 46:28 Isa 8:10; 43:25 Eze 11:16,\allowbreak17 Mt 1:23}
\crossref{Jer}{30}{12}{30:15; 14:17; 15:18 2Ch 36:16 Isa 1:5,\allowbreak6 Eze 37:11}
\crossref{Jer}{30}{13}{Ps 106:23; 142:4 Isa 59:16 Eze 22:30 1Ti 2:5,\allowbreak6 1Jo 2:1}
\crossref{Jer}{30}{14}{Jer 2:36; 4:30; 22:20,\allowbreak22; 38:22 La 1:2,\allowbreak19 Eze 23:9,\allowbreak22 Ho 2:5,\allowbreak10-\allowbreak16}
\crossref{Jer}{30}{15}{Jer 15:18 Jos 9:10,\allowbreak11 La 3:39 Mic 7:9}
\crossref{Jer}{30}{16}{Jer 10:25; 12:14; 25:12,\allowbreak26-\allowbreak29; 50:7-\allowbreak11,\allowbreak17,\allowbreak18,\allowbreak28,\allowbreak33-\allowbreak40; 51:34-\allowbreak37}
\crossref{Jer}{30}{17}{30:13; 3:22; 33:6 Ex 15:26 Ps 23:3; 103:3; 107:20 Isa 30:26}
\crossref{Jer}{30}{18}{30:3; 23:3; 29:14; 33:7,\allowbreak11; 46:27; 49:6,\allowbreak39 Ps 85:1; 102:13}
\crossref{Jer}{30}{19}{Jer 31:4,\allowbreak12,\allowbreak13; 33:10,\allowbreak11 Ezr 3:10-\allowbreak13; 6:22 Ne 8:12,\allowbreak17; 12:43-\allowbreak46}
\crossref{Jer}{30}{20}{Jer 32:39 Ge 17:5-\allowbreak9 Ps 90:16,\allowbreak17; 102:18,\allowbreak28 Isa 1:26,\allowbreak27}
\crossref{Jer}{30}{21}{Ge 49:10 Ezr 2:2; 7:25,\allowbreak26 Ne 2:9,\allowbreak10; 7:2}
\crossref{Jer}{30}{22}{Jer 24:7; 31:1,\allowbreak33 De 26:17-\allowbreak19 So 2:16 Eze 11:20; 36:28; 37:27}
\crossref{Jer}{30}{23}{Jer 23:19,\allowbreak20; 25:32 Ps 58:9 Pr 1:27 Zec 9:14}
\crossref{Jer}{30}{24}{Jer 4:28 1Sa 3:12 Job 23:13,\allowbreak14 Isa 14:24,\allowbreak26,\allowbreak27; 46:11 Eze 20:47,\allowbreak48}
\crossref{Jer}{31}{1}{Jer 30:24}
\crossref{Jer}{31}{2}{Ex 1:16,\allowbreak22; 2:23; 5:21; 12:37; 14:8-\allowbreak12; 15:9,\allowbreak10; 17:8-\allowbreak13}
\crossref{Jer}{31}{3}{De 7:7-\allowbreak9; 10:15; 33:3,\allowbreak26 Ho 11:1 Mal 1:2 Ro 9:13 1Jo 4:19}
\crossref{Jer}{31}{4}{Jer 1:10; 30:18; 33:7 Ps 51:18; 69:35 Am 9:11 Ac 15:16 Eph 2:20-\allowbreak22}
\crossref{Jer}{31}{5}{De 28:30 Isa 62:8,\allowbreak9; 65:21,\allowbreak22 Am 9:14 Mic 4:4 Zec 3:10}
\crossref{Jer}{31}{6}{Jer 6:17 Isa 40:9; 52:7,\allowbreak8; 62:6 Eze 3:17; 33:2 Ho 9:8}
\crossref{Jer}{31}{7}{De 32:43 Ps 67:1; 96:1-\allowbreak3; 98:1-\allowbreak4; 117:1,\allowbreak2; 138:4,\allowbreak5 Isa 12:4-\allowbreak6}
\crossref{Jer}{31}{8}{Jer 3:12; 23:8; 29:14 Ps 107:3 Zec 2:6}
\crossref{Jer}{31}{9}{Jer 3:4; 50:4 Ps 126:5,\allowbreak6 Ho 12:4 Zec 12:10 Da 9:17,\allowbreak18 Mt 5:4}
\crossref{Jer}{31}{10}{Ge 10:5 Ps 72:10 Isa 24:14; 41:1; 42:4,\allowbreak10; 60:9; 66:19 Zep 2:11}
\crossref{Jer}{31}{11}{Jer 15:21; 50:33 Isa 44:23; 48:20; 49:24 Ho 13:14 Mt 20:28}
\crossref{Jer}{31}{12}{31:4; 33:9-\allowbreak11 Isa 12:1-\allowbreak6; 35:10; 51:11}
\crossref{Jer}{31}{13}{31:4 Ne 12:27,\allowbreak43 Ps 30:11; 149:3 Zec 8:4,\allowbreak5,\allowbreak19}
\crossref{Jer}{31}{14}{De 33:8-\allowbreak11 2Ch 6:41 Ne 10:39 Ps 132:9,\allowbreak16 Isa 61:6 1Pe 2:9}
\crossref{Jer}{31}{15}{Eze 2:10 Mt 2:16}
\crossref{Jer}{31}{16}{Ge 43:31; 45:1 Ps 30:5 Mr 5:38,\allowbreak39 Joh 20:13-\allowbreak15 1Th 4:14}
\crossref{Jer}{31}{17}{Jer 29:11-\allowbreak16; 46:27,\allowbreak28 Ps 102:13,\allowbreak14 Isa 6:13; 11:11-\allowbreak16}
\crossref{Jer}{31}{18}{Job 33:27,\allowbreak28 Ps 102:19,\allowbreak20 Isa 57:15-\allowbreak18 Ho 5:15; 6:1,\allowbreak2 Lu 15:20}
\crossref{Jer}{31}{19}{De 30:2,\allowbreak6-\allowbreak8 Eze 36:26,\allowbreak31 Zec 12:10 Lu 15:17-\allowbreak19 Joh 6:44,\allowbreak45}
\crossref{Jer}{31}{20}{31:9; 3:19 Ps 103:13 Pr 3:12 Lu 15:24,\allowbreak32}
\crossref{Jer}{31}{21}{Isa 57:14; 62:10}
\crossref{Jer}{31}{22}{Jer 2:18,\allowbreak23,\allowbreak36; 4:14; 13:27 Ho 8:5}
\crossref{Jer}{31}{23}{Jer 23:5-\allowbreak8; 33:15-\allowbreak26 Isa 1:26; 60:21 Zec 8:3}
\crossref{Jer}{31}{24}{Jer 33:11-\allowbreak13 Eze 36:10 Zec 2:4; 8:4-\allowbreak8}
\crossref{Jer}{31}{25}{31:14 Ps 107:9 Isa 32:2; 50:4 Mt 5:6; 11:28 Lu 1:53 Joh 4:14}
\crossref{Jer}{31}{26}{Ps 127:2 Zec 4:1,\allowbreak2}
\crossref{Jer}{31}{27}{31:31}
\crossref{Jer}{31}{28}{Jer 44:27 Da 9:14}
\crossref{Jer}{31}{29}{31:30 La 5:7 Eze 18:2,\allowbreak3}
\crossref{Jer}{31}{30}{De 24:16 Isa 3:11 Eze 3:18,\allowbreak19,\allowbreak24; 18:4,\allowbreak20; 33:8,\allowbreak13,\allowbreak18}
\crossref{Jer}{31}{31}{31:27; 23:5; 30:3; 33:14-\allowbreak16 Am 9:13}
\crossref{Jer}{31}{32}{31:1; 34:14 Ex 19:5; 24:6-\allowbreak8 De 5:3; 29:1 1Ki 8:9 Eze 16:8,\allowbreak60-\allowbreak62}
\crossref{Jer}{31}{33}{Jer 32:40}
\crossref{Jer}{31}{34}{1Th 4:9 Heb 5:12 1Jo 2:27}
\crossref{Jer}{31}{35}{Ge 1:14-\allowbreak18 De 4:19 Job 38:33 Ps 19:1-\allowbreak6; 72:5,\allowbreak17; 74:16}
\crossref{Jer}{31}{36}{Jer 33:20-\allowbreak26 Ps 72:5,\allowbreak17; 89:36,\allowbreak37; 102:28; 119:89; 148:6 Isa 54:9,\allowbreak10}
\crossref{Jer}{31}{37}{Jer 33:22 Job 11:7-\allowbreak9 Ps 89:2 Pr 30:4 Isa 40:12}
\crossref{Jer}{31}{38}{31:27}
\crossref{Jer}{31}{39}{Eze 40:8 Zec 2:1,\allowbreak2}
\crossref{Jer}{31}{40}{Jer 7:32; 19:11-\allowbreak13; 32:36 Eze 37:2}
\crossref{Jer}{32}{1}{Jer 39:1,\allowbreak2; 52:4,\allowbreak5 2Ki 25:1,\allowbreak2 2Ch 36:11}
\crossref{Jer}{32}{2}{32:3,\allowbreak8; 33:1; 36:5; 37:21; 38:6; 39:13-\allowbreak15 Mt 5:12}
\crossref{Jer}{32}{3}{Jer 2:30; 5:3 2Ki 6:31,\allowbreak32 2Ch 28:22}
\crossref{Jer}{32}{4}{Jer 37:17; 38:18,\allowbreak23; 39:4-\allowbreak7; 52:8-\allowbreak11 2Ki 25:4-\allowbreak7 Eze 12:12,\allowbreak13}
\crossref{Jer}{32}{5}{Jer 27:22; 34:4,\allowbreak5}
\crossref{Jer}{32}{6}{}
\crossref{Jer}{32}{7}{1Ki 14:5 Mr 11:2-\allowbreak6; 14:13-\allowbreak16}
\crossref{Jer}{32}{8}{32:2; 33:1}
\crossref{Jer}{32}{9}{Ge 23:15,\allowbreak16 1Ki 20:39 Es 3:9 Isa 55:2}
\crossref{Jer}{32}{10}{32:12,\allowbreak44 Isa 44:5}
\crossref{Jer}{32}{11}{Lu 2:27 Ac 26:3 1Co 11:16}
\crossref{Jer}{32}{12}{32:16; 36:4,\allowbreak5,\allowbreak16-\allowbreak19,\allowbreak26; 43:3-\allowbreak6; 45:1-\allowbreak5}
\crossref{Jer}{32}{13}{}
\crossref{Jer}{32}{14}{32:10-\allowbreak12}
\crossref{Jer}{32}{15}{32:37,\allowbreak43,\allowbreak44}
\crossref{Jer}{32}{16}{Jer 12:1 Ge 32:9-\allowbreak12 2Sa 7:18-\allowbreak25 Eze 36:35-\allowbreak37 Php 4:6,\allowbreak7}
\crossref{Jer}{32}{17}{Jer 1:6; 4:10; 14:13 Eze 9:8; 11:13}
\crossref{Jer}{32}{18}{Ex 20:5,\allowbreak6; 34:7 Nu 14:18 De 5:9,\allowbreak10; 7:9,\allowbreak10}
\crossref{Jer}{32}{19}{Isa 9:6; 28:29; 40:13; 46:10,\allowbreak11 Ro 11:33,\allowbreak34 Eph 1:11}
\crossref{Jer}{32}{20}{Ex 7:3; 10:2 De 4:34; 6:22; 7:19 Ne 9:10 Ps 78:43-\allowbreak51; 105:27-\allowbreak36}
\crossref{Jer}{32}{21}{Ex 6:6; 13:14 Ps 105:37,\allowbreak43; 106:8-\allowbreak11}
\crossref{Jer}{32}{22}{Ge 13:15; 15:18-\allowbreak21; 17:7,\allowbreak8; 24:7; 28:13-\allowbreak15; 35:11,\allowbreak12; 50:24 Ex 13:5}
\crossref{Jer}{32}{23}{Ne 9:15,\allowbreak22-\allowbreak25 Ps 44:2,\allowbreak3; 78:54,\allowbreak55; 105:44,\allowbreak45}
\crossref{Jer}{32}{24}{Jer 33:4 Eze 21:22}
\crossref{Jer}{32}{25}{32:8-\allowbreak15}
\crossref{Jer}{32}{26}{}
\crossref{Jer}{32}{27}{Nu 16:22; 27:16 Ps 65:2 Isa 64:8 Lu 3:6 Joh 17:2 Ro 3:29,\allowbreak30}
\crossref{Jer}{32}{28}{32:3,\allowbreak24,\allowbreak36; 19:7-\allowbreak12; 20:5}
\crossref{Jer}{32}{29}{Jer 17:27; 21:10; 27:8-\allowbreak10; 37:7-\allowbreak10; 39:8; 52:13 2Ki 25:9 2Ch 36:19}
\crossref{Jer}{32}{30}{Jer 2:7; 3:25; 7:22-\allowbreak26 De 9:7-\allowbreak12,\allowbreak22-\allowbreak24 2Ki 17:9-\allowbreak20 Ne 9:16-\allowbreak37}
\crossref{Jer}{32}{31}{Jer 5:9-\allowbreak11; 6:6,\allowbreak7; 23:14,\allowbreak15 1Ki 11:7,\allowbreak8 2Ki 21:4-\allowbreak7,\allowbreak16; 22:16,\allowbreak17}
\crossref{Jer}{32}{32}{Jer 2:26 Ezr 9:7 Ne 9:32-\allowbreak34 Isa 1:4-\allowbreak6,\allowbreak23; 9:14,\allowbreak15 Eze 22:6,\allowbreak25-\allowbreak29}
\crossref{Jer}{32}{33}{Jer 2:27; 7:24; 18:17 Eze 8:16 Ho 11:2 Zec 7:11}
\crossref{Jer}{32}{34}{Jer 7:30; 23:11 2Ki 21:4-\allowbreak7; 23:6 2Ch 33:4-\allowbreak7,\allowbreak15 Eze 8:5-\allowbreak16}
\crossref{Jer}{32}{35}{Jer 7:31; 19:5,\allowbreak6 2Ki 23:10 2Ch 28:2,\allowbreak3; 33:6 Ps 106:37,\allowbreak38 Isa 57:5}
\crossref{Jer}{32}{36}{Jer 16:12-\allowbreak15 Isa 43:24,\allowbreak25; 57:17,\allowbreak18 Eze 36:31,\allowbreak32 Ho 2:14 Ro 5:20}
\crossref{Jer}{32}{37}{Jer 23:6; 33:16 Eze 34:25-\allowbreak28 Joe 3:20 Zec 2:4,\allowbreak5; 3:10; 14:11}
\crossref{Jer}{32}{38}{Jer 24:7; 30:22; 31:1,\allowbreak33 Ge 17:7 De 26:17-\allowbreak19 Ps 144:15 Eze 11:19,\allowbreak20}
\crossref{Jer}{32}{39}{2Ch 30:12 Isa 52:8 Eze 11:19,\allowbreak20; 36:26; 37:22 Joh 17:21 Ac 4:32}
\crossref{Jer}{32}{40}{Jer 31:31-\allowbreak33; 50:5 Ge 17:7-\allowbreak13 2Sa 23:4 Isa 24:5; 55:3; 61:8}
\crossref{Jer}{32}{41}{De 30:9 Isa 62:5; 65:19 Zep 3:17}
\crossref{Jer}{32}{42}{Jer 31:28 Jos 23:14,\allowbreak15 Zec 8:14,\allowbreak15 Mt 24:35}
\crossref{Jer}{32}{43}{32:36 Eze 37:11-\allowbreak14}
\crossref{Jer}{32}{44}{32:6-\allowbreak15}
\crossref{Jer}{33}{1}{Jer 32:2,\allowbreak3,\allowbreak8; 37:21; 38:28 2Ti 2:9}
\crossref{Jer}{33}{2}{Jer 32:18 Ex 3:14,\allowbreak15; 6:3; 15:3 Am 5:8; 9:6}
\crossref{Jer}{33}{3}{Jer 29:12 De 4:7,\allowbreak29 1Ki 8:47-\allowbreak50 Ps 50:15; 91:15; 145:18 Isa 55:6,\allowbreak7}
\crossref{Jer}{33}{4}{Jer 32:24 Eze 4:2; 21:22; 26:8 Hab 1:10}
\crossref{Jer}{33}{5}{Jer 21:4-\allowbreak7; 32:5; 37:9,\allowbreak10}
\crossref{Jer}{33}{6}{Jer 17:14; 30:12-\allowbreak17 De 32:39 Ps 67:2 Isa 30:26; 58:8 Ho 6:1; 7:1}
\crossref{Jer}{33}{7}{33:11,\allowbreak26}
\crossref{Jer}{33}{8}{Jer 31:34; 50:20 Ps 51:2; 65:3; 85:2,\allowbreak3 Isa 4:2; 44:22; 56:7}
\crossref{Jer}{33}{9}{Jer 13:11; 31:4 Ps 126:2,\allowbreak3 Isa 62:2,\allowbreak3,\allowbreak7,\allowbreak12 Zep 3:17-\allowbreak20 Zec 8:20-\allowbreak23}
\crossref{Jer}{33}{10}{Jer 32:36 Eze 37:11}
\crossref{Jer}{33}{11}{Jer 7:34; 16:9; 25:10 Joh 3:29 Re 18:23}
\crossref{Jer}{33}{12}{Jer 32:43; 36:29; 51:62}
\crossref{Jer}{33}{13}{Le 27:32 Lu 15:4 Joh 10:3,\allowbreak4}
\crossref{Jer}{33}{14}{Jer 23:5; 29:10; 31:27,\allowbreak31-\allowbreak34; 32:38-\allowbreak41 Ge 22:18; 49:10 1Ch 17:13,\allowbreak14}
\crossref{Jer}{33}{15}{Jer 23:5,\allowbreak6 Isa 4:2; 11:1-\allowbreak5; 53:2 Eze 17:22,\allowbreak23 Zec 3:8; 6:12,\allowbreak13}
\crossref{Jer}{33}{16}{Jer 23:6 Isa 45:17,\allowbreak22 Ro 11:26}
\crossref{Jer}{33}{17}{}
\crossref{Jer}{33}{18}{Isa 56:7; 61:6 Eze 43:19-\allowbreak27; 44:9-\allowbreak11; 45:5 Ro 1:21; 15:16}
\crossref{Jer}{33}{19}{33:19}
\crossref{Jer}{33}{20}{33:25,\allowbreak26}
\crossref{Jer}{33}{21}{2Sa 23:5 2Ch 7:18; 21:7 Ps 89:34; 132:11,\allowbreak12,\allowbreak17 Isa 55:3}
\crossref{Jer}{33}{22}{Jer 31:37 Ge 13:16; 15:5; 22:17; 28:14 Ho 1:10 Heb 11:12 Re 7:9,\allowbreak10}
\crossref{Jer}{33}{23}{}
\crossref{Jer}{33}{24}{33:21,\allowbreak22 Ps 94:14 Ro 11:1-\allowbreak6}
\crossref{Jer}{33}{25}{33:20 Ge 8:22; 9:9-\allowbreak17}
\crossref{Jer}{33}{26}{Jer 31:37 Ge 49:10}
\crossref{Jer}{34}{1}{34:7; 32:2; 39:1-\allowbreak3; 52:4-\allowbreak11 2Ki 25:1-\allowbreak9 2Ch 36:12-\allowbreak17}
\crossref{Jer}{34}{2}{Jer 22:1,\allowbreak2; 37:1-\allowbreak4 2Ch 36:11,\allowbreak12}
\crossref{Jer}{34}{3}{34:21; 21:7; 32:4; 37:17; 38:18; 39:4,\allowbreak5; 52:7-\allowbreak9 2Ki 25:4,\allowbreak5}
\crossref{Jer}{34}{4}{Jer 38:17,\allowbreak20}
\crossref{Jer}{34}{5}{2Ki 22:20 2Ch 34:28 Eze 17:16}
\crossref{Jer}{34}{6}{1Sa 3:18; 15:16-\allowbreak24 2Sa 12:7-\allowbreak12 1Ki 21:19; 22:14 Eze 2:7 Mt 14:4}
\crossref{Jer}{34}{7}{34:1; 4:5; 8:14; 11:12 De 28:52}
\crossref{Jer}{34}{8}{2Ki 11:17; 23:2,\allowbreak3 2Ch 15:12-\allowbreak15; 23:16; 29:10; 34:30-\allowbreak33 Ne 9:38}
\crossref{Jer}{34}{9}{Ge 14:13; 40:15 Ex 2:6; 3:18 De 15:12 1Sa 4:6,\allowbreak9; 14:11 2Co 11:22}
\crossref{Jer}{34}{10}{Jer 26:10,\allowbreak16; 36:12,\allowbreak24,\allowbreak25; 38:4}
\crossref{Jer}{34}{11}{34:21; 37:5 Ex 8:8,\allowbreak15; 9:28,\allowbreak34,\allowbreak35; 10:17-\allowbreak20; 14:3-\allowbreak9 1Sa 19:6-\allowbreak11}
\crossref{Jer}{34}{12}{Ge 19:24}
\crossref{Jer}{34}{13}{Jer 31:32 Ex 24:3,\allowbreak7,\allowbreak8 De 5:2,\allowbreak3,\allowbreak27; 29:1 Heb 8:10,\allowbreak11}
\crossref{Jer}{34}{14}{34:8,\allowbreak9 Ex 21:1-\allowbreak4; 23:10,\allowbreak11 De 15:12 1Ki 9:22 2Ch 28:10 Isa 58:6}
\crossref{Jer}{34}{15}{1Ki 21:27-\allowbreak29 2Ki 10:30,\allowbreak31; 12:2; 14:3 Isa 58:2 Mt 15:8}
\crossref{Jer}{34}{16}{34:11 1Sa 15:11 Eze 3:20; 18:24; 33:12,\allowbreak13 Lu 8:13-\allowbreak15}
\crossref{Jer}{34}{17}{Jer 15:2; 21:7; 24:10; 32:24,\allowbreak36; 47:6,\allowbreak7 Eze 14:17-\allowbreak21}
\crossref{Jer}{34}{18}{De 17:2 Jos 7:11; 23:16 Ho 6:7; 8:1}
\crossref{Jer}{34}{19}{34:10 Eze 22:27-\allowbreak31 Da 9:6,\allowbreak8,\allowbreak12 Mic 7:1-\allowbreak5 Zep 3:3,\allowbreak4}
\crossref{Jer}{34}{20}{Jer 4:30; 11:21; 21:7; 22:25; 38:16; 44:30; 49:37}
\crossref{Jer}{34}{21}{34:3-\allowbreak5; 39:6; 52:10,\allowbreak24-\allowbreak27 2Ki 25:18-\allowbreak21 La 4:20 Eze 17:16}
\crossref{Jer}{34}{22}{2Sa 16:11 2Ki 24:2,\allowbreak3 2Ch 36:17 Isa 10:5-\allowbreak7; 13:3; 37:26; 45:1-\allowbreak3}
\crossref{Jer}{35}{1}{Jer 1:3; 22:13-\allowbreak19; 25:1; 26:1; 36:1,\allowbreak9,\allowbreak29; 46:2 2Ki 23:35; 24:1-\allowbreak6}
\crossref{Jer}{35}{2}{35:8 2Ki 10:15,\allowbreak16 1Ch 2:55}
\crossref{Jer}{35}{3}{}
\crossref{Jer}{35}{4}{Jer 36:10-\allowbreak12}
\crossref{Jer}{35}{5}{35:2 Ec 9:7 Am 2:12 2Co 2:9}
\crossref{Jer}{35}{6}{2Ki 10:15 1Ch 2:55}
\crossref{Jer}{35}{7}{35:10 Ge 25:27 Le 23:42,\allowbreak43 Ne 8:14-\allowbreak16 Eph 5:18 Heb 11:9-\allowbreak13}
\crossref{Jer}{35}{8}{Pr 1:8,\allowbreak9; 4:1,\allowbreak2,\allowbreak10; 6:20; 13:1 Col 3:20}
\crossref{Jer}{35}{9}{35:7 Nu 16:14 2Ki 5:26 Ps 37:16 1Ti 6:6}
\crossref{Jer}{35}{10}{}
\crossref{Jer}{35}{11}{2Ki 24:2 Da 1:1,\allowbreak2}
\crossref{Jer}{35}{12}{}
\crossref{Jer}{35}{13}{Jer 5:3; 6:8-\allowbreak10; 9:12; 32:33 Ps 32:8,\allowbreak9 Pr 8:10; 19:20 Isa 28:9-\allowbreak12}
\crossref{Jer}{35}{14}{35:6-\allowbreak10}
\crossref{Jer}{35}{15}{Lu 10:16 1Th 4:8}
\crossref{Jer}{35}{16}{35:14 Isa 1:3 Mal 1:6 Mt 11:28-\allowbreak30 Lu 15:11-\allowbreak13,\allowbreak28-\allowbreak30}
\crossref{Jer}{35}{17}{Jer 11:8; 15:3,\allowbreak4; 19:7-\allowbreak13; 21:4-\allowbreak10 Ge 6:17 Le 26:14-\allowbreak46 De 28:15-\allowbreak68}
\crossref{Jer}{35}{18}{Ex 20:12 De 5:16 Eph 6:1-\allowbreak3}
\crossref{Jer}{35}{19}{Jer 15:19; 33:17,\allowbreak18 Ps 5:5 Lu 21:36 Jude 1:24}
\crossref{Jer}{36}{1}{Jer 25:1; 35:1 2Ki 24:1,\allowbreak2}
\crossref{Jer}{36}{2}{36:6,\allowbreak23,\allowbreak29; 30:2; 45:1; 51:60 Ex 17:14 De 31:24 Ezr 6:2 Job 31:35}
\crossref{Jer}{36}{3}{36:7; 18:8; 26:3 De 5:29 Eze 12:3 Zep 2:3 Lu 20:13 2Ti 2:25,\allowbreak26}
\crossref{Jer}{36}{4}{36:26; 32:12; 43:3}
\crossref{Jer}{36}{5}{Jer 20:2; 32:2; 33:1; 37:15; 38:6,\allowbreak28; 40:4 2Co 11:23 Eph 3:1; 6:20}
\crossref{Jer}{36}{6}{36:8 Eze 2:3-\allowbreak7}
\crossref{Jer}{36}{7}{36:3 1Ki 8:33-\allowbreak36 2Ch 33:12,\allowbreak13 Da 9:13 Ho 5:15; 6:1; 14:1-\allowbreak3}
\crossref{Jer}{36}{8}{36:4; 1:17 Mt 16:24 1Co 16:10 Php 2:19-\allowbreak22}
\crossref{Jer}{36}{9}{36:1}
\crossref{Jer}{36}{10}{36:6,\allowbreak8}
\crossref{Jer}{36}{11}{36:10 2Ki 22:12-\allowbreak14; 25:22 2Ch 34:20}
\crossref{Jer}{36}{12}{36:20,\allowbreak21; 41:1}
\crossref{Jer}{36}{13}{2Ki 22:10,\allowbreak19 2Ch 34:16-\allowbreak18,\allowbreak24 Jon 3:6}
\crossref{Jer}{36}{14}{Jer 40:8; 41:1,\allowbreak2,\allowbreak16,\allowbreak18 2Ki 25:23}
\crossref{Jer}{36}{15}{36:21}
\crossref{Jer}{36}{16}{36:24 Ac 24:25,\allowbreak26}
\crossref{Jer}{36}{17}{Joh 9:10,\allowbreak11,\allowbreak15,\allowbreak26,\allowbreak27}
\crossref{Jer}{36}{18}{36:2,\allowbreak4; 43:2,\allowbreak3 Pr 26:4,\allowbreak5}
\crossref{Jer}{36}{19}{36:26; 26:20-\allowbreak24 1Ki 17:3; 18:4,\allowbreak10 2Ch 25:15 Pr 28:12 Am 7:12}
\crossref{Jer}{36}{20}{36:12,\allowbreak21}
\crossref{Jer}{36}{21}{36:14}
\crossref{Jer}{36}{22}{}
\crossref{Jer}{36}{23}{36:29-\allowbreak31 De 29:19-\allowbreak21 1Ki 22:8,\allowbreak27 Ps 50:17 Pr 1:30; 5:12; 13:13}
\crossref{Jer}{36}{24}{36:16 Job 15:4 Ps 36:1; 64:5 Isa 26:11 Ro 3:18}
\crossref{Jer}{36}{25}{36:12; 26:22}
\crossref{Jer}{36}{26}{Jer 2:30; 26:21-\allowbreak23 1Ki 19:1-\allowbreak3,\allowbreak10,\allowbreak14 Mt 23:34-\allowbreak37; 26:47-\allowbreak50}
\crossref{Jer}{36}{27}{36:23}
\crossref{Jer}{36}{28}{Jer 28:13,\allowbreak14; 44:28 Job 23:13 Zec 1:5,\allowbreak6 Mt 24:35 2Ti 2:13}
\crossref{Jer}{36}{29}{De 29:19 Job 15:24; 40:8 Isa 45:9 Ac 5:39 1Co 10:22}
\crossref{Jer}{36}{30}{Jer 22:30 2Ki 24:12-\allowbreak15}
\crossref{Jer}{36}{31}{Jer 23:34}
\crossref{Jer}{36}{32}{36:28-\allowbreak30}
\crossref{Jer}{37}{1}{2Ki 24:17 1Ch 3:15 2Ch 36:10}
\crossref{Jer}{37}{2}{2Ki 24:19,\allowbreak20 2Ch 36:12-\allowbreak16 Pr 29:12 Eze 21:25 1Th 4:8}
\crossref{Jer}{37}{3}{Jer 21:1,\allowbreak2; 29:21,\allowbreak25; 52:24}
\crossref{Jer}{37}{4}{37:15; 32:2,\allowbreak3}
\crossref{Jer}{37}{5}{37:11; 34:21}
\crossref{Jer}{37}{6}{}
\crossref{Jer}{37}{7}{37:3; 21:2 2Ki 22:18}
\crossref{Jer}{37}{8}{Jer 32:29; 34:21,\allowbreak22; 38:23; 39:2-\allowbreak8}
\crossref{Jer}{37}{9}{Ob 1:3 Mt 24:4,\allowbreak5 Ga 6:3,\allowbreak7 Eph 5:6 2Th 2:3 Jas 1:22}
\crossref{Jer}{37}{10}{Jer 21:4-\allowbreak7; 49:20; 50:45 Le 26:36-\allowbreak38 Isa 10:4; 30:17}
\crossref{Jer}{37}{11}{37:5}
\crossref{Jer}{37}{12}{1Ki 19:3,\allowbreak9 Ne 6:11 Mt 10:23 1Th 5:22}
\crossref{Jer}{37}{13}{Jer 38:7 Zec 14:10}
\crossref{Jer}{37}{14}{Jer 40:4-\allowbreak6 Ne 6:8 Ps 27:12; 35:11; 52:1,\allowbreak2 Mt 5:11,\allowbreak12 Lu 6:22,\allowbreak23,\allowbreak26}
\crossref{Jer}{37}{15}{Jer 20:1-\allowbreak3; 26:16 Mt 21:35; 23:34; 26:67,\allowbreak68 Lu 20:10,\allowbreak11; 22:64}
\crossref{Jer}{37}{16}{Jer 38:6,\allowbreak10-\allowbreak13 Ge 40:15 La 3:53,\allowbreak55}
\crossref{Jer}{37}{17}{Jer 38:5,\allowbreak14-\allowbreak16,\allowbreak24-\allowbreak27 1Ki 14:1-\allowbreak4}
\crossref{Jer}{37}{18}{Jer 26:19 Ge 31:36 1Sa 24:9-\allowbreak15; 26:18-\allowbreak21 Pr 17:13,\allowbreak26 Da 6:22}
\crossref{Jer}{37}{19}{Jer 2:28 De 32:36,\allowbreak37 2Ki 3:13}
\crossref{Jer}{37}{20}{Jer 36:7}
\crossref{Jer}{37}{21}{Jer 32:2,\allowbreak8; 38:13,\allowbreak28}
\crossref{Jer}{38}{1}{Ezr 2:3 Ne 7:9}
\crossref{Jer}{38}{2}{38:17-\allowbreak23; 21:8,\allowbreak9; 24:8; 27:13; 29:18; 34:17; 42:17,\allowbreak22; 44:13}
\crossref{Jer}{38}{3}{Jer 21:10; 32:3-\allowbreak5}
\crossref{Jer}{38}{4}{Jer 26:11,\allowbreak21-\allowbreak23; 36:12-\allowbreak16 2Ch 24:21 Eze 22:27 Mic 3:1-\allowbreak3 Zep 3:1-\allowbreak3}
\crossref{Jer}{38}{5}{1Sa 15:24; 29:9 2Sa 3:39; 19:22 Pr 29:25 Joh 19:12-\allowbreak16}
\crossref{Jer}{38}{6}{Jer 37:21 Ps 109:5 Lu 3:19,\allowbreak20}
\crossref{Jer}{38}{7}{Jer 39:16-\allowbreak18}
\crossref{Jer}{38}{8}{}
\crossref{Jer}{38}{9}{38:1-\allowbreak6 Es 7:4-\allowbreak6 Job 31:34 Pr 24:11,\allowbreak12; 31:8,\allowbreak9}
\crossref{Jer}{38}{10}{Es 5:2; 8:7 Ps 75:10 Pr 21:1}
\crossref{Jer}{38}{11}{38:6}
\crossref{Jer}{38}{12}{Ro 12:10,\allowbreak15 Eph 4:32}
\crossref{Jer}{38}{13}{38:6}
\crossref{Jer}{38}{14}{Jer 21:1,\allowbreak2; 37:17}
\crossref{Jer}{38}{15}{Lu 22:67,\allowbreak68}
\crossref{Jer}{38}{16}{Jer 37:17 Joh 3:2}
\crossref{Jer}{38}{17}{Ps 80:7,\allowbreak14 Am 5:27}
\crossref{Jer}{38}{18}{2Ki 24:12; 25:27-\allowbreak30}
\crossref{Jer}{38}{19}{38:5 1Sa 15:24 Job 31:34 Pr 29:25 Isa 51:12,\allowbreak13; 57:11 Joh 12:42}
\crossref{Jer}{38}{20}{Jer 26:13 2Ch 20:20 Da 4:27 Ac 26:29 2Co 5:11,\allowbreak20; 6:1 Phm 1:8-\allowbreak10}
\crossref{Jer}{38}{21}{Jer 5:3 Ex 10:3,\allowbreak4; 16:28 Job 34:33 Pr 1:24-\allowbreak31 Isa 1:19,\allowbreak20}
\crossref{Jer}{38}{22}{Jer 41:10; 43:6 La 5:11}
\crossref{Jer}{38}{23}{38:18; 39:6; 41:10; 52:8-\allowbreak13 2Ki 25:7 2Ch 36:20,\allowbreak21}
\crossref{Jer}{38}{24}{Jer 37:17 1Sa 16:2}
\crossref{Jer}{38}{25}{38:4-\allowbreak6,\allowbreak27}
\crossref{Jer}{38}{26}{Jer 37:15,\allowbreak20; 42:2 Es 4:8}
\crossref{Jer}{38}{27}{1Sa 10:15,\allowbreak16; 16:2-\allowbreak5 2Ki 6:19 Ac 23:6}
\crossref{Jer}{38}{28}{38:13; 15:20,\allowbreak21; 37:21; 39:14 Ps 23:4 2Ti 3:11; 4:17,\allowbreak18}
\crossref{Jer}{39}{1}{Jer 52:4-\allowbreak7 2Ki 25:1,\allowbreak2-\allowbreak7 Eze 24:1,\allowbreak2 Zec 8:19}
\crossref{Jer}{39}{2}{Jer 5:10; 52:6,\allowbreak7 2Ki 25:4 Eze 33:21 Mic 2:12,\allowbreak13 Zep 1:10}
\crossref{Jer}{39}{3}{Jer 1:15; 21:4; 38:17}
\crossref{Jer}{39}{4}{Jer 38:18-\allowbreak20 Le 26:17,\allowbreak36 De 28:25; 32:24-\allowbreak30 2Ki 25:4-\allowbreak7}
\crossref{Jer}{39}{5}{Jer 32:4,\allowbreak5; 38:18,\allowbreak23 2Ch 33:11 La 1:3; 4:20}
\crossref{Jer}{39}{6}{Jer 52:10 2Ki 25:7}
\crossref{Jer}{39}{7}{Jer 32:4,\allowbreak5; 52:11 2Ki 25:7 Eze 12:13}
\crossref{Jer}{39}{8}{Jer 7:20; 9:10-\allowbreak12; 17:27; 21:10; 34:2,\allowbreak22; 37:10; 38:18; 52:13 2Ki 25:9}
\crossref{Jer}{39}{9}{39:13; 40:1; 52:12-\allowbreak16,\allowbreak26 2Ki 25:11,\allowbreak20}
\crossref{Jer}{39}{10}{Jer 40:7 2Ki 25:12 Eze 33:24}
\crossref{Jer}{39}{11}{Jer 15:11,\allowbreak21 Job 5:19 Ac 24:23}
\crossref{Jer}{39}{12}{Jer 24:6; 40:4}
\crossref{Jer}{39}{13}{39:3,\allowbreak9}
\crossref{Jer}{39}{14}{39:15; 37:21; 38:13,\allowbreak28; 40:1-\allowbreak4 Ps 105:19}
\crossref{Jer}{39}{15}{39:14; 32:1,\allowbreak2; 36:1-\allowbreak5; 37:21 2Ti 2:9}
\crossref{Jer}{39}{16}{Jer 38:7-\allowbreak13}
\crossref{Jer}{39}{17}{Jer 1:19 Job 5:19-\allowbreak21 Ps 41:1,\allowbreak2; 50:15; 91:14,\allowbreak15 Da 6:16 Mt 10:40-\allowbreak42}
\crossref{Jer}{39}{18}{Jer 21:9; 38:2; 45:4,\allowbreak5}
\crossref{Jer}{40}{1}{Jer 39:11-\allowbreak14}
\crossref{Jer}{40}{2}{Jer 22:8,\allowbreak9 De 29:24-\allowbreak28 1Ki 9:8,\allowbreak9 2Ch 7:20-\allowbreak22 La 2:15-\allowbreak17}
\crossref{Jer}{40}{3}{Jer 50:7 Ne 9:28,\allowbreak33 Da 9:11,\allowbreak12 Ro 2:5; 3:19}
\crossref{Jer}{40}{4}{40:1}
\crossref{Jer}{40}{5}{Jer 39:14; 41:2 2Ki 25:22-\allowbreak24}
\crossref{Jer}{40}{6}{Jos 15:38 Jud 20:1; 21:1 1Sa 7:5,\allowbreak6}
\crossref{Jer}{40}{7}{Jer 39:4 2Ki 25:4,\allowbreak22,\allowbreak23-\allowbreak26}
\crossref{Jer}{40}{8}{40:6,\allowbreak11,\allowbreak12}
\crossref{Jer}{40}{9}{1Sa 20:16,\allowbreak17 2Ki 25:24}
\crossref{Jer}{40}{10}{Jer 35:19 De 1:38 Pr 22:29 Lu 21:36}
\crossref{Jer}{40}{11}{Jer 24:9 Isa 16:4 Eze 5:3,\allowbreak12; 25:2,\allowbreak6,\allowbreak8,\allowbreak12; 35:5,\allowbreak15 Ob 1:11-\allowbreak14}
\crossref{Jer}{40}{12}{Jer 43:5}
\crossref{Jer}{40}{13}{40:6-\allowbreak8}
\crossref{Jer}{40}{14}{Jer 25:21; 41:10; 49:1-\allowbreak6 1Sa 11:1-\allowbreak3 2Sa 10:1-\allowbreak6 Eze 25:2-\allowbreak6}
\crossref{Jer}{40}{15}{1Sa 24:4; 26:8 Job 31:31}
\crossref{Jer}{40}{16}{Jer 41:2 Mt 10:16,\allowbreak17 Ro 3:8}
\crossref{Jer}{41}{1}{Jer 40:6,\allowbreak8}
\crossref{Jer}{41}{2}{2Ki 25:25}
\crossref{Jer}{41}{3}{41:11,\allowbreak12 2Ki 25:25 Ec 9:18 La 1:2}
\crossref{Jer}{41}{4}{1Sa 27:11 Ps 52:1,\allowbreak2}
\crossref{Jer}{41}{5}{2Ki 10:13,\allowbreak14}
\crossref{Jer}{41}{6}{Jer 50:4 2Sa 1:2-\allowbreak16; 3:16 Pr 26:23-\allowbreak26}
\crossref{Jer}{41}{7}{1Ki 15:28,\allowbreak29; 16:10-\allowbreak12 2Ki 11:1,\allowbreak2; 15:25 Ps 55:23 Pr 1:16}
\crossref{Jer}{41}{8}{Job 2:4 Ps 49:6-\allowbreak8 Pr 13:8 Mt 6:25; 16:26 Mr 8:36,\allowbreak37 Php 3:7-\allowbreak9}
\crossref{Jer}{41}{9}{1Ki 15:17-\allowbreak22 2Ch 16:1-\allowbreak10}
\crossref{Jer}{41}{10}{Jer 40:11,\allowbreak12}
\crossref{Jer}{41}{11}{41:2,\allowbreak3,\allowbreak7; 40:7,\allowbreak8,\allowbreak13-\allowbreak16; 42:1,\allowbreak3; 43:2-\allowbreak5}
\crossref{Jer}{41}{12}{Ge 14:14-\allowbreak16 1Sa 30:1-\allowbreak8,\allowbreak18-\allowbreak20}
\crossref{Jer}{41}{13}{41:13,\allowbreak16; 40:13}
\crossref{Jer}{41}{14}{41:10,\allowbreak16}
\crossref{Jer}{41}{15}{1Sa 30:17 1Ki 20:20 Job 21:30 Pr 28:17 Ec 8:11,\allowbreak12 Ac 28:4}
\crossref{Jer}{41}{16}{}
\crossref{Jer}{41}{17}{2Sa 19:37,\allowbreak38}
\crossref{Jer}{41}{18}{Jer 42:11,\allowbreak16; 43:2,\allowbreak3 2Ki 25:25 Isa 30:16,\allowbreak17; 51:12,\allowbreak13; 57:11}
\crossref{Jer}{42}{1}{42:8; 40:8,\allowbreak13; 41:11,\allowbreak16; 43:4,\allowbreak5}
\crossref{Jer}{42}{2}{Jer 36:7; 37:20}
\crossref{Jer}{42}{3}{Jer 6:16 De 5:26,\allowbreak29 1Ki 8:36 Ezr 8:21 Ps 25:4,\allowbreak5; 27:11; 86:11}
\crossref{Jer}{42}{4}{Ex 8:29 1Sa 12:23 Ro 10:1}
\crossref{Jer}{42}{5}{Jer 5:2 Ge 31:50 Ex 20:7 Jud 11:10 1Sa 12:5; 20:42 Mic 1:2}
\crossref{Jer}{42}{6}{Ro 7:7,\allowbreak13; 8:7}
\crossref{Jer}{42}{7}{}
\crossref{Jer}{42}{8}{42:1; 40:8,\allowbreak13; 41:11-\allowbreak16; 43:2-\allowbreak5}
\crossref{Jer}{42}{9}{42:2 2Ki 19:4,\allowbreak6,\allowbreak20-\allowbreak37; 22:15-\allowbreak20}
\crossref{Jer}{42}{10}{Ge 26:2,\allowbreak3 Ps 37:3}
\crossref{Jer}{42}{11}{Jer 27:12,\allowbreak17; 41:18 2Ki 25:26 Mt 10:28}
\crossref{Jer}{42}{12}{Ne 1:11 Ps 106:45,\allowbreak46 Pr 16:7}
\crossref{Jer}{42}{13}{42:10; 44:16 Ex 5:2}
\crossref{Jer}{42}{14}{Jer 41:17; 43:7 De 29:19 Isa 30:16; 31:1}
\crossref{Jer}{42}{15}{}
\crossref{Jer}{42}{16}{42:13; 44:13,\allowbreak27 De 28:15,\allowbreak22,\allowbreak45 Pr 13:21 Eze 11:8 Am 9:1-\allowbreak4}
\crossref{Jer}{42}{17}{42:22; 24:10; 44:14}
\crossref{Jer}{42}{18}{Jer 18:16; 24:9; 25:9; 26:6; 29:18,\allowbreak22; 44:12 De 29:21,\allowbreak22 1Ki 9:7-\allowbreak9}
\crossref{Jer}{42}{19}{Jer 38:21 Eze 3:21 Ac 20:26,\allowbreak27}
\crossref{Jer}{42}{20}{42:2}
\crossref{Jer}{42}{21}{De 11:26,\allowbreak27 Eze 2:7; 3:17 Ac 20:20,\allowbreak26,\allowbreak27}
\crossref{Jer}{42}{22}{42:17; 43:11 Eze 5:3,\allowbreak4; 6:11}
\crossref{Jer}{43}{1}{Jer 26:8; 42:22; 51:63}
\crossref{Jer}{43}{2}{Jer 40:8; 43:1}
\crossref{Jer}{43}{3}{43:6; 36:4,\allowbreak10,\allowbreak26; 45:1-\allowbreak3}
\crossref{Jer}{43}{4}{Jer 42:5,\allowbreak6; 44:5 2Ch 25:16 Ec 9:16}
\crossref{Jer}{43}{5}{Jer 40:11,\allowbreak12; 41:15,\allowbreak16 1Sa 26:19}
\crossref{Jer}{43}{6}{Jer 41:10; 52:10}
\crossref{Jer}{43}{7}{2Ch 25:16}
\crossref{Jer}{43}{8}{Ps 139:7 2Ti 2:9}
\crossref{Jer}{43}{9}{Jer 13:1-\allowbreak11; 18:2-\allowbreak12; 19:1-\allowbreak15; 51:63,\allowbreak64 1Ki 11:29-\allowbreak31 Isa 20:1-\allowbreak4}
\crossref{Jer}{43}{10}{Jer 1:15; 25:6-\allowbreak26; 27:6-\allowbreak8 Eze 29:18-\allowbreak20 Da 2:21; 5:18,\allowbreak19}
\crossref{Jer}{43}{11}{Jer 25:19; 46:1-\allowbreak26 Isa 19:1-\allowbreak25 Eze 29:19,\allowbreak20; 30:1-\allowbreak32:32}
\crossref{Jer}{43}{12}{Jer 46:25; 48:7; 50:2; 51:44 Ex 12:12 2Sa 5:21 Isa 19:1; 21:9; 46:1}
\crossref{Jer}{43}{13}{43:12}
\crossref{Jer}{44}{1}{Jer 42:15-\allowbreak18; 43:5-\allowbreak7}
\crossref{Jer}{44}{2}{Jer 39:1-\allowbreak8 Ex 19:4 De 29:2 Jos 23:3 Zec 1:6}
\crossref{Jer}{44}{3}{Jer 2:17-\allowbreak19; 4:17,\allowbreak18; 5:19,\allowbreak29; 9:12-\allowbreak14; 11:17; 16:11,\allowbreak12; 19:3,\allowbreak4; 22:9}
\crossref{Jer}{44}{4}{Jer 7:13,\allowbreak25; 25:3,\allowbreak4; 26:5; 29:19; 32:33; 35:17 2Ch 36:15 Zec 7:7}
\crossref{Jer}{44}{5}{Jer 7:24 2Ch 36:16 Ps 81:11-\allowbreak13 Isa 48:4,\allowbreak18 Zec 7:11,\allowbreak12 Re 2:21,\allowbreak22}
\crossref{Jer}{44}{6}{Jer 4:4; 7:20; 21:5,\allowbreak12; 36:7; 42:18 Le 26:28 Isa 51:17,\allowbreak20 Eze 5:13}
\crossref{Jer}{44}{7}{Jer 7:19; 25:7; 42:20}
\crossref{Jer}{44}{8}{Jer 25:6,\allowbreak7 De 32:16,\allowbreak17 2Ki 17:15-\allowbreak17 Isa 3:8 1Co 10:21,\allowbreak22 Heb 3:16}
\crossref{Jer}{44}{9}{Jos 22:17-\allowbreak20 Ezr 9:7-\allowbreak15 Da 9:5-\allowbreak8}
\crossref{Jer}{44}{10}{Jer 8:12 Ex 9:17; 10:3 1Ki 21:29 2Ch 12:6-\allowbreak12; 32:26; 33:12,\allowbreak19; 34:27}
\crossref{Jer}{44}{11}{Jer 21:10 Le 17:10; 20:5,\allowbreak6; 26:17 Ps 34:16 Eze 14:7,\allowbreak8; 15:7 Am 9:4}
\crossref{Jer}{44}{12}{Jer 42:15-\allowbreak18,\allowbreak22}
\crossref{Jer}{44}{13}{44:27,\allowbreak28}
\crossref{Jer}{44}{14}{Isa 30:1-\allowbreak3}
\crossref{Jer}{44}{15}{Jer 5:1-\allowbreak5 Ge 19:4 Ne 13:26 Pr 11:21 Isa 1:5 Mt 7:13 2Pe 2:1,\allowbreak2}
\crossref{Jer}{44}{16}{Jer 16:15-\allowbreak17; 8:6,\allowbreak12; 18:18; 38:4 Ex 5:2 Job 15:25-\allowbreak27; 21:14,\allowbreak15}
\crossref{Jer}{44}{17}{44:25 Nu 30:2,\allowbreak12 De 23:23 Jud 11:36 Ps 12:4 Mr 6:26}
\crossref{Jer}{44}{18}{Jer 40:12 Nu 11:5,\allowbreak6 Job 21:14,\allowbreak15 Ps 73:9-\allowbreak15 Mal 3:13-\allowbreak15}
\crossref{Jer}{44}{19}{44:15; 7:18}
\crossref{Jer}{44}{20}{Jer 43:6}
\crossref{Jer}{44}{21}{44:9,\allowbreak17; 11:13 Eze 16:24}
\crossref{Jer}{44}{22}{Jer 15:6 Ge 6:3,\allowbreak5-\allowbreak7 Ps 95:10,\allowbreak11 Isa 1:24; 7:13; 43:24 Eze 5:13}
\crossref{Jer}{44}{23}{44:8,\allowbreak18,\allowbreak21; 32:31-\allowbreak33 2Ch 36:16 La 1:8 1Co 10:20 2Co 6:16}
\crossref{Jer}{44}{24}{44:16; 42:15 1Ki 22:19 Isa 1:10; 28:14 Eze 2:7 Am 7:16 Mt 11:15}
\crossref{Jer}{44}{25}{44:15-\allowbreak19 Isa 28:15 Jude 1:13}
\crossref{Jer}{44}{26}{Jer 46:18 Ge 22:16 Nu 14:21-\allowbreak23,\allowbreak28 De 32:40-\allowbreak42 Ps 89:34 Isa 62:8}
\crossref{Jer}{44}{27}{Jer 1:10}
\crossref{Jer}{44}{28}{44:14 Isa 10:19,\allowbreak22; 27:12,\allowbreak13}
\crossref{Jer}{44}{29}{44:30 1Sa 2:34 Mt 24:15,\allowbreak16,\allowbreak32-\allowbreak34 Mr 13:14-\allowbreak16 Lu 21:20,\allowbreak21}
\crossref{Jer}{44}{30}{Jer 43:9-\allowbreak13; 46:13-\allowbreak26 Eze 29:1-\allowbreak30:26; 31:18; 32:1-\allowbreak32}
\crossref{Jer}{45}{1}{Jer 32:12,\allowbreak16; 43:3-\allowbreak6}
\crossref{Jer}{45}{2}{Isa 63:9 Mr 16:7 2Co 1:4; 7:6 Heb 2:18; 4:15}
\crossref{Jer}{45}{3}{Jer 9:1; 15:10-\allowbreak21; 20:7-\allowbreak18 Ps 120:5}
\crossref{Jer}{45}{4}{Jer 1:10; 18:7-\allowbreak10; 31:28 Ge 6:6,\allowbreak7 Ps 80:8-\allowbreak16 Isa 5:2-\allowbreak7}
\crossref{Jer}{45}{5}{2Ki 5:26 Ro 12:16 1Co 7:26-\allowbreak32 1Ti 6:6-\allowbreak9 Heb 13:5}
\crossref{Jer}{46}{1}{Jer 1:10; 4:7; 25:15-\allowbreak29 Ge 10:5 Nu 23:9 Zec 2:8 Ro 3:29}
\crossref{Jer}{46}{2}{46:14; 25:9,\allowbreak19 Eze 29:1-\allowbreak32:32}
\crossref{Jer}{46}{3}{}
\crossref{Jer}{46}{4}{Eze 21:9-\allowbreak11,\allowbreak28}
\crossref{Jer}{46}{5}{Re 6:15}
\crossref{Jer}{46}{6}{Jud 4:15-\allowbreak21 Ps 33:16,\allowbreak17; 147:10,\allowbreak11 Ec 9:11 Isa 30:16,\allowbreak17}
\crossref{Jer}{46}{7}{So 3:6; 8:5 Isa 63:1}
\crossref{Jer}{46}{8}{Eze 29:3; 32:2}
\crossref{Jer}{46}{9}{Na 2:3,\allowbreak4}
\crossref{Jer}{46}{10}{Jer 51:6 Isa 13:6; 34:6,\allowbreak8; 61:2; 63:4 Joe 1:15; 2:1 Zep 1:14,\allowbreak15}
\crossref{Jer}{46}{11}{Jer 8:22; 51:8 Ge 37:25; 43:11 Eze 27:17}
\crossref{Jer}{46}{12}{Eze 32:9-\allowbreak12 Na 3:8-\allowbreak10}
\crossref{Jer}{46}{13}{Jer 43:10-\allowbreak13; 44:30 Isa 19:1-\allowbreak25; 29:1-\allowbreak32:20}
\crossref{Jer}{46}{14}{Jer 43:9; 44:1 Ex 14:2 Eze 30:16-\allowbreak18}
\crossref{Jer}{46}{15}{46:5,\allowbreak21 Jud 5:20,\allowbreak21 Isa 66:15,\allowbreak16}
\crossref{Jer}{46}{16}{Le 26:36,\allowbreak37}
\crossref{Jer}{46}{17}{Ex 15:9 1Ki 20:10,\allowbreak18 Isa 19:11-\allowbreak16; 31:3; 37:27-\allowbreak29 Eze 29:3}
\crossref{Jer}{46}{18}{Jer 10:10; 44:26; 48:15; 51:17 Isa 47:4; 48:2 Mal 1:14 Mt 5:35}
\crossref{Jer}{46}{19}{Jer 48:18}
\crossref{Jer}{46}{20}{Jer 50:11 Ho 10:11}
\crossref{Jer}{46}{21}{46:9,\allowbreak16 2Sa 10:6 2Ki 7:6 Eze 27:10,\allowbreak11; 30:4-\allowbreak6}
\crossref{Jer}{46}{22}{Isa 29:4 Mic 1:8; 7:16}
\crossref{Jer}{46}{23}{Isa 10:18 Eze 20:46}
\crossref{Jer}{46}{24}{46:11,\allowbreak19 Ps 137:8}
\crossref{Jer}{46}{25}{Eze 30:14 Na 3:8}
\crossref{Jer}{46}{26}{Jer 44:30 Eze 32:11}
\crossref{Jer}{46}{27}{Jer 30:10,\allowbreak11 Isa 41:13,\allowbreak14; 43:1,\allowbreak5; 44:2}
\crossref{Jer}{46}{28}{Jer 1:19; 15:20; 30:11 Jos 1:5,\allowbreak9 Ps 46:7,\allowbreak11 Isa 8:9,\allowbreak10; 41:10; 43:2}
\crossref{Jer}{47}{1}{Ex 25:15-\allowbreak17 Am 1:6-\allowbreak8 Zep 2:4-\allowbreak7 Zec 9:5-\allowbreak7}
\crossref{Jer}{47}{2}{Jer 46:7,\allowbreak8 Isa 8:7,\allowbreak8; 28:17; 59:19 Da 11:22 Am 9:5,\allowbreak6 Na 1:8}
\crossref{Jer}{47}{3}{Jer 8:16; 46:9 Jud 5:22 Job 39:19-\allowbreak25 Eze 26:10,\allowbreak11 Na 2:4; 3:2,\allowbreak3}
\crossref{Jer}{47}{4}{Jer 46:10 Ps 37:13 Isa 10:3 Eze 7:5-\allowbreak7,\allowbreak12; 21:25,\allowbreak29 Ho 9:7 Lu 21:22}
\crossref{Jer}{47}{5}{Jer 48:37 Isa 15:2 Eze 7:18 Mic 1:16}
\crossref{Jer}{47}{6}{Jer 12:12; 15:3; 25:27; 51:20-\allowbreak23 De 32:41,\allowbreak42 Ps 17:13 Isa 10:5,\allowbreak15}
\crossref{Jer}{47}{7}{1Sa 15:3 Isa 10:6; 13:3; 37:26; 45:1-\allowbreak3; 46:10,\allowbreak11 Eze 14:17 Am 3:6}
\crossref{Jer}{48}{1}{Jer 9:26; 25:21; 27:3 Ge 19:37 Nu 24:17 2Ch 20:10 Isa 15:1-\allowbreak16:14}
\crossref{Jer}{48}{2}{48:17 Isa 16:14}
\crossref{Jer}{48}{3}{Jer 4:20,\allowbreak21; 47:2 Isa 15:2,\allowbreak8; 16:7-\allowbreak11; 22:4}
\crossref{Jer}{48}{4}{Es 8:11 Ps 137:9}
\crossref{Jer}{48}{5}{}
\crossref{Jer}{48}{6}{Jer 51:6 Ge 19:17 Ps 11:1 Pr 6:4,\allowbreak5 Mt 24:16-\allowbreak18 Lu 3:7; 17:31-\allowbreak33}
\crossref{Jer}{48}{7}{Jer 9:23; 13:25 Ps 40:4; 49:6,\allowbreak7; 52:7; 62:8-\allowbreak10 Isa 59:4-\allowbreak6 Eze 28:2-\allowbreak5}
\crossref{Jer}{48}{8}{48:18; 6:26; 15:8; 25:9; 51:56}
\crossref{Jer}{48}{9}{48:28 Ps 11:1; 55:6 Isa 16:2 Re 12:14}
\crossref{Jer}{48}{10}{Jer 50:25 Nu 31:14-\allowbreak18 Jud 5:23 1Sa 15:3,\allowbreak9,\allowbreak13-\allowbreak35 1Ki 20:42}
\crossref{Jer}{48}{11}{Ps 55:19; 73:4-\allowbreak8; 123:4 Pr 1:32}
\crossref{Jer}{48}{12}{48:8,\allowbreak15; 25:9 Isa 16:2 Eze 25:9,\allowbreak10}
\crossref{Jer}{48}{13}{48:7,\allowbreak39,\allowbreak46 Jud 11:24 1Sa 5:3-\allowbreak7 1Ki 11:7; 18:26-\allowbreak29,\allowbreak40 Isa 2:20}
\crossref{Jer}{48}{14}{Jer 8:8 Ps 11:1 Isa 36:4,\allowbreak5}
\crossref{Jer}{48}{15}{48:8,\allowbreak9-\allowbreak25}
\crossref{Jer}{48}{16}{Jer 1:12 De 32:35 Isa 13:22; 16:13,\allowbreak14 Eze 12:23,\allowbreak28 2Pe 2:3}
\crossref{Jer}{48}{17}{48:31-\allowbreak33; 9:17-\allowbreak20 Isa 16:8 Re 18:14-\allowbreak20}
\crossref{Jer}{48}{18}{Jer 46:18,\allowbreak19 Isa 47:1}
\crossref{Jer}{48}{19}{Nu 32:34 De 2:36 2Sa 24:5 1Ch 5:8}
\crossref{Jer}{48}{20}{48:1-\allowbreak5 Isa 15:1-\allowbreak5,\allowbreak8; 16:7-\allowbreak11}
\crossref{Jer}{48}{21}{48:8 Eze 25:9 Zep 2:9}
\crossref{Jer}{48}{22}{48:1,\allowbreak18 Nu 32:34}
\crossref{Jer}{48}{23}{48:1 Ge 14:5}
\crossref{Jer}{48}{24}{48:41 Am 2:2}
\crossref{Jer}{48}{25}{Ps 75:10 La 2:3 Da 7:8; 8:7-\allowbreak9,\allowbreak21 Zec 1:19-\allowbreak21}
\crossref{Jer}{48}{26}{Jer 13:13,\allowbreak14; 25:15-\allowbreak17,\allowbreak27-\allowbreak29; 51:7,\allowbreak39,\allowbreak57 Ps 60:3; 75:8 Isa 29:9}
\crossref{Jer}{48}{27}{Ps 44:13; 79:4 Pr 24:17,\allowbreak18 La 2:15-\allowbreak17 Eze 25:8; 26:2,\allowbreak3; 35:15}
\crossref{Jer}{48}{28}{48:9 Jud 6:2 1Sa 13:6 Isa 2:19 Ob 1:3,\allowbreak4}
\crossref{Jer}{48}{29}{Pr 8:13 Isa 16:6 Zep 2:8-\allowbreak15}
\crossref{Jer}{48}{30}{Isa 16:6; 37:28,\allowbreak29}
\crossref{Jer}{48}{31}{Isa 15:5; 16:7-\allowbreak11}
\crossref{Jer}{48}{32}{Nu 32:38}
\crossref{Jer}{48}{33}{Jer 25:9,\allowbreak10 Isa 9:3; 16:9; 24:7-\allowbreak12; 32:9-\allowbreak14 Joe 1:12,\allowbreak16 Re 18:22,\allowbreak23}
\crossref{Jer}{48}{34}{48:2 Isa 15:4-\allowbreak6}
\crossref{Jer}{48}{35}{48:7 Nu 22:40,\allowbreak41; 28:14,\allowbreak28-\allowbreak30 Isa 15:2; 16:12}
\crossref{Jer}{48}{36}{Jer 4:19 Isa 15:5; 16:11; 63:15}
\crossref{Jer}{48}{37}{Jer 16:6; 41:5; 47:5 Isa 3:24; 15:2,\allowbreak3 Eze 7:18; 27:31 Am 8:10}
\crossref{Jer}{48}{38}{Isa 15:3; 22:1}
\crossref{Jer}{48}{39}{48:17 La 1:1; 2:1; 4:1 Re 18:9,\allowbreak10,\allowbreak15,\allowbreak16}
\crossref{Jer}{48}{40}{Jer 4:13 De 28:49 La 4:19 Eze 17:3 Da 7:4 Ho 8:1}
\crossref{Jer}{48}{41}{48:24}
\crossref{Jer}{48}{42}{48:2; 30:11 Es 3:8-\allowbreak13 Ps 83:4-\allowbreak8 Isa 7:8 Mt 7:2}
\crossref{Jer}{48}{43}{De 32:23-\allowbreak25 Ps 11:6 Isa 24:17,\allowbreak18 La 3:47}
\crossref{Jer}{48}{44}{Jer 16:16 1Ki 19:17; 20:30 Isa 37:36-\allowbreak38 Am 2:14,\allowbreak15; 5:19; 9:1-\allowbreak4}
\crossref{Jer}{48}{45}{Nu 21:28 Am 2:2}
\crossref{Jer}{48}{46}{Nu 21:29}
\crossref{Jer}{48}{47}{Jer 23:20; 30:24 Nu 24:14 De 4:30; 31:29 Job 19:25 Eze 38:8 Da 2:28}
\crossref{Jer}{49}{1}{49:7,\allowbreak23,\allowbreak28; 48:1}
\crossref{Jer}{49}{2}{Jer 4:19 Eze 25:4-\allowbreak6 Am 1:14}
\crossref{Jer}{49}{3}{Jer 48:20; 51:8 Isa 13:6; 14:31; 15:2; 16:7; 23:1,\allowbreak6 Jas 5:1}
\crossref{Jer}{49}{4}{Jer 9:23 Isa 28:1-\allowbreak4; 47:7,\allowbreak8 Re 18:7}
\crossref{Jer}{49}{5}{49:29; 15:8; 20:4; 48:41-\allowbreak44 Jos 2:9 2Ki 7:6,\allowbreak7; 19:7 Job 15:21}
\crossref{Jer}{49}{6}{49:39; 46:26; 48:47 Isa 19:18-\allowbreak23; 23:18 Eze 16:53}
\crossref{Jer}{49}{7}{Jer 25:9,\allowbreak21 Ge 25:30; 27:41; 36:8 Nu 20:14-\allowbreak21; 24:17,\allowbreak18 De 23:7}
\crossref{Jer}{49}{8}{49:30; 6:1; 48:6 Mt 24:15-\allowbreak18 Re 6:15}
\crossref{Jer}{49}{9}{Isa 17:6 Ob 1:5,\allowbreak6}
\crossref{Jer}{49}{10}{Mal 1:3,\allowbreak4 Ro 9:13}
\crossref{Jer}{49}{11}{De 10:18 Ps 10:14-\allowbreak18; 68:5; 82:3; 146:9 Pr 23:10,\allowbreak11 Ho 14:3}
\crossref{Jer}{49}{12}{Jer 25:28,\allowbreak29; 30:11; 46:27 Pr 17:5 La 4:21,\allowbreak22 Ob 1:16 1Pe 4:17,\allowbreak18}
\crossref{Jer}{49}{13}{Jer 44:26 Ge 22:16 Isa 45:23 Eze 35:11 Am 6:8}
\crossref{Jer}{49}{14}{Jer 51:46 Isa 37:7 Eze 7:25,\allowbreak26 Ob 1:1 Mt 24:6}
\crossref{Jer}{49}{15}{1Sa 2:7,\allowbreak8,\allowbreak30 Ps 53:5 Ob 1:2 Mic 7:10 Lu 1:51}
\crossref{Jer}{49}{16}{Jer 48:29 Pr 16:18; 18:21; 29:23 Isa 25:4,\allowbreak5; 49:25 Ob 1:3}
\crossref{Jer}{49}{17}{49:13 Isa 34:9-\allowbreak15 Eze 25:13; 35:7,\allowbreak15}
\crossref{Jer}{49}{18}{Jer 50:40 Ge 19:24,\allowbreak25 De 29:23 Ps 11:6 Isa 13:19-\allowbreak22 Am 4:11}
\crossref{Jer}{49}{19}{Jer 4:7; 50:44-\allowbreak46 Zec 11:3}
\crossref{Jer}{49}{20}{Jer 50:45 Ps 33:11 Pr 19:21 Isa 14:24-\allowbreak27; 46:10,\allowbreak11 Ac 4:28}
\crossref{Jer}{49}{21}{Jer 50:46 Isa 14:4-\allowbreak15 Eze 26:15-\allowbreak18; 21:16; 32:10 Re 18:10}
\crossref{Jer}{49}{22}{Jer 4:13; 48:40,\allowbreak41 De 28:49 Da 7:4 Ho 8:1}
\crossref{Jer}{49}{23}{Ge 14:15; 15:2 1Ki 11:24 Isa 17:1-\allowbreak3; 37:13 Am 1:3-\allowbreak5 Zec 9:1,\allowbreak2}
\crossref{Jer}{49}{24}{49:22}
\crossref{Jer}{49}{25}{Jer 33:9; 48:2,\allowbreak39; 51:41 Ps 37:35,\allowbreak36 Isa 1:26; 14:4-\allowbreak6 Da 4:30}
\crossref{Jer}{49}{26}{Jer 9:21; 11:22; 50:30; 51:3,\allowbreak4 La 2:21 Eze 27:27 Am 4:10}
\crossref{Jer}{49}{27}{Am 1:3-\allowbreak5}
\crossref{Jer}{49}{28}{Jer 2:10 Ge 25:13 1Ch 1:29 So 1:5 Isa 21:13,\allowbreak16,\allowbreak17; 42:11 Eze 27:21}
\crossref{Jer}{49}{29}{Ps 120:5 Isa 13:20; 60:7}
\crossref{Jer}{49}{30}{49:8}
\crossref{Jer}{49}{31}{Jer 48:11 Ps 123:4 Isa 32:9,\allowbreak11}
\crossref{Jer}{49}{32}{49:29}
\crossref{Jer}{49}{33}{49:17,\allowbreak18; 9:11; 10:22; 50:39,\allowbreak40; 51:37 Isa 13:20-\allowbreak22; 14:23; 34:9-\allowbreak17}
\crossref{Jer}{49}{34}{}
\crossref{Jer}{49}{35}{Jer 50:14,\allowbreak29; 51:56 Ps 46:9 Isa 22:6}
\crossref{Jer}{49}{36}{Da 7:2,\allowbreak3; 8:8,\allowbreak22; 11:4 Re 7:1}
\crossref{Jer}{49}{37}{49:5,\allowbreak22,\allowbreak24,\allowbreak29; 48:39; 50:36 Ps 48:4-\allowbreak6 Eze 32:23}
\crossref{Jer}{49}{38}{Jer 43:10 Da 7:9-\allowbreak14}
\crossref{Jer}{49}{39}{Jer 48:47 Isa 2:2 Eze 38:16 Da 2:28; 10:14 Ho 3:5 Mic 4:1}
\crossref{Jer}{50}{1}{Jer 25:26,\allowbreak27; 27:7; 51:1-\allowbreak14 Ps 137:8,\allowbreak9 Isa 13:1-\allowbreak3; 14:4; 21:1-\allowbreak10}
\crossref{Jer}{50}{2}{Jer 6:18; 31:10; 46:14 Ps 64:9; 96:3 Isa 12:4; 48:6; 66:18,\allowbreak19}
\crossref{Jer}{50}{3}{50:12,\allowbreak13,\allowbreak35-\allowbreak40; 51:8,\allowbreak9,\allowbreak25,\allowbreak26,\allowbreak37-\allowbreak44,\allowbreak62 Isa 13:6-\allowbreak10,\allowbreak19-\allowbreak22; 14:22-\allowbreak24}
\crossref{Jer}{50}{4}{50:20; 3:16-\allowbreak18; 33:15; 51:47,\allowbreak48 Isa 63:4}
\crossref{Jer}{50}{5}{Jer 6:16 Ps 25:8,\allowbreak9; 84:7 Isa 35:8 Joh 7:17}
\crossref{Jer}{50}{6}{50:17 Ps 119:176 Isa 53:6 Mt 9:36; 10:6; 15:24; 18:11-\allowbreak13 Lu 15:4-\allowbreak7}
\crossref{Jer}{50}{7}{50:17,\allowbreak33; 12:7-\allowbreak11 Ps 79:7 Isa 9:12; 56:9}
\crossref{Jer}{50}{8}{Jer 51:6,\allowbreak45 Nu 16:26 Isa 48:20; 52:1 Zec 2:6,\allowbreak7 2Co 6:17 Re 18:4}
\crossref{Jer}{50}{9}{50:3,\allowbreak21,\allowbreak26,\allowbreak41,\allowbreak42; 15:14; 51:1-\allowbreak4,\allowbreak11,\allowbreak27,\allowbreak28 Ezr 1:1,\allowbreak2 Isa 13:2-\allowbreak5,\allowbreak17}
\crossref{Jer}{50}{10}{Jer 25:12; 27:7}
\crossref{Jer}{50}{11}{Pr 17:5 La 1:21; 2:15,\allowbreak16; 4:21,\allowbreak22 Eze 25:3-\allowbreak8,\allowbreak15-\allowbreak17; 26:2,\allowbreak3 Ob 1:12}
\crossref{Jer}{50}{12}{Jer 49:2 Ga 4:26 Re 17:5}
\crossref{Jer}{50}{13}{Zec 1:15}
\crossref{Jer}{50}{14}{50:9; 51:2,\allowbreak11,\allowbreak12,\allowbreak27 1Sa 17:20 2Sa 10:9 Isa 13:4,\allowbreak17,\allowbreak18}
\crossref{Jer}{50}{15}{Jer 51:14 Jos 6:5,\allowbreak20 Eze 21:22}
\crossref{Jer}{50}{16}{Jer 51:23 Joe 1:11 Am 5:16}
\crossref{Jer}{50}{17}{50:6; 23:1,\allowbreak2 Eze 34:5,\allowbreak6,\allowbreak12 Joe 3:2 Mt 9:36-\allowbreak38 Lu 15:4-\allowbreak6}
\crossref{Jer}{50}{18}{Isa 37:36-\allowbreak38 Eze 31:3-\allowbreak17 Na 1:1-\allowbreak3:19 Zep 2:13-\allowbreak15}
\crossref{Jer}{50}{19}{50:4,\allowbreak5; 3:18; 23:3; 24:6,\allowbreak7; 30:10,\allowbreak18; 31:8-\allowbreak10; 32:37; 33:7-\allowbreak12 Isa 65:9}
\crossref{Jer}{50}{20}{50:4; 33:15}
\crossref{Jer}{50}{21}{50:3,\allowbreak9,\allowbreak15}
\crossref{Jer}{50}{22}{Jer 4:19-\allowbreak21; 51:54-\allowbreak56 Isa 21:2-\allowbreak4}
\crossref{Jer}{50}{23}{Jer 51:20-\allowbreak24 Isa 14:4-\allowbreak6,\allowbreak12-\allowbreak17 Re 18:16}
\crossref{Jer}{50}{24}{Ec 9:12}
\crossref{Jer}{50}{25}{50:35-\allowbreak38; 51:11,\allowbreak20 Ps 45:3,\allowbreak5 Isa 13:2-\allowbreak5,\allowbreak17,\allowbreak18; 21:7-\allowbreak9}
\crossref{Jer}{50}{26}{50:41; 51:27,\allowbreak28 Isa 5:26}
\crossref{Jer}{50}{27}{50:11; 46:21 Ps 22:12 Isa 34:7 Eze 39:17-\allowbreak20 Re 19:17}
\crossref{Jer}{50}{28}{Jer 51:50,\allowbreak51 Isa 48:20}
\crossref{Jer}{50}{29}{50:9,\allowbreak14,\allowbreak26}
\crossref{Jer}{50}{30}{Jer 9:21; 18:21; 48:15; 49:26; 51:3,\allowbreak4 Isa 13:15-\allowbreak18}
\crossref{Jer}{50}{31}{Jer 21:13; 51:25 Eze 5:8; 29:3,\allowbreak9,\allowbreak10; 38:3; 39:1 Na 2:13; 3:5}
\crossref{Jer}{50}{32}{Pr 16:18; 18:12 Isa 10:12-\allowbreak15; 14:13-\allowbreak15 Eze 28:2-\allowbreak9 Da 5:20,\allowbreak23-\allowbreak30}
\crossref{Jer}{50}{33}{50:7,\allowbreak17,\allowbreak18; 51:34-\allowbreak36 Isa 14:17; 47:6; 49:24-\allowbreak26; 51:23; 52:4-\allowbreak6}
\crossref{Jer}{50}{34}{Ex 6:6 Pr 23:11 Isa 41:14; 43:14; 44:6,\allowbreak23,\allowbreak24; 47:4; 54:5 Mic 4:10}
\crossref{Jer}{50}{35}{Jer 47:6 Le 26:25 Isa 66:16 Eze 14:2 Ho 11:6 Zec 11:17}
\crossref{Jer}{50}{36}{Jer 48:30 Isa 43:14}
\crossref{Jer}{50}{37}{Jer 51:21 Ps 20:7,\allowbreak8; 46:9; 76:6 Eze 39:20 Na 2:2-\allowbreak4,\allowbreak13 Hag 2:22}
\crossref{Jer}{50}{38}{50:12; 51:32-\allowbreak36 Isa 44:27 Re 16:12; 17:15,\allowbreak16}
\crossref{Jer}{50}{39}{50:12,\allowbreak13; 25:12; 51:26,\allowbreak37,\allowbreak38,\allowbreak43,\allowbreak62-\allowbreak64 Isa 13:20-\allowbreak22; 14:23; 34:11-\allowbreak17}
\crossref{Jer}{50}{40}{Jer 49:18; 51:26 Ge 19:24,\allowbreak25 De 29:23 Isa 1:9; 13:19,\allowbreak20 Ho 11:8,\allowbreak9}
\crossref{Jer}{50}{41}{50:2,\allowbreak3,\allowbreak9; 6:22,\allowbreak23; 25:14; 51:1,\allowbreak2,\allowbreak11,\allowbreak27,\allowbreak28 Isa 13:2-\allowbreak5,\allowbreak17,\allowbreak18 Re 17:16}
\crossref{Jer}{50}{42}{Jer 6:22,\allowbreak23}
\crossref{Jer}{50}{43}{Jer 51:31 Isa 13:6-\allowbreak8; 21:3,\allowbreak4 Da 5:5,\allowbreak6}
\crossref{Jer}{50}{44}{Jer 25:38; 49:19-\allowbreak21}
\crossref{Jer}{50}{45}{Jer 51:10,\allowbreak11 Ps 33:10,\allowbreak11 Isa 14:24-\allowbreak27; 46:10,\allowbreak11 Ac 4:28 Eph 1:11}
\crossref{Jer}{50}{46}{Jer 49:21 Isa 14:9,\allowbreak10 Eze 26:18; 31:16; 32:10 Re 18:9-\allowbreak19}
\crossref{Jer}{51}{1}{Jer 50:9,\allowbreak14-\allowbreak16,\allowbreak21 Isa 13:3-\allowbreak5 Am 3:6}
\crossref{Jer}{51}{2}{Jer 15:7 Isa 41:16 Eze 5:12 Mt 3:12}
\crossref{Jer}{51}{3}{Jer 50:14,\allowbreak41,\allowbreak42}
\crossref{Jer}{51}{4}{Jer 49:26; 50:30,\allowbreak37 Isa 13:15; 14:19}
\crossref{Jer}{51}{5}{Jer 33:24-\allowbreak26; 46:28; 50:4,\allowbreak5,\allowbreak20 1Sa 12:22 1Ki 6:13 Ezr 9:9 Ps 94:14}
\crossref{Jer}{51}{6}{51:9,\allowbreak45,\allowbreak50; 50:8,\allowbreak28 Isa 48:20 Zec 2:6,\allowbreak7 Re 18:4}
\crossref{Jer}{51}{7}{Isa 14:4 Da 2:32,\allowbreak38 Re 17:4}
\crossref{Jer}{51}{8}{51:41; 50:2 Isa 21:9; 47:9 Re 14:8; 18:2,\allowbreak8}
\crossref{Jer}{51}{9}{Jer 8:20; 46:16,\allowbreak21; 50:16 Isa 13:14; 47:15 Mt 25:10-\allowbreak13}
\crossref{Jer}{51}{10}{Ps 37:6 Mic 7:9,\allowbreak10}
\crossref{Jer}{51}{11}{Jer 46:4,\allowbreak9; 50:9,\allowbreak14,\allowbreak25,\allowbreak28,\allowbreak29 Isa 21:5}
\crossref{Jer}{51}{12}{Jer 46:3-\allowbreak5 Pr 21:30 Isa 8:9,\allowbreak10; 13:2 Joe 3:2,\allowbreak9-\allowbreak14 Na 2:1; 3:14,\allowbreak15}
\crossref{Jer}{51}{13}{51:36 Re 17:1,\allowbreak15}
\crossref{Jer}{51}{14}{Jer 49:13 Am 6:8 Heb 6:13}
\crossref{Jer}{51}{15}{Jer 10:12-\allowbreak16; 32:17 Ge 1:1-\allowbreak6 Ps 107:25; 146:5,\allowbreak6; 148:1-\allowbreak5 Isa 40:26}
\crossref{Jer}{51}{16}{Jer 10:12,\allowbreak13 Job 37:2-\allowbreak11; 40:9 Ps 18:13; 29:3-\allowbreak10; 46:6; 68:33; 104:7}
\crossref{Jer}{51}{17}{Jer 10:14 Ps 14:2; 53:1,\allowbreak2; 92:5,\allowbreak6; 115:5,\allowbreak8; 135:18 Isa 44:18-\allowbreak20}
\crossref{Jer}{51}{18}{Jer 10:8,\allowbreak15 Jon 2:8 Ac 14:15}
\crossref{Jer}{51}{19}{Jer 10:16 Ps 16:5; 73:26; 115:3 La 3:24}
\crossref{Jer}{51}{20}{Jer 50:23 Isa 10:5,\allowbreak15; 13:5; 14:5,\allowbreak6; 37:26; 41:15,\allowbreak16 Mic 4:13}
\crossref{Jer}{51}{21}{Jer 50:37 Ex 15:1,\allowbreak21 Ps 46:9; 76:6 Eze 39:20 Mic 5:10 Na 2:13}
\crossref{Jer}{51}{22}{Jer 6:11 De 32:25 1Sa 15:3 2Ch 36:17 Isa 20:4 La 2:11 Eze 9:6}
\crossref{Jer}{51}{23}{51:23}
\crossref{Jer}{51}{24}{51:11,\allowbreak35,\allowbreak49; 50:15,\allowbreak17,\allowbreak18,\allowbreak28,\allowbreak29,\allowbreak33,\allowbreak34 Ps 137:8,\allowbreak9 Isa 47:6-\allowbreak9}
\crossref{Jer}{51}{25}{Jer 50:31}
\crossref{Jer}{51}{26}{51:37,\allowbreak43; 50:12,\allowbreak13 Isa 13:19-\allowbreak22; 14:23}
\crossref{Jer}{51}{27}{51:12; 6:1; 50:2,\allowbreak41 Isa 13:2-\allowbreak5; 18:3 Am 3:6 Zec 14:2}
\crossref{Jer}{51}{28}{51:11; 25:25 Ge 10:2 1Ch 1:5}
\crossref{Jer}{51}{29}{Jer 8:16; 10:10; 50:36,\allowbreak43 Isa 13:13,\allowbreak14; 14:16 Joe 2:10 Am 8:8}
\crossref{Jer}{51}{30}{Jer 50:36}
\crossref{Jer}{51}{31}{Jer 4:20; 50:24 1Sa 4:12-\allowbreak18 2Sa 18:19-\allowbreak31 2Ch 30:6 Es 3:13-\allowbreak15}
\crossref{Jer}{51}{32}{Jer 50:38 Isa 44:27}
\crossref{Jer}{51}{33}{Isa 21:10; 41:15,\allowbreak16 Am 1:3 Mic 4:13 Hab 3:12}
\crossref{Jer}{51}{34}{51:49; 39:1-\allowbreak8; 50:7,\allowbreak17 La 1:1,\allowbreak14,\allowbreak15}
\crossref{Jer}{51}{35}{Jer 50:29 Jud 9:20,\allowbreak24,\allowbreak56,\allowbreak57 Ps 9:12; 12:5; 137:8,\allowbreak9 Isa 26:20,\allowbreak21}
\crossref{Jer}{51}{36}{Jer 50:33,\allowbreak34 Ps 140:12 Pr 22:23; 23:11 Isa 43:14; 47:6-\allowbreak9; 49:25,\allowbreak26}
\crossref{Jer}{51}{37}{51:25,\allowbreak26,\allowbreak29; 25:9,\allowbreak12,\allowbreak18; 50:12,\allowbreak13,\allowbreak23-\allowbreak26,\allowbreak38-\allowbreak40 Isa 13:19-\allowbreak22; 14:23}
\crossref{Jer}{51}{38}{Jer 2:15 Job 4:10,\allowbreak11 Ps 34:10; 58:6 Isa 35:9 Na 2:11-\allowbreak13 Zec 11:3}
\crossref{Jer}{51}{39}{Jer 25:27 Isa 21:4,\allowbreak5; 22:12-\allowbreak14 Da 5:1-\allowbreak4,\allowbreak30 Na 1:10; 3:11}
\crossref{Jer}{51}{40}{Jer 50:27 Ps 37:20; 44:22 Isa 34:6 Eze 39:18}
\crossref{Jer}{51}{41}{Jer 49:25; 50:23 Isa 13:19; 14:4 Da 2:38; 4:22,\allowbreak30; 5:4,\allowbreak5 Re 18:10-\allowbreak19}
\crossref{Jer}{51}{42}{Ps 18:4,\allowbreak16; 42:7; 65:7; 93:3 Isa 8:7,\allowbreak8 Eze 27:26-\allowbreak34 Da 9:26}
\crossref{Jer}{51}{43}{51:29,\allowbreak37; 50:39,\allowbreak40}
\crossref{Jer}{51}{44}{51:18,\allowbreak47; 50:2 Isa 46:1,\allowbreak2}
\crossref{Jer}{51}{45}{51:6,\allowbreak9,\allowbreak50; 50:8 Isa 48:20 Zec 2:7 Re 14:8-\allowbreak11; 18:4}
\crossref{Jer}{51}{46}{2Ki 19:7 Mt 24:6-\allowbreak8 Mr 13:7,\allowbreak8 Lu 21:9-\allowbreak19,\allowbreak28}
\crossref{Jer}{51}{47}{51:52}
\crossref{Jer}{51}{48}{51:10 Ps 58:10,\allowbreak11 Pr 11:10 Isa 44:23; 48:20; 49:13 Re 15:1-\allowbreak4}
\crossref{Jer}{51}{49}{}
\crossref{Jer}{51}{50}{51:6,\allowbreak45; 31:21; 44:28; 50:8 Isa 48:20; 51:11; 52:2,\allowbreak11,\allowbreak12 Zec 2:7-\allowbreak9}
\crossref{Jer}{51}{51}{Jer 3:22-\allowbreak25; 31:19 Ps 74:18-\allowbreak21; 79:4,\allowbreak12; 123:3,\allowbreak4; 137:1-\allowbreak3 La 2:15-\allowbreak17}
\crossref{Jer}{51}{52}{51:47; 50:38}
\crossref{Jer}{51}{53}{51:25,\allowbreak58; 49:16 Ge 11:4 Ps 139:8-\allowbreak10 Isa 14:12-\allowbreak15; 47:5,\allowbreak7}
\crossref{Jer}{51}{54}{Jer 48:3-\allowbreak5; 50:22,\allowbreak27,\allowbreak43,\allowbreak46 Isa 13:6-\allowbreak9; 15:5 Zep 1:10 Re 18:17-\allowbreak19}
\crossref{Jer}{51}{55}{51:38,\allowbreak39; 25:10; 50:10-\allowbreak15 Isa 15:1; 24:8-\allowbreak11; 47:5 Re 18:22,\allowbreak23}
\crossref{Jer}{51}{56}{51:48; 50:10 Isa 21:2 Hab 2:8 Re 17:16}
\crossref{Jer}{51}{57}{51:39; 25:27 Isa 21:4,\allowbreak5 Da 5:1-\allowbreak4,\allowbreak30,\allowbreak31 Na 1:10 Hab 2:15-\allowbreak17}
\crossref{Jer}{51}{58}{51:30 Isa 45:1,\allowbreak2}
\crossref{Jer}{51}{59}{Jer 32:12; 36:4; 45:1}
\crossref{Jer}{51}{60}{Jer 30:2,\allowbreak3; 36:2-\allowbreak4,\allowbreak32 Isa 8:1-\allowbreak4; 30:8 Da 12:4 Hab 2:2,\allowbreak3 Re 1:11,\allowbreak19}
\crossref{Jer}{51}{61}{Mt 24:1 Mr 13:1}
\crossref{Jer}{51}{62}{51:25,\allowbreak26,\allowbreak29,\allowbreak37; 50:3,\allowbreak13,\allowbreak39,\allowbreak40 Isa 13:19-\allowbreak22; 14:22,\allowbreak23 Re 18:20-\allowbreak23}
\crossref{Jer}{51}{63}{}
\crossref{Jer}{51}{64}{51:42; 25:27 Na 1:8,\allowbreak9 Re 14:8; 18:2,\allowbreak21}
\crossref{Jer}{52}{1}{2Ki 24:18 2Ch 36:11}
\crossref{Jer}{52}{2}{1Ki 14:22 2Ki 24:19,\allowbreak20 2Ch 36:12,\allowbreak13 Eze 17:16-\allowbreak20; 21:25}
\crossref{Jer}{52}{3}{2Sa 24:1 1Ki 10:9 Pr 28:2 Ec 10:16 Isa 3:4,\allowbreak5; 19:4}
\crossref{Jer}{52}{4}{Jer 39:1 2Ki 25:1-\allowbreak27 Eze 24:1,\allowbreak2}
\crossref{Jer}{52}{5}{}
\crossref{Jer}{52}{6}{Jer 39:2 2Ki 25:3 Zec 8:19}
\crossref{Jer}{52}{7}{Jer 34:2,\allowbreak3 2Ki 25:4}
\crossref{Jer}{52}{8}{Jer 21:7; 32:4; 34:21; 37:18; 38:23; 39:5 Isa 30:16,\allowbreak17 La 4:19,\allowbreak20}
\crossref{Jer}{52}{9}{Jer 32:4,\allowbreak5 2Ch 33:11 Eze 21:25-\allowbreak27}
\crossref{Jer}{52}{10}{Jer 22:30; 39:6,\allowbreak7 Ge 21:16; 44:34 De 28:34 2Ki 25:7}
\crossref{Jer}{52}{11}{Jer 34:3-\allowbreak5 Eze 12:13}
\crossref{Jer}{52}{12}{2Ki 25:8 Zec 7:3-\allowbreak5; 8:19}
\crossref{Jer}{52}{13}{Jer 7:14 2Ki 25:9 2Ch 36:19 Ps 74:6-\allowbreak8; 79:1 Isa 64:10,\allowbreak11 La 2:7}
\crossref{Jer}{52}{14}{2Ki 25:10 Ne 1:3}
\crossref{Jer}{52}{15}{Jer 15:1,\allowbreak2 Zec 14:2}
\crossref{Jer}{52}{16}{Jer 39:9,\allowbreak10; 40:5-\allowbreak7 2Ki 25:12 Eze 33:24}
\crossref{Jer}{52}{17}{52:21-\allowbreak23; 27:19-\allowbreak22 1Ki 7:15-\allowbreak22,\allowbreak27,\allowbreak50 2Ki 25:13-\allowbreak17 2Ch 4:12,\allowbreak13}
\crossref{Jer}{52}{18}{Ex 27:3; 38:3 2Ki 25:14-\allowbreak16 Eze 46:20-\allowbreak24}
\crossref{Jer}{52}{19}{Le 26:12 Nu 16:46 Re 8:3-\allowbreak5}
\crossref{Jer}{52}{20}{52:17}
\crossref{Jer}{52}{21}{1Ki 7:15-\allowbreak21 2Ki 25:17 2Ch 3:15-\allowbreak17}
\crossref{Jer}{52}{22}{Ex 28:14-\allowbreak22,\allowbreak25; 39:15-\allowbreak18 1Ki 7:17 2Ch 3:15; 4:12,\allowbreak13}
\crossref{Jer}{52}{23}{1Ki 7:20}
\crossref{Jer}{52}{24}{52:12,\allowbreak15 2Ki 25:18}
\crossref{Jer}{52}{25}{2Ki 25:19}
\crossref{Jer}{52}{26}{}
\crossref{Jer}{52}{27}{Jer 6:13-\allowbreak15 2Ki 25:20,\allowbreak21 Eze 8:11-\allowbreak18; 11:1-\allowbreak11}
\crossref{Jer}{52}{28}{2Ki 24:2,\allowbreak3,\allowbreak12-\allowbreak16 Da 1:1-\allowbreak3}
\crossref{Jer}{52}{29}{52:12; 39:9 2Ki 25:11 2Ch 36:20}
\crossref{Jer}{52}{30}{52:15; 6:9}
\crossref{Jer}{52}{31}{2Ki 25:27-\allowbreak30}
\crossref{Jer}{52}{32}{Pr 12:25}
\crossref{Jer}{52}{33}{2Sa 9:7,\allowbreak13 1Ki 2:7}
\crossref{Jer}{52}{34}{2Sa 9:10 Mt 6:11}

% Lam
\crossref{Lam}{1}{1}{La 2:10 Isa 3:26; 47:1-\allowbreak15; 50:5; 52:2,\allowbreak7 Jer 9:11 Eze 26:16}
\crossref{Lam}{1}{2}{1:16; 2:11,\allowbreak18,\allowbreak19 Job 7:3 Ps 6:6; 77:2-\allowbreak6 Jer 9:1,\allowbreak17-\allowbreak19; 13:17}
\crossref{Lam}{1}{3}{2Ki 24:14,\allowbreak15; 25:11,\allowbreak21 2Ch 36:20,\allowbreak21 Jer 39:9; 52:15,\allowbreak27-\allowbreak30}
\crossref{Lam}{1}{4}{La 2:6,\allowbreak7; 5:13 Isa 24:4-\allowbreak6 Jer 14:2 Mic 3:12}
\crossref{Lam}{1}{5}{La 2:17; 3:46 Le 26:17 De 28:43,\allowbreak44 Ps 80:6; 89:42 Isa 63:18}
\crossref{Lam}{1}{6}{2Ki 19:21 Ps 48:2,\allowbreak3 Isa 1:21; 4:5; 12:6 Zep 3:14-\allowbreak17}
\crossref{Lam}{1}{7}{Job 29:2-\allowbreak25; 30:1 Ps 42:4; 77:3,\allowbreak5-\allowbreak9 Ho 2:7 Lu 15:17; 16:25}
\crossref{Lam}{1}{8}{1:5,\allowbreak20 1Ki 8:46,\allowbreak47; 9:7,\allowbreak9 Isa 59:2-\allowbreak13 Jer 6:28 Eze 14:13-\allowbreak21}
\crossref{Lam}{1}{9}{1:17 Jer 2:34; 13:27 Eze 24:12,\allowbreak13}
\crossref{Lam}{1}{10}{1:7 Isa 5:13,\allowbreak14 Jer 15:13; 20:5; 52:17-\allowbreak20}
\crossref{Lam}{1}{11}{1:19; 2:12; 4:4-\allowbreak10 De 28:52-\allowbreak57 2Ki 6:25 Jer 19:9; 38:9; 52:6}
\crossref{Lam}{1}{12}{}
\crossref{Lam}{1}{13}{La 2:3,\allowbreak4 De 32:21-\allowbreak25 Job 30:30 Ps 22:14; 31:10; 102:3-\allowbreak5 Na 1:6}
\crossref{Lam}{1}{14}{De 28:48 Pr 5:22 Isa 14:25; 47:6 Jer 27:8,\allowbreak12; 28:14}
\crossref{Lam}{1}{15}{2Ki 9:33; 24:14-\allowbreak16; 25:4-\allowbreak7 Ps 119:118 Isa 5:5; 28:18}
\crossref{Lam}{1}{16}{1:2,\allowbreak9; 2:11,\allowbreak18; 3:48,\allowbreak49 Ps 119:136 Jer 9:1,\allowbreak10; 13:17; 14:17}
\crossref{Lam}{1}{17}{1Ki 8:22,\allowbreak38 Isa 1:15 Jer 4:31}
\crossref{Lam}{1}{18}{Ex 9:27 De 32:4 Jud 1:7 Ezr 9:13 Ne 9:33 Ps 119:75; 145:17}
\crossref{Lam}{1}{19}{1:2; 4:17 Job 19:13-\allowbreak19 Jer 2:28; 30:14; 37:7-\allowbreak9}
\crossref{Lam}{1}{20}{1:9,\allowbreak11 Isa 38:14}
\crossref{Lam}{1}{21}{1:2,\allowbreak8,\allowbreak11,\allowbreak12,\allowbreak16,\allowbreak22}
\crossref{Lam}{1}{22}{Ne 4:4,\allowbreak5 Ps 109:14,\allowbreak15; 137:7-\allowbreak9 Jer 10:25; 18:23; 51:35 Lu 23:31}
\crossref{Lam}{2}{1}{La 1:1; 4:1}
\crossref{Lam}{2}{2}{2:17,\allowbreak21; 3:43 Job 2:3}
\crossref{Lam}{2}{3}{Job 16:15 Ps 75:5,\allowbreak10; 89:24; 132:17 Jer 48:25 Lu 1:69}
\crossref{Lam}{2}{4}{2:5; 3:3,\allowbreak12,\allowbreak13 Job 6:4; 16:12-\allowbreak14 Isa 63:10 Jer 21:5; 30:14}
\crossref{Lam}{2}{5}{2:4 Jer 15:1; 30:14}
\crossref{Lam}{2}{6}{Isa 1:8}
\crossref{Lam}{2}{7}{2:1 Le 26:31,\allowbreak44 Ps 78:59-\allowbreak61 Isa 64:10,\allowbreak11 Jer 7:12-\allowbreak14; 26:6,\allowbreak18}
\crossref{Lam}{2}{8}{2:17 Isa 5:5 Jer 5:10}
\crossref{Lam}{2}{9}{Ne 1:3 Jer 39:2,\allowbreak8; 51:30; 52:14}
\crossref{Lam}{2}{10}{La 4:5,\allowbreak16; 5:12,\allowbreak14 Job 2:13 Isa 3:26; 47:1,\allowbreak5}
\crossref{Lam}{2}{11}{La 1:16; 3:48-\allowbreak51 1Sa 30:4 Ps 6:7; 31:9; 69:3 Isa 38:14}
\crossref{Lam}{2}{12}{Eze 30:24}
\crossref{Lam}{2}{13}{La 1:12 Da 9:12}
\crossref{Lam}{2}{14}{Isa 9:15,\allowbreak16 Jer 2:8; 5:31; 6:13,\allowbreak14; 8:10,\allowbreak11; 14:13-\allowbreak15; 23:11-\allowbreak17}
\crossref{Lam}{2}{15}{La 3:46 Job 16:9,\allowbreak10 Ps 22:13; 35:21; 109:2}
\crossref{Lam}{2}{16}{2:8 Le 26:14-\allowbreak46 De 28:15-\allowbreak68; 29:18-\allowbreak23; 31:16,\allowbreak17; 32:15-\allowbreak27}
\crossref{Lam}{2}{17}{Ps 119:145 Isa 26:16,\allowbreak17 Ho 7:14}
\crossref{Lam}{2}{18}{Ps 42:8; 62:8; 119:55,\allowbreak147,\allowbreak148 Isa 26:9 Mr 1:35 Lu 6:12}
\crossref{Lam}{2}{19}{Ex 32:11 De 9:26 Isa 63:16-\allowbreak19; 64:8-\allowbreak12 Jer 14:20-\allowbreak21}
\crossref{Lam}{2}{20}{De 28:50 Jos 6:21 1Sa 15:3 2Ch 36:17 Es 3:13 Jer 51:22}
\crossref{Lam}{2}{21}{Ps 31:13 Isa 24:17,\allowbreak18 Jer 6:25; 20:3; 46:5 Am 9:1-\allowbreak4}
\crossref{Lam}{2}{22}{La 1:12-\allowbreak14 Job 19:21 Ps 71:20; 88:7,\allowbreak15,\allowbreak16 Isa 53:3 Jer 15:17,\allowbreak18}
\crossref{Lam}{3}{1}{3:53-\allowbreak55; 2:1 De 28:29 Job 18:18; 30:26 Isa 59:9 Jer 13:16}
\crossref{Lam}{3}{2}{La 2:4-\allowbreak7 De 29:20 Job 31:21 Isa 1:25; 63:10}
\crossref{Lam}{3}{3}{Job 16:8,\allowbreak9 Ps 31:9,\allowbreak10; 32:3; 38:2-\allowbreak8; 102:3-\allowbreak5}
\crossref{Lam}{3}{4}{3:7-\allowbreak9 Job 19:8}
\crossref{Lam}{3}{5}{Ps 88:5,\allowbreak6; 143:3,\allowbreak7}
\crossref{Lam}{3}{6}{3:9 Job 3:23; 19:8 Ps 88:8 Jer 38:6 Ho 2:6}
\crossref{Lam}{3}{7}{3:44 Job 19:7; 30:20 Ps 22:2; 80:4 Hab 1:2 Mt 27:46}
\crossref{Lam}{3}{8}{3:11 Isa 30:28; 63:17}
\crossref{Lam}{3}{9}{Job 10:16 Isa 38:13 Ho 5:14; 6:1; 13:7,\allowbreak8 Am 5:18-\allowbreak20}
\crossref{Lam}{3}{10}{Job 16:12,\allowbreak13 Ps 50:22 Jer 5:6; 51:20-\allowbreak22 Da 2:40-\allowbreak44; 7:23}
\crossref{Lam}{3}{11}{Job 6:4; 7:20; 16:12,\allowbreak13 Ps 7:12,\allowbreak13; 38:2}
\crossref{Lam}{3}{12}{De 32:23 Job 6:4; 41:28}
\crossref{Lam}{3}{13}{3:63 Ne 4:2-\allowbreak4 Job 30:1-\allowbreak9 Ps 22:6,\allowbreak7; 35:15,\allowbreak16; 44:13; 69:11,\allowbreak12}
\crossref{Lam}{3}{14}{3:19 Ru 1:20 Job 9:18 Ps 60:3 Isa 51:17-\allowbreak22 Jer 9:15; 23:15}
\crossref{Lam}{3}{15}{Job 4:10 Ps 3:7; 58:6}
\crossref{Lam}{3}{16}{La 1:16 Ps 119:155 Isa 38:17; 54:10; 59:11 Jer 8:15; 14:19; 16:5}
\crossref{Lam}{3}{17}{1Sa 27:1 Job 6:11; 17:15 Ps 31:22; 116:11 Eze 37:11}
\crossref{Lam}{3}{18}{Ne 9:32 Job 7:7 Ps 89:47,\allowbreak50; 132:1}
\crossref{Lam}{3}{19}{Job 21:6}
\crossref{Lam}{3}{20}{Ps 77:7-\allowbreak11}
\crossref{Lam}{3}{21}{Ezr 9:8,\allowbreak9,\allowbreak13-\allowbreak15 Ne 9:31 Ps 78:38; 106:45 Eze 20:8,\allowbreak9,\allowbreak13,\allowbreak14,\allowbreak21}
\crossref{Lam}{3}{22}{Ps 30:5 Isa 33:2 Zep 3:5}
\crossref{Lam}{3}{23}{Ps 16:5; 73:26; 119:57; 142:5 Jer 10:16; 51:19}
\crossref{Lam}{3}{24}{3:26 Ge 49:18 Ps 25:8; 27:14; 37:7,\allowbreak34; 39:7; 40:1-\allowbreak5; 61:1,\allowbreak5}
\crossref{Lam}{3}{25}{Ps 52:9; 54:6; 73:28; 92:1 Ga 4:18}
\crossref{Lam}{3}{26}{Ps 90:12; 94:12; 119:71 Ec 12:1 Mt 11:29,\allowbreak30 Heb 12:5-\allowbreak12}
\crossref{Lam}{3}{27}{La 2:10 Ps 39:9; 102:7 Jer 15:17}
\crossref{Lam}{3}{28}{2Ch 33:12 Job 40:4; 42:5,\allowbreak6 Eze 16:63 Ro 3:19}
\crossref{Lam}{3}{29}{Job 16:10 Isa 50:6 Mic 5:1 Mt 5:39; 26:67 Lu 6:29 2Co 11:20}
\crossref{Lam}{3}{30}{1Sa 12:22 Ps 77:7; 94:14; 103:8-\allowbreak10 Isa 54:7-\allowbreak10; 57:16 Jer 31:37}
\crossref{Lam}{3}{31}{3:22 Ex 2:23; 3:7 Jud 10:16 2Ki 13:23 Ps 30:5; 78:38; 103:11}
\crossref{Lam}{3}{32}{Isa 28:21 Eze 18:32; 33:11 Heb 12:9,\allowbreak10}
\crossref{Lam}{3}{33}{Isa 51:22,\allowbreak23 Jer 50:17,\allowbreak33,\allowbreak34; 51:33-\allowbreak36}
\crossref{Lam}{3}{34}{Ps 12:5; 140:12 Pr 17:15; 22:22; 23:10 Zec 1:15,\allowbreak16}
\crossref{Lam}{3}{35}{}
\crossref{Lam}{3}{36}{2Sa 11:27 Isa 59:15 Hab 1:13}
\crossref{Lam}{3}{37}{Ps 33:9-\allowbreak11 Pr 16:9; 19:21; 21:30 Isa 46:10 Da 4:35 Ro 9:15}
\crossref{Lam}{3}{38}{Job 2:10 Ps 75:7 Pr 29:26 Isa 45:7 Am 3:6}
\crossref{Lam}{3}{39}{3:22 Nu 11:11 Pr 19:3 Isa 38:17-\allowbreak19}
\crossref{Lam}{3}{40}{1Ch 15:12,\allowbreak13 Job 11:13-\allowbreak15; 34:31,\allowbreak32 Ps 4:4; 119:59; 139:23,\allowbreak24}
\crossref{Lam}{3}{41}{Ps 25:1; 86:4; 143:6-\allowbreak8}
\crossref{Lam}{3}{42}{La 1:18; 5:16 Ne 9:26 Job 33:27,\allowbreak28 Jer 3:13 Da 9:5-\allowbreak14 Lu 15:18,\allowbreak19}
\crossref{Lam}{3}{43}{La 2:1 Ps 44:19}
\crossref{Lam}{3}{44}{Ps 97:2}
\crossref{Lam}{3}{45}{3:14; 2:15; 4:14,\allowbreak15 De 28:13,\allowbreak37,\allowbreak44 1Co 4:13}
\crossref{Lam}{3}{46}{La 2:16 Ex 11:7 Job 30:9-\allowbreak11 Ps 22:6-\allowbreak8; 44:13,\allowbreak14; 79:4,\allowbreak10}
\crossref{Lam}{3}{47}{Isa 24:17,\allowbreak18; 51:19 Jer 48:43,\allowbreak44 Lu 21:35}
\crossref{Lam}{3}{48}{La 2:11,\allowbreak18 Ps 119:136 Jer 4:19; 9:1,\allowbreak18; 13:17 Ro 9:1-\allowbreak3}
\crossref{Lam}{3}{49}{La 1:16 Ps 77:2 Jer 14:17}
\crossref{Lam}{3}{50}{La 2:20; 5:1 Ps 80:14-\allowbreak16; 102:19,\allowbreak20 Isa 62:6,\allowbreak7; 63:15; 64:1}
\crossref{Lam}{3}{51}{Ge 44:34 1Sa 30:3,\allowbreak4 Jer 4:19-\allowbreak21; 14:18 Lu 19:41-\allowbreak44}
\crossref{Lam}{3}{52}{Jer 37:15,\allowbreak16; 38:4-\allowbreak6}
\crossref{Lam}{3}{53}{Jer 37:20; 38:6,\allowbreak9}
\crossref{Lam}{3}{54}{Ps 18:4; 69:1,\allowbreak2,\allowbreak15; 124:4,\allowbreak5 Jon 2:3-\allowbreak5}
\crossref{Lam}{3}{55}{2Ch 33:11,\allowbreak12 Ps 18:5,\allowbreak6; 40:1,\allowbreak2; 69:13-\allowbreak18; 116:3,\allowbreak4; 130:1,\allowbreak2}
\crossref{Lam}{3}{56}{2Ch 33:13,\allowbreak19 Job 34:28 Ps 3:4; 6:8,\allowbreak9; 34:6; 66:19; 116:1,\allowbreak2}
\crossref{Lam}{3}{57}{Ps 69:18; 145:18 Isa 58:9 Jas 4:8}
\crossref{Lam}{3}{58}{1Sa 25:39 Ps 35:1 Jer 51:36}
\crossref{Lam}{3}{59}{Jer 11:19-\allowbreak21; 15:10; 18:18-\allowbreak23; 20:7-\allowbreak10; 37:1-\allowbreak38:28}
\crossref{Lam}{3}{60}{3:59 Ps 10:14 Jer 11:19,\allowbreak20}
\crossref{Lam}{3}{61}{3:30; 5:1 Ps 74:18; 89:50 Zep 2:8}
\crossref{Lam}{3}{62}{Ps 59:7,\allowbreak12; 140:3 Eze 36:3}
\crossref{Lam}{3}{63}{Ps 139:2}
\crossref{Lam}{3}{64}{Ps 28:4 Jer 11:20; 50:29 2Ti 4:14 Re 6:10; 18:6}
\crossref{Lam}{3}{65}{De 2:30 Isa 6:10}
\crossref{Lam}{3}{66}{3:43 Ps 35:6; 73:15}
\crossref{Lam}{4}{1}{2Ki 25:9,\allowbreak10 Isa 1:21; 14:12 Eze 7:19-\allowbreak22}
\crossref{Lam}{4}{2}{Isa 51:18 Zec 9:13}
\crossref{Lam}{4}{3}{La 2:20; 4:10 Le 26:29 De 28:52-\allowbreak57 2Ki 6:26-\allowbreak29 Isa 49:15 Jer 19:9}
\crossref{Lam}{4}{4}{Ps 22:15; 137:6}
\crossref{Lam}{4}{5}{De 28:54-\allowbreak56 Isa 3:16-\allowbreak26; 24:6-\allowbreak12; 32:9-\allowbreak14 Jer 6:2,\allowbreak3 Am 6:3-\allowbreak7}
\crossref{Lam}{4}{6}{4:9 Ge 19:25 Da 9:12 Mt 24:21}
\crossref{Lam}{4}{7}{Nu 6:2-\allowbreak21 Jud 13:5,\allowbreak7; 16:17 Am 2:11,\allowbreak12 Lu 1:15}
\crossref{Lam}{4}{8}{La 5:10 Job 30:17-\allowbreak19,\allowbreak30 Joe 2:6 Na 2:10}
\crossref{Lam}{4}{9}{Le 26:39 Eze 24:23; 33:10}
\crossref{Lam}{4}{10}{4:3; 2:20 2Ki 6:26-\allowbreak29}
\crossref{Lam}{4}{11}{4:22; 2:8,\allowbreak17 De 32:21-\allowbreak25 Jer 6:11,\allowbreak12; 7:20; 9:9-\allowbreak11; 13:14}
\crossref{Lam}{4}{12}{De 29:24-\allowbreak28 1Ki 9:8,\allowbreak9 Ps 48:4-\allowbreak6}
\crossref{Lam}{4}{13}{La 2:14 Jer 5:31; 6:13; 14:14; 23:11-\allowbreak21 Eze 22:26-\allowbreak28 Mic 3:11,\allowbreak12}
\crossref{Lam}{4}{14}{De 28:28,\allowbreak29 Isa 29:10-\allowbreak12; 56:10; 59:9-\allowbreak11 Mic 3:6,\allowbreak7 Mt 15:14}
\crossref{Lam}{4}{15}{Nu 16:26; 19:16 Ps 6:8; 139:19 Mic 2:10 2Co 6:17}
\crossref{Lam}{4}{16}{Ge 49:7 Le 26:33-\allowbreak39 De 28:25,\allowbreak64,\allowbreak65; 32:26 Jer 15:4; 24:9}
\crossref{Lam}{4}{17}{La 1:19 2Ki 24:7 Isa 20:5; 30:1-\allowbreak7; 31:1-\allowbreak3 Jer 2:18,\allowbreak36; 8:20}
\crossref{Lam}{4}{18}{La 3:52 1Sa 24:14 2Ki 25:4,\allowbreak5 Job 10:16 Ps 140:11 Jer 16:16}
\crossref{Lam}{4}{19}{De 28:49 Isa 5:26-\allowbreak28; 30:16,\allowbreak17 Jer 4:13 Ho 8:1 Hab 1:8}
\crossref{Lam}{4}{20}{La 2:9 Ge 2:7; 44:30 2Sa 18:3}
\crossref{Lam}{4}{21}{Ps 83:3-\allowbreak12; 137:7 Ec 11:9 Eze 25:6,\allowbreak8; 26:2; 35:11-\allowbreak15 Ob 1:10-\allowbreak16}
\crossref{Lam}{4}{22}{4:6}
\crossref{Lam}{5}{1}{La 1:20; 2:20; 3:19 Ne 1:8 Job 7:7; 10:9 Jer 15:15 Hab 3:2}
\crossref{Lam}{5}{2}{De 28:30-\allowbreak68 Ps 79:1,\allowbreak2 Isa 1:7; 5:17; 63:18 Jer 6:12 Eze 7:21,\allowbreak24}
\crossref{Lam}{5}{3}{Ex 22:24 Jer 18:21 Ho 14:3}
\crossref{Lam}{5}{4}{De 28:48 Isa 3:1 Eze 4:9-\allowbreak17}
\crossref{Lam}{5}{5}{Ne 9:36,\allowbreak37}
\crossref{Lam}{5}{6}{Ge 24:2 2Ki 10:15 Jer 50:15 Eze 17:18}
\crossref{Lam}{5}{7}{Ex 20:5 Jer 16:12; 31:29 Eze 18:2 Mt 23:32-\allowbreak36}
\crossref{Lam}{5}{8}{Ge 9:25 De 28:43 Ne 2:19; 5:15 Pr 30:22}
\crossref{Lam}{5}{9}{Jud 6:11 2Sa 23:17 Jer 40:9-\allowbreak12; 41:1-\allowbreak10,\allowbreak18; 42:14,\allowbreak16}
\crossref{Lam}{5}{10}{La 3:4; 4:8 Job 30:30 Ps 119:83}
\crossref{Lam}{5}{11}{De 28:30 Isa 13:16 Zec 14:2}
\crossref{Lam}{5}{12}{La 2:10,\allowbreak20; 4:16 Isa 47:6 Jer 39:6,\allowbreak7; 52:10,\allowbreak11,\allowbreak25-\allowbreak27}
\crossref{Lam}{5}{13}{Ex 11:5 Jud 16:21 Job 31:10 Isa 47:2}
\crossref{Lam}{5}{14}{La 1:4,\allowbreak19; 2:10 De 16:18 Job 29:7-\allowbreak17; 30:1 Isa 3:2,\allowbreak3}
\crossref{Lam}{5}{15}{Ps 30:11 Am 6:4-\allowbreak7; 8:10 Jas 4:9,\allowbreak10}
\crossref{Lam}{5}{16}{La 1:1 Job 19:9 Ps 89:39 Jer 13:18 Eze 21:26 Re 2:10; 3:11}
\crossref{Lam}{5}{17}{La 1:13,\allowbreak22 Le 26:36 Isa 1:5 Jer 8:18; 46:5 Eze 21:7,\allowbreak15 Mic 6:13}
\crossref{Lam}{5}{18}{La 2:8,\allowbreak9 1Ki 9:7,\allowbreak8 Ps 74:2,\allowbreak3 Jer 17:3; 26:9; 52:13 Mic 3:12}
\crossref{Lam}{5}{19}{De 33:27 Ps 9:7; 10:16; 29:10; 90:2; 102:12,\allowbreak25-\allowbreak27 Hab 1:12}
\crossref{Lam}{5}{20}{Ps 13:1; 44:24; 74:1; 77:7-\allowbreak10; 79:5; 85:5; 89:46; 94:3,\allowbreak4 Isa 64:9-\allowbreak12}
\crossref{Lam}{5}{21}{1Ki 18:37 Ps 80:3,\allowbreak7,\allowbreak19; 85:4 Jer 31:18; 32:39,\allowbreak40 Eze 11:19,\allowbreak20}
\crossref{Lam}{5}{22}{}

% Ezek
\crossref{Ezek}{1}{1}{Nu 4:3 Lu 3:23}
\crossref{Ezek}{1}{2}{Eze 8:1; 20:1; 29:1,\allowbreak17; 31:1; 40:1 2Ki 24:12-\allowbreak15}
\crossref{Ezek}{1}{3}{Jer 1:2,\allowbreak4 Ho 1:1 Joe 1:1 1Ti 4:1}
\crossref{Ezek}{1}{4}{Isa 21:1 Jer 1:13,\allowbreak14; 4:6; 6:1; 23:19; 25:9,\allowbreak32 Hab 1:8,\allowbreak9}
\crossref{Ezek}{1}{5}{Re 4:6; 6:6}
\crossref{Ezek}{1}{6}{1:8-\allowbreak11 Ex 25:20 1Ki 6:24-\allowbreak27 Isa 6:2}
\crossref{Ezek}{1}{7}{Le 11:3,\allowbreak47}
\crossref{Ezek}{1}{8}{Eze 8:3; 10:2,\allowbreak7,\allowbreak8,\allowbreak18,\allowbreak21 Isa 6:6}
\crossref{Ezek}{1}{9}{1:11 2Ch 3:11,\allowbreak12 1Co 1:10}
\crossref{Ezek}{1}{10}{Eze 10:14 Re 4:7}
\crossref{Ezek}{1}{11}{Eze 10:16,\allowbreak19}
\crossref{Ezek}{1}{12}{1:9,\allowbreak17; 10:22}
\crossref{Ezek}{1}{13}{1:7 Ge 15:17 Ps 104:4 Da 10:5,\allowbreak6 Mt 28:3 Re 4:5; 10:1; 18:1}
\crossref{Ezek}{1}{14}{Ps 147:15 Da 9:21 Zec 2:3,\allowbreak4; 4:10 Mt 24:27,\allowbreak31 Mr 13:27}
\crossref{Ezek}{1}{15}{1:19-\allowbreak21; 10:9,\allowbreak13-\allowbreak17 Da 7:9}
\crossref{Ezek}{1}{16}{Eze 10:9 Ex 39:13 Da 10:6}
\crossref{Ezek}{1}{17}{1:9,\allowbreak12; 10:1-\allowbreak11:25 Isa 55:11}
\crossref{Ezek}{1}{18}{Job 37:22-\allowbreak24 Ps 77:16-\allowbreak19; 97:2-\allowbreak5 Isa 55:9}
\crossref{Ezek}{1}{19}{Eze 10:16 Ps 103:20}
\crossref{Ezek}{1}{20}{1:12 1Co 14:32}
\crossref{Ezek}{1}{21}{1:19,\allowbreak20; 10:17}
\crossref{Ezek}{1}{22}{1:26; 10:1 Ex 24:10 Job 37:22 Re 4:3,\allowbreak6; 21:11}
\crossref{Ezek}{1}{23}{1:12,\allowbreak24}
\crossref{Ezek}{1}{24}{Eze 43:2 Re 1:15; 19:6}
\crossref{Ezek}{1}{25}{1:24}
\crossref{Ezek}{1}{26}{Mt 28:18 Eph 1:21,\allowbreak22 Php 2:9,\allowbreak10 1Pe 3:22}
\crossref{Ezek}{1}{27}{1:4; 8:2}
\crossref{Ezek}{1}{28}{Ge 9:13-\allowbreak16 Isa 54:8-\allowbreak10 Re 4:3; 10:1}
\crossref{Ezek}{2}{1}{2:3,\allowbreak6,\allowbreak8; 3:1,\allowbreak4,\allowbreak10,\allowbreak17; 4:1; 5:1; 7:2; 12:3; 13:2; 14:3,\allowbreak13; 15:2; 16:2}
\crossref{Ezek}{2}{2}{Eze 3:12,\allowbreak14,\allowbreak24; 36:27 Nu 11:25,\allowbreak26 Jud 13:25 1Sa 16:13 Ne 9:30}
\crossref{Ezek}{2}{3}{Eze 3:4-\allowbreak8 2Ch 36:15,\allowbreak16 Isa 6:8-\allowbreak10 Jer 1:7; 7:2; 25:3-\allowbreak7; 26:2-\allowbreak6; 36:2}
\crossref{Ezek}{2}{4}{Eze 3:7 De 10:16; 31:27 2Ch 30:8; 36:13 Ps 95:8 Isa 48:4 Jer 3:3}
\crossref{Ezek}{2}{5}{2:7; 3:10,\allowbreak11,\allowbreak27 Mt 10:12-\allowbreak15 Ac 13:46 Ro 3:3 2Co 2:15-\allowbreak17}
\crossref{Ezek}{2}{6}{Eze 3:8,\allowbreak9 2Ki 1:15 Isa 51:12 Jer 1:8,\allowbreak17 Mic 3:8 Mt 10:28 Lu 12:4}
\crossref{Ezek}{2}{7}{Eze 3:10,\allowbreak17 Jer 1:7,\allowbreak17; 23:28; 26:2 Jon 3:2 Mt 28:20}
\crossref{Ezek}{2}{8}{Le 10:3 Nu 20:10-\allowbreak13,\allowbreak24 1Ki 13:21,\allowbreak22 Isa 50:5 1Pe 5:3}
\crossref{Ezek}{2}{9}{Eze 8:3 Jer 1:9 Da 5:5; 10:10,\allowbreak16-\allowbreak18}
\crossref{Ezek}{2}{10}{Isa 30:8-\allowbreak11 Hab 2:2}
\crossref{Ezek}{3}{1}{3:11,\allowbreak15,\allowbreak17-\allowbreak21; 2:3 Jer 24:1-\allowbreak7}
\crossref{Ezek}{3}{2}{Jer 25:17 Ac 26:19}
\crossref{Ezek}{3}{3}{Eze 2:10 Job 32:18,\allowbreak19 Jer 6:11; 20:9 Joh 7:38 Col 3:16}
\crossref{Ezek}{3}{4}{3:11; 2:3,\allowbreak7 Mt 10:5,\allowbreak6; 15:24 Ac 1:8}
\crossref{Ezek}{3}{5}{Jon 1:2; 3:2-\allowbreak4 Ac 26:17,\allowbreak18}
\crossref{Ezek}{3}{6}{}
\crossref{Ezek}{3}{7}{1Sa 8:7 Jer 25:3,\allowbreak4; 44:4,\allowbreak5,\allowbreak16 Lu 10:16; 13:34; 19:14 Joh 5:40-\allowbreak47}
\crossref{Ezek}{3}{8}{Ex 4:15,\allowbreak16; 11:4-\allowbreak8 1Ki 21:20 Isa 50:7 Jer 1:18; 15:20 Mic 3:8}
\crossref{Ezek}{3}{9}{Zec 7:12}
\crossref{Ezek}{3}{10}{3:1-\allowbreak3; 2:8 Job 22:22 Ps 119:11 Pr 8:10; 19:20 Lu 8:15 1Th 2:13}
\crossref{Ezek}{3}{11}{3:15; 11:24,\allowbreak25 Da 6:13}
\crossref{Ezek}{3}{12}{3:14; 2:2; 8:3; 11:1,\allowbreak24; 40:1,\allowbreak2 1Ki 18:12 2Ki 2:16 Ac 8:39}
\crossref{Ezek}{3}{13}{Eze 1:24; 10:5 2Sa 5:24}
\crossref{Ezek}{3}{14}{3:12; 8:3; 37:1}
\crossref{Ezek}{3}{15}{3:23; 1:1; 10:15; 43:3}
\crossref{Ezek}{3}{16}{Jer 42:7}
\crossref{Ezek}{3}{17}{Eze 33:2-\allowbreak9 1Co 12:28}
\crossref{Ezek}{3}{18}{Eze 18:4,\allowbreak13,\allowbreak20; 33:6,\allowbreak8 Ge 2:17; 3:3,\allowbreak4 Nu 26:65 2Ki 1:4 Isa 3:11}
\crossref{Ezek}{3}{19}{2Ki 17:13-\allowbreak23 2Ch 36:15,\allowbreak16 Pr 29:1 Jer 42:19-\allowbreak22; 44:4,\allowbreak5}
\crossref{Ezek}{3}{20}{Eze 18:24,\allowbreak26; 33:12,\allowbreak13 2Ch 24:2,\allowbreak17-\allowbreak22 Ps 36:3; 125:5 Zep 1:6}
\crossref{Ezek}{3}{21}{Mt 24:24,\allowbreak25 Ac 20:31 1Co 4:14; 10:12 Ga 1:6-\allowbreak10; 5:2-\allowbreak7}
\crossref{Ezek}{3}{22}{3:14; 1:3; 37:1}
\crossref{Ezek}{3}{23}{Eze 1:4,\allowbreak28; 9:3; 10:18 Nu 16:19,\allowbreak42 Ac 7:55}
\crossref{Ezek}{3}{24}{Eze 2:2; 37:10 Da 10:8-\allowbreak10,\allowbreak19}
\crossref{Ezek}{3}{25}{Eze 4:8 Mr 3:21 Joh 21:18 Ac 9:16; 20:23; 21:11-\allowbreak13}
\crossref{Ezek}{3}{26}{Eze 24:27 Ps 51:15; 137:6 Jer 1:17 Lu 1:20-\allowbreak22}
\crossref{Ezek}{3}{27}{Eze 11:25; 24:27; 29:21; 33:32 Ex 4:11,\allowbreak12 Lu 21:15 Eph 6:19}
\crossref{Ezek}{4}{1}{Eze 5:1-\allowbreak17; 12:3-\allowbreak16 1Sa 15:27,\allowbreak28 1Ki 11:30,\allowbreak31 Isa 20:2-\allowbreak4}
\crossref{Ezek}{4}{2}{Jer 39:1,\allowbreak2; 52:4 Lu 19:42-\allowbreak44}
\crossref{Ezek}{4}{3}{Le 2:5}
\crossref{Ezek}{4}{4}{4:5,\allowbreak8}
\crossref{Ezek}{4}{5}{Isa 53:6}
\crossref{Ezek}{4}{6}{}
\crossref{Ezek}{4}{7}{4:3; 6:2}
\crossref{Ezek}{4}{8}{Eze 3:25}
\crossref{Ezek}{4}{9}{4:13,\allowbreak16}
\crossref{Ezek}{4}{10}{4:16; 14:13 Le 26:26 De 28:51-\allowbreak68 Isa 3:1}
\crossref{Ezek}{4}{11}{4:16 Isa 5:13 Joh 3:34}
\crossref{Ezek}{4}{12}{Ge 18:6}
\crossref{Ezek}{4}{13}{Da 1:8 Ho 9:3,\allowbreak4}
\crossref{Ezek}{4}{14}{Eze 9:8; 20:49 Jer 1:6}
\crossref{Ezek}{4}{15}{}
\crossref{Ezek}{4}{16}{Eze 5:16; 14:13 Le 26:26 Ps 105:16 Isa 3:1}
\crossref{Ezek}{4}{17}{Eze 24:23 Le 26:39}
\crossref{Ezek}{5}{1}{Eze 44:20 Le 21:5 Isa 7:20}
\crossref{Ezek}{5}{2}{5:12 Jer 9:21,\allowbreak22; 15:2; 24:10; 38:2}
\crossref{Ezek}{5}{3}{2Ki 25:12 Jer 39:10; 40:6; 52:16 Mt 7:14 Lu 13:23,\allowbreak24 1Pe 4:18}
\crossref{Ezek}{5}{4}{2Ki 25:25 Jer 41:1-\allowbreak44:30; 52:30}
\crossref{Ezek}{5}{5}{Eze 4:1 Jer 6:6 Lu 22:19,\allowbreak20 1Co 10:4}
\crossref{Ezek}{5}{6}{Eze 16:47 De 32:15-\allowbreak21 2Ki 17:8-\allowbreak20 Ps 106:20 Ro 1:23-\allowbreak25 1Co 5:1}
\crossref{Ezek}{5}{7}{5:11; 16:47,\allowbreak48,\allowbreak54 2Ki 21:9-\allowbreak11 2Ch 33:9 Jer 2:10,\allowbreak11}
\crossref{Ezek}{5}{8}{Eze 15:7; 21:3; 26:3; 28:22; 35:3; 39:1 Le 26:17-\allowbreak46 De 29:20}
\crossref{Ezek}{5}{9}{La 4:6,\allowbreak9 Da 9:12 Am 3:2 Mt 24:21}
\crossref{Ezek}{5}{10}{Le 26:29 De 28:53-\allowbreak57,\allowbreak64 2Ki 6:29 Isa 9:20; 49:26 Jer 19:9}
\crossref{Ezek}{5}{11}{Nu 14:28-\allowbreak35 Ps 95:11 Am 8:7 Heb 6:13}
\crossref{Ezek}{5}{12}{5:2; 6:12 Jer 15:2; 21:9 Zec 13:7-\allowbreak9}
\crossref{Ezek}{5}{13}{Eze 6:12; 7:8; 13:15; 20:8,\allowbreak21 Jer 25:12 La 4:11,\allowbreak22 Da 9:2; 11:36}
\crossref{Ezek}{5}{14}{Eze 22:4 Le 26:31,\allowbreak32 De 28:37 2Ch 7:20,\allowbreak21 Ne 2:17 Ps 74:3-\allowbreak10}
\crossref{Ezek}{5}{15}{De 29:24-\allowbreak28 1Ki 9:7 Ps 79:4 Isa 26:9 Jer 22:8,\allowbreak9 1Co 10:11}
\crossref{Ezek}{5}{16}{De 32:23,\allowbreak24 Ps 7:13; 91:5-\allowbreak7 La 3:12}
\crossref{Ezek}{5}{17}{Eze 14:15,\allowbreak21; 33:27; 34:25-\allowbreak28 Ex 23:29 Le 26:22 De 32:24 2Ki 17:25}
\crossref{Ezek}{6}{1}{6:1}
\crossref{Ezek}{6}{2}{Eze 4:7; 13:17; 20:46; 21:2; 25:2; 38:2,\allowbreak3}
\crossref{Ezek}{6}{3}{Eze 36:1-\allowbreak4,\allowbreak8 Jer 22:29 Mic 6:2}
\crossref{Ezek}{6}{4}{6:6 2Ch 14:5; 34:4 Jer 43:13}
\crossref{Ezek}{6}{5}{6:5}
\crossref{Ezek}{6}{6}{Isa 6:11 Jer 9:19 Zep 3:7}
\crossref{Ezek}{6}{7}{Eze 9:7 Jer 14:18; 18:21; 25:33 La 2:20,\allowbreak21; 4:9}
\crossref{Ezek}{6}{8}{Eze 5:2,\allowbreak12; 12:16; 14:22 Isa 6:13; 27:7,\allowbreak8 Jer 30:11; 44:14,\allowbreak28; 46:28}
\crossref{Ezek}{6}{9}{Le 26:40,\allowbreak41 De 4:29-\allowbreak31; 30:1-\allowbreak3 Ps 137:1 Jer 51:50 Da 9:2,\allowbreak3}
\crossref{Ezek}{6}{10}{6:7; 14:22,\allowbreak23 Jer 5:12-\allowbreak14; 44:28 Da 9:12 Zec 1:6}
\crossref{Ezek}{6}{11}{Eze 21:14-\allowbreak17 Nu 24:10 Isa 58:1 Jer 9:1,\allowbreak10}
\crossref{Ezek}{6}{12}{Da 9:7}
\crossref{Ezek}{6}{13}{6:4-\allowbreak7 Isa 37:20,\allowbreak36-\allowbreak38}
\crossref{Ezek}{6}{14}{Eze 16:27; 20:33,\allowbreak34 Isa 5:25; 9:12,\allowbreak17,\allowbreak21; 10:4; 26:11}
\crossref{Ezek}{7}{1}{7:1}
\crossref{Ezek}{7}{2}{Eze 12:22; 21:2; 40:2 2Ch 34:7}
\crossref{Ezek}{7}{3}{7:8,\allowbreak9; 5:13; 6:3-\allowbreak7,\allowbreak12,\allowbreak13}
\crossref{Ezek}{7}{4}{7:9; 5:11; 8:18; 9:10; 24:14 Jer 13:14 Zec 11:6}
\crossref{Ezek}{7}{5}{Eze 5:9 2Ki 21:12,\allowbreak13 Da 9:12 Am 3:2 Na 1:9 Mt 24:21}
\crossref{Ezek}{7}{6}{Zec 13:7}
\crossref{Ezek}{7}{7}{Ge 19:15,\allowbreak24 Isa 17:14 Am 4:13}
\crossref{Ezek}{7}{8}{Eze 9:8; 14:19; 20:8,\allowbreak13,\allowbreak21,\allowbreak33; 22:31; 30:15; 36:18 2Ch 34:21 Ps 79:6}
\crossref{Ezek}{7}{9}{7:4}
\crossref{Ezek}{7}{10}{7:6 1Th 5:3}
\crossref{Ezek}{7}{11}{7:23 Isa 5:7; 9:4; 14:29; 59:6-\allowbreak8 Jer 6:7 Am 3:10; 6:3 Mic 2:2; 3:3}
\crossref{Ezek}{7}{12}{7:5-\allowbreak7,\allowbreak10 1Co 7:29-\allowbreak31 Jas 5:8,\allowbreak9}
\crossref{Ezek}{7}{13}{Ec 8:8 Le 25:24-\allowbreak28,\allowbreak31}
\crossref{Ezek}{7}{14}{Jer 4:5; 6:1; 51:27}
\crossref{Ezek}{7}{15}{Eze 5:12 De 32:23-\allowbreak25 Jer 14:18; 15:2,\allowbreak3 La 1:20}
\crossref{Ezek}{7}{16}{Eze 6:8 Ezr 9:15 Isa 1:9; 37:31 Jer 44:14,\allowbreak28}
\crossref{Ezek}{7}{17}{Eze 21:7 Isa 13:7,\allowbreak8 Jer 6:24 Heb 12:12}
\crossref{Ezek}{7}{18}{Isa 3:24; 15:2,\allowbreak3 Jer 48:37 Am 8:10}
\crossref{Ezek}{7}{19}{2Ki 7:7,\allowbreak8,\allowbreak15 Pr 11:4 Isa 2:20; 30:22 Zep 1:18 Mt 16:26}
\crossref{Ezek}{7}{20}{Eze 24:21 1Ch 29:1,\allowbreak2 2Ch 2:9; 3:1-\allowbreak17 Ezr 3:12 Ps 48:2; 50:2; 87:2,\allowbreak3}
\crossref{Ezek}{7}{21}{2Ki 24:13; 25:9,\allowbreak13-\allowbreak16 2Ch 36:18,\allowbreak19 Ps 74:2-\allowbreak8; 79:1 Jer 52:13-\allowbreak23}
\crossref{Ezek}{7}{22}{Ps 10:11; 35:22; 74:10,\allowbreak11,\allowbreak18-\allowbreak23 Jer 18:17}
\crossref{Ezek}{7}{23}{Eze 19:3-\allowbreak6 Jer 27:2; 40:1 La 3:7 Na 3:10}
\crossref{Ezek}{7}{24}{Eze 21:31; 28:7 Ps 106:41 Jer 4:7; 12:12 Hab 1:6-\allowbreak10}
\crossref{Ezek}{7}{25}{Isa 57:21; 59:8-\allowbreak12 Jer 8:15,\allowbreak16 La 4:17,\allowbreak18 Mic 1:12}
\crossref{Ezek}{7}{26}{Le 26:18,\allowbreak21,\allowbreak24,\allowbreak28 De 32:23 Jer 4:20}
\crossref{Ezek}{7}{27}{Eze 12:10-\allowbreak22; 17:15-\allowbreak21; 21:25 Jer 52:8-\allowbreak11}
\crossref{Ezek}{8}{1}{Eze 1:2; 20:1; 24:1; 26:1; 29:1,\allowbreak17; 31:1; 32:17; 40:1}
\crossref{Ezek}{8}{2}{Eze 1:4,\allowbreak26,\allowbreak27 Da 7:9,\allowbreak10 Re 1:14,\allowbreak15}
\crossref{Ezek}{8}{3}{Eze 2:9 Da 5:5; 10:10,\allowbreak18}
\crossref{Ezek}{8}{4}{Eze 1:26-\allowbreak28; 3:22,\allowbreak23; 9:3; 10:1-\allowbreak4; 11:22,\allowbreak23; 43:2-\allowbreak4 Ex 25:22; 40:34,\allowbreak35}
\crossref{Ezek}{8}{5}{Jer 3:2 Zec 5:5-\allowbreak11}
\crossref{Ezek}{8}{6}{8:12,\allowbreak17 Jer 3:6; 7:17}
\crossref{Ezek}{8}{7}{1Ki 7:12 2Ki 21:5}
\crossref{Ezek}{8}{8}{Job 34:22 Isa 29:15 Jer 2:34}
\crossref{Ezek}{8}{9}{Eze 20:8}
\crossref{Ezek}{8}{10}{Ex 20:4 Le 11:10-\allowbreak12,\allowbreak29-\allowbreak31,\allowbreak42-\allowbreak44 De 4:18; 14:3,\allowbreak7,\allowbreak8 Isa 57:6-\allowbreak10}
\crossref{Ezek}{8}{11}{Ex 24:1,\allowbreak9 Nu 11:16,\allowbreak25 Jer 5:5; 19:1 Da 9:8}
\crossref{Ezek}{8}{12}{8:6,\allowbreak15,\allowbreak17}
\crossref{Ezek}{8}{13}{8:6,\allowbreak15 Jer 9:3 2Ti 3:13}
\crossref{Ezek}{8}{14}{Eze 44:4; 46:9}
\crossref{Ezek}{8}{15}{8:6,\allowbreak12 2Ti 3:13}
\crossref{Ezek}{8}{16}{Eze 10:3; 40:28; 43:5; 45:19}
\crossref{Ezek}{8}{17}{}
\crossref{Ezek}{8}{18}{Eze 5:11-\allowbreak13; 7:4-\allowbreak9; 9:5,\allowbreak10; 16:42; 24:13 Na 1:2}
\crossref{Ezek}{9}{1}{Eze 43:6,\allowbreak7 Isa 6:8 Am 3:7,\allowbreak8 Re 1:10,\allowbreak11; 14:7}
\crossref{Ezek}{9}{2}{Jer 1:15; 5:15-\allowbreak17; 8:16,\allowbreak17; 25:9}
\crossref{Ezek}{9}{3}{Eze 3:23; 8:4; 10:4; 11:22,\allowbreak23; 43:2-\allowbreak4}
\crossref{Ezek}{9}{4}{Ex 12:7,\allowbreak13 Mal 3:16 2Co 1:22 Eph 4:30 2Ti 2:19 Re 7:2,\allowbreak3; 9:4}
\crossref{Ezek}{9}{5}{1Sa 9:15 Isa 5:9; 22:14}
\crossref{Ezek}{9}{6}{Nu 31:15-\allowbreak17 De 2:34; 3:6 Jos 6:17-\allowbreak21 1Sa 15:3 2Ch 36:17}
\crossref{Ezek}{9}{7}{Eze 7:20-\allowbreak22 2Ch 36:17 Ps 79:1-\allowbreak3 La 2:4-\allowbreak7 Lu 13:1}
\crossref{Ezek}{9}{8}{Nu 14:5; 16:4,\allowbreak21,\allowbreak22,\allowbreak45 De 9:18 Jos 7:6 1Ch 21:16 Ezr 9:5}
\crossref{Ezek}{9}{9}{Eze 7:23; 22:2-\allowbreak12,\allowbreak25-\allowbreak31 De 31:29; 32:5,\allowbreak15-\allowbreak22 2Ki 17:7-\allowbreak23}
\crossref{Ezek}{9}{10}{9:5; 5:11; 7:4; 8:18; 21:31,\allowbreak32}
\crossref{Ezek}{9}{11}{Ps 103:20 Isa 46:10,\allowbreak11 Zec 1:10,\allowbreak11; 6:7,\allowbreak8 Re 16:2,\allowbreak17}
\crossref{Ezek}{10}{1}{Isa 21:8,\allowbreak9 Hab 2:1}
\crossref{Ezek}{10}{2}{10:7; 9:2,\allowbreak3,\allowbreak11}
\crossref{Ezek}{10}{3}{10:3; 9:3; 43:4}
\crossref{Ezek}{10}{4}{10:18; 1:28; 9:3; 11:22,\allowbreak23 Nu 16:19}
\crossref{Ezek}{10}{5}{Eze 1:24}
\crossref{Ezek}{10}{6}{10:2 Ps 80:1; 99:1}
\crossref{Ezek}{10}{7}{10:6; 1:13}
\crossref{Ezek}{10}{8}{10:21; 1:8 Isa 6:6}
\crossref{Ezek}{10}{9}{Eze 1:15-\allowbreak17}
\crossref{Ezek}{10}{10}{Eze 1:16 Ps 36:6; 97:2; 104:24 Ro 11:33}
\crossref{Ezek}{10}{11}{10:22; 1:17}
\crossref{Ezek}{10}{12}{Eze 1:18 Re 4:6,\allowbreak8}
\crossref{Ezek}{10}{13}{}
\crossref{Ezek}{10}{14}{10:21; 1:6-\allowbreak10 1Ki 7:29,\allowbreak36 Re 4:7}
\crossref{Ezek}{10}{15}{10:18,\allowbreak19; 8:6; 11:22 Ho 9:12}
\crossref{Ezek}{10}{16}{Eze 1:19-\allowbreak21}
\crossref{Ezek}{10}{17}{Eze 1:12,\allowbreak20,\allowbreak21}
\crossref{Ezek}{10}{18}{10:4; 7:20-\allowbreak22 Ps 78:60,\allowbreak61 Jer 6:8; 7:12-\allowbreak14 Ho 9:12 Mt 23:37-\allowbreak39}
\crossref{Ezek}{10}{19}{Eze 1:17-\allowbreak21; 11:22,\allowbreak23}
\crossref{Ezek}{10}{20}{10:15; 1:22-\allowbreak28; 3:23}
\crossref{Ezek}{10}{21}{10:14; 1:8-\allowbreak10; 41:18,\allowbreak19 Re 4:7}
\crossref{Ezek}{10}{22}{Eze 1:10}
\crossref{Ezek}{11}{1}{11:24; 3:12,\allowbreak14; 8:3; 37:1; 40:1,\allowbreak2; 41:1 1Ki 18:12 2Ki 2:16 Ac 8:39}
\crossref{Ezek}{11}{2}{Es 8:3 Ps 2:1,\allowbreak2; 36:4; 52:2 Isa 30:1; 59:4 Jer 5:5; 18:18}
\crossref{Ezek}{11}{3}{Eze 7:7; 12:22,\allowbreak27 Isa 5:19 Jer 1:11,\allowbreak12 Am 6:5 2Pe 3:4}
\crossref{Ezek}{11}{4}{Eze 3:2-\allowbreak15,\allowbreak17-\allowbreak21; 20:46,\allowbreak47; 21:2; 25:2 Isa 58:1 Ho 6:5; 8:1}
\crossref{Ezek}{11}{5}{Eze 2:2; 3:24,\allowbreak27; 8:1 Nu 11:25,\allowbreak26 1Sa 10:6,\allowbreak10 Ac 10:44; 11:15}
\crossref{Ezek}{11}{6}{Eze 7:23; 9:9; 22:2-\allowbreak6,\allowbreak9,\allowbreak12,\allowbreak27; 24:6-\allowbreak9 2Ki 21:16 Isa 1:15 Jer 2:30,\allowbreak34}
\crossref{Ezek}{11}{7}{Eze 24:3-\allowbreak13 Mic 3:2,\allowbreak3}
\crossref{Ezek}{11}{8}{Job 3:25; 20:24 Pr 10:24 Isa 24:17,\allowbreak18; 30:16,\allowbreak17; 66:4}
\crossref{Ezek}{11}{9}{Eze 21:31 De 28:36,\allowbreak49,\allowbreak50 2Ki 24:4 Ne 9:36,\allowbreak37 Ps 106:41}
\crossref{Ezek}{11}{10}{2Ki 25:19-\allowbreak21 Jer 39:6; 52:9,\allowbreak10,\allowbreak24-\allowbreak27}
\crossref{Ezek}{11}{11}{11:3,\allowbreak7-\allowbreak10}
\crossref{Ezek}{11}{12}{11:21; 20:16,\allowbreak21,\allowbreak24 Le 26:40 1Ki 11:33 2Ki 21:22 Ezr 9:7 Ne 9:34}
\crossref{Ezek}{11}{13}{11:1; 37:7 Nu 14:35-\allowbreak37 De 7:4 1Ki 13:4 Pr 6:15 Jer 28:15-\allowbreak17}
\crossref{Ezek}{11}{14}{}
\crossref{Ezek}{11}{15}{Jer 24:1-\allowbreak5}
\crossref{Ezek}{11}{16}{Le 26:44 De 30:3,\allowbreak4 2Ki 24:12-\allowbreak16 Ps 44:11 Jer 24:5,\allowbreak6; 30:11}
\crossref{Ezek}{11}{17}{Eze 28:25; 34:13; 36:24; 37:21-\allowbreak28; 39:27-\allowbreak29 Isa 11:11-\allowbreak16 Jer 3:12,\allowbreak18}
\crossref{Ezek}{11}{18}{11:21; 5:11; 7:20; 37:23; 42:7,\allowbreak8 Isa 1:25-\allowbreak27; 30:22 Jer 16:18 Ho 14:8}
\crossref{Ezek}{11}{19}{Eze 36:26,\allowbreak27 De 30:6 2Ch 30:12 Jer 24:7; 32:39,\allowbreak40 Zep 3:9}
\crossref{Ezek}{11}{20}{11:12 Ps 105:45; 119:4,\allowbreak5,\allowbreak32 Lu 1:6,\allowbreak74,\allowbreak75 Ro 16:26 1Co 11:2}
\crossref{Ezek}{11}{21}{Ec 11:9 Jer 17:9 Mr 7:21-\allowbreak23 Heb 3:12,\allowbreak13; 10:38 Jas 1:14,\allowbreak15}
\crossref{Ezek}{11}{22}{Eze 1:19,\allowbreak20; 10:19}
\crossref{Ezek}{11}{23}{Eze 8:4; 9:3; 10:4,\allowbreak18; 43:4 Zec 14:4 Mt 23:37-\allowbreak39; 24:1,\allowbreak2}
\crossref{Ezek}{11}{24}{11:1; 8:3 2Ki 2:16 2Co 12:3}
\crossref{Ezek}{11}{25}{Eze 2:7; 3:4,\allowbreak17,\allowbreak27}
\crossref{Ezek}{12}{1}{12:1}
\crossref{Ezek}{12}{2}{Eze 2:3,\allowbreak6-\allowbreak8; 3:9,\allowbreak26,\allowbreak27; 17:12; 24:3; 44:6 De 9:7,\allowbreak24; 31:27 Ps 78:40}
\crossref{Ezek}{12}{3}{12:10-\allowbreak12; 4:1-\allowbreak17 Jer 13:1-\allowbreak11; 18:2-\allowbreak12; 19:1-\allowbreak15; 27:2}
\crossref{Ezek}{12}{4}{12:12 2Ki 25:4 Jer 39:4; 52:7}
\crossref{Ezek}{12}{5}{2Ki 25:4 Jer 39:2-\allowbreak4}
\crossref{Ezek}{12}{6}{1Sa 28:8 2Sa 15:30 Job 24:17}
\crossref{Ezek}{12}{7}{Eze 2:8; 24:18; 37:7,\allowbreak10 Jer 32:8-\allowbreak12 Mt 21:6,\allowbreak7 Mr 14:16 Joh 2:5-\allowbreak8}
\crossref{Ezek}{12}{8}{}
\crossref{Ezek}{12}{9}{12:1-\allowbreak3; 2:5-\allowbreak8}
\crossref{Ezek}{12}{10}{2Ki 9:25 Isa 13:1; 14:28 Mal 1:1}
\crossref{Ezek}{12}{11}{12:6}
\crossref{Ezek}{12}{12}{12:6 2Ki 25:4 Jer 39:4; 42:7}
\crossref{Ezek}{12}{13}{2Ki 25:5-\allowbreak7 Jer 34:3; 39:7; 52:8-\allowbreak11}
\crossref{Ezek}{12}{14}{Eze 5:10-\allowbreak12; 17:21 2Ki 25:4,\allowbreak5}
\crossref{Ezek}{12}{15}{12:16,\allowbreak20; 5:13; 6:7,\allowbreak14; 7:4; 11:10; 24:27; 25:11; 26:6; 28:26; 33:33}
\crossref{Ezek}{12}{16}{Eze 6:8-\allowbreak10; 14:22,\allowbreak23 Isa 1:9; 6:13; 10:22; 24:13 Jer 4:27; 30:11}
\crossref{Ezek}{12}{17}{12:17}
\crossref{Ezek}{12}{18}{Eze 4:16,\allowbreak17; 23:33 Le 26:26,\allowbreak36 De 28:48,\allowbreak65 Job 3:24 Ps 60:2,\allowbreak3}
\crossref{Ezek}{12}{19}{1Ki 17:10-\allowbreak12}
\crossref{Ezek}{12}{20}{Eze 15:6,\allowbreak8 Isa 3:26; 7:23,\allowbreak24; 24:3,\allowbreak12; 64:10,\allowbreak11 Jer 4:7,\allowbreak23-\allowbreak29}
\crossref{Ezek}{12}{21}{}
\crossref{Ezek}{12}{22}{Eze 18:2,\allowbreak3 Jer 23:33-\allowbreak40}
\crossref{Ezek}{12}{23}{Eze 18:3 Isa 28:22}
\crossref{Ezek}{12}{24}{Eze 13:23 1Ki 22:11-\allowbreak13,\allowbreak17 Pr 26:28 Jer 14:13-\allowbreak16; 23:14-\allowbreak29 La 2:14}
\crossref{Ezek}{12}{25}{12:28; 6:10 Nu 14:28-\allowbreak34 Isa 14:24; 55:11 La 2:17 Da 9:12 Zec 1:6}
\crossref{Ezek}{12}{26}{}
\crossref{Ezek}{12}{27}{12:22 Isa 28:14,\allowbreak15 Da 10:14 2Pe 3:4}
\crossref{Ezek}{12}{28}{12:23-\allowbreak25 Jer 4:7; 44:28 Mt 24:48-\allowbreak51 Mr 13:32-\allowbreak37 Lu 21:34-\allowbreak36}
\crossref{Ezek}{13}{1}{13:1}
\crossref{Ezek}{13}{2}{Eze 14:9,\allowbreak10; 22:25,\allowbreak28 2Ch 18:18-\allowbreak24 Isa 9:15; 56:9-\allowbreak12 Jer 5:31}
\crossref{Ezek}{13}{3}{13:18; 34:2 Jer 23:1 Mt 23:13-\allowbreak29 Lu 11:42-\allowbreak47,\allowbreak52 1Co 9:16}
\crossref{Ezek}{13}{4}{So 2:15 Mic 2:11; 3:5 Mt 7:15 Ro 16:18 2Co 11:13-\allowbreak15 Ga 2:4}
\crossref{Ezek}{13}{5}{Eze 22:30 Ex 17:9-\allowbreak13; 32:11,\allowbreak12 Nu 16:21,\allowbreak22,\allowbreak47,\allowbreak48 1Sa 12:23}
\crossref{Ezek}{13}{6}{13:23; 12:23,\allowbreak24; 22:28 La 2:14 2Pe 2:18}
\crossref{Ezek}{13}{7}{13:2,\allowbreak3,\allowbreak6 Mt 24:23,\allowbreak24}
\crossref{Ezek}{13}{8}{Eze 5:8; 21:3; 26:3; 28:22; 29:3,\allowbreak4,\allowbreak10; 35:3; 38:3,\allowbreak4; 39:1 Jer 50:31,\allowbreak32}
\crossref{Ezek}{13}{9}{Eze 11:13; 14:9,\allowbreak10 Ps 101:7 Jer 20:3-\allowbreak6; 28:15-\allowbreak17; 29:21,\allowbreak22,\allowbreak31,\allowbreak32}
\crossref{Ezek}{13}{10}{2Ki 21:9 Pr 12:26 Jer 23:13-\allowbreak15 1Ti 4:1 2Ti 3:13 1Jo 2:26}
\crossref{Ezek}{13}{11}{}
\crossref{Ezek}{13}{12}{De 32:37 Jud 9:38; 10:14 2Ki 3:13 Jer 2:28; 29:31,\allowbreak32; 37:19}
\crossref{Ezek}{13}{13}{Le 26:28 Isa 30:30 Ps 107:25; 148:8 Jer 23:19 Jon 1:4}
\crossref{Ezek}{13}{14}{Ps 11:3 Mic 1:6 Hab 3:13 Mt 7:26,\allowbreak27 Lu 6:49 1Co 3:11-\allowbreak15}
\crossref{Ezek}{13}{15}{Ne 4:3 Ps 62:3 Isa 30:13}
\crossref{Ezek}{13}{16}{13:10 Jer 5:31; 6:14; 8:11; 28:1,\allowbreak9-\allowbreak17; 29:31}
\crossref{Ezek}{13}{17}{Eze 4:3; 20:46; 21:2}
\crossref{Ezek}{13}{18}{13:3}
\crossref{Ezek}{13}{19}{Eze 20:39; 22:26}
\crossref{Ezek}{13}{20}{13:8,\allowbreak9,\allowbreak15,\allowbreak16}
\crossref{Ezek}{13}{21}{13:9}
\crossref{Ezek}{13}{22}{Eze 9:4 Jer 4:10; 14:13-\allowbreak17; 23:9,\allowbreak14 La 2:11-\allowbreak14}
\crossref{Ezek}{13}{23}{13:6-\allowbreak16; 12:24 De 18:20 Mic 3:6 Zec 13:3 2Ti 3:9}
\crossref{Ezek}{14}{1}{Eze 8:1; 20:1 2Ki 6:32 Ac 4:5,\allowbreak8}
\crossref{Ezek}{14}{2}{1Ki 14:4 Am 3:7}
\crossref{Ezek}{14}{3}{14:4,\allowbreak7; 6:9; 11:21; 20:16; 36:25 Jer 17:1,\allowbreak2,\allowbreak9 Eph 5:5}
\crossref{Ezek}{14}{4}{Eze 2:7; 3:4,\allowbreak17-\allowbreak21}
\crossref{Ezek}{14}{5}{14:9,\allowbreak10 Ho 10:2 Zec 7:11-\allowbreak14 2Th 2:9-\allowbreak11}
\crossref{Ezek}{14}{6}{Eze 18:30 1Sa 7:3 1Ki 8:47-\allowbreak49 Ne 1:8,\allowbreak9 Isa 55:6,\allowbreak7 Jer 8:5,\allowbreak6}
\crossref{Ezek}{14}{7}{Ex 12:48; 20:10 Le 16:29; 20:2; 24:22 Nu 15:15,\allowbreak29}
\crossref{Ezek}{14}{8}{Eze 15:7 Le 17:10; 20:3-\allowbreak6; 26:17 Ps 34:16 Jer 21:10; 44:11}
\crossref{Ezek}{14}{9}{Eze 20:25 2Sa 12:11,\allowbreak12 1Ki 22:20-\allowbreak23 Job 12:16 Ps 81:11,\allowbreak12}
\crossref{Ezek}{14}{10}{Eze 17:18-\allowbreak20; 23:49 Ge 4:13 Nu 5:31 Mic 7:9 Ga 6:5}
\crossref{Ezek}{14}{11}{Eze 34:10-\allowbreak31; 44:10,\allowbreak15; 48:11 De 13:11; 19:20 Ps 119:67 Isa 9:16}
\crossref{Ezek}{14}{12}{}
\crossref{Ezek}{14}{13}{Eze 9:9 Ezr 9:6 Isa 24:20 La 1:8,\allowbreak20 Da 9:5,\allowbreak10-\allowbreak12}
\crossref{Ezek}{14}{14}{14:16,\allowbreak18,\allowbreak20}
\crossref{Ezek}{14}{15}{Eze 5:17 Le 26:22 1Ki 20:36 2Ki 17:25 Jer 15:3}
\crossref{Ezek}{14}{16}{14:14,\allowbreak18 Mt 18:19,\allowbreak20 Jas 5:16}
\crossref{Ezek}{14}{17}{Eze 5:12,\allowbreak17; 21:3,\allowbreak4,\allowbreak9-\allowbreak15; 29:8; 38:21,\allowbreak22 Le 26:25 Jer 25:9; 47:6}
\crossref{Ezek}{14}{18}{}
\crossref{Ezek}{14}{19}{Eze 5:12; 38:22 Nu 14:12; 16:46-\allowbreak50 De 28:21,\allowbreak22,\allowbreak59-\allowbreak61 2Sa 24:13,\allowbreak15}
\crossref{Ezek}{14}{20}{14:14,\allowbreak16}
\crossref{Ezek}{14}{21}{14:13,\allowbreak15,\allowbreak17,\allowbreak19; 5:12,\allowbreak17; 6:11,\allowbreak12; 33:27 Jer 15:2,\allowbreak3 Am 4:6-\allowbreak12}
\crossref{Ezek}{14}{22}{Eze 6:8 De 4:31 2Ch 36:20 Isa 6:13; 10:20-\allowbreak22; 17:4-\allowbreak6; 24:13; 40:1,\allowbreak2}
\crossref{Ezek}{14}{23}{Eze 8:6-\allowbreak18; 9:8,\allowbreak9 Ge 18:22-\allowbreak33 De 8:2 Ne 9:33 Pr 26:2 Jer 7:17-\allowbreak28}
\crossref{Ezek}{15}{1}{15:1}
\crossref{Ezek}{15}{2}{Isa 44:23 Mic 3:12 Zec 11:2}
\crossref{Ezek}{15}{3}{Jer 24:8 Mt 5:13 Mr 9:50 Lu 14:34,\allowbreak35}
\crossref{Ezek}{15}{4}{Ps 80:16 Isa 27:11 Joh 15:6 Heb 6:8}
\crossref{Ezek}{15}{5}{Jer 3:16}
\crossref{Ezek}{15}{6}{15:2; 17:3-\allowbreak10; 20:47,\allowbreak48 Isa 5:1-\allowbreak6,\allowbreak24,\allowbreak25 Jer 4:7; 7:20; 21:7; 24:8-\allowbreak10}
\crossref{Ezek}{15}{7}{Eze 14:8 Le 17:10; 20:3-\allowbreak6; 26:17 Ps 34:16 Jer 21:10}
\crossref{Ezek}{15}{8}{Eze 6:14; 14:13-\allowbreak21; 33:29 Isa 6:11; 24:3-\allowbreak12 Jer 25:10,\allowbreak11 Zep 1:18}
\crossref{Ezek}{16}{1}{16:1}
\crossref{Ezek}{16}{2}{Eze 20:4; 22:2; 23:36; 33:7-\allowbreak9 Isa 58:1 Ho 8:1}
\crossref{Ezek}{16}{3}{16:45; 21:30 Ge 11:25,\allowbreak29 Jos 24:14 Ne 9:7 Isa 1:10; 51:1,\allowbreak2}
\crossref{Ezek}{16}{4}{Eze 20:8,\allowbreak13 Ge 15:13 Ex 1:11-\allowbreak14; 2:23,\allowbreak24; 5:16-\allowbreak21 De 5:6; 15:15}
\crossref{Ezek}{16}{5}{Eze 2:6 Isa 49:15 La 2:11,\allowbreak19; 4:3,\allowbreak10}
\crossref{Ezek}{16}{6}{Ex 2:24,\allowbreak25; 3:7,\allowbreak8 Ac 7:34}
\crossref{Ezek}{16}{7}{Ge 22:17 Ex 1:7; 12:37 Ac 7:17}
\crossref{Ezek}{16}{8}{16:6 De 7:6-\allowbreak8 Ru 3:9 1Sa 12:22 Isa 41:8,\allowbreak9; 43:4; 63:7-\allowbreak9 Jer 2:2,\allowbreak3}
\crossref{Ezek}{16}{9}{16:4; 36:25 Ps 51:7 Isa 4:4 Joh 13:8-\allowbreak10 1Co 6:11; 10:2 Heb 9:10-\allowbreak14}
\crossref{Ezek}{16}{10}{16:7 Ps 45:13,\allowbreak14 Isa 61:3,\allowbreak10 Lu 15:22 Re 21:2}
\crossref{Ezek}{16}{11}{Ge 24:22,\allowbreak47,\allowbreak53}
\crossref{Ezek}{16}{12}{Ge 24:22}
\crossref{Ezek}{16}{13}{16:19 De 8:8; 32:13,\allowbreak14 Ps 45:13,\allowbreak14; 81:16; 147:14 Ho 2:5}
\crossref{Ezek}{16}{14}{De 4:6-\allowbreak8,\allowbreak32-\allowbreak38 Jos 2:9-\allowbreak11; 9:6-\allowbreak9 1Ki 10:1-\allowbreak13,\allowbreak24 2Ch 2:11,\allowbreak12}
\crossref{Ezek}{16}{15}{Eze 33:13 De 32:15 Isa 48:1 Jer 7:4 Mic 3:11 Zep 3:11 Mt 3:9}
\crossref{Ezek}{16}{16}{Eze 7:20 2Ki 23:7 2Ch 28:24 Ho 2:8}
\crossref{Ezek}{16}{17}{Eze 7:19; 23:14-\allowbreak21 Ex 32:1-\allowbreak4 Ho 2:13; 10:1}
\crossref{Ezek}{16}{18}{16:13 De 32:14-\allowbreak17 Ho 2:8-\allowbreak13}
\crossref{Ezek}{16}{19}{16:21; 23:4 Ge 17:7 Ex 13:2,\allowbreak12 De 29:11,\allowbreak12}
\crossref{Ezek}{16}{20}{Ps 106:37}
\crossref{Ezek}{16}{21}{16:3-\allowbreak7,\allowbreak43,\allowbreak60-\allowbreak63 Jer 2:2 Ho 2:3; 11:1}
\crossref{Ezek}{16}{22}{Eze 2:10; 13:3,\allowbreak18; 24:6 Jer 13:27 Zep 3:1 Mt 11:21; 23:13-\allowbreak29 Re 8:13}
\crossref{Ezek}{16}{23}{16:31,\allowbreak39; 20:28,\allowbreak29 2Ki 21:3-\allowbreak7; 23:5-\allowbreak7,\allowbreak11,\allowbreak12 2Ch 33:3-\allowbreak7}
\crossref{Ezek}{16}{24}{16:31 Ge 38:14,\allowbreak21 Pr 9:14,\allowbreak15 Isa 3:9 Jer 2:23,\allowbreak24; 3:2; 6:15}
\crossref{Ezek}{16}{25}{}
\crossref{Ezek}{16}{26}{Eze 8:10,\allowbreak14; 20:7,\allowbreak8; 23:3,\allowbreak8,\allowbreak19-\allowbreak21 Ex 32:4 De 29:16,\allowbreak17 Jos 24:14}
\crossref{Ezek}{16}{27}{Eze 14:9 Isa 5:25; 9:12,\allowbreak17}
\crossref{Ezek}{16}{28}{Eze 23:5-\allowbreak9,\allowbreak12-\allowbreak21 Jud 10:6 2Ki 16:7,\allowbreak10-\allowbreak18; 21:11 2Ch 28:23}
\crossref{Ezek}{16}{29}{Eze 13:14-\allowbreak23 Jud 2:12-\allowbreak19 2Ki 21:9}
\crossref{Ezek}{16}{30}{Pr 9:13 Isa 1:3 Jer 2:12,\allowbreak13; 4:22}
\crossref{Ezek}{16}{31}{16:25 Ho 12:11}
\crossref{Ezek}{16}{32}{16:8; 23:37,\allowbreak45 Jer 2:25,\allowbreak28; 3:1,\allowbreak8,\allowbreak9,\allowbreak20 Ho 2:2; 3:1 2Co 11:2,\allowbreak3}
\crossref{Ezek}{16}{33}{Ge 38:16-\allowbreak18 De 23:17,\allowbreak18 Ho 2:12 Joe 3:3 Mic 1:7 Lu 15:30}
\crossref{Ezek}{16}{34}{Ho 8:9}
\crossref{Ezek}{16}{35}{Isa 1:21; 23:15,\allowbreak16 Jer 3:1,\allowbreak6-\allowbreak8 Ho 2:5 Na 3:4 Joh 4:10,\allowbreak18}
\crossref{Ezek}{16}{36}{16:15-\allowbreak22; 22:15; 23:8; 24:13; 36:25 La 1:9 Zep 3:1}
\crossref{Ezek}{16}{37}{Eze 23:9,\allowbreak10,\allowbreak22-\allowbreak30 Jer 4:30; 13:22,\allowbreak26; 22:20 La 1:8,\allowbreak19 Ho 2:3,\allowbreak10}
\crossref{Ezek}{16}{38}{16:40; 23:45-\allowbreak47 Ge 38:11,\allowbreak24 Le 20:10 De 22:22-\allowbreak24 Mt 1:18,\allowbreak19}
\crossref{Ezek}{16}{39}{16:24,\allowbreak25,\allowbreak31; 7:22-\allowbreak24 Isa 27:9}
\crossref{Ezek}{16}{40}{Hab 1:6-\allowbreak10 Joh 8:5-\allowbreak7}
\crossref{Ezek}{16}{41}{De 13:16 2Ki 25:9 Jer 39:8; 52:13 Mic 3:12}
\crossref{Ezek}{16}{42}{Eze 5:13; 21:17 2Sa 21:14 Isa 1:24 Zec 6:8}
\crossref{Ezek}{16}{43}{16:22 Ps 78:42; 106:13 Jer 2:32}
\crossref{Ezek}{16}{44}{Eze 18:2,\allowbreak3 1Sa 24:13}
\crossref{Ezek}{16}{45}{16:8,\allowbreak15,\allowbreak20,\allowbreak21; 23:37-\allowbreak39 De 5:9; 12:31 Isa 1:4}
\crossref{Ezek}{16}{46}{16:51; 23:4,\allowbreak11,\allowbreak31-\allowbreak33 Jer 3:8-\allowbreak11 Mic 1:5}
\crossref{Ezek}{16}{47}{16:48,\allowbreak51; 5:6,\allowbreak7 2Ki 21:9,\allowbreak16 Joh 15:21,\allowbreak22 1Co 5:1}
\crossref{Ezek}{16}{48}{Mt 10:15; 11:24 Mr 6:11 Lu 10:12 Ac 7:52}
\crossref{Ezek}{16}{49}{Eze 28:2,\allowbreak9,\allowbreak17; 29:3 Ge 19:9 Ps 138:6 Pr 16:5,\allowbreak18; 18:12; 21:4 Isa 3:9}
\crossref{Ezek}{16}{50}{Ge 13:13; 18:20; 19:5 Le 18:22 De 23:17 2Ki 23:7 Pr 16:18}
\crossref{Ezek}{16}{51}{Lu 12:47,\allowbreak48 Ro 3:9-\allowbreak20}
\crossref{Ezek}{16}{52}{16:56 Mt 7:1-\allowbreak5 Lu 6:37 Ro 2:1,\allowbreak10,\allowbreak26,\allowbreak27}
\crossref{Ezek}{16}{53}{16:60,\allowbreak61; 29:14; 39:25 Job 42:10 Ps 14:7; 85:1; 126:1 Isa 1:9}
\crossref{Ezek}{16}{54}{16:52,\allowbreak63; 36:31,\allowbreak32 Jer 2:26}
\crossref{Ezek}{16}{55}{16:53; 36:11 Mal 3:4}
\crossref{Ezek}{16}{56}{Isa 65:5 Zep 3:11 Lu 15:28-\allowbreak30; 18:11}
\crossref{Ezek}{16}{57}{16:36,\allowbreak37; 21:24; 23:18,\allowbreak19 Ps 50:21 La 4:22 Ho 2:10; 7:1 1Co 4:5}
\crossref{Ezek}{16}{58}{Eze 23:49 Ge 4:13 La 5:7}
\crossref{Ezek}{16}{59}{Eze 7:4,\allowbreak8,\allowbreak9; 14:4 Isa 3:11 Jer 2:19 Mt 7:1,\allowbreak2 Ro 2:8,\allowbreak9}
\crossref{Ezek}{16}{60}{16:8 Le 26:42,\allowbreak45 Ne 1:5-\allowbreak11 Ps 105:8; 106:45 Jer 2:2; 33:20-\allowbreak26}
\crossref{Ezek}{16}{61}{16:63; 20:43; 36:31,\allowbreak32 Job 42:5,\allowbreak6 Ps 119:59 Jer 31:18-\allowbreak20; 50:4,\allowbreak5}
\crossref{Ezek}{16}{62}{16:60 Da 9:27 Ho 2:18-\allowbreak23}
\crossref{Ezek}{16}{63}{16:61; 36:31,\allowbreak32 Ezr 9:6 Da 9:7,\allowbreak8}
\crossref{Ezek}{17}{1}{17:1}
\crossref{Ezek}{17}{2}{Eze 20:49 Jud 9:8-\allowbreak15; 14:12-\allowbreak19 2Sa 12:1-\allowbreak4 Ho 12:10 Mt 13:13,\allowbreak14,\allowbreak35}
\crossref{Ezek}{17}{3}{}
\crossref{Ezek}{17}{4}{Isa 43:14; 47:15 Jer 51:13 Re 18:3,\allowbreak11-\allowbreak19}
\crossref{Ezek}{17}{5}{}
\crossref{Ezek}{17}{6}{17:14 Pr 16:18,\allowbreak19}
\crossref{Ezek}{17}{7}{}
\crossref{Ezek}{17}{8}{17:5,\allowbreak6}
\crossref{Ezek}{17}{9}{}
\crossref{Ezek}{17}{10}{Eze 19:12-\allowbreak14 Ho 12:1; 13:15 Mt 21:19 Mr 11:20 Joh 15:6 Jude 1:12}
\crossref{Ezek}{17}{11}{}
\crossref{Ezek}{17}{12}{Eze 2:5,\allowbreak8; 3:9; 12:9 Isa 1:2}
\crossref{Ezek}{17}{13}{17:5 2Ki 24:17 Jer 37:1}
\crossref{Ezek}{17}{14}{17:6; 29:14 De 28:43 1Sa 2:7,\allowbreak30 Ne 9:36,\allowbreak37 La 5:10 Mt 22:17-\allowbreak21}
\crossref{Ezek}{17}{15}{17:7 2Ki 24:20 2Ch 36:13 Jer 52:3}
\crossref{Ezek}{17}{16}{17:18,\allowbreak19; 16:59 Ex 20:7 Nu 30:2 Jos 9:20 2Sa 21:2 Ps 15:4}
\crossref{Ezek}{17}{17}{Eze 29:6,\allowbreak7 Isa 36:6 Jer 37:7 La 4:17}
\crossref{Ezek}{17}{18}{1Ch 29:24 2Ch 30:8}
\crossref{Ezek}{17}{19}{Eze 21:23-\allowbreak27 De 5:11 Jer 5:2,\allowbreak9; 7:9-\allowbreak15}
\crossref{Ezek}{17}{20}{Eze 12:13; 32:3 Jos 10:16-\allowbreak18 2Sa 18:9 2Ch 33:11 Job 10:16 Ec 9:12}
\crossref{Ezek}{17}{21}{Eze 5:12; 12:14 2Ki 25:5,\allowbreak11 Jer 48:44; 52:8 Am 9:1,\allowbreak9,\allowbreak10}
\crossref{Ezek}{17}{22}{Eze 34:29 Ps 80:15 Isa 4:2; 11:1-\allowbreak5 Jer 23:5,\allowbreak6; 33:15,\allowbreak16 Zec 3:8}
\crossref{Ezek}{17}{23}{Ps 92:12,\allowbreak13 Isa 27:6 Joh 12:24; 15:5-\allowbreak8}
\crossref{Ezek}{17}{24}{Ps 96:11,\allowbreak12 Isa 55:12,\allowbreak13}
\crossref{Ezek}{18}{1}{18:1}
\crossref{Ezek}{18}{2}{Eze 17:12 Isa 3:15 Ro 9:20}
\crossref{Ezek}{18}{3}{18:19,\allowbreak20,\allowbreak30; 33:11-\allowbreak20; 36:31,\allowbreak32 Ro 3:19}
\crossref{Ezek}{18}{4}{Nu 16:22; 27:16 Zec 12:1 Heb 12:9}
\crossref{Ezek}{18}{5}{Ps 15:2-\allowbreak5; 24:4-\allowbreak6 Mt 7:21-\allowbreak27 Ro 2:7-\allowbreak10 Jas 1:22-\allowbreak25; 2:14-\allowbreak26}
\crossref{Ezek}{18}{6}{18:11,\allowbreak15; 6:13; 20:28; 22:9 Ex 34:15 Nu 25:2 1Co 10:20}
\crossref{Ezek}{18}{7}{18:12,\allowbreak16,\allowbreak18; 22:12,\allowbreak13,\allowbreak27-\allowbreak29 Ex 22:21-\allowbreak24; 23:9 Le 19:15; 25:14}
\crossref{Ezek}{18}{8}{18:13,\allowbreak17; 22:12 Ex 22:25 Le 25:35-\allowbreak37 De 23:19,\allowbreak20 Ne 5:1-\allowbreak11}
\crossref{Ezek}{18}{9}{18:17; 20:13; 33:15; 36:27; 37:24 De 4:1; 5:1; 6:1,\allowbreak2; 10:12,\allowbreak13; 11:1}
\crossref{Ezek}{18}{10}{Le 19:13 Mal 3:8,\allowbreak9 Joh 18:40}
\crossref{Ezek}{18}{11}{18:7 Mt 7:21-\allowbreak27 Lu 11:28 Joh 13:17; 15:14 Php 4:9 Jas 2:17}
\crossref{Ezek}{18}{12}{18:7,\allowbreak16 Ho 12:7 Am 4:1 Zec 7:10 Jas 2:6}
\crossref{Ezek}{18}{13}{18:8,\allowbreak17}
\crossref{Ezek}{18}{14}{18:10 Pr 17:21; 23:24}
\crossref{Ezek}{18}{15}{18:6,\allowbreak7,\allowbreak11-\allowbreak13}
\crossref{Ezek}{18}{16}{Job 22:7; 31:19 Pr 22:9; 25:21; 31:20 Ec 11:1,\allowbreak2 Isa 58:7-\allowbreak10}
\crossref{Ezek}{18}{17}{18:8 Job 29:16 Pr 14:31; 29:7,\allowbreak14 Jer 22:16 Da 4:27 Mt 18:27-\allowbreak35}
\crossref{Ezek}{18}{18}{18:4,\allowbreak20,\allowbreak24,\allowbreak26; 3:18 Isa 3:11 Joh 8:21,\allowbreak24}
\crossref{Ezek}{18}{19}{Ex 20:5 De 5:9 2Ki 23:26; 24:3,\allowbreak4 Jer 15:4 La 5:7}
\crossref{Ezek}{18}{20}{18:4,\allowbreak13 De 24:16 1Ki 14:13 2Ki 14:6; 22:18-\allowbreak20 2Ch 25:4}
\crossref{Ezek}{18}{21}{18:27,\allowbreak28,\allowbreak30; 33:11-\allowbreak16,\allowbreak19 2Ch 33:12,\allowbreak13 Pr 28:13 Isa 1:16-\allowbreak20; 55:6,\allowbreak7}
\crossref{Ezek}{18}{22}{18:24; 33:16 1Ki 17:18 Ps 25:7; 32:1,\allowbreak2; 51:1; 103:12 Isa 43:25}
\crossref{Ezek}{18}{23}{18:32; 33:11 La 3:33 Ho 11:8 1Ti 2:4 2Pe 3:9}
\crossref{Ezek}{18}{24}{18:26; 3:20,\allowbreak21; 33:12,\allowbreak13,\allowbreak18 1Sa 15:11 2Ch 24:2,\allowbreak17-\allowbreak22 Ps 36:3,\allowbreak4}
\crossref{Ezek}{18}{25}{18:29; 33:17,\allowbreak20 Job 32:2; 34:5-\allowbreak10; 35:2; 40:8; 42:4-\allowbreak6 Mal 2:17}
\crossref{Ezek}{18}{26}{18:24}
\crossref{Ezek}{18}{27}{18:21 Isa 1:18; 55:7 Mt 9:13; 21:28-\allowbreak32 Ac 3:19; 20:21; 26:20}
\crossref{Ezek}{18}{28}{18:14; 12:3 De 32:29 Ps 119:1,\allowbreak6,\allowbreak59 Jer 31:18-\allowbreak20 Lu 15:17,\allowbreak18}
\crossref{Ezek}{18}{29}{18:2,\allowbreak25 Pr 19:3}
\crossref{Ezek}{18}{30}{Eze 7:3,\allowbreak8,\allowbreak9,\allowbreak27; 33:20; 34:20 Ec 3:17; 12:14 1Pe 1:17 Re 20:12}
\crossref{Ezek}{18}{31}{Eze 20:7 Ps 34:14 Isa 1:16,\allowbreak17; 30:22; 55:7 Ro 8:13 Eph 4:22-\allowbreak32}
\crossref{Ezek}{18}{32}{18:23 La 3:33 2Pe 3:9}
\crossref{Ezek}{19}{1}{19:14; 2:10; 26:17; 27:2; 32:16,\allowbreak18 Jer 9:1,\allowbreak10,\allowbreak17,\allowbreak18; 13:17,\allowbreak18}
\crossref{Ezek}{19}{2}{}
\crossref{Ezek}{19}{3}{19:6 2Ki 23:31,\allowbreak32 2Ch 36:1,\allowbreak2}
\crossref{Ezek}{19}{4}{}
\crossref{Ezek}{19}{5}{}
\crossref{Ezek}{19}{6}{19:3}
\crossref{Ezek}{19}{7}{Eze 22:25 Pr 19:12; 28:3,\allowbreak15,\allowbreak16}
\crossref{Ezek}{19}{8}{2Ki 24:1-\allowbreak6}
\crossref{Ezek}{19}{9}{2Ch 36:6 Jer 22:18,\allowbreak19; 36:30,\allowbreak31}
\crossref{Ezek}{19}{10}{19:2 Ho 2:2,\allowbreak5}
\crossref{Ezek}{19}{11}{Eze 31:3 Da 4:11,\allowbreak20,\allowbreak21}
\crossref{Ezek}{19}{12}{Eze 17:10 Jer 4:11,\allowbreak12 Ho 13:15}
\crossref{Ezek}{19}{13}{19:10 De 28:47,\allowbreak48 Jer 52:27-\allowbreak31}
\crossref{Ezek}{19}{14}{19:11; 21:25-\allowbreak27 Ge 49:10 Ne 9:37 Ps 79:7; 80:15,\allowbreak16 Ho 3:4; 10:3}
\crossref{Ezek}{20}{1}{Eze 14:1-\allowbreak3; 33:30-\allowbreak33 1Ki 14:2-\allowbreak6; 22:15-\allowbreak28 2Ki 3:13 Isa 29:13; 58:2}
\crossref{Ezek}{20}{2}{}
\crossref{Ezek}{20}{3}{Isa 1:12 Mt 3:7 Lu 3:7}
\crossref{Ezek}{20}{4}{Eze 14:14,\allowbreak20; 22:2; 23:36,\allowbreak45 Isa 5:3 Jer 7:16; 11:14; 14:11-\allowbreak14; 15:1}
\crossref{Ezek}{20}{5}{Ex 6:6,\allowbreak7; 19:4-\allowbreak6 De 4:37; 7:6; 14:2 Ps 33:12 Isa 41:8,\allowbreak9; 43:10}
\crossref{Ezek}{20}{6}{20:5,\allowbreak15,\allowbreak23,\allowbreak42}
\crossref{Ezek}{20}{7}{20:8; 18:6,\allowbreak15,\allowbreak31 Isa 2:20,\allowbreak21; 31:7}
\crossref{Ezek}{20}{8}{De 9:7 Ne 9:26 Isa 63:10}
\crossref{Ezek}{20}{9}{20:14,\allowbreak22; 36:21,\allowbreak22; 39:7 Ex 32:12 Nu 14:13-\allowbreak25 De 9:28; 32:26,\allowbreak27}
\crossref{Ezek}{20}{10}{Ex 13:17,\allowbreak18; 14:17-\allowbreak22; 15:22; 20:2}
\crossref{Ezek}{20}{11}{De 4:8 Ne 9:13,\allowbreak14 Ps 147:19,\allowbreak20 Ro 3:2}
\crossref{Ezek}{20}{12}{Ge 2:3 Ex 16:29; 20:8-\allowbreak11; 35:2 Le 23:3,\allowbreak24,\allowbreak32,\allowbreak39; 25:4 De 5:12-\allowbreak15}
\crossref{Ezek}{20}{13}{20:8 Ex 16:28; 32:8 Nu 14:22 De 9:12-\allowbreak24; 31:27 1Sa 8:8}
\crossref{Ezek}{20}{14}{20:9,\allowbreak22; 36:22,\allowbreak23 Eph 1:6,\allowbreak12}
\crossref{Ezek}{20}{15}{20:23 Nu 14:23-\allowbreak30; 26:64,\allowbreak65 De 1:34,\allowbreak35 Ps 95:11; 106:26 Heb 3:11}
\crossref{Ezek}{20}{16}{20:13,\allowbreak14}
\crossref{Ezek}{20}{17}{Eze 8:18; 9:10 1Sa 24:10 Ne 9:19 Ps 78:37,\allowbreak38}
\crossref{Ezek}{20}{18}{Nu 14:32,\allowbreak33; 32:13-\allowbreak15 De 4:3-\allowbreak6 Ps 78:6-\allowbreak8}
\crossref{Ezek}{20}{19}{Ex 20:2,\allowbreak3 De 5:6,\allowbreak7; 7:4-\allowbreak6 Ps 81:9,\allowbreak10 Jer 3:22,\allowbreak23}
\crossref{Ezek}{20}{20}{20:12; 44:24 Ex 20:11; 31:13-\allowbreak17 Ne 13:15-\allowbreak22 Isa 58:13}
\crossref{Ezek}{20}{21}{Nu 21:5; 25:1-\allowbreak8 De 9:23,\allowbreak24; 31:27 Ps 106:29-\allowbreak33 Ac 13:18}
\crossref{Ezek}{20}{22}{20:17 Job 13:21 Ps 78:38 La 2:8}
\crossref{Ezek}{20}{23}{20:15 De 32:40 Re 10:5,\allowbreak6}
\crossref{Ezek}{20}{24}{20:13,\allowbreak16}
\crossref{Ezek}{20}{25}{}
\crossref{Ezek}{20}{26}{20:31 Isa 63:17 Ro 11:7-\allowbreak10}
\crossref{Ezek}{20}{27}{Eze 2:7; 3:4,\allowbreak11,\allowbreak27}
\crossref{Ezek}{20}{28}{Jos 23:3,\allowbreak4,\allowbreak14 Ne 9:22-\allowbreak26 Ps 78:55-\allowbreak58}
\crossref{Ezek}{20}{29}{}
\crossref{Ezek}{20}{30}{Nu 32:14 Jud 2:19 Jer 7:26; 9:14; 16:12 Mt 23:32 Ac 7:51}
\crossref{Ezek}{20}{31}{20:26 De 18:10-\allowbreak12 Ps 106:37-\allowbreak39 Jer 7:31; 19:5}
\crossref{Ezek}{20}{32}{Eze 11:5; 38:10 Ps 139:2 Pr 19:21 La 3:37}
\crossref{Ezek}{20}{33}{Eze 8:18 Jer 21:5; 42:18; 44:6 La 2:4 Da 9:11,\allowbreak12}
\crossref{Ezek}{20}{34}{20:38; 34:16 Isa 27:9-\allowbreak13 Am 9:9,\allowbreak10}
\crossref{Ezek}{20}{35}{20:36; 19:13; 38:8 Ho 2:14 Mic 4:10; 7:13-\allowbreak15 Re 12:14}
\crossref{Ezek}{20}{36}{20:13,\allowbreak21 Ex 32:7-\allowbreak35 Nu 11:1-\allowbreak35; 14:1-\allowbreak45; 16:1-\allowbreak50; 25:1-\allowbreak18}
\crossref{Ezek}{20}{37}{Eze 34:17 Le 27:32 Jer 33:13 Mt 25:32,\allowbreak33}
\crossref{Ezek}{20}{38}{Eze 11:21; 34:17,\allowbreak20-\allowbreak22 Nu 14:28-\allowbreak30 Am 9:9,\allowbreak10 Zec 13:8,\allowbreak9 Mal 3:3}
\crossref{Ezek}{20}{39}{20:25,\allowbreak26 Jud 10:14 2Ki 3:13 Ps 81:12 Ho 4:17 Am 4:4,\allowbreak5}
\crossref{Ezek}{20}{40}{Eze 37:22-\allowbreak28 Isa 56:7; 60:7; 66:23 Zec 8:20-\allowbreak23 Mal 1:11; 3:4 Re 12:1}
\crossref{Ezek}{20}{41}{20:28; 6:13 Ge 8:21 Le 1:9,\allowbreak13,\allowbreak17 Eph 5:2 Php 4:18}
\crossref{Ezek}{20}{42}{20:38,\allowbreak44; 24:24; 26:13; 36:23; 38:23 Jer 24:7; 31:34 Joh 17:3}
\crossref{Ezek}{20}{43}{Eze 6:9 Le 26:39-\allowbreak41 Ne 1:8-\allowbreak10 Ho 5:15}
\crossref{Ezek}{20}{44}{20:38; 24:24}
\crossref{Ezek}{20}{45}{}
\crossref{Ezek}{20}{46}{Eze 4:7; 6:2}
\crossref{Ezek}{20}{47}{Eze 17:24 Lu 23:31}
\crossref{Ezek}{20}{48}{De 29:24-\allowbreak28 2Ch 7:20-\allowbreak22 Isa 26:11 Jer 40:2,\allowbreak3 La 2:16,\allowbreak17}
\crossref{Ezek}{20}{49}{}
\crossref{Ezek}{21}{1}{21:1}
\crossref{Ezek}{21}{2}{Eze 4:3,\allowbreak7; 20:46; 25:2; 28:21; 29:2; 38:2 Eph 6:19}
\crossref{Ezek}{21}{3}{Eze 5:8; 26:3 Jer 21:13; 50:31; 51:25 Na 2:13; 3:5}
\crossref{Ezek}{21}{4}{Eze 6:11-\allowbreak14; 7:2; 20:47}
\crossref{Ezek}{21}{5}{Eze 20:48 Nu 14:21-\allowbreak23 De 29:24-\allowbreak28 1Ki 9:7-\allowbreak9}
\crossref{Ezek}{21}{6}{21:12; 6:11; 9:4 Isa 22:4 Jer 4:19; 9:17-\allowbreak21 Joh 11:33-\allowbreak35}
\crossref{Ezek}{21}{7}{Eze 12:9-\allowbreak11; 20:49; 24:19}
\crossref{Ezek}{21}{8}{}
\crossref{Ezek}{21}{9}{21:3,\allowbreak15,\allowbreak28 De 32:41,\allowbreak42 Job 20:25 Isa 66:16 Jer 12:12; 15:2}
\crossref{Ezek}{21}{10}{Jer 46:4 Na 3:3 Hab 3:11}
\crossref{Ezek}{21}{11}{21:19 Jer 25:9,\allowbreak33; 51:20-\allowbreak23}
\crossref{Ezek}{21}{12}{21:6; 9:8; 30:2 Jer 25:34 Joe 1:13 Mic 1:8}
\crossref{Ezek}{21}{13}{21:10,\allowbreak25}
\crossref{Ezek}{21}{14}{21:17; 6:11 Nu 24:10}
\crossref{Ezek}{21}{15}{21:22; 15:7 Jer 17:27}
\crossref{Ezek}{21}{16}{21:4,\allowbreak20; 14:17; 16:46}
\crossref{Ezek}{21}{17}{21:14; 22:13 Nu 24:10}
\crossref{Ezek}{21}{18}{21:18}
\crossref{Ezek}{21}{19}{Eze 4:1-\allowbreak3; 5:1-\allowbreak17 Jer 1:10}
\crossref{Ezek}{21}{20}{Eze 25:5 De 3:11 2Sa 12:26 Jer 49:2 Am 1:14}
\crossref{Ezek}{21}{21}{Pr 16:33; 21:1}
\crossref{Ezek}{21}{22}{Eze 4:2}
\crossref{Ezek}{21}{23}{Eze 11:3; 12:22 Isa 28:14,\allowbreak15}
\crossref{Ezek}{21}{24}{Eze 16:16-\allowbreak22; 22:3-\allowbreak12,\allowbreak24-\allowbreak31; 23:5-\allowbreak21; 24:7 Isa 3:9 Jer 2:34; 3:2}
\crossref{Ezek}{21}{25}{Eze 17:19 2Ch 36:13 Jer 24:8; 52:2}
\crossref{Ezek}{21}{26}{Eze 12:12,\allowbreak13; 16:12 2Ki 25:6,\allowbreak27 Jer 13:18; 39:6,\allowbreak7; 52:9-\allowbreak11,\allowbreak31-\allowbreak34}
\crossref{Ezek}{21}{27}{21:13; 17:22,\allowbreak23; 34:23; 37:24,\allowbreak25 Ge 49:10 Nu 24:19 Ps 2:6; 72:7-\allowbreak10}
\crossref{Ezek}{21}{28}{21:20; 25:2-\allowbreak7 Jer 49:1-\allowbreak5 Am 1:13-\allowbreak15 Zep 2:8-\allowbreak10}
\crossref{Ezek}{21}{29}{Eze 12:24; 13:23; 22:28 Isa 44:25; 47:13 Jer 27:9}
\crossref{Ezek}{21}{30}{21:4,\allowbreak5 Jer 47:6,\allowbreak7}
\crossref{Ezek}{21}{31}{Eze 7:8; 14:19; 22:22 Na 1:6}
\crossref{Ezek}{21}{32}{Eze 20:47,\allowbreak48 Mal 4:1 Mt 3:10,\allowbreak12}
\crossref{Ezek}{22}{1}{22:1}
\crossref{Ezek}{22}{2}{Eze 20:4; 23:36}
\crossref{Ezek}{22}{3}{22:27; 24:6-\allowbreak9 Zep 3:3}
\crossref{Ezek}{22}{4}{22:2 2Ki 21:16}
\crossref{Ezek}{22}{5}{}
\crossref{Ezek}{22}{6}{22:27 Ne 9:34 Isa 1:23 Jer 2:26,\allowbreak27; 5:5; 32:32 Da 9:8}
\crossref{Ezek}{22}{7}{Ex 21:17 Le 20:9 De 27:16 Pr 20:20; 30:11,\allowbreak17 Mt 15:4-\allowbreak6 Mr 7:10}
\crossref{Ezek}{22}{8}{22:26; 20:13,\allowbreak21,\allowbreak24; 23:38,\allowbreak39 Le 19:30 Am 8:4-\allowbreak6 Mal 1:6-\allowbreak8,\allowbreak12}
\crossref{Ezek}{22}{9}{Ex 20:16; 23:1 Le 19:16 1Ki 21:10-\allowbreak13 Ps 50:20; 101:5 Pr 10:18}
\crossref{Ezek}{22}{10}{Ge 35:22; 49:4 Le 18:7,\allowbreak8; 20:11 De 27:20,\allowbreak23 2Sa 16:21,\allowbreak22}
\crossref{Ezek}{22}{11}{Eze 18:11 Le 18:20; 20:10 De 22:22 Job 31:9-\allowbreak11 Jer 5:7,\allowbreak8; 9:2; 29:23}
\crossref{Ezek}{22}{12}{Ex 23:7,\allowbreak8 De 16:19; 27:25 Isa 1:23 Mic 7:2,\allowbreak3 Zep 3:3,\allowbreak4}
\crossref{Ezek}{22}{13}{Eze 21:14,\allowbreak17 Nu 24:10}
\crossref{Ezek}{22}{14}{Eze 21:7; 28:9 Job 40:9 Isa 31:3; 45:9 Jer 13:21 1Co 10:22}
\crossref{Ezek}{22}{15}{Eze 5:12; 12:14,\allowbreak15; 34:6; 36:19 Le 26:33 De 4:27; 28:25,\allowbreak64 Ne 1:8}
\crossref{Ezek}{22}{16}{Eze 6:7; 39:6,\allowbreak7,\allowbreak28 Ex 8:22 1Ki 20:13,\allowbreak28 Ps 9:16; 83:18 Isa 37:20}
\crossref{Ezek}{22}{17}{}
\crossref{Ezek}{22}{18}{Ps 119:119 Isa 1:22 Jer 6:28-\allowbreak30}
\crossref{Ezek}{22}{19}{}
\crossref{Ezek}{22}{20}{22:21; 21:31,\allowbreak32 Isa 54:16}
\crossref{Ezek}{22}{21}{Eze 15:6,\allowbreak7; 20:47,\allowbreak48; 22:20-\allowbreak22 De 4:24; 29:20; 32:22 2Ki 25:9 Ps 21:9}
\crossref{Ezek}{22}{22}{22:16,\allowbreak31; 20:8,\allowbreak33 Ho 5:10 Re 16:1}
\crossref{Ezek}{22}{23}{22:23}
\crossref{Ezek}{22}{24}{2Ch 28:22; 36:14-\allowbreak16 Isa 1:5; 9:13 Jer 2:30; 5:3; 6:29; 44:16-\allowbreak19}
\crossref{Ezek}{22}{25}{Eze 13:10-\allowbreak16 1Ki 22:11-\allowbreak13,\allowbreak23 Jer 5:30,\allowbreak31; 6:13 La 2:14; 4:13}
\crossref{Ezek}{22}{26}{1Sa 2:12-\allowbreak17,\allowbreak22 Jer 2:8,\allowbreak26,\allowbreak27 La 4:13 Mic 3:11,\allowbreak12 Zep 3:3,\allowbreak4}
\crossref{Ezek}{22}{27}{22:6; 19:3-\allowbreak6; 22:6; 45:9 Isa 1:23 Ho 7:1-\allowbreak7 Mic 3:2,\allowbreak3,\allowbreak9-\allowbreak11; 7:8}
\crossref{Ezek}{22}{28}{Eze 13:22,\allowbreak23; 21:29 Jer 23:25-\allowbreak32 La 2:14 Zep 3:4}
\crossref{Ezek}{22}{29}{22:7; 18:12 Isa 5:7; 10:2; 59:3-\allowbreak7 Jer 5:26-\allowbreak28,\allowbreak31; 6:13 Am 3:10}
\crossref{Ezek}{22}{30}{Eze 13:5 Ge 18:23-\allowbreak32 Ex 32:10-\allowbreak14 Ps 106:23 Jer 15:1}
\crossref{Ezek}{22}{31}{22:21,\allowbreak22}
\crossref{Ezek}{23}{1}{23:1}
\crossref{Ezek}{23}{2}{Eze 16:44,\allowbreak46 Jer 3:7-\allowbreak10}
\crossref{Ezek}{23}{3}{Eze 20:8 Le 17:7 De 29:16 Jos 24:14}
\crossref{Ezek}{23}{4}{Eze 16:40 1Ki 12:20}
\crossref{Ezek}{23}{5}{23:7,\allowbreak9,\allowbreak12,\allowbreak16,\allowbreak20; 16:37 Jer 50:38}
\crossref{Ezek}{23}{6}{23:12-\allowbreak15}
\crossref{Ezek}{23}{7}{23:30; 20:7; 22:3,\allowbreak4 Ps 106:39 Ho 5:3; 6:10}
\crossref{Ezek}{23}{8}{23:3,\allowbreak19,\allowbreak21 Ex 32:4 1Ki 12:28 2Ki 10:29; 17:16}
\crossref{Ezek}{23}{9}{2Ki 15:29; 17:3-\allowbreak6,\allowbreak23; 18:9-\allowbreak12 1Ch 5:26 Ho 11:5 Re 17:12,\allowbreak13,\allowbreak16}
\crossref{Ezek}{23}{10}{23:29; 16:37-\allowbreak41 Ho 2:3,\allowbreak10}
\crossref{Ezek}{23}{11}{23:4 Jer 3:8}
\crossref{Ezek}{23}{12}{23:5; 16:28 2Ki 16:7-\allowbreak15 2Ch 28:16-\allowbreak23}
\crossref{Ezek}{23}{13}{23:31 2Ki 17:18,\allowbreak19 Ho 12:1,\allowbreak2}
\crossref{Ezek}{23}{14}{Eze 8:10 Isa 46:1 Jer 50:2}
\crossref{Ezek}{23}{15}{1Sa 18:4 Isa 22:21}
\crossref{Ezek}{23}{16}{23:40,\allowbreak41; 16:17,\allowbreak29 2Pe 2:14}
\crossref{Ezek}{23}{17}{Ge 10:10; 11:9}
\crossref{Ezek}{23}{18}{Eze 16:36; 21:24 Isa 3:9 Jer 8:12 Ho 7:1}
\crossref{Ezek}{23}{19}{23:14; 16:25,\allowbreak29,\allowbreak51 Am 4:4}
\crossref{Ezek}{23}{20}{23:16}
\crossref{Ezek}{23}{21}{}
\crossref{Ezek}{23}{22}{23:9,\allowbreak28; 16:37 Isa 10:5,\allowbreak6; 39:3,\allowbreak4 Hab 1:6-\allowbreak10 Re 17:16}
\crossref{Ezek}{23}{23}{Eze 21:19-\allowbreak27 2Ki 20:14-\allowbreak17; 25:1-\allowbreak3}
\crossref{Ezek}{23}{24}{Eze 26:10 Jer 47:3 Na 2:3,\allowbreak4; 3:2,\allowbreak3}
\crossref{Ezek}{23}{25}{Eze 5:13; 8:1-\allowbreak18; 16:38-\allowbreak42 Ex 34:14 De 29:20; 32:21,\allowbreak22 Pr 6:34}
\crossref{Ezek}{23}{26}{23:29; 16:16,\allowbreak37,\allowbreak39 Jer 13:22 Ho 2:3,\allowbreak9,\allowbreak10 Re 17:16; 18:14-\allowbreak17}
\crossref{Ezek}{23}{27}{Eze 16:41; 22:15 Isa 27:9 Mic 5:10-\allowbreak14 Zec 13:2}
\crossref{Ezek}{23}{28}{23:17,\allowbreak22; 16:37 Jer 21:7-\allowbreak10; 24:8; 34:20}
\crossref{Ezek}{23}{29}{23:25,\allowbreak26,\allowbreak45-\allowbreak47; 16:39 De 28:47-\allowbreak51 2Sa 13:15}
\crossref{Ezek}{23}{30}{23:12-\allowbreak21; 6:9 Ps 106:35-\allowbreak38 Jer 2:18-\allowbreak20; 16:11,\allowbreak12; 22:8,\allowbreak9}
\crossref{Ezek}{23}{31}{23:13; 16:47-\allowbreak51 Jer 3:8-\allowbreak11}
\crossref{Ezek}{23}{32}{Ps 60:3 Isa 51:17 Jer 25:15-\allowbreak28; 48:26 Mt 20:22,\allowbreak23 Re 16:19}
\crossref{Ezek}{23}{33}{Jer 25:27 Hab 2:16}
\crossref{Ezek}{23}{34}{Ps 75:8 Isa 51:17}
\crossref{Ezek}{23}{35}{Eze 22:12 Isa 17:10 Jer 2:32; 3:21; 13:25; 23:27; 32:33 Ho 8:14; 13:6}
\crossref{Ezek}{23}{36}{Eze 20:4; 22:2 Jer 1:10 1Co 6:2,\allowbreak3}
\crossref{Ezek}{23}{37}{23:5; 16:32 Ho 1:2; 3:1}
\crossref{Ezek}{23}{38}{Eze 7:20; 8:5-\allowbreak16 2Ki 21:4,\allowbreak7; 23:11,\allowbreak12}
\crossref{Ezek}{23}{39}{Isa 3:9 Jer 7:8-\allowbreak11; 11:15 Mic 3:11 Joh 18:28}
\crossref{Ezek}{23}{40}{23:13 Isa 57:9}
\crossref{Ezek}{23}{41}{Es 1:6 Pr 7:16,\allowbreak17 Isa 57:7 Am 2:8; 6:4}
\crossref{Ezek}{23}{42}{Job 1:15 Joe 3:8}
\crossref{Ezek}{23}{43}{Ezr 9:7 Ps 106:6 Jer 13:23 Da 9:16}
\crossref{Ezek}{23}{44}{23:3,\allowbreak9-\allowbreak13}
\crossref{Ezek}{23}{45}{23:37-\allowbreak39; 16:38-\allowbreak43 Le 20:10; 21:9 De 22:21-\allowbreak24 Joh 8:7}
\crossref{Ezek}{23}{46}{23:22-\allowbreak26; 16:40 Jer 25:9}
\crossref{Ezek}{23}{47}{23:25,\allowbreak29; 9:6; 16:41 Jer 33:4,\allowbreak5}
\crossref{Ezek}{23}{48}{23:27; 6:6; 22:15; 36:25 Mic 5:11-\allowbreak14 Zep 1:3}
\crossref{Ezek}{23}{49}{Eze 7:4,\allowbreak9; 9:10; 11:21; 16:43; 22:31 Isa 59:18}
\crossref{Ezek}{24}{1}{}
\crossref{Ezek}{24}{2}{Isa 8:1; 30:8,\allowbreak9 Hab 2:2,\allowbreak3}
\crossref{Ezek}{24}{3}{Eze 17:2; 19:2-\allowbreak14; 20:49 Ps 78:2 Mic 2:4 Mr 12:12 Lu 8:10}
\crossref{Ezek}{24}{4}{Eze 22:18-\allowbreak22 Mic 3:2,\allowbreak3 Mt 7:2}
\crossref{Ezek}{24}{5}{Eze 20:47; 34:16,\allowbreak17,\allowbreak20 Jer 39:6; 52:10,\allowbreak24-\allowbreak27 Re 19:20}
\crossref{Ezek}{24}{6}{24:9; 11:6,\allowbreak7; 22:2,\allowbreak6-\allowbreak9,\allowbreak12,\allowbreak27; 23:37-\allowbreak45 2Ki 21:16; 24:4 Mic 7:2}
\crossref{Ezek}{24}{7}{1Ki 21:19 Isa 3:9 Jer 2:34; 6:15}
\crossref{Ezek}{24}{8}{Eze 5:13; 8:17,\allowbreak18; 22:30,\allowbreak31 De 32:21,\allowbreak22 2Ki 22:17 2Ch 34:25}
\crossref{Ezek}{24}{9}{24:6 Na 3:1 Hab 2:12 Lu 13:34,\allowbreak35 Re 14:20; 16:6,\allowbreak19}
\crossref{Ezek}{24}{10}{Jer 17:3; 20:5 La 1:10; 2:16}
\crossref{Ezek}{24}{11}{Jer 21:10; 32:29; 37:10; 38:18; 39:8; 52:13}
\crossref{Ezek}{24}{12}{Isa 47:13; 57:9,\allowbreak10 Jer 2:13; 9:5; 10:14,\allowbreak15; 51:58 Ho 12:1}
\crossref{Ezek}{24}{13}{24:11; 23:36-\allowbreak48 2Co 7:1}
\crossref{Ezek}{24}{14}{Nu 23:19 1Sa 15:29 Ps 33:9 Isa 55:11 Jer 23:20 Mt 24:35}
\crossref{Ezek}{24}{15}{}
\crossref{Ezek}{24}{16}{24:18,\allowbreak21,\allowbreak25 Pr 5:19 So 7:10}
\crossref{Ezek}{24}{17}{Ps 37:7}
\crossref{Ezek}{24}{18}{1Co 7:29,\allowbreak30}
\crossref{Ezek}{24}{19}{Eze 12:9; 17:12; 20:49; 21:7; 37:18 Mal 3:7,\allowbreak8,\allowbreak13}
\crossref{Ezek}{24}{20}{}
\crossref{Ezek}{24}{21}{Eze 7:20-\allowbreak22; 9:7 Ps 74:7; 79:1 Isa 65:11 Jer 7:14 La 1:10; 2:6,\allowbreak7}
\crossref{Ezek}{24}{22}{24:16,\allowbreak17 Job 27:15 Ps 78:64 Jer 16:4-\allowbreak7; 47:3 Am 6:9,\allowbreak10}
\crossref{Ezek}{24}{23}{Eze 4:17; 33:10 Le 26:39}
\crossref{Ezek}{24}{24}{Eze 4:3; 12:6,\allowbreak11 Isa 8:18; 20:3 Ho 1:2-\allowbreak9; 3:1-\allowbreak4 Lu 11:29,\allowbreak30}
\crossref{Ezek}{24}{25}{24:21 Ps 48:2; 50:2; 122:1-\allowbreak9 Jer 7:4}
\crossref{Ezek}{24}{26}{Eze 33:21,\allowbreak22 1Sa 4:12-\allowbreak18 Job 1:15-\allowbreak19}
\crossref{Ezek}{24}{27}{Eze 3:26,\allowbreak27; 29:21; 33:22 Ex 6:11,\allowbreak12 Ps 51:15 Lu 21:15 Eph 6:19}
\crossref{Ezek}{25}{1}{25:1}
\crossref{Ezek}{25}{2}{Eze 6:2; 20:46; 21:2; 35:2}
\crossref{Ezek}{25}{3}{25:6,\allowbreak8; 26:2-\allowbreak21; 35:10-\allowbreak15; 36:2 Ps 70:2,\allowbreak3 Pr 17:5; 24:17,\allowbreak18}
\crossref{Ezek}{25}{4}{Jud 6:3,\allowbreak33; 7:12; 8:10 1Ki 4:30}
\crossref{Ezek}{25}{5}{Eze 21:20}
\crossref{Ezek}{25}{6}{Job 27:23; 34:37 Jer 48:27 La 2:15 Na 3:19 Zep 2:15}
\crossref{Ezek}{25}{7}{25:13,\allowbreak16; 14:9; 35:3 Zep 1:4}
\crossref{Ezek}{25}{8}{Nu 24:17 Ps 83:4-\allowbreak8 Isa 15:1-\allowbreak16:14; 25:10 Jer 25:21; 48:1-\allowbreak47}
\crossref{Ezek}{25}{9}{Jos 13:20}
\crossref{Ezek}{25}{10}{25:4}
\crossref{Ezek}{25}{11}{25:17; 5:8,\allowbreak10,\allowbreak15; 11:9; 16:41; 30:14,\allowbreak19; 39:21 Ps 9:16; 149:7 Jude 1:15}
\crossref{Ezek}{25}{12}{25:8; 35:1-\allowbreak15 2Ch 28:17,\allowbreak18 Ps 137:7 Jer 49:7-\allowbreak22}
\crossref{Ezek}{25}{13}{25:7,\allowbreak16 Isa 34:1-\allowbreak17; 63:1-\allowbreak6 La 4:21,\allowbreak22 Mal 1:3,\allowbreak4}
\crossref{Ezek}{25}{14}{De 32:35,\allowbreak36 Ps 58:10,\allowbreak11 Na 1:2-\allowbreak4 Heb 10:30,\allowbreak31 Re 6:16,\allowbreak17}
\crossref{Ezek}{25}{15}{25:6,\allowbreak12 Isa 14:29-\allowbreak31 Jer 25:20; 47:1-\allowbreak7 Joe 3:4-\allowbreak21 Zep 2:4-\allowbreak7}
\crossref{Ezek}{25}{16}{1Sa 30:14 2Sa 15:18 Zep 2:4,\allowbreak5-\allowbreak15}
\crossref{Ezek}{25}{17}{25:11; 5:15}
\crossref{Ezek}{26}{1}{Eze 1:2; 8:1; 20:1 Jer 39:2}
\crossref{Ezek}{26}{2}{Eze 27:1-\allowbreak28:26 Jos 19:29 Ps 83:7 Isa 23:1-\allowbreak18 Jer 25:22; 27:3; 47:4}
\crossref{Ezek}{26}{3}{Eze 5:8; 21:3; 28:22; 38:3 Jer 21:13; 50:31 Na 2:12}
\crossref{Ezek}{26}{4}{26:9 Isa 23:11 Jer 5:10 Am 1:10 Zec 9:3}
\crossref{Ezek}{26}{5}{26:14,\allowbreak19; 27:32; 47:10}
\crossref{Ezek}{26}{6}{26:8; 16:46,\allowbreak48 Jer 49:2}
\crossref{Ezek}{26}{7}{26:3; 28:7; 29:18-\allowbreak20; 30:10,\allowbreak11; 32:11,\allowbreak12 Jer 25:9,\allowbreak22; 27:3-\allowbreak6}
\crossref{Ezek}{26}{8}{Eze 21:22 2Sa 20:15 Jer 52:4}
\crossref{Ezek}{26}{9}{2Ch 26:15}
\crossref{Ezek}{26}{10}{26:7 Jer 47:3}
\crossref{Ezek}{26}{11}{Isa 5:28 Jer 51:27 Hab 1:8}
\crossref{Ezek}{26}{12}{26:5 Mt 6:19,\allowbreak20}
\crossref{Ezek}{26}{13}{Eze 28:13 Isa 14:11; 22:2; 23:7,\allowbreak16; 24:8,\allowbreak9 Jer 7:34; 16:9; 25:10}
\crossref{Ezek}{26}{14}{26:4,\allowbreak5,\allowbreak12}
\crossref{Ezek}{26}{15}{26:18; 27:28,\allowbreak35; 31:16; 32:10 Isa 2:19 Jer 49:21 Heb 12:26,\allowbreak27}
\crossref{Ezek}{26}{16}{Eze 27:29-\allowbreak36; 32:21-\allowbreak32 Isa 14:9-\allowbreak13; 23:1-\allowbreak8 Re 18:11-\allowbreak19}
\crossref{Ezek}{26}{17}{Eze 19:1,\allowbreak14; 27:2,\allowbreak32; 28:12-\allowbreak19; 32:2,\allowbreak16 Jer 6:26; 7:29; 9:20 Mic 2:4}
\crossref{Ezek}{26}{18}{26:15; 27:28-\allowbreak30}
\crossref{Ezek}{26}{19}{26:3 Isa 8:7,\allowbreak8 Da 9:26; 11:40 Re 17:15}
\crossref{Ezek}{26}{20}{Eze 32:18-\allowbreak32; 34:1-\allowbreak31 Nu 16:30,\allowbreak33 Ps 28:1 Isa 14:11-\allowbreak19 Lu 10:15}
\crossref{Ezek}{26}{21}{26:15,\allowbreak16; 27:36; 28:19}
\crossref{Ezek}{27}{1}{27:1}
\crossref{Ezek}{27}{2}{27:32; 19:1; 26:17; 28:12; 32:2 Jer 7:20; 9:10,\allowbreak17-\allowbreak20 Am 5:1,\allowbreak16}
\crossref{Ezek}{27}{3}{27:12-\allowbreak36 Isa 23:3,\allowbreak8,\allowbreak11 Re 18:3,\allowbreak11-\allowbreak15}
\crossref{Ezek}{27}{4}{Eze 26:5}
\crossref{Ezek}{27}{5}{De 3:9 So 4:8}
\crossref{Ezek}{27}{6}{Isa 2:13 Zec 11:2}
\crossref{Ezek}{27}{7}{1Ki 10:28 Pr 7:16 Isa 19:9}
\crossref{Ezek}{27}{8}{Ge 10:15}
\crossref{Ezek}{27}{9}{Jos 13:5 1Ki 5:18}
\crossref{Ezek}{27}{10}{Eze 38:5 Da 5:28}
\crossref{Ezek}{27}{11}{27:8}
\crossref{Ezek}{27}{12}{Eze 38:13 Ge 10:4 1Ki 10:22; 22:48 2Ch 20:36,\allowbreak37 Ps 72:10 Isa 2:16}
\crossref{Ezek}{27}{13}{Ge 10:2,\allowbreak4 1Ch 1:5,\allowbreak7 Isa 66:19 Da 8:21; 10:20; 11:2}
\crossref{Ezek}{27}{14}{Eze 38:6 Ge 10:3 1Ch 1:6}
\crossref{Ezek}{27}{15}{27:20 Ge 10:7; 25:3 1Ch 1:9,\allowbreak32 Jer 25:23; 49:8}
\crossref{Ezek}{27}{16}{Ge 10:22}
\crossref{Ezek}{27}{17}{De 8:8; 32:14 1Ki 5:9 2Ch 2:10 Ezr 3:7 Ac 12:20}
\crossref{Ezek}{27}{18}{Ge 15:2 1Ki 11:24,\allowbreak25 Isa 7:8 Ac 9:2}
\crossref{Ezek}{27}{19}{Jud 18:29}
\crossref{Ezek}{27}{20}{27:15 Ge 25:3}
\crossref{Ezek}{27}{21}{1Ki 10:15 Jer 25:24 Ac 2:11 Ga 4:25}
\crossref{Ezek}{27}{22}{Ge 10:7 1Ki 10:1-\allowbreak13 1Ch 1:9 2Ch 9:1-\allowbreak12 Ps 72:10,\allowbreak15 Isa 60:6}
\crossref{Ezek}{27}{23}{Ge 11:31,\allowbreak32; 12:4 2Ki 19:12 Isa 37:12 Ac 7:4}
\crossref{Ezek}{27}{24}{}
\crossref{Ezek}{27}{25}{1Ki 10:22 Ps 48:7 Isa 2:16; 23:14; 60:9}
\crossref{Ezek}{27}{26}{Isa 33:23}
\crossref{Ezek}{27}{27}{Eze 26:14,\allowbreak21}
\crossref{Ezek}{27}{28}{27:35; 26:10,\allowbreak15-\allowbreak18; 31:16 Ex 15:14 Na 2:3}
\crossref{Ezek}{27}{29}{Re 18:17-\allowbreak24}
\crossref{Ezek}{27}{30}{1Sa 4:12 2Sa 1:2 Job 2:12 La 2:10 Re 18:19}
\crossref{Ezek}{27}{31}{Eze 7:18 Le 21:5 De 14:1 Isa 15:2; 22:12 Jer 16:6; 47:5; 48:37}
\crossref{Ezek}{27}{32}{27:2; 26:17}
\crossref{Ezek}{27}{33}{27:3,\allowbreak12-\allowbreak36 Isa 23:3-\allowbreak8 Re 18:3,\allowbreak12-\allowbreak15,\allowbreak19}
\crossref{Ezek}{27}{34}{27:26,\allowbreak27; 26:12-\allowbreak15,\allowbreak19-\allowbreak21 Zec 9:3,\allowbreak4}
\crossref{Ezek}{27}{35}{Eze 26:15-\allowbreak18 Isa 23:6}
\crossref{Ezek}{27}{36}{Eze 26:2 1Ki 9:8 Jer 18:16; 19:8 La 2:15 Zep 2:15}
\crossref{Ezek}{28}{1}{28:1}
\crossref{Ezek}{28}{2}{28:5,\allowbreak17; 31:10 De 8:14 2Ch 26:16 Pr 16:18; 18:12 Isa 2:12 Da 5:22}
\crossref{Ezek}{28}{3}{Da 1:20; 2:48; 5:11,\allowbreak12 Zec 9:2,\allowbreak3}
\crossref{Ezek}{28}{4}{Eze 29:3 De 8:17,\allowbreak18 Pr 18:11; 23:4,\allowbreak5 Ec 9:11 Hab 1:16 Zec 9:2-\allowbreak4}
\crossref{Ezek}{28}{5}{Pr 26:12 Isa 5:21 Ro 12:16}
\crossref{Ezek}{28}{6}{28:2 Ex 9:17 Job 9:4; 40:9-\allowbreak12 1Co 10:22 2Th 2:4 Jas 1:11}
\crossref{Ezek}{28}{7}{Eze 26:7-\allowbreak14 Isa 23:8,\allowbreak9 Am 3:6}
\crossref{Ezek}{28}{8}{Eze 32:18-\allowbreak30 Job 17:16; 33:18,\allowbreak28 Ps 28:1; 30:9; 55:15; 88:4,\allowbreak5 Pr 1:12}
\crossref{Ezek}{28}{9}{28:2 Da 4:31,\allowbreak32; 5:23-\allowbreak30 Ac 12:22,\allowbreak23}
\crossref{Ezek}{28}{10}{Eze 31:18; 32:19,\allowbreak21,\allowbreak24-\allowbreak30; 44:7,\allowbreak9 Le 26:41 1Sa 17:26,\allowbreak36 Jer 6:10}
\crossref{Ezek}{28}{11}{}
\crossref{Ezek}{28}{12}{28:2; 19:1,\allowbreak14; 26:17; 27:2,\allowbreak32; 32:2,\allowbreak16 2Ch 35:25 Isa 14:4}
\crossref{Ezek}{28}{13}{Eze 31:8,\allowbreak9; 36:35 Ge 2:8; 3:23,\allowbreak24; 13:10 Isa 51:3 Joe 2:3 Re 2:7}
\crossref{Ezek}{28}{14}{28:16 Ex 25:17-\allowbreak20; 30:26; 40:9}
\crossref{Ezek}{28}{15}{28:3-\allowbreak6,\allowbreak12; 27:3,\allowbreak4}
\crossref{Ezek}{28}{16}{Eze 27:12-\allowbreak36 Isa 23:17,\allowbreak18 Ho 12:7 Lu 19:45,\allowbreak46 Joh 2:16 1Ti 6:9,\allowbreak10}
\crossref{Ezek}{28}{17}{28:2,\allowbreak5; 16:14,\allowbreak15; 31:10 Pr 11:2; 16:18 Lu 14:11 Jas 4:6}
\crossref{Ezek}{28}{18}{28:2,\allowbreak13,\allowbreak14,\allowbreak16}
\crossref{Ezek}{28}{19}{Eze 27:35,\allowbreak36 Ps 76:12 Isa 14:16-\allowbreak19 Re 18:9,\allowbreak10,\allowbreak15-\allowbreak19}
\crossref{Ezek}{28}{20}{}
\crossref{Ezek}{28}{21}{Eze 6:2; 25:2; 29:2}
\crossref{Ezek}{28}{22}{Eze 5:8; 21:3; 26:3; 29:3,\allowbreak10; 38:3; 39:1-\allowbreak3 Jer 21:13; 50:31 Na 1:6; 2:13}
\crossref{Ezek}{28}{23}{Eze 5:12; 38:22 Jer 15:2}
\crossref{Ezek}{28}{24}{Nu 33:55 Jos 23:13 Jud 2:3 Isa 35:9; 55:13 Jer 12:14 Mic 7:4}
\crossref{Ezek}{28}{25}{Eze 11:17; 20:41; 34:13; 36:24; 37:21; 39:27 Le 26:44,\allowbreak45 De 30:3,\allowbreak4}
\crossref{Ezek}{28}{26}{Eze 34:25-\allowbreak28; 38:8 Le 25:18,\allowbreak19 De 12:10 Jer 23:6-\allowbreak8; 33:16}
\crossref{Ezek}{29}{1}{29:17; 1:2; 8:1; 20:1; 26:1; 40:1}
\crossref{Ezek}{29}{2}{Eze 6:2; 20:46; 21:2; 25:2; 28:21,\allowbreak22}
\crossref{Ezek}{29}{3}{29:10; 28:22 Ps 76:7 Jer 44:30 Na 1:6}
\crossref{Ezek}{29}{4}{Eze 38:4 2Ki 19:28 Job 41:1,\allowbreak2 Isa 37:29 Am 4:2}
\crossref{Ezek}{29}{5}{Eze 31:18; 32:4-\allowbreak6; 39:4-\allowbreak6,\allowbreak11-\allowbreak20 Ps 110:5,\allowbreak6 Jer 8:2; 16:4; 25:33}
\crossref{Ezek}{29}{6}{Eze 28:22-\allowbreak24,\allowbreak26 Ex 9:14; 14:18}
\crossref{Ezek}{29}{7}{Eze 17:15-\allowbreak17 Jer 37:5-\allowbreak11}
\crossref{Ezek}{29}{8}{29:19,\allowbreak20; 14:17; 30:4,\allowbreak10 Jer 46:13-\allowbreak26}
\crossref{Ezek}{29}{9}{29:10-\allowbreak12; 30:7,\allowbreak13-\allowbreak17 Jer 43:10-\allowbreak13}
\crossref{Ezek}{29}{10}{29:11; 30:12 Hab 3:8}
\crossref{Ezek}{29}{11}{Eze 30:10-\allowbreak13; 31:12; 32:13; 33:28; 36:28 Jer 43:11,\allowbreak12}
\crossref{Ezek}{29}{12}{Eze 30:7 Jer 25:15-\allowbreak19; 27:6-\allowbreak11}
\crossref{Ezek}{29}{13}{Isa 19:22 Jer 46:26}
\crossref{Ezek}{29}{14}{Eze 30:14 Ge 10:14 1Ch 1:12}
\crossref{Ezek}{29}{15}{Eze 17:6,\allowbreak14; 30:13 Zec 10:11}
\crossref{Ezek}{29}{16}{29:6,\allowbreak7; 17:15-\allowbreak17 Isa 20:5; 30:1-\allowbreak6; 31:1-\allowbreak3; 36:4-\allowbreak6 Jer 2:18,\allowbreak19,\allowbreak36,\allowbreak37}
\crossref{Ezek}{29}{17}{29:1; 1:2}
\crossref{Ezek}{29}{18}{Eze 26:7-\allowbreak12 Jer 25:9; 27:6}
\crossref{Ezek}{29}{19}{29:8-\allowbreak10; 30:10-\allowbreak12 Jer 43:10-\allowbreak13}
\crossref{Ezek}{29}{20}{2Ki 10:30 Isa 10:6,\allowbreak7; 45:1-\allowbreak3 Jer 25:9}
\crossref{Ezek}{29}{21}{Eze 28:25,\allowbreak26 1Sa 2:10 Ps 92:10; 112:9; 132:17; 148:14 Isa 27:6}
\crossref{Ezek}{30}{1}{30:1}
\crossref{Ezek}{30}{2}{Eze 21:12 Isa 13:6; 14:31; 15:2; 16:7; 23:1,\allowbreak6; 65:14 Jer 4:8; 47:2}
\crossref{Ezek}{30}{3}{Eze 7:7,\allowbreak12 Ps 37:13 Ob 1:15 Joe 2:1 Zep 1:7,\allowbreak14 Mt 24:33 Php 4:5}
\crossref{Ezek}{30}{4}{Eze 29:8 Isa 19:2 Jer 50:35-\allowbreak37}
\crossref{Ezek}{30}{5}{Isa 18:1; 20:4 Jer 46:9 Na 3:8,\allowbreak9}
\crossref{Ezek}{30}{6}{Job 9:13 Isa 20:3-\allowbreak6; 31:3 Na 3:9}
\crossref{Ezek}{30}{7}{Eze 29:12; 32:18-\allowbreak32 Jer 25:18-\allowbreak26; 46:1-\allowbreak51:64}
\crossref{Ezek}{30}{8}{Eze 28:24,\allowbreak26; 29:6,\allowbreak9,\allowbreak16 Ps 58:11}
\crossref{Ezek}{30}{9}{30:5,\allowbreak6 Isa 18:1,\allowbreak2; 20:3,\allowbreak5 Zep 2:12}
\crossref{Ezek}{30}{10}{Eze 29:4,\allowbreak5,\allowbreak19; 32:11-\allowbreak16}
\crossref{Ezek}{30}{11}{Eze 28:7; 31:12; 32:12 De 28:50 Isa 14:4-\allowbreak6 Jer 51:20-\allowbreak23 Hab 1:6-\allowbreak9}
\crossref{Ezek}{30}{12}{Eze 29:3 Isa 19:4-\allowbreak10; 44:27 Jer 50:38; 51:36 Na 1:4 Re 16:12}
\crossref{Ezek}{30}{13}{Ex 12:12 Isa 19:1-\allowbreak15 Jer 43:12,\allowbreak13; 46:25 Zep 2:11 Zec 13:2}
\crossref{Ezek}{30}{14}{Eze 29:14}
\crossref{Ezek}{30}{15}{Ps 11:6 Na 1:6 Re 16:1}
\crossref{Ezek}{30}{16}{30:8,\allowbreak9; 28:18}
\crossref{Ezek}{30}{17}{Ge 41:45}
\crossref{Ezek}{30}{18}{Jer 2:16}
\crossref{Ezek}{30}{19}{30:14; 5:8,\allowbreak15; 25:11; 39:21 Nu 33:4 Ps 9:16; 149:7 Ro 2:5 Re 17:1}
\crossref{Ezek}{30}{20}{Eze 1:2; 26:1; 29:1,\allowbreak17}
\crossref{Ezek}{30}{21}{30:24 Ps 10:15; 37:17 Jer 48:25}
\crossref{Ezek}{30}{22}{Eze 29:3 Jer 46:25}
\crossref{Ezek}{30}{23}{30:17,\allowbreak18,\allowbreak26; 29:12,\allowbreak13}
\crossref{Ezek}{30}{24}{30:25 Ne 6:9 Ps 18:32,\allowbreak39; 144:1 Isa 45:1,\allowbreak5 Jer 27:6-\allowbreak8}
\crossref{Ezek}{30}{25}{30:19,\allowbreak26; 29:16,\allowbreak21; 32:15; 38:16,\allowbreak23; 39:21,\allowbreak22 Ps 9:16}
\crossref{Ezek}{30}{26}{30:17,\allowbreak18,\allowbreak23; 6:13; 29:12 Da 11:42}
\crossref{Ezek}{31}{1}{}
\crossref{Ezek}{31}{2}{Jer 1:5,\allowbreak17 Re 10:11}
\crossref{Ezek}{31}{3}{Na 3:1-\allowbreak19 Zep 2:13}
\crossref{Ezek}{31}{4}{Eze 17:5,\allowbreak8 Pr 14:28 Jer 51:36 Re 17:1,\allowbreak15}
\crossref{Ezek}{31}{5}{}
\crossref{Ezek}{31}{6}{Eze 17:23 Da 4:12,\allowbreak21 Mt 13:32}
\crossref{Ezek}{31}{7}{31:9,\allowbreak12; 17:6 Da 4:14}
\crossref{Ezek}{31}{8}{Eze 28:13 Ge 2:8; 13:10 Ps 80:10 Isa 51:3}
\crossref{Ezek}{31}{9}{Eze 16:14 Ex 9:16 Ps 75:6,\allowbreak7 Da 2:21,\allowbreak37,\allowbreak38; 4:22-\allowbreak25; 5:20-\allowbreak23}
\crossref{Ezek}{31}{10}{Mt 23:12}
\crossref{Ezek}{31}{11}{Eze 11:9; 21:31; 23:28 Jud 16:23 1Ti 1:20}
\crossref{Ezek}{31}{12}{Eze 28:7; 30:11 Hab 1:6,\allowbreak11}
\crossref{Ezek}{31}{13}{Eze 29:5; 32:4 Isa 18:6 Re 19:17,\allowbreak18}
\crossref{Ezek}{31}{14}{De 13:11; 21:21 Ne 13:18 Da 4:32; 5:22,\allowbreak23 1Co 10:11 2Pe 2:6}
\crossref{Ezek}{31}{15}{Mal 3:4}
\crossref{Ezek}{31}{16}{Eze 26:10,\allowbreak15; 27:28 Na 2:3 Hag 2:7 Heb 12:26,\allowbreak27 Re 11:13; 18:9-\allowbreak24}
\crossref{Ezek}{31}{17}{Eze 32:20-\allowbreak30 Ps 9:17 Isa 14:9}
\crossref{Ezek}{31}{18}{31:2; 32:19}
\crossref{Ezek}{32}{1}{}
\crossref{Ezek}{32}{2}{32:16,\allowbreak18; 19:1; 27:2,\allowbreak32; 28:12 Jer 9:18}
\crossref{Ezek}{32}{3}{Eze 12:13; 17:20 Ec 9:12 Jer 16:16 La 1:13 Ho 7:12 Hab 1:14-\allowbreak17}
\crossref{Ezek}{32}{4}{Eze 29:5; 31:12,\allowbreak13; 39:4,\allowbreak5,\allowbreak17-\allowbreak20 1Sa 17:44-\allowbreak46 Ps 63:10; 74:14}
\crossref{Ezek}{32}{5}{Ge 2:24}
\crossref{Ezek}{32}{6}{Ex 7:17 Isa 34:3,\allowbreak7 Re 14:20; 16:6}
\crossref{Ezek}{32}{7}{Job 18:5,\allowbreak6 Pr 13:9}
\crossref{Ezek}{32}{8}{Ge 1:14}
\crossref{Ezek}{32}{9}{Re 11:18; 18:10-\allowbreak15}
\crossref{Ezek}{32}{10}{Eze 27:35 De 29:24 1Ki 9:8}
\crossref{Ezek}{32}{11}{Eze 26:7; 30:4,\allowbreak22-\allowbreak25 Jer 43:10; 46:13,\allowbreak24-\allowbreak26}
\crossref{Ezek}{32}{12}{Eze 28:7; 30:11; 31:11 De 28:49,\allowbreak50 Hab 1:6,\allowbreak7}
\crossref{Ezek}{32}{13}{Eze 29:8; 30:12}
\crossref{Ezek}{32}{14}{}
\crossref{Ezek}{32}{15}{Eze 6:7; 30:26 Ex 7:5; 14:4,\allowbreak18 Ps 9:16; 83:17,\allowbreak18}
\crossref{Ezek}{32}{16}{32:2; 26:17 2Sa 1:17; 3:33,\allowbreak34 2Ch 35:25 Jer 9:17}
\crossref{Ezek}{32}{17}{32:1; 1:2}
\crossref{Ezek}{32}{18}{32:2,\allowbreak16; 21:6,\allowbreak7 Isa 16:9 Mic 1:8 Lu 19:41 Ro 12:15}
\crossref{Ezek}{32}{19}{Eze 27:3,\allowbreak4; 28:12-\allowbreak17; 31:2,\allowbreak18}
\crossref{Ezek}{32}{20}{32:23-\allowbreak26,\allowbreak29,\allowbreak30; 29:8-\allowbreak12}
\crossref{Ezek}{32}{21}{32:19,\allowbreak24,\allowbreak25 Nu 16:30-\allowbreak34 Ps 9:17; 55:15 Pr 14:32}
\crossref{Ezek}{32}{22}{32:24,\allowbreak26,\allowbreak29,\allowbreak30; 31:3-\allowbreak18 Nu 24:24 Ps 83:8-\allowbreak10}
\crossref{Ezek}{32}{23}{32:24-\allowbreak27,\allowbreak32; 26:17,\allowbreak20 Isa 14:16; 51:12,\allowbreak13}
\crossref{Ezek}{32}{24}{Ge 10:22; 14:1 1Ch 1:17 Jer 25:25; 49:34-\allowbreak39 Da 8:2}
\crossref{Ezek}{32}{25}{Ps 139:8 Re 2:22}
\crossref{Ezek}{32}{26}{}
\crossref{Ezek}{32}{27}{32:21 Job 3:13-\allowbreak15 Isa 14:18,\allowbreak19}
\crossref{Ezek}{32}{28}{Da 2:34,\allowbreak35}
\crossref{Ezek}{32}{29}{Eze 25:1-\allowbreak17; 35:1-\allowbreak15 Ge 25:30; 36:1-\allowbreak19 Isa 34:1-\allowbreak17; 63:1-\allowbreak6}
\crossref{Ezek}{32}{30}{32:24,\allowbreak25}
\crossref{Ezek}{32}{31}{}
\crossref{Ezek}{32}{32}{32:27 Ge 35:5 Job 31:23 Jer 25:15-\allowbreak38 Zep 3:6-\allowbreak8 2Co 5:11}
\crossref{Ezek}{33}{1}{33:1}
\crossref{Ezek}{33}{2}{33:17,\allowbreak30; 3:11,\allowbreak27}
\crossref{Ezek}{33}{3}{33:8,\allowbreak9 Ne 4:18,\allowbreak20 Isa 58:1 Jer 4:5; 6:1; 51:27 Ho 8:1 Joe 2:1}
\crossref{Ezek}{33}{4}{2Ch 25:16 Pr 29:1 Jer 6:17; 42:20-\allowbreak22 Zec 1:2-\allowbreak4 Jas 1:22}
\crossref{Ezek}{33}{5}{Ps 95:7 Heb 2:1-\allowbreak3}
\crossref{Ezek}{33}{6}{Isa 56:10,\allowbreak11}
\crossref{Ezek}{33}{7}{Eze 3:17-\allowbreak21 So 3:3; 5:7 Isa 62:6 Jer 6:27; 31:6 Mic 7:4 Eph 4:11}
\crossref{Ezek}{33}{8}{33:14; 18:4,\allowbreak10-\allowbreak13,\allowbreak18,\allowbreak20 Ge 2:17; 3:4 Pr 11:21 Ec 8:13 Isa 3:11}
\crossref{Ezek}{33}{9}{Eze 3:19,\allowbreak21 Ac 13:40; 18:5,\allowbreak6; 28:23-\allowbreak28 Ga 5:19-\allowbreak21; 6:7,\allowbreak8 Eph 5:3-\allowbreak6}
\crossref{Ezek}{33}{10}{Eze 37:11 Ps 130:7 Isa 49:14; 51:20 Jer 2:25}
\crossref{Ezek}{33}{11}{Eze 5:11; 14:16-\allowbreak18; 16:48 Nu 14:21,\allowbreak28 Isa 49:18 Jer 22:24; 46:18}
\crossref{Ezek}{33}{12}{33:2}
\crossref{Ezek}{33}{13}{Eze 3:20; 18:24 Lu 18:9-\allowbreak14 Ro 10:3 Php 3:9 Heb 10:38 2Pe 2:20-\allowbreak22}
\crossref{Ezek}{33}{14}{33:8; 3:18,\allowbreak19; 18:27 Isa 3:11 Jer 18:7,\allowbreak8 Lu 13:3-\allowbreak5}
\crossref{Ezek}{33}{15}{Ex 22:1-\allowbreak4 Le 6:2-\allowbreak5 Nu 5:6-\allowbreak8 Lu 19:8}
\crossref{Ezek}{33}{16}{Eze 18:22 Isa 1:18; 43:25; 44:22 Mic 7:18,\allowbreak19 Ro 5:16,\allowbreak21 1Jo 2:1-\allowbreak3}
\crossref{Ezek}{33}{17}{33:20; 18:25,\allowbreak29 Job 35:2; 40:8 Mt 25:24-\allowbreak26 Lu 19:21,\allowbreak22}
\crossref{Ezek}{33}{18}{33:12,\allowbreak13; 18:26,\allowbreak27 2Pe 2:20-\allowbreak22 Heb 10:38}
\crossref{Ezek}{33}{19}{33:14; 18:27,\allowbreak28}
\crossref{Ezek}{33}{20}{33:17; 18:25,\allowbreak29 Pr 19:3}
\crossref{Ezek}{33}{21}{Eze 24:26,\allowbreak27}
\crossref{Ezek}{33}{22}{Eze 1:3; 3:22; 37:1; 40:1}
\crossref{Ezek}{33}{23}{}
\crossref{Ezek}{33}{24}{33:27; 36:4}
\crossref{Ezek}{33}{25}{Ge 9:4 Le 3:17; 7:26,\allowbreak27; 17:10-\allowbreak14; 19:26 De 12:16 1Sa 14:32-\allowbreak34}
\crossref{Ezek}{33}{26}{Ge 27:40 Mic 2:1,\allowbreak2 Zep 3:3}
\crossref{Ezek}{33}{27}{33:24; 5:12-\allowbreak17; 6:11-\allowbreak14 Jer 15:2-\allowbreak4; 42:22; 44:12}
\crossref{Ezek}{33}{28}{Eze 6:14; 12:20; 15:8; 36:34,\allowbreak35 2Ch 36:21 Isa 6:11 Jer 9:11; 16:16}
\crossref{Ezek}{33}{29}{Eze 6:7; 7:27; 23:49; 25:11 Ex 14:18 Ps 9:16; 83:17,\allowbreak18}
\crossref{Ezek}{33}{30}{Jer 11:18,\allowbreak19; 18:18}
\crossref{Ezek}{33}{31}{De 5:28,\allowbreak29 Ps 78:36,\allowbreak37 Isa 29:13 Jas 2:14-\allowbreak16 1Jo 3:17,\allowbreak18}
\crossref{Ezek}{33}{32}{Mr 4:16,\allowbreak17; 6:20 Joh 5:35}
\crossref{Ezek}{33}{33}{1Sa 3:19,\allowbreak20 Jer 28:9}
\crossref{Ezek}{34}{1}{34:1}
\crossref{Ezek}{34}{2}{34:8-\allowbreak10; 13:19 Jer 23:1 Mic 3:1-\allowbreak3,\allowbreak11,\allowbreak12 Zep 3:3,\allowbreak4 Zec 11:17}
\crossref{Ezek}{34}{3}{Isa 56:11,\allowbreak12 Zec 11:5,\allowbreak16}
\crossref{Ezek}{34}{4}{34:16 Isa 56:10 Jer 8:22 Zec 11:15,\allowbreak16 Mt 9:36 Heb 12:12}
\crossref{Ezek}{34}{5}{34:6; 33:21,\allowbreak28 1Ki 22:17 2Ch 18:16 Jer 23:2; 50:6,\allowbreak17 Zec 13:7}
\crossref{Ezek}{34}{6}{Eze 7:16 Jer 13:16; 40:11,\allowbreak12 Heb 11:37,\allowbreak38 1Pe 2:25}
\crossref{Ezek}{34}{7}{34:9 Ps 82:1-\allowbreak7 Isa 1:10 Jer 13:13,\allowbreak18; 22:2,\allowbreak3 Mic 3:8,\allowbreak9 Mal 2:1}
\crossref{Ezek}{34}{8}{34:5,\allowbreak6,\allowbreak31}
\crossref{Ezek}{34}{9}{34:7}
\crossref{Ezek}{34}{10}{Eze 5:8; 13:8; 21:3; 35:3 Jer 21:13; 50:31 Na 2:13 Zec 10:3 1Pe 3:12}
\crossref{Ezek}{34}{11}{Eze 5:8; 6:3 Ge 6:17 Le 26:28 De 32:39 Isa 45:12; 48:15; 51:12}
\crossref{Ezek}{34}{12}{Eze 30:3 Isa 50:10 Jer 13:16 Joe 2:1-\allowbreak3 Am 5:18-\allowbreak20 Zep 1:15}
\crossref{Ezek}{34}{13}{Eze 11:17; 20:41; 28:25,\allowbreak26; 36:24; 37:21,\allowbreak22; 38:8; 39:27 Ps 106:47}
\crossref{Ezek}{34}{14}{34:27 Ps 23:1,\allowbreak2; 31:8-\allowbreak10 Isa 25:6; 30:23,\allowbreak24; 40:11 Jer 31:12-\allowbreak14,\allowbreak25}
\crossref{Ezek}{34}{15}{Ps 23:2 So 1:7,\allowbreak8 Isa 11:6,\allowbreak7; 27:10; 65:9,\allowbreak10 Jer 3:15 Ho 2:18}
\crossref{Ezek}{34}{16}{34:4,\allowbreak11 Isa 40:11; 61:1-\allowbreak3 Mic 4:6,\allowbreak7 Mt 15:24; 18:11-\allowbreak14 Mr 2:17}
\crossref{Ezek}{34}{17}{34:20-\allowbreak22; 20:37,\allowbreak38 Zec 10:3 Mt 25:32,\allowbreak33}
\crossref{Ezek}{34}{18}{Eze 16:20,\allowbreak47 Ge 30:15 Nu 16:9,\allowbreak13 2Sa 7:19 Isa 7:13}
\crossref{Ezek}{34}{19}{}
\crossref{Ezek}{34}{20}{34:10,\allowbreak17 Ps 22:12-\allowbreak16 Mt 25:31-\allowbreak46}
\crossref{Ezek}{34}{21}{34:3-\allowbreak5 Da 8:3-\allowbreak10 Zec 11:5,\allowbreak16,\allowbreak17}
\crossref{Ezek}{34}{22}{34:10 Ps 72:12-\allowbreak14 Jer 23:2,\allowbreak3 Zec 11:7-\allowbreak9}
\crossref{Ezek}{34}{23}{Ec 12:11 Isa 40:11 Jer 23:4-\allowbreak6 Mic 5:2-\allowbreak5 Zec 13:7 Joh 10:11}
\crossref{Ezek}{34}{24}{34:30,\allowbreak31; 36:28; 37:23,\allowbreak27; 39:22 Ex 29:45,\allowbreak46 Isa 43:2,\allowbreak3 Jer 31:1,\allowbreak33}
\crossref{Ezek}{34}{25}{Eze 37:26 Isa 55:3 Jer 31:31-\allowbreak33 Zec 6:13 Heb 13:20}
\crossref{Ezek}{34}{26}{Ge 12:2 Isa 19:24 Zec 8:13,\allowbreak23}
\crossref{Ezek}{34}{27}{Eze 47:12 Le 26:4 Ps 85:12; 92:12-\allowbreak14 Isa 4:2; 35:1,\allowbreak2; 61:3}
\crossref{Ezek}{34}{28}{34:25,\allowbreak29 Jer 30:10; 46:27}
\crossref{Ezek}{34}{29}{Isa 4:2; 11:1-\allowbreak6; 53:2 Jer 23:5; 33:15 Zec 3:8; 6:12}
\crossref{Ezek}{34}{30}{34:24; 16:62; 37:27 Ps 46:7,\allowbreak11 Isa 8:9,\allowbreak10 Mt 1:23; 28:20}
\crossref{Ezek}{34}{31}{Eze 36:38 Ps 78:52; 80:1; 95:7; 100:3 Isa 40:11 Mic 7:14 Lu 12:32}
\crossref{Ezek}{35}{1}{Eze 21:1; 22:1; 34:1 2Pe 1:21}
\crossref{Ezek}{35}{2}{Eze 6:2; 20:46; 21:2; 25:2 Isa 50:7 Eph 6:19}
\crossref{Ezek}{35}{3}{Eze 5:8; 21:3; 29:3,\allowbreak10 Jer 21:13 Na 2:13; 3:5}
\crossref{Ezek}{35}{4}{35:9; 6:6 Joe 3:19 Mal 1:3,\allowbreak4}
\crossref{Ezek}{35}{5}{35:12; 25:12 Ge 27:41,\allowbreak42 Ps 137:7 Am 1:11 Ob 1:10-\allowbreak16}
\crossref{Ezek}{35}{6}{Ps 109:16 Isa 63:2-\allowbreak6 Ob 1:15 Mt 7:2 Re 16:5-\allowbreak7; 18:6,\allowbreak24; 19:2,\allowbreak3}
\crossref{Ezek}{35}{7}{35:3,\allowbreak9; 33:28}
\crossref{Ezek}{35}{8}{Eze 31:12; 32:4,\allowbreak5; 39:4,\allowbreak5 Isa 34:2-\allowbreak7}
\crossref{Ezek}{35}{9}{35:4; 25:13 Jer 49:17,\allowbreak18 Zep 2:9 Mal 1:3,\allowbreak4}
\crossref{Ezek}{35}{10}{Eze 36:5 Ps 83:4-\allowbreak12 Jer 49:1 Ob 1:13}
\crossref{Ezek}{35}{11}{Ps 137:7 Am 1:11 Mt 7:2 Jas 2:13}
\crossref{Ezek}{35}{12}{35:9; 6:7}
\crossref{Ezek}{35}{13}{1Sa 2:3 2Ch 32:15,\allowbreak19 Isa 10:13-\allowbreak19; 36:20; 37:10,\allowbreak23,\allowbreak29 Da 11:36}
\crossref{Ezek}{35}{14}{Isa 14:7,\allowbreak8; 65:13-\allowbreak15}
\crossref{Ezek}{35}{15}{Eze 36:2-\allowbreak5 Ps 137:7 Pr 17:5 La 4:21 Ob 1:12,\allowbreak15}
\crossref{Ezek}{36}{1}{Eze 6:2,\allowbreak3; 33:28; 34:14; 37:22}
\crossref{Ezek}{36}{2}{36:5; 25:3; 26:2}
\crossref{Ezek}{36}{3}{Eze 13:10 Le 26:43}
\crossref{Ezek}{36}{4}{36:1,\allowbreak6 De 11:11}
\crossref{Ezek}{36}{5}{Eze 38:19 De 4:24 Isa 66:15,\allowbreak16 Zep 3:8 Zec 1:15}
\crossref{Ezek}{36}{6}{36:4,\allowbreak5,\allowbreak15; 34:29 Ps 74:10,\allowbreak18,\allowbreak23; 123:3,\allowbreak4}
\crossref{Ezek}{36}{7}{Eze 20:5,\allowbreak15 De 32:40 Re 10:5,\allowbreak6}
\crossref{Ezek}{36}{8}{Eze 34:26-\allowbreak29 Ps 67:6; 85:12 Isa 4:2; 27:6; 30:23 Ho 2:21-\allowbreak23}
\crossref{Ezek}{36}{9}{Ps 46:11; 99:8 Ho 2:21-\allowbreak23 Joe 3:18 Hag 2:19 Zec 8:12}
\crossref{Ezek}{36}{10}{36:37 Isa 27:6; 41:17-\allowbreak23 Jer 30:19; 31:27,\allowbreak28; 33:12 Zec 8:3-\allowbreak6}
\crossref{Ezek}{36}{11}{Jer 31:27; 33:12}
\crossref{Ezek}{36}{12}{Jer 32:15,\allowbreak44 Ob 1:17-\allowbreak21}
\crossref{Ezek}{36}{13}{Nu 13:32 2Ki 17:25,\allowbreak26}
\crossref{Ezek}{36}{14}{Eze 37:25-\allowbreak28 Isa 60:21 Am 9:15}
\crossref{Ezek}{36}{15}{36:6; 34:29 Isa 54:4; 60:14 Mic 7:8-\allowbreak10 Zep 3:19,\allowbreak20}
\crossref{Ezek}{36}{16}{}
\crossref{Ezek}{36}{17}{Le 18:24-\allowbreak28 Nu 35:33,\allowbreak34 Ps 106:37,\allowbreak38 Isa 24:5 Jer 2:7; 3:1,\allowbreak2,\allowbreak9}
\crossref{Ezek}{36}{18}{Eze 7:8; 14:19; 21:31 2Ch 34:21,\allowbreak26 Isa 42:25 Jer 7:20; 44:6 La 2:4}
\crossref{Ezek}{36}{19}{Eze 5:12; 22:15 Le 26:38 De 28:64 Am 9:9}
\crossref{Ezek}{36}{20}{Ex 32:11-\allowbreak13 Nu 14:15,\allowbreak16 Jos 7:9 2Ki 18:30,\allowbreak35; 19:10-\allowbreak12}
\crossref{Ezek}{36}{21}{Eze 20:9,\allowbreak14,\allowbreak22 De 32:26,\allowbreak27 Ps 74:18 Isa 37:35; 48:9}
\crossref{Ezek}{36}{22}{36:32 De 7:7,\allowbreak8; 9:5-\allowbreak7 Ps 106:8; 115:1,\allowbreak2}
\crossref{Ezek}{36}{23}{Eze 20:41; 38:22,\allowbreak23 Nu 20:12,\allowbreak13 Ps 46:10 Isa 5:16 1Pe 3:15}
\crossref{Ezek}{36}{24}{Eze 11:17; 34:13; 37:21,\allowbreak25; 39:27,\allowbreak28 De 30:3-\allowbreak5 Ps 107:2,\allowbreak3}
\crossref{Ezek}{36}{25}{Le 14:5-\allowbreak7 Nu 8:7; 19:13-\allowbreak20 Ps 51:7 Isa 52:15 Joh 3:5 Tit 3:5,\allowbreak6}
\crossref{Ezek}{36}{26}{De 30:6 Ps 51:10 Jer 32:39 Joh 3:3-\allowbreak5 2Co 3:18; 5:17 Ga 6:15}
\crossref{Ezek}{36}{27}{Eze 37:14; 39:29 Pr 1:23 Isa 44:3,\allowbreak4; 59:21 Joe 2:28,\allowbreak29 Zec 12:10}
\crossref{Ezek}{36}{28}{36:10; 28:25; 37:25; 39:28}
\crossref{Ezek}{36}{29}{36:25 Jer 33:8 Ho 14:2,\allowbreak4,\allowbreak8 Joe 3:21 Mic 7:19 Zec 13:1 Mt 1:21}
\crossref{Ezek}{36}{30}{De 29:23-\allowbreak28 Joe 2:17,\allowbreak26}
\crossref{Ezek}{36}{31}{Eze 6:9; 16:61-\allowbreak63; 20:43 Le 26:39 Ezr 9:6-\allowbreak15 Ne 9:26-\allowbreak35}
\crossref{Ezek}{36}{32}{36:22 De 9:5 Da 9:18,\allowbreak19 2Ti 1:9 Tit 3:3-\allowbreak6}
\crossref{Ezek}{36}{33}{Zec 8:7,\allowbreak8}
\crossref{Ezek}{36}{34}{Eze 6:14 De 29:23-\allowbreak28 2Ch 36:21 Jer 25:9-\allowbreak11}
\crossref{Ezek}{36}{35}{Ps 58:11; 64:9; 126:2 Jer 33:9}
\crossref{Ezek}{36}{36}{Eze 17:24; 34:30; 37:28; 39:27-\allowbreak29 Mic 7:15-\allowbreak17}
\crossref{Ezek}{36}{37}{Eze 14:3; 20:3,\allowbreak31 Ps 10:17; 102:17 Isa 55:6,\allowbreak7 Jer 29:11-\allowbreak13; 50:4,\allowbreak5}
\crossref{Ezek}{36}{38}{Ex 23:17; 34:23 De 16:16 2Ch 7:8; 30:21-\allowbreak27; 35:7-\allowbreak19 Zec 8:19-\allowbreak23}
\crossref{Ezek}{37}{1}{Eze 8:3; 11:24 1Ki 18:12 2Ki 2:16 Lu 4:1 Ac 8:39}
\crossref{Ezek}{37}{2}{De 11:30}
\crossref{Ezek}{37}{3}{Joh 6:5,\allowbreak6}
\crossref{Ezek}{37}{4}{37:11,\allowbreak15,\allowbreak16 Nu 20:8 1Ki 13:2 Mt 21:21 Joh 2:5}
\crossref{Ezek}{37}{5}{37:9,\allowbreak10,\allowbreak14 Ge 2:7 Ps 104:29,\allowbreak30 Joh 20:22 Ro 8:2 Eph 2:5}
\crossref{Ezek}{37}{6}{37:8-\allowbreak10}
\crossref{Ezek}{37}{7}{Jer 13:5-\allowbreak7; 26:8 Ac 4:19; 5:20-\allowbreak29}
\crossref{Ezek}{37}{8}{Isa 66:14}
\crossref{Ezek}{37}{9}{37:5,\allowbreak14 So 4:16 Joh 3:8}
\crossref{Ezek}{37}{10}{Ps 104:30 Re 11:11; 20:4,\allowbreak5}
\crossref{Ezek}{37}{11}{37:16,\allowbreak19; 36:10; 39:25 Jer 31:1; 33:24-\allowbreak26 Ho 1:11 Ro 11:26}
\crossref{Ezek}{37}{12}{Job 35:14,\allowbreak15}
\crossref{Ezek}{37}{13}{37:6; 16:62 Ps 126:2,\allowbreak3}
\crossref{Ezek}{37}{14}{37:9; 11:19; 36:27; 39:29 Isa 32:15 Joe 2:28,\allowbreak29 Zec 12:10}
\crossref{Ezek}{37}{15}{}
\crossref{Ezek}{37}{16}{Nu 17:2,\allowbreak3}
\crossref{Ezek}{37}{17}{37:22-\allowbreak24 Isa 11:13 Jer 50:4 Ho 1:11 Zep 3:9}
\crossref{Ezek}{37}{18}{Eze 12:9; 17:12; 20:49; 24:19}
\crossref{Ezek}{37}{19}{37:16,\allowbreak17 1Ch 9:1-\allowbreak3 Zec 10:6 Eph 2:13,\allowbreak14 Col 3:11}
\crossref{Ezek}{37}{20}{Eze 12:3 Nu 17:6-\allowbreak9 Ho 12:10}
\crossref{Ezek}{37}{21}{Eze 34:13; 36:24; 39:25 De 30:3,\allowbreak4 Isa 11:11-\allowbreak16; 27:12,\allowbreak13; 43:6; 49:12}
\crossref{Ezek}{37}{22}{Isa 11:12,\allowbreak13 Jer 3:18; 32:39; 50:4 Ho 1:11 Eph 2:19-\allowbreak22}
\crossref{Ezek}{37}{23}{Eze 20:43; 36:25,\allowbreak29,\allowbreak31; 43:7,\allowbreak8 Isa 2:18 Ho 14:8 Zec 13:1,\allowbreak2; 14:21}
\crossref{Ezek}{37}{24}{37:25 Isa 55:3,\allowbreak4 Jer 23:5; 30:9 Ho 3:5 Lu 1:32}
\crossref{Ezek}{37}{25}{37:21; 28:25; 36:28; 37:26 Jer 30:3; 31:24; 32:41}
\crossref{Ezek}{37}{26}{Eze 34:25 Ge 17:7 2Sa 23:5 Ps 89:3,\allowbreak4 Isa 55:3; 59:20,\allowbreak21 Jer 32:40}
\crossref{Ezek}{37}{27}{Joh 1:14 Col 2:9,\allowbreak10 Re 21:3,\allowbreak22}
\crossref{Ezek}{37}{28}{Eze 36:23,\allowbreak36; 38:23; 39:7,\allowbreak23 Ps 79:10; 102:15; 126:2 Ro 11:15}
\crossref{Ezek}{38}{1}{38:1}
\crossref{Ezek}{38}{2}{Eze 2:1; 39:1}
\crossref{Ezek}{38}{3}{Eze 13:8; 29:3; 35:3; 39:1,\allowbreak2-\allowbreak10}
\crossref{Ezek}{38}{4}{Eze 29:4; 39:2 2Ki 19:28 Isa 37:29}
\crossref{Ezek}{38}{5}{Eze 27:10}
\crossref{Ezek}{38}{6}{Ge 10:2 1Ch 1:5}
\crossref{Ezek}{38}{7}{2Ch 25:8 Ps 2:1-\allowbreak4 Isa 8:9,\allowbreak10; 37:22 Jer 46:3-\allowbreak5,\allowbreak14-\allowbreak16; 51:12}
\crossref{Ezek}{38}{8}{38:16 Ge 49:1 Nu 24:14 De 4:30 Jer 48:47; 49:39 Ho 3:3-\allowbreak5 Hab 2:3}
\crossref{Ezek}{38}{9}{Eze 13:11 Isa 21:1,\allowbreak2; 25:4; 28:2 Da 11:40}
\crossref{Ezek}{38}{10}{Ps 83:3,\allowbreak4; 139:2 Pr 19:21 Isa 10:7 Mr 7:21 Joh 13:2 Ac 5:3,\allowbreak9}
\crossref{Ezek}{38}{11}{Ex 15:9 Ps 10:9 Pr 1:11-\allowbreak16 Isa 37:24,\allowbreak25 Ro 3:15}
\crossref{Ezek}{38}{12}{Isa 1:24,\allowbreak25 Am 1:8 Zec 13:7}
\crossref{Ezek}{38}{13}{Eze 27:12,\allowbreak15,\allowbreak20,\allowbreak22,\allowbreak23,\allowbreak25}
\crossref{Ezek}{38}{14}{Isa 4:1,\allowbreak2}
\crossref{Ezek}{38}{15}{Eze 39:2 Da 11:40}
\crossref{Ezek}{38}{16}{38:9}
\crossref{Ezek}{38}{17}{38:10,\allowbreak11,\allowbreak16 Ps 110:5,\allowbreak6 Isa 27:1; 34:1-\allowbreak6; 63:1-\allowbreak6; 66:15,\allowbreak16}
\crossref{Ezek}{38}{18}{Eze 36:5,\allowbreak6 De 32:22 Ps 18:7,\allowbreak8; 89:46 Na 1:2 Heb 12:29}
\crossref{Ezek}{38}{19}{Eze 39:25 De 29:20 Isa 42:13 Joe 2:18 Zec 1:14}
\crossref{Ezek}{38}{20}{Jer 4:23-\allowbreak26 Ho 4:3 Na 1:4-\allowbreak6 Zec 14:4,\allowbreak5 Re 6:12,\allowbreak13}
\crossref{Ezek}{38}{21}{Eze 14:17 Ps 105:16}
\crossref{Ezek}{38}{22}{Isa 66:16 Jer 25:31 Zec 14:12-\allowbreak15}
\crossref{Ezek}{38}{23}{Eze 36:23}
\crossref{Ezek}{39}{1}{Eze 38:2,\allowbreak3}
\crossref{Ezek}{39}{2}{}
\crossref{Ezek}{39}{3}{Eze 20:21-\allowbreak24 Ps 46:9; 76:3 Jer 21:4,\allowbreak5 Ho 1:5}
\crossref{Ezek}{39}{4}{39:17-\allowbreak20; 38:21}
\crossref{Ezek}{39}{5}{Eze 29:5; 32:4 Jer 8:2; 22:19}
\crossref{Ezek}{39}{6}{Eze 38:11 Jud 18:7}
\crossref{Ezek}{39}{7}{39:22; 38:16,\allowbreak23}
\crossref{Ezek}{39}{8}{Eze 38:17 2Pe 3:8}
\crossref{Ezek}{39}{9}{Ps 111:2,\allowbreak3 Isa 66:24 Mal 1:5}
\crossref{Ezek}{39}{10}{Ex 3:22; 12:36 Isa 14:2; 33:1 Mic 5:8 Hab 3:8 Zep 2:9,\allowbreak10 Mt 7:2}
\crossref{Ezek}{39}{11}{Eze 47:18 Nu 34:11 Lu 5:1 Joh 6:1}
\crossref{Ezek}{39}{12}{39:14,\allowbreak16 Nu 19:16 De 21:23}
\crossref{Ezek}{39}{13}{De 26:19 Ps 149:6-\allowbreak9 Jer 33:9 Zep 3:19,\allowbreak20 1Pe 1:7}
\crossref{Ezek}{39}{14}{Nu 19:11-\allowbreak19}
\crossref{Ezek}{39}{15}{Lu 11:44}
\crossref{Ezek}{39}{16}{39:12}
\crossref{Ezek}{39}{17}{Ge 31:54 1Sa 9:13; 16:3 Isa 56:9 Jer 12:9 Zep 1:7 Re 19:17,\allowbreak18}
\crossref{Ezek}{39}{18}{Eze 29:5; 34:8 Re 19:17,\allowbreak18,\allowbreak21}
\crossref{Ezek}{39}{19}{Isa 23:18}
\crossref{Ezek}{39}{20}{Eze 38:4 Ps 76:5,\allowbreak6 Hag 2:22 Re 19:18}
\crossref{Ezek}{39}{21}{Eze 36:23; 38:16,\allowbreak23 Ex 9:16; 14:4 Isa 26:11; 37:20 Mal 1:11}
\crossref{Ezek}{39}{22}{39:7,\allowbreak28; 28:26; 34:30 Ps 9:16 Jer 24:7; 31:34 Joh 17:3 1Jo 5:20}
\crossref{Ezek}{39}{23}{Eze 36:18-\allowbreak23,\allowbreak36 2Ch 7:21,\allowbreak22 Jer 22:8,\allowbreak9; 40:2,\allowbreak3 La 1:8; 2:15-\allowbreak17}
\crossref{Ezek}{39}{24}{Eze 36:19 Le 26:24 2Ki 17:7-\allowbreak23 Isa 1:20; 3:11; 59:17,\allowbreak18 Jer 2:17,\allowbreak19}
\crossref{Ezek}{39}{25}{Eze 20:40; 37:21,\allowbreak22 Jer 31:1 Ho 1:11}
\crossref{Ezek}{39}{26}{Eze 16:52,\allowbreak57,\allowbreak58,\allowbreak63; 32:25,\allowbreak30 Ps 99:8 Jer 3:24,\allowbreak25; 30:11 Da 9:16}
\crossref{Ezek}{39}{27}{39:25; 28:25,\allowbreak26}
\crossref{Ezek}{39}{28}{39:22; 34:30 Ho 2:20}
\crossref{Ezek}{39}{29}{39:23-\allowbreak25; 37:26,\allowbreak27 Isa 45:17; 54:8-\allowbreak10}
\crossref{Ezek}{40}{1}{Eze 33:21 2Ki 25:1-\allowbreak30 Jer 39:1-\allowbreak18; 52:1-\allowbreak34}
\crossref{Ezek}{40}{2}{Eze 17:22,\allowbreak23 Isa 2:2,\allowbreak3 Da 2:34,\allowbreak35 Mic 4:1 Re 21:10}
\crossref{Ezek}{40}{3}{Eze 1:7,\allowbreak27 Da 10:5,\allowbreak6 Re 1:15}
\crossref{Ezek}{40}{4}{Eze 2:7,\allowbreak8; 3:17; 43:10; 44:5 Mt 10:27; 13:9,\allowbreak51,\allowbreak52}
\crossref{Ezek}{40}{5}{Eze 42:20 Ps 125:2 Isa 26:1; 60:18 Zec 2:5 Re 21:12}
\crossref{Ezek}{40}{6}{40:20; 8:16; 11:1; 43:1; 44:1; 46:1,\allowbreak12 1Ch 9:18,\allowbreak24 Ne 3:29 Jer 19:2}
\crossref{Ezek}{40}{7}{Eze 42:5 1Ki 6:5-\allowbreak10 1Ch 9:26; 23:28 2Ch 3:9; 31:11 Ezr 8:29}
\crossref{Ezek}{40}{8}{}
\crossref{Ezek}{40}{9}{Eze 45:19}
\crossref{Ezek}{40}{10}{40:7}
\crossref{Ezek}{40}{11}{}
\crossref{Ezek}{40}{12}{40:12}
\crossref{Ezek}{40}{13}{}
\crossref{Ezek}{40}{14}{Eze 8:7; 42:1 Ex 27:9; 35:17 Le 6:16 1Ch 28:6 Ps 100:4 Isa 62:9}
\crossref{Ezek}{40}{15}{}
\crossref{Ezek}{40}{16}{Eze 41:16 1Ki 6:4 1Co 13:12}
\crossref{Ezek}{40}{17}{Eze 10:5; 42:1; 46:21 Re 11:2}
\crossref{Ezek}{40}{18}{}
\crossref{Ezek}{40}{19}{40:23,\allowbreak27; 46:1,\allowbreak2}
\crossref{Ezek}{40}{20}{40:6}
\crossref{Ezek}{40}{21}{40:7,\allowbreak10-\allowbreak16,\allowbreak29,\allowbreak30,\allowbreak36,\allowbreak37}
\crossref{Ezek}{40}{22}{40:16,\allowbreak31,\allowbreak37 1Ki 6:29,\allowbreak32,\allowbreak35; 7:36 2Ch 3:5 Re 7:9}
\crossref{Ezek}{40}{23}{Ex 27:9-\allowbreak18; 38:9-\allowbreak12}
\crossref{Ezek}{40}{24}{40:6,\allowbreak20,\allowbreak35; 46:9}
\crossref{Ezek}{40}{25}{40:16,\allowbreak22,\allowbreak29 Joh 12:46 1Co 13:12 2Pe 1:19}
\crossref{Ezek}{40}{26}{40:6,\allowbreak22,\allowbreak29 2Pe 3:18}
\crossref{Ezek}{40}{27}{40:23,\allowbreak32}
\crossref{Ezek}{40}{28}{40:32,\allowbreak35}
\crossref{Ezek}{40}{29}{40:16,\allowbreak22,\allowbreak25}
\crossref{Ezek}{40}{30}{40:21,\allowbreak25,\allowbreak29,\allowbreak33,\allowbreak36}
\crossref{Ezek}{40}{31}{40:26,\allowbreak34}
\crossref{Ezek}{40}{32}{40:28-\allowbreak31,\allowbreak35}
\crossref{Ezek}{40}{33}{40:21,\allowbreak25,\allowbreak36}
\crossref{Ezek}{40}{34}{40:6,\allowbreak22,\allowbreak26,\allowbreak31,\allowbreak34,\allowbreak37,\allowbreak49}
\crossref{Ezek}{40}{35}{40:27,\allowbreak32; 44:4; 47:2}
\crossref{Ezek}{40}{36}{40:21,\allowbreak29,\allowbreak36}
\crossref{Ezek}{40}{37}{40:31,\allowbreak34}
\crossref{Ezek}{40}{38}{40:12; 41:10,\allowbreak11 1Ki 6:8}
\crossref{Ezek}{40}{39}{Eze 41:22; 44:16 Mal 1:7,\allowbreak12 Lu 22:30 1Co 10:16-\allowbreak21}
\crossref{Ezek}{40}{40}{40:35}
\crossref{Ezek}{40}{41}{}
\crossref{Ezek}{40}{42}{Ex 20:25 La 3:9 Am 5:11}
\crossref{Ezek}{40}{43}{Le 1:6,\allowbreak8; 8:20}
\crossref{Ezek}{40}{44}{40:23,\allowbreak27}
\crossref{Ezek}{40}{45}{Eze 8:5}
\crossref{Ezek}{40}{46}{Eze 44:15 Le 6:12,\allowbreak13 Nu 18:5}
\crossref{Ezek}{40}{47}{40:19,\allowbreak23,\allowbreak27}
\crossref{Ezek}{40}{48}{}
\crossref{Ezek}{40}{49}{1Ki 7:15-\allowbreak21 2Ch 3:17 Jer 52:17-\allowbreak23 Re 3:12}
\crossref{Ezek}{41}{1}{Eze 40:2,\allowbreak3,\allowbreak17}
\crossref{Ezek}{41}{2}{1Ki 6:2,\allowbreak17 2Ch 3:3}
\crossref{Ezek}{41}{3}{}
\crossref{Ezek}{41}{4}{1Ki 6:20 2Ch 3:8 Re 21:16}
\crossref{Ezek}{41}{5}{41:6,\allowbreak7; 42:3-\allowbreak14 1Ki 6:5,\allowbreak6}
\crossref{Ezek}{41}{6}{1Ki 6:6,\allowbreak10}
\crossref{Ezek}{41}{7}{1Ki 6:8 Mt 13:32 Heb 6:1}
\crossref{Ezek}{41}{8}{Eze 40:5 Re 21:16}
\crossref{Ezek}{41}{9}{41:5}
\crossref{Ezek}{41}{10}{}
\crossref{Ezek}{41}{11}{41:9; 42:4}
\crossref{Ezek}{41}{12}{41:13-\allowbreak15; 42:1,\allowbreak10,\allowbreak13 Re 21:27; 22:14,\allowbreak15}
\crossref{Ezek}{41}{13}{}
\crossref{Ezek}{41}{14}{}
\crossref{Ezek}{41}{15}{Eze 42:3 So 1:17; 7:5 Zec 3:7}
\crossref{Ezek}{41}{16}{41:26; 40:16,\allowbreak25 1Ki 6:4 1Co 13:12}
\crossref{Ezek}{41}{17}{Eze 42:15}
\crossref{Ezek}{41}{18}{1Ki 6:29-\allowbreak32; 7:36 2Ch 3:7}
\crossref{Ezek}{41}{19}{Eze 1:6,\allowbreak10; 10:14}
\crossref{Ezek}{41}{20}{2Ch 3:7}
\crossref{Ezek}{41}{21}{Eze 40:14 1Ki 6:33}
\crossref{Ezek}{41}{22}{Eze 23:41; 44:16 Ex 25:28-\allowbreak30 Le 24:6 Pr 9:2 So 1:12 Mal 1:7,\allowbreak12}
\crossref{Ezek}{41}{23}{1Ki 6:31-\allowbreak35 2Ch 4:22}
\crossref{Ezek}{41}{24}{Eze 40:48}
\crossref{Ezek}{41}{25}{41:16-\allowbreak20}
\crossref{Ezek}{41}{26}{41:16; 40:16}
\crossref{Ezek}{42}{1}{Eze 40:2,\allowbreak3,\allowbreak24; 41:1}
\crossref{Ezek}{42}{2}{}
\crossref{Ezek}{42}{3}{Eze 41:10}
\crossref{Ezek}{42}{4}{42:11}
\crossref{Ezek}{42}{5}{Eze 41:7}
\crossref{Ezek}{42}{6}{Eze 41:6 1Ki 6:8}
\crossref{Ezek}{42}{7}{}
\crossref{Ezek}{42}{8}{}
\crossref{Ezek}{42}{9}{Eze 46:19}
\crossref{Ezek}{42}{10}{}
\crossref{Ezek}{42}{11}{42:2-\allowbreak8}
\crossref{Ezek}{42}{12}{42:9}
\crossref{Ezek}{42}{13}{Ex 29:31 Le 6:14-\allowbreak16,\allowbreak26; 7:6; 10:13,\allowbreak14,\allowbreak17; 24:9 Nu 18:9}
\crossref{Ezek}{42}{14}{Eze 44:19 Ex 28:40-\allowbreak43; 29:4-\allowbreak9 Le 8:7,\allowbreak13,\allowbreak33-\allowbreak35 Lu 9:62}
\crossref{Ezek}{42}{15}{Eze 41:2-\allowbreak5,\allowbreak15}
\crossref{Ezek}{42}{16}{}
\crossref{Ezek}{42}{17}{42:17}
\crossref{Ezek}{42}{18}{42:18}
\crossref{Ezek}{42}{19}{}
\crossref{Ezek}{42}{20}{Eze 40:5 So 2:9 Isa 25:1; 26:1; 60:18 Mic 7:11 Zec 2:5}
\crossref{Ezek}{43}{1}{Eze 40:6; 42:15; 44:1; 46:1}
\crossref{Ezek}{43}{2}{Eze 1:28; 3:23; 9:3; 10:18,\allowbreak19 Isa 6:3 Joh 12:41}
\crossref{Ezek}{43}{3}{Eze 1:4-\allowbreak28; 8:4; 9:3; 10:1-\allowbreak22; 11:22,\allowbreak23}
\crossref{Ezek}{43}{4}{Eze 10:18,\allowbreak19; 44:2}
\crossref{Ezek}{43}{5}{Eze 3:12-\allowbreak14; 8:3; 11:24; 37:1; 40:2 1Ki 18:12 2Ki 2:16 Ac 8:39}
\crossref{Ezek}{43}{6}{Le 1:1 Isa 66:6 Re 16:1}
\crossref{Ezek}{43}{7}{Eze 1:26; 10:1 Ps 47:8; 99:1 Isa 6:1 Jer 3:17; 14:21; 17:12}
\crossref{Ezek}{43}{8}{Eze 5:11; 8:3-\allowbreak16; 23:39; 44:7 2Ki 16:14,\allowbreak15; 21:4-\allowbreak7; 23:11,\allowbreak12}
\crossref{Ezek}{43}{9}{43:7; 37:23}
\crossref{Ezek}{43}{10}{Eze 40:4 Ex 25:40 1Ch 28:11,\allowbreak19}
\crossref{Ezek}{43}{11}{Eze 40:1-\allowbreak42:20; 44:5,\allowbreak6 Heb 8:5}
\crossref{Ezek}{43}{12}{Eze 40:2; 42:20 Ps 93:5 Joe 3:17 Zec 14:20,\allowbreak21 Re 21:27}
\crossref{Ezek}{43}{13}{Ex 27:1-\allowbreak8 2Ch 4:1}
\crossref{Ezek}{43}{14}{}
\crossref{Ezek}{43}{15}{}
\crossref{Ezek}{43}{16}{Ex 27:1 2Ch 4:1 Ezr 3:3}
\crossref{Ezek}{43}{17}{Ex 25:25; 30:3 1Ki 18:32}
\crossref{Ezek}{43}{18}{Eze 45:18,\allowbreak19 Ex 40:29 Le 1:5-\allowbreak17; 8:18-\allowbreak21; 16:19 Heb 9:21-\allowbreak13; 10:4-\allowbreak12}
\crossref{Ezek}{43}{19}{Eze 40:46; 44:15; 48:11 1Sa 2:35,\allowbreak36 1Ki 2:27,\allowbreak35 Isa 61:6; 66:22}
\crossref{Ezek}{43}{20}{43:15 Ex 29:12,\allowbreak36 Le 4:25,\allowbreak30,\allowbreak34; 8:15; 9:9}
\crossref{Ezek}{43}{21}{Ex 29:14 Le 4:12; 8:17 Heb 13:11,\allowbreak12}
\crossref{Ezek}{43}{22}{43:25 Ex 29:15-\allowbreak18 Le 8:18-\allowbreak21 Isa 53:6,\allowbreak10 1Pe 1:19}
\crossref{Ezek}{43}{23}{}
\crossref{Ezek}{43}{24}{Le 2:13 Nu 18:19 2Ch 13:5 Mt 5:13 Mr 9:49,\allowbreak50 Col 4:6}
\crossref{Ezek}{43}{25}{Ex 29:35-\allowbreak37 Le 8:33}
\crossref{Ezek}{43}{26}{Le 8:34}
\crossref{Ezek}{43}{27}{Le 9:1}
\crossref{Ezek}{44}{1}{Eze 43:1,\allowbreak4; 46:1}
\crossref{Ezek}{44}{2}{Eze 43:2-\allowbreak4 Ex 24:10 Isa 6:1-\allowbreak5}
\crossref{Ezek}{44}{3}{Eze 46:2,\allowbreak8 2Ch 23:13; 34:31}
\crossref{Ezek}{44}{4}{Eze 40:20,\allowbreak40}
\crossref{Ezek}{44}{5}{Eze 40:4 Ex 9:21}
\crossref{Ezek}{44}{6}{Eze 2:5-\allowbreak8; 3:9,\allowbreak26,\allowbreak27}
\crossref{Ezek}{44}{7}{44:9; 7:20; 22:26; 43:7,\allowbreak8 Le 22:25 Ac 21:28}
\crossref{Ezek}{44}{8}{Le 22:2-\allowbreak33 Nu 18:3-\allowbreak5 Ac 7:53 1Ti 6:13 2Ti 4:1}
\crossref{Ezek}{44}{9}{44:7 Ps 50:16; 93:5 Joe 3:17 Zec 14:21 Mr 16:16 Joh 3:3-\allowbreak5}
\crossref{Ezek}{44}{10}{44:15; 22:26; 48:11 2Ki 23:8,\allowbreak9 2Ch 29:4,\allowbreak5 Ne 9:34 Jer 23:11}
\crossref{Ezek}{44}{11}{44:14; 40:45 1Ch 26:1-\allowbreak19}
\crossref{Ezek}{44}{12}{1Sa 2:29,\allowbreak30 2Ki 16:10-\allowbreak16 Isa 9:16 Ho 4:6; 5:1 Mal 2:8,\allowbreak9}
\crossref{Ezek}{44}{13}{Nu 18:3 2Ki 23:9}
\crossref{Ezek}{44}{14}{Nu 18:4 1Ch 23:28-\allowbreak32}
\crossref{Ezek}{44}{15}{Eze 40:46; 43:19; 48:11 1Sa 2:35 1Ki 2:35 1Ti 3:3-\allowbreak10 2Ti 2:2}
\crossref{Ezek}{44}{16}{Re 1:6}
\crossref{Ezek}{44}{17}{Ex 28:39,\allowbreak40,\allowbreak43; 39:27-\allowbreak29 Le 16:4 Re 4:4; 19:8}
\crossref{Ezek}{44}{18}{Ex 28:40,\allowbreak41; 39:28 1Co 11:4-\allowbreak10}
\crossref{Ezek}{44}{19}{Eze 42:13,\allowbreak14 Le 6:10,\allowbreak11}
\crossref{Ezek}{44}{20}{Le 21:5-\allowbreak24 De 14:1}
\crossref{Ezek}{44}{21}{Le 10:9 Lu 1:15 1Ti 3:8; 5:23 Tit 1:7,\allowbreak8}
\crossref{Ezek}{44}{22}{Le 21:7,\allowbreak13,\allowbreak14 1Ti 3:2,\allowbreak4,\allowbreak5,\allowbreak11,\allowbreak12 Tit 1:6}
\crossref{Ezek}{44}{23}{Eze 22:26 Le 10:10,\allowbreak11 De 33:10 Ho 4:6 Mic 3:9-\allowbreak11 Zep 3:4}
\crossref{Ezek}{44}{24}{1Ti 3:15}
\crossref{Ezek}{44}{25}{Le 21:1-\allowbreak6; 22:4 Mt 8:21,\allowbreak22 Lu 9:59,\allowbreak60 2Co 5:16 1Th 4:13-\allowbreak15}
\crossref{Ezek}{44}{26}{Nu 6:10-\allowbreak21; 19:11-\allowbreak13 Heb 9:13,\allowbreak14}
\crossref{Ezek}{44}{27}{44:17}
\crossref{Ezek}{44}{28}{Eze 45:4; 48:9-\allowbreak11 Nu 18:20 De 10:9; 18:1,\allowbreak2 Jos 13:14,\allowbreak33 1Pe 5:2-\allowbreak4}
\crossref{Ezek}{44}{29}{Le 2:3,\allowbreak10; 6:14-\allowbreak18,\allowbreak26,\allowbreak29; 7:6 Nu 18:9-\allowbreak11 1Co 9:13,\allowbreak14 Heb 13:10}
\crossref{Ezek}{44}{30}{Ex 13:2,\allowbreak12; 22:29; 23:19 Nu 3:13; 15:19-\allowbreak21; 18:12-\allowbreak18,\allowbreak27-\allowbreak30}
\crossref{Ezek}{44}{31}{Ex 22:31 Le 17:15; 22:8 De 14:21 Ro 14:20 1Co 8:13}
\crossref{Ezek}{45}{1}{Eze 47:21; 48:29 Nu 34:13 Jos 13:6; 14:2 Ps 16:5,\allowbreak6}
\crossref{Ezek}{45}{2}{Eze 42:16-\allowbreak20}
\crossref{Ezek}{45}{3}{Eze 48:10}
\crossref{Ezek}{45}{4}{45:1; 44:28; 48:11}
\crossref{Ezek}{45}{5}{Eze 48:10,\allowbreak13,\allowbreak20}
\crossref{Ezek}{45}{6}{Eze 48:15-\allowbreak18,\allowbreak30-\allowbreak35}
\crossref{Ezek}{45}{7}{Eze 34:24; 37:24; 46:16-\allowbreak18; 48:21 Ps 2:8,\allowbreak9 Isa 9:5,\allowbreak6 Lu 1:32,\allowbreak33}
\crossref{Ezek}{45}{8}{Jos 11:23}
\crossref{Ezek}{45}{9}{Eze 44:6 1Pe 4:3}
\crossref{Ezek}{45}{10}{Le 19:35,\allowbreak36 Pr 11:1; 16:11; 20:10; 21:3 Am 8:4-\allowbreak6 Mic 6:10,\allowbreak11}
\crossref{Ezek}{45}{11}{}
\crossref{Ezek}{45}{12}{Ex 30:13 Le 27:25 Nu 3:47}
\crossref{Ezek}{45}{13}{Eze 44:30}
\crossref{Ezek}{45}{14}{45:11}
\crossref{Ezek}{45}{15}{Pr 3:9,\allowbreak10 Mal 1:8,\allowbreak14}
\crossref{Ezek}{45}{16}{Ex 30:14,\allowbreak15}
\crossref{Ezek}{45}{17}{Le 23:1-\allowbreak44 Nu 28:1-\allowbreak29:40 Isa 66:23}
\crossref{Ezek}{45}{18}{Le 22:20 Heb 7:26; 9:14 1Pe 1:19}
\crossref{Ezek}{45}{19}{Eze 43:14,\allowbreak20 Le 16:18-\allowbreak20}
\crossref{Ezek}{45}{20}{Le 4:27-\allowbreak35 Ps 19:12 Ro 16:18,\allowbreak19 Heb 5:2}
\crossref{Ezek}{45}{21}{Ex 12:1-\allowbreak51 Le 23:5-\allowbreak8 Nu 9:2-\allowbreak14; 28:16-\allowbreak25 De 16:1-\allowbreak8 1Co 5:7,\allowbreak8}
\crossref{Ezek}{45}{22}{Mt 20:28; 26:26-\allowbreak28}
\crossref{Ezek}{45}{23}{Le 23:8}
\crossref{Ezek}{45}{24}{Eze 46:5-\allowbreak7 Nu 28:12-\allowbreak15}
\crossref{Ezek}{45}{25}{}
\crossref{Ezek}{46}{1}{}
\crossref{Ezek}{46}{2}{Joh 10:1-\allowbreak3}
\crossref{Ezek}{46}{3}{Lu 1:10 Joh 10:9 Heb 10:19-\allowbreak22}
\crossref{Ezek}{46}{4}{}
\crossref{Ezek}{46}{5}{46:7,\allowbreak11,\allowbreak12; 45:24 Nu 28:12}
\crossref{Ezek}{46}{6}{46:6}
\crossref{Ezek}{46}{7}{}
\crossref{Ezek}{46}{8}{46:2; 44:1-\allowbreak3 Col 1:18}
\crossref{Ezek}{46}{9}{Ex 23:14-\allowbreak17; 34:23 De 16:16 Ps 84:7 Mal 4:4}
\crossref{Ezek}{46}{10}{2Sa 6:14-\allowbreak19 1Ch 29:20,\allowbreak22 2Ch 6:2-\allowbreak4; 7:4,\allowbreak5; 20:27,\allowbreak28; 29:28,\allowbreak29}
\crossref{Ezek}{46}{11}{Le 23:1-\allowbreak44 Nu 15:1-\allowbreak41; 28:1-\allowbreak29:40 De 16:1-\allowbreak22}
\crossref{Ezek}{46}{12}{Le 1:3; 23:38 Nu 29:39 1Ki 3:4 1Ch 29:21 2Ch 5:6; 7:5-\allowbreak7; 29:31}
\crossref{Ezek}{46}{13}{Ex 12:5 Le 12:6}
\crossref{Ezek}{46}{14}{Nu 28:5}
\crossref{Ezek}{46}{15}{Heb 7:27; 9:26; 10:1-\allowbreak10}
\crossref{Ezek}{46}{16}{Ge 25:5,\allowbreak6 2Ch 21:3 Ps 37:18 Mt 25:34 Lu 10:42 Joh 8:35,\allowbreak36}
\crossref{Ezek}{46}{17}{}
\crossref{Ezek}{46}{18}{Eze 45:8 Ps 72:2-\allowbreak4; 78:72 Isa 11:3,\allowbreak4; 32:1,\allowbreak2 Jer 23:5,\allowbreak6}
\crossref{Ezek}{46}{19}{}
\crossref{Ezek}{46}{20}{Eze 44:29 1Sa 2:13-\allowbreak15 2Ch 35:13}
\crossref{Ezek}{46}{21}{}
\crossref{Ezek}{46}{22}{}
\crossref{Ezek}{46}{23}{46:23}
\crossref{Ezek}{46}{24}{46:20 Mt 24:45 Joh 21:15-\allowbreak17 1Pe 5:2}
\crossref{Ezek}{47}{1}{Eze 41:2,\allowbreak23-\allowbreak26}
\crossref{Ezek}{47}{2}{Eze 44:2,\allowbreak4}
\crossref{Ezek}{47}{3}{Eze 40:3 Zec 2:1 Re 11:1; 21:15}
\crossref{Ezek}{47}{4}{Ac 19:10-\allowbreak20 Ro 15:19 Col 1:6}
\crossref{Ezek}{47}{5}{Isa 11:9 Da 2:34,\allowbreak35 Hab 2:14 Mt 13:31,\allowbreak32 Re 7:9; 11:15; 20:2-\allowbreak4}
\crossref{Ezek}{47}{6}{Eze 8:17; 40:4; 44:5 Jer 1:11-\allowbreak13 Zec 4:2; 5:2 Mt 13:51}
\crossref{Ezek}{47}{7}{1Ki 9:26 2Ki 2:13}
\crossref{Ezek}{47}{8}{Isa 35:1,\allowbreak7; 41:17-\allowbreak19; 43:20; 44:3-\allowbreak5; 49:9 Jer 31:9}
\crossref{Ezek}{47}{9}{Joh 3:16; 11:26}
\crossref{Ezek}{47}{10}{2Ch 20:2}
\crossref{Ezek}{47}{11}{De 29:23 Jud 9:45 Ps 107:34 Jer 17:6 Mr 9:48,\allowbreak49}
\crossref{Ezek}{47}{12}{47:7 Ps 92:12 Isa 60:21; 61:3}
\crossref{Ezek}{47}{13}{Nu 34:2-\allowbreak12}
\crossref{Ezek}{47}{14}{Eze 20:5,\allowbreak6,\allowbreak28,\allowbreak42 Ge 12:7; 13:15; 15:7; 17:8; 26:3; 28:13 Nu 14:16,\allowbreak30}
\crossref{Ezek}{47}{15}{47:17-\allowbreak20}
\crossref{Ezek}{47}{16}{Nu 13:21; 34:8 1Ki 8:65 Am 6:14 Zec 9:2}
\crossref{Ezek}{47}{17}{Eze 48:1 Nu 34:9}
\crossref{Ezek}{47}{18}{Ge 31:23,\allowbreak47}
\crossref{Ezek}{47}{19}{Eze 48:28}
\crossref{Ezek}{47}{20}{Nu 34:6}
\crossref{Ezek}{47}{21}{Nu 34:13-\allowbreak15}
\crossref{Ezek}{47}{22}{47:13,\allowbreak14}
\crossref{Ezek}{47}{23}{47:22 De 10:19 Zec 7:10 Mal 3:5}
\crossref{Ezek}{48}{1}{Ex 1:1-\allowbreak5 Nu 1:5-\allowbreak15; 13:4-\allowbreak15 Re 7:4-\allowbreak8}
\crossref{Ezek}{48}{2}{Ge 30:12,\allowbreak13 Jos 19:24-\allowbreak31}
\crossref{Ezek}{48}{3}{Ge 30:7,\allowbreak8 Jos 19:32-\allowbreak39}
\crossref{Ezek}{48}{4}{Ge 30:22-\allowbreak24; 41:51; 48:5,\allowbreak14-\allowbreak20 Jos 13:29-\allowbreak31; 17:1-\allowbreak11}
\crossref{Ezek}{48}{5}{Jos 16:1-\allowbreak10; 17:8-\allowbreak10,\allowbreak14-\allowbreak18}
\crossref{Ezek}{48}{6}{Ge 29:32; 49:3,\allowbreak4 Jos 13:15-\allowbreak21}
\crossref{Ezek}{48}{7}{Ge 29:35 Jos 15:1-\allowbreak63; 19:9}
\crossref{Ezek}{48}{8}{Eze 45:1-\allowbreak6}
\crossref{Ezek}{48}{9}{48:8,\allowbreak10,\allowbreak21 Eze 44:30}
\crossref{Ezek}{48}{10}{Eze 44:28; 45:4 Nu 35:1-\allowbreak9 Jos 21:1-\allowbreak45 Mt 10:10 1Co 9:13,\allowbreak14}
\crossref{Ezek}{48}{11}{Mt 24:45,\allowbreak45,\allowbreak46 2Ti 4:7,\allowbreak8 1Pe 5:4 Re 2:10}
\crossref{Ezek}{48}{12}{Eze 45:4 Le 27:21}
\crossref{Ezek}{48}{13}{Eze 45:3 De 12:19 Lu 10:7}
\crossref{Ezek}{48}{14}{Ex 22:29 Le 27:10,\allowbreak28,\allowbreak33}
\crossref{Ezek}{48}{15}{Eze 22:26; 42:20; 44:23; 45:6}
\crossref{Ezek}{48}{16}{48:16}
\crossref{Ezek}{48}{17}{}
\crossref{Ezek}{48}{18}{Jos 9:27 Ezr 2:43-\allowbreak58 Ne 7:46-\allowbreak62}
\crossref{Ezek}{48}{19}{Eze 45:6 1Ki 4:7-\allowbreak23 Ne 11:1-\allowbreak36}
\crossref{Ezek}{48}{20}{Heb 12:17 Re 21:16}
\crossref{Ezek}{48}{21}{48:22; 34:23,\allowbreak24; 37:24; 45:7,\allowbreak8 Ho 1:11}
\crossref{Ezek}{48}{22}{}
\crossref{Ezek}{48}{23}{48:1-\allowbreak7 Ge 35:16-\allowbreak19 Jos 18:21-\allowbreak28}
\crossref{Ezek}{48}{24}{Ge 29:33; 49:5-\allowbreak7 Jos 19:1-\allowbreak9}
\crossref{Ezek}{48}{25}{Ge 30:14-\allowbreak18 Jos 19:17-\allowbreak23}
\crossref{Ezek}{48}{26}{Ge 30:19,\allowbreak20 Jos 19:10-\allowbreak16}
\crossref{Ezek}{48}{27}{Ge 30:10,\allowbreak11 Jos 13:24-\allowbreak28}
\crossref{Ezek}{48}{28}{Eze 47:19 2Ch 20:2}
\crossref{Ezek}{48}{29}{Eze 47:13-\allowbreak22 Nu 34:2,\allowbreak13 Jos 13:1-\allowbreak21:45}
\crossref{Ezek}{48}{30}{48:16,\allowbreak32-\allowbreak35 Re 21:16}
\crossref{Ezek}{48}{31}{Isa 26:1,\allowbreak2; 54:12; 60:11 Re 21:12,\allowbreak13,\allowbreak21,\allowbreak25}
\crossref{Ezek}{48}{32}{48:32}
\crossref{Ezek}{48}{33}{48:33}
\crossref{Ezek}{48}{34}{}
\crossref{Ezek}{48}{35}{Ge 22:14 Jer 33:16 Zec 14:21}

% Dan
\crossref{Dan}{1}{1}{2Ki 24:1,\allowbreak2,\allowbreak13 2Ch 36:5-\allowbreak7}
\crossref{Dan}{1}{2}{Da 2:37,\allowbreak38; 5:18 De 28:49-\allowbreak52; 32:30 Jud 2:14; 3:8; 4:2 Ps 106:41,\allowbreak42}
\crossref{Dan}{1}{3}{2Ki 20:17,\allowbreak18 Isa 39:7 Jer 41:1}
\crossref{Dan}{1}{4}{Le 21:18-\allowbreak21; 24:19,\allowbreak20 Jud 8:18 2Sa 14:25 Ac 7:20 Eph 5:27}
\crossref{Dan}{1}{5}{1:19 Ge 41:46 1Sa 16:22 1Ki 10:8 2Ch 9:7 Jer 15:19 Lu 1:19}
\crossref{Dan}{1}{6}{Da 2:17 Eze 14:14,\allowbreak20; 28:3 Mt 24:15 Mr 13:14}
\crossref{Dan}{1}{7}{1:3,\allowbreak10,\allowbreak11}
\crossref{Dan}{1}{8}{Ru 1:17,\allowbreak18 1Ki 5:5 Ps 119:106,\allowbreak115 Ac 11:23 1Co 7:37 2Co 9:7}
\crossref{Dan}{1}{9}{Ge 32:28; 39:21 1Ki 8:50 Ezr 7:27,\allowbreak28 Ne 1:11; 2:4 Ps 4:3; 106:46}
\crossref{Dan}{1}{10}{Pr 29:25 Joh 12:42,\allowbreak43}
\crossref{Dan}{1}{11}{1:16}
\crossref{Dan}{1}{12}{}
\crossref{Dan}{1}{13}{La 4:7}
\crossref{Dan}{1}{14}{}
\crossref{Dan}{1}{15}{Ex 23:25 De 28:1-\allowbreak14 2Ki 4:42-\allowbreak44 Ps 37:16 Pr 10:22 Hag 1:6,\allowbreak9}
\crossref{Dan}{1}{16}{}
\crossref{Dan}{1}{17}{Da 2:21,\allowbreak23 1Ki 3:12,\allowbreak28; 4:29-\allowbreak31 2Ch 1:10,\allowbreak12 Job 32:8}
\crossref{Dan}{1}{18}{1:3 Ge 37:36}
\crossref{Dan}{1}{19}{1:5 Ge 41:46 1Ki 17:1 Pr 22:29 Jer 15:19}
\crossref{Dan}{1}{20}{1Ki 4:29-\allowbreak34; 10:1-\allowbreak3,\allowbreak23,\allowbreak24 Ps 119:99}
\crossref{Dan}{1}{21}{Da 6:28; 10:1}
\crossref{Dan}{2}{1}{Da 1:1-\allowbreak5 2Ch 36:5-\allowbreak7}
\crossref{Dan}{2}{2}{Da 1:20; 4:6; 5:7 Ge 41:8 Ex 7:11 De 18:10-\allowbreak12 Isa 8:19; 19:3}
\crossref{Dan}{2}{3}{2:1 Ge 40:8; 41:15}
\crossref{Dan}{2}{4}{Ge 31:47 Ezr 4:7 Isa 36:11}
\crossref{Dan}{2}{5}{De 13:16 Jos 6:26 2Ki 10:27 Ezr 6:11}
\crossref{Dan}{2}{6}{2:48; 5:7,\allowbreak16,\allowbreak29 Nu 22:7,\allowbreak17,\allowbreak37; 24:11}
\crossref{Dan}{2}{7}{2:4,\allowbreak9 Ec 10:4}
\crossref{Dan}{2}{8}{Eph 5:16 Col 4:5}
\crossref{Dan}{2}{9}{Da 3:15 Es 4:11}
\crossref{Dan}{2}{10}{2:10,\allowbreak27; 4:7,\allowbreak9; 5:11}
\crossref{Dan}{2}{11}{Ex 29:45 Nu 35:34 1Ki 8:27 2Ch 6:18 Ps 68:18; 113:5,\allowbreak6; 132:14}
\crossref{Dan}{2}{12}{Da 3:13 Job 5:2 Ps 76:10 Pr 16:14; 19:12; 20:2; 27:3,\allowbreak4; 29:22}
\crossref{Dan}{2}{13}{Da 6:9-\allowbreak15 Es 3:12-\allowbreak15 Ps 94:20 Pr 28:15-\allowbreak17 Isa 10:1}
\crossref{Dan}{2}{14}{2Sa 20:16-\allowbreak22 Ec 9:13-\allowbreak18}
\crossref{Dan}{2}{15}{2:9}
\crossref{Dan}{2}{16}{2:9-\allowbreak11; 1:18,\allowbreak19}
\crossref{Dan}{2}{17}{Da 1:7,\allowbreak11; 3:12}
\crossref{Dan}{2}{18}{Da 3:17 1Sa 17:37 Es 4:15-\allowbreak17 Ps 50:15; 91:15 Pr 3:5,\allowbreak6 Isa 37:4}
\crossref{Dan}{2}{19}{2:22,\allowbreak27-\allowbreak29; 4:9 2Ki 6:8-\allowbreak12 Ps 25:14 Am 3:7 1Co 2:9,\allowbreak10}
\crossref{Dan}{2}{20}{Ge 14:20 1Ki 8:56 1Ch 29:10,\allowbreak20 2Ch 20:21 Ps 41:13; 50:23}
\crossref{Dan}{2}{21}{2:9; 7:25; 11:6 1Ch 29:30 Es 1:13 Job 34:24-\allowbreak29 Ps 31:14,\allowbreak15}
\crossref{Dan}{2}{22}{2:11,\allowbreak28,\allowbreak29 Ge 37:5-\allowbreak9; 41:16,\allowbreak25-\allowbreak28 Job 12:22 Ps 25:14}
\crossref{Dan}{2}{23}{1Ch 29:13 Ps 50:14; 103:1-\allowbreak4 Isa 12:1 Mt 11:25 Lu 10:21}
\crossref{Dan}{2}{24}{2:15}
\crossref{Dan}{2}{25}{Pr 24:11 Ec 9:10}
\crossref{Dan}{2}{26}{Da 1:7; 4:8,\allowbreak19; 5:12}
\crossref{Dan}{2}{27}{2:2,\allowbreak10,\allowbreak11; 5:7,\allowbreak8 Job 5:12,\allowbreak13 Isa 19:3; 44:25; 47:12-\allowbreak14}
\crossref{Dan}{2}{28}{Ps 115:3 Mt 6:9}
\crossref{Dan}{2}{29}{Eze 38:10}
\crossref{Dan}{2}{30}{Ge 41:16 Ac 3:12 1Co 15:8-\allowbreak12}
\crossref{Dan}{2}{31}{Da 7:3-\allowbreak17 Mt 4:8 Lu 4:5}
\crossref{Dan}{2}{32}{2:37,\allowbreak38; 4:22,\allowbreak30; 7:4 Isa 14:4 Jer 51:7 Re 17:4}
\crossref{Dan}{2}{33}{2:40-\allowbreak43; 7:7,\allowbreak8,\allowbreak19-\allowbreak26}
\crossref{Dan}{2}{34}{2:44,\allowbreak45; 7:13,\allowbreak14,\allowbreak27 Ps 118:22 Isa 28:16 Zec 12:3 Mt 16:18}
\crossref{Dan}{2}{35}{Ps 1:4,\allowbreak5 Isa 17:13,\allowbreak14; 41:15,\allowbreak16 Ho 13:3 Mic 4:13}
\crossref{Dan}{2}{36}{2:23,\allowbreak24}
\crossref{Dan}{2}{37}{1Ki 4:24 Ezr 7:12 Isa 10:8; 47:5 Jer 27:6,\allowbreak7 Eze 26:7 Ho 8:10}
\crossref{Dan}{2}{38}{Da 4:21,\allowbreak22 Ps 50:10,\allowbreak11 Jer 27:5-\allowbreak7}
\crossref{Dan}{2}{39}{}
\crossref{Dan}{2}{40}{Da 7:7 Jer 15:12 Am 1:3}
\crossref{Dan}{2}{41}{2:33-\allowbreak35; 7:7,\allowbreak24 Re 12:3; 13:1; 17:12}
\crossref{Dan}{2}{42}{Da 7:24 Re 13:1}
\crossref{Dan}{2}{43}{}
\crossref{Dan}{2}{44}{2:28,\allowbreak37}
\crossref{Dan}{2}{45}{2:24,\allowbreak35 Isa 28:16 Zec 12:3 Mt 21:24}
\crossref{Dan}{2}{46}{Lu 17:16 Ac 10:25; 14:13; 28:6 Re 11:16; 19:10; 22:8}
\crossref{Dan}{2}{47}{Da 11:36 De 10:17 Jos 22:22 Ps 136:2}
\crossref{Dan}{2}{48}{2:6; 5:16 Ge 41:39-\allowbreak43 Nu 22:16,\allowbreak17; 24:11 1Sa 17:25; 25:2 2Sa 19:32}
\crossref{Dan}{2}{49}{2:17; 1:17; 3:12-\allowbreak30 Pr 28:12}
\crossref{Dan}{3}{1}{Da 2:31,\allowbreak32; 5:23 Ex 20:23; 32:2-\allowbreak4,\allowbreak31 De 7:25 Jud 8:26,\allowbreak27 1Ki 12:28}
\crossref{Dan}{3}{2}{Ex 32:4-\allowbreak6 Nu 25:2 Jud 16:23 1Ki 12:32 Pr 29:12 Re 17:2}
\crossref{Dan}{3}{3}{}
\crossref{Dan}{3}{4}{Da 4:14 Pr 9:13-\allowbreak15 Isa 40:9; 58:1}
\crossref{Dan}{3}{5}{3:10,\allowbreak15}
\crossref{Dan}{3}{6}{3:11,\allowbreak15 Ex 20:5 Isa 44:17 Mt 4:9 Re 13:15-\allowbreak17}
\crossref{Dan}{3}{7}{3:10}
\crossref{Dan}{3}{8}{Da 6:12,\allowbreak13 Ezr 4:12-\allowbreak16 Es 3:6,\allowbreak8,\allowbreak9 Ac 16:20-\allowbreak22; 17:6-\allowbreak8; 28:22}
\crossref{Dan}{3}{9}{3:4,\allowbreak5}
\crossref{Dan}{3}{10}{3:4-\allowbreak7; 6:12 Ex 1:16,\allowbreak22 Es 3:12-\allowbreak14 Ps 94:20 Ec 3:16 Isa 10:1}
\crossref{Dan}{3}{11}{}
\crossref{Dan}{3}{12}{Da 2:49; 6:13 1Sa 18:7-\allowbreak11 Es 3:8 Pr 27:4 Ec 4:4}
\crossref{Dan}{3}{13}{3:19; 2:12 Ge 4:5 1Sa 20:30-\allowbreak33 Es 3:5,\allowbreak6 Pr 17:12; 27:3; 29:22}
\crossref{Dan}{3}{14}{Ex 21:13,\allowbreak14}
\crossref{Dan}{3}{15}{3:10}
\crossref{Dan}{3}{16}{De 1:41}
\crossref{Dan}{3}{17}{Da 4:35; 6:20-\allowbreak22,\allowbreak27 Ge 17:1; 18:14 1Sa 17:37,\allowbreak46 Job 5:19; 34:29}
\crossref{Dan}{3}{18}{Job 13:15 Pr 28:1 Isa 51:12,\allowbreak13 Mt 10:28,\allowbreak32,\allowbreak33,\allowbreak39; 16:2}
\crossref{Dan}{3}{19}{3:13 Pr 21:24 Isa 51:23 Lu 12:4,\allowbreak5 Ac 5:33; 7:54}
\crossref{Dan}{3}{20}{}
\crossref{Dan}{3}{21}{3:21}
\crossref{Dan}{3}{22}{Ex 12:33}
\crossref{Dan}{3}{23}{Da 6:16,\allowbreak17 Ps 34:19; 66:11,\allowbreak12; 124:1-\allowbreak5 Jer 38:6 La 3:52-\allowbreak54}
\crossref{Dan}{3}{24}{Da 5:6 Ac 5:23-\allowbreak25; 9:6; 12:13}
\crossref{Dan}{3}{25}{Isa 43:2}
\crossref{Dan}{3}{26}{3:17; 2:47; 6:20 Ezr 5:11 Ac 16:17; 27:23 Ga 1:10 Re 19:5}
\crossref{Dan}{3}{27}{3:2,\allowbreak3 1Sa 17:46,\allowbreak47 2Ki 19:19 Ps 83:18; 96:7-\allowbreak9 Isa 26:11}
\crossref{Dan}{3}{28}{Da 2:47; 4:34; 6:26 Ge 9:26 Ezr 1:3; 7:23-\allowbreak28}
\crossref{Dan}{3}{29}{Da 6:26,\allowbreak27}
\crossref{Dan}{3}{30}{1Sa 2:30 Ps 91:14 Joh 12:26 Ro 8:31}
\crossref{Dan}{4}{1}{Da 3:4,\allowbreak29; 7:14 Es 3:12; 8:9 Zec 8:23 Ac 2:6}
\crossref{Dan}{4}{2}{Jos 7:19 Ps 51:14; 71:18; 92:1,\allowbreak2}
\crossref{Dan}{4}{3}{Da 6:27 De 4:34 Ps 71:19,\allowbreak20; 72:18; 77:19; 86:10; 92:5; 104:24; 105:27}
\crossref{Dan}{4}{4}{Ps 30:6,\allowbreak7 Isa 47:7,\allowbreak8; 56:12 Jer 48:11 Eze 28:2-\allowbreak5,\allowbreak17; 29:3}
\crossref{Dan}{4}{5}{Da 2:1; 5:5,\allowbreak6,\allowbreak10; 7:28 Ge 41:1 Job 7:13,\allowbreak14}
\crossref{Dan}{4}{6}{Da 2:2 Ge 41:7,\allowbreak8 Isa 8:19; 47:12-\allowbreak14}
\crossref{Dan}{4}{7}{Da 2:1,\allowbreak2}
\crossref{Dan}{4}{8}{Da 1:7; 5:12 Isa 46:1 Jer 50:2}
\crossref{Dan}{4}{9}{Da 1:20; 2:48; 5:11}
\crossref{Dan}{4}{10}{}
\crossref{Dan}{4}{11}{4:21,\allowbreak22 Ge 11:4 De 9:1 Mt 11:23}
\crossref{Dan}{4}{12}{Jer 27:6,\allowbreak7 Eze 17:23; 31:6}
\crossref{Dan}{4}{13}{4:5,\allowbreak10; 7:1}
\crossref{Dan}{4}{14}{Da 3:4 Re 10:3; 18:2}
\crossref{Dan}{4}{15}{4:25-\allowbreak27 Job 14:7-\allowbreak9 Eze 29:14,\allowbreak15}
\crossref{Dan}{4}{16}{4:32,\allowbreak33 Isa 6:10 Heb 1:11 Mr 5:4,\allowbreak5 Lu 8:27-\allowbreak29}
\crossref{Dan}{4}{17}{4:13,\allowbreak14 1Ki 22:19,\allowbreak20 1Ti 5:21}
\crossref{Dan}{4}{18}{4:7; 2:7; 5:8,\allowbreak15 Ge 41:8,\allowbreak15 Isa 19:3; 47:12-\allowbreak14}
\crossref{Dan}{4}{19}{4:8; 1:7; 2:26; 5:12}
\crossref{Dan}{4}{20}{4:10-\allowbreak12 Eze 31:3,\allowbreak16}
\crossref{Dan}{4}{21}{4:12}
\crossref{Dan}{4}{22}{Da 2:37,\allowbreak38 2Sa 12:7 Mt 14:4}
\crossref{Dan}{4}{23}{4:13-\allowbreak17}
\crossref{Dan}{4}{24}{4:17 Job 20:29}
\crossref{Dan}{4}{25}{4:32,\allowbreak33; 5:21-\allowbreak31 Job 30:3-\allowbreak8 Mr 5:3,\allowbreak4}
\crossref{Dan}{4}{26}{4:15}
\crossref{Dan}{4}{27}{Ge 41:33-\allowbreak37 Ps 119:46 Ac 24:25 2Co 5:11}
\crossref{Dan}{4}{28}{Nu 23:19 Pr 10:24 Zec 1:6 Mt 24:35}
\crossref{Dan}{4}{29}{Ge 6:3 Ec 8:11 1Pe 3:20 2Pe 3:9,\allowbreak10,\allowbreak15 Re 2:21}
\crossref{Dan}{4}{30}{Da 5:20 Ps 73:8 Pr 16:18 Hab 1:15,\allowbreak16; 2:4,\allowbreak5 Lu 12:19,\allowbreak20; 14:11}
\crossref{Dan}{4}{31}{Da 5:4,\allowbreak5 Ex 15:9,\allowbreak10 Job 20:23 Lu 12:20 Ac 12:22,\allowbreak23 1Th 5:3}
\crossref{Dan}{4}{32}{4:14-\allowbreak16,\allowbreak25,\allowbreak26; 5:21 Job 30:5-\allowbreak7}
\crossref{Dan}{4}{33}{Da 5:5 Job 20:5 Isa 30:14 1Th 5:2}
\crossref{Dan}{4}{34}{4:16,\allowbreak26,\allowbreak32}
\crossref{Dan}{4}{35}{Job 34:14,\allowbreak15,\allowbreak19-\allowbreak24 Isa 40:15-\allowbreak17,\allowbreak22-\allowbreak24}
\crossref{Dan}{4}{36}{4:15,\allowbreak16,\allowbreak32 2Ch 33:12,\allowbreak13}
\crossref{Dan}{4}{37}{4:3,\allowbreak34; 5:4,\allowbreak23 1Pe 2:9,\allowbreak10}
\crossref{Dan}{5}{1}{Ge 40:20 Es 1:3 Isa 21:4,\allowbreak5; 22:12,\allowbreak14 Jer 51:39,\allowbreak57 Na 1:10}
\crossref{Dan}{5}{2}{Da 1:2 2Ki 24:13; 25:15 2Ch 36:10,\allowbreak18 Ezr 1:7-\allowbreak11 Jer 27:16-\allowbreak22}
\crossref{Dan}{5}{3}{2Ki 25:12 Jer 39:10; 40:6; 52:16 Mt 7:14 Lu 13:23,\allowbreak24 1Pe 4:18}
\crossref{Dan}{5}{4}{5:23}
\crossref{Dan}{5}{5}{Da 4:31,\allowbreak33 Job 20:5 Ps 78:30,\allowbreak31 Pr 29:1 Lu 12:19,\allowbreak20 1Th 5:2,\allowbreak3}
\crossref{Dan}{5}{6}{5:9; 2:1; 3:19 Job 15:20-\allowbreak27; 20:19-\allowbreak27 Ps 73:18-\allowbreak20 Isa 21:2-\allowbreak4}
\crossref{Dan}{5}{7}{Da 4:14}
\crossref{Dan}{5}{8}{}
\crossref{Dan}{5}{9}{5:6; 2:1 Job 18:11-\allowbreak14 Ps 18:14 Re 6:15}
\crossref{Dan}{5}{10}{Da 2:4; 3:9; 6:6,\allowbreak21 1Ki 1:31}
\crossref{Dan}{5}{11}{Da 2:47; 4:8,\allowbreak9,\allowbreak18 Ge 41:11-\allowbreak15}
\crossref{Dan}{5}{12}{5:14; 6:3 Ps 16:3 Pr 12:26; 17:27 Col 1:29}
\crossref{Dan}{5}{13}{Da 2:25; 6:13 Ezr 4:1; 6:16,\allowbreak19,\allowbreak20; 10:7,\allowbreak16}
\crossref{Dan}{5}{14}{5:11,\allowbreak12}
\crossref{Dan}{5}{15}{5:7,\allowbreak8; 2:3-\allowbreak11 Isa 29:10-\allowbreak12; 47:12}
\crossref{Dan}{5}{16}{Ge 40:8}
\crossref{Dan}{5}{17}{Da 2:6}
\crossref{Dan}{5}{18}{Da 3:17,\allowbreak18; 4:22; 6:22 Ac 26:13,\allowbreak19}
\crossref{Dan}{5}{19}{Da 3:4; 4:22 Jer 25:9-\allowbreak14; 27:5-\allowbreak7 Hab 2:5 Ro 13:1}
\crossref{Dan}{5}{20}{Da 4:30-\allowbreak33,\allowbreak37 Ex 9:17; 18:11 Job 15:25-\allowbreak27; 40:11,\allowbreak12 Pr 16:5,\allowbreak18}
\crossref{Dan}{5}{21}{Da 4:25,\allowbreak32,\allowbreak33 Job 30:3-\allowbreak7}
\crossref{Dan}{5}{22}{5:18 Ps 119:46 Mt 14:4 Ac 4:8-\allowbreak13}
\crossref{Dan}{5}{23}{5:3,\allowbreak4 2Ki 14:10 Isa 2:12; 33:10; 37:23 Jer 50:29 Eze 28:2,\allowbreak5,\allowbreak17}
\crossref{Dan}{5}{24}{5:5}
\crossref{Dan}{5}{25}{}
\crossref{Dan}{5}{26}{Da 9:2 Job 14:14 Isa 13:1-\allowbreak14:32; 21:1-\allowbreak10; 47:1-\allowbreak15 Jer 25:11,\allowbreak12}
\crossref{Dan}{5}{27}{Job 31:6 Ps 62:9 Jer 6:30 Eze 22:18-\allowbreak20}
\crossref{Dan}{5}{28}{5:31; 6:28; 8:3,\allowbreak4,\allowbreak20; 9:1 Isa 13:17; 21:2; 45:1,\allowbreak2}
\crossref{Dan}{5}{29}{5:7,\allowbreak16}
\crossref{Dan}{5}{30}{5:1,\allowbreak2 Isa 21:4-\allowbreak9; 47:9 Jer 51:11,\allowbreak31,\allowbreak39,\allowbreak57}
\crossref{Dan}{5}{31}{}
\crossref{Dan}{6}{1}{Da 5:31 1Pe 2:14}
\crossref{Dan}{6}{2}{Da 2:48,\allowbreak49; 5:16,\allowbreak29 1Sa 2:30 Pr 3:16}
\crossref{Dan}{6}{3}{Pr 22:29}
\crossref{Dan}{6}{4}{Da 3:8 Ge 43:18 Jud 14:4 Ps 37:12,\allowbreak13,\allowbreak32,\allowbreak33 Pr 29:27 Ec 4:4}
\crossref{Dan}{6}{5}{1Sa 24:17 Es 3:8 Joh 19:6,\allowbreak7 Ac 24:13-\allowbreak16,\allowbreak20,\allowbreak21}
\crossref{Dan}{6}{6}{6:11 Ps 56:6; 62:3; 64:2-\allowbreak6 Mt 27:23-\allowbreak25 Lu 23:23-\allowbreak25 Ac 22:22,\allowbreak23}
\crossref{Dan}{6}{7}{6:2,\allowbreak3; 3:2,\allowbreak27}
\crossref{Dan}{6}{8}{Es 3:12; 8:10 Isa 10:1}
\crossref{Dan}{6}{9}{Ps 62:9,\allowbreak10; 118:9; 146:3 Pr 6:2 Isa 2:22}
\crossref{Dan}{6}{10}{Lu 14:26 Ac 4:17-\allowbreak19}
\crossref{Dan}{6}{11}{6:6 Ps 10:9; 37:32,\allowbreak33}
\crossref{Dan}{6}{12}{Da 3:8-\allowbreak12 Ac 16:19,\allowbreak24; 24:2-\allowbreak9}
\crossref{Dan}{6}{13}{Da 1:6; 2:25; 5:13}
\crossref{Dan}{6}{14}{}
\crossref{Dan}{6}{15}{6:8,\allowbreak12 Es 8:8 Ps 94:20,\allowbreak21}
\crossref{Dan}{6}{16}{2Sa 3:39 Pr 29:25 Jer 26:14; 38:5 Mt 14:8-\allowbreak10; 27:23-\allowbreak26}
\crossref{Dan}{6}{17}{La 3:53 Mt 27:60-\allowbreak66 Ac 12:4; 16:23,\allowbreak24}
\crossref{Dan}{6}{18}{2Sa 12:16,\allowbreak17; 19:24 1Ki 21:27 Job 21:12 Ps 137:2 Ec 2:8}
\crossref{Dan}{6}{19}{Mt 28:1 Mr 16:2 2Co 2:13 1Th 3:5}
\crossref{Dan}{6}{20}{6:16,\allowbreak27; 3:15,\allowbreak17,\allowbreak28,\allowbreak29}
\crossref{Dan}{6}{21}{6:6; 2:4 Ne 2:3}
\crossref{Dan}{6}{22}{6:20 2Sa 22:7 Ps 31:14; 38:21; 118:28 Mic 7:7 Mt 27:46 Joh 20:17}
\crossref{Dan}{6}{23}{6:14,\allowbreak18 Ex 18:9 1Ki 5:7 2Ch 2:11,\allowbreak12}
\crossref{Dan}{6}{24}{De 19:18-\allowbreak20 Es 7:10; 9:25 Pr 11:8}
\crossref{Dan}{6}{25}{Da 4:1 Ezr 1:1,\allowbreak2 Es 3:12; 8:9}
\crossref{Dan}{6}{26}{Da 3:29 Ezr 6:8-\allowbreak12; 7:12,\allowbreak13}
\crossref{Dan}{6}{27}{Job 36:15 Ps 18:48,\allowbreak50; 32:7; 35:17; 97:10 Lu 1:74,\allowbreak75 2Co 1:8-\allowbreak10}
\crossref{Dan}{6}{28}{Da 1:21 2Ch 36:22,\allowbreak23 Ezr 1:1,\allowbreak2 Isa 44:28; 45:1}
\crossref{Dan}{7}{1}{Da 5:1,\allowbreak22,\allowbreak30; 8:1 Jer 27:7}
\crossref{Dan}{7}{2}{Re 7:1}
\crossref{Dan}{7}{3}{7:4-\allowbreak8,\allowbreak17 Ps 76:4 Eze 19:3-\allowbreak8 Re 13:1}
\crossref{Dan}{7}{4}{De 28:49 2Sa 1:23 Isa 5:28,\allowbreak29 Jer 4:7,\allowbreak13; 25:38; 48:40 La 4:19}
\crossref{Dan}{7}{5}{Da 2:39; 8:3 2Ki 2:24 Pr 17:12 Ho 13:8}
\crossref{Dan}{7}{6}{Da 2:39; 8:5-\allowbreak7,\allowbreak20,\allowbreak21; 10:20; 11:3-\allowbreak20 Ho 13:7 Re 13:2}
\crossref{Dan}{7}{7}{7:2,\allowbreak13}
\crossref{Dan}{7}{8}{7:20-\allowbreak25; 8:9-\allowbreak12 Re 13:11-\allowbreak13}
\crossref{Dan}{7}{9}{Da 2:34,\allowbreak35,\allowbreak44,\allowbreak45 1Co 15:24,\allowbreak25 Re 19:18-\allowbreak21; 20:1-\allowbreak4}
\crossref{Dan}{7}{10}{Ps 50:3; 97:2,\allowbreak3 Isa 30:27,\allowbreak33; 66:15,\allowbreak16 Na 1:5,\allowbreak6}
\crossref{Dan}{7}{11}{7:8,\allowbreak25 2Pe 2:18 Jude 1:16 Re 13:5,\allowbreak6; 20:4,\allowbreak12}
\crossref{Dan}{7}{12}{7:4-\allowbreak6; 8:7}
\crossref{Dan}{7}{13}{Ps 8:4,\allowbreak5 Isa 9:6,\allowbreak7 Eze 1:26 Mt 13:41; 24:30; 25:31; 26:64}
\crossref{Dan}{7}{14}{7:27 Ps 2:6-\allowbreak8; 8:6; 110:1,\allowbreak2 Mt 11:27; 28:18 Lu 10:22; 19:11,\allowbreak12}
\crossref{Dan}{7}{15}{7:28; 8:27 Jer 15:17,\allowbreak18; 17:16 Hab 3:16 Lu 19:41-\allowbreak44 Ro 9:2,\allowbreak3}
\crossref{Dan}{7}{16}{7:10; 8:13-\allowbreak16; 10:5,\allowbreak6,\allowbreak11,\allowbreak12; 12:5,\allowbreak6 Zec 1:8-\allowbreak11; 2:3; 3:7 Re 5:5}
\crossref{Dan}{7}{17}{7:3,\allowbreak4; 2:37-\allowbreak40; 8:19-\allowbreak22}
\crossref{Dan}{7}{18}{7:22,\allowbreak27 Ps 45:16; 149:5-\allowbreak9 Isa 60:12-\allowbreak14 2Ti 2:11,\allowbreak12 Re 2:26,\allowbreak27}
\crossref{Dan}{7}{19}{7:7; 2:40-\allowbreak43}
\crossref{Dan}{7}{20}{Da 11:36,\allowbreak37}
\crossref{Dan}{7}{21}{Da 8:12,\allowbreak24; 11:31; 12:7 Re 11:7-\allowbreak9; 12:3,\allowbreak4; 13:5-\allowbreak7,\allowbreak8-\allowbreak18; 17:6,\allowbreak14; 19:19}
\crossref{Dan}{7}{22}{7:9-\allowbreak11 2Th 2:8 Re 11:11-\allowbreak18; 14:8-\allowbreak20; 19:11-\allowbreak21; 20:9-\allowbreak15}
\crossref{Dan}{7}{23}{7:7; 2:40 Lu 2:1}
\crossref{Dan}{7}{24}{7:20 Re 12:3; 13:1; 17:3,\allowbreak12,\allowbreak13,\allowbreak16-\allowbreak18}
\crossref{Dan}{7}{25}{}
\crossref{Dan}{7}{26}{7:10,\allowbreak11,\allowbreak22 2Th 2:8 Re 11:13; 20:10,\allowbreak11}
\crossref{Dan}{7}{27}{7:14,\allowbreak18,\allowbreak22 Ps 149:5-\allowbreak9 Isa 49:23-\allowbreak26; 54:3; 60:11-\allowbreak16 Zep 3:19,\allowbreak20}
\crossref{Dan}{7}{28}{Da 8:17,\allowbreak19; 11:27; 12:9,\allowbreak13}
\crossref{Dan}{8}{1}{Da 7:1}
\crossref{Dan}{8}{2}{8:3}
\crossref{Dan}{8}{3}{Da 10:5 Nu 24:2 Jos 5:13 1Ch 21:16 Zec 1:18; 2:1; 5:1,\allowbreak5,\allowbreak9; 6:1}
\crossref{Dan}{8}{4}{Da 5:30; 7:5; 11:2 Isa 45:1-\allowbreak5 Jer 50:1-\allowbreak51:64}
\crossref{Dan}{8}{5}{8:21; 2:32,\allowbreak39; 7:6}
\crossref{Dan}{8}{6}{8:3}
\crossref{Dan}{8}{7}{Da 11:11}
\crossref{Dan}{8}{8}{De 31:20 Es 9:4 Jer 5:27 Eze 16:7}
\crossref{Dan}{8}{9}{8:23,\allowbreak24; 7:8,\allowbreak20-\allowbreak26; 11:21,\allowbreak25-\allowbreak45}
\crossref{Dan}{8}{10}{8:24,\allowbreak25; 11:28,\allowbreak30,\allowbreak33-\allowbreak36 Isa 14:13 Re 12:4}
\crossref{Dan}{8}{11}{8:25; 5:23; 7:25; 11:36 2Ki 19:22,\allowbreak23 2Ch 32:15-\allowbreak22 Isa 37:23,\allowbreak29}
\crossref{Dan}{8}{12}{Ps 119:43,\allowbreak142 Isa 59:14 2Th 2:10-\allowbreak12}
\crossref{Dan}{8}{13}{Da 4:13; 7:16; 12:5,\allowbreak6 De 33:2 Zec 1:9-\allowbreak12,\allowbreak19; 2:3,\allowbreak4; 14:5 1Th 3:13}
\crossref{Dan}{8}{14}{Da 7:25; 12:7,\allowbreak11 Re 11:2,\allowbreak3; 12:14; 13:5}
\crossref{Dan}{8}{15}{Da 7:28}
\crossref{Dan}{8}{16}{Da 10:11,\allowbreak12 Ac 9:7; 10:13 Re 1:12}
\crossref{Dan}{8}{17}{Da 10:7,\allowbreak8,\allowbreak16 Ge 17:3 Eze 1:28 Mt 17:8 Mr 9:4,\allowbreak5 Re 1:17; 19:9,\allowbreak10}
\crossref{Dan}{8}{18}{8:17,\allowbreak27; 10:8,\allowbreak9 Lu 9:32; 22:45}
\crossref{Dan}{8}{19}{8:15-\allowbreak17 Re 1:1}
\crossref{Dan}{8}{20}{8:3; 11:1,\allowbreak2}
\crossref{Dan}{8}{21}{8:5-\allowbreak7; 10:20}
\crossref{Dan}{8}{22}{8:3; 11:4}
\crossref{Dan}{8}{23}{Da 10:14 Nu 24:24 Eze 38:8,\allowbreak16 1Ti 4:1}
\crossref{Dan}{8}{24}{Re 13:3-\allowbreak9; 17:12,\allowbreak13,\allowbreak17}
\crossref{Dan}{8}{25}{8:23,\allowbreak24; 7:8; 11:21-\allowbreak25,\allowbreak32,\allowbreak33}
\crossref{Dan}{8}{26}{8:11-\allowbreak15; 10:1}
\crossref{Dan}{8}{27}{8:7; 7:28; 10:8,\allowbreak16 Hab 3:16}
\crossref{Dan}{9}{1}{Da 1:21; 5:31; 6:1,\allowbreak28; 11:1}
\crossref{Dan}{9}{2}{Da 8:15,\allowbreak16 Ps 119:24,\allowbreak99,\allowbreak100 Mt 24:15 Mr 13:14 Ac 8:34 1Ti 4:13}
\crossref{Dan}{9}{3}{Da 6:10 Ne 1:4-\allowbreak11 Ps 102:13-\allowbreak17 Jer 29:10-\allowbreak13; 33:3 Eze 36:37}
\crossref{Dan}{9}{4}{9:5-\allowbreak12 Le 26:40-\allowbreak42 1Ki 8:47-\allowbreak49 2Ch 7:14 Ne 9:2,\allowbreak3 Ps 32:5}
\crossref{Dan}{9}{5}{9:15 1Ki 8:47-\allowbreak50 2Ch 6:37-\allowbreak39 Ezr 9:6 Ne 1:6-\allowbreak8; 9:33,\allowbreak34 Ps 106:6}
\crossref{Dan}{9}{6}{9:10 2Ki 17:13,\allowbreak14 2Ch 33:10; 36:15,\allowbreak16 Isa 30:10,\allowbreak11 Jer 6:16,\allowbreak17}
\crossref{Dan}{9}{7}{9:8,\allowbreak14 De 32:4 Ezr 9:13 Ne 9:33 Ps 51:4,\allowbreak14; 119:137 Jer 12:1}
\crossref{Dan}{9}{8}{9:6,\allowbreak7}
\crossref{Dan}{9}{9}{9:5 Ne 9:18,\allowbreak19,\allowbreak26-\allowbreak28 Ps 106:43-\allowbreak45 Jer 14:7 Eze 20:8,\allowbreak9,\allowbreak13}
\crossref{Dan}{9}{10}{9:6 2Ki 17:13; 18:12 Ezr 9:10,\allowbreak11 Ne 9:13-\allowbreak17 Heb 1:1}
\crossref{Dan}{9}{11}{2Ki 17:18-\allowbreak23 Isa 1:4-\allowbreak6 Jer 8:5-\allowbreak10; 9:26 Eze 22:26-\allowbreak31}
\crossref{Dan}{9}{12}{Isa 44:26 La 2:17 Eze 13:6 Zec 1:8 Mt 5:18 Ro 15:8}
\crossref{Dan}{9}{13}{9:11 Le 26:14-\allowbreak46 De 28:15-\allowbreak68 Isa 42:9 La 2:15-\allowbreak17 Joh 10:35}
\crossref{Dan}{9}{14}{Jer 31:28; 44:27}
\crossref{Dan}{9}{15}{Ex 6:1,\allowbreak6; 14:1-\allowbreak15:27; 32:11 1Ki 8:51 Ne 1:10 Jer 32:20-\allowbreak23}
\crossref{Dan}{9}{16}{1Sa 2:7 Ne 9:8 Ps 31:1; 71:2; 143:1 Mic 6:4,\allowbreak5 2Th 1:6 1Jo 1:9}
\crossref{Dan}{9}{17}{Nu 6:23-\allowbreak26 Ps 4:6; 67:1; 80:1,\allowbreak3,\allowbreak7,\allowbreak19; 119:135 Re 21:23}
\crossref{Dan}{9}{18}{1Ki 8:29 2Ki 19:16 Ps 17:6,\allowbreak7 Isa 37:17; 63:15-\allowbreak19; 64:12}
\crossref{Dan}{9}{19}{Nu 14:19 1Ki 8:30-\allowbreak39 2Ch 6:21,\allowbreak25-\allowbreak30,\allowbreak39 Am 7:2 Lu 11:8}
\crossref{Dan}{9}{20}{Da 10:2 Ps 32:5; 145:18 Isa 58:9; 65:24 Ac 4:31; 10:30,\allowbreak31}
\crossref{Dan}{9}{21}{Da 8:16; 10:16 Lu 1:19}
\crossref{Dan}{9}{22}{9:24-\allowbreak27; 8:16; 10:21 Zec 1:9,\allowbreak14; 6:4,\allowbreak5 Re 4:1}
\crossref{Dan}{9}{23}{Da 10:12}
\crossref{Dan}{9}{24}{Mt 1:21 1Jo 3:8}
\crossref{Dan}{9}{25}{9:23 Mt 13:23; 24:15 Mr 13:14 Ac 8:30}
\crossref{Dan}{9}{26}{Ps 22:15 Isa 53:8 Mr 9:12 Lu 24:26,\allowbreak46 Joh 11:51,\allowbreak52; 12:32-\allowbreak34}
\crossref{Dan}{9}{27}{Isa 42:6; 53:11; 55:3 Jer 31:31-\allowbreak34; 32:40-\allowbreak42 Eze 16:60-\allowbreak63}
\crossref{Dan}{10}{1}{Da 1:21; 6:28 2Ch 36:22,\allowbreak23 Ezr 1:1,\allowbreak2,\allowbreak7,\allowbreak8; 3:7; 4:3,\allowbreak5; 5:13-\allowbreak17; 6:3,\allowbreak14}
\crossref{Dan}{10}{2}{Ezr 9:4,\allowbreak5 Ne 1:4 Ps 42:9; 43:2; 137:1-\allowbreak5 Isa 66:10 Jer 9:1}
\crossref{Dan}{10}{3}{Da 6:18 Isa 24:6-\allowbreak11 1Co 9:27}
\crossref{Dan}{10}{4}{Da 8:2 Eze 1:3}
\crossref{Dan}{10}{5}{Da 12:6,\allowbreak7 Jos 5:13 Zec 1:8 Re 1:13-\allowbreak15}
\crossref{Dan}{10}{6}{Ex 28:20 Eze 1:16; 10:9 Re 21:20}
\crossref{Dan}{10}{7}{2Ki 6:17 Ac 9:7; 22:9}
\crossref{Dan}{10}{8}{Ge 32:24 Ex 3:3 Joh 16:32 2Co 12:2,\allowbreak3}
\crossref{Dan}{10}{9}{Da 8:18 Ge 2:21; 15:12 Job 4:13; 33:15 So 5:2 Lu 9:32; 22:45}
\crossref{Dan}{10}{10}{10:16,\allowbreak18; 8:18; 9:21 Jer 1:9 Re 1:17}
\crossref{Dan}{10}{11}{Da 9:23 Joh 13:23; 21:20}
\crossref{Dan}{10}{12}{10:19 Isa 35:4; 41:10,\allowbreak14 Mt 28:5,\allowbreak10 Mr 16:6 Lu 1:13,\allowbreak30; 2:10; 24:38}
\crossref{Dan}{10}{13}{10:20 Ezr 4:4-\allowbreak6,\allowbreak24 Zec 3:1,\allowbreak2 Eph 6:12 1Th 2:18}
\crossref{Dan}{10}{14}{Da 2:28 Ge 49:1 De 4:30; 31:21 Isa 2:2 Ho 3:5 Mic 4:1 2Ti 3:1}
\crossref{Dan}{10}{15}{10:9; 8:18 Eze 24:27; 33:22 Lu 1:20}
\crossref{Dan}{10}{16}{10:5,\allowbreak6,\allowbreak18; 8:15; 9:21 Eze 1:26 Php 2:7,\allowbreak8 Re 1:13}
\crossref{Dan}{10}{17}{Mt 22:43,\allowbreak44 Mr 12:36}
\crossref{Dan}{10}{18}{10:10,\allowbreak16; 8:18}
\crossref{Dan}{10}{19}{10:11; 9:23 Joh 11:3,\allowbreak5,\allowbreak36; 15:9-\allowbreak14; 19:26; 21:20}
\crossref{Dan}{10}{20}{10:13 Isa 37:36 Ac 12:23}
\crossref{Dan}{10}{21}{Da 8:26; 11:1-\allowbreak12:13 Isa 41:22,\allowbreak23; 43:8,\allowbreak9 Am 3:7 Ac 15:15,\allowbreak18}
\crossref{Dan}{11}{1}{Da 5:31; 9:1}
\crossref{Dan}{11}{2}{Da 8:26; 10:1,\allowbreak21 Pr 22:21 Am 3:7 Joh 10:35; 18:37,\allowbreak38 Re 21:5}
\crossref{Dan}{11}{3}{11:16,\allowbreak36; 4:35; 5:19; 8:4-\allowbreak14 Eph 1:11 Heb 2:4 Jas 1:18}
\crossref{Dan}{11}{4}{Job 20:5-\allowbreak7 Ps 37:35,\allowbreak36; 49:6-\allowbreak12; 73:17-\allowbreak20 Lu 12:20}
\crossref{Dan}{11}{5}{11:3,\allowbreak4}
\crossref{Dan}{11}{6}{11:13 Eze 38:8,\allowbreak9}
\crossref{Dan}{11}{7}{Job 14:7 Isa 9:14; 11:1 Jer 12:2 Mal 4:1}
\crossref{Dan}{11}{8}{Ge 31:30 Ex 12:12 Nu 33:4 De 12:3 Jud 18:24 Isa 37:19; 46:1,\allowbreak2}
\crossref{Dan}{11}{9}{}
\crossref{Dan}{11}{10}{11:22,\allowbreak40; 9:26 Isa 8:7,\allowbreak8 Jer 46:7,\allowbreak8; 51:42}
\crossref{Dan}{11}{11}{11:5,\allowbreak9}
\crossref{Dan}{11}{12}{Da 5:19,\allowbreak20,\allowbreak23; 8:25 De 8:14 2Ki 14:10 2Ch 25:19; 26:16; 32:25}
\crossref{Dan}{11}{13}{11:6,\allowbreak7}
\crossref{Dan}{11}{14}{Ac 4:25-\allowbreak28 Re 17:17}
\crossref{Dan}{11}{15}{Jer 5:10; 6:6; 33:4; 52:4 Eze 17:17}
\crossref{Dan}{11}{16}{11:3,\allowbreak36; 8:4,\allowbreak7}
\crossref{Dan}{11}{17}{11:19 2Ki 12:17 2Ch 20:3 Pr 19:21 Eze 4:3,\allowbreak7; 25:2 Lu 9:51}
\crossref{Dan}{11}{18}{Ge 10:4,\allowbreak5 Jer 2:10; 31:10 Eze 27:6 Zep 2:11}
\crossref{Dan}{11}{19}{}
\crossref{Dan}{11}{20}{11:7,\allowbreak21}
\crossref{Dan}{11}{21}{11:7,\allowbreak20}
\crossref{Dan}{11}{22}{11:10; 9:26 Isa 8:7,\allowbreak8 Am 8:8; 9:5 Na 1:8 Re 12:15,\allowbreak16}
\crossref{Dan}{11}{23}{Da 8:25 Ge 34:13 Ps 52:2 Pr 11:18 Eze 17:13-\allowbreak19 Ro 1:29 2Co 11:3}
\crossref{Dan}{11}{24}{Da 7:25 Pr 23:7 Eze 38:10 Mt 9:4}
\crossref{Dan}{11}{25}{11:2,\allowbreak10 Pr 15:18; 28:25}
\crossref{Dan}{11}{26}{2Sa 4:2-\allowbreak12 2Ki 8:14; 10:6-\allowbreak9 Ps 41:9 Mic 7:5,\allowbreak6 Mt 26:23 Mr 14:20}
\crossref{Dan}{11}{27}{2Sa 13:26 Ps 12:2; 52:1; 58:2; 64:6 Pr 12:20; 23:6-\allowbreak8; 26:23}
\crossref{Dan}{11}{28}{11:22,\allowbreak30-\allowbreak32; 8:24 Ac 3:25}
\crossref{Dan}{11}{29}{Da 8:19; 10:1 Isa 14:31 Ac 17:26 Ga 4:2}
\crossref{Dan}{11}{30}{11:28; 7:25 Re 12:12}
\crossref{Dan}{11}{31}{Da 8:24 Re 17:12}
\crossref{Dan}{11}{32}{Pr 19:5; 26:28}
\crossref{Dan}{11}{33}{Da 12:3,\allowbreak4,\allowbreak10 Isa 32:3,\allowbreak4 Zec 8:20-\allowbreak23 Mal 2:7 Mt 13:11,\allowbreak51,\allowbreak52; 28:20}
\crossref{Dan}{11}{34}{Re 12:2-\allowbreak6,\allowbreak13-\allowbreak17; 13:1-\allowbreak4}
\crossref{Dan}{11}{35}{11:33; 8:10 Mt 16:17; 26:56,\allowbreak69-\allowbreak75 Joh 20:25 Ac 13:13; 15:37-\allowbreak39}
\crossref{Dan}{11}{36}{11:16; 8:4 Joh 5:30; 6:38}
\crossref{Dan}{11}{37}{Ge 3:16 De 5:21; 21:11 So 7:10 Eze 24:16 1Ti 4:3}
\crossref{Dan}{11}{38}{}
\crossref{Dan}{11}{39}{}
\crossref{Dan}{11}{40}{11:35; 8:17; 12:4}
\crossref{Dan}{11}{41}{11:45 Eze 38:8-\allowbreak13}
\crossref{Dan}{11}{42}{Eze 29:14 Zec 10:10,\allowbreak11; 14:17 Re 11:8}
\crossref{Dan}{11}{43}{Jer 46:9,\allowbreak10 Eze 38:5}
\crossref{Dan}{11}{44}{11:11,\allowbreak30 Eze 38:9-\allowbreak12 Re 16:12; 17:13; 19:19-\allowbreak21}
\crossref{Dan}{11}{45}{Joe 2:20 Zec 14:8}
\crossref{Dan}{12}{1}{Da 11:35,\allowbreak 45}
\crossref{Dan}{12}{2}{Job 19:25-\allowbreak27 Isa 26:19 Eze 37:1-\allowbreak4,\allowbreak12 Ho 13:14 Mt 22:29-\allowbreak32}
\crossref{Dan}{12}{3}{Da 11:33,\allowbreak35 Pr 11:30 Mt 24:45 1Co 3:10 2Pe 3:15}
\crossref{Dan}{12}{4}{Da 8:26 Re 10:4; 22:10}
\crossref{Dan}{12}{5}{Da 10:5,\allowbreak6,\allowbreak10,\allowbreak16}
\crossref{Dan}{12}{6}{Da 8:16 Zec 1:12,\allowbreak13 Eph 3:10 1Pe 1:12}
\crossref{Dan}{12}{7}{De 32:40 Re 10:5,\allowbreak7}
\crossref{Dan}{12}{8}{Lu 18:34 Joh 12:16 Ac 1:7 1Pe 1:11}
\crossref{Dan}{12}{9}{12:13}
\crossref{Dan}{12}{10}{Da 11:35 Ps 51:7 Isa 1:18 Eze 36:25 Zec 13:9 1Co 6:11 2Co 7:1}
\crossref{Dan}{12}{11}{Da 8:11,\allowbreak12,\allowbreak26; 11:31}
\crossref{Dan}{12}{12}{Ro 11:15 Re 20:4}
\crossref{Dan}{12}{13}{12:9}

% Hos
\crossref{Hos}{1}{1}{Jer 1:2,\allowbreak4 Eze 1:3 Joe 1:1 Jon 1:1 Zec 1:1 Joh 10:35 2Pe 1:21}
\crossref{Hos}{1}{2}{Mr 1:1}
\crossref{Hos}{1}{3}{Isa 8:1-\allowbreak3}
\crossref{Hos}{1}{4}{1:6,\allowbreak9 Isa 7:14; 9:6 Mt 1:21 Lu 1:13,\allowbreak31,\allowbreak63 Joh 1:42}
\crossref{Hos}{1}{5}{Ho 2:18 Ps 37:15; 46:9 Jer 49:34,\allowbreak35; 51:56}
\crossref{Hos}{1}{6}{Ho 2:23 1Pe 2:10}
\crossref{Hos}{1}{7}{Ho 11:12 2Ki 19:35 Isa 36:1-\allowbreak37:38}
\crossref{Hos}{1}{8}{1Sa 1:22}
\crossref{Hos}{1}{9}{Jer 15:1}
\crossref{Hos}{1}{10}{Ge 13:16; 32:12 Isa 48:19 Ro 9:27,\allowbreak28 Heb 11:12}
\crossref{Hos}{1}{11}{Ho 3:5,\allowbreak Isa 11:12,\allowbreak13 Jer 3:18,\allowbreak19; 23:5-\allowbreak8; 30:3; 31:1-\allowbreak9,\allowbreak 33:15-\allowbreak26}
\crossref{Hos}{2}{1}{Ho 1:9-\allowbreak11}
\crossref{Hos}{2}{2}{Ho 1:9-\allowbreak11}
\crossref{Hos}{2}{3}{Isa 58:1 Jer 2:2; 19:3 Eze 20:4; 23:45 Mt 23:37-\allowbreak39 Ac 7:51-\allowbreak53}
\crossref{Hos}{2}{4}{Isa 58:1 Jer 2:2; 19:3 Eze 20:4; 23:45 Mt 23:37-\allowbreak39 Ac 7:51-\allowbreak53}
\crossref{Hos}{2}{5}{2:10 Isa 47:3 Jer 13:22,\allowbreak26 Eze 16:37-\allowbreak39; 23:26-\allowbreak29 Re 17:16}
\crossref{Hos}{2}{6}{Ho 1:6 Isa 27:11 Jer 13:14; 16:5 Eze 8:18; 9:10 Zec 1:12 Ro 9:18}
\crossref{Hos}{2}{7}{2:2; 3:1; 4:5,\allowbreak12-\allowbreak15 Isa 1:21; 50:1 Jer 2:20,\allowbreak25; 3:1-\allowbreak9 Eze 16:15,\allowbreak16}
\crossref{Hos}{2}{8}{Job 3:23; 19:8 La 3:7-\allowbreak9 Lu 15:14-\allowbreak16; 19:43}
\crossref{Hos}{2}{9}{Ho 5:13 2Ch 28:20-\allowbreak22 Isa 30:2,\allowbreak3,\allowbreak16; 31:1-\allowbreak3 Jer 2:28,\allowbreak36,\allowbreak37; 30:12-\allowbreak15}
\crossref{Hos}{2}{10}{Isa 1:3 Hab 1:16 Ac 17:23-\allowbreak25 Ro 1:28}
\crossref{Hos}{2}{11}{Da 11:13 Joe 2:14 Mal 1:4; 3:18}
\crossref{Hos}{2}{12}{2:3 Isa 3:17 Jer 13:22,\allowbreak26 Eze 16:36; 23:29 Lu 12:2,\allowbreak3 1Co 4:5}
\crossref{Hos}{2}{13}{Ho 9:1-\allowbreak5 Isa 24:7-\allowbreak11 Jer 7:34; 16:9; 25:10 Eze 26:13 Na 1:10}
\crossref{Hos}{2}{14}{2:5; 9:1}
\crossref{Hos}{2}{15}{Ho 9:7,\allowbreak9 Ex 32:34 Jer 23:2}
\crossref{Hos}{2}{16}{Isa 30:18 Jer 16:14}
\crossref{Hos}{2}{17}{2:12 Le 26:40-\allowbreak45 De 30:3-\allowbreak5 Ne 1:8,\allowbreak9 Isa 65:21 Jer 32:15}
\crossref{Hos}{2}{18}{2:7 Isa 54:5 Jer 3:14 Joh 3:29 2Co 11:2 Eph 5:25-\allowbreak27 Re 19:7}
\crossref{Hos}{2}{19}{Ex 23:13 Jos 23:7 Ps 16:4 Zec 13:2}
\crossref{Hos}{2}{20}{Isa 2:11,\allowbreak17; 26:1 Zec 2:11; 14:4,\allowbreak9}
\crossref{Hos}{2}{21}{Isa 54:5; 62:3-\allowbreak5 Jer 3:14,\allowbreak15 Joh 3:29 Ro 7:4 2Co 11:2}
\crossref{Hos}{2}{22}{Jer 9:24; 24:7; 31:33,\allowbreak34 Eze 38:23 Mt 11:27 Lu 10:22 Joh 8:55}
\crossref{Hos}{2}{23}{Isa 65:24 Zec 8:12; 13:9 Mt 6:33 Ro 8:32 1Co 3:21-\allowbreak23}
\crossref{Hos}{3}{1}{Ho 1:2,\allowbreak3}
\crossref{Hos}{3}{2}{Ge 31:41; 34:12 Ex 22:17 1Sa 18:25}
\crossref{Hos}{3}{3}{De 21:13}
\crossref{Hos}{3}{4}{Ho 10:3 Ge 49:10 Jer 15:4,\allowbreak5 Joh 19:15}
\crossref{Hos}{3}{5}{Ho 5:6,\allowbreak15 Isa 27:12,\allowbreak13 Jer 3:22,\allowbreak23; 31:6-\allowbreak10; 50:4,\allowbreak5}
\crossref{Hos}{4}{1}{1Ki 22:19 Isa 1:10; 28:14; 34:1; 66:5 Jer 2:4; 7:2; 9:20; 19:3; 34:4}
\crossref{Hos}{4}{2}{Isa 24:5; 48:1; 59:2-\allowbreak8,\allowbreak12-\allowbreak15 Jer 5:1,\allowbreak2,\allowbreak7-\allowbreak9,\allowbreak26,\allowbreak27; 6:7; 7:6-\allowbreak10}
\crossref{Hos}{4}{3}{Isa 24:4-\allowbreak12 Jer 4:27 Joe 1:10-\allowbreak13 Am 1:2; 5:16; 8:8 Na 1:4}
\crossref{Hos}{4}{4}{4:17 Am 5:13; 6:10 Mt 7:3-\allowbreak6}
\crossref{Hos}{4}{5}{Ho 9:7,\allowbreak8 Isa 9:13-\allowbreak17 Jer 6:4,\allowbreak5,\allowbreak12-\allowbreak15; 8:10-\allowbreak12; 14:15,\allowbreak16}
\crossref{Hos}{4}{6}{4:12 Isa 1:3; 3:12; 5:13 Jer 4:22; 8:7}
\crossref{Hos}{4}{7}{4:10; 5:1; 6:9; 13:6,\allowbreak14 Ezr 9:7}
\crossref{Hos}{4}{8}{Le 6:26; 7:6,\allowbreak7}
\crossref{Hos}{4}{9}{Isa 9:14-\allowbreak16; 24:2 Jer 5:31; 8:10-\allowbreak12; 23:11,\allowbreak12 Eze 22:26-\allowbreak31}
\crossref{Hos}{4}{10}{Le 26:26 Pr 13:25 Isa 65:13-\allowbreak16 Mic 6:14 Hag 1:6 Mal 2:1-\allowbreak3}
\crossref{Hos}{4}{11}{4:12 Pr 6:32; 20:1; 23:27-\allowbreak35 Ec 7:7 Isa 5:12; 28:7 Lu 21:34}
\crossref{Hos}{4}{12}{Jer 2:27; 10:8 Eze 21:21 Hab 2:19}
\crossref{Hos}{4}{13}{Isa 1:29; 57:5,\allowbreak7 Jer 3:6,\allowbreak13 Eze 6:13; 16:16,\allowbreak25; 20:28,\allowbreak29}
\crossref{Hos}{4}{14}{4:17 Isa 1:5 Heb 12:8}
\crossref{Hos}{4}{15}{4:12 Jer 3:6-\allowbreak10 Eze 23:4-\allowbreak8}
\crossref{Hos}{4}{16}{Ho 11:7 1Sa 15:11 Jer 3:6,\allowbreak8,\allowbreak11; 5:6; 7:24; 8:5; 14:7 Zec 7:11}
\crossref{Hos}{4}{17}{Ho 11:2; 12:1; 13:2}
\crossref{Hos}{4}{18}{De 32:32,\allowbreak33 Isa 1:21,\allowbreak22 Jer 2:21}
\crossref{Hos}{4}{19}{Jer 4:11,\allowbreak12; 51:1 Zec 5:9-\allowbreak11}
\crossref{Hos}{5}{1}{Ho 4:1,\allowbreak6,\allowbreak7; 6:9 Mal 1:6; 2:1}
\crossref{Hos}{5}{2}{Ho 6:9; 9:15 Jer 6:28}
\crossref{Hos}{5}{3}{Am 3:2 Heb 4:13 Re 3:15}
\crossref{Hos}{5}{4}{Ho 4:12 Jer 50:38}
\crossref{Hos}{5}{5}{Ho 7:10 Pr 30:13 Isa 3:9; 9:9,\allowbreak10; 28:1-\allowbreak3}
\crossref{Hos}{5}{6}{Ex 10:9,\allowbreak24-\allowbreak26 Pr 15:8; 21:27 Jer 7:4 Mic 6:6,\allowbreak7}
\crossref{Hos}{5}{7}{Ho 6:7 Isa 48:8; 59:13 Jer 3:20; 5:11}
\crossref{Hos}{5}{8}{Ho 8:1 Jer 4:5; 6:1 Joe 2:1,\allowbreak15}
\crossref{Hos}{5}{9}{5:12,\allowbreak14; 8:8; 9:11-\allowbreak17; 11:5,\allowbreak6; 13:1-\allowbreak3,\allowbreak15,\allowbreak16 Job 12:14 Isa 28:1-\allowbreak4}
\crossref{Hos}{5}{10}{5:5}
\crossref{Hos}{5}{11}{De 28:33 2Ki 15:16-\allowbreak20,\allowbreak29 Am 5:11,\allowbreak12}
\crossref{Hos}{5}{12}{Job 13:28 Isa 50:9; 51:8}
\crossref{Hos}{5}{13}{Jer 30:12,\allowbreak14 Mic 1:9}
\crossref{Hos}{5}{14}{Ho 13:7,\allowbreak8 Job 10:16 Ps 7:2 La 3:10 Am 3:4-\allowbreak8}
\crossref{Hos}{5}{15}{5:6 Ex 25:21,\allowbreak22 1Ki 8:10-\allowbreak13 Ps 132:14 Isa 26:21 Eze 8:6; 10:4}
\crossref{Hos}{6}{1}{Ho 5:15; 14:1 Isa 2:3; 55:7 Jer 3:22; 50:4 La 3:40,\allowbreak41 Zep 2:1}
\crossref{Hos}{6}{2}{Ho 13:14 2Ki 20:5 Ps 30:4 Isa 26:19 Eze 37:11-\allowbreak13 1Co 15:4}
\crossref{Hos}{6}{3}{Ho 2:20 Isa 54:13 Jer 24:7 Mic 4:2 Joh 17:3}
\crossref{Hos}{6}{4}{Ho 11:8 Isa 5:3,\allowbreak4 Jer 3:19; 5:7,\allowbreak9,\allowbreak23; 9:7 Lu 13:7-\allowbreak9; 19:41,\allowbreak42}
\crossref{Hos}{6}{5}{1Sa 13:13; 15:22 1Ki 14:6; 17:1; 18:17 2Ki 1:16 2Ch 21:12}
\crossref{Hos}{6}{6}{1Sa 15:22 Ps 50:8 Pr 21:3 Ec 5:1 Isa 1:11; 58:6 Jer 7:22}
\crossref{Hos}{6}{7}{Ge 3:6,\allowbreak11 Job 31:33}
\crossref{Hos}{6}{8}{Ho 12:11 Jos 21:38}
\crossref{Hos}{6}{9}{Ho 7:1 Ezr 8:31 Job 1:15-\allowbreak17 Pr 1:11-\allowbreak19}
\crossref{Hos}{6}{10}{Jer 2:12,\allowbreak13; 5:30,\allowbreak31; 18:13; 23:14}
\crossref{Hos}{6}{11}{Jer 51:33 Joe 3:13 Mic 4:12 Re 14:15}
\crossref{Hos}{7}{1}{Jer 51:9 Mt 23:37 Lu 13:34; 19:42}
\crossref{Hos}{7}{2}{De 32:29 Ps 50:22 Isa 1:3; 5:12; 44:19}
\crossref{Hos}{7}{3}{Ho 5:11 1Ki 22:6,\allowbreak13 Jer 5:31; 9:2; 28:1-\allowbreak4; 37:19 Am 7:10-\allowbreak13 Mic 6:16}
\crossref{Hos}{7}{4}{Ho 4:2,\allowbreak12 Jer 5:7,\allowbreak8; 9:2 Jas 4:4}
\crossref{Hos}{7}{5}{Ge 40:20 Da 5:1-\allowbreak4 Mt 14:6 Mr 6:21}
\crossref{Hos}{7}{6}{7:4,\allowbreak7 1Sa 19:11-\allowbreak15 2Sa 13:28,\allowbreak29 Ps 10:8,\allowbreak9 Pr 4:16 Mic 2:1}
\crossref{Hos}{7}{7}{Ho 8:4 1Ki 15:28; 16:9-\allowbreak11,\allowbreak18,\allowbreak22 2Ki 9:24,\allowbreak33; 10:7,\allowbreak14}
\crossref{Hos}{7}{8}{Ho 5:7,\allowbreak13; 9:3 Ezr 9:1,\allowbreak12 Ne 13:23-\allowbreak25 Ps 106:35 Eze 23:4-\allowbreak11}
\crossref{Hos}{7}{9}{Ho 8:7 2Ki 13:3-\allowbreak7,\allowbreak22; 15:19 Pr 23:35 Isa 42:22-\allowbreak25; 57:1}
\crossref{Hos}{7}{10}{Ho 5:5 Jer 3:3}
\crossref{Hos}{7}{11}{Ho 11:11}
\crossref{Hos}{7}{12}{Job 19:6 Jer 16:16 Eze 12:13; 17:20; 32:3}
\crossref{Hos}{7}{13}{Ho 9:12 Isa 31:1 La 5:16 Eze 16:23 Mt 23:13-\allowbreak29 Re 8:13}
\crossref{Hos}{7}{14}{Job 35:9,\allowbreak10 Ps 78:34-\allowbreak37 Isa 29:13 Jer 3:10 Zec 7:5}
\crossref{Hos}{7}{15}{2Ki 13:5,\allowbreak23; 14:25-\allowbreak27 Ps 106:43-\allowbreak45}
\crossref{Hos}{7}{16}{Ho 6:4; 8:14; 11:7 Ps 78:37 Jer 3:10 Lu 8:13; 11:24-\allowbreak26}
\crossref{Hos}{8}{1}{Ho 5:8 Isa 18:3; 58:1 Jer 4:5; 6:1; 51:27 Eze 7:14; 33:3-\allowbreak6 Joe 2:1,\allowbreak15}
\crossref{Hos}{8}{2}{Ho 5:15; 7:13,\allowbreak14 2Ki 10:16,\allowbreak29 Ps 78:34-\allowbreak37 Isa 48:1,\allowbreak2 Jer 7:4}
\crossref{Hos}{8}{3}{Ps 36:3; 81:10,\allowbreak11 Am 1:11 1Ti 5:12}
\crossref{Hos}{8}{4}{1Ki 12:16-\allowbreak20 2Ki 15:10-\allowbreak30}
\crossref{Hos}{8}{5}{8:6; 10:5 Isa 45:20 Ac 7:41}
\crossref{Hos}{8}{6}{Ps 106:19,\allowbreak20}
\crossref{Hos}{8}{7}{Ho 10:12 Job 4:8 Pr 22:8 Ec 5:16 Ga 6:7}
\crossref{Hos}{8}{8}{2Ki 17:1-\allowbreak6; 18:11 Jer 50:17; 51:34 La 2:2,\allowbreak5,\allowbreak16 Eze 36:3}
\crossref{Hos}{8}{9}{Ho 5:13; 7:11 2Ki 15:19 Eze 23:5-\allowbreak9}
\crossref{Hos}{8}{10}{Ho 10:10 Eze 16:37; 23:9,\allowbreak10,\allowbreak22-\allowbreak26,\allowbreak46,\allowbreak47}
\crossref{Hos}{8}{11}{Ho 10:1,\allowbreak2,\allowbreak8; 12:11 Isa 10:10,\allowbreak11}
\crossref{Hos}{8}{12}{De 4:6-\allowbreak8 Ne 9:13,\allowbreak14 Ps 119:18; 147:19,\allowbreak20 Pr 22:20 Eze 20:11}
\crossref{Hos}{8}{13}{Ho 5:6; 9:4; 12:11 1Sa 15:22 Pr 21:27 Isa 1:11 Jer 14:10 Am 5:22}
\crossref{Hos}{8}{14}{Ho 13:6 De 32:18 Ps 106:21 Isa 17:10 Jer 2:32; 3:21; 23:27}
\crossref{Hos}{9}{1}{Ho 10:5 Isa 17:11; 22:12 La 4:21 Eze 21:10 Am 6:6,\allowbreak7,\allowbreak13; 8:10}
\crossref{Hos}{9}{2}{Ho 2:9,\allowbreak12 Isa 24:7-\allowbreak12 Joe 1:3-\allowbreak7,\allowbreak9-\allowbreak13 Am 4:6-\allowbreak9,\allowbreak5-\allowbreak11 Mic 6:13-\allowbreak16}
\crossref{Hos}{9}{3}{Le 18:28; 20:22 De 4:26; 28:63 Jos 23:15 1Ki 9:7 Mic 2:10}
\crossref{Hos}{9}{4}{Ho 3:4 Joe 1:13; 2:14}
\crossref{Hos}{9}{5}{Isa 10:3 Jer 5:31}
\crossref{Hos}{9}{6}{De 28:63,\allowbreak64 1Sa 13:6 2Ki 13:7}
\crossref{Hos}{9}{7}{Isa 10:3 Jer 10:15; 11:23; 46:21 Eze 7:2-\allowbreak7; 12:22-\allowbreak28 Am 8:2}
\crossref{Hos}{9}{8}{So 3:3 Isa 62:6 Jer 6:17; 31:6 Eze 3:17; 33:7 Mic 7:4 Heb 13:17}
\crossref{Hos}{9}{9}{Isa 24:5; 31:6}
\crossref{Hos}{9}{10}{Ho 11:1 Ex 19:4-\allowbreak6 De 32:10 Jer 2:2,\allowbreak3; 31:2}
\crossref{Hos}{9}{11}{Ge 41:52; 48:16-\allowbreak20; 49:22 De 33:17 Job 18:5,\allowbreak18,\allowbreak19}
\crossref{Hos}{9}{12}{9:13,\allowbreak16 De 28:32,\allowbreak41,\allowbreak42; 32:25 Job 27:14 Jer 15:7; 16:3,\allowbreak4 La 2:20}
\crossref{Hos}{9}{13}{Eze 26:1-\allowbreak28:26}
\crossref{Hos}{9}{14}{9:13,\allowbreak16 Mt 24:19 Mr 13:17 Lu 21:23; 23:29 1Co 7:26}
\crossref{Hos}{9}{15}{Ho 4:15; 12:11 Jos 4:19-\allowbreak24; 5:2-\allowbreak9; 10:43 1Sa 7:16 Am 4:4; 5:5 Mic 6:5}
\crossref{Hos}{9}{16}{9:11-\allowbreak13 Job 18:16 Isa 5:24; 40:24 Mal 4:1}
\crossref{Hos}{9}{17}{2Ch 18:13 Ne 5:19 Ps 31:14 Isa 7:13 Mic 7:7 Joh 20:17,\allowbreak28}
\crossref{Hos}{10}{1}{Isa 5:1-\allowbreak7 Eze 15:1-\allowbreak5 Na 2:2 Joh 15:1-\allowbreak6}
\crossref{Hos}{10}{2}{Ho 7:8 1Ki 18:21 Isa 44:18 Zep 1:5 Mt 6:24 Lu 16:13 2Th 2:11,\allowbreak12}
\crossref{Hos}{10}{3}{10:7,\allowbreak15; 3:4; 11:5; 13:11 Ge 49:10 Mic 4:9 Joh 19:15}
\crossref{Hos}{10}{4}{Ho 6:7 2Ki 17:3,\allowbreak4 Eze 17:13-\allowbreak19 Ro 1:31 2Ti 3:3}
\crossref{Hos}{10}{5}{Ho 8:5,\allowbreak6; 13:2 1Ki 12:28-\allowbreak32 2Ki 10:29; 17:16 2Ch 11:15; 13:8}
\crossref{Hos}{10}{6}{Ho 8:6 Isa 46:1,\allowbreak2 Jer 43:12,\allowbreak13 Da 11:8}
\crossref{Hos}{10}{7}{1Ki 21:1 2Ki 1:3}
\crossref{Hos}{10}{8}{10:5; 4:15; 5:8}
\crossref{Hos}{10}{9}{Ho 9:9 Jud 19:22-\allowbreak30; 20:5,\allowbreak13,\allowbreak14}
\crossref{Hos}{10}{10}{De 28:63 Isa 1:24 Jer 15:6 Eze 5:13; 16:42}
\crossref{Hos}{10}{11}{Ho 4:16 Jer 50:11}
\crossref{Hos}{10}{12}{Ho 8:7 Ps 126:5,\allowbreak6 Pr 11:18; 18:21 Ec 11:6 Isa 32:20 Jas 3:18}
\crossref{Hos}{10}{13}{Ho 8:7 Job 4:8 Pr 22:8 Ga 6:7,\allowbreak8}
\crossref{Hos}{10}{14}{Ho 13:16 Isa 22:1-\allowbreak4; 33:14 Am 3:8; 9:5}
\crossref{Hos}{10}{15}{10:5 Am 7:9-\allowbreak17}
\crossref{Hos}{11}{1}{Ho 2:15 De 7:7 Jer 2:2 Eze 16:6 Mal 1:2}
\crossref{Hos}{11}{2}{11:7 De 29:2-\allowbreak4 1Sa 8:7-\allowbreak9 2Ki 17:13-\allowbreak15 2Ch 36:15,\allowbreak16 Ne 9:30}
\crossref{Hos}{11}{3}{Ex 19:4 Nu 11:11,\allowbreak12 De 1:31; 8:2; 32:10-\allowbreak12 Isa 46:3; 63:9}
\crossref{Hos}{11}{4}{So 1:4 Isa 63:9 Joh 6:44; 12:32 2Co 5:14}
\crossref{Hos}{11}{5}{Ho 7:16; 8:13; 9:3,\allowbreak6}
\crossref{Hos}{11}{6}{Ho 10:14; 13:16 Le 26:31,\allowbreak33 De 28:52; 32:25 Jer 5:17 Mic 5:11}
\crossref{Hos}{11}{7}{Ho 4:16; 14:4 Ps 78:57,\allowbreak58 Pr 14:14 Jer 3:6,\allowbreak8,\allowbreak11; 8:5; 14:7}
\crossref{Hos}{11}{8}{Ho 6:4 Jer 9:7 La 3:33 Mt 23:37 Lu 19:41,\allowbreak42}
\crossref{Hos}{11}{9}{Ho 14:4 Ex 32:10-\allowbreak14 De 32:26,\allowbreak27 Ps 78:38 Isa 27:4-\allowbreak8; 48:9}
\crossref{Hos}{11}{10}{Isa 2:5; 49:10 Jer 2:2; 7:6,\allowbreak9; 31:9 Mic 4:5 Zec 10:12 Joh 8:12}
\crossref{Hos}{11}{11}{Ho 3:5; 9:3-\allowbreak6 Isa 11:11 Zec 10:10}
\crossref{Hos}{11}{12}{Ho 7:16; 12:1,\allowbreak7 Ps 78:36 Isa 29:13; 44:20; 59:3,\allowbreak4 Mic 6:12}
\crossref{Hos}{12}{1}{Ho 8:7 Job 15:2 Jer 22:22}
\crossref{Hos}{12}{2}{Ho 4:1 Jer 25:31 Mic 6:2}
\crossref{Hos}{12}{3}{Ge 25:26 Ro 9:11}
\crossref{Hos}{12}{4}{Ge 32:29; 48:15 Ex 3:2-\allowbreak5 Isa 63:9 Mal 3:1 Ac 7:30-\allowbreak35}
\crossref{Hos}{12}{5}{Ge 28:16; 32:30}
\crossref{Hos}{12}{6}{Ho 14:1 Pr 1:23 Isa 31:6; 55:6,\allowbreak7 Jer 3:14-\allowbreak22 La 3:39-\allowbreak41 Joe 2:13}
\crossref{Hos}{12}{7}{Eze 16:3 Zec 14:21 Joh 2:16}
\crossref{Hos}{12}{8}{Job 31:24,\allowbreak25 Ps 49:6; 52:7; 62:10 Zec 11:5 Lu 12:19; 16:13}
\crossref{Hos}{12}{9}{Ho 13:4 Ex 20:2 Le 19:36; 26:13 Nu 15:41 Ps 81:10 Mic 6:4}
\crossref{Hos}{12}{10}{1Ki 13:1; 14:7-\allowbreak16; 17:1; 18:21-\allowbreak40; 19:10 2Ki 17:13 Ne 9:30}
\crossref{Hos}{12}{11}{Ho 6:8 1Ki 17:1}
\crossref{Hos}{12}{12}{Ge 27:43; 28:1-\allowbreak29:35 De 26:5}
\crossref{Hos}{12}{13}{Ho 13:4,\allowbreak5 Ex 12:50,\allowbreak51; 13:3 1Sa 12:8 Ps 77:20 Isa 63:11,\allowbreak12}
\crossref{Hos}{12}{14}{2Ki 17:7-\allowbreak18 Eze 23:2-\allowbreak10}
\crossref{Hos}{13}{1}{1Sa 15:17 Pr 18:12 Isa 66:2 Lu 14:11}
\crossref{Hos}{13}{2}{Nu 32:14 2Ch 28:13; 33:23 Isa 1:5; 30:1 Ro 2:5 2Ti 3:13}
\crossref{Hos}{13}{3}{Ho 6:4}
\crossref{Hos}{13}{4}{Ho 12:9 Ex 20:2 Ps 81:9,\allowbreak10 Isa 43:3,\allowbreak10; 44:6-\allowbreak8}
\crossref{Hos}{13}{5}{Ex 2:25 Ps 1:6; 31:7; 142:3 Na 1:7 1Co 8:3 Ga 4:9}
\crossref{Hos}{13}{6}{Ho 10:1 De 8:12-\allowbreak14; 32:13-\allowbreak15 Ne 9:25,\allowbreak26,\allowbreak35 Jer 2:31}
\crossref{Hos}{13}{7}{Ho 5:14 Isa 42:13 Jer 5:6 La 3:10 Am 1:2; 3:4,\allowbreak8}
\crossref{Hos}{13}{8}{2Sa 17:8 Pr 17:12 Am 9:1-\allowbreak3}
\crossref{Hos}{13}{9}{Ho 14:1 2Ki 17:7-\allowbreak17 Pr 6:32; 8:36 Isa 3:9,\allowbreak11 Jer 2:17,\allowbreak19; 4:18; 5:25}
\crossref{Hos}{13}{10}{13:4; 10:3 De 32:37-\allowbreak39 Jer 2:28}
\crossref{Hos}{13}{11}{Ho 10:3 1Sa 8:7-\allowbreak9; 10:19; 12:13; 15:22,\allowbreak23; 16:1; 31:1-\allowbreak7}
\crossref{Hos}{13}{12}{De 32:34,\allowbreak35 Job 14:17; 21:19 Ro 2:5}
\crossref{Hos}{13}{13}{Ps 48:6 Isa 13:8; 21:3 Jer 4:31; 13:21; 22:23; 30:6; 49:24}
\crossref{Hos}{13}{14}{Ho 6:2 Job 19:25-\allowbreak27; 33:24 Ps 16:10; 30:3; 49:15; 71:20; 86:13}
\crossref{Hos}{13}{15}{Ge 41:52; 48:19; 49:22 De 33:17}
\crossref{Hos}{13}{16}{2Ki 17:6,\allowbreak18; 19:9-\allowbreak11 Isa 7:8,\allowbreak9; 8:4; 17:3 Am 3:9-\allowbreak15; 4:1; 6:1-\allowbreak8; 9:1}
\crossref{Hos}{14}{1}{Ho 6:1; 12:6 1Sa 7:3,\allowbreak4 2Ch 30:6-\allowbreak9 Isa 55:6,\allowbreak7 Jer 3:12-\allowbreak14; 4:1}
\crossref{Hos}{14}{2}{Job 34:31,\allowbreak32 Joe 2:17 Mt 6:9-\allowbreak13 Lu 11:2-\allowbreak4; 18:13}
\crossref{Hos}{14}{3}{Ho 5:13; 7:11; 8:9; 12:1 2Ch 16:7 Ps 146:3 Jer 31:18-\allowbreak22}
\crossref{Hos}{14}{4}{Ho 11:7 Ex 15:26 Isa 57:18 Jer 3:22; 5:6; 8:22; 14:7; 17:14; 33:6}
\crossref{Hos}{14}{5}{De 32:2 2Sa 23:4 Job 29:19 Ps 72:6 Pr 19:12 Isa 18:4; 26:19}
\crossref{Hos}{14}{6}{Ps 80:9-\allowbreak11 Eze 17:5-\allowbreak8; 31:3-\allowbreak10 Da 4:10-\allowbreak15 Mt 13:31 Joh 15:1}
\crossref{Hos}{14}{7}{Ps 91:1 So 2:3 Isa 32:1,\allowbreak2}
\crossref{Hos}{14}{8}{14:2,\allowbreak3 Job 34:32 Ac 19:18-\allowbreak20 1Th 1:9 1Pe 1:14-\allowbreak16; 4:3,\allowbreak4}
\crossref{Hos}{14}{9}{Ps 107:43 Pr 1:5,\allowbreak6; 4:18 Jer 9:12 Da 12:10 Mt 13:11,\allowbreak12 Joh 8:47}

% Joel
\crossref{Joel}{1}{1}{Jer 1:2 Eze 1:3 Ho 1:1 2Pe 1:21}
\crossref{Joel}{1}{2}{Ps 49:1 Isa 34:1 Jer 5:21 Ho 5:1 Am 3:1; 4:1; 5:1 Mic 1:2}
\crossref{Joel}{1}{3}{Ex 10:1,\allowbreak2; 13:14 De 6:7 Jos 4:6,\allowbreak7,\allowbreak21,\allowbreak22 Ps 44:1; 71:18; 78:3-\allowbreak8}
\crossref{Joel}{1}{4}{Ex 10:12-\allowbreak15 De 28:38,\allowbreak42 1Ki 8:37 2Ch 6:28; 7:13}
\crossref{Joel}{1}{5}{Isa 24:7-\allowbreak11 Am 6:3-\allowbreak7 Lu 21:34-\allowbreak36 Ro 13:11-\allowbreak14}
\crossref{Joel}{1}{6}{Joe 2:2-\allowbreak11,\allowbreak25 Pr 30:25-\allowbreak27}
\crossref{Joel}{1}{7}{1:12 Ex 10:15 Ps 105:33 Isa 5:6; 24:7 Jer 8:13 Ho 2:12 Hab 3:17}
\crossref{Joel}{1}{8}{1:13-\allowbreak15; 2:12-\allowbreak14 Isa 22:12; 24:7-\allowbreak12; 32:11 Jer 9:17-\allowbreak19 Jas 4:8,\allowbreak9}
\crossref{Joel}{1}{9}{1:13,\allowbreak16; 2:14 Ho 9:4}
\crossref{Joel}{1}{10}{1:17-\allowbreak20 Le 26:20 Isa 24:3,\allowbreak4 Jer 12:4,\allowbreak11; 14:2-\allowbreak6 Ho 4:3}
\crossref{Joel}{1}{11}{Jer 14:3,\allowbreak4 Ro 5:5}
\crossref{Joel}{1}{12}{Nu 13:23 Ps 92:12 So 2:3; 4:13; 7:7-\allowbreak9}
\crossref{Joel}{1}{13}{1:8,\allowbreak9; 2:17 Jer 4:8; 9:10 Eze 7:18}
\crossref{Joel}{1}{14}{Joe 2:15,\allowbreak16 2Ch 20:3,\allowbreak4}
\crossref{Joel}{1}{15}{Joe 2:2 Jer 30:7 Am 5:16-\allowbreak18}
\crossref{Joel}{1}{16}{1:5-\allowbreak9,\allowbreak13 Am 4:6,\allowbreak7}
\crossref{Joel}{1}{17}{Ge 23:16}
\crossref{Joel}{1}{18}{1:20 1Ki 18:5 Jer 12:4; 14:5,\allowbreak6 Ho 4:3 Ro 8:22}
\crossref{Joel}{1}{19}{Ps 50:15; 91:15 Mic 7:7 Hab 3:17,\allowbreak18 Lu 18:1,\allowbreak7 Php 4:6,\allowbreak7}
\crossref{Joel}{1}{20}{Job 38:41 Ps 104:21; 145:15; 147:9}
\crossref{Joel}{2}{1}{2:15 Nu 10:3,\allowbreak8 Jer 4:5 Ho 8:1}
\crossref{Joel}{2}{2}{Am 4:13}
\crossref{Joel}{2}{3}{Joe 1:19,\allowbreak20 Ps 50:3 Am 7:4}
\crossref{Joel}{2}{4}{Re 9:7}
\crossref{Joel}{2}{5}{Na 2:3,\allowbreak4; 3:2,\allowbreak3 Re 9:9}
\crossref{Joel}{2}{6}{Ps 119:83 Isa 13:8 Jer 8:21; 30:6 La 4:8 Na 2:10}
\crossref{Joel}{2}{7}{2:9 2Sa 5:8 Jer 5:10}
\crossref{Joel}{2}{8}{2Ch 23:10; 32:5}
\crossref{Joel}{2}{9}{Ex 10:6 Jer 9:21 Joh 10:1}
\crossref{Joel}{2}{10}{Ps 18:7; 114:7 Na 1:5 Mt 27:51 Re 6:12; 20:11}
\crossref{Joel}{2}{11}{Joe 3:16 2Sa 22:14,\allowbreak15 Ps 46:6 Isa 7:18; 13:4; 42:13 Jer 25:30 Am 1:2}
\crossref{Joel}{2}{12}{De 4:29,\allowbreak30 1Sa 7:3 1Ki 8:47-\allowbreak49 2Ch 6:38,\allowbreak39; 7:13,\allowbreak14 Isa 55:6,\allowbreak7}
\crossref{Joel}{2}{13}{Ge 37:29,\allowbreak34 2Sa 1:11 1Ki 21:27 2Ki 5:7; 6:30; 22:11 Job 1:20}
\crossref{Joel}{2}{14}{Ex 32:30 Jos 14:12 1Sa 6:5 2Sa 12:22 2Ki 19:4 Am 5:15 Jon 1:6}
\crossref{Joel}{2}{15}{2:1 Nu 10:3}
\crossref{Joel}{2}{16}{Ex 19:10,\allowbreak15,\allowbreak22 Jos 7:13 1Sa 16:5 2Ch 29:5,\allowbreak23,\allowbreak24; 30:17,\allowbreak19; 35:6}
\crossref{Joel}{2}{17}{Joe 1:9,\allowbreak13}
\crossref{Joel}{2}{18}{Isa 42:13 Zec 1:14; 8:2}
\crossref{Joel}{2}{19}{2:24; 1:10 Isa 62:8,\allowbreak9; 65:21-\allowbreak24 Ho 2:15 Am 9:13,\allowbreak14 Hag 2:16-\allowbreak19}
\crossref{Joel}{2}{20}{2:2-\allowbreak11; 1:4-\allowbreak6 Ex 10:19}
\crossref{Joel}{2}{21}{Ge 15:1 Isa 41:10; 54:4 Jer 30:9,\allowbreak10 Zep 3:16,\allowbreak17 Zec 8:15}
\crossref{Joel}{2}{22}{Joe 1:18-\allowbreak20 Ps 36:6; 104:11-\allowbreak14,\allowbreak27-\allowbreak29; 145:15,\allowbreak16; 147:8,\allowbreak9 Isa 30:23,\allowbreak24}
\crossref{Joel}{2}{23}{Ps 149:2 La 4:2 Zec 9:13 Ga 4:26,\allowbreak27}
\crossref{Joel}{2}{24}{Joe 3:13,\allowbreak18 Le 26:10 Pr 3:9,\allowbreak10 Am 9:13 Mal 3:10}
\crossref{Joel}{2}{25}{2:2-\allowbreak11; 1:4-\allowbreak7 Zec 10:6}
\crossref{Joel}{2}{26}{Le 26:5,\allowbreak26 De 6:11,\allowbreak12; 8:10 Ne 9:25 Ps 22:26; 103:5 Pr 13:25}
\crossref{Joel}{2}{27}{Joe 3:17 Le 26:11,\allowbreak12 De 23:14 Ps 46:5; 68:18 Isa 12:6 Eze 37:26-\allowbreak28}
\crossref{Joel}{2}{28}{Pr 1:23 Isa 32:15; 44:3 Eze 39:29 Joh 7:39 Ac 2:16-\allowbreak18}
\crossref{Joel}{2}{29}{1Co 12:13 Ga 3:28 Col 3:11}
\crossref{Joel}{2}{30}{Mt 24:29 Mr 13:24 Lu 21:11,\allowbreak25,\allowbreak26 Ac 2:19,\allowbreak20 Re 6:12-\allowbreak17}
\crossref{Joel}{2}{31}{2:10; 3:1,\allowbreak15 Isa 13:9,\allowbreak10; 34:4,\allowbreak5 Mt 24:29; 27:45 Mr 13:24,\allowbreak25}
\crossref{Joel}{2}{32}{Ps 50:15 Jer 33:3 Zec 13:9 Ac 2:21 Ro 10:11-\allowbreak14 1Co 1:2}
\crossref{Joel}{3}{1}{Joe 2:29 Da 12:1 Zep 3:19,\allowbreak20}
\crossref{Joel}{3}{2}{Zep 3:8 Zec 14:2-\allowbreak4 Re 16:14,\allowbreak16; 19:19-\allowbreak21; 20:8}
\crossref{Joel}{3}{3}{2Ch 28:8,\allowbreak9 Am 2:6 Ob 1:11 Na 3:10 Re 18:13}
\crossref{Joel}{3}{4}{Jud 11:12 2Ch 21:16; 28:17,\allowbreak18 Ac 9:4}
\crossref{Joel}{3}{5}{2Ki 12:18; 16:8; 18:15,\allowbreak16; 24:13; 25:13-\allowbreak17 Jer 50:28; 51:11}
\crossref{Joel}{3}{6}{3:3,\allowbreak8 De 28:32,\allowbreak68 Eze 27:13}
\crossref{Joel}{3}{7}{Isa 11:12; 43:5,\allowbreak6; 49:12 Jer 23:8; 30:10,\allowbreak16; 31:8; 32:37}
\crossref{Joel}{3}{8}{De 32:30 Jud 2:14; 4:2,\allowbreak9}
\crossref{Joel}{3}{9}{Ps 96:10 Isa 34:1 Jer 31:10; 50:2}
\crossref{Joel}{3}{10}{Isa 2:4 Mic 4:3 Lu 22:36}
\crossref{Joel}{3}{11}{3:2 Eze 38:9-\allowbreak18 Mic 4:12 Zep 3:8 Zec 14:2,\allowbreak3 Re 16:14-\allowbreak16; 19:19,\allowbreak20}
\crossref{Joel}{3}{12}{3:2,\allowbreak14 2Ch 20:26 Eze 39:11 Zec 14:4}
\crossref{Joel}{3}{13}{De 16:9 Mr 4:29 Re 14:15,\allowbreak16}
\crossref{Joel}{3}{14}{3:2 Isa 34:2-\allowbreak8; 63:1-\allowbreak7 Eze 38:8-\allowbreak23; 39:8-\allowbreak20 Re 16:14-\allowbreak16; 19:19-\allowbreak21}
\crossref{Joel}{3}{15}{Joe 2:10,\allowbreak31 Isa 13:10 Mt 24:29 Lu 21:25,\allowbreak26 Re 6:12,\allowbreak13}
\crossref{Joel}{3}{16}{Joe 2:11 Isa 42:13 Jer 25:30,\allowbreak31 Ho 11:10 Am 1:2; 3:8}
\crossref{Joel}{3}{17}{3:21; 2:27 Ps 9:11; 76:2 Isa 12:6 Eze 48:35 Mic 4:7 Zep 3:14-\allowbreak16}
\crossref{Joel}{3}{18}{Job 29:6 Isa 55:12,\allowbreak13 Am 9:13,\allowbreak14}
\crossref{Joel}{3}{19}{Isa 11:15; 19:1-\allowbreak15 Zec 10:10; 14:18,\allowbreak19}
\crossref{Joel}{3}{20}{Isa 33:20 Eze 37:25 Am 9:15}
\crossref{Joel}{3}{21}{Isa 4:4 Eze 36:25,\allowbreak29 Mt 27:25}

% Amos
\crossref{Amos}{1}{1}{Jer 1:1; 7:27}
\crossref{Amos}{1}{2}{Am 3:7,\allowbreak8 Pr 20:2 Isa 42:13 Jer 25:30 Ho 13:8 Joe 2:11; 3:16}
\crossref{Amos}{1}{3}{1:6,\allowbreak9,\allowbreak11,\allowbreak13; 2:1,\allowbreak4,\allowbreak6 Job 5:19; 19:3 Pr 6:16 Ec 11:2}
\crossref{Amos}{1}{4}{1:7,\allowbreak10,\allowbreak12,\allowbreak14; 2:2,\allowbreak5 Jud 9:19,\allowbreak20,\allowbreak57 Jer 17:27; 49:27 Eze 30:8}
\crossref{Amos}{1}{5}{Isa 43:14 Jer 50:36; 51:30 La 2:9 Na 3:13}
\crossref{Amos}{1}{6}{1:3,\allowbreak9,\allowbreak11}
\crossref{Amos}{1}{7}{De 32:35,\allowbreak41-\allowbreak43 Ps 75:7,\allowbreak8; 94:1-\allowbreak5 Zep 2:4 Ro 12:19}
\crossref{Amos}{1}{8}{Isa 20:1 Jer 47:5 Eze 25:16}
\crossref{Amos}{1}{9}{Isa 23:1-\allowbreak18 Jer 47:4 Eze 26:1-\allowbreak28:26 Joe 3:4-\allowbreak8 Zec 9:2-\allowbreak4}
\crossref{Amos}{1}{10}{1:4,\allowbreak7-\allowbreak2:16 Eze 26:12 Zec 9:4}
\crossref{Amos}{1}{11}{Isa 21:11,\allowbreak12; 34:1-\allowbreak17; 63:1-\allowbreak7 Jer 49:7-\allowbreak22 Eze 25:12-\allowbreak14; 35:1-\allowbreak15}
\crossref{Amos}{1}{12}{Ge 36:11 Jer 49:7,\allowbreak20 Ob 1:9,\allowbreak10}
\crossref{Amos}{1}{13}{De 2:19 Jer 49:1-\allowbreak6 Eze 25:2-\allowbreak7 Zep 2:8}
\crossref{Amos}{1}{14}{De 3:11 2Sa 12:26 Jer 49:2 Eze 25:5}
\crossref{Amos}{1}{15}{Jer 49:3}
\crossref{Amos}{2}{1}{2:4,\allowbreak6; 1:3,\allowbreak6,\allowbreak9,\allowbreak11,\allowbreak13 Nu 22:1-\allowbreak25:18 De 23:4,\allowbreak5 Ps 83:4-\allowbreak7 Mic 6:5}
\crossref{Amos}{2}{2}{Jer 48:24,\allowbreak41}
\crossref{Amos}{2}{3}{Nu 24:17 Jer 48:7,\allowbreak25}
\crossref{Amos}{2}{4}{De 31:16-\allowbreak18; 32:15-\allowbreak27}
\crossref{Amos}{2}{5}{Jer 17:27; 21:10; 37:8-\allowbreak10; 39:8; 52:13 Ho 8:14}
\crossref{Amos}{2}{6}{Am 6:3-\allowbreak7 2Ki 17:7-\allowbreak18; 18:12 Eze 23:5-\allowbreak9 Ho 4:1,\allowbreak2,\allowbreak11-\allowbreak14; 7:7-\allowbreak10; 8:4-\allowbreak6}
\crossref{Amos}{2}{7}{Am 4:1 1Ki 21:4 Pr 28:21 Mic 2:2,\allowbreak9; 7:2,\allowbreak3 Zep 3:3}
\crossref{Amos}{2}{8}{Ex 22:26,\allowbreak27 De 24:12-\allowbreak17 Eze 18:7,\allowbreak12}
\crossref{Amos}{2}{9}{Ge 15:16 Ex 3:8; 34:11 Nu 21:24 De 2:24-\allowbreak33 Jos 3:10; 24:8-\allowbreak12}
\crossref{Amos}{2}{10}{Ex 12:51 Ne 9:8-\allowbreak12 Ps 105:42,\allowbreak43; 136:10,\allowbreak11 Jer 32:20,\allowbreak21}
\crossref{Amos}{2}{11}{1Sa 3:20; 19:20 1Ki 17:1; 18:4; 19:16; 20:13,\allowbreak35,\allowbreak41; 22:8 2Ki 2:2-\allowbreak5}
\crossref{Amos}{2}{12}{Ex 32:21-\allowbreak23}
\crossref{Amos}{2}{13}{Ps 78:40 Isa 1:14; 7:13; 43:24 Eze 6:9; 16:43 Mal 2:17}
\crossref{Amos}{2}{14}{Am 9:1-\allowbreak3 Job 11:20}
\crossref{Amos}{2}{15}{Ps 33:16,\allowbreak17}
\crossref{Amos}{2}{16}{Jer 48:41}
\crossref{Amos}{3}{1}{2Ch 20:15 Isa 46:3; 48:12 Ho 4:1; 5:1 Mic 3:1 Re 2:29}
\crossref{Amos}{3}{2}{Ex 19:5,\allowbreak6 De 7:6; 10:15; 26:18; 32:9 Ps 147:19 Isa 63:19}
\crossref{Amos}{3}{3}{Ge 5:22; 6:9; 17:1 2Co 6:14-\allowbreak16}
\crossref{Amos}{3}{4}{3:8; 1:2 Ps 104:21 Ho 11:10}
\crossref{Amos}{3}{5}{Ec 9:12 Jer 31:28 Da 9:14}
\crossref{Amos}{3}{6}{Jer 4:5; 6:1 Eze 33:3 Ho 5:8 Zep 1:16}
\crossref{Amos}{3}{7}{Ge 6:13; 18:17 1Ki 22:19-\allowbreak23 2Ki 3:17-\allowbreak20; 6:12; 22:13,\allowbreak20}
\crossref{Amos}{3}{8}{3:4; 1:2 Re 5:5}
\crossref{Amos}{3}{9}{2Sa 1:20 Jer 2:10,\allowbreak11; 31:7-\allowbreak9; 46:14; 50:2}
\crossref{Amos}{3}{10}{Ps 14:4 Jer 4:22; 5:4 2Pe 3:5}
\crossref{Amos}{3}{11}{Am 6:14 2Ki 15:19,\allowbreak29; 17:3-\allowbreak6; 18:9-\allowbreak11 Isa 7:17-\allowbreak25; 8:7,\allowbreak8; 10:5,\allowbreak6,\allowbreak9-\allowbreak11}
\crossref{Amos}{3}{12}{1Sa 17:34-\allowbreak37 Isa 31:4}
\crossref{Amos}{3}{13}{De 8:19; 30:18,\allowbreak19 2Ki 17:13,\allowbreak15 2Ch 24:19 Ac 2:40; 18:5,\allowbreak6; 20:21}
\crossref{Amos}{3}{14}{Ex 32:34}
\crossref{Amos}{3}{15}{Jer 36:22}
\crossref{Amos}{4}{1}{Am 6:1 1Ki 16:24}
\crossref{Amos}{4}{2}{Am 6:8}
\crossref{Amos}{4}{3}{2Ki 25:4 Eze 12:5,\allowbreak12}
\crossref{Amos}{4}{4}{Am 5:5 Ho 4:15; 9:15; 12:11}
\crossref{Amos}{4}{5}{Le 7:12,\allowbreak13; 23:17}
\crossref{Amos}{4}{6}{Le 26:26 De 28:38 1Ki 17:1; 18:2 2Ki 4:38; 6:25-\allowbreak29; 8:1 Eze 16:27}
\crossref{Amos}{4}{7}{Le 26:18-\allowbreak21,\allowbreak23,\allowbreak24,\allowbreak27,\allowbreak28 De 28:23,\allowbreak24 1Ki 8:35,\allowbreak36 2Ch 7:13,\allowbreak14}
\crossref{Amos}{4}{8}{1Ki 18:5 Isa 41:17,\allowbreak18 Jer 14:3}
\crossref{Amos}{4}{9}{De 28:22 1Ki 8:37 2Ch 6:28 Hag 2:17}
\crossref{Amos}{4}{10}{Ex 9:3-\allowbreak6; 12:29,\allowbreak30; 15:26 Le 26:16,\allowbreak25 De 7:15; 28:22,\allowbreak27,\allowbreak60}
\crossref{Amos}{4}{11}{Ge 19:24,\allowbreak25 Isa 13:19 Jer 49:18 Ho 11:8 2Pe 2:6 Jude 1:7}
\crossref{Amos}{4}{12}{4:2,\allowbreak3; 2:14; 9:1-\allowbreak4}
\crossref{Amos}{4}{13}{Ps 135:7; 147:18 Jer 10:13; 51:16}
\crossref{Amos}{5}{1}{Am 3:1; 4:1}
\crossref{Amos}{5}{2}{Isa 37:22 Jer 14:17; 18:13; 31:4 La 2:13}
\crossref{Amos}{5}{3}{De 4:27; 28:62 Isa 1:9; 10:22 Eze 12:16 Ro 9:27}
\crossref{Amos}{5}{4}{5:6 De 30:1-\allowbreak8 1Ch 28:9 2Ch 15:2; 20:3; 34:3 Ps 14:2; 27:8}
\crossref{Amos}{5}{5}{Am 4:4 Ho 4:15; 9:15; 10:14,\allowbreak15; 12:11}
\crossref{Amos}{5}{6}{5:4 Eze 33:11}
\crossref{Amos}{5}{7}{5:11,\allowbreak12; 6:12 De 29:18 Isa 1:23; 5:7; 10:1; 59:13,\allowbreak14 Ho 10:4}
\crossref{Amos}{5}{8}{Job 9:9; 38:31,\allowbreak32}
\crossref{Amos}{5}{9}{2Ki 13:17,\allowbreak25 Jer 37:10 Heb 11:34}
\crossref{Amos}{5}{10}{Am 7:10-\allowbreak17 1Ki 18:17; 21:20; 22:8 2Ch 24:20-\allowbreak22; 25:16; 36:16 Pr 9:7,\allowbreak8}
\crossref{Amos}{5}{11}{Am 4:1 Isa 5:7,\allowbreak8; 59:13,\allowbreak14 Mic 2:2; 3:1-\allowbreak3 Jas 2:6 Re 11:8-\allowbreak10}
\crossref{Amos}{5}{12}{De 31:21 Isa 66:18 Jer 29:23 Heb 4:12,\allowbreak13}
\crossref{Amos}{5}{13}{Am 6:10 Ec 3:7 Isa 36:21 Ho 4:4 Mic 7:5-\allowbreak7 Mt 27:12-\allowbreak14}
\crossref{Amos}{5}{14}{Ps 34:12-\allowbreak16 Pr 11:27 Isa 1:16,\allowbreak17; 55:2 Mic 6:8 Mt 6:33 Ro 2:7-\allowbreak9}
\crossref{Amos}{5}{15}{Ps 34:14; 36:4; 37:27; 97:10; 119:104; 139:21,\allowbreak22 Ro 7:15,\allowbreak16,\allowbreak22; 8:7}
\crossref{Amos}{5}{16}{5:27; 3:13}
\crossref{Amos}{5}{17}{Isa 16:10; 32:10-\allowbreak12 Jer 48:33 Ho 9:1,\allowbreak2}
\crossref{Amos}{5}{18}{Isa 5:19; 28:15-\allowbreak22 Jer 17:15 Eze 12:22,\allowbreak27 Mal 3:1,\allowbreak2 2Pe 3:4}
\crossref{Amos}{5}{19}{}
\crossref{Amos}{5}{20}{Job 3:4-\allowbreak6; 10:21,\allowbreak22 Isa 13:10 Eze 34:12 Na 1:8 Mt 22:13}
\crossref{Amos}{5}{21}{Pr 15:8; 21:27; 28:9 Isa 1:11-\allowbreak16; 66:3 Jer 6:20; 7:21-\allowbreak23 Ho 8:13}
\crossref{Amos}{5}{22}{Ps 50:8-\allowbreak13 Isa 66:3 Mic 6:6,\allowbreak7}
\crossref{Amos}{5}{23}{Am 6:5; 8:3,\allowbreak10}
\crossref{Amos}{5}{24}{5:7,\allowbreak14,\allowbreak15 Job 29:12-\allowbreak17 Pr 21:3 Ho 6:6 Mic 6:8 Mr 12:32-\allowbreak34}
\crossref{Amos}{5}{25}{Le 17:7 De 32:17-\allowbreak19 Jos 24:14 Ne 9:18,\allowbreak21 Isa 43:23,\allowbreak24}
\crossref{Amos}{5}{26}{Le 18:21; 20:2-\allowbreak5 1Ki 11:33 2Ki 23:12,\allowbreak13}
\crossref{Amos}{5}{27}{2Ki 15:29; 17:6 Ac 7:43}
\crossref{Amos}{6}{1}{Jud 18:7 Isa 32:9-\allowbreak11; 33:14 Jer 48:11; 49:31 Lu 6:24,\allowbreak25}
\crossref{Amos}{6}{2}{Jer 2:10,\allowbreak11 Na 3:8}
\crossref{Amos}{6}{3}{Am 5:18; 9:10 Ec 8:11 Isa 47:7; 56:12 Eze 12:22,\allowbreak27 Mt 24:48 1Th 5:3}
\crossref{Amos}{6}{4}{Isa 5:11,\allowbreak12; 22:13 Lu 16:19 Ro 13:13,\allowbreak14 Jas 5:5}
\crossref{Amos}{6}{5}{Ge 31:27 Job 21:11,\allowbreak12 Ec 2:8 Isa 5:12 1Pe 4:3 Re 18:22}
\crossref{Amos}{6}{6}{Ho 3:1 1Ti 5:23}
\crossref{Amos}{6}{7}{Am 5:5,\allowbreak27; 7:11 De 28:41 Lu 21:24}
\crossref{Amos}{6}{8}{Am 4:2 Jer 51:14 Heb 6:13-\allowbreak17}
\crossref{Amos}{6}{9}{Am 5:3 1Sa 2:33 Es 5:11; 9:10 Job 1:2,\allowbreak19; 20:28 Ps 109:13 Isa 14:21}
\crossref{Amos}{6}{10}{Am 8:3 1Sa 31:12 2Ki 23:16 Jer 16:6}
\crossref{Amos}{6}{11}{Am 3:6,\allowbreak7; 9:1,\allowbreak9 Ps 105:16,\allowbreak31,\allowbreak34 Isa 10:5,\allowbreak6; 13:3; 46:10,\allowbreak11; 55:11}
\crossref{Amos}{6}{12}{Isa 48:4 Jer 5:3; 6:29,\allowbreak30 Zec 7:11,\allowbreak12}
\crossref{Amos}{6}{13}{Ex 32:18,\allowbreak19 Jud 9:19,\allowbreak20,\allowbreak27; 16:23-\allowbreak25 1Sa 4:5 Job 31:25,\allowbreak29}
\crossref{Amos}{6}{14}{2Ki 15:29; 17:6 Isa 7:20; 8:4-\allowbreak8; 10:5,\allowbreak6 Jer 5:15-\allowbreak17 Ho 10:5}
\crossref{Amos}{7}{1}{7:4,\allowbreak7; 8:1 Jer 1:11-\allowbreak16; 24:1 Eze 11:25 Zec 1:20}
\crossref{Amos}{7}{2}{Ex 10:15 Re 9:4}
\crossref{Amos}{7}{3}{7:6 De 32:36 1Ch 21:15 Ps 106:45 Ho 11:8 Joe 2:14 Jon 3:10}
\crossref{Amos}{7}{4}{7:1,\allowbreak7 Re 4:1}
\crossref{Amos}{7}{5}{7:2 Ps 85:4 Isa 10:25}
\crossref{Amos}{7}{6}{Jud 2:18; 10:16 Ps 90:13; 135:14 Jer 26:19 Jon 4:2}
\crossref{Amos}{7}{7}{2Sa 8:2 2Ki 21:13 Isa 28:17; 34:11 La 2:8 Eze 40:3 Zec 2:1,\allowbreak2}
\crossref{Amos}{7}{8}{Jer 1:11-\allowbreak13 Zec 5:2}
\crossref{Amos}{7}{9}{Am 3:14; 5:5; 8:14}
\crossref{Amos}{7}{10}{1Ki 12:31,\allowbreak32; 13:33 2Ki 14:23,\allowbreak24 2Ch 13:8,\allowbreak9 Jer 20:1-\allowbreak3}
\crossref{Amos}{7}{11}{Jer 26:9; 28:10,\allowbreak11 Ac 6:14}
\crossref{Amos}{7}{12}{1Sa 9:9 2Ch 16:10 Isa 30:10}
\crossref{Amos}{7}{13}{Am 2:12 Ac 4:17,\allowbreak18; 5:28,\allowbreak40}
\crossref{Amos}{7}{14}{1Ki 20:35 2Ki 2:3,\allowbreak5,\allowbreak7; 4:38; 6:1 2Ch 16:7; 19:2; 20:34}
\crossref{Amos}{7}{15}{2Sa 7:8 Ps 78:70-\allowbreak72 Mt 4:18,\allowbreak19; 9:9}
\crossref{Amos}{7}{16}{1Sa 15:16 1Ki 22:19 Jer 28:15-\allowbreak17}
\crossref{Amos}{7}{17}{Isa 13:16 Jer 20:6; 28:12,\allowbreak16; 29:21,\allowbreak25,\allowbreak31,\allowbreak32 La 5:11 Ho 4:13,\allowbreak14}
\crossref{Amos}{8}{1}{Am 7:1,\allowbreak4,\allowbreak7}
\crossref{Amos}{8}{2}{Am 7:8 Jer 1:11-\allowbreak14 Eze 8:6,\allowbreak12,\allowbreak17 Zec 1:18-\allowbreak21; 5:2,\allowbreak5,\allowbreak6}
\crossref{Amos}{8}{3}{8:10; 5:23 Ho 10:5,\allowbreak6 Joe 1:5,\allowbreak11,\allowbreak13 Zec 11:1-\allowbreak3}
\crossref{Amos}{8}{4}{Am 7:16 1Ki 22:19 Isa 1:10; 28:14 Jer 5:21; 28:15}
\crossref{Amos}{8}{5}{Nu 10:10; 28:11-\allowbreak15 2Ki 4:23 Ps 81:3,\allowbreak4 Isa 1:13 Col 2:16}
\crossref{Amos}{8}{6}{8:4; 2:6 Le 25:39-\allowbreak42 Ne 5:1-\allowbreak5,\allowbreak8 Joe 3:3,\allowbreak6}
\crossref{Amos}{8}{7}{Am 6:8 De 33:26-\allowbreak29 Ps 47:4; 68:34 Lu 2:32}
\crossref{Amos}{8}{8}{8:10; 9:5 Jer 12:4 Ho 4:3; 10:5 Mt 24:30}
\crossref{Amos}{8}{9}{Ex 10:21-\allowbreak23 Mt 27:45 Mr 15:33 Lu 23:44}
\crossref{Amos}{8}{10}{8:3; 5:23; 6:4-\allowbreak7 De 16:14 1Sa 25:36-\allowbreak38 2Sa 13:28-\allowbreak31 Job 20:23}
\crossref{Amos}{8}{11}{1Sa 3:1; 28:6,\allowbreak15 Ps 74:9 Isa 5:6; 30:20,\allowbreak21 Eze 7:26 Mic 3:6}
\crossref{Amos}{8}{12}{Pr 14:6 Da 12:4 Mt 11:25-\allowbreak27; 12:30; 24:23-\allowbreak26 Ro 9:31-\allowbreak33}
\crossref{Amos}{8}{13}{De 32:25 Ps 63:1; 144:12-\allowbreak15 Isa 40:30; 41:17-\allowbreak20 Jer 48:18}
\crossref{Amos}{8}{14}{Ho 4:15 Zep 1:5}
\crossref{Amos}{9}{1}{2Ch 18:18 Isa 6:1 Eze 1:28 Joh 1:18,\allowbreak32 Ac 26:13 Re 1:17}
\crossref{Amos}{9}{2}{Job 26:6 Ps 139:7-\allowbreak10 Isa 2:19}
\crossref{Amos}{9}{3}{Job 34:22 Jer 23:23,\allowbreak24}
\crossref{Amos}{9}{4}{Le 26:33,\allowbreak36-\allowbreak39 De 28:64,\allowbreak65 Eze 5:2,\allowbreak12 Zec 13:8,\allowbreak9}
\crossref{Amos}{9}{5}{Ps 46:6; 144:5 Isa 64:1 Mic 1:3 Na 1:6 Hab 3:10 Re 20:11}
\crossref{Amos}{9}{6}{Ps 104:3,\allowbreak13}
\crossref{Amos}{9}{7}{Jer 9:25,\allowbreak26; 13:23}
\crossref{Amos}{9}{8}{9:4 Ps 11:4-\allowbreak6 Pr 5:21; 15:3 Jer 44:27}
\crossref{Amos}{9}{9}{Le 26:33 De 28:64}
\crossref{Amos}{9}{10}{Isa 33:14 Eze 20:38; 34:16,\allowbreak17 Zep 3:11-\allowbreak13 Zec 13:8,\allowbreak9 Mal 3:2-\allowbreak5}
\crossref{Amos}{9}{11}{Ac 15:15-\allowbreak17}
\crossref{Amos}{9}{12}{Isa 11:14; 14:1,\allowbreak2 Joe 3:8 Ob 1:18-\allowbreak21}
\crossref{Amos}{9}{13}{Le 26:5 Eze 36:35 Ho 2:21-\allowbreak23 Joh 4:35}
\crossref{Amos}{9}{14}{Ps 53:6 Jer 30:3,\allowbreak18; 31:23 Eze 16:53; 39:25 Joe 3:1,\allowbreak2}
\crossref{Amos}{9}{15}{}

% Obad
\crossref{Obad}{1}{1}{Ps 137:7 Isa 21:11; 34:1-\allowbreak17; 63:1-\allowbreak6 Jer 9:25,\allowbreak26; 25:17,\allowbreak21}
\crossref{Obad}{1}{2}{Nu 24:18 1Sa 2:7,\allowbreak8 Job 34:25-\allowbreak29 Ps 107:39,\allowbreak40 Isa 23:9}
\crossref{Obad}{1}{3}{Pr 16:18; 18:12; 29:23 Isa 10:14-\allowbreak16; 16:6 Jer 48:29,\allowbreak30; 49:16}
\crossref{Obad}{1}{4}{Job 20:6,\allowbreak7; 39:27,\allowbreak28 Jer 49:16 Hab 2:9}
\crossref{Obad}{1}{5}{Jer 49:9}
\crossref{Obad}{1}{6}{Ps 139:1 Isa 10:13,\allowbreak14; 45:3 Jer 49:10; 50:37 Mt 6:19,\allowbreak20}
\crossref{Obad}{1}{7}{Jer 20:10; 38:22}
\crossref{Obad}{1}{8}{Job 5:12-\allowbreak14 Ps 33:10 Isa 19:3,\allowbreak13,\allowbreak14; 29:14 1Co 3:19,\allowbreak20}
\crossref{Obad}{1}{9}{Ps 76:5,\allowbreak6 Isa 19:16,\allowbreak17 Jer 49:22; 50:36,\allowbreak37 Am 2:16 Na 3:13}
\crossref{Obad}{1}{10}{Ge 27:11,\allowbreak41 Nu 20:14-\allowbreak21 Ps 83:5-\allowbreak9; 137:7 La 4:21 Eze 25:12}
\crossref{Obad}{1}{11}{2Ki 24:10-\allowbreak16; 25:11 Jer 52:28-\allowbreak30}
\crossref{Obad}{1}{12}{Ps 22:17; 37:13; 54:7; 59:10; 92:11 Mic 4:11; 7:8-\allowbreak10 Mt 27:40-\allowbreak43}
\crossref{Obad}{1}{13}{2Sa 16:12 Ps 22:17 Zec 1:15}
\crossref{Obad}{1}{14}{Am 1:6,\allowbreak9}
\crossref{Obad}{1}{15}{Ps 110:5,\allowbreak6 Jer 9:25,\allowbreak26; 25:15-\allowbreak29; 49:12}
\crossref{Obad}{1}{16}{Ps 75:8,\allowbreak9 Isa 49:25,\allowbreak26; 51:22,\allowbreak23 Jer 25:15,\allowbreak16,\allowbreak27-\allowbreak29; 49:12}
\crossref{Obad}{1}{17}{Isa 46:13 Joe 2:32}
\crossref{Obad}{1}{18}{Isa 10:17; 31:9 Mic 5:8 Zec 12:6}
\crossref{Obad}{1}{19}{Nu 24:18,\allowbreak19 Jos 15:21 Jer 32:44 Am 9:12 Mal 1:4,\allowbreak5}
\crossref{Obad}{1}{20}{Jer 3:18; 33:26 Eze 34:12,\allowbreak13 Ho 1:10,\allowbreak11 Am 9:14,\allowbreak15 Zec 10:6-\allowbreak10}
\crossref{Obad}{1}{21}{Jud 2:16; 3:9 2Ki 13:5 Isa 19:20 Da 12:3 Joe 2:32 Mic 5:4-\allowbreak9}

% Jonah
\crossref{Jonah}{1}{1}{2Ki 14:25 Mt 12:39; 16:4 Lu 11:29,\allowbreak30,\allowbreak32}
\crossref{Jonah}{1}{2}{Jon 3:2; 4:11 Ge 10:11 2Ki 19:36 Na 1:1; 2:1-\allowbreak3:19 Zep 2:13-\allowbreak15}
\crossref{Jonah}{1}{3}{Jon 4:2 Ex 4:13,\allowbreak14 1Ki 19:3,\allowbreak9 Jer 20:7-\allowbreak9 Eze 3:14 Lu 9:62 Ac 15:38}
\crossref{Jonah}{1}{4}{Ex 10:13,\allowbreak19; 14:21; 15:10 Nu 11:31 Ps 107:24-\allowbreak31; 135:7 Am 4:13}
\crossref{Jonah}{1}{5}{1:6,\allowbreak14,\allowbreak16 1Ki 18:26 Isa 44:17-\allowbreak20; 45:20 Jer 2:28 Ho 7:14}
\crossref{Jonah}{1}{6}{Isa 3:15 Eze 18:2 Ac 21:13 Ro 13:11 Eph 5:14}
\crossref{Jonah}{1}{7}{Jud 7:13,\allowbreak14 Isa 41:6,\allowbreak7}
\crossref{Jonah}{1}{8}{Jos 7:19 1Sa 14:43 Jas 5:16}
\crossref{Jonah}{1}{9}{Ge 14:13; 39:14 Php 3:5}
\crossref{Jonah}{1}{10}{Joh 19:8}
\crossref{Jonah}{1}{11}{1Sa 6:2,\allowbreak3 2Sa 21:1-\allowbreak6; 24:11-\allowbreak13 Mic 6:6,\allowbreak7}
\crossref{Jonah}{1}{12}{2Sa 24:17 Joh 11:50}
\crossref{Jonah}{1}{13}{Job 34:29 Pr 21:30}
\crossref{Jonah}{1}{14}{1:5,\allowbreak16 Ps 107:28 Isa 26:16}
\crossref{Jonah}{1}{15}{Jos 7:24-\allowbreak26 2Sa 21:8,\allowbreak9}
\crossref{Jonah}{1}{16}{1:10 Isa 26:9 Da 4:34-\allowbreak37; 6:26 Mr 4:31 Ac 5:11}
\crossref{Jonah}{1}{17}{Jon 4:6 Ge 1:21 Ps 104:25,\allowbreak26 Hab 3:2}
\crossref{Jonah}{2}{1}{2Ch 33:11-\allowbreak13 Ps 50:15; 91:15 Isa 26:16 Ho 5:15; 6:1-\allowbreak3 Jas 5:13}
\crossref{Jonah}{2}{2}{Ge 32:7-\allowbreak12,\allowbreak24-\allowbreak28 1Sa 30:6 Ps 4:1; 18:4-\allowbreak6; 22:24; 34:6; 65:2}
\crossref{Jonah}{2}{3}{Jon 1:12-\allowbreak16 Ps 69:1,\allowbreak2,\allowbreak14,\allowbreak15; 88:5-\allowbreak8 La 3:54}
\crossref{Jonah}{2}{4}{Ps 31:22; 77:1-\allowbreak7 Isa 38:10-\allowbreak14,\allowbreak17; 49:14 Eze 37:11}
\crossref{Jonah}{2}{5}{Ps 40:2; 69:1,\allowbreak2 La 3:54}
\crossref{Jonah}{2}{6}{De 32:22 Ps 65:6; 104:6,\allowbreak8 Isa 40:12 Hab 3:6,\allowbreak10}
\crossref{Jonah}{2}{7}{Ps 22:14; 27:13; 119:81-\allowbreak83 Heb 12:3}
\crossref{Jonah}{2}{8}{1Sa 12:21 2Ki 17:15 Ps 31:6 Jer 2:13; 10:8,\allowbreak14,\allowbreak15; 16:19}
\crossref{Jonah}{2}{9}{Ge 35:3 Ps 50:14,\allowbreak23; 66:13-\allowbreak15; 107:22; 116:17,\allowbreak18 Jer 33:11}
\crossref{Jonah}{2}{10}{Jon 1:17 Ge 1:3,\allowbreak7,\allowbreak9,\allowbreak11,\allowbreak14 Ps 33:9; 105:31,\allowbreak34 Isa 50:2}
\crossref{Jonah}{3}{1}{Jon 1:1}
\crossref{Jonah}{3}{2}{Jer 1:17; 15:19-\allowbreak21 Eze 2:7; 3:17 Mt 3:8 Joh 5:14}
\crossref{Jonah}{3}{3}{Ge 22:3 Mt 21:28,\allowbreak29 2Ti 4:11}
\crossref{Jonah}{3}{4}{3:10 De 18:22 2Ki 20:1,\allowbreak6 Jer 18:7-\allowbreak10}
\crossref{Jonah}{3}{5}{Ex 9:18-\allowbreak21 Mt 12:41 Lu 11:32 Ac 27:25 Heb 11:1,\allowbreak7}
\crossref{Jonah}{3}{6}{Jer 13:18}
\crossref{Jonah}{3}{7}{3:5 2Ch 20:3 Ezr 8:21 Joe 2:15,\allowbreak16}
\crossref{Jonah}{3}{8}{Jon 1:6,\allowbreak14 Ps 130:1,\allowbreak2}
\crossref{Jonah}{3}{9}{Jon 1:6 2Sa 12:22 Ps 106:45 Joe 2:13,\allowbreak14 Am 5:15 Lu 15:18-\allowbreak20}
\crossref{Jonah}{3}{10}{1Ki 21:27-\allowbreak29 Job 33:27,\allowbreak28 Jer 31:18-\allowbreak20 Lu 11:32; 15:20}
\crossref{Jonah}{4}{1}{Jon 4:9 Mt 20:15 Lu 7:39; 15:28 Ac 13:46 Jas 4:5,\allowbreak6}
\crossref{Jonah}{4}{2}{1Ki 19:4 Jer 20:7}
\crossref{Jonah}{4}{3}{Nu 11:15; 20:3 1Ki 19:4 Job 3:20,\allowbreak21; 6:8,\allowbreak9 Jer 20:14-\allowbreak18}
\crossref{Jonah}{4}{4}{4:9 Nu 20:11,\allowbreak12,\allowbreak24 Ps 106:32,\allowbreak33 Mic 6:3 Mt 20:15 Jas 1:19,\allowbreak20}
\crossref{Jonah}{4}{5}{Jon 1:5 1Ki 19:9,\allowbreak13 Isa 57:17 Jer 20:9}
\crossref{Jonah}{4}{6}{Jon 1:17 Ps 103:10-\allowbreak14}
\crossref{Jonah}{4}{7}{Job 1:21 Ps 30:6,\allowbreak7; 102:10}
\crossref{Jonah}{4}{8}{4:6,\allowbreak7; 1:4,\allowbreak17 Eze 19:12 Re 3:19}
\crossref{Jonah}{4}{9}{4:4}
\crossref{Jonah}{4}{10}{}
\crossref{Jonah}{4}{11}{4:1 Isa 1:18 Mt 18:33 Lu 15:28-\allowbreak32}

% Mic
\crossref{Mic}{1}{1}{1:14,\allowbreak15 Jer 26:18}
\crossref{Mic}{1}{2}{Mic 6:1,\allowbreak2 De 32:1 Ps 49:1,\allowbreak2; 50:1 Isa 1:2 Jer 22:29 Mr 7:14-\allowbreak16}
\crossref{Mic}{1}{3}{Isa 26:21; 64:1,\allowbreak2 Eze 3:12 Ho 5:14,\allowbreak15}
\crossref{Mic}{1}{4}{Jud 5:4 Ps 97:5 Isa 64:1-\allowbreak3 Am 9:5 Na 1:5 Hab 3:6,\allowbreak10}
\crossref{Mic}{1}{5}{2Ki 17:7-\allowbreak23 2Ch 36:14-\allowbreak16 Isa 50:1,\allowbreak2; 59:1-\allowbreak15 Jer 2:17,\allowbreak19}
\crossref{Mic}{1}{6}{Mic 3:12 2Ki 19:25 Isa 25:2,\allowbreak12 Jer 9:11; 51:37 Ho 13:16}
\crossref{Mic}{1}{7}{Le 26:30 2Ki 23:14,\allowbreak15 2Ch 31:1; 34:6,\allowbreak7 Isa 27:9 Ho 8:6; 10:5,\allowbreak6}
\crossref{Mic}{1}{8}{Isa 16:9; 21:3; 22:4 Jer 4:19; 9:1,\allowbreak10,\allowbreak19; 48:36-\allowbreak39}
\crossref{Mic}{1}{9}{2Ki 18:9-\allowbreak13 Isa 8:7,\allowbreak8}
\crossref{Mic}{1}{10}{2Sa 1:20 Am 5:13; 6:10}
\crossref{Mic}{1}{11}{Isa 16:2 Jer 48:6,\allowbreak9}
\crossref{Mic}{1}{12}{Ru 1:20}
\crossref{Mic}{1}{13}{Jos 15:39 2Ki 18:13,\allowbreak14,\allowbreak17 2Ch 11:9; 32:9 Isa 37:8}
\crossref{Mic}{1}{14}{2Sa 8:2 2Ki 16:8; 18:14-\allowbreak16 2Ch 16:1-\allowbreak3 Isa 30:6}
\crossref{Mic}{1}{15}{Isa 7:17-\allowbreak25; 10:5,\allowbreak6 Jer 49:1}
\crossref{Mic}{1}{16}{Job 1:20 Isa 15:2; 22:12 Jer 6:26; 7:29; 16:6 Am 8:10}
\crossref{Mic}{2}{1}{Es 3:8; 5:14; 9:25 Ps 7:14-\allowbreak16; 140:1-\allowbreak8 Pr 6:12-\allowbreak19; 12:2 Isa 32:7}
\crossref{Mic}{2}{2}{Ex 20:17 1Ki 21:2-\allowbreak19 Job 31:38 Isa 5:8 Jer 22:17 Am 8:4}
\crossref{Mic}{2}{3}{Jer 8:3 Am 3:1,\allowbreak2}
\crossref{Mic}{2}{4}{Nu 23:7,\allowbreak18; 24:3,\allowbreak15 Job 27:1 Isa 14:4 Eze 16:44 Hab 2:6}
\crossref{Mic}{2}{5}{De 32:8 Jos 18:4,\allowbreak10 Ps 16:6 Ho 9:3}
\crossref{Mic}{2}{6}{Ps 74:9 Eze 3:26 Am 8:11-\allowbreak13}
\crossref{Mic}{2}{7}{Mic 3:9 Isa 48:1,\allowbreak2; 58:1 Jer 2:4 Mt 3:8 Joh 8:39 Ro 2:28,\allowbreak29; 9:6-\allowbreak13}
\crossref{Mic}{2}{8}{2Ch 28:5-\allowbreak8 Isa 9:21}
\crossref{Mic}{2}{9}{2:2 Mt 23:14 Mr 12:40 Lu 20:47}
\crossref{Mic}{2}{10}{De 4:26; 30:18 Jos 23:15,\allowbreak16 1Ki 9:7 2Ki 15:29; 17:6 2Ch 7:20}
\crossref{Mic}{2}{11}{1Ki 13:18; 22:21-\allowbreak23 2Ch 18:19-\allowbreak22 Isa 9:15 Jer 14:14; 23:14,\allowbreak25,\allowbreak32}
\crossref{Mic}{2}{12}{Mic 4:6,\allowbreak7 Isa 11:11; 27:12 Jer 3:18; 31:8 Eze 37:21 Ho 1:11}
\crossref{Mic}{2}{13}{Isa 42:7,\allowbreak13-\allowbreak16; 45:1,\allowbreak2; 49:9,\allowbreak24,\allowbreak25; 51:9,\allowbreak10; 55:4; 59:16-\allowbreak19}
\crossref{Mic}{3}{1}{3:9,\allowbreak10 Isa 1:10 Jer 13:15-\allowbreak18; 22:2,\allowbreak3 Ho 5:1 Am 4:1}
\crossref{Mic}{3}{2}{1Ki 21:20; 22:6-\allowbreak8 Am 5:10-\allowbreak14 Lu 19:14 Joh 7:7; 15:18,\allowbreak19,\allowbreak23,\allowbreak24}
\crossref{Mic}{3}{3}{Ps 14:4}
\crossref{Mic}{3}{4}{Mic 2:3,\allowbreak4 Jer 5:31}
\crossref{Mic}{3}{5}{3:11 Isa 9:15,\allowbreak16 Jer 14:14,\allowbreak15; 23:9-\allowbreak17,\allowbreak27,\allowbreak32; 28:15-\allowbreak17; 29:21-\allowbreak23}
\crossref{Mic}{3}{6}{Ps 74:9 Isa 8:20-\allowbreak22 Jer 13:16 Eze 13:22,\allowbreak23 Zec 13:2-\allowbreak4}
\crossref{Mic}{3}{7}{Ex 8:18,\allowbreak19; 9:11 1Sa 9:9 Isa 44:25; 47:12-\allowbreak14 Da 2:9-\allowbreak11 Zec 13:4}
\crossref{Mic}{3}{8}{Job 32:18 Isa 11:2,\allowbreak3; 58:1 Jer 1:18; 6:11; 15:19-\allowbreak21; 20:9 Eze 3:14}
\crossref{Mic}{3}{9}{3:1 Ex 3:16 Ho 5:1}
\crossref{Mic}{3}{10}{Jer 22:13-\allowbreak17 Eze 22:25-\allowbreak28 Hab 2:9-\allowbreak12 Zep 3:3 Mt 27:25}
\crossref{Mic}{3}{11}{Mic 7:3 Nu 16:15 1Sa 8:3; 12:3,\allowbreak4 Isa 1:23 Eze 22:12,\allowbreak27 Ho 4:18}
\crossref{Mic}{3}{12}{Mic 1:6 Ps 79:1; 107:34 Jer 26:18 Mt 24:2 Ac 6:13,\allowbreak14}
\crossref{Mic}{4}{1}{Ge 49:1 Isa 2:1-\allowbreak3 Jer 48:47 Eze 38:16 Da 2:28; 10:14 Ho 3:5}
\crossref{Mic}{4}{2}{Isa 2:3 Jer 31:6; 50:4,\allowbreak5 Zec 8:20-\allowbreak23}
\crossref{Mic}{4}{3}{1Sa 2:10 Ps 82:8; 96:13; 98:9 Isa 11:3-\allowbreak5; 51:5 Mt 25:31,\allowbreak32}
\crossref{Mic}{4}{4}{1Ki 4:25 Isa 26:16 Zec 3:10}
\crossref{Mic}{4}{5}{2Ki 17:29,\allowbreak34 Jer 2:10,\allowbreak11}
\crossref{Mic}{4}{6}{Mic 2:12 Ps 38:17 Isa 35:3-\allowbreak6 Jer 31:8 Eze 34:13-\allowbreak17 Zep 3:19}
\crossref{Mic}{4}{7}{Mic 2:12; 5:3,\allowbreak7,\allowbreak8; 7:18 Isa 6:13; 10:21,\allowbreak22; 11:11-\allowbreak16; 49:21-\allowbreak23; 60:22}
\crossref{Mic}{4}{8}{Ge 35:21}
\crossref{Mic}{4}{9}{Jer 4:21; 8:19; 30:6,\allowbreak7}
\crossref{Mic}{4}{10}{Isa 66:7-\allowbreak9 Ho 13:13 Joh 16:20-\allowbreak22}
\crossref{Mic}{4}{11}{Isa 5:25-\allowbreak30; 8:7,\allowbreak8 Jer 52:4 La 2:15,\allowbreak16 Joe 3:2-\allowbreak15}
\crossref{Mic}{4}{12}{Isa 55:8 Jer 29:11 Ro 11:33,\allowbreak34}
\crossref{Mic}{4}{13}{Isa 41:15,\allowbreak16 Jer 51:33}
\crossref{Mic}{5}{1}{De 28:49 2Ki 24:2 Isa 8:9; 10:6 Jer 4:7; 25:9 Joe 3:9 Hab 1:6}
\crossref{Mic}{5}{2}{Mt 2:6 Joh 7:42}
\crossref{Mic}{5}{3}{Mic 7:13 Ho 2:9,\allowbreak14}
\crossref{Mic}{5}{4}{Mic 7:14 Ps 23:1,\allowbreak2 Isa 40:10,\allowbreak11; 49:9,\allowbreak10 Eze 34:22-\allowbreak24 Joh 10:27-\allowbreak30}
\crossref{Mic}{5}{5}{Ps 72:7 Isa 9:6,\allowbreak7 Zec 9:10 Lu 2:14 Joh 14:27; 16:33 Eph 2:14-\allowbreak17}
\crossref{Mic}{5}{6}{Isa 14:2; 33:1 Na 2:11-\allowbreak13; 3:1-\allowbreak3}
\crossref{Mic}{5}{7}{5:3,\allowbreak8 Eze 14:22 Joe 2:32 Am 5:15 Zep 3:13 Ro 11:5,\allowbreak6}
\crossref{Mic}{5}{8}{Mic 4:13 Ps 2:8-\allowbreak12; 110:5,\allowbreak6 Isa 41:15,\allowbreak16 Ob 1:18,\allowbreak19 Zec 9:15; 10:5}
\crossref{Mic}{5}{9}{Ps 21:8; 106:26 Isa 1:25; 11:14; 14:2-\allowbreak4; 26:11; 33:10; 37:36}
\crossref{Mic}{5}{10}{Ps 20:7,\allowbreak8; 33:16,\allowbreak17 Jer 3:23 Ho 1:7; 14:3 Zec 9:10}
\crossref{Mic}{5}{11}{Isa 2:12-\allowbreak17 Eze 38:11 Zec 4:6}
\crossref{Mic}{5}{12}{Isa 2:6-\allowbreak8,\allowbreak18,\allowbreak20; 8:19,\allowbreak20; 27:9 Zec 13:2-\allowbreak4 Re 19:20; 22:15}
\crossref{Mic}{5}{13}{Isa 17:7,\allowbreak8 Eze 6:9; 36:25; 37:23 Ho 2:16,\allowbreak17; 14:3,\allowbreak8}
\crossref{Mic}{5}{14}{5:11}
\crossref{Mic}{5}{15}{5:8 Ps 149:7 2Th 1:8}
\crossref{Mic}{6}{1}{Mic 1:2 1Sa 15:16 Jer 13:15 Am 3:1 Heb 3:7,\allowbreak8}
\crossref{Mic}{6}{2}{De 32:22 2Sa 22:8,\allowbreak16 Ps 104:5 Pr 8:29 Jer 31:37}
\crossref{Mic}{6}{3}{6:5 Ps 50:7; 81:8,\allowbreak13}
\crossref{Mic}{6}{4}{Ex 12:51; 14:30,\allowbreak31; 20:2 De 4:20,\allowbreak34; 5:6; 9:26 Ne 9:9-\allowbreak11}
\crossref{Mic}{6}{5}{De 8:2,\allowbreak18; 9:7; 16:3 Ps 103:1,\allowbreak2; 111:4 Eph 2:11}
\crossref{Mic}{6}{6}{2Sa 21:3 Mt 19:16 Lu 10:25 Joh 6:26 Ac 2:37; 16:30 Ro 10:2,\allowbreak3}
\crossref{Mic}{6}{7}{1Sa 15:22 Ps 10:8-\allowbreak13; 50:9; 51:16 Isa 1:11-\allowbreak15; 40:16 Jer 7:21,\allowbreak22}
\crossref{Mic}{6}{8}{Ro 9:20 1Co 7:16 Jas 2:20}
\crossref{Mic}{6}{9}{Mic 3:12 Isa 24:10-\allowbreak12; 27:10; 32:13,\allowbreak14; 40:6-\allowbreak8; 66:6 Jer 19:11-\allowbreak13}
\crossref{Mic}{6}{10}{Le 19:35,\allowbreak36 De 25:13-\allowbreak16 Pr 11:1; 20:10,\allowbreak23 Eze 45:9-\allowbreak12 Ho 12:7,\allowbreak8}
\crossref{Mic}{6}{11}{Ho 12:7}
\crossref{Mic}{6}{12}{Mic 2:1,\allowbreak2; 3:1-\allowbreak3,\allowbreak9-\allowbreak11; 7:2-\allowbreak6 Isa 1:23; 5:7 Jer 5:5,\allowbreak6,\allowbreak26-\allowbreak29; 6:6,\allowbreak7}
\crossref{Mic}{6}{13}{Le 26:16 De 28:21,\allowbreak22 Job 33:19-\allowbreak22 Ps 107:17,\allowbreak18 Isa 1:5,\allowbreak6}
\crossref{Mic}{6}{14}{Le 26:26 Isa 65:13 Eze 4:16,\allowbreak17 Ho 4:10 Hag 1:6; 2:16}
\crossref{Mic}{6}{15}{Le 26:20 De 28:38-\allowbreak40 Isa 62:8,\allowbreak9; 65:21,\allowbreak22 Jer 12:13 Joe 1:10-\allowbreak12}
\crossref{Mic}{6}{16}{1Ki 16:30-\allowbreak33; 18:4; 21:25,\allowbreak26 2Ki 16:3; 21:3 Isa 9:16 Re 2:20}
\crossref{Mic}{7}{1}{Ps 120:5 Isa 6:5; 24:16 Jer 4:31; 15:10; 45:3}
\crossref{Mic}{7}{2}{Ps 12:1; 14:1-\allowbreak3 Isa 57:1 Ro 3:10-\allowbreak18}
\crossref{Mic}{7}{3}{Pr 4:16,\allowbreak17 Jer 3:5 Eze 22:6}
\crossref{Mic}{7}{4}{2Sa 23:6,\allowbreak7 Isa 55:13 Eze 2:6 Heb 6:8}
\crossref{Mic}{7}{5}{Job 6:14,\allowbreak15 Ps 118:8,\allowbreak9 Jer 9:4 Mt 10:16}
\crossref{Mic}{7}{6}{Ge 9:22-\allowbreak24; 49:4 2Sa 15:10-\allowbreak12; 16:11,\allowbreak21-\allowbreak23 Pr 30:11,\allowbreak17}
\crossref{Mic}{7}{7}{Ps 34:5,\allowbreak6; 55:16,\allowbreak17; 109:4; 142:4,\allowbreak5 Isa 8:17; 45:22 Hab 3:17-\allowbreak19}
\crossref{Mic}{7}{8}{Job 31:29 Ps 13:4-\allowbreak6; 35:15,\allowbreak16,\allowbreak19,\allowbreak24-\allowbreak26; 38:16 Pr 24:17,\allowbreak18}
\crossref{Mic}{7}{9}{Le 26:41 1Sa 3:18 2Sa 16:11,\allowbreak12; 24:17 Job 34:31,\allowbreak32 La 1:18}
\crossref{Mic}{7}{10}{Ps 35:26; 109:29 Jer 51:51 Eze 7:18 Ob 1:10}
\crossref{Mic}{7}{11}{Ne 2:17; 3:1-\allowbreak16; 4:3,\allowbreak6 Da 9:25 Am 9:11-\allowbreak15}
\crossref{Mic}{7}{12}{Isa 11:16; 19:23-\allowbreak25; 27:12,\allowbreak13; 43:6; 49:12; 60:4-\allowbreak9; 66:19,\allowbreak20}
\crossref{Mic}{7}{13}{Mic 3:12 Job 4:8 Pr 1:31; 5:22; 31:31 Isa 3:10,\allowbreak11 Jer 17:10; 21:14}
\crossref{Mic}{7}{14}{Mic 5:4}
\crossref{Mic}{7}{15}{Ps 68:22; 78:12-\allowbreak72 Isa 11:16; 51:9; 63:11-\allowbreak15 Jer 23:7,\allowbreak8}
\crossref{Mic}{7}{16}{Mic 5:8 Ps 126:2 Isa 26:11; 66:18 Eze 38:23; 39:17-\allowbreak21 Zec 8:20-\allowbreak23}
\crossref{Mic}{7}{17}{Ge 3:14,\allowbreak15 Ps 72:9 Isa 49:23; 60:14; 65:25 La 3:29 Re 3:9}
\crossref{Mic}{7}{18}{Ex 15:11 De 33:26 1Ki 8:23 Ps 35:10; 71:19; 89:6,\allowbreak8; 113:5,\allowbreak6}
\crossref{Mic}{7}{19}{De 30:3; 32:36 Ezr 9:8,\allowbreak9 Ps 90:13,\allowbreak14 Isa 63:15-\allowbreak17 Jer 31:20}
\crossref{Mic}{7}{20}{Ge 12:2,\allowbreak3; 17:7,\allowbreak8; 22:16-\allowbreak18; 26:3,\allowbreak4; 28:13,\allowbreak14 Ps 105:8-\allowbreak10}

% Nah
\crossref{Nah}{1}{1}{Isa 13:1; 14:28; 15:1; 21:1; 22:1; 23:1 Jer 23:33-\allowbreak37 Zec 9:1}
\crossref{Nah}{1}{2}{De 32:35,\allowbreak42 Ps 94:1 Isa 59:17,\allowbreak18 Ro 12:19; 13:4 Heb 10:30}
\crossref{Nah}{1}{3}{Ex 34:6,\allowbreak7 Ne 9:17 Ps 103:8; 145:8 Joe 2:13 Jon 4:2 Jas 1:19}
\crossref{Nah}{1}{4}{Job 38:11 Ps 104:7; 106:9; 114:3,\allowbreak5 Isa 50:2,\allowbreak3; 51:10 Am 5:8}
\crossref{Nah}{1}{5}{2Sa 22:8 Ps 29:5,\allowbreak6; 68:8; 97:4,\allowbreak5; 114:4,\allowbreak6 Isa 2:12-\allowbreak14 Jer 4:24}
\crossref{Nah}{1}{6}{Ps 2:12; 76:7; 90:11 Isa 27:4 Jer 10:10 Mal 3:2 Re 6:17}
\crossref{Nah}{1}{7}{1Ch 16:34 Ezr 3:11 Ps 25:8; 100:5; 136:1-\allowbreak26; 145:6-\allowbreak10 Jer 33:11}
\crossref{Nah}{1}{8}{Isa 8:7,\allowbreak8; 28:17 Eze 13:13 Da 9:26; 11:10,\allowbreak22,\allowbreak40 Am 8:8; 9:5,\allowbreak6}
\crossref{Nah}{1}{9}{1:11 Ps 2:1-\allowbreak4; 21:11; 33:10 Pr 21:30 Isa 8:9,\allowbreak10 Eze 38:10,\allowbreak11}
\crossref{Nah}{1}{10}{2Sa 23:6,\allowbreak7 Mic 7:4 1Th 5:2,\allowbreak3}
\crossref{Nah}{1}{11}{1:9 2Ki 18:13,\allowbreak14,\allowbreak30; 19:22-\allowbreak25 2Ch 32:15-\allowbreak19 Isa 10:7-\allowbreak15}
\crossref{Nah}{1}{12}{Isa 7:20}
\crossref{Nah}{1}{13}{Isa 9:4; 10:27; 14:25 Jer 2:20 Mic 5:5,\allowbreak6}
\crossref{Nah}{1}{14}{Ps 71:3 Isa 33:13}
\crossref{Nah}{1}{15}{Isa 40:9,\allowbreak10; 52:7 Lu 2:10,\allowbreak14 Ac 10:36 Ro 10:15}
\crossref{Nah}{2}{1}{Isa 14:6 Jer 25:9; 50:23; 51:20-\allowbreak23}
\crossref{Nah}{2}{2}{Isa 10:5-\allowbreak12 Jer 25:29}
\crossref{Nah}{2}{3}{Isa 63:1-\allowbreak3 Zec 1:8; 6:2 Re 6:4; 12:3}
\crossref{Nah}{2}{4}{Na 3:2,\allowbreak3 Isa 37:24; 66:15 Jer 4:13 Eze 26:10 Da 11:40}
\crossref{Nah}{2}{5}{Isa 21:5 Jer 50:29; 51:27,\allowbreak28}
\crossref{Nah}{2}{6}{Isa 45:1,\allowbreak2}
\crossref{Nah}{2}{7}{}
\crossref{Nah}{2}{8}{Ge 10:11}
\crossref{Nah}{2}{9}{Isa 33:1,\allowbreak4 Jer 51:56}
\crossref{Nah}{2}{10}{Na 3:7 Ge 1:2 Isa 13:19-\allowbreak22; 14:23; 24:1; 34:10-\allowbreak15 Jer 4:23-\allowbreak26}
\crossref{Nah}{2}{11}{Na 3:1 Job 4:10,\allowbreak11 Isa 5:29 Jer 2:15; 4:7; 50:17,\allowbreak44 Eze 19:2-\allowbreak8}
\crossref{Nah}{2}{12}{Ps 17:12 Isa 10:6-\allowbreak14 Jer 51:34}
\crossref{Nah}{2}{13}{Na 3:5 Jer 21:13; 50:31; 51:25 Eze 5:8; 26:3; 28:22; 29:3,\allowbreak10}
\crossref{Nah}{3}{1}{Isa 24:2 Eze 22:2,\allowbreak3; 24:6-\allowbreak9 Hab 2:12 Zep 3:1-\allowbreak3}
\crossref{Nah}{3}{2}{Na 2:3,\allowbreak4 Jud 5:22 Job 39:22-\allowbreak25 Isa 9:5 Jer 47:3}
\crossref{Nah}{3}{3}{Isa 37:36 Eze 31:3-\allowbreak13; 39:4}
\crossref{Nah}{3}{4}{Isa 23:15-\allowbreak17; 47:9,\allowbreak12,\allowbreak13 Re 17:1-\allowbreak5; 18:2,\allowbreak3,\allowbreak9,\allowbreak23}
\crossref{Nah}{3}{5}{Na 2:13 Eze 23:25}
\crossref{Nah}{3}{6}{Job 9:31; 30:19 Ps 38:5-\allowbreak7 La 3:16 Mal 2:2 1Co 4:13}
\crossref{Nah}{3}{7}{Nu 16:34 Jer 51:9 Re 18:10}
\crossref{Nah}{3}{8}{Eze 31:2,\allowbreak3 Am 6:2}
\crossref{Nah}{3}{9}{Isa 20:5 Jer 46:9}
\crossref{Nah}{3}{10}{Ps 33:16,\allowbreak17 Isa 20:4}
\crossref{Nah}{3}{11}{Na 1:10 Ps 75:8 Isa 29:9; 49:26; 63:6 Jer 25:15-\allowbreak27; 51:57}
\crossref{Nah}{3}{12}{Hab 1:10 Re 6:13}
\crossref{Nah}{3}{13}{Isa 19:16 Jer 50:37; 51:30}
\crossref{Nah}{3}{14}{2Ch 32:3,\allowbreak4,\allowbreak11 Isa 22:9-\allowbreak11; 37:25}
\crossref{Nah}{3}{15}{3:13; 2:13 Zep 2:13}
\crossref{Nah}{3}{16}{Ge 15:5; 22:17 Ne 9:23 Jer 33:22}
\crossref{Nah}{3}{17}{Re 9:7}
\crossref{Nah}{3}{18}{Jer 50:18 Eze 31:3-\allowbreak18; 32:22,\allowbreak23}
\crossref{Nah}{3}{19}{Jer 30:13-\allowbreak15; 46:11 Eze 30:21,\allowbreak22 Mic 1:9 Zep 2:13-\allowbreak15}

% Hab
\crossref{Hab}{1}{1}{Isa 22:1 Na 1:1}
\crossref{Hab}{1}{2}{Ps 13:1,\allowbreak2; 74:9,\allowbreak10; 94:3 Re 6:10}
\crossref{Hab}{1}{3}{Ps 12:1,\allowbreak2; 55:9-\allowbreak11; 73:3-\allowbreak9; 120:5,\allowbreak6 Ec 4:1; 5:8 Jer 9:2-\allowbreak6 Eze 2:6}
\crossref{Hab}{1}{4}{Ps 11:3; 119:126 Mr 7:9 Ro 3:31}
\crossref{Hab}{1}{5}{De 4:27 Jer 9:25,\allowbreak26; 25:14-\allowbreak29}
\crossref{Hab}{1}{6}{De 28:49-\allowbreak52 2Ki 24:2 2Ch 36:6,\allowbreak17 Isa 23:13; 39:6,\allowbreak7 Jer 1:15,\allowbreak16}
\crossref{Hab}{1}{7}{}
\crossref{Hab}{1}{8}{De 28:49 Isa 5:26-\allowbreak28}
\crossref{Hab}{1}{9}{1:6; 2:5-\allowbreak13 De 28:51,\allowbreak52 Jer 4:7; 5:15-\allowbreak17; 25:9}
\crossref{Hab}{1}{10}{2Ki 24:12; 25:6,\allowbreak7 2Ch 36:6,\allowbreak10}
\crossref{Hab}{1}{11}{Da 4:30-\allowbreak34}
\crossref{Hab}{1}{12}{De 33:27 Ps 90:2; 93:2,\allowbreak Isa 40:28; 57:15 La 5:19 Mic 5:2}
\crossref{Hab}{1}{13}{Job 15:15 Ps 5:4,\allowbreak5; 11:4-\allowbreak7; 34:15,\allowbreak16 1Pe 1:15,\allowbreak16}
\crossref{Hab}{1}{14}{Pr 6:7}
\crossref{Hab}{1}{15}{Jer 16:16 Eze 29:4,\allowbreak5 Am 4:2 Mt 17:27}
\crossref{Hab}{1}{16}{1:11 De 8:17 Isa 10:13-\allowbreak15; 37:24 Eze 28:3; 29:3 Da 4:30; 5:23}
\crossref{Hab}{1}{17}{1:9,\allowbreak10; 2:5-\allowbreak8,\allowbreak17 Isa 14:16,\allowbreak17 Jer 25:9-\allowbreak26; 46:1-\allowbreak49:39; 52:1-\allowbreak34}
\crossref{Hab}{2}{1}{Ps 73:16,\allowbreak17 Isa 21:8,\allowbreak11,\allowbreak12}
\crossref{Hab}{2}{2}{De 27:8; 31:19,\allowbreak22 Isa 8:1; 30:8 Jer 36:2-\allowbreak4,\allowbreak27-\allowbreak32 Da 12:4}
\crossref{Hab}{2}{3}{Jer 27:7 Da 8:19; 9:24-\allowbreak27; 10:1,\allowbreak14; 11:27,\allowbreak35 Ac 1:7; 17:26 Ga 4:2}
\crossref{Hab}{2}{4}{Job 40:11,\allowbreak12 Da 4:30,\allowbreak37; 5:20-\allowbreak23 Lu 18:14 2Th 2:4 1Pe 5:5}
\crossref{Hab}{2}{5}{Pr 20:1; 23:29-\allowbreak33; 31:4,\allowbreak5 Isa 5:11,\allowbreak12,\allowbreak22,\allowbreak23; 21:5 Jer 51:39}
\crossref{Hab}{2}{6}{Nu 23:7,\allowbreak18 Isa 14:4-\allowbreak19 Jer 29:22; 50:13 Eze 32:21 Mic 2:4}
\crossref{Hab}{2}{7}{Pr 29:1 Isa 13:1-\allowbreak5,\allowbreak16-\allowbreak18; 21:2-\allowbreak9; 41:25; 45:1-\allowbreak3; 46:11; 47:11}
\crossref{Hab}{2}{8}{2:10,\allowbreak17 Isa 33:1,\allowbreak4 Jer 27:7; 30:16; 50:10,\allowbreak37; 51:13,\allowbreak44,\allowbreak48,\allowbreak55,\allowbreak56}
\crossref{Hab}{2}{9}{Ps 10:3-\allowbreak6; 49:11; 52:7 Pr 18:11,\allowbreak12 Isa 28:15; 47:7-\allowbreak9 Jer 49:16}
\crossref{Hab}{2}{10}{2Ki 9:26; 10:7 Isa 14:20-\allowbreak22 Jer 22:30; 36:31 Na 1:14 Mt 27:25}
\crossref{Hab}{2}{11}{Ge 4:10 Jos 24:27 Job 31:38-\allowbreak40 Lu 19:40 Heb 12:24 Jas 5:3,\allowbreak4}
\crossref{Hab}{2}{12}{Ge 4:11-\allowbreak17 Jos 6:26 1Ki 16:34 Jer 22:13-\allowbreak17 Eze 24:9 Da 4:27-\allowbreak31}
\crossref{Hab}{2}{13}{Ge 11:6-\allowbreak9 2Sa 15:31 Job 5:13,\allowbreak14 Ps 39:6; 127:1,\allowbreak2 Pr 21:30}
\crossref{Hab}{2}{14}{Ps 22:27; 67:1,\allowbreak2; 72:19; 86:9; 98:1-\allowbreak3 Isa 6:3; 11:9 Zec 14:8,\allowbreak9}
\crossref{Hab}{2}{15}{Ge 19:32-\allowbreak35 2Sa 11:13; 13:26-\allowbreak28 Jer 25:15; 51:7 Re 17:2,\allowbreak6; 18:3}
\crossref{Hab}{2}{16}{Pr 3:35 Isa 47:3 Ho 4:7 Php 3:19}
\crossref{Hab}{2}{17}{Zec 11:1}
\crossref{Hab}{2}{18}{Isa 37:38; 42:17; 44:9,\allowbreak10; 45:16,\allowbreak20; 46:1,\allowbreak2,\allowbreak6-\allowbreak8 Jer 2:27,\allowbreak28; 10:3-\allowbreak5}
\crossref{Hab}{2}{19}{1Ki 18:26-\allowbreak29 Ps 97:7 Isa 44:17 Jer 51:47 Da 3:7,\allowbreak18,\allowbreak29; 5:23}
\crossref{Hab}{2}{20}{Ps 11:4; 115:3; 132:13,\allowbreak14 Isa 6:1; 66:1,\allowbreak6 Jon 2:4,\allowbreak7 Eph 2:21,\allowbreak22}
\crossref{Hab}{3}{1}{Ps 86:1-\allowbreak17}
\crossref{Hab}{3}{2}{3:16; 1:5-\allowbreak10 Ex 9:20,\allowbreak21 2Ch 34:27,\allowbreak28 Job 4:12-\allowbreak21 Ps 119:120}
\crossref{Hab}{3}{3}{Jud 5:4,\allowbreak5 Ps 68:7,\allowbreak8 Isa 64:3}
\crossref{Hab}{3}{4}{Ex 13:21; 14:20 Ne 9:12 Ps 104:2 Isa 60:19,\allowbreak20 Mt 17:2 1Ti 6:16}
\crossref{Hab}{3}{5}{Ex 12:29,\allowbreak30 Nu 14:12; 16:46-\allowbreak49 Ps 78:50,\allowbreak51 Na 1:2,\allowbreak3}
\crossref{Hab}{3}{6}{Ex 15:17; 21:31 Nu 34:1-\allowbreak29 De 32:8 Ac 17:26}
\crossref{Hab}{3}{7}{Ex 15:14-\allowbreak16 Nu 22:3,\allowbreak4 Jos 2:10; 9:24}
\crossref{Hab}{3}{8}{Ex 14:21,\allowbreak22 Jos 3:16,\allowbreak17 Ps 114:3,\allowbreak5 Isa 50:2 Na 1:4 Mr 4:39}
\crossref{Hab}{3}{9}{De 32:23 Ps 7:12,\allowbreak13; 35:1-\allowbreak3 Isa 51:9,\allowbreak10; 52:10 La 2:4}
\crossref{Hab}{3}{10}{3:6 Ex 19:16-\allowbreak18 Jud 5:4,\allowbreak5 Ps 68:7,\allowbreak8; 77:18; 97:4,\allowbreak5; 114:4,\allowbreak6}
\crossref{Hab}{3}{11}{Jos 10:12,\allowbreak13 Isa 28:21; 38:8}
\crossref{Hab}{3}{12}{Nu 21:23-\allowbreak35 Jos 6:1-\allowbreak12:24 Ne 9:22-\allowbreak24 Ps 44:1-\allowbreak3; 78:55 Ac 13:19}
\crossref{Hab}{3}{13}{Ex 14:13,\allowbreak14; 15:1,\allowbreak2 Ps 68:7,\allowbreak19-\allowbreak23}
\crossref{Hab}{3}{14}{Ex 11:4-\allowbreak7; 12:12,\allowbreak13,\allowbreak29,\allowbreak30; 14:17,\allowbreak18 Ps 78:50,\allowbreak51; 83:9-\allowbreak11}
\crossref{Hab}{3}{15}{3:8 Ps 77:19}
\crossref{Hab}{3}{16}{3:2; 1:5-\allowbreak11}
\crossref{Hab}{3}{17}{De 28:15-\allowbreak18,\allowbreak30-\allowbreak41 Jer 14:2-\allowbreak8 Joe 1:10-\allowbreak13,\allowbreak16-\allowbreak18 Am 4:6-\allowbreak10}
\crossref{Hab}{3}{18}{De 12:18 1Sa 2:1 Job 13:15 Ps 33:1; 46:1-\allowbreak5; 85:6; 97:12; 104:34}
\crossref{Hab}{3}{19}{Ps 18:1; 27:1; 46:1 Isa 12:2; 45:24 Zec 10:12 2Co 12:9,\allowbreak10}

% Zeph
\crossref{Zeph}{1}{1}{Eze 1:3 Ho 1:1 2Ti 3:16 2Pe 1:19}
\crossref{Zeph}{1}{2}{}
\crossref{Zeph}{1}{3}{Jer 4:23-\allowbreak29; 12:4 Ho 4:3}
\crossref{Zeph}{1}{4}{Ex 15:12 2Ki 21:13 Isa 14:26,\allowbreak27}
\crossref{Zeph}{1}{5}{2Ki 23:12 Jer 19:13; 32:29}
\crossref{Zeph}{1}{6}{1Sa 15:11 Ps 36:3; 125:5 Isa 1:4 Jer 2:13,\allowbreak17; 3:10; 15:6 Eze 3:20}
\crossref{Zeph}{1}{7}{1Sa 2:9,\allowbreak10 Job 40:4,\allowbreak5 Ps 46:10; 76:8,\allowbreak9 Isa 6:5 Am 6:10 Hab 2:20}
\crossref{Zeph}{1}{8}{Isa 10:12; 24:21}
\crossref{Zeph}{1}{9}{1Sa 5:5}
\crossref{Zeph}{1}{10}{1:7,\allowbreak15 Jer 39:2}
\crossref{Zeph}{1}{11}{Jer 4:8; 25:34 Eze 21:12 Joe 1:5,\allowbreak13 Zec 11:2,\allowbreak3 Jas 5:1}
\crossref{Zeph}{1}{12}{Jer 16:16,\allowbreak17 Am 9:1-\allowbreak3 Ob 1:6}
\crossref{Zeph}{1}{13}{1:9 Isa 6:11; 24:1-\allowbreak3 Jer 4:7,\allowbreak20; 5:17; 9:11,\allowbreak19; 12:10-\allowbreak13 Eze 7:19,\allowbreak21}
\crossref{Zeph}{1}{14}{1:7 Jer 30:7 Eze 30:3 Joe 2:1,\allowbreak11,\allowbreak31 Mal 4:5 Ac 2:20 Re 6:17}
\crossref{Zeph}{1}{15}{1:18; 2:2 Isa 22:5 Jer 30:7 Am 5:18-\allowbreak20 Lu 21:22,\allowbreak23 Ro 2:5 2Pe 3:7}
\crossref{Zeph}{1}{16}{Isa 59:10 Jer 4:19,\allowbreak20; 6:1; 8:16 Ho 5:8; 8:1 Am 3:6 Hab 1:6-\allowbreak10}
\crossref{Zeph}{1}{17}{De 28:28,\allowbreak29 Ps 79:3 Isa 29:10; 59:9,\allowbreak10 La 4:14 Mt 15:14}
\crossref{Zeph}{1}{18}{1:11 Ps 49:6-\allowbreak9; 52:5-\allowbreak7 Pr 11:4; 18:11 Isa 2:20,\allowbreak21 Jer 9:23,\allowbreak24}
\crossref{Zeph}{2}{1}{2Ch 20:4 Ne 8:1; 9:1 Es 4:16 Joe 1:14; 2:12-\allowbreak18 Mt 18:20}
\crossref{Zeph}{2}{2}{Zep 3:8 2Ki 22:16,\allowbreak17; 23:26,\allowbreak27 Eze 12:25 Mt 24:35 2Pe 3:4-\allowbreak10}
\crossref{Zeph}{2}{3}{Ps 105:4 Isa 55:6,\allowbreak7 Jer 3:13,\allowbreak14; 4:1,\allowbreak2; 29:12,\allowbreak13 Ho 7:10; 10:12}
\crossref{Zeph}{2}{4}{Jer 25:20; 47:1-\allowbreak7 Eze 25:15-\allowbreak17 Am 1:6-\allowbreak8 Zec 9:5-\allowbreak7}
\crossref{Zeph}{2}{5}{Jer 47:7 Eze 25:16}
\crossref{Zeph}{2}{6}{2:14,\allowbreak15 Isa 17:2 Eze 25:5}
\crossref{Zeph}{2}{7}{Isa 14:29-\allowbreak32 Ob 1:19 Zec 9:6,\allowbreak7 Ac 8:26,\allowbreak40}
\crossref{Zeph}{2}{8}{Jer 48:27-\allowbreak29 Eze 25:8-\allowbreak11}
\crossref{Zeph}{2}{9}{Nu 14:21 Isa 49:18 Jer 46:18 Ro 14:11}
\crossref{Zeph}{2}{10}{2:8 Isa 16:6 Jer 48:29 Da 4:37; 5:20-\allowbreak23 Ob 1:3 1Pe 5:5}
\crossref{Zeph}{2}{11}{De 32:38 Ho 2:17 Zec 13:2}
\crossref{Zeph}{2}{12}{Isa 18:1-\allowbreak7; 20:4,\allowbreak5; 43:3 Jer 46:9,\allowbreak10 Eze 30:4-\allowbreak9}
\crossref{Zeph}{2}{13}{Ps 83:8,\allowbreak9 Isa 10:12,\allowbreak16; 11:11 Eze 31:3-\allowbreak18}
\crossref{Zeph}{2}{14}{2:6 Isa 13:19-\allowbreak22; 34:11-\allowbreak17 Re 18:2}
\crossref{Zeph}{2}{15}{Isa 10:12-\allowbreak14; 22:2; 47:7 Re 18:7-\allowbreak10}
\crossref{Zeph}{3}{1}{Le 1:16}
\crossref{Zeph}{3}{2}{De 28:15-\allowbreak68 Ne 9:26 Jer 7:23-\allowbreak28; 22:21 Zec 7:11-\allowbreak14}
\crossref{Zeph}{3}{3}{Job 4:8-\allowbreak11 Ps 10:8-\allowbreak10 Pr 28:15 Isa 1:23 Jer 22:17}
\crossref{Zeph}{3}{4}{Isa 9:15; 56:10-\allowbreak12 Jer 5:31; 6:13,\allowbreak14; 8:10; 14:13-\allowbreak15; 23:9-\allowbreak17,\allowbreak25-\allowbreak27}
\crossref{Zeph}{3}{5}{De 32:4 Ps 99:3,\allowbreak4; 145:17 Ec 3:16,\allowbreak17 Isa 45:21 Hab 1:3 Zec 9:9}
\crossref{Zeph}{3}{6}{Isa 10:1-\allowbreak34; 15:1-\allowbreak16:14; 19:1-\allowbreak25; 37:11-\allowbreak13,\allowbreak24-\allowbreak26,\allowbreak36 Jer 25:9-\allowbreak11}
\crossref{Zeph}{3}{7}{3:2 Isa 5:4; 63:8 Jer 8:6; 36:3 Lu 19:42-\allowbreak44 2Pe 3:9}
\crossref{Zeph}{3}{8}{Ps 27:14; 37:7,\allowbreak34; 62:1,\allowbreak5; 123:2; 130:5,\allowbreak6 Pr 20:22 Isa 30:18}
\crossref{Zeph}{3}{9}{Isa 19:18 Mt 12:35 Eph 4:29}
\crossref{Zeph}{3}{10}{Ps 68:31; 72:8-\allowbreak11 Isa 11:11; 18:1,\allowbreak7-\allowbreak19:15; 27:12,\allowbreak13; 49:20-\allowbreak23}
\crossref{Zeph}{3}{11}{3:19,\allowbreak20 Ps 49:5 Isa 45:17; 54:4; 61:7; 65:13,\allowbreak14 Joe 2:26,\allowbreak27 Ro 9:33}
\crossref{Zeph}{3}{12}{Isa 14:32; 61:1-\allowbreak3 Zec 11:11; 13:8,\allowbreak9 Mt 5:3; 11:5 1Co 1:27,\allowbreak28}
\crossref{Zeph}{3}{13}{Zep 2:7 Isa 6:13; 10:20-\allowbreak22 Mic 4:7 Ro 11:4-\allowbreak7}
\crossref{Zeph}{3}{14}{Ezr 3:11-\allowbreak13 Ne 12:43 Ps 14:7; 47:5-\allowbreak7; 81:1-\allowbreak3; 95:1,\allowbreak2; 100:1,\allowbreak2}
\crossref{Zeph}{3}{15}{Ge 30:23 Ps 85:3 Isa 25:8; 40:1,\allowbreak2; 51:22 Mic 7:18-\allowbreak20 Zec 1:14-\allowbreak16}
\crossref{Zeph}{3}{16}{Isa 35:3,\allowbreak4; 40:9; 41:10,\allowbreak13,\allowbreak14; 43:1,\allowbreak2; 44:2; 54:4 Jer 46:27,\allowbreak28}
\crossref{Zeph}{3}{17}{3:5,\allowbreak15}
\crossref{Zeph}{3}{18}{3:20 Jer 23:3; 31:8,\allowbreak9 Eze 34:13; 36:24 Ho 1:11 Ro 11:25,\allowbreak26}
\crossref{Zeph}{3}{19}{3:15 Isa 25:9-\allowbreak12; 26:11; 41:11-\allowbreak16; 43:14-\allowbreak17; 49:25,\allowbreak26; 51:22,\allowbreak23}
\crossref{Zeph}{3}{20}{Isa 11:11,\allowbreak12; 27:12,\allowbreak13; 56:8 Eze 28:25; 34:16; 37:21; 39:28 Am 9:14}

% Hag
\crossref{Hag}{1}{1}{Hag 2:1,\allowbreak10,\allowbreak20 Ezr 4:24; 5:1,\allowbreak2 Zec 1:1}
\crossref{Hag}{1}{2}{Nu 13:31 Ezr 4:23,\allowbreak24; 5:1,\allowbreak2 Ne 4:10 Pr 22:13; 26:13-\allowbreak16; 29:25}
\crossref{Hag}{1}{3}{Ezr 5:1 Zec 1:1}
\crossref{Hag}{1}{4}{2Sa 7:2 Ps 132:3-\allowbreak5 Mt 6:33 Php 2:21}
\crossref{Hag}{1}{5}{1:7; 2:15-\allowbreak18 La 3:40 Eze 18:28 Lu 15:17 2Co 13:5 Ga 6:4}
\crossref{Hag}{1}{6}{1:9; 2:16 Le 26:20 De 28:38-\allowbreak40 2Sa 21:1 Ps 107:34 Isa 5:10}
\crossref{Hag}{1}{7}{1:5 Ps 119:59,\allowbreak60 Isa 28:10 Php 3:1}
\crossref{Hag}{1}{8}{2Ch 2:8-\allowbreak10 Ezr 3:7; 6:4 Zec 11:1,\allowbreak2}
\crossref{Hag}{1}{9}{2Sa 22:16 2Ki 19:7 Isa 40:7 Mal 2:2}
\crossref{Hag}{1}{10}{Le 26:19 De 28:23,\allowbreak24 1Ki 8:35; 17:1 Jer 14:1-\allowbreak6 Ho 2:9}
\crossref{Hag}{1}{11}{De 28:22 1Ki 17:1 2Ki 8:1 Job 34:29 La 1:21 Am 5:8; 7:4; 9:6}
\crossref{Hag}{1}{12}{1:14 Ezr 5:2 Isa 55:10,\allowbreak11 Col 1:6 1Th 1:5,\allowbreak6; 2:13,\allowbreak14}
\crossref{Hag}{1}{13}{Jud 2:1}
\crossref{Hag}{1}{14}{1Ch 5:26 2Ch 36:22 Ezr 1:1,\allowbreak5; 7:27,\allowbreak28 Ps 110:3 1Co 12:4-\allowbreak11}
\crossref{Hag}{1}{15}{1:1; 2:1,\allowbreak10,\allowbreak20}
\crossref{Hag}{2}{1}{2:10,\allowbreak20; 1:15}
\crossref{Hag}{2}{2}{Hag 1:14 Ezr 1:8; 2:63 Ne 8:9}
\crossref{Hag}{2}{3}{Ezr 3:12 Zec 4:9,\allowbreak10}
\crossref{Hag}{2}{4}{De 31:23 Jos 1:6,\allowbreak9 1Ch 22:13; 28:20 Zec 8:9 1Co 16:13 Eph 6:10}
\crossref{Hag}{2}{5}{Ex 29:45,\allowbreak46; 33:12-\allowbreak14; 34:8,\allowbreak10}
\crossref{Hag}{2}{6}{2:21,\allowbreak22 Heb 12:26-\allowbreak28}
\crossref{Hag}{2}{7}{Eze 21:27 Da 2:44,\allowbreak45; 7:20-\allowbreak25 Joe 3:9-\allowbreak16 Lu 21:10,\allowbreak11}
\crossref{Hag}{2}{8}{1Ki 6:20-\allowbreak35 1Ch 29:14-\allowbreak16 Ps 24:1; 50:10-\allowbreak12 Isa 60:13,\allowbreak17}
\crossref{Hag}{2}{9}{Ps 24:7-\allowbreak10 Joh 1:14 2Co 3:9,\allowbreak10 1Ti 3:16 Jas 2:1}
\crossref{Hag}{2}{10}{2:1,\allowbreak20; 1:1,\allowbreak15}
\crossref{Hag}{2}{11}{Le 10:10,\allowbreak11 De 33:10 Eze 44:23,\allowbreak24 Mal 2:7 Tit 1:9}
\crossref{Hag}{2}{12}{Ex 29:37 Le 6:27,\allowbreak29; 7:6 Eze 44:19 Mt 23:19}
\crossref{Hag}{2}{13}{Nu 5:2,\allowbreak3; 9:6-\allowbreak10; 19:11-\allowbreak22}
\crossref{Hag}{2}{14}{Hag 1:4-\allowbreak11 Pr 15:8; 21:4,\allowbreak27; 28:9 Isa 1:11-\allowbreak15 Tit 1:15 Jude 1:23}
\crossref{Hag}{2}{15}{2:18; 1:5,\allowbreak7 Ps 107:43 Isa 5:12 Ho 14:9 Mal 3:8-\allowbreak11 Ro 6:21}
\crossref{Hag}{2}{16}{Hag 1:6,\allowbreak9-\allowbreak11 Pr 3:9,\allowbreak10 Zec 8:10-\allowbreak12 Mal 2:2}
\crossref{Hag}{2}{17}{Hag 1:9 Ge 42:6,\allowbreak23,\allowbreak27 De 28:22 1Ki 8:37 2Ch 6:28 Isa 37:27 Am 4:9}
\crossref{Hag}{2}{18}{2:15 De 32:29 Lu 15:17-\allowbreak20}
\crossref{Hag}{2}{19}{Hab 3:17,\allowbreak18}
\crossref{Hag}{2}{20}{2:10}
\crossref{Hag}{2}{21}{Hag 1:1,\allowbreak14 1Ch 3:19 Ezr 2:2; 5:2 Zec 4:6-\allowbreak10}
\crossref{Hag}{2}{22}{Isa 60:12 Eze 21:27 Da 2:34,\allowbreak35,\allowbreak44,\allowbreak45; 7:25-\allowbreak27; 8:25 Mic 5:8,\allowbreak15}
\crossref{Hag}{2}{23}{So 8:6 Jer 22:24 Joh 6:27 2Ti 2:19}

% Zech
\crossref{Zech}{1}{1}{1:7; 7:1 Ezr 4:24; 6:15 Hag 1:1,\allowbreak15; 2:1,\allowbreak10,\allowbreak20}
\crossref{Zech}{1}{2}{2Ki 22:16,\allowbreak17,\allowbreak19; 23:26 2Ch 36:13-\allowbreak20 Ezr 9:6,\allowbreak7,\allowbreak13 Ne 9:26,\allowbreak27}
\crossref{Zech}{1}{3}{De 4:30,\allowbreak31; 30:2-\allowbreak10 1Ki 8:47,\allowbreak48 2Ch 15:4; 30:6-\allowbreak9 Ne 9:28}
\crossref{Zech}{1}{4}{2Ch 29:6-\allowbreak10; 30:7; 34:21 Ezr 9:7 Ne 9:16 Ps 78:8; 106:6,\allowbreak7}
\crossref{Zech}{1}{5}{Job 14:10-\allowbreak12 Ps 90:10 Ec 1:4; 9:1-\allowbreak3; 12:5,\allowbreak7 Ac 13:36 Heb 7:23,\allowbreak24}
\crossref{Zech}{1}{6}{Isa 55:1}
\crossref{Zech}{1}{7}{1:1}
\crossref{Zech}{1}{8}{Ge 20:3 1Ki 3:5 Job 4:13 Da 2:19; 7:2,\allowbreak13}
\crossref{Zech}{1}{9}{1:19; 4:4,\allowbreak11; 6:4 Da 7:16; 8:15 Re 7:13,\allowbreak14}
\crossref{Zech}{1}{10}{1:8,\allowbreak11; 13:7 Ge 32:24-\allowbreak31 Ho 12:3-\allowbreak5}
\crossref{Zech}{1}{11}{1:8,\allowbreak10 Ps 68:17; 103:20,\allowbreak21 Mt 13:41,\allowbreak49; 24:30,\allowbreak31; 25:31 2Th 1:7}
\crossref{Zech}{1}{12}{1:8,\allowbreak10,\allowbreak11 Ex 23:20-\allowbreak23 Isa 63:9 Heb 7:25}
\crossref{Zech}{1}{13}{1:14-\allowbreak16; 2:4-\allowbreak12; 8:2-\allowbreak8,\allowbreak19 Isa 40:1,\allowbreak2 Jer 29:10; 30:10-\allowbreak22; 31:3-\allowbreak14}
\crossref{Zech}{1}{14}{1:9,\allowbreak13; 2:3,\allowbreak4; 4:1}
\crossref{Zech}{1}{15}{1:2,\allowbreak11 Isa 47:7-\allowbreak9 Jer 48:11-\allowbreak13 Am 6:1 Re 18:7,\allowbreak8}
\crossref{Zech}{1}{16}{Zec 2:10,\allowbreak11; 8:3 Isa 12:1; 54:8-\allowbreak10 Jer 31:22-\allowbreak25; 33:10-\allowbreak12}
\crossref{Zech}{1}{17}{Ne 11:3,\allowbreak20 Ps 69:35 Isa 44:26; 61:4-\allowbreak6 Jer 31:23,\allowbreak24; 32:43,\allowbreak44}
\crossref{Zech}{1}{18}{Zec 2:1; 5:1,\allowbreak5,\allowbreak9 Jos 5:13 Da 8:3}
\crossref{Zech}{1}{19}{1:9,\allowbreak21; 2:2; 4:11-\allowbreak14 Re 7:13,\allowbreak14}
\crossref{Zech}{1}{20}{Zec 9:12-\allowbreak16; 10:3-\allowbreak5; 12:2-\allowbreak6 De 33:25 Jud 11:16,\allowbreak18 1Sa 12:11 Ne 9:27}
\crossref{Zech}{1}{21}{1:19 Da 12:7}
\crossref{Zech}{2}{1}{Zec 1:18}
\crossref{Zech}{2}{2}{Zec 5:10 Joh 16:5}
\crossref{Zech}{2}{3}{Zec 1:9,\allowbreak13,\allowbreak14,\allowbreak19; 4:1,\allowbreak5; 5:5}
\crossref{Zech}{2}{4}{Jer 1:6 Da 1:17 1Ti 4:12}
\crossref{Zech}{2}{5}{Zec 9:8 Ps 46:7-\allowbreak11; 48:3,\allowbreak12 Isa 4:5; 12:6; 26:1,\allowbreak2; 33:21; 60:18,\allowbreak19}
\crossref{Zech}{2}{6}{Ru 4:1 Isa 55:1}
\crossref{Zech}{2}{7}{Ge 19:17 Nu 16:26,\allowbreak34 Isa 48:20; 52:11 Jer 50:8; 51:6,\allowbreak45 Ac 2:40}
\crossref{Zech}{2}{8}{2:4,\allowbreak5; 1:15,\allowbreak16 Isa 60:7-\allowbreak14}
\crossref{Zech}{2}{9}{Isa 10:32; 11:15; 13:2; 19:16}
\crossref{Zech}{2}{10}{Zec 9:9 Ps 47:1-\allowbreak9; 98:1-\allowbreak3 Isa 12:6; 35:10; 40:9; 42:10; 51:11; 52:9,\allowbreak10}
\crossref{Zech}{2}{11}{Zec 8:20-\allowbreak23 Ps 22:27-\allowbreak30; 68:29-\allowbreak31; 72:8-\allowbreak11,\allowbreak17 Isa 2:2-\allowbreak5; 11:9,\allowbreak10}
\crossref{Zech}{2}{12}{Ex 19:5,\allowbreak6 De 32:9 Ps 82:8; 135:4 Jer 10:16; 51:19}
\crossref{Zech}{2}{13}{Ps 46:10 Hab 2:20 Zep 1:7 Ro 3:19; 9:20}
\crossref{Zech}{3}{1}{Zec 1:9,\allowbreak13,\allowbreak19; 2:3}
\crossref{Zech}{3}{2}{Ps 109:31 Lu 22:32 Ro 16:20 1Jo 3:8}
\crossref{Zech}{3}{3}{2Ch 30:18-\allowbreak20 Ezr 9:15 Isa 64:6 Da 9:18 Mt 22:11-\allowbreak13}
\crossref{Zech}{3}{4}{3:1,\allowbreak7 1Ki 22:19 Isa 6:2,\allowbreak3 Lu 1:19 Re 5:11}
\crossref{Zech}{3}{5}{Zec 6:11 Ex 28:2-\allowbreak4; 29:6 Le 8:6-\allowbreak9 Heb 2:8,\allowbreak9 Re 4:4,\allowbreak10; 5:8-\allowbreak14}
\crossref{Zech}{3}{6}{3:1 Ge 22:15,\allowbreak16; 28:13-\allowbreak17; 48:15,\allowbreak16 Ex 23:20,\allowbreak21 Isa 63:9 Ho 12:4}
\crossref{Zech}{3}{7}{Ge 26:5 Le 8:35; 10:3 1Ki 2:3 1Ch 23:32 Eze 44:8,\allowbreak15,\allowbreak16; 48:11}
\crossref{Zech}{3}{8}{Ps 71:7 Isa 8:18; 20:3 1Co 4:9-\allowbreak13}
\crossref{Zech}{3}{9}{Ps 118:22 Isa 8:14,\allowbreak15; 28:16 Mt 21:42-\allowbreak44 Ac 4:11 Ro 9:33}
\crossref{Zech}{3}{10}{Zec 2:11}
\crossref{Zech}{4}{1}{Zec 1:9,\allowbreak13,\allowbreak19; 2:3; 3:6,\allowbreak7}
\crossref{Zech}{4}{2}{Zec 5:2 Jer 1:11-\allowbreak13}
\crossref{Zech}{4}{3}{4:11,\allowbreak12,\allowbreak14 Jud 9:9 Ro 11:17,\allowbreak24 Re 11:4}
\crossref{Zech}{4}{4}{4:12-\allowbreak14; 1:9,\allowbreak19; 5:6; 6:4 Da 7:16-\allowbreak19; 12:8 Mt 13:36 Re 7:13,\allowbreak14}
\crossref{Zech}{4}{5}{4:13 Mr 4:13}
\crossref{Zech}{4}{6}{Zec 9:13-\allowbreak15 Nu 27:16 2Ch 14:11 Isa 11:2-\allowbreak4; 30:1; 32:15; 63:10-\allowbreak14}
\crossref{Zech}{4}{7}{Zec 14:4,\allowbreak5 Ps 114:4,\allowbreak6 Isa 40:3,\allowbreak4; 41:15; 64:1-\allowbreak3 Jer 51:25 Da 2:34,\allowbreak35}
\crossref{Zech}{4}{8}{}
\crossref{Zech}{4}{9}{Ezr 3:8-\allowbreak13; 5:16}
\crossref{Zech}{4}{10}{Ezr 3:12,\allowbreak13 Ne 4:2-\allowbreak4 Job 8:7 Pr 4:18 Da 2:34,\allowbreak35 Ho 6:3 Hag 2:3}
\crossref{Zech}{4}{11}{4:3 Re 11:4}
\crossref{Zech}{4}{12}{Mt 20:23 Re 11:4}
\crossref{Zech}{4}{13}{4:5 Heb 5:11,\allowbreak12}
\crossref{Zech}{4}{14}{Zec 6:13 Ex 29:7; 40:15 Le 8:12 1Sa 10:1; 16:1,\allowbreak12,\allowbreak13 Ps 2:6}
\crossref{Zech}{5}{1}{5:2 Isa 8:1 Jer 36:1-\allowbreak6,\allowbreak20-\allowbreak24,\allowbreak27-\allowbreak32 Eze 2:9,\allowbreak10 Re 5:1-\allowbreak14}
\crossref{Zech}{5}{2}{Zec 4:2 Jer 1:11-\allowbreak14 Am 7:8}
\crossref{Zech}{5}{3}{De 11:28,\allowbreak29; 27:15-\allowbreak26; 28:15-\allowbreak68; 29:19-\allowbreak28 Ps 109:17-\allowbreak20 Pr 3:33}
\crossref{Zech}{5}{4}{Le 14:34-\allowbreak45 De 7:26 Job 18:15; 20:26 Pr 3:33 Hab 2:9-\allowbreak11}
\crossref{Zech}{5}{5}{Zec 1:9,\allowbreak14,\allowbreak19; 2:3; 4:5}
\crossref{Zech}{5}{6}{}
\crossref{Zech}{5}{7}{Isa 13:1; 15:1; 22:11}
\crossref{Zech}{5}{8}{Ge 15:16 Mt 23:32 1Th 2:16}
\crossref{Zech}{5}{9}{De 28:49 Da 9:26,\allowbreak27 Ho 8:1 Mt 24:28}
\crossref{Zech}{5}{10}{}
\crossref{Zech}{5}{11}{De 28:59 Jer 29:28 Ho 3:4 Lu 21:24}
\crossref{Zech}{6}{1}{Zec 5:1}
\crossref{Zech}{6}{2}{Zec 1:8 Re 6:2-\allowbreak5; 12:3; 17:3}
\crossref{Zech}{6}{3}{Re 6:2; 19:11; 20:11}
\crossref{Zech}{6}{4}{Zec 1:9,\allowbreak19-\allowbreak21; 5:5,\allowbreak6,\allowbreak10}
\crossref{Zech}{6}{5}{Zec 1:10,\allowbreak11 Ps 68:17; 104:3,\allowbreak4 Eze 1:5-\allowbreak28; 10:9-\allowbreak19; 11:22 Heb 1:7,\allowbreak14}
\crossref{Zech}{6}{6}{Jer 1:14,\allowbreak15; 4:6; 6:1; 25:9; 46:10; 51:48 Eze 1:4}
\crossref{Zech}{6}{7}{Zec 1:10 Ge 13:17 2Ch 16:9 Job 1:6,\allowbreak7; 2:1,\allowbreak2 Da 7:19,\allowbreak24}
\crossref{Zech}{6}{8}{Zec 1:15 Jud 8:3; 15:7 Ec 10:4 Isa 1:24; 18:3,\allowbreak4; 42:13-\allowbreak15; 48:14}
\crossref{Zech}{6}{9}{Zec 1:1; 7:1; 8:1}
\crossref{Zech}{6}{10}{Ezr 7:14-\allowbreak16; 8:26-\allowbreak30 Isa 66:20 Ac 24:17 Ro 15:25,\allowbreak26}
\crossref{Zech}{6}{11}{Zec 3:5 Ex 28:36-\allowbreak38; 29:6; 39:30 Le 8:9 Ps 21:3 So 3:11 Heb 2:9}
\crossref{Zech}{6}{12}{Zec 13:7 Isa 32:1,\allowbreak2 Mic 5:5 Mr 15:39 Joh 19:5 Ac 13:38; 17:31}
\crossref{Zech}{6}{13}{Ps 21:5; 45:3,\allowbreak4; 72:17-\allowbreak19 Isa 9:6; 11:10; 22:24; 49:5,\allowbreak6 Jer 23:6}
\crossref{Zech}{6}{14}{6:10}
\crossref{Zech}{6}{15}{Isa 56:6-\allowbreak8; 57:19; 60:10 Ac 2:39 1Co 3:10-\allowbreak15 Eph 2:13-\allowbreak22}
\crossref{Zech}{7}{1}{Zec 1:1 Ezr 6:14,\allowbreak15 Hag 2:10,\allowbreak20}
\crossref{Zech}{7}{2}{Zec 6:10 Ezr 6:10; 7:15-\allowbreak23; 8:28-\allowbreak30 Isa 60:7}
\crossref{Zech}{7}{3}{De 17:9-\allowbreak11; 33:10 Eze 44:23,\allowbreak24 Ho 4:6 Hag 2:11 Mal 2:7}
\crossref{Zech}{7}{4}{Isa 10:16}
\crossref{Zech}{7}{5}{Isa 58:5}
\crossref{Zech}{7}{6}{De 12:7; 14:26 1Sa 16:7 1Ch 29:22 Jer 17:9,\allowbreak10 Ho 8:13; 9:4}
\crossref{Zech}{7}{7}{Zec 1:3-\allowbreak6 Isa 1:16-\allowbreak20 Jer 7:5,\allowbreak23; 36:2,\allowbreak3 Eze 18:30-\allowbreak32 Da 9:6-\allowbreak14}
\crossref{Zech}{7}{8}{}
\crossref{Zech}{7}{9}{7:7; 8:16,\allowbreak17 Le 19:15,\allowbreak35-\allowbreak37 De 10:18,\allowbreak19; 15:7-\allowbreak14; 16:18-\allowbreak20}
\crossref{Zech}{7}{10}{Ex 22:21-\allowbreak24; 23:9 De 24:14-\allowbreak18; 27:19 Ps 72:4 Pr 22:22,\allowbreak23; 23:10}
\crossref{Zech}{7}{11}{Zec 1:4 Ex 10:3 2Ki 17:13-\allowbreak15 2Ch 33:10 Ne 9:17,\allowbreak26,\allowbreak29 Pr 1:24-\allowbreak32}
\crossref{Zech}{7}{12}{Ne 9:29 Job 9:4 Isa 48:4 Jer 5:3 Eze 2:4; 3:7-\allowbreak9; 11:19; 36:26}
\crossref{Zech}{7}{13}{Ps 81:8-\allowbreak12 Pr 1:24-\allowbreak28 Isa 50:2 Jer 6:16,\allowbreak17 Lu 13:34,\allowbreak35}
\crossref{Zech}{7}{14}{Zec 2:6; 9:14 Le 26:33 De 4:27; 28:33,\allowbreak64 Ps 58:9 Isa 17:13; 21:1}
\crossref{Zech}{8}{1}{8:1}
\crossref{Zech}{8}{2}{Zec 1:14-\allowbreak16 Ps 78:58,\allowbreak59 Isa 42:13,\allowbreak14; 59:17; 63:4-\allowbreak6,\allowbreak15 Eze 36:5,\allowbreak6}
\crossref{Zech}{8}{3}{Zec 1:16 Jer 30:10,\allowbreak11}
\crossref{Zech}{8}{4}{1Sa 2:31 Job 5:26; 42:17 Isa 65:20-\allowbreak22 La 2:20,\allowbreak21,\allowbreak22; 5:11-\allowbreak15}
\crossref{Zech}{8}{5}{Zec 2:4 Ps 128:3,\allowbreak4; 144:12-\allowbreak15 Jer 30:19,\allowbreak20; 31:27; 33:11 La 2:19}
\crossref{Zech}{8}{6}{Ge 18:14 Nu 11:22,\allowbreak23 2Ki 7:2 Jer 32:17,\allowbreak27 Lu 1:20,\allowbreak37; 18:27}
\crossref{Zech}{8}{7}{Ps 107:2,\allowbreak3 Isa 11:11-\allowbreak16; 27:12,\allowbreak13; 43:5,\allowbreak6; 49:12; 59:19; 66:19,\allowbreak20}
\crossref{Zech}{8}{8}{Jer 3:17,\allowbreak18; 23:8; 32:41 Eze 37:25 Joe 3:20 Am 9:14,\allowbreak15 Ob 1:17-\allowbreak21}
\crossref{Zech}{8}{9}{8:13,\allowbreak18 Jos 1:6,\allowbreak8 1Ch 22:13; 28:20 Isa 35:4 Hag 2:4-\allowbreak9 Eph 6:10}
\crossref{Zech}{8}{10}{Hag 1:6-\allowbreak11; 2:16-\allowbreak18}
\crossref{Zech}{8}{11}{8:8,\allowbreak9 Ps 103:9 Isa 11:13; 12:1 Hag 2:19 Mal 3:9-\allowbreak11}
\crossref{Zech}{8}{12}{Ge 26:12 Le 26:4,\allowbreak5 De 28:4-\allowbreak12 Ps 67:6,\allowbreak7 Pr 3:9,\allowbreak10 Isa 30:23}
\crossref{Zech}{8}{13}{De 28:37; 29:23-\allowbreak28 1Ki 9:7,\allowbreak8 2Ch 7:20-\allowbreak22 Ps 44:13,\allowbreak14,\allowbreak16}
\crossref{Zech}{8}{14}{Zec 1:6 Ps 33:11 Isa 14:24 Jer 31:28}
\crossref{Zech}{8}{15}{Jer 29:11-\allowbreak14; 32:42 Mic 4:10-\allowbreak13; 7:18-\allowbreak20}
\crossref{Zech}{8}{16}{De 10:12,\allowbreak13; 11:7,\allowbreak8 Mic 6:8 Lu 3:8-\allowbreak14 Eph 4:17 1Pe 1:13-\allowbreak16}
\crossref{Zech}{8}{17}{Zec 7:10 Pr 3:29; 6:14 Jer 4:14 Mic 2:1-\allowbreak3 Mt 5:28; 12:35; 15:19}
\crossref{Zech}{8}{18}{}
\crossref{Zech}{8}{19}{2Ki 25:3,\allowbreak4 Jer 39:2; 52:6,\allowbreak7}
\crossref{Zech}{8}{20}{Zec 2:11; 14:16,\allowbreak17 1Ki 8:41,\allowbreak43 2Ch 6:32,\allowbreak33 Ps 22:27; 67:1-\allowbreak4; 72:17}
\crossref{Zech}{8}{21}{Ps 122:1-\allowbreak9}
\crossref{Zech}{8}{22}{Isa 25:7; 55:5; 60:3-\allowbreak22; 66:23 Jer 4:2 Mic 4:3 Hag 2:7 Ga 3:8}
\crossref{Zech}{8}{23}{Ge 31:7,\allowbreak41 Nu 14:22 Job 19:3 Ec 11:2 Mic 5:5 Mt 18:21,\allowbreak22}
\crossref{Zech}{9}{1}{Isa 13:1 Jer 23:33-\allowbreak38 Mal 1:1}
\crossref{Zech}{9}{2}{Nu 13:21 2Ki 23:33; 25:21 Jer 49:23 Am 6:14}
\crossref{Zech}{9}{3}{Jos 19:29 2Sa 24:7}
\crossref{Zech}{9}{4}{Pr 10:2; 11:4 Isa 23:1-\allowbreak7 Eze 28:16 Joe 3:8}
\crossref{Zech}{9}{5}{Isa 14:29-\allowbreak31 Jer 47:1,\allowbreak4-\allowbreak7 Eze 25:15-\allowbreak17 Zep 2:4-\allowbreak7 Ac 8:26}
\crossref{Zech}{9}{6}{Ec 2:18-\allowbreak21; 6:2 Am 1:8 Isa 2:12-\allowbreak17; 23:9; 28:1 Da 4:37 Zep 2:10}
\crossref{Zech}{9}{7}{1Sa 17:34-\allowbreak36 Ps 3:7; 58:6 Am 3:12}
\crossref{Zech}{9}{8}{Zec 2:1-\allowbreak13; 12:8 Ge 32:1,\allowbreak2 Ps 34:7; 46:1-\allowbreak5; 125:1,\allowbreak2 Isa 4:5; 26:1; 31:5}
\crossref{Zech}{9}{9}{Zec 2:10 Ps 97:6-\allowbreak8 Isa 12:6; 40:9; 52:9,\allowbreak10; 62:11 Zep 3:14,\allowbreak15}
\crossref{Zech}{9}{10}{Ho 1:7; 2:18 Mic 5:10,\allowbreak11 Hag 2:22 2Co 10:4,\allowbreak5}
\crossref{Zech}{9}{11}{De 5:31 2Sa 13:13 2Ch 7:17 Da 2:29}
\crossref{Zech}{9}{12}{Isa 52:2 Jer 31:6; 50:4,\allowbreak5,\allowbreak28; 51:10 Mic 4:8 Na 1:7 Heb 6:18}
\crossref{Zech}{9}{13}{Zec 1:21; 10:3-\allowbreak7; 12:2-\allowbreak8 Mic 5:4-\allowbreak9 Re 17:14}
\crossref{Zech}{9}{14}{Zec 2:5; 12:8; 14:3 Ex 14:24,\allowbreak25 Jos 10:11-\allowbreak14,\allowbreak42 Mt 28:20 Ac 4:10,\allowbreak11}
\crossref{Zech}{9}{15}{Zec 10:5; 12:6 Mic 5:8 Re 19:13-\allowbreak21}
\crossref{Zech}{9}{16}{Ps 100:3 Isa 40:10 Jer 23:3 Eze 34:22-\allowbreak26,\allowbreak31 Mic 5:4; 7:14}
\crossref{Zech}{9}{17}{Ps 31:19; 36:7; 86:5,\allowbreak15; 145:7 Isa 63:7,\allowbreak15 Joh 3:16 Ro 5:8,\allowbreak20}
\crossref{Zech}{10}{1}{Eze 36:37 Mt 7:7,\allowbreak8 Joh 16:23 Jas 5:16-\allowbreak18}
\crossref{Zech}{10}{2}{Ge 31:19 Jud 18:14 Isa 44:9; 46:5 Jer 10:8; 14:22 Ho 3:4}
\crossref{Zech}{10}{3}{Zec 11:5-\allowbreak8,\allowbreak17 Isa 56:9-\allowbreak12 Jer 10:21; 23:1,\allowbreak2; 50:6 Eze 34:2,\allowbreak7-\allowbreak10}
\crossref{Zech}{10}{4}{Zec 1:20,\allowbreak21; 9:13-\allowbreak16; 12:6-\allowbreak8 Nu 24:17 Isa 41:14-\allowbreak16; 49:2; 54:16}
\crossref{Zech}{10}{5}{Zec 9:13; 12:8 1Sa 16:18 2Sa 22:8 Ps 45:3 Lu 24:19 Ac 7:22; 18:24}
\crossref{Zech}{10}{6}{10:12 Ps 89:21 Isa 41:10 Eze 37:16 Ob 1:18 Mic 4:6,\allowbreak13; 5:8; 7:16}
\crossref{Zech}{10}{7}{Zec 9:15,\allowbreak17 Ge 43:34 Ps 104:15 Pr 31:6,\allowbreak7 Ac 2:13-\allowbreak18 Eph 5:18,\allowbreak19}
\crossref{Zech}{10}{8}{Isa 5:26; 7:18; 11:11,\allowbreak12; 27:12,\allowbreak13; 55:1-\allowbreak3 Mt 11:28 Re 22:17}
\crossref{Zech}{10}{9}{Es 8:17 Jer 31:27 Da 3:1-\allowbreak6:28 Ho 2:23 Am 9:9 Mic 5:7}
\crossref{Zech}{10}{10}{Zec 8:7 Isa 11:11-\allowbreak16; 19:23-\allowbreak25; 27:12,\allowbreak13 Ho 11:11 Mic 7:11,\allowbreak12}
\crossref{Zech}{10}{11}{Ps 66:10-\allowbreak12 Isa 11:15,\allowbreak16; 42:15,\allowbreak16; 43:2}
\crossref{Zech}{10}{12}{10:6; 12:5 Ps 68:34,\allowbreak35 Isa 41:10; 45:24 Eph 6:10 Php 4:13 2Ti 2:1}
\crossref{Zech}{11}{1}{Zec 10:10 Jer 22:6,\allowbreak7,\allowbreak23 Hab 2:8,\allowbreak17 Hag 1:8}
\crossref{Zech}{11}{2}{Isa 2:12-\allowbreak17; 10:33,\allowbreak34 Eze 31:2,\allowbreak3,\allowbreak17 Am 6:1 Na 3:8-\allowbreak19 Lu 23:31}
\crossref{Zech}{11}{3}{11:8,\allowbreak15-\allowbreak17 Jer 25:34-\allowbreak36 Joe 1:13 Am 8:8 Zep 1:10 Mt 15:14}
\crossref{Zech}{11}{4}{Zec 14:5 Isa 49:4,\allowbreak5 Joh 20:17 Eph 1:3}
\crossref{Zech}{11}{5}{Jer 23:1,\allowbreak2 Eze 22:25-\allowbreak27; 34:2,\allowbreak3,\allowbreak10 Mic 3:1-\allowbreak3,\allowbreak9-\allowbreak12 Mt 23:14}
\crossref{Zech}{11}{6}{11:5 Isa 27:11 Eze 8:18; 9:10 Ho 1:6 Mt 18:33-\allowbreak35; 22:7; 23:35-\allowbreak38}
\crossref{Zech}{11}{7}{11:4,\allowbreak11; 13:8,\allowbreak9}
\crossref{Zech}{11}{8}{Ho 5:7 Mt 23:34-\allowbreak36; 24:50,\allowbreak51}
\crossref{Zech}{11}{9}{Jer 23:33,\allowbreak39 Mt 13:10,\allowbreak11; 21:43; 23:38,\allowbreak39 Joh 8:21,\allowbreak24; 12:35}
\crossref{Zech}{11}{10}{11:7 Ps 50:2; 90:17 Eze 7:20-\allowbreak22; 24:21 Da 9:26 Lu 21:5,\allowbreak6,\allowbreak32}
\crossref{Zech}{11}{11}{Isa 8:17; 26:8,\allowbreak9; 40:31 La 3:25,\allowbreak26 Mic 7:7 Lu 2:25,\allowbreak38; 23:51}
\crossref{Zech}{11}{12}{1Ki 21:2 2Ch 30:4}
\crossref{Zech}{11}{13}{Isa 54:7-\allowbreak10 Mt 27:3-\allowbreak10,\allowbreak12 Ac 1:18,\allowbreak19}
\crossref{Zech}{11}{14}{11:9 Isa 9:21; 11:13 Eze 37:16-\allowbreak20 Mt 24:10 Ac 23:7-\allowbreak10 Ga 5:15}
\crossref{Zech}{11}{15}{Isa 6:10-\allowbreak12 Jer 2:26,\allowbreak27 La 2:14 Eze 13:3 Mt 15:14; 23:17}
\crossref{Zech}{11}{16}{Jer 23:2,\allowbreak22 Eze 34:2-\allowbreak6,\allowbreak16 Mt 23:2-\allowbreak4,\allowbreak13-\allowbreak29 Lu 12:45,\allowbreak46}
\crossref{Zech}{11}{17}{Jer 22:1 Eze 13:3; 34:2 Mt 23:13,\allowbreak16 Lu 11:42-\allowbreak52}
\crossref{Zech}{12}{1}{Zec 9:1 La 2:14 Mal 1:1}
\crossref{Zech}{12}{2}{Ps 75:8 Isa 51:17,\allowbreak22,\allowbreak23 Jer 25:15,\allowbreak17; 49:12; 51:7 Hab 2:16}
\crossref{Zech}{12}{3}{12:4,\allowbreak6,\allowbreak8,\allowbreak9,\allowbreak11; 2:8,\allowbreak9; 10:3-\allowbreak5; 13:1; 14:2,\allowbreak3,\allowbreak4,\allowbreak6,\allowbreak8,\allowbreak9,\allowbreak13 Isa 60:12}
\crossref{Zech}{12}{4}{12:3,\allowbreak6,\allowbreak8,\allowbreak9,\allowbreak11 Isa 24:21}
\crossref{Zech}{12}{5}{12:6 Jud 5:9 Isa 1:10,\allowbreak23,\allowbreak26; 29:10; 32:1; 60:17 Jer 30:21; 33:26}
\crossref{Zech}{12}{6}{Isa 10:16,\allowbreak17 Ob 1:18 Re 20:9}
\crossref{Zech}{12}{7}{Zec 4:6; 11:11 Isa 2:11-\allowbreak17; 23:9 Jer 9:23,\allowbreak24 Mt 11:25,\allowbreak26 Lu 1:51-\allowbreak53}
\crossref{Zech}{12}{8}{Zec 2:5; 9:8,\allowbreak15,\allowbreak16 Joe 3:16,\allowbreak17}
\crossref{Zech}{12}{9}{12:2 Isa 54:17 Hag 2:22}
\crossref{Zech}{12}{10}{Pr 1:23 Isa 32:15; 44:3,\allowbreak4; 59:19-\allowbreak21 Eze 39:29 Joe 2:28,\allowbreak29}
\crossref{Zech}{12}{11}{2Ki 23:29 2Ch 35:24}
\crossref{Zech}{12}{12}{Jer 3:21; 4:28; 31:18 Mt 24:30 Re 1:7}
\crossref{Zech}{12}{13}{Ex 6:16-\allowbreak26 Nu 3:1-\allowbreak4:49 Mal 2:4-\allowbreak9}
\crossref{Zech}{12}{14}{Pr 9:12}
\crossref{Zech}{13}{1}{Zec 12:3,\allowbreak8,\allowbreak11}
\crossref{Zech}{13}{2}{Ex 22:13 De 12:3 Jos 23:7 Ps 16:4 Isa 2:18,\allowbreak20 Eze 30:13; 36:25}
\crossref{Zech}{13}{3}{Ex 32:27,\allowbreak28 De 13:6-\allowbreak11; 18:20; 33:9 Mt 10:37 Lu 14:26 2Co 5:16}
\crossref{Zech}{13}{4}{Jer 2:26 Mic 3:6,\allowbreak7}
\crossref{Zech}{13}{5}{Am 7:14 Ac 19:17-\allowbreak20}
\crossref{Zech}{13}{6}{1Ki 18:28 Re 13:16,\allowbreak17; 14:11}
\crossref{Zech}{13}{7}{De 32:41,\allowbreak42 Isa 27:1 Jer 47:6 Eze 21:4,\allowbreak5,\allowbreak9,\allowbreak10,\allowbreak28}
\crossref{Zech}{13}{8}{Zec 11:6-\allowbreak9 De 28:49-\allowbreak68 Isa 65:12-\allowbreak15; 66:4-\allowbreak6,\allowbreak24 Eze 5:2-\allowbreak4,\allowbreak12 Da 9:27}
\crossref{Zech}{13}{9}{Ps 66:10-\allowbreak12 Isa 43:2 1Co 3:11-\allowbreak13 1Pe 4:12}
\crossref{Zech}{14}{1}{}
\crossref{Zech}{14}{2}{De 28:9-\allowbreak14 Isa 5:26 Jer 34:1 Da 2:40-\allowbreak43 Joe 3:2 Mt 22:7 Lu 2:1}
\crossref{Zech}{14}{3}{Zec 2:8,\allowbreak9; 10:4,\allowbreak5; 12:2-\allowbreak6,\allowbreak9 Isa 63:1-\allowbreak6; 66:15,\allowbreak16 Da 2:34,\allowbreak35,\allowbreak44,\allowbreak45}
\crossref{Zech}{14}{4}{14:7 Eze 11:23; 43:2 Ac 1:11,\allowbreak12}
\crossref{Zech}{14}{5}{Isa 29:6 Am 1:1}
\crossref{Zech}{14}{6}{Ps 97:10,\allowbreak11; 112:4 Pr 4:18,\allowbreak19 Isa 50:10; 60:1-\allowbreak3 Ho 6:3}
\crossref{Zech}{14}{7}{Re 21:23; 22:5}
\crossref{Zech}{14}{8}{Eze 47:1-\allowbreak12 Joe 3:18 Lu 24:47 Joh 4:10,\allowbreak14; 7:38 Re 22:1,\allowbreak2,\allowbreak17}
\crossref{Zech}{14}{9}{Zec 8:20-\allowbreak23 Ge 49:10 1Sa 2:10 Ps 2:6-\allowbreak8; 22:27-\allowbreak31; 47:2-\allowbreak9; 67:4}
\crossref{Zech}{14}{10}{Zec 4:6,\allowbreak7 Isa 40:3,\allowbreak4 Lu 3:4-\allowbreak6}
\crossref{Zech}{14}{11}{Nu 21:3 Isa 60:18 Jer 31:40 Eze 37:26 Joe 3:17,\allowbreak20 Am 9:15}
\crossref{Zech}{14}{12}{14:3; 12:9 Ps 110:5,\allowbreak6 Isa 34:1-\allowbreak17; 66:15,\allowbreak16 Eze 38:18-\allowbreak22}
\crossref{Zech}{14}{13}{Zec 12:4 Jud 7:22 1Sa 14:15-\allowbreak23 2Ch 20:22-\allowbreak24 Eze 38:21 Re 17:12-\allowbreak17}
\crossref{Zech}{14}{14}{Zec 10:4,\allowbreak5; 12:5-\allowbreak7}
\crossref{Zech}{14}{15}{}
\crossref{Zech}{14}{16}{Zec 8:20-\allowbreak23; 9:7 Isa 60:6-\allowbreak9; 66:18-\allowbreak21,\allowbreak23 Joe 2:32 Ac 15:17}
\crossref{Zech}{14}{17}{Ps 2:8-\allowbreak12; 110:5,\allowbreak6 Isa 45:23; 60:12 Jer 10:25 Ro 14:10,\allowbreak11}
\crossref{Zech}{14}{18}{De 11:10,\allowbreak11}
\crossref{Zech}{14}{19}{Joh 3:19}
\crossref{Zech}{14}{20}{Pr 21:3,\allowbreak4 Isa 23:18 Ob 1:17 Zep 2:11 Mal 1:11 Lu 11:41 Ac 10:15}
\crossref{Zech}{14}{21}{Zec 7:6 De 12:7,\allowbreak12 Ne 8:10 Ro 14:6,\allowbreak7 1Co 10:31 1Ti 4:3-\allowbreak5}

% Mal
\crossref{Mal}{1}{1}{Isa 13:1 Hab 1:1 Zec 9:1; 12:1}
\crossref{Mal}{1}{2}{1:6,\allowbreak7; 2:17; 3:7,\allowbreak8,\allowbreak13,\allowbreak14 Jer 2:5,\allowbreak31 Lu 10:29}
\crossref{Mal}{1}{3}{Ge 29:30,\allowbreak31 De 21:15,\allowbreak16 Lu 14:26}
\crossref{Mal}{1}{4}{Isa 9:9,\allowbreak10 Jas 4:13-\allowbreak16}
\crossref{Mal}{1}{5}{De 4:3; 11:7 Jos 24:7 1Sa 12:16 2Ch 29:8 Lu 10:23,\allowbreak24}
\crossref{Mal}{1}{6}{Ex 20:12 Le 19:3 De 5:16 Pr 30:11,\allowbreak17 Mt 15:4,\allowbreak6; 19:19 Mr 7:10}
\crossref{Mal}{1}{7}{Le 2:11,\allowbreak 21:6 De 15:21}
\crossref{Mal}{1}{8}{1:14 Le 22:19-\allowbreak25 De 15:21}
\crossref{Mal}{1}{9}{2Ch 30:27 Jer 27:18 Joe 1:13,\allowbreak14; 2:17 Zec 3:1-\allowbreak5 Joh 9:31}
\crossref{Mal}{1}{10}{Job 1:9-\allowbreak11 Isa 56:11,\allowbreak12 Jer 6:13; 8:10 Mic 3:11 Joh 10:12}
\crossref{Mal}{1}{11}{1:14 Ps 22:27-\allowbreak31; 67:2; 72:11-\allowbreak17; 98:1-\allowbreak3 Isa 11:9,\allowbreak10; 45:22,\allowbreak23}
\crossref{Mal}{1}{12}{1:6,\allowbreak8; 2:8 2Sa 12:14 Eze 36:21-\allowbreak23 Am 2:7 Ro 2:24}
\crossref{Mal}{1}{13}{1Sa 2:29 Isa 43:22 Am 8:5 Mic 6:3 Mr 14:4,\allowbreak5,\allowbreak37,\allowbreak38}
\crossref{Mal}{1}{14}{Mal 3:9 Ge 27:12 Jos 7:11,\allowbreak12 Jer 48:10 Mt 24:51 Lu 12:1,\allowbreak2,\allowbreak46}
\crossref{Mal}{2}{1}{Mal 1:6 Jer 13:13 La 4:13 Ho 5:1}
\crossref{Mal}{2}{2}{Le 26:14-\allowbreak46 De 28:15-\allowbreak68; 30:17,\allowbreak18 Ps 81:11,\allowbreak12 Isa 30:8-\allowbreak13}
\crossref{Mal}{2}{3}{Joe 1:17}
\crossref{Mal}{2}{4}{1Ki 22:25 Isa 26:11 Jer 28:9 Eze 33:33; 38:23 Lu 10:11}
\crossref{Mal}{2}{5}{Nu 3:45; 8:15; 16:9,\allowbreak10; 18:8-\allowbreak24; 25:12,\allowbreak13 De 33:8-\allowbreak10 Ps 106:30,\allowbreak31}
\crossref{Mal}{2}{6}{Ps 37:30 Eze 44:23,\allowbreak24 Ho 4:6 Mt 22:16 Mr 12:14 Lu 20:21}
\crossref{Mal}{2}{7}{Le 10:11 De 17:8-\allowbreak11; 21:5; 24:8 2Ch 17:8,\allowbreak9; 30:22 Ezr 7:10}
\crossref{Mal}{2}{8}{Ps 18:21; 119:102 Isa 30:11; 59:13 Jer 17:5,\allowbreak13 Eze 44:10}
\crossref{Mal}{2}{9}{2:3 1Sa 2:30 Pr 10:7 Da 12:2,\allowbreak3 Mic 3:6,\allowbreak7}
\crossref{Mal}{2}{10}{Mal 1:6 Jos 24:3 Isa 51:2; 63:16; 64:8 Eze 33:24 Mt 3:9 Lu 1:73; 3:8}
\crossref{Mal}{2}{11}{Le 18:24-\allowbreak30 Jer 7:10 Eze 18:13; 22:11 Re 21:8}
\crossref{Mal}{2}{12}{Le 18:29; 20:3 Nu 15:30,\allowbreak31 Jos 23:12,\allowbreak13 1Sa 2:31-\allowbreak34}
\crossref{Mal}{2}{13}{De 15:9 1Sa 1:9,\allowbreak10 2Sa 13:19,\allowbreak20 Ps 78:34-\allowbreak37 Ec 4:1}
\crossref{Mal}{2}{14}{Mal 1:6,\allowbreak7; 3:8 Pr 30:20 Isa 58:3 Jer 8:12}
\crossref{Mal}{2}{15}{Ge 1:27; 2:20-\allowbreak24 Mt 19:4-\allowbreak6 Mr 10:6-\allowbreak8 1Co 7:2}
\crossref{Mal}{2}{16}{De 24:1-\allowbreak4 Isa 50:1 Mt 5:31,\allowbreak32; 19:3-\allowbreak9 Mr 10:2-\allowbreak12 Lu 16:18}
\crossref{Mal}{2}{17}{Ps 95:9,\allowbreak10 Isa 1:14; 7:13; 43:24 Jer 15:6 Eze 16:43 Am 2:13}
\crossref{Mal}{3}{1}{Mal 2:7; 4:5 Mt 11:10,\allowbreak11 Mr 1:2,\allowbreak3 Lu 1:76; 7:26-\allowbreak28 Joh 1:6,\allowbreak7}
\crossref{Mal}{3}{2}{Mal 4:1 Am 5:18-\allowbreak20 Mt 3:7-\allowbreak12; 21:31-\allowbreak44; 23:13-\allowbreak35; 25:10 Lu 2:34}
\crossref{Mal}{3}{3}{Ps 66:10 Pr 17:3; 25:4 Isa 1:25; 48:10 Jer 6:28-\allowbreak30 Eze 22:18-\allowbreak22}
\crossref{Mal}{3}{4}{Isa 1:26,\allowbreak27; 56:7 Jer 30:18-\allowbreak20; 31:23,\allowbreak24 Eze 20:40,\allowbreak41; 43:26,\allowbreak27}
\crossref{Mal}{3}{5}{Mal 2:17 Ps 50:3-\allowbreak6; 96:13; 98:9 Eze 34:20-\allowbreak22 Heb 10:30,\allowbreak31 Jas 5:8,\allowbreak9}
\crossref{Mal}{3}{6}{Ge 15:7,\allowbreak18; 22:16 Ex 3:14,\allowbreak15 Ne 9:7,\allowbreak8 Isa 41:13; 42:5-\allowbreak8; 43:11,\allowbreak12}
\crossref{Mal}{3}{7}{De 9:7-\allowbreak21; 31:20,\allowbreak27-\allowbreak29 Ne 9:16,\allowbreak17,\allowbreak26,\allowbreak28-\allowbreak30 Ps 78:8-\allowbreak10}
\crossref{Mal}{3}{8}{Ps 29:2 Pr 3:9,\allowbreak10 Mt 22:21 Mr 12:17 Lu 20:25 Ro 13:7}
\crossref{Mal}{3}{9}{Mal 2:2 De 28:15-\allowbreak19 Jos 7:12,\allowbreak13; 22:20 Isa 43:28 Hag 1:6-\allowbreak11; 2:14-\allowbreak17}
\crossref{Mal}{3}{10}{2Ch 31:4-\allowbreak10 Ne 10:33-\allowbreak39 Pr 3:9,\allowbreak10}
\crossref{Mal}{3}{11}{Joe 2:20 Am 4:9; 7:1-\allowbreak3 Hag 2:17}
\crossref{Mal}{3}{12}{De 4:6,\allowbreak7 2Ch 32:23 Ps 72:17 Isa 61:9 Jer 33:9 Zep 3:19,\allowbreak20}
\crossref{Mal}{3}{13}{Mal 2:17 Ex 5:2 2Ch 32:14-\allowbreak19 Job 34:7,\allowbreak8 Ps 10:11 Isa 5:19; 28:14,\allowbreak15}
\crossref{Mal}{3}{14}{Job 21:14,\allowbreak15; 22:17; 34:9; 35:3 Ps 73:8-\allowbreak13 Isa 58:3 Zep 1:12}
\crossref{Mal}{3}{15}{Mal 4:1 Es 5:10 Ps 10:3; 49:18; 73:12 Da 4:30,\allowbreak37; 5:20-\allowbreak28 Ac 12:21}
\crossref{Mal}{3}{16}{3:5; 4:2 Ge 22:12 1Ki 18:3,\allowbreak12 Job 28:28 Ps 33:18; 111:10; 112:1}
\crossref{Mal}{3}{17}{So 2:16 Jer 31:33; 32:38,\allowbreak39 Eze 16:8; 36:27,\allowbreak28 Zec 13:9}
\crossref{Mal}{3}{18}{3:14,\allowbreak15; 1:4 Job 6:29; 17:10 Jer 12:15 Joe 2:14 Zec 1:6}
\crossref{Mal}{4}{1}{4:5; 3:2 Eze 7:10 Joe 2:1,\allowbreak31 Zep 1:14 Zec 14:1 Lu 19:43; 21:20}
\crossref{Mal}{4}{2}{Mal 3:16 Ps 85:9 Isa 50:10; 66:1,\allowbreak2 Lu 1:50 Ac 13:26 Re 11:18}
\crossref{Mal}{4}{3}{Ge 3:15 Jos 10:24,\allowbreak25 2Sa 22:43 Job 40:12 Ps 91:13}
\crossref{Mal}{4}{4}{Ex 20:3-\allowbreak21 De 4:5,\allowbreak6 Ps 147:19,\allowbreak20 Isa 8:20; 42:21}
\crossref{Mal}{4}{5}{Mal 3:1 Isa 40:3 Mt 11:13,\allowbreak14; 17:10-\allowbreak13; 27:47-\allowbreak49 Mr 9:11-\allowbreak13}
\crossref{Mal}{4}{6}{Lu 1:16,\allowbreak17,\allowbreak76}

% Matt
\crossref{Matt}{1}{1}{Ge 2:4; 5:1 Isa 53:8 Lu 3:23-\allowbreak38 Ro 9:5}
\crossref{Matt}{1}{2}{Ge 21:2-\allowbreak5 Jos 24:2,\allowbreak3 1Ch 1:28 Isa 51:2 Lu 3:34 Ac 7:8 Ro 9:7-\allowbreak9}
\crossref{Matt}{1}{3}{Ge 38:27,\allowbreak29,\allowbreak30; 46:12}
\crossref{Matt}{1}{4}{Ru 4:19,\allowbreak20 1Ch 2:10-\allowbreak12}
\crossref{Matt}{1}{5}{Ru 4:21 1Ch 2:11,\allowbreak12}
\crossref{Matt}{1}{6}{Ru 4:22 1Sa 16:1,\allowbreak11-\allowbreak13; 17:12,\allowbreak58; 20:30,\allowbreak31; 22:8 2Sa 23:1}
\crossref{Matt}{1}{7}{1Ki 11:43; 12:1-\allowbreak24}
\crossref{Matt}{1}{8}{1Ki 15:24; 22:2-\allowbreak50}
\crossref{Matt}{1}{9}{2Ki 15:7,\allowbreak32-\allowbreak38 1Ch 3:11-\allowbreak13 2Ch 26:21; 27:1-\allowbreak9}
\crossref{Matt}{1}{10}{2Ki 20:21; 21:1-\allowbreak18; 24:3,\allowbreak4 1Ch 3:13-\allowbreak15 2Ch 32:33; 33:1-\allowbreak19}
\crossref{Matt}{1}{11}{2Ki 23:31-\allowbreak37; 24:1-\allowbreak20 1Ch 3:15-\allowbreak17 2Ch 36:1-\allowbreak8}
\crossref{Matt}{1}{12}{2Ki 25:27}
\crossref{Matt}{1}{13}{1:13}
\crossref{Matt}{1}{14}{1:14}
\crossref{Matt}{1}{15}{}
\crossref{Matt}{1}{16}{1:18-\allowbreak25; 2:13 Lu 1:27; 2:4,\allowbreak5,\allowbreak48; 3:23; 4:22}
\crossref{Matt}{1}{17}{}
\crossref{Matt}{1}{18}{Lu 1:27-\allowbreak38}
\crossref{Matt}{1}{19}{Le 19:20 De 22:23,\allowbreak24}
\crossref{Matt}{1}{20}{Ps 25:8,\allowbreak9; 94:19; 119:125; 143:8 Pr 3:5,\allowbreak6; 12:5 Isa 26:3; 30:21}
\crossref{Matt}{1}{21}{Ge 17:19,\allowbreak21; 18:10 Jud 13:3 2Ki 4:16,\allowbreak17 Lu 1:13,\allowbreak35,\allowbreak36}
\crossref{Matt}{1}{22}{Mt 2:15,\allowbreak23; 5:17; 8:17; 12:17; 13:35,\allowbreak21 1Ki 8:15,\allowbreak24 Ezr 1:1}
\crossref{Matt}{1}{23}{Isa 7:14}
\crossref{Matt}{1}{24}{Ge 6:22; 7:5; 22:2,\allowbreak3 Ex 40:16,\allowbreak19,\allowbreak25,\allowbreak27,\allowbreak32 2Ki 5:11-\allowbreak14 Joh 2:5-\allowbreak8}
\crossref{Matt}{1}{25}{Ex 13:2; 22:29 Lu 2:7 Ro 8:29}
\crossref{Matt}{2}{1}{Mt 1:25 Lu 2:4-\allowbreak7}
\crossref{Matt}{2}{2}{Mt 21:5 Ps 2:6 Isa 9:6,\allowbreak7; 32:1,\allowbreak2 Jer 23:5 Zec 9:9 Lu 2:11; 19:38}
\crossref{Matt}{2}{3}{Mt 8:29; 23:37 1Ki 18:17,\allowbreak18 Joh 11:47,\allowbreak48 Ac 4:2,\allowbreak24-\allowbreak27; 5:24-\allowbreak28}
\crossref{Matt}{2}{4}{Mt 21:15,\allowbreak23; 26:3,\allowbreak47; 27:1 1Ch 24:4-\allowbreak19}
\crossref{Matt}{2}{5}{Ge 35:19 Jos 19:15 Ru 1:1,\allowbreak19; 2:4; 4:11 1Sa 16:1}
\crossref{Matt}{2}{6}{2:1 Mic 5:2 Joh 7:42}
\crossref{Matt}{2}{7}{Mt 26:3-\allowbreak5 Ex 1:10 1Sa 18:21 Ps 10:9,\allowbreak10; 55:21; 64:4-\allowbreak6; 83:3,\allowbreak4}
\crossref{Matt}{2}{8}{1Sa 23:22,\allowbreak23 2Sa 17:14 1Ki 19:2 Job 5:12,\allowbreak13 Ps 33:10,\allowbreak11}
\crossref{Matt}{2}{9}{2:2 Ps 25:12 Pr 2:1-\allowbreak6; 8:17 2Pe 1:19}
\crossref{Matt}{2}{10}{De 32:13 Ps 67:4; 105:3 Lu 2:10,\allowbreak20 Ac 13:46-\allowbreak48 Ro 15:9-\allowbreak13}
\crossref{Matt}{2}{11}{Lu 2:16,\allowbreak26-\allowbreak32,\allowbreak38}
\crossref{Matt}{2}{12}{2:22; 1:20; 27:19 Ge 20:6,\allowbreak7; 31:24 Job 33:15-\allowbreak17 Da 2:19}
\crossref{Matt}{2}{13}{2:19; 1:20 Ac 5:19; 10:7,\allowbreak22; 12:11 Heb 1:13,\allowbreak14}
\crossref{Matt}{2}{14}{2:20,\allowbreak21; 1:24 Ac 26:21}
\crossref{Matt}{2}{15}{2:19 Ac 12:1-\allowbreak4,\allowbreak23,\allowbreak24}
\crossref{Matt}{2}{16}{Ge 39:14,\allowbreak17 Nu 22:29; 24:10 Jud 16:10 Job 12:4}
\crossref{Matt}{2}{17}{}
\crossref{Matt}{2}{18}{Jer 31:15}
\crossref{Matt}{2}{19}{Ps 76:10 Isa 51:12 Da 8:25; 11:45}
\crossref{Matt}{2}{20}{2:13 Pr 3:5,\allowbreak6}
\crossref{Matt}{2}{21}{Ge 6:22 Heb 11:8}
\crossref{Matt}{2}{22}{Ge 19:17-\allowbreak21 1Sa 16:2 Ac 9:13,\allowbreak14}
\crossref{Matt}{2}{23}{Joh 18:5,\allowbreak7; 19:19 Ac 2:22}
\crossref{Matt}{3}{1}{Lu 3:1,\allowbreak2}
\crossref{Matt}{3}{2}{Mt 4:17; 11:20; 12:41; 21:29-\allowbreak32 1Ki 8:47 Job 42:6 Eze 18:30-\allowbreak32; 33:11}
\crossref{Matt}{3}{3}{Isa 40:3 Mr 1:3 Lu 3:3-\allowbreak6 Joh 1:23}
\crossref{Matt}{3}{4}{Mt 11:8 2Ki 1:8 Zec 13:4 Mal 4:5 Mr 1:6 Lu 1:17 Re 11:3}
\crossref{Matt}{3}{5}{Mt 4:25; 11:7-\allowbreak12 Mr 1:5 Lu 3:7; 16:16 Joh 3:23; 5:35}
\crossref{Matt}{3}{6}{3:11,\allowbreak13-\allowbreak16 Eze 36:25 Mr 1:8,\allowbreak9 Lu 3:16 Joh 1:25-\allowbreak28,\allowbreak31-\allowbreak33; 3:23-\allowbreak25}
\crossref{Matt}{3}{7}{Mt 5:20; 12:24; 15:12; 16:6,\allowbreak11,\allowbreak12; 22:15,\allowbreak23,\allowbreak34; 23:13 etc.}
\crossref{Matt}{3}{8}{Mt 21:28-\allowbreak30,\allowbreak32 Isa 1:16,\allowbreak17 Lu 3:8,\allowbreak10-\allowbreak14 Ac 26:20 Ro 2:4-\allowbreak7}
\crossref{Matt}{3}{9}{Mr 7:21 Lu 3:8; 5:22; 7:39; 12:17}
\crossref{Matt}{3}{10}{Mal 3:1-\allowbreak3; 4:1 Heb 3:1-\allowbreak3; 10:28-\allowbreak31; 12:25}
\crossref{Matt}{3}{11}{3:6 Mr 1:4,\allowbreak8 Lu 3:3,\allowbreak16 Joh 1:26,\allowbreak33 Ac 1:5; 11:16; 13:24; 19:4}
\crossref{Matt}{3}{12}{Isa 30:24; 41:16 Jer 4:11; 15:7; 51:2 Lu 3:17}
\crossref{Matt}{3}{13}{Mt 2:22 Mr 1:9 Lu 3:21}
\crossref{Matt}{3}{14}{Lu 1:43 Joh 13:6-\allowbreak8}
\crossref{Matt}{3}{15}{Joh 13:7-\allowbreak9}
\crossref{Matt}{3}{16}{Mr 1:10}
\crossref{Matt}{3}{17}{Joh 5:37; 12:28-\allowbreak30 Re 14:2}
\crossref{Matt}{4}{1}{Mr 1:12,\allowbreak13 etc.}
\crossref{Matt}{4}{2}{Ex 24:18; 34:28 De 9:9,\allowbreak18,\allowbreak25; 18:18 1Ki 19:8 Lu 4:2}
\crossref{Matt}{4}{3}{Job 1:9-\allowbreak12; 2:4-\allowbreak7 Lu 22:31,\allowbreak32 1Th 3:5 Re 2:10; 12:9-\allowbreak11}
\crossref{Matt}{4}{4}{4:7,\allowbreak10 Lu 4:4,\allowbreak8,\allowbreak12 Ro 15:4 Eph 6:17}
\crossref{Matt}{4}{5}{Lu 4:9 Joh 19:11}
\crossref{Matt}{4}{6}{4:4 2Co 11:14}
\crossref{Matt}{4}{7}{4:4,\allowbreak10; 21:16,\allowbreak42; 22:31,\allowbreak32 Isa 8:20}
\crossref{Matt}{4}{8}{4:5 Lu 4:5-\allowbreak7}
\crossref{Matt}{4}{9}{Mt 26:15 Joh 13:3}
\crossref{Matt}{4}{10}{Mt 16:23 Jas 4:7 1Pe 5:9}
\crossref{Matt}{4}{11}{Lu 4:13; 22:53 Joh 14:30}
\crossref{Matt}{4}{12}{Mr 1:14; 6:17 Lu 3:20; 4:14,\allowbreak31 Joh 4:43,\allowbreak54}
\crossref{Matt}{4}{13}{Lu 4:30,\allowbreak31}
\crossref{Matt}{4}{14}{Mt 1:22; 2:15,\allowbreak23; 8:17; 12:17-\allowbreak21; 26:54,\allowbreak56 Lu 22:37; 24:44}
\crossref{Matt}{4}{15}{Jos 20:7; 21:32 1Ki 9:11 2Ki 15:29}
\crossref{Matt}{4}{16}{Ps 107:10-\allowbreak14 Isa 9:2; 42:6,\allowbreak7; 60:1-\allowbreak3 Mic 7:8 Lu 1:78,\allowbreak79; 2:32}
\crossref{Matt}{4}{17}{Mr 1:14}
\crossref{Matt}{4}{18}{Mt 1:16-\allowbreak18 Lu 5:2}
\crossref{Matt}{4}{19}{Mt 8:22; 9:9; 16:24; 19:21 Mr 2:14 Lu 5:27; 9:59 Joh 1:43; 12:26; 21:22}
\crossref{Matt}{4}{20}{Mt 10:37; 19:27 1Ki 19:21 Ps 119:60 Mr 10:28-\allowbreak31 Lu 18:28-\allowbreak30}
\crossref{Matt}{4}{21}{Mt 10:2; 17:1; 20:20,\allowbreak21; 26:37 Mr 1:19,\allowbreak20; 3:17; 5:37 Lu 5:10,\allowbreak11}
\crossref{Matt}{4}{22}{Mt 10:37 De 33:9,\allowbreak10 Mr 1:20 Lu 9:59,\allowbreak60; 14:26,\allowbreak33 2Co 5:16}
\crossref{Matt}{4}{23}{Mt 9:35 Mr 6:6 Joh 7:1 Ac 10:38}
\crossref{Matt}{4}{24}{Mt 9:26,\allowbreak31; 14:1 Jos 6:27 1Ki 4:31; 10:1 1Ch 14:17 Mr 1:28}
\crossref{Matt}{4}{25}{Mt 5:1; 8:1; 12:15; 19:2 Mr 3:7; 6:2 Lu 6:17,\allowbreak19}
\crossref{Matt}{5}{1}{Mt 4:25; 13:2 Mr 4:1}
\crossref{Matt}{5}{2}{Mt 13:35 Job 3:1 Ps 78:1,\allowbreak2 Pr 8:6; 31:8,\allowbreak9 Lu 6:20 etc.}
\crossref{Matt}{5}{3}{5:4-\allowbreak11; 11:6; 13:16; 24:46 Ps 1:1; 2:12; 32:1,\allowbreak2; 41:1; 84:12; 112:1}
\crossref{Matt}{5}{4}{Ps 6:1-\allowbreak9; 13:1-\allowbreak5; 30:7-\allowbreak11; 32:3-\allowbreak7; 40:1-\allowbreak3; 69:29-\allowbreak30; 116:3-\allowbreak7; 126:5,\allowbreak6}
\crossref{Matt}{5}{5}{Mt 11:29; 21:5 Nu 12:3 Ps 22:26; 25:9; 69:32}
\crossref{Matt}{5}{6}{Ps 42:1,\allowbreak2; 63:1,\allowbreak2; 84:2; 107:9 Am 8:11-\allowbreak13 Lu 1:53; 6:21,\allowbreak25}
\crossref{Matt}{5}{7}{Mt 6:14,\allowbreak15; 18:33-\allowbreak35 2Sa 22:26 Job 31:16-\allowbreak22 Ps 18:25; 37:26}
\crossref{Matt}{5}{8}{Mt 23:25-\allowbreak28 1Ch 29:17-\allowbreak19 Ps 15:2; 18:26; 24:4; 51:6,\allowbreak10; 73:1}
\crossref{Matt}{5}{9}{1Ch 12:17 Ps 34:12; 120:6; 122:6-\allowbreak8 Ac 7:26 Ro 12:18; 14:1-\allowbreak7,\allowbreak17-\allowbreak19}
\crossref{Matt}{5}{10}{Mt 10:23 Ps 37:12 Mr 10:30 Lu 6:22; 21:12 Joh 15:20 Ac 5:40}
\crossref{Matt}{5}{11}{Mt 10:25; 27:39 Ps 35:11 Isa 66:5 Lu 7:33,\allowbreak34 Joh 9:28 1Pe 2:23}
\crossref{Matt}{5}{12}{Lu 6:23 Ac 5:41; 16:25 Ro 5:3 2Co 4:17 Php 2:17 Col 1:24}
\crossref{Matt}{5}{13}{Le 2:13 Col 4:6}
\crossref{Matt}{5}{14}{Pr 4:18 Joh 5:35; 12:36 Ro 2:19,\allowbreak20 2Co 6:14 Eph 5:8-\allowbreak14 Php 2:15}
\crossref{Matt}{5}{15}{Mr 4:21 Lu 8:16; 11:33}
\crossref{Matt}{5}{16}{Pr 4:18 Isa 58:8; 60:1-\allowbreak3 Ro 13:11-\allowbreak14 Eph 5:8 Php 2:15,\allowbreak16}
\crossref{Matt}{5}{17}{Lu 16:17 Joh 8:5 Ac 6:13; 18:13; 21:28 Ro 3:31; 10:4 Ga 3:17-\allowbreak24}
\crossref{Matt}{5}{18}{5:26; 6:2,\allowbreak16; 8:10; 10:15,\allowbreak23,\allowbreak42; 11:11; 13:17; 16:28; 17:20; 18:3,\allowbreak18}
\crossref{Matt}{5}{19}{De 27:26 Ps 119:6,\allowbreak128 Ga 3:10-\allowbreak13 Jas 2:10,\allowbreak11}
\crossref{Matt}{5}{20}{Mt 23:2-\allowbreak5,\allowbreak23-\allowbreak28 Lu 11:39,\allowbreak40,\allowbreak44; 12:1; 16:14,\allowbreak15; 18:10-\allowbreak14; 20:46,\allowbreak47}
\crossref{Matt}{5}{21}{5:27,\allowbreak33,\allowbreak43 2Sa 20:18 Job 8:8-\allowbreak10}
\crossref{Matt}{5}{22}{5:28,\allowbreak34,\allowbreak44; 3:17; 17:5 De 18:18,\allowbreak19 Ac 3:20-\allowbreak23; 7:37 Heb 5:9; 12:25}
\crossref{Matt}{5}{23}{Mt 8:4; 23:19 De 16:16,\allowbreak17 1Sa 15:22 Isa 1:10-\allowbreak17 Ho 6:6 Am 5:21-\allowbreak24}
\crossref{Matt}{5}{24}{Mt 18:15-\allowbreak17 Job 42:8 Pr 25:9 Mr 9:50 Ro 12:17,\allowbreak18 1Co 6:7,\allowbreak8}
\crossref{Matt}{5}{25}{Ge 32:3-\allowbreak8,\allowbreak13-\allowbreak22; 33:3-\allowbreak11 1Sa 25:17-\allowbreak35 Pr 6:1-\allowbreak5; 25:8 Lu 12:58,\allowbreak59}
\crossref{Matt}{5}{26}{Mt 18:34; 25:41,\allowbreak46 Lu 12:59; 16:26 2Th 1:9 Jas 2:13}
\crossref{Matt}{5}{27}{Ex 20:14 Le 20:10 De 5:18; 22:22-\allowbreak24 Pr 6:32}
\crossref{Matt}{5}{28}{5:22,\allowbreak39; 7:28,\allowbreak29}
\crossref{Matt}{5}{29}{Mt 18:8,\allowbreak9 Mr 9:43-\allowbreak48}
\crossref{Matt}{5}{30}{Mt 11:6; 13:21; 16:23; 18:6,\allowbreak7; 26:31 Lu 17:2 Ro 9:33; 14:20,\allowbreak21}
\crossref{Matt}{5}{31}{Mt 19:3,\allowbreak7 De 24:1-\allowbreak4 Jer 3:1 Mr 10:2-\allowbreak9}
\crossref{Matt}{5}{32}{5:28 Lu 9:30,\allowbreak35}
\crossref{Matt}{5}{33}{Mt 23:16}
\crossref{Matt}{5}{34}{De 23:21-\allowbreak23 Ec 9:2 Jas 5:12}
\crossref{Matt}{5}{35}{Ps 99:5}
\crossref{Matt}{5}{36}{Mt 23:16-\allowbreak21}
\crossref{Matt}{5}{37}{2Co 1:17-\allowbreak20 Col 4:6 Jas 5:12}
\crossref{Matt}{5}{38}{Ex 21:22-\allowbreak27 Le 24:19,\allowbreak20 De 19:19}
\crossref{Matt}{5}{39}{Le 19:18 1Sa 24:10-\allowbreak15; 25:31-\allowbreak34; 26:8-\allowbreak10 Job 31:29-\allowbreak31 Pr 20:22}
\crossref{Matt}{5}{40}{Lu 6:29 1Co 6:7}
\crossref{Matt}{5}{41}{Mt 27:32 Mr 15:21 Lu 23:26}
\crossref{Matt}{5}{42}{Mt 25:35-\allowbreak40 De 15:7-\allowbreak14 Job 31:16-\allowbreak20 Ps 37:21,\allowbreak25,\allowbreak26; 112:5-\allowbreak9}
\crossref{Matt}{5}{43}{Mt 19:19; 22:39,\allowbreak40 Le 19:18 Mr 12:31-\allowbreak34 Lu 10:27-\allowbreak29}
\crossref{Matt}{5}{44}{Ex 23:4,\allowbreak5 2Ki 6:22 2Ch 28:9-\allowbreak15 Ps 7:4; 35:13,\allowbreak14 Pr 25:21,\allowbreak22}
\crossref{Matt}{5}{45}{5:9 Lu 6:35 Joh 13:35 Eph 5:1 1Jo 3:9}
\crossref{Matt}{5}{46}{Mt 6:1 Lu 6:32-\allowbreak35 1Pe 2:20-\allowbreak23}
\crossref{Matt}{5}{47}{Mt 10:12 Lu 6:32; 10:4,\allowbreak5}
\crossref{Matt}{5}{48}{Ge 17:1 Le 11:44; 19:2; 20:26 De 18:13 Job 1:1,\allowbreak2,\allowbreak3 Ps 37:37}
\crossref{Matt}{6}{1}{Mt 16:6 Mr 8:15 Lu 11:35; 12:1,\allowbreak15 Heb 2:1}
\crossref{Matt}{6}{2}{Job 31:16-\allowbreak20 Ps 37:21; 112:9 Pr 19:17 Ec 11:2 Isa 58:7,\allowbreak10-\allowbreak12}
\crossref{Matt}{6}{3}{Mt 8:4; 9:30; 12:19 Mr 1:44 Joh 7:4}
\crossref{Matt}{6}{4}{6:6,\allowbreak18 Ps 17:3; 44:21; 139:1-\allowbreak3,\allowbreak12 Jer 17:10; 23:24 Heb 4:13}
\crossref{Matt}{6}{5}{Mt 7:7,\allowbreak8; 9:38; 21:22 Ps 5:2; 55:17 Pr 15:8 Isa 55:6,\allowbreak7 Jer 29:12}
\crossref{Matt}{6}{6}{Mt 14:23; 26:36-\allowbreak39 Ge 32:24-\allowbreak29 2Ki 4:33 Isa 26:20 Joh 1:48}
\crossref{Matt}{6}{7}{1Ki 18:26-\allowbreak29 Ec 5:2,\allowbreak3,\allowbreak7 Ac 19:34}
\crossref{Matt}{6}{8}{6:32 Ps 38:9; 69:17-\allowbreak19 Lu 12:30 Joh 16:23-\allowbreak27 Php 4:6}
\crossref{Matt}{6}{9}{Lu 11:1,\allowbreak2}
\crossref{Matt}{6}{10}{Mt 3:2; 4:17; 16:28 Ps 2:6 Isa 2:2 Jer 23:5 Da 2:44; 7:13,\allowbreak27}
\crossref{Matt}{6}{11}{Mt 4:4 Ex 16:16-\allowbreak35 Job 23:12 Ps 33:18,\allowbreak19; 34:10 Pr 30:8}
\crossref{Matt}{6}{12}{Ex 34:7 1Ki 8:30,\allowbreak34,\allowbreak39,\allowbreak50 Ps 32:1; 130:4 Isa 1:18 Da 9:19}
\crossref{Matt}{6}{13}{Mt 26:41 Ge 22:1 De 8:2,\allowbreak16 Pr 30:8 Lu 22:31-\allowbreak46 1Co 10:13}
\crossref{Matt}{6}{14}{6:12; 7:2; 18:21-\allowbreak35 Pr 21:13 Mr 11:25,\allowbreak26 Eph 4:32 Col 3:13}
\crossref{Matt}{6}{15}{}
\crossref{Matt}{6}{16}{Mt 9:14,\allowbreak15 2Sa 12:16,\allowbreak21 Ne 1:4 Es 4:16 Ps 35:13; 69:10; 109:24}
\crossref{Matt}{6}{17}{Ru 3:3 2Sa 14:2 Ec 9:8 Da 10:2,\allowbreak3}
\crossref{Matt}{6}{18}{2Co 5:9; 10:18 Col 3:22-\allowbreak24 1Pe 2:13}
\crossref{Matt}{6}{19}{Job 31:24 Ps 39:6; 62:10 Pr 11:4; 16:16; 23:5 Ec 2:26; 5:10-\allowbreak14}
\crossref{Matt}{6}{20}{Mt 19:21 Isa 33:6 Lu 12:33; 18:22 1Ti 6:17 Heb 10:34; 11:26}
\crossref{Matt}{6}{21}{Isa 33:6 Lu 12:34 2Co 4:18}
\crossref{Matt}{6}{22}{Lu 11:34-\allowbreak36}
\crossref{Matt}{6}{23}{Mt 20:15 Isa 44:18-\allowbreak20 Mr 7:22 Eph 4:18; 5:8 1Jo 2:11}
\crossref{Matt}{6}{24}{Mt 4:10 Jos 24:15,\allowbreak19,\allowbreak20 1Sa 7:3 1Ki 18:21 2Ki 17:33,\allowbreak34,\allowbreak41}
\crossref{Matt}{6}{25}{Mt 5:22-\allowbreak28 Lu 12:4,\allowbreak5,\allowbreak8,\allowbreak9,\allowbreak22}
\crossref{Matt}{6}{26}{Mt 10:29-\allowbreak31 Ge 1:29-\allowbreak31 Job 35:11; 38:41 Ps 104:11,\allowbreak12,\allowbreak27,\allowbreak28}
\crossref{Matt}{6}{27}{Mt 5:36 Ps 39:6 Ec 3:14 Lu 12:25,\allowbreak26 1Co 12:18}
\crossref{Matt}{6}{28}{6:25,\allowbreak31; 10:10 Lu 3:11; 22:35,\allowbreak36}
\crossref{Matt}{6}{29}{1Ki 10:5-\allowbreak7 2Ch 9:4-\allowbreak6,\allowbreak20-\allowbreak22 1Ti 2:9,\allowbreak10 1Pe 3:2-\allowbreak5}
\crossref{Matt}{6}{30}{Ps 90:5,\allowbreak6; 92:7 Isa 40:6-\allowbreak8 Lu 12:28 Jas 1:10,\allowbreak11 1Pe 1:24}
\crossref{Matt}{6}{31}{Mt 4:4; 15:33 Le 25:20-\allowbreak22 2Ch 25:9 Ps 37:3; 55:22; 78:18-\allowbreak31}
\crossref{Matt}{6}{32}{Mt 5:46,\allowbreak47; 20:25,\allowbreak26 Ps 17:14 Lu 12:30 Eph 4:17 1Th 4:5}
\crossref{Matt}{6}{33}{1Ki 3:11-\allowbreak13; 17:13 2Ch 1:7-\allowbreak12; 31:20,\allowbreak21 Pr 2:1-\allowbreak9; 3:9,\allowbreak10}
\crossref{Matt}{6}{34}{6:11,\allowbreak25 Ex 16:18-\allowbreak20 La 3:23}
\crossref{Matt}{7}{1}{Isa 66:5 Eze 16:52-\allowbreak56 Lu 6:37 Ro 2:1,\allowbreak2; 14:3,\allowbreak4,\allowbreak10-\allowbreak13 1Co 4:3-\allowbreak5}
\crossref{Matt}{7}{2}{Jud 1:7 Ps 18:25,\allowbreak26; 137:7,\allowbreak8 Jer 51:24 Ob 1:15 Mr 4:24 Lu 6:38}
\crossref{Matt}{7}{3}{Lu 6:41,\allowbreak42; 18:11}
\crossref{Matt}{7}{4}{}
\crossref{Matt}{7}{5}{Mt 22:18; 23:14 etc.}
\crossref{Matt}{7}{6}{Mt 10:14,\allowbreak15; 15:26 Pr 9:7,\allowbreak8; 23:9; 26:11 Ac 13:45-\allowbreak47 Php 3:2}
\crossref{Matt}{7}{7}{7:11; 21:22 1Ki 3:5 Ps 10:17; 50:15; 86:5; 145:18,\allowbreak19 Isa 55:6,\allowbreak7}
\crossref{Matt}{7}{8}{Mt 15:22-\allowbreak28 2Ch 33:1,\allowbreak2,\allowbreak19 Ps 81:10,\allowbreak16 Joh 2:2; 3:8-\allowbreak10}
\crossref{Matt}{7}{9}{Lu 11:11-\allowbreak13}
\crossref{Matt}{7}{10}{}
\crossref{Matt}{7}{11}{Ge 6:5; 8:21 Job 15:16 Jer 17:9 Ro 3:9,\allowbreak19 Ga 3:22 Eph 2:1-\allowbreak3}
\crossref{Matt}{7}{12}{Lu 6:31}
\crossref{Matt}{7}{13}{Mt 3:2,\allowbreak8; 18:2,\allowbreak3; 23:13 Pr 9:6 Isa 55:7 Eze 18:27-\allowbreak32 Lu 9:33; 13:24}
\crossref{Matt}{7}{14}{Mt 16:24,\allowbreak25 Pr 4:26,\allowbreak27; 8:20 Isa 30:21; 35:8; 57:14 Jer 6:16}
\crossref{Matt}{7}{15}{Mt 10:17; 16:6,\allowbreak11 Mr 12:38 Lu 12:15 Ac 13:40 Php 3:2 Col 2:8}
\crossref{Matt}{7}{16}{7:20; 12:33 2Pe 2:10-\allowbreak18 Jude 1:10-\allowbreak19}
\crossref{Matt}{7}{17}{Ps 1:3; 92:13,\allowbreak14 Isa 5:3-\allowbreak5; 61:3 Jer 11:19; 17:8 Lu 13:6-\allowbreak9}
\crossref{Matt}{7}{18}{Ga 5:17 1Jo 3:9,\allowbreak10}
\crossref{Matt}{7}{19}{Mt 3:10; 21:19,\allowbreak20 Isa 5:5-\allowbreak7; 27:11 Eze 15:2-\allowbreak7 Lu 3:9; 13:6-\allowbreak9}
\crossref{Matt}{7}{20}{7:16 Ac 5:38}
\crossref{Matt}{7}{21}{Mt 25:11,\allowbreak12 Ho 8:2,\allowbreak3 Lu 6:46; 13:25 Ac 19:13 etc.}
\crossref{Matt}{7}{22}{7:21}
\crossref{Matt}{7}{23}{Mt 25:12 Joh 10:14,\allowbreak27-\allowbreak30 2Ti 2:19}
\crossref{Matt}{7}{24}{7:7,\allowbreak8,\allowbreak13,\allowbreak14; 5:3 etc.}
\crossref{Matt}{7}{25}{Eze 13:11 etc.}
\crossref{Matt}{7}{26}{1Sa 2:30 Pr 14:1 Jer 8:9 Lu 6:49 Jas 2:20}
\crossref{Matt}{7}{27}{Mt 12:43-\allowbreak45; 13:19-\allowbreak22 Eze 13:10-\allowbreak16 1Co 3:13 Heb 10:26-\allowbreak31}
\crossref{Matt}{7}{28}{Mt 13:54 Ps 45:2 Mr 1:22; 6:2 Lu 4:22,\allowbreak32; 19:48 Joh 7:15,\allowbreak46}
\crossref{Matt}{7}{29}{Mt 5:20,\allowbreak28,\allowbreak32,\allowbreak44; 21:23-\allowbreak27; 28:18 De 18:18,\allowbreak19 Ec 8:4 Isa 50:4}
\crossref{Matt}{8}{1}{Mt 5:1}
\crossref{Matt}{8}{2}{Mr 1:40 etc.}
\crossref{Matt}{8}{3}{2Ki 5:11}
\crossref{Matt}{8}{4}{Mt 6:1; 9:30; 12:16-\allowbreak19; 16:20; 17:9 Mr 1:43,\allowbreak44; 5:43; 7:36}
\crossref{Matt}{8}{5}{Mt 4:13; 9:1; 11:23 Mr 2:1 Lu 7:1}
\crossref{Matt}{8}{6}{Job 31:13,\allowbreak14 Ac 10:7 Col 3:11; 4:1 1Ti 6:2 Phm 1:16}
\crossref{Matt}{8}{7}{Mt 9:18,\allowbreak19 Mr 5:23,\allowbreak24 Lu 7:6}
\crossref{Matt}{8}{8}{Mt 3:11,\allowbreak14; 15:26,\allowbreak27 Ge 32:10 Ps 10:17 Lu 5:8; 7:6,\allowbreak7; 15:19,\allowbreak21}
\crossref{Matt}{8}{9}{Job 38:34,\allowbreak35 Ps 107:25-\allowbreak29; 119:91; 148:8 Jer 47:6,\allowbreak7 Eze 14:17-\allowbreak21}
\crossref{Matt}{8}{10}{Mr 6:6 Lu 7:9}
\crossref{Matt}{8}{11}{Mt 24:31 Ge 12:3; 22:18; 28:14; 49:10 Ps 22:27; 98:3 Isa 2:2,\allowbreak3; 11:10}
\crossref{Matt}{8}{12}{Mt 3:9,\allowbreak10; 7:22,\allowbreak23; 21:43 Ac 3:25 Ro 9:4}
\crossref{Matt}{8}{13}{8:4 Ec 9:7 Mr 7:29 Joh 4:50}
\crossref{Matt}{8}{14}{8:20; 17:25 Mr 1:29-\allowbreak31 Lu 4:38,\allowbreak39}
\crossref{Matt}{8}{15}{8:3; 9:20,\allowbreak29; 14:36; 20:34 2Ki 13:21 Isa 6:7 Mr 1:41 Lu 8:54}
\crossref{Matt}{8}{16}{Mr 1:32-\allowbreak34 Lu 4:40}
\crossref{Matt}{8}{17}{Mt 1:22; 2:15,\allowbreak23}
\crossref{Matt}{8}{18}{8:1 Mr 1:35-\allowbreak38 Lu 4:42,\allowbreak43 Joh 6:15}
\crossref{Matt}{8}{19}{Ezr 7:6 Mr 12:32-\allowbreak34 Lu 9:57,\allowbreak58 1Co 1:20}
\crossref{Matt}{8}{20}{Ps 84:3; 104:17}
\crossref{Matt}{8}{21}{Lu 9:59-\allowbreak62}
\crossref{Matt}{8}{22}{Mt 4:18-\allowbreak22; 9:9 Joh 1:43}
\crossref{Matt}{8}{23}{Mt 9:1 Mr 4:36 Lu 7:22}
\crossref{Matt}{8}{24}{Ps 107:23-\allowbreak27 Isa 54:11 Jon 1:4,\allowbreak5 Mr 4:37,\allowbreak38 Ac 27:14 etc.}
\crossref{Matt}{8}{25}{Ps 10:1; 44:22,\allowbreak23 Isa 51:9,\allowbreak10 Mr 4:38,\allowbreak39 Lu 8:24}
\crossref{Matt}{8}{26}{Mt 6:30; 14:30,\allowbreak31; 16:8 Isa 41:10-\allowbreak14 Mr 4:40 Lu 8:25 Ro 4:20}
\crossref{Matt}{8}{27}{Mt 14:33; 15:31 Mr 1:27; 6:51; 7:37}
\crossref{Matt}{8}{28}{Mr 5:1 etc.}
\crossref{Matt}{8}{29}{2Sa 16:10; 19:22 Joe 3:4 Mr 1:24; 5:7 Lu 4:34; 8:28 Joh 2:4}
\crossref{Matt}{8}{30}{Le 11:7 De 14:8 Isa 65:3,\allowbreak4; 66:3 Mr 5:11 Lu 8:32; 15:15,\allowbreak16}
\crossref{Matt}{8}{31}{Mr 5:7,\allowbreak12 Lu 8:30-\allowbreak33 Re 12:12; 20:1,\allowbreak2}
\crossref{Matt}{8}{32}{1Ki 22:22 Job 1:10-\allowbreak12; 2:3-\allowbreak6 Ac 2:23; 4:28 Re 20:7}
\crossref{Matt}{8}{33}{Mr 5:14-\allowbreak16 Lu 8:34-\allowbreak36 Ac 19:15-\allowbreak17}
\crossref{Matt}{8}{34}{8:29 De 5:25 1Sa 16:4 1Ki 17:18; 18:17 Job 21:14; 22:17}
\crossref{Matt}{9}{1}{Mt 7:6; 8:18,\allowbreak23 Mr 5:21 Lu 8:37 Re 22:11}
\crossref{Matt}{9}{2}{Mt 4:24; 8:16 Mr 1:32; 2:1-\allowbreak3 Lu 5:18,\allowbreak19 Ac 5:15,\allowbreak16; 19:12}
\crossref{Matt}{9}{3}{Mt 7:29 Mr 2:6,\allowbreak7; 7:21 Lu 5:21; 7:39,\allowbreak40}
\crossref{Matt}{9}{4}{Mt 12:25; 16:7,\allowbreak8 Ps 44:21; 139:2 Mr 2:8; 8:16,\allowbreak17; 12:15 Lu 5:22}
\crossref{Matt}{9}{5}{Mr 2:9-\allowbreak12 Lu 5:23-\allowbreak25}
\crossref{Matt}{9}{6}{Isa 43:25 Mic 7:18 Mr 2:7,\allowbreak10 Lu 5:21 Joh 5:21-\allowbreak23; 10:28; 17:2}
\crossref{Matt}{9}{7}{}
\crossref{Matt}{9}{8}{Mt 12:23; 15:31 Mr 2:12; 7:37 Lu 5:26; 7:16}
\crossref{Matt}{9}{9}{Mt 21:31,\allowbreak32 Mr 2:14 etc.}
\crossref{Matt}{9}{10}{Mr 2:15,\allowbreak16 etc.}
\crossref{Matt}{9}{11}{Mr 2:16; 9:14-\allowbreak16}
\crossref{Matt}{9}{12}{Ps 6:2; 41:4; 147:3 Jer 17:14; 30:17; 33:6 Ho 14:4 Mr 2:17 Lu 5:31}
\crossref{Matt}{9}{13}{Mt 12:3,\allowbreak5,\allowbreak7; 19:4; 21:42; 22:31,\allowbreak32 Mr 12:26 Lu 10:26 Joh 10:34}
\crossref{Matt}{9}{14}{Mt 11:2 Joh 3:25; 4:1}
\crossref{Matt}{9}{15}{Mt 25:1-\allowbreak10 Jud 14:11 etc.}
\crossref{Matt}{9}{16}{Ge 33:14 Ps 125:3 Isa 40:11 Joh 16:12 1Co 3:1,\allowbreak2; 13:13}
\crossref{Matt}{9}{17}{Jos 9:4 Job 32:19 Ps 119:83}
\crossref{Matt}{9}{18}{Mr 5:22 etc.}
\crossref{Matt}{9}{19}{Mt 8:7 Joh 4:34 Ac 10:38 Ga 6:9,\allowbreak10}
\crossref{Matt}{9}{20}{Mr 5:25 etc.}
\crossref{Matt}{9}{21}{Mr 5:26-\allowbreak33 Lu 8:45-\allowbreak47 Ac 19:12}
\crossref{Matt}{9}{22}{9:2 Mr 5:34 Lu 8:48}
\crossref{Matt}{9}{23}{9:18,\allowbreak19 Mr 5:35-\allowbreak38 Lu 8:49-\allowbreak51}
\crossref{Matt}{9}{24}{1Ki 17:18-\allowbreak24 Ac 9:40; 20:10}
\crossref{Matt}{9}{25}{2Ki 4:32-\allowbreak36 Ac 9:40,\allowbreak41}
\crossref{Matt}{9}{26}{Mt 4:24; 14:1,\allowbreak2 Mr 1:45; 6:14 Ac 26:26}
\crossref{Matt}{9}{27}{Mt 11:5; 12:22; 20:30 Mr 8:22,\allowbreak23; 10:46 Lu 7:21 Joh 9:1 etc.}
\crossref{Matt}{9}{28}{Mt 8:14; 13:36}
\crossref{Matt}{9}{29}{Mt 20:34 Joh 9:6,\allowbreak7}
\crossref{Matt}{9}{30}{Ps 146:8 Isa 35:5; 42:7; 52:13 Joh 9:7-\allowbreak26}
\crossref{Matt}{9}{31}{Mr 1:44,\allowbreak45; 7:36}
\crossref{Matt}{9}{32}{Mt 12:22,\allowbreak23 Mr 9:17-\allowbreak27 Lu 11:14}
\crossref{Matt}{9}{33}{Mt 15:30,\allowbreak31 Ex 4:11,\allowbreak12 Isa 35:6 Mr 7:32-\allowbreak37 Lu 11:14}
\crossref{Matt}{9}{34}{Mt 12:23,\allowbreak24 Mr 3:22 Lu 11:15 Joh 3:20}
\crossref{Matt}{9}{35}{Mt 4:23,\allowbreak24; 11:1,\allowbreak5 Mr 1:32-\allowbreak39; 6:6,\allowbreak56 Lu 4:43,\allowbreak44; 13:22}
\crossref{Matt}{9}{36}{Mt 14:14; 15:32 Mr 6:34 Heb 4:15; 5:2}
\crossref{Matt}{9}{37}{Mt 28:19 Mr 16:15 Lu 10:2; 24:47 Joh 4:35,\allowbreak36 Ac 16:9; 18:10}
\crossref{Matt}{9}{38}{Lu 6:12,\allowbreak13 Ac 13:2 2Th 3:1}
\crossref{Matt}{10}{1}{Mt 19:28; 26:20,\allowbreak47 Mr 3:13,\allowbreak14; 6:7 etc.}
\crossref{Matt}{10}{2}{Lu 6:13; 9:10; 11:49; 22:14 Ac 1:26 Eph 4:11 Heb 3:1 Re 18:20}
\crossref{Matt}{10}{3}{Mr 3:18 Lu 6:14 Joh 1:43-\allowbreak46; 6:5-\allowbreak7; 12:21,\allowbreak22; 14:9}
\crossref{Matt}{10}{4}{Mr 3:18 Lu 6:15}
\crossref{Matt}{10}{5}{Mt 22:3 Lu 9:2; 10:1 Joh 20:21}
\crossref{Matt}{10}{6}{Mt 15:24-\allowbreak26 Lu 24:47 Ac 3:26; 13:46; 18:6; 26:20; 28:25-\allowbreak28}
\crossref{Matt}{10}{7}{Mt 4:17; 11:1 Isa 61:1 Joh 3:2 Mr 6:12 Lu 9:60; 16:16 Ac 4:2}
\crossref{Matt}{10}{8}{10:1 Mr 16:18 Lu 10:9 Ac 4:9,\allowbreak10,\allowbreak30; 5:12-\allowbreak15}
\crossref{Matt}{10}{9}{Mr 6:8 Lu 9:3; 10:4; 22:35 1Co 9:7 etc.}
\crossref{Matt}{10}{10}{1Sa 9:7; 17:40}
\crossref{Matt}{10}{11}{Ge 19:1-\allowbreak3 Jud 19:16-\allowbreak21 1Ki 17:9 etc.}
\crossref{Matt}{10}{12}{Lu 10:5,\allowbreak6 Ac 10:36 2Co 5:20 3Jo 1:14}
\crossref{Matt}{10}{13}{Ps 35:13 Lu 10:6 2Co 2:16}
\crossref{Matt}{10}{14}{10:40,\allowbreak41; 18:5 Mr 6:11; 9:37 Lu 9:5,\allowbreak48; 10:10,\allowbreak11 Joh 13:20}
\crossref{Matt}{10}{15}{Mt 5:18; 24:34,\allowbreak35}
\crossref{Matt}{10}{16}{Lu 10:3 Ac 20:29}
\crossref{Matt}{10}{17}{Mic 7:5 Mr 13:9,\allowbreak12 Ac 14:5,\allowbreak6; 17:14; 23:12-\allowbreak22 2Co 11:24-\allowbreak26}
\crossref{Matt}{10}{18}{Ps 2:1-\allowbreak6 Ac 5:25-\allowbreak27; 12:1-\allowbreak4; 23:33,\allowbreak34; 24:1-\allowbreak26:32 2Ti 4:16,\allowbreak17}
\crossref{Matt}{10}{19}{Mr 13:11-\allowbreak13 Lu 12:11; 21:14,\allowbreak15}
\crossref{Matt}{10}{20}{2Sa 23:2 Mr 12:36 Lu 11:13; 21:15 Ac 2:4; 4:8; 6:10; 7:55,\allowbreak56}
\crossref{Matt}{10}{21}{10:34-\allowbreak36; 24:10 Mic 7:5,\allowbreak6 Zec 13:3 Mr 13:12,\allowbreak13 Lu 12:51-\allowbreak53; 21:16,\allowbreak17}
\crossref{Matt}{10}{22}{Mt 24:9 Isa 66:5,\allowbreak6 Lu 6:22 Joh 7:7; 15:18,\allowbreak19; 17:14 1Jo 3:13}
\crossref{Matt}{10}{23}{Mt 2:13; 4:12; 12:14,\allowbreak15 Lu 4:29-\allowbreak31 Joh 7:1; 10:39-\allowbreak42; 11:53,\allowbreak54}
\crossref{Matt}{10}{24}{2Sa 11:11 Lu 6:40 Joh 13:16; 15:20 Heb 12:2-\allowbreak4}
\crossref{Matt}{10}{25}{Mt 9:34; 12:24 Mr 3:22 Lu 11:15 Joh 7:20; 8:48,\allowbreak52; 10:20}
\crossref{Matt}{10}{26}{10:28 Pr 28:1; 29:25 Isa 41:10,\allowbreak14; 43:1,\allowbreak2; 51:7,\allowbreak8,\allowbreak12,\allowbreak13}
\crossref{Matt}{10}{27}{Mt 13:1-\allowbreak17,\allowbreak34,\allowbreak35 Lu 8:10 Joh 16:1,\allowbreak13,\allowbreak25,\allowbreak29 2Co 3:12}
\crossref{Matt}{10}{28}{10:26 Isa 8:12,\allowbreak13; 51:7,\allowbreak12 Da 3:10-\allowbreak18 Lu 12:4,\allowbreak5 Ac 20:23,\allowbreak24}
\crossref{Matt}{10}{29}{Lu 12:6,\allowbreak7}
\crossref{Matt}{10}{30}{1Sa 14:45 2Sa 14:11 1Ki 1:52 Lu 12:7; 21:18 Ac 27:34}
\crossref{Matt}{10}{31}{Mt 6:26; 12:11,\allowbreak12 Ps 8:5 Lu 12:24 1Co 9:9,\allowbreak10}
\crossref{Matt}{10}{32}{Ps 119:46 Lu 12:8,\allowbreak9 Joh 9:22 Ro 10:9,\allowbreak10 1Ti 6:12,\allowbreak13}
\crossref{Matt}{10}{33}{Mt 26:70-\allowbreak75 Mr 14:30,\allowbreak72 Lu 9:26; 12:9 2Ti 2:12 2Pe 2:1 1Jo 2:23}
\crossref{Matt}{10}{34}{Jer 15:10 Lu 12:49-\allowbreak53 Joh 7:40-\allowbreak52 Ac 13:45-\allowbreak50; 14:2,\allowbreak4}
\crossref{Matt}{10}{35}{10:21; 24:10 Mic 7:5 Mr 13:12 Lu 21:16}
\crossref{Matt}{10}{36}{Ge 3:15; 4:8-\allowbreak10; 37:17-\allowbreak28 1Sa 17:28 2Sa 16:11 Job 19:13-\allowbreak19}
\crossref{Matt}{10}{37}{Mt 22:37 De 33:9 Lu 14:26 Joh 5:23; 21:15-\allowbreak17 2Co 5:14,\allowbreak15}
\crossref{Matt}{10}{38}{Mt 16:24; 27:32 Mr 8:34; 10:21 Lu 9:23,\allowbreak24; 14:27 Joh 19:17}
\crossref{Matt}{10}{39}{Mt 16:25,\allowbreak26 Mr 8:35,\allowbreak36 Lu 17:33 Joh 12:25 Php 1:20,\allowbreak21}
\crossref{Matt}{10}{40}{Mt 18:5; 25:40,\allowbreak45 Lu 9:48; 10:16 Joh 13:20; 20:21 2Co 5:20}
\crossref{Matt}{10}{41}{Ge 20:7 1Ki 17:9-\allowbreak15,\allowbreak20-\allowbreak24; 18:3,\allowbreak4 2Ki 4:8-\allowbreak10,\allowbreak16,\allowbreak17,\allowbreak32-\allowbreak37}
\crossref{Matt}{10}{42}{Mt 8:5,\allowbreak6; 18:3-\allowbreak6,\allowbreak10,\allowbreak14; 25:40 Zec 13:7 Mr 9:42 Lu 17:2 1Co 8:10-\allowbreak13}
\crossref{Matt}{11}{1}{Mt 28:20 Joh 15:10,\allowbreak14 Ac 1:2; 10:42 1Th 4:2 2Th 3:6,\allowbreak10 1Ti 6:14}
\crossref{Matt}{11}{2}{Mt 4:12; 14:3 Mr 6:17 Lu 3:19; 7:18-\allowbreak23 Joh 3:24}
\crossref{Matt}{11}{3}{Mt 2:2-\allowbreak6 Ge 3:15; 12:3; 49:10 Nu 24:17 De 18:15-\allowbreak18 Ps 2:6-\allowbreak12}
\crossref{Matt}{11}{4}{}
\crossref{Matt}{11}{5}{Mt 9:30 Ps 146:8 Isa 29:18; 35:4-\allowbreak6; 42:6,\allowbreak7 Lu 4:18; 7:21,\allowbreak22}
\crossref{Matt}{11}{6}{Mt 5:3-\allowbreak12 Ps 1:1,\allowbreak2; 32:1,\allowbreak2; 119:1 Lu 11:27,\allowbreak28}
\crossref{Matt}{11}{7}{Lu 7:24-\allowbreak30}
\crossref{Matt}{11}{8}{Mt 3:4 2Ki 1:8 Isa 20:2 Zec 13:4 1Co 4:11 2Co 11:27 Re 11:3}
\crossref{Matt}{11}{9}{11:13,\allowbreak14; 14:5; 17:12,\allowbreak13; 21:24-\allowbreak26 Mr 9:11-\allowbreak13 Lu 1:15-\allowbreak17,\allowbreak76}
\crossref{Matt}{11}{10}{Mt 3:3 Isa 40:3 Mal 3:1; 4:5 Mr 1:2 Lu 7:26,\allowbreak27 Joh 1:23}
\crossref{Matt}{11}{11}{Job 14:1,\allowbreak4; 15:14; 25:4 Ps 51:5 Eph 2:3}
\crossref{Matt}{11}{12}{Mt 21:23-\allowbreak32 Lu 7:29,\allowbreak30; 13:24; 16:16 Joh 6:27 Eph 6:11-\allowbreak13}
\crossref{Matt}{11}{13}{Mt 5:17,\allowbreak18 Mal 4:6 Lu 24:27,\allowbreak44 Joh 5:46,\allowbreak47 Ac 3:22-\allowbreak24; 13:27}
\crossref{Matt}{11}{14}{Eze 2:5; 3:10,\allowbreak11 Joh 16:12 1Co 3:2}
\crossref{Matt}{11}{15}{Mt 13:9,\allowbreak43 Mr 4:9,\allowbreak23; 7:16 Lu 8:8 Re 2:7,\allowbreak11,\allowbreak17,\allowbreak29; 3:6,\allowbreak13,\allowbreak22}
\crossref{Matt}{11}{16}{La 2:13 Mr 4:30 Lu 13:18}
\crossref{Matt}{11}{17}{Isa 28:9-\allowbreak13 1Co 9:19-\allowbreak23}
\crossref{Matt}{11}{18}{Mt 3:4 Jer 15:17; 16:8,\allowbreak9 Lu 1:15 1Co 9:27}
\crossref{Matt}{11}{19}{Lu 5:29,\allowbreak30; 7:34,\allowbreak36; 14:1 Joh 2:2; 12:2 etc.}
\crossref{Matt}{11}{20}{Lu 10:13-\allowbreak15}
\crossref{Matt}{11}{21}{Mt 18:7; 23:13-\allowbreak29; 26:24 Jer 13:27 Lu 11:42-\allowbreak52 Jude 1:11}
\crossref{Matt}{11}{22}{11:24; 10:15 Lu 10:14; 12:47,\allowbreak48 Heb 2:3; 6:4-\allowbreak8; 10:26-\allowbreak31}
\crossref{Matt}{11}{23}{Mt 4:13; 8:5; 17:24 Lu 4:23 Joh 4:46 etc.}
\crossref{Matt}{11}{24}{11:22; 10:15 La 4:6 Mr 6:11 Lu 10:12}
\crossref{Matt}{11}{25}{Lu 10:21 etc.}
\crossref{Matt}{11}{26}{Job 33:13 Isa 46:10 Lu 10:21 Ro 9:18; 11:33-\allowbreak36 Eph 1:9,\allowbreak11; 3:11}
\crossref{Matt}{11}{27}{Mt 28:18 Joh 3:35; 5:21-\allowbreak29; 13:3; 17:2 1Co 15:25-\allowbreak27 Eph 1:20-\allowbreak23}
\crossref{Matt}{11}{28}{Isa 45:22-\allowbreak25; 53:2,\allowbreak3; 55:1-\allowbreak3 Joh 6:37; 7:37 Re 22:17}
\crossref{Matt}{11}{29}{Mt 7:24; 17:5 Joh 13:17; 14:21-\allowbreak24; 15:10-\allowbreak14 1Co 9:21 2Co 10:5}
\crossref{Matt}{11}{30}{Pr 3:17 Mic 6:8 Ac 15:10,\allowbreak28 Ga 5:1,\allowbreak18 1Jo 5:3}
\crossref{Matt}{12}{1}{Mr 2:23-\allowbreak28 Lu 6:1-\allowbreak5}
\crossref{Matt}{12}{2}{12:10 Ex 20:9-\allowbreak11; 23:12; 31:15-\allowbreak17; 35:2 Nu 15:32-\allowbreak36 Isa 58:13}
\crossref{Matt}{12}{3}{12:5; 19:4; 21:16; 22:31 Mr 12:10,\allowbreak26 Lu 6:3; 10:26}
\crossref{Matt}{12}{4}{Ex 25:30 Le 24:5-\allowbreak9}
\crossref{Matt}{12}{5}{Nu 28:9,\allowbreak10 Joh 7:22,\allowbreak23}
\crossref{Matt}{12}{6}{12:41,\allowbreak42; 23:17-\allowbreak21 2Ch 6:18 Hag 2:7-\allowbreak9 Mal 3:1 Joh 2:19-\allowbreak21}
\crossref{Matt}{12}{7}{Mt 9:13; 22:29 Ac 13:27}
\crossref{Matt}{12}{8}{Mt 9:6 Mr 2:28; 9:4-\allowbreak7 Lu 6:5 Joh 5:17-\allowbreak23 1Co 9:21; 16:2}
\crossref{Matt}{12}{9}{Mr 3:1-\allowbreak5 Lu 6:6-\allowbreak11}
\crossref{Matt}{12}{10}{1Ki 13:4-\allowbreak6 Zec 11:17 Joh 5:3}
\crossref{Matt}{12}{11}{Ex 23:4,\allowbreak5 De 22:4}
\crossref{Matt}{12}{12}{Mt 6:26 Lu 12:24}
\crossref{Matt}{12}{13}{Lu 13:13 Ac 3:7,\allowbreak8}
\crossref{Matt}{12}{14}{Mt 27:1 Mr 3:6 Lu 6:11 Joh 5:18; 10:39; 11:53,\allowbreak57}
\crossref{Matt}{12}{15}{Mt 10:23 Lu 6:12 Joh 7:1; 10:40-\allowbreak42; 11:54}
\crossref{Matt}{12}{16}{Mt 9:30; 17:9 Mr 7:36 Lu 5:14,\allowbreak15}
\crossref{Matt}{12}{17}{Mt 8:17; 13:35; 21:4 Isa 41:22,\allowbreak23; 42:9; 44:26 Lu 21:22; 24:44}
\crossref{Matt}{12}{18}{Isa 49:5,\allowbreak6; 52:13; 53:11 Zec 3:8 Php 2:6,\allowbreak7}
\crossref{Matt}{12}{19}{Mt 11:29 Zec 9:9 Lu 17:20 Joh 18:36-\allowbreak38 2Co 10:1 2Ti 2:24,\allowbreak25}
\crossref{Matt}{12}{20}{Mt 11:28 2Ki 18:21 Ps 51:17; 147:3 Isa 40:11; 57:15; 61:1-\allowbreak3}
\crossref{Matt}{12}{21}{Isa 11:10 Ro 15:12,\allowbreak13 Eph 1:12,\allowbreak13 Col 1:27}
\crossref{Matt}{12}{22}{Mt 9:32 Mr 3:11 Lu 11:14}
\crossref{Matt}{12}{23}{Mt 9:33; 15:30,\allowbreak31}
\crossref{Matt}{12}{24}{Mt 9:34 Mr 3:22 Lu 11:15}
\crossref{Matt}{12}{25}{Mt 9:4 Ps 139:2 Jer 17:10 Am 4:13 Mr 2:8 Joh 2:24,\allowbreak25; 21:17}
\crossref{Matt}{12}{26}{Joh 12:31; 14:30; 16:11 2Co 4:4 Col 1:13 1Jo 5:19 Re 9:11}
\crossref{Matt}{12}{27}{12:24}
\crossref{Matt}{12}{28}{12:18 Mr 16:17 Lu 11:20 Ac 10:38}
\crossref{Matt}{12}{29}{Isa 49:24; 53:12 Mr 3:27 Lu 11:21,\allowbreak22 1Jo 3:8; 4:4 Re 12:7-\allowbreak10}
\crossref{Matt}{12}{30}{Mt 6:24 Jos 5:13; 24:15 1Ch 12:17,\allowbreak18 Mr 9:40 Lu 9:50; 11:23}
\crossref{Matt}{12}{31}{Isa 1:18; 55:7 Eze 33:11 1Ti 1:13-\allowbreak15 Heb 6:4 etc.}
\crossref{Matt}{12}{32}{Mt 11:19; 13:55 Lu 7:34; 23:34 Joh 7:12,\allowbreak52 Ac 3:14,\allowbreak15,\allowbreak19}
\crossref{Matt}{12}{33}{Mt 23:26 Eze 18:31 Am 5:15 Lu 11:39,\allowbreak40 Jas 4:8}
\crossref{Matt}{12}{34}{Mt 3:7; 23:33 Lu 3:7 Joh 8:44 1Jo 3:10}
\crossref{Matt}{12}{35}{Mt 13:52 Ps 37:30,\allowbreak31 Pr 10:20,\allowbreak21; 12:6,\allowbreak17-\allowbreak19; 15:4,\allowbreak23,\allowbreak28}
\crossref{Matt}{12}{36}{Ec 12:14 Ro 2:16 Eph 6:4-\allowbreak6 Jude 1:14,\allowbreak15 Re 20:12}
\crossref{Matt}{12}{37}{Pr 13:3}
\crossref{Matt}{12}{38}{Mt 16:1-\allowbreak4 Mr 8:11,\allowbreak12 Lu 11:16,\allowbreak29 Joh 2:18; 4:48 1Co 1:22}
\crossref{Matt}{12}{39}{Isa 57:3 Mr 8:38 Jas 4:4}
\crossref{Matt}{12}{40}{Jon 1:17}
\crossref{Matt}{12}{41}{Lu 11:32}
\crossref{Matt}{12}{42}{1Ki 10:1 etc.}
\crossref{Matt}{12}{43}{Lu 11:24 Ac 8:13}
\crossref{Matt}{12}{44}{12:29 Lu 11:21,\allowbreak22 Joh 13:27 Eph 2:2 1Jo 4:4}
\crossref{Matt}{12}{45}{12:24 Mr 5:9; 16:9 Eph 6:12}
\crossref{Matt}{12}{46}{Mr 2:21; 3:31 etc.}
\crossref{Matt}{12}{47}{}
\crossref{Matt}{12}{48}{Mt 10:37 De 33:9 Mr 3:32,\allowbreak33 Lu 2:49,\allowbreak52 Joh 2:3,\allowbreak4 2Co 5:16}
\crossref{Matt}{12}{49}{Mt 28:7 Mr 3:34 Joh 17:8,\allowbreak9,\allowbreak20; 20:17-\allowbreak20}
\crossref{Matt}{12}{50}{Mt 7:20,\allowbreak21; 17:5 Mr 3:35 Lu 8:21; 11:27,\allowbreak28 Joh 6:29,\allowbreak40; 15:14}
\crossref{Matt}{13}{1}{Mr 2:13; 4:1}
\crossref{Matt}{13}{2}{Mt 4:25; 15:30 Ge 49:10 Lu 8:4-\allowbreak8}
\crossref{Matt}{13}{3}{13:10-\allowbreak13,\allowbreak34,\allowbreak35,\allowbreak53; 22:1; 24:32 Jud 9:8-\allowbreak20 2Sa 12:1-\allowbreak7 Ps 49:4; 78:2}
\crossref{Matt}{13}{4}{13:18,\allowbreak19}
\crossref{Matt}{13}{5}{13:20 Eze 11:19; 36:26 Am 6:12 Zec 7:12}
\crossref{Matt}{13}{6}{13:21 Isa 49:10 Jas 1:11,\allowbreak12 Re 7:16}
\crossref{Matt}{13}{7}{13:22 Ge 3:18 Jer 4:3,\allowbreak4 Mr 4:18,\allowbreak19}
\crossref{Matt}{13}{8}{13:23 Lu 8:15 Ro 7:18}
\crossref{Matt}{13}{9}{13:16; 11:15 Mr 4:9,\allowbreak23; 7:14-\allowbreak16 Re 2:7,\allowbreak11,\allowbreak17,\allowbreak29; 3:6,\allowbreak13,\allowbreak22; 13:8,\allowbreak9}
\crossref{Matt}{13}{10}{Mr 4:10,\allowbreak33,\allowbreak34}
\crossref{Matt}{13}{11}{Mt 11:25,\allowbreak26; 16:17 Ps 25:8,\allowbreak9,\allowbreak14 Isa 29:10; 35:8 Mr 4:11 Lu 8:10}
\crossref{Matt}{13}{12}{Mt 25:29 Mr 4:24,\allowbreak25 Lu 8:18; 9:26; 19:24-\allowbreak26 Joh 15:2-\allowbreak5}
\crossref{Matt}{13}{13}{13:16 De 29:3,\allowbreak4 Isa 42:18-\allowbreak20; 44:18 Jer 5:21 Eze 12:2}
\crossref{Matt}{13}{14}{Isa 6:9,\allowbreak10 Eze 12:2 Mr 4:12 Lu 8:10 Joh 12:39,\allowbreak40 Ac 28:25-\allowbreak27}
\crossref{Matt}{13}{15}{Ps 119:70}
\crossref{Matt}{13}{16}{Mt 5:3-\allowbreak11; 16:17 Lu 2:29,\allowbreak30; 10:23,\allowbreak24 Joh 20:29 Ac 26:18}
\crossref{Matt}{13}{17}{Lu 10:24 Joh 8:56 Eph 3:5,\allowbreak6 Heb 11:13,\allowbreak39,\allowbreak40 1Pe 1:10-\allowbreak12}
\crossref{Matt}{13}{18}{13:11,\allowbreak12 Mr 4:14 etc.}
\crossref{Matt}{13}{19}{Mt 4:23 Lu 8:11 etc.}
\crossref{Matt}{13}{20}{13:5,\allowbreak6}
\crossref{Matt}{13}{21}{13:6; 7:22,\allowbreak23,\allowbreak26,\allowbreak27 Job 19:28 Pr 12:3,\allowbreak12 Lu 8:13 Joh 6:26,\allowbreak61-\allowbreak65}
\crossref{Matt}{13}{22}{13:7 Mr 4:18 Lu 8:14; 18:24 2Ti 4:10}
\crossref{Matt}{13}{23}{13:8 Mr 4:20 Lu 8:15}
\crossref{Matt}{13}{24}{Mt 21:33 Jud 14:12,\allowbreak13 Isa 28:10,\allowbreak13 Eze 17:2}
\crossref{Matt}{13}{25}{Mt 25:5 Isa 56:9,\allowbreak10 Ac 20:30,\allowbreak31 Ga 2:4 2Ti 4:3-\allowbreak5 Heb 12:15}
\crossref{Matt}{13}{26}{Mr 4:26-\allowbreak29}
\crossref{Matt}{13}{27}{1Co 3:5-\allowbreak9; 12:28,\allowbreak29; 16:10 2Co 5:18-\allowbreak20; 6:1,\allowbreak4 Eph 4:11,\allowbreak12}
\crossref{Matt}{13}{28}{Lu 9:49-\allowbreak54 1Co 5:3-\allowbreak7 2Co 2:6-\allowbreak11 1Th 5:14 Jude 1:22,\allowbreak23}
\crossref{Matt}{13}{29}{}
\crossref{Matt}{13}{30}{13:39; 3:12; 22:10-\allowbreak14; 25:6-\allowbreak13,\allowbreak32 Mal 3:18 1Co 4:5}
\crossref{Matt}{13}{31}{13:24 Lu 19:11; 20:9}
\crossref{Matt}{13}{32}{Ps 72:16-\allowbreak19 Isa 2:2-\allowbreak4 Eze 47:1-\allowbreak5 Da 2:34,\allowbreak35,\allowbreak44,\allowbreak45}
\crossref{Matt}{13}{33}{Mr 13:20}
\crossref{Matt}{13}{34}{13:13 Mr 4:33,\allowbreak34}
\crossref{Matt}{13}{35}{13:14; 21:4,\allowbreak5}
\crossref{Matt}{13}{36}{Mt 14:22; 15:39 Mr 6:45; 8:9}
\crossref{Matt}{13}{37}{13:24,\allowbreak27}
\crossref{Matt}{13}{38}{Mt 24:14; 28:18-\allowbreak20 Mr 16:15-\allowbreak20 Lu 24:47 Ro 10:18; 16:26}
\crossref{Matt}{13}{39}{13:25,\allowbreak28 2Co 2:17; 11:3,\allowbreak13-\allowbreak15 Eph 2:2; 6:11,\allowbreak12 2Th 2:8-\allowbreak11}
\crossref{Matt}{13}{40}{}
\crossref{Matt}{13}{41}{Mt 24:31 Mr 13:27 Heb 1:6,\allowbreak7,\allowbreak14 Re 5:11,\allowbreak12}
\crossref{Matt}{13}{42}{Mt 3:12; 25:41 Ps 21:9 Da 3:6,\allowbreak15-\allowbreak17,\allowbreak21,\allowbreak22 Mr 9:43-\allowbreak49}
\crossref{Matt}{13}{43}{Mt 25:34,\allowbreak36 Da 12:3 1Co 15:41-\allowbreak54,\allowbreak58 Re 21:3-\allowbreak5,\allowbreak22,\allowbreak23}
\crossref{Matt}{13}{44}{Mt 6:21 Pr 2:2-\allowbreak5; 16:16; 17:16; 18:1 Joh 6:35 Ro 15:4}
\crossref{Matt}{13}{45}{Mt 16:26; 22:5 Pr 3:13-\allowbreak18; 8:10,\allowbreak11,\allowbreak18-\allowbreak20}
\crossref{Matt}{13}{46}{Pr 2:4 Isa 33:6 1Co 3:21-\allowbreak23 Eph 3:8 Col 2:3 1Jo 5:11,\allowbreak12}
\crossref{Matt}{13}{47}{Mt 4:19 Mr 1:17 Lu 5:10}
\crossref{Matt}{13}{48}{13:30,\allowbreak40-\allowbreak43; 3:12}
\crossref{Matt}{13}{49}{13:39; 24:31}
\crossref{Matt}{13}{50}{13:42}
\crossref{Matt}{13}{51}{13:11,\allowbreak19; 15:17; 16:11; 24:15 Mr 4:34; 7:18; 8:17,\allowbreak18}
\crossref{Matt}{13}{52}{Mt 23:34 Ezr 7:6,\allowbreak10,\allowbreak21 Lu 11:49 2Co 3:4-\allowbreak6 Col 1:7}
\crossref{Matt}{13}{53}{Mr 4:33-\allowbreak35}
\crossref{Matt}{13}{54}{Mt 2:23 Mr 6:1,\allowbreak2 Lu 4:16-\allowbreak30 Joh 1:11}
\crossref{Matt}{13}{55}{Ps 22:6 Isa 49:7; 53:2,\allowbreak3 Mr 6:3 Lu 3:23; 4:22 Joh 1:45,\allowbreak46; 6:42}
\crossref{Matt}{13}{56}{}
\crossref{Matt}{13}{57}{Mt 11:6 Isa 8:14; 49:7; 53:3 Mr 6:3 Lu 2:34,\allowbreak35; 7:23 Joh 6:42,\allowbreak61}
\crossref{Matt}{13}{58}{Mr 6:5,\allowbreak6 Lu 4:25-\allowbreak29 Ro 11:20 Heb 3:12-\allowbreak19; 4:6-\allowbreak11}
\crossref{Matt}{14}{1}{Lu 3:1}
\crossref{Matt}{14}{2}{Mt 11:11; 16:14 Mr 8:28 Joh 10:41}
\crossref{Matt}{14}{3}{Mt 4:12 Mr 6:17 Lu 3:19,\allowbreak20 Joh 3:23,\allowbreak24}
\crossref{Matt}{14}{4}{Le 18:16; 20:21 De 25:5,\allowbreak6 2Sa 12:7 1Ki 21:19 2Ch 26:18,\allowbreak19}
\crossref{Matt}{14}{5}{Mr 6:19,\allowbreak20; 14:1,\allowbreak2 Ac 4:21; 5:26}
\crossref{Matt}{14}{6}{Ge 40:20 Es 1:2-\allowbreak9; 2:18 Da 5:1-\allowbreak4 Ho 1:5,\allowbreak6 Mr 6:21-\allowbreak23}
\crossref{Matt}{14}{7}{Es 5:3,\allowbreak6; 7:2}
\crossref{Matt}{14}{8}{2Ch 22:2,\allowbreak3 Mr 6:24}
\crossref{Matt}{14}{9}{14:1 Mr 6:14}
\crossref{Matt}{14}{10}{Mt 17:12; 21:35,\allowbreak36; 22:3-\allowbreak6; 23:34-\allowbreak36 2Ch 36:16 Jer 2:30}
\crossref{Matt}{14}{11}{Ge 49:7 Pr 27:4; 29:10 Jer 22:17 Eze 16:3,\allowbreak4; 19:2,\allowbreak3; 35:6 Re 16:6}
\crossref{Matt}{14}{12}{Mt 27:58-\allowbreak61 Ac 8:2}
\crossref{Matt}{14}{13}{14:1,\allowbreak2; 10:23; 12:15 Mr 6:30-\allowbreak33 Lu 9:10 etc.}
\crossref{Matt}{14}{14}{Mt 9:36; 15:32 etc.}
\crossref{Matt}{14}{15}{Mr 6:35,\allowbreak36 Lu 9:12}
\crossref{Matt}{14}{16}{2Ki 4:42-\allowbreak44 Job 31:16,\allowbreak17 Pr 11:24 Ec 11:2 Lu 3:11 Joh 13:29}
\crossref{Matt}{14}{17}{Mt 15:33,\allowbreak34 Nu 11:21-\allowbreak23 Ps 78:19,\allowbreak20 Mr 6:37,\allowbreak38; 8:4,\allowbreak5 Lu 9:13}
\crossref{Matt}{14}{18}{}
\crossref{Matt}{14}{19}{Mt 15:35 Mr 6:39; 8:6 Lu 9:14 Joh 6:10}
\crossref{Matt}{14}{20}{Mt 5:6; 15:33 Ex 16:8,\allowbreak12 Le 26:26 1Ki 17:12-\allowbreak16 2Ki 4:43,\allowbreak44}
\crossref{Matt}{14}{21}{Joh 6:10 Ac 4:4,\allowbreak34 2Co 9:8-\allowbreak11 Php 4:19}
\crossref{Matt}{14}{22}{Mr 6:45}
\crossref{Matt}{14}{23}{Mt 6:6; 26:36 Mr 6:46 Lu 6:12 Ac 6:4}
\crossref{Matt}{14}{24}{Mt 8:24 Isa 54:11 Mr 6:48 Joh 6:18}
\crossref{Matt}{14}{25}{}
\crossref{Matt}{14}{26}{Lu 24:5,\allowbreak45 Ac 12:15 Re 1:17}
\crossref{Matt}{14}{27}{Mt 9:2 Joh 16:33 Ac 23:11}
\crossref{Matt}{14}{28}{Mt 19:27; 26:33-\allowbreak35 Mr 14:31 Lu 22:31-\allowbreak34,\allowbreak49,\allowbreak50 Joh 6:68; 13:36-\allowbreak38}
\crossref{Matt}{14}{29}{Mt 17:20; 21:21 Mr 9:23; 11:22,\allowbreak23 Lu 17:6 Ac 3:16 Ro 4:19}
\crossref{Matt}{14}{30}{Mt 26:69-\allowbreak75 2Ki 6:15 Mr 14:38,\allowbreak66-\allowbreak72 Lu 22:54-\allowbreak61 Joh 18:25-\allowbreak27}
\crossref{Matt}{14}{31}{Ps 138:7 Isa 63:12 Mr 1:31,\allowbreak41; 5:41 Ac 4:30}
\crossref{Matt}{14}{32}{Ps 107:29,\allowbreak30 Mr 4:41; 6:51 Joh 6:21}
\crossref{Matt}{14}{33}{Mt 15:25; 28:9,\allowbreak17 Lu 24:52}
\crossref{Matt}{14}{34}{Mr 6:53-\allowbreak56}
\crossref{Matt}{14}{35}{Mt 4:24,\allowbreak25 Mr 1:28-\allowbreak34; 2:1 etc.}
\crossref{Matt}{14}{36}{Mt 9:20,\allowbreak21 Mr 3:10 Lu 6:19 Ac 19:11,\allowbreak12}
\crossref{Matt}{15}{1}{Mr 7:1 etc.}
\crossref{Matt}{15}{2}{Mr 7:2,\allowbreak5 Ge 1:14 Col 2:8,\allowbreak20-\allowbreak23 1Pe 1:18}
\crossref{Matt}{15}{3}{Mt 7:3-\allowbreak5 Mr 7:6-\allowbreak8,\allowbreak13 Col 2:8,\allowbreak23 Tit 1:14}
\crossref{Matt}{15}{4}{Mt 4:10; 5:17-\allowbreak19 Isa 8:20 Ro 3:31}
\crossref{Matt}{15}{5}{Mt 23:16-\allowbreak18 Am 7:15-\allowbreak17 Mr 7:10-\allowbreak13 Ac 4:19; 5:29}
\crossref{Matt}{15}{6}{1Ti 5:3,\allowbreak4,\allowbreak8,\allowbreak16}
\crossref{Matt}{15}{7}{Mt 7:5; 23:23-\allowbreak29}
\crossref{Matt}{15}{8}{Isa 29:13 Eze 33:31 Joh 1:47 1Pe 3:10}
\crossref{Matt}{15}{9}{Ex 20:7 Le 26:16,\allowbreak20 1Sa 25:21 Ps 39:6; 73:13 Ec 5:2-\allowbreak7}
\crossref{Matt}{15}{10}{1Ki 22:28 Mr 7:14,\allowbreak16 Lu 20:45-\allowbreak47}
\crossref{Matt}{15}{11}{Mr 7:15 Lu 11:38-\allowbreak41 Ac 10:14,\allowbreak15; 11:8,\allowbreak9 Ro 14:14,\allowbreak17,\allowbreak20}
\crossref{Matt}{15}{12}{Mt 17:27 1Ki 22:13,\allowbreak14 1Co 10:32,\allowbreak33 2Co 6:3 Ga 2:5 Jas 3:17}
\crossref{Matt}{15}{13}{Mt 13:40,\allowbreak41 Ps 92:13 Isa 60:21 Joh 15:2,\allowbreak6 1Co 3:12-\allowbreak15}
\crossref{Matt}{15}{14}{Ho 4:17 1Ti 6:5}
\crossref{Matt}{15}{15}{Mt 13:36 Mr 4:34; 7:17 Joh 16:29}
\crossref{Matt}{15}{16}{15:10; 13:51; 16:9,\allowbreak11 Isa 28:9,\allowbreak10 Mr 6:52; 7:18; 8:17,\allowbreak18; 9:32}
\crossref{Matt}{15}{17}{Mt 7:19,\allowbreak20 Lu 6:45 1Co 6:13 Col 2:21,\allowbreak22 Jas 3:6}
\crossref{Matt}{15}{18}{15:11; 12:34 1Sa 24:13 Ps 36:3 Pr 6:12; 10:32; 15:2,\allowbreak28}
\crossref{Matt}{15}{19}{Ge 6:5; 8:21 Pr 4:23; 6:14; 22:15; 24:9 Jer 17:9 Mr 7:21-\allowbreak23}
\crossref{Matt}{15}{20}{1Co 3:16,\allowbreak17; 6:9-\allowbreak11,\allowbreak18-\allowbreak20 Eph 5:3-\allowbreak6 Re 21:8,\allowbreak27}
\crossref{Matt}{15}{21}{Mr 7:24}
\crossref{Matt}{15}{22}{Mt 3:8,\allowbreak9 Ps 45:12 Eze 3:6 Mr 7:26}
\crossref{Matt}{15}{23}{Ge 42:7 De 8:2 Ps 28:1 La 3:8}
\crossref{Matt}{15}{24}{Mt 9:36; 10:5,\allowbreak6 Isa 53:6 Jer 50:6,\allowbreak7 Eze 34:5,\allowbreak6,\allowbreak16,\allowbreak23 Lu 15:4-\allowbreak6}
\crossref{Matt}{15}{25}{Mt 20:31 Ge 32:26 Ho 12:4 Lu 11:8-\allowbreak10; 18:1 etc.}
\crossref{Matt}{15}{26}{Mt 7:6 Mr 7:27,\allowbreak28 Ac 22:21,\allowbreak22 Ro 9:4 Ga 2:15 Eph 2:12 Php 3:2}
\crossref{Matt}{15}{27}{Mt 8:8 Ge 32:10 Job 40:4,\allowbreak5; 42:2-\allowbreak6 Ps 51:4,\allowbreak5 Eze 16:63 Da 9:18}
\crossref{Matt}{15}{28}{Job 13:15; 23:10 La 3:32}
\crossref{Matt}{15}{29}{Mr 7:31}
\crossref{Matt}{15}{30}{Mt 4:23,\allowbreak24; 11:4,\allowbreak5; 14:35,\allowbreak36 Ps 103:3 Isa 35:5,\allowbreak6 Mr 1:32-\allowbreak34}
\crossref{Matt}{15}{31}{Mt 9:33 Mr 7:37}
\crossref{Matt}{15}{32}{Mt 9:36; 14:14; 20:34 Mr 8:1,\allowbreak2; 9:22 Lu 7:13}
\crossref{Matt}{15}{33}{Nu 11:21,\allowbreak22 2Ki 4:42-\allowbreak44 Mr 6:37; 8:4,\allowbreak5 Joh 6:5-\allowbreak7}
\crossref{Matt}{15}{34}{Mt 16:9,\allowbreak10}
\crossref{Matt}{15}{35}{Mt 14:19 etc.}
\crossref{Matt}{15}{36}{Mt 26:26,\allowbreak27 1Sa 9:13 Lu 22:19; 24:30 Joh 6:11 Ac 27:35}
\crossref{Matt}{15}{37}{15:33; 14:20,\allowbreak21 Ps 107:9 Lu 1:53}
\crossref{Matt}{15}{38}{}
\crossref{Matt}{15}{39}{Mt 14:22 Mr 8:10}
\crossref{Matt}{16}{1}{Mt 5:20; 9:11; 12:14; 15:1; 22:15,\allowbreak34; 23:2; 27:62}
\crossref{Matt}{16}{2}{Lu 12:54-\allowbreak56}
\crossref{Matt}{16}{3}{Mt 7:5; 15:7; 22:18; 23:13 Lu 11:44; 13:15}
\crossref{Matt}{16}{4}{Mt 12:39,\allowbreak40 Mr 8:12,\allowbreak38 Ac 2:40}
\crossref{Matt}{16}{5}{Mt 15:39 Mr 8:13,\allowbreak14}
\crossref{Matt}{16}{6}{Lu 12:15}
\crossref{Matt}{16}{7}{Mr 8:16-\allowbreak18; 9:10 Lu 9:46}
\crossref{Matt}{16}{8}{Joh 2:24,\allowbreak25; 16:30 Heb 4:13 Re 2:23}
\crossref{Matt}{16}{9}{Mt 15:16,\allowbreak17 Mr 7:18 Lu 24:25-\allowbreak27 Re 3:19}
\crossref{Matt}{16}{10}{Mt 15:34,\allowbreak38 Mr 8:5-\allowbreak9,\allowbreak17-\allowbreak21}
\crossref{Matt}{16}{11}{Mr 4:40; 8:21 Lu 12:56 Joh 8:43}
\crossref{Matt}{16}{12}{Mt 15:4-\allowbreak9; 23:13 etc.}
\crossref{Matt}{16}{13}{Mt 15:21 Ac 10:38}
\crossref{Matt}{16}{14}{Mt 14:2 Mr 8:28}
\crossref{Matt}{16}{15}{Mt 13:11 Mr 8:29 Lu 9:20}
\crossref{Matt}{16}{16}{Mt 14:33; 26:63; 27:54 Ps 2:7 Mr 14:61 Joh 1:49; 6:69; 11:27; 20:31}
\crossref{Matt}{16}{17}{Mt 5:3-\allowbreak11; 13:16,\allowbreak17 Lu 10:23,\allowbreak24; 22:32 1Pe 1:3-\allowbreak5; 5:1}
\crossref{Matt}{16}{18}{Mt 10:2 Joh 1:42 Ga 2:9}
\crossref{Matt}{16}{19}{Ac 2:14 etc.}
\crossref{Matt}{16}{20}{Mt 8:4; 17:9 Mr 8:30; 9:9 Lu 9:21,\allowbreak36}
\crossref{Matt}{16}{21}{Mt 17:22,\allowbreak23; 20:17-\allowbreak19,\allowbreak28; 26:2 Mr 8:31; 9:31,\allowbreak32; 10:32-\allowbreak34}
\crossref{Matt}{16}{22}{16:16,\allowbreak17; 26:51-\allowbreak53 Mr 8:32 Joh 13:6-\allowbreak8}
\crossref{Matt}{16}{23}{Mt 4:10 Ge 3:1-\allowbreak6,\allowbreak17 Mr 8:33 Lu 4:8 2Co 11:14,\allowbreak15}
\crossref{Matt}{16}{24}{Mt 10:38 Mr 8:34; 10:21 Lu 9:23-\allowbreak27; 14:27 Ac 14:22 Col 1:24}
\crossref{Matt}{16}{25}{Mt 10:39 Es 4:14,\allowbreak16 Mr 8:35 Lu 17:33 Joh 12:25 Ac 20:23,\allowbreak24}
\crossref{Matt}{16}{26}{Mt 5:29 Job 2:4 Mr 8:36 Lu 9:25}
\crossref{Matt}{16}{27}{Mt 24:30; 25:31; 26:64 Mr 8:38; 14:62 Lu 9:26; 21:27; 22:69}
\crossref{Matt}{16}{28}{Mr 9:1 Lu 9:27}
\crossref{Matt}{17}{1}{Mt 26:37 Mr 5:37 Lu 8:51 2Co 13:1}
\crossref{Matt}{17}{2}{Lu 9:29 Ro 12:2 Php 2:6,\allowbreak7}
\crossref{Matt}{17}{3}{Mr 9:4 Lu 9:30,\allowbreak31}
\crossref{Matt}{17}{4}{Mr 9:5,\allowbreak6 Lu 9:33}
\crossref{Matt}{17}{5}{Ex 40:34,\allowbreak35 1Ki 8:10-\allowbreak12 Ps 18:10,\allowbreak11 Lu 9:34 Ac 1:9 Re 1:7}
\crossref{Matt}{17}{6}{Le 9:24 Jud 13:20,\allowbreak22 1Ch 21:16 Eze 3:23; 43:3 Da 8:17; 10:7-\allowbreak9}
\crossref{Matt}{17}{7}{Da 8:18; 9:21; 10:10,\allowbreak18 Re 1:17}
\crossref{Matt}{17}{8}{Mr 9:8 Lu 9:36 Ac 12:10,\allowbreak11}
\crossref{Matt}{17}{9}{Mt 16:20 Mr 8:30; 9:9,\allowbreak10 Lu 8:56; 9:21,\allowbreak22}
\crossref{Matt}{17}{10}{17:3,\allowbreak4; 11:14; 27:47-\allowbreak49 Mal 4:5,\allowbreak6 Mr 9:11 Joh 1:21,\allowbreak25}
\crossref{Matt}{17}{11}{Mal 4:6 Lu 1:16,\allowbreak17; 3:3-\allowbreak14 Ac 3:21}
\crossref{Matt}{17}{12}{Mt 11:9-\allowbreak15; 21:23-\allowbreak25,\allowbreak32 Mr 9:12,\allowbreak13; 11:30-\allowbreak32 Lu 7:33 Joh 1:11}
\crossref{Matt}{17}{13}{Mt 11:14}
\crossref{Matt}{17}{14}{Mr 9:14 etc.}
\crossref{Matt}{17}{15}{Mt 15:22 Mr 5:22,\allowbreak23; 9:22 Lu 9:38-\allowbreak42 Joh 4:46,\allowbreak47}
\crossref{Matt}{17}{16}{17:19,\allowbreak20 2Ki 4:29-\allowbreak31 Lu 9:40 Ac 3:16; 19:15,\allowbreak16}
\crossref{Matt}{17}{17}{Mt 6:30; 8:26; 13:58; 16:8 Mr 9:19; 16:14 Lu 9:41; 24:25 Joh 20:27}
\crossref{Matt}{17}{18}{Mt 12:22 Mr 1:34; 5:8; 9:25-\allowbreak27 Lu 4:35,\allowbreak36,\allowbreak41; 8:29; 9:42}
\crossref{Matt}{17}{19}{Mr 4:10; 9:28}
\crossref{Matt}{17}{20}{17:17; 14:30,\allowbreak31 Heb 3:19}
\crossref{Matt}{17}{21}{Mt 12:45}
\crossref{Matt}{17}{22}{Mt 16:21; 20:17,\allowbreak18 Mr 8:31; 9:30,\allowbreak31; 10:33,\allowbreak34 Lu 9:22,\allowbreak44}
\crossref{Matt}{17}{23}{Ps 22:15,\allowbreak22 etc.}
\crossref{Matt}{17}{24}{Mr 9:33}
\crossref{Matt}{17}{25}{Mt 3:15; 22:21 Ro 13:6,\allowbreak7}
\crossref{Matt}{17}{26}{}
\crossref{Matt}{17}{27}{Mt 15:12-\allowbreak14 Ro 14:21; 15:1-\allowbreak3 1Co 8:9,\allowbreak13; 9:19-\allowbreak22; 10:32,\allowbreak33}
\crossref{Matt}{18}{1}{Mr 9:33 etc.}
\crossref{Matt}{18}{2}{Mt 19:13,\allowbreak14 1Ki 3:7 Jer 1:7 Mr 9:36,\allowbreak37}
\crossref{Matt}{18}{3}{Mt 5:18; 6:2,\allowbreak5,\allowbreak16 Joh 1:51; 3:3}
\crossref{Matt}{18}{4}{Mt 23:11,\allowbreak12 Ps 131:1,\allowbreak2 Isa 57:15 Lu 14:11 1Pe 5:5 Jas 4:10}
\crossref{Matt}{18}{5}{Mt 10:40-\allowbreak42; 25:40,\allowbreak45 Mr 9:41 Lu 9:48; 17:1,\allowbreak2}
\crossref{Matt}{18}{6}{Ps 105:15 Zec 2:8 Mr 9:42 Lu 17:1,\allowbreak2 Ac 9:5 Ro 14:13-\allowbreak15,\allowbreak21}
\crossref{Matt}{18}{7}{Ge 13:7 1Sa 2:17,\allowbreak22-\allowbreak25 2Sa 12:14 Lu 17:1 Ro 2:23,\allowbreak24}
\crossref{Matt}{18}{8}{Mt 5:29,\allowbreak30; 14:3,\allowbreak4 De 13:6-\allowbreak8 Mr 9:43-\allowbreak48 Lu
14:26,\allowbreak27,\allowbreak33; 18:22,\allowbreak23}
\crossref{Matt}{18}{9}{Mt 19:17,\allowbreak23,\allowbreak24 Ac 14:22 Heb 4:11 Re 21:27}
\crossref{Matt}{18}{10}{18:6,\allowbreak14; 12:20 Ps 15:4 Zec 4:10 Lu 10:16 Ro 14:1-\allowbreak3,\allowbreak10,\allowbreak13-\allowbreak15,\allowbreak21; 15:1}
\crossref{Matt}{18}{11}{Mt 9:12,\allowbreak13; 10:6; 15:24 Lu 9:56; 15:24,\allowbreak32; 19:10 Joh 3:17; 10:10}
\crossref{Matt}{18}{12}{Mt 21:28; 22:42 1Co 10:15}
\crossref{Matt}{18}{13}{Ps 147:11 Isa 53:11; 62:5 Jer 32:37-\allowbreak41 Mic 7:18 Zep 3:17}
\crossref{Matt}{18}{14}{Lu 12:32 Joh 6:39,\allowbreak40; 10:27-\allowbreak30; 17:12 Ro 8:28-\allowbreak39 Eph 1:5-\allowbreak7}
\crossref{Matt}{18}{15}{18:35 Le 6:2-\allowbreak7 Lu 17:3,\allowbreak4 1Co 6:6-\allowbreak8; 8:12 2Co 7:12 Col 3:13}
\crossref{Matt}{18}{16}{Nu 35:30 De 17:6; 19:15 1Ki 21:13 Joh 8:17 2Co 13:1 1Ti 5:19}
\crossref{Matt}{18}{17}{Ac 6:1-\allowbreak3; 15:6,\allowbreak7 1Co 5:4,\allowbreak5; 6:1-\allowbreak4 2Co 2:6,\allowbreak7 3Jo 1:9,\allowbreak10}
\crossref{Matt}{18}{18}{Mt 16:19 Joh 20:23 Ac 15:23-\allowbreak31 1Co 5:4,\allowbreak5 2Co 2:10 Re 3:7,\allowbreak8}
\crossref{Matt}{18}{19}{Mt 5:24; 21:22 Mr 11:24 Joh 15:7,\allowbreak16 Ac 1:14; 2:1,\allowbreak2; 4:24-\allowbreak31; 6:4}
\crossref{Matt}{18}{20}{Ge 49:10 Joh 20:19,\allowbreak26 1Co 5:4 1Th 1:1 Phm 1:2}
\crossref{Matt}{18}{21}{18:15 Lu 17:3,\allowbreak4}
\crossref{Matt}{18}{22}{Mt 6:11,\allowbreak12,\allowbreak14,\allowbreak15 Isa 55:7 Mic 7:19 Mr 11:25,\allowbreak26 Ro 12:21}
\crossref{Matt}{18}{23}{Mt 3:2; 13:24,\allowbreak31,\allowbreak33,\allowbreak44,\allowbreak45,\allowbreak47,\allowbreak52; 25:1,\allowbreak14}
\crossref{Matt}{18}{24}{Lu 7:41,\allowbreak42; 13:4}
\crossref{Matt}{18}{25}{Le 25:39 2Ki 4:1 Ne 5:5,\allowbreak8 Isa 50:1}
\crossref{Matt}{18}{26}{18:29 Lu 7:43 Ro 10:3}
\crossref{Matt}{18}{27}{Jud 10:16 Ne 9:17 Ps 78:38; 86:5,\allowbreak15; 145:8 Ho 11:8}
\crossref{Matt}{18}{28}{}
\crossref{Matt}{18}{29}{18:26; 6:12 Phm 1:18,\allowbreak19}
\crossref{Matt}{18}{30}{1Ki 21:27-\allowbreak29; 22:27}
\crossref{Matt}{18}{31}{Ps 119:136,\allowbreak158 Jer 9:1 Mr 3:5 Lu 19:41 Ro 9:1-\allowbreak3; 12:15}
\crossref{Matt}{18}{32}{Mt 25:26 Lu 19:22 Ro 3:19}
\crossref{Matt}{18}{33}{Mt 5:44,\allowbreak45 Lu 6:35,\allowbreak36 Eph 4:32; 5:1,\allowbreak2 Col 3:13}
\crossref{Matt}{18}{34}{Mt 5:25,\allowbreak26 Lu 12:58,\allowbreak59 2Th 1:8,\allowbreak9 Re 14:10,\allowbreak11}
\crossref{Matt}{18}{35}{Mt 6:12,\allowbreak14,\allowbreak15; 7:1,\allowbreak2 Pr 21:13 Mr 11:26 Lu 6:37,\allowbreak38 Jas 2:13}
\crossref{Matt}{19}{1}{Mr 10:1 Joh 10:40}
\crossref{Matt}{19}{2}{Mt 4:23-\allowbreak25; 9:35,\allowbreak36; 12:15; 14:35,\allowbreak36; 15:30,\allowbreak31 Mr 6:55,\allowbreak56}
\crossref{Matt}{19}{3}{Mt 16:1; 22:16-\allowbreak18,\allowbreak35 Mr 10:2; 12:13,\allowbreak15 Lu 11:53,\allowbreak54 Joh 8:6}
\crossref{Matt}{19}{4}{Mt 12:3; 21:6,\allowbreak42; 22:31 Mr 2:25; 12:10,\allowbreak26 Lu 6:3; 10:26}
\crossref{Matt}{19}{5}{Ge 2:21-\allowbreak24 Ps 45:10 Mr 10:5-\allowbreak9 Eph 5:31}
\crossref{Matt}{19}{6}{Pr 2:17 Mal 2:14 Mr 10:9 Ro 7:2 1Co 7:10-\allowbreak14 Eph 5:28 Heb 13:4}
\crossref{Matt}{19}{7}{Mt 5:31 De 24:1-\allowbreak4 Isa 50:1 Jer 3:8 Mr 10:4}
\crossref{Matt}{19}{8}{Ps 95:8 Zec 7:12 Mal 2:13,\allowbreak14 Mr 10:5}
\crossref{Matt}{19}{9}{Mt 5:32 Mr 10:11,\allowbreak12 Lu 16:18 1Co 7:10-\allowbreak13,\allowbreak39}
\crossref{Matt}{19}{10}{Ge 2:18 Pr 5:15-\allowbreak19; 18:22; 19:13,\allowbreak14; 21:9,\allowbreak19 1Co 7:1,\allowbreak2,\allowbreak8,\allowbreak26-\allowbreak28}
\crossref{Matt}{19}{11}{1Co 7:2,\allowbreak7,\allowbreak9,\allowbreak17,\allowbreak35}
\crossref{Matt}{19}{12}{Isa 39:7; 56:3,\allowbreak4}
\crossref{Matt}{19}{13}{Mt 18:2-\allowbreak5 Ge 48:1,\allowbreak9-\allowbreak20 1Sa 1:24 Ps 115:14,\allowbreak15 Jer 32:39}
\crossref{Matt}{19}{14}{Ge 17:7,\allowbreak8,\allowbreak24-\allowbreak26; 21:4 Jud 13:7 1Sa 1:11,\allowbreak22,\allowbreak24; 2:18 Mr 10:14}
\crossref{Matt}{19}{15}{Isa 40:11 Mr 10:16 1Co 7:14 2Ti 3:15}
\crossref{Matt}{19}{16}{Mr 10:17 Lu 18:18}
\crossref{Matt}{19}{17}{1Sa 2:2 Ps 52:1; 145:7-\allowbreak9 Jas 1:17 1Jo 4:8-\allowbreak10,\allowbreak16}
\crossref{Matt}{19}{18}{Ga 3:10 Jas 2:10,\allowbreak11}
\crossref{Matt}{19}{19}{Mt 15:4-\allowbreak6 Le 19:3 Pr 30:17 Eph 6:1,\allowbreak2}
\crossref{Matt}{19}{20}{Mr 10:20 Lu 15:7,\allowbreak29; 18:11,\allowbreak12,\allowbreak21 Joh 8:7 Ro 3:19-\allowbreak23; 7:9}
\crossref{Matt}{19}{21}{Mt 5:19,\allowbreak20,\allowbreak48 Ge 6:9; 17:1 Job 1:1 Ps 37:37 Lu 6:40}
\crossref{Matt}{19}{22}{Mt 13:22; 14:9 Jud 18:23,\allowbreak24 Da 6:14-\allowbreak17 Mr 6:26; 10:22 Lu 18:23}
\crossref{Matt}{19}{23}{Mt 13:22 De 6:10-\allowbreak12; 8:10-\allowbreak18 Job 31:24,\allowbreak25 Ps 49:6,\allowbreak7,\allowbreak16-\allowbreak19}
\crossref{Matt}{19}{24}{}
\crossref{Matt}{19}{25}{Mt 24:22 Mr 13:20 Lu 13:23,\allowbreak24 Ro 10:13; 11:5-\allowbreak7}
\crossref{Matt}{19}{26}{Ge 18:14 Nu 11:23 Job 42:2 Ps 3:8; 62:11 Jer 32:27 Zec 8:6}
\crossref{Matt}{19}{27}{Mt 4:20-\allowbreak22; 9:9 De 33:9 Mr 1:17-\allowbreak20; 2:14; 10:28 Lu 5:11,\allowbreak27,\allowbreak28}
\crossref{Matt}{19}{28}{Isa 65:17; 66:22 Ac 3:21 2Pe 3:13 Re 21:5}
\crossref{Matt}{19}{29}{Mt 16:25 Mr 10:29,\allowbreak30 Lu 18:29,\allowbreak30 1Co 2:9}
\crossref{Matt}{19}{30}{Mt 8:11,\allowbreak12; 20:16; 21:31,\allowbreak32 Mr 10:31 Lu 7:29,\allowbreak30; 13:30; 18:13,\allowbreak14}
\crossref{Matt}{20}{1}{Mt 9:37,\allowbreak38; 21:33-\allowbreak43 So 8:11,\allowbreak12 Isa 5:1,\allowbreak2 Joh 15:1}
\crossref{Matt}{20}{2}{20:13 Ex 19:5,\allowbreak6 De 5:27-\allowbreak30}
\crossref{Matt}{20}{3}{Mr 15:25 Ac 2:15}
\crossref{Matt}{20}{4}{Mt 9:9; 21:23-\allowbreak31 Lu 19:7-\allowbreak10 Ro 6:16-\allowbreak22 1Co 6:11 1Ti 1:12,\allowbreak13}
\crossref{Matt}{20}{5}{Mt 27:45 Mr 15:33,\allowbreak34 Lu 23:44-\allowbreak46 Joh 1:39; 4:6; 11:9 Ac 3:1; 10:3,\allowbreak9}
\crossref{Matt}{20}{6}{Ec 9:10 Lu 23:40-\allowbreak43 Joh 9:4}
\crossref{Matt}{20}{7}{Ac 4:16; 17:30,\allowbreak31 Ro 10:14-\allowbreak17; 16:25 Eph 2:11,\allowbreak12; 3:5,\allowbreak6 Col 1:26}
\crossref{Matt}{20}{8}{Mt 13:39,\allowbreak40; 25:19,\allowbreak31 Ro 2:6-\allowbreak10 2Co 5:10 Heb 9:28 Re 20:11,\allowbreak12}
\crossref{Matt}{20}{9}{20:2,\allowbreak6,\allowbreak7 Lu 23:40-\allowbreak43 Ro 4:3-\allowbreak6; 5:20,\allowbreak21 Eph 1:6-\allowbreak8; 2:8-\allowbreak10}
\crossref{Matt}{20}{10}{}
\crossref{Matt}{20}{11}{Lu 5:30; 15:2,\allowbreak28-\allowbreak30; 19:7 Ac 11:2,\allowbreak3; 13:45; 22:21,\allowbreak22 1Th 2:16}
\crossref{Matt}{20}{12}{Lu 14:10,\allowbreak11 Ro 3:22-\allowbreak24,\allowbreak30 Eph 3:6}
\crossref{Matt}{20}{13}{Mt 22:12; 26:50}
\crossref{Matt}{20}{14}{Mt 6:2,\allowbreak6,\allowbreak16 2Ki 10:16,\allowbreak30,\allowbreak31 Eze 29:18-\allowbreak20 Lu 15:31; 16:25}
\crossref{Matt}{20}{15}{Mt 11:25 Ex 33:19 De 7:6-\allowbreak8 1Ch 28:4,\allowbreak5 Jer 27:5-\allowbreak7 Joh 17:2}
\crossref{Matt}{20}{16}{Mt 8:11,\allowbreak12; 19:30; 21:31 Mr 10:31 Lu 7:47; 13:28-\allowbreak30; 15:7; 17:17,\allowbreak18}
\crossref{Matt}{20}{17}{Mr 10:32-\allowbreak34 Lu 18:31-\allowbreak34 Joh 12:12}
\crossref{Matt}{20}{18}{Mt 16:21; 17:22,\allowbreak23; 26:2 Ps 2:1-\allowbreak3; 22:1 etc.}
\crossref{Matt}{20}{19}{Mt 27:2 etc.}
\crossref{Matt}{20}{20}{Mr 10:35}
\crossref{Matt}{20}{21}{20:32 1Ki 3:5 Es 5:3 Mr 6:22; 10:36,\allowbreak51 Lu 18:41 Joh 15:7}
\crossref{Matt}{20}{22}{Mr 10:38 Ro 8:26 Jas 4:3}
\crossref{Matt}{20}{23}{Ac 12:2 Ro 8:17 2Co 1:7 Col 1:24 2Ti 2:11,\allowbreak12 Re 1:9}
\crossref{Matt}{20}{24}{Pr 13:10 Mr 10:41 Lu 22:23-\allowbreak25 1Co 13:4 Php 2:3 Jas 3:14-\allowbreak18}
\crossref{Matt}{20}{25}{Mt 11:29; 18:3,\allowbreak4 Joh 13:12-\allowbreak17}
\crossref{Matt}{20}{26}{Mt 23:8-\allowbreak12 Mr 9:35; 10:43,\allowbreak45 Lu 14:7-\allowbreak11; 18:14 Joh 18:36}
\crossref{Matt}{20}{27}{Mt 18:4 Mr 9:33-\allowbreak35 Lu 22:26 Ac 20:34,\allowbreak35 Ro 1:14 1Co 9:19-\allowbreak23}
\crossref{Matt}{20}{28}{Lu 22:27 Joh 13:4-\allowbreak17 Php 2:4-\allowbreak8 Heb 5:8}
\crossref{Matt}{20}{29}{Mr 10:46-\allowbreak52 Lu 18:35-\allowbreak43}
\crossref{Matt}{20}{30}{Mt 9:27-\allowbreak31; 12:22; 21:14 Ps 146:8 Isa 29:18; 35:5,\allowbreak6; 42:16,\allowbreak18}
\crossref{Matt}{20}{31}{Mt 15:23; 19:13}
\crossref{Matt}{20}{32}{20:21 Eze 36:37 Ac 10:29 Php 4:6}
\crossref{Matt}{20}{33}{Ps 119:18 Eph 1:17-\allowbreak19}
\crossref{Matt}{20}{34}{Mt 9:36; 14:14; 15:32 Ps 145:8 Lu 7:13 Joh 11:33-\allowbreak35 Heb 2:17}
\crossref{Matt}{21}{1}{Mr 11:1 Lu 19:28}
\crossref{Matt}{21}{2}{Mt 26:18 Mr 11:2,\allowbreak3; 14:13-\allowbreak16 Lu 19:30-\allowbreak32 Joh 2:5-\allowbreak8}
\crossref{Matt}{21}{3}{1Ch 29:14-\allowbreak16 Ps 24:1; 50:10,\allowbreak11 Hag 2:8,\allowbreak9 Joh 3:35; 17:2 Ac 17:25}
\crossref{Matt}{21}{4}{Mt 1:22; 26:56 Joh 19:36,\allowbreak37}
\crossref{Matt}{21}{5}{Ps 9:14 Isa 12:6; 40:9; 62:11 Zep 3:14,\allowbreak15 Mr 11:4 etc.}
\crossref{Matt}{21}{6}{Ge 6:22; 12:4 Ex 39:43; 40:16 1Sa 15:11 Joh 15:14}
\crossref{Matt}{21}{7}{Mr 11:4-\allowbreak8 Lu 19:32-\allowbreak35}
\crossref{Matt}{21}{8}{Le 23:40 Joh 12:13}
\crossref{Matt}{21}{9}{21:15 Ps 118:24-\allowbreak26 Mr 11:9,\allowbreak10}
\crossref{Matt}{21}{10}{Mt 2:3 Ru 1:19 1Sa 16:4 Joh 12:16-\allowbreak19}
\crossref{Matt}{21}{11}{Mt 16:13,\allowbreak14 De 18:15-\allowbreak19 Lu 7:16 Joh 7:40; 9:17 Ac 3:22,\allowbreak23; 7:37}
\crossref{Matt}{21}{12}{Mal 3:1,\allowbreak2 Mr 11:11}
\crossref{Matt}{21}{13}{Mt 2:5 Joh 15:25}
\crossref{Matt}{21}{14}{Mt 9:35; 11:4,\allowbreak5 Isa 35:5 Ac 3:1-\allowbreak9; 10:38}
\crossref{Matt}{21}{15}{21:23; 26:3,\allowbreak59; 27:1,\allowbreak20 Isa 26:11 Mr 11:18 Lu 19:39,\allowbreak40; 20:1; 22:2,\allowbreak66}
\crossref{Matt}{21}{16}{Lu 19:39,\allowbreak40 Joh 11:47,\allowbreak48 Ac 4:16-\allowbreak18}
\crossref{Matt}{21}{17}{Mt 16:4 Jer 6:8 Ho 9:12 Mr 3:7 Lu 8:37,\allowbreak38}
\crossref{Matt}{21}{18}{Mr 11:12,\allowbreak13}
\crossref{Matt}{21}{19}{Isa 5:4,\allowbreak5 Lu 3:9; 13:6-\allowbreak9 Joh 15:2,\allowbreak6 2Ti 3:5 Tit 1:16}
\crossref{Matt}{21}{20}{Isa 40:6-\allowbreak8 Mr 11:20,\allowbreak21 Jas 1:10,\allowbreak11}
\crossref{Matt}{21}{21}{Mt 17:20 Mr 11:22,\allowbreak23 Lu 17:6,\allowbreak7 Ro 4:19,\allowbreak20 1Co 13:2 Jas 1:6}
\crossref{Matt}{21}{22}{Mt 7:7,\allowbreak11; 18:19 Mr 11:24 Lu 11:8-\allowbreak10 Joh 14:13; 15:7; 16:24}
\crossref{Matt}{21}{23}{Mr 11:27,\allowbreak28 Lu 19:47,\allowbreak48; 20:1,\allowbreak2}
\crossref{Matt}{21}{24}{Mt 10:16 Pr 26:4,\allowbreak5 Lu 6:9 Col 4:6}
\crossref{Matt}{21}{25}{Mt 3:1 etc.}
\crossref{Matt}{21}{26}{21:46; 14:5 Isa 57:11 Mr 11:32; 12:12 Lu 20:6,\allowbreak19; 22:2 Joh 9:22 Ac 5:26}
\crossref{Matt}{21}{27}{Mt 15:14; 16:3; 23:16 etc.}
\crossref{Matt}{21}{28}{Mt 17:25; 22:17 Lu 13:4 1Co 10:15}
\crossref{Matt}{21}{29}{21:31 Jer 44:16 Eph 4:17-\allowbreak19}
\crossref{Matt}{21}{30}{Mt 23:3 Eze 33:31 Ro 2:17-\allowbreak25 Tit 1:16}
\crossref{Matt}{21}{31}{Mt 7:21; 12:50 Eze 33:11 Lu 15:10 Ac 17:30 2Pe 3:9}
\crossref{Matt}{21}{32}{Mt 3:1-\allowbreak8 Isa 35:8 Jer 6:16 Lu 3:8-\allowbreak13 2Pe 2:21}
\crossref{Matt}{21}{33}{Ps 80:8-\allowbreak16 So 8:11,\allowbreak12 Isa 5:1-\allowbreak4 Jer 2:21 Mr 12:1 Lu 20:9 etc.}
\crossref{Matt}{21}{34}{2Ki 17:13,\allowbreak14 etc.}
\crossref{Matt}{21}{35}{Mt 5:12; 23:31-\allowbreak37 1Ki 18:4,\allowbreak13; 19:2,\allowbreak10; 22:24 2Ch 16:10; 24:21,\allowbreak22}
\crossref{Matt}{21}{36}{}
\crossref{Matt}{21}{37}{Mt 3:17 Mr 12:6 Lu 20:13 Joh 1:18,\allowbreak34; 3:16,\allowbreak35,\allowbreak36 Heb 1:1,\allowbreak2}
\crossref{Matt}{21}{38}{Mt 2:13-\allowbreak16; 26:3,\allowbreak4; 27:1,\allowbreak2 Ge 37:18-\allowbreak20 Ps 2:2-\allowbreak8 Mr 12:7,\allowbreak8}
\crossref{Matt}{21}{39}{Mt 26:50,\allowbreak57 Mr 14:46-\allowbreak53 Lu 22:52-\allowbreak54 Joh 18:12,\allowbreak24 Ac 2:23; 4:25-\allowbreak27}
\crossref{Matt}{21}{40}{Mr 12:9 Lu 20:15,\allowbreak16 Heb 10:29}
\crossref{Matt}{21}{41}{Mt 3:12; 22:6,\allowbreak7; 23:35-\allowbreak38; 24:21,\allowbreak22 Le 26:14 etc.}
\crossref{Matt}{21}{42}{21:16}
\crossref{Matt}{21}{43}{21:41; 8:11,\allowbreak12; 12:28 Isa 28:2 Lu 17:20,\allowbreak21 Joh 3:3,\allowbreak5}
\crossref{Matt}{21}{44}{Ps 2:12 Isa 8:14,\allowbreak15; 60:12 Zec 12:3 Lu 20:18 Ro 9:33 2Co 4:3,\allowbreak4}
\crossref{Matt}{21}{45}{Mt 12:12 Lu 11:45; 20:19}
\crossref{Matt}{21}{46}{2Sa 12:7-\allowbreak13 Pr 9:7-\allowbreak9; 15:12 Isa 29:1 Joh 7:7}
\crossref{Matt}{22}{1}{Mt 9:15-\allowbreak17; 12:43-\allowbreak45; 13:3-\allowbreak11; 20:1-\allowbreak16; 21:28-\allowbreak46 Mr 4:33,\allowbreak34}
\crossref{Matt}{22}{2}{Mt 13:24,\allowbreak31-\allowbreak33,\allowbreak44-\allowbreak47; 25:1,\allowbreak14}
\crossref{Matt}{22}{3}{Mt 3:2; 10:6,\allowbreak7 Ps 68:11 Pr 9:1-\allowbreak3 Isa 55:1,\allowbreak2 Jer 25:4; 35:15}
\crossref{Matt}{22}{4}{Lu 10:1-\allowbreak16; 24:46,\allowbreak47 Ac 1:8; 11:19,\allowbreak20; 13:46; 28:17 etc.}
\crossref{Matt}{22}{5}{Ge 19:14; 25:34 Ps 106:24,\allowbreak25 Pr 1:7,\allowbreak24,\allowbreak25 Ac 2:13; 24:25 Ro 2:4}
\crossref{Matt}{22}{6}{Mt 5:10-\allowbreak12; 10:12-\allowbreak18,\allowbreak22-\allowbreak25; 21:35-\allowbreak39; 23:34-\allowbreak37 Joh 15:19,\allowbreak20; 16:2,\allowbreak3}
\crossref{Matt}{22}{7}{Mt 21:40,\allowbreak41 Da 9:26 Zec 14:1,\allowbreak2 Lu 19:27,\allowbreak42-\allowbreak44; 21:21,\allowbreak24}
\crossref{Matt}{22}{8}{22:4}
\crossref{Matt}{22}{9}{Pr 1:20-\allowbreak23; 8:1-\allowbreak5; 9:4-\allowbreak6 Isa 55:1-\allowbreak3,\allowbreak6,\allowbreak7 Mr 16:15,\allowbreak16}
\crossref{Matt}{22}{10}{22:11,\allowbreak12; 13:38,\allowbreak47,\allowbreak48; 25:1,\allowbreak2 1Co 6:9-\allowbreak11 2Co 12:21}
\crossref{Matt}{22}{11}{Mt 3:12; 13:30; 25:31,\allowbreak32 Zep 1:12 1Co 4:5 Heb 4:12,\allowbreak13 Re 2:23}
\crossref{Matt}{22}{12}{Mt 20:13; 26:50}
\crossref{Matt}{22}{13}{Mt 12:29; 13:30 Isa 52:1 Da 3:20 Joh 21:18 Ac 21:11 Re 21:27}
\crossref{Matt}{22}{14}{Mt 7:13,\allowbreak14; 20:16 Lu 13:23,\allowbreak24}
\crossref{Matt}{22}{15}{Ps 2:2 Mr 12:13-\allowbreak17 Lu 20:20-\allowbreak26}
\crossref{Matt}{22}{16}{Mt 16:11,\allowbreak12 Mr 3:6; 8:15}
\crossref{Matt}{22}{17}{Jer 42:2,\allowbreak3,\allowbreak20 Ac 28:22}
\crossref{Matt}{22}{18}{Mr 2:8 Lu 5:22; 9:47; 20:23 Joh 2:25 Re 2:23}
\crossref{Matt}{22}{19}{}
\crossref{Matt}{22}{20}{Lu 20:24}
\crossref{Matt}{22}{21}{Mt 17:25-\allowbreak27 Pr 24:21 Lu 23:2 Ro 13:7}
\crossref{Matt}{22}{22}{22:33,\allowbreak46; 10:16 Pr 26:4,\allowbreak5 Lu 20:25,\allowbreak26; 21:15 Ac 6:10 Col 4:6}
\crossref{Matt}{22}{23}{Mr 12:18 etc.}
\crossref{Matt}{22}{24}{22:16,\allowbreak36; 7:21 Lu 6:46}
\crossref{Matt}{22}{25}{Mr 12:19-\allowbreak23 Lu 20:29-\allowbreak33 Heb 9:27}
\crossref{Matt}{22}{26}{22:26}
\crossref{Matt}{22}{27}{22:27}
\crossref{Matt}{22}{28}{}
\crossref{Matt}{22}{29}{Job 19:25-\allowbreak27 Ps 16:9-\allowbreak11; 17:15; 49:14,\allowbreak15; 73:25,\allowbreak26 Isa 25:8; 26:19}
\crossref{Matt}{22}{30}{Mr 12:24,\allowbreak25 Lu 20:34-\allowbreak36 Joh 5:28,\allowbreak29 1Co 7:29-\allowbreak31 1Jo 3:1,\allowbreak2}
\crossref{Matt}{22}{31}{Mt 9:13; 12:3,\allowbreak7; 21:16,\allowbreak42}
\crossref{Matt}{22}{32}{Ex 3:6,\allowbreak15,\allowbreak16 Ac 7:32 Heb 11:16}
\crossref{Matt}{22}{33}{22:22; 7:28,\allowbreak29 Mr 6:2; 12:17 Lu 2:47; 4:22; 20:39,\allowbreak40 Joh 7:46}
\crossref{Matt}{22}{34}{Mr 12:28}
\crossref{Matt}{22}{35}{Lu 7:30; 10:25 etc.}
\crossref{Matt}{22}{36}{Mt 5:19,\allowbreak20; 15:6; 23:23,\allowbreak24 Ho 8:12 Mr 12:28-\allowbreak33 Lu 11:42}
\crossref{Matt}{22}{37}{De 6:5; 10:12; 30:6 Mr 12:29,\allowbreak30,\allowbreak33 Lu 10:27 Ro 8:7 Heb 10:16,\allowbreak17}
\crossref{Matt}{22}{38}{}
\crossref{Matt}{22}{39}{Mt 19:19 Le 19:18 Mr 12:31 Lu 10:27,\allowbreak28 Ro 13:9,\allowbreak10 Ga 5:14 Jas 2:8}
\crossref{Matt}{22}{40}{Mt 7:12 Joh 1:17 Ro 3:19-\allowbreak21; 13:9 1Ti 1:5 1Jo 4:7-\allowbreak11,\allowbreak19-\allowbreak21 Jas 2:8}
\crossref{Matt}{22}{41}{22:15,\allowbreak34 Mr 12:35 etc.}
\crossref{Matt}{22}{42}{Mt 2:4-\allowbreak6; 14:33; 16:13-\allowbreak17 Joh 1:49; 6:68,\allowbreak69; 20:28 Php 2:9-\allowbreak11; 3:7-\allowbreak10}
\crossref{Matt}{22}{43}{2Sa 23:2 Mr 12:36 Lu 2:26,\allowbreak27 Ac 1:16; 2:30,\allowbreak31 Heb 3:7 2Pe 1:21}
\crossref{Matt}{22}{44}{Joh 20:28 1Co 1:2 Php 3:8}
\crossref{Matt}{22}{45}{Joh 8:58 Ro 1:3,\allowbreak4; 9:5 Php 2:6-\allowbreak8 1Ti 3:16 Heb 2:14 Re 22:16}
\crossref{Matt}{22}{46}{Mt 21:27 Job 32:15,\allowbreak16 Isa 50:2-\allowbreak9 Lu 13:17; 14:6 Joh 8:7-\allowbreak9 Ac 4:14}
\crossref{Matt}{23}{1}{Mt 15:10 etc.}
\crossref{Matt}{23}{2}{Ne 8:4-\allowbreak8 Mal 2:7 Mr 12:38 Lu 20:46}
\crossref{Matt}{23}{3}{Mt 15:2-\allowbreak9 Ex 18:19,\allowbreak20,\allowbreak23 De 4:5; 5:27; 17:9-\allowbreak12 2Ch 30:12 Ac 5:29}
\crossref{Matt}{23}{4}{23:23; 11:28-\allowbreak30 Lu 11:46 Ac 15:10,\allowbreak28 Ga 6:13 Re 2:24}
\crossref{Matt}{23}{5}{Mt 6:1-\allowbreak16 2Ki 10:16 Lu 16:15; 20:47; 21:1 Joh 5:44; 7:18; 12:43}
\crossref{Matt}{23}{6}{Mt 20:21 Pr 25:6,\allowbreak7 Mr 12:38,\allowbreak39 Lu 11:43 etc.}
\crossref{Matt}{23}{7}{Joh 1:38,\allowbreak49; 3:2,\allowbreak26; 6:25; 20:16}
\crossref{Matt}{23}{8}{23:10 2Co 1:24; 4:5 Jas 3:1 1Pe 5:3}
\crossref{Matt}{23}{9}{2Ki 2:12; 6:21; 13:14 Job 32:21,\allowbreak22 Ac 22:1 1Co 4:15 1Ti 5:1,\allowbreak2}
\crossref{Matt}{23}{10}{23:10}
\crossref{Matt}{23}{11}{Mt 20:26,\allowbreak27 Mr 10:43,\allowbreak44 Lu 22:26,\allowbreak27 Joh 13:14,\allowbreak15 1Co 9:19}
\crossref{Matt}{23}{12}{Mt 5:3; 18:4 Job 22:29 Ps 138:6 Pr 15:33; 16:18,\allowbreak19; 29:23 Isa 57:15}
\crossref{Matt}{23}{13}{23:14,\allowbreak15,\allowbreak27,\allowbreak29 Isa 9:14,\allowbreak15; 33:14 Zec 11:17 Lu 11:43,\allowbreak44}
\crossref{Matt}{23}{14}{}
\crossref{Matt}{23}{15}{Ga 4:17; 6:12}
\crossref{Matt}{23}{16}{23:17,\allowbreak19,\allowbreak24,\allowbreak26; 15:14 Isa 56:10,\allowbreak11 Joh 9:39-\allowbreak41}
\crossref{Matt}{23}{17}{Ps 94:8}
\crossref{Matt}{23}{18}{23:15}
\crossref{Matt}{23}{19}{Ex 29:37; 30:29}
\crossref{Matt}{23}{20}{}
\crossref{Matt}{23}{21}{1Ki 8:13,\allowbreak27 2Ch 6:2; 7:2 Ps 26:8; 132:13,\allowbreak14 Eph 2:22 Col 2:9}
\crossref{Matt}{23}{22}{Mt 5:34 Ps 11:4 Isa 66:1 Ac 7:49 Re 4:2,\allowbreak3}
\crossref{Matt}{23}{23}{Lu 11:42}
\crossref{Matt}{23}{24}{Mt 7:4; 15:2-\allowbreak6; 19:24; 27:6-\allowbreak8 Lu 6:7-\allowbreak10 Joh 18:28,\allowbreak40}
\crossref{Matt}{23}{25}{Mt 15:19,\allowbreak20 Mr 7:4 etc.}
\crossref{Matt}{23}{26}{Mt 12:33 Isa 55:7 Jer 4:14; 13:27 Eze 18:31 Lu 6:45 2Co 7:1}
\crossref{Matt}{23}{27}{Isa 58:1,\allowbreak2 Lu 11:44 Ac 23:3}
\crossref{Matt}{23}{28}{23:5 1Sa 16:7 Ps 51:6 Jer 17:9,\allowbreak10 Lu 16:15 Heb 4:12,\allowbreak13}
\crossref{Matt}{23}{29}{Lu 11:47,\allowbreak48 Ac 2:29}
\crossref{Matt}{23}{30}{23:34,\allowbreak35; 21:35,\allowbreak36 2Ch 36:15 Jer 2:30}
\crossref{Matt}{23}{31}{Jos 24:22 Job 15:5,\allowbreak6 Ps 64:8 Lu 19:22}
\crossref{Matt}{23}{32}{Ge 15:16 Nu 32:14 Zec 5:6-\allowbreak11}
\crossref{Matt}{23}{33}{Mt 3:7; 12:34; 21:34,\allowbreak35 Ge 3:15 Ps 58:3-\allowbreak5 Isa 57:3,\allowbreak4 Lu 3:7}
\crossref{Matt}{23}{34}{Mt 10:16; 28:19,\allowbreak20 Lu 11:49; 24:47 Joh 20:21 Ac 1:8 1Co 12:3-\allowbreak11}
\crossref{Matt}{23}{35}{Ge 9:5,\allowbreak6 Nu 35:33 De 21:7,\allowbreak8 2Ki 21:16; 24:4 Isa 26:21}
\crossref{Matt}{23}{36}{Mt 24:34 Eze 12:21-\allowbreak28 Mr 13:30,\allowbreak31 Lu 21:32,\allowbreak33}
\crossref{Matt}{23}{37}{Jer 4:14; 6:8 Lu 13:34 Re 11:8}
\crossref{Matt}{23}{38}{Mt 24:2 2Ch 7:20,\allowbreak21 Ps 69:24 Isa 64:10-\allowbreak12 Jer 7:9-\allowbreak14 Da 9:26}
\crossref{Matt}{23}{39}{Ho 3:4 Lu 2:26-\allowbreak30; 10:22,\allowbreak23; 17:22 Joh 8:21,\allowbreak24,\allowbreak56; 14:9,\allowbreak19}
\crossref{Matt}{24}{1}{Mt 23:39 Jer 6:8 Eze 8:6; 10:17-\allowbreak19; 11:22,\allowbreak23 Ho 9:12}
\crossref{Matt}{24}{2}{}
\crossref{Matt}{24}{3}{Mt 21:1 Mr 13:3,\allowbreak4}
\crossref{Matt}{24}{4}{Jer 29:8 Mr 13:5,\allowbreak6,\allowbreak22 Lu 21:8 2Co 11:13-\allowbreak15 Eph 4:14; 5:6}
\crossref{Matt}{24}{5}{24:11,\allowbreak24 Jer 14:14; 23:21,\allowbreak25 Joh 5:43 Ac 5:36,\allowbreak37; 8:9,\allowbreak10 Re 13:8}
\crossref{Matt}{24}{6}{Jer 4:19-\allowbreak22; 6:22-\allowbreak24; 8:15,\allowbreak16; 47:6 Eze 7:24-\allowbreak26; 14:17-\allowbreak21}
\crossref{Matt}{24}{7}{2Ch 15:6 Isa 9:19-\allowbreak21; 19:2 Eze 21:27 Hag 2:21,\allowbreak22 Zec 14:2,\allowbreak3,\allowbreak13}
\crossref{Matt}{24}{8}{Le 26:18-\allowbreak29 De 28:59 Isa 9:12,\allowbreak17,\allowbreak21; 10:4 1Th 5:3 1Pe 4:17,\allowbreak18}
\crossref{Matt}{24}{9}{Mt 10:17-\allowbreak22; 22:6; 23:34 Mr 13:9-\allowbreak13 Lu 11:49; 21:12,\allowbreak16,\allowbreak17 Joh 15:19}
\crossref{Matt}{24}{10}{Mt 11:6; 13:21,\allowbreak57; 26:31-\allowbreak34 Mr 4:17 Joh 6:60,\allowbreak61,\allowbreak66,\allowbreak67 2Ti 1:15}
\crossref{Matt}{24}{11}{24:5,\allowbreak24; 7:15 Mr 13:22 Ac 20:30 1Ti 4:1 2Pe 2:1 1Jo 2:18,\allowbreak26; 4:1}
\crossref{Matt}{24}{12}{Jas 4:1-\allowbreak4; 5:1-\allowbreak6}
\crossref{Matt}{24}{13}{24:6; 10:22 Mr 13:13 Lu 8:15 Ro 2:7 1Co 1:8 Heb 3:6,\allowbreak14; 10:39}
\crossref{Matt}{24}{14}{Mt 4:23; 9:35; 10:7 Ac 20:25}
\crossref{Matt}{24}{15}{Mr 13:14 Lu 19:43; 21:20}
\crossref{Matt}{24}{16}{Ge 19:15-\allowbreak17 Ex 9:20,\allowbreak21 Pr 22:3 Jer 6:1; 37:11,\allowbreak12 Lu 21:21,\allowbreak22}
\crossref{Matt}{24}{17}{Mt 6:25 Job 2:4 Pr 6:4,\allowbreak5 Mr 13:15,\allowbreak16 Lu 17:31-\allowbreak33}
\crossref{Matt}{24}{18}{24:18}
\crossref{Matt}{24}{19}{De 28:53-\allowbreak56 2Sa 4:4 2Ki 15:16 La 4:3,\allowbreak4,\allowbreak10 Ho 13:16 Mr 13:17,\allowbreak18}
\crossref{Matt}{24}{20}{Ex 16:29 Ac 1:12}
\crossref{Matt}{24}{21}{Ps 69:22-\allowbreak28 Isa 65:12-\allowbreak16; 66:15,\allowbreak16 Da 9:26; 12:1 Joe 1:2; 2:2}
\crossref{Matt}{24}{22}{Mr 13:20}
\crossref{Matt}{24}{23}{De 13:1-\allowbreak3 Mr 13:21 Lu 17:23,\allowbreak24; 21:8 Joh 5:43}
\crossref{Matt}{24}{24}{24:5,\allowbreak11 2Pe 2:1-\allowbreak3; 3:17}
\crossref{Matt}{24}{25}{Isa 44:7,\allowbreak8; 46:10,\allowbreak11; 48:5,\allowbreak6 Lu 21:13 Joh 16:1}
\crossref{Matt}{24}{26}{Mt 3:1 Isa 40:3 Lu 3:2,\allowbreak3 Ac 21:38}
\crossref{Matt}{24}{27}{Job 37:3; 38:35 Isa 30:30 Zec 9:14 Lu 17:24 etc.}
\crossref{Matt}{24}{28}{De 28:49 Job 39:27-\allowbreak30 Jer 16:16 Am 9:1-\allowbreak4 Lu 17:37}
\crossref{Matt}{24}{29}{24:8 Da 7:11,\allowbreak12 Mr 13:24,\allowbreak25}
\crossref{Matt}{24}{30}{24:3 Da 7:13 Mr 13:4 Re 1:7}
\crossref{Matt}{24}{31}{Mt 28:18 Mr 16:15,\allowbreak16 Lu 24:47 Ac 26:19,\allowbreak20}
\crossref{Matt}{24}{32}{Mr 13:28,\allowbreak29 Lu 21:29,\allowbreak30}
\crossref{Matt}{24}{33}{24:3}
\crossref{Matt}{24}{34}{Mt 12:45; 16:28; 23:36 Mr 13:30,\allowbreak31 Lu 11:50; 21:32,\allowbreak33}
\crossref{Matt}{24}{35}{Mt 5:18 Ps 102:26 Isa 34:4; 51:6; 54:10 Jer 31:35,\allowbreak36 Heb 1:11,\allowbreak12}
\crossref{Matt}{24}{36}{24:42,\allowbreak44; 25:13 Zec 14:7 Mr 13:32 Ac 1:7 1Th 5:2 2Pe 3:10}
\crossref{Matt}{24}{37}{Ge 6:1-\allowbreak7:24 Job 22:15-\allowbreak17 Lu 17:26,\allowbreak27 Heb 11:7 1Pe 3:20,\allowbreak21}
\crossref{Matt}{24}{38}{Ge 6:2 1Sa 25:36-\allowbreak38; 30:16,\allowbreak17 Isa 22:12-\allowbreak14 Eze 16:49,\allowbreak50 Am 6:3-\allowbreak6}
\crossref{Matt}{24}{39}{Mt 13:13-\allowbreak15 Jud 20:34 Pr 23:35; 24:12; 29:7 Isa 42:25; 44:18,\allowbreak19}
\crossref{Matt}{24}{40}{2Ch 33:12-\allowbreak24 Lu 17:34-\allowbreak37; 23:39-\allowbreak43 1Co 4:7 2Pe 2:5,\allowbreak7-\allowbreak9}
\crossref{Matt}{24}{41}{Ex 11:5 Isa 47:2}
\crossref{Matt}{24}{42}{Mt 25:13; 26:38-\allowbreak41 Mr 13:33-\allowbreak37 Lu 12:35-\allowbreak40; 21:36 Ro 13:11 1Co 16:13}
\crossref{Matt}{24}{43}{Mt 20:11 Pr 7:19}
\crossref{Matt}{24}{44}{Mt 25:10,\allowbreak13 Lu 12:40 Php 4:5 Jas 5:9 Re 19:7}
\crossref{Matt}{24}{45}{Lu 12:41-\allowbreak43; 16:10-\allowbreak12; 19:17 Ac 20:28 1Co 4:1,\allowbreak2 1Ti 1:12 2Ti 2:2}
\crossref{Matt}{24}{46}{Mt 25:34 Lu 12:37,\allowbreak43 Php 1:21-\allowbreak23 2Ti 4:6-\allowbreak8 2Pe 1:13-\allowbreak15}
\crossref{Matt}{24}{47}{Mt 25:21,\allowbreak23 Da 12:3 Lu 12:37,\allowbreak44; 19:17; 22:29,\allowbreak30 Joh 12:26}
\crossref{Matt}{24}{48}{Mt 18:32; 25:26 Lu 19:22}
\crossref{Matt}{24}{49}{Isa 66:5 2Co 11:20 1Pe 5:3 3Jo 1:9,\allowbreak10 Re 13:7; 16:6; 17:6}
\crossref{Matt}{24}{50}{24:42-\allowbreak44 Pr 29:1 1Th 5:2,\allowbreak3 Re 3:3}
\crossref{Matt}{24}{51}{Job 20:29 Isa 33:14 Lu 12:46}
\crossref{Matt}{25}{1}{Mt 24:42-\allowbreak51 Lu 21:34-\allowbreak36}
\crossref{Matt}{25}{2}{Mt 7:24-\allowbreak27; 13:19-\allowbreak23,\allowbreak38-\allowbreak43,\allowbreak47,\allowbreak48; 22:10,\allowbreak11 Jer 24:2 1Co 10:1-\allowbreak5}
\crossref{Matt}{25}{3}{Mt 23:25,\allowbreak26 Isa 48:1,\allowbreak2; 58:2 Eze 33:3 2Ti 3:5 Heb 12:15}
\crossref{Matt}{25}{4}{Ps 45:7 Zec 4:2,\allowbreak3 Joh 1:15,\allowbreak16; 3:34 Ro 8:9 2Co 1:22 Ga 5:22,\allowbreak23}
\crossref{Matt}{25}{5}{25:19; 24:48 Hab 2:3 Lu 12:45; 20:9 Heb 10:36,\allowbreak37 2Pe 3:4-\allowbreak9 Re 2:25}
\crossref{Matt}{25}{6}{Mt 24:44 Mr 13:33-\allowbreak37 Lu 12:20,\allowbreak38-\allowbreak40,\allowbreak46 1Th 5:1-\allowbreak3 Re 16:15}
\crossref{Matt}{25}{7}{Lu 12:35 2Pe 3:14 Re 2:4,\allowbreak5; 3:2,\allowbreak19,\allowbreak20}
\crossref{Matt}{25}{8}{Mt 3:9 Lu 16:24 Ac 8:24 Re 3:9}
\crossref{Matt}{25}{9}{Ps 49:7-\allowbreak9 Jer 15:1 Eze 14:14-\allowbreak16,\allowbreak20}
\crossref{Matt}{25}{10}{25:6 Re 1:7; 22:12,\allowbreak20}
\crossref{Matt}{25}{11}{Mt 7:21-\allowbreak23 Heb 12:16,\allowbreak17}
\crossref{Matt}{25}{12}{Ps 1:6; 5:5 Hab 1:13 Lu 13:26-\allowbreak30 Joh 9:31; 10:27 1Co 8:3 Ga 4:9}
\crossref{Matt}{25}{13}{Mt 24:42-\allowbreak44 Mr 13:33-\allowbreak37 Lu 21:36 Ac 20:31 1Co 16:13 1Th 5:6}
\crossref{Matt}{25}{14}{Mt 21:33 Mr 13:34 Lu 19:12,\allowbreak13; 20:9}
\crossref{Matt}{25}{15}{}
\crossref{Matt}{25}{16}{2Sa 7:1-\allowbreak3 1Ch 13:1-\allowbreak3; 22:1-\allowbreak26:32; 28:2 etc.}
\crossref{Matt}{25}{17}{Ge 18:19 2Sa 19:32 1Ki 18:3,\allowbreak4 2Ki 4:8-\allowbreak10 Job 29:11-\allowbreak17; 31:16-\allowbreak22}
\crossref{Matt}{25}{18}{Pr 18:9; 26:13-\allowbreak16 Hag 1:2-\allowbreak4 Mal 1:10 Lu 19:20 Heb 6:12 2Pe 1:8}
\crossref{Matt}{25}{19}{25:5; 24:48}
\crossref{Matt}{25}{20}{Lu 19:16,\allowbreak17 Ac 20:24 1Co 15:10 Col 1:29 2Ti 4:1-\allowbreak8 Jas 2:18}
\crossref{Matt}{25}{21}{2Ch 31:20,\allowbreak21 Lu 16:10 Ro 2:29 1Co 4:5 2Co 5:9; 10:18 1Pe 1:7}
\crossref{Matt}{25}{22}{Lu 19:18,\allowbreak19 Ro 12:6-\allowbreak8 2Co 8:1-\allowbreak3,\allowbreak7,\allowbreak8,\allowbreak12}
\crossref{Matt}{25}{23}{25:21 Mr 12:41-\allowbreak44; 14:8,\allowbreak9}
\crossref{Matt}{25}{24}{Mt 7:21 Lu 6:46}
\crossref{Matt}{25}{25}{2Sa 6:9,\allowbreak10 Pr 26:13 Isa 57:11 Ro 8:15 2Ti 1:6,\allowbreak7 Re 21:8}
\crossref{Matt}{25}{26}{Mt 18:32 Job 15:5,\allowbreak6}
\crossref{Matt}{25}{27}{Lu 19:22,\allowbreak23 Ro 3:19 Jude 1:15}
\crossref{Matt}{25}{28}{Lu 10:42; 19:24}
\crossref{Matt}{25}{29}{Mt 13:12 Mr 4:25 Lu 8:18; 16:9-\allowbreak12; 19:25,\allowbreak26 Joh 15:2}
\crossref{Matt}{25}{30}{Mt 3:10; 5:13 Jer 15:1,\allowbreak2 Eze 15:2-\allowbreak5 Lu 14:34,\allowbreak35 Joh 15:6 Tit 3:14}
\crossref{Matt}{25}{31}{25:6; 16:27; 19:28; 26:64 Da 7:13,\allowbreak14 Zec 14:5 Mr 8:38; 14:62 Lu 9:26}
\crossref{Matt}{25}{32}{Ps 96:13; 98:9 Ac 17:30,\allowbreak31 Ro 2:12,\allowbreak16; 14:10-\allowbreak12 2Co 5:10}
\crossref{Matt}{25}{33}{Ps 79:13; 95:7; 100:3 Joh 10:26-\allowbreak28; 21:15-\allowbreak17}
\crossref{Matt}{25}{34}{Mt 21:5; 22:11-\allowbreak13; 27:37 Ps 2:6; 24:7-\allowbreak10 Isa 9:7; 32:1,\allowbreak2; 33:22}
\crossref{Matt}{25}{35}{25:40; 10:40-\allowbreak42; 26:11 De 15:7-\allowbreak11 Job 29:13-\allowbreak16; 31:16-\allowbreak21}
\crossref{Matt}{25}{36}{Job 31:19,\allowbreak20 Lu 3:11 Jas 2:14-\allowbreak16}
\crossref{Matt}{25}{37}{Mt 6:3 1Ch 29:14 Pr 15:33 Isa 64:6 1Co 15:10 1Pe 5:5,\allowbreak6}
\crossref{Matt}{25}{38}{25:38}
\crossref{Matt}{25}{39}{}
\crossref{Matt}{25}{40}{25:34 Pr 25:6,\allowbreak7}
\crossref{Matt}{25}{41}{25:33}
\crossref{Matt}{25}{42}{25:35; 10:37,\allowbreak38; 12:30 Am 6:6 Joh 5:23; 8:42-\allowbreak44; 14:21 1Co 16:22}
\crossref{Matt}{25}{43}{}
\crossref{Matt}{25}{44}{25:24-\allowbreak27; 7:22 1Sa 15:13-\allowbreak15,\allowbreak20,\allowbreak21 Jer 2:23,\allowbreak35 Mal 1:6; 2:17; 3:13 Lu 10:29}
\crossref{Matt}{25}{45}{25:40 Ge 12:3 Nu 24:9 Ps 105:15 Pr 14:31; 17:5; 21:13 Zec 2:8}
\crossref{Matt}{25}{46}{25:41 Da 12:2 Mr 9:44,\allowbreak46,\allowbreak48,\allowbreak49 Lu 16:26 Joh 5:29 2Th 1:9}
\crossref{Matt}{26}{1}{Mt 19:1}
\crossref{Matt}{26}{2}{Mr 14:1,\allowbreak2 Lu 22:1,\allowbreak2,\allowbreak15 Joh 13:1}
\crossref{Matt}{26}{3}{Mt 21:45,\allowbreak46 Ps 2:1,\allowbreak2; 56:6; 64:4-\allowbreak6; 94:20,\allowbreak21 Jer 11:19; 18:18-\allowbreak20}
\crossref{Matt}{26}{4}{Ps 2:2}
\crossref{Matt}{26}{5}{Ps 76:10 Pr 19:21; 21:30 Isa 46:10 La 3:37 Mr 14:2,\allowbreak12,\allowbreak27 Lu 22:7}
\crossref{Matt}{26}{6}{Mt 21:17 Mr 11:12 Joh 11:1,\allowbreak2; 12:1}
\crossref{Matt}{26}{7}{Joh 12:2,\allowbreak3}
\crossref{Matt}{26}{8}{1Sa 17:28,\allowbreak29 Ec 4:4 Mr 14:4 Joh 12:4-\allowbreak6}
\crossref{Matt}{26}{9}{Jos 7:20,\allowbreak21 1Sa 15:9,\allowbreak21 2Ki 5:20 Mr 14:5 Joh 12:5,\allowbreak6 2Pe 2:15}
\crossref{Matt}{26}{10}{Job 13:7 Mr 14:6 Lu 7:44-\allowbreak50 Ga 1:7; 5:12; 6:17}
\crossref{Matt}{26}{11}{Mt 25:34-\allowbreak40,\allowbreak42-\allowbreak45 De 15:11 Mr 14:7 Joh 12:8 Ga 2:10 1Jo 3:17}
\crossref{Matt}{26}{12}{2Ch 16:14 Mr 14:8; 16:1 Lu 23:56; 24:1 Joh 12:7; 19:39,\allowbreak40}
\crossref{Matt}{26}{13}{Mt 24:14; 28:19 Ps 98:2,\allowbreak3 Isa 52:9 Mr 13:10; 16:15 Lu 24:47}
\crossref{Matt}{26}{14}{Mr 14:10 Lu 22:3-\allowbreak6 Joh 13:2,\allowbreak30}
\crossref{Matt}{26}{15}{Ge 38:16 Jud 16:5; 17:10; 18:19,\allowbreak20 Isa 56:11 1Ti 3:3; 6:9,\allowbreak10}
\crossref{Matt}{26}{16}{Mr 14:11 Lu 22:6}
\crossref{Matt}{26}{17}{Ex 12:6,\allowbreak18-\allowbreak20; 13:6-\allowbreak8 Le 23:5,\allowbreak6 Nu 28:16,\allowbreak17 De 16:1-\allowbreak4 Mr 14:12}
\crossref{Matt}{26}{18}{Mr 14:13-\allowbreak16 Lu 22:10-\allowbreak13}
\crossref{Matt}{26}{19}{Mt 21:6 Joh 2:5; 15:14}
\crossref{Matt}{26}{20}{Mr 14:17-\allowbreak21 Lu 22:14-\allowbreak16 Joh 13:21}
\crossref{Matt}{26}{21}{26:2,\allowbreak14-\allowbreak16 Ps 55:12-\allowbreak14 Joh 6:70,\allowbreak71; 13:21 Heb 4:13 Re 2:23}
\crossref{Matt}{26}{22}{Mr 14:19,\allowbreak20 Lu 22:23 Joh 13:22-\allowbreak25; 21:17}
\crossref{Matt}{26}{23}{Ps 41:9 Lu 22:21 Joh 13:18,\allowbreak26-\allowbreak28}
\crossref{Matt}{26}{24}{26:54,\allowbreak56 Ge 3:15 Ps 22:1-\allowbreak31; 69:1-\allowbreak21 Isa 50:5,\allowbreak6; 53:1-\allowbreak12 Da 9:26}
\crossref{Matt}{26}{25}{2Ki 5:25 Pr 30:20}
\crossref{Matt}{26}{26}{Mr 14:22 Lu 22:19}
\crossref{Matt}{26}{27}{Mr 14:23,\allowbreak24 Lu 22:20}
\crossref{Matt}{26}{28}{Ex 24:7,\allowbreak8 Le 17:11 Jer 31:31 Zec 9:11 Mr 14:24 Lu 22:19}
\crossref{Matt}{26}{29}{Ps 4:7; 104:15 Isa 24:9-\allowbreak11 Mr 14:25 Lu 22:15-\allowbreak18}
\crossref{Matt}{26}{30}{Ps 81:1-\allowbreak4 Mr 14:26 Eph 5:19,\allowbreak20 Col 3:16,\allowbreak17}
\crossref{Matt}{26}{31}{26:56; 11:6; 24:9,\allowbreak10 Mr 14:27,\allowbreak28 Lu 22:31,\allowbreak32 Joh 16:32}
\crossref{Matt}{26}{32}{Mt 16:21; 20:19; 27:63,\allowbreak64 Mr 9:9,\allowbreak10 Lu 18:33,\allowbreak34}
\crossref{Matt}{26}{33}{Mr 14:29 Lu 22:33 Joh 13:36-\allowbreak38; 21:15}
\crossref{Matt}{26}{34}{Mr 14:30,\allowbreak31 Lu 22:34 Joh 13:38}
\crossref{Matt}{26}{35}{Mt 20:22,\allowbreak23 Pr 28:14; 29:23 Ro 11:20 1Co 10:12 Php 2:12 1Pe 1:17}
\crossref{Matt}{26}{36}{Mr 14:32-\allowbreak35 Lu 22:39 etc.}
\crossref{Matt}{26}{37}{Mt 4:18,\allowbreak21; 17:1; 20:20 Mr 5:37}
\crossref{Matt}{26}{38}{Job 6:2-\allowbreak4 Ps 88:1-\allowbreak7,\allowbreak14-\allowbreak16; 116:3 Isa 53:3,\allowbreak10 Ro 8:32 2Co 5:21}
\crossref{Matt}{26}{39}{Ge 17:3 Nu 14:5; 16:22 1Ch 21:16 Eze 1:28 Lu 17:16 Ac 10:25}
\crossref{Matt}{26}{40}{26:43; 25:5 So 5:2 Mr 14:37 Lu 9:32; 22:45}
\crossref{Matt}{26}{41}{Mt 24:42; 25:13 Mr 13:33-\allowbreak37; 14:38 Lu 21:36; 22:40,\allowbreak46 1Co 16:13}
\crossref{Matt}{26}{42}{26:39 Ps 22:1,\allowbreak2; 69:1-\allowbreak3,\allowbreak17,\allowbreak18; 88:1,\allowbreak2 Mr 14:39,\allowbreak40 Heb 4:15; 5:7,\allowbreak8}
\crossref{Matt}{26}{43}{Pr 23:34 Jon 1:6 Lu 9:32 Ac 20:9 Ro 13:1 1Th 5:6-\allowbreak8}
\crossref{Matt}{26}{44}{Mt 6:7 Da 9:17-\allowbreak19 Lu 18:1 2Co 12:8}
\crossref{Matt}{26}{45}{26:2,\allowbreak14,\allowbreak15 Mr 14:41,\allowbreak42 Lu 22:53 Joh 13:1; 17:1}
\crossref{Matt}{26}{46}{1Sa 17:48 Lu 9:51; 12:50; 22:15 Joh 14:31 Ac 21:13}
\crossref{Matt}{26}{47}{26:55 Mr 14:43 Lu 22:47,\allowbreak48 Joh 18:1-\allowbreak8 Ac 1:16}
\crossref{Matt}{26}{48}{2Sa 3:27; 20:9,\allowbreak10 Ps 28:3; 55:20,\allowbreak21}
\crossref{Matt}{26}{49}{Mt 27:29,\allowbreak30 Mr 15:18 Joh 19:3}
\crossref{Matt}{26}{50}{}
\crossref{Matt}{26}{51}{26:35 Mr 14:47 Lu 9:55; 22:36-\allowbreak38,\allowbreak49-\allowbreak51 Joh 18:10,\allowbreak11,\allowbreak36 2Co 10:4}
\crossref{Matt}{26}{52}{Mt 5:39 Ro 12:19 1Co 4:11,\allowbreak12 1Th 5:15 1Pe 2:21-\allowbreak23; 3:9}
\crossref{Matt}{26}{53}{Mt 4:11; 25:31 2Ki 6:17 Da 7:10 2Th 1:7 Jude 1:14}
\crossref{Matt}{26}{54}{26:24 Ps 22:1-\allowbreak31; 69:1-\allowbreak36 Isa 53:1-\allowbreak12 Da 9:24-\allowbreak26 Zec 13:7}
\crossref{Matt}{26}{55}{Mr 14:48-\allowbreak50 Lu 22:52,\allowbreak53}
\crossref{Matt}{26}{56}{26:54 Ge 3:15 Isa 44:26 La 4:20 Da 9:24,\allowbreak26 Zec 13:7 Ac 1:16; 2:23}
\crossref{Matt}{26}{57}{Ps 56:5,\allowbreak6 Mr 14:53,\allowbreak54 Lu 22:54,\allowbreak55 Joh 11:49; 18:12-\allowbreak14,\allowbreak24}
\crossref{Matt}{26}{58}{Joh 18:15,\allowbreak16,\allowbreak25}
\crossref{Matt}{26}{59}{De 19:16-\allowbreak21 1Ki 21:8-\allowbreak13 Ps 27:12; 35:11,\allowbreak12; 94:20,\allowbreak21 Pr 25:18}
\crossref{Matt}{26}{60}{Da 6:4,\allowbreak5 Tit 2:8 1Pe 3:16}
\crossref{Matt}{26}{61}{26:71; 12:24 Ge 19:9 1Ki 22:27 2Ki 9:11 Ps 22:6,\allowbreak7 Isa 49:7; 53:3}
\crossref{Matt}{26}{62}{Mt 27:12-\allowbreak14 Mr 14:60 Lu 23:9 Joh 18:19-\allowbreak24; 19:9-\allowbreak11}
\crossref{Matt}{26}{63}{Ps 38:12-\allowbreak14 Isa 53:7 Da 3:16 Ac 8:32-\allowbreak35 1Pe 2:23}
\crossref{Matt}{26}{64}{26:25; 27:11 Mr 14:62 Lu 22:70 Joh 18:37}
\crossref{Matt}{26}{65}{Le 21:20 2Ki 18:37; 19:1-\allowbreak3 Jer 36:24 Mr 14:63,\allowbreak64}
\crossref{Matt}{26}{66}{Le 24:11-\allowbreak16 Joh 19:7 Ac 7:52; 13:27,\allowbreak28 Jas 5:6}
\crossref{Matt}{26}{67}{Mt 27:30 Nu 12:14 De 25:9 Job 30:9-\allowbreak11 Isa 50:6; 52:14; 53:3 Mr 14:65}
\crossref{Matt}{26}{68}{Mt 27:39-\allowbreak44 Ge 37:19,\allowbreak20 Jud 16:25 Mr 14:65 Lu 22:63-\allowbreak65}
\crossref{Matt}{26}{69}{26:58 1Ki 19:9,\allowbreak13 Ps 1:1 Mr 14:66-\allowbreak68 Lu 22:55-\allowbreak57 Joh 18:16,\allowbreak17,\allowbreak25}
\crossref{Matt}{26}{70}{26:34,\allowbreak35,\allowbreak40-\allowbreak43,\allowbreak51,\allowbreak56,\allowbreak58 Ps 119:115-\allowbreak117 Pr 28:26; 29:23,\allowbreak25}
\crossref{Matt}{26}{71}{Mr 14:68,\allowbreak69 Lu 22:58 Joh 18:25-\allowbreak27}
\crossref{Matt}{26}{72}{Mt 5:34-\allowbreak36 Ex 20:7 Isa 48:1 Zec 5:3,\allowbreak4; 8:17 Mal 3:5 Ac 5:3,\allowbreak4}
\crossref{Matt}{26}{73}{Lu 22:59,\allowbreak60 Joh 18:26,\allowbreak27}
\crossref{Matt}{26}{74}{Mt 27:25 Jud 17:2; 21:18 1Sa 14:24-\allowbreak28 Mr 14:71 Ac 23:12-\allowbreak14 Ro 9:3}
\crossref{Matt}{26}{75}{26:34 Lu 22:61,\allowbreak62 Joh 13:38}
\crossref{Matt}{27}{1}{Jud 16:2 1Sa 19:11 Pr 4:16-\allowbreak18 Mic 2:1 Lu 22:66 Ac 5:21}
\crossref{Matt}{27}{2}{Ge 22:9 Joh 18:12,\allowbreak24 Ac 9:2; 12:6; 21:33; 22:25,\allowbreak29; 24:27; 28:20}
\crossref{Matt}{27}{3}{Mt 26:14-\allowbreak16,\allowbreak47-\allowbreak50 Mr 14:10,\allowbreak11,\allowbreak43-\allowbreak46 Lu 22:2-\allowbreak6,\allowbreak47,\allowbreak48 Joh 13:2,\allowbreak27}
\crossref{Matt}{27}{4}{Ge 42:21,\allowbreak22 Ex 9:27; 10:16,\allowbreak17; 12:31 1Sa 15:24,\allowbreak30 1Ki 21:27}
\crossref{Matt}{27}{5}{Jud 9:54 1Sa 31:4,\allowbreak5 2Sa 17:23 1Ki 16:18 Job 2:9; 7:15 Ps 55:23}
\crossref{Matt}{27}{6}{Mt 23:24 Lu 6:7-\allowbreak9 Joh 18:28}
\crossref{Matt}{27}{7}{}
\crossref{Matt}{27}{8}{Ac 1:19}
\crossref{Matt}{27}{9}{Zec 11:12,\allowbreak13}
\crossref{Matt}{27}{10}{}
\crossref{Matt}{27}{11}{Mt 10:18,\allowbreak25 Mr 15:2 Lu 23:3 Joh 18:33-\allowbreak36}
\crossref{Matt}{27}{12}{27:14; 26:63 Ps 38:13,\allowbreak14 Isa 53:7 Mr 15:3-\allowbreak5 Joh 19:9-\allowbreak11 Ac 8:32}
\crossref{Matt}{27}{13}{Mt 26:62 Joh 18:35 Ac 22:24}
\crossref{Matt}{27}{14}{Ps 71:7 Isa 8:18 Zec 3:8 1Co 4:9}
\crossref{Matt}{27}{15}{Mt 26:5 Mr 15:6,\allowbreak8 Lu 23:16,\allowbreak17 Joh 18:38,\allowbreak39 Ac 24:27; 25:9}
\crossref{Matt}{27}{16}{Mr 15:7 Lu 23:18,\allowbreak19,\allowbreak25 Joh 18:40 Ac 3:14 Ro 1:32}
\crossref{Matt}{27}{17}{27:21 Jos 24:15 1Ki 18:21}
\crossref{Matt}{27}{18}{Ge 37:11 1Sa 18:7-\allowbreak11 Ps 106:16 Pr 27:4 Ec 4:4 Isa 26:11}
\crossref{Matt}{27}{19}{Ge 20:3-\allowbreak6; 31:24,\allowbreak29 Job 33:14-\allowbreak17 Pr 29:1}
\crossref{Matt}{27}{20}{Mr 15:11 Ac 14:18,\allowbreak19; 19:23-\allowbreak29}
\crossref{Matt}{27}{21}{}
\crossref{Matt}{27}{22}{27:17 Job 31:31 Ps 22:8,\allowbreak9 Isa 49:7; 53:2,\allowbreak3 Zec 11:8 Mr 14:55}
\crossref{Matt}{27}{23}{Ge 37:18,\allowbreak19 1Sa 19:3-\allowbreak15; 20:31-\allowbreak33; 22:14-\allowbreak19}
\crossref{Matt}{27}{24}{De 21:6,\allowbreak7 Job 9:30,\allowbreak31 Ps 26:6 Jer 2:27,\allowbreak35}
\crossref{Matt}{27}{25}{Mt 21:44; 23:30-\allowbreak37 Nu 35:33 De 19:10,\allowbreak13 Jos 2:19 2Sa 1:16; 3:28,\allowbreak29}
\crossref{Matt}{27}{26}{Mr 15:15 Lu 23:25}
\crossref{Matt}{27}{27}{Mr 15:16 Joh 18:28,\allowbreak33; 19:8,\allowbreak9 Ac 23:35}
\crossref{Matt}{27}{28}{Mr 15:17 Lu 23:11 Joh 19:2-\allowbreak5}
\crossref{Matt}{27}{29}{Mt 20:19 Ps 35:15,\allowbreak16; 69:7,\allowbreak19,\allowbreak20 Isa 49:7; 53:3 Jer 20:7 Heb 12:2,\allowbreak3}
\crossref{Matt}{27}{30}{Mt 26:67 Job 30:8-\allowbreak10 Isa 49:7; 50:6; 52:14; 53:3,\allowbreak7 Mic 5:1 Mr 15:19}
\crossref{Matt}{27}{31}{Mt 20:19; 21:39 Nu 15:35 1Ki 21:10,\allowbreak13 Isa 53:7 Joh 19:16,\allowbreak27 Ac 7:58}
\crossref{Matt}{27}{32}{Le 4:3,\allowbreak12,\allowbreak21 Nu 15:35,\allowbreak36 1Ki 21:10,\allowbreak13 Ac 7:58 Heb 13:11,\allowbreak12}
\crossref{Matt}{27}{33}{Mr 15:22 Lu 23:27-\allowbreak33 Joh 19:17}
\crossref{Matt}{27}{34}{27:48 Ps 69:21 Mr 15:23 Joh 19:28-\allowbreak30}
\crossref{Matt}{27}{35}{Ps 22:16 Joh 20:20,\allowbreak25,\allowbreak27 Ac 4:10}
\crossref{Matt}{27}{36}{27:54 Mr 15:39,\allowbreak44}
\crossref{Matt}{27}{37}{Mr 15:26 Lu 23:38 Joh 19:19-\allowbreak22}
\crossref{Matt}{27}{38}{27:44 Isa 53:12 Mr 15:27,\allowbreak28 Lu 22:37; 23:32,\allowbreak33,\allowbreak39-\allowbreak43 Joh 19:18}
\crossref{Matt}{27}{39}{Ps 22:6,\allowbreak7,\allowbreak17; 31:11-\allowbreak13; 35:15-\allowbreak21; 69:7-\allowbreak12,\allowbreak20; 109:2,\allowbreak25 La 1:12}
\crossref{Matt}{27}{40}{Ge 37:19,\allowbreak20 Re 11:10}
\crossref{Matt}{27}{41}{Job 13:9 Ps 22:12,\allowbreak13; 35:26 Isa 28:22; 49:7 Zec 11:8 Mr 15:31,\allowbreak32}
\crossref{Matt}{27}{42}{Joh 9:24; 12:47 Ac 4:14}
\crossref{Matt}{27}{43}{Ps 3:2; 14:6; 22:8; 42:10; 71:11 Isa 36:15,\allowbreak18; 37:10}
\crossref{Matt}{27}{44}{27:38 Job 30:7-\allowbreak9 Ps 35:15 Mr 15:32 Lu 23:39,\allowbreak40}
\crossref{Matt}{27}{45}{Mr 15:25,\allowbreak33,\allowbreak34 Lu 23:44,\allowbreak45}
\crossref{Matt}{27}{46}{Mr 15:34 Lu 23:46 Joh 19:28-\allowbreak30 Heb 5:7}
\crossref{Matt}{27}{47}{Mt 11:14 Mal 4:5 Mr 15:35,\allowbreak36}
\crossref{Matt}{27}{48}{27:34 Ps 69:21 Lu 23:36 Joh 19:29,\allowbreak30}
\crossref{Matt}{27}{49}{27:43}
\crossref{Matt}{27}{50}{Mr 15:37 Lu 23:46 Joh 19:30}
\crossref{Matt}{27}{51}{Ex 26:31-\allowbreak37; 40:21 Le 16:2,\allowbreak12-\allowbreak15; 21:23 2Ch 3:14 Isa 25:7}
\crossref{Matt}{27}{52}{Isa 25:8; 26:19 Ho 13:14 Joh 5:25-\allowbreak29 1Co 15:20}
\crossref{Matt}{27}{53}{Mt 4:5 Ne 11:1 Isa 48:2 Da 9:24 Re 11:2; 21:2; 22:19}
\crossref{Matt}{27}{54}{27:36; 8:5 Ac 10:1; 21:32; 23:17,\allowbreak23; 27:1,\allowbreak43}
\crossref{Matt}{27}{55}{Lu 23:27,\allowbreak28,\allowbreak48,\allowbreak49 Joh 19:25-\allowbreak27}
\crossref{Matt}{27}{56}{27:61; 28:1 Mr 15:40,\allowbreak41; 16:1,\allowbreak9 Lu 24:10 Joh 20:1,\allowbreak18}
\crossref{Matt}{27}{57}{Mr 15:42,\allowbreak43 Lu 23:50,\allowbreak51 Joh 19:38-\allowbreak42}
\crossref{Matt}{27}{58}{Mr 15:44-\allowbreak46 Lu 23:52,\allowbreak53}
\crossref{Matt}{27}{59}{}
\crossref{Matt}{27}{60}{Isa 53:9}
\crossref{Matt}{27}{61}{27:56}
\crossref{Matt}{27}{62}{Mt 26:17 Mr 15:42 Lu 23:54-\allowbreak56 Joh 19:14,\allowbreak42}
\crossref{Matt}{27}{63}{Lu 23:2 Joh 7:12,\allowbreak47 2Co 6:8}
\crossref{Matt}{27}{64}{Mt 28:13}
\crossref{Matt}{27}{65}{Mt 28:11-\allowbreak15 Ps 76:10 Pr 21:30}
\crossref{Matt}{27}{66}{Da 6:17 2Ti 2:19}
\crossref{Matt}{28}{1}{Mt 27:56,\allowbreak61}
\crossref{Matt}{28}{2}{Mt 27:51-\allowbreak53 Ac 16:26 Re 11:19}
\crossref{Matt}{28}{3}{Mt 17:2 Ps 104:4 Eze 1:4-\allowbreak14 Da 10:5,\allowbreak6 Re 1:14-\allowbreak16; 10:1; 18:1}
\crossref{Matt}{28}{4}{28:11; 27:65,\allowbreak66}
\crossref{Matt}{28}{5}{Isa 35:4; 41:10,\allowbreak14 Da 10:12,\allowbreak19 Mr 16:6 Lu 1:12,\allowbreak13,\allowbreak30 Heb 1:14}
\crossref{Matt}{28}{6}{Mt 12:40; 16:21; 17:9,\allowbreak23; 20:19; 26:31,\allowbreak32; 27:63 Mr 8:31 Lu 24:6-\allowbreak8}
\crossref{Matt}{28}{7}{28:10 Mr 16:7,\allowbreak8,\allowbreak10,\allowbreak13 Lu 24:9,\allowbreak10,\allowbreak22-\allowbreak24,\allowbreak34 Joh 20:17,\allowbreak18}
\crossref{Matt}{28}{8}{Ezr 3:12,\allowbreak13 Ps 2:11 Mr 16:8 Lu 24:36-\allowbreak41 Joh 16:20,\allowbreak22; 20:20,\allowbreak21}
\crossref{Matt}{28}{9}{Isa 64:5 Mr 16:9,\allowbreak10 Joh 20:14-\allowbreak16}
\crossref{Matt}{28}{10}{28:5; 14:27 Lu 24:36-\allowbreak38 Joh 6:20}
\crossref{Matt}{28}{11}{28:4; 27:65,\allowbreak66}
\crossref{Matt}{28}{12}{Mt 26:3,\allowbreak4; 27:1,\allowbreak2,\allowbreak62-\allowbreak64 Ps 2:1-\allowbreak7 Joh 11:47; 12:10,\allowbreak11 Ac 4:5-\allowbreak22}
\crossref{Matt}{28}{13}{Mt 26:64}
\crossref{Matt}{28}{14}{Ac 12:19}
\crossref{Matt}{28}{15}{Mt 26:15 1Ti 6:10}
\crossref{Matt}{28}{16}{Mr 16:14 Joh 6:70 Ac 1:13-\allowbreak26 1Co 15:15}
\crossref{Matt}{28}{17}{Mt 16:28}
\crossref{Matt}{28}{18}{Mt 11:27; 16:28 Ps 2:6-\allowbreak9; 89:19,\allowbreak27; 110:1-\allowbreak3 Isa 9:6,\allowbreak7 Da 7:14}
\crossref{Matt}{28}{19}{Ps 22:27,\allowbreak28; 98:2,\allowbreak3 Isa 42:1-\allowbreak4; 49:6; 52:10; 66:18,\allowbreak19 Mr 16:15,\allowbreak16}
\crossref{Matt}{28}{20}{Mt 7:24-\allowbreak27 De 5:32; 12:32 Ac 2:42; 20:20,\allowbreak21,\allowbreak27 1Co 11:2,\allowbreak23; 14:37}

% Mark
\crossref{Mark}{1}{1}{Lu 1:2,\allowbreak3; 2:10,\allowbreak11 Ac 1:1,\allowbreak2}
\crossref{Mark}{1}{2}{Ps 40:7 Mt 2:5; 26:24,\allowbreak31 Lu 1:70; 18:31}
\crossref{Mark}{1}{3}{Isa 40:3-\allowbreak5 Mt 3:3 Lu 3:4-\allowbreak6 Joh 1:15,\allowbreak19-\allowbreak34; 3:28-\allowbreak36}
\crossref{Mark}{1}{4}{Mt 3:1,\allowbreak2,\allowbreak6,\allowbreak11 Lu 3:2,\allowbreak3 Joh 3:23 Ac 10:37; 13:24,\allowbreak25; 19:3,\allowbreak4}
\crossref{Mark}{1}{5}{Mt 3:5,\allowbreak6; 4:25}
\crossref{Mark}{1}{6}{2Ki 1:8 Zec 13:4 Mt 3:4}
\crossref{Mark}{1}{7}{Mt 3:11,\allowbreak14 Lu 3:16; 7:6,\allowbreak7 Joh 1:27; 3:28-\allowbreak31 Ac 13:25}
\crossref{Mark}{1}{8}{Mt 3:11}
\crossref{Mark}{1}{9}{Mt 3:13-\allowbreak15 Lu 3:21}
\crossref{Mark}{1}{10}{Mt 3:16 Joh 1:31-\allowbreak34}
\crossref{Mark}{1}{11}{Mt 3:17 Joh 5:37; 12:28-\allowbreak30 2Pe 1:17,\allowbreak18}
\crossref{Mark}{1}{12}{Mt 4:1 etc.}
\crossref{Mark}{1}{13}{Ex 24:18; 34:28 De 9:11,\allowbreak18,\allowbreak25 1Ki 19:8}
\crossref{Mark}{1}{14}{Mt 4:12; 11:2; 14:2 Lu 3:20 Joh 3:22-\allowbreak24}
\crossref{Mark}{1}{15}{Da 2:44; 9:25 Ga 4:4 Eph 1:10}
\crossref{Mark}{1}{16}{Mt 4:18 etc.}
\crossref{Mark}{1}{17}{Eze 47:10 Mt 4:19,\allowbreak20 Lu 5:10 Ac 2:38-\allowbreak41}
\crossref{Mark}{1}{18}{Mr 10:28-\allowbreak31 Mt 19:27-\allowbreak30 Lu 5:11; 14:33; 18:28-\allowbreak30 Php 3:8}
\crossref{Mark}{1}{19}{Mr 3:17; 5:37; 9:2; 10:35; 14:33 Mt 4:21 Ac 1:13; 12:2}
\crossref{Mark}{1}{20}{Mr 10:29 De 33:9 1Ki 19:20 Mt 4:21,\allowbreak22; 8:21,\allowbreak22; 10:37 Lu 14:26}
\crossref{Mark}{1}{21}{Mr 2:1 Mt 4:13 Lu 4:31; 10:15}
\crossref{Mark}{1}{22}{Jer 23:29 Mt 7:28,\allowbreak29; 13:54 Lu 4:32; 21:15 Joh 7:46 Ac 6:10}
\crossref{Mark}{1}{23}{1:34; 5:2; 7:25; 9:25 Mt 12:43 Lu 4:33-\allowbreak37}
\crossref{Mark}{1}{24}{Mr 5:7 Ex 14:12 Mt 8:29 Lu 8:28,\allowbreak37 Jas 2:19}
\crossref{Mark}{1}{25}{1:34; 3:11,\allowbreak12; 9:25 Ps 50:16 Lu 4:35,\allowbreak41 Ac 16:17}
\crossref{Mark}{1}{26}{Mr 9:20,\allowbreak26 Lu 9:39,\allowbreak42; 11:22}
\crossref{Mark}{1}{27}{Mr 7:37 Mt 9:33; 12:22,\allowbreak23; 15:31}
\crossref{Mark}{1}{28}{1:45 Mic 5:4 Mt 4:24; 9:31 Lu 4:17,\allowbreak37}
\crossref{Mark}{1}{29}{Mt 8:14,\allowbreak15 Lu 4:38,\allowbreak39; 9:58}
\crossref{Mark}{1}{30}{1Co 9:5}
\crossref{Mark}{1}{31}{Mr 5:41 Ac 9:41}
\crossref{Mark}{1}{32}{1:21; 3:2 Mt 8:16 Lu 4:40}
\crossref{Mark}{1}{33}{1:5 Ac 13:44}
\crossref{Mark}{1}{34}{1:25; 3:12 Lu 4:41 Ac 16:16-\allowbreak18}
\crossref{Mark}{1}{35}{Mr 6:46-\allowbreak48 Ps 5:3; 109:4 Lu 4:42; 6:12; 22:39-\allowbreak46 Joh 4:34; 6:15}
\crossref{Mark}{1}{36}{}
\crossref{Mark}{1}{37}{1:5 Zec 11:11 Joh 3:26; 11:48; 12:19}
\crossref{Mark}{1}{38}{Lu 4:43}
\crossref{Mark}{1}{39}{1:21 Mt 4:23 Lu 4:43,\allowbreak44}
\crossref{Mark}{1}{40}{Mt 8:2-\allowbreak4 Lu 5:12-\allowbreak14}
\crossref{Mark}{1}{41}{Mr 6:34 Mt 9:36 Lu 7:12,\allowbreak13 Heb 2:17; 4:15}
\crossref{Mark}{1}{42}{1:31; 5:29 Ps 33:9 Mt 15:28 Joh 4:50-\allowbreak53; 15:3}
\crossref{Mark}{1}{43}{Mr 3:12; 5:43; 7:36 Mt 9:30 Lu 8:56}
\crossref{Mark}{1}{44}{Le 14:2-\allowbreak32 Mt 23:2,\allowbreak3 Lu 5:14; 17:14}
\crossref{Mark}{1}{45}{Ps 77:11 Mt 9:31 Lu 5:15 Tit 1:10}
\crossref{Mark}{2}{1}{Mr 1:45 Mt 9:1 Lu 5:18}
\crossref{Mark}{2}{2}{2:13; 1:33,\allowbreak37,\allowbreak45; 4:1,\allowbreak2 Lu 5:17; 12:1}
\crossref{Mark}{2}{3}{Mt 9:1,\allowbreak2 etc.}
\crossref{Mark}{2}{4}{De 22:8 Lu 5:19}
\crossref{Mark}{2}{5}{Ge 22:12 Joh 2:25 Ac 11:23; 14:9 Eph 2:8 1Th 1:3,\allowbreak4 Jas 2:18-\allowbreak22}
\crossref{Mark}{2}{6}{Mr 8:17 Mt 16:7,\allowbreak8 Lu 5:21,\allowbreak22 2Co 10:5}
\crossref{Mark}{2}{7}{Mr 14:64 Mt 9:3; 26:65 Joh 10:33,\allowbreak36}
\crossref{Mark}{2}{8}{1Ch 29:17 Mt 9:4 Lu 5:22; 6:8; 7:39,\allowbreak40 Joh 2:24,\allowbreak25; 6:64; 21:17}
\crossref{Mark}{2}{9}{Mt 9:5 Lu 5:22-\allowbreak25}
\crossref{Mark}{2}{10}{Da 7:13,\allowbreak14 Mt 9:6-\allowbreak8; 16:13 Joh 5:20-\allowbreak27 Ac 5:31 1Ti 1:13-\allowbreak16}
\crossref{Mark}{2}{11}{Mr 1:41 Joh 5:8-\allowbreak10; 6:63}
\crossref{Mark}{2}{12}{Mr 1:27 Mt 9:8; 12:23 Lu 7:16}
\crossref{Mark}{2}{13}{Mt 9:9; 13:1}
\crossref{Mark}{2}{14}{Mr 3:18 Mt 9:9 Lu 5:27}
\crossref{Mark}{2}{15}{Mt 9:10,\allowbreak11; 21:31,\allowbreak32 Lu 5:29,\allowbreak30; 6:17; 15:1}
\crossref{Mark}{2}{16}{2:7 Isa 65:5 Lu 15:2 etc.}
\crossref{Mark}{2}{17}{Mt 9:12,\allowbreak13 Lu 5:31,\allowbreak32; 15:7,\allowbreak29; 16:15 Joh 9:34,\allowbreak40}
\crossref{Mark}{2}{18}{Mt 9:14-\allowbreak17 Lu 5:33-\allowbreak39}
\crossref{Mark}{2}{19}{Ge 29:22 Jud 14:10,\allowbreak11 Ps 45:14 So 6:8 Mt 25:1-\allowbreak10}
\crossref{Mark}{2}{20}{Ps 45:11 So 3:11 Isa 54:5; 62:5 Joh 3:29 2Co 11:2 Re 19:7; 21:9}
\crossref{Mark}{2}{21}{Ps 103:13-\allowbreak15 Isa 57:16 1Co 10:13}
\crossref{Mark}{2}{22}{Jos 9:4,\allowbreak13 Job 32:19 Ps 119:80,\allowbreak83 Mt 9:17 Lu 5:37,\allowbreak38}
\crossref{Mark}{2}{23}{Mt 12:1-\allowbreak8 Lu 6:1-\allowbreak5}
\crossref{Mark}{2}{24}{2:7,\allowbreak16 Mt 7:3-\allowbreak5; 15:2,\allowbreak3; 23:23,\allowbreak24 Heb 12:3}
\crossref{Mark}{2}{25}{Mr 12:20,\allowbreak26 Mt 19:4; 21:16,\allowbreak42; 22:31 Lu 10:26}
\crossref{Mark}{2}{26}{Ex 29:32,\allowbreak33 Le 24:5-\allowbreak9}
\crossref{Mark}{2}{27}{Ex 23:12 De 5:14 Ne 9:13,\allowbreak14 Isa 58:13 Eze 20:12,\allowbreak20 Lu 6:9}
\crossref{Mark}{2}{28}{Mr 3:4 Mt 12:8 Lu 6:5; 13:15,\allowbreak16 Joh 5:9-\allowbreak11,\allowbreak17; 9:5-\allowbreak11,\allowbreak14,\allowbreak16}
\crossref{Mark}{3}{1}{Mr 1:21 Mt 12:9-\allowbreak14 Lu 6:6-\allowbreak11}
\crossref{Mark}{3}{2}{Ps 37:32 Isa 29:20,\allowbreak21 Jer 20:10 Da 6:4 Lu 6:7; 11:53,\allowbreak54; 14:1}
\crossref{Mark}{3}{3}{Isa 42:4 Da 6:10 Lu 6:8 Joh 9:4 1Co 15:58 Ga 6:9}
\crossref{Mark}{3}{4}{Mr 2:27,\allowbreak28 Ho 6:6 Mt 12:10-\allowbreak12 Lu 6:9; 13:13-\allowbreak17; 14:1-\allowbreak5}
\crossref{Mark}{3}{5}{Ge 6:6 Jud 10:16 Ne 13:8 Ps 95:10 Isa 63:9,\allowbreak10 Lu 19:40-\allowbreak44}
\crossref{Mark}{3}{6}{Ps 109:3,\allowbreak4 Mt 12:14 Lu 6:11; 20:19,\allowbreak20; 22:2 Joh 11:53}
\crossref{Mark}{3}{7}{Mt 10:23; 12:15 Lu 6:12 Joh 10:39-\allowbreak41; 11:53,\allowbreak54 Ac 14:5,\allowbreak6}
\crossref{Mark}{3}{8}{Isa 34:5 Eze 35:15; 36:5 Mal 1:2-\allowbreak4}
\crossref{Mark}{3}{9}{Mr 5:30 Joh 6:15}
\crossref{Mark}{3}{10}{Mt 12:15; 14:14}
\crossref{Mark}{3}{11}{Mr 1:23,\allowbreak24; 5:5,\allowbreak6 Mt 8:31 Lu 4:41 Ac 16:17; 19:13-\allowbreak17 Jas 2:19}
\crossref{Mark}{3}{12}{Mr 1:25,\allowbreak34 Mt 12:16 Ac 16:18}
\crossref{Mark}{3}{13}{Mt 10:1 etc.}
\crossref{Mark}{3}{14}{Joh 15:16 Ac 1:24,\allowbreak25 Ga 1:1,\allowbreak15-\allowbreak20}
\crossref{Mark}{3}{15}{}
\crossref{Mark}{3}{16}{Mr 1:16 Mt 16:16-\allowbreak18 Joh 1:42 1Co 1:12; 3:22; 9:5 Ga 2:7-\allowbreak9}
\crossref{Mark}{3}{17}{Mr 1:19,\allowbreak20; 5:37; 9:2; 10:35; 14:33 Joh 21:2,\allowbreak20-\allowbreak25 Ac 12:1}
\crossref{Mark}{3}{18}{Joh 1:40; 6:8; 12:21,\allowbreak22 Ac 1:13}
\crossref{Mark}{3}{19}{Mt 26:14-\allowbreak16,\allowbreak47; 27:3-\allowbreak5 Joh 6:64,\allowbreak71; 12:4-\allowbreak6; 13:2,\allowbreak26-\allowbreak30}
\crossref{Mark}{3}{20}{3:9; 6:31 Lu 6:17 Joh 4:31-\allowbreak34}
\crossref{Mark}{3}{21}{3:31 Joh 7:3-\allowbreak10}
\crossref{Mark}{3}{22}{Mr 7:1 Mt 15:1 Lu 5:17}
\crossref{Mark}{3}{23}{Ps 49:4 Mt 13:34}
\crossref{Mark}{3}{24}{Jud 9:23 etc.}
\crossref{Mark}{3}{25}{Ge 13:7,\allowbreak8; 37:4 Ps 133:1 Ga 5:15 Jas 3:16}
\crossref{Mark}{3}{26}{3:26}
\crossref{Mark}{3}{27}{Ge 3:15 Isa 27:1; 49:24-\allowbreak26; 53:12; 61:1 Mt 12:29 Lu 10:17-\allowbreak20}
\crossref{Mark}{3}{28}{Mt 12:31,\allowbreak32 Lu 12:10 Heb 6:4-\allowbreak8; 10:26-\allowbreak31 1Jo 5:16}
\crossref{Mark}{3}{29}{Mr 12:40 Mt 25:46 2Th 1:9 Jude 1:7,\allowbreak13}
\crossref{Mark}{3}{30}{3:22 Joh 10:20}
\crossref{Mark}{3}{31}{Mt 12:46-\allowbreak48 Lu 8:19-\allowbreak21}
\crossref{Mark}{3}{32}{}
\crossref{Mark}{3}{33}{De 33:9 Lu 2:49 Joh 2:4 2Co 5:16}
\crossref{Mark}{3}{34}{Ps 22:22 So 4:9,\allowbreak10; 5:1,\allowbreak2 Mt 12:49,\allowbreak50; 25:40-\allowbreak45; 28:10}
\crossref{Mark}{3}{35}{Mt 7:21 Joh 7:17 Jas 1:25 1Jo 2:17; 3:22,\allowbreak23}
\crossref{Mark}{4}{1}{Mr 2:13 Mt 13:1,\allowbreak2 etc.}
\crossref{Mark}{4}{2}{4:11,\allowbreak34; 3:23 Ps 49:4; 78:2 Mt 13:3,\allowbreak10,\allowbreak34,\allowbreak35}
\crossref{Mark}{4}{3}{4:9,\allowbreak23; 7:14,\allowbreak16 De 4:1 Ps 34:11; 45:10 Pr 7:24; 8:32}
\crossref{Mark}{4}{4}{4:15 Ge 15:11 Mt 13:4,\allowbreak19 Lu 8:5,\allowbreak12}
\crossref{Mark}{4}{5}{4:16,\allowbreak17 Eze 11:19; 36:26 Ho 10:12 Am 6:12 Mt 13:5,\allowbreak6,\allowbreak20}
\crossref{Mark}{4}{6}{So 1:6 Isa 25:4 Jon 4:8 Jas 1:11 Re 7:16}
\crossref{Mark}{4}{7}{4:18,\allowbreak19 Ge 3:17,\allowbreak18 Jer 4:3 Mt 13:7,\allowbreak22 Lu 8:7,\allowbreak14; 12:15; 21:34}
\crossref{Mark}{4}{8}{4:20 Isa 58:1 Jer 23:29 Mt 13:8,\allowbreak23 Lu 8:8,\allowbreak15 Joh 1:12,\allowbreak13}
\crossref{Mark}{4}{9}{4:3,\allowbreak23,\allowbreak24}
\crossref{Mark}{4}{10}{4:34; 7:17 Pr 13:20 Mt 13:10 etc.}
\crossref{Mark}{4}{11}{Mt 11:25; 13:11,\allowbreak12,\allowbreak16; 16:17 Lu 8:10; 10:21-\allowbreak24 1Co 4:7}
\crossref{Mark}{4}{12}{Jer 31:18-\allowbreak20 Eze 18:27-\allowbreak32 Ac 3:19 2Ti 2:25 Heb 6:6}
\crossref{Mark}{4}{13}{Mr 7:17,\allowbreak18 Mt 13:51,\allowbreak52; 15:15-\allowbreak17; 16:8,\allowbreak9 Lu 24:25}
\crossref{Mark}{4}{14}{4:3 Isa 32:20 Mt 13:19,\allowbreak37 Lu 8:11}
\crossref{Mark}{4}{15}{4:4 Ge 19:14 Isa 53:1 Mt 22:5 Lu 8:12; 14:18,\allowbreak19}
\crossref{Mark}{4}{16}{Mr 6:20; 10:17-\allowbreak22 Eze 33:31,\allowbreak32 Mt 8:19,\allowbreak20; 13:20,\allowbreak21}
\crossref{Mark}{4}{17}{4:5,\allowbreak6 Job 19:28; 27:8-\allowbreak10 Mt 12:31 Lu 12:10 Joh 8:31; 15:2-\allowbreak7}
\crossref{Mark}{4}{18}{4:7 Jer 4:3 Mt 13:22 Lu 8:14}
\crossref{Mark}{4}{19}{Lu 10:41; 12:17-\allowbreak21,\allowbreak29,\allowbreak30; 14:18-\allowbreak20; 21:34 Php 4:6 2Ti 4:10}
\crossref{Mark}{4}{20}{4:8 Mt 13:23 Lu 8:15 Joh 15:4,\allowbreak5 Ro 7:4 Ga 5:22,\allowbreak23}
\crossref{Mark}{4}{21}{Isa 60:1-\allowbreak3 Mt 5:15 Lu 8:16; 11:33 1Co 12:7 Eph 5:3-\allowbreak15}
\crossref{Mark}{4}{22}{Ps 40:9,\allowbreak10; 78:2-\allowbreak4 Ec 12:14 Mt 10:26,\allowbreak27 Lu 8:17; 12:2,\allowbreak3}
\crossref{Mark}{4}{23}{4:9 Mt 11:15 Re 2:7,\allowbreak11,\allowbreak17,\allowbreak29}
\crossref{Mark}{4}{24}{Pr 19:27 Lu 8:18 Ac 17:11 Heb 2:1 1Jo 4:1 1Pe 2:2 2Pe 2:1-\allowbreak3}
\crossref{Mark}{4}{25}{Mt 13:12; 25:28,\allowbreak29 Lu 8:18; 16:9-\allowbreak12; 19:24-\allowbreak26 Joh 15:2}
\crossref{Mark}{4}{26}{Mt 3:2; 4:17; 13:11,\allowbreak31,\allowbreak33 Lu 13:18}
\crossref{Mark}{4}{27}{Ec 8:17; 11:5 Joh 3:7,\allowbreak8 1Co 15:37,\allowbreak38 2Th 1:3 2Pe 3:18}
\crossref{Mark}{4}{28}{Ge 1:11,\allowbreak12; 2:4,\allowbreak5,\allowbreak9; 4:11,\allowbreak12 Isa 61:11}
\crossref{Mark}{4}{29}{Job 5:26 2Ti 4:7,\allowbreak8}
\crossref{Mark}{4}{30}{La 2:13 Mt 11:16 Lu 13:18,\allowbreak20,\allowbreak21}
\crossref{Mark}{4}{31}{Mt 13:31-\allowbreak33 Lu 13:18,\allowbreak19}
\crossref{Mark}{4}{32}{Pr 4:18 Isa 11:9}
\crossref{Mark}{4}{33}{Mt 13:34,\allowbreak35}
\crossref{Mark}{4}{34}{4:10; 7:17-\allowbreak23 Mt 13:36 etc.}
\crossref{Mark}{4}{35}{Mt 8:23 Lu 8:22}
\crossref{Mark}{4}{36}{4:1; 3:9}
\crossref{Mark}{4}{37}{Mt 8:23,\allowbreak24 Lu 8:22,\allowbreak23}
\crossref{Mark}{4}{38}{Joh 4:6 Heb 2:17; 4:15}
\crossref{Mark}{4}{39}{Ex 14:16,\allowbreak22,\allowbreak28,\allowbreak29 Job 38:11 Ps 29:10; 93:3,\allowbreak4; 104:7-\allowbreak9}
\crossref{Mark}{4}{40}{Ps 46:1-\allowbreak3 Isa 42:3; 43:2 Mt 8:26; 14:31 Lu 8:25 Joh 6:19,\allowbreak20}
\crossref{Mark}{4}{41}{Mr 5:33 1Sa 12:18-\allowbreak20,\allowbreak24 Ps 89:7 Jon 1:9,\allowbreak10,\allowbreak15,\allowbreak16 Mal 2:5}
\crossref{Mark}{5}{1}{Mr 4:35 Mt 8:28-\allowbreak34 Lu 8:26 etc.}
\crossref{Mark}{5}{2}{Isa 65:4 Lu 8:27}
\crossref{Mark}{5}{3}{Mr 9:18-\allowbreak22 Isa 65:4 Da 4:32,\allowbreak33 Lu 8:29}
\crossref{Mark}{5}{4}{Jas 3:7,\allowbreak8}
\crossref{Mark}{5}{5}{1Ki 18:28 Job 2:7,\allowbreak8 Joh 8:44}
\crossref{Mark}{5}{6}{Ps 66:3}
\crossref{Mark}{5}{7}{Mr 1:24 Ho 14:8 Mt 8:29 Lu 4:34}
\crossref{Mark}{5}{8}{Mr 1:25; 9:25,\allowbreak26 Ac 16:18}
\crossref{Mark}{5}{9}{Lu 8:30; 11:21-\allowbreak26}
\crossref{Mark}{5}{10}{5:13; 3:22}
\crossref{Mark}{5}{11}{Le 11:7,\allowbreak8 De 14:8 Isa 65:4; 66:3 Mt 8:30 Lu 8:32}
\crossref{Mark}{5}{12}{Job 1:10-\allowbreak12; 2:5 Lu 22:31,\allowbreak32 2Co 2:11 1Pe 5:8}
\crossref{Mark}{5}{13}{1Ki 22:22 Job 1:12; 2:6 Mt 8:32 1Pe 3:22 Re 13:5-\allowbreak7; 20:7}
\crossref{Mark}{5}{14}{Mt 8:33 Lu 8:34}
\crossref{Mark}{5}{15}{5:4 Isa 49:24,\allowbreak25 Mt 9:33; 12:29 Lu 8:35,\allowbreak36; 10:39 Col 1:13}
\crossref{Mark}{5}{16}{5:16}
\crossref{Mark}{5}{17}{5:7; 1:24 Ge 26:16 De 5:25 1Ki 17:18 Job 21:14,\allowbreak15}
\crossref{Mark}{5}{18}{5:7,\allowbreak17 Ps 116:12 Lu 8:38,\allowbreak39; 17:15-\allowbreak17; 23:42,\allowbreak43 Php 1:23,\allowbreak24}
\crossref{Mark}{5}{19}{Ps 66:16 Isa 38:9-\allowbreak20 Da 4:1-\allowbreak3,\allowbreak37; 6:25-\allowbreak27 Jon 2:1 etc.}
\crossref{Mark}{5}{20}{Mr 7:31 Mt 4:25}
\crossref{Mark}{5}{21}{Mt 9:1 Lu 8:40}
\crossref{Mark}{5}{22}{Mt 9:18,\allowbreak19 Lu 8:41,\allowbreak42}
\crossref{Mark}{5}{23}{Mr 7:25-\allowbreak27; 9:21,\allowbreak22 2Sa 12:15,\allowbreak16 Ps 50:15; 107:19 Lu 4:38; 7:2,\allowbreak3,\allowbreak12}
\crossref{Mark}{5}{24}{Lu 7:6 Ac 10:38}
\crossref{Mark}{5}{25}{Mt 9:20-\allowbreak22 Lu 8:43,\allowbreak44}
\crossref{Mark}{5}{26}{Ps 108:12}
\crossref{Mark}{5}{27}{Mr 6:56 2Ki 13:21 Mt 14:36 Ac 5:15; 19:12}
\crossref{Mark}{5}{28}{}
\crossref{Mark}{5}{29}{Ex 15:26 Job 33:24,\allowbreak25 Ps 30:2; 103:3; 107:20; 147:3}
\crossref{Mark}{5}{30}{Lu 6:19; 8:46 1Pe 2:9}
\crossref{Mark}{5}{31}{Lu 8:45; 9:12}
\crossref{Mark}{5}{32}{}
\crossref{Mark}{5}{33}{Mr 4:41 Lu 1:12,\allowbreak29; 8:47}
\crossref{Mark}{5}{34}{Mt 9:2,\allowbreak22 Lu 8:48}
\crossref{Mark}{5}{35}{Lu 8:49}
\crossref{Mark}{5}{36}{5:34; 9:23 2Ch 20:20 Mt 9:28,\allowbreak29; 17:20 Lu 8:50 Joh 4:48-\allowbreak50; 11:40}
\crossref{Mark}{5}{37}{Lu 8:51 Ac 9:40}
\crossref{Mark}{5}{38}{Jer 9:17-\allowbreak20 Mt 9:23,\allowbreak24; 11:17 Lu 8:52,\allowbreak53 Ac 9:39}
\crossref{Mark}{5}{39}{Da 12:2 Joh 11:11-\allowbreak13 Ac 20:10 1Co 11:30 1Th 4:13,\allowbreak14; 5:10}
\crossref{Mark}{5}{40}{Ge 19:14 Ne 2:19 Job 12:4 Ps 22:7; 123:3,\allowbreak4 Lu 16:14 Ac 17:32}
\crossref{Mark}{5}{41}{Mr 1:31 Ac 9:40,\allowbreak41}
\crossref{Mark}{5}{42}{Mr 1:27; 4:41; 6:51; 7:37 Ac 3:10-\allowbreak13}
\crossref{Mark}{5}{43}{Mr 1:43; 3:12; 7:36 Mt 8:4; 9:30; 12:16-\allowbreak18; 17:9 Lu 5:14; 8:56}
\crossref{Mark}{6}{1}{Mt 13:54 etc.}
\crossref{Mark}{6}{2}{Mr 1:21,\allowbreak22,\allowbreak39 Lu 4:15,\allowbreak31,\allowbreak32}
\crossref{Mark}{6}{3}{Mt 13:55,\allowbreak56 Lu 4:22 Joh 6:42}
\crossref{Mark}{6}{4}{Jer 11:21; 12:6 Mt 13:57 Lu 4:24 Joh 4:44}
\crossref{Mark}{6}{5}{Mr 9:23 Ge 19:22; 32:25 Isa 59:1,\allowbreak2 Mt 13:58 Heb 4:2}
\crossref{Mark}{6}{6}{Isa 59:16 Jer 2:11 Mt 8:10 Joh 9:30}
\crossref{Mark}{6}{7}{Mr 3:13,\allowbreak14 Mt 10:1 etc.}
\crossref{Mark}{6}{8}{Mt 10:9,\allowbreak10 Lu 10:4; 22:35}
\crossref{Mark}{6}{9}{Eph 6:15}
\crossref{Mark}{6}{10}{Mt 10:11-\allowbreak13 Lu 9:4; 10:7,\allowbreak8 Ac 16:15; 17:5-\allowbreak7}
\crossref{Mark}{6}{11}{Ne 5:13 Mt 10:14 Lu 9:5; 10:10,\allowbreak11 Ac 13:50,\allowbreak51; 18:6}
\crossref{Mark}{6}{12}{Mr 1:3,\allowbreak15 Eze 18:30 Mt 3:2,\allowbreak8; 4:17; 9:13; 11:20 Lu 11:32}
\crossref{Mark}{6}{13}{6:7 Lu 10:17}
\crossref{Mark}{6}{14}{6:22,\allowbreak26,\allowbreak27 Mt 14:1,\allowbreak2 Lu 3:1; 9:7 etc.}
\crossref{Mark}{6}{15}{Mr 8:28; 9:12,\allowbreak13; 15:35,\allowbreak36 Mal 4:5}
\crossref{Mark}{6}{16}{Ge 40:10,\allowbreak11 Ps 53:5 Mt 14:2; 27:4 Lu 9:9 Re 11:10-\allowbreak13}
\crossref{Mark}{6}{17}{Mt 4:12; 11:2; 14:3 etc.}
\crossref{Mark}{6}{18}{Le 18:16; 20:21 1Ki 22:14 Eze 3:18,\allowbreak19 Mt 14:3,\allowbreak4 Ac 20:26,\allowbreak27}
\crossref{Mark}{6}{19}{Ge 39:17-\allowbreak20 1Ki 21:20}
\crossref{Mark}{6}{20}{Mr 11:18 Ex 11:3 1Ki 21:20 2Ki 3:12,\allowbreak13; 6:21; 13:14}
\crossref{Mark}{6}{21}{Ge 27:41 2Sa 13:23-\allowbreak29 Es 3:7 Ps 37:12,\allowbreak13 Ac 12:2-\allowbreak4}
\crossref{Mark}{6}{22}{Es 1:10-\allowbreak12 Isa 3:16 etc.}
\crossref{Mark}{6}{23}{1Sa 28:10 2Ki 6:31 Mt 5:34-\allowbreak37; 14:7}
\crossref{Mark}{6}{24}{Ge 27:8-\allowbreak11 2Ch 22:3,\allowbreak4 Eze 19:2,\allowbreak3 Mt 14:8}
\crossref{Mark}{6}{25}{Pr 1:16 Ro 3:15}
\crossref{Mark}{6}{26}{Mt 14:9; 27:3-\allowbreak5,\allowbreak24,\allowbreak25}
\crossref{Mark}{6}{27}{Mt 14:10,\allowbreak11}
\crossref{Mark}{6}{28}{}
\crossref{Mark}{6}{29}{1Ki 13:29,\allowbreak30 2Ch 24:16 Mt 14:12; 27:57-\allowbreak60 Ac 8:2}
\crossref{Mark}{6}{30}{6:7 etc.}
\crossref{Mark}{6}{31}{Mr 1:45; 3:7,\allowbreak20 Mt 14:13 Joh 6:1}
\crossref{Mark}{6}{32}{Mt 14:13}
\crossref{Mark}{6}{33}{6:54,\allowbreak55 Mt 15:29-\allowbreak31 Joh 6:2 Jas 1:19}
\crossref{Mark}{6}{34}{Mt 14:14; 15:32 Lu 9:11 Ro 15:2,\allowbreak3 Heb 2:17; 4:15}
\crossref{Mark}{6}{35}{Mt 14:15 etc.}
\crossref{Mark}{6}{36}{Mr 3:21; 5:31 Mt 15:23; 16:22}
\crossref{Mark}{6}{37}{Mr 8:2,\allowbreak3 2Ki 4:42-\allowbreak44 Mt 14:16; 15:32 Lu 9:13 Joh 6:4-\allowbreak10}
\crossref{Mark}{6}{38}{Mr 8:5 Mt 14:17,\allowbreak18; 15:34 Lu 9:13 Joh 6:9}
\crossref{Mark}{6}{39}{1Ki 10:5 Es 1:5,\allowbreak6 Mt 15:35 1Co 14:33,\allowbreak40}
\crossref{Mark}{6}{40}{}
\crossref{Mark}{6}{41}{Mr 7:34 Mt 14:19 Lu 9:16 Joh 11:41; 17:1}
\crossref{Mark}{6}{42}{Mr 8:8,\allowbreak9 De 8:3 2Ki 4:42-\allowbreak44 Ps 145:15,\allowbreak16 Mt 14:20,\allowbreak21; 15:37,\allowbreak38}
\crossref{Mark}{6}{43}{Mr 8:19,\allowbreak20}
\crossref{Mark}{6}{44}{}
\crossref{Mark}{6}{45}{Mt 14:22 etc.}
\crossref{Mark}{6}{46}{Mr 1:35 Mt 6:6; 14:23 Lu 6:12 1Pe 2:21}
\crossref{Mark}{6}{47}{Mt 14:23 Joh 6:16,\allowbreak17}
\crossref{Mark}{6}{48}{Isa 54:11 Joh 1:13 Mt 14:24}
\crossref{Mark}{6}{49}{Job 9:8}
\crossref{Mark}{6}{50}{Isa 43:2 Mt 14:27 Lu 24:38-\allowbreak41 Joh 6:19,\allowbreak20; 20:19,\allowbreak20}
\crossref{Mark}{6}{51}{Mr 4:39 Ps 93:3,\allowbreak4; 107:28-\allowbreak30 Mt 8:26,\allowbreak27; 14:28-\allowbreak32 Lu 8:24,\allowbreak25}
\crossref{Mark}{6}{52}{Mr 7:18; 8:17,\allowbreak18,\allowbreak21 Mt 16:9-\allowbreak11 Lu 24:25}
\crossref{Mark}{6}{53}{Mt 14:34-\allowbreak36 Lu 5:1 Joh 6:24}
\crossref{Mark}{6}{54}{Ps 9:10 Php 3:10}
\crossref{Mark}{6}{55}{Mr 2:1-\allowbreak3; 3:7-\allowbreak11 Mt 4:24}
\crossref{Mark}{6}{56}{Ac 5:15}
\crossref{Mark}{7}{1}{Mr 3:22 Mt 15:1 Lu 5:17; 11:53,\allowbreak54}
\crossref{Mark}{7}{2}{Ac 10:14,\allowbreak15,\allowbreak28}
\crossref{Mark}{7}{3}{7:7-\allowbreak10,\allowbreak13 Mt 15:2-\allowbreak6 Ga 1:14 Col 2:8,\allowbreak21-\allowbreak23 1Pe 1:18}
\crossref{Mark}{7}{4}{Job 9:30,\allowbreak31 Ps 26:6 Isa 1:16 Jer 4:14 Mt 27:24 Lu 11:38,\allowbreak39}
\crossref{Mark}{7}{5}{Mr 2:16-\allowbreak18 Mt 15:2 Ac 21:21,\allowbreak24 Ro 4:12 2Th 3:6,\allowbreak11}
\crossref{Mark}{7}{6}{Isa 29:13 Mt 15:7-\allowbreak9 Ac 28:25}
\crossref{Mark}{7}{7}{1Sa 12:21 Mal 3:14 Mt 6:7; 15:9 1Co 15:14,\allowbreak58 Tit 3:9 Jas 1:26}
\crossref{Mark}{7}{8}{Isa 1:12}
\crossref{Mark}{7}{9}{2Ki 16:10-\allowbreak16 Isa 24:5; 29:13 Jer 44:16,\allowbreak17 Da 7:25; 11:36}
\crossref{Mark}{7}{10}{Mr 10:19 Ex 20:12 De 5:16}
\crossref{Mark}{7}{11}{}
\crossref{Mark}{7}{12}{}
\crossref{Mark}{7}{13}{7:9 Isa 8:20 Jer 8:8,\allowbreak9 Ho 8:12 Mt 5:17-\allowbreak20; 15:6 Tit 1:14}
\crossref{Mark}{7}{14}{1Ki 18:21; 22:28 Ps 49:1,\allowbreak2; 94:8 Mt 15:10 Lu 12:1,\allowbreak54-\allowbreak57}
\crossref{Mark}{7}{15}{7:18-\allowbreak20 Le 11:42-\allowbreak47 Ac 10:14-\allowbreak16,\allowbreak28; 11:8-\allowbreak10; 15:20,\allowbreak21}
\crossref{Mark}{7}{16}{Mr 4:9,\allowbreak23 Mt 11:15 Re 2:7,\allowbreak11,\allowbreak17,\allowbreak29; 3:6,\allowbreak13,\allowbreak22}
\crossref{Mark}{7}{17}{Mr 4:10,\allowbreak34 Mt 13:10,\allowbreak36; 15:15}
\crossref{Mark}{7}{18}{Mr 4:13 Isa 28:9,\allowbreak10 Jer 5:4,\allowbreak5 Mt 15:16,\allowbreak17; 16:11 Lu 24:25}
\crossref{Mark}{7}{19}{Mt 15:17 1Co 6:13 Col 2:21,\allowbreak22}
\crossref{Mark}{7}{20}{7:15 Ps 41:6 Heb 7:6 Mic 2:1 Mt 12:34-\allowbreak37 Jas 1:14,\allowbreak15; 3:6; 4:1}
\crossref{Mark}{7}{21}{Ge 6:5; 8:21 Job 14:4; 15:14-\allowbreak16; 25:4 Ps 14:1,\allowbreak3; 53:1,\allowbreak3; 58:2,\allowbreak3}
\crossref{Mark}{7}{22}{2Ch 32:25,\allowbreak26,\allowbreak31 Ps 10:4 Ob 1:3,\allowbreak4 2Co 10:5 1Pe 5:5}
\crossref{Mark}{7}{23}{7:15,\allowbreak18 1Co 3:17 Tit 1:15 Jude 1:8}
\crossref{Mark}{7}{24}{Mt 15:21 etc.}
\crossref{Mark}{7}{25}{Mt 15:22}
\crossref{Mark}{7}{26}{Isa 49:12 Ga 3:28 Col 3:11}
\crossref{Mark}{7}{27}{Mt 7:6; 10:5; 15:23-\allowbreak28 Ac 22:21 Ro 15:8 Eph 2:12}
\crossref{Mark}{7}{28}{Ps 145:16 Isa 45:22; 49:6 Mt 5:45 Lu 7:6-\allowbreak8; 15:30-\allowbreak32 Ac 11:17,\allowbreak18}
\crossref{Mark}{7}{29}{Isa 57:15,\allowbreak16; 66:2 Mt 5:3; 8:9-\allowbreak13 1Jo 3:8}
\crossref{Mark}{7}{30}{Joh 4:50-\allowbreak52}
\crossref{Mark}{7}{31}{7:24 Mt 15:29 etc.}
\crossref{Mark}{7}{32}{Mt 9:32,\allowbreak33 Lu 11:14}
\crossref{Mark}{7}{33}{Mr 5:40; 8:23 1Ki 17:19-\allowbreak22 2Ki 4:4-\allowbreak6,\allowbreak33,\allowbreak34 Joh 9:6,\allowbreak7}
\crossref{Mark}{7}{34}{Mr 6:41 Joh 11:41; 17:1}
\crossref{Mark}{7}{35}{Mr 2:12 Ps 33:9 Isa 32:3,\allowbreak4; 35:5,\allowbreak6 Mt 11:5}
\crossref{Mark}{7}{36}{Mr 1:44,\allowbreak45; 3:12; 5:43; 8:26}
\crossref{Mark}{7}{37}{Mr 1:27; 2:12; 4:41; 5:42; 6:51 Ps 139:14 Ac 2:7-\allowbreak12; 3:10-\allowbreak13; 14:11}
\crossref{Mark}{8}{1}{Mt 15:32 etc.}
\crossref{Mark}{8}{2}{Mr 1:41; 5:19; 6:34; 9:22 Ps 103:13; 145:8,\allowbreak15 Mic 7:19 Mt 9:36}
\crossref{Mark}{8}{3}{Jud 8:4-\allowbreak6 1Sa 14:28-\allowbreak31; 30:10-\allowbreak12 Isa 40:31}
\crossref{Mark}{8}{4}{Mr 6:36,\allowbreak37,\allowbreak52 Nu 11:21-\allowbreak23 2Ki 4:42-\allowbreak44; 7:2 Ps 78:19,\allowbreak20}
\crossref{Mark}{8}{5}{Mr 6:38 Mt 14:15-\allowbreak17; 15:34 Lu 9:13}
\crossref{Mark}{8}{6}{Mr 6:39,\allowbreak40 Mt 14:18,\allowbreak19; 15:35,\allowbreak36 Lu 9:14,\allowbreak15; 12:37 Joh 2:5; 6:10}
\crossref{Mark}{8}{7}{Lu 24:41,\allowbreak42 Joh 21:5,\allowbreak8,\allowbreak9}
\crossref{Mark}{8}{8}{1Ki 17:14-\allowbreak16 2Ki 4:2-\allowbreak7,\allowbreak42-\allowbreak44}
\crossref{Mark}{8}{9}{}
\crossref{Mark}{8}{10}{Mt 15:39}
\crossref{Mark}{8}{11}{Mr 2:16; 7:1,\allowbreak2 Mt 12:38; 16:1-\allowbreak4; 19:3; 21:23; 22:15,\allowbreak18,\allowbreak23,\allowbreak34,\allowbreak35}
\crossref{Mark}{8}{12}{Mr 3:5; 7:34; 9:19 Isa 53:3 Lu 19:41 Joh 11:33-\allowbreak38}
\crossref{Mark}{8}{13}{Ps 81:12 Jer 23:33 Ho 4:17; 9:12 Zec 11:8,\allowbreak9 Mt 7:6; 15:14}
\crossref{Mark}{8}{14}{Mt 16:5}
\crossref{Mark}{8}{15}{Nu 27:19-\allowbreak23 1Ch 28:9,\allowbreak10,\allowbreak20 1Ti 5:21; 6:13 2Ti 2:14}
\crossref{Mark}{8}{16}{Mt 16:7,\allowbreak8 Lu 9:46; 20:5}
\crossref{Mark}{8}{17}{Mr 2:8 Joh 2:24,\allowbreak25; 16:30; 21:17 Heb 4:12,\allowbreak13 Re 2:23}
\crossref{Mark}{8}{18}{Mr 4:12 De 29:4 Ps 69:23; 115:5-\allowbreak8 Isa 6:9,\allowbreak10; 42:18-\allowbreak20; 44:18}
\crossref{Mark}{8}{19}{Mr 6:38-\allowbreak44 Mt 14:17-\allowbreak21 Lu 9:12-\allowbreak17 Joh 6:5-\allowbreak13}
\crossref{Mark}{8}{20}{8:1-\allowbreak9 Mt 15:34-\allowbreak38}
\crossref{Mark}{8}{21}{8:12,\allowbreak17; 6:52; 9:19 Ps 94:8 Mt 16:11,\allowbreak12 Joh 14:9 1Co 6:5; 15:34}
\crossref{Mark}{8}{22}{Mr 6:45 Mt 11:21 Lu 9:10; 10:13 Joh 1:44; 12:21}
\crossref{Mark}{8}{23}{Isa 51:18 Jer 31:32 Ac 9:8 Heb 8:9}
\crossref{Mark}{8}{24}{Jud 9:36 Isa 29:18; 32:3 1Co 13:9-\allowbreak12}
\crossref{Mark}{8}{25}{Pr 4:18 Mt 13:12 Php 1:6 1Pe 2:9 2Pe 3:18}
\crossref{Mark}{8}{26}{Mr 5:43; 7:36 Mt 8:4; 9:30; 12:16}
\crossref{Mark}{8}{27}{Mt 16:13 etc.}
\crossref{Mark}{8}{28}{Mr 6:14-\allowbreak16 Mt 14:2; 16:14 Lu 9:7-\allowbreak9}
\crossref{Mark}{8}{29}{Mr 4:11 Mt 16:15 Lu 9:20 1Pe 2:7}
\crossref{Mark}{8}{30}{8:26; 7:36; 9:9 Mt 16:20 Lu 9:21}
\crossref{Mark}{8}{31}{Mr 9:31,\allowbreak32; 10:33,\allowbreak34 Mt 16:21; 17:22; 20:17-\allowbreak19 Lu 9:22; 18:31-\allowbreak34}
\crossref{Mark}{8}{32}{Joh 16:25,\allowbreak29}
\crossref{Mark}{8}{33}{Mr 3:5,\allowbreak34 Lu 22:61}
\crossref{Mark}{8}{34}{Mr 7:14 Lu 9:23; 20:45}
\crossref{Mark}{8}{35}{Es 4:11-\allowbreak16 Jer 26:20-\allowbreak24 Mt 10:39; 16:25 Lu 9:24; 17:33}
\crossref{Mark}{8}{36}{Job 2:4 Ps 49:17; 73:18-\allowbreak20 Mt 4:8-\allowbreak10; 16:26 Lu 9:25; 12:19,\allowbreak20}
\crossref{Mark}{8}{37}{Ps 49:7,\allowbreak8 1Pe 1:18,\allowbreak19}
\crossref{Mark}{8}{38}{Mt 10:32,\allowbreak33 Lu 19:26; 12:8,\allowbreak9 Ac 5:41 Ro 1:16 Ga 6:14}
\crossref{Mark}{9}{1}{Mt 16:28 Lu 9:27}
\crossref{Mark}{9}{2}{Mt 17:11 etc.}
\crossref{Mark}{9}{3}{Ps 104:1,\allowbreak2 Da 7:9 Mt 28:3 Ac 10:30}
\crossref{Mark}{9}{4}{Mt 11:13; 17:3,\allowbreak4 Lu 9:19,\allowbreak30,\allowbreak31; 24:27,\allowbreak44 Joh 5:39,\allowbreak45-\allowbreak47}
\crossref{Mark}{9}{5}{Ex 33:17-\allowbreak23 Ps 62:2,\allowbreak3; 84:10 Joh 14:8,\allowbreak9,\allowbreak21-\allowbreak23 Php 1:23 1Jo 3:2}
\crossref{Mark}{9}{6}{Mr 16:5-\allowbreak8 Da 10:15-\allowbreak19 Re 1:17}
\crossref{Mark}{9}{7}{Ex 40:34 1Ki 8:10-\allowbreak12 Ps 97:2 Da 7:13 Mt 17:5-\allowbreak7; 26:64}
\crossref{Mark}{9}{8}{Lu 9:36; 24:31 Ac 8:39,\allowbreak40; 10:16}
\crossref{Mark}{9}{9}{Mr 5:43; 8:29,\allowbreak30 Mt 12:19; 17:9}
\crossref{Mark}{9}{10}{Ge 37:11 Lu 2:50,\allowbreak51; 24:7,\allowbreak8 Joh 16:17-\allowbreak19}
\crossref{Mark}{9}{11}{9:4 Mal 3:1; 4:5 Mt 11:14; 17:10,\allowbreak11}
\crossref{Mark}{9}{12}{Mr 1:2-\allowbreak8 Isa 40:3-\allowbreak5 Mal 4:6 Mt 3:1 etc.}
\crossref{Mark}{9}{13}{Mt 11:14; 17:12,\allowbreak13 Lu 1:17}
\crossref{Mark}{9}{14}{Mt 17:14 etc.}
\crossref{Mark}{9}{15}{9:2,\allowbreak3 Ex 34:30}
\crossref{Mark}{9}{16}{Mr 8:11 Lu 5:30-\allowbreak32}
\crossref{Mark}{9}{17}{Mr 5:23; 7:26; 10:13 Mt 17:15 Lu 9:38 Joh 4:47}
\crossref{Mark}{9}{18}{9:26 Mt 15:22 Lu 9:39}
\crossref{Mark}{9}{19}{Mr 16:14 Nu 14:11,\allowbreak22,\allowbreak27; 32:13,\allowbreak14 De 32:20 Ps 78:6-\allowbreak8,\allowbreak22; 106:21-\allowbreak25}
\crossref{Mark}{9}{20}{9:18,\allowbreak26; 1:26; 5:3-\allowbreak5 Job 1:10 etc.}
\crossref{Mark}{9}{21}{Mr 5:25 Job 5:7; 14:1 Ps 51:5 Lu 8:43; 13:16 Joh 5:5,\allowbreak6; 9:1,\allowbreak20,\allowbreak21}
\crossref{Mark}{9}{22}{Mr 1:40-\allowbreak42 Mt 8:2,\allowbreak8,\allowbreak9; 9:28; 14:31}
\crossref{Mark}{9}{23}{Mr 11:23 2Ch 20:20 Mt 17:20; 21:21,\allowbreak22 Lu 17:6 Joh 4:48-\allowbreak50; 11:40}
\crossref{Mark}{9}{24}{2Sa 16:12}
\crossref{Mark}{9}{25}{Mr 1:25-\allowbreak27; 5:7,\allowbreak8 Zec 3:2 Mt 17:18 Lu 4:35,\allowbreak41; 9:42 Jude 1:9}
\crossref{Mark}{9}{26}{9:18,\allowbreak20; 1:26 Ex 5:23 Re 12:12}
\crossref{Mark}{9}{27}{Mr 1:31,\allowbreak41; 5:41; 8:23 Isa 41:13 Ac 3:7; 9:41}
\crossref{Mark}{9}{28}{Mr 4:10,\allowbreak34 Mt 13:10,\allowbreak36; 15:15}
\crossref{Mark}{9}{29}{Mt 12:45 Lu 11:26}
\crossref{Mark}{9}{30}{Mt 27:22,\allowbreak23}
\crossref{Mark}{9}{31}{9:12; 8:31 Mt 16:21; 20:18,\allowbreak19,\allowbreak28; 21:38,\allowbreak39; 26:2 Lu 9:44; 18:31-\allowbreak33}
\crossref{Mark}{9}{32}{9:10 Lu 2:50; 9:45; 18:34; 24:45}
\crossref{Mark}{9}{33}{Mt 17:24}
\crossref{Mark}{9}{34}{Mt 18:1 etc.}
\crossref{Mark}{9}{35}{Mr 10:42-\allowbreak45 Pr 13:10 Jer 45:5 Mt 20:25-\allowbreak28 Lu 14:10,\allowbreak11; 18:14}
\crossref{Mark}{9}{36}{Mr 10:16 Mt 18:2; 19:14,\allowbreak15}
\crossref{Mark}{9}{37}{Mt 10:40-\allowbreak42; 18:3-\allowbreak5,\allowbreak10; 25:40 Lu 9:48}
\crossref{Mark}{9}{38}{Nu 11:26-\allowbreak29 Lu 9:49,\allowbreak50; 11:19}
\crossref{Mark}{9}{39}{Mr 10:13,\allowbreak14 Mt 13:28,\allowbreak29 Php 1:18}
\crossref{Mark}{9}{40}{Mt 12:30 Lu 11:23}
\crossref{Mark}{9}{41}{Mt 10:42; 25:40}
\crossref{Mark}{9}{42}{Mt 18:6,\allowbreak10 Lu 17:1,\allowbreak2 Ro 14:13; 15:21; 16:17 1Co 8:10-\allowbreak13}
\crossref{Mark}{9}{43}{De 13:6-\allowbreak8 Mt 5:29,\allowbreak30; 18:8,\allowbreak9 Ro 8:13 1Co 9:27 Ga 5:24 Col 3:5}
\crossref{Mark}{9}{44}{9:46,\allowbreak48 Isa 66:24}
\crossref{Mark}{9}{45}{9:43,\allowbreak44}
\crossref{Mark}{9}{46}{Lu 16:24-\allowbreak26}
\crossref{Mark}{9}{47}{Ge 3:6 Job 31:1 Ps 119:37 Mt 5:28,\allowbreak29; 10:37-\allowbreak39 Lu 14:26}
\crossref{Mark}{9}{48}{9:44,\allowbreak46}
\crossref{Mark}{9}{49}{Le 2:13 Eze 43:24}
\crossref{Mark}{9}{50}{Job 6:6 Mt 5:13 Lu 14:34,\allowbreak35}
\crossref{Mark}{10}{1}{Mt 19:1 etc.}
\crossref{Mark}{10}{2}{Mr 8:15 Mt 9:34; 15:12; 23:13 Lu 5:30; 6:7; 7:30; 11:39,\allowbreak53,\allowbreak54; 16:14}
\crossref{Mark}{10}{3}{Isa 8:20 Lu 10:25 Joh 5:39 Ga 4:21}
\crossref{Mark}{10}{4}{De 24:1-\allowbreak4 Isa 50:1 Jer 3:1 Mt 1:19; 5:31,\allowbreak32; 19:7}
\crossref{Mark}{10}{5}{De 9:6; 31:27 Ne 9:16,\allowbreak17,\allowbreak26 Mt 19:8 Ac 7:51 Heb 3:7-\allowbreak10}
\crossref{Mark}{10}{6}{Ge 1:1 2Pe 3:4}
\crossref{Mark}{10}{7}{Ge 2:24 Mt 19:5,\allowbreak6 Eph 5:31}
\crossref{Mark}{10}{8}{1Co 6:16 Eph 5:28}
\crossref{Mark}{10}{9}{Ro 7:1-\allowbreak3 1Ch 7:10-\allowbreak13}
\crossref{Mark}{10}{10}{Mr 4:10; 9:28,\allowbreak33}
\crossref{Mark}{10}{11}{Mt 5:31,\allowbreak32; 19:9 Lu 16:18 Ro 7:3 1Co 7:4,\allowbreak10,\allowbreak11 Heb 13:4}
\crossref{Mark}{10}{12}{}
\crossref{Mark}{10}{13}{Mt 19:13-\allowbreak15 Lu 18:15,\allowbreak16}
\crossref{Mark}{10}{14}{Mr 3:5; 8:33 Lu 9:54-\allowbreak56 Eph 4:26}
\crossref{Mark}{10}{15}{Mt 18:3 Lu 18:17 Joh 3:3-\allowbreak6}
\crossref{Mark}{10}{16}{Ge 48:14-\allowbreak16 De 28:3 Isa 40:11 Lu 2:28-\allowbreak34; 24:50,\allowbreak51 Joh 21:15-\allowbreak17}
\crossref{Mark}{10}{17}{Mt 19:16 etc.}
\crossref{Mark}{10}{18}{Mt 19:17 Lu 18:19 Joh 5:41-\allowbreak44 Ro 3:12}
\crossref{Mark}{10}{19}{Mr 12:28-\allowbreak34 Isa 8:20 Mt 5:17-\allowbreak20; 19:17-\allowbreak19 Lu 10:26-\allowbreak28; 18:20}
\crossref{Mark}{10}{20}{Isa 58:2 Eze 5:14; 33:31,\allowbreak32 Mal 3:8 Mt 19:20 Lu 10:29; 18:11,\allowbreak12}
\crossref{Mark}{10}{21}{Ge 34:19 Isa 63:8-\allowbreak10 Lu 19:41 2Co 12:15}
\crossref{Mark}{10}{22}{Mr 6:20,\allowbreak26 Mt 19:22; 27:3,\allowbreak24-\allowbreak26 Lu 18:23 2Co 7:10 2Ti 4:10}
\crossref{Mark}{10}{23}{Mr 3:5; 5:32}
\crossref{Mark}{10}{24}{Mt 19:25 Lu 18:26,\allowbreak27 Joh 6:60}
\crossref{Mark}{10}{25}{Jer 13:23 Mt 7:3-\allowbreak5; 19:24,\allowbreak25; 23:24 Lu 18:25}
\crossref{Mark}{10}{26}{Mr 6:51; 7:37 2Co 11:23}
\crossref{Mark}{10}{27}{Ge 18:13,\allowbreak14 Nu 11:21-\allowbreak23 2Ki 7:2 Zec 8:6 Mt 19:26 Lu 18:27}
\crossref{Mark}{10}{28}{Mr 1:16-\allowbreak20 Mt 19:27-\allowbreak30 Lu 14:33; 18:28-\allowbreak30 Php 3:7-\allowbreak9}
\crossref{Mark}{10}{29}{Ge 12:1-\allowbreak3; 45:20 De 33:9-\allowbreak11 Lu 22:28-\allowbreak30 Heb 11:24-\allowbreak26}
\crossref{Mark}{10}{30}{2Ch 25:9 Ps 84:11 Pr 3:9,\allowbreak10; 16:16 Mal 3:10 Mt 13:44-\allowbreak46}
\crossref{Mark}{10}{31}{Mt 8:11,\allowbreak12; 19:30; 20:16; 21:31 Lu 7:29,\allowbreak30,\allowbreak40-\allowbreak47; 13:30; 18:11-\allowbreak14}
\crossref{Mark}{10}{32}{Mt 20:17 etc.}
\crossref{Mark}{10}{33}{Ac 20:22}
\crossref{Mark}{10}{34}{Mr 14:65; 15:17-\allowbreak20,\allowbreak29-\allowbreak31 Ps 22:6-\allowbreak8,\allowbreak13 Isa 53:3 Mt 27:27-\allowbreak44}
\crossref{Mark}{10}{35}{Mt 20:20 etc.}
\crossref{Mark}{10}{36}{10:51 1Ki 3:5 etc.}
\crossref{Mark}{10}{37}{Mr 16:19 1Ki 22:19 Ps 45:9; 110:1}
\crossref{Mark}{10}{38}{1Ki 2:22 Jer 45:5 Mt 20:21,\allowbreak22 Ro 8:26 Jas 4:3}
\crossref{Mark}{10}{39}{Mr 14:31 Joh 13:37}
\crossref{Mark}{10}{40}{Mt 20:23; 25:34 Joh 17:2,\allowbreak24 Heb 11:16}
\crossref{Mark}{10}{41}{Mr 9:33-\allowbreak36 Pr 13:10 Mt 20:24 Lu 22:24 Ro 12:10 Php 2:3}
\crossref{Mark}{10}{42}{Mt 20:25 Lu 22:25 1Pe 5:3}
\crossref{Mark}{10}{43}{Joh 18:36 Ro 12:2}
\crossref{Mark}{10}{44}{}
\crossref{Mark}{10}{45}{Mt 20:28 Lu 22:26,\allowbreak27 Joh 13:14 Php 2:5-\allowbreak8 Heb 5:8}
\crossref{Mark}{10}{46}{Mt 20:29 etc.}
\crossref{Mark}{10}{47}{Mt 2:23; 21:11; 26:71 Lu 4:16; 18:36,\allowbreak37 Joh 1:46; 7:41,\allowbreak52; 19:19}
\crossref{Mark}{10}{48}{Mr 5:35 Mt 19:13; 20:31 Lu 18:39}
\crossref{Mark}{10}{49}{Ps 86:15; 145:8 Mt 20:32-\allowbreak34 Lu 18:40 Heb 2:17; 4:15}
\crossref{Mark}{10}{50}{Php 3:7-\allowbreak9 Heb 12:1}
\crossref{Mark}{10}{51}{10:36 2Ch 1:7 Mt 6:8; 7:7 Lu 18:41-\allowbreak43 Php 4:6}
\crossref{Mark}{10}{52}{Mr 5:34 Mt 9:22,\allowbreak28-\allowbreak30; 15:28 Lu 7:50; 9:48}
\crossref{Mark}{11}{1}{Mt 21:1 etc.}
\crossref{Mark}{11}{2}{Mt 21:2,\allowbreak3 Lu 19:30,\allowbreak31}
\crossref{Mark}{11}{3}{Ps 24:1 Ac 10:36; 17:25 2Co 8:9 Heb 2:7-\allowbreak9}
\crossref{Mark}{11}{4}{Mt 21:6,\allowbreak7; 26:19 Lu 19:32-\allowbreak34 Joh 2:5 Heb 11:8}
\crossref{Mark}{11}{5}{11:5}
\crossref{Mark}{11}{6}{}
\crossref{Mark}{11}{7}{Zec 9:9 Mt 21:4,\allowbreak5 Lu 19:35}
\crossref{Mark}{11}{8}{Le 23:40}
\crossref{Mark}{11}{9}{Ps 118:25,\allowbreak26 Mt 21:9; 23:39 Lu 19:37,\allowbreak38 Joh 12:13; 19:15}
\crossref{Mark}{11}{10}{Isa 9:6,\allowbreak7 Jer 33:15-\allowbreak17,\allowbreak26 Eze 34:23,\allowbreak24; 37:24,\allowbreak25 Ho 3:5}
\crossref{Mark}{11}{11}{Mal 3:1 Mt 21:10-\allowbreak16 Lu 19:41-\allowbreak45}
\crossref{Mark}{11}{12}{Mt 21:18 etc.}
\crossref{Mark}{11}{13}{Mt 21:19 Lu 13:6-\allowbreak9}
\crossref{Mark}{11}{14}{11:20,\allowbreak21 Isa 5:5,\allowbreak6 Mt 3:10; 7:19; 12:33-\allowbreak35; 21:19,\allowbreak33,\allowbreak44}
\crossref{Mark}{11}{15}{Mt 21:12-\allowbreak16 Lu 19:45 Joh 2:13-\allowbreak17}
\crossref{Mark}{11}{16}{}
\crossref{Mark}{11}{17}{1Ki 8:41-\allowbreak48 Isa 56:7; 60:7 Lu 19:46}
\crossref{Mark}{11}{18}{Mr 3:6; 12:12; 14:1,\allowbreak2 Isa 49:7 Mt 21:15,\allowbreak38,\allowbreak39,\allowbreak45,\allowbreak46; 26:3,\allowbreak4 Lu 19:47}
\crossref{Mark}{11}{19}{11:11 Lu 21:37 Joh 12:36}
\crossref{Mark}{11}{20}{11:14 Job 18:16,\allowbreak17; 20:5-\allowbreak7 Isa 5:4; 40:24 Mt 13:6; 15:13; 21:19,\allowbreak20}
\crossref{Mark}{11}{21}{Pr 3:33 Zec 5:3,\allowbreak4 Mt 25:41 1Co 16:22}
\crossref{Mark}{11}{22}{Mr 9:23 2Ch 20:20 Ps 62:8 Isa 7:9 Joh 14:1 Tit 1:1}
\crossref{Mark}{11}{23}{Mt 17:20; 21:21 Lu 17:6 1Co 13:2}
\crossref{Mark}{11}{24}{Mt 7:7-\allowbreak11; 18:19; 21:22 Lu 11:9-\allowbreak13; 18:1-\allowbreak8 Joh 14:13; 15:7}
\crossref{Mark}{11}{25}{Zec 3:1 Lu 18:11 Re 11:4}
\crossref{Mark}{11}{26}{}
\crossref{Mark}{11}{27}{Mal 3:1 Mt 21:23-\allowbreak27 Lu 20:1-\allowbreak8 Joh 10:23; 18:20}
\crossref{Mark}{11}{28}{Ex 2:14 Nu 16:3,\allowbreak13 Ac 7:27,\allowbreak28,\allowbreak38,\allowbreak39,\allowbreak51}
\crossref{Mark}{11}{29}{Isa 52:13 Mt 21:24 Lu 20:3-\allowbreak8}
\crossref{Mark}{11}{30}{Mr 1:1-\allowbreak11; 9:13 Mt 3:1-\allowbreak17 Lu 3:1-\allowbreak20 Joh 1:6-\allowbreak8,\allowbreak15-\allowbreak36; 3:25-\allowbreak36}
\crossref{Mark}{11}{31}{Mt 11:7-\allowbreak14; 21:25-\allowbreak27,\allowbreak31,\allowbreak32 Joh 1:15,\allowbreak29,\allowbreak34,\allowbreak36; 3:29-\allowbreak36}
\crossref{Mark}{11}{32}{Mr 6:20; 12:12 Mt 14:5; 21:46 Lu 20:19; 22:2 Ac 5:26}
\crossref{Mark}{11}{33}{Isa 1:3; 6:9,\allowbreak10; 29:9-\allowbreak14; 42:19,\allowbreak20; 56:10 Jer 8:7-\allowbreak9 Ho 4:6}
\crossref{Mark}{12}{1}{Mr 4:2,\allowbreak11-\allowbreak13,\allowbreak33,\allowbreak34 Eze 20:49 Mt 13:10-\allowbreak15,\allowbreak34,\allowbreak35; 21:28-\allowbreak33; 22:1-\allowbreak14}
\crossref{Mark}{12}{2}{Ps 1:3 Mt 21:34 Lu 20:10}
\crossref{Mark}{12}{3}{1Ki 18:4,\allowbreak13; 19:10,\allowbreak14; 22:27 2Ch 16:10; 24:19-\allowbreak21; 36:16 Ne 9:26}
\crossref{Mark}{12}{4}{}
\crossref{Mark}{12}{5}{Mr 9:13 Ne 9:30 Jer 7:25 etc.}
\crossref{Mark}{12}{6}{Ps 2:7 Mt 1:23; 11:27; 26:63 Joh 1:14,\allowbreak18,\allowbreak34,\allowbreak49; 3:16-\allowbreak18}
\crossref{Mark}{12}{7}{12:12 Ge 3:15; 37:20 Ps 2:2,\allowbreak3; 22:12-\allowbreak15 Isa 49:7; 53:7,\allowbreak8}
\crossref{Mark}{12}{8}{Mt 21:33,\allowbreak39 Lu 20:15 Heb 13:11-\allowbreak13}
\crossref{Mark}{12}{9}{Mt 21:40,\allowbreak41}
\crossref{Mark}{12}{10}{12:26; 2:25; 13:14 Mt 12:3; 19:4; 21:16; 22:31 Lu 6:3}
\crossref{Mark}{12}{11}{Nu 23:23 Hab 1:5 Ac 2:12,\allowbreak32-\allowbreak36; 3:12-\allowbreak16; 13:40,\allowbreak41 Eph 3:8-\allowbreak11}
\crossref{Mark}{12}{12}{Mr 11:18,\allowbreak32 Mt 21:26,\allowbreak45,\allowbreak46 Lu 20:6,\allowbreak19 Joh 7:25,\allowbreak30,\allowbreak44}
\crossref{Mark}{12}{13}{Ps 38:12; 56:5,\allowbreak6; 140:5 Isa 29:21 Jer 18:18 Mt 22:15,\allowbreak16 Lu 11:54}
\crossref{Mark}{12}{14}{Mr 14:45 Ps 12:2-\allowbreak4; 55:21; 120:2 Pr 26:23-\allowbreak26 Jer 42:2,\allowbreak3,\allowbreak20}
\crossref{Mark}{12}{15}{Mt 22:18 Lu 20:23 Joh 2:24,\allowbreak25; 21:17 Heb 4:13 Re 2:23}
\crossref{Mark}{12}{16}{Mt 22:19-\allowbreak22 Lu 20:24-\allowbreak26 2Ti 2:19 Re 3:12}
\crossref{Mark}{12}{17}{Pr 24:21 Mt 17:25-\allowbreak27 Ro 13:7 1Pe 2:17}
\crossref{Mark}{12}{18}{Mt 22:23 etc.}
\crossref{Mark}{12}{19}{Ge 38:8 De 25:5-\allowbreak10 Ru 4:5}
\crossref{Mark}{12}{20}{Mt 22:25-\allowbreak28 Lu 20:29-\allowbreak33}
\crossref{Mark}{12}{21}{12:21}
\crossref{Mark}{12}{22}{12:22}
\crossref{Mark}{12}{23}{}
\crossref{Mark}{12}{24}{Job 19:25-\allowbreak27 Isa 25:8; 26:19 Eze 37:1-\allowbreak14 Da 12:2 Ho 6:2; 13:14}
\crossref{Mark}{12}{25}{Mt 22:30 Lu 20:35,\allowbreak36 1Co 15:42-\allowbreak54 Heb 12:22,\allowbreak23 1Jo 3:2}
\crossref{Mark}{12}{26}{12:10 Mt 22:31,\allowbreak32}
\crossref{Mark}{12}{27}{Ro 4:17; 14:9 Heb 11:13-\allowbreak16}
\crossref{Mark}{12}{28}{Mt 22:34-\allowbreak40}
\crossref{Mark}{12}{29}{12:32,\allowbreak33 De 6:4; 10:12; 30:6 Pr 23:26 Mt 10:37 Lu 10:27 1Ti 1:5}
\crossref{Mark}{12}{30}{}
\crossref{Mark}{12}{31}{Le 19:13 Mt 7:12; 19:18,\allowbreak19; 22:39 Lu 10:27,\allowbreak36,\allowbreak37 Ro 13:8,\allowbreak9}
\crossref{Mark}{12}{32}{De 4:39; 5:7; 6:4 Isa 44:8; 45:5,\allowbreak6,\allowbreak14,\allowbreak18,\allowbreak21,\allowbreak22; 46:9 Jer 10:10-\allowbreak12}
\crossref{Mark}{12}{33}{1Sa 15:22 Ps 50:8-\allowbreak15,\allowbreak23 Pr 21:3 Isa 1:11-\allowbreak17; 58:5-\allowbreak7}
\crossref{Mark}{12}{34}{Mt 12:20 Ro 3:20; 7:9 Ga 2:19}
\crossref{Mark}{12}{35}{Mr 11:27 Lu 19:47; 20:1; 21:37 Joh 18:20}
\crossref{Mark}{12}{36}{2Sa 23:2 Ne 9:30 Mt 22:43-\allowbreak45 Ac 1:16; 28:25 2Ti 3:16 Heb 3:7,\allowbreak8}
\crossref{Mark}{12}{37}{Mt 1:23 Ro 1:3,\allowbreak4; 9:5 1Ti 3:16 Re 22:16}
\crossref{Mark}{12}{38}{Mr 4:2}
\crossref{Mark}{12}{39}{Jas 2:2,\allowbreak3}
\crossref{Mark}{12}{40}{Eze 22:25 Mic 2:2; 3:1-\allowbreak4 Mt 23:14 Lu 20:47 2Ti 3:6}
\crossref{Mark}{12}{41}{Mt 27:6 Lu 21:2 etc.}
\crossref{Mark}{12}{42}{}
\crossref{Mark}{12}{43}{Ex 35:21-\allowbreak29 Mt 10:42 Ac 11:29 2Co 8:2,\allowbreak12; 9:6-\allowbreak8}
\crossref{Mark}{12}{44}{Mr 14:8 1Ch 29:2-\allowbreak17 2Ch 24:10-\allowbreak14; 31:5-\allowbreak10; 35:7,\allowbreak8 Ezr 2:68,\allowbreak69}
\crossref{Mark}{13}{1}{Mt 24:1 etc.}
\crossref{Mark}{13}{2}{1Ki 9:7,\allowbreak8 2Ch 7:20,\allowbreak21 Jer 26:18 Mic 3:12 Mt 24:2 Lu 19:41-\allowbreak44}
\crossref{Mark}{13}{3}{Mt 24:3}
\crossref{Mark}{13}{4}{Da 12:6,\allowbreak8 Mt 24:3 Lu 21:7 Joh 21:21,\allowbreak22 Ac 1:6,\allowbreak7}
\crossref{Mark}{13}{5}{Jer 29:8 Mt 24:4,\allowbreak5 Lu 21:8 1Co 15:33 Eph 5:6 Col 2:8}
\crossref{Mark}{13}{6}{13:22 Jer 14:14; 23:21-\allowbreak25 Joh 5:43 1Jo 4:1}
\crossref{Mark}{13}{7}{Ps 27:3; 46:1-\allowbreak3; 112:7 Pr 3:25 Isa 8:12 Jer 4:19-\allowbreak21; 51:46}
\crossref{Mark}{13}{8}{2Ch 15:6 Isa 19:2 Jer 25:32 Hag 2:22 Zec 14:13 Re 6:4}
\crossref{Mark}{13}{9}{13:5 Mt 10:17,\allowbreak18; 23:34-\allowbreak37; 24:9,\allowbreak10 Lu 21:16-\allowbreak18 Joh 15:20; 16:2}
\crossref{Mark}{13}{10}{Mr 16:15 Mt 24:14; 28:18,\allowbreak19 Ro 1:8; 10:18; 15:19 Col 1:6,\allowbreak23 Re 14:6}
\crossref{Mark}{13}{11}{13:9 Mt 10:17,\allowbreak21 Ac 3:13}
\crossref{Mark}{13}{12}{Eze 38:21 Mic 7:4-\allowbreak6 Mt 10:21; 24:10 Lu 12:52,\allowbreak53; 21:16}
\crossref{Mark}{13}{13}{Mt 5:11,\allowbreak12; 24:9 Lu 6:22; 21:17 Joh 15:18,\allowbreak19; 17:14 1Jo 3:13}
\crossref{Mark}{13}{14}{Da 8:13; 9:27; 12:11 Mt 24:15 etc.}
\crossref{Mark}{13}{15}{Ge 19:15-\allowbreak17,\allowbreak22,\allowbreak26 Job 2:4 Pr 6:4,\allowbreak5; 22:3 Mt 24:16-\allowbreak18}
\crossref{Mark}{13}{16}{13:16}
\crossref{Mark}{13}{17}{De 28:56,\allowbreak57 La 2:19,\allowbreak20; 4:3,\allowbreak4,\allowbreak10 Ho 9:14; 13:16 Mt 24:19-\allowbreak21}
\crossref{Mark}{13}{18}{}
\crossref{Mark}{13}{19}{De 28:59; 29:22-\allowbreak28 Isa 65:12-\allowbreak15 La 1:12; 2:13; 4:6 Da 9:12,\allowbreak26}
\crossref{Mark}{13}{20}{Isa 1:9; 6:13; 65:8,\allowbreak9 Zec 13:8,\allowbreak9 Mt 24:22 Ro 11:5-\allowbreak7,\allowbreak23,\allowbreak24,\allowbreak28-\allowbreak32}
\crossref{Mark}{13}{21}{De 13:1-\allowbreak3 Mt 24:5,\allowbreak23-\allowbreak25 Lu 17:23,\allowbreak24; 21:8 Joh 5:43}
\crossref{Mark}{13}{22}{13:6 Mt 24:24 Joh 10:27,\allowbreak28 2Th 2:8-\allowbreak14 2Ti 2:19}
\crossref{Mark}{13}{23}{13:5,\allowbreak9,\allowbreak33 Mt 7:15 Lu 21:8,\allowbreak34 2Pe 3:17}
\crossref{Mark}{13}{24}{Isa 13:10; 24:20-\allowbreak23 Jer 4:23-\allowbreak25,\allowbreak28 Eze 32:7 Da 7:10; 12:1}
\crossref{Mark}{13}{25}{13:25}
\crossref{Mark}{13}{26}{Mr 8:38; 14:62 Da 7:9-\allowbreak14 Mt 16:17,\allowbreak27; 24:30; 25:31 Ac 1:11}
\crossref{Mark}{13}{27}{Mt 13:41,\allowbreak49; 24:31 Lu 16:22 Re 7:1-\allowbreak3; 15:6,\allowbreak7}
\crossref{Mark}{13}{28}{Mt 24:32,\allowbreak33 Lu 21:29-\allowbreak31}
\crossref{Mark}{13}{29}{Eze 7:10-\allowbreak12; 12:25-\allowbreak28 Heb 10:25-\allowbreak37 Jas 5:9 1Pe 4:17,\allowbreak18}
\crossref{Mark}{13}{30}{Mt 16:28; 23:36; 24:34 Lu 21:32}
\crossref{Mark}{13}{31}{Ps 102:25-\allowbreak27 Isa 51:6 Mt 5:18; 24:35 Heb 1:10-\allowbreak12 2Pe 3:10-\allowbreak12}
\crossref{Mark}{13}{32}{13:26,\allowbreak27 Mt 24:36-\allowbreak42; 25:6,\allowbreak13,\allowbreak19 Ac 1:7 1Th 5:2 2Pe 3:10 Re 3:3}
\crossref{Mark}{13}{33}{13:23,\allowbreak35-\allowbreak37; 14:37,\allowbreak38 Mt 24:42-\allowbreak44; 25:13; 26:40,\allowbreak41 Lu 12:40; 21:34-\allowbreak36}
\crossref{Mark}{13}{34}{Mt 24:45; 25:14 etc.}
\crossref{Mark}{13}{35}{13:33 Mt 24:42,\allowbreak44}
\crossref{Mark}{13}{36}{Mr 14:37,\allowbreak40 Pr 6:9-\allowbreak11; 24:33,\allowbreak34 So 3:1; 5:2 Isa 56:10}
\crossref{Mark}{13}{37}{13:33,\allowbreak35 Lu 12:41-\allowbreak46}
\crossref{Mark}{14}{1}{Mt 6:2 Lu 22:1,\allowbreak2 Joh 11:53-\allowbreak57; 13:1}
\crossref{Mark}{14}{2}{Pr 19:21; 21:30 La 3:27 Mt 26:5}
\crossref{Mark}{14}{3}{Mt 26:6,\allowbreak7 Joh 11:2; 12:1-\allowbreak3}
\crossref{Mark}{14}{4}{Ec 4:4 Mt 26:8,\allowbreak9 Joh 12:4,\allowbreak5}
\crossref{Mark}{14}{5}{Mt 18:28}
\crossref{Mark}{14}{6}{Job 42:7,\allowbreak8 Isa 54:17 2Co 10:18}
\crossref{Mark}{14}{7}{De 15:11 Mt 25:35-\allowbreak45; 26:11 Joh 12:7,\allowbreak8 2Co 9:13,\allowbreak14 Phm 1:7}
\crossref{Mark}{14}{8}{Mr 15:42-\allowbreak47; 16:1 Lu 23:53-\allowbreak56; 24:1-\allowbreak3 Joh 12:7; 19:32-\allowbreak42}
\crossref{Mark}{14}{9}{Mr 16:15 Mt 26:12,\allowbreak13}
\crossref{Mark}{14}{10}{Mt 26:14-\allowbreak16 Lu 22:3-\allowbreak6 Joh 13:2,\allowbreak30}
\crossref{Mark}{14}{11}{Ho 7:3 Lu 22:5}
\crossref{Mark}{14}{12}{Ex 12:6,\allowbreak8,\allowbreak18; 13:3 Le 23:5,\allowbreak6 Nu 28:16-\allowbreak18 De 16:1-\allowbreak4 Mt 26:17}
\crossref{Mark}{14}{13}{Mr 11:2,\allowbreak3 Mt 8:9; 26:18,\allowbreak19 Lu 19:30-\allowbreak33; 22:10-\allowbreak13 Joh 2:5; 15:14}
\crossref{Mark}{14}{14}{Mr 10:17; 11:3 Joh 11:28; 13:13}
\crossref{Mark}{14}{15}{2Ch 6:30 Ps 110:3 Pr 16:1; 21:1,\allowbreak2 Joh 2:24,\allowbreak25; 21:17 2Ti 2:19}
\crossref{Mark}{14}{16}{Lu 22:13,\allowbreak35 Joh 16:4}
\crossref{Mark}{14}{17}{Mt 26:20 Lu 22:14}
\crossref{Mark}{14}{18}{Mt 26:21}
\crossref{Mark}{14}{19}{Mt 26:22 Lu 22:21-\allowbreak23 Joh 13:22}
\crossref{Mark}{14}{20}{14:43 Mt 26:47 Lu 22:47 Joh 6:71}
\crossref{Mark}{14}{21}{14:49 Ge 3:15 Ps 22:1 etc.}
\crossref{Mark}{14}{22}{Mt 26:26-\allowbreak29 Lu 22:19,\allowbreak20 1Co 10:16,\allowbreak17; 11:23-\allowbreak29}
\crossref{Mark}{14}{23}{14:22 Lu 22:17 Ro 14:6 1Co 10:16}
\crossref{Mark}{14}{24}{Ex 24:8 Zec 9:11 Joh 6:53 1Co 10:16; 11:25 Heb 9:15-\allowbreak23; 13:20,\allowbreak21}
\crossref{Mark}{14}{25}{Ps 104:15 Mt 26:29 Lu 22:16-\allowbreak18,\allowbreak29,\allowbreak30}
\crossref{Mark}{14}{26}{Ps 47:6,\allowbreak7 Ac 16:25 1Co 14:15 Eph 5:18-\allowbreak20 Col 3:16 Jas 5:13}
\crossref{Mark}{14}{27}{Mt 26:31 Lu 22:31,\allowbreak32 Joh 16:1,\allowbreak32 2Ti 4:16}
\crossref{Mark}{14}{28}{Mr 16:7 Mt 16:21; 26:32; 28:7,\allowbreak10,\allowbreak16 Joh 21:1 1Co 15:4-\allowbreak6}
\crossref{Mark}{14}{29}{Mt 26:33-\allowbreak35 Lu 22:33,\allowbreak34 Joh 13:36-\allowbreak38; 21:15}
\crossref{Mark}{14}{30}{Ge 1:5,\allowbreak8,\allowbreak13,\allowbreak19,\allowbreak23}
\crossref{Mark}{14}{31}{2Ki 8:13 Job 40:4,\allowbreak5 Ps 30:6 Pr 16:18; 18:24; 29:23}
\crossref{Mark}{14}{32}{Mt 26:36 etc.}
\crossref{Mark}{14}{33}{Mr 1:16-\allowbreak19; 5:37; 9:2}
\crossref{Mark}{14}{34}{Isa 53:3,\allowbreak4,\allowbreak12 La 1:12 Joh 12:27}
\crossref{Mark}{14}{35}{Ge 17:3 De 9:18 1Ch 21:15,\allowbreak16 2Ch 7:3 Mt 26:39 Lu 17:15,\allowbreak16}
\crossref{Mark}{14}{36}{Mt 6:9 Ro 8:15,\allowbreak16 Ga 4:6}
\crossref{Mark}{14}{37}{14:40,\allowbreak41 Lu 9:31,\allowbreak32; 22:45,\allowbreak46}
\crossref{Mark}{14}{38}{14:34 Mt 24:42; 25:13; 26:41 Lu 21:36; 22:40,\allowbreak46 1Co 16:13}
\crossref{Mark}{14}{39}{Mt 6:7; 26:42-\allowbreak44 Lu 18:1 2Co 12:8}
\crossref{Mark}{14}{40}{Mr 9:33,\allowbreak34 Ge 44:16 Ro 3:19}
\crossref{Mark}{14}{41}{Mr 7:9 Jud 10:14 1Ki 18:27; 22:15 2Ki 3:13 Ec 11:9 Eze 20:39}
\crossref{Mark}{14}{42}{Mt 26:46 Joh 18:1,\allowbreak2}
\crossref{Mark}{14}{43}{Mt 26:47 Lu 22:47,\allowbreak48 Joh 18:3-\allowbreak9 Ac 1:16}
\crossref{Mark}{14}{44}{Ex 12:13 Jos 2:12 Php 1:28 2Th 3:17}
\crossref{Mark}{14}{45}{Mr 12:14 Isa 1:3 Mal 1:6 Mt 23:8-\allowbreak10 Lu 6:46 Joh 13:13,\allowbreak14; 20:16}
\crossref{Mark}{14}{46}{Jud 16:21 La 4:20 Joh 18:12 Ac 2:23}
\crossref{Mark}{14}{47}{Mt 26:51-\allowbreak54 Lu 22:49-\allowbreak51 Joh 18:10,\allowbreak11}
\crossref{Mark}{14}{48}{1Sa 24:14,\allowbreak15; 26:18 Mt 26:55 Lu 22:52,\allowbreak53}
\crossref{Mark}{14}{49}{Mr 11:15-\allowbreak18,\allowbreak27; 12:35 Mt 21:23 etc.}
\crossref{Mark}{14}{50}{14:27 Job 19:13,\allowbreak14 Ps 38:11; 88:7,\allowbreak8,\allowbreak18 Isa 63:3 Joh 16:32; 18:8}
\crossref{Mark}{14}{51}{14:51}
\crossref{Mark}{14}{52}{Mr 13:14-\allowbreak16 Ge 39:12 Job 2:4}
\crossref{Mark}{14}{53}{Isa 53:7 Mt 26:57 etc.}
\crossref{Mark}{14}{54}{14:29-\allowbreak31,\allowbreak38 1Sa 13:7 Mt 26:58}
\crossref{Mark}{14}{55}{1Ki 21:10,\allowbreak13 Ps 27:12; 35:11 Mt 26:59,\allowbreak60 Ac 6:11-\allowbreak13; 24:1-\allowbreak13}
\crossref{Mark}{14}{56}{}
\crossref{Mark}{14}{57}{Mr 15:29 Jer 26:8,\allowbreak9,\allowbreak18 Mt 26:60,\allowbreak61; 27:40 Joh 2:18-\allowbreak21}
\crossref{Mark}{14}{58}{Da 2:34,\allowbreak45 Ac 7:48 2Co 5:1 Heb 9:11,\allowbreak24}
\crossref{Mark}{14}{59}{14:56}
\crossref{Mark}{14}{60}{Mr 15:3-\allowbreak5 Mt 26:62,\allowbreak63 Joh 19:9,\allowbreak10}
\crossref{Mark}{14}{61}{Ps 39:1,\allowbreak2,\allowbreak9 Isa 53:7 Mt 27:12-\allowbreak14 Ac 8:32 1Pe 2:23}
\crossref{Mark}{14}{62}{Mr 15:2 Mt 26:64; 27:11 Lu 23:3}
\crossref{Mark}{14}{63}{Isa 36:22; 37:1 Jer 36:23,\allowbreak24 Ac 14:13,\allowbreak14}
\crossref{Mark}{14}{64}{Le 24:16 1Ki 21:9-\allowbreak13 Mt 26:65,\allowbreak66 Lu 22:71 Joh 5:18; 8:58,\allowbreak59}
\crossref{Mark}{14}{65}{Mr 15:19 Nu 12:14 Job 30:10 Isa 50:6; 52:14; 53:3 Mic 5:1}
\crossref{Mark}{14}{66}{14:54 Mt 26:58,\allowbreak69,\allowbreak70 Lu 22:55-\allowbreak57}
\crossref{Mark}{14}{67}{Mr 10:47 Mt 2:23; 21:11 Joh 1:45-\allowbreak49; 19:19 Ac 10:38}
\crossref{Mark}{14}{68}{14:29-\allowbreak31 Joh 13:36-\allowbreak38 2Ti 2:12,\allowbreak13}
\crossref{Mark}{14}{69}{14:38 Lu 22:58 Joh 18:25 Ga 6:1}
\crossref{Mark}{14}{70}{Mt 26:73,\allowbreak74 Lu 22:59,\allowbreak60 Joh 18:26,\allowbreak27}
\crossref{Mark}{14}{71}{2Ki 8:12-\allowbreak15; 10:32 Jer 17:9 1Co 10:12}
\crossref{Mark}{14}{72}{14:30,\allowbreak68 Mt 26:34,\allowbreak74}
\crossref{Mark}{15}{1}{Ps 2:2 Mt 27:1,\allowbreak2 Lu 22:66 Ac 4:5,\allowbreak6,\allowbreak25-\allowbreak28}
\crossref{Mark}{15}{2}{Mt 2:2; 27:11 Lu 23:3 Joh 18:33-\allowbreak37; 19:19-\allowbreak22 1Ti 6:13}
\crossref{Mark}{15}{3}{Mt 27:12 Lu 23:2-\allowbreak5 Joh 18:29-\allowbreak31; 19:6,\allowbreak7,\allowbreak12}
\crossref{Mark}{15}{4}{Mt 26:62; 27:13 Joh 19:10}
\crossref{Mark}{15}{5}{Isa 53:7 Joh 19:9}
\crossref{Mark}{15}{6}{Mt 26:2,\allowbreak5; 27:15 Lu 23:16,\allowbreak17 Joh 18:39,\allowbreak40 Ac 24:27; 25:9}
\crossref{Mark}{15}{7}{Mt 27:16 Lu 23:18,\allowbreak19,\allowbreak25}
\crossref{Mark}{15}{8}{}
\crossref{Mark}{15}{9}{Mt 27:17-\allowbreak21 Joh 18:39; 19:4,\allowbreak5,\allowbreak14-\allowbreak16 Ac 3:13-\allowbreak15}
\crossref{Mark}{15}{10}{Ge 4:4-\allowbreak6; 37:11 1Sa 18:8,\allowbreak9 Pr 27:4 Ec 4:4 Mt 27:18 Ac 13:45}
\crossref{Mark}{15}{11}{Ho 5:1 Mt 27:20 Joh 18:40 Ac 3:14}
\crossref{Mark}{15}{12}{Mt 27:22,\allowbreak23 Lu 23:20-\allowbreak24 Joh 19:14-\allowbreak16}
\crossref{Mark}{15}{13}{}
\crossref{Mark}{15}{14}{Isa 53:9 Mt 27:4,\allowbreak19,\allowbreak24,\allowbreak54 Lu 23:4,\allowbreak14,\allowbreak15,\allowbreak21,\allowbreak41,\allowbreak47 Joh 18:38}
\crossref{Mark}{15}{15}{Pr 29:25 Ps 57:11 Mt 27:26 Lu 23:24,\allowbreak25 Joh 19:1,\allowbreak16 Ac 24:27}
\crossref{Mark}{15}{16}{Mt 27:27}
\crossref{Mark}{15}{17}{Mt 27:28-\allowbreak30 Lu 23:11 Joh 19:2-\allowbreak5}
\crossref{Mark}{15}{18}{15:29-\allowbreak32 Ge 37:10,\allowbreak20 Mt 27:42,\allowbreak43 Lu 23:36,\allowbreak37 Joh 19:14,\allowbreak15}
\crossref{Mark}{15}{19}{Mr 9:12; 10:34; 14:65 Job 13:9; 30:8-\allowbreak12 Ps 22:6,\allowbreak7; 35:15-\allowbreak17}
\crossref{Mark}{15}{20}{Mt 27:31 Joh 19:16}
\crossref{Mark}{15}{21}{Mt 27:32 Lu 23:26}
\crossref{Mark}{15}{22}{Mt 27:33 etc.}
\crossref{Mark}{15}{23}{Mt 27:34 Lu 23:36 Joh 19:28-\allowbreak30}
\crossref{Mark}{15}{24}{De 21:23 Ps 22:16,\allowbreak17 Isa 53:4-\allowbreak8 Ac 5:30 2Co 5:21 Ga 3:13}
\crossref{Mark}{15}{25}{15:33 Mt 27:45 Lu 23:44 Joh 19:14 Ac 2:15}
\crossref{Mark}{15}{26}{De 23:5 Ps 76:10 Pr 21:1 Isa 10:7; 46:10}
\crossref{Mark}{15}{27}{Mt 27:38 Lu 23:32,\allowbreak33 Joh 19:18}
% skipped verse
%\crossref{Mark}{15}{28}{Isa 53:12 Lu 22:37 Heb 12:2}
\crossref{Mark}{15}{29}{Ps 22:7,\allowbreak8,\allowbreak12-\allowbreak14; 35:15-\allowbreak21; 69:7,\allowbreak19,\allowbreak20,\allowbreak26; 109:25 La 1:12; 2:15}
\crossref{Mark}{15}{30}{}
\crossref{Mark}{15}{31}{Ps 2:1-\allowbreak4; 22:16,\allowbreak17 Mt 27:41-\allowbreak43 Lu 23:35-\allowbreak37}
\crossref{Mark}{15}{32}{Mr 14:61,\allowbreak62 Isa 44:6 Zep 3:15 Zec 9:9 Joh 1:49; 12:13; 19:12-\allowbreak15}
\crossref{Mark}{15}{33}{15:25 Mt 27:45 Lu 23:44,\allowbreak45}
\crossref{Mark}{15}{34}{Da 9:21 Lu 23:46 Ac 10:3}
\crossref{Mark}{15}{35}{Mr 9:11-\allowbreak13 Mt 17:11-\allowbreak13; 27:47-\allowbreak49}
\crossref{Mark}{15}{36}{15:23 Ps 69:21 Lu 23:36 Joh 19:28-\allowbreak30}
\crossref{Mark}{15}{37}{Mt 27:50 Lu 23:46 Joh 19:30}
\crossref{Mark}{15}{38}{Ex 26:31-\allowbreak34; 40:20,\allowbreak21 Le 16:2 etc.}
\crossref{Mark}{15}{39}{Mt 27:43,\allowbreak54 Lu 23:47,\allowbreak48}
\crossref{Mark}{15}{40}{Ps 38:11 Mt 27:55,\allowbreak56 Lu 23:49 Joh 19:25-\allowbreak27}
\crossref{Mark}{15}{41}{Mt 27:56 Lu 8:2,\allowbreak3}
\crossref{Mark}{15}{42}{Mt 27:57,\allowbreak62 Lu 23:50-\allowbreak54 Joh 19:38}
\crossref{Mark}{15}{43}{Mr 10:23-\allowbreak27}
\crossref{Mark}{15}{44}{Joh 19:31-\allowbreak37}
\crossref{Mark}{15}{45}{Mt 27:58 Joh 19:38}
\crossref{Mark}{15}{46}{Mt 27:59,\allowbreak60 Lu 23:53 Joh 19:38-\allowbreak42}
\crossref{Mark}{15}{47}{15:40; 16:1 Mt 27:61; 28:1 Lu 23:55,\allowbreak56; 24:1,\allowbreak2}
\crossref{Mark}{16}{1}{Mr 15:42 Mt 28:1 etc.}
\crossref{Mark}{16}{2}{}
\crossref{Mark}{16}{3}{Mr 15:46,\allowbreak47 Mt 27:60-\allowbreak66}
\crossref{Mark}{16}{4}{Mt 28:2-\allowbreak4 Lu 24:2 Joh 20:1}
\crossref{Mark}{16}{5}{Lu 24:3 Joh 20:8}
\crossref{Mark}{16}{6}{Mt 14:26,\allowbreak27; 28:4,\allowbreak5 Re 1:17,\allowbreak18}
\crossref{Mark}{16}{7}{Mr 14:50,\allowbreak66-\allowbreak72 Mt 28:7 2Co 2:7}
\crossref{Mark}{16}{8}{Mt 28:8 Lu 24:9-\allowbreak11,\allowbreak22-\allowbreak24}
\crossref{Mark}{16}{9}{Joh 20:19 Ac 20:7 1Co 16:2 Re 1:10}
\crossref{Mark}{16}{10}{Mr 14:72 Mt 9:15; 24:30 Lu 24:17 Joh 16:6,\allowbreak20-\allowbreak22}
\crossref{Mark}{16}{11}{16:13,\allowbreak14; 9:19 Ex 6:9 Job 9:16 Lu 24:11,\allowbreak23-\allowbreak35}
\crossref{Mark}{16}{12}{Lu 24:13-\allowbreak32}
\crossref{Mark}{16}{13}{Lu 24:33-\allowbreak35}
\crossref{Mark}{16}{14}{Lu 24:36-\allowbreak43 Joh 20:19,\allowbreak20 1Co 15:5}
\crossref{Mark}{16}{15}{Mt 10:5,\allowbreak6; 28:19 Lu 14:21-\allowbreak23; 24:47,\allowbreak48 Joh 15:16; 20:21}
\crossref{Mark}{16}{16}{Mr 1:15 Lu 8:12 Joh 1:12,\allowbreak13; 3:15,\allowbreak16,\allowbreak18,\allowbreak36; 5:24; 6:29,\allowbreak35,\allowbreak40; 7:37,\allowbreak38}
\crossref{Mark}{16}{17}{Joh 14:12}
\crossref{Mark}{16}{18}{Ge 3:15 Ps 91:13 Lu 10:19 Ac 28:3-\allowbreak6 Ro 16:20}
\crossref{Mark}{16}{19}{Mt 28:18-\allowbreak20 Lu 24:44-\allowbreak50 Joh 21:15,\allowbreak22 Ac 1:2,\allowbreak3}
\crossref{Mark}{16}{20}{Ac 2:1-\allowbreak28:31}

% Luke
\crossref{Luke}{1}{1}{Joh 20:31 Ac 1:1-\allowbreak3 1Ti 3:16 2Pe 1:16-\allowbreak19}
\crossref{Luke}{1}{2}{Lu 24:48 Mr 1:1 Joh 15:27 Ac 1:3,\allowbreak8,\allowbreak21,\allowbreak22; 4:20; 10:39-\allowbreak41 Heb 2:3}
\crossref{Luke}{1}{3}{Ac 15:19,\allowbreak25,\allowbreak28 1Co 7:40; 16:12}
\crossref{Luke}{1}{4}{Joh 20:31 2Pe 1:15,\allowbreak16}
\crossref{Luke}{1}{5}{Mt 2:1}
\crossref{Luke}{1}{6}{Lu 16:15 Ge 6:9; 7:1; 17:1 Job 1:1,\allowbreak8; 9:2 Ro 3:9-\allowbreak25 Php 3:6-\allowbreak9}
\crossref{Luke}{1}{7}{Ge 15:2,\allowbreak3; 16:1,\allowbreak2; 25:21; 30:1 Jud 13:2,\allowbreak3 1Sa 1:2,\allowbreak5-\allowbreak8}
\crossref{Luke}{1}{8}{Ex 28:1,\allowbreak41; 29:1,\allowbreak9,\allowbreak44; 30:30 Nu 18:7 1Ch 24:2 2Ch 11:14}
\crossref{Luke}{1}{9}{Ex 30:7,\allowbreak8; 37:25-\allowbreak29 Nu 16:40 1Sa 2:28 1Ch 6:49; 23:13 2Ch 26:16}
\crossref{Luke}{1}{10}{Le 16:17 Heb 4:14; 9:24 Re 8:3}
\crossref{Luke}{1}{11}{1:19,\allowbreak28; 2:10 Jud 13:3,\allowbreak9 Ac 10:3,\allowbreak4 Heb 1:14}
\crossref{Luke}{1}{12}{1:29; 2:9,\allowbreak10 Jud 6:22; 13:22 Job 4:14,\allowbreak15 Da 10:7 Mr 16:5 Ac 10:4}
\crossref{Luke}{1}{13}{Lu 24:36-\allowbreak40 Jud 6:23 Da 10:12 Mt 28:5 Mr 16:6}
\crossref{Luke}{1}{14}{1:58 Ge 21:6 Pr 15:20; 23:15,\allowbreak24}
\crossref{Luke}{1}{15}{Lu 7:28 Ge 12:2; 48:19 Jos 3:7; 4:14 1Ch 17:8; 29:12 Mt 11:9-\allowbreak19}
\crossref{Luke}{1}{16}{1:76 Isa 40:3-\allowbreak5; 49:6 Da 12:3 Mal 3:1 Mt 3:1-\allowbreak6; 21:32}
\crossref{Luke}{1}{17}{1:16 Joh 1:13,\allowbreak23-\allowbreak30,\allowbreak34; 3:28}
\crossref{Luke}{1}{18}{1:34 Ge 15:8; 17:17; 18:12 Jud 6:36-\allowbreak40 Isa 38:22}
\crossref{Luke}{1}{19}{1:26 Da 8:16; 9:21-\allowbreak23 Mt 18:10 Heb 4:14}
\crossref{Luke}{1}{20}{1:22,\allowbreak62,\allowbreak63 Ex 4:11 Eze 3:26; 24:27}
\crossref{Luke}{1}{21}{Nu 6:23-\allowbreak27}
\crossref{Luke}{1}{22}{Joh 13:24 Ac 12:17; 19:33; 21:40}
\crossref{Luke}{1}{23}{2Ki 11:5-\allowbreak7 1Ch 9:25}
\crossref{Luke}{1}{24}{}
\crossref{Luke}{1}{25}{1:13 Ge 21:1,\allowbreak2; 25:21; 30:22 1Sa 1:19,\allowbreak20; 2:21,\allowbreak22 Heb 11:11}
\crossref{Luke}{1}{26}{1:24}
\crossref{Luke}{1}{27}{Lu 2:4,\allowbreak5 Ge 3:15 Isa 7:14 Jer 31:22 Mt 1:18,\allowbreak21,\allowbreak23}
\crossref{Luke}{1}{28}{Da 9:21-\allowbreak23; 10:19}
\crossref{Luke}{1}{29}{1:12 Mr 6:49,\allowbreak50; 16:5,\allowbreak6 Ac 10:4}
\crossref{Luke}{1}{30}{1:13; 12:32 Isa 41:10,\allowbreak14; 43:1-\allowbreak4; 44:2 Mt 28:5 Ac 18:9,\allowbreak10; 27:24}
\crossref{Luke}{1}{31}{1:27 Isa 7:14 Mt 1:23 Ga 4:4}
\crossref{Luke}{1}{32}{1:15; 3:16 Mt 3:11; 12:42 Php 2:9-\allowbreak11}
\crossref{Luke}{1}{33}{Ps 45:6; 89:35-\allowbreak37 Da 2:44; 7:13,\allowbreak14,\allowbreak27 Ob 1:21 Mic 4:7 1Co 15:24,\allowbreak25}
\crossref{Luke}{1}{34}{Jud 13:8-\allowbreak12 Ac 9:6}
\crossref{Luke}{1}{35}{1:27,\allowbreak31 Mt 1:20}
\crossref{Luke}{1}{36}{1:24-\allowbreak26}
\crossref{Luke}{1}{37}{Lu 18:27 Ge 18:14 Nu 11:23 Job 13:2 Jer 32:17,\allowbreak27 Zec 8:6 Mt 19:26}
\crossref{Luke}{1}{38}{2Sa 7:25-\allowbreak29 Ps 116:16 Ro 4:20,\allowbreak21}
\crossref{Luke}{1}{39}{Jos 10:40; 15:48-\allowbreak59; 21:9-\allowbreak11}
\crossref{Luke}{1}{40}{}
\crossref{Luke}{1}{41}{1:15,\allowbreak44 Ge 25:22 Ps 22:10}
\crossref{Luke}{1}{42}{1:28,\allowbreak48 Jud 5:24}
\crossref{Luke}{1}{43}{Lu 7:7 Ru 2:10 1Sa 25:41 Mt 3:14 Joh 13:5-\allowbreak8 Php 2:3}
\crossref{Luke}{1}{44}{1:41}
\crossref{Luke}{1}{45}{1:20; 11:27,\allowbreak28 2Ch 20:20 Joh 11:40; 20:29}
\crossref{Luke}{1}{46}{1Sa 2:1 Ps 34:2,\allowbreak3; 35:9; 103:1,\allowbreak2 Isa 24:15,\allowbreak16; 45:25; 61:10}
\crossref{Luke}{1}{47}{Lu 2:11 Isa 12:2,\allowbreak3; 45:21,\allowbreak22 Zep 3:14-\allowbreak17 Zec 9:9 1Ti 1:1}
\crossref{Luke}{1}{48}{1Sa 1:11; 2:8 2Sa 7:8,\allowbreak18,\allowbreak19 Ps 102:17; 113:7,\allowbreak8; 136:23; 138:6}
\crossref{Luke}{1}{49}{Ge 17:1 Ps 24:8 Isa 1:24; 63:1 Jer 10:6; 20:11}
\crossref{Luke}{1}{50}{Ge 17:7 Ex 20:6; 34:6,\allowbreak7 Ps 31:19; 85:9; 103:11,\allowbreak17,\allowbreak18; 115:13; 118:4}
\crossref{Luke}{1}{51}{Ex 15:6,\allowbreak7,\allowbreak12,\allowbreak13 De 4:34 Ps 52:9; 63:5; 89:13; 98:1; 118:15}
\crossref{Luke}{1}{52}{Lu 18:14 1Sa 2:4,\allowbreak6-\allowbreak8 Job 5:11-\allowbreak13; 34:24-\allowbreak28 Ps 107:40,\allowbreak41; 113:6-\allowbreak8}
\crossref{Luke}{1}{53}{Lu 6:21 1Sa 2:5 Ps 34:10; 107:8,\allowbreak9; 146:7 Eze 34:29 Mt 5:6}
\crossref{Luke}{1}{54}{1:70-\allowbreak75 Ps 98:3 Isa 44:21; 46:3,\allowbreak4; 49:14-\allowbreak16; 54:6-\allowbreak10; 63:7-\allowbreak16}
\crossref{Luke}{1}{55}{Ge 12:3; 17:19; 22:18; 26:4; 28:14 Ps 105:6-\allowbreak10; 132:11-\allowbreak17}
\crossref{Luke}{1}{56}{1:56}
\crossref{Luke}{1}{57}{1:13; 2:6,\allowbreak7 Ge 21:2,\allowbreak3 Nu 23:19}
\crossref{Luke}{1}{58}{1:25 Ru 4:14-\allowbreak17 Ps 113:9}
\crossref{Luke}{1}{59}{Lu 2:21 Ge 17:12; 21:3,\allowbreak4 Le 12:3 Ac 7:8 Php 3:5}
\crossref{Luke}{1}{60}{1:13 2Sa 12:25 Isa 8:3 Mt 1:25}
\crossref{Luke}{1}{61}{1:61}
\crossref{Luke}{1}{62}{}
\crossref{Luke}{1}{63}{Pr 3:3 Isa 30:8 Jer 17:1 Hab 2:2}
\crossref{Luke}{1}{64}{1:20 Ex 4:15,\allowbreak16 Ps 51:15 Jer 1:9 Eze 3:27; 29:21; 33:22 Mt 9:33}
\crossref{Luke}{1}{65}{Lu 7:16 Ac 2:43; 5:5,\allowbreak11; 19:17 Re 11:11}
\crossref{Luke}{1}{66}{Lu 2:19,\allowbreak51; 9:44 Ge 37:11 Ps 119:11}
\crossref{Luke}{1}{67}{1:15,\allowbreak41 Nu 11:25 2Sa 23:2 Joe 2:28 2Pe 1:21}
\crossref{Luke}{1}{68}{Ge 9:26; 14:20 1Ki 1:48 1Ch 29:10,\allowbreak20 Ps 41:13; 72:17-\allowbreak19; 106:48}
\crossref{Luke}{1}{69}{1Sa 2:10 2Sa 22:3 Ps 18:2; 132:17,\allowbreak18 Eze 29:21}
\crossref{Luke}{1}{70}{2Sa 23:2 Jer 30:10 Mr 12:36 Ac 28:25 Heb 3:7 2Pe 1:21 Re 19:10}
\crossref{Luke}{1}{71}{1:74 De 33:29 Ps 106:10,\allowbreak47 Isa 14:1-\allowbreak3; 44:24-\allowbreak26; 54:7-\allowbreak17 Jer 23:6}
\crossref{Luke}{1}{72}{1:54,\allowbreak55 Ge 12:3; 22:18; 26:4; 28:14 Ps 98:3 Ac 3:25,\allowbreak26 Ro 11:28}
\crossref{Luke}{1}{73}{Ge 22:16,\allowbreak17; 24:7; 26:3 De 7:8,\allowbreak12 Ps 105:9 Jer 11:5 Heb 6:16,\allowbreak17}
\crossref{Luke}{1}{74}{1:71 Isa 35:9,\allowbreak10; 45:17; 54:13,\allowbreak14; 65:21-\allowbreak25 Eze 34:25-\allowbreak28; 39:28,\allowbreak29}
\crossref{Luke}{1}{75}{De 6:2 Ps 105:44,\allowbreak45 Jer 31:33,\allowbreak34; 32:39,\allowbreak40 Eze 36:24-\allowbreak27 Mt 1:21}
\crossref{Luke}{1}{76}{Lu 7:28 Mt 14:5; 21:26 Mr 11:32}
\crossref{Luke}{1}{77}{Lu 3:3,\allowbreak6 Mr 1:3,\allowbreak4 Joh 1:7-\allowbreak9,\allowbreak15-\allowbreak17,\allowbreak29,\allowbreak34; 3:27-\allowbreak36 Ac 19:4}
\crossref{Luke}{1}{78}{Ps 25:6 Isa 63:7,\allowbreak15 Joh 3:16 Eph 2:4,\allowbreak5 Php 1:8; 2:1 Col 3:12}
\crossref{Luke}{1}{79}{Lu 2:32 Isa 9:2; 42:7,\allowbreak16; 49:6,\allowbreak9; 60:1-\allowbreak3 Mt 4:16 Joh 1:9; 8:12; 9:5}
\crossref{Luke}{1}{80}{1:15; 2:40,\allowbreak52 Jud 13:24,\allowbreak25 1Sa 3:19,\allowbreak20}
\crossref{Luke}{2}{1}{Lu 3:1 Ac 11:28; 25:11,\allowbreak21 Php 4:22}
\crossref{Luke}{2}{2}{Ac 5:37}
\crossref{Luke}{2}{3}{}
\crossref{Luke}{2}{4}{Lu 1:26,\allowbreak27; 3:23}
\crossref{Luke}{2}{5}{De 22:22-\allowbreak27 Mt 1:18,\allowbreak19}
\crossref{Luke}{2}{6}{Ps 33:11 Pr 19:21 Mic 5:2}
\crossref{Luke}{2}{7}{Isa 7:14 Mt 1:25 Ga 4:4}
\crossref{Luke}{2}{8}{Ge 31:39,\allowbreak40 Ex 3:1,\allowbreak2 1Sa 17:34,\allowbreak35 Ps 78:70,\allowbreak71 Eze 34:8}
\crossref{Luke}{2}{9}{Lu 1:11,\allowbreak28 Jud 6:11,\allowbreak12 Mt 1:20 Ac 27:23 1Ti 3:16}
\crossref{Luke}{2}{10}{Lu 1:13,\allowbreak30 Da 10:11,\allowbreak12,\allowbreak19 Mt 28:5 Re 1:17,\allowbreak18}
\crossref{Luke}{2}{11}{Lu 1:69 Isa 9:6 Mt 1:21 Ga 4:4,\allowbreak5 2Ti 1:9,\allowbreak10 Tit 2:10-\allowbreak14; 3:4-\allowbreak7}
\crossref{Luke}{2}{12}{Ex 3:12 1Sa 10:2-\allowbreak7 Ps 22:6 Isa 53:1,\allowbreak2}
\crossref{Luke}{2}{13}{Ge 28:12; 32:1,\allowbreak2 1Ki 22:19 Job 38:7 Ps 68:17; 103:20,\allowbreak21; 148:2}
\crossref{Luke}{2}{14}{Lu 19:38 Ps 69:34,\allowbreak35; 85:9-\allowbreak12; 96:11-\allowbreak13 Isa 44:23; 49:13 Joh 17:4}
\crossref{Luke}{2}{15}{Lu 24:51 2Ki 2:1,\allowbreak11 1Pe 3:22}
\crossref{Luke}{2}{16}{Lu 1:39 Ec 9:10}
\crossref{Luke}{2}{17}{2:38; 8:39 Ps 16:9,\allowbreak10; 66:16; 71:17,\allowbreak18 Mal 3:16 Joh 1:41-\allowbreak46; 4:28,\allowbreak29}
\crossref{Luke}{2}{18}{2:33,\allowbreak47; 1:65,\allowbreak66; 4:36; 5:9,\allowbreak10 Isa 8:18}
\crossref{Luke}{2}{19}{2:51; 1:66; 9:43,\allowbreak44 Ge 37:11 1Sa 21:12 Pr 4:4 Ho 14:9}
\crossref{Luke}{2}{20}{Lu 18:43; 19:37,\allowbreak38 1Ch 29:10-\allowbreak12 Ps 72:17-\allowbreak19; 106:48; 107:8,\allowbreak15,\allowbreak21}
\crossref{Luke}{2}{21}{Lu 1:59 Ge 17:12 Le 12:3 Mt 3:15 Ga 4:4,\allowbreak5 Php 2:8}
\crossref{Luke}{2}{22}{Le 12:2-\allowbreak6}
\crossref{Luke}{2}{23}{Ex 13:2,\allowbreak12-\allowbreak15; 22:29; 34:19 Nu 3:13; 8:16,\allowbreak17; 18:15}
\crossref{Luke}{2}{24}{Le 12:2,\allowbreak6-\allowbreak8 2Co 8:9}
\crossref{Luke}{2}{25}{Lu 1:6 Ge 6:9 Job 1:1,\allowbreak8 Da 6:22,\allowbreak23 Mic 6:8 Ac 10:2,\allowbreak22; 24:16}
\crossref{Luke}{2}{26}{Ps 25:14 Am 3:7}
\crossref{Luke}{2}{27}{Lu 4:1 Mt 4:1 Ac 8:29; 10:19; 11:12; 16:7 Re 1:10; 17:3}
\crossref{Luke}{2}{28}{Mr 9:36; 10:16}
\crossref{Luke}{2}{29}{Ge 15:15; 46:30 Ps 37:37 Isa 57:1,\allowbreak2 Php 1:23 Re 14:13}
\crossref{Luke}{2}{30}{2:10,\allowbreak11; 3:6 Ge 49:18 2Sa 23:1-\allowbreak5 Isa 49:6 Ac 4:10-\allowbreak12}
\crossref{Luke}{2}{31}{Ps 96:1-\allowbreak3,\allowbreak10-\allowbreak13; 97:6-\allowbreak8; 98:2,\allowbreak3 Isa 42:1-\allowbreak4,\allowbreak10-\allowbreak12; 45:21-\allowbreak25}
\crossref{Luke}{2}{32}{Isa 9:2; 42:6,\allowbreak7; 49:6; 60:1-\allowbreak3,\allowbreak19 Mt 4:16 Ac 13:47,\allowbreak48; 28:28}
\crossref{Luke}{2}{33}{2:48; 1:65,\allowbreak66 Isa 8:18}
\crossref{Luke}{2}{34}{Ge 14:19; 47:7 Ex 39:43 Le 9:22,\allowbreak23 Heb 7:1,\allowbreak7}
\crossref{Luke}{2}{35}{Ps 42:10 Joh 19:25}
\crossref{Luke}{2}{36}{Ex 15:20 Jud 4:4 2Ki 22:14 Ac 2:18; 21:9 1Co 12:1}
\crossref{Luke}{2}{37}{Ex 38:8 1Sa 2:2 Ps 23:6; 27:4; 84:4,\allowbreak10; 92:13; 135:1,\allowbreak2 Re 3:12}
\crossref{Luke}{2}{38}{2:27}
\crossref{Luke}{2}{39}{2:21-\allowbreak24; 1:6 De 12:32 Mt 3:15 Ga 4:4,\allowbreak5}
\crossref{Luke}{2}{40}{2:52 Jud 13:24 1Sa 2:18,\allowbreak26; 3:19 Ps 22:9 Isa 53:1,\allowbreak2}
\crossref{Luke}{2}{41}{Ex 23:14-\allowbreak17; 34:23 De 12:5-\allowbreak7,\allowbreak11,\allowbreak18; 16:1-\allowbreak8,\allowbreak16 1Sa 1:3,\allowbreak21}
\crossref{Luke}{2}{42}{2:42}
\crossref{Luke}{2}{43}{2Ch 30:21-\allowbreak23; 25:17}
\crossref{Luke}{2}{44}{Ps 42:4; 122:1-\allowbreak4 Isa 2:3}
\crossref{Luke}{2}{45}{}
\crossref{Luke}{2}{46}{2:44,\allowbreak45 1Ki 12:5,\allowbreak12 Mt 12:40; 16:21; 27:63,\allowbreak64}
\crossref{Luke}{2}{47}{Lu 4:22,\allowbreak32 Ps 119:99 Mt 7:28 Mr 1:22 Joh 7:15,\allowbreak46}
\crossref{Luke}{2}{48}{}
\crossref{Luke}{2}{49}{2:48 Ps 40:8 Mal 3:1 Mt 21:12 Joh 2:16,\allowbreak17; 4:34; 5:17; 6:38; 8:29}
\crossref{Luke}{2}{50}{Lu 9:45; 18:34}
\crossref{Luke}{2}{51}{2:39}
\crossref{Luke}{2}{52}{2:40; 1:80 1Sa 2:26}
\crossref{Luke}{3}{1}{Lu 2:1}
\crossref{Luke}{3}{2}{Joh 11:49-\allowbreak51; 18:13,\allowbreak14,\allowbreak24 Ac 4:6}
\crossref{Luke}{3}{3}{Mt 3:5 Mr 1:4,\allowbreak5 Joh 1:28; 3:26}
\crossref{Luke}{3}{4}{Isa 40:3-\allowbreak5 Mt 3:3 Mr 1:3 Joh 1:23}
\crossref{Luke}{3}{5}{Lu 1:51-\allowbreak53 Isa 2:11-\allowbreak17; 35:6-\allowbreak8; 40:4; 49:11; 61:1-\allowbreak3 Eze 17:24 Jas 1:9}
\crossref{Luke}{3}{6}{Lu 2:10,\allowbreak11,\allowbreak30-\allowbreak32 Ps 98:2,\allowbreak3 Isa 40:5; 49:6; 52:10 Mr 16:15 Ro 10:12}
\crossref{Luke}{3}{7}{Ge 3:15 Ps 58:4,\allowbreak5 Isa 59:5 Mt 3:7-\allowbreak10; 23:33 Joh 8:44 Ac 13:10}
\crossref{Luke}{3}{8}{Isa 1:16-\allowbreak18 Eze 18:27-\allowbreak31 Ac 26:20 2Co 7:10,\allowbreak11 Ga 5:22-\allowbreak24}
\crossref{Luke}{3}{9}{Lu 13:7,\allowbreak9; 23:29-\allowbreak31 Isa 10:33,\allowbreak34 Eze 15:2-\allowbreak4; 31:18 Da 4:14,\allowbreak23}
\crossref{Luke}{3}{10}{3:8 Ac 2:37; 9:6; 16:30}
\crossref{Luke}{3}{11}{Lu 11:41; 18:22; 19:8 Isa 58:7-\allowbreak11 Da 4:27 Mt 25:40 Mr 14:5-\allowbreak8}
\crossref{Luke}{3}{12}{Lu 7:29; 15:1,\allowbreak2; 18:13 Mt 21:31,\allowbreak32}
\crossref{Luke}{3}{13}{Lu 19:8 Ps 18:23 Pr 28:13 Isa 1:16,\allowbreak17; 55:6,\allowbreak7 Eze 18:21,\allowbreak22,\allowbreak27,\allowbreak28}
\crossref{Luke}{3}{14}{Mt 8:5 Ac 10:7}
\crossref{Luke}{3}{15}{Joh 10:24}
\crossref{Luke}{3}{16}{Mt 3:11 Mr 1:7,\allowbreak8 Joh 1:26,\allowbreak33 Ac 1:5; 11:16; 13:24,\allowbreak25; 19:4,\allowbreak5}
\crossref{Luke}{3}{17}{Jer 15:7 Mt 3:12}
\crossref{Luke}{3}{18}{Joh 1:15,\allowbreak29,\allowbreak34; 3:29-\allowbreak36 Ac 2:40}
\crossref{Luke}{3}{19}{Pr 9:7,\allowbreak8; 15:12 Mt 11:2; 14:3,\allowbreak4 Mr 6:17,\allowbreak18}
\crossref{Luke}{3}{20}{Lu 13:31-\allowbreak34 2Ki 21:16; 24:4 2Ch 24:17-\allowbreak22; 36:16 Ne 9:26 Jer 2:30}
\crossref{Luke}{3}{21}{Mt 3:13-\allowbreak15 Mr 1:9 Joh 1:32 etc.}
\crossref{Luke}{3}{22}{Lu 9:34,\allowbreak35 Ps 2:7 Isa 42:1 Mt 12:18; 17:5; 27:43 Col 1:13 1Pe 2:4}
\crossref{Luke}{3}{23}{Ge 41:46 Nu 4:3,\allowbreak35,\allowbreak39,\allowbreak43,\allowbreak47}
\crossref{Luke}{3}{24}{3:24}
\crossref{Luke}{3}{25}{3:25}
\crossref{Luke}{3}{26}{3:26}
\crossref{Luke}{3}{27}{3:27}
\crossref{Luke}{3}{28}{3:28}
\crossref{Luke}{3}{29}{3:29}
\crossref{Luke}{3}{30}{}
\crossref{Luke}{3}{31}{2Sa 5:14 1Ch 3:5; 14:4 Zec 12:12}
\crossref{Luke}{3}{32}{Ru 4:18-\allowbreak22 1Sa 17:58; 20:31 1Ki 12:16 1Ch 2:10-\allowbreak15 Ps 72:20}
\crossref{Luke}{3}{33}{Ru 4:19,\allowbreak20 1Ch 2:9,\allowbreak10}
\crossref{Luke}{3}{34}{Ge 21:3; 25:26 1Ch 1:34 Mt 1:2 Ac 7:8}
\crossref{Luke}{3}{35}{Ge 11:18-\allowbreak21}
\crossref{Luke}{3}{36}{Ge 5:32; 7:13; 9:18,\allowbreak26,\allowbreak27; 10:21,\allowbreak22; 11:10 etc.}
\crossref{Luke}{3}{37}{Ge 5:6-\allowbreak28 1Ch 1:1-\allowbreak3}
\crossref{Luke}{3}{38}{Ge 4:25,\allowbreak26; 5:3}
\crossref{Luke}{4}{1}{Mt 4:1 etc.}
\crossref{Luke}{4}{2}{Ex 24:18; 34:28 De 9:9,\allowbreak18,\allowbreak25 1Ki 19:8 Mt 4:2}
\crossref{Luke}{4}{3}{Lu 3:22 Mt 4:3}
\crossref{Luke}{4}{4}{4:8,\allowbreak10 Isa 8:20 Joh 10:34,\allowbreak35 Eph 6:17}
\crossref{Luke}{4}{5}{Mr 4:8,\allowbreak9 1Co 7:31 Eph 2:2; 6:12 1Jo 2:15,\allowbreak16}
\crossref{Luke}{4}{6}{Joh 8:44 2Co 11:14 Re 12:9; 20:2,\allowbreak3}
\crossref{Luke}{4}{7}{Lu 8:28; 17:16 Ps 72:11 Isa 45:14; 46:6 Mt 2:11 Re 4:10; 5:8; 22:8}
\crossref{Luke}{4}{8}{Mt 4:10; 16:23 Jas 4:7 1Pe 5:9}
\crossref{Luke}{4}{9}{Job 2:6 Mt 4:5}
\crossref{Luke}{4}{10}{4:3,\allowbreak8 2Co 11:14}
\crossref{Luke}{4}{11}{}
\crossref{Luke}{4}{12}{De 6:16 Ps 95:9; 106:14 Mal 3:15 Mt 4:7 1Co 10:9 Heb 3:8,\allowbreak9}
\crossref{Luke}{4}{13}{Mt 4:11 Joh 14:30 Heb 4:15 Jas 4:7}
\crossref{Luke}{4}{14}{Mt 4:12 Mr 1:14 Joh 4:43 Ac 10:37}
\crossref{Luke}{4}{15}{4:16; 13:10 Mt 4:23; 9:35; 13:54 Mr 1:39}
\crossref{Luke}{4}{16}{Lu 1:26,\allowbreak27; 2:39,\allowbreak51 Mt 2:23; 13:54 Mr 6:1}
\crossref{Luke}{4}{17}{Lu 20:42 Ac 7:42; 13:15,\allowbreak27}
\crossref{Luke}{4}{18}{Ps 45:7 Isa 11:2-\allowbreak5; 42:1-\allowbreak4; 50:4; 59:21}
\crossref{Luke}{4}{19}{Lu 19:42 Le 25:8-\allowbreak13,\allowbreak50-\allowbreak54 Nu 36:4 Isa 61:2; 63:4 2Co 6:1}
\crossref{Luke}{4}{20}{4:17 Mt 20:26-\allowbreak28}
\crossref{Luke}{4}{21}{Lu 10:23,\allowbreak24 Mt 13:14 Joh 4:25,\allowbreak26; 5:39 Ac 2:16-\allowbreak18,\allowbreak29-\allowbreak33; 3:18}
\crossref{Luke}{4}{22}{Lu 2:47; 21:15 Ps 45:2,\allowbreak4 Pr 10:32; 16:21; 25:11 Ec 12:10,\allowbreak11 So 5:16}
\crossref{Luke}{4}{23}{Lu 6:42 Ro 2:21,\allowbreak22}
\crossref{Luke}{4}{24}{Mt 13:57 Mr 6:4,\allowbreak5 Joh 4:41,\allowbreak44 Ac 22:3,\allowbreak18-\allowbreak22}
\crossref{Luke}{4}{25}{Lu 10:21 Isa 55:8 Mt 20:15 Mr 7:26-\allowbreak29 Ro 9:15,\allowbreak20 Eph 1:9,\allowbreak11}
\crossref{Luke}{4}{26}{1Ki 17:9 etc.}
\crossref{Luke}{4}{27}{1Ki 19:19-\allowbreak21}
\crossref{Luke}{4}{28}{Lu 6:11; 11:53,\allowbreak54 2Ch 16:10; 24:20,\allowbreak21 Jer 37:15,\allowbreak16; 38:6 Ac 5:33}
\crossref{Luke}{4}{29}{Joh 8:37,\allowbreak40,\allowbreak59; 15:24,\allowbreak25 Ac 7:57,\allowbreak58; 16:23,\allowbreak24; 21:28-\allowbreak32}
\crossref{Luke}{4}{30}{Joh 8:59; 10:39; 18:6,\allowbreak7 Ac 12:18}
\crossref{Luke}{4}{31}{Mt 4:13 Mr 1:21}
\crossref{Luke}{4}{32}{4:36 Jer 23:28,\allowbreak29 Mt 7:28,\allowbreak29 Mr 1:22 Joh 6:63 1Co 2:4,\allowbreak5; 14:24,\allowbreak25}
\crossref{Luke}{4}{33}{Mr 1:23}
\crossref{Luke}{4}{34}{Lu 8:37 Ac 16:39}
\crossref{Luke}{4}{35}{4:39,\allowbreak41 Ps 50:16 Zec 3:2 Mt 8:26; 17:18 Mr 3:11,\allowbreak12 Ac 16:17,\allowbreak18}
\crossref{Luke}{4}{36}{Mt 9:33; 12:22,\allowbreak23 Mr 1:27; 7:37}
\crossref{Luke}{4}{37}{}
\crossref{Luke}{4}{38}{Mt 8:14,\allowbreak15 Mr 1:29-\allowbreak31 1Co 9:5}
\crossref{Luke}{4}{39}{4:35; 8:24}
\crossref{Luke}{4}{40}{Mt 8:16,\allowbreak17 Mr 1:32-\allowbreak34}
\crossref{Luke}{4}{41}{4:34,\allowbreak35 Mr 1:25,\allowbreak34; 3:11}
\crossref{Luke}{4}{42}{Lu 6:12 Mr 1:35 Joh 4:34}
\crossref{Luke}{4}{43}{Mr 1:14,\allowbreak15,\allowbreak38,\allowbreak39 Joh 9:4 Ac 10:38 2Ti 4:2}
\crossref{Luke}{4}{44}{4:15 Mt 4:23 Mr 1:39}
\crossref{Luke}{5}{1}{Lu 8:45; 12:1 Mt 4:18 etc.}
\crossref{Luke}{5}{2}{Mt 4:21 Mr 1:19}
\crossref{Luke}{5}{3}{Mt 4:18 Joh 1:41,\allowbreak42}
\crossref{Luke}{5}{4}{Mt 17:27 Joh 21:6}
\crossref{Luke}{5}{5}{Ps 127:1,\allowbreak2 Eze 37:11,\allowbreak12 Joh 21:3}
\crossref{Luke}{5}{6}{2Ki 4:3-\allowbreak7 Ec 11:6 Joh 21:6-\allowbreak11 Ac 2:41; 4:4 1Co 15:58 Ga 6:9}
\crossref{Luke}{5}{7}{Ex 23:5 Pr 18:24 Ac 11:25 Ro 16:2-\allowbreak4 Ga 6:2 Php 4:3}
\crossref{Luke}{5}{8}{Mt 2:11 Joh 11:32 Ac 10:25,\allowbreak26 Re 1:17; 22:8,\allowbreak9}
\crossref{Luke}{5}{9}{Lu 4:32,\allowbreak36 Ps 8:6,\allowbreak8 Mr 9:6}
\crossref{Luke}{5}{10}{Lu 6:14 Mt 4:21; 20:20}
\crossref{Luke}{5}{11}{Lu 18:28-\allowbreak30 Mt 4:20; 10:37; 19:27 Mr 1:18-\allowbreak25; 10:21,\allowbreak29,\allowbreak30 Php 3:7,\allowbreak8}
\crossref{Luke}{5}{12}{Mt 8:2-\allowbreak4 Mr 1:40-\allowbreak45}
\crossref{Luke}{5}{13}{Ge 1:3,\allowbreak9 Ps 33:9 2Ki 5:10,\allowbreak14 Eze 36:25-\allowbreak27,\allowbreak29 Ho 14:4}
\crossref{Luke}{5}{14}{Mt 8:4; 9:30; 12:16}
\crossref{Luke}{5}{15}{Pr 15:33 1Ti 5:25}
\crossref{Luke}{5}{16}{Lu 6:12 Mt 14:23 Mr 1:35,\allowbreak36; 6:46 Joh 6:15}
\crossref{Luke}{5}{17}{5:21,\allowbreak30; 7:30; 11:52-\allowbreak54; 15:2 Joh 3:21}
\crossref{Luke}{5}{18}{Mt 9:2-\allowbreak8 Mr 2:3-\allowbreak12 Joh 5:5,\allowbreak6 Ac 9:33}
\crossref{Luke}{5}{19}{Mr 2:4}
\crossref{Luke}{5}{20}{Ge 22:12 Joh 2:25 Ac 11:23; 14:9 Jas 2:18}
\crossref{Luke}{5}{21}{5:17; 7:49 Mr 2:6,\allowbreak7}
\crossref{Luke}{5}{22}{1Ch 28:9 Ps 139:2 Pr 15:26 Isa 66:18 Eze 38:10 Mt 9:4; 12:25}
\crossref{Luke}{5}{23}{Mt 9:5 Mr 2:9}
\crossref{Luke}{5}{24}{Da 7:13 Mt 16:13; 25:31; 26:64 Joh 3:13; 5:27 Re 1:13}
\crossref{Luke}{5}{25}{5:13 Ge 1:3 Ps 33:9}
\crossref{Luke}{5}{26}{Lu 7:16 Mt 9:8; 12:23 Mr 2:12 Ac 4:21 Ga 1:24}
\crossref{Luke}{5}{27}{Mt 9:9 etc.}
\crossref{Luke}{5}{28}{5:11; 9:59-\allowbreak62 1Ki 19:19-\allowbreak21 Mt 19:22-\allowbreak27}
\crossref{Luke}{5}{29}{Joh 12:2}
\crossref{Luke}{5}{30}{5:17,\allowbreak21; 7:29,\allowbreak30,\allowbreak34,\allowbreak39; 15:1,\allowbreak2; 18:11; 19:7 Isa 65:5 Mt 21:28-\allowbreak32 Mr 7:3}
\crossref{Luke}{5}{31}{Jer 8:22 Mt 9:12,\allowbreak13 Mr 2:17}
\crossref{Luke}{5}{32}{Lu 4:18,\allowbreak19; 15:7,\allowbreak10; 18:10-\allowbreak14; 19:10; 24:47 Isa 55:6,\allowbreak7; 57:15 Mt 18:11}
\crossref{Luke}{5}{33}{Lu 18:12 Isa 58:3-\allowbreak6 Zec 7:6 Mt 9:14-\allowbreak17 Mr 2:18-\allowbreak22}
\crossref{Luke}{5}{34}{Jud 14:10,\allowbreak11 Ps 45:14 So 2:6,\allowbreak7; 3:10,\allowbreak11; 5:8; 6:1 Mt 25:1-\allowbreak10}
\crossref{Luke}{5}{35}{Lu 24:17-\allowbreak21 Da 9:26 Zec 13:7 Joh 12:8; 13:33; 14:3,\allowbreak4; 16:4-\allowbreak7,\allowbreak16-\allowbreak22}
\crossref{Luke}{5}{36}{Mt 9:16,\allowbreak17 Mr 2:21,\allowbreak22}
\crossref{Luke}{5}{37}{Jos 9:4,\allowbreak13 Ps 119:83}
\crossref{Luke}{5}{38}{Eze 36:26 2Co 5:17 Ga 2:4,\allowbreak12-\allowbreak14; 4:9-\allowbreak11; 5:1-\allowbreak6; 6:13,\allowbreak14 Php 3:5-\allowbreak7}
\crossref{Luke}{5}{39}{Jer 6:16 Mr 7:7-\allowbreak13 Ro 4:11,\allowbreak12 Heb 11:1,\allowbreak2,\allowbreak39}
\crossref{Luke}{6}{1}{Ex 12:15 Le 23:7,\allowbreak10,\allowbreak11,\allowbreak15 De 16:9}
\crossref{Luke}{6}{2}{6:7-\allowbreak9; 5:33 Mt 12:2; 15:2; 23:23,\allowbreak24 Mr 2:24 Joh 5:9-\allowbreak11,\allowbreak16; 9:14-\allowbreak16}
\crossref{Luke}{6}{3}{Mt 12:3,\allowbreak5; 19:4; 21:16,\allowbreak42; 22:31 Mr 2:25; 12:10,\allowbreak26}
\crossref{Luke}{6}{4}{Le 24:5-\allowbreak9}
\crossref{Luke}{6}{5}{Mt 11:5-\allowbreak8 Mr 2:27; 9:7 Re 1:10}
\crossref{Luke}{6}{6}{Mt 12:9-\allowbreak14 Mr 3:1-\allowbreak6}
\crossref{Luke}{6}{7}{Lu 13:14; 14:1-\allowbreak6 Ps 37:32,\allowbreak33; 38:12 Isa 29:21 Jer 20:10 Mr 3:2}
\crossref{Luke}{6}{8}{Lu 5:22 1Ch 28:9; 29:17 Job 42:2 Ps 44:21 Joh 2:25; 21:17 Heb 4:13}
\crossref{Luke}{6}{9}{Lu 14:3 Mt 12:12,\allowbreak13 Mr 3:4 Joh 7:19-\allowbreak23}
\crossref{Luke}{6}{10}{Mr 3:5}
\crossref{Luke}{6}{11}{Lu 4:28 Ps 2:1,\allowbreak2 Ec 9:3 Ac 5:33; 7:54; 26:11}
\crossref{Luke}{6}{12}{Ps 55:15-\allowbreak17; 109:3,\allowbreak4 Da 6:10 Mt 6:6 Mr 1:35; 14:34-\allowbreak36 Heb 5:7}
\crossref{Luke}{6}{13}{Lu 9:1,\allowbreak2 Mt 9:36-\allowbreak38; 10:1-\allowbreak4 Mr 3:13-\allowbreak19; 6:7}
\crossref{Luke}{6}{14}{Lu 5:8 Joh 1:40-\allowbreak42; 21:15-\allowbreak20 Ac 1:13 2Pe 1:1}
\crossref{Luke}{6}{15}{Lu 5:27}
\crossref{Luke}{6}{16}{Mt 10:3}
\crossref{Luke}{6}{17}{Mt 4:23-\allowbreak25; 12:15 Mr 3:7 etc.}
\crossref{Luke}{6}{18}{Mt 15:22; 17:15 Ac 5:16}
\crossref{Luke}{6}{19}{Nu 21:8,\allowbreak9 2Ki 13:21 Mt 9:20,\allowbreak21; 14:36 Mr 3:10; 6:56; 8:22}
\crossref{Luke}{6}{20}{Mt 5:2 etc.}
\crossref{Luke}{6}{21}{6:25; 1:53 Ps 42:1,\allowbreak2; 143:6 Isa 55:1,\allowbreak2 1Co 4:11 2Co 11:27; 12:10}
\crossref{Luke}{6}{22}{Mt 5:10-\allowbreak12; 10:22 Mr 13:9-\allowbreak13 Joh 7:7; 15:18-\allowbreak20; 17:14}
\crossref{Luke}{6}{23}{Ac 5:41 Ro 5:3 2Co 12:10 Col 1:24 Jas 1:2}
\crossref{Luke}{6}{24}{Lu 12:15-\allowbreak21; 18:23-\allowbreak25 Job 21:7-\allowbreak15 Ps 49:6,\allowbreak7,\allowbreak16-\allowbreak19; 73:3-\allowbreak12 Pr 1:32}
\crossref{Luke}{6}{25}{De 6:11,\allowbreak12 1Sa 2:5 Pr 30:9 Isa 28:7; 65:13 Php 4:12,\allowbreak13 Re 3:17}
\crossref{Luke}{6}{26}{Mic 2:11 Joh 7:7; 15:19 Ro 16:18 2Th 2:8-\allowbreak12 Jas 4:4 2Pe 2:18,\allowbreak19}
\crossref{Luke}{6}{27}{Lu 8:8,\allowbreak15,\allowbreak18 Mr 4:24}
\crossref{Luke}{6}{28}{Lu 23:34 Ac 7:60 Ro 12:14 1Co 4:12 Jas 3:10 1Pe 3:9}
\crossref{Luke}{6}{29}{Mt 5:39}
\crossref{Luke}{6}{30}{6:38; 11:41; 12:33; 18:22 De 15:7-\allowbreak10 Ps 41:1; 112:9 Pr 3:27,\allowbreak28; 11:24,\allowbreak25}
\crossref{Luke}{6}{31}{Mt 7:12; 22:39 Ga 5:14 Jas 2:8-\allowbreak16}
\crossref{Luke}{6}{32}{Mt 5:46,\allowbreak47}
\crossref{Luke}{6}{33}{6:33}
\crossref{Luke}{6}{34}{6:35; 14:12-\allowbreak14 De 15:8-\allowbreak11 Mt 5:42}
\crossref{Luke}{6}{35}{6:27-\allowbreak31 Le 25:35-\allowbreak37 Ps 37:26; 112:5 Pr 19:17; 22:9 Ro 5:8-\allowbreak10}
\crossref{Luke}{6}{36}{Mt 5:48 Eph 4:31; 5:1,\allowbreak2 1Pe 1:15,\allowbreak16}
\crossref{Luke}{6}{37}{Isa 65:5 Mt 7:1 Ro 2:1,\allowbreak2; 14:3,\allowbreak4,\allowbreak10-\allowbreak16 1Co 4:3-\allowbreak5 Jas 4:11,\allowbreak12}
\crossref{Luke}{6}{38}{6:30 De 15:10 Ezr 7:27,\allowbreak28 Job 31:16-\allowbreak20; 42:11 Pr 3:9,\allowbreak10; 10:22}
\crossref{Luke}{6}{39}{Isa 9:16; 56:10 Mt 15:14; 23:16-\allowbreak26 Ro 2:19 1Ti 6:3-\allowbreak5 2Ti 3:13}
\crossref{Luke}{6}{40}{Mt 10:24,\allowbreak25 Joh 13:16; 15:20}
\crossref{Luke}{6}{41}{Mt 7:3-\allowbreak5 Ro 2:1,\allowbreak21-\allowbreak24}
\crossref{Luke}{6}{42}{Lu 13:15 Mt 23:13-\allowbreak15 Ac 8:21; 13:10}
\crossref{Luke}{6}{43}{Ps 92:12-\allowbreak14 Isa 5:4; 61:3 Jer 2:21 Mt 3:10; 7:16-\allowbreak20; 12:33}
\crossref{Luke}{6}{44}{Ga 5:19-\allowbreak23 Tit 2:11-\allowbreak13 Jas 3:12 Jude 1:12}
\crossref{Luke}{6}{45}{Ps 37:30,\allowbreak31; 40:8-\allowbreak10; 71:15-\allowbreak18 Pr 10:20,\allowbreak21; 12:18; 15:23; 22:17,\allowbreak18}
\crossref{Luke}{6}{46}{Lu 13:25-\allowbreak27 Mal 1:6 Mt 7:21-\allowbreak23; 25:11,\allowbreak24,\allowbreak44 Joh 13:13-\allowbreak17 Ga 6:7}
\crossref{Luke}{6}{47}{Lu 14:26 Isa 55:3 Mt 11:28 Joh 5:40; 6:35,\allowbreak37,\allowbreak44,\allowbreak45 1Pe 2:4}
\crossref{Luke}{6}{48}{Pr 10:25 Isa 28:16 Mt 7:25,\allowbreak26 1Co 3:10-\allowbreak12 Eph 2:20 2Ti 2:19}
\crossref{Luke}{6}{49}{6:46; 8:5-\allowbreak7; 19:14,\allowbreak27 Jer 44:16,\allowbreak17 Eze 33:31 Mt 21:29,\allowbreak30; 23:3}
\crossref{Luke}{7}{1}{Mt 7:28,\allowbreak29}
\crossref{Luke}{7}{2}{Lu 23:47 Mt 27:54 Ac 10:1; 22:26; 23:17; 27:1,\allowbreak3,\allowbreak43}
\crossref{Luke}{7}{3}{Lu 8:41; 9:38 Mt 8:5 Joh 4:47 Phm 1:10}
\crossref{Luke}{7}{4}{7:6,\allowbreak7; 20:35 Mt 10:11,\allowbreak13,\allowbreak37,\allowbreak38 Re 3:4}
\crossref{Luke}{7}{5}{1Ki 5:1 2Ch 2:11,\allowbreak12 Ga 5:6 1Jo 3:14; 5:1-\allowbreak3}
\crossref{Luke}{7}{6}{Mt 20:28 Mr 5:24 Ac 10:38}
\crossref{Luke}{7}{7}{Lu 4:36; 5:13 Ex 15:26 De 32:39 1Sa 2:6 Ps 33:9; 107:20 Mr 1:27}
\crossref{Luke}{7}{8}{Ac 22:25,\allowbreak26; 23:17,\allowbreak23,\allowbreak26; 24:23; 25:26}
\crossref{Luke}{7}{9}{Mt 8:10; 15:28}
\crossref{Luke}{7}{10}{Mt 8:13; 15:28 Mr 9:23 Joh 4:50-\allowbreak53}
\crossref{Luke}{7}{11}{Ac 10:38}
\crossref{Luke}{7}{12}{Lu 8:42 Ge 22:2,\allowbreak12 2Sa 14:7 1Ki 17:9,\allowbreak12,\allowbreak18,\allowbreak23 2Ki 4:16,\allowbreak20}
\crossref{Luke}{7}{13}{Jud 10:16 Ps 86:5,\allowbreak15; 103:13 Isa 63:9 Jer 31:20 La 3:32,\allowbreak33}
\crossref{Luke}{7}{14}{Lu 8:54,\allowbreak55 1Ki 17:21 Job 14:12,\allowbreak14 Ps 33:9 Isa 26:19 Eze 37:3-\allowbreak10}
\crossref{Luke}{7}{15}{1Ki 17:23,\allowbreak24 2Ki 4:32-\allowbreak37; 13:21}
\crossref{Luke}{7}{16}{Lu 1:65; 5:8,\allowbreak26; 8:37 Jer 33:9 Mt 28:8 Ac 5:5,\allowbreak11-\allowbreak13}
\crossref{Luke}{7}{17}{7:14 Mt 4:24; 9:31 Mr 1:28; 6:14}
\crossref{Luke}{7}{18}{Mt 11:2-\allowbreak6 Joh 3:26}
\crossref{Luke}{7}{19}{Lu 10:1 Jos 2:1 Mr 6:7 Ac 10:7,\allowbreak8 Re 11:3}
\crossref{Luke}{7}{20}{}
\crossref{Luke}{7}{21}{1Ki 8:37 Ps 90:7-\allowbreak9 Mr 3:10; 5:29,\allowbreak34 1Co 11:30-\allowbreak32 Heb 12:6}
\crossref{Luke}{7}{22}{Joh 1:46}
\crossref{Luke}{7}{23}{Lu 2:34 Isa 8:14,\allowbreak15 Mt 11:6; 13:57,\allowbreak58 Joh 6:60-\allowbreak66 Ro 9:32,\allowbreak33}
\crossref{Luke}{7}{24}{Mt 11:7,\allowbreak8}
\crossref{Luke}{7}{25}{2Ki 1:8 Isa 59:17 Mt 3:4 1Pe 3:3,\allowbreak4}
\crossref{Luke}{7}{26}{Lu 1:76; 20:6}
\crossref{Luke}{7}{27}{Lu 1:15-\allowbreak17,\allowbreak76 Isa 40:3 Mal 3:1; 4:5,\allowbreak6 Joh 1:23}
\crossref{Luke}{7}{28}{Lu 1:14,\allowbreak15; 3:16}
\crossref{Luke}{7}{29}{7:35 Jud 1:7 Ps 51:4 Ro 3:4-\allowbreak6; 10:3 Re 15:3; 16:5}
\crossref{Luke}{7}{30}{Lu 13:34 Jer 8:8 Ro 10:21 2Co 6:1 Ga 2:21}
\crossref{Luke}{7}{31}{La 2:13 Mt 11:16 etc.}
\crossref{Luke}{7}{32}{Pr 17:16 Isa 28:9-\allowbreak13; 29:11,\allowbreak12 Jer 5:3-\allowbreak5}
\crossref{Luke}{7}{33}{Lu 1:15 Jer 16:8-\allowbreak10 Mt 3:4 Mr 1:6}
\crossref{Luke}{7}{34}{7:36; 5:29; 11:37; 14:1 Joh 2:2; 12:2}
\crossref{Luke}{7}{35}{7:29 Pr 8:32-\allowbreak36; 17:16 Ho 14:9 Mt 11:19 1Co 2:14,\allowbreak15}
\crossref{Luke}{7}{36}{Mt 26:6 etc.}
\crossref{Luke}{7}{37}{7:34,\allowbreak39; 5:30,\allowbreak32; 18:13; 19:7 Mt 21:31 Joh 9:24,\allowbreak31 Ro 5:8 1Ti 1:9,\allowbreak15}
\crossref{Luke}{7}{38}{Lu 6:21; 22:62 Jud 2:4,\allowbreak5 Ezr 10:1 Ps 6:6-\allowbreak8; 38:18; 51:17; 126:5,\allowbreak6}
\crossref{Luke}{7}{39}{Lu 3:8; 12:17; 16:3; 18:4 2Ki 5:20 Pr 23:7 Mr 2:6,\allowbreak7; 7:21}
\crossref{Luke}{7}{40}{Lu 5:22,\allowbreak31; 6:8 Joh 16:19,\allowbreak30}
\crossref{Luke}{7}{41}{Lu 11:4; 13:4}
\crossref{Luke}{7}{42}{Ps 49:7,\allowbreak8 Mt 18:25,\allowbreak26,\allowbreak34 Ro 5:6 Ga 3:10}
\crossref{Luke}{7}{43}{7:47 1Co 15:9,\allowbreak10 2Co 5:14,\allowbreak15 1Ti 1:13-\allowbreak16}
\crossref{Luke}{7}{44}{7:37-\allowbreak39}
\crossref{Luke}{7}{45}{Ge 29:11; 33:4 2Sa 15:5; 19:39 Mt 26:48 Ro 16:16 1Co 16:20}
\crossref{Luke}{7}{46}{Ru 3:3 2Sa 14:2 Ps 23:5; 104:15 Ec 9:8 Da 10:3 Am 6:6 Mic 6:15}
\crossref{Luke}{7}{47}{7:42; 5:20,\allowbreak21 Ex 34:6,\allowbreak7}
\crossref{Luke}{7}{48}{Mt 9:2 Mr 2:5}
\crossref{Luke}{7}{49}{Lu 5:20,\allowbreak21 Mt 9:3 Mr 2:7}
\crossref{Luke}{7}{50}{Lu 8:18,\allowbreak42,\allowbreak48; 18:42 Hab 2:4 Mt 9:22 Mr 5:34; 10:52 Eph 2:8-\allowbreak10}
\crossref{Luke}{8}{1}{Lu 4:43,\allowbreak44 Mt 4:23; 9:35; 11:1 Mr 1:39 Ac 10:38}
\crossref{Luke}{8}{2}{Lu 23:27 Mt 27:55,\allowbreak56 Mr 15:40,\allowbreak41; 16:1 Joh 19:25 Ac 1:14}
\crossref{Luke}{8}{3}{Lu 24:10}
\crossref{Luke}{8}{4}{Mt 13:2 Mr 4:1}
\crossref{Luke}{8}{5}{8:11 Mt 13:3,\allowbreak4,\allowbreak18,\allowbreak19,\allowbreak24-\allowbreak26,\allowbreak37 Mr 4:2-\allowbreak4,\allowbreak15,\allowbreak26-\allowbreak29}
\crossref{Luke}{8}{6}{8:13 Jer 5:3 Eze 11:19; 36:26 Am 6:12 Mt 13:5,\allowbreak6,\allowbreak20,\allowbreak21 Mr 4:5,\allowbreak6,\allowbreak16}
\crossref{Luke}{8}{7}{8:14; 21:34 Ge 3:18 Jer 4:3 Mt 13:7,\allowbreak22 Mr 4:7,\allowbreak18,\allowbreak19 Heb 6:7,\allowbreak8}
\crossref{Luke}{8}{8}{8:15 Mt 13:8,\allowbreak23 Mr 4:8,\allowbreak20 Joh 1:12,\allowbreak13; 3:3-\allowbreak5 Eph 2:10 Col 1:10}
\crossref{Luke}{8}{9}{Ho 6:3 Mt 13:10,\allowbreak18,\allowbreak36; 15:15 Mr 4:10,\allowbreak34; 7:17,\allowbreak18 Joh 15:15}
\crossref{Luke}{8}{10}{Lu 10:21-\allowbreak24 Ps 25:14 Mt 11:25; 13:11,\allowbreak12; 16:17 Mr 4:11 Ro 16:25}
\crossref{Luke}{8}{11}{Isa 8:20 Mt 13:19 Mr 4:14 etc.}
\crossref{Luke}{8}{12}{8:5 Pr 1:24-\allowbreak26,\allowbreak29 Mt 13:19 Mr 4:15 Jas 1:23,\allowbreak24}
\crossref{Luke}{8}{13}{Ps 106:12-\allowbreak14 Isa 58:2 Eze 33:32 Mt 13:20,\allowbreak21 Mr 4:16,\allowbreak17; 6:20}
\crossref{Luke}{8}{14}{8:7; 16:13; 17:26-\allowbreak30; 18:24,\allowbreak25; 21:34 Mt 6:24,\allowbreak25; 13:22 Mr 4:19}
\crossref{Luke}{8}{15}{Lu 6:45 De 30:6 Ps 51:10 Jer 31:33; 32:29 Eze 36:26,\allowbreak27 Ro 7:18}
\crossref{Luke}{8}{16}{Lu 11:33 Mt 5:15,\allowbreak16 Mr 4:21,\allowbreak22 Ac 26:18 Php 2:15,\allowbreak16 Re 1:20; 2:1}
\crossref{Luke}{8}{17}{Lu 12:2,\allowbreak3 Ec 12:14 Mt 10:26 1Co 4:5}
\crossref{Luke}{8}{18}{Lu 9:44 De 32:46,\allowbreak47 Pr 2:2-\allowbreak5 Mr 4:23,\allowbreak24; 13:14 Ac 10:33; 17:11}
\crossref{Luke}{8}{19}{Mt 12:46-\allowbreak50 Mr 3:21,\allowbreak31-\allowbreak35}
\crossref{Luke}{8}{20}{Mt 13:55,\allowbreak56 Mr 6:3 Joh 7:3-\allowbreak6 Ac 1:14 1Co 9:5 Ga 1:19}
\crossref{Luke}{8}{21}{Lu 11:27,\allowbreak28 Mt 25:40,\allowbreak45; 28:10 Joh 15:14,\allowbreak15; 20:17 2Co 5:16; 6:18}
\crossref{Luke}{8}{22}{Mt 8:18,\allowbreak23-\allowbreak27 Mr 4:35-\allowbreak41 Joh 6:1}
\crossref{Luke}{8}{23}{Ps 44:23 Isa 51:9,\allowbreak10 Heb 4:15}
\crossref{Luke}{8}{24}{Ps 69:1,\allowbreak2; 116:3,\allowbreak4; 142:4,\allowbreak5 La 3:54-\allowbreak56 Joh 2:2-\allowbreak6 Mt 14:30}
\crossref{Luke}{8}{25}{Lu 12:28 Mt 6:30; 8:26; 14:31; 17:20 Mr 4:40,\allowbreak41 Joh 11:40}
\crossref{Luke}{8}{26}{Mt 8:28 etc.}
\crossref{Luke}{8}{27}{Mr 5:2-\allowbreak5}
\crossref{Luke}{8}{28}{Lu 4:33-\allowbreak36 Mt 8:29 Mr 1:24-\allowbreak27; 5:6-\allowbreak8 Ac 16:16-\allowbreak18}
\crossref{Luke}{8}{29}{Mr 5:8 Ac 19:12-\allowbreak16}
\crossref{Luke}{8}{30}{Mt 26:53 Mr 5:9}
\crossref{Luke}{8}{31}{8:28 Job 1:11; 2:5 Php 2:10,\allowbreak11}
\crossref{Luke}{8}{32}{Le 11:7 Isa 65:4; 66:3 Mt 8:30-\allowbreak33 Mr 5:11-\allowbreak13}
\crossref{Luke}{8}{33}{Joh 8:44 1Pe 5:8 Re 9:11}
\crossref{Luke}{8}{34}{Mt 8:33; 28:11 Mr 5:14 Ac 19:16,\allowbreak17}
\crossref{Luke}{8}{35}{Isa 49:24,\allowbreak25; 53:12 Heb 2:14,\allowbreak15 1Jo 3:8}
\crossref{Luke}{8}{36}{}
\crossref{Luke}{8}{37}{8:28; 5:8 De 5:25 1Sa 6:20 2Sa 6:8,\allowbreak9 1Ki 17:18 Job 21:14,\allowbreak15 Mt 8:34}
\crossref{Luke}{8}{38}{8:28,\allowbreak37 De 10:20,\allowbreak21 Ps 27:4; 32:7; 116:12,\allowbreak16 Mr 5:18 Php 1:23}
\crossref{Luke}{8}{39}{1Ti 5:8}
\crossref{Luke}{8}{40}{Mt 9:1 Mr 5:21}
\crossref{Luke}{8}{41}{Mt 9:18-\allowbreak25 Mr 5:22 etc.}
\crossref{Luke}{8}{42}{Lu 7:12 Ge 44:20-\allowbreak22 Job 1:18,\allowbreak19 Zec 12:10}
\crossref{Luke}{8}{43}{Le 15:25 etc.}
\crossref{Luke}{8}{44}{Lu 7:38}
\crossref{Luke}{8}{45}{Lu 9:13 Mr 5:30-\allowbreak32}
\crossref{Luke}{8}{46}{Lu 6:19 1Pe 2:9}
\crossref{Luke}{8}{47}{Ps 38:9 Ho 5:3}
\crossref{Luke}{8}{48}{Mt 9:2,\allowbreak22; 12:20 2Co 6:18}
\crossref{Luke}{8}{49}{8:41-\allowbreak43 Mt 9:23-\allowbreak26 Mr 5:35 etc.}
\crossref{Luke}{8}{50}{8:48 Isa 50:10 Mr 5:36; 9:23; 11:22-\allowbreak24 Joh 11:25,\allowbreak40 Ro 4:17,\allowbreak20}
\crossref{Luke}{8}{51}{1Ki 17:19-\allowbreak23 2Ki 4:4-\allowbreak6,\allowbreak34-\allowbreak36 Isa 42:2 Mt 6:5,\allowbreak6 Ac 9:40}
\crossref{Luke}{8}{52}{Ge 23:2; 27:34,\allowbreak35 2Sa 18:33 Jer 9:17-\allowbreak21 Ex 24:17 Zec 12:10}
\crossref{Luke}{8}{53}{Lu 16:14 Job 12:4; 17:2 Ps 22:7 Isa 53:3}
\crossref{Luke}{8}{54}{8:51 Mr 5:40}
\crossref{Luke}{8}{55}{Lu 24:41-\allowbreak43 Mr 5:43 Joh 11:44}
\crossref{Luke}{8}{56}{Lu 5:14 Mt 8:4; 9:30 Mr 5:42,\allowbreak43}
\crossref{Luke}{9}{1}{Lu 6:13-\allowbreak16 Mt 10:2-\allowbreak5 Mr 3:13-\allowbreak19; 6:7-\allowbreak13}
\crossref{Luke}{9}{2}{Lu 10:1,\allowbreak9,\allowbreak11; 16:16 Mt 3:2; 10:7,\allowbreak8; 13:19; 24:14 Mr 1:14,\allowbreak15; 6:12}
\crossref{Luke}{9}{3}{Lu 10:4 etc.}
\crossref{Luke}{9}{4}{Lu 10:5-\allowbreak8 Mt 10:11 Mr 6:10 Ac 16:15}
\crossref{Luke}{9}{5}{9:48; 10:10-\allowbreak12,\allowbreak16 Mt 10:14,\allowbreak15 Mr 6:11; 9:37 Ac 13:51; 18:6}
\crossref{Luke}{9}{6}{9:1,\allowbreak2 Mr 6:12,\allowbreak13; 16:20 Ac 4:30; 5:15}
\crossref{Luke}{9}{7}{Job 18:11,\allowbreak12 Ps 73:19 Mt 14:1-\allowbreak12 Mr 6:14-\allowbreak28}
\crossref{Luke}{9}{8}{9:19 Mt 17:10 Mr 6:15; 8:28 Joh 1:21}
\crossref{Luke}{9}{9}{9:7}
\crossref{Luke}{9}{10}{Lu 10:17 Zec 1:10 Mr 6:30 Heb 13:17}
\crossref{Luke}{9}{11}{Mt 14:14 Mr 6:33,\allowbreak34 Ro 10:14,\allowbreak17}
\crossref{Luke}{9}{12}{Mt 14:15 etc.}
\crossref{Luke}{9}{13}{2Ki 4:42,\allowbreak43 Mt 14:16,\allowbreak17 Mr 6:37,\allowbreak38 Joh 6:5-\allowbreak9}
\crossref{Luke}{9}{14}{Mr 6:39,\allowbreak40; 8:6 1Co 14:40}
\crossref{Luke}{9}{15}{}
\crossref{Luke}{9}{16}{Ps 121:1,\allowbreak2 Mt 14:19 Mr 7:34}
\crossref{Luke}{9}{17}{Ps 37:16 Pr 13:25 Mt 14:20,\allowbreak21; 15:37,\allowbreak38 Mr 6:42-\allowbreak44; 8:8,\allowbreak9}
\crossref{Luke}{9}{18}{Lu 11:1; 22:39-\allowbreak41 Mt 26:36}
\crossref{Luke}{9}{19}{9:7,\allowbreak8 Mal 4:5 Mt 14:2 Joh 1:21,\allowbreak25}
\crossref{Luke}{9}{20}{Mt 5:47; 16:15; 22:42}
\crossref{Luke}{9}{21}{Mt 16:20; 17:9 Mr 8:31}
\crossref{Luke}{9}{22}{9:44; 18:31-\allowbreak34; 24:7,\allowbreak26 Ge 3:15 Ps 22:1-\allowbreak31; 69:1-\allowbreak36 Isa 53:1-\allowbreak12}
\crossref{Luke}{9}{23}{Lu 14:26,\allowbreak27 Mt 10:38,\allowbreak39; 16:22-\allowbreak25 Mr 8:34-\allowbreak38 Joh 12:25,\allowbreak26 Ro 8:13}
\crossref{Luke}{9}{24}{Lu 17:33 Ac 20:23,\allowbreak24 Heb 11:35 Re 2:10; 12:11}
\crossref{Luke}{9}{25}{Lu 4:5-\allowbreak7; 12:19-\allowbreak21; 16:24,\allowbreak25 Ps 49:6-\allowbreak8 Mt 16:26 Mr 8:36; 9:43-\allowbreak48}
\crossref{Luke}{9}{26}{Lu 12:8,\allowbreak9 Ps 22:6-\allowbreak8 Isa 53:3 Mt 10:32,\allowbreak33 Mr 8:38 Joh 5:44; 12:43}
\crossref{Luke}{9}{27}{Mt 16:28 Mr 9:1 Joh 14:2; 16:7}
\crossref{Luke}{9}{28}{Mt 17:1 etc.}
\crossref{Luke}{9}{29}{Ex 34:29-\allowbreak35 Isa 33:17; 53:2 Mt 17:2 Mr 9:2,\allowbreak3 Joh 1:14 Ac 6:15}
\crossref{Luke}{9}{30}{Lu 24:27,\allowbreak44 Mt 17:3,\allowbreak4 Mr 9:4-\allowbreak6 Joh 1:17 Ro 3:21-\allowbreak23 2Co 3:7-\allowbreak11}
\crossref{Luke}{9}{31}{2Co 3:18 Php 3:21 Col 3:4 1Pe 5:10}
\crossref{Luke}{9}{32}{Lu 22:45,\allowbreak46 Da 8:18; 10:9 Mt 26:40-\allowbreak43}
\crossref{Luke}{9}{33}{Ps 4:6,\allowbreak7; 27:4; 63:2-\allowbreak5; 73:28 Joh 14:8 2Co 4:6}
\crossref{Luke}{9}{34}{Ex 14:19,\allowbreak20; 40:34-\allowbreak38 Ps 18:9-\allowbreak11 Isa 19:1 Mt 17:5-\allowbreak7 Mr 9:7,\allowbreak8}
\crossref{Luke}{9}{35}{Lu 3:22 Mt 3:17 Joh 3:16,\allowbreak35,\allowbreak36 2Pe 1:17,\allowbreak18}
\crossref{Luke}{9}{36}{Ec 3:7 Mt 17:9 Mr 9:6,\allowbreak10}
\crossref{Luke}{9}{37}{Mt 17:14-\allowbreak21 Mr 9:14-\allowbreak29}
\crossref{Luke}{9}{38}{Lu 7:12; 8:41,\allowbreak42 Mt 15:22 Joh 4:47}
\crossref{Luke}{9}{39}{Lu 4:35; 8:29 Mr 5:4,\allowbreak5; 9:20,\allowbreak26 Joh 8:44 1Pe 5:8 Re 9:11}
\crossref{Luke}{9}{40}{9:1; 10:17-\allowbreak19 2Ki 4:31 Mt 17:20,\allowbreak21 Ac 19:13-\allowbreak16}
\crossref{Luke}{9}{41}{Lu 8:25 Mr 9:19 Joh 20:27 Heb 3:19; 4:2,\allowbreak11}
\crossref{Luke}{9}{42}{9:39 Mr 1:26,\allowbreak27; 9:20,\allowbreak26,\allowbreak27 Re 12:12}
\crossref{Luke}{9}{43}{Lu 4:36; 5:9,\allowbreak26; 8:25 Ps 139:14 Zec 8:6 Mr 6:51 Ac 3:10-\allowbreak13}
\crossref{Luke}{9}{44}{Lu 1:66; 2:19,\allowbreak51 Isa 32:9,\allowbreak10 Joh 16:4 1Th 3:3,\allowbreak4 Heb 2:1; 12:2-\allowbreak5}
\crossref{Luke}{9}{45}{9:46; 2:50; 18:34 Mt 16:22 Mr 8:16-\allowbreak18,\allowbreak32,\allowbreak33; 9:10,\allowbreak32 Joh 12:16,\allowbreak34; 14:5}
\crossref{Luke}{9}{46}{Lu 14:7-\allowbreak11; 22:24-\allowbreak27 Mt 18:1 etc.}
\crossref{Luke}{9}{47}{Lu 5:22; 7:39,\allowbreak40 Ps 139:2,\allowbreak23 Jer 17:10 Joh 2:25; 16:30; 21:17}
\crossref{Luke}{9}{48}{Lu 10:16 Mt 10:40-\allowbreak42; 18:5,\allowbreak6,\allowbreak10,\allowbreak14; 25:40,\allowbreak45 Mr 9:37 Joh 12:44,\allowbreak45}
\crossref{Luke}{9}{49}{Nu 11:27-\allowbreak29 Mr 9:38-\allowbreak40; 10:13,\allowbreak14 Ac 4:18,\allowbreak19; 5:28 1Th 2:16}
\crossref{Luke}{9}{50}{Jos 9:14 Pr 3:5,\allowbreak6 Mt 13:28-\allowbreak30; 17:24,\allowbreak26 Php 1:15-\allowbreak18}
\crossref{Luke}{9}{51}{Lu 24:51 2Ki 2:1-\allowbreak3,\allowbreak11 Mr 16:19 Joh 6:62; 13:1; 16:5,\allowbreak28; 17:11}
\crossref{Luke}{9}{52}{Lu 7:27; 10:1 Mal 3:1}
\crossref{Luke}{9}{53}{9:48 Joh 4:9,\allowbreak40-\allowbreak42}
\crossref{Luke}{9}{54}{2Sa 21:2 2Ki 10:16,\allowbreak31 Jas 1:19,\allowbreak20; 3:14-\allowbreak18}
\crossref{Luke}{9}{55}{1Sa 24:4-\allowbreak7; 26:8-\allowbreak11 2Sa 19:22 Job 31:29-\allowbreak31 Pr 9:8 Mt 16:23}
\crossref{Luke}{9}{56}{Lu 19:10 Mt 18:11; 20:28 Joh 3:17; 10:10; 12:47 1Ti 1:15}
\crossref{Luke}{9}{57}{Ex 19:8 Mt 8:19,\allowbreak20 Joh 13:37}
\crossref{Luke}{9}{58}{Lu 14:26-\allowbreak33; 18:22,\allowbreak23 Jos 24:19-\allowbreak22 Joh 6:60-\allowbreak66}
\crossref{Luke}{9}{59}{Mt 4:19-\allowbreak22; 9:9; 16:24}
\crossref{Luke}{9}{60}{Lu 15:32 Eph 2:1,\allowbreak5 1Ti 5:6 Re 3:1}
\crossref{Luke}{9}{61}{Lu 14:18-\allowbreak20,\allowbreak26 De 33:9 1Ki 19:20 Ec 9:10 Mt 10:37,\allowbreak38}
\crossref{Luke}{9}{62}{Lu 17:31,\allowbreak32 Ps 78:8,\allowbreak9 Ac 15:37,\allowbreak38 2Ti 4:10 Heb 10:38 Jas 1:6-\allowbreak8}
\crossref{Luke}{10}{1}{Mt 10:1 etc.}
\crossref{Luke}{10}{2}{Mt 9:37,\allowbreak38 Joh 4:35-\allowbreak38 1Co 3:6-\allowbreak9}
\crossref{Luke}{10}{3}{Ps 22:12-\allowbreak16,\allowbreak21 Eze 2:3-\allowbreak6 Mt 10:16,\allowbreak22 Joh 15:20; 16:2 Ac 9:2,\allowbreak16}
\crossref{Luke}{10}{4}{Lu 9:3 etc.}
\crossref{Luke}{10}{5}{Lu 19:9 1Sa 25:6 Isa 57:19 Mt 10:12,\allowbreak13 Ac 10:36 2Co 5:18-\allowbreak20}
\crossref{Luke}{10}{6}{1Sa 25:17 Isa 9:6 Eph 2:2,\allowbreak3; 5:6 2Th 3:16 1Pe 1:14}
\crossref{Luke}{10}{7}{Lu 9:4 Mt 10:11 Mr 6:10 Ac 16:15,\allowbreak34,\allowbreak40}
\crossref{Luke}{10}{8}{10:10; 9:48 Mt 10:40 Joh 13:20}
\crossref{Luke}{10}{9}{Lu 9:2 Mt 10:8 Mr 6:13 Ac 28:7-\allowbreak10}
\crossref{Luke}{10}{10}{Lu 9:5 Mt 10:14 Ac 13:51; 18:6}
\crossref{Luke}{10}{11}{10:9 De 30:11-\allowbreak14 Ac 13:26,\allowbreak40,\allowbreak46 Ro 10:8,\allowbreak21 Heb 1:3}
\crossref{Luke}{10}{12}{La 4:6 Eze 16:48-\allowbreak50 Mt 10:15; 11:24 Mr 6:11}
\crossref{Luke}{10}{13}{Mt 11:20-\allowbreak22}
\crossref{Luke}{10}{14}{Lu 12:47,\allowbreak48 Am 3:2 Joh 3:19; 15:22-\allowbreak25 Ro 2:1,\allowbreak27}
\crossref{Luke}{10}{15}{Lu 7:1,\allowbreak2 Mt 4:13}
\crossref{Luke}{10}{16}{Lu 9:48 Mt 10:40; 18:5 Mr 9:37 Joh 12:44,\allowbreak48; 13:20 1Th 4:8}
\crossref{Luke}{10}{17}{10:1,\allowbreak9; 9:1 Ro 16:20}
\crossref{Luke}{10}{18}{Joh 12:31; 16:11 Heb 2:14 1Jo 3:8 Re 9:1; 12:7-\allowbreak9; 20:2}
\crossref{Luke}{10}{19}{Ps 91:13 Isa 11:8 Eze 2:6 Mr 16:18 Ac 28:5 Ro 16:20}
\crossref{Luke}{10}{20}{Mt 7:22,\allowbreak23; 10:1; 26:24; 27:5 1Co 13:2,\allowbreak3}
\crossref{Luke}{10}{21}{Lu 15:5,\allowbreak9 Isa 53:11; 62:5 Zep 3:17}
\crossref{Luke}{10}{22}{Joh 1:18; 6:44-\allowbreak46; 10:15; 17:5,\allowbreak26 2Co 4:6 1Jo 5:20 2Jo 1:9}
\crossref{Luke}{10}{23}{Mt 13:16,\allowbreak17}
\crossref{Luke}{10}{24}{Joh 8:56 Heb 11:13,\allowbreak39 1Pe 1:10,\allowbreak11}
\crossref{Luke}{10}{25}{Lu 7:30; 11:45,\allowbreak46 Mt 22:35}
\crossref{Luke}{10}{26}{Isa 8:20 Ro 3:19; 4:14-\allowbreak16; 10:5 Ga 3:12,\allowbreak13,\allowbreak21,\allowbreak22}
\crossref{Luke}{10}{27}{De 6:5; 10:12; 30:6 Mt 22:37-\allowbreak40 Mr 12:30,\allowbreak31,\allowbreak33,\allowbreak34 Heb 8:10}
\crossref{Luke}{10}{28}{Lu 7:43 Mr 12:34}
\crossref{Luke}{10}{29}{Lu 16:15; 18:9-\allowbreak11 Le 19:34 Job 32:2 Ro 4:2; 10:3 Ga 3:11 Jas 2:24}
\crossref{Luke}{10}{30}{Ps 88:4 Jer 51:52 La 2:12 Eze 30:24}
\crossref{Luke}{10}{31}{Ru 2:3}
\crossref{Luke}{10}{32}{Ps 109:25 Pr 27:10 Ac 18:17 2Ti 3:2}
\crossref{Luke}{10}{33}{Lu 9:52,\allowbreak53; 17:16-\allowbreak18 Pr 27:10 Jer 38:7-\allowbreak13; 39:16-\allowbreak18 Joh 4:9; 8:48}
\crossref{Luke}{10}{34}{10:34 Ex 23:4,\allowbreak5 Pr 24:17,\allowbreak18; 25:21,\allowbreak22 Mt 5:43-\allowbreak45 Ro 12:20 1Th 5:15}
\crossref{Luke}{10}{35}{Mt 20:2}
\crossref{Luke}{10}{36}{Lu 7:42 Mt 17:25; 21:28-\allowbreak31; 22:42}
\crossref{Luke}{10}{37}{Pr 14:21 Ho 6:6 Mic 6:8 Mt 20:28; 23:23 2Co 8:9 Eph 3:18,\allowbreak19; 5:2}
\crossref{Luke}{10}{38}{Joh 11:1-\allowbreak5; 12:1-\allowbreak3}
\crossref{Luke}{10}{39}{Lu 2:46; 8:35 De 33:3 Pr 8:34 Ac 22:3 1Co 7:32 etc.}
\crossref{Luke}{10}{40}{Lu 12:29 Joh 6:27}
\crossref{Luke}{10}{41}{Lu 8:14; 21:34 Mr 4:19 1Co 7:32-\allowbreak35 Php 4:6}
\crossref{Luke}{10}{42}{Lu 18:22 Ps 27:4; 73:25 Ec 12:13 Mr 8:36 Joh 17:3 1Co 13:3 Ga 5:6}
\crossref{Luke}{11}{1}{Lu 6:12; 9:18,\allowbreak28; 22:39-\allowbreak45 Heb 5:7}
\crossref{Luke}{11}{2}{Ec 5:2 Ho 14:2 Mt 6:6-\allowbreak8}
\crossref{Luke}{11}{3}{Ex 16:15-\allowbreak22 Pr 30:8 Isa 33:16 Mt 6:11,\allowbreak34 Joh 6:27-\allowbreak33}
\crossref{Luke}{11}{4}{1Ki 8:34,\allowbreak36 Ps 25:11,\allowbreak18; 32:1-\allowbreak5; 51:1-\allowbreak3; 130:3,\allowbreak4 Isa 43:25,\allowbreak26}
\crossref{Luke}{11}{5}{Lu 18:1-\allowbreak8}
\crossref{Luke}{11}{6}{11:6}
\crossref{Luke}{11}{7}{Lu 7:6 Ga 6:17}
\crossref{Luke}{11}{8}{Lu 18:1-\allowbreak8 Ge 32:26 Mt 15:22-\allowbreak28 Ro 15:30 2Co 12:8 Col 2:1; 4:12}
\crossref{Luke}{11}{9}{Lu 13:24 Mt 6:29; 21:31 Mr 13:37 Re 2:24}
\crossref{Luke}{11}{10}{Lu 18:1 Ps 31:22 La 3:8,\allowbreak18,\allowbreak54-\allowbreak58 Jon 2:2-\allowbreak8 Jas 4:3; 5:11}
\crossref{Luke}{11}{11}{Isa 49:15 Mt 7:9}
\crossref{Luke}{11}{12}{Lu 10:19 Eze 2:6 Re 9:10}
\crossref{Luke}{11}{13}{Ge 6:5,\allowbreak6; 8:21 Job 15:14-\allowbreak16 Ps 51:5 Joh 3:5,\allowbreak6 Ro 7:18 Tit 3:3}
\crossref{Luke}{11}{14}{Mt 9:32,\allowbreak33; 12:22,\allowbreak23 Mr 7:32-\allowbreak37}
\crossref{Luke}{11}{15}{Mt 9:34; 12:24-\allowbreak30 Mr 3:22-\allowbreak30 Joh 7:20; 8:48,\allowbreak52; 10:20}
\crossref{Luke}{11}{16}{Mt 12:38,\allowbreak39; 16:1-\allowbreak4 Mr 8:11,\allowbreak12 Joh 6:30 1Co 1:22}
\crossref{Luke}{11}{17}{Mt 9:4; 12:25 Mr 3:23-\allowbreak26 Joh 2:25 Re 2:23}
\crossref{Luke}{11}{18}{Mt 12:26}
\crossref{Luke}{11}{19}{Lu 9:49 Mt 12:27,\allowbreak28}
\crossref{Luke}{11}{20}{Ex 8:19 Mt 12:28}
\crossref{Luke}{11}{21}{Mt 12:29 Mr 3:27}
\crossref{Luke}{11}{22}{Ge 3:15 Isa 27:1; 49:24,\allowbreak25; 53:12; 63:1-\allowbreak4 Col 2:15 1Jo 3:8; 4:4}
\crossref{Luke}{11}{23}{Lu 9:50 Mt 12:30 Re 3:15,\allowbreak16}
\crossref{Luke}{11}{24}{Mt 12:43-\allowbreak45}
\crossref{Luke}{11}{25}{2Ch 24:17-\allowbreak22 Ps 36:3; 81:11,\allowbreak12; 125:5 Mt 12:44,\allowbreak45 2Th 2:9-\allowbreak12}
\crossref{Luke}{11}{26}{Mt 23:15}
\crossref{Luke}{11}{27}{Lu 1:28,\allowbreak42,\allowbreak48}
\crossref{Luke}{11}{28}{Lu 6:47,\allowbreak48; 8:21 Ps 1:1-\allowbreak3; 112:1; 119:1-\allowbreak6; 128:1 Isa 48:17,\allowbreak18}
\crossref{Luke}{11}{29}{Lu 12:1; 14:25,\allowbreak26}
\crossref{Luke}{11}{30}{Lu 24:46,\allowbreak47 Jon 1:17; 2:10; 3:2 etc.}
\crossref{Luke}{11}{31}{1Ki 10:1,\allowbreak2 etc.}
\crossref{Luke}{11}{32}{Jon 3:5-\allowbreak10}
\crossref{Luke}{11}{33}{Lu 8:16,\allowbreak17 Mt 5:15 Mr 4:21,\allowbreak22}
\crossref{Luke}{11}{34}{Ps 119:18 Mt 6:22,\allowbreak23 Mr 8:18 Ac 26:18 Eph 1:17}
\crossref{Luke}{11}{35}{Pr 16:25; 26:12 Isa 5:20,\allowbreak21 Jer 8:8,\allowbreak9 Joh 7:48,\allowbreak49; 9:39-\allowbreak41}
\crossref{Luke}{11}{36}{Ps 119:97-\allowbreak105 Pr 1:5; 2:1-\allowbreak11; 4:18,\allowbreak19; 6:23; 20:27 Isa 8:20; 42:16}
\crossref{Luke}{11}{37}{Lu 7:36; 14:1 1Co 9:19-\allowbreak23}
\crossref{Luke}{11}{38}{Mt 15:2,\allowbreak3 Mr 7:2-\allowbreak5 Joh 3:25}
\crossref{Luke}{11}{39}{Mt 23:25 Ga 1:14 2Ti 3:5 Tit 1:15}
\crossref{Luke}{11}{40}{Lu 12:20; 24:25 Ps 14:1; 75:4,\allowbreak5; 94:8 Pr 1:22; 8:5 Jer 5:21 Mt 23:17}
\crossref{Luke}{11}{41}{Lu 12:33; 14:12-\allowbreak14; 16:9; 18:22; 19:8 De 15:8-\allowbreak10 Job 13:16-\allowbreak20 Ps 41:1}
\crossref{Luke}{11}{42}{Mt 23:13,\allowbreak23,\allowbreak27}
\crossref{Luke}{11}{43}{Lu 14:7-\allowbreak11; 20:46 Pr 16:18 Mt 23:6 Mr 12:38,\allowbreak39 Ro 12:10 Php 2:3}
\crossref{Luke}{11}{44}{Nu 19:16 Ps 5:9 Ho 9:8 Mt 23:27,\allowbreak28 Ac 23:3}
\crossref{Luke}{11}{45}{1Ki 22:8 Jer 6:10; 20:8 Am 7:10-\allowbreak13 Joh 7:7,\allowbreak48; 9:40}
\crossref{Luke}{11}{46}{Isa 10:1 Mt 23:2-\allowbreak4 Ga 6:13}
\crossref{Luke}{11}{47}{Mt 23:29-\allowbreak33 Ac 7:51 1Th 2:15}
\crossref{Luke}{11}{48}{Jos 24:22 Job 15:6 Ps 64:8 Eze 18:19}
\crossref{Luke}{11}{49}{Lu 24:47 Mt 23:34 Ac 1:8 Eph 4:11}
\crossref{Luke}{11}{50}{Ge 9:5,\allowbreak6 Nu 35:33 2Ki 24:4 Ps 9:12 Isa 26:21 Re 18:20-\allowbreak24}
\crossref{Luke}{11}{51}{Ge 4:8-\allowbreak11 Heb 11:4; 12:24 1Jo 3:12}
\crossref{Luke}{11}{52}{Lu 19:39,\allowbreak40 Mal 2:7 Mt 23:13 Joh 7:47-\allowbreak52; 9:24-\allowbreak34 Ac 4:17,\allowbreak18; 5:40}
\crossref{Luke}{11}{53}{Ps 22:12,\allowbreak13 Isa 9:12}
\crossref{Luke}{11}{54}{Ps 37:32,\allowbreak33; 56:5,\allowbreak6 Mt 22:15,\allowbreak18,\allowbreak35 Mr 12:13}
\crossref{Luke}{12}{1}{Lu 5:1,\allowbreak15; 6:17 Ac 21:20}
\crossref{Luke}{12}{2}{Lu 8:17 Ec 12:14 Mt 10:26 Mr 4:22 Ro 2:16 1Co 4:5 2Co 5:10}
\crossref{Luke}{12}{3}{Job 24:14,\allowbreak15 Ec 10:12,\allowbreak13,\allowbreak20 Mt 12:36 Jude 1:14,\allowbreak15}
\crossref{Luke}{12}{4}{So 5:1,\allowbreak16 Isa 41:8 Joh 15:14 Jas 2:23}
\crossref{Luke}{12}{5}{Mr 13:23 1Th 4:6}
\crossref{Luke}{12}{6}{Mt 10:29}
\crossref{Luke}{12}{7}{Lu 21:18 1Sa 14:45 2Sa 14:11 Mt 10:30 Ac 27:34}
\crossref{Luke}{12}{8}{1Sa 2:30 Ps 119:46 Mt 10:32,\allowbreak33 Ro 10:9,\allowbreak10 2Ti 2:12 1Jo 2:23}
\crossref{Luke}{12}{9}{Lu 9:26 Mt 10:33 Mr 8:38 Ac 3:13,\allowbreak14 2Ti 2:12 Re 3:8}
\crossref{Luke}{12}{10}{Lu 23:34 Mt 12:31,\allowbreak32 Mr 3:28,\allowbreak29 1Ti 1:13 Heb 6:4-\allowbreak8; 10:26-\allowbreak31}
\crossref{Luke}{12}{11}{Lu 21:12-\allowbreak14 Mt 10:17-\allowbreak20; 23:34 Mr 13:9-\allowbreak11 Ac 4:5-\allowbreak7; 5:27-\allowbreak32; 6:9-\allowbreak15}
\crossref{Luke}{12}{12}{Lu 21:15 Ex 4:11 Ac 4:8; 6:10; 7:2 etc.}
\crossref{Luke}{12}{13}{Lu 6:45 Ps 17:14 Eze 33:31 Ac 8:18,\allowbreak19 1Ti 6:5}
\crossref{Luke}{12}{14}{Lu 5:20; 22:58 Ro 2:1,\allowbreak3; 9:20}
\crossref{Luke}{12}{15}{Lu 8:14; 16:14; 21:34 Jos 7:21 Job 31:24,\allowbreak25 Ps 10:3; 62:10; 119:36,\allowbreak37}
\crossref{Luke}{12}{16}{Ge 26:12-\allowbreak14; 41:47-\allowbreak49 Job 12:6 Ps 73:3,\allowbreak12 Ho 2:8 Mt 5:45}
\crossref{Luke}{12}{17}{12:22,\allowbreak29; 10:25; 16:3 Ac 2:37; 16:30}
\crossref{Luke}{12}{18}{12:21; 18:4,\allowbreak6 Ps 17:14 Jas 3:15; 4:15}
\crossref{Luke}{12}{19}{De 6:11,\allowbreak12; 8:12-\allowbreak14 Job 31:24,\allowbreak25 Ps 49:5-\allowbreak13,\allowbreak18; 52:5-\allowbreak7; 62:10}
\crossref{Luke}{12}{20}{Lu 16:22,\allowbreak23 Ex 16:9,\allowbreak10 1Sa 25:36-\allowbreak38 2Sa 13:28,\allowbreak29 1Ki 16:9,\allowbreak10}
\crossref{Luke}{12}{21}{12:33; 6:24 Ho 10:1 Hab 2:9 Mt 6:19,\allowbreak20 Ro 2:5 1Ti 6:19 Jas 5:1-\allowbreak3}
\crossref{Luke}{12}{22}{12:29 Mt 6:25 etc.}
\crossref{Luke}{12}{23}{Ge 19:17 Job 1:12; 2:4,\allowbreak6 Pr 13:8 Ac 27:18,\allowbreak19,\allowbreak38}
\crossref{Luke}{12}{24}{12:7,\allowbreak30-\allowbreak32 Job 35:11 Mt 10:31}
\crossref{Luke}{12}{25}{Lu 19:3 Mt 5:36; 6:27}
\crossref{Luke}{12}{26}{12:29 Ps 39:6 Ec 7:13 1Pe 5:7}
\crossref{Luke}{12}{27}{12:24 Mt 6:28-\allowbreak30 Jas 1:10,\allowbreak11}
\crossref{Luke}{12}{28}{Isa 40:6 1Pe 1:24}
\crossref{Luke}{12}{29}{12:22; 10:7,\allowbreak8; 22:35 Mt 6:31}
\crossref{Luke}{12}{30}{Mt 5:47; 6:32 Eph 4:17 1Th 4:5 1Pe 4:2-\allowbreak4}
\crossref{Luke}{12}{31}{Lu 10:42 1Ki 3:11-\allowbreak13 Ps 34:9; 37:3,\allowbreak19,\allowbreak25; 84:11 Isa 33:16 Mt 6:33}
\crossref{Luke}{12}{32}{So 1:7,\allowbreak8 Isa 40:11; 41:14}
\crossref{Luke}{12}{33}{Lu 18:22 Mt 19:21 Ac 2:45; 4:34,\allowbreak35 2Co 8:2}
\crossref{Luke}{12}{34}{Mt 6:21 Php 3:20 Col 3:1-\allowbreak3}
\crossref{Luke}{12}{35}{1Ki 18:46 Pr 31:17 Isa 5:27; 11:5 Eph 6:14 1Pe 1:13}
\crossref{Luke}{12}{36}{Lu 2:25-\allowbreak30 Ge 49:18 Isa 64:4 La 3:25,\allowbreak26 Mt 24:42-\allowbreak44 Mr 13:34-\allowbreak37}
\crossref{Luke}{12}{37}{12:43; 21:36 Mt 24:45-\allowbreak47; 25:20-\allowbreak23 Php 1:21,\allowbreak23 2Ti 4:7,\allowbreak8}
\crossref{Luke}{12}{38}{Mt 25:6 1Th 5:4,\allowbreak5}
\crossref{Luke}{12}{39}{Mt 24:43,\allowbreak44 1Th 5:2,\allowbreak3 2Pe 3:10 Re 3:3; 16:15}
\crossref{Luke}{12}{40}{Lu 21:34-\allowbreak36 Mt 24:42,\allowbreak44; 25:13 Mr 13:33-\allowbreak36 Ro 13:11,\allowbreak14 1Th 5:6}
\crossref{Luke}{12}{41}{Mr 13:37; 14:37 1Pe 4:7; 5:8}
\crossref{Luke}{12}{42}{Lu 19:15-\allowbreak19 Mt 24:45,\allowbreak46; 25:20-\allowbreak23 1Co 4:1,\allowbreak2 Tit 1:7}
\crossref{Luke}{12}{43}{12:37}
\crossref{Luke}{12}{44}{Lu 19:17-\allowbreak19; 22:29,\allowbreak30 Da 12:2,\allowbreak3 Mt 24:47 Re 3:18}
\crossref{Luke}{12}{45}{Eze 12:22,\allowbreak27,\allowbreak28 Mt 24:48-\allowbreak50 2Pe 2:3,\allowbreak4}
\crossref{Luke}{12}{46}{12:19,\allowbreak20,\allowbreak40 Re 16:15}
\crossref{Luke}{12}{47}{Lu 10:12-\allowbreak15 Nu 15:30,\allowbreak31 Mt 11:22-\allowbreak24 Joh 9:41; 12:48; 15:22-\allowbreak24; 19:11}
\crossref{Luke}{12}{48}{Le 5:17 Ac 17:30 Ro 2:12-\allowbreak16 1Ti 1:13}
\crossref{Luke}{12}{49}{12:51,\allowbreak52 Isa 11:4 Joe 2:30,\allowbreak31 Mal 3:2,\allowbreak3; 4:1 Mt 3:10-\allowbreak12}
\crossref{Luke}{12}{50}{Mt 20:17-\allowbreak22 Mr 10:32-\allowbreak38}
\crossref{Luke}{12}{51}{12:49 Zec 11:7,\allowbreak8,\allowbreak10,\allowbreak11,\allowbreak14 Mt 10:34-\allowbreak36; 24:7-\allowbreak10}
\crossref{Luke}{12}{52}{Ps 41:9 Mic 7:5,\allowbreak6 Joh 7:41-\allowbreak43; 9:16; 10:19-\allowbreak21; 15:18-\allowbreak21; 16:2}
\crossref{Luke}{12}{53}{Mic 7:6 Zec 13:2-\allowbreak6 Mt 10:21,\allowbreak22; 24:10}
\crossref{Luke}{12}{54}{1Ki 18:44,\allowbreak45 Mt 16:2 etc.}
\crossref{Luke}{12}{55}{Job 37:17}
\crossref{Luke}{12}{56}{1Ch 12:32 Mt 11:25; 16:3; 24:32,\allowbreak33}
\crossref{Luke}{12}{57}{De 32:29 Mt 15:10-\allowbreak14; 21:21,\allowbreak32 Ac 2:40; 13:26-\allowbreak38 1Co 11:14}
\crossref{Luke}{12}{58}{Pr 25:8,\allowbreak9 Mt 5:23-\allowbreak26}
\crossref{Luke}{12}{59}{Lu 16:26 Mt 18:34; 25:41,\allowbreak46 2Th 1:3}
\crossref{Luke}{13}{1}{La 2:20 Eze 9:5-\allowbreak7 1Pe 4:17,\allowbreak18}
\crossref{Luke}{13}{2}{13:4 Job 22:5-\allowbreak16 Joh 9:2 Ac 28:4}
\crossref{Luke}{13}{3}{13:5; 24:47 Mt 3:2,\allowbreak10-\allowbreak12 Ac 2:38-\allowbreak40; 3:19 Re 2:21,\allowbreak22}
\crossref{Luke}{13}{4}{Ne 3:15 Joh 9:7,\allowbreak11}
\crossref{Luke}{13}{5}{13:3 Isa 28:10-\allowbreak13 Eze 18:30}
\crossref{Luke}{13}{6}{Ps 80:8-\allowbreak13 Isa 5:1-\allowbreak4 Jer 2:21 Mt 21:19,\allowbreak20 Mr 11:12-\allowbreak14}
\crossref{Luke}{13}{7}{Le 19:23; 25:21 Ro 2:4,\allowbreak5}
\crossref{Luke}{13}{8}{Ex 32:11-\allowbreak13,\allowbreak30-\allowbreak32; 34:9 Nu 14:11-\allowbreak20 Jos 7:7-\allowbreak9 Ps 106:23}
\crossref{Luke}{13}{9}{Ezr 9:14,\allowbreak15 Ps 69:22-\allowbreak28 Da 9:5-\allowbreak8 Joh 15:2 1Th 2:15 Heb 6:8}
\crossref{Luke}{13}{10}{Lu 4:15,\allowbreak16,\allowbreak44}
\crossref{Luke}{13}{11}{13:16; 8:2 Job 2:7 Ps 6:2 Mt 9:32,\allowbreak33}
\crossref{Luke}{13}{12}{Lu 6:8-\allowbreak10 Ps 107:20 Isa 65:1 Mt 8:16}
\crossref{Luke}{13}{13}{Lu 4:40 Mr 6:5; 8:25; 16:18 Ac 9:17}
\crossref{Luke}{13}{14}{Lu 8:41 Ac 13:15; 18:8,\allowbreak17}
\crossref{Luke}{13}{15}{Lu 6:42; 12:1 Job 34:30 Pr 11:9 Isa 29:20 Mt 7:5; 15:7,\allowbreak14; 23:13,\allowbreak28}
\crossref{Luke}{13}{16}{Lu 3:8; 16:24; 19:9 Ac 13:26 Ro 4:12-\allowbreak16}
\crossref{Luke}{13}{17}{Lu 14:6; 20:40 Ps 40:14; 109:29; 132:18 Isa 45:24 2Ti 3:9 1Pe 3:16}
\crossref{Luke}{13}{18}{13:20; 7:31 La 2:13 Mt 13:31}
\crossref{Luke}{13}{19}{Mt 13:31,\allowbreak32; 17:20 Mr 4:31,\allowbreak32}
\crossref{Luke}{13}{20}{}
\crossref{Luke}{13}{21}{Mt 13:33}
\crossref{Luke}{13}{22}{Lu 4:43,\allowbreak44 Mt 9:35 Mr 6:6 Ac 10:38}
\crossref{Luke}{13}{23}{Mt 7:14; 19:25; 20:16; 22:14}
\crossref{Luke}{13}{24}{Lu 21:36 Ge 32:25,\allowbreak26 Mt 11:12 Joh 6:27 1Co 9:24-\allowbreak27 Php 2:12,\allowbreak13}
\crossref{Luke}{13}{25}{Ps 32:6 Isa 55:6 2Co 6:2 Heb 3:7,\allowbreak8; 12:17}
\crossref{Luke}{13}{26}{Isa 58:2 2Ti 3:5 Tit 1:16}
\crossref{Luke}{13}{27}{13:25 Ps 1:6 Mt 7:22,\allowbreak23; 25:12,\allowbreak41 1Co 8:3 Ga 4:9 2Ti 2:19}
\crossref{Luke}{13}{28}{Ps 112:10 Mt 8:12; 13:42,\allowbreak50; 22:13; 24:51; 25:30}
\crossref{Luke}{13}{29}{Ge 28:14 Isa 43:6; 49:6; 54:2,\allowbreak3; 66:18-\allowbreak20 Mal 1:11 Mr 13:27}
\crossref{Luke}{13}{30}{Mt 3:9,\allowbreak10; 8:11,\allowbreak12; 19:30; 20:16; 21:28-\allowbreak31 Mr 10:31}
\crossref{Luke}{13}{31}{Ne 6:9-\allowbreak11 Ps 11:1,\allowbreak2 Am 7:12,\allowbreak13}
\crossref{Luke}{13}{32}{Lu 9:7 Mr 6:14 Joh 10:32; 11:8-\allowbreak10}
\crossref{Luke}{13}{33}{Joh 4:34; 9:4; 11:54; 12:35 Ac 10:38}
\crossref{Luke}{13}{34}{Lu 19:41,\allowbreak42 Mt 23:37-\allowbreak39}
\crossref{Luke}{13}{35}{Lu 21:5,\allowbreak6,\allowbreak24 Le 26:31,\allowbreak32 Ps 69:25 Isa 1:7,\allowbreak8; 5:5,\allowbreak6; 64:10,\allowbreak11}
\crossref{Luke}{14}{1}{Lu 7:34-\allowbreak36; 11:37 1Co 9:19-\allowbreak22}
\crossref{Luke}{14}{2}{}
\crossref{Luke}{14}{3}{Lu 11:44,\allowbreak45}
\crossref{Luke}{14}{4}{Mt 21:25-\allowbreak27; 22:46}
\crossref{Luke}{14}{5}{Lu 13:15 Ex 23:4,\allowbreak5 Da 4:24 Mt 12:11,\allowbreak12}
\crossref{Luke}{14}{6}{Lu 13:17; 20:26,\allowbreak40; 21:15 Ac 6:10}
\crossref{Luke}{14}{7}{Jud 14:12 Pr 8:1 Eze 17:2 Mt 13:34}
\crossref{Luke}{14}{8}{}
\crossref{Luke}{14}{9}{Es 6:6-\allowbreak12 Pr 3:35; 11:2; 16:18 Eze 28:2-\allowbreak10 Da 4:30-\allowbreak34}
\crossref{Luke}{14}{10}{1Sa 15:17 Pr 15:33; 25:6,\allowbreak7}
\crossref{Luke}{14}{11}{Lu 1:51; 18:14 1Sa 15:17 Job 22:29; 40:10-\allowbreak12 Ps 18:27; 138:6}
\crossref{Luke}{14}{12}{Lu 1:53 Pr 14:20; 22:16 Jas 2:1-\allowbreak6}
\crossref{Luke}{14}{13}{14:21; 11:41 De 14:29; 16:11,\allowbreak14; 26:12,\allowbreak13 2Sa 6:19 2Ch 30:24 Ne 8:10,\allowbreak12}
\crossref{Luke}{14}{14}{Pr 19:17 Mt 6:4; 10:41,\allowbreak42; 25:34-\allowbreak40 Php 4:18,\allowbreak19}
\crossref{Luke}{14}{15}{Lu 12:37; 13:29; 22:30 Mt 8:11; 25:10 Joh 6:27 etc.}
\crossref{Luke}{14}{16}{Pr 9:1,\allowbreak2 Isa 25:6,\allowbreak7 Jer 31:12-\allowbreak14 Zec 10:7 Mt 22:2-\allowbreak14}
\crossref{Luke}{14}{17}{Lu 3:4-\allowbreak6; 9:1-\allowbreak5; 10:1 etc.}
\crossref{Luke}{14}{18}{Lu 20:4,\allowbreak5 Isa 28:12,\allowbreak13; 29:11,\allowbreak12 Jer 5:4,\allowbreak5; 6:10,\allowbreak16,\allowbreak17 Mt 22:5,\allowbreak6}
\crossref{Luke}{14}{19}{14:19}
\crossref{Luke}{14}{20}{14:26-\allowbreak28; 18:29,\allowbreak30 1Co 7:29-\allowbreak31,\allowbreak33}
\crossref{Luke}{14}{21}{Lu 9:10 1Sa 25:12 Mt 15:12; 18:31 Heb 13:17}
\crossref{Luke}{14}{22}{Ac 1:1-\allowbreak9:43}
\crossref{Luke}{14}{23}{Ps 98:3 Isa 11:10; 19:24,\allowbreak25; 27:13; 49:5,\allowbreak6; 66:19,\allowbreak20 Zec 14:8,\allowbreak9}
\crossref{Luke}{14}{24}{Pr 1:24-\allowbreak32 Mt 21:43; 22:8; 23:38,\allowbreak39 Joh 3:19,\allowbreak36; 8:21,\allowbreak24 Ac 13:46}
\crossref{Luke}{14}{25}{Lu 12:1 Joh 6:24-\allowbreak27}
\crossref{Luke}{14}{26}{De 13:6-\allowbreak8; 33:9 Ps 73:25,\allowbreak26 Mt 10:37 Php 3:8}
\crossref{Luke}{14}{27}{Lu 9:23-\allowbreak25 Mt 10:38; 16:24-\allowbreak26 Mr 8:34-\allowbreak37; 10:21; 15:21 Joh 19:17}
\crossref{Luke}{14}{28}{Ge 11:4-\allowbreak9 Pr 24:27}
\crossref{Luke}{14}{29}{14:29}
\crossref{Luke}{14}{30}{Mt 7:27; 27:3-\allowbreak8 Ac 1:18,\allowbreak19 1Co 3:11-\allowbreak14 Heb 6:4-\allowbreak8,\allowbreak11; 10:38}
\crossref{Luke}{14}{31}{1Ki 20:11 2Ki 18:20-\allowbreak22 Pr 20:18; 25:8}
\crossref{Luke}{14}{32}{Lu 12:58 1Ki 20:31-\allowbreak34 2Ki 10:4,\allowbreak5 Job 40:9 Mt 5:25 Ac 12:20}
\crossref{Luke}{14}{33}{14:26; 5:11,\allowbreak28; 18:22,\allowbreak23,\allowbreak28-\allowbreak30 Ac 5:1-\allowbreak5; 8:19-\allowbreak22 Php 3:7,\allowbreak8 2Ti 4:10}
\crossref{Luke}{14}{34}{Mt 5:13 Mr 9:49,\allowbreak50 Col 4:6 Heb 2:4-\allowbreak8}
\crossref{Luke}{14}{35}{Joh 15:6}
\crossref{Luke}{15}{1}{Lu 5:29-\allowbreak32; 7:29; 13:30 Eze 18:27 Mt 9:10-\allowbreak13; 21:28-\allowbreak31 Ro 5:20}
\crossref{Luke}{15}{2}{15:29,\allowbreak30; 5:30; 7:34,\allowbreak39; 19:7 Mt 9:11 Ac 11:3 1Co 5:9-\allowbreak11 Ga 2:12}
\crossref{Luke}{15}{3}{}
\crossref{Luke}{15}{4}{Lu 13:15 Mt 12:11; 18:12 Ro 2:1}
\crossref{Luke}{15}{5}{Lu 19:9; 23:43 Isa 62:12 Joh 4:34,\allowbreak35 Ac 9:1-\allowbreak16 Ro 10:20,\allowbreak21}
\crossref{Luke}{15}{6}{15:7,\allowbreak10,\allowbreak24; 2:13,\allowbreak14 Isa 66:10,\allowbreak11 Joh 3:29; 15:14 Ac 11:23; 15:3}
\crossref{Luke}{15}{7}{15:32; 5:32 Mt 18:13}
\crossref{Luke}{15}{8}{Lu 19:10 Eze 34:12 Joh 10:16; 11:52 Eph 2:17}
\crossref{Luke}{15}{9}{15:6,\allowbreak7}
\crossref{Luke}{15}{10}{Lu 2:1-\allowbreak14 Eze 18:23,\allowbreak32; 33:11 Mt 18:10,\allowbreak11; 28:5-\allowbreak7 Ac 5:19; 10:3-\allowbreak5}
\crossref{Luke}{15}{11}{Mt 21:23-\allowbreak31}
\crossref{Luke}{15}{12}{De 21:16,\allowbreak17 Ps 16:5,\allowbreak6; 17:14}
\crossref{Luke}{15}{13}{2Ch 33:1-\allowbreak10 Job 21:13-\allowbreak15; 22:17,\allowbreak18 Ps 10:4-\allowbreak6; 73:27 Pr 27:8}
\crossref{Luke}{15}{14}{2Ch 33:11 Eze 16:27 Ho 2:9-\allowbreak14 Am 8:9-\allowbreak12}
\crossref{Luke}{15}{15}{15:13 Ex 10:3 2Ch 28:22 Isa 1:5,\allowbreak9,\allowbreak10-\allowbreak13; 57:17 Jer 5:3; 8:4-\allowbreak6}
\crossref{Luke}{15}{16}{Isa 44:20; 55:2 La 4:5 Ho 12:1 Ro 6:19-\allowbreak21}
\crossref{Luke}{15}{17}{Lu 8:35; 16:23 Ps 73:20 Ec 9:3 Jer 31:19 Eze 18:28 Ac 2:37; 16:29}
\crossref{Luke}{15}{18}{1Ki 20:30,\allowbreak31 2Ki 7:3,\allowbreak4 2Ch 33:12,\allowbreak13,\allowbreak19 Ps 32:5; 116:3-\allowbreak7}
\crossref{Luke}{15}{19}{Lu 5:8; 7:6,\allowbreak7 Ge 32:10 Job 42:6 1Co 15:9 1Ti 1:13-\allowbreak16}
\crossref{Luke}{15}{20}{De 30:2-\allowbreak4 Job 33:27,\allowbreak28 Ps 86:5,\allowbreak15; 103:10-\allowbreak13 Isa 49:15; 55:6-\allowbreak9}
\crossref{Luke}{15}{21}{15:18,\allowbreak19 Jer 3:13 Eze 16:63 Ro 2:4}
\crossref{Luke}{15}{22}{Ps 45:13; 132:9,\allowbreak16 Isa 61:10 Eze 16:9-\allowbreak13 Zec 3:3-\allowbreak5 Mt 22:11,\allowbreak12}
\crossref{Luke}{15}{23}{Ge 18:7 Ps 63:5 Pr 9:2 Isa 25:6; 65:13,\allowbreak14 Mt 22:2 etc.}
\crossref{Luke}{15}{24}{15:32 Mr 8:22 Joh 5:21,\allowbreak24,\allowbreak25; 11:25 Ro 6:11,\allowbreak13; 8:2 2Co 5:14,\allowbreak15}
\crossref{Luke}{15}{25}{15:11,\allowbreak12}
\crossref{Luke}{15}{26}{}
\crossref{Luke}{15}{27}{15:30 Ac 9:17; 22:13 Phm 1:16}
\crossref{Luke}{15}{28}{15:2; 5:30; 7:39 1Sa 17:28; 18:8 Isa 65:5; 66:5 Jon 4:1-\allowbreak3 Mt 20:11}
\crossref{Luke}{15}{29}{Lu 17:10; 18:9,\allowbreak11,\allowbreak12,\allowbreak20,\allowbreak21 1Sa 15:13,\allowbreak14 Isa 58:2,\allowbreak3; 65:5 Zec 7:3}
\crossref{Luke}{15}{30}{15:32; 18:11 Ex 32:7,\allowbreak11}
\crossref{Luke}{15}{31}{Lu 19:22,\allowbreak23 Mt 20:13-\allowbreak16 Mr 7:27,\allowbreak28 Ro 9:4; 11:1,\allowbreak35}
\crossref{Luke}{15}{32}{Lu 7:34 Ps 51:8 Isa 35:10 Ho 14:9 Jon 4:10,\allowbreak11 Ro 3:4,\allowbreak19; 15:9-\allowbreak13}
\crossref{Luke}{16}{1}{Mt 18:23,\allowbreak24; 25:14 etc.}
\crossref{Luke}{16}{2}{Ge 3:9-\allowbreak11; 4:9,\allowbreak10; 18:20,\allowbreak21 1Sa 2:23,\allowbreak24 1Co 1:11 1Ti 5:24}
\crossref{Luke}{16}{3}{Lu 18:4 Es 6:6}
\crossref{Luke}{16}{4}{Pr 30:9 Jer 4:22 Jas 3:15}
\crossref{Luke}{16}{5}{Lu 7:41,\allowbreak42 Mt 18:24}
\crossref{Luke}{16}{6}{16:9,\allowbreak12 Tit 2:10}
\crossref{Luke}{16}{7}{Lu 20:9,\allowbreak12 So 8:11,\allowbreak12}
\crossref{Luke}{16}{8}{16:10; 18:6}
\crossref{Luke}{16}{9}{Lu 11:41; 14:14 Pr 19:17 Ec 11:1 Isa 58:7,\allowbreak8 Da 4:27 Mt 6:19; 19:21}
\crossref{Luke}{16}{10}{16:11,\allowbreak12; 19:17 Mt 25:21 Heb 3:2}
\crossref{Luke}{16}{11}{16:9}
\crossref{Luke}{16}{12}{Lu 19:13-\allowbreak26 1Ch 29:14-\allowbreak16 Job 1:21 Eze 16:16-\allowbreak21 Ho 2:8 Mt 25:14-\allowbreak29}
\crossref{Luke}{16}{13}{Lu 9:50; 11:23 Jos 24:15 Mt 4:10; 6:24 Ro 6:16-\allowbreak22; 8:5-\allowbreak8 Jas 4:4}
\crossref{Luke}{16}{14}{Lu 12:15; 20:47 Isa 56:11 Jer 6:13; 8:10 Eze 22:25-\allowbreak29; 33:31}
\crossref{Luke}{16}{15}{Lu 10:29; 11:39; 18:11,\allowbreak21; 20:20,\allowbreak47 Pr 20:6 Mt 6:2,\allowbreak5,\allowbreak16; 23:5,\allowbreak25-\allowbreak27}
\crossref{Luke}{16}{16}{16:29,\allowbreak31 Mt 11:9-\allowbreak14 Joh 1:45 Ac 3:18,\allowbreak24,\allowbreak25}
\crossref{Luke}{16}{17}{Lu 21:33 Ps 102:25-\allowbreak27 Isa 51:6 Mt 5:18 2Pe 3:10 Re 20:11; 21:1,\allowbreak4}
\crossref{Luke}{16}{18}{Mt 5:32; 19:9 Mr 10:11,\allowbreak12 1Co 7:4,\allowbreak10-\allowbreak12}
\crossref{Luke}{16}{19}{Lu 12:16-\allowbreak21; 18:24,\allowbreak25 Jas 5:1-\allowbreak5}
\crossref{Luke}{16}{20}{Lu 18:35-\allowbreak43 1Sa 2:8 Jas 1:9; 2:5}
\crossref{Luke}{16}{21}{1Co 4:11 2Co 11:27}
\crossref{Luke}{16}{22}{Job 3:13-\allowbreak19 Isa 57:1,\allowbreak2 Re 14:13}
\crossref{Luke}{16}{23}{Ps 9:17; 16:10; 49:15; 86:13 Pr 5:5; 7:27; 9:18; 15:24 Isa 14:9,\allowbreak15}
\crossref{Luke}{16}{24}{16:30; 3:8 Mt 3:9 Joh 8:33-\allowbreak39,\allowbreak53-\allowbreak56 Ro 4:12; 9:7,\allowbreak8}
\crossref{Luke}{16}{25}{16:24}
\crossref{Luke}{16}{26}{1Sa 25:36 Ps 49:14 Eze 28:24 Mal 3:18 2Th 1:4-\allowbreak10 Jas 1:11,\allowbreak12}
\crossref{Luke}{16}{27}{}
\crossref{Luke}{16}{28}{Ps 49:12,\allowbreak13}
\crossref{Luke}{16}{29}{16:16 Isa 8:20; 34:16 Mal 4:2-\allowbreak4 Joh 5:39-\allowbreak45 Ac 15:21; 17:11,\allowbreak12}
\crossref{Luke}{16}{30}{Lu 13:3,\allowbreak5 Re 16:9-\allowbreak11}
\crossref{Luke}{16}{31}{Joh 11:43-\allowbreak53; 12:10,\allowbreak11 2Co 4:3}
\crossref{Luke}{17}{1}{Mt 16:23; 18:7 Ro 14:13,\allowbreak20,\allowbreak21; 16:17 1Co 8:13; 10:32; 11:19}
\crossref{Luke}{17}{2}{Mt 18:6; 26:24 Mr 9:42 1Co 9:15 2Pe 2:1-\allowbreak3}
\crossref{Luke}{17}{3}{Lu 21:34 Ex 34:12 De 4:9,\allowbreak15,\allowbreak23 2Ch 19:6,\allowbreak7 Eph 5:15 Heb 12:15}
\crossref{Luke}{17}{4}{Mt 18:21,\allowbreak22,\allowbreak35 1Co 13:4-\allowbreak7 Eph 4:31,\allowbreak32 Col 3:12,\allowbreak13}
\crossref{Luke}{17}{5}{Mr 9:24 2Co 12:8-\allowbreak10 Php 4:13 2Th 1:3 Heb 12:2 1Pe 1:22,\allowbreak23}
\crossref{Luke}{17}{6}{Mt 17:20,\allowbreak21; 21:21 Mr 9:23; 11:22,\allowbreak23 1Co 13:2}
\crossref{Luke}{17}{7}{Lu 13:15; 14:5 Mt 12:11}
\crossref{Luke}{17}{8}{Ge 43:16 2Sa 12:20}
\crossref{Luke}{17}{9}{17:9}
\crossref{Luke}{17}{10}{1Ch 29:14-\allowbreak16 Job 22:2,\allowbreak3; 35:6,\allowbreak7 Ps 16:2,\allowbreak3; 35:6,\allowbreak7 Pr 16:2,\allowbreak3}
\crossref{Luke}{17}{11}{Lu 9:51,\allowbreak52 Joh 4:4}
\crossref{Luke}{17}{12}{Lu 5:12; 18:13 Le 13:45,\allowbreak46 Nu 5:2,\allowbreak3; 12:14 2Ki 5:27; 7:3}
\crossref{Luke}{17}{13}{Lu 18:38,\allowbreak39 Mt 9:27; 15:22; 20:30,\allowbreak31 Mr 9:22}
\crossref{Luke}{17}{14}{Lu 5:14 Le 13:1,\allowbreak2 etc.}
\crossref{Luke}{17}{15}{17:17,\allowbreak18 2Ch 32:24-\allowbreak26 Ps 30:1,\allowbreak2,\allowbreak11,\allowbreak12; 103:1-\allowbreak4; 107:20-\allowbreak22; 116:12-\allowbreak15}
\crossref{Luke}{17}{16}{Lu 5:8 Ge 17:3 Mt 2:11 Mr 5:33 Joh 5:23 Ac 10:25,\allowbreak26 Re 4:10; 5:14}
\crossref{Luke}{17}{17}{Ge 3:9 Ps 106:13 Joh 8:7-\allowbreak10 Ro 1:21}
\crossref{Luke}{17}{18}{Ps 29:1,\allowbreak2; 50:23; 106:13 Isa 42:12 Re 14:7}
\crossref{Luke}{17}{19}{Lu 7:50; 8:48; 18:42 Mt 9:22 Mr 5:34; 10:52}
\crossref{Luke}{17}{20}{Lu 10:11; 16:16; 19:11 Ac 1:6,\allowbreak7}
\crossref{Luke}{17}{21}{Lu 21:8 Mt 24:23-\allowbreak28 Mr 13:21}
\crossref{Luke}{17}{22}{Lu 5:35; 13:35 Mt 9:15 Joh 7:33-\allowbreak36; 8:21-\allowbreak24; 12:35; 13:33; 16:5-\allowbreak7}
\crossref{Luke}{17}{23}{17:21; 21:8 Mt 24:23-\allowbreak26 Mr 13:21-\allowbreak23}
\crossref{Luke}{17}{24}{Job 37:3,\allowbreak4 Zec 9:14 Mt 24:27}
\crossref{Luke}{17}{25}{Lu 9:22; 18:31,\allowbreak33; 24:25,\allowbreak26,\allowbreak46 Mt 16:21; 17:22,\allowbreak23; 20:18,\allowbreak19 Mr 8:31}
\crossref{Luke}{17}{26}{Ge 7:7-\allowbreak23}
\crossref{Luke}{17}{27}{Lu 12:19,\allowbreak20; 16:19-\allowbreak23 De 6:10-\allowbreak12; 8:12-\allowbreak14 1Sa 25:36-\allowbreak38 Job 21:9-\allowbreak13}
\crossref{Luke}{17}{28}{Ge 13:13; 18:20,\allowbreak21; 19:4-\allowbreak15 Eze 16:49,\allowbreak50 Jas 5:1-\allowbreak5}
\crossref{Luke}{17}{29}{Ge 19:16-\allowbreak25 De 29:23-\allowbreak25 Isa 1:9; 13:19 Jer 50:40 Ho 11:8}
\crossref{Luke}{17}{30}{17:24; 21:22,\allowbreak27,\allowbreak34-\allowbreak36 Mt 24:3,\allowbreak27-\allowbreak31; 26:64 Mr 13:26 2Th 1:7 1Pe 1:13}
\crossref{Luke}{17}{31}{}
\crossref{Luke}{17}{32}{Ge 19:17,\allowbreak26 1Co 10:6-\allowbreak12 Heb 10:38,\allowbreak39 2Pe 2:18-\allowbreak22}
\crossref{Luke}{17}{33}{Lu 9:24,\allowbreak25 Mt 10:39; 16:25 Mr 8:35-\allowbreak37 Joh 12:25 Re 2:10}
\crossref{Luke}{17}{34}{Lu 13:3,\allowbreak5,\allowbreak24 Isa 42:9 Mt 24:25 Mr 13:23; 14:29}
\crossref{Luke}{17}{35}{Ex 11:5 Jud 16:21}
% skipped verse
%\crossref{Luke}{17}{36}{17:36}
\crossref{Luke}{17}{37}{Job 39:29,\allowbreak30 Da 9:26,\allowbreak27 Am 9:1-\allowbreak4 Zec 13:8,\allowbreak9; 14:2 Mt 24:28}
\crossref{Luke}{18}{1}{Lu 11:5-\allowbreak8; 21:36 Ge 32:9-\allowbreak12,\allowbreak24-\allowbreak26 Job 27:8-\allowbreak10 Ps 55:16,\allowbreak17; 65:2}
\crossref{Luke}{18}{2}{18:4 Ex 18:21,\allowbreak22 2Ch 19:3-\allowbreak9 Job 29:7-\allowbreak17 Ps 8:1-\allowbreak4 Jer 22:16,\allowbreak17}
\crossref{Luke}{18}{3}{De 27:19 2Sa 14:5 etc.}
\crossref{Luke}{18}{4}{Lu 12:17; 16:3 Heb 4:12,\allowbreak13}
\crossref{Luke}{18}{5}{Lu 11:8 Jud 16:16 2Sa 13:24-\allowbreak27}
\crossref{Luke}{18}{6}{}
\crossref{Luke}{18}{7}{Lu 11:13 Mt 7:11}
\crossref{Luke}{18}{8}{Ps 46:5; 143:7-\allowbreak9 2Pe 2:3; 3:8,\allowbreak9}
\crossref{Luke}{18}{9}{Lu 10:29; 15:29; 16:15 Pr 30:12 Isa 65:5; 66:5 Joh 9:28,\allowbreak34 Ro 7:9}
\crossref{Luke}{18}{10}{Lu 1:9,\allowbreak10; 19:46 1Ki 8:30 Ac 3:1}
\crossref{Luke}{18}{11}{Ps 134:1; 135:2 Mt 6:5 Mr 11:25}
\crossref{Luke}{18}{12}{Lu 17:10 Nu 23:4 1Sa 15:13 2Ki 10:16 Isa 1:15; 58:2,\allowbreak3 Zec 7:5,\allowbreak6}
\crossref{Luke}{18}{13}{Lu 5:8; 7:6,\allowbreak7; 17:12 Ezr 9:6 Job 42:6 Ps 40:12 Isa 6:5 Eze 16:63}
\crossref{Luke}{18}{14}{Lu 5:24,\allowbreak25; 7:47-\allowbreak50 1Sa 1:18 Ec 9:7}
\crossref{Luke}{18}{15}{1Sa 1:24 Mt 19:13-\allowbreak15 Mr 10:13-\allowbreak16}
\crossref{Luke}{18}{16}{Ge 47:10-\allowbreak14; 21:4 De 29:11; 31:12 2Ch 20:13 Jer 32:39 Ac 2:39}
\crossref{Luke}{18}{17}{Ps 131:1,\allowbreak2 Mr 10:15 1Pe 1:14}
\crossref{Luke}{18}{18}{Mt 19:16 etc.}
\crossref{Luke}{18}{19}{Lu 1:35; 11:13 Job 14:4; 15:14-\allowbreak16; 25:4 1Ti 3:16 Heb 7:26 Jas 1:17}
\crossref{Luke}{18}{20}{Lu 10:26-\allowbreak28 Isa 8:20 Mt 19:17-\allowbreak19 Mr 10:18,\allowbreak19 Ro 3:20; 7:7-\allowbreak11}
\crossref{Luke}{18}{21}{18:11,\allowbreak12; 15:7,\allowbreak29 Mt 19:20,\allowbreak21 Mr 10:20,\allowbreak21 Ro 10:2,\allowbreak3 Php 3:6}
\crossref{Luke}{18}{22}{Lu 10:42 Ps 27:4 Php 3:13 2Pe 3:8}
\crossref{Luke}{18}{23}{Lu 8:14; 12:15; 19:8; 21:34 Jud 18:23,\allowbreak24 Job 31:24,\allowbreak25 Eze 33:31}
\crossref{Luke}{18}{24}{Mr 6:26 2Co 7:9,\allowbreak10}
\crossref{Luke}{18}{25}{}
\crossref{Luke}{18}{26}{Lu 13:23}
\crossref{Luke}{18}{27}{Lu 1:37 Ge 18:14 Job 42:2 Jer 32:17 Da 4:35 Zec 8:6 Mt 19:26}
\crossref{Luke}{18}{28}{Lu 5:11 Mt 4:19-\allowbreak22; 9:9; 19:27 Mr 10:28 Php 3:7}
\crossref{Luke}{18}{29}{Lu 14:26-\allowbreak28 De 33:9 Mt 10:37-\allowbreak39; 19:28-\allowbreak30 Mr 10:29-\allowbreak31}
\crossref{Luke}{18}{30}{Lu 12:31,\allowbreak32 Job 42:10 Ps 37:16; 63:4,\allowbreak5; 84:10-\allowbreak12; 119:72,\allowbreak103,\allowbreak111,\allowbreak127}
\crossref{Luke}{18}{31}{Lu 9:22; 24:6,\allowbreak7 Mt 16:21; 17:22,\allowbreak23; 20:17-\allowbreak19 Mr 8:31,\allowbreak9,\allowbreak30,\allowbreak31}
\crossref{Luke}{18}{32}{Lu 23:1,\allowbreak11 Mt 27:2 Mr 15:1 Joh 18:28,\allowbreak30,\allowbreak35 Ac 3:13}
\crossref{Luke}{18}{33}{Lu 24:7,\allowbreak21 Mt 27:63 1Co 15:3,\allowbreak4}
\crossref{Luke}{18}{34}{Lu 2:50; 9:45; 24:25,\allowbreak45 Mr 9:32 Joh 10:6; 12:16; 16:1-\allowbreak19}
\crossref{Luke}{18}{35}{Mt 20:29,\allowbreak30 Mr 10:46,\allowbreak47}
\crossref{Luke}{18}{36}{Lu 15:26 Mt 21:10,\allowbreak11}
\crossref{Luke}{18}{37}{Mr 2:1-\allowbreak3 Joh 12:35,\allowbreak36 2Co 6:2}
\crossref{Luke}{18}{38}{Ps 62:12 Isa 9:6,\allowbreak7; 11:1 Jer 23:5 Mt 9:27; 12:23; 15:22; 21:9,\allowbreak15}
\crossref{Luke}{18}{39}{18:15; 8:49; 11:52; 19:39}
\crossref{Luke}{18}{40}{Mt 20:31-\allowbreak34 Mr 10:48-\allowbreak52}
\crossref{Luke}{18}{41}{1Ki 3:5 etc.}
\crossref{Luke}{18}{42}{Ps 33:9; 107:20 Mt 8:3; 15:28}
\crossref{Luke}{18}{43}{Ps 30:2; 146:8 Isa 29:18,\allowbreak19; 35:5; 42:16; 43:8 Mt 9:28-\allowbreak30; 11:5}
\crossref{Luke}{19}{1}{Jos 2:1; 6:1 etc.}
\crossref{Luke}{19}{2}{Lu 18:24-\allowbreak27 2Ch 17:5,\allowbreak6}
\crossref{Luke}{19}{3}{Lu 9:7-\allowbreak9; 23:8 Joh 12:21}
\crossref{Luke}{19}{4}{Lu 5:19}
\crossref{Luke}{19}{5}{Ps 139:1-\allowbreak3 Eze 16:6 Joh 1:48; 4:7-\allowbreak10}
\crossref{Luke}{19}{6}{Lu 2:16 Ge 18:6,\allowbreak7 Ps 119:59,\allowbreak60 Ga 1:15,\allowbreak16}
\crossref{Luke}{19}{7}{Lu 5:30; 7:34,\allowbreak39; 15:2; 18:9-\allowbreak14 Mt 9:11; 21:28-\allowbreak31}
\crossref{Luke}{19}{8}{Lu 3:8-\allowbreak13; 11:41; 12:33; 16:9; 18:22,\allowbreak23 Ps 41:1 Ac 2:44-\allowbreak46; 4:34,\allowbreak35}
\crossref{Luke}{19}{9}{Lu 2:30; 13:30 Joh 4:38-\allowbreak42 Ac 16:30-\allowbreak32 1Co 6:9-\allowbreak11 1Pe 2:10}
\crossref{Luke}{19}{10}{Lu 5:31,\allowbreak32; 15:4-\allowbreak7,\allowbreak32 Eze 34:16 Mt 1:21; 9:12,\allowbreak13; 10:6; 15:24; 18:11}
\crossref{Luke}{19}{11}{Lu 17:20 Ac 1:6 2Th 2:1-\allowbreak3}
\crossref{Luke}{19}{12}{Mt 25:14-\allowbreak30 Mr 13:34-\allowbreak37}
\crossref{Luke}{19}{13}{Mt 25:14 Joh 12:26 Ga 1:10 Jas 1:1 2Pe 1:1}
\crossref{Luke}{19}{14}{19:27 1Sa 8:7 Ps 2:1-\allowbreak3 Isa 49:7 Zec 11:8 Joh 1:11; 15:18,\allowbreak23,\allowbreak24}
\crossref{Luke}{19}{15}{Ps 2:4-\allowbreak6}
\crossref{Luke}{19}{16}{1Ch 29:14-\allowbreak16 1Co 15:10 Col 1:28,\allowbreak29 2Ti 4:7,\allowbreak8 Jas 2:18-\allowbreak26}
\crossref{Luke}{19}{17}{Ge 39:4 1Sa 2:30 Mt 25:21 Ro 2:29 1Co 4:5 2Ti 2:10 1Pe 1:7; 5:4}
\crossref{Luke}{19}{18}{Mt 13:23 Mr 4:20 2Co 8:12}
\crossref{Luke}{19}{19}{Isa 3:10 1Co 3:8; 15:41,\allowbreak42,\allowbreak58 2Co 9:6 2Jo 1:8}
\crossref{Luke}{19}{20}{19:13; 3:9; 6:46 Pr 26:13-\allowbreak16 Mt 25:24 Jas 4:17}
\crossref{Luke}{19}{21}{Ex 20:19,\allowbreak20 1Sa 12:20 Mt 25:24,\allowbreak25 Ro 8:15 2Ti 1:7 Jas 2:10}
\crossref{Luke}{19}{22}{2Sa 1:16 Job 15:5,\allowbreak6 Mt 12:37; 22:12 Ro 3:19}
\crossref{Luke}{19}{23}{Ro 2:4,\allowbreak5}
\crossref{Luke}{19}{24}{Lu 12:20; 16:2}
\crossref{Luke}{19}{25}{Lu 16:2 2Sa 7:19 Isa 55:8,\allowbreak9}
\crossref{Luke}{19}{26}{Lu 8:18 Mt 13:12; 25:28,\allowbreak29 Mr 4:25 Joh 5:1-\allowbreak3}
\crossref{Luke}{19}{27}{19:14,\allowbreak42-\allowbreak44; 21:22,\allowbreak24 Nu 14:36,\allowbreak37; 16:30-\allowbreak35 Ps 2:3-\allowbreak5,\allowbreak9; 21:8,\allowbreak9; 69:22-\allowbreak28}
\crossref{Luke}{19}{28}{Lu 9:51; 12:50; 18:31 Ps 40:6-\allowbreak8 Mr 10:32-\allowbreak34 Joh 18:11 Heb 12:2}
\crossref{Luke}{19}{29}{Mt 21:1 etc.}
\crossref{Luke}{19}{30}{19:32; 22:8-\allowbreak13 1Sa 10:2-\allowbreak9 Joh 14:29}
\crossref{Luke}{19}{31}{Ps 24:1; 50:10-\allowbreak12 Mt 21:2,\allowbreak3 Mr 11:3-\allowbreak6 Ac 10:36}
\crossref{Luke}{19}{32}{19:32}
\crossref{Luke}{19}{33}{19:33}
\crossref{Luke}{19}{34}{Zec 9:9 Joh 10:35; 12:16 2Co 8:9}
\crossref{Luke}{19}{35}{2Ki 9:13 Mt 21:7 Mr 11:7,\allowbreak8 Joh 12:14 Ga 4:15,\allowbreak16}
\crossref{Luke}{19}{36}{Mt 21:8}
\crossref{Luke}{19}{37}{19:20 Mr 13:3; 14:26}
\crossref{Luke}{19}{38}{Lu 13:35 Ps 72:17-\allowbreak19; 118:22-\allowbreak26 Zec 9:9 Mt 21:9 Mr 11:9,\allowbreak10}
\crossref{Luke}{19}{39}{Isa 26:11 Mt 23:13 Joh 11:47,\allowbreak48; 12:10,\allowbreak19 Ac 4:1,\allowbreak2,\allowbreak16-\allowbreak18}
\crossref{Luke}{19}{40}{Ps 96:11; 98:7-\allowbreak9; 114:1-\allowbreak8 Isa 55:12 Hab 2:11 Mt 3:9; 21:15,\allowbreak16}
\crossref{Luke}{19}{41}{Ps 119:53,\allowbreak136,\allowbreak158 Jer 9:1; 13:17; 17:16 Ho 11:8 Joh 11:35}
\crossref{Luke}{19}{42}{De 5:29; 32:29 Ps 81:13 Isa 48:18 Eze 18:31,\allowbreak32; 33:11}
\crossref{Luke}{19}{43}{Lu 21:20-\allowbreak24 De 28:49-\allowbreak58 Ps 37:12,\allowbreak13 Da 9:26,\allowbreak27 Mt 22:7; 23:37-\allowbreak39}
\crossref{Luke}{19}{44}{1Ki 9:7,\allowbreak8 Mic 3:12}
\crossref{Luke}{19}{45}{Mt 21:12,\allowbreak13 Mr 11:15-\allowbreak17 Joh 2:13-\allowbreak17}
\crossref{Luke}{19}{46}{Ps 93:5 Isa 56:7 Jer 7:11 Eze 43:12 Ho 12:7 Mt 23:14}
\crossref{Luke}{19}{47}{Lu 21:37,\allowbreak38 Mt 21:23 Mr 11:27 etc.}
\crossref{Luke}{19}{48}{Lu 20:19,\allowbreak20; 22:2-\allowbreak4 Mt 22:15,\allowbreak16}
\crossref{Luke}{20}{1}{Lu 19:47,\allowbreak48 Mr 11:27 Joh 18:20}
\crossref{Luke}{20}{2}{Lu 19:35-\allowbreak40,\allowbreak45,\allowbreak46 Mt 21:23-\allowbreak27 Mr 11:28-\allowbreak33}
\crossref{Luke}{20}{3}{Lu 22:68 Mt 15:2,\allowbreak3 Col 4:6}
\crossref{Luke}{20}{4}{Lu 7:28-\allowbreak35 Mt 11:7-\allowbreak19; 17:11,\allowbreak12; 21:25-\allowbreak32 Joh 1:6,\allowbreak19-\allowbreak28}
\crossref{Luke}{20}{5}{Joh 1:15-\allowbreak18,\allowbreak30,\allowbreak34; 3:26,\allowbreak36; 3:26,\allowbreak36; 5:33-\allowbreak35 Ac 13:25}
\crossref{Luke}{20}{6}{Mt 21:26,\allowbreak46; 26:5 Mr 12:12 Ac 5:26}
\crossref{Luke}{20}{7}{Isa 6:9,\allowbreak10; 26:11; 29:9-\allowbreak12,\allowbreak14; 41:28; 42:19,\allowbreak20; 44:18 Jer 8:7-\allowbreak9}
\crossref{Luke}{20}{8}{Lu 22:68 Job 5:12,\allowbreak13 Pr 26:4,\allowbreak5 Mt 15:14; 16:4; 21:27 Mr 11:33}
\crossref{Luke}{20}{9}{Mt 21:33 etc.}
\crossref{Luke}{20}{10}{Ps 1:3 Jer 5:24 Mt 21:34-\allowbreak36 Mr 12:2-\allowbreak5}
\crossref{Luke}{20}{11}{Mt 23:30-\allowbreak37 Ac 7:52 1Th 2:2 Heb 11:36,\allowbreak37}
\crossref{Luke}{20}{12}{}
\crossref{Luke}{20}{13}{Isa 5:4 Ho 6:4; 11:8}
\crossref{Luke}{20}{14}{Ps 2:1-\allowbreak6,\allowbreak8; 89:27 Mt 2:2-\allowbreak16 Ro 8:17 Heb 1:2}
\crossref{Luke}{20}{15}{Heb 13:12}
\crossref{Luke}{20}{16}{Lu 19:27 Ps 2:8,\allowbreak9; 21:8-\allowbreak10 Mt 21:41; 22:7 Ac 13:46}
\crossref{Luke}{20}{17}{Lu 19:41; 22:61 Mr 3:5; 10:23}
\crossref{Luke}{20}{18}{}
\crossref{Luke}{20}{19}{20:14; 19:47,\allowbreak48 Mt 21:45,\allowbreak46; 26:3,\allowbreak4 Mr 12:12}
\crossref{Luke}{20}{20}{Ps 37:32,\allowbreak33; 38:12 Isa 29:20,\allowbreak21 Jer 11:19; 18:18; 20:10}
\crossref{Luke}{20}{21}{Ps 12:2; 55:21 Jer 42:2,\allowbreak3 Mt 22:16; 26:49,\allowbreak50 Mr 12:14 Joh 3:2}
\crossref{Luke}{20}{22}{De 17:15 Ezr 4:13,\allowbreak19-\allowbreak22; 9:7 Ne 5:4; 9:37 Mt 22:17-\allowbreak21}
\crossref{Luke}{20}{23}{Lu 5:22; 6:8; 11:17 Joh 2:24,\allowbreak25 1Co 3:19 Heb 4:13}
\crossref{Luke}{20}{24}{Mt 18:28; 20:2}
\crossref{Luke}{20}{25}{Pr 24:21 Mt 17:27; 22:21 Mr 12:17 Ro 13:6,\allowbreak7 1Pe 2:13-\allowbreak17}
\crossref{Luke}{20}{26}{20:20,\allowbreak39,\allowbreak40 Job 5:12,\allowbreak13 Pr 26:4,\allowbreak5 2Ti 3:8,\allowbreak9}
\crossref{Luke}{20}{27}{Mt 16:1,\allowbreak6,\allowbreak12; 22:23 etc.}
\crossref{Luke}{20}{28}{Ge 38:8,\allowbreak11,\allowbreak26 De 25:5-\allowbreak10 Ru 1:11,\allowbreak12}
\crossref{Luke}{20}{29}{Le 20:20 Jer 22:30}
\crossref{Luke}{20}{30}{20:30}
\crossref{Luke}{20}{31}{}
\crossref{Luke}{20}{32}{Jud 2:10 Ec 1:4; 9:5 Heb 9:27}
\crossref{Luke}{20}{33}{Mt 22:24-\allowbreak28 Mr 12:19-\allowbreak23}
\crossref{Luke}{20}{34}{Lu 16:8}
\crossref{Luke}{20}{35}{Lu 21:36 Ac 5:41 2Th 1:5 Re 3:4}
\crossref{Luke}{20}{36}{Isa 25:8 Ho 13:14 1Co 15:26,\allowbreak42,\allowbreak53,\allowbreak54 Php 3:21 1Th 4:13-\allowbreak17}
\crossref{Luke}{20}{37}{Ex 3:2-\allowbreak6 De 33:16 Ac 7:30-\allowbreak32}
\crossref{Luke}{20}{38}{Ps 16:5-\allowbreak11; 22:23-\allowbreak26; 145:1,\allowbreak2 Heb 11:16}
\crossref{Luke}{20}{39}{Mt 22:34-\allowbreak40 Mr 12:28-\allowbreak34 Ac 23:9}
\crossref{Luke}{20}{40}{Pr 26:5 Mt 22:46 Mr 12:34}
\crossref{Luke}{20}{41}{Mt 22:41,\allowbreak42 Mr 12:35 etc.}
\crossref{Luke}{20}{42}{Lu 24:44 2Sa 23:1,\allowbreak2 Mt 22:43 Mr 12:36,\allowbreak37 Ac 1:20; 13:33-\allowbreak35 Heb 3:7}
\crossref{Luke}{20}{43}{Lu 19:27 Ps 2:1-\allowbreak9; 21:8-\allowbreak12; 72:9; 109:4-\allowbreak20; 110:5,\allowbreak6 Re 19:14-\allowbreak21}
\crossref{Luke}{20}{44}{Lu 1:31-\allowbreak35; 2:11 Isa 7:14 Mt 1:23 Ro 9:5 Ga 4:4 1Ti 3:16 Re 22:16}
\crossref{Luke}{20}{45}{Mt 15:10; 23:1 Mr 8:34; 12:38 1Ti 5:20}
\crossref{Luke}{20}{46}{Lu 12:1 Mt 16:6 Mr 8:15 2Ti 4:15}
\crossref{Luke}{20}{47}{Isa 10:2 Jer 7:6-\allowbreak10 Eze 22:7 Am 2:7; 8:4-\allowbreak6 Mic 2:2,\allowbreak8; 3:2}
\crossref{Luke}{21}{1}{Mr 7:11-\allowbreak13; 12:41-\allowbreak44}
\crossref{Luke}{21}{2}{Mr 12:42}
\crossref{Luke}{21}{3}{Lu 4:25; 9:27; 12:44 Ac 4:27; 10:34}
\crossref{Luke}{21}{4}{Lu 8:43; 15:12 Ac 2:44,\allowbreak45; 4:34}
\crossref{Luke}{21}{5}{Mt 24:1 etc.}
\crossref{Luke}{21}{6}{Lu 19:44 etc.}
\crossref{Luke}{21}{7}{21:32 Da 12:6,\allowbreak8 Mt 24:3 Mr 13:3,\allowbreak4 Joh 21:21,\allowbreak22 Ac 1:6,\allowbreak7}
\crossref{Luke}{21}{8}{Jer 29:8 Mt 24:4,\allowbreak5,\allowbreak11,\allowbreak23-\allowbreak25 Mr 13:5,\allowbreak6,\allowbreak21-\allowbreak23 2Co 11:13-\allowbreak15}
\crossref{Luke}{21}{9}{21:18,\allowbreak19 Ps 27:1-\allowbreak3; 46:1,\allowbreak2; 112:7 Pr 3:25,\allowbreak26 Isa 8:12; 51:12,\allowbreak13}
\crossref{Luke}{21}{10}{}
\crossref{Luke}{21}{11}{21:25-\allowbreak27 Mt 24:29,\allowbreak30}
\crossref{Luke}{21}{12}{Lu 11:49-\allowbreak51 Mt 10:16-\allowbreak25; 22:6; 23:34-\allowbreak36; 24:9,\allowbreak10 Mr 13:9-\allowbreak13}
\crossref{Luke}{21}{13}{Php 1:28 1Th 3:3,\allowbreak4 2Th 1:5}
\crossref{Luke}{21}{14}{Lu 12:11,\allowbreak12 Mt 10:19,\allowbreak20 Mr 13:11}
\crossref{Luke}{21}{15}{Lu 24:45 Ex 4:11,\allowbreak12 Pr 2:6 Jer 1:9 Ac 2:4; 4:8-\allowbreak13,\allowbreak31-\allowbreak33 Eph 6:19}
\crossref{Luke}{21}{16}{Jer 9:4; 12:6 Mic 7:5,\allowbreak6 Mt 10:21 Mr 13:12}
\crossref{Luke}{21}{17}{Mt 10:22; 24:9 Mr 13:13 Joh 7:7; 15:19; 17:14}
\crossref{Luke}{21}{18}{Lu 12:7 1Sa 14:45; 25:29 2Sa 14:11 Mt 10:30 Ac 27:34}
\crossref{Luke}{21}{19}{Lu 8:15 Ps 27:13,\allowbreak14; 37:7; 40:1 Ro 2:7; 5:3; 8:25; 15:4 1Th 1:3}
\crossref{Luke}{21}{20}{21:7; 19:43 Da 9:27 Mt 24:15 Mr 13:14}
\crossref{Luke}{21}{21}{Nu 16:26 Jer 6:1; 35:11; 37:12 Re 18:4}
\crossref{Luke}{21}{22}{Isa 34:8; 61:2 Jer 51:6 Ro 2:5 2Pe 2:9; 3:7}
\crossref{Luke}{21}{23}{Lu 23:29 De 28:56,\allowbreak57 La 4:10 Heb 9:12-\allowbreak17; 13:16 Mt 24:19 Mr 13:17}
\crossref{Luke}{21}{24}{De 28:64-\allowbreak68}
\crossref{Luke}{21}{25}{Isa 13:10,\allowbreak13,\allowbreak14; 24:23 Jer 4:23 Eze 32:7,\allowbreak8 Joe 2:30,\allowbreak31}
\crossref{Luke}{21}{26}{Le 26:36 De 28:32-\allowbreak34,\allowbreak65-\allowbreak67 Heb 10:26,\allowbreak27}
\crossref{Luke}{21}{27}{Da 7:13 Mt 24:30; 26:64 Mr 13:26 Ac 1:9-\allowbreak11 Re 1:7; 14:14}
\crossref{Luke}{21}{28}{Ps 98:5-\allowbreak9 Isa 12:1-\allowbreak3; 25:8,\allowbreak9; 60:1,\allowbreak2}
\crossref{Luke}{21}{29}{Mt 24:32-\allowbreak35 Mr 13:28-\allowbreak30}
\crossref{Luke}{21}{30}{}
\crossref{Luke}{21}{31}{Lu 12:51-\allowbreak57 Mt 16:1-\allowbreak4}
\crossref{Luke}{21}{32}{Lu 11:50,\allowbreak51 Mt 16:28; 23:36; 24:34 Mr 13:30}
\crossref{Luke}{21}{33}{Ps 102:26 Isa 40:8; 51:6 Mt 5:18; 24:35 Mr 13:31 1Pe 1:25}
\crossref{Luke}{21}{34}{21:8; 17:3 Mr 13:9 Heb 12:15}
\crossref{Luke}{21}{35}{Ps 11:6 Ec 9:12 Isa 24:17,\allowbreak18 Jer 48:43,\allowbreak44 Re 16:15}
\crossref{Luke}{21}{36}{Lu 12:37-\allowbreak40 Mt 24:42; 25:13; 26:41 Mr 13:33,\allowbreak37 1Co 16:13 2Ti 4:5}
\crossref{Luke}{21}{37}{Lu 22:39 Mt 21:17 Mr 11:12 Joh 12:1}
\crossref{Luke}{21}{38}{Joh 8:1,\allowbreak2}
\crossref{Luke}{22}{1}{Ex 12:6-\allowbreak23 Le 23:5,\allowbreak6 Mt 26:2 Mr 14:1,\allowbreak2,\allowbreak12 Joh 11:55-\allowbreak57}
\crossref{Luke}{22}{2}{Lu 19:47,\allowbreak48; 20:19 Ps 2:1-\allowbreak5 Mt 21:38,\allowbreak45,\allowbreak46; 26:3-\allowbreak5 Joh 11:47-\allowbreak53,\allowbreak57}
\crossref{Luke}{22}{3}{Mt 26:14 Mr 14:10 etc.}
\crossref{Luke}{22}{4}{Mt 26:14 Mr 14:10,\allowbreak11}
\crossref{Luke}{22}{5}{Zec 11:12,\allowbreak13 Mt 26:15-\allowbreak16; 27:3-\allowbreak5 Ac 1:18; 8:20 1Ti 6:9,\allowbreak10}
\crossref{Luke}{22}{6}{Mt 26:5 Mr 14:2}
\crossref{Luke}{22}{7}{22:1 Ex 12:6,\allowbreak18 Mt 26:17 Mr 14:12}
\crossref{Luke}{22}{8}{Mr 14:13-\allowbreak16}
\crossref{Luke}{22}{9}{22:9}
\crossref{Luke}{22}{10}{Lu 19:29 etc.}
\crossref{Luke}{22}{11}{Lu 19:31,\allowbreak34 Mt 21:3 Joh 11:28}
\crossref{Luke}{22}{12}{Joh 2:25; 21:17 Ac 16:14,\allowbreak15}
\crossref{Luke}{22}{13}{Lu 21:33 Joh 2:5; 11:40 Heb 11:8}
\crossref{Luke}{22}{14}{De 16:6,\allowbreak7 Mt 26:20 Mr 14:17}
\crossref{Luke}{22}{15}{Lu 12:50 Joh 4:34; 13:1; 17:1}
\crossref{Luke}{22}{16}{22:18-\allowbreak20}
\crossref{Luke}{22}{17}{Ps 23:5; 116:13 Jer 16:7}
\crossref{Luke}{22}{18}{22:16 Mt 26:29 Mr 14:23; 15:23}
\crossref{Luke}{22}{19}{Mt 26:26-\allowbreak28 Mr 14:22-\allowbreak24 1Co 10:16; 11:23-\allowbreak29}
\crossref{Luke}{22}{20}{Ex 24:8 Zec 9:11 1Co 10:16-\allowbreak21; 11:25 Heb 8:6-\allowbreak13; 9:17; 12:24}
\crossref{Luke}{22}{21}{Job 19:19 Ps 41:9 Mic 7:5,\allowbreak6 Mt 26:21-\allowbreak23 Mr 14:18}
\crossref{Luke}{22}{22}{Lu 24:25-\allowbreak27,\allowbreak46 Ge 3:15 Ps 22:1-\allowbreak31; 69:1-\allowbreak36 Isa 53:1-\allowbreak12 Da 9:24-\allowbreak26}
\crossref{Luke}{22}{23}{Mt 26:22 Mr 14:19 Joh 13:22-\allowbreak25}
\crossref{Luke}{22}{24}{Lu 9:46 Mt 20:20-\allowbreak24 Mr 9:34; 10:37-\allowbreak41 Ro 12:10 1Co 13:4 Php 2:3-\allowbreak5}
\crossref{Luke}{22}{25}{Mt 20:25-\allowbreak28 Mr 10:41-\allowbreak45}
\crossref{Luke}{22}{26}{Lu 9:48 Mt 18:3-\allowbreak5; 23:8-\allowbreak12 Ro 12:2 1Pe 5:3 3Jo 1:9,\allowbreak10}
\crossref{Luke}{22}{27}{Lu 12:37; 17:7-\allowbreak9 Mt 20:28 Joh 13:5-\allowbreak16 2Co 8:9 Php 2:7,\allowbreak8}
\crossref{Luke}{22}{28}{Mt 19:28,\allowbreak29; 24:13 Joh 6:67,\allowbreak68; 8:31 Ac 1:25 Heb 2:18; 4:15}
\crossref{Luke}{22}{29}{Lu 12:32; 19:17 Mt 24:47; 25:34 1Co 9:25 2Co 1:7 2Ti 2:12 Jas 2:5}
\crossref{Luke}{22}{30}{22:16-\allowbreak18; 12:37; 14:15 2Sa 9:9,\allowbreak10; 19:28 Mt 8:11 Re 19:9}
\crossref{Luke}{22}{31}{Lu 10:41 Ac 9:4}
\crossref{Luke}{22}{32}{Zec 3:2-\allowbreak4 Joh 14:19; 17:9-\allowbreak11,\allowbreak15-\allowbreak21 Ro 5:9,\allowbreak10; 8:32,\allowbreak34 Heb 7:25}
\crossref{Luke}{22}{33}{2Ki 8:12,\allowbreak13 Pr 28:26 Jer 10:23; 17:9 Mt 20:22; 26:33-\allowbreak35,\allowbreak40,\allowbreak41}
\crossref{Luke}{22}{34}{Mt 26:34,\allowbreak74 Mr 14:30,\allowbreak71,\allowbreak72 Joh 13:38; 18:27}
\crossref{Luke}{22}{35}{Lu 9:3; 10:4 Mt 10:9,\allowbreak10 Mr 6:8,\allowbreak9}
\crossref{Luke}{22}{36}{Mt 10:22-\allowbreak25 Joh 15:20; 16:33 1Th 2:14,\allowbreak15; 3:4 1Pe 4:1}
\crossref{Luke}{22}{37}{22:22; 18:31; 24:44-\allowbreak46 Mt 26:54-\allowbreak56 Joh 10:35; 19:28-\allowbreak30 Ac 13:27-\allowbreak29}
\crossref{Luke}{22}{38}{Mt 26:52-\allowbreak54 Joh 18:36 2Co 10:3,\allowbreak4 Eph 6:10-\allowbreak18 1Th 5:8 1Pe 5:9}
\crossref{Luke}{22}{39}{Mt 26:36-\allowbreak38 Mr 14:32-\allowbreak34 Joh 18:1,\allowbreak2}
\crossref{Luke}{22}{40}{22:46; 11:4 1Ch 4:10 Ps 17:5; 19:13; 119:116,\allowbreak117,\allowbreak133 Pr 30:8,\allowbreak9 Mt 6:13}
\crossref{Luke}{22}{41}{Mt 26:39 Mr 14:35}
\crossref{Luke}{22}{42}{Mt 26:42,\allowbreak44 Mr 14:36 Joh 12:27,\allowbreak28}
\crossref{Luke}{22}{43}{Lu 4:10,\allowbreak11 Ps 91:11,\allowbreak12 Mt 4:6,\allowbreak11; 26:53 1Ti 3:16 Heb 1:6,\allowbreak14}
\crossref{Luke}{22}{44}{Ge 32:24-\allowbreak28 Ps 22:1,\allowbreak2,\allowbreak12-\allowbreak21; 40:1-\allowbreak3; 69:14-\allowbreak18; 88:1-\allowbreak18; 130:1,\allowbreak2}
\crossref{Luke}{22}{45}{Mt 26:40,\allowbreak43 Mr 14:37,\allowbreak40,\allowbreak41}
\crossref{Luke}{22}{46}{22:40; 21:34-\allowbreak36 Pr 6:4-\allowbreak11 Jon 1:6}
\crossref{Luke}{22}{47}{Mt 26:45-\allowbreak47 Mr 14:41-\allowbreak43 Joh 18:2-\allowbreak9}
\crossref{Luke}{22}{48}{2Sa 20:9,\allowbreak10 Ps 55:21 Pr 27:6 Mt 26:48-\allowbreak50 Mr 14:44-\allowbreak46}
\crossref{Luke}{22}{49}{22:49}
\crossref{Luke}{22}{50}{Mt 26:51-\allowbreak54 Mr 14:47 Joh 18:10,\allowbreak11 Ro 12:19 2Co 10:4}
\crossref{Luke}{22}{51}{Joh 17:12; 18:8,\allowbreak9}
\crossref{Luke}{22}{52}{Mt 26:55 Mr 14:48,\allowbreak49}
\crossref{Luke}{22}{53}{Lu 21:37,\allowbreak38 Mt 21:12-\allowbreak15,\allowbreak23,\allowbreak45,\allowbreak46 Joh 7:25,\allowbreak26,\allowbreak30,\allowbreak45}
\crossref{Luke}{22}{54}{22:33,\allowbreak34 2Ch 32:31}
\crossref{Luke}{22}{55}{22:44 Mt 26:69 Mr 14:66 Joh 18:17,\allowbreak18}
\crossref{Luke}{22}{56}{Mt 26:69 Mr 14:6,\allowbreak17,\allowbreak66-\allowbreak68 Joh 18:17}
\crossref{Luke}{22}{57}{22:33,\allowbreak34; 12:9 Mt 10:33; 26:70 Joh 18:25,\allowbreak27 Ac 3:13,\allowbreak14,\allowbreak19 2Ti 2:10-\allowbreak12}
\crossref{Luke}{22}{58}{}
\crossref{Luke}{22}{59}{Mt 26:73,\allowbreak74 Mr 14:69,\allowbreak70 Joh 18:26,\allowbreak27}
\crossref{Luke}{22}{60}{22:34 Mt 26:74,\allowbreak75 Mr 14:71,\allowbreak72 Joh 18:27}
\crossref{Luke}{22}{61}{Lu 10:41 Mr 5:30}
\crossref{Luke}{22}{62}{Ps 38:18; 126:5,\allowbreak6; 130:1-\allowbreak4; 143:1-\allowbreak4 Jer 31:18 Eze 7:16 Zec 12:10}
\crossref{Luke}{22}{63}{Mt 26:59-\allowbreak68 Mr 14:55-\allowbreak65 Joh 18:22}
\crossref{Luke}{22}{64}{Jud 16:21,\allowbreak25}
\crossref{Luke}{22}{65}{Lu 12:10 Mt 12:31,\allowbreak32 Ac 26:11 1Ti 1:13,\allowbreak14}
\crossref{Luke}{22}{66}{Mt 27:1 Mr 15:1}
\crossref{Luke}{22}{67}{Mt 11:3-\allowbreak5; 26:63 etc.}
\crossref{Luke}{22}{68}{Lu 20:3-\allowbreak7,\allowbreak41-\allowbreak44}
\crossref{Luke}{22}{69}{Mt 26:64 Mr 14:62}
\crossref{Luke}{22}{70}{Lu 4:41 Ps 2:7,\allowbreak12 Mt 3:17; 27:43,\allowbreak54 Joh 1:34,\allowbreak49; 10:30,\allowbreak36; 19:7}
\crossref{Luke}{22}{71}{Mt 26:65,\allowbreak66 Mr 14:63,\allowbreak64}
\crossref{Luke}{23}{1}{Lu 22:66 Mt 27:1,\allowbreak2,\allowbreak11 etc.}
\crossref{Luke}{23}{2}{Zec 11:8 Mr 15:3-\allowbreak5 Joh 18:30}
\crossref{Luke}{23}{3}{Mt 27:11 Mr 15:2 Joh 18:33-\allowbreak37 1Ti 6:13}
\crossref{Luke}{23}{4}{23:14,\allowbreak15 Mt 27:19,\allowbreak24 Mr 15:14 Joh 18:38; 19:4-\allowbreak6 Heb 7:26 1Pe 1:19}
\crossref{Luke}{23}{5}{23:23; 11:53 Ps 22:12,\allowbreak13,\allowbreak16; 57:4; 69:4 Mt 27:24 Joh 19:15 Ac 5:33; 7:54}
\crossref{Luke}{23}{6}{Lu 13:1 Ac 5:37}
\crossref{Luke}{23}{7}{Lu 3:1; 13:31}
\crossref{Luke}{23}{8}{Lu 9:7-\allowbreak9 Mt 14:1 Mr 6:14}
\crossref{Luke}{23}{9}{Lu 13:32 Ps 38:13,\allowbreak14; 39:1,\allowbreak2,\allowbreak9 Isa 53:7 Mt 7:6; 27:14 Ac 8:32}
\crossref{Luke}{23}{10}{23:2,\allowbreak5,\allowbreak14,\allowbreak15; 11:53 Ac 24:5}
\crossref{Luke}{23}{11}{Ac 4:27,\allowbreak28}
\crossref{Luke}{23}{12}{Ps 83:4-\allowbreak6 Ac 4:27 Mt 16:1 Re 17:13,\allowbreak14}
\crossref{Luke}{23}{13}{Mt 27:21-\allowbreak23 Mr 15:14 Joh 18:38; 19:4}
\crossref{Luke}{23}{14}{23:1,\allowbreak2,\allowbreak5}
\crossref{Luke}{23}{15}{}
\crossref{Luke}{23}{16}{Isa 53:5 Mt 27:26 Mr 15:15 Joh 19:1-\allowbreak4 Ac 5:40,\allowbreak41}
\crossref{Luke}{23}{17}{Mt 27:15 Mr 15:6 Joh 18:39}
\crossref{Luke}{23}{18}{Mt 27:16-\allowbreak23 Mr 15:7-\allowbreak14 Joh 18:40 Ac 3:14}
\crossref{Luke}{23}{19}{23:2,\allowbreak5 Ac 3:14}
\crossref{Luke}{23}{20}{Mt 14:8,\allowbreak9; 27:19 Mr 15:15 Joh 19:12}
\crossref{Luke}{23}{21}{23:23 Mt 27:22-\allowbreak25 Mr 15:13 Joh 19:15}
\crossref{Luke}{23}{22}{23:14,\allowbreak20 1Pe 1:19; 3:18}
\crossref{Luke}{23}{23}{23:5 Ps 22:12,\allowbreak13; 57:4 Zec 11:8}
\crossref{Luke}{23}{24}{Mt 27:26 Mr 15:15 Joh 19:1}
\crossref{Luke}{23}{25}{23:2,\allowbreak5 Mr 15:7 Joh 18:40}
\crossref{Luke}{23}{26}{Mt 27:32 etc.}
\crossref{Luke}{23}{27}{23:55; 8:2 Mt 27:55 Mr 15:40}
\crossref{Luke}{23}{28}{So 1:5; 2:7; 3:5,\allowbreak10; 5:8,\allowbreak16; 8:4}
\crossref{Luke}{23}{29}{De 28:53-\allowbreak57 Ho 9:12-\allowbreak16; 13:16}
\crossref{Luke}{23}{30}{Isa 2:19 Ho 10:8 Re 6:16; 9:6}
\crossref{Luke}{23}{31}{Pr 11:31 Jer 25:29 Eze 15:2-\allowbreak7; 20:47,\allowbreak48; 21:3,\allowbreak4 Da 9:26 Mt 3:12}
\crossref{Luke}{23}{32}{Lu 22:37 Isa 53:12 Mt 27:38 Mr 15:27,\allowbreak28 Joh 19:18 Heb 12:2}
\crossref{Luke}{23}{33}{Mt 27:33,\allowbreak34 Mr 15:22,\allowbreak23 Joh 19:17,\allowbreak18 Heb 13:12,\allowbreak13}
\crossref{Luke}{23}{34}{23:47,\allowbreak48; 6:27,\allowbreak28 Ge 50:17 Ps 106:16-\allowbreak23 Mt 5:44 Ac 7:60 Ro 12:14}
\crossref{Luke}{23}{35}{Ps 22:12,\allowbreak13,\allowbreak17 Zec 12:10 Mt 27:38-\allowbreak43 Mr 15:29-\allowbreak32}
\crossref{Luke}{23}{36}{23:11 Ps 69:21 Mt 27:29,\allowbreak30,\allowbreak34,\allowbreak48 Mr 15:19,\allowbreak20,\allowbreak36 Joh 19:28-\allowbreak30}
\crossref{Luke}{23}{37}{23:37}
\crossref{Luke}{23}{38}{23:3 Mt 27:11,\allowbreak37 Mr 15:18,\allowbreak26,\allowbreak32 Joh 19:3,\allowbreak19-\allowbreak22}
\crossref{Luke}{23}{39}{Lu 17:34-\allowbreak36 Mt 27:44 Mr 15:32}
\crossref{Luke}{23}{40}{Le 19:17 Eph 5:11}
\crossref{Luke}{23}{41}{Lu 15:18,\allowbreak19 Le 26:40,\allowbreak41 Jos 7:19,\allowbreak20 2Ch 33:12 Ezr 9:13 Ne 9:3}
\crossref{Luke}{23}{42}{Lu 18:13 Ps 106:4,\allowbreak5 Joh 20:28 Ac 16:31; 20:21 Ro 10:9-\allowbreak14}
\crossref{Luke}{23}{43}{Lu 15:4,\allowbreak5,\allowbreak20-\allowbreak24; 19:10 Job 33:27-\allowbreak30 Ps 32:5; 50:15 Isa 1:18,\allowbreak19}
\crossref{Luke}{23}{44}{Mt 27:45 Mr 15:33}
\crossref{Luke}{23}{45}{Ex 26:31 Le 16:12-\allowbreak16 2Ch 3:14 Mt 27:51 Mr 15:38 Eph 2:14-\allowbreak18}
\crossref{Luke}{23}{46}{Mt 27:46-\allowbreak49 Mr 15:34-\allowbreak36}
\crossref{Luke}{23}{47}{23:41 Mt 27:54 Mr 15:39 Joh 19:7}
\crossref{Luke}{23}{48}{Lu 18:13 Jer 31:19 Ac 2:37}
\crossref{Luke}{23}{49}{Job 19:13 Ps 38:11; 88:18; 142:4}
\crossref{Luke}{23}{50}{Mt 27:57,\allowbreak58 Mr 15:42-\allowbreak45 Joh 19:38}
\crossref{Luke}{23}{51}{Ge 37:21,\allowbreak22; 42:21,\allowbreak22 Ex 23:2 Pr 1:10 Isa 8:12}
\crossref{Luke}{23}{52}{Joh 19:38-\allowbreak42}
\crossref{Luke}{23}{53}{Isa 53:9 Mt 27:59,\allowbreak60 Mr 15:46}
\crossref{Luke}{23}{54}{Mt 27:62 Joh 19:14,\allowbreak31,\allowbreak42}
\crossref{Luke}{23}{55}{23:49; 8:2 Mt 27:61 Mr 15:47}
\crossref{Luke}{23}{56}{Lu 24:1 2Ch 16:14 Mr 16:1}
\crossref{Luke}{24}{1}{Mt 28:1 Mr 16:1,\allowbreak2 Joh 20:1,\allowbreak2}
\crossref{Luke}{24}{2}{Mt 27:60-\allowbreak66; 28:2 Mr 15:46,\allowbreak47; 16:3,\allowbreak4 Joh 20:1,\allowbreak2}
\crossref{Luke}{24}{3}{24:23 Mt 16:5 Joh 20:6,\allowbreak7}
\crossref{Luke}{24}{4}{Ge 18:2 Mt 28:2-\allowbreak6 Mr 16:5 Joh 20:11,\allowbreak12 Ac 1:10}
\crossref{Luke}{24}{5}{Lu 1:12,\allowbreak13,\allowbreak29 Da 8:17,\allowbreak18; 10:7-\allowbreak12,\allowbreak16,\allowbreak19 Mt 28:3-\allowbreak5 Mr 16:5,\allowbreak6}
\crossref{Luke}{24}{6}{24:44-\allowbreak46; 9:22; 18:31-\allowbreak33 Mt 12:40; 16:21; 17:22,\allowbreak23; 20:18,\allowbreak19; 27:63; 28:6}
\crossref{Luke}{24}{7}{24:7}
\crossref{Luke}{24}{8}{Joh 2:19-\allowbreak22; 12:16; 14:26}
\crossref{Luke}{24}{9}{24:22-\allowbreak24 Mt 28:7,\allowbreak8 Mr 16:7,\allowbreak8,\allowbreak10}
\crossref{Luke}{24}{10}{Lu 8:2,\allowbreak3 Mr 15:40,\allowbreak41; 16:9-\allowbreak11 Joh 20:11-\allowbreak18}
\crossref{Luke}{24}{11}{24:25 Ge 19:14 2Ki 7:2 Job 9:16 Ps 126:1 Ac 12:9}
\crossref{Luke}{24}{12}{Joh 20:3-\allowbreak10}
\crossref{Luke}{24}{13}{24:18 Mr 16:12,\allowbreak13}
\crossref{Luke}{24}{14}{Lu 6:45 De 6:7 Mal 3:6}
\crossref{Luke}{24}{15}{24:36 Mt 18:20 Joh 14:18,\allowbreak19}
\crossref{Luke}{24}{16}{24:31 2Ki 6:18-\allowbreak20 Mr 16:12 Joh 20:14; 21:4}
\crossref{Luke}{24}{17}{Eze 9:4-\allowbreak6 Joh 16:6,\allowbreak20-\allowbreak22}
\crossref{Luke}{24}{18}{Joh 19:25}
\crossref{Luke}{24}{19}{Lu 7:16 Mt 21:11 Joh 3:2; 4:19; 6:14; 7:40-\allowbreak42,\allowbreak52 Ac 2:22; 10:38}
\crossref{Luke}{24}{20}{Lu 22:66-\allowbreak71; 23:1-\allowbreak5 Mt 27:1,\allowbreak2,\allowbreak20 Mr 15:1 Ac 3:13-\allowbreak15; 4:8-\allowbreak10}
\crossref{Luke}{24}{21}{Lu 1:68; 2:38 Ps 130:8 Isa 59:20 Ac 1:6 1Pe 1:18,\allowbreak19 Re 5:9}
\crossref{Luke}{24}{22}{24:9-\allowbreak11 Mt 28:7,\allowbreak8 Mr 16:9,\allowbreak10 Joh 20:1,\allowbreak2,\allowbreak18}
\crossref{Luke}{24}{23}{}
\crossref{Luke}{24}{24}{24:12 Joh 20:1-\allowbreak10}
\crossref{Luke}{24}{25}{}
\crossref{Luke}{24}{26}{24:46 Ps 22:1-\allowbreak31; 69:1-\allowbreak36 Isa 53:1-\allowbreak12 Zec 13:7 Ac 17:3 1Co 15:3,\allowbreak4}
\crossref{Luke}{24}{27}{24:44 Ge 3:15; 22:18; 26:4; 49:10 Nu 21:6-\allowbreak9 De 18:15}
\crossref{Luke}{24}{28}{}
\crossref{Luke}{24}{29}{Lu 14:23 Ge 19:3 2Ki 4:8 Ac 16:14}
\crossref{Luke}{24}{30}{24:35; 9:16; 22:19 Mt 14:19; 15:36; 26:26 Mr 6:41; 8:6; 14:22}
\crossref{Luke}{24}{31}{24:16 Joh 20:13-\allowbreak16}
\crossref{Luke}{24}{32}{Ps 39:3; 104:34 Pr 27:9,\allowbreak17 Isa 50:4 Jer 15:16; 20:9; 23:29}
\crossref{Luke}{24}{33}{Joh 20:19-\allowbreak26}
\crossref{Luke}{24}{34}{Lu 22:54-\allowbreak62 Mr 16:7 1Co 15:5}
\crossref{Luke}{24}{35}{Mr 16:12,\allowbreak13}
\crossref{Luke}{24}{36}{Mr 16:14 Joh 20:19-\allowbreak23 1Co 15:5}
\crossref{Luke}{24}{37}{Lu 16:30 1Sa 28:13 Job 4:14-\allowbreak16 Mt 14:26,\allowbreak27 Mr 6:49,\allowbreak50 Ac 12:15}
\crossref{Luke}{24}{38}{Jer 4:14 Da 4:5,\allowbreak19 Mt 16:8 Heb 4:13}
\crossref{Luke}{24}{39}{Joh 20:20,\allowbreak25,\allowbreak27 Ac 1:3 1Jo 1:1}
\crossref{Luke}{24}{40}{}
\crossref{Luke}{24}{41}{Ge 45:26-\allowbreak28 Job 9:16 Ps 126:1,\allowbreak2 Joh 16:22}
\crossref{Luke}{24}{42}{24:42}
\crossref{Luke}{24}{43}{Ac 10:41}
\crossref{Luke}{24}{44}{24:6,\allowbreak7; 9:22; 18:31-\allowbreak33 Mt 16:21; 17:22,\allowbreak23; 20:18,\allowbreak19}
\crossref{Luke}{24}{45}{Ex 4:11 Job 33:16 Ps 119:18 Isa 29:10-\allowbreak12,\allowbreak18,\allowbreak19 Ac 16:14; 26:18}
\crossref{Luke}{24}{46}{24:26,\allowbreak27,\allowbreak44 Ps 22:1-\allowbreak31 Isa 50:6; 53:2 etc.}
\crossref{Luke}{24}{47}{Da 9:24 Mt 3:2; 9:13 Ac 2:38; 3:19; 5:31; 11:18; 13:38,\allowbreak39,\allowbreak46}
\crossref{Luke}{24}{48}{Joh 15:27 Ac 1:8,\allowbreak22; 2:32; 3:15; 4:33; 5:32; 10:39,\allowbreak41; 13:31; 22:15}
\crossref{Luke}{24}{49}{Isa 44:3,\allowbreak4; 59:20,\allowbreak21 Joe 2:28 etc.}
\crossref{Luke}{24}{50}{Mr 11:1 Ac 1:12}
\crossref{Luke}{24}{51}{2Ki 2:11 Mr 16:19 Joh 20:17 Ac 1:9 Eph 4:8-\allowbreak10 Heb 1:3; 4:14}
\crossref{Luke}{24}{52}{Mt 28:9,\allowbreak17 Joh 20:28}
\crossref{Luke}{24}{53}{Ac 2:46,\allowbreak47; 5:41,\allowbreak42}

% John
\crossref{John}{1}{1}{Ge 1:1 Pr 8:22-\allowbreak31 Eph 3:9 Col 1:17 Heb 1:10; 7:3; 13:8}
\crossref{John}{1}{2}{}
\crossref{John}{1}{3}{1:10; 5:17-\allowbreak19 Ge 1:1,\allowbreak26 Ps 33:6; 102:25 Isa 45:12,\allowbreak18 Eph 3:9}
\crossref{John}{1}{4}{Joh 5:21,\allowbreak26; 11:25; 14:6 1Co 15:45 Col 3:4 1Jo 1:2; 5:11 Re 22:1}
\crossref{John}{1}{5}{1:10; 3:19,\allowbreak20; 12:36-\allowbreak40 Job 24:13-\allowbreak17 Pr 1:22,\allowbreak29,\allowbreak30 Ro 1:28 1Co 2:14}
\crossref{John}{1}{6}{1:33; 3:28 Isa 40:3-\allowbreak5 Mal 3:1; 4:5,\allowbreak6 Mt 3:1-\allowbreak11}
\crossref{John}{1}{7}{1:19,\allowbreak26,\allowbreak27,\allowbreak32-\allowbreak34,\allowbreak36; 3:26-\allowbreak36; 5:33-\allowbreak35 Ac 19:4}
\crossref{John}{1}{8}{1:20; 3:28 Ac 19:4}
\crossref{John}{1}{9}{1:4; 6:32; 14:6; 15:1 Isa 49:6 Mt 6:23 1Jo 1:8; 2:8; 5:20}
\crossref{John}{1}{10}{1:18; 5:17 Ge 11:6-\allowbreak9; 16:13; 17:1; 18:33 Ex 3:4-\allowbreak6 Ac 14:17; 17:24-\allowbreak27}
\crossref{John}{1}{11}{Mt 15:24 Ac 3:25,\allowbreak26; 13:26,\allowbreak46 Ro 9:1,\allowbreak5; 15:8 Ga 4:4}
\crossref{John}{1}{12}{Mt 10:40; 18:5 Col 2:6}
\crossref{John}{1}{13}{Joh 3:3,\allowbreak5 Jas 1:18 1Pe 1:3,\allowbreak23; 2:2 1Jo 3:9; 4:7; 5:1,\allowbreak4,\allowbreak18}
\crossref{John}{1}{14}{1:1 Isa 7:14 Mt 1:16,\allowbreak20-\allowbreak23 Lu 1:31-\allowbreak35; 2:7,\allowbreak11 Ro 1:3,\allowbreak4; 9:5}
\crossref{John}{1}{15}{1:7,\allowbreak8,\allowbreak29-\allowbreak34; 3:26-\allowbreak36; 5:33-\allowbreak36 Mt 3:11,\allowbreak13-\allowbreak17 Mr 1:7 Lu 3:16}
\crossref{John}{1}{16}{Joh 3:34; 15:1-\allowbreak5 Mt 3:11,\allowbreak14 Lu 21:15 Ac 3:12-\allowbreak16 Ro 8:9 1Co 1:4,\allowbreak5}
\crossref{John}{1}{17}{Joh 5:45; 9:29 Ex 20:1-\allowbreak17}
\crossref{John}{1}{18}{Joh 6:46 Ex 33:20 De 4:12 Mt 11:27 Lu 10:22 Col 1:15 1Ti 1:17}
\crossref{John}{1}{19}{Joh 5:33-\allowbreak36 De 17:9-\allowbreak11; 24:8 Mt 21:23-\allowbreak32 Lu 3:15-\allowbreak18}
\crossref{John}{1}{20}{Joh 3:28-\allowbreak36 Mt 3:11,\allowbreak12 Mr 1:7,\allowbreak8 Lu 3:15-\allowbreak17}
\crossref{John}{1}{21}{Mal 4:5 Mt 11:14; 17:10-\allowbreak12 Lu 1:17}
\crossref{John}{1}{22}{2Sa 24:13}
\crossref{John}{1}{23}{Joh 3:28 Mt 3:3 Mr 1:3 Lu 1:16,\allowbreak17,\allowbreak76-\allowbreak79; 3:4-\allowbreak6}
\crossref{John}{1}{24}{Joh 3:1,\allowbreak2; 7:47-\allowbreak49 Mt 23:13-\allowbreak15,\allowbreak26 Lu 7:30; 11:39-\allowbreak44,\allowbreak53; 16:14 Ac 23:8}
\crossref{John}{1}{25}{Mt 21:23 Ac 4:5-\allowbreak7; 5:28}
\crossref{John}{1}{26}{Mt 3:11 Mr 1:8 Lu 3:16 Ac 1:5; 11:16}
\crossref{John}{1}{27}{1:15,\allowbreak30 Ac 19:4}
\crossref{John}{1}{28}{Joh 10:40 Jud 7:24}
\crossref{John}{1}{29}{1:36 Ge 22:7,\allowbreak8 Ex 12:3-\allowbreak13}
\crossref{John}{1}{30}{1:15,\allowbreak27 Lu 3:16}
\crossref{John}{1}{31}{1:33 Lu 1:80; 2:39-\allowbreak42}
\crossref{John}{1}{32}{Joh 5:32 Mt 3:16 Mr 1:10 Lu 3:22}
\crossref{John}{1}{33}{1:31 Mt 3:13-\allowbreak15}
\crossref{John}{1}{34}{1:18,\allowbreak49; 3:16-\allowbreak18,\allowbreak35,\allowbreak36; 5:23-\allowbreak27; 6:69; 10:30,\allowbreak36; 11:27; 19:7; 20:28,\allowbreak31}
\crossref{John}{1}{35}{Joh 3:25,\allowbreak26 Mal 3:16}
\crossref{John}{1}{36}{1:29 Isa 45:22; 65:1,\allowbreak2 Heb 12:2 1Pe 1:19,\allowbreak20}
\crossref{John}{1}{37}{1:43; 4:39-\allowbreak42 Pr 15:23 Zec 8:21 Ro 10:17 Eph 4:29 Re 22:17}
\crossref{John}{1}{38}{Lu 14:25; 15:20; 19:5; 22:61}
\crossref{John}{1}{39}{1:46; 6:37; 14:22,\allowbreak23 Pr 8:17 Mt 11:28-\allowbreak30}
\crossref{John}{1}{40}{Joh 6:8 Mt 4:18; 10:2 Ac 1:13}
\crossref{John}{1}{41}{1:36,\allowbreak37,\allowbreak45; 4:28,\allowbreak29 2Ki 7:9 Isa 2:3-\allowbreak5 Lu 2:17,\allowbreak38 Ac 13:32,\allowbreak33 1Jo 1:3}
\crossref{John}{1}{42}{1:47,\allowbreak48; 2:24,\allowbreak25; 6:70,\allowbreak71; 13:18}
\crossref{John}{1}{43}{Isa 65:1 Mt 4:18-\allowbreak21; 9:9 Lu 19:10 Php 3:12 1Jo 4:19}
\crossref{John}{1}{44}{Joh 12:21; 14:8,\allowbreak9 Mt 10:3 Mr 3:18 Lu 6:14 Ac 1:13}
\crossref{John}{1}{45}{Joh 21:2}
\crossref{John}{1}{46}{Joh 7:41,\allowbreak42,\allowbreak52 Lu 4:28,\allowbreak29}
\crossref{John}{1}{47}{Joh 8:31,\allowbreak39 Ro 2:28,\allowbreak29; 9:6 Php 3:3}
\crossref{John}{1}{48}{Joh 2:25 Ge 32:24-\allowbreak30 Ps 139:1,\allowbreak2 Isa 65:24 Mt 6:6 1Co 4:5; 14:25}
\crossref{John}{1}{49}{1:38}
\crossref{John}{1}{50}{Joh 20:29 Lu 1:45; 7:9}
\crossref{John}{1}{51}{Joh 3:3,\allowbreak5; 5:19,\allowbreak24,\allowbreak25; 6:26,\allowbreak32,\allowbreak47,\allowbreak53; 8:34,\allowbreak51,\allowbreak58; 10:1,\allowbreak7; 12:24; 13:16}
\crossref{John}{2}{1}{Joh 1:43}
\crossref{John}{2}{2}{Mt 12:19 Lu 7:34-\allowbreak38 1Co 7:39; 10:31 Col 3:17 Re 3:20}
\crossref{John}{2}{3}{Ps 104:15 Ec 10:19 Isa 24:11 Mt 26:28}
\crossref{John}{2}{4}{Joh 19:26,\allowbreak27; 20:13,\allowbreak15 Mt 15:28}
\crossref{John}{2}{5}{Joh 15:14 Ge 6:22 Jud 13:14 Lu 5:5,\allowbreak6; 6:46-\allowbreak49 Ac 9:6 Heb 5:9; 11:8}
\crossref{John}{2}{6}{Joh 3:25 Mr 7:2-\allowbreak5 Eph 5:26 Heb 6:2; 9:10,\allowbreak19; 10:22}
\crossref{John}{2}{7}{2:3,\allowbreak5 Nu 21:6-\allowbreak9 Jos 6:3-\allowbreak5 1Ki 17:13 2Ki 4:2-\allowbreak6; 5:10-\allowbreak14}
\crossref{John}{2}{8}{2:9 Pr 3:5,\allowbreak6 Ec 9:6}
\crossref{John}{2}{9}{Joh 4:46}
\crossref{John}{2}{10}{Ge 43:34 So 5:1}
\crossref{John}{2}{11}{Joh 1:17 Ex 4:9; 7:19-\allowbreak21 Ec 9:7 Mal 2:2 2Co 4:17 Ga 3:10-\allowbreak13}
\crossref{John}{2}{12}{Joh 6:17 Mt 4:13; 11:23}
\crossref{John}{2}{13}{2:23; 5:1; 6:4; 11:55 Ex 12:6-\allowbreak14 Nu 28:16-\allowbreak25 De 16:1-\allowbreak8,\allowbreak16 Lu 2:41}
\crossref{John}{2}{14}{De 14:23-\allowbreak26 Mt 21:12 Mr 11:15 Lu 19:45,\allowbreak46}
\crossref{John}{2}{15}{Joh 18:6 Zec 4:6 2Co 10:4}
\crossref{John}{2}{16}{Isa 56:5-\allowbreak11 Jer 7:11 Ho 12:7,\allowbreak8 Mt 21:13 Mr 11:17 Ac 19:24-\allowbreak27}
\crossref{John}{2}{17}{Ps 69:9; 119:139}
\crossref{John}{2}{18}{Joh 6:30 Mt 12:38 etc.}
\crossref{John}{2}{19}{Mt 26:60,\allowbreak61; 27:40 Mr 14:58; 15:29}
\crossref{John}{2}{20}{}
\crossref{John}{2}{21}{Joh 1:14}
\crossref{John}{2}{22}{2:17; 12:16; 14:26; 16:4 Lu 24:7,\allowbreak8,\allowbreak44 Ac 11:16}
\crossref{John}{2}{23}{Joh 3:2; 6:14; 7:31; 8:30,\allowbreak31; 12:42,\allowbreak43 Mt 13:20,\allowbreak21 Mr 4:16,\allowbreak17 Lu 8:13}
\crossref{John}{2}{24}{Joh 6:15 Mt 10:16,\allowbreak17}
\crossref{John}{2}{25}{2:25}
\crossref{John}{3}{1}{3:10; 7:47-\allowbreak49}
\crossref{John}{3}{2}{Joh 7:50,\allowbreak51; 12:42,\allowbreak43; 19:38,\allowbreak39 Jud 6:27 Isa 51:7 Php 1:14}
\crossref{John}{3}{3}{Joh 1:51 Mt 5:18 2Co 1:19,\allowbreak20 Re 3:14}
\crossref{John}{3}{4}{3:3; 4:11,\allowbreak12; 6:53,\allowbreak60 1Co 1:18; 2:14}
\crossref{John}{3}{5}{3:3 Isa 44:3,\allowbreak4 Eze 36:25-\allowbreak27 Mt 3:11 Mr 16:16 Ac 2:38 Eph 5:26}
\crossref{John}{3}{6}{Ge 5:3; 6:5,\allowbreak12 Job 14:4; 15:14-\allowbreak16; 25:4 Ps 51:10 Ro 7:5,\allowbreak18,\allowbreak25}
\crossref{John}{3}{7}{3:12; 5:28; 6:61-\allowbreak63}
\crossref{John}{3}{8}{Job 37:10-\allowbreak13,\allowbreak16,\allowbreak17,\allowbreak21-\allowbreak23 Ps 107:25,\allowbreak29 Ec 11:4,\allowbreak5 Eze 37:9}
\crossref{John}{3}{9}{3:4; 6:52,\allowbreak60 Pr 4:18 Isa 42:16 Mr 8:24,\allowbreak25 Lu 1:34}
\crossref{John}{3}{10}{Isa 9:16; 29:10-\allowbreak12; 56:10 Jer 8:8,\allowbreak9 Mt 11:25; 15:14; 22:29}
\crossref{John}{3}{11}{3:3,\allowbreak5}
\crossref{John}{3}{12}{3:3,\allowbreak5,\allowbreak8 1Co 3:1,\allowbreak2 Heb 5:11 1Pe 2:1-\allowbreak3}
\crossref{John}{3}{13}{Joh 1:18; 6:46 De 30:12 Pr 30:4 Ac 2:34 Ro 10:6 Eph 4:9}
\crossref{John}{3}{14}{Nu 21:7-\allowbreak9 2Ki 18:4}
\crossref{John}{3}{15}{3:16,\allowbreak36; 1:12; 6:40,\allowbreak47; 11:25,\allowbreak26; 12:44-\allowbreak46; 20:31 Isa 45:22 Mr 16:16}
\crossref{John}{3}{16}{Lu 2:14 Ro 5:8 2Co 5:19-\allowbreak21 Tit 3:4 1Jo 4:9,\allowbreak10,\allowbreak19}
\crossref{John}{3}{17}{Joh 5:45; 8:15,\allowbreak16; 12:47,\allowbreak48 Lu 9:56}
\crossref{John}{3}{18}{3:36; 5:24; 6:40,\allowbreak47; 20:31 Ro 5:1; 8:1,\allowbreak34 1Jo 5:12}
\crossref{John}{3}{19}{Joh 1:4,\allowbreak9-\allowbreak11; 8:12; 9:39-\allowbreak41; 15:22-\allowbreak25 Mt 11:20-\allowbreak24 Lu 10:11-\allowbreak16; 12:47}
\crossref{John}{3}{20}{Joh 7:7 1Ki 22:8 Job 24:13-\allowbreak17 Ps 50:17 Pr 1:29; 4:18; 5:12; 15:12}
\crossref{John}{3}{21}{Joh 1:47; 5:39 Ps 1:1-\allowbreak3; 119:80,\allowbreak105; 139:23,\allowbreak24 Isa 8:20 Ac 17:11,\allowbreak12}
\crossref{John}{3}{22}{Joh 2:13; 4:3; 7:3}
\crossref{John}{3}{23}{Ge 33:18}
\crossref{John}{3}{24}{Mt 4:12; 14:3 Mr 6:17 Lu 3:19,\allowbreak20; 9:7-\allowbreak9}
\crossref{John}{3}{25}{Joh 2:6 Mt 3:11 Mr 7:2-\allowbreak5,\allowbreak8 Heb 6:2; 9:10,\allowbreak13,\allowbreak14,\allowbreak23 1Pe 3:21}
\crossref{John}{3}{26}{Nu 11:26-\allowbreak29 Ec 4:4 1Co 3:3-\allowbreak5 Ga 5:20,\allowbreak21; 6:12,\allowbreak13 Jas 3:14-\allowbreak18}
\crossref{John}{3}{27}{Nu 16:9-\allowbreak11; 17:5 1Ch 28:4,\allowbreak5 Jer 1:5; 17:16 Am 7:15 Mt 25:15}
\crossref{John}{3}{28}{Joh 1:20,\allowbreak25,\allowbreak27}
\crossref{John}{3}{29}{Ps 45:9-\allowbreak17 So 3:11; 4:8-\allowbreak12 Isa 54:5; 62:4,\allowbreak5 Jer 2:2 Eze 16:8}
\crossref{John}{3}{30}{Ps 72:17-\allowbreak19 Isa 9:7; 53:2,\allowbreak3,\allowbreak12 Da 2:34,\allowbreak35,\allowbreak44,\allowbreak45 Mt 13:31-\allowbreak33}
\crossref{John}{3}{31}{3:13; 6:33; 8:23 Eph 1:20,\allowbreak21; 4:8-\allowbreak10}
\crossref{John}{3}{32}{3:11; 5:20; 8:26; 15:15}
\crossref{John}{3}{33}{Ro 3:3,\allowbreak4; 4:18-\allowbreak21 2Co 1:18 Tit 1:1,\allowbreak2 Heb 6:17 1Jo 5:9,\allowbreak10}
\crossref{John}{3}{34}{Joh 7:16; 8:26-\allowbreak28,\allowbreak40,\allowbreak47}
\crossref{John}{3}{35}{Joh 5:20,\allowbreak22; 15:9; 17:23,\allowbreak26 Pr 8:30 Isa 42:1 Mt 3:17; 17:5}
\crossref{John}{3}{36}{3:15,\allowbreak16; 1:12; 5:24; 6:47-\allowbreak54; 10:28 Hab 2:4 Ro 1:17; 8:1 1Jo 3:14,\allowbreak15}
\crossref{John}{4}{1}{Lu 1:76; 2:11; 19:31,\allowbreak34 Ac 10:36 1Co 2:8; 15:47 2Co 4:5 Jas 2:1}
\crossref{John}{4}{2}{Ac 10:48 1Co 1:13-\allowbreak17}
\crossref{John}{4}{3}{Joh 3:32; 10:40; 11:54 Mt 10:23 Mr 3:7}
\crossref{John}{4}{4}{Mt 10:5,\allowbreak6 Lu 2:49; 9:51,\allowbreak52; 17:11}
\crossref{John}{4}{5}{Ge 33:19; 48:22 Jos 24:32}
\crossref{John}{4}{6}{Mt 4:2; 8:24 Heb 2:17; 4:15}
\crossref{John}{4}{7}{4:10; 19:28 Ge 24:43 2Sa 23:15-\allowbreak17 1Ki 17:10 Mt 10:42}
\crossref{John}{4}{8}{Joh 6:5-\allowbreak7 Lu 9:13}
\crossref{John}{4}{9}{4:27; 8:48 Lu 10:33; 17:16-\allowbreak19}
\crossref{John}{4}{10}{Joh 3:16 Isa 9:6; 42:6; 49:6-\allowbreak8 Lu 11:13 Ro 8:32 1Co 1:30 2Co 9:15}
\crossref{John}{4}{11}{Joh 3:4 1Co 2:14}
\crossref{John}{4}{12}{Joh 8:53 Isa 53:2,\allowbreak3 Mt 12:42 Heb 3:3}
\crossref{John}{4}{13}{Joh 6:27,\allowbreak49 Isa 65:13,\allowbreak14 Lu 16:24}
\crossref{John}{4}{14}{Joh 6:35,\allowbreak58; 11:26; 17:2,\allowbreak3 Isa 49:10 Ro 6:23 Re 7:16}
\crossref{John}{4}{15}{Joh 6:26,\allowbreak34; 17:2,\allowbreak3 Ps 4:6 Ro 6:23; 8:5 1Co 2:14 1Jo 5:20 Jas 4:3}
\crossref{John}{4}{16}{4:18; 1:42,\allowbreak47,\allowbreak48; 2:24,\allowbreak25; 21:17 Heb 4:13 Re 2:23}
\crossref{John}{4}{17}{}
\crossref{John}{4}{18}{Ge 20:3; 34:2,\allowbreak7,\allowbreak8,\allowbreak31 Nu 5:29 Ru 4:10 Jer 3:20 Eze 16:32}
\crossref{John}{4}{19}{4:29; 1:48,\allowbreak49 2Ki 5:26; 6:12 Lu 7:39 1Co 14:24,\allowbreak25}
\crossref{John}{4}{20}{Ge 12:6,\allowbreak7; 33:18-\allowbreak20 De 27:12 Jos 8:33-\allowbreak35 Jud 9:6,\allowbreak7 2Ki 17:26-\allowbreak33}
\crossref{John}{4}{21}{Eze 14:3; 20:3}
\crossref{John}{4}{22}{2Ki 17:27-\allowbreak29,\allowbreak41 Ezr 4:2 Ac 17:23,\allowbreak30}
\crossref{John}{4}{23}{Joh 5:25; 12:23}
\crossref{John}{4}{24}{2Co 3:17 1Ti 1:17}
\crossref{John}{4}{25}{4:42; 1:41,\allowbreak42,\allowbreak49 Da 9:24-\allowbreak26}
\crossref{John}{4}{26}{Joh 9:37 Mt 16:20; 20:15; 26:63,\allowbreak64 Mr 14:61,\allowbreak62 Lu 13:30 Ro 10:20,\allowbreak21}
\crossref{John}{4}{27}{4:9 Lu 7:39}
\crossref{John}{4}{28}{4:7 Mt 28:8 Mr 16:8-\allowbreak10 Lu 24:9,\allowbreak33}
\crossref{John}{4}{29}{4:17,\allowbreak18,\allowbreak25; 1:41-\allowbreak49 1Co 14:24,\allowbreak25 Re 22:17}
\crossref{John}{4}{30}{Isa 60:8 Mt 2:1-\allowbreak3; 8:11,\allowbreak12; 11:20-\allowbreak24; 12:40-\allowbreak42; 20:16 Lu 17:16-\allowbreak18}
\crossref{John}{4}{31}{Ge 24:33 Ac 16:30-\allowbreak34}
\crossref{John}{4}{32}{4:34 Job 23:12 Ps 63:5; 119:103 Pr 18:20 Isa 53:11 Jer 15:16}
\crossref{John}{4}{33}{Mt 16:6-\allowbreak11 Lu 9:45}
\crossref{John}{4}{34}{4:32; 6:33,\allowbreak38 Job 23:12 Ps 40:8 Isa 61:1-\allowbreak3 Lu 15:4-\allowbreak6,\allowbreak10; 19:10}
\crossref{John}{4}{35}{4:30 Mt 9:37,\allowbreak38 Lu 10:3}
\crossref{John}{4}{36}{Pr 11:30 Da 12:3 Ro 1:13; 6:22 1Co 9:19-\allowbreak23 Php 2:15,\allowbreak16 1Th 2:19}
\crossref{John}{4}{37}{Jud 6:3 Mic 6:15 Lu 19:21}
\crossref{John}{4}{38}{Ac 2:41; 4:4,\allowbreak32; 5:14; 6:7; 8:4-\allowbreak8,\allowbreak14-\allowbreak17}
\crossref{John}{4}{39}{Joh 10:41,\allowbreak42; 11:45}
\crossref{John}{4}{40}{Ge 32:26 Pr 4:13 So 3:4 Jer 14:8 Lu 8:38; 10:39; 24:29 Ac 16:15}
\crossref{John}{4}{41}{Ge 49:10 Ac 1:8; 8:12,\allowbreak25; 15:3}
\crossref{John}{4}{42}{Joh 1:45-\allowbreak49; 17:8 Ac 17:11,\allowbreak12}
\crossref{John}{4}{43}{Mt 15:21-\allowbreak24 Mr 7:27,\allowbreak28 Ro 15:8}
\crossref{John}{4}{44}{Mt 13:57 Mr 6:4 Lu 4:24}
\crossref{John}{4}{45}{Mt 4:23,\allowbreak24 Lu 8:40}
\crossref{John}{4}{46}{Ps 50:15; 78:34 Ho 5:15 Mt 9:18; 15:22; 17:14,\allowbreak15 Lu 7:2; 8:42}
\crossref{John}{4}{47}{Mr 2:1-\allowbreak3; 6:55,\allowbreak56; 10:47}
\crossref{John}{4}{48}{4:41,\allowbreak42; 2:18; 12:37; 15:24; 20:29 Nu 14:11 Mt 16:1; 27:42 Lu 10:18}
\crossref{John}{4}{49}{Ps 40:17; 88:10-\allowbreak12 Mr 5:23,\allowbreak35,\allowbreak36}
\crossref{John}{4}{50}{Joh 11:40 1Ki 17:13-\allowbreak15 Mt 8:13 Mr 7:29,\allowbreak30; 9:23,\allowbreak24 Lu 17:14}
\crossref{John}{4}{51}{4:50,\allowbreak53 1Ki 17:23}
\crossref{John}{4}{52}{}
\crossref{John}{4}{53}{Ps 33:9; 107:20 Mt 8:8,\allowbreak9,\allowbreak13}
\crossref{John}{4}{54}{Joh 2:1-\allowbreak11}
\crossref{John}{5}{1}{Joh 2:13 Ex 23:14-\allowbreak17; 34:23 Le 23:2 etc.}
\crossref{John}{5}{2}{Ne 3:1; 12:39}
\crossref{John}{5}{3}{Mt 15:30 Lu 7:22}
\crossref{John}{5}{4}{Ps 119:60 Pr 6:4; 8:17 Ec 9:10 Ho 13:13 Mt 6:33; 11:12}
\crossref{John}{5}{5}{5:14; 9:1,\allowbreak21 Mr 9:21 Lu 8:43; 13:16 Ac 3:2; 4:22; 9:33; 14:8}
\crossref{John}{5}{6}{Joh 21:17 Ps 142:3 Heb 4:13,\allowbreak15}
\crossref{John}{5}{7}{De 32:36 Ps 72:12; 142:4 Ro 5:6 2Co 1:8-\allowbreak10}
\crossref{John}{5}{8}{Mt 9:6 Mr 2:11 Lu 5:24 Ac 9:34}
\crossref{John}{5}{9}{5:14 Mr 1:31,\allowbreak42; 5:29,\allowbreak41,\allowbreak42; 10:52 Ac 3:7,\allowbreak8}
\crossref{John}{5}{10}{Ex 20:8-\allowbreak11; 31:12-\allowbreak17 Ne 13:15-\allowbreak21 Isa 58:13 Jer 17:21,\allowbreak27}
\crossref{John}{5}{11}{Joh 9:16 Mr 2:9-\allowbreak11}
\crossref{John}{5}{12}{Jud 6:29 1Sa 14:38 Mt 21:23 Ro 10:2}
\crossref{John}{5}{13}{Joh 14:9}
\crossref{John}{5}{14}{Le 7:12 Ps 9:13; 27:6; 66:13-\allowbreak15; 107:20-\allowbreak22; 116:12-\allowbreak19; 118:18}
\crossref{John}{5}{15}{Joh 4:29; 9:11,\allowbreak12 Mr 1:45}
\crossref{John}{5}{16}{Joh 15:20 Ac 9:4,\allowbreak5}
\crossref{John}{5}{17}{1Co 12:6 Col 1:16 Heb 1:3}
\crossref{John}{5}{18}{Joh 7:19}
\crossref{John}{5}{19}{5:24,\allowbreak25}
\crossref{John}{5}{20}{Joh 3:35; 17:26 Mt 3:17; 17:5 2Pe 1:17}
\crossref{John}{5}{21}{De 32:39 1Ki 17:21 2Ki 4:32-\allowbreak35; 5:7 Ac 26:8 Ro 4:17-\allowbreak19}
\crossref{John}{5}{22}{5:27; 3:35; 17:2 Ps 9:7,\allowbreak8; 50:3-\allowbreak6; 96:13; 98:9 Ec 11:9; 12:14 Mt 11:27}
\crossref{John}{5}{23}{Joh 14:1 Ps 146:3-\allowbreak5 Jer 17:5-\allowbreak7 Mt 12:21 Ro 15:12 2Co 1:9}
\crossref{John}{5}{24}{Joh 3:16,\allowbreak18,\allowbreak36; 6:40,\allowbreak47; 8:51; 11:26; 12:44; 20:31 Mr 16:16 Ro 10:11-\allowbreak13}
\crossref{John}{5}{25}{Joh 4:23; 13:1; 17:1}
\crossref{John}{5}{26}{Ex 3:14 Ps 36:9; 90:2 Jer 10:10 Ac 17:25 1Ti 1:17; 6:16}
\crossref{John}{5}{27}{5:22 Ps 2:6-\allowbreak9; 110:1,\allowbreak2,\allowbreak6 Ac 10:42; 17:31 1Co 15:25 Eph 1:20-\allowbreak23}
\crossref{John}{5}{28}{5:20; 3:7 Ac 3:12}
\crossref{John}{5}{29}{Da 12:2,\allowbreak3 Mt 25:31-\allowbreak46 Ac 24:15}
\crossref{John}{5}{30}{5:19; 8:28,\allowbreak42; 14:10}
\crossref{John}{5}{31}{Joh 8:13,\allowbreak14,\allowbreak54 Pr 27:2 Re 3:14}
\crossref{John}{5}{32}{5:36,\allowbreak37; 1:33; 8:17,\allowbreak18; 12:28-\allowbreak30 Mt 3:17; 17:5 Mr 1:11 Lu 3:22 1Jo 5:6-\allowbreak9}
\crossref{John}{5}{33}{Joh 1:19-\allowbreak27}
\crossref{John}{5}{34}{5:41; 8:54}
\crossref{John}{5}{35}{Joh 1:7,\allowbreak8 Mt 11:11 Lu 1:15-\allowbreak17,\allowbreak76,\allowbreak77; 7:28 2Pe 1:19}
\crossref{John}{5}{36}{5:32 1Jo 5:9,\allowbreak11,\allowbreak12}
\crossref{John}{5}{37}{Joh 6:27; 8:18 Mt 3:17; 17:5}
\crossref{John}{5}{38}{5:42,\allowbreak46,\allowbreak47; 8:37,\allowbreak46,\allowbreak47; 15:7 De 6:6-\allowbreak9 Jos 1:8 Ps 119:11 Pr 2:1,\allowbreak2; 7:1,\allowbreak2}
\crossref{John}{5}{39}{5:46; 7:52 De 11:18-\allowbreak20; 17:18,\allowbreak19 Jos 1:8 Ps 1:2; 119:11,\allowbreak97-\allowbreak99 Pr 6:23}
\crossref{John}{5}{40}{5:44; 1:11; 3:19; 8:45,\allowbreak46; 12:37-\allowbreak41 Ps 81:11 Isa 49:7; 50:2; 53:1-\allowbreak3}
\crossref{John}{5}{41}{5:34; 6:15; 7:18; 8:50,\allowbreak54 1Th 2:6 1Pe 2:21 2Pe 1:17}
\crossref{John}{5}{42}{Joh 1:47-\allowbreak49; 2:25; 21:17 Lu 16:15 Heb 4:12,\allowbreak13 Re 2:23}
\crossref{John}{5}{43}{Joh 3:16; 6:38; 8:28,\allowbreak29; 10:25; 12:28; 17:4-\allowbreak6 Eze 23:21 Heb 5:4,\allowbreak5}
\crossref{John}{5}{44}{Joh 3:20; 8:43; 12:43 Jer 13:23 Ro 8:7,\allowbreak8 Heb 3:12}
\crossref{John}{5}{45}{Joh 7:19; 8:5,\allowbreak9 Ro 2:12,\allowbreak17 etc.}
\crossref{John}{5}{46}{Ga 2:19; 3:10,\allowbreak13,\allowbreak24; 4:21-\allowbreak31}
\crossref{John}{5}{47}{Lu 16:29,\allowbreak31}
\crossref{John}{6}{1}{Mt 14:13,\allowbreak15 etc.}
\crossref{John}{6}{2}{Mt 4:24,\allowbreak25; 8:1; 12:15; 13:2; 14:14; 15:30,\allowbreak31 Mr 6:33}
\crossref{John}{6}{3}{6:15 Mt 14:23; 15:29 Lu 6:12,\allowbreak13; 9:28}
\crossref{John}{6}{4}{Joh 2:13; 5:1; 11:55; 12:1; 13:1 Ex 12:6 etc.}
\crossref{John}{6}{5}{Joh 4:35 Mt 14:14,\allowbreak15 Mr 6:34,\allowbreak35 Lu 9:12}
\crossref{John}{6}{6}{Ge 22:1 De 8:2,\allowbreak16; 13:3; 33:8 2Ch 32:31}
\crossref{John}{6}{7}{Joh 12:5 Mt 18:28}
\crossref{John}{6}{8}{Joh 1:40-\allowbreak44 Mt 4:18}
\crossref{John}{6}{9}{Mt 14:17; 16:9 Mr 6:38; 8:19 Lu 9:13}
\crossref{John}{6}{10}{Mt 14:18,\allowbreak19; 15:35,\allowbreak36 Mr 6:39-\allowbreak41; 8:6,\allowbreak7 Lu 9:14-\allowbreak16}
\crossref{John}{6}{11}{6:23 1Sa 9:13 Lu 24:30 Ac 27:35 Ro 14:6 1Co 10:31 1Th 5:18}
\crossref{John}{6}{12}{Ne 9:25 Mt 14:20,\allowbreak21; 15:37,\allowbreak38 Mr 6:42-\allowbreak44; 8:8,\allowbreak9 Lu 1:53; 9:17}
\crossref{John}{6}{13}{1Ki 7:15,\allowbreak16 2Ki 4:2-\allowbreak7 2Ch 25:9 Pr 11:24,\allowbreak25 2Co 9:8,\allowbreak9 Php 4:19}
\crossref{John}{6}{14}{Joh 1:21; 4:19,\allowbreak25,\allowbreak42; 7:40 Ge 49:10 De 18:15-\allowbreak18 Mt 11:3; 21:11}
\crossref{John}{6}{15}{Joh 2:24,\allowbreak25 Heb 4:13}
\crossref{John}{6}{16}{}
\crossref{John}{6}{17}{6:24,\allowbreak25; 2:12; 4:46 Mr 6:45}
\crossref{John}{6}{18}{Ps 107:25; 135:7 Mt 14:24}
\crossref{John}{6}{19}{Eze 27:26 Jon 1:13 Mr 6:47,\allowbreak48}
\crossref{John}{6}{20}{Ps 35:3 Isa 41:10,\allowbreak14; 43:1,\allowbreak2; 44:8 Mt 14:27-\allowbreak31 Mr 6:50; 16:6}
\crossref{John}{6}{21}{Ps 24:7-\allowbreak10 So 3:4 Mt 14:32,\allowbreak33 Mr 6:51 Re 3:20}
\crossref{John}{6}{22}{6:16,\allowbreak17 Mt 14:22 Mr 6:45}
\crossref{John}{6}{23}{6:24}
\crossref{John}{6}{24}{6:17,\allowbreak23}
\crossref{John}{6}{25}{Joh 1:38,\allowbreak39}
\crossref{John}{6}{26}{6:47,\allowbreak53; 3:3,\allowbreak5}
\crossref{John}{6}{27}{6:28,\allowbreak29 Ga 5:6 Php 2:13 Col 1:29 1Th 1:3}
\crossref{John}{6}{28}{De 5:27 Jer 42:3-\allowbreak6,\allowbreak20 Mic 6:7,\allowbreak8 Mt 19:16 Lu 10:25 Ac 2:37; 9:6}
\crossref{John}{6}{29}{Joh 3:16-\allowbreak18,\allowbreak36; 5:39 De 18:18,\allowbreak19 Ps 2:12 Mt 17:5 Mr 16:16 Ac 16:31}
\crossref{John}{6}{30}{Joh 2:18; 4:8 Ex 4:8 1Ki 13:3,\allowbreak5 Isa 7:11-\allowbreak14 Mt 12:38,\allowbreak39; 16:1-\allowbreak4}
\crossref{John}{6}{31}{6:49 Ex 16:4-\allowbreak15,\allowbreak35 Nu 11:6-\allowbreak9 De 8:3 Jos 5:12 Ne 9:20 Ps 105:40}
\crossref{John}{6}{32}{Ex 16:4,\allowbreak8 Ps 78:23}
\crossref{John}{6}{33}{6:38,\allowbreak48; 3:13; 8:42; 13:3; 16:28; 17:8 1Ti 1:15 1Jo 1:1,\allowbreak2}
\crossref{John}{6}{34}{6:26; 4:15 Ps 4:6}
\crossref{John}{6}{35}{6:41,\allowbreak48-\allowbreak58 1Co 10:16-\allowbreak18; 11:23-\allowbreak29}
\crossref{John}{6}{36}{6:26,\allowbreak30,\allowbreak40,\allowbreak64; 12:37; 15:24 Lu 16:31 1Pe 1:8,\allowbreak9}
\crossref{John}{6}{37}{6:39,\allowbreak45; 17:2,\allowbreak6,\allowbreak8,\allowbreak9,\allowbreak11,\allowbreak24}
\crossref{John}{6}{38}{6:33; 3:13,\allowbreak31 Eph 4:9}
\crossref{John}{6}{39}{6:40 Mt 18:14 Lu 12:32 Ro 8:28-\allowbreak31 2Th 2:13,\allowbreak14 2Ti 2:19}
\crossref{John}{6}{40}{6:36,\allowbreak37; 1:14; 4:14; 8:56 Isa 45:21,\allowbreak22; 52:10; 53:2 Lu 2:30 2Co 4:6}
\crossref{John}{6}{41}{6:43,\allowbreak52,\allowbreak60,\allowbreak66; 7:12 Lu 5:30; 15:2; 19:7 1Co 10:10 Jude 1:16}
\crossref{John}{6}{42}{Joh 7:27 Mt 13:55,\allowbreak56 Mr 6:3 Lu 4:22 Ro 1:3,\allowbreak4; 9:5 1Co 15:47 Ga 4:4}
\crossref{John}{6}{43}{6:64; 16:19 Mt 16:8 Mr 9:33 Heb 4:13}
\crossref{John}{6}{44}{6:65; 5:44; 8:43; 12:37-\allowbreak40 Isa 44:18-\allowbreak20 Jer 13:23 Mt 12:34 Ro 8:7,\allowbreak8}
\crossref{John}{6}{45}{Mr 1:2 Lu 1:70; 18:31}
\crossref{John}{6}{46}{Joh 1:18; 5:37; 8:19; 14:9,\allowbreak10; 15:24 Col 1:15 1Ti 6:16 1Jo 4:12}
\crossref{John}{6}{47}{6:40,\allowbreak54; 3:16,\allowbreak18,\allowbreak36; 5:24; 14:19 Ro 5:9,\allowbreak10 Col 3:3,\allowbreak4 1Jo 5:12,\allowbreak13}
\crossref{John}{6}{48}{6:33-\allowbreak35,\allowbreak41,\allowbreak51 1Co 10:16,\allowbreak17; 11:24,\allowbreak25}
\crossref{John}{6}{49}{6:31}
\crossref{John}{6}{50}{6:33,\allowbreak42,\allowbreak51; 3:13}
\crossref{John}{6}{51}{Joh 3:13; 4:10,\allowbreak11; 7:38 1Pe 2:4}
\crossref{John}{6}{52}{6:41; 7:40-\allowbreak43; 9:16; 10:19}
\crossref{John}{6}{53}{6:26,\allowbreak47}
\crossref{John}{6}{54}{6:27,\allowbreak40,\allowbreak63; 4:14 Ps 22:26 Pr 9:4-\allowbreak6 Isa 25:6-\allowbreak8; 55:1-\allowbreak3 Ga 2:20 Php 3:7-\allowbreak10}
\crossref{John}{6}{55}{6:32; 1:9,\allowbreak47; 8:31,\allowbreak36; 15:1 Ps 4:7 Heb 8:2 1Jo 5:20}
\crossref{John}{6}{56}{La 3:24}
\crossref{John}{6}{57}{Ps 18:46 Jer 10:10 1Th 1:9 Heb 9:14}
\crossref{John}{6}{58}{6:32,\allowbreak34,\allowbreak41,\allowbreak47-\allowbreak51}
\crossref{John}{6}{59}{6:24; 18:20 Ps 40:9,\allowbreak10 Pr 1:20-\allowbreak23; 8:1-\allowbreak3 Lu 4:31}
\crossref{John}{6}{60}{6:66; 8:31}
\crossref{John}{6}{61}{6:64; 2:24,\allowbreak25; 21:17 Heb 4:13 Re 2:23}
\crossref{John}{6}{62}{Joh 3:13; 16:28; 17:4,\allowbreak5,\allowbreak11 Mr 16:19 Lu 24:51 Ac 1:9 Eph 4:8-\allowbreak10}
\crossref{John}{6}{63}{Ge 2:7 Ro 8:2 1Co 15:45 2Co 3:6 Ga 5:25 1Pe 3:18}
\crossref{John}{6}{64}{6:36,\allowbreak61; 5:42; 8:23,\allowbreak38-\allowbreak47,\allowbreak55; 10:26; 13:10,\allowbreak18-\allowbreak21}
\crossref{John}{6}{65}{6:37,\allowbreak44,\allowbreak45; 10:16,\allowbreak26,\allowbreak27; 12:37-\allowbreak41 Eph 2:8,\allowbreak9 Php 1:29 1Ti 1:14 2Ti 2:25}
\crossref{John}{6}{66}{6:60; 8:31 Zep 1:6 Mt 12:40-\allowbreak45; 13:20,\allowbreak21; 19:22; 21:8-\allowbreak11; 27:20-\allowbreak25}
\crossref{John}{6}{67}{Jos 24:15-\allowbreak22 Ru 1:11-\allowbreak18 2Sa 15:19,\allowbreak20 Lu 14:25-\allowbreak33}
\crossref{John}{6}{68}{Ps 73:25}
\crossref{John}{6}{69}{Joh 1:29,\allowbreak41,\allowbreak45-\allowbreak49; 11:27; 20:28,\allowbreak31 Mt 16:16 Mr 1:1; 8:29 Lu 9:20}
\crossref{John}{6}{70}{6:64; 13:18; 17:12 Mt 10:1-\allowbreak4 Lu 6:13-\allowbreak16 Ac 1:17}
\crossref{John}{6}{71}{Ps 109:6-\allowbreak8 Ac 1:16-\allowbreak20; 2:23 Jude 1:4}
\crossref{John}{7}{1}{Joh 4:3,\allowbreak54; 10:39,\allowbreak40; 11:54 Lu 13:31-\allowbreak33 Ac 10:38}
\crossref{John}{7}{2}{Ex 23:16,\allowbreak17 Le 23:34-\allowbreak43 Nu 29:12-\allowbreak38 De 16:13-\allowbreak16 1Ki 8:2,\allowbreak65}
\crossref{John}{7}{3}{7:5 Mt 12:46,\allowbreak47 Mr 3:31 Lu 8:19 Ac 2:14}
\crossref{John}{7}{4}{Pr 18:1,\allowbreak2 Mt 6:1,\allowbreak2,\allowbreak5,\allowbreak16; 23:5 Lu 6:45}
\crossref{John}{7}{5}{Joh 1:11-\allowbreak13 Mic 7:5,\allowbreak6 Mr 3:21}
\crossref{John}{7}{6}{7:8,\allowbreak30; 2:4; 8:20; 13:1; 17:1 Ps 102:13 Ec 3:1 etc.}
\crossref{John}{7}{7}{Joh 15:19 Lu 6:26 Jas 4:4 1Jo 4:5}
\crossref{John}{7}{8}{7:6,\allowbreak30; 8:20,\allowbreak30; 11:6,\allowbreak7 1Co 2:15,\allowbreak16}
\crossref{John}{7}{9}{}
\crossref{John}{7}{10}{Ps 26:8; 40:8 Mt 3:15 Ga 4:4}
\crossref{John}{7}{11}{Joh 11:56}
\crossref{John}{7}{12}{7:32; 9:16 Php 2:14}
\crossref{John}{7}{13}{Joh 3:2; 9:22,\allowbreak34; 12:42,\allowbreak43; 19:38; 20:19 Pr 29:25 Ga 2:12,\allowbreak13}
\crossref{John}{7}{14}{7:2,\allowbreak37 Nu 29:12,\allowbreak13,\allowbreak17,\allowbreak20,\allowbreak23 etc.}
\crossref{John}{7}{15}{7:46 Mt 7:28,\allowbreak29; 22:22,\allowbreak33 Lu 2:47}
\crossref{John}{7}{16}{Joh 3:11,\allowbreak31; 8:28; 12:49,\allowbreak50; 14:10,\allowbreak24; 17:8,\allowbreak14 Re 1:1}
\crossref{John}{7}{17}{Joh 1:46-\allowbreak49; 8:31,\allowbreak32,\allowbreak43,\allowbreak47 Ps 25:8,\allowbreak9,\allowbreak12; 119:10,\allowbreak101,\allowbreak102 Isa 35:8}
\crossref{John}{7}{18}{Joh 5:41; 8:49,\allowbreak50 1Co 10:31-\allowbreak33 Ga 6:12-\allowbreak14 Php 2:3-\allowbreak5 1Th 2:6}
\crossref{John}{7}{19}{Joh 1:17; 5:45; 9:28,\allowbreak29 Ex 24:2,\allowbreak3 De 33:4; 1:17 Ac 7:38 Ga 3:19}
\crossref{John}{7}{20}{Joh 8:48,\allowbreak52; 10:20 Mt 10:25; 11:18,\allowbreak19; 12:24 Mr 3:21,\allowbreak22,\allowbreak30 Ac 26:24}
\crossref{John}{7}{21}{Joh 5:9-\allowbreak11}
\crossref{John}{7}{22}{Ge 17:10-\allowbreak14 Le 12:3 Ro 4:9-\allowbreak11 Ga 3:17}
\crossref{John}{7}{23}{}
\crossref{John}{7}{24}{Joh 8:15 De 1:16,\allowbreak17; 16:18,\allowbreak19 Ps 58:1,\allowbreak2; 82:2; 94:20,\allowbreak21 Pr 17:15}
\crossref{John}{7}{25}{7:10,\allowbreak11}
\crossref{John}{7}{26}{Ps 40:9,\allowbreak10; 71:15,\allowbreak16 Pr 28:1 Isa 42:4; 50:7,\allowbreak8 Mt 22:16 Ac 4:13}
\crossref{John}{7}{27}{7:15; 6:42 Mt 13:54-\allowbreak57 Mr 6:3 Lu 4:22}
\crossref{John}{7}{28}{Joh 1:46; 8:14 Mt 2:23 Lu 2:4,\allowbreak11,\allowbreak39,\allowbreak51}
\crossref{John}{7}{29}{Joh 1:18; 8:55; 10:15; 17:25,\allowbreak26}
\crossref{John}{7}{30}{7:19,\allowbreak32; 8:37,\allowbreak59; 10:31,\allowbreak39; 11:57 Mr 11:18 Lu 19:47,\allowbreak48; 20:19}
\crossref{John}{7}{31}{Joh 2:23,\allowbreak24; 4:39; 6:14,\allowbreak15; 8:30-\allowbreak32; 12:42 Mt 12:23 Lu 8:13 Ac 8:13}
\crossref{John}{7}{32}{7:47-\allowbreak53; 11:47,\allowbreak48; 12:19 Mt 12:23,\allowbreak24; 23:13}
\crossref{John}{7}{33}{Joh 12:35,\allowbreak36; 13:1,\allowbreak3,\allowbreak33; 16:5,\allowbreak16-\allowbreak22; 17:11,\allowbreak13 Mr 16:19}
\crossref{John}{7}{34}{Joh 8:21-\allowbreak24; 13:33-\allowbreak36; 14:3,\allowbreak6; 17:24 Pr 1:24-\allowbreak31 Ho 5:6 Mt 23:39}
\crossref{John}{7}{35}{Isa 11:12; 27:12,\allowbreak13 Zep 3:10 Ac 21:21 Jas 1:1 1Pe 1:1}
\crossref{John}{7}{36}{Joh 3:4,\allowbreak9; 6:41,\allowbreak52,\allowbreak60; 12:34; 16:17,\allowbreak18}
\crossref{John}{7}{37}{Le 23:36,\allowbreak39 Nu 29:35 1Ki 8:65,\allowbreak66}
\crossref{John}{7}{38}{De 18:15}
\crossref{John}{7}{39}{Joh 14:16,\allowbreak17,\allowbreak26 Pr 1:23 Isa 12:3; 32:15; 44:3 Joe 2:28 Lu 3:16}
\crossref{John}{7}{40}{}
\crossref{John}{7}{41}{7:31; 1:41,\allowbreak49; 4:25,\allowbreak29,\allowbreak42; 6:69 Mt 16:14-\allowbreak16}
\crossref{John}{7}{42}{7:27 Ps 132:11 Isa 11:1 Jer 23:5 Mic 5:2 Mt 2:5 Lu 2:4,\allowbreak11}
\crossref{John}{7}{43}{7:12; 9:16; 10:19 Mt 10:35 Lu 12:51 Ac 14:4; 23:7-\allowbreak10}
\crossref{John}{7}{44}{7:30; 8:20; 18:5,\allowbreak6 Ac 18:10; 23:11; 27:23-\allowbreak25}
\crossref{John}{7}{45}{7:32 Ac 5:21-\allowbreak27}
\crossref{John}{7}{46}{7:26 Mt 7:29 Lu 4:22}
\crossref{John}{7}{47}{7:12; 9:27-\allowbreak34 2Ki 18:29,\allowbreak32 2Ch 32:15 Mt 27:63 2Co 6:8}
\crossref{John}{7}{48}{7:26,\allowbreak50; 12:42 Jer 5:4,\allowbreak5 Mt 11:25 Ac 6:7 1Co 1:20,\allowbreak22-\allowbreak28; 2:8}
\crossref{John}{7}{49}{Joh 9:34,\allowbreak40 Isa 5:21; 28:14; 29:14-\allowbreak19; 65:5 1Co 1:20,\allowbreak21; 3:18-\allowbreak20}
\crossref{John}{7}{50}{Joh 3:1,\allowbreak2; 19:39}
\crossref{John}{7}{51}{De 1:17; 17:8-\allowbreak11; 19:15-\allowbreak19 Pr 18:13}
\crossref{John}{7}{52}{Joh 9:34 Ge 19:9 Ex 2:14 1Ki 22:24 Pr 9:7,\allowbreak8}
\crossref{John}{7}{53}{Job 5:12,\allowbreak13 Ps 33:10; 76:5,\allowbreak10}
\crossref{John}{8}{1}{Mt 21:1 Mr 11:1; 13:3 Lu 19:37}
\crossref{John}{8}{2}{Joh 4:34 Ec 9:10 Jer 25:3; 44:4 Lu 21:37}
\crossref{John}{8}{3}{8:3}
\crossref{John}{8}{4}{}
\crossref{John}{8}{5}{Le 20:10 De 22:21-\allowbreak24 Eze 16:38-\allowbreak40; 23:47}
\crossref{John}{8}{6}{Nu 14:22 Mt 19:3 Lu 10:25; 11:53,\allowbreak54; 20:20-\allowbreak23 1Co 10:9}
\crossref{John}{8}{7}{Joh 7:46 Pr 12:18; 26:4,\allowbreak5 Jer 23:29 1Co 14:24,\allowbreak25 Col 4:6}
\crossref{John}{8}{8}{}
\crossref{John}{8}{9}{Ge 42:21,\allowbreak22 1Ki 2:44; 17:18 Ps 50:21 Ec 7:22 Mr 6:14-\allowbreak16}
\crossref{John}{8}{10}{Isa 41:11,\allowbreak12}
\crossref{John}{8}{11}{8:15; 3:17; 18:36 De 16:18; 17:9 Lu 9:56; 12:13,\allowbreak14 Ro 13:3,\allowbreak4 1Co 5:12}
\crossref{John}{8}{12}{Joh 1:4-\allowbreak9; 3:19; 9:5; 12:35 Isa 9:2; 42:6,\allowbreak7; 49:6; 60:1-\allowbreak3 Ho 6:3 Mal 4:2}
\crossref{John}{8}{13}{Joh 5:31-\allowbreak47 De 19:15}
\crossref{John}{8}{14}{Nu 12:3 Ne 5:14-\allowbreak19 2Co 11:31; 12:11,\allowbreak19}
\crossref{John}{8}{15}{Joh 7:24 1Sa 16:7 Ps 58:1,\allowbreak2; 94:20,\allowbreak21 Am 5:7; 6:12 Hab 1:4 Ro 2:1}
\crossref{John}{8}{16}{Joh 5:22-\allowbreak30 1Sa 16:7 Ps 45:6,\allowbreak7; 72:1,\allowbreak2; 98:9; 99:4 Isa 9:7; 11:2-\allowbreak5}
\crossref{John}{8}{17}{Joh 10:34; 15:25 Ga 3:24; 4:21}
\crossref{John}{8}{18}{8:12,\allowbreak25,\allowbreak38,\allowbreak51,\allowbreak58; 10:9,\allowbreak11,\allowbreak14,\allowbreak30; 11:25; 14:6 Re 1:17,\allowbreak18}
\crossref{John}{8}{19}{8:54,\allowbreak55; 1:10; 7:28; 10:14,\allowbreak15; 15:21; 16:3 Jer 22:16; 24:7 1Co 15:34}
\crossref{John}{8}{20}{1Ch 9:26 Mt 27:6 Mr 12:41,\allowbreak43}
\crossref{John}{8}{21}{Joh 7:34; 12:33,\allowbreak35 1Ki 18:10 2Ki 2:16,\allowbreak17 Mt 23:39; 24:23,\allowbreak24}
\crossref{John}{8}{22}{8:48,\allowbreak52; 7:20; 10:20 Ps 22:6; 31:18; 123:4 Heb 12:3; 13:13}
\crossref{John}{8}{23}{Joh 1:14; 3:13,\allowbreak31 Ps 17:4 Ro 8:7,\allowbreak8 1Co 15:47,\allowbreak48 Php 3:19-\allowbreak21}
\crossref{John}{8}{24}{8:21}
\crossref{John}{8}{25}{Joh 1:19,\allowbreak22; 10:24; 19:9 Lu 22:67}
\crossref{John}{8}{26}{Joh 16:12 Heb 5:11,\allowbreak12}
\crossref{John}{8}{27}{8:43,\allowbreak47 Isa 6:9; 42:18-\allowbreak20; 59:10 Ro 11:7-\allowbreak10 2Co 4:3,\allowbreak4}
\crossref{John}{8}{28}{Joh 3:14; 12:32-\allowbreak34; 19:18}
\crossref{John}{8}{29}{8:16; 14:10,\allowbreak11; 16:32 Isa 42:1,\allowbreak6; 49:4-\allowbreak8; 50:4-\allowbreak9 2Ti 4:17,\allowbreak22}
\crossref{John}{8}{30}{Joh 2:23; 6:14; 7:31; 10:42; 11:45}
\crossref{John}{8}{31}{Joh 6:66-\allowbreak71; 15:4-\allowbreak9 1Sa 12:14 Mt 24:13 Ac 13:43; 14:22; 26:22 Ro 2:7}
\crossref{John}{8}{32}{Joh 6:45; 7:17; 14:6; 16:13 Ps 25:5,\allowbreak8,\allowbreak9 Pr 1:23,\allowbreak29; 2:1-\allowbreak7; 4:18}
\crossref{John}{8}{33}{8:39 Le 25:42 Mt 3:9 Lu 16:24-\allowbreak26}
\crossref{John}{8}{34}{Joh 3:3 Mt 5:18}
\crossref{John}{8}{35}{Ge 21:10 Eze 46:17 Mt 21:41-\allowbreak43 Ga 4:30,\allowbreak31}
\crossref{John}{8}{36}{8:31,\allowbreak32 Ps 19:13; 119:32,\allowbreak133 Isa 49:24,\allowbreak25; 61:1 Zec 9:11,\allowbreak12}
\crossref{John}{8}{37}{8:33 Ac 13:26 Ro 9:7}
\crossref{John}{8}{38}{8:26; 3:32; 5:19,\allowbreak30; 12:49,\allowbreak50; 14:10,\allowbreak24; 17:8}
\crossref{John}{8}{39}{8:33}
\crossref{John}{8}{40}{8:37 Ps 37:12,\allowbreak32 Ga 4:16,\allowbreak29 1Jo 3:12-\allowbreak15 Re 12:4,\allowbreak12,\allowbreak13,\allowbreak17}
\crossref{John}{8}{41}{8:38,\allowbreak44}
\crossref{John}{8}{42}{Joh 5:23; 15:23,\allowbreak24 Mal 1:6 1Co 16:22 1Jo 5:1,\allowbreak2}
\crossref{John}{8}{43}{8:27; 5:43; 7:17; 12:39,\allowbreak40 Pr 28:5 Isa 44:18 Ho 14:9 Mic 4:12 Ro 3:11}
\crossref{John}{8}{44}{8:38,\allowbreak41; 6:70 Ge 3:15 Mt 13:38 Ac 13:10 1Jo 3:8-\allowbreak10,\allowbreak12}
\crossref{John}{8}{45}{Joh 3:19,\allowbreak20; 7:7 Ga 4:16 2Th 2:10 2Ti 4:3,\allowbreak4}
\crossref{John}{8}{46}{8:7; 14:30; 15:10; 16:8 2Co 5:21 Heb 4:15; 7:26 1Pe 2:22}
\crossref{John}{8}{47}{8:37,\allowbreak43,\allowbreak45; 1:12,\allowbreak13; 6:45,\allowbreak46,\allowbreak65; 10:26,\allowbreak27; 17:6-\allowbreak8 1Jo 3:10; 4:1-\allowbreak6}
\crossref{John}{8}{48}{8:52; 13:13 Mt 15:7 Jas 2:19}
\crossref{John}{8}{49}{Pr 26:4,\allowbreak5 1Pe 2:23}
\crossref{John}{8}{50}{Joh 5:41; 7:18}
\crossref{John}{8}{51}{Joh 3:15,\allowbreak16; 5:24; 6:50; 11:25,\allowbreak26}
\crossref{John}{8}{52}{8:48; 9:24}
\crossref{John}{8}{53}{8:58; 4:12; 10:29,\allowbreak30; 12:34 Isa 9:6 Mt 12:6,\allowbreak41,\allowbreak42 Ro 9:5 Heb 3:2,\allowbreak3}
\crossref{John}{8}{54}{8:50; 2:11; 5:31,\allowbreak32; 7:18 Pr 25:27 2Co 10:18 Heb 5:4,\allowbreak5}
\crossref{John}{8}{55}{8:19; 7:28,\allowbreak29; 15:21; 16:3; 17:25 Jer 4:22; 9:3 Ho 5:4 Ac 17:23 2Co 4:6}
\crossref{John}{8}{56}{Ge 22:18 Lu 2:28-\allowbreak30; 10:24 Ga 3:7-\allowbreak9 Heb 11:13,\allowbreak39 1Pe 1:10-\allowbreak12}
\crossref{John}{8}{57}{}
\crossref{John}{8}{58}{8:34,\allowbreak51}
\crossref{John}{8}{59}{8:5,\allowbreak6; 10:30-\allowbreak33; 11:8; 18:31 Le 24:16 Lu 4:29 Ac 7:57}
\crossref{John}{9}{1}{9:32}
\crossref{John}{9}{2}{9:34 Mt 16:14}
\crossref{John}{9}{3}{Job 1:8-\allowbreak12; 2:3-\allowbreak6; 21:27; 22:5 etc.}
\crossref{John}{9}{4}{Joh 4:34; 5:19,\allowbreak36; 10:32,\allowbreak37; 17:4 Lu 13:32-\allowbreak34 Ac 4:20}
\crossref{John}{9}{5}{Joh 1:4-\allowbreak9; 3:19-\allowbreak21; 8:12; 12:35,\allowbreak36,\allowbreak46 Isa 42:6,\allowbreak7; 49:6; 60:1-\allowbreak3 Mal 4:2}
\crossref{John}{9}{6}{Mr 7:33; 8:23 Re 3:18}
\crossref{John}{9}{7}{2Ki 5:10-\allowbreak14}
\crossref{John}{9}{8}{Ru 1:19 1Sa 21:11}
\crossref{John}{9}{9}{9:9}
\crossref{John}{9}{10}{9:15,\allowbreak21,\allowbreak26; 3:9 Ec 11:5 Mr 4:27 1Co 15:35}
\crossref{John}{9}{11}{9:6,\allowbreak7,\allowbreak27 Jer 36:17}
\crossref{John}{9}{12}{Joh 5:11-\allowbreak13; 7:11 Ex 2:18-\allowbreak20}
\crossref{John}{9}{13}{Joh 8:3-\allowbreak8; 11:46,\allowbreak47,\allowbreak57; 12:19,\allowbreak42}
\crossref{John}{9}{14}{Joh 5:9,\allowbreak16; 7:21-\allowbreak23 Mt 12:1-\allowbreak14 Mr 2:23-\allowbreak28; 3:1-\allowbreak6 Lu 6:1-\allowbreak11; 13:10-\allowbreak17}
\crossref{John}{9}{15}{9:10,\allowbreak11,\allowbreak26,\allowbreak27}
\crossref{John}{9}{16}{9:24,\allowbreak30-\allowbreak33; 3:2; 5:36; 14:11; 15:24}
\crossref{John}{9}{17}{Joh 4:19; 6:14 Lu 24:19 Ac 2:22; 3:22-\allowbreak26; 10:38}
\crossref{John}{9}{18}{Joh 5:44; 12:37-\allowbreak40 Ge 19:14 Isa 26:11; 53:1 Lu 16:31 Heb 3:15-\allowbreak19}
\crossref{John}{9}{19}{9:8,\allowbreak9 Ac 3:10; 4:14}
\crossref{John}{9}{20}{9:20}
\crossref{John}{9}{21}{}
\crossref{John}{9}{22}{Joh 7:13; 12:42,\allowbreak43; 19:38; 20:19 Ps 27:1,\allowbreak2 Pr 29:25 Isa 51:7,\allowbreak12; 57:11}
\crossref{John}{9}{23}{9:21}
\crossref{John}{9}{24}{Joh 5:23; 8:49; 16:2 Jos 7:19 1Sa 6:5-\allowbreak9 Ps 50:14,\allowbreak15 Isa 66:5}
\crossref{John}{9}{25}{9:30; 5:11 1Jo 5:10}
\crossref{John}{9}{26}{}
\crossref{John}{9}{27}{9:10-\allowbreak15 Lu 22:67}
\crossref{John}{9}{28}{9:34; 7:47-\allowbreak52 Isa 51:7 Mt 5:11; 27:39 1Co 4:12; 6:10 1Pe 2:23}
\crossref{John}{9}{29}{Joh 1:17 Nu 12:2-\allowbreak7; 16:28 De 34:10 Ps 103:7; 105:26; 106:16 Mal 4:4}
\crossref{John}{9}{30}{Joh 3:10; 12:37 Isa 29:14 Mr 6:6}
\crossref{John}{9}{31}{Job 27:8,\allowbreak9; 35:12; 42:8 Ps 18:41; 34:15; 66:18-\allowbreak20 Pr 1:28,\allowbreak29; 15:29}
\crossref{John}{9}{32}{Job 20:4 Isa 64:4 Lu 1:70 Re 16:18}
\crossref{John}{9}{33}{9:16; 3:2 Ac 5:38,\allowbreak39}
\crossref{John}{9}{34}{9:2; 8:41 Job 14:4; 15:14-\allowbreak16; 25:4 Ps 51:5 Ga 2:15 Eph 2:3}
\crossref{John}{9}{35}{Joh 5:14 Ps 27:10 Ro 10:20}
\crossref{John}{9}{36}{Joh 1:38 Pr 30:3,\allowbreak4 So 5:9 Mt 11:3}
\crossref{John}{9}{37}{Joh 4:26; 7:17; 14:21-\allowbreak23 Ps 25:8,\allowbreak9,\allowbreak14 Mt 11:25; 13:11,\allowbreak12 Ac 10:31-\allowbreak33}
\crossref{John}{9}{38}{Joh 20:28 Ps 2:12; 45:11 Mt 14:33; 28:9,\allowbreak17 Lu 24:52 Re 5:9-\allowbreak14}
\crossref{John}{9}{39}{Joh 3:17; 5:22-\allowbreak27; 8:15 Jer 1:9,\allowbreak10 Lu 2:34; 13:30 2Co 2:16}
\crossref{John}{9}{40}{9:34; 7:47-\allowbreak52 Mt 15:12-\allowbreak14; 23:16 etc.}
\crossref{John}{9}{41}{Joh 15:22-\allowbreak24 Pr 26:12 Isa 5:21 Jer 2:35 Lu 12:47; 18:14 Heb 10:26}
\crossref{John}{10}{1}{Joh 3:3}
\crossref{John}{10}{2}{10:7,\allowbreak9 Ac 20:28 1Ti 3:2-\allowbreak7; 4:14 Tit 1:5 Re 1:20; 2:1}
\crossref{John}{10}{3}{Isa 53:10-\allowbreak12 1Co 16:9 Col 4:3 1Pe 1:12 Re 3:7,\allowbreak8,\allowbreak20}
\crossref{John}{10}{4}{Joh 12:26; 13:15; 14:2,\allowbreak3 De 1:30 Mic 2:12,\allowbreak13 Mt 16:24 1Co 11:1}
\crossref{John}{10}{5}{1Ki 22:7 Pr 19:27 Mr 4:24 Lu 8:18 Eph 4:11-\allowbreak15 Col 2:6-\allowbreak10}
\crossref{John}{10}{6}{Joh 6:52,\allowbreak60; 7:36; 8:27,\allowbreak43 Ps 82:5; 106:7 Pr 28:5 Isa 6:9,\allowbreak10; 56:11}
\crossref{John}{10}{7}{10:1,\allowbreak9; 14:6 Eph 2:18 Heb 10:19-\allowbreak22}
\crossref{John}{10}{8}{10:1 Isa 56:10-\allowbreak12 Eze 22:25-\allowbreak28; 34:2 Zep 3:3 Zec 11:4-\allowbreak9,\allowbreak16 Ac 5:36}
\crossref{John}{10}{9}{10:1,\allowbreak7; 14:6 Ro 5:1,\allowbreak2 Eph 2:18 Heb 10:19-\allowbreak22}
\crossref{John}{10}{10}{10:1; 12:6 Isa 56:11 Eze 34:2-\allowbreak4 Ho 7:1 Mt 21:13; 23:14 Mr 11:17}
\crossref{John}{10}{11}{10:14 Ps 23:1; 80:1 Isa 40:11 Eze 34:12,\allowbreak23; 37:24 Mic 5:4 Zec 13:7}
\crossref{John}{10}{12}{10:3 Isa 56:10-\allowbreak12 Eze 34:2-\allowbreak6 Zec 11:16,\allowbreak17 1Ti 3:3,\allowbreak8 2Ti 4:10}
\crossref{John}{10}{13}{Joh 12:6 Ac 18:17 Php 2:20}
\crossref{John}{10}{14}{10:11}
\crossref{John}{10}{15}{Joh 1:18; 6:46; 8:55; 17:25 Mt 11:27 Lu 10:21 Re 5:2-\allowbreak9}
\crossref{John}{10}{16}{Joh 11:52 Ge 49:10 Ps 22:26-\allowbreak31; 72:17-\allowbreak19; 86:9; 98:2,\allowbreak3 Isa 11:10}
\crossref{John}{10}{17}{Joh 3:25; 15:9,\allowbreak10; 17:4,\allowbreak5,\allowbreak24-\allowbreak26 Isa 42:1,\allowbreak21; 53:7-\allowbreak12 Heb 2:9}
\crossref{John}{10}{18}{Joh 18:5,\allowbreak6; 19:11 Mt 26:53-\allowbreak56}
\crossref{John}{10}{19}{Joh 7:40-\allowbreak43; 9:16 Mt 10:34,\allowbreak35 Lu 12:51-\allowbreak53 Ac 14:4; 23:7-\allowbreak10 1Co 3:3}
\crossref{John}{10}{20}{Joh 7:20; 8:48,\allowbreak52 Mt 9:34; 10:25 Mr 3:21 Ac 26:24}
\crossref{John}{10}{21}{Joh 9:6,\allowbreak32 Ex 4:11; 8:19 Ps 94:9; 146:8 Pr 20:12 Isa 35:5,\allowbreak6 Mt 11:5}
\crossref{John}{10}{22}{}
\crossref{John}{10}{23}{Ac 3:11; 5:12}
\crossref{John}{10}{24}{1Ki 18:21 Mt 11:3 Lu 3:15}
\crossref{John}{10}{25}{Joh 5:17-\allowbreak43; 8:12,\allowbreak24,\allowbreak58}
\crossref{John}{10}{26}{10:4,\allowbreak27; 6:37,\allowbreak44,\allowbreak45,\allowbreak65; 8:47; 12:37-\allowbreak40 Ro 11:7,\allowbreak8 2Co 4:3,\allowbreak4 1Jo 4:6}
\crossref{John}{10}{27}{10:4,\allowbreak8,\allowbreak16; 5:25; 8:43 Mt 17:5 Ac 3:23 Heb 3:7 Re 3:20}
\crossref{John}{10}{28}{Joh 3:16,\allowbreak36; 5:39,\allowbreak40; 6:27,\allowbreak40,\allowbreak47,\allowbreak68; 11:25; 17:2 Ro 5:21; 6:23 1Ti 1:16}
\crossref{John}{10}{29}{Joh 6:37; 17:2,\allowbreak6,\allowbreak9,\allowbreak11}
\crossref{John}{10}{30}{Joh 1:1,\allowbreak2; 5:17,\allowbreak23; 8:58; 14:9,\allowbreak23; 16:15; 17:10,\allowbreak21 Mt 11:27; 28:19}
\crossref{John}{10}{31}{Joh 5:18; 8:59; 11:8 Ex 17:4 1Sa 30:6 Mt 21:35; 23:35 Ac 7:52,\allowbreak58,\allowbreak59}
\crossref{John}{10}{32}{10:25,\allowbreak37; 5:19,\allowbreak20,\allowbreak36 Mt 11:5 Ac 2:22; 10:38}
\crossref{John}{10}{33}{Le 24:14 1Ki 21:10}
\crossref{John}{10}{34}{Joh 12:34; 15:25 Ro 3:10-\allowbreak19}
\crossref{John}{10}{35}{Ge 15:1 De 18:15,\allowbreak18-\allowbreak20 1Sa 14:36,\allowbreak37; 15:1; 23:9-\allowbreak11; 28:6; 30:8}
\crossref{John}{10}{36}{Joh 3:34; 6:27 Ps 2:2,\allowbreak6-\allowbreak12 Isa 11:2-\allowbreak5; 42:1,\allowbreak3; 49:1-\allowbreak3,\allowbreak6-\allowbreak8; 55:4}
\crossref{John}{10}{37}{10:25,\allowbreak32; 5:31; 12:37-\allowbreak40; 14:10; 15:24 Mt 11:20-\allowbreak24}
\crossref{John}{10}{38}{Joh 3:2; 5:36 Ac 2:22; 4:8-\allowbreak12}
\crossref{John}{10}{39}{10:31; 7:30,\allowbreak44; 8:59 Lu 4:29,\allowbreak30}
\crossref{John}{10}{40}{Joh 1:28; 3:26}
\crossref{John}{10}{41}{Joh 3:26 Mt 4:23-\allowbreak25 Mr 1:37 Lu 5:1; 12:1}
\crossref{John}{10}{42}{Joh 2:23; 4:39,\allowbreak41; 8:30; 11:45; 12:42}
\crossref{John}{11}{1}{11:3,\allowbreak6 Ge 48:1 2Ki 20:1-\allowbreak12 Ac 9:37}
\crossref{John}{11}{2}{Joh 12:3 Mt 26:6,\allowbreak7 Mr 14:3}
\crossref{John}{11}{3}{11:1,\allowbreak5; 13:23 Ge 22:2 Ps 16:3 Php 2:26,\allowbreak27 2Ti 4:20 Heb 12:6,\allowbreak7}
\crossref{John}{11}{4}{Joh 9:3 Mr 5:39-\allowbreak42 Ro 11:11}
\crossref{John}{11}{5}{11:8,\allowbreak36; 15:9-\allowbreak13; 16:27; 17:26}
\crossref{John}{11}{6}{Ge 22:14; 42:24; 43:29-\allowbreak31; 44:1-\allowbreak5; 45:1-\allowbreak5 Isa 30:18; 55:8,\allowbreak9}
\crossref{John}{11}{7}{Joh 10:40-\allowbreak42 Lu 9:51 Ac 15:36; 20:22-\allowbreak24}
\crossref{John}{11}{8}{Joh 10:31,\allowbreak39 Ps 11:1-\allowbreak3 Mt 16:21-\allowbreak23 Ac 21:12,\allowbreak13}
\crossref{John}{11}{9}{Joh 9:4 Lu 13:31-\allowbreak33}
\crossref{John}{11}{10}{Ps 27:2 Pr 4:18,\allowbreak19 Ec 2:14 Jer 13:16; 20:11 1Jo 2:10,\allowbreak11}
\crossref{John}{11}{11}{Joh 3:29; 15:13-\allowbreak15 Ex 33:11 2Ch 20:7 Isa 41:8 Jas 2:23}
\crossref{John}{11}{12}{11:12}
\crossref{John}{11}{13}{}
\crossref{John}{11}{14}{Joh 10:24; 16:25,\allowbreak29}
\crossref{John}{11}{15}{11:35,\allowbreak36}
\crossref{John}{11}{16}{Joh 20:24-\allowbreak29; 21:2 Mt 10:3 Mr 3:18 Lu 6:15}
\crossref{John}{11}{17}{11:39; 2:19 Ho 6:2 Ac 2:27-\allowbreak31}
\crossref{John}{11}{18}{Joh 6:19 Lu 24:13 Re 14:20; 21:16}
\crossref{John}{11}{19}{Ge 37:35 2Sa 10:2 1Ch 7:21,\allowbreak22 Job 2:11; 42:11 Ec 7:2 Isa 51:19}
\crossref{John}{11}{20}{11:30 Mt 25:1,\allowbreak6 Ac 10:25; 28:15 1Th 4:17}
\crossref{John}{11}{21}{11:32,\allowbreak37; 4:47-\allowbreak49 1Ki 17:18 Ps 78:19,\allowbreak41 Mt 9:18 Lu 7:6-\allowbreak10,\allowbreak13-\allowbreak15; 8:49-\allowbreak55}
\crossref{John}{11}{22}{11:41,\allowbreak42; 9:31 Mr 9:23 Heb 11:17-\allowbreak19}
\crossref{John}{11}{23}{11:43,\allowbreak44}
\crossref{John}{11}{24}{Joh 5:28,\allowbreak29 Ps 17:15; 49:14,\allowbreak15 Isa 25:8; 26:19 Eze 37:1-\allowbreak10 Da 12:2,\allowbreak3}
\crossref{John}{11}{25}{Joh 5:21; 6:39,\allowbreak40,\allowbreak44 Ro 5:17-\allowbreak19; 8:11 1Co 15:20-\allowbreak26,\allowbreak43-\allowbreak57 2Co 4:14}
\crossref{John}{11}{26}{Joh 3:15-\allowbreak18; 4:14; 5:24; 6:50,\allowbreak54-\allowbreak58; 8:52,\allowbreak53; 10:28 Ro 8:13 1Jo 5:10-\allowbreak12}
\crossref{John}{11}{27}{Joh 1:49; 4:42; 6:69; 9:36-\allowbreak38; 20:28-\allowbreak31 Mt 16:16 Ac 8:37 1Jo 5:1}
\crossref{John}{11}{28}{11:20; 1:41,\allowbreak45; 21:7 Zec 3:10 Lu 10:38-\allowbreak42 1Th 4:17,\allowbreak18; 5:11 Heb 12:12}
\crossref{John}{11}{29}{Ps 27:8; 119:59,\allowbreak60 Pr 15:23; 27:17 So 3:1-\allowbreak4}
\crossref{John}{11}{30}{}
\crossref{John}{11}{31}{11:19}
\crossref{John}{11}{32}{Lu 5:8; 8:41; 17:16 Re 5:8,\allowbreak14; 22:8}
\crossref{John}{11}{33}{Ro 12:15}
\crossref{John}{11}{34}{Joh 1:39; 20:2 Mt 28:6 Mr 15:47; 16:6}
\crossref{John}{11}{35}{11:33 Ge 43:30 Job 30:25 Ps 35:13-\allowbreak15; 119:136 Isa 53:3; 63:9}
\crossref{John}{11}{36}{Joh 14:21-\allowbreak23; 21:15-\allowbreak17 2Co 8:8,\allowbreak9 Eph 5:2,\allowbreak25 1Jo 3:1; 4:9,\allowbreak10 Re 1:5}
\crossref{John}{11}{37}{Joh 9:6 Ps 78:19,\allowbreak20 Mt 27:40-\allowbreak42 Mr 15:32 Lu 23:35,\allowbreak39}
\crossref{John}{11}{38}{11:33 Eze 9:4; 21:6 Mr 8:12}
\crossref{John}{11}{39}{Mr 16:3}
\crossref{John}{11}{40}{11:23-\allowbreak26 2Ch 20:20 Ro 4:17-\allowbreak25}
\crossref{John}{11}{41}{Joh 12:28-\allowbreak30; 17:1 Ps 123:1 Lu 18:13}
\crossref{John}{11}{42}{11:22; 8:29; 12:27,\allowbreak28 Mt 26:53 Heb 5:7; 7:25}
\crossref{John}{11}{43}{1Ki 17:21,\allowbreak22 2Ki 4:33-\allowbreak36 Mr 4:41 Lu 7:14,\allowbreak15 Ac 3:6,\allowbreak12; 9:34,\allowbreak40}
\crossref{John}{11}{44}{11:25,\allowbreak26; 5:21,\allowbreak25; 10:30 Ge 1:3 1Sa 2:6 Ps 33:9 Eze 37:3-\allowbreak10 Ho 13:14}
\crossref{John}{11}{45}{11:19,\allowbreak31; 2:23; 10:41; 12:9-\allowbreak11,\allowbreak17-\allowbreak19,\allowbreak42}
\crossref{John}{11}{46}{Joh 5:15,\allowbreak16; 9:13; 12:37 Lu 16:30,\allowbreak31 Ac 5:25}
\crossref{John}{11}{47}{Ps 2:2-\allowbreak4 Mt 26:3; 27:1,\allowbreak2 Mr 14:1 Lu 22:2 Ac 4:5,\allowbreak6,\allowbreak27,\allowbreak28; 5:21}
\crossref{John}{11}{48}{Ac 5:28,\allowbreak38-\allowbreak40}
\crossref{John}{11}{49}{Joh 18:13,\allowbreak14 Lu 3:2 Ac 4:6}
\crossref{John}{11}{50}{11:48; 18:14; 19:12 Lu 24:46 Ro 3:8}
\crossref{John}{11}{51}{Ex 28:30 Jud 20:27,\allowbreak28 1Sa 23:9; 28:6}
\crossref{John}{11}{52}{Joh 1:29; 12:32 Ps 22:15,\allowbreak27; 72:19 Isa 49:6 Lu 2:32 Ro 3:29 1Jo 2:2}
\crossref{John}{11}{53}{Ne 4:16; 13:21 Ps 113:2 Mt 16:21; 22:46}
\crossref{John}{11}{54}{Joh 4:1-\allowbreak3; 7:1; 10:40; 18:20}
\crossref{John}{11}{55}{Joh 2:13; 5:1; 6:4 Ex 12:11 etc.}
\crossref{John}{11}{56}{11:8; 7:11; 11:7}
\crossref{John}{11}{57}{Joh 5:16-\allowbreak18; 8:59; 9:22; 10:39 Ps 109:4}
\crossref{John}{12}{1}{Joh 11:55}
\crossref{John}{12}{2}{So 4:16; 5:1 Lu 5:29; 14:12 Re 3:20}
\crossref{John}{12}{3}{Joh 11:2,\allowbreak28,\allowbreak32 Mt 26:6,\allowbreak7 etc.}
\crossref{John}{12}{4}{1Sa 17:28,\allowbreak29 Ec 4:4}
\crossref{John}{12}{5}{Ex 5:8,\allowbreak17 Am 8:5 Mal 1:10-\allowbreak13 Mt 26:8 Mr 14:4 Lu 6:41}
\crossref{John}{12}{6}{Joh 10:13 Ps 14:1 Pr 29:7 Eze 33:31 Ga 2:10 Jas 2:2,\allowbreak6}
\crossref{John}{12}{7}{Ps 109:31 Zec 3:2 Mt 26:10 Mr 14:6}
\crossref{John}{12}{8}{De 15:11 Mt 26:11 Mr 14:7}
\crossref{John}{12}{9}{Joh 11:43-\allowbreak45 Ac 3:10,\allowbreak11; 4:14}
\crossref{John}{12}{10}{Joh 11:47-\allowbreak53,\allowbreak57 Ge 4:4-\allowbreak10 Ex 10:3 Job 15:25,\allowbreak26; 40:8,\allowbreak9 Ec 9:3}
\crossref{John}{12}{11}{12:18; 11:45,\allowbreak48; 15:18-\allowbreak25 Ac 13:45 Jas 3:14-\allowbreak16}
\crossref{John}{12}{12}{Mt 21:8}
\crossref{John}{12}{13}{Le 23:40 Re 7:9}
\crossref{John}{12}{14}{Mt 21:1-\allowbreak7 Mr 11:1-\allowbreak7 Lu 19:29-\allowbreak35}
\crossref{John}{12}{15}{Isa 35:4,\allowbreak5; 40:9,\allowbreak10; 41:14; 62:11 Mic 4:8 Zep 3:16,\allowbreak17 Zec 2:9-\allowbreak11}
\crossref{John}{12}{16}{Lu 9:45; 18:34; 24:25,\allowbreak45}
\crossref{John}{12}{17}{12:9; 11:31,\allowbreak45,\allowbreak46 Ps 145:6,\allowbreak7}
\crossref{John}{12}{18}{12:9-\allowbreak11}
\crossref{John}{12}{19}{Joh 11:47-\allowbreak50 Mt 21:15 Lu 19:47,\allowbreak48 Ac 4:16,\allowbreak17; 5:27,\allowbreak28}
\crossref{John}{12}{20}{Joh 7:35 Mr 7:26 Ac 14:1; 16:1; 17:4; 20:21; 21:28 Ro 1:16; 10:12}
\crossref{John}{12}{21}{Joh 1:43-\allowbreak47; 6:5-\allowbreak7; 14:8,\allowbreak9}
\crossref{John}{12}{22}{Joh 1:40,\allowbreak41; 6:8}
\crossref{John}{12}{23}{Joh 13:31,\allowbreak32; 17:1-\allowbreak5,\allowbreak9,\allowbreak10 Isa 49:5,\allowbreak6; 53:10-\allowbreak12; 55:5; 60:9 Mt 25:31}
\crossref{John}{12}{24}{Ps 72:16 1Co 15:36-\allowbreak38}
\crossref{John}{12}{25}{Mt 10:39; 16:25; 19:29 Mr 8:35 Lu 9:23,\allowbreak24; 17:33 Ac 20:24; 21:13}
\crossref{John}{12}{26}{Joh 13:16; 14:15; 15:20 Lu 6:46 Ro 1:1; 14:18 2Co 4:5 Ga 1:10}
\crossref{John}{12}{27}{Joh 11:33-\allowbreak35; 13:21 Ps 69:1-\allowbreak3; 88:3 Isa 53:3 Mt 26:38,\allowbreak39,\allowbreak42}
\crossref{John}{12}{28}{Joh 18:11 Mt 26:42 Mr 14:36}
\crossref{John}{12}{29}{Ex 19:16; 20:18 Job 37:2-\allowbreak5; 40:9 Eze 10:5 Re 6:1; 8:5; 11:19; 14:2}
\crossref{John}{12}{30}{Joh 5:34; 11:15,\allowbreak42 2Co 8:9}
\crossref{John}{12}{31}{Joh 5:22-\allowbreak27; 16:8-\allowbreak10}
\crossref{John}{12}{32}{Joh 3:14; 8:28; 19:17 De 21:22,\allowbreak23 2Sa 18:9 Ps 22:16-\allowbreak18 Ga 3:13}
\crossref{John}{12}{33}{Joh 18:32; 21:19}
\crossref{John}{12}{34}{Joh 10:34; 15:25 Ro 3:19; 5:18}
\crossref{John}{12}{35}{Joh 7:33; 9:4; 16:16 Heb 3:7,\allowbreak8}
\crossref{John}{12}{36}{Joh 1:7; 3:21 Isa 60:1 Ac 13:47,\allowbreak48}
\crossref{John}{12}{37}{Joh 1:11; 11:42; 15:24 Mt 11:20 Lu 16:31}
\crossref{John}{12}{38}{Joh 15:25; 17:12; 19:24,\allowbreak36,\allowbreak37 Mt 27:35 Ac 13:27-\allowbreak29}
\crossref{John}{12}{39}{Joh 5:44; 6:44; 10:38 Isa 44:18-\allowbreak20 2Pe 2:14}
\crossref{John}{12}{40}{Joh 9:39 1Ki 22:20 Isa 29:10 Eze 14:9 Mt 13:13-\allowbreak15; 15:14 Mr 4:12}
\crossref{John}{12}{41}{Isa 6:1-\allowbreak5,\allowbreak9,\allowbreak10}
\crossref{John}{12}{42}{Joh 3:2; 7:48-\allowbreak51; 11:45; 19:38}
\crossref{John}{12}{43}{Joh 5:41,\allowbreak44 Mt 6:2; 23:5-\allowbreak7 Lu 16:15 Ps 22:29 1Th 2:6}
\crossref{John}{12}{44}{Joh 7:28; 11:43 Pr 1:20; 8:1 Isa 55:1-\allowbreak3}
\crossref{John}{12}{45}{12:41; 14:9,\allowbreak10; 15:24 2Co 4:6 Col 1:15 Heb 1:3 1Jo 5:20}
\crossref{John}{12}{46}{12:35,\allowbreak36; 1:4,\allowbreak5; 3:19; 8:12; 9:5,\allowbreak39 Ps 36:9 Isa 40:1 Mal 4:2 Mt 4:16}
\crossref{John}{12}{47}{12:48; 5:45; 8:15,\allowbreak16,\allowbreak26}
\crossref{John}{12}{48}{De 18:19 1Sa 8:7; 10:19 Isa 53:3 Mt 21:42 Mr 8:31; 12:10}
\crossref{John}{12}{49}{Joh 3:11,\allowbreak32; 5:30; 6:38-\allowbreak40; 8:26,\allowbreak42; 14:10; 15:15; 17:8 De 18:18 Re 1:1}
\crossref{John}{12}{50}{Joh 6:63,\allowbreak68; 17:3; 20:31 1Ti 1:16 1Jo 2:25; 3:23,\allowbreak24; 5:11-\allowbreak13,\allowbreak20}
\crossref{John}{13}{1}{Joh 6:4 Mt 26:2 etc.}
\crossref{John}{13}{2}{13:4,\allowbreak26}
\crossref{John}{13}{3}{Joh 3:35; 5:22-\allowbreak27; 17:2 Mt 11:27; 28:18 Lu 10:22 Ac 2:36 1Co 15:27}
\crossref{John}{13}{4}{}
\crossref{John}{13}{5}{Joh 19:34 2Ki 3:11 Eze 36:25 Zec 13:1 Eph 5:26 1Jo 5:6}
\crossref{John}{13}{6}{Joh 1:27 Mt 3:11-\allowbreak14 Lu 5:8}
\crossref{John}{13}{7}{13:10-\allowbreak12; 12:16; 14:26 Jer 32:24,\allowbreak25,\allowbreak43 Da 12:8,\allowbreak12 Hab 2:1-\allowbreak3 Jas 5:7-\allowbreak11}
\crossref{John}{13}{8}{Ge 42:38 Mt 16:22; 21:29; 26:33,\allowbreak35 Col 2:18,\allowbreak23}
\crossref{John}{13}{9}{Ps 26:6; 51:2,\allowbreak7 Jer 4:14 Mt 27:24 Heb 10:22 1Pe 3:21}
\crossref{John}{13}{10}{Le 16:26,\allowbreak28; 17:15,\allowbreak16 Nu 19:7,\allowbreak8,\allowbreak12,\allowbreak13,\allowbreak19-\allowbreak21 Heb 9:10}
\crossref{John}{13}{11}{13:18,\allowbreak21,\allowbreak26; 2:25; 6:64-\allowbreak71; 17:12 Mt 26:24,\allowbreak25}
\crossref{John}{13}{12}{13:7 Eze 24:19,\allowbreak24 Mt 13:51 Mr 4:13}
\crossref{John}{13}{13}{Joh 11:28 Mt 7:21,\allowbreak22; 23:8-\allowbreak10 Lu 6:46 Ro 14:8,\allowbreak9 1Co 8:6; 12:3}
\crossref{John}{13}{14}{Mt 20:26-\allowbreak28 Mr 10:43-\allowbreak45 Lu 22:26,\allowbreak27 2Co 8:9 Php 2:5-\allowbreak8}
\crossref{John}{13}{15}{Mt 11:29 Ro 15:5}
\crossref{John}{13}{16}{Joh 3:3,\allowbreak5}
\crossref{John}{13}{17}{Joh 15:14 Ge 6:22 Ex 40:16 Ps 19:11; 119:1-\allowbreak5 Eze 36:27 Mt 7:24,\allowbreak25}
\crossref{John}{13}{18}{13:11; 17:12; 21:17 2Co 4:5 Heb 4:13 Re 2:23}
\crossref{John}{13}{19}{Joh 14:29; 16:4 Isa 41:23; 48:5 Mt 24:25 Lu 21:13}
\crossref{John}{13}{20}{Joh 12:44-\allowbreak48 Mt 10:40-\allowbreak42; 25:40 Mr 9:37 Lu 9:48; 10:16 Ga 4:14}
\crossref{John}{13}{21}{Joh 11:33,\allowbreak35,\allowbreak38; 12:27 Mt 26:38 Mr 3:5 Ac 17:16 Ro 9:2,\allowbreak3}
\crossref{John}{13}{22}{Ge 42:1 Mt 26:22 Mr 14:19 Lu 22:23}
\crossref{John}{13}{23}{13:25; 1:18; 21:20 2Sa 12:3}
\crossref{John}{13}{24}{Lu 1:22; 5:7 Ac 12:17; 13:16; 21:40}
\crossref{John}{13}{25}{Ge 44:4-\allowbreak12 Es 7:5}
\crossref{John}{13}{26}{13:30 Mt 26:23 Mr 14:19,\allowbreak20 Lu 22:21}
\crossref{John}{13}{27}{13:2 Ps 109:6 Mt 12:45 Lu 8:32,\allowbreak33; 22:3 Ac 5:3}
\crossref{John}{13}{28}{}
\crossref{John}{13}{29}{Joh 12:5 Ac 20:34,\allowbreak35 Ga 2:10 Eph 4:28}
\crossref{John}{13}{30}{Pr 4:16 Isa 59:7 Ro 3:15}
\crossref{John}{13}{31}{Joh 7:39; 11:4; 12:23; 16:14 Lu 12:50 Ac 2:36; 3:13 Col 2:14,\allowbreak15}
\crossref{John}{13}{32}{Joh 17:4-\allowbreak6,\allowbreak21-\allowbreak24 Isa 53:10-\allowbreak12 Heb 1:2,\allowbreak3 1Pe 3:22 Re 3:21; 21:22,\allowbreak23}
\crossref{John}{13}{33}{Ga 4:19 1Jo 2:1; 4:4; 5:21}
\crossref{John}{13}{34}{Ga 6:2 1Jo 2:8-\allowbreak10; 3:14-\allowbreak18,\allowbreak23 2Jo 1:5}
\crossref{John}{13}{35}{Joh 17:21 Ge 13:7,\allowbreak8 Ac 4:32-\allowbreak35; 5:12-\allowbreak14 1Jo 2:5,\allowbreak10; 3:10-\allowbreak14; 4:20,\allowbreak21}
\crossref{John}{13}{36}{13:33; 14:4,\allowbreak5; 16:17; 21:21}
\crossref{John}{13}{37}{Joh 21:15 Mt 26:31-\allowbreak35 Mr 14:27-\allowbreak31 Lu 22:31-\allowbreak34 Ac 20:24; 21:13}
\crossref{John}{13}{38}{Pr 16:18; 28:26; 29:23 1Co 10:12}
\crossref{John}{14}{1}{14:27,\allowbreak28; 11:33}
\crossref{John}{14}{2}{2Co 5:1 Heb 11:10,\allowbreak14-\allowbreak16; 13:14 Re 3:12,\allowbreak21; 21:10-\allowbreak27}
\crossref{John}{14}{3}{14:18-\allowbreak23,\allowbreak28; 12:26; 17:24 Mt 25:32-\allowbreak34 Ac 1:11; 7:59,\allowbreak60 Ro 8:17 2Co 5:6-\allowbreak8}
\crossref{John}{14}{4}{14:2,\allowbreak28; 13:3; 16:28 Lu 24:26}
\crossref{John}{14}{5}{Joh 20:25-\allowbreak28}
\crossref{John}{14}{6}{Joh 10:9 Isa 35:8,\allowbreak9 Mt 11:27 Ac 4:12 Ro 5:2 Eph 2:18 Heb 7:25}
\crossref{John}{14}{7}{14:9,\allowbreak10,\allowbreak20; 1:18; 8:19; 15:24; 16:3; 17:3,\allowbreak21,\allowbreak23 Mt 11:27 Lu 10:22 2Co 4:6}
\crossref{John}{14}{8}{Joh 1:43-\allowbreak46; 6:5-\allowbreak7; 12:21,\allowbreak22}
\crossref{John}{14}{9}{Mr 9:19}
\crossref{John}{14}{10}{14:20; 1:1-\allowbreak3; 10:30,\allowbreak38; 11:26; 17:21-\allowbreak23 1Jo 5:7}
\crossref{John}{14}{11}{Joh 5:36; 10:25,\allowbreak32,\allowbreak38; 12:38-\allowbreak40 Mt 11:4,\allowbreak5 Lu 7:21-\allowbreak23 Ac 2:22 Heb 2:4}
\crossref{John}{14}{12}{Mt 21:21 Mr 11:13; 16:17 Lu 10:17-\allowbreak19 Ac 3:6-\allowbreak8; 4:9-\allowbreak12,\allowbreak16,\allowbreak33; 8:7}
\crossref{John}{14}{13}{Joh 15:7,\allowbreak16; 16:23,\allowbreak26 Mt 7:7; 21:22 Mr 11:24 Lu 11:9 Eph 3:20}
\crossref{John}{14}{14}{14:14}
\crossref{John}{14}{15}{14:21-\allowbreak24; 8:42; 15:10-\allowbreak14; 21:15-\allowbreak17 Mt 10:37; 25:34-\allowbreak40 1Co 16:22}
\crossref{John}{14}{16}{14:14; 16:26,\allowbreak27; 17:9-\allowbreak11,\allowbreak15,\allowbreak20 Ro 8:34 Heb 7:25 1Jo 2:1}
\crossref{John}{14}{17}{Joh 15:26; 16:13 1Jo 2:27; 4:6}
\crossref{John}{14}{18}{14:16,\allowbreak27; 16:33 Ps 23:4 Isa 43:1; 51:12; 66:11-\allowbreak13 2Co 1:2-\allowbreak6 2Th 2:16}
\crossref{John}{14}{19}{Joh 7:33; 8:21; 12:35; 13:33; 16:16,\allowbreak22}
\crossref{John}{14}{20}{14:10; 10:38; 17:7,\allowbreak11,\allowbreak21-\allowbreak23,\allowbreak26 2Co 5:19 Col 1:19; 2:9}
\crossref{John}{14}{21}{14:15,\allowbreak23,\allowbreak24; 15:14 Ge 26:3-\allowbreak5 De 10:12,\allowbreak13; 11:13; 30:6-\allowbreak8 Ps 119:4-\allowbreak6}
\crossref{John}{14}{22}{Mt 10:3}
\crossref{John}{14}{23}{14:15,\allowbreak21}
\crossref{John}{14}{24}{14:15,\allowbreak21-\allowbreak23 Mt 19:21; 25:41-\allowbreak46 2Co 8:8,\allowbreak9 1Jo 3:16-\allowbreak20}
\crossref{John}{14}{25}{14:29; 13:19; 15:11; 16:1-\allowbreak4,\allowbreak12; 17:6-\allowbreak8}
\crossref{John}{14}{26}{14:16}
\crossref{John}{14}{27}{Joh 16:33; 20:19,\allowbreak21,\allowbreak26 Nu 6:26 Ps 29:11; 72:2,\allowbreak7; 85:10 Isa 9:6}
\crossref{John}{14}{28}{14:3,\allowbreak18; 16:16-\allowbreak22}
\crossref{John}{14}{29}{Joh 13:19; 16:4-\allowbreak30,\allowbreak31 Mt 24:24,\allowbreak25}
\crossref{John}{14}{30}{Joh 16:12 Lu 24:44-\allowbreak49 Ac 1:3}
\crossref{John}{14}{31}{Joh 4:34; 10:18; 12:27; 15:9; 18:11 Ps 40:8 Mt 26:39 Php 2:8}
\crossref{John}{15}{1}{Joh 1:9,\allowbreak17; 6:32,\allowbreak55 1Jo 2:8}
\crossref{John}{15}{2}{Joh 17:12 Mt 3:10; 15:13; 21:19 Lu 8:13; 13:7-\allowbreak9 1Co 13:1 Heb 6:7,\allowbreak8}
\crossref{John}{15}{3}{Joh 13:10; 17:17 Eph 5:26 1Pe 1:22}
\crossref{John}{15}{4}{Joh 6:68,\allowbreak69; 8:31 So 8:5 Lu 8:15 Ac 11:23; 14:22 Ga 2:20 Col 1:23}
\crossref{John}{15}{5}{Ro 12:5 1Co 10:16; 12:12,\allowbreak27 1Pe 2:4}
\crossref{John}{15}{6}{Job 15:30 Ps 80:15 Isa 14:19; 27:10 Eze 15:3-\allowbreak7; 17:9; 19:12-\allowbreak14}
\crossref{John}{15}{7}{Joh 8:37 De 6:6 Job 23:12 Ps 119:11 Pr 4:4 Jer 15:16 Col 3:16}
\crossref{John}{15}{8}{Ps 92:12-\allowbreak15 Isa 60:21; 61:3 Hag 1:8 Mt 5:16 1Co 6:20; 10:31}
\crossref{John}{15}{9}{15:13; 17:23,\allowbreak26 Eph 3:18 Re 1:5}
\crossref{John}{15}{10}{Joh 14:15,\allowbreak21 1Co 7:19 1Th 4:1 2Pe 2:21 1Jo 2:5; 3:21-\allowbreak24; 5:3}
\crossref{John}{15}{11}{Isa 53:11; 62:4 Jer 32:41; 33:9 Zep 3:17 Lu 15:5,\allowbreak9,\allowbreak23,\allowbreak32 1Jo 1:4}
\crossref{John}{15}{12}{Joh 13:34 Ro 12:10 Eph 5:2 1Th 3:12; 4:9 2Th 1:3 1Pe 1:22; 3:8; 4:8}
\crossref{John}{15}{13}{Joh 10:11,\allowbreak15 Ro 5:6-\allowbreak8 Eph 5:2 1Jo 4:7-\allowbreak11}
\crossref{John}{15}{14}{Joh 14:15,\allowbreak28 2Ch 20:7 So 5:1 Isa 41:8 Mt 12:50 Lu 12:4 Jas 2:23}
\crossref{John}{15}{15}{15:20; 12:26; 13:16; 20:17 Ga 4:6 Phm 1:16 Jas 1:1 2Pe 1:1 Jude 1:1}
\crossref{John}{15}{16}{15:19; 6:70; 13:18 Lu 6:13 Ac 1:24; 9:15; 10:41; 22:14 Ro 9:11-\allowbreak16,\allowbreak21}
\crossref{John}{15}{17}{15:12 1Pe 2:17 1Jo 3:14-\allowbreak17}
\crossref{John}{15}{18}{15:23-\allowbreak25; 3:20; 7:7 1Ki 22:8 Isa 49:7; 53:3 Zec 11:8 Mt 5:11; 10:22; 24:9}
\crossref{John}{15}{19}{Lu 6:32 1Jo 4:4,\allowbreak5}
\crossref{John}{15}{20}{Joh 5:16; 7:32; 8:59; 10:31; 11:57; 13:16 Mt 10:24 Lu 2:34; 6:40}
\crossref{John}{15}{21}{Joh 16:3 Ps 69:7 Isa 66:5 Mt 5:11; 10:18,\allowbreak22,\allowbreak39; 24:9 Lu 6:22 Ac 9:16}
\crossref{John}{15}{22}{Joh 3:18-\allowbreak21; 9:41; 12:48; 19:11 Eze 2:5; 33:31-\allowbreak33 Lu 12:46 Ac 17:30}
\crossref{John}{15}{23}{Joh 8:40-\allowbreak42 1Jo 2:23 2Jo 1:9}
\crossref{John}{15}{24}{Joh 3:2; 5:36; 7:31; 9:32; 10:32,\allowbreak37; 11:47-\allowbreak50; 12:10,\allowbreak37-\allowbreak40 Mt 9:33; 11:5}
\crossref{John}{15}{25}{Joh 10:34; 19:36 Lu 24:44 Ro 3:19}
\crossref{John}{15}{26}{Joh 14:16,\allowbreak17,\allowbreak26; 16:7,\allowbreak13,\allowbreak14 Lu 24:49 Ac 2:33}
\crossref{John}{15}{27}{Joh 21:24 Lu 24:48 Ac 1:8,\allowbreak21,\allowbreak22; 3:15; 4:20,\allowbreak33; 10:39-\allowbreak42; 13:31}
\crossref{John}{16}{1}{16:4; 15:11 Mt 11:6; 13:21,\allowbreak57; 24:10; 26:31-\allowbreak33 Ro 14:21 Php 1:10 1Pe 2:8}
\crossref{John}{16}{2}{Joh 9:22,\allowbreak34; 12:42 Lu 6:22 1Co 4:13}
\crossref{John}{16}{3}{Joh 8:19,\allowbreak55; 15:21,\allowbreak23; 17:3,\allowbreak25 Lu 10:22 1Co 2:8 2Co 4:3-\allowbreak6 2Th 1:8}
\crossref{John}{16}{4}{Joh 13:19; 14:29 Isa 41:22,\allowbreak23 Mt 10:7; 24:25 Mr 13:23 Lu 21:12,\allowbreak13}
\crossref{John}{16}{5}{16:10,\allowbreak16,\allowbreak28; 6:62; 7:33; 13:3; 14:28; 17:4,\allowbreak13 Eph 4:7-\allowbreak11 Heb 1:3; 12:2}
\crossref{John}{16}{6}{16:20-\allowbreak22; 14:1,\allowbreak27,\allowbreak28; 20:11-\allowbreak15 Lu 22:45; 24:17}
\crossref{John}{16}{7}{Joh 8:45,\allowbreak46 Lu 4:25; 9:27 Ac 10:34}
\crossref{John}{16}{8}{Zec 12:10 Ac 2:37; 16:29,\allowbreak30}
\crossref{John}{16}{9}{Joh 3:18-\allowbreak21; 5:40-\allowbreak44; 8:23,\allowbreak24,\allowbreak42-\allowbreak47; 12:47,\allowbreak48; 15:22-\allowbreak25 Mr 16:16}
\crossref{John}{16}{10}{Isa 42:21; 45:24,\allowbreak25 Jer 23:5,\allowbreak6 Da 9:24 Ac 2:32 Ro 1:17; 3:21-\allowbreak26}
\crossref{John}{16}{11}{Joh 5:22-\allowbreak27 Mt 12:18,\allowbreak36 Ac 10:42; 17:30,\allowbreak31; 24:25; 26:18 Ro 2:2-\allowbreak4,\allowbreak16}
\crossref{John}{16}{12}{Joh 14:30; 15:15 Ac 1:3}
\crossref{John}{16}{13}{Joh 14:17; 15:26 1Jo 4:6}
\crossref{John}{16}{14}{16:9,\allowbreak10 Ac 2:32-\allowbreak36; 4:10-\allowbreak12 1Co 12:3 1Pe 1:10-\allowbreak12; 2:7}
\crossref{John}{16}{15}{Joh 3:35; 10:29,\allowbreak30; 13:3; 17:2,\allowbreak10 Mt 11:27; 28:18 Lu 10:22 Col 1:19}
\crossref{John}{16}{16}{16:5,\allowbreak10,\allowbreak17-\allowbreak19; 7:33; 12:35; 13:33; 14:19}
\crossref{John}{16}{17}{16:1,\allowbreak5,\allowbreak19; 12:16; 14:5,\allowbreak22 Mr 9:10,\allowbreak32 Lu 9:45; 18:34}
\crossref{John}{16}{18}{Mt 16:9-\allowbreak11 Lu 24:25 Heb 5:12}
\crossref{John}{16}{19}{16:30; 2:24,\allowbreak25; 21:17 Ps 139:1-\allowbreak4 Mt 6:8; 9:4 Mr 9:33,\allowbreak34 Heb 4:13}
\crossref{John}{16}{20}{16:6,\allowbreak33; 19:25-\allowbreak27 Mr 14:72; 16:10 Lu 22:45,\allowbreak62; 23:47-\allowbreak49; 24:17,\allowbreak21}
\crossref{John}{16}{21}{Ge 3:16 Isa 26:16-\allowbreak18 Jer 30:6,\allowbreak7 Ho 13:13,\allowbreak14 Mic 4:10 Re 12:2-\allowbreak5}
\crossref{John}{16}{22}{16:6,\allowbreak20}
\crossref{John}{16}{23}{16:19; 13:36,\allowbreak37; 14:5,\allowbreak22; 15:15; 21:20,\allowbreak21}
\crossref{John}{16}{24}{Ge 32:9 1Ki 18:36 2Ki 19:15 Mt 6:9 Eph 1:16,\allowbreak17 1Th 3:11-\allowbreak13}
\crossref{John}{16}{25}{16:12,\allowbreak16,\allowbreak17 Ps 49:4; 78:2 Pr 1:6 Mt 13:10,\allowbreak11,\allowbreak34,\allowbreak35 Mr 4:13}
\crossref{John}{16}{26}{16:23}
\crossref{John}{16}{27}{Joh 14:21,\allowbreak23; 17:23,\allowbreak26 Zep 3:17 Heb 12:6 Jude 1:20,\allowbreak21 Re 3:9,\allowbreak19}
\crossref{John}{16}{28}{Joh 8:14; 13:1,\allowbreak3}
\crossref{John}{16}{29}{16:25}
\crossref{John}{16}{30}{16:17; 5:20; 21:17 Heb 4:13}
\crossref{John}{16}{31}{Joh 13:38 Lu 9:44,\allowbreak45}
\crossref{John}{16}{32}{Joh 4:21,\allowbreak23; 5:25,\allowbreak28; 12:23}
\crossref{John}{16}{33}{Joh 14:27 Ps 85:8-\allowbreak11 Isa 9:6,\allowbreak7 Mic 5:5 Lu 2:14; 19:38 Ro 5:1,\allowbreak2}
\crossref{John}{17}{1}{Joh 11:41 Ps 121:1,\allowbreak2; 123:1 Isa 38:14 Lu 18:13}
\crossref{John}{17}{2}{Joh 3:35; 5:21-\allowbreak29 Ps 2:6-\allowbreak12; 110:1 Da 7:14 Mt 11:27; 28:18 1Co 15:25}
\crossref{John}{17}{3}{17:25; 8:19,\allowbreak54,\allowbreak55 1Ch 28:9 Ps 9:10 Isa 53:11 Jer 9:23,\allowbreak24; 31:33,\allowbreak34}
\crossref{John}{17}{4}{Joh 12:28; 13:31,\allowbreak32; 14:13}
\crossref{John}{17}{5}{17:24; 1:18; 3:13; 10:30; 14:9 Pr 8:22-\allowbreak31 Php 2:6 Col 1:15-\allowbreak17 Heb 1:3,\allowbreak10}
\crossref{John}{17}{6}{17:26; 1:18; 12:28 Ex 3:13-\allowbreak15; 9:16; 34:5-\allowbreak7 Ps 22:22; 71:17-\allowbreak19}
\crossref{John}{17}{7}{Joh 7:16,\allowbreak17; 14:7-\allowbreak10,\allowbreak20; 16:27-\allowbreak30}
\crossref{John}{17}{8}{17:14; 6:68; 14:10 Pr 1:23 Mt 13:11 Eph 3:2-\allowbreak8; 4:11,\allowbreak12}
\crossref{John}{17}{9}{Joh 14:16; 16:26,\allowbreak27 Lu 22:32; 8:34 Heb 7:25; 9:24 1Jo 2:1,\allowbreak2; 5:19}
\crossref{John}{17}{10}{Joh 10:30; 16:14,\allowbreak15 1Co 3:21-\allowbreak23 Col 1:15-\allowbreak19; 2:9}
\crossref{John}{17}{11}{17:13; 13:1,\allowbreak3; 16:28 Ac 1:9-\allowbreak11; 3:21 Heb 1:3; 9:24}
\crossref{John}{17}{12}{Joh 6:37,\allowbreak39,\allowbreak40; 10:27,\allowbreak28 Heb 2:13}
\crossref{John}{17}{13}{17:1; 13:3 Heb 12:2}
\crossref{John}{17}{14}{17:8}
\crossref{John}{17}{15}{Ps 30:9 Ec 9:10 Isa 38:18,\allowbreak19; 57:1 Lu 8:38,\allowbreak39 Php 1:20-\allowbreak26}
\crossref{John}{17}{16}{}
\crossref{John}{17}{17}{17:19; 8:32; 15:3 Ps 19:7-\allowbreak9; 119:9,\allowbreak11,\allowbreak104 Lu 8:11,\allowbreak15 Ac 15:9 2Co 3:18}
\crossref{John}{17}{18}{Joh 20:21 Isa 61:1-\allowbreak3 Mt 23:34 2Co 5:20 Eph 3:7}
\crossref{John}{17}{19}{Isa 62:1 2Co 4:15; 8:9 1Th 4:7 2Ti 2:10}
\crossref{John}{17}{20}{17:6-\allowbreak11 Eph 4:11}
\crossref{John}{17}{21}{17:11,\allowbreak22,\allowbreak23; 10:16 Jer 32:39 Eze 37:16-\allowbreak19,\allowbreak22-\allowbreak25 Zep 3:9 Zec 14:9}
\crossref{John}{17}{22}{Joh 1:16; 15:18,\allowbreak19; 20:21-\allowbreak23 Mr 6:7; 16:17-\allowbreak20 Lu 22:30 Ac 5:41}
\crossref{John}{17}{23}{Joh 6:56; 14:10,\allowbreak23 Ro 8:10,\allowbreak11 1Co 1:30 2Co 5:21 Ga 3:28 1Jo 1:3}
\crossref{John}{17}{24}{Joh 12:26; 14:3 Mt 25:21,\allowbreak23; 26:29 Lu 12:37; 22:28-\allowbreak30; 23:43 2Co 5:8}
\crossref{John}{17}{25}{17:11 Isa 45:21 Ro 3:26}
\crossref{John}{17}{26}{17:6; 8:50; 15:15 Ps 22:22 Heb 2:12}
\crossref{John}{18}{1}{Joh 13:31 etc.}
\crossref{John}{18}{2}{Mr 11:11,\allowbreak12 Lu 21:37; 22:39}
\crossref{John}{18}{3}{Joh 13:2,\allowbreak27-\allowbreak30 Mt 26:47,\allowbreak55 Mr 14:43,\allowbreak44,\allowbreak48 Lu 22:47 etc.}
\crossref{John}{18}{4}{Joh 10:17,\allowbreak18; 13:1; 19:28 Mt 16:21; 17:22,\allowbreak23; 20:18,\allowbreak19; 26:2,\allowbreak21,\allowbreak31}
\crossref{John}{18}{5}{Joh 1:46; 19:19 Mt 2:23; 21:11}
\crossref{John}{18}{6}{}
\crossref{John}{18}{7}{}
\crossref{John}{18}{8}{Isa 53:6 Eph 5:25}
\crossref{John}{18}{9}{Joh 17:12}
\crossref{John}{18}{10}{18:26 Mt 26:51-\allowbreak54 Mr 14:30,\allowbreak47 Lu 22:33,\allowbreak49-\allowbreak51}
\crossref{John}{18}{11}{18:36 2Co 6:7; 10:4 Eph 6:11-\allowbreak17}
\crossref{John}{18}{12}{18:3 Mt 26:57 Mr 14:53 Lu 22:54}
\crossref{John}{18}{13}{Mt 26:57}
\crossref{John}{18}{14}{Joh 11:49-\allowbreak52}
\crossref{John}{18}{15}{Mt 26:58 etc.}
\crossref{John}{18}{16}{}
\crossref{John}{18}{17}{18:16 Mt 26:69,\allowbreak70 Mr 14:66-\allowbreak68 Lu 22:54,\allowbreak56,\allowbreak57}
\crossref{John}{18}{18}{18:25 Mr 14:54 Lu 22:55,\allowbreak56}
\crossref{John}{18}{19}{Lu 11:53,\allowbreak54; 20:20}
\crossref{John}{18}{20}{Joh 7:14,\allowbreak26,\allowbreak28; 8:2; 10:23-\allowbreak39 Ps 22:22; 40:9 Mt 4:23; 9:35; 21:23 etc.}
\crossref{John}{18}{21}{Mt 26:59,\allowbreak60 Mr 14:55-\allowbreak59 Lu 22:67 Ac 24:12,\allowbreak13,\allowbreak18-\allowbreak20}
\crossref{John}{18}{22}{Job 16:10; 30:10-\allowbreak12 Isa 50:5-\allowbreak7 Jer 20:2 Mic 5:1 Mt 26:67,\allowbreak68}
\crossref{John}{18}{23}{2Co 10:1 1Pe 2:20-\allowbreak23}
\crossref{John}{18}{24}{18:13}
\crossref{John}{18}{25}{18:18 Mr 14:37,\allowbreak38,\allowbreak67 Lu 22:56}
\crossref{John}{18}{26}{18:10}
\crossref{John}{18}{27}{Joh 13:38 Mt 26:34,\allowbreak74,\allowbreak75 Mr 14:30,\allowbreak68,\allowbreak71,\allowbreak72 Lu 22:34,\allowbreak60-\allowbreak62}
\crossref{John}{18}{28}{Mt 27:1,\allowbreak2 etc.}
\crossref{John}{18}{29}{Mt 27:23 Ac 23:28-\allowbreak30; 25:16}
\crossref{John}{18}{30}{Joh 19:12 Mr 15:3 Lu 20:19-\allowbreak26; 23:2-\allowbreak5}
\crossref{John}{18}{31}{Joh 19:6,\allowbreak7 Ac 25:18-\allowbreak20}
\crossref{John}{18}{32}{Joh 3:14; 10:31,\allowbreak33; 12:32,\allowbreak33 Mt 20:19; 26:2 Lu 18:32,\allowbreak33; 24:7,\allowbreak8}
\crossref{John}{18}{33}{18:37 Mt 27:11 Mr 15:2 Lu 23:3,\allowbreak4 1Ti 6:13}
\crossref{John}{18}{34}{18:36}
\crossref{John}{18}{35}{Ezr 4:12 Ne 4:2 Ac 18:14-\allowbreak16; 23:29; 25:19,\allowbreak20 Ro 3:1,\allowbreak2}
\crossref{John}{18}{36}{1Ti 6:13}
\crossref{John}{18}{37}{Mt 26:64; 27:11 Mr 14:62; 15:2 Lu 23:3 1Ti 6:13}
\crossref{John}{18}{38}{Ac 17:19,\allowbreak20,\allowbreak32; 24:25,\allowbreak26}
\crossref{John}{18}{39}{Mt 27:15-\allowbreak18 Mr 15:6,\allowbreak8 Lu 23:17,\allowbreak20}
\crossref{John}{18}{40}{Mt 27:16,\allowbreak26 Mr 15:7,\allowbreak15 Lu 23:18,\allowbreak19,\allowbreak25 Ac 3:13,\allowbreak14}
\crossref{John}{19}{1}{Mt 27:26 etc.}
\crossref{John}{19}{2}{19:5 Ps 22:6 Isa 49:7; 53:3 Mt 27:27-\allowbreak31 Mr 15:17-\allowbreak20 Lu 23:11}
\crossref{John}{19}{3}{Mt 26:49; 27:29 Lu 1:28}
\crossref{John}{19}{4}{19:6; 18:38 Mt 27:4,\allowbreak19,\allowbreak24,\allowbreak54 Lu 23:41,\allowbreak47 2Co 5:21 Heb 7:26 1Pe 1:19}
\crossref{John}{19}{5}{Joh 1:29 Isa 7:14; 40:9; 43:1 La 1:12 Heb 12:2}
\crossref{John}{19}{6}{19:15 Mt 27:22 Mr 15:12-\allowbreak15 Lu 22:21-\allowbreak23 Ac 2:23; 3:13-\allowbreak15}
\crossref{John}{19}{7}{Le 24:16 De 18:20}
\crossref{John}{19}{8}{19:13 Ac 14:11-\allowbreak19}
\crossref{John}{19}{9}{Joh 8:14; 9:29,\allowbreak30 Jud 13:6}
\crossref{John}{19}{10}{Joh 18:39 Da 3:14,\allowbreak15; 5:19}
\crossref{John}{19}{11}{Joh 3:27; 7:30 Ge 45:7,\allowbreak8 Ex 9:14-\allowbreak16 1Ch 29:11 Ps 39:9; 62:11}
\crossref{John}{19}{12}{Mr 6:16-\allowbreak26 Ac 24:24-\allowbreak27}
\crossref{John}{19}{13}{19:8 Pr 29:25 Isa 51:12,\allowbreak13; 57:11 Lu 12:5 Ac 4:19}
\crossref{John}{19}{14}{19:31,\allowbreak32,\allowbreak42 Mt 27:62 Mr 15:42 Lu 23:54}
\crossref{John}{19}{15}{19:6 Lu 23:18 Ac 21:36; 22:22}
\crossref{John}{19}{16}{Mt 27:26-\allowbreak31 Mr 15:15-\allowbreak20 Lu 23:24}
\crossref{John}{19}{17}{Mt 10:38; 16:24; 27:31-\allowbreak33 Mr 8:34; 10:21; 15:21,\allowbreak22 Lu 9:23; 14:27}
\crossref{John}{19}{18}{Joh 18:32 Ps 22:16 Isa 53:12 Mt 27:35-\allowbreak38,\allowbreak44 Mr 15:24-\allowbreak28}
\crossref{John}{19}{19}{Mt 27:37 Mr 15:26 Lu 23:38}
\crossref{John}{19}{20}{19:13; 5:2 Ac 21:40; 22:2; 26:14 Re 16:16}
\crossref{John}{19}{21}{}
\crossref{John}{19}{22}{19:12 Ps 65:7; 76:10 Pr 8:29}
\crossref{John}{19}{23}{Mt 27:35 Mr 15:24 Lu 23:34}
\crossref{John}{19}{24}{19:28,\allowbreak36,\allowbreak37; 10:35; 12:38,\allowbreak39}
\crossref{John}{19}{25}{Lu 2:35}
\crossref{John}{19}{26}{Joh 13:23; 20:2; 21:7,\allowbreak20,\allowbreak24}
\crossref{John}{19}{27}{Ge 45:8; 47:12 Mt 12:48-\allowbreak50; 25:40 Mr 3:34 1Ti 5:2-\allowbreak4}
\crossref{John}{19}{28}{19:30; 13:1; 18:4 Lu 9:31; 12:50; 18:31; 22:37 Ac 13:29}
\crossref{John}{19}{29}{Mt 27:34,\allowbreak48 Mr 15:36 Lu 23:36}
\crossref{John}{19}{30}{19:28}
\crossref{John}{19}{31}{19:14,\allowbreak42 Mt 27:62 Mr 15:42}
\crossref{John}{19}{32}{19:18 Lu 23:39-\allowbreak43}
\crossref{John}{19}{33}{}
\crossref{John}{19}{34}{Joh 13:8-\allowbreak10 Ps 51:7 Eze 36:25 Zec 13:1 Mt 27:62 Ac 22:16 1Co 1:30}
\crossref{John}{19}{35}{19:26; 21:24 Ac 10:39 Heb 2:3,\allowbreak4 1Pe 5:1 1Jo 1:1-\allowbreak3}
\crossref{John}{19}{36}{Ex 12:46 Nu 9:12 Ps 22:14; 34:20; 35:10}
\crossref{John}{19}{37}{Ps 22:16,\allowbreak17 Zec 12:10 Re 1:7}
\crossref{John}{19}{38}{Mt 27:57-\allowbreak60 Mr 15:42-\allowbreak46 Lu 23:50}
\crossref{John}{19}{39}{Joh 3:1 etc.}
\crossref{John}{19}{40}{Joh 11:44; 20:5-\allowbreak7 Ac 5:6}
\crossref{John}{19}{41}{Joh 20:15 2Ki 23:30 Isa 22:16 Mt 27:60,\allowbreak64-\allowbreak66 Lu 23:53}
\crossref{John}{19}{42}{Ps 22:15 Isa 53:9 Mt 12:40 Ac 13:29 1Co 15:4 Col 2:12}
\crossref{John}{20}{1}{20:19,\allowbreak26 Ac 20:7 1Co 16:2 Re 1:10}
\crossref{John}{20}{2}{Joh 13:23; 19:26; 21:7,\allowbreak20,\allowbreak24}
\crossref{John}{20}{3}{Lu 24:12}
\crossref{John}{20}{4}{2Sa 18:23 Le 13:30 1Co 9:24 2Co 8:12}
\crossref{John}{20}{5}{Joh 11:44; 19:40}
\crossref{John}{20}{6}{Joh 6:67-\allowbreak69; 18:17,\allowbreak25-\allowbreak27; 21:7,\allowbreak15-\allowbreak17 Mt 16:15,\allowbreak16 Lu 22:31,\allowbreak32}
\crossref{John}{20}{7}{Joh 11:44}
\crossref{John}{20}{8}{20:25,\allowbreak29; 1:50}
\crossref{John}{20}{9}{Mt 16:21,\allowbreak22 Mr 8:31-\allowbreak33; 9:9,\allowbreak10,\allowbreak31,\allowbreak32 Lu 9:45; 18:33,\allowbreak34; 24:26}
\crossref{John}{20}{10}{Joh 7:53; 16:32}
\crossref{John}{20}{11}{}
\crossref{John}{20}{12}{Mt 28:3-\allowbreak5 Mr 16:5,\allowbreak6 Lu 24:3-\allowbreak7,\allowbreak22,\allowbreak23}
\crossref{John}{20}{13}{Joh 2:4; 19:26}
\crossref{John}{20}{14}{So 3:3,\allowbreak4 Mt 28:9 Mr 16:9}
\crossref{John}{20}{15}{Joh 1:38; 18:4,\allowbreak7 So 3:2; 6:1 Mt 28:5 Mr 16:6 Lu 24:5}
\crossref{John}{20}{16}{Joh 10:3 Ge 22:1,\allowbreak11 Ex 3:4; 33:17 1Sa 3:6,\allowbreak10 Isa 43:1 Lu 10:41}
\crossref{John}{20}{17}{Ps 22:22 Mt 12:50; 25:40; 28:10 Ro 8:29 Heb 2:11-\allowbreak13}
\crossref{John}{20}{18}{Mt 28:10 Mr 16:10-\allowbreak13 Lu 24:10}
\crossref{John}{20}{19}{Mr 16:14 Lu 24:36-\allowbreak49 1Co 15:5}
\crossref{John}{20}{20}{20:27 Lu 24:39,\allowbreak40 1Jo 1:1}
\crossref{John}{20}{21}{Joh 14:27}
\crossref{John}{20}{22}{Ge 2:7 Job 33:4 Ps 33:6 Eze 37:9}
\crossref{John}{20}{23}{Mt 16:19; 18:18 Mr 2:5-\allowbreak10 Ac 2:38; 10:43; 13:38,\allowbreak39 1Co 5:4}
\crossref{John}{20}{24}{Joh 11:16; 14:5; 21:2 Mt 10:3}
\crossref{John}{20}{25}{20:14-\allowbreak20; 1:41; 21:7 Mr 16:11 Lu 24:34-\allowbreak40 Ac 5:30-\allowbreak32; 10:40,\allowbreak41 1Co 15:5-\allowbreak8}
\crossref{John}{20}{26}{20:19 Mt 17:1 Lu 9:28}
\crossref{John}{20}{27}{20:25 Ps 78:38; 103:13,\allowbreak14 Ro 5:20 1Ti 1:14-\allowbreak16 1Jo 1:1,\allowbreak2}
\crossref{John}{20}{28}{}
\crossref{John}{20}{29}{20:8; 4:48 Lu 1:45 2Co 5:7 Heb 11:1,\allowbreak27,\allowbreak39 1Pe 1:8}
\crossref{John}{20}{30}{Joh 21:25 Lu 1:3,\allowbreak4 Ro 15:4 1Co 10:11 2Ti 3:15-\allowbreak17 2Pe 3:1,\allowbreak2}
\crossref{John}{20}{31}{20:28; 1:49; 6:69,\allowbreak70; 9:35-\allowbreak38 Ps 2:7,\allowbreak12 Mt 16:16; 27:54 Lu 1:4}
\crossref{John}{21}{1}{Joh 20:19-\allowbreak29}
\crossref{John}{21}{2}{Joh 20:28}
\crossref{John}{21}{3}{2Ki 6:1-\allowbreak7 Mt 4:18-\allowbreak20 Lu 5:10,\allowbreak11 Ac 18:3; 20:34 1Co 9:6 1Th 2:9}
\crossref{John}{21}{4}{Joh 20:14 Mr 16:12 Lu 24:15,\allowbreak16,\allowbreak31}
\crossref{John}{21}{5}{1Jo 2:13,\allowbreak18}
\crossref{John}{21}{6}{Mt 7:27 Lu 5:4-\allowbreak7}
\crossref{John}{21}{7}{21:20,\allowbreak24; 13:23; 19:26; 20:2}
\crossref{John}{21}{8}{De 3:11}
\crossref{John}{21}{9}{1Ki 19:5,\allowbreak6 Mt 4:11 Mr 8:3 Lu 12:29-\allowbreak31}
\crossref{John}{21}{10}{}
\crossref{John}{21}{11}{Lu 5:6-\allowbreak8 Ac 2:41}
\crossref{John}{21}{12}{Ac 10:41}
\crossref{John}{21}{13}{Lu 24:42,\allowbreak43 Ac 10:41}
\crossref{John}{21}{14}{}
\crossref{John}{21}{15}{21:16,\allowbreak17; 1:42}
\crossref{John}{21}{16}{Joh 18:17,\allowbreak25 Mt 26:72}
\crossref{John}{21}{17}{Joh 13:38; 18:27 Mt 26:73,\allowbreak74 Re 3:19}
\crossref{John}{21}{18}{Joh 13:36 Ac 12:3,\allowbreak4}
\crossref{John}{21}{19}{Php 1:20 1Pe 4:11-\allowbreak14 2Pe 1:14}
\crossref{John}{21}{20}{21:7,\allowbreak24; 20:2}
\crossref{John}{21}{21}{Mt 24:3,\allowbreak4 Lu 13:23,\allowbreak24 Ac 1:6,\allowbreak7}
\crossref{John}{21}{22}{Mt 16:27,\allowbreak28; 24:3,\allowbreak27,\allowbreak44; 25:31 Mr 9:1 1Co 4:5; 11:26 Re 1:7}
\crossref{John}{21}{23}{De 29:29 Job 28:28; 33:13 Da 4:35}
\crossref{John}{21}{24}{Joh 19:35 1Jo 1:1,\allowbreak2; 5:6 3Jo 1:12}
\crossref{John}{21}{25}{Joh 20:30,\allowbreak31 Job 26:14 Ps 40:5; 71:15 Ec 12:12 Mt 11:5}

% Acts
\crossref{Acts}{1}{1}{Lu 1:24}
\crossref{Acts}{1}{2}{1:9 Mr 16:19 Lu 9:51; 24:51 Joh 6:62; 13:1,\allowbreak3; 16:28; 17:13; 20:17}
\crossref{Acts}{1}{3}{Ac 13:31 Mt 28:9,\allowbreak16 Mr 16:10-\allowbreak14 Lu 24:1-\allowbreak53 Joh 20:1-\allowbreak21:25}
\crossref{Acts}{1}{4}{Ac 10:41 Lu 24:41-\allowbreak43}
\crossref{Acts}{1}{5}{Ac 11:15; 19:4 Mt 3:11 Lu 3:16 Joh 1:31 1Co 12:13 Tit 3:5}
\crossref{Acts}{1}{6}{Mt 24:3 Joh 21:21}
\crossref{Acts}{1}{7}{Ac 17:26 De 29:29 Da 2:21 Mt 24:36 Mr 13:32 Lu 21:24 Eph 1:10}
\crossref{Acts}{1}{8}{Ac 2:1-\allowbreak4; 6:8; 8:19 Mic 3:8 Zec 4:6 Lu 10:19 Re 11:3-\allowbreak6}
\crossref{Acts}{1}{9}{1:2 Ps 68:18 Mr 16:19 Lu 24:50,\allowbreak51 Joh 6:62 Eph 4:8-\allowbreak12}
\crossref{Acts}{1}{10}{2Ki 2:11,\allowbreak12}
\crossref{Acts}{1}{11}{Ac 2:7; 13:31 Mr 14:70}
\crossref{Acts}{1}{12}{Zec 14:4 Mt 21:1; 24:3; 26:30 Lu 21:37; 24:52}
\crossref{Acts}{1}{13}{Ac 9:37-\allowbreak39; 20:8 Mr 14:15 Lu 22:12}
\crossref{Acts}{1}{14}{Ac 2:1,\allowbreak42,\allowbreak46; 4:24-\allowbreak31; 6:4 Mt 18:19,\allowbreak20; 21:22 Lu 11:13; 18:1; 24:53}
\crossref{Acts}{1}{15}{Ps 32:5,\allowbreak6; 51:9-\allowbreak13 Lu 22:32 Joh 21:15-\allowbreak17}
\crossref{Acts}{1}{16}{Ac 2:29,\allowbreak37; 7:2; 13:15,\allowbreak26,\allowbreak38; 15:7,\allowbreak13; 22:1; 23:1,\allowbreak6; 28:17}
\crossref{Acts}{1}{17}{Mt 10:4 Mr 3:19 Lu 6:16; 22:47 Joh 6:70,\allowbreak71; 17:12}
\crossref{Acts}{1}{18}{Mt 27:3-\allowbreak10}
\crossref{Acts}{1}{19}{Ac 2:22 Mt 28:15}
\crossref{Acts}{1}{20}{Ac 13:33 Lu 20:42; 24:44}
\crossref{Acts}{1}{21}{Lu 10:1,\allowbreak2 Joh 15:27}
\crossref{Acts}{1}{22}{Ac 13:24,\allowbreak25 Mt 3:1-\allowbreak17 Mr 1:1,\allowbreak3-\allowbreak8 Lu 3:1-\allowbreak18 Joh 1:28-\allowbreak51}
\crossref{Acts}{1}{23}{Ac 15:22}
\crossref{Acts}{1}{24}{Ac 13:2,\allowbreak3 Pr 3:5,\allowbreak6 Lu 6:12,\allowbreak13}
\crossref{Acts}{1}{25}{1:17,\allowbreak20}
\crossref{Acts}{1}{26}{Ac 13:19 Le 16:8 Jos 18:10 1Sa 14:41,\allowbreak42 1Ch 24:5 Pr 16:22 Jon 1:7}
\crossref{Acts}{2}{1}{Ac 20:16 Ex 23:16; 34:22 Le 23:15-\allowbreak21 Nu 28:16-\allowbreak31 De 16:9-\allowbreak12}
\crossref{Acts}{2}{2}{Ac 16:25,\allowbreak26 Isa 65:24 Mal 3:1 Lu 2:13}
\crossref{Acts}{2}{3}{2:4,\allowbreak11 Ge 11:6 Ps 55:9 1Co 12:10 Re 14:6}
\crossref{Acts}{2}{4}{Ac 1:5; 4:8,\allowbreak31; 6:3,\allowbreak5,\allowbreak8; 7:55; 9:17; 11:24; 13:9,\allowbreak52 Lu 1:15,\allowbreak41,\allowbreak67; 4:1}
\crossref{Acts}{2}{5}{2:1; 8:27 Ex 23:16 Isa 66:18 Zec 8:18 Lu 24:18 Joh 12:20}
\crossref{Acts}{2}{6}{Ac 3:11 1Co 16:9 2Co 2:12}
\crossref{Acts}{2}{7}{2:12; 3:10; 14:11,\allowbreak12 Mr 1:27; 2:12}
\crossref{Acts}{2}{8}{}
\crossref{Acts}{2}{9}{2Ki 17:6 Ezr 6:2 Da 8:20}
\crossref{Acts}{2}{10}{Ac 16:6; 18:23}
\crossref{Acts}{2}{11}{Ac 27:7,\allowbreak12 Tit 1:5,\allowbreak12}
\crossref{Acts}{2}{12}{Ac 10:17; 17:20 Lu 15:26; 18:36}
\crossref{Acts}{2}{13}{2:15 1Sa 1:14 Job 32:19 So 7:9 Isa 25:6 Zec 9:15,\allowbreak17; 10:7}
\crossref{Acts}{2}{14}{Ac 1:26}
\crossref{Acts}{2}{15}{1Sa 1:15}
\crossref{Acts}{2}{16}{Joe 2:28-\allowbreak32}
\crossref{Acts}{2}{17}{Ge 49:1 Isa 2:2 Da 10:14 Ho 3:5 Mic 4:1 Heb 1:2 Jas 5:3}
\crossref{Acts}{2}{18}{1Co 7:21,\allowbreak22 Ga 3:28 Col 3:11}
\crossref{Acts}{2}{19}{Joe 2:30,\allowbreak31 Zep 1:14-\allowbreak18 Mal 4:1-\allowbreak6}
\crossref{Acts}{2}{20}{Isa 13:9,\allowbreak15; 24:23 Jer 4:23 Am 8:9 Mt 24:29; 27:45 Mr 13:24}
\crossref{Acts}{2}{21}{Ac 9:11,\allowbreak15; 22:16 Ps 86:5 Joe 2:32 Mt 28:19 Ro 10:12,\allowbreak13 1Co 1:2}
\crossref{Acts}{2}{22}{Ac 3:12; 5:35; 13:16; 21:28 Isa 41:14}
\crossref{Acts}{2}{23}{Ac 3:18; 4:28; 13:27; 15:18 Ps 76:10 Isa 10:6,\allowbreak7; 46:10,\allowbreak11 Da 4:35}
\crossref{Acts}{2}{24}{2:32; 3:15,\allowbreak26; 10:40,\allowbreak41; 13:30,\allowbreak34; 17:31 Mt 27:63 Lu 24:1-\allowbreak53}
\crossref{Acts}{2}{25}{2:29,\allowbreak30; 13:32-\allowbreak36}
\crossref{Acts}{2}{26}{Ps 16:9; 22:22-\allowbreak24; 30:11; 63:5; 71:23}
\crossref{Acts}{2}{27}{Ps 49:15; 86:13; 116:3 Lu 16:23 1Co 15:55 Re 1:18; 20:13}
\crossref{Acts}{2}{28}{Ps 16:11; 21:4; 25:4 Pr 2:19; 8:20 Joh 11:25,\allowbreak26; 14:6}
\crossref{Acts}{2}{29}{Ac 26:26}
\crossref{Acts}{2}{30}{Ac 1:16 2Sa 23:2 Mt 27:35 Mr 12:36 Lu 24:44 Heb 3:7; 4:7 2Pe 1:21}
\crossref{Acts}{2}{31}{1Pe 1:11,\allowbreak12}
\crossref{Acts}{2}{32}{2:24; 1:8,\allowbreak22; 3:15; 4:33; 5:31,\allowbreak32; 10:39-\allowbreak41 Lu 24:46-\allowbreak48}
\crossref{Acts}{2}{33}{Ac 5:31 Ps 89:19,\allowbreak24; 118:16,\allowbreak22,\allowbreak23 Isa 52:13; 53:12 Mt 28:18}
\crossref{Acts}{2}{34}{Ps 110:1 Mt 22:42-\allowbreak45 Mr 12:36 Lu 20:42,\allowbreak43 1Co 15:25 Eph 1:22}
\crossref{Acts}{2}{35}{Ge 3:15 Jos 10:24,\allowbreak25 Ps 2:8-\allowbreak12; 18:40-\allowbreak42; 21:8-\allowbreak12; 72:9 Isa 49:23}
\crossref{Acts}{2}{36}{Jer 2:4; 9:26; 31:31; 33:14 Eze 34:30; 39:25-\allowbreak29 Zec 13:1 Ro 9:3-\allowbreak6}
\crossref{Acts}{2}{37}{Ac 5:33; 7:54 Eze 7:16 Zec 12:10 Lu 3:10 Joh 8:9; 16:8-\allowbreak11 Ro 7:9}
\crossref{Acts}{2}{38}{Ac 3:19; 17:30; 20:21; 26:20 Mt 3:2,\allowbreak8,\allowbreak9; 4:17; 21:28-\allowbreak32 Lu 15:1-\allowbreak32}
\crossref{Acts}{2}{39}{Ac 3:25,\allowbreak26 Ge 17:7,\allowbreak8 Ps 115:14,\allowbreak15 Jer 32:39,\allowbreak40 Eze 37:25 Joe 2:28}
\crossref{Acts}{2}{40}{Ac 15:32; 20:2,\allowbreak9,\allowbreak11; 28:23 Joh 21:25}
\crossref{Acts}{2}{41}{2:37; 8:6-\allowbreak8; 13:48; 16:31-\allowbreak34 Mt 13:44-\allowbreak46 Ga 4:14,\allowbreak15 1Th 1:6}
\crossref{Acts}{2}{42}{2:46; 11:23; 14:22 Mr 4:16,\allowbreak17 Joh 8:31,\allowbreak32 1Co 11:2 Ga 1:6 Eph 2:20}
\crossref{Acts}{2}{43}{Ac 5:11,\allowbreak13 Es 8:17 Jer 33:9 Ho 3:5 Lu 7:16; 8:37}
\crossref{Acts}{2}{44}{Ac 4:32; 5:4; 6:1-\allowbreak3 2Co 8:9,\allowbreak14,\allowbreak15; 9:6-\allowbreak15 1Jo 3:16-\allowbreak18}
\crossref{Acts}{2}{45}{Ac 4:34-\allowbreak37; 5:1,\allowbreak2; 11:29 Lu 12:33,\allowbreak34; 16:9; 18:22; 19:8}
\crossref{Acts}{2}{46}{Ac 1:14; 3:1; 5:42 Lu 24:53}
\crossref{Acts}{2}{47}{Ac 4:21,\allowbreak33 Lu 2:52; 19:48 Ro 14:18}
\crossref{Acts}{3}{1}{Ac 4:13; 8:14 Mt 17:1; 26:37 Joh 13:23-\allowbreak25; 20:2-\allowbreak9; 21:7,\allowbreak18-\allowbreak22 Ga 2:9}
\crossref{Acts}{3}{2}{Ac 4:22; 14:8 Joh 1:9-\allowbreak30}
\crossref{Acts}{3}{3}{}
\crossref{Acts}{3}{4}{Ac 11:6; 14:9,\allowbreak10 Lu 4:20}
\crossref{Acts}{3}{5}{}
\crossref{Acts}{3}{6}{Mt 10:9 1Co 4:11 2Co 6:10; 8:9 Jas 2:5}
\crossref{Acts}{3}{7}{Ac 9:41 Mr 1:31; 5:41; 9:27 Lu 13:13}
\crossref{Acts}{3}{8}{Ac 14:10 Isa 35:6 Lu 6:23 Joh 5:8,\allowbreak9,\allowbreak14}
\crossref{Acts}{3}{9}{Ac 14:11 Mr 2:11,\allowbreak12 Lu 13:17}
\crossref{Acts}{3}{10}{3:2; 4:14-\allowbreak16,\allowbreak21,\allowbreak22 Joh 9:3,\allowbreak18-\allowbreak21}
\crossref{Acts}{3}{11}{Lu 8:38}
\crossref{Acts}{3}{12}{Ac 2:22; 13:26 Ro 9:4; 11:1}
\crossref{Acts}{3}{13}{Ac 5:30; 7:32 Ex 3:6 Ps 105:6-\allowbreak10 Mt 22:32 Heb 11:9-\allowbreak16}
\crossref{Acts}{3}{14}{Ac 2:27; 4:27; 7:52; 22:14 Ps 16:10 Zec 9:9 Mr 1:24 Lu 1:35 Jas 5:6}
\crossref{Acts}{3}{15}{Joh 1:4; 4:10,\allowbreak14; 5:26; 10:28; 11:25,\allowbreak26; 14:6; 17:2 Ro 8:1,\allowbreak2}
\crossref{Acts}{3}{16}{3:6; 4:7,\allowbreak10,\allowbreak30; 16:18 Mt 9:22}
\crossref{Acts}{3}{17}{Ac 7:40 Ge 21:26; 39:8; 44:15 Ex 32:1 Nu 22:6 Ro 11:2 Php 1:22}
\crossref{Acts}{3}{18}{Ac 17:2,\allowbreak3; 26:22,\allowbreak23; 28:23 Lu 24:26,\allowbreak27,\allowbreak44 1Co 15:3,\allowbreak4 1Pe 1:10,\allowbreak11}
\crossref{Acts}{3}{19}{Ac 2:38; 11:18 2Ti 2:25}
\crossref{Acts}{3}{20}{Ac 17:31 Mt 16:27; 24:3,\allowbreak30-\allowbreak36 Mr 13:26,\allowbreak30-\allowbreak37 Lu 19:11; 21:27}
\crossref{Acts}{3}{21}{Ac 1:11}
\crossref{Acts}{3}{22}{Ac 7:37 De 18:15-\allowbreak19}
\crossref{Acts}{3}{23}{Ac 13:38-\allowbreak41 De 18:19 Mr 16:16 Joh 3:18-\allowbreak20; 8:24; 12:48 2Th 1:7-\allowbreak9}
\crossref{Acts}{3}{24}{3:19,\allowbreak21 Ro 3:21}
\crossref{Acts}{3}{25}{Ac 2:39; 13:26 Ge 20:7; 27:36-\allowbreak40; 48:14-\allowbreak20; 49:1-\allowbreak33 Ps 105:8-\allowbreak15}
\crossref{Acts}{3}{26}{Ac 1:8; 13:26,\allowbreak32,\allowbreak33,\allowbreak46,\allowbreak47; 18:4-\allowbreak6; 26:20; 28:23-\allowbreak28 Mt 10:5,\allowbreak6 Lu 24:47}
\crossref{Acts}{4}{1}{4:6; 6:7,\allowbreak12 Mt 26:3,\allowbreak4; 27:1,\allowbreak2,\allowbreak20,\allowbreak41 Joh 15:20; 18:3}
\crossref{Acts}{4}{2}{Ac 5:17; 13:45; 19:23 Ne 2:10 Joh 11:47,\allowbreak48}
\crossref{Acts}{4}{3}{Ac 5:18; 6:12; 8:3; 9:2; 12:1-\allowbreak3; 16:19-\allowbreak24 Mt 10:16,\allowbreak17 Lu 22:52,\allowbreak54}
\crossref{Acts}{4}{4}{Ac 28:24 2Co 2:14-\allowbreak17 Php 1:12-\allowbreak18 2Ti 2:9,\allowbreak10}
\crossref{Acts}{4}{5}{Ac 5:20,\allowbreak21 Mic 2:1 Mt 27:1,\allowbreak2}
\crossref{Acts}{4}{6}{Lu 3:2 Joh 11:49; 18:13,\allowbreak14,\allowbreak24}
\crossref{Acts}{4}{7}{Ac 5:27 1Ki 21:12-\allowbreak14 Joh 8:3,\allowbreak9}
\crossref{Acts}{4}{8}{4:31; 2:4; 7:55 Mt 10:19,\allowbreak20 Lu 12:11,\allowbreak12; 21:14,\allowbreak15}
\crossref{Acts}{4}{9}{Ac 3:7 Joh 7:23; 10:32 1Pe 3:15-\allowbreak17; 4:14}
\crossref{Acts}{4}{10}{Ac 13:38; 28:28 Jer 42:19,\allowbreak20 Da 3:18}
\crossref{Acts}{4}{11}{Ps 118:22,\allowbreak23 Isa 28:16 Mt 21:42-\allowbreak45 Mr 12:10-\allowbreak12 Lu 20:16-\allowbreak18}
\crossref{Acts}{4}{12}{Ac 10:42,\allowbreak43 Mt 1:21 Mr 16:15,\allowbreak16 Joh 3:36; 14:6 1Co 3:11 1Ti 2:5,\allowbreak6}
\crossref{Acts}{4}{13}{Ac 2:7-\allowbreak12 Mt 4:18-\allowbreak22; 11:25 Joh 7:15,\allowbreak49 1Co 1:27}
\crossref{Acts}{4}{14}{4:10; 3:8-\allowbreak12}
\crossref{Acts}{4}{15}{Ac 5:34 etc.}
\crossref{Acts}{4}{16}{Joh 11:47,\allowbreak48; 12:18}
\crossref{Acts}{4}{17}{Ac 5:39 Ps 2:1-\allowbreak4 Da 2:34,\allowbreak35 Ro 10:16-\allowbreak18; 15:18-\allowbreak22 1Th 1:8}
\crossref{Acts}{4}{18}{Ac 5:40}
\crossref{Acts}{4}{19}{2Co 4:2 Eph 6:1 1Ti 2:3}
\crossref{Acts}{4}{20}{Ac 2:4,\allowbreak32; 17:16,\allowbreak17; 18:5 Nu 22:38; 23:20 2Sa 23:2 Job 32:18-\allowbreak20}
\crossref{Acts}{4}{21}{4:17; 5:40}
\crossref{Acts}{4}{22}{Ac 3:2; 9:33 Mt 9:20 Lu 13:11 Joh 5:5; 9:1}
\crossref{Acts}{4}{23}{Ac 1:13,\allowbreak14; 2:44-\allowbreak46; 12:11,\allowbreak12; 16:40 Ps 16:3; 42:4; 119:63 Pr 13:20}
\crossref{Acts}{4}{24}{Ac 16:25 Ps 55:16-\allowbreak18; 62:5-\allowbreak8; 69:29,\allowbreak30; 109:29-\allowbreak31 Jer 20:13}
\crossref{Acts}{4}{25}{Ac 1:16; 2:30}
\crossref{Acts}{4}{26}{Ps 83:2-\allowbreak8 Joe 3:9-\allowbreak14 Re 17:12-\allowbreak14,\allowbreak17; 19:16-\allowbreak21}
\crossref{Acts}{4}{27}{Mt 26:3 Lu 22:1; 23:1,\allowbreak8 etc.}
\crossref{Acts}{4}{28}{Ac 2:23; 3:18; 13:27-\allowbreak29 Ge 50:20 Ps 76:10 Mt 26:24,\allowbreak54 Lu 22:22}
\crossref{Acts}{4}{29}{4:17,\allowbreak18,\allowbreak21 Isa 37:17-\allowbreak20; 63:15 La 3:50; 5:1 Da 9:18}
\crossref{Acts}{4}{30}{Ex 6:6 De 4:34 Jer 15:15; 20:11,\allowbreak12 Lu 9:54-\allowbreak56; 22:49-\allowbreak51}
\crossref{Acts}{4}{31}{Ac 2:2; 16:25,\allowbreak26}
\crossref{Acts}{4}{32}{Ac 1:14; 2:1; 5:12 2Ch 30:12 Jer 32:39 Eze 11:19,\allowbreak20 Joh 17:11,\allowbreak21-\allowbreak23}
\crossref{Acts}{4}{33}{4:30; 1:8,\allowbreak22; 2:32,\allowbreak33; 3:15,\allowbreak16; 5:12-\allowbreak16 Mr 16:20 Lu 24:48,\allowbreak49}
\crossref{Acts}{4}{34}{De 2:7 Ps 34:9,\allowbreak10 Lu 22:35 1Th 4:12}
\crossref{Acts}{4}{35}{Ac 3:6; 5:2; 6:1-\allowbreak6 2Co 8:20,\allowbreak21}
\crossref{Acts}{4}{36}{Ac 11:22-\allowbreak25,\allowbreak30; 12:25; 13:1; 15:2,\allowbreak12,\allowbreak37 1Co 9:6 Ga 2:1,\allowbreak9,\allowbreak13}
\crossref{Acts}{4}{37}{4:34,\allowbreak35; 5:1,\allowbreak2 Mt 19:29}
\crossref{Acts}{5}{1}{Le 10:1-\allowbreak3 Jos 6:1 Mt 13:47,\allowbreak48 Joh 6:37 2Ti 2:20}
\crossref{Acts}{5}{2}{Jos 7:11,\allowbreak12 2Ki 5:21-\allowbreak25 Mal 1:14; 3:8,\allowbreak9 Joh 12:6 1Ti 6:10}
\crossref{Acts}{5}{3}{Ge 3:13-\allowbreak17 1Ki 22:21,\allowbreak22 1Ch 21:1-\allowbreak3 Mt 4:3-\allowbreak11; 13:19 Lu 22:3}
\crossref{Acts}{5}{4}{Ex 35:21,\allowbreak22,\allowbreak29 1Ch 29:3,\allowbreak5,\allowbreak9,\allowbreak17 1Co 8:8; 9:5-\allowbreak17 Phm 1:14}
\crossref{Acts}{5}{5}{5:10,\allowbreak11; 13:11 Nu 16:26-\allowbreak33 2Ki 1:10-\allowbreak14; 2:24 Jer 5:14 1Co 4:21}
\crossref{Acts}{5}{6}{Le 10:4-\allowbreak6 De 21:23 2Sa 18:17 Joh 19:40}
\crossref{Acts}{5}{7}{5:7}
\crossref{Acts}{5}{8}{}
\crossref{Acts}{5}{9}{Ge 3:9-\allowbreak13 Lu 16:2 Ro 3:19}
\crossref{Acts}{5}{10}{5:5}
\crossref{Acts}{5}{11}{5:5; 19:17 Ps 89:7 Jer 32:40 1Co 10:11,\allowbreak12 Php 2:12 Heb 4:1; 11:7}
\crossref{Acts}{5}{12}{Ac 2:43; 3:6,\allowbreak7; 4:30,\allowbreak33; 9:33,\allowbreak40; 14:3,\allowbreak8-\allowbreak10; 16:18; 19:11 Mr 16:17,\allowbreak18}
\crossref{Acts}{5}{13}{5:5 Nu 17:12,\allowbreak13; 24:8-\allowbreak10 1Sa 16:4,\allowbreak5 1Ki 17:18 Isa 33:14 Lu 12:1,\allowbreak2}
\crossref{Acts}{5}{14}{Ac 2:41,\allowbreak47; 4:4; 6:7; 9:31,\allowbreak35,\allowbreak42 Isa 44:3-\allowbreak5; 45:24; 55:11-\allowbreak13}
\crossref{Acts}{5}{15}{Ac 19:11,\allowbreak12 Mt 9:21; 14:36 Joh 14:12}
\crossref{Acts}{5}{16}{Mt 4:24; 8:16; 15:30,\allowbreak31 Mr 2:3,\allowbreak4; 6:54-\allowbreak56 Joh 14:12}
\crossref{Acts}{5}{17}{Ac 4:26 Ps 2:1-\allowbreak3 Joh 11:47-\allowbreak49; 12:10,\allowbreak19}
\crossref{Acts}{5}{18}{Ac 4:3; 8:3; 12:5-\allowbreak7; 16:23-\allowbreak27 Lu 21:12 2Co 11:23 Heb 11:36 Re 2:10}
\crossref{Acts}{5}{19}{Ac 12:7-\allowbreak11; 16:26 Ps 34:7; 105:17-\allowbreak20; 146:7 Isa 61:1}
\crossref{Acts}{5}{20}{Isa 58:1 Jer 7:2; 19:14,\allowbreak15; 20:2,\allowbreak3; 22:1,\allowbreak2; 26:2; 36:10 Mt 21:23}
\crossref{Acts}{5}{21}{5:25 Lu 21:37,\allowbreak38 Joh 8:2}
\crossref{Acts}{5}{22}{}
\crossref{Acts}{5}{23}{5:19 Ps 2:4; 33:10 Pr 21:30 La 3:37,\allowbreak55-\allowbreak58 Da 3:11-\allowbreak25; 6:22-\allowbreak24}
\crossref{Acts}{5}{24}{5:26; 4:1 Lu 22:4,\allowbreak52}
\crossref{Acts}{5}{25}{5:18-\allowbreak21}
\crossref{Acts}{5}{26}{5:13 Mt 14:5; 21:26; 26:5 Lu 20:6,\allowbreak19; 22:2}
\crossref{Acts}{5}{27}{Ac 4:7; 6:12; 22:30; 23:1 Lu 22:66}
\crossref{Acts}{5}{28}{5:40; 4:18-\allowbreak21}
\crossref{Acts}{5}{29}{Ac 4:19 Ge 3:17 1Sa 15:24 Mr 7:7-\allowbreak9 Re 14:8-\allowbreak12}
\crossref{Acts}{5}{30}{Ac 3:13-\allowbreak15; 22:14 1Ch 12:17; 29:18 Ezr 7:27 Lu 1:55,\allowbreak72}
\crossref{Acts}{5}{31}{Ac 2:33,\allowbreak36; 4:11 Ps 89:19,\allowbreak24; 110:1,\allowbreak2 Eze 17:24 Mt 28:18}
\crossref{Acts}{5}{32}{5:29; 1:8; 2:32; 10:39-\allowbreak41; 13:31 Lu 24:47,\allowbreak48 Joh 15:27 2Co 13:1 Heb 2:3}
\crossref{Acts}{5}{33}{Ac 2:37; 7:54; 22:22 Lu 4:28,\allowbreak29; 6:11; 11:50-\allowbreak54; 19:45-\allowbreak48; 20:19}
\crossref{Acts}{5}{34}{Ac 23:7-\allowbreak9 Ps 76:10 Joh 7:50-\allowbreak53}
\crossref{Acts}{5}{35}{Ac 19:36; 22:26 Jer 26:19 Mt 27:19}
\crossref{Acts}{5}{36}{Ac 8:9 Mt 24:24 2Th 2:3-\allowbreak7 2Pe 2:18 Jude 1:16 Re 17:3,\allowbreak5}
\crossref{Acts}{5}{37}{Lu 2:1; 13:1}
\crossref{Acts}{5}{38}{5:35 Joh 11:48}
\crossref{Acts}{5}{39}{Ac 6:10 Ge 24:50 2Sa 5:2 1Ki 12:24 Job 34:29 Isa 43:13; 46:10}
\crossref{Acts}{5}{40}{Ac 4:18}
\crossref{Acts}{5}{41}{Ac 16:23-\allowbreak25 Isa 61:10; 65:14; 66:5 Mt 5:10-\allowbreak12 Lu 6:22 Ro 5:3}
\crossref{Acts}{5}{42}{5:20,\allowbreak21; 2:46; 3:1,\allowbreak2 etc.}
\crossref{Acts}{6}{1}{6:7; 2:41,\allowbreak47; 4:4; 5:14,\allowbreak28 Ps 72:16; 110:3 Isa 27:6 Jer 30:19}
\crossref{Acts}{6}{2}{Ac 21:22}
\crossref{Acts}{6}{3}{Ac 9:30; 15:23 Mt 23:8 1Jo 3:14-\allowbreak16}
\crossref{Acts}{6}{4}{Ac 2:42; 20:19-\allowbreak31 Ro 12:6-\allowbreak8 1Co 9:16 Col 4:17 1Ti 4:13-\allowbreak16 2Ti 4:2}
\crossref{Acts}{6}{5}{Ac 15:22 Ge 41:37 Pr 15:1,\allowbreak23; 25:11,\allowbreak12}
\crossref{Acts}{6}{6}{Ac 1:24; 8:17; 9:17; 13:3 1Ti 4:14; 5:22 2Ti 1:6}
\crossref{Acts}{6}{7}{Ac 12:24; 19:20 Col 1:6 2Ti 2:9}
\crossref{Acts}{6}{8}{6:3,\allowbreak5,\allowbreak10,\allowbreak15; 7:55 Eph 4:11 1Ti 3:13}
\crossref{Acts}{6}{9}{Ac 13:45; 17:17,\allowbreak18}
\crossref{Acts}{6}{10}{Ac 5:39; 7:51 Ex 4:12 Isa 54:17 Jer 1:18,\allowbreak19; 15:20 Eze 3:27}
\crossref{Acts}{6}{11}{Ac 23:12-\allowbreak15; 24:1-\allowbreak13; 25:3,\allowbreak7 1Ki 21:10,\allowbreak13 Mt 26:59,\allowbreak60; 28:12-\allowbreak15}
\crossref{Acts}{6}{12}{Ac 13:50; 14:2; 17:5,\allowbreak13; 21:27 Pr 15:18}
\crossref{Acts}{6}{13}{6:11 Ps 27:12; 35:11; 56:5}
\crossref{Acts}{6}{14}{Ac 25:8}
\crossref{Acts}{6}{15}{Ex 34:29-\allowbreak35 Ec 8:1 Mt 13:43; 17:2 2Co 3:7,\allowbreak8,\allowbreak18}
\crossref{Acts}{7}{1}{Ac 6:13,\allowbreak14 Mt 26:61,\allowbreak62 Mr 14:58-\allowbreak60 Joh 18:19-\allowbreak21,\allowbreak33-\allowbreak35}
\crossref{Acts}{7}{2}{Ac 22:1; 23:7}
\crossref{Acts}{7}{3}{Ge 12:1 Mt 10:37 Lu 14:33 2Co 6:17 Heb 11:8}
\crossref{Acts}{7}{4}{Ge 11:31,\allowbreak32; 12:4,\allowbreak5 Isa 41:2,\allowbreak9}
\crossref{Acts}{7}{5}{Ge 23:4 Ps 105:11,\allowbreak12 Heb 11:9,\allowbreak10,\allowbreak13-\allowbreak16}
\crossref{Acts}{7}{6}{Ge 15:13,\allowbreak16}
\crossref{Acts}{7}{7}{Ge 15:14-\allowbreak16 Ex 7:1-\allowbreak14:31 Ne 9:9-\allowbreak11 Ps 74:12-\allowbreak14; 78:43-\allowbreak51}
\crossref{Acts}{7}{8}{Ge 17:9-\allowbreak14 Joh 7:22 Ro 4:10 Ga 3:15,\allowbreak17}
\crossref{Acts}{7}{9}{Ge 37:4-\allowbreak11; 49:23 Mt 27:18}
\crossref{Acts}{7}{10}{Ge 48:16 Ps 22:24; 34:17-\allowbreak19; 37:40; 40:1-\allowbreak3 2Ti 4:18 Jas 5:11}
\crossref{Acts}{7}{11}{Ge 41:54-\allowbreak57; 43:1; 45:5,\allowbreak6,\allowbreak11; 47:13-\allowbreak15 Ps 105:16}
\crossref{Acts}{7}{12}{Ge 42:1 etc.}
\crossref{Acts}{7}{13}{Ge 45:1-\allowbreak18; 46:31-\allowbreak34; 47:1-\allowbreak10}
\crossref{Acts}{7}{14}{Ge 45:9-\allowbreak11 Ps 105:23}
\crossref{Acts}{7}{15}{Ge 46:3-\allowbreak7 Nu 20:15 De 10:22; 26:5 Jos 24:4}
\crossref{Acts}{7}{16}{Ge 33:9-\allowbreak20; 35:19; 49:29-\allowbreak32}
\crossref{Acts}{7}{17}{7:6 Ge 15:13-\allowbreak16 2Pe 3:8,\allowbreak9}
\crossref{Acts}{7}{18}{Ex 1:8}
\crossref{Acts}{7}{19}{Ex 1:9-\allowbreak22 Ps 83:4,\allowbreak5; 105:25; 129:1-\allowbreak3 Re 12:4,\allowbreak5}
\crossref{Acts}{7}{20}{Ex 2:2-\allowbreak10}
\crossref{Acts}{7}{21}{Ex 2:2-\allowbreak10 De 32:26}
\crossref{Acts}{7}{22}{1Ki 4:29 2Ch 9:22 Isa 19:11 Da 1:4,\allowbreak17-\allowbreak20}
\crossref{Acts}{7}{23}{Ex 2:11,\allowbreak12 Heb 11:24-\allowbreak26}
\crossref{Acts}{7}{24}{7:28 Joh 18:10,\allowbreak11,\allowbreak25-\allowbreak27}
\crossref{Acts}{7}{25}{Ac 14:27; 15:4,\allowbreak7; 21:19 1Sa 14:45; 19:5 2Ki 5:1 Ro 15:18 1Co 3:9}
\crossref{Acts}{7}{26}{Ex 2:13-\allowbreak15}
\crossref{Acts}{7}{27}{7:54; 5:33 Ge 19:19 1Sa 25:14,\allowbreak15 Pr 9:7,\allowbreak8}
\crossref{Acts}{7}{28}{7:28}
\crossref{Acts}{7}{29}{Ex 2:14-\allowbreak22; 4:19,\allowbreak20}
\crossref{Acts}{7}{30}{7:17 Ex 7:7}
\crossref{Acts}{7}{31}{Ex 3:3,\allowbreak4}
\crossref{Acts}{7}{32}{Ac 3:13 Ge 50:24 Ex 3:6,\allowbreak15; 4:5 Mt 22:32 Heb 11:16}
\crossref{Acts}{7}{33}{Ex 3:5 Jos 5:15 Ec 5:1 2Pe 1:18}
\crossref{Acts}{7}{34}{Ex 2:23-\allowbreak25; 3:7,\allowbreak9; 4:31; 6:5,\allowbreak6 Jud 2:18; 10:15,\allowbreak16 Ne 9:9 Ps 106:44}
\crossref{Acts}{7}{35}{7:9-\allowbreak15,\allowbreak27,\allowbreak28,\allowbreak51 1Sa 8:7,\allowbreak8; 10:27 Lu 19:14 Joh 18:40; 19:15}
\crossref{Acts}{7}{36}{Ex 12:41; 33:1}
\crossref{Acts}{7}{37}{7:38 2Ch 28:22 Da 6:13}
\crossref{Acts}{7}{38}{Ex 19:3-\allowbreak17; 20:19,\allowbreak20 Nu 16:3 etc.}
\crossref{Acts}{7}{39}{7:51,\allowbreak52 Ne 9:16 Ps 106:16,\allowbreak32,\allowbreak33 Eze 20:6-\allowbreak14}
\crossref{Acts}{7}{40}{Ex 32:1}
\crossref{Acts}{7}{41}{Ex 32:2-\allowbreak8,\allowbreak17-\allowbreak20 De 9:12-\allowbreak18 Ne 9:18 Ps 106:19-\allowbreak21}
\crossref{Acts}{7}{42}{Ps 81:11,\allowbreak12 Isa 66:4 Eze 14:7-\allowbreak10; 20:25,\allowbreak39 Ho 4:17 Ro 1:24-\allowbreak28}
\crossref{Acts}{7}{43}{Le 18:21; 20:2-\allowbreak5 2Ki 17:16-\allowbreak18; 21:6}
\crossref{Acts}{7}{44}{Ex 38:21 Nu 1:50-\allowbreak53; 9:15; 10:11; 17:7,\allowbreak8; 18:2 Jos 18:1 2Ch 24:6}
\crossref{Acts}{7}{45}{Jos 3:11-\allowbreak14; 18:1 Jud 18:31 1Sa 4:4 1Ki 8:4 1Ch 16:39; 21:29}
\crossref{Acts}{7}{46}{Ac 13:22 1Sa 15:28; 16:1,\allowbreak11-\allowbreak13 2Sa 6:21; 7:1,\allowbreak8,\allowbreak18,\allowbreak19 1Ch 28:4,\allowbreak5}
\crossref{Acts}{7}{47}{2Sa 7:13 1Ki 5:1-\allowbreak18; 6:1,\allowbreak37,\allowbreak38; 7:13-\allowbreak51; 8:20 1Ch 17:1}
\crossref{Acts}{7}{48}{De 32:8 Ps 7:17; 46:4; 91:1,\allowbreak9; 92:8 Da 4:17,\allowbreak24,\allowbreak25,\allowbreak34 Ho 7:16}
\crossref{Acts}{7}{49}{1Ki 22:19 Ps 11:4 Jer 23:24 Mt 5:34,\allowbreak35; 23:22 Re 3:21}
\crossref{Acts}{7}{50}{Ac 14:15 Ex 20:11 Ps 33:6-\allowbreak9; 50:9-\allowbreak12; 146:5,\allowbreak6 Isa 40:28; 44:24}
\crossref{Acts}{7}{51}{Ex 32:9; 33:3,\allowbreak5; 34:9 De 9:6,\allowbreak13; 31:27 2Ch 30:8 Ne 9:16 Ps 75:5}
\crossref{Acts}{7}{52}{1Sa 8:7,\allowbreak8 1Ki 19:10,\allowbreak14 2Ch 24:19-\allowbreak22; 36:16 Ne 9:26 Jer 2:30}
\crossref{Acts}{7}{53}{Ex 19:1-\allowbreak20:26 De 33:2 Ps 68:17 Ga 3:19 Heb 2:2}
\crossref{Acts}{7}{54}{Ac 5:33; 22:22,\allowbreak23}
\crossref{Acts}{7}{55}{Ac 2:4; 4:8; 6:3,\allowbreak5,\allowbreak8,\allowbreak10; 13:9,\allowbreak10 Mic 3:8}
\crossref{Acts}{7}{56}{Ac 10:11,\allowbreak16 Eze 1:1 Mt 3:16 Mr 1:10 Lu 3:21 Re 4:1; 11:19; 19:11}
\crossref{Acts}{7}{57}{7:54; 21:27-\allowbreak31; 23:27}
\crossref{Acts}{7}{58}{Nu 15:35 1Ki 21:13 Lu 4:29 Heb 13:12,\allowbreak13}
\crossref{Acts}{7}{59}{Ac 2:21; 9:14,\allowbreak21; 22:16 Joe 2:32 Ro 10:12-\allowbreak14 1Co 1:2}
\crossref{Acts}{7}{60}{Ac 9:40; 20:36; 21:5 Ezr 9:5 Da 6:10 Lu 22:41}
\crossref{Acts}{8}{1}{Ac 5:33,\allowbreak40; 7:54 Mt 10:25-\allowbreak28; 22:6; 23:34 Lu 11:49,\allowbreak50 Joh 15:20; 16:2}
\crossref{Acts}{8}{2}{Ac 2:5; 10:2 Lu 2:25}
\crossref{Acts}{8}{3}{Ac 7:58; 9:1-\allowbreak13,\allowbreak21; 22:3,\allowbreak4; 26:9-\allowbreak11 1Co 15:9 Ga 1:13 Php 3:6}
\crossref{Acts}{8}{4}{Ac 11:19; 14:2-\allowbreak7 Mt 10:23 1Th 2:2}
\crossref{Acts}{8}{5}{8:1,\allowbreak14,\allowbreak15,\allowbreak40; 6:5; 21:8}
\crossref{Acts}{8}{6}{Ac 13:44 2Ch 30:12 Mt 20:15,\allowbreak16 Joh 4:41,\allowbreak42}
\crossref{Acts}{8}{7}{Ac 5:16 Mt 10:1 Mr 9:26; 16:17,\allowbreak18 Lu 10:17 Joh 14:12 Heb 2:4}
\crossref{Acts}{8}{8}{Ac 13:48,\allowbreak52 Ps 96:10-\allowbreak12; 98:2-\allowbreak6 Isa 35:1,\allowbreak2; 42:10-\allowbreak12 Lu 2:10,\allowbreak11}
\crossref{Acts}{8}{9}{Ac 13:6; 16:16-\allowbreak18; 19:18-\allowbreak20 Ex 7:11,\allowbreak22; 8:18,\allowbreak19; 9:11 Le 20:6}
\crossref{Acts}{8}{10}{2Co 11:19 Eph 4:14 2Pe 2:2 Re 13:3}
\crossref{Acts}{8}{11}{Isa 8:19; 44:25; 47:9-\allowbreak13 Ga 3:1}
\crossref{Acts}{8}{12}{8:35-\allowbreak38; 2:38,\allowbreak41; 16:14,\allowbreak15,\allowbreak31-\allowbreak34 Mt 28:19 Mr 16:15 Ro 10:10 1Pe 3:21}
\crossref{Acts}{8}{13}{8:21 Ps 78:35-\allowbreak37; 106:12,\allowbreak13 Lu 8:13 Joh 2:23-\allowbreak25; 8:30,\allowbreak31}
\crossref{Acts}{8}{14}{8:1; 11:1,\allowbreak19-\allowbreak22; 15:4 1Th 3:2}
\crossref{Acts}{8}{15}{Ac 2:38 Mt 18:19 Joh 14:13,\allowbreak14; 16:23,\allowbreak24 Php 1:19}
\crossref{Acts}{8}{16}{Ac 10:44-\allowbreak46; 11:15-\allowbreak17; 19:2}
\crossref{Acts}{8}{17}{8:18; 6:6; 9:17; 13:3; 19:6 Nu 8:10; 27:18 1Ti 4:14; 5:22 2Ti 1:6 Heb 6:2}
\crossref{Acts}{8}{18}{2Ki 5:15,\allowbreak16; 8:9 Eze 13:19 Mt 10:8 1Ti 6:5}
\crossref{Acts}{8}{19}{8:9-\allowbreak11,\allowbreak17 Mt 18:1-\allowbreak3 Lu 14:7-\allowbreak11 Joh 5:44 1Co 15:8,\allowbreak9 3Jo 1:9}
\crossref{Acts}{8}{20}{Ac 1:18 De 7:26 Jos 7:24,\allowbreak25 2Ki 5:26,\allowbreak27 Da 5:17 Hab 2:9,\allowbreak10}
\crossref{Acts}{8}{21}{Jos 22:25 Eze 14:3 Re 20:6; 22:19}
\crossref{Acts}{8}{22}{Ac 2:38; 3:19; 17:30 Ro 2:4 2Ti 2:25,\allowbreak26 Re 2:21}
\crossref{Acts}{8}{23}{De 29:18-\allowbreak20; 32:32,\allowbreak33 Job 20:14 Jer 4:18; 9:15 La 3:5,\allowbreak19}
\crossref{Acts}{8}{24}{Ge 20:7,\allowbreak17 Ex 8:8; 10:17; 12:32 Nu 21:7 1Sa 12:19,\allowbreak23 1Ki 13:6}
\crossref{Acts}{8}{25}{Ac 1:8; 18:5; 20:21; 26:22,\allowbreak23; 28:23,\allowbreak28,\allowbreak31 Joh 15:27 1Pe 5:12}
\crossref{Acts}{8}{26}{Ac 5:19; 10:7,\allowbreak22; 12:8-\allowbreak11,\allowbreak23; 27:23 2Ki 1:3 Heb 1:14}
\crossref{Acts}{8}{27}{Mt 21:2-\allowbreak6 Mr 14:13-\allowbreak16 Joh 2:5-\allowbreak8 Heb 11:8}
\crossref{Acts}{8}{28}{Ac 17:11,\allowbreak12 De 6:6,\allowbreak7; 11:18-\allowbreak20; 17:18,\allowbreak19 Jos 1:8 Ps 1:2,\allowbreak3}
\crossref{Acts}{8}{29}{Ac 10:19; 11:12; 13:2-\allowbreak4; 16:6,\allowbreak7; 20:22,\allowbreak23 Isa 65:24 Ho 6:3 1Co 12:11}
\crossref{Acts}{8}{30}{8:27 Ps 119:32 Ec 9:10 Joh 4:34}
\crossref{Acts}{8}{31}{Ps 25:8,\allowbreak9; 73:16,\allowbreak17,\allowbreak22 Pr 30:2,\allowbreak3 Isa 29:18,\allowbreak19; 35:8 Mt 18:3,\allowbreak4}
\crossref{Acts}{8}{32}{Isa 53:7,\allowbreak8}
\crossref{Acts}{8}{33}{Php 2:8,\allowbreak9}
\crossref{Acts}{8}{34}{Mt 2:2-\allowbreak4; 13:36; 15:15}
\crossref{Acts}{8}{35}{Ac 10:34 Mt 5:2 2Co 6:11}
\crossref{Acts}{8}{36}{Ac 10:47 Eze 36:25 Joh 3:5,\allowbreak23 Tit 3:5,\allowbreak6 1Jo 5:6}
% skipped verse
%\crossref{Acts}{8}{37}{8:12,\allowbreak13,\allowbreak21; 2:38,\allowbreak39 Mt 28:19 Mr 16:16 Ro 10:10}
\crossref{Acts}{8}{38}{Joh 3:22,\allowbreak23; 4:1}
\crossref{Acts}{8}{39}{Mt 3:16 Mr 1:10}
\crossref{Acts}{8}{40}{Jos 15:46,\allowbreak47 1Sa 5:1 Zec 9:6}
\crossref{Acts}{9}{1}{9:11-\allowbreak13,\allowbreak19-\allowbreak21; 7:58; 8:3; 22:3,\allowbreak4; 26:9-\allowbreak11 1Co 15:9 Ga 1:13 Php 3:6}
\crossref{Acts}{9}{2}{9:14; 7:19; 22:5; 26:12 Es 3:8-\allowbreak13 Ps 82:2-\allowbreak4}
\crossref{Acts}{9}{3}{9:17; 22:6; 26:12,\allowbreak13 1Co 15:8}
\crossref{Acts}{9}{4}{Ac 5:10 Nu 16:45 Joh 18:6 Ro 11:22 1Co 4:7}
\crossref{Acts}{9}{5}{1Sa 3:4-\allowbreak10 1Ti 1:13}
\crossref{Acts}{9}{6}{Ac 16:29; 24:25,\allowbreak26 1Sa 28:5 Isa 66:2 Hab 3:16 Php 2:12}
\crossref{Acts}{9}{7}{Ac 22:9; 26:13,\allowbreak14 Da 10:7 Mt 24:40,\allowbreak41}
\crossref{Acts}{9}{8}{9:18; 13:11; 22:11 Ge 19:11 Ex 4:11 2Ki 6:17-\allowbreak20}
\crossref{Acts}{9}{9}{9:11,\allowbreak12 2Ch 33:12,\allowbreak13,\allowbreak18,\allowbreak19 Es 4:16 Jon 3:6-\allowbreak8}
\crossref{Acts}{9}{10}{Ac 22:12}
\crossref{Acts}{9}{11}{Ac 8:26; 10:5,\allowbreak6; 11:13}
\crossref{Acts}{9}{12}{9:10,\allowbreak17,\allowbreak18}
\crossref{Acts}{9}{13}{Ex 4:13-\allowbreak19 1Sa 16:2 1Ki 18:9-\allowbreak14 Jer 20:9,\allowbreak10 Eze 3:14 Jon 1:2,\allowbreak3}
\crossref{Acts}{9}{14}{9:2,\allowbreak3}
\crossref{Acts}{9}{15}{Ex 4:12-\allowbreak14 Jer 1:7 Jon 3:1,\allowbreak2}
\crossref{Acts}{9}{16}{Ac 20:22,\allowbreak23; 21:11 Isa 33:1 Mt 10:21-\allowbreak25 Joh 15:20; 16:1-\allowbreak4}
\crossref{Acts}{9}{17}{Ac 22:12,\allowbreak13}
\crossref{Acts}{9}{18}{2Co 3:14; 4:6}
\crossref{Acts}{9}{19}{Ac 27:33-\allowbreak36 1Sa 30:12 Ec 9:7}
\crossref{Acts}{9}{20}{9:27,\allowbreak28 Ga 1:23,\allowbreak24}
\crossref{Acts}{9}{21}{Ac 2:6,\allowbreak12; 4:13 Nu 23:23 Ps 71:7 Isa 8:18 Zec 3:8 2Th 1:10 1Jo 3:1}
\crossref{Acts}{9}{22}{Ge 49:24 Job 17:9 Ps 84:7 Isa 40:29 2Co 12:9,\allowbreak10 Php 4:13}
\crossref{Acts}{9}{23}{9:16; 13:50; 14:2,\allowbreak19; 22:21-\allowbreak23 Jos 10:1-\allowbreak6 Mt 10:16-\allowbreak23 2Co 11:26}
\crossref{Acts}{9}{24}{9:29,\allowbreak30; 14:5,\allowbreak6; 17:10-\allowbreak15; 23:12-\allowbreak21; 25:3,\allowbreak11 Jud 16:2,\allowbreak3 2Co 11:32}
\crossref{Acts}{9}{25}{Jos 2:15 1Sa 19:11,\allowbreak12 2Co 11:33}
\crossref{Acts}{9}{26}{Ac 22:17; 26:20 Ga 1:17-\allowbreak19}
\crossref{Acts}{9}{27}{Ac 4:36; 11:22,\allowbreak25; 12:25; 13:2; 15:2,\allowbreak25,\allowbreak26,\allowbreak35-\allowbreak39 1Co 9:6 Ga 2:9,\allowbreak13}
\crossref{Acts}{9}{28}{Ac 1:21 Nu 27:16,\allowbreak17 2Sa 5:2 1Ki 3:7 Ps 121:8 Joh 10:9 Ga 1:18}
\crossref{Acts}{9}{29}{9:20-\allowbreak22,\allowbreak27}
\crossref{Acts}{9}{30}{9:24,\allowbreak25; 17:10,\allowbreak15 Mt 10:23}
\crossref{Acts}{9}{31}{Ac 8:1 De 12:10 Jos 21:44 Jud 3:30 1Ch 22:9,\allowbreak18 Ps 94:13 Pr 16:7}
\crossref{Acts}{9}{32}{Ac 1:8; 8:14,\allowbreak25 Ga 2:7-\allowbreak9}
\crossref{Acts}{9}{33}{Ac 3:2; 4:22; 14:8 Mr 5:25; 9:21 Lu 13:16 Joh 5:5; 9:1,\allowbreak21}
\crossref{Acts}{9}{34}{Ac 3:6,\allowbreak12,\allowbreak16; 4:10; 16:18 Mt 8:3; 9:6,\allowbreak28-\allowbreak30 Joh 2:11}
\crossref{Acts}{9}{35}{Ac 4:4; 5:12-\allowbreak14; 6:7; 19:10,\allowbreak20 Ps 110:3 Isa 66:8}
\crossref{Acts}{9}{36}{Ac 10:5 2Ch 2:16 Ezr 3:7 Jon 1:3}
\crossref{Acts}{9}{37}{Joh 11:3,\allowbreak4,\allowbreak36,\allowbreak37}
\crossref{Acts}{9}{38}{9:32,\allowbreak36}
\crossref{Acts}{9}{39}{9:41; 8:2 2Sa 1:24 Pr 10:7 1Th 4:13}
\crossref{Acts}{9}{40}{Mr 5:40; 9:25 Lu 8:54}
\crossref{Acts}{9}{41}{Ac 3:7 Mr 1:31}
\crossref{Acts}{9}{42}{9:35; 11:21; 19:17,\allowbreak18 Joh 11:4,\allowbreak45; 12:11,\allowbreak44}
\crossref{Acts}{9}{43}{Ac 10:6,\allowbreak32}
\crossref{Acts}{10}{1}{Ac 8:40; 21:8; 23:23,\allowbreak33; 25:1,\allowbreak13}
\crossref{Acts}{10}{2}{10:7,\allowbreak22; 2:5; 8:2; 13:50; 16:14; 22:12 Lu 2:25}
\crossref{Acts}{10}{3}{Job 4:15,\allowbreak16 Da 9:20,\allowbreak21}
\crossref{Acts}{10}{4}{Da 10:11 Lu 1:12,\allowbreak29; 24:5}
\crossref{Acts}{10}{5}{10:32; 9:38; 15:7; 16:9}
\crossref{Acts}{10}{6}{Ac 9:43}
\crossref{Acts}{10}{7}{10:2 Ge 24:1-\allowbreak10,\allowbreak52 Jud 7:10 1Sa 14:6,\allowbreak7 1Ti 6:2 Phm 1:16}
\crossref{Acts}{10}{8}{10:33; 26:19 Ps 119:59,\allowbreak60 Ec 9:10 Ga 1:16}
\crossref{Acts}{10}{9}{10:8; 11:5-\allowbreak10 1Sa 9:25 Zep 1:5 Mt 6:6 Mr 1:35; 6:46 1Ti 2:8}
\crossref{Acts}{10}{10}{Mt 4:2; 12:1-\allowbreak3; 21:18}
\crossref{Acts}{10}{11}{Ac 7:56 Eze 1:1 Lu 3:21 Joh 1:51 Re 4:1; 11:19; 19:11}
\crossref{Acts}{10}{12}{Ge 7:8,\allowbreak9 Isa 11:6-\allowbreak9; 65:25 Joh 7:37 1Co 6:9-\allowbreak11}
\crossref{Acts}{10}{13}{10:10 Jer 35:2-\allowbreak5 Joh 4:31-\allowbreak34}
\crossref{Acts}{10}{14}{Ge 19:18 Ex 10:11 Mt 16:22; 25:9 Lu 1:60}
\crossref{Acts}{10}{15}{10:28; 11:9; 15:9,\allowbreak20,\allowbreak29 Mt 15:11 Re 14:14-\allowbreak17,\allowbreak20 1Co 10:25 Ga 2:12,\allowbreak13}
\crossref{Acts}{10}{16}{Ge 41:32 Joh 21:17 2Co 13:1}
\crossref{Acts}{10}{17}{10:19; 2:12; 5:24; 25:20 Joh 13:12 1Pe 1:11}
\crossref{Acts}{10}{18}{10:5,\allowbreak6; 11:11}
\crossref{Acts}{10}{19}{Ac 8:29; 11:12; 13:2; 16:6,\allowbreak7; 21:4 Joh 16:13 1Co 12:11 1Ti 4:1}
\crossref{Acts}{10}{20}{Ac 8:26; 9:15; 15:7 Mr 16:15}
\crossref{Acts}{10}{21}{Joh 1:38,\allowbreak39; 18:4-\allowbreak8}
\crossref{Acts}{10}{22}{10:1-\allowbreak5}
\crossref{Acts}{10}{23}{Ge 19:2,\allowbreak3; 24:31,\allowbreak32 Jud 19:19-\allowbreak21 Heb 13:2 1Pe 4:9}
\crossref{Acts}{10}{24}{10:9}
\crossref{Acts}{10}{25}{Ac 14:11-\allowbreak13 Da 2:30,\allowbreak46 Mt 8:2; 14:33 Re 19:10; 22:8,\allowbreak9}
\crossref{Acts}{10}{26}{Ac 14:14,\allowbreak15 Isa 42:8; 48:13 Mt 4:10 2Th 2:3,\allowbreak4 Re 13:8; 19:10; 22:9}
\crossref{Acts}{10}{27}{10:24; 14:27 Joh 4:35 1Co 16:9 2Co 2:12 Col 4:3}
\crossref{Acts}{10}{28}{Ac 11:2,\allowbreak3; 22:21,\allowbreak22 Joh 4:9,\allowbreak27; 18:28 Ga 2:12-\allowbreak14}
\crossref{Acts}{10}{29}{10:19,\allowbreak20 Ps 119:60 1Pe 3:15}
\crossref{Acts}{10}{30}{10:7-\allowbreak9,\allowbreak23,\allowbreak24}
\crossref{Acts}{10}{31}{Isa 38:5 Da 9:23; 10:12 Lu 1:13}
\crossref{Acts}{10}{32}{10:5-\allowbreak8}
\crossref{Acts}{10}{33}{Ac 17:11,\allowbreak12; 28:28 De 5:25-\allowbreak29 2Ch 30:12 Pr 1:5; 9:9,\allowbreak10; 18:15; 25:12}
\crossref{Acts}{10}{34}{Ac 8:35 Mt 5:2 Eph 6:19,\allowbreak20}
\crossref{Acts}{10}{35}{Ac 15:9 Isa 56:3-\allowbreak8 Ro 2:13,\allowbreak25-\allowbreak29; 3:22,\allowbreak29,\allowbreak30; 10:12,\allowbreak13 1Co 12:13}
\crossref{Acts}{10}{36}{Ac 2:38,\allowbreak39; 3:25,\allowbreak26; 11:19; 13:46 Mt 10:6 Lu 24:47}
\crossref{Acts}{10}{37}{Ac 2:22; 26:26; 28:22}
\crossref{Acts}{10}{38}{Ac 2:22; 4:27 Ps 2:2,\allowbreak6}
\crossref{Acts}{10}{39}{10:41; 1:8,\allowbreak22; 2:32; 3:15; 5:30-\allowbreak32; 13:31 Lu 1:2; 24:48 Joh 15:27}
\crossref{Acts}{10}{40}{Ac 13:30,\allowbreak31; 17:31 Mt 28:1,\allowbreak2 Ro 1:4; 4:24,\allowbreak25; 6:4-\allowbreak11; 8:11; 14:9}
\crossref{Acts}{10}{41}{10:39; 1:2,\allowbreak3,\allowbreak22; 13:31 Joh 14:17,\allowbreak22; 20:1-\allowbreak21:25}
\crossref{Acts}{10}{42}{Ac 1:8; 4:19,\allowbreak20; 5:20,\allowbreak29-\allowbreak32 Mt 28:19,\allowbreak20 Mr 16:15,\allowbreak16 Lu 24:47,\allowbreak48}
\crossref{Acts}{10}{43}{Ac 26:22 Isa 53:11 Jer 31:34 Da 9:24 Mic 7:18 Zec 13:1 Mal 4:2}
\crossref{Acts}{10}{44}{Ac 2:2-\allowbreak4; 4:31; 8:15-\allowbreak17; 11:15; 19:6}
\crossref{Acts}{10}{45}{10:23; 11:3,\allowbreak15-\allowbreak18 Ga 3:13,\allowbreak14}
\crossref{Acts}{10}{46}{Ac 2:4,\allowbreak11; 19:6 1Co 14:20-\allowbreak25}
\crossref{Acts}{10}{47}{Ac 8:12,\allowbreak36; 11:15-\allowbreak17; 15:8,\allowbreak9 Ge 17:24-\allowbreak26 Ro 4:11; 10:12}
\crossref{Acts}{10}{48}{Joh 4:2 1Co 1:13-\allowbreak17 Ga 3:27}
\crossref{Acts}{11}{1}{Ac 8:14,\allowbreak15 Ga 1:17-\allowbreak22}
\crossref{Acts}{11}{2}{Ac 10:9,\allowbreak45; 15:1,\allowbreak5; 21:20-\allowbreak23 Ga 2:12-\allowbreak14}
\crossref{Acts}{11}{3}{Ac 10:23,\allowbreak28,\allowbreak48 Lu 15:2 1Co 5:11 2Jo 1:10}
\crossref{Acts}{11}{4}{Ac 14:27 Jos 22:21-\allowbreak31 Pr 15:1 Lu 1:3}
\crossref{Acts}{11}{5}{Ac 10:9-\allowbreak18}
\crossref{Acts}{11}{6}{Ac 3:4 Lu 4:20}
\crossref{Acts}{11}{7}{}
\crossref{Acts}{11}{8}{Mr 7:2 Ro 14:14}
\crossref{Acts}{11}{9}{Ac 10:28,\allowbreak34,\allowbreak35; 15:9 1Ti 4:5 Heb 9:13,\allowbreak14}
\crossref{Acts}{11}{10}{Nu 24:10 Joh 13:38; 21:17 2Co 12:8}
\crossref{Acts}{11}{11}{Ac 9:10-\allowbreak12; 10:17,\allowbreak18 Ex 4:14,\allowbreak27}
\crossref{Acts}{11}{12}{Ac 8:29; 10:19,\allowbreak20; 13:2,\allowbreak4; 15:7; 16:6,\allowbreak7 Joh 16:13 2Th 2:2 Re 22:17}
\crossref{Acts}{11}{13}{Ac 10:3-\allowbreak6,\allowbreak22,\allowbreak30-\allowbreak32; 12:11 Heb 1:14}
\crossref{Acts}{11}{14}{Ac 10:6,\allowbreak22,\allowbreak32,\allowbreak33,\allowbreak43; 16:31 Ps 19:7-\allowbreak11 Mr 16:16 Joh 6:63,\allowbreak68; 12:50}
\crossref{Acts}{11}{15}{Ac 10:34-\allowbreak44}
\crossref{Acts}{11}{16}{Ac 20:35 Lu 24:8 Joh 14:26; 16:4 2Pe 3:1}
\crossref{Acts}{11}{17}{11:15; 15:8,\allowbreak9 Mt 20:14,\allowbreak15 Ro 9:15,\allowbreak16,\allowbreak23,\allowbreak24; 11:34-\allowbreak36}
\crossref{Acts}{11}{18}{Le 10:19,\allowbreak20 Jos 22:30}
\crossref{Acts}{11}{19}{Ac 8:1-\allowbreak4}
\crossref{Acts}{11}{20}{Ac 2:10; 6:9; 13:1 Mt 27:32}
\crossref{Acts}{11}{21}{2Ch 30:12 Ezr 7:9; 8:18 Ne 2:8,\allowbreak18 Isa 53:1; 59:1 Lu 1:66}
\crossref{Acts}{11}{22}{11:1; 8:14; 15:2 1Th 3:6}
\crossref{Acts}{11}{23}{Mr 2:5 Col 1:6 1Th 1:3,\allowbreak4 2Ti 1:4,\allowbreak5 2Pe 1:4-\allowbreak9 3Jo 1:4}
\crossref{Acts}{11}{24}{Ac 24:16 2Sa 18:27 Ps 37:23; 112:5 Pr 12:2; 13:22; 14:14 Mt 12:35}
\crossref{Acts}{11}{25}{Ac 9:11,\allowbreak27,\allowbreak30; 21:39}
\crossref{Acts}{11}{26}{Ac 13:1,\allowbreak2}
\crossref{Acts}{11}{27}{Ac 2:17; 13:1; 15:32; 21:4,\allowbreak9 Mt 23:34 1Co 12:28; 14:32 Eph 4:11}
\crossref{Acts}{11}{28}{Ac 21:10}
\crossref{Acts}{11}{29}{Ezr 2:69 Ne 5:8 1Co 16:2 2Co 8:2-\allowbreak4,\allowbreak12-\allowbreak14 1Pe 4:9-\allowbreak11}
\crossref{Acts}{11}{30}{Ac 14:23; 15:4,\allowbreak6,\allowbreak23; 16:4; 20:17 1Ti 5:17 Tit 1:5 Jas 5:14 1Pe 5:1}
\crossref{Acts}{12}{1}{Ac 4:30; 9:31 Lu 22:53}
\crossref{Acts}{12}{2}{Mt 4:21,\allowbreak22; 20:23 Mr 10:35,\allowbreak38}
\crossref{Acts}{12}{3}{Ac 24:27; 25:9 Joh 12:43 Ga 1:10 1Th 2:4}
\crossref{Acts}{12}{4}{Ac 4:3; 5:18; 8:3 Mt 24:9 Lu 21:12; 22:33 Joh 13:36-\allowbreak38; 21:18}
\crossref{Acts}{12}{5}{}
\crossref{Acts}{12}{6}{Ge 22:14 De 32:26 1Sa 23:26,\allowbreak27 Ps 3:5,\allowbreak6; 4:8 Isa 26:3,\allowbreak4}
\crossref{Acts}{12}{7}{12:23; 5:19; 10:30; 27:23,\allowbreak24 1Ki 19:5,\allowbreak7 Ps 34:7; 37:32,\allowbreak33 Isa 37:30}
\crossref{Acts}{12}{8}{}
\crossref{Acts}{12}{9}{Ac 26:19 Ge 6:22 Joh 2:5 Heb 11:8}
\crossref{Acts}{12}{10}{12:4 Ge 40:3; 42:17 Nu 15:34 Isa 21:8}
\crossref{Acts}{12}{11}{Ge 15:13; 18:13; 26:9}
\crossref{Acts}{12}{12}{Ac 4:23; 16:40}
\crossref{Acts}{12}{13}{12:16 Lu 13:25}
\crossref{Acts}{12}{14}{Mt 28:8 Lu 24:41}
\crossref{Acts}{12}{15}{Ac 26:24 Job 9:16 Mr 16:11,\allowbreak14 Lu 24:11}
\crossref{Acts}{12}{16}{}
\crossref{Acts}{12}{17}{Ac 13:16; 19:33; 21:40 Lu 1:22 Joh 13:24}
\crossref{Acts}{12}{18}{Ac 5:22-\allowbreak25; 16:27; 19:23}
\crossref{Acts}{12}{19}{1Sa 23:14 Ps 37:32,\allowbreak33 Jer 36:26 Mt 2:13}
\crossref{Acts}{12}{20}{Pr 17:14; 20:18; 25:8 Ec 10:4 Isa 27:4,\allowbreak5 Lu 14:31,\allowbreak32}
\crossref{Acts}{12}{21}{12:21}
\crossref{Acts}{12}{22}{Ac 14:10-\allowbreak13 Ps 12:2 Da 6:7 Jude 1:16 Re 13:4}
\crossref{Acts}{12}{23}{Ex 12:12,\allowbreak23,\allowbreak29 1Sa 25:38 2Sa 24:17 1Ch 21:14-\allowbreak18 2Ch 32:21}
\crossref{Acts}{12}{24}{Ac 5:39; 6:7; 11:21; 19:20 Pr 28:28 Isa 41:10-\allowbreak13; 54:14-\allowbreak17; 55:10}
\crossref{Acts}{12}{25}{Ac 11:29,\allowbreak30; 13:1-\allowbreak3}
\crossref{Acts}{13}{1}{Ac 11:22-\allowbreak24; 14:26,\allowbreak27}
\crossref{Acts}{13}{2}{Ac 6:4 De 10:8 1Sa 2:11 1Ch 16:4,\allowbreak37 etc.}
\crossref{Acts}{13}{3}{13:2; 6:6; 8:15-\allowbreak17; 9:17; 14:23 Nu 27:23 1Ti 4:14; 5:22 2Ti 1:6; 2:2}
\crossref{Acts}{13}{4}{Ac 20:23}
\crossref{Acts}{13}{5}{13:14,\allowbreak46; 14:1; 17:1-\allowbreak3,\allowbreak17; 18:4; 19:8}
\crossref{Acts}{13}{6}{Ac 8:9-\allowbreak11; 19:18,\allowbreak19 Ex 22:18 Le 20:6 De 18:10-\allowbreak12 1Ch 10:13}
\crossref{Acts}{13}{7}{13:12; 18:12; 19:38}
\crossref{Acts}{13}{8}{13:6; 9:36 Joh 1:41}
\crossref{Acts}{13}{9}{13:7}
\crossref{Acts}{13}{10}{Ac 8:20-\allowbreak23 Ec 9:3 Mt 3:7; 15:19; 23:25-\allowbreak33 Lu 11:39 2Co 11:3}
\crossref{Acts}{13}{11}{Ex 9:3 1Sa 5:6,\allowbreak9,\allowbreak11 Job 19:21 Ps 32:4; 38:2; 39:10,\allowbreak11}
\crossref{Acts}{13}{12}{13:7; 28:7}
\crossref{Acts}{13}{13}{13:6; 27:13}
\crossref{Acts}{13}{14}{Ac 14:19,\allowbreak21-\allowbreak24}
\crossref{Acts}{13}{15}{13:27; 15:21 Lu 4:16-\allowbreak18}
\crossref{Acts}{13}{16}{Ac 12:17; 19:33; 21:40}
\crossref{Acts}{13}{17}{Ac 7:2 etc.}
\crossref{Acts}{13}{18}{Ac 7:36,\allowbreak39-\allowbreak43 Ex 16:2,\allowbreak35 Nu 14:22,\allowbreak33,\allowbreak34 De 9:7,\allowbreak21-\allowbreak24 Ne 9:16-\allowbreak21}
\crossref{Acts}{13}{19}{Ac 7:45 De 7:1 Jos 24:11 Ne 9:24 Ps 78:55}
\crossref{Acts}{13}{20}{Jud 2:16; 3:10 Ru 1:1 1Sa 12:11 2Sa 7:11 2Ki 23:22 1Ch 17:6}
\crossref{Acts}{13}{21}{1Sa 8:5-\allowbreak22; 12:12-\allowbreak19}
\crossref{Acts}{13}{22}{1Sa 12:25; 13:13; 15:11,\allowbreak23,\allowbreak26,\allowbreak28; 16:1; 28:16; 31:6 2Sa 7:15}
\crossref{Acts}{13}{23}{Ac 2:30 2Sa 7:12 Ps 89:35-\allowbreak37; 132:11 Isa 7:13; 11:1,\allowbreak10 Jer 23:5,\allowbreak6}
\crossref{Acts}{13}{24}{Ac 1:22; 10:37; 19:3,\allowbreak4 Mt 3:1-\allowbreak11 Mr 1:2-\allowbreak8 Lu 1:76; 3:2,\allowbreak3 etc.}
\crossref{Acts}{13}{25}{13:36; 20:24 Mr 6:16-\allowbreak28 Joh 4:34; 19:28-\allowbreak30 2Ti 4:7 Re 11:7}
\crossref{Acts}{13}{26}{13:15,\allowbreak17,\allowbreak46; 3:26 2Ch 20:7 Ps 105:6; 147:19,\allowbreak20 Isa 41:8; 48:1; 51:1,\allowbreak2}
\crossref{Acts}{13}{27}{Ac 3:17 Lu 22:34 Joh 8:28; 15:21; 16:3 Ro 11:8-\allowbreak10,\allowbreak25 1Co 2:8}
\crossref{Acts}{13}{28}{Ac 3:13,\allowbreak14 Mt 27:19,\allowbreak22-\allowbreak25 Mr 15:13-\allowbreak15 Lu 23:4,\allowbreak5,\allowbreak14-\allowbreak16,\allowbreak21-\allowbreak25}
\crossref{Acts}{13}{29}{13:27; 2:23; 4:28 Lu 18:31-\allowbreak33; 24:44 Joh 19:28,\allowbreak30,\allowbreak36,\allowbreak37}
\crossref{Acts}{13}{30}{Ac 2:24,\allowbreak32; 3:13,\allowbreak15,\allowbreak26; 4:10; 5:30,\allowbreak31; 10:40; 17:31 Mt 28:6 Joh 2:19}
\crossref{Acts}{13}{31}{Ac 1:3,\allowbreak11; 10:41 Mt 28:16 Mr 16:12-\allowbreak14 Lu 24:36-\allowbreak42 Joh 20:19-\allowbreak29}
\crossref{Acts}{13}{32}{13:38 Isa 40:9; 41:27; 52:7; 61:1 Lu 1:19; 2:10 Ro 10:15}
\crossref{Acts}{13}{33}{Ps 2:7 Heb 1:5,\allowbreak6; 5:5}
\crossref{Acts}{13}{34}{Ro 6:9}
\crossref{Acts}{13}{35}{Ac 2:27-\allowbreak31 Ps 16:10}
\crossref{Acts}{13}{36}{13:22 1Ch 11:2; 13:2-\allowbreak4; 15:12-\allowbreak16,\allowbreak25-\allowbreak29; 18:14; 22:1-\allowbreak29:30 Ps 78:71,\allowbreak72}
\crossref{Acts}{13}{37}{13:30; 2:24}
\crossref{Acts}{13}{38}{Ac 2:14; 4:10; 28:28 Eze 36:32 Da 3:18}
\crossref{Acts}{13}{39}{Isa 53:11 Hab 2:4 Lu 18:14 Joh 5:24 Ro 3:24-\allowbreak30; 4:5-\allowbreak8,\allowbreak24; 5:1,\allowbreak9}
\crossref{Acts}{13}{40}{Mal 3:2; 4:1 Mt 3:9-\allowbreak12 Heb 2:3; 3:12; 12:25}
\crossref{Acts}{13}{41}{Pr 1:24-\allowbreak32; 5:12 Isa 5:24; 28:14-\allowbreak22 Lu 16:14; 23:35 Heb 10:28-\allowbreak30}
\crossref{Acts}{13}{42}{Ac 10:33; 28:28 Eze 3:6 Mt 11:21; 19:30}
\crossref{Acts}{13}{43}{Ac 2:10; 6:5}
\crossref{Acts}{13}{44}{Ge 49:10 Ps 110:3 Isa 11:10; 60:8}
\crossref{Acts}{13}{45}{Ac 5:17}
\crossref{Acts}{13}{46}{Ac 4:13,\allowbreak29-\allowbreak31 Pr 28:1 Ro 10:20 Eph 6:19,\allowbreak20 Php 1:14 Heb 11:34}
\crossref{Acts}{13}{47}{Ac 1:8; 9:15; 22:21; 26:17,\allowbreak18 Mt 28:19 Mr 16:15 Lu 24:47}
\crossref{Acts}{13}{48}{13:42; 2:41; 8:8; 15:31 Lu 2:10-\allowbreak11 Ro 15:9-\allowbreak12}
\crossref{Acts}{13}{49}{Ac 6:7; 9:42; 12:24; 19:10,\allowbreak26 Php 1:13,\allowbreak14}
\crossref{Acts}{13}{50}{13:45; 6:12; 14:2,\allowbreak19; 17:13; 21:27 1Ki 21:25}
\crossref{Acts}{13}{51}{Ac 18:6 Mt 10:14 Mr 6:11 Lu 9:5}
\crossref{Acts}{13}{52}{Ac 2:46; 5:41 Mt 5:12 Lu 6:22,\allowbreak23 Joh 16:22,\allowbreak23 Ro 5:3; 14:17; 15:13}
\crossref{Acts}{14}{1}{Ac 13:51}
\crossref{Acts}{14}{2}{14:19; 13:45,\allowbreak50; 17:5,\allowbreak13; 18:12; 21:27-\allowbreak30 Mr 15:10,\allowbreak11 1Th 2:15,\allowbreak16}
\crossref{Acts}{14}{3}{Ac 18:9-\allowbreak11; 19:10 1Co 16:8,\allowbreak9}
\crossref{Acts}{14}{4}{Mic 7:6 Mt 10:34-\allowbreak36 Lu 2:34; 11:21-\allowbreak23; 12:51-\allowbreak53 Joh 7:43}
\crossref{Acts}{14}{5}{Ac 4:25-\allowbreak29; 17:5 Ps 2:1-\allowbreak3; 83:5 2Ti 3:11}
\crossref{Acts}{14}{6}{Ac 9:24; 17:13,\allowbreak14; 23:12 etc.}
\crossref{Acts}{14}{7}{14:21; 8:4; 11:19; 17:2 1Th 2:2 2Ti 4:2}
\crossref{Acts}{14}{8}{Ac 4:9 Joh 5:3,\allowbreak7}
\crossref{Acts}{14}{9}{Ac 3:4}
\crossref{Acts}{14}{10}{Ac 3:6-\allowbreak8; 9:33,\allowbreak34 Isa 35:6 Lu 7:14; 13:11-\allowbreak13 Joh 5:8,\allowbreak9; 14:12}
\crossref{Acts}{14}{11}{Ac 8:10; 12:22; 28:6}
\crossref{Acts}{14}{12}{Ac 19:35}
\crossref{Acts}{14}{13}{Ac 10:25 Da 2:46}
\crossref{Acts}{14}{14}{14:4 1Co 9:5,\allowbreak6}
\crossref{Acts}{14}{15}{Ac 7:26; 16:30; 27:10,\allowbreak21,\allowbreak25}
\crossref{Acts}{14}{16}{Ac 17:30 Ps 81:12; 147:20 Ho 4:17 Ro 1:21-\allowbreak25,\allowbreak28 Eph 2:12 1Pe 4:3}
\crossref{Acts}{14}{17}{Ac 17:27,\allowbreak28 Ps 19:1-\allowbreak4 Ro 1:19,\allowbreak20}
\crossref{Acts}{14}{18}{Ge 11:6; 19:9 Ex 32:21-\allowbreak23 Jer 44:16,\allowbreak17 Joh 6:15}
\crossref{Acts}{14}{19}{Ac 13:45,\allowbreak50,\allowbreak51; 17:13}
\crossref{Acts}{14}{20}{Ac 20:9-\allowbreak12 2Co 1:9,\allowbreak10; 6:9 Re 11:7-\allowbreak12}
\crossref{Acts}{14}{21}{Mt 28:19}
\crossref{Acts}{14}{22}{Ac 15:32,\allowbreak41; 18:23 Isa 35:3 1Co 1:8 1Th 3:2-\allowbreak4,\allowbreak13 1Pe 5:10}
\crossref{Acts}{14}{23}{Ac 1:22 Mr 3:14 1Ti 5:22 2Ti 2:2 Tit 1:5}
\crossref{Acts}{14}{24}{}
\crossref{Acts}{14}{25}{}
\crossref{Acts}{14}{26}{Ac 11:19,\allowbreak26; 13:1; 15:22,\allowbreak30 Ga 2:11}
\crossref{Acts}{14}{27}{Ac 15:4-\allowbreak6; 21:20-\allowbreak22 1Co 5:4; 11:18; 14:23}
\crossref{Acts}{14}{28}{Ac 11:26; 15:35}
\crossref{Acts}{15}{1}{Ac 21:20 Ga 2:4,\allowbreak12,\allowbreak13}
\crossref{Acts}{15}{2}{15:7 Ga 1:6-\allowbreak10; 2:5 Jude 1:3}
\crossref{Acts}{15}{3}{Ac 21:5; 28:15 Ro 15:24 1Co 16:6,\allowbreak11 Tit 3:13 3Jo 1:6-\allowbreak8}
\crossref{Acts}{15}{4}{Ac 18:27; 21:17 Mt 10:40 Ro 15:7 Col 4:10 2Jo 1:10 3Jo 1:8-\allowbreak10}
\crossref{Acts}{15}{5}{Ac 21:20; 26:5,\allowbreak6 Php 3:5-\allowbreak8}
\crossref{Acts}{15}{6}{15:25; 6:2; 21:18 Pr 15:22 Mt 18:20 Heb 13:7,\allowbreak17}
\crossref{Acts}{15}{7}{15:2,\allowbreak39 Php 2:14}
\crossref{Acts}{15}{8}{Ac 1:24 1Sa 16:7 1Ki 8:39 1Ch 28:9; 29:17 Ps 44:21; 139:1,\allowbreak2}
\crossref{Acts}{15}{9}{Ac 14:1,\allowbreak27 Ro 3:9,\allowbreak22,\allowbreak29,\allowbreak30; 4:11,\allowbreak12; 9:24; 10:11-\allowbreak13 1Co 7:18 Ga 3:28}
\crossref{Acts}{15}{10}{Ex 17:2 Isa 7:12 Mt 4:7 Heb 3:9}
\crossref{Acts}{15}{11}{Ro 3:24; 5:20,\allowbreak21; 6:23 1Co 16:23 2Co 8:9; 13:14 Ga 1:6; 2:16}
\crossref{Acts}{15}{12}{15:4; 14:27; 21:19}
\crossref{Acts}{15}{13}{1Co 14:30-\allowbreak33 Jas 1:19}
\crossref{Acts}{15}{14}{2Pe 1:1}
\crossref{Acts}{15}{15}{Ac 13:47 Ro 15:8-\allowbreak12}
\crossref{Acts}{15}{16}{Am 9:11,\allowbreak12}
\crossref{Acts}{15}{17}{Ge 22:18; 49:10 Ps 22:26,\allowbreak27; 67:1-\allowbreak3; 72:17-\allowbreak19 Isa 2:2,\allowbreak3; 11:10}
\crossref{Acts}{15}{18}{Ac 17:26 Nu 23:19 Isa 41:22,\allowbreak23; 44:7; 46:9,\allowbreak10 Mt 13:35; 25:34}
\crossref{Acts}{15}{19}{15:10,\allowbreak24,\allowbreak28 Ga 1:7-\allowbreak10; 2:4; 5:11,\allowbreak12}
\crossref{Acts}{15}{20}{15:29 Ge 35:2 Ex 20:3-\allowbreak5,\allowbreak23; 34:15,\allowbreak16 Nu 25:2 Ps 106:37-\allowbreak39}
\crossref{Acts}{15}{21}{Ac 13:15,\allowbreak27 Ne 8:1 etc.}
\crossref{Acts}{15}{22}{15:23,\allowbreak25; 6:4,\allowbreak5 2Sa 3:36 2Ch 30:4,\allowbreak12}
\crossref{Acts}{15}{23}{15:4,\allowbreak22}
\crossref{Acts}{15}{24}{Jer 23:16 Ga 2:4; 5:4,\allowbreak12 2Ti 2:14 Tit 1:10,\allowbreak11 1Jo 2:19}
\crossref{Acts}{15}{25}{15:28 Mt 11:26 Lu 1:3}
\crossref{Acts}{15}{26}{Ac 13:50; 14:19 Jud 5:18 1Co 15:30 2Co 11:23-\allowbreak27 Php 2:29,\allowbreak30}
\crossref{Acts}{15}{27}{15:22}
\crossref{Acts}{15}{28}{Joh 16:13 1Co 7:25,\allowbreak40; 14:37 1Th 4:8 1Pe 1:12}
\crossref{Acts}{15}{29}{15:20; 21:25 Le 17:14 Ro 14:14,\allowbreak15,\allowbreak20,\allowbreak21 1Co 10:18-\allowbreak20 Re 2:14,\allowbreak20}
\crossref{Acts}{15}{30}{Ac 6:2; 21:22}
\crossref{Acts}{15}{31}{15:1,\allowbreak10; 16:5 Ga 2:4,\allowbreak5; 5:1 Php 3:3}
\crossref{Acts}{15}{32}{Ac 2:17,\allowbreak18; 11:23,\allowbreak27; 13:1 Mt 23:34 Lu 11:49 Ro 12:6 1Co 12:28,\allowbreak29}
\crossref{Acts}{15}{33}{Ac 16:36 Ge 26:29 Ex 4:18 1Co 16:11 Heb 11:31 2Jo 1:10}
% skipped verse
%\crossref{Acts}{15}{34}{Ac 11:25,\allowbreak26; 18:27 1Co 16:12}
\crossref{Acts}{15}{35}{Ac 13:1; 14:28}
\crossref{Acts}{15}{36}{Ac 7:23 Ex 4:18 Jer 23:2 Mt 25:36,\allowbreak43}
\crossref{Acts}{15}{37}{Ac 12:12,\allowbreak25; 13:5,\allowbreak13 Col 4:10 2Ti 4:11 Phm 1:24}
\crossref{Acts}{15}{38}{Ac 13:13 Ps 78:9 Pr 25:19 Lu 9:61; 14:27-\allowbreak34 Jas 1:8}
\crossref{Acts}{15}{39}{15:2; 6:1 Ps 106:33; 119:96 Ec 7:20 Ro 7:18-\allowbreak21 Jas 3:2}
\crossref{Acts}{15}{40}{15:22,\allowbreak32; 16:1-\allowbreak3}
\crossref{Acts}{15}{41}{15:23; 18:18; 21:3 Ga 1:21}
\crossref{Acts}{16}{1}{Ac 14:6,\allowbreak21 2Ti 3:11}
\crossref{Acts}{16}{2}{Ac 6:3 1Ti 3:7; 5:10,\allowbreak25 2Ti 3:15 Heb 11:2}
\crossref{Acts}{16}{3}{Ac 15:37,\allowbreak40}
\crossref{Acts}{16}{4}{Ac 15:6,\allowbreak28,\allowbreak29}
\crossref{Acts}{16}{5}{Ac 15:41 2Ch 20:20 Isa 7:9 Ro 16:25 1Co 15:58 Ga 5:1 Eph 4:13-\allowbreak16}
\crossref{Acts}{16}{6}{Ac 2:10; 18:23}
\crossref{Acts}{16}{7}{1Pe 1:1}
\crossref{Acts}{16}{8}{16:11; 20:5 2Co 2:12 2Ti 4:13}
\crossref{Acts}{16}{9}{Ac 2:17,\allowbreak18; 9:10-\allowbreak12; 10:3,\allowbreak10-\allowbreak17,\allowbreak30; 11:5-\allowbreak12; 18:9,\allowbreak10; 22:17-\allowbreak21}
\crossref{Acts}{16}{10}{Ac 10:29; 26:13 Ps 119:60 Pr 3:27,\allowbreak28 2Co 2:12,\allowbreak13}
\crossref{Acts}{16}{11}{}
\crossref{Acts}{16}{12}{Ac 20:6 Php 1:1 1Th 2:2}
\crossref{Acts}{16}{13}{Ac 13:14,\allowbreak42; 17:2; 18:4; 20:7}
\crossref{Acts}{16}{14}{16:40}
\crossref{Acts}{16}{15}{16:33; 8:12,\allowbreak38; 11:14; 18:8 1Co 1:13-\allowbreak16}
\crossref{Acts}{16}{16}{16:13}
\crossref{Acts}{16}{17}{Ac 19:13 Mt 8:29 Mr 1:24 Lu 4:34,\allowbreak41}
\crossref{Acts}{16}{18}{Ac 14:13-\allowbreak15 Mr 1:25,\allowbreak26,\allowbreak34}
\crossref{Acts}{16}{19}{Ac 19:24-\allowbreak27 1Ti 6:10}
\crossref{Acts}{16}{20}{Ac 18:2; 19:34 Ezr 4:12-\allowbreak15 Es 3:8,\allowbreak9}
\crossref{Acts}{16}{21}{Ac 26:3 Jer 10:3}
\crossref{Acts}{16}{22}{Ac 17:5; 18:12; 19:28 etc.}
\crossref{Acts}{16}{23}{Ac 5:18; 8:3; 9:2; 12:4 Lu 21:12 Eph 3:1; 4:1 2Ti 2:9 Phm 1:9 Re 1:9}
\crossref{Acts}{16}{24}{1Ki 22:27 Jer 37:15,\allowbreak16; 38:26 La 3:53-\allowbreak55}
\crossref{Acts}{16}{25}{Job 35:10 Ps 22:2; 42:8; 77:6; 119:55,\allowbreak62 Isa 30:29}
\crossref{Acts}{16}{26}{Ac 4:31; 5:19; 12:7,\allowbreak10 Mt 28:2 Re 6:12; 11:13}
\crossref{Acts}{16}{27}{16:23,\allowbreak24}
\crossref{Acts}{16}{28}{Le 19:18 Ps 7:4; 35:14 Pr 24:11,\allowbreak12 Mt 5:44 Lu 6:27,\allowbreak28; 10:32-\allowbreak37}
\crossref{Acts}{16}{29}{Ac 9:5,\allowbreak6; 24:25 Ps 99:1; 119:120 Isa 66:2,\allowbreak5 Jer 5:22; 10:10 Da 6:26}
\crossref{Acts}{16}{30}{16:24 Job 34:32 Isa 1:16,\allowbreak17; 58:6,\allowbreak9 Mt 3:8; 5:7 Jas 2:13}
\crossref{Acts}{16}{31}{Ac 2:38,\allowbreak39; 4:12; 8:37; 11:13,\allowbreak14; 13:38,\allowbreak39; 15:11 Isa 45:22 Hab 2:4}
\crossref{Acts}{16}{32}{Ac 10:33-\allowbreak43 Mr 16:15 Eph 3:8 Col 1:27,\allowbreak28 1Th 2:8 1Ti 1:13-\allowbreak16}
\crossref{Acts}{16}{33}{16:23 Pr 16:7 Isa 11:6-\allowbreak9 Mt 25:35-\allowbreak40 Lu 10:33,\allowbreak34 Ga 5:6,\allowbreak13}
\crossref{Acts}{16}{34}{Lu 5:29; 19:6 Php 4:17 1Th 4:9,\allowbreak10 Phm 1:7 Jas 2:14-\allowbreak17 1Jo 3:18}
\crossref{Acts}{16}{35}{Ac 4:21; 5:40 Ps 76:10 Jer 5:22}
\crossref{Acts}{16}{36}{Ac 15:33 Ex 4:18 Jud 18:6 1Sa 1:17; 20:42; 25:35; 29:7 2Ki 5:19}
\crossref{Acts}{16}{37}{16:20-\allowbreak24; 22:25-\allowbreak28 Ps 58:1,\allowbreak2; 82:1,\allowbreak2; 94:20 Pr 28:1}
\crossref{Acts}{16}{38}{Ac 22:29 Mt 14:5; 21:46}
\crossref{Acts}{16}{39}{Ex 11:8 Isa 45:14; 49:23; 60:14 Mic 7:9,\allowbreak10 Re 3:9}
\crossref{Acts}{16}{40}{16:14; 4:23; 12:12-\allowbreak17}
\crossref{Acts}{17}{1}{Ac 20:4; 27:2 Php 4:16 1Th 1:1 2Th 1:1 2Ti 4:10}
\crossref{Acts}{17}{2}{Lu 4:16 Joh 18:20}
\crossref{Acts}{17}{3}{Ac 2:16-\allowbreak36; 3:22-\allowbreak26; 13:26-\allowbreak39}
\crossref{Acts}{17}{4}{17:34; 2:41,\allowbreak42,\allowbreak44; 4:23; 5:12-\allowbreak14; 14:1,\allowbreak4; 28:24 Pr 9:6; 13:20 So 1:7,\allowbreak8; 6:1}
\crossref{Acts}{17}{5}{17:13; 7:9; 13:45; 14:2,\allowbreak19; 18:12 Pr 14:30 Isa 26:11 Mt 27:18 1Co 3:3}
\crossref{Acts}{17}{6}{Ac 6:12,\allowbreak13; 16:19,\allowbreak20; 18:12,\allowbreak13}
\crossref{Acts}{17}{7}{Ac 16:21; 25:8-\allowbreak11 Ezr 4:12-\allowbreak15 Da 3:12; 6:13 Lu 23:2 Joh 19:12}
\crossref{Acts}{17}{8}{Mt 2:3 Joh 11:48}
\crossref{Acts}{17}{9}{}
\crossref{Acts}{17}{10}{17:14; 9:25; 23:23,\allowbreak24 Jos 2:15,\allowbreak16 1Sa 19:12-\allowbreak17; 20:42}
\crossref{Acts}{17}{11}{Pr 1:5; 9:9 Jer 2:21 Joh 1:45-\allowbreak49}
\crossref{Acts}{17}{12}{17:2-\allowbreak4; 13:46; 14:1 Ps 25:8,\allowbreak9 Joh 1:45-\allowbreak49; 7:17 Eph 5:14 Jas 1:21}
\crossref{Acts}{17}{13}{17:5 Mt 23:13 1Th 2:14-\allowbreak16}
\crossref{Acts}{17}{14}{17:10; 9:25,\allowbreak30 Mt 10:23}
\crossref{Acts}{17}{15}{Ac 18:1 1Th 3:1}
\crossref{Acts}{17}{16}{Ex 32:19,\allowbreak20 Nu 25:6-\allowbreak11 1Ki 19:10,\allowbreak14 Job 32:2,\allowbreak3,\allowbreak18-\allowbreak20 Ps 69:9}
\crossref{Acts}{17}{17}{17:2-\allowbreak4; 14:1-\allowbreak4}
\crossref{Acts}{17}{18}{Ro 1:22 1Co 1:20,\allowbreak21 Col 2:8}
\crossref{Acts}{17}{19}{17:22}
\crossref{Acts}{17}{20}{Ho 8:12 Mt 19:23-\allowbreak25 Mr 10:24-\allowbreak26 Joh 6:60; 7:35,\allowbreak36 1Co 1:18,\allowbreak23}
\crossref{Acts}{17}{21}{Eph 5:16 Col 4:5 2Th 3:11,\allowbreak12 1Ti 5:13 2Ti 2:16,\allowbreak17}
\crossref{Acts}{17}{22}{17:19}
\crossref{Acts}{17}{23}{Ro 1:23-\allowbreak25 1Co 8:5 2Th 2:4}
\crossref{Acts}{17}{24}{17:26-\allowbreak28; 4:24; 14:15 Ps 146:5 Isa 40:12,\allowbreak28; 45:18 Jer 10:11; 32:17}
\crossref{Acts}{17}{25}{Job 22:2; 35:6,\allowbreak7 Ps 16:2; 50:8-\allowbreak13 Jer 7:20-\allowbreak23 Am 5:21-\allowbreak23 Mt 9:13}
\crossref{Acts}{17}{26}{Ge 3:20; 9:19 Mal 2:10 Ro 5:12-\allowbreak19 1Co 15:22,\allowbreak47}
\crossref{Acts}{17}{27}{Ac 15:17 Ps 19:1-\allowbreak6 Ro 1:20; 2:4}
\crossref{Acts}{17}{28}{1Sa 25:29 Job 12:10 Ps 36:9; 66:9 Lu 20:38 Joh 5:26; 11:25}
\crossref{Acts}{17}{29}{Ps 94:7-\allowbreak9; 106:20; 115:4-\allowbreak8 Isa 40:12-\allowbreak18; 44:9-\allowbreak20 Hab 2:19,\allowbreak20}
\crossref{Acts}{17}{30}{Ac 14:16 Ps 50:21 Ro 1:28; 3:23,\allowbreak25}
\crossref{Acts}{17}{31}{Ac 10:42 Mt 25:31 etc.}
\crossref{Acts}{17}{32}{17:18; 2:13; 13:41; 25:19; 26:8,\allowbreak24,\allowbreak25 Ge 19:14 2Ch 30:9-\allowbreak11; 36:16}
\crossref{Acts}{17}{33}{}
\crossref{Acts}{17}{34}{17:4; 13:48 Isa 55:10-\allowbreak11 Mt 20:16 Ro 11:5,\allowbreak6}
\crossref{Acts}{18}{1}{Ac 17:32,\allowbreak33}
\crossref{Acts}{18}{2}{18:26 Ro 16:3,\allowbreak4 1Co 16:19 2Ti 4:19}
\crossref{Acts}{18}{3}{Ac 20:34,\allowbreak35 1Co 4:12; 9:6-\allowbreak12 2Co 11:9 1Th 2:9 2Th 3:8,\allowbreak9}
\crossref{Acts}{18}{4}{Ac 13:14 etc.}
\crossref{Acts}{18}{5}{Ac 17:14,\allowbreak15 1Th 3:2}
\crossref{Acts}{18}{6}{Ac 13:45; 19:9; 26:11 Lu 22:65 1Th 2:14-\allowbreak16 2Ti 2:25 Jas 2:6,\allowbreak7}
\crossref{Acts}{18}{7}{Col 4:11}
\crossref{Acts}{18}{8}{1Co 1:14}
\crossref{Acts}{18}{9}{Ac 16:9; 22:18; 23:11; 27:23-\allowbreak25 2Co 12:1-\allowbreak3}
\crossref{Acts}{18}{10}{Ex 4:12 Jos 1:5,\allowbreak9 Jud 2:18 Isa 8:10; 41:10; 43:2 Jer 1:18,\allowbreak19}
\crossref{Acts}{18}{11}{Ac 14:3; 19:10; 20:31}
\crossref{Acts}{18}{12}{Ac 13:7,\allowbreak12}
\crossref{Acts}{18}{13}{18:4; 6:13; 21:28; 24:5,\allowbreak6; 25:8}
\crossref{Acts}{18}{14}{Ac 21:39,\allowbreak40; 22:1,\allowbreak2; 26:1,\allowbreak2 Lu 21:12-\allowbreak15 1Pe 3:14,\allowbreak15}
\crossref{Acts}{18}{15}{Ac 23:29; 25:11,\allowbreak19; 26:3 1Ti 1:4; 6:4 2Ti 2:23 Tit 3:9}
\crossref{Acts}{18}{16}{Ps 76:10 Ro 13:3,\allowbreak4 Re 12:16}
\crossref{Acts}{18}{17}{1Co 1:1}
\crossref{Acts}{18}{18}{Ac 15:23,\allowbreak41; 21:3 Ga 1:21}
\crossref{Acts}{18}{19}{18:24; 19:1,\allowbreak17,\allowbreak26; 20:16 1Co 16:8 Eph 1:1 1Ti 1:3 2Ti 1:18; 4:12}
\crossref{Acts}{18}{20}{Ac 20:16; 21:13,\allowbreak14 Mr 1:37,\allowbreak38 1Co 16:12}
\crossref{Acts}{18}{21}{Ac 15:29 Lu 9:61 2Co 13:11}
\crossref{Acts}{18}{22}{Ac 8:40; 10:1,\allowbreak24; 11:11; 18:22; 23:23}
\crossref{Acts}{18}{23}{Ac 16:6 1Co 16:1 Ga 1:2; 4:14}
\crossref{Acts}{18}{24}{Ac 19:1 1Co 1:12; 3:5,\allowbreak6; 4:6; 16:12 Tit 3:13}
\crossref{Acts}{18}{25}{Ac 13:10; 16:17; 19:9,\allowbreak23 Ge 18:19 Jud 2:22 1Sa 12:23 Ps 25:8,\allowbreak9}
\crossref{Acts}{18}{26}{Ac 14:3 Isa 58:1 Eph 6:19,\allowbreak20}
\crossref{Acts}{18}{27}{Ac 9:27 Ro 16:1,\allowbreak2 1Co 16:3 2Co 3:1,\allowbreak2}
\crossref{Acts}{18}{28}{18:5,\allowbreak25; 9:22; 17:3; 26:22,\allowbreak23 Lu 24:27,\allowbreak44 1Co 15:3,\allowbreak4 Heb 7:1-\allowbreak10:39}
\crossref{Acts}{19}{1}{Ac 18:24-\allowbreak28 1Co 1:12; 3:4-\allowbreak7; 16:12}
\crossref{Acts}{19}{2}{19:5; 2:17,\allowbreak38,\allowbreak39; 8:15-\allowbreak17; 10:44; 11:15-\allowbreak17 Ro 1:11}
\crossref{Acts}{19}{3}{Mt 28:19 1Co 12:13}
\crossref{Acts}{19}{4}{Ac 1:5; 11:16; 13:23-\allowbreak25 Mt 3:11,\allowbreak12; 11:3-\allowbreak5; 21:25-\allowbreak32 Mr 1:1-\allowbreak12}
\crossref{Acts}{19}{5}{Ac 2:38; 8:12,\allowbreak16 Ro 6:3,\allowbreak4 1Co 1:13-\allowbreak15; 10:2}
\crossref{Acts}{19}{6}{Ac 6:6; 8:17-\allowbreak19; 9:17 1Ti 5:22 2Ti 1:6}
\crossref{Acts}{19}{7}{}
\crossref{Acts}{19}{8}{Ac 13:14,\allowbreak46; 14:1; 26:22,\allowbreak23}
\crossref{Acts}{19}{9}{Ac 7:51; 13:45,\allowbreak46; 18:6 2Ki 17:14 2Ch 30:8; 36:16 Ne 9:16,\allowbreak17,\allowbreak29}
\crossref{Acts}{19}{10}{Ac 18:11; 20:18,\allowbreak31 Ro 10:18}
\crossref{Acts}{19}{11}{Ac 5:12; 14:3; 15:12; 16:18 Mr 16:17-\allowbreak20 Joh 14:12 Ro 15:18,\allowbreak19 Ga 3:5}
\crossref{Acts}{19}{12}{Ac 5:15 2Ki 4:29-\allowbreak31; 13:20,\allowbreak21}
\crossref{Acts}{19}{13}{Ge 4:12,\allowbreak14 Ps 109:10}
\crossref{Acts}{19}{14}{19:14}
\crossref{Acts}{19}{15}{Ac 16:17,\allowbreak18 Ge 3:1-\allowbreak5 1Ki 22:21-\allowbreak23 Mt 8:29-\allowbreak31 Mr 1:24,\allowbreak34; 5:9-\allowbreak13}
\crossref{Acts}{19}{16}{Mr 5:3,\allowbreak4,\allowbreak15 Lu 8:29,\allowbreak35}
\crossref{Acts}{19}{17}{19:10}
\crossref{Acts}{19}{18}{Le 16:21; 26:40 Job 33:27,\allowbreak28 Ps 32:5 Pr 28:13 Jer 3:13}
\crossref{Acts}{19}{19}{Ac 8:9-\allowbreak11; 13:6,\allowbreak8 Ex 7:11,\allowbreak22 De 18:10-\allowbreak12 1Sa 28:7-\allowbreak9 1Ch 10:13}
\crossref{Acts}{19}{20}{Ac 6:7; 12:24 Isa 55:11 2Th 3:1}
\crossref{Acts}{19}{21}{Ro 15:25-\allowbreak28 Ga 2:1}
\crossref{Acts}{19}{22}{Ac 16:9,\allowbreak10; 18:5; 20:1 2Co 1:16; 2:13; 8:1; 11:9 1Th 1:8}
\crossref{Acts}{19}{23}{2Co 1:8-\allowbreak10; 6:9}
\crossref{Acts}{19}{24}{19:27,\allowbreak28,\allowbreak34,\allowbreak35}
\crossref{Acts}{19}{25}{Ac 16:19 Ho 4:8; 12:7,\allowbreak8 2Pe 2:3 Re 18:3,\allowbreak11-\allowbreak19}
\crossref{Acts}{19}{26}{19:10,\allowbreak18-\allowbreak20 1Co 16:8,\allowbreak9 1Th 1:9}
\crossref{Acts}{19}{27}{19:21 Zep 2:11 Mt 23:14 1Ti 6:5}
\crossref{Acts}{19}{28}{Ac 7:54; 16:19-\allowbreak24; 21:28-\allowbreak31 Ps 2:2 Re 12:12}
\crossref{Acts}{19}{29}{19:32; 17:8; 21:30,\allowbreak38}
\crossref{Acts}{19}{30}{Ac 14:14-\allowbreak18; 17:22-\allowbreak31; 21:39}
\crossref{Acts}{19}{31}{19:10; 16:6 Pr 16:7}
\crossref{Acts}{19}{32}{19:29; 21:34}
\crossref{Acts}{19}{33}{1Ti 1:20 2Ti 4:14}
\crossref{Acts}{19}{34}{19:26; 16:20 Ro 2:22}
\crossref{Acts}{19}{35}{Eph 2:12}
\crossref{Acts}{19}{36}{Ac 5:35-\allowbreak39 Pr 14:29; 25:8}
\crossref{Acts}{19}{37}{Ac 25:8 1Co 10:32 2Co 6:3}
\crossref{Acts}{19}{38}{19:24}
\crossref{Acts}{19}{39}{19:39}
\crossref{Acts}{19}{40}{Ac 17:5-\allowbreak8}
\crossref{Acts}{19}{41}{Pr 15:1,\allowbreak2 Ec 9:17}
\crossref{Acts}{20}{1}{Ac 19:23-\allowbreak41}
\crossref{Acts}{20}{2}{20:6; 16:12; 17:1,\allowbreak10}
\crossref{Acts}{20}{3}{20:19; 9:23,\allowbreak24; 23:12-\allowbreak15; 25:3 Ezr 8:31 Pr 1:11 Jer 5:26 2Co 7:5; 11:26}
\crossref{Acts}{20}{4}{Ro 16:21}
\crossref{Acts}{20}{5}{}
\crossref{Acts}{20}{6}{Ac 16:12 Php 1:1 1Th 2:2}
\crossref{Acts}{20}{7}{Joh 20:1,\allowbreak19,\allowbreak26 1Co 16:2 Re 1:10}
\crossref{Acts}{20}{8}{Ac 1:13 Lu 22:12}
\crossref{Acts}{20}{9}{Jon 1:5,\allowbreak6 Mt 26:40,\allowbreak41 Mr 13:36}
\crossref{Acts}{20}{10}{1Ki 17:21,\allowbreak22 2Ki 4:34,\allowbreak35}
\crossref{Acts}{20}{11}{20:7}
\crossref{Acts}{20}{12}{20:10}
\crossref{Acts}{20}{13}{Mr 1:35; 6:31-\allowbreak33,\allowbreak46}
\crossref{Acts}{20}{14}{}
\crossref{Acts}{20}{15}{20:17 2Ti 4:20}
\crossref{Acts}{20}{16}{20:13; 18:21; 19:21; 21:4; 12:13; 24:17 Ro 15:24-\allowbreak28}
\crossref{Acts}{20}{17}{20:28; 11:30; 14:23; 15:4,\allowbreak6,\allowbreak23; 16:4 1Ti 5:17 Tit 1:5 Jas 5:14 1Pe 5:1}
\crossref{Acts}{20}{18}{Ac 18:19; 19:1,\allowbreak10}
\crossref{Acts}{20}{19}{Ac 27:23 Joh 12:26 Ro 1:1,\allowbreak9; 12:11 Ga 1:10 Eph 6:7 Col 3:24}
\crossref{Acts}{20}{20}{20:27,\allowbreak31; 5:2 De 4:5 Ps 40:9,\allowbreak10 Eze 33:7-\allowbreak9 1Co 15:3 Col 1:28}
\crossref{Acts}{20}{21}{20:24; 2:40; 8:25; 18:5; 28:23 1Jo 5:11-\allowbreak13}
\crossref{Acts}{20}{22}{Ac 19:21; 21:11-\allowbreak14 Lu 9:51; 12:50 2Co 5:14}
\crossref{Acts}{20}{23}{Ac 9:16; 14:22; 21:4,\allowbreak11 Joh 16:33 1Th 3:3 2Ti 2:12}
\crossref{Acts}{20}{24}{Ac 21:13 Ro 8:35-\allowbreak39 1Co 15:58 2Co 4:1,\allowbreak8,\allowbreak9,\allowbreak16-\allowbreak18; 6:4-\allowbreak10; 7:4; 12:10}
\crossref{Acts}{20}{25}{20:38 Ro 15:23}
\crossref{Acts}{20}{26}{Job 16:19 Joh 12:17; 19:35 Ro 10:2 2Co 1:23; 8:3 1Th 2:10-\allowbreak12}
\crossref{Acts}{20}{27}{20:20,\allowbreak35; 26:22,\allowbreak23 2Co 4:2 Ga 1:7-\allowbreak10; 4:16 1Th 2:4}
\crossref{Acts}{20}{28}{2Ch 19:6,\allowbreak7 Mr 13:9 Lu 21:34 1Co 9:26,\allowbreak27 Col 4:17 1Ti 4:16}
\crossref{Acts}{20}{29}{Zep 3:3 Mt 7:15; 10:16 Lu 10:3 Joh 10:12 2Pe 2:1}
\crossref{Acts}{20}{30}{Mt 26:21-\allowbreak25 1Ti 1:19,\allowbreak20 2Ti 2:17,\allowbreak18; 4:3,\allowbreak4 2Pe 2:1-\allowbreak3 1Jo 2:19}
\crossref{Acts}{20}{31}{Mt 13:25 Mr 13:34-\allowbreak37 Lu 21:36 2Ti 4:5 Heb 13:17 Re 16:15}
\crossref{Acts}{20}{32}{Ac 14:23,\allowbreak26; 15:40 Ge 50:24 Jer 49:11 Jude 1:24,\allowbreak25}
\crossref{Acts}{20}{33}{Nu 16:15 1Sa 12:3-\allowbreak5 1Co 9:12,\allowbreak15,\allowbreak18 2Co 7:2; 11:9; 12:14,\allowbreak17}
\crossref{Acts}{20}{34}{Ac 18:3 1Co 4:12 1Th 2:9 2Th 3:8,\allowbreak9}
\crossref{Acts}{20}{35}{20:20,\allowbreak27}
\crossref{Acts}{20}{36}{Ac 7:60; 21:5 2Ch 6:13 Da 6:10 Lu 22:41 Eph 3:14 Php 4:6}
\crossref{Acts}{20}{37}{1Sa 20:41 2Sa 15:30 2Ki 20:3 Ezr 10:1 Job 2:12 Ps 126:5}
\crossref{Acts}{20}{38}{20:25}
\crossref{Acts}{21}{1}{Ac 20:37,\allowbreak38 1Sa 20:41,\allowbreak42 1Th 2:17}
\crossref{Acts}{21}{2}{Ac 27:6 Jon 1:3}
\crossref{Acts}{21}{3}{21:16; 4:36; 11:19; 13:4; 15:39; 27:4}
\crossref{Acts}{21}{4}{Ac 19:1 Mt 10:11 2Ti 1:17}
\crossref{Acts}{21}{5}{Ac 15:3; 17:10; 20:38}
\crossref{Acts}{21}{6}{2Co 2:13}
\crossref{Acts}{21}{7}{21:19; 18:22; 25:13 1Sa 10:4; 13:10 Mt 5:47 Heb 13:24}
\crossref{Acts}{21}{8}{Ac 16:10,\allowbreak13,\allowbreak16; 20:6,\allowbreak13; 27:1; 28:11,\allowbreak16}
\crossref{Acts}{21}{9}{1Co 7:25-\allowbreak34,\allowbreak38}
\crossref{Acts}{21}{10}{21:4,\allowbreak7; 20:16}
\crossref{Acts}{21}{11}{1Sa 15:27,\allowbreak28 1Ki 11:29-\allowbreak31 2Ki 13:15-\allowbreak19 Jer 13:1-\allowbreak11; 19:10,\allowbreak11}
\crossref{Acts}{21}{12}{21:4; 20:22 Mt 16:21-\allowbreak23}
\crossref{Acts}{21}{13}{1Sa 15:14 Isa 3:15 Eze 18:2 Jon 1:6}
\crossref{Acts}{21}{14}{Ge 43:14 1Sa 3:18 2Sa 15:25,\allowbreak26 2Ki 20:19 Mt 6:10; 26:39,\allowbreak42}
\crossref{Acts}{21}{15}{Ac 18:22; 25:1,\allowbreak6,\allowbreak9}
\crossref{Acts}{21}{16}{21:8; 10:24,\allowbreak48}
\crossref{Acts}{21}{17}{Ac 15:4 Ro 15:7 Heb 13:1,\allowbreak2 3Jo 1:7,\allowbreak8}
\crossref{Acts}{21}{18}{Ac 15:13 Mt 10:2 Ga 1:19; 2:9 Jas 1:1}
\crossref{Acts}{21}{19}{Ac 11:4 etc.}
\crossref{Acts}{21}{20}{Ac 4:21; 11:18 Ps 22:23,\allowbreak27; 72:17-\allowbreak19; 98:1-\allowbreak3 Isa 55:10-\allowbreak13; 66:9-\allowbreak14}
\crossref{Acts}{21}{21}{Ac 6:13,\allowbreak14; 16:3; 28:17 Ro 14:1-\allowbreak6 1Co 9:19-\allowbreak21 Ga 5:1-\allowbreak6; 6:12-\allowbreak15}
\crossref{Acts}{21}{22}{Ac 15:12,\allowbreak22; 19:32}
\crossref{Acts}{21}{23}{Ac 18:18 Nu 6:2-\allowbreak7}
\crossref{Acts}{21}{24}{21:26; 24:18 Ex 19:10,\allowbreak14 Nu 19:17-\allowbreak22 2Ch 30:18,\allowbreak19 Job 1:5; 41:25}
\crossref{Acts}{21}{25}{Ac 15:20,\allowbreak29}
\crossref{Acts}{21}{26}{1Co 9:20}
\crossref{Acts}{21}{27}{Ac 24:18}
\crossref{Acts}{21}{28}{Ac 19:26-\allowbreak28; 24:5,\allowbreak6}
\crossref{Acts}{21}{29}{Ac 20:4 2Ti 4:20}
\crossref{Acts}{21}{30}{Ac 16:20-\allowbreak22; 19:29; 26:21 Mt 2:3; 21:10}
\crossref{Acts}{21}{31}{Ac 22:22; 26:9,\allowbreak10 Joh 16:2 2Co 11:23 etc.}
\crossref{Acts}{21}{32}{Ac 23:23,\allowbreak24}
\crossref{Acts}{21}{33}{21:11; 12:6; 20:23; 22:25,\allowbreak29; 28:20 Jud 15:13; 16:8,\allowbreak12,\allowbreak21 Eph 6:20}
\crossref{Acts}{21}{34}{Ac 19:32}
\crossref{Acts}{21}{35}{Ge 6:11,\allowbreak12 Ps 55:9; 58:2 Jer 23:10 Hab 1:2,\allowbreak3}
\crossref{Acts}{21}{36}{Ac 7:54; 22:22 Lu 23:18 Joh 19:15 1Co 4:13}
\crossref{Acts}{21}{37}{21:19; 19:30 Mt 10:18-\allowbreak20 Lu 21:15}
\crossref{Acts}{21}{38}{}
\crossref{Acts}{21}{39}{Ac 9:11,\allowbreak30; 22:3; 23:34}
\crossref{Acts}{21}{40}{21:35 2Ki 9:13}
\crossref{Acts}{22}{1}{Ac 7:2; 13:26; 23:1,\allowbreak6; 28:17}
\crossref{Acts}{22}{2}{Ac 21:40}
\crossref{Acts}{22}{3}{Ac 21:39 Ro 11:1 2Co 11:22 Php 3:5}
\crossref{Acts}{22}{4}{22:19,\allowbreak20; 7:58; 8:1-\allowbreak4; 9:1,\allowbreak2,\allowbreak13,\allowbreak14,\allowbreak21; 26:9-\allowbreak11 1Co 15:9 Php 3:6 1Ti 1:13-\allowbreak15}
\crossref{Acts}{22}{5}{Ac 9:1,\allowbreak2,\allowbreak14; 26:10,\allowbreak12}
\crossref{Acts}{22}{6}{Ac 9:3-\allowbreak5; 26:12}
\crossref{Acts}{22}{7}{Ge 3:9; 16:8; 22:1,\allowbreak11 Ex 3:4 1Sa 3:10}
\crossref{Acts}{22}{8}{Ac 3:6; 4:10; 6:14 Mt 2:23}
\crossref{Acts}{22}{9}{Ac 9:7 Da 10:7}
\crossref{Acts}{22}{10}{Ac 2:37; 9:6; 10:33; 16:30 Ps 25:8,\allowbreak9; 143:8-\allowbreak10}
\crossref{Acts}{22}{11}{Ac 9:8,\allowbreak9}
\crossref{Acts}{22}{12}{Ac 9:10-\allowbreak18}
\crossref{Acts}{22}{13}{Ac 9:17 Phm 1:16}
\crossref{Acts}{22}{14}{Ac 3:13; 5:30; 13:17; 24:14 Ex 3:13-\allowbreak16; 15:2 2Ki 21:22 1Ch 12:17}
\crossref{Acts}{22}{15}{Ac 1:8,\allowbreak22; 10:39-\allowbreak41; 23:11; 26:16 etc.}
\crossref{Acts}{22}{16}{Ps 119:60 Jer 8:14}
\crossref{Acts}{22}{17}{Ac 9:26-\allowbreak28 Ga 1:18}
\crossref{Acts}{22}{18}{22:14}
\crossref{Acts}{22}{19}{22:4; 8:3; 9:1; 26:9-\allowbreak12}
\crossref{Acts}{22}{20}{Re 2:13; 17:6}
\crossref{Acts}{22}{21}{Ac 9:15}
\crossref{Acts}{22}{22}{Ac 7:54-\allowbreak57; 21:36; 25:24 Lu 23:18 Joh 19:15}
\crossref{Acts}{22}{23}{Ac 7:53; 26:11 Ec 10:3}
\crossref{Acts}{22}{24}{22:25-\allowbreak29; 16:22,\allowbreak23,\allowbreak37 Joh 19:1 Heb 11:35}
\crossref{Acts}{22}{25}{Ac 10:1; 23:17; 27:1,\allowbreak3,\allowbreak43 Mt 8:8; 27:54}
\crossref{Acts}{22}{26}{22:29; 23:27}
\crossref{Acts}{22}{27}{}
\crossref{Acts}{22}{28}{}
\crossref{Acts}{22}{29}{22:24 Heb 11:35}
\crossref{Acts}{22}{30}{Ac 21:11,\allowbreak33; 23:28; 26:29 Mt 27:2}
\crossref{Acts}{23}{1}{23:6; 6:15; 22:5 Pr 28:1}
\crossref{Acts}{23}{2}{Ac 24:1}
\crossref{Acts}{23}{3}{Mt 23:27,\allowbreak28}
\crossref{Acts}{23}{4}{}
\crossref{Acts}{23}{5}{Ex 22:28 Ec 10:20 2Pe 2:10 Jude 1:8,\allowbreak9}
\crossref{Acts}{23}{6}{Mt 10:16}
\crossref{Acts}{23}{7}{Ac 14:4 Ps 55:9 Mt 10:34 Joh 7:40-\allowbreak43}
\crossref{Acts}{23}{8}{Ac 4:1 Mt 22:23 Mr 12:18 Lu 20:27}
\crossref{Acts}{23}{9}{Ac 25:25; 26:31 1Sa 24:17 Pr 16:7 Lu 23:4,\allowbreak14,\allowbreak15,\allowbreak22}
\crossref{Acts}{23}{10}{23:27; 19:28-\allowbreak31; 21:30-\allowbreak36 Ps 7:2; 50:22 Mic 3:3 Jas 1:19; 3:14-\allowbreak18; 4:1,\allowbreak2}
\crossref{Acts}{23}{11}{Ac 2:25; 18:9; 27:23,\allowbreak24 Ps 46:1,\allowbreak2; 109:31 Isa 41:10,\allowbreak14; 43:2}
\crossref{Acts}{23}{12}{23:21,\allowbreak30; 25:3 Ps 2:1-\allowbreak3; 64:2-\allowbreak6 Isa 8:9,\allowbreak10 Jer 11:19 Mt 26:4}
\crossref{Acts}{23}{13}{2Sa 15:12,\allowbreak31 Joh 16:2}
\crossref{Acts}{23}{14}{Ps 52:1,\allowbreak2 Isa 3:9 Jer 6:15; 8:12 Ho 4:9 Mic 7:3}
\crossref{Acts}{23}{15}{Ac 25:3 Ps 21:11; 37:32,\allowbreak33 Pr 1:11,\allowbreak12,\allowbreak16; 4:16 Isa 59:7 Ro 3:14-\allowbreak16}
\crossref{Acts}{23}{16}{Job 5:13 Pr 21:30 La 3:37 1Co 3:19}
\crossref{Acts}{23}{17}{23:23; 22:26 Pr 22:3 Mt 8:8,\allowbreak9; 10:16}
\crossref{Acts}{23}{18}{Ac 16:25; 27:1; 28:17 Ge 40:14,\allowbreak15 Eph 3:1; 4:1 Phm 1:9}
\crossref{Acts}{23}{19}{Jer 31:32 Mr 8:23; 9:27}
\crossref{Acts}{23}{20}{23:12}
\crossref{Acts}{23}{21}{Ex 23:2}
\crossref{Acts}{23}{22}{Jos 2:14 Mr 1:44}
\crossref{Acts}{23}{23}{23:17}
\crossref{Acts}{23}{24}{Ne 2:12 Es 8:12 Lu 10:34}
\crossref{Acts}{23}{25}{}
\crossref{Acts}{23}{26}{Ac 24:3; 26:25}
\crossref{Acts}{23}{27}{23:10; 21:31-\allowbreak33; 24:7}
\crossref{Acts}{23}{28}{Ac 22:30}
\crossref{Acts}{23}{29}{23:6-\allowbreak9; 18:15; 24:5,\allowbreak6,\allowbreak10-\allowbreak21; 25:19,\allowbreak20}
\crossref{Acts}{23}{30}{23:16-\allowbreak24}
\crossref{Acts}{23}{31}{23:23,\allowbreak24 Lu 7:8 2Ti 2:3,\allowbreak4}
\crossref{Acts}{23}{32}{}
\crossref{Acts}{23}{33}{23:25-\allowbreak30}
\crossref{Acts}{23}{34}{Ac 25:1 Es 1:1; 8:9 Da 2:49; 6:1 Lu 23:6}
\crossref{Acts}{23}{35}{23:30; 24:1,\allowbreak10,\allowbreak22,\allowbreak24-\allowbreak27; 25:16}
\crossref{Acts}{24}{1}{24:11; 21:27}
\crossref{Acts}{24}{2}{}
\crossref{Acts}{24}{3}{Ac 23:26}
\crossref{Acts}{24}{4}{Heb 11:32}
\crossref{Acts}{24}{5}{Ac 6:13; 16:20,\allowbreak21; 17:6,\allowbreak7; 21:28; 22:22; 28:22 1Ki 18:17,\allowbreak18 Jer 38:4}
\crossref{Acts}{24}{6}{24:12; 19:37; 21:27-\allowbreak29}
% skipped verse
%\crossref{Acts}{24}{7}{Ac 21:31-\allowbreak33; 23:23-\allowbreak32 Pr 4:16}
\crossref{Acts}{24}{8}{Ac 23:30,\allowbreak35; 25:5,\allowbreak15,\allowbreak16}
\crossref{Acts}{24}{9}{Ac 6:11-\allowbreak13 Ps 4:2; 62:3,\allowbreak4; 64:2-\allowbreak8 Isa 59:4-\allowbreak7 Jer 9:3-\allowbreak6 Eze 22:27-\allowbreak29}
\crossref{Acts}{24}{10}{Ac 12:17; 13:16; 19:33; 21:40; 26:1}
\crossref{Acts}{24}{11}{24:1; 21:18,\allowbreak27; 22:30; 23:11,\allowbreak23,\allowbreak32,\allowbreak33}
\crossref{Acts}{24}{12}{24:5; 25:8; 28:17}
\crossref{Acts}{24}{13}{Ac 25:7 1Pe 3:16}
\crossref{Acts}{24}{14}{Ps 119:46 Mt 10:32}
\crossref{Acts}{24}{15}{24:21; 26:6,\allowbreak7; 28:20 etc.}
\crossref{Acts}{24}{16}{Ac 23:1 Ro 2:15; 9:1 1Co 4:4 2Co 1:12; 4:2 1Th 2:10 1Ti 1:5,\allowbreak19; 3:9}
\crossref{Acts}{24}{17}{Ac 11:29,\allowbreak30; 20:16 Ro 15:25,\allowbreak26 1Co 16:1,\allowbreak2 2Co 8:9 Ga 2:10}
\crossref{Acts}{24}{18}{Ac 21:26-\allowbreak30; 26:21}
\crossref{Acts}{24}{19}{Ac 23:30; 25:16}
\crossref{Acts}{24}{20}{}
\crossref{Acts}{24}{21}{Ac 4:2; 23:6; 26:6-\allowbreak8; 28:20}
\crossref{Acts}{24}{22}{24:10,\allowbreak24; 26:3}
\crossref{Acts}{24}{23}{24:26; 27:3; 28:16,\allowbreak31 Pr 16:7}
\crossref{Acts}{24}{24}{Ac 26:22 Mr 6:20 Lu 19:3; 23:8}
\crossref{Acts}{24}{25}{Ac 17:2 1Sa 12:7 Isa 1:18; 41:21 Ro 12:1 1Pe 3:15}
\crossref{Acts}{24}{26}{24:2,\allowbreak3 Ex 23:8 De 16:19 1Sa 8:3; 12:3 2Ch 19:7 Job 15:34}
\crossref{Acts}{24}{27}{Ac 28:30}
\crossref{Acts}{25}{1}{Ac 23:34}
\crossref{Acts}{25}{2}{25:15; 24:1 Job 31:31 Pr 4:16 Ro 3:12-\allowbreak19}
\crossref{Acts}{25}{3}{Ac 9:2 1Sa 23:19-\allowbreak21 Jer 38:4 Mr 6:23-\allowbreak25 Lu 23:8-\allowbreak24}
\crossref{Acts}{25}{4}{}
\crossref{Acts}{25}{5}{25:16; 23:30; 24:8}
\crossref{Acts}{25}{6}{}
\crossref{Acts}{25}{7}{25:24; 21:28; 24:5,\allowbreak6,\allowbreak13 Ezr 4:15 Es 3:8 Ps 27:12; 35:11 Mt 5:11,\allowbreak12}
\crossref{Acts}{25}{8}{25:10; 6:13,\allowbreak14; 23:1; 24:6,\allowbreak12,\allowbreak17-\allowbreak21; 28:17,\allowbreak21 Ge 40:15 Jer 37:18 Da 6:22}
\crossref{Acts}{25}{9}{25:3,\allowbreak20; 12:3; 24:27 Mr 15:15}
\crossref{Acts}{25}{10}{25:25; 23:29; 26:31; 28:18 Mt 27:18,\allowbreak23,\allowbreak24 2Co 4:2}
\crossref{Acts}{25}{11}{Ac 18:14 Jos 22:22 1Sa 12:3-\allowbreak5 Job 31:21,\allowbreak38-\allowbreak40 Ps 7:3-\allowbreak5}
\crossref{Acts}{25}{12}{25:21; 19:21; 23:11; 26:32; 27:1; 28:16 Ps 76:10 Isa 46:10,\allowbreak11 La 3:37}
\crossref{Acts}{25}{13}{25:22,\allowbreak23; 26:1,\allowbreak27,\allowbreak28}
\crossref{Acts}{25}{14}{Ac 24:27}
\crossref{Acts}{25}{15}{25:1-\allowbreak3 Es 3:9 Lu 18:3-\allowbreak5; 23:23}
\crossref{Acts}{25}{16}{25:4,\allowbreak5}
\crossref{Acts}{25}{17}{25:6}
\crossref{Acts}{25}{18}{}
\crossref{Acts}{25}{19}{25:7; 18:15,\allowbreak19; 23:29}
\crossref{Acts}{25}{20}{}
\crossref{Acts}{25}{21}{25:10; 26:32 2Ti 4:16}
\crossref{Acts}{25}{22}{Ac 9:15 Isa 52:15 Mt 10:18 Lu 21:12}
\crossref{Acts}{25}{23}{Ac 12:21 Es 1:4 Ec 1:2 Isa 5:14; 14:11 Eze 7:24; 30:18; 32:12; 33:28}
\crossref{Acts}{25}{24}{25:2,\allowbreak3,\allowbreak7}
\crossref{Acts}{25}{25}{Ac 23:9,\allowbreak29; 26:31 Lu 23:4,\allowbreak14 Joh 18:38}
\crossref{Acts}{25}{26}{Ac 26:2,\allowbreak3}
\crossref{Acts}{25}{27}{Pr 18:13 Joh 7:51}
\crossref{Acts}{26}{1}{Ac 25:16 Pr 18:13,\allowbreak17 Joh 7:51}
\crossref{Acts}{26}{2}{}
\crossref{Acts}{26}{3}{26:26; 6:14; 21:21; 24:10; 25:19,\allowbreak20,\allowbreak26; 28:17 De 17:18 1Co 13:2}
\crossref{Acts}{26}{4}{2Ti 3:10}
\crossref{Acts}{26}{5}{Ac 22:5}
\crossref{Acts}{26}{6}{26:8; 23:6; 24:15,\allowbreak21; 28:20}
\crossref{Acts}{26}{7}{Ezr 6:17; 8:35 Mt 19:28 Lu 22:30 Jas 1:1 Re 7:4-\allowbreak8}
\crossref{Acts}{26}{8}{Ac 4:2; 10:40-\allowbreak42; 13:30,\allowbreak31; 17:31,\allowbreak32; 25:19 Ge 18:14 Mt 22:29-\allowbreak32}
\crossref{Acts}{26}{9}{Joh 16:2,\allowbreak3 Ro 10:2 Ga 1:13,\allowbreak14 Php 3:6 1Ti 1:13}
\crossref{Acts}{26}{10}{Ac 7:58; 8:1,\allowbreak3; 9:13,\allowbreak26; 22:4,\allowbreak19,\allowbreak20 1Co 15:9 Ga 1:13}
\crossref{Acts}{26}{11}{Ac 22:19 Mt 10:17 Mr 13:9 Lu 21:12}
\crossref{Acts}{26}{12}{Ac 9:1,\allowbreak2; 22:5}
\crossref{Acts}{26}{13}{Ac 9:3; 22:6}
\crossref{Acts}{26}{14}{Ac 21:40; 22:2}
\crossref{Acts}{26}{15}{Ex 16:8 Mt 25:40,\allowbreak45 Joh 15:20,\allowbreak21}
\crossref{Acts}{26}{16}{Ac 9:6-\allowbreak9; 22:10}
\crossref{Acts}{26}{17}{Ac 9:23-\allowbreak25,\allowbreak29,\allowbreak30; 13:50; 14:5,\allowbreak6,\allowbreak19,\allowbreak20; 16:39; 17:10,\allowbreak14; 18:10,\allowbreak12-\allowbreak16}
\crossref{Acts}{26}{18}{Ac 9:17,\allowbreak18 Ps 119:18; 146:8 Isa 29:18; 32:3; 35:5; 42:7; 43:8 Lu 4:18}
\crossref{Acts}{26}{19}{26:2,\allowbreak26,\allowbreak27}
\crossref{Acts}{26}{20}{Ac 9:19-\allowbreak22; 11:26 etc.}
\crossref{Acts}{26}{21}{Ac 21:30,\allowbreak31; 22:22; 23:12-\allowbreak15; 25:3}
\crossref{Acts}{26}{22}{26:17; 14:19,\allowbreak20; 16:25,\allowbreak26; 18:9,\allowbreak10; 21:31-\allowbreak33; 23:10,\allowbreak11,\allowbreak16 etc.}
\crossref{Acts}{26}{23}{Ge 3:15 Ps 22:1-\allowbreak69:36 Isa 53:1-\allowbreak12 Da 9:24-\allowbreak26 Zec 12:10; 13:7}
\crossref{Acts}{26}{24}{Ac 22:1}
\crossref{Acts}{26}{25}{Joh 8:49 1Pe 2:21-\allowbreak23; 3:9,\allowbreak15}
\crossref{Acts}{26}{26}{26:2,\allowbreak3; 25:22}
\crossref{Acts}{26}{27}{26:22,\allowbreak23}
\crossref{Acts}{26}{28}{26:29; 24:25 Eze 33:31 Mt 10:18 Mr 6:20; 10:17-\allowbreak22 2Co 4:2 Jas 1:23,\allowbreak24}
\crossref{Acts}{26}{29}{Ex 16:3 Nu 11:29 2Sa 18:33 2Ki 5:3 1Co 4:8; 7:7 2Co 11:1}
\crossref{Acts}{26}{30}{Ac 18:15; 28:22}
\crossref{Acts}{26}{31}{Ac 23:9,\allowbreak29; 25:25; 28:18 2Sa 24:17 Lu 23:4,\allowbreak14,\allowbreak15 1Pe 3:16; 4:14-\allowbreak16}
\crossref{Acts}{26}{32}{Ac 25:11,\allowbreak12,\allowbreak25; 28:18}
\crossref{Acts}{27}{1}{Ac 19:21; 23:11; 25:12,\allowbreak25 Ge 50:20 Ps 33:11; 76:10 Pr 19:21 La 3:27}
\crossref{Acts}{27}{2}{Ac 21:1 Lu 8:22}
\crossref{Acts}{27}{3}{Ac 12:20 Ge 10:15; 49:13 Isa 23:2-\allowbreak4,\allowbreak12 Zec 9:2}
\crossref{Acts}{27}{4}{Ac 4:36; 11:19,\allowbreak20; 13:4; 15:39; 21:3,\allowbreak16}
\crossref{Acts}{27}{5}{Ac 6:9; 15:23,\allowbreak41; 21:39; 22:3 Ga 1:21}
\crossref{Acts}{27}{6}{27:1}
\crossref{Acts}{27}{7}{27:12,\allowbreak13,\allowbreak21; 2:11 Tit 1:5,\allowbreak12}
\crossref{Acts}{27}{8}{}
\crossref{Acts}{27}{9}{}
\crossref{Acts}{27}{10}{27:21-\allowbreak26,\allowbreak31,\allowbreak34 Ge 41:16-\allowbreak25,\allowbreak38,\allowbreak39 2Ki 6:9,\allowbreak10 Ps 25:14 Da 2:30}
\crossref{Acts}{27}{11}{27:21 Ex 9:20,\allowbreak21 2Ki 6:10 Pr 27:12 Eze 3:17,\allowbreak18; 33:4 Heb 11:7}
\crossref{Acts}{27}{12}{27:8 Ps 107:30}
\crossref{Acts}{27}{13}{Job 37:17 Ps 78:26 So 4:16 Lu 12:55}
\crossref{Acts}{27}{14}{Ex 14:21-\allowbreak27 Jon 1:3-\allowbreak5}
\crossref{Acts}{27}{15}{27:27 Jas 3:4}
\crossref{Acts}{27}{16}{}
\crossref{Acts}{27}{17}{27:29,\allowbreak41}
\crossref{Acts}{27}{18}{Ps 107:27}
\crossref{Acts}{27}{19}{Job 2:4 Jon 1:5 Mr 8:35-\allowbreak37 Lu 9:24,\allowbreak25}
\crossref{Acts}{27}{20}{Ex 10:21-\allowbreak23 Ps 105:28 Mt 24:29}
\crossref{Acts}{27}{21}{27:33-\allowbreak35 Ps 107:5,\allowbreak6}
\crossref{Acts}{27}{22}{27:25,\allowbreak36; 23:11 1Sa 30:6 Ezr 10:2 Job 22:29,\allowbreak30 Ps 112:7 Isa 43:1,\allowbreak2}
\crossref{Acts}{27}{23}{Ac 5:19; 12:8-\allowbreak11,\allowbreak23; 23:11 Da 6:22 Heb 1:14 Re 22:16}
\crossref{Acts}{27}{24}{Ac 18:9,\allowbreak10 Ge 15:1; 46:3 1Ki 17:13 2Ki 6:16 Isa 41:10-\allowbreak14; 43:1-\allowbreak5}
\crossref{Acts}{27}{25}{27:11,\allowbreak21 Nu 23:19 2Ch 20:20 Lu 1:45 Ro 4:20,\allowbreak21 2Ti 1:12}
\crossref{Acts}{27}{26}{Ac 28:1}
\crossref{Acts}{27}{27}{27:18-\allowbreak20}
\crossref{Acts}{27}{28}{}
\crossref{Acts}{27}{29}{27:17,\allowbreak41}
\crossref{Acts}{27}{30}{27:16,\allowbreak32}
\crossref{Acts}{27}{31}{27:11,\allowbreak21,\allowbreak42,\allowbreak43}
\crossref{Acts}{27}{32}{Lu 16:8 Php 3:7-\allowbreak9}
\crossref{Acts}{27}{33}{27:29}
\crossref{Acts}{27}{34}{Mt 15:32 Mr 8:2,\allowbreak3 Php 2:5 1Ti 5:23}
\crossref{Acts}{27}{35}{Ac 2:46,\allowbreak47 1Sa 9:13 Mt 15:36 Mr 8:6 Lu 24:30 Joh 6:11,\allowbreak23 Ro 14:6}
\crossref{Acts}{27}{36}{Ps 27:14 2Co 1:4-\allowbreak6}
\crossref{Acts}{27}{37}{27:24}
\crossref{Acts}{27}{38}{27:18,\allowbreak19 Job 2:4 Jon 1:5 Mt 6:25; 16:26 Heb 12:1}
\crossref{Acts}{27}{39}{}
\crossref{Acts}{27}{40}{}
\crossref{Acts}{27}{41}{27:17,\allowbreak26-\allowbreak29 2Co 11:25}
\crossref{Acts}{27}{42}{Ps 74:20 Pr 12:10 Ec 9:3 Mr 15:15-\allowbreak20 Lu 23:40,\allowbreak41}
\crossref{Acts}{27}{43}{27:3,\allowbreak11,\allowbreak31; 23:10,\allowbreak24 Pr 16:7 2Co 11:25}
\crossref{Acts}{27}{44}{27:22,\allowbreak24 Ps 107:28-\allowbreak30 Am 9:9 Joh 6:39,\allowbreak40 2Co 1:8-\allowbreak10 1Pe 4:18}
\crossref{Acts}{28}{1}{Ac 27:26,\allowbreak44}
\crossref{Acts}{28}{2}{28:4 Ro 1:14 1Co 14:11 Col 3:11}
\crossref{Acts}{28}{3}{Job 20:16 Isa 30:6; 41:24; 59:5 Mt 3:7; 12:34; 23:33}
\crossref{Acts}{28}{4}{28:2}
\crossref{Acts}{28}{5}{Nu 21:6-\allowbreak9 Ps 91:13 Mr 16:18 Lu 10:19 Joh 3:14,\allowbreak15 Ro 16:20}
\crossref{Acts}{28}{6}{Ac 12:22; 14:11-\allowbreak13 Mt 21:9; 27:22}
\crossref{Acts}{28}{7}{Ac 13:7; 18:12; 23:24}
\crossref{Acts}{28}{8}{Mr 1:30,\allowbreak31}
\crossref{Acts}{28}{9}{Ac 5:12,\allowbreak15 Mt 4:24 Mr 6:54-\allowbreak56}
\crossref{Acts}{28}{10}{Mt 15:5,\allowbreak6 1Th 2:6 1Ti 5:3,\allowbreak4,\allowbreak17,\allowbreak18}
\crossref{Acts}{28}{11}{Ac 6:9; 27:6}
\crossref{Acts}{28}{12}{}
\crossref{Acts}{28}{13}{Ac 27:13}
\crossref{Acts}{28}{14}{Ac 9:42,\allowbreak43; 19:1; 21:4,\allowbreak7,\allowbreak8 Ps 119:63 Mt 10:11}
\crossref{Acts}{28}{15}{Ac 10:25; 21:5 Ex 4:14 Joh 12:13 Ro 15:24 Ga 4:14 Heb 13:3}
\crossref{Acts}{28}{16}{Ac 27:3,\allowbreak31,\allowbreak43}
\crossref{Acts}{28}{17}{Ac 23:1 etc.}
\crossref{Acts}{28}{18}{Ac 22:24,\allowbreak25,\allowbreak30; 24:10,\allowbreak22; 25:7,\allowbreak8; 26:31}
\crossref{Acts}{28}{19}{Ac 25:10-\allowbreak12,\allowbreak21,\allowbreak25; 26:32}
\crossref{Acts}{28}{20}{28:17; 10:29,\allowbreak33}
\crossref{Acts}{28}{21}{Ex 11:7 Isa 41:11; 50:8; 54:17}
\crossref{Acts}{28}{22}{Ac 16:20,\allowbreak21; 17:6,\allowbreak7; 24:5,\allowbreak6,\allowbreak14 Lu 2:34 1Pe 2:12; 3:16; 4:14-\allowbreak16}
\crossref{Acts}{28}{23}{Phm 1:2}
\crossref{Acts}{28}{24}{Ac 13:48-\allowbreak50; 14:4; 17:4,\allowbreak5; 18:6-\allowbreak8; 19:8,\allowbreak9 Ro 3:3; 11:4-\allowbreak6}
\crossref{Acts}{28}{25}{28:29}
\crossref{Acts}{28}{26}{Isa 6:9,\allowbreak10 Eze 12:2 Mt 13:14,\allowbreak15 Mr 4:12 Lu 8:10 Joh 12:38-\allowbreak40}
\crossref{Acts}{28}{27}{}
\crossref{Acts}{28}{28}{Ac 2:14; 4:10; 13:38 Eze 36:32}
% skipped verse
%\crossref{Acts}{28}{29}{28:25 Mt 10:34-\allowbreak36 Lu 12:51 Joh 7:40-\allowbreak53}
\crossref{Acts}{28}{30}{28:16}
\crossref{Acts}{28}{31}{28:23; 8:12; 20:25 Mt 4:23 Mr 1:14 Lu 8:1}

% Rom
\crossref{Rom}{1}{1}{Ac 13:9; 21:40; 22:7,\allowbreak13; 26:1,\allowbreak14}
\crossref{Rom}{1}{2}{Lu 24:26,\allowbreak27 Ac 10:43; 26:6 Tit 1:2}
\crossref{Rom}{1}{3}{1:9; 8:2,\allowbreak3,\allowbreak29-\allowbreak32 Ps 2:7 Mt 3:17; 26:63; 27:43 Lu 1:35 Joh 1:34,\allowbreak49}
\crossref{Rom}{1}{4}{1:3 Joh 2:18-\allowbreak21 Ac 2:24,\allowbreak32; 3:15; 4:10-\allowbreak12; 5:30-\allowbreak32; 13:33-\allowbreak35; 17:31}
\crossref{Rom}{1}{5}{Ro 12:3; 15:15,\allowbreak16 Joh 1:16 1Co 15:10 2Co 3:5,\allowbreak6 Ga 1:15,\allowbreak16}
\crossref{Rom}{1}{6}{Eph 1:11 Col 1:6,\allowbreak21}
\crossref{Rom}{1}{7}{Ac 15:23 1Co 1:2 2Co 1:1 Php 1:1 Col 1:2 Jas 1:1 1Pe 1:1,\allowbreak2}
\crossref{Rom}{1}{8}{Ro 6:17}
\crossref{Rom}{1}{9}{Ro 9:1 Job 16:19 2Co 1:23; 11:10,\allowbreak11,\allowbreak31 Ga 1:20 Php 1:8 1Th 2:5-\allowbreak10}
\crossref{Rom}{1}{10}{Ro 15:22-\allowbreak24,\allowbreak30-\allowbreak32 Php 4:6 1Th 2:18; 3:10,\allowbreak11 Phm 1:22 Heb 13:19}
\crossref{Rom}{1}{11}{Ro 15:23,\allowbreak32 Ge 31:30 2Sa 13:39; 23:15 2Co 9:14 Php 1:8; 2:26; 4:1}
\crossref{Rom}{1}{12}{Ro 15:24,\allowbreak32 Ac 11:23 2Co 2:1-\allowbreak3; 7:4-\allowbreak7,\allowbreak13 1Th 2:17-\allowbreak20; 3:7-\allowbreak10}
\crossref{Rom}{1}{13}{Ro 11:25 1Co 10:1; 12:1 2Co 1:8 1Th 4:13}
\crossref{Rom}{1}{14}{Ro 8:12; 13:8}
\crossref{Rom}{1}{15}{Ro 12:18 1Ki 8:18 Mr 14:8 2Co 8:12}
\crossref{Rom}{1}{16}{Ps 40:9,\allowbreak10; 71:15,\allowbreak16; 119:46 Mr 8:38 Lu 9:26 1Co 2:2}
\crossref{Rom}{1}{17}{Ro 3:21}
\crossref{Rom}{1}{18}{Ro 4:15}
\crossref{Rom}{1}{19}{1:20 Ps 19:1-\allowbreak6 Isa 40:26 Jer 10:10-\allowbreak13 Ac 14:16; 17:23-\allowbreak30}
\crossref{Rom}{1}{20}{Joh 1:18 Col 1:15 1Ti 1:17; 6:16 Heb 11:27}
\crossref{Rom}{1}{21}{1:19,\allowbreak28 Joh 3:19}
\crossref{Rom}{1}{22}{Ro 11:25 Pr 25:14; 26:12 Isa 47:10 Jer 8:8,\allowbreak9; 10:14 Mt 6:23}
\crossref{Rom}{1}{23}{1:25 Ps 106:20 Jer 2:11}
\crossref{Rom}{1}{24}{Ps 81:11,\allowbreak12 Ho 4:17,\allowbreak18 Mt 15:14 Ac 7:42; 14:16; 17:29,\allowbreak30}
\crossref{Rom}{1}{25}{1:23}
\crossref{Rom}{1}{26}{1:24}
\crossref{Rom}{1}{27}{1:23,\allowbreak24}
\crossref{Rom}{1}{28}{1:18,\allowbreak21 Job 21:14,\allowbreak15 Pr 1:7,\allowbreak22,\allowbreak29; 5:12,\allowbreak13; 17:16 Jer 4:22; 9:6}
\crossref{Rom}{1}{29}{Ro 3:10}
\crossref{Rom}{1}{30}{Pr 25:23}
\crossref{Rom}{1}{31}{1:20,\allowbreak21; 3:11 Pr 18:2 Isa 27:11 Jer 4:22 Mt 15:16}
\crossref{Rom}{1}{32}{1:18,\allowbreak21; 2:1-\allowbreak5,\allowbreak21-\allowbreak23}
\crossref{Rom}{2}{1}{Ro 1:18-\allowbreak20}
\crossref{Rom}{2}{2}{2:5; 3:4,\allowbreak5; 9:14 Ge 18:25 Job 34:17-\allowbreak19,\allowbreak23 Ps 9:4,\allowbreak7,\allowbreak8; 11:5-\allowbreak7; 36:5,\allowbreak6}
\crossref{Rom}{2}{3}{2Sa 10:3 Job 35:2 Ps 50:21 Mt 26:53}
\crossref{Rom}{2}{4}{Ro 6:1,\allowbreak15 Ps 10:11 Ec 8:11 Jer 7:10 Eze 12:22,\allowbreak23 Mt 24:48,\allowbreak49}
\crossref{Rom}{2}{5}{Ro 11:25}
\crossref{Rom}{2}{6}{Ro 14:22 Job 34:11 Ps 62:12 Pr 24:2 Isa 3:10,\allowbreak11 Jer 17:10; 32:19}
\crossref{Rom}{2}{7}{Ro 8:24,\allowbreak25 Job 17:9 Ps 27:14; 37:3,\allowbreak34 La 3:25,\allowbreak26 Mt 24:12,\allowbreak13}
\crossref{Rom}{2}{8}{Pr 13:10 1Co 11:16 1Ti 6:3,\allowbreak4 Tit 3:9}
\crossref{Rom}{2}{9}{Pr 1:27,\allowbreak28 2Th 1:6}
\crossref{Rom}{2}{10}{2:7; 9:21,\allowbreak23 1Sa 2:30 Ps 112:6-\allowbreak9 Pr 3:16,\allowbreak17; 4:7-\allowbreak9; 8:18 Lu 9:48}
\crossref{Rom}{2}{11}{De 10:17; 16:19 2Ch 19:7 Job 34:19 Pr 24:23,\allowbreak24 Mt 22:16}
\crossref{Rom}{2}{12}{2:14,\allowbreak15; 1:18-\allowbreak21,\allowbreak32 Eze 16:49,\allowbreak50 Mt 11:22,\allowbreak24 Lu 10:12-\allowbreak15; 12:47,\allowbreak48}
\crossref{Rom}{2}{13}{2:25 De 4:1; 5:1; 6:3; 30:12-\allowbreak14 Eze 20:11; 33:30-\allowbreak33 Mt 7:21-\allowbreak27}
\crossref{Rom}{2}{14}{2:12; 3:1,\allowbreak2 De 4:7 Ps 147:19,\allowbreak20 Ac 14:16; 17:30 Eph 2:12}
\crossref{Rom}{2}{15}{Ro 1:18,\allowbreak19}
\crossref{Rom}{2}{16}{2:5; 3:6; 14:10-\allowbreak12 Ge 18:25 Ps 9:7,\allowbreak8; 50:6; 96:13; 98:9 Ec 3:17; 11:9}
\crossref{Rom}{2}{17}{2:28,\allowbreak29; 9:4-\allowbreak7 Ps 135:4 Isa 48:1,\allowbreak2 Mt 3:9; 8:11,\allowbreak12 Joh 8:33 2Co 11:22}
\crossref{Rom}{2}{18}{De 4:8 Ne 9:13,\allowbreak14 Ps 147:19,\allowbreak20 Lu 12:47 Joh 13:17 1Co 8:1,\allowbreak2}
\crossref{Rom}{2}{19}{Pr 26:12 Isa 5:21; 56:10 Mt 6:23; 15:14; 23:16-\allowbreak26 Mr 10:15}
\crossref{Rom}{2}{20}{Mt 11:25 1Co 3:1 Heb 5:13 1Pe 2:2}
\crossref{Rom}{2}{21}{Ps 50:16-\allowbreak21 Mt 23:3 etc.}
\crossref{Rom}{2}{22}{Jer 5:7; 7:9,\allowbreak10; 9:2 Eze 22:11 Mt 12:39; 16:4 Jas 4:4}
\crossref{Rom}{2}{23}{2:17; 3:2; 9:4 Jer 8:8,\allowbreak9 Mt 19:17-\allowbreak20 Lu 10:26-\allowbreak29; 18:11 Joh 5:45}
\crossref{Rom}{2}{24}{Isa 52:5 La 2:15,\allowbreak16 Eze 36:20-\allowbreak23 Mt 18:7 1Ti 5:14; 6:1}
\crossref{Rom}{2}{25}{2:28,\allowbreak29; 3:1,\allowbreak2; 4:11,\allowbreak12 De 30:6 Jer 4:4 Ga 5:3-\allowbreak6; 6:15 Eph 2:11,\allowbreak12}
\crossref{Rom}{2}{26}{Isa 56:6,\allowbreak7 Mt 8:11,\allowbreak12; 15:28 Ac 10:2-\allowbreak4,\allowbreak34,\allowbreak35; 11:3 etc.}
\crossref{Rom}{2}{27}{Ro 8:4; 13:10 Mt 3:15; 5:17-\allowbreak20 Ac 13:22 Ga 5:14}
\crossref{Rom}{2}{28}{Ro 9:6-\allowbreak8 Ps 73:1 Isa 1:9-\allowbreak15; 48:1,\allowbreak2 Ho 1:6-\allowbreak9 Mt 3:9 Joh 1:47}
\crossref{Rom}{2}{29}{1Sa 16:7 1Ch 29:17 Ps 45:13 Jer 4:14 Mt 23:25-\allowbreak28 Lu 11:39}
\crossref{Rom}{3}{1}{Ro 2:25-\allowbreak29 Ge 25:32 Ec 6:8,\allowbreak11 Isa 1:11-\allowbreak15 Mal 3:14 1Co 15:32}
\crossref{Rom}{3}{2}{3:3; 11:1,\allowbreak2,\allowbreak15-\allowbreak23,\allowbreak28,\allowbreak29}
\crossref{Rom}{3}{3}{Ro 9:6; 10:16; 11:1-\allowbreak7 Heb 4:2}
\crossref{Rom}{3}{4}{3:6,\allowbreak31; 6:2,\allowbreak15; 7:7,\allowbreak13; 9:14; 11:1,\allowbreak11 Lu 20:16 1Co 6:15 Ga 2:17}
\crossref{Rom}{3}{5}{3:7,\allowbreak25,\allowbreak26; 8:20,\allowbreak21}
\crossref{Rom}{3}{6}{3:4}
\crossref{Rom}{3}{7}{Ge 37:8,\allowbreak9,\allowbreak20; 44:1-\allowbreak14; 50:18-\allowbreak20 Ex 3:19; 14:5,\allowbreak30 1Ki 13:17,\allowbreak18,\allowbreak26-\allowbreak32}
\crossref{Rom}{3}{8}{Mt 5:11 1Pe 3:16,\allowbreak17}
\crossref{Rom}{3}{9}{3:5; 6:15; 11:7 1Co 10:19; 14:15 Php 1:18}
\crossref{Rom}{3}{10}{3:4; 11:8; 15:3,\allowbreak4 Isa 8:20 1Pe 1:16}
\crossref{Rom}{3}{11}{Ro 1:22,\allowbreak28 Ps 14:2-\allowbreak4; 53:2,\allowbreak4; 94:8 Pr 1:7,\allowbreak22,\allowbreak29,\allowbreak30 Isa 27:11}
\crossref{Rom}{3}{12}{Ex 32:8 Ps 14:3 Ec 7:29 Isa 53:6; 59:8 Jer 2:13 Eph 2:3}
\crossref{Rom}{3}{13}{Ps 5:9 Jer 5:16 Mt 23:27,\allowbreak28}
\crossref{Rom}{3}{14}{Ps 10:7; 59:12; 109:17,\allowbreak18 Jas 3:10}
\crossref{Rom}{3}{15}{Pr 1:16; 6:18 Isa 59:7,\allowbreak8}
\crossref{Rom}{3}{16}{3:16}
\crossref{Rom}{3}{17}{Ro 5:1 Isa 57:21; 59:8 Mt 7:14 Lu 1:79}
\crossref{Rom}{3}{18}{Ge 20:11 Ps 36:1 Pr 8:13; 16:6; 23:17 Lu 23:40 Re 19:5}
\crossref{Rom}{3}{19}{3:2; 2:12-\allowbreak18 Joh 10:34,\allowbreak35; 15:25 1Co 9:20,\allowbreak21 Ga 3:23; 4:5,\allowbreak21; 5:18}
\crossref{Rom}{3}{20}{3:28; 2:13; 4:13; 9:32 Ac 13:39 Ga 2:16,\allowbreak19; 3:10-\allowbreak13; 5:4 Eph 2:8,\allowbreak9}
\crossref{Rom}{3}{21}{Ro 1:17; 5:19,\allowbreak21; 10:3,\allowbreak4 Ge 15:6 Isa 45:24,\allowbreak25; 46:13; 51:8; 54:17}
\crossref{Rom}{3}{22}{Ro 4:3-\allowbreak13,\allowbreak20-\allowbreak22; 5:1 etc.}
\crossref{Rom}{3}{23}{3:9,\allowbreak19; 1:28-\allowbreak32; 2:1 etc.}
\crossref{Rom}{3}{24}{Ro 4:16; 5:16-\allowbreak19 1Co 6:11 Eph 2:7-\allowbreak10 Tit 3:5-\allowbreak7}
\crossref{Rom}{3}{25}{Ac 2:23; 3:18; 4:28; 15:18 1Pe 1:18-\allowbreak20 Re 13:8}
\crossref{Rom}{3}{26}{De 32:4 Ps 85:10,\allowbreak11 Isa 42:21; 45:21 Zep 3:5,\allowbreak15 Zec 9:9}
\crossref{Rom}{3}{27}{3:19; 2:17,\allowbreak23; 4:2 Eze 16:62,\allowbreak63; 36:31,\allowbreak32 Zep 3:11 Lu 18:9-\allowbreak14}
\crossref{Rom}{3}{28}{3:20-\allowbreak22,\allowbreak26; 4:5; 5:1; 8:3 Joh 3:14-\allowbreak18; 5:24; 6:40 Ac 13:38,\allowbreak39 1Co 6:11}
\crossref{Rom}{3}{29}{Ro 1:16; 9:24-\allowbreak26; 11:12,\allowbreak13; 15:9-\allowbreak13,\allowbreak16 Ge 17:7,\allowbreak8,\allowbreak18 Ps 22:7; 67:2}
\crossref{Rom}{3}{30}{3:28; 4:11,\allowbreak12; 10:12,\allowbreak13 Ga 2:14-\allowbreak16; 3:8,\allowbreak20,\allowbreak28; 5:6; 6:15 Php 3:3}
\crossref{Rom}{3}{31}{Ro 4:14 Ps 119:126 Jer 8:8,\allowbreak9 Mt 5:17; 15:6 Ga 2:21; 3:17-\allowbreak19}
\crossref{Rom}{4}{1}{Ro 6:1; 7:7; 8:31}
\crossref{Rom}{4}{2}{Ro 3:20-\allowbreak28 Php 3:9}
\crossref{Rom}{4}{3}{Ro 9:17; 10:11; 11:2 Isa 8:20 Mr 12:10 Jas 4:5 2Pe 1:20,\allowbreak21}
\crossref{Rom}{4}{4}{Ro 9:32; 11:6,\allowbreak35 Mt 20:1-\allowbreak16}
\crossref{Rom}{4}{5}{4:24,\allowbreak25; 3:22; 5:1,\allowbreak2; 10:3,\allowbreak9,\allowbreak10 Ac 13:38,\allowbreak39 Ga 2:16,\allowbreak17; 3:9-\allowbreak14 Php 3:9}
\crossref{Rom}{4}{6}{4:9 De 33:29 Ps 1:1-\allowbreak3; 112:1; 146:5,\allowbreak6 Mt 5:3-\allowbreak12 Ga 3:8,\allowbreak9,\allowbreak14; 4:15}
\crossref{Rom}{4}{7}{Ps 32:1,\allowbreak2; 51:8,\allowbreak9; 85:2; 130:3,\allowbreak4 Isa 40:1,\allowbreak2 Jer 33:8,\allowbreak9}
\crossref{Rom}{4}{8}{Isa 53:10-\allowbreak12 2Co 5:19-\allowbreak20 Phm 1:18,\allowbreak19 1Pe 2:24; 3:18}
\crossref{Rom}{4}{9}{Ro 3:29,\allowbreak30; 9:23,\allowbreak24; 10:12,\allowbreak13; 15:8-\allowbreak19 Isa 49:6 Lu 2:32}
\crossref{Rom}{4}{10}{Ge 15:5,\allowbreak6,\allowbreak16; 16:1-\allowbreak3; 17:1,\allowbreak10,\allowbreak23-\allowbreak27 1Co 7:18,\allowbreak19 Ga 5:6; 6:15}
\crossref{Rom}{4}{11}{Ge 17:10 Ex 12:13; 31:13,\allowbreak17 Eze 20:12,\allowbreak20}
\crossref{Rom}{4}{12}{Ro 9:6,\allowbreak7 Mt 3:9 Lu 16:23-\allowbreak31 Joh 8:39,\allowbreak40 Ga 4:22-\allowbreak31}
\crossref{Rom}{4}{13}{Ge 12:3; 17:4,\allowbreak5,\allowbreak16; 22:17,\allowbreak18; 28:14; 49:10 Ps 2:8; 72:11}
\crossref{Rom}{4}{14}{4:16 Ga 2:21; 3:18-\allowbreak24; 5:4 Php 3:9 Heb 7:19,\allowbreak28}
\crossref{Rom}{4}{15}{Ro 1:17; 2:5,\allowbreak6; 3:19,\allowbreak20; 5:13,\allowbreak20,\allowbreak21; 7:7-\allowbreak11 Nu 32:14 De 29:20-\allowbreak28}
\crossref{Rom}{4}{16}{Ro 3:24-\allowbreak26; 5:1 Ga 3:7-\allowbreak12,\allowbreak22 Eph 2:5,\allowbreak8 Tit 3:7}
\crossref{Rom}{4}{17}{Ge 17:4,\allowbreak5,\allowbreak16,\allowbreak20; 25:1-\allowbreak34; 28:3 Heb 11:12}
\crossref{Rom}{4}{18}{4:19; 5:5; 8:24 Ru 1:11-\allowbreak13 Pr 13:12 Eze 37:11 Mr 5:35,\allowbreak36 Lu 1:18}
\crossref{Rom}{4}{19}{4:20,\allowbreak21; 14:21 Mt 6:30; 8:26; 14:31 Mr 9:23,\allowbreak24 Joh 20:27,\allowbreak28}
\crossref{Rom}{4}{20}{Nu 11:13-\allowbreak23 2Ki 7:2,\allowbreak19 2Ch 20:15-\allowbreak20 Isa 7:9 Jer 32:16-\allowbreak27}
\crossref{Rom}{4}{21}{Ro 8:38 2Ti 1:12 Heb 11:13}
\crossref{Rom}{4}{22}{4:3,\allowbreak6}
\crossref{Rom}{4}{23}{Ro 15:4 1Co 9:10; 10:6,\allowbreak11 2Ti 3:16,\allowbreak17}
\crossref{Rom}{4}{24}{Ac 2:39}
\crossref{Rom}{4}{25}{Ro 3:25; 5:6-\allowbreak8; 8:3,\allowbreak32 Isa 53:5,\allowbreak6,\allowbreak10-\allowbreak12 Da 9:24,\allowbreak26 Zec 13:7}
\crossref{Rom}{5}{1}{5:9,\allowbreak18; 1:17; 3:22,\allowbreak26-\allowbreak28,\allowbreak30; 4:5,\allowbreak24,\allowbreak25; 9:30; 10:10 Hab 2:4 Joh 3:16-\allowbreak18}
\crossref{Rom}{5}{2}{Joh 10:7,\allowbreak9; 14:6 Ac 14:27 Eph 2:18; 3:12 Heb 10:19,\allowbreak20 1Pe 3:18}
\crossref{Rom}{5}{3}{Ro 8:35-\allowbreak37 Mt 5:10-\allowbreak12 Lu 6:22,\allowbreak23 Ac 5:41 2Co 11:23-\allowbreak30; 12:9,\allowbreak10}
\crossref{Rom}{5}{4}{Ro 15:4 2Co 1:4-\allowbreak6; 4:8-\allowbreak12; 6:9,\allowbreak10 Jas 1:12 1Pe 1:6,\allowbreak7; 5:10}
\crossref{Rom}{5}{5}{Job 27:8 Ps 22:4,\allowbreak5 Isa 28:15-\allowbreak18; 45:16,\allowbreak17; 49:23 Jer 17:5-\allowbreak8}
\crossref{Rom}{5}{6}{Eze 16:4-\allowbreak8 Eph 2:1-\allowbreak5 Col 2:13 Tit 3:3-\allowbreak5}
\crossref{Rom}{5}{7}{Joh 15:13 1Jo 3:16}
\crossref{Rom}{5}{8}{5:20; 3:5 Joh 15:13 Eph 1:6-\allowbreak8; 2:7 1Ti 1:16}
\crossref{Rom}{5}{9}{5:1; 3:24-\allowbreak26 Eph 2:13 Heb 9:14,\allowbreak22 1Jo 1:7}
\crossref{Rom}{5}{10}{Ro 8:7 2Co 5:18,\allowbreak19,\allowbreak21 Col 1:20,\allowbreak21}
\crossref{Rom}{5}{11}{Ro 2:17; 3:29,\allowbreak30 1Sa 2:1 Ps 32:11; 33:1; 43:4; 104:34; 149:2 Isa 61:10}
\crossref{Rom}{5}{12}{5:19 Ge 3:6}
\crossref{Rom}{5}{13}{Ge 4:7-\allowbreak11; 6:5,\allowbreak6,\allowbreak11; 8:21; 13:13; 18:20; 19:4,\allowbreak32,\allowbreak36; 38:7,\allowbreak10}
\crossref{Rom}{5}{14}{5:17,\allowbreak21 Ge 4:8; 5:5-\allowbreak31; 7:22; 19:25 Ex 1:6 Heb 9:27}
\crossref{Rom}{5}{15}{5:16,\allowbreak17,\allowbreak20 Isa 55:8,\allowbreak9 Joh 3:16; 4:10}
\crossref{Rom}{5}{16}{Ge 3:6-\allowbreak19 Ga 3:10 Jas 2:10}
\crossref{Rom}{5}{17}{5:12 Ge 3:6,\allowbreak19 1Co 15:21,\allowbreak22,\allowbreak49}
\crossref{Rom}{5}{18}{5:12,\allowbreak15,\allowbreak19; 3:19,\allowbreak20}
\crossref{Rom}{5}{19}{5:12-\allowbreak14}
\crossref{Rom}{5}{20}{Ro 3:19,\allowbreak20; 4:15; 6:14; 7:5-\allowbreak13 Joh 15:22 2Co 3:7-\allowbreak9 Ga 3:19-\allowbreak25}
\crossref{Rom}{5}{21}{5:14; 6:12,\allowbreak14,\allowbreak16}
\crossref{Rom}{6}{1}{Ro 3:5}
\crossref{Rom}{6}{2}{Ro 3:1-\allowbreak4:25}
\crossref{Rom}{6}{3}{6:16; 7:1 1Co 3:16; 5:6; 6:2,\allowbreak3,\allowbreak9,\allowbreak15,\allowbreak16,\allowbreak19; 9:13,\allowbreak24 2Co 13:5 Jud 4:4}
\crossref{Rom}{6}{4}{6:3 Col 2:12,\allowbreak13; 3:1-\allowbreak3 1Pe 3:21}
\crossref{Rom}{6}{5}{6:8-\allowbreak12 Eph 2:5,\allowbreak6 Php 3:10,\allowbreak11}
\crossref{Rom}{6}{6}{Ga 2:20; 5:24; 6:14 Eph 4:22 Col 3:5,\allowbreak9,\allowbreak10}
\crossref{Rom}{6}{7}{6:2,\allowbreak8; 7:2,\allowbreak4 Col 3:1-\allowbreak3 1Pe 4:1}
\crossref{Rom}{6}{8}{6:3-\allowbreak5 2Ti 2:11,\allowbreak12}
\crossref{Rom}{6}{9}{Ps 16:9-\allowbreak11 Ac 2:24-\allowbreak28 Heb 7:16,\allowbreak25; 10:12,\allowbreak13 Re 1:18}
\crossref{Rom}{6}{10}{Ro 8:3 2Co 5:21 Heb 9:26-\allowbreak28 1Pe 3:18}
\crossref{Rom}{6}{11}{Ro 8:18}
\crossref{Rom}{6}{12}{6:16; 5:21; 7:23,\allowbreak24 Nu 33:55 De 7:2 Jos 23:12,\allowbreak13 Jud 2:3 Ps 19:13}
\crossref{Rom}{6}{13}{6:16,\allowbreak19; 7:5,\allowbreak23 1Co 6:15 Col 3:5 Jas 3:5,\allowbreak6; 4:1}
\crossref{Rom}{6}{14}{6:12; 5:20,\allowbreak21; 8:2 Ps 130:7,\allowbreak8 Mic 7:19 Mt 1:21 Joh 8:36 Tit 2:14}
\crossref{Rom}{6}{15}{Ro 3:9}
\crossref{Rom}{6}{16}{6:3}
\crossref{Rom}{6}{17}{Ro 1:8 1Ch 29:12-\allowbreak16 Ezr 7:27 Mt 11:25,\allowbreak26 Ac 11:18; 28:15 1Co 1:4}
\crossref{Rom}{6}{18}{6:14 Ps 116:16; 119:32,\allowbreak45 Lu 1:74,\allowbreak75 Joh 8:32,\allowbreak36 1Co 7:21,\allowbreak22}
\crossref{Rom}{6}{19}{Ro 3:5 1Co 9:8; 15:32 Ga 3:15}
\crossref{Rom}{6}{20}{6:16,\allowbreak17 Joh 8:34}
\crossref{Rom}{6}{21}{Ro 7:5 Pr 1:31; 5:10-\allowbreak13; 9:17,\allowbreak18 Isa 3:10 Jer 17:10; 44:20-\allowbreak24}
\crossref{Rom}{6}{22}{6:14,\allowbreak18; 8:2 Joh 8:32 2Co 3:17 Ga 5:13}
\crossref{Rom}{6}{23}{Ro 5:12 Ge 2:17; 3:19 Isa 3:11 Eze 18:4,\allowbreak20 1Co 6:9,\allowbreak10 Ga 3:10}
\crossref{Rom}{7}{1}{Ro 6:3}
\crossref{Rom}{7}{2}{}
\crossref{Rom}{7}{3}{Ex 20:14 Le 20:10 Nu 5:13 etc.}
\crossref{Rom}{7}{4}{7:6; 6:14; 8:2 Ga 2:19,\allowbreak20; 3:13; 5:18 Eph 2:15 Col 2:14,\allowbreak20}
\crossref{Rom}{7}{5}{Ro 8:8,\allowbreak9 Joh 3:6 Ga 5:16,\allowbreak17,\allowbreak24 Eph 2:3,\allowbreak11 Tit 3:3}
\crossref{Rom}{7}{6}{7:4; 6:14,\allowbreak15 Ga 3:13,\allowbreak23-\allowbreak25; 4:4,\allowbreak5}
\crossref{Rom}{7}{7}{Ro 3:5; 4:1; 6:15}
\crossref{Rom}{7}{8}{7:11,\allowbreak13,\allowbreak17; 4:15; 5:20}
\crossref{Rom}{7}{9}{Mt 19:20 Lu 10:25-\allowbreak29; 15:29; 18:9-\allowbreak12,\allowbreak21 Php 3:5,\allowbreak6}
\crossref{Rom}{7}{10}{Ro 10:5 Le 18:5 Eze 20:11,\allowbreak13,\allowbreak21 Lu 10:27-\allowbreak29 2Co 3:7}
\crossref{Rom}{7}{11}{7:8,\allowbreak13}
\crossref{Rom}{7}{12}{7:14; 3:31; 12:2 De 4:8; 10:12 Ne 9:13 Ps 19:7-\allowbreak12; 119:38,\allowbreak86,\allowbreak127,\allowbreak137}
\crossref{Rom}{7}{13}{Ro 8:3 Ga 3:21}
\crossref{Rom}{7}{14}{Le 19:18 De 6:5 Ps 51:6 Mt 5:22,\allowbreak28; 22:37-\allowbreak40 Heb 4:12}
\crossref{Rom}{7}{15}{Ro 14:22 Lu 11:48}
\crossref{Rom}{7}{16}{7:12,\allowbreak14,\allowbreak22 Ps 119:127,\allowbreak128}
\crossref{Rom}{7}{17}{7:20; 4:7,\allowbreak8 2Co 8:12 Php 3:8,\allowbreak9}
\crossref{Rom}{7}{18}{Ge 6:5; 8:21 Job 14:4; 15:14-\allowbreak16; 25:4 Ps 51:5 Isa 64:6 Mt 15:19}
\crossref{Rom}{7}{19}{}
\crossref{Rom}{7}{20}{7:17}
\crossref{Rom}{7}{21}{7:23; 6:12,\allowbreak14; 8:2 Ps 19:13; 119:133 Joh 8:34 Eph 6:11-\allowbreak13 2Pe 2:19}
\crossref{Rom}{7}{22}{Ro 8:7 Job 23:12 Ps 1:2; 19:8-\allowbreak10; 40:8; 119:16,\allowbreak24,\allowbreak35,\allowbreak47,\allowbreak48,\allowbreak72,\allowbreak92}
\crossref{Rom}{7}{23}{7:5,\allowbreak21,\allowbreak25; 8:2 Ec 7:20 Ga 5:17 1Ti 6:11,\allowbreak12 Heb 12:4 Jas 3:2; 4:1 1Pe 2:11}
\crossref{Rom}{7}{24}{Ro 8:26 1Ki 8:38 Ps 6:6; 32:3,\allowbreak4; 38:2,\allowbreak8-\allowbreak10; 77:3-\allowbreak9; 119:20,\allowbreak81-\allowbreak83,\allowbreak131}
\crossref{Rom}{7}{25}{Ro 6:14,\allowbreak17 Ps 107:15,\allowbreak16; 116:16,\allowbreak17 Isa 12:1; 49:9,\allowbreak13 Mt 1:21}
\crossref{Rom}{8}{1}{Ro 4:7,\allowbreak8; 5:1; 7:17,\allowbreak20 Isa 54:17 Joh 3:18,\allowbreak19; 5:24 Ga 3:13}
\crossref{Rom}{8}{2}{Ro 3:27 Joh 8:36}
\crossref{Rom}{8}{3}{Ro 3:20; 7:5-\allowbreak11 Ac 13:39 Ga 3:21 Heb 7:18,\allowbreak19; 10:1-\allowbreak10,\allowbreak14}
\crossref{Rom}{8}{4}{Ga 5:22-\allowbreak24 Eph 5:26,\allowbreak27 Col 1:22 Heb 12:23 1Jo 3:2 Jude 1:24}
\crossref{Rom}{8}{5}{8:12,\allowbreak13 Joh 3:6 1Co 15:48 2Co 10:3 2Pe 2:10}
\crossref{Rom}{8}{6}{Ro 5:1,\allowbreak10; 14:17 Joh 14:6,\allowbreak27; 17:5 Ga 5:22}
\crossref{Rom}{8}{7}{Ro 1:28,\allowbreak30; 5:10 Ex 20:5 2Ch 19:2 Ps 53:1 Joh 7:7; 15:23,\allowbreak24}
\crossref{Rom}{8}{8}{8:9; 7:5 Joh 3:3,\allowbreak5,\allowbreak6}
\crossref{Rom}{8}{9}{8:2 Eze 11:19; 36:26,\allowbreak27 Joh 3:6}
\crossref{Rom}{8}{10}{Joh 6:56; 14:20,\allowbreak23; 15:5; 17:23 2Co 13:5 Eph 3:17 Col 1:27}
\crossref{Rom}{8}{11}{8:9; 4:24,\allowbreak25 Ac 2:24,\allowbreak32,\allowbreak33 Eph 1:19,\allowbreak20 Heb 13:20 1Pe 1:21}
\crossref{Rom}{8}{12}{Ro 6:2-\allowbreak15 Ps 116:16 1Co 6:19,\allowbreak20 1Pe 4:2,\allowbreak3}
\crossref{Rom}{8}{13}{8:1,\allowbreak4-\allowbreak6; 6:21,\allowbreak23; 7:5 Ga 5:19-\allowbreak21; 6:8 Eph 5:3-\allowbreak5 Col 3:5,\allowbreak6 Jas 1:14,\allowbreak15}
\crossref{Rom}{8}{14}{8:5,\allowbreak9 Ps 143:10 Pr 8:20 Isa 48:16,\allowbreak17 Ga 4:6; 5:16,\allowbreak18,\allowbreak22-\allowbreak25}
\crossref{Rom}{8}{15}{Ex 20:19 Nu 17:12 Lu 8:28,\allowbreak37 Joh 16:8 Ac 2:37; 16:29 1Co 2:12}
\crossref{Rom}{8}{16}{8:23,\allowbreak26 2Co 1:22; 5:5 Eph 1:13; 4:30 1Jo 4:13}
\crossref{Rom}{8}{17}{8:3,\allowbreak29,\allowbreak30; 5:9,\allowbreak10,\allowbreak17 Lu 12:32 Ac 26:18 Ga 3:29; 4:7 Eph 3:6 Tit 3:7}
\crossref{Rom}{8}{18}{Mt 5:11,\allowbreak12 Ac 20:24 2Co 4:17,\allowbreak18 Heb 11:25,\allowbreak26,\allowbreak35 1Pe 1:6,\allowbreak7}
\crossref{Rom}{8}{19}{8:23 Php 1:20}
\crossref{Rom}{8}{20}{8:22 Ge 3:17-\allowbreak19; 5:29; 6:13 Job 12:6-\allowbreak10 Isa 24:5,\allowbreak6 Jer 12:4,\allowbreak11}
\crossref{Rom}{8}{21}{2Pe 3:13}
\crossref{Rom}{8}{22}{8:20 Mr 16:15 Col 1:23}
\crossref{Rom}{8}{23}{8:15,\allowbreak16; 5:5 2Co 5:5 Ga 5:22,\allowbreak23 Eph 1:14; 5:9}
\crossref{Rom}{8}{24}{Ro 5:2; 12:12; 15:4,\allowbreak13 Ps 33:18,\allowbreak22; 146:5 Pr 14:32 Jer 17:7 Zec 9:12}
\crossref{Rom}{8}{25}{8:23; 2:7; 12:12 Ge 49:18 Ps 27:14; 37:7-\allowbreak9; 62:1,\allowbreak5,\allowbreak6; 130:5-\allowbreak7 Isa 25:9}
\crossref{Rom}{8}{26}{Ro 15:1 2Co 12:5-\allowbreak10 Heb 4:15; 5:2}
\crossref{Rom}{8}{27}{1Ch 28:9; 29:17 Ps 7:9; 44:21 Pr 17:3 Jer 11:20; 17:10; 20:12}
\crossref{Rom}{8}{28}{8:35-\allowbreak39; 5:3,\allowbreak4 Ge 50:20 De 8:2,\allowbreak3,\allowbreak16 Ps 46:1,\allowbreak2 Jer 24:5-\allowbreak7 Zec 13:9}
\crossref{Rom}{8}{29}{Ro 11:2 Ex 33:12,\allowbreak17 Ps 1:6 Jer 1:5 Mt 7:23 2Ti 2:19 1Pe 1:2}
\crossref{Rom}{8}{30}{8:28; 1:6; 9:23,\allowbreak24 Isa 41:9 1Co 1:2,\allowbreak9 Eph 4:4 Heb 9:15 1Pe 2:9}
\crossref{Rom}{8}{31}{Ro 4:1}
\crossref{Rom}{8}{32}{Ro 5:6-\allowbreak10; 11:21 Ge 22:12 Isa 53:10 Mt 3:17 Joh 3:16 2Co 5:21}
\crossref{Rom}{8}{33}{8:1 Job 1:9-\allowbreak11; 2:4-\allowbreak6; 22:6 etc.}
\crossref{Rom}{8}{34}{8:1; 14:13 Job 34:29 Ps 37:33; 109:31 Jer 50:20}
\crossref{Rom}{8}{35}{8:39 Ps 103:17 Jer 31:3 Joh 10:28; 13:1 2Th 2:13,\allowbreak14,\allowbreak16 Re 1:5}
\crossref{Rom}{8}{36}{Ps 44:22; 141:7 Joh 16:2 1Co 15:30 2Co 4:11}
\crossref{Rom}{8}{37}{2Ch 20:25-\allowbreak27 Isa 25:8 1Co 15:54,\allowbreak57 2Co 2:14; 12:9,\allowbreak19 1Jo 4:4}
\crossref{Rom}{8}{38}{Ro 4:21 2Co 4:13 2Ti 1:12 Heb 11:13}
\crossref{Rom}{8}{39}{Eph 3:18,\allowbreak19}
\crossref{Rom}{9}{1}{Ro 1:9 2Co 1:23; 11:31; 12:19 Ga 1:20 Php 1:8 1Th 2:5 1Ti 2:7; 5:21}
\crossref{Rom}{9}{2}{Ro 10:1 1Sa 15:35 Ps 119:136 Isa 66:10 Jer 9:1; 13:17 La 1:12}
\crossref{Rom}{9}{3}{Ex 32:32}
\crossref{Rom}{9}{4}{9:6 Ge 32:28 Ex 19:3-\allowbreak6 De 7:6 Ps 73:1 Isa 41:8; 46:3 Joh 1:47}
\crossref{Rom}{9}{5}{Ro 11:28 De 10:15}
\crossref{Rom}{9}{6}{Ro 3:3; 11:1,\allowbreak2 Nu 23:19 Isa 55:11 Mt 24:35 Joh 10:35 2Ti 2:13}
\crossref{Rom}{9}{7}{Lu 3:8; 16:24,\allowbreak25,\allowbreak30 Joh 8:37-\allowbreak39 Php 3:3}
\crossref{Rom}{9}{8}{Ro 4:11-\allowbreak16 Ga 4:22-\allowbreak31}
\crossref{Rom}{9}{9}{Ge 17:21; 18:10,\allowbreak14; 21:2}
\crossref{Rom}{9}{10}{Ro 5:3,\allowbreak11 Lu 16:26}
\crossref{Rom}{9}{11}{Ro 4:17 Ps 51:5 Eph 2:3}
\crossref{Rom}{9}{12}{Ge 25:22,\allowbreak23 2Sa 8:14 1Ki 22:47}
\crossref{Rom}{9}{13}{Mal 1:2,\allowbreak3}
\crossref{Rom}{9}{14}{Ro 3:1,\allowbreak5}
\crossref{Rom}{9}{15}{9:16,\allowbreak18,\allowbreak19 Ex 33:19; 34:6,\allowbreak7 Isa 27:11 Mic 7:18}
\crossref{Rom}{9}{16}{9:11 Ge 27:1-\allowbreak4,\allowbreak9-\allowbreak14 Ps 110:3 Isa 65:1 Mt 11:25,\allowbreak26 Lu 10:21}
\crossref{Rom}{9}{17}{Ro 11:4 Ga 3:8,\allowbreak22; 4:30}
\crossref{Rom}{9}{18}{9:15,\allowbreak16; 5:20,\allowbreak21 Eph 1:6}
\crossref{Rom}{9}{19}{Ro 3:8 1Co 15:12,\allowbreak35 Jas 1:13}
\crossref{Rom}{9}{20}{Ro 2:1 Mic 6:8 1Co 7:16 Jas 2:20}
\crossref{Rom}{9}{21}{9:11,\allowbreak18 Pr 16:4 Isa 64:8 Jer 18:3-\allowbreak6}
\crossref{Rom}{9}{22}{9:17; 1:18; 2:4,\allowbreak5 Ex 9:16 Ps 90:11 Pr 16:4 Re 6:16,\allowbreak17}
\crossref{Rom}{9}{23}{Ro 2:4; 5:20,\allowbreak21 Eph 1:6-\allowbreak8,\allowbreak18; 2:4,\allowbreak7,\allowbreak10; 3:8,\allowbreak16 Col 1:27 2Th 1:10-\allowbreak12}
\crossref{Rom}{9}{24}{Ro 8:28-\allowbreak30 1Co 1:9 Heb 3:1 1Pe 5:10 Re 19:9}
\crossref{Rom}{9}{25}{Ho 1:1,\allowbreak2}
\crossref{Rom}{9}{26}{Ho 1:9,\allowbreak10}
\crossref{Rom}{9}{27}{Isa 1:1}
\crossref{Rom}{9}{28}{Isa 28:22; 30:12-\allowbreak14 Da 9:26,\allowbreak27 Mt 24:21}
\crossref{Rom}{9}{29}{Isa 1:9; 6:13 La 3:22}
\crossref{Rom}{9}{30}{9:14; 3:5}
\crossref{Rom}{9}{31}{9:30-\allowbreak32; 10:2-\allowbreak4 Ga 3:21 Php 3:6}
\crossref{Rom}{9}{32}{Ro 4:16; 10:3 Mt 19:16-\allowbreak20 Joh 6:27-\allowbreak29 Ac 16:30-\allowbreak34 1Jo 5:9-\allowbreak12}
\crossref{Rom}{9}{33}{Ps 118:22 Isa 8:14,\allowbreak15; 28:16 Mt 21:42,\allowbreak44 1Pe 2:7,\allowbreak8}
\crossref{Rom}{10}{1}{Ro 9:1-\allowbreak3 Ex 32:10,\allowbreak13 1Sa 12:23; 15:11,\allowbreak35; 16:1 Jer 17:16; 18:20}
\crossref{Rom}{10}{2}{2Ki 10:16 Joh 16:2 Ac 21:20,\allowbreak28; 22:3,\allowbreak22; 26:9,\allowbreak10 Ga 1:14; 4:17,\allowbreak18}
\crossref{Rom}{10}{3}{Ro 9:31,\allowbreak32 Isa 57:12; 64:6 Lu 10:29; 16:15; 18:9-\allowbreak12 Ga 5:3,\allowbreak4 Php 3:9}
\crossref{Rom}{10}{4}{Ro 3:25-\allowbreak31; 8:3,\allowbreak4 Isa 53:11 Mt 3:15; 5:17,\allowbreak18 Joh 1:17 Ac 13:38,\allowbreak39}
\crossref{Rom}{10}{5}{Le 18:5 Ne 9:29 Eze 20:11,\allowbreak13,\allowbreak21 Lu 10:27,\allowbreak28 Ga 3:12}
\crossref{Rom}{10}{6}{Ro 3:22,\allowbreak25; 4:13; 9:31 Php 3:9 Heb 11:7}
\crossref{Rom}{10}{7}{Ro 4:25 Heb 13:20 1Pe 3:18,\allowbreak22 Re 1:18}
\crossref{Rom}{10}{8}{De 30:14}
\crossref{Rom}{10}{9}{Ro 14:11 Mt 10:32,\allowbreak33 Lu 12:8 Joh 9:22; 12:42,\allowbreak43 Php 2:11 1Jo 4:2,\allowbreak3}
\crossref{Rom}{10}{10}{Lu 8:15 Joh 1:12,\allowbreak13; 3:19-\allowbreak21 Heb 3:12; 10:22}
\crossref{Rom}{10}{11}{Ro 9:33 Isa 28:16; 49:23 Jer 17:7 1Pe 2:6}
\crossref{Rom}{10}{12}{Ro 3:22,\allowbreak29,\allowbreak30; 4:11,\allowbreak12; 9:24 Ac 10:34,\allowbreak35; 15:8,\allowbreak9 Ga 3:28 Eph 2:18-\allowbreak22}
\crossref{Rom}{10}{13}{Joe 2:32 Ac 2:21}
\crossref{Rom}{10}{14}{1Ki 8:41-\allowbreak43 Jon 1:5,\allowbreak9-\allowbreak11,\allowbreak16; 3:5-\allowbreak9 Heb 11:6 Jas 5:15}
\crossref{Rom}{10}{15}{Jer 23:32 Mt 9:38; 10:1-\allowbreak6; 28:18-\allowbreak20 Lu 10:1 Joh 20:21 Ac 9:15}
\crossref{Rom}{10}{16}{Ro 3:3; 11:17 Joh 10:26 Ac 28:24 Heb 4:2 1Pe 2:8}
\crossref{Rom}{10}{17}{10:14; 1:16 Lu 16:29-\allowbreak31 1Co 1:18-\allowbreak24 Col 1:4-\allowbreak6 1Th 2:13 2Th 2:13,\allowbreak14}
\crossref{Rom}{10}{18}{Ac 2:5-\allowbreak11; 26:20; 28:23}
\crossref{Rom}{10}{19}{10:18; 3:26 1Co 1:12; 7:29; 10:19; 11:22; 15:50}
\crossref{Rom}{10}{20}{Pr 28:1 Isa 58:1 Eph 6:19,\allowbreak20}
\crossref{Rom}{10}{21}{Pr 1:24 Isa 65:2-\allowbreak5 Jer 25:4; 35:15 Mt 20:1-\allowbreak15; 21:33-\allowbreak43; 22:3-\allowbreak7}
\crossref{Rom}{11}{1}{1Sa 12:22 2Ki 23:27 Ps 77:7; 89:31-\allowbreak37; 94:14 Jer 31:36,\allowbreak37}
\crossref{Rom}{11}{2}{Ro 8:29,\allowbreak30; 9:6,\allowbreak23 Ac 13:48; 15:18 1Pe 1:2}
\crossref{Rom}{11}{3}{1Ki 18:4,\allowbreak13; 19:10-\allowbreak18 Ne 9:26 Jer 2:30}
\crossref{Rom}{11}{4}{1Ki 19:18}
\crossref{Rom}{11}{5}{11:6,\allowbreak7}
\crossref{Rom}{11}{6}{Ro 3:27,\allowbreak28; 4:4,\allowbreak5; 5:20,\allowbreak21 De 9:4-\allowbreak6 1Co 15:10 Ga 2:21; 5:4 Eph 2:4-\allowbreak9}
\crossref{Rom}{11}{7}{Ro 3:9; 6:15 1Co 10:19 Php 1:18}
\crossref{Rom}{11}{8}{Isa 29:10}
\crossref{Rom}{11}{9}{Ps 69:22,\allowbreak23}
\crossref{Rom}{11}{10}{11:8; 1:21 Ps 69:23 Zec 11:17 Eph 4:18 2Pe 2:4,\allowbreak17 Jude 1:6,\allowbreak13}
\crossref{Rom}{11}{11}{Eze 18:23,\allowbreak32; 33:11}
\crossref{Rom}{11}{12}{11:15,\allowbreak33; 9:23 Eph 3:8 Col 1:27}
\crossref{Rom}{11}{13}{Ro 15:16-\allowbreak19 Ac 9:15; 13:2; 22:21; 26:17,\allowbreak18 Ga 1:16; 2:2,\allowbreak7-\allowbreak9 Eph 3:8}
\crossref{Rom}{11}{14}{1Co 7:16; 9:20-\allowbreak22 2Ti 2:10}
\crossref{Rom}{11}{15}{11:1,\allowbreak2,\allowbreak11,\allowbreak12}
\crossref{Rom}{11}{16}{Ex 22:29; 23:16,\allowbreak19 Le 23:10 Nu 15:17-\allowbreak21 De 18:4; 26:10}
\crossref{Rom}{11}{17}{Ps 80:11-\allowbreak16 Isa 6:13; 27:11 Jer 11:16 Eze 15:6-\allowbreak8 Mt 8:11,\allowbreak12}
\crossref{Rom}{11}{18}{11:20; 3:27 1Ki 20:11 Pr 16:18 Mt 26:33 Lu 18:9-\allowbreak11 1Co 10:12}
\crossref{Rom}{11}{19}{11:11,\allowbreak12,\allowbreak17,\allowbreak23,\allowbreak24}
\crossref{Rom}{11}{20}{Joh 4:17,\allowbreak18 Jas 2:19}
\crossref{Rom}{11}{21}{11:17,\allowbreak19; 8:32 Jer 25:29; 49:12 1Co 10:1-\allowbreak12 2Pe 2:4-\allowbreak9 Jude 1:5}
\crossref{Rom}{11}{22}{Ro 2:4,\allowbreak5; 9:22,\allowbreak23 Nu 14:18-\allowbreak22 De 32:39-\allowbreak43 Jos 23:15,\allowbreak16}
\crossref{Rom}{11}{23}{Zec 12:10 Mt 23:39 2Co 3:16}
\crossref{Rom}{11}{24}{11:17,\allowbreak18,\allowbreak30}
\crossref{Rom}{11}{25}{Ps 107:43 Ho 14:9 1Co 10:1; 12:1 2Pe 3:8}
\crossref{Rom}{11}{26}{Isa 11:11-\allowbreak16; 45:17; 54:6-\allowbreak10 Jer 3:17-\allowbreak23; 30:17-\allowbreak22; 31:31-\allowbreak37}
\crossref{Rom}{11}{27}{Isa 55:3; 59:21 Jer 31:31-\allowbreak34; 32:38-\allowbreak40 Heb 8:8-\allowbreak12; 10:16}
\crossref{Rom}{11}{28}{11:11,\allowbreak30 Mt 21:43 Ac 13:45,\allowbreak46; 14:2; 18:6 1Th 2:15,\allowbreak16}
\crossref{Rom}{11}{29}{Nu 23:19 Ho 13:14 Mal 3:6}
\crossref{Rom}{11}{30}{1Co 6:9-\allowbreak11 Eph 2:1,\allowbreak2,\allowbreak12,\allowbreak13,\allowbreak19-\allowbreak21 Col 3:7 Tit 3:3-\allowbreak7}
\crossref{Rom}{11}{31}{Ro 10:16; 11:15,\allowbreak25}
\crossref{Rom}{11}{32}{Ro 3:9,\allowbreak22 Ga 3:22}
\crossref{Rom}{11}{33}{Ps 107:8 etc.}
\crossref{Rom}{11}{34}{Job 15:8; 36:22 Isa 40:13 Jer 23:18 1Co 2:16}
\crossref{Rom}{11}{35}{Job 35:7; 41:11 Mt 20:15 1Co 4:7}
\crossref{Rom}{11}{36}{1Ch 29:11,\allowbreak12 Ps 33:6 Pr 16:4 Da 2:20-\allowbreak23; 4:3,\allowbreak34 Mt 6:13}
\crossref{Rom}{12}{1}{Ro 15:30 1Co 1:10 2Co 5:20; 6:1; 10:1 Eph 4:1 1Th 4:1,\allowbreak10; 5:12}
\crossref{Rom}{12}{2}{Ex 23:2 Le 18:29,\allowbreak30 De 18:9-\allowbreak14 Joh 7:7; 14:30; 15:19; 17:14}
\crossref{Rom}{12}{3}{12:6-\allowbreak8; 1:5; 15:15,\allowbreak16 1Co 3:10; 15:10 Ga 2:8,\allowbreak9 Eph 3:2,\allowbreak4,\allowbreak7,\allowbreak8; 4:7-\allowbreak12}
\crossref{Rom}{12}{4}{1Co 12:4,\allowbreak12,\allowbreak27 Eph 4:15,\allowbreak16}
\crossref{Rom}{12}{5}{12:4 1Co 10:17; 12:12-\allowbreak14,\allowbreak20,\allowbreak27,\allowbreak28}
\crossref{Rom}{12}{6}{Ro 1:11 1Co 1:5-\allowbreak7; 4:6,\allowbreak7; 12:4-\allowbreak11,\allowbreak28-\allowbreak31; 13:2 1Pe 4:10,\allowbreak11}
\crossref{Rom}{12}{7}{Isa 21:8 Eze 3:17-\allowbreak21; 33:7-\allowbreak9 Mt 24:45-\allowbreak47 Lu 12:42-\allowbreak44}
\crossref{Rom}{12}{8}{Ac 13:15; 15:32; 20:2 1Co 14:3 1Th 2:3 1Ti 4:13 Heb 10:25; 13:22}
\crossref{Rom}{12}{9}{2Sa 20:9,\allowbreak10 Ps 55:21 Pr 26:25 Eze 33:31 Mt 26:49 Joh 12:6}
\crossref{Rom}{12}{10}{Joh 13:34,\allowbreak35; 15:17; 17:21 Ac 4:32 Ga 5:6,\allowbreak13,\allowbreak22 Eph 4:1-\allowbreak3}
\crossref{Rom}{12}{11}{Ex 5:17 Pr 6:6-\allowbreak9; 10:26; 13:4; 18:9; 22:29; 24:30-\allowbreak34; 26:13-\allowbreak16}
\crossref{Rom}{12}{12}{Ro 5:2,\allowbreak3; 15:13 Ps 16:9-\allowbreak11; 71:20-\allowbreak23; 73:24-\allowbreak26 Pr 10:28; 14:32}
\crossref{Rom}{12}{13}{12:8; 15:25-\allowbreak28 Ps 41:1 Ac 4:35; 9:36-\allowbreak41; 10:4; 20:34,\allowbreak35 1Co 16:1,\allowbreak2}
\crossref{Rom}{12}{14}{12:21 Job 31:29,\allowbreak30 Mt 5:44 Lu 6:28; 23:34 Ac 7:60 1Co 4:12,\allowbreak13}
\crossref{Rom}{12}{15}{Isa 66:10-\allowbreak14 Lu 1:58; 15:5-\allowbreak10 Ac 11:23 1Co 12:26 2Co 2:3}
\crossref{Rom}{12}{16}{Ro 15:5; 6:2 2Ch 30:12 Jer 32:39 Ac 4:32 1Co 1:10 Php 1:27; 2:2,\allowbreak3}
\crossref{Rom}{12}{17}{12:19 Pr 20:22 Mt 5:39 1Th 5:15 1Pe 3:9}
\crossref{Rom}{12}{18}{Ro 14:17,\allowbreak19 2Sa 20:19 Ps 34:14; 120:5-\allowbreak7 Pr 12:20 Mt 5:5,\allowbreak9 Mr 9:50}
\crossref{Rom}{12}{19}{12:14,\allowbreak17 Le 19:18 1Sa 25:26,\allowbreak33 Pr 24:17-\allowbreak19,\allowbreak29 Eze 25:12}
\crossref{Rom}{12}{20}{Ex 23:4,\allowbreak5 1Sa 24:16-\allowbreak19; 26:21 Pr 25:21,\allowbreak22 Mt 5:44}
\crossref{Rom}{12}{21}{Pr 16:32 Lu 6:27-\allowbreak30 1Pe 3:9}
\crossref{Rom}{13}{1}{De 17:12 Eph 5:21 Tit 3:1 1Pe 2:13-\allowbreak17 2Pe 2:10,\allowbreak11 Jude 1:8}
\crossref{Rom}{13}{2}{Jer 23:8-\allowbreak17; 44:14-\allowbreak17 Tit 3:1}
\crossref{Rom}{13}{3}{13:4 De 25:1 Pr 14:35; 20:2 Ec 10:4-\allowbreak6 Jer 22:15-\allowbreak18}
\crossref{Rom}{13}{4}{13:6 1Ki 10:9 2Ch 19:6 Ps 82:2-\allowbreak4 Pr 24:23,\allowbreak24; 31:8,\allowbreak9 Ec 8:2-\allowbreak5}
\crossref{Rom}{13}{5}{1Sa 24:5,\allowbreak6 Ec 8:2 Tit 3:1,\allowbreak2 1Pe 2:13-\allowbreak15}
\crossref{Rom}{13}{6}{Ezr 4:13,\allowbreak20; 6:8 Ne 5:4 Mt 17:24-\allowbreak27; 22:17-\allowbreak21 Mr 12:14-\allowbreak17}
\crossref{Rom}{13}{7}{Lu 20:25}
\crossref{Rom}{13}{8}{13:7 De 24:14,\allowbreak15 Pr 3:27,\allowbreak28 Mt 7:12; 22:39,\allowbreak40}
\crossref{Rom}{13}{9}{Ex 20:12-\allowbreak17 De 5:16-\allowbreak21 Mt 19:18,\allowbreak19 Mr 10:19 Lu 18:20}
\crossref{Rom}{13}{10}{1Co 13:4-\allowbreak7}
\crossref{Rom}{13}{11}{Isa 21:11,\allowbreak12 Mt 16:3; 24:42-\allowbreak44 1Th 5:1-\allowbreak3}
\crossref{Rom}{13}{12}{So 2:17 1Jo 2:8}
\crossref{Rom}{13}{13}{Lu 1:6 Ga 5:16,\allowbreak25 Eph 4:1,\allowbreak17; 5:2,\allowbreak8,\allowbreak15 Php 1:27; 3:16-\allowbreak20; 4:8,\allowbreak9}
\crossref{Rom}{13}{14}{Ga 3:27 Eph 4:24 Col 3:10-\allowbreak12}
\crossref{Rom}{14}{1}{14:21; 4:19; 15:1,\allowbreak7 Job 4:3 Isa 35:3,\allowbreak4; 40:11; 42:3 Eze 34:4,\allowbreak16}
\crossref{Rom}{14}{2}{14:14 1Co 10:25 Ga 2:12 1Ti 4:4 Tit 1:15 Heb 9:10; 13:9}
\crossref{Rom}{14}{3}{14:10,\allowbreak15,\allowbreak21 Zec 4:10 Mt 18:10 Lu 18:9 1Co 8:11-\allowbreak13}
\crossref{Rom}{14}{4}{Ro 9:20 Ac 11:17 1Co 4:4,\allowbreak5 Jas 4:11,\allowbreak12}
\crossref{Rom}{14}{5}{Ga 4:9,\allowbreak10 Col 2:16,\allowbreak17}
\crossref{Rom}{14}{6}{Ga 4:10}
\crossref{Rom}{14}{7}{14:9 1Co 6:19,\allowbreak20 2Co 5:15 Ga 2:19,\allowbreak20}
\crossref{Rom}{14}{8}{Joh 21:19 Ac 13:36; 20:24; 21:13 Php 2:17,\allowbreak30 1Th 5:10}
\crossref{Rom}{14}{9}{Isa 53:10-\allowbreak12 Lu 24:26 2Co 5:14 Heb 12:2 1Pe 1:21 Re 1:18}
\crossref{Rom}{14}{10}{14:3,\allowbreak4 Lu 23:11 Ac 4:11}
\crossref{Rom}{14}{11}{Nu 14:21,\allowbreak28 Isa 49:18 Jer 22:24 Eze 5:11 Zep 2:9}
\crossref{Rom}{14}{12}{Ec 11:9 Mt 12:36; 18:23 etc.}
\crossref{Rom}{14}{13}{14:4,\allowbreak10 Jas 2:4; 4:11}
\crossref{Rom}{14}{14}{Ac 10:28}
\crossref{Rom}{14}{15}{Eze 13:22 1Co 8:12}
\crossref{Rom}{14}{16}{Ro 12:17 1Co 10:29,\allowbreak30 2Co 8:20,\allowbreak21 1Th 5:22}
\crossref{Rom}{14}{17}{Da 2:44 Mt 3:2; 6:33 Lu 14:15; 17:20,\allowbreak21 Joh 3:3,\allowbreak5 1Co 4:20; 6:9}
\crossref{Rom}{14}{18}{14:4; 6:22; 12:11; 16:18 Mr 13:34 Joh 12:26 1Co 7:22 Ga 6:15,\allowbreak16}
\crossref{Rom}{14}{19}{Ro 12:18 Ps 34:14; 133:1 Mt 5:9 Mr 9:50 2Co 13:11 Eph 4:3-\allowbreak7}
\crossref{Rom}{14}{20}{14:15 Mt 18:6 1Co 6:12,\allowbreak13; 8:8,\allowbreak13; 10:31}
\crossref{Rom}{14}{21}{14:17; 15:1,\allowbreak2 1Co 8:13}
\crossref{Rom}{14}{22}{14:2,\allowbreak5,\allowbreak14,\allowbreak23 Ga 6:1 Jas 3:13}
\crossref{Rom}{14}{23}{1Co 8:7}
\crossref{Rom}{15}{1}{Ro 4:20 1Co 4:10 2Co 12:10 Eph 6:10 2Ti 2:1 1Jo 2:14}
\crossref{Rom}{15}{2}{Ro 14:19 1Co 9:19-\allowbreak22; 10:24,\allowbreak33; 11:1; 13:5 Php 2:4,\allowbreak5 Tit 2:9,\allowbreak10}
\crossref{Rom}{15}{3}{Ps 40:6-\allowbreak8 Mt 26:39,\allowbreak42 Joh 4:34; 5:30; 6:38; 8:29; 12:27,\allowbreak28; 14:30}
\crossref{Rom}{15}{4}{Ro 4:23,\allowbreak24 1Co 9:9,\allowbreak10; 10:11 2Ti 3:16,\allowbreak17 2Pe 1:20,\allowbreak21}
\crossref{Rom}{15}{5}{15:13 Ex 34:6 Ps 86:5 1Pe 3:20 2Pe 3:9,\allowbreak15}
\crossref{Rom}{15}{6}{15:9-\allowbreak11 Zep 3:9 Zec 13:9 Ac 4:24,\allowbreak32}
\crossref{Rom}{15}{7}{Ro 14:1-\allowbreak3 Mt 10:40 Mr 9:37 Lu 9:48}
\crossref{Rom}{15}{8}{Ro 3:26 1Co 1:12; 10:19,\allowbreak29; 15:50}
\crossref{Rom}{15}{9}{2Sa 22:50 Ps 18:49}
\crossref{Rom}{15}{10}{De 32:43 Ps 66:1-\allowbreak4; 67:3,\allowbreak4; 68:32; 97:1; 98:3,\allowbreak4; 138:4,\allowbreak5}
\crossref{Rom}{15}{11}{Ps 117:1}
\crossref{Rom}{15}{12}{Isa 11:1,\allowbreak10 Re 5:5; 22:16}
\crossref{Rom}{15}{13}{15:5 Jer 14:8 Joe 3:16 1Ti 1:1}
\crossref{Rom}{15}{14}{Php 1:7 2Ti 1:5 Phm 1:21 Heb 6:9 2Pe 1:12 1Jo 2:21}
\crossref{Rom}{15}{15}{Heb 13:22 1Pe 5:12 1Jo 2:12-\allowbreak14; 5:13 Jude 1:3-\allowbreak5}
\crossref{Rom}{15}{16}{15:18; 11:13 Ac 9:15; 13:2; 22:21; 26:17,\allowbreak18 1Co 3:5; 4:1 2Co 5:20; 11:23}
\crossref{Rom}{15}{17}{Ro 4:2 2Co 2:14-\allowbreak16; 3:4-\allowbreak6; 7:4; 11:16-\allowbreak30; 12:1,\allowbreak11 etc.}
\crossref{Rom}{15}{18}{Pr 25:14 2Co 10:13-\allowbreak18; 11:31; 12:6 Jude 1:9}
\crossref{Rom}{15}{19}{Ac 14:10; 15:12; 16:18; 19:11,\allowbreak12 2Co 12:12 Ga 3:5 Heb 2:4}
\crossref{Rom}{15}{20}{2Co 10:14-\allowbreak16}
\crossref{Rom}{15}{21}{Isa 52:15; 65:1}
\crossref{Rom}{15}{22}{Ro 1:13 1Th 2:17,\allowbreak18}
\crossref{Rom}{15}{23}{15:32; 1:10-\allowbreak12 1Th 3:10 2Ti 1:4}
\crossref{Rom}{15}{24}{15:28 Ac 19:21}
\crossref{Rom}{15}{25}{15:26-\allowbreak31 Ac 18:21; 19:21; 20:16,\allowbreak22; 24:17 1Co 16:1-\allowbreak3 Ga 2:10}
\crossref{Rom}{15}{26}{Ac 11:27-\allowbreak30 2Co 8:1-\allowbreak9:15 Ga 6:6-\allowbreak10}
\crossref{Rom}{15}{27}{Ro 11:17 1Co 9:11 Ga 6:6 Phm 1:19}
\crossref{Rom}{15}{28}{Php 4:17 Col 1:6}
\crossref{Rom}{15}{29}{Ro 1:11,\allowbreak12 Ps 16:11 Eze 34:26 Eph 1:3; 3:8,\allowbreak19; 4:13}
\crossref{Rom}{15}{30}{2Co 4:5,\allowbreak11; 12:10 1Ti 6:13,\allowbreak14 2Ti 4:1}
\crossref{Rom}{15}{31}{Ac 21:27-\allowbreak31; 22:24; 23:12-\allowbreak24; 24:1-\allowbreak9; 25:2,\allowbreak24 1Th 2:15 2Th 3:2}
\crossref{Rom}{15}{32}{15:23,\allowbreak24; 1:10-\allowbreak13 Ac 27:1,\allowbreak41-\allowbreak43; 28:15,\allowbreak16,\allowbreak30,\allowbreak31 Php 1:12-\allowbreak14}
\crossref{Rom}{15}{33}{Ro 16:20 1Co 14:33 2Co 5:19,\allowbreak20; 13:11 Php 4:9 1Th 5:23 2Th 3:16}
\crossref{Rom}{16}{1}{2Co 3:1}
\crossref{Rom}{16}{2}{Ro 15:7 Mt 10:40-\allowbreak42; 25:40 Php 2:29 Col 4:10 Phm 1:12,\allowbreak17 2Jo 1:10}
\crossref{Rom}{16}{3}{Ac 18:2 etc.}
\crossref{Rom}{16}{4}{Ro 5:7 Joh 15:13 Php 2:30 1Jo 3:16}
\crossref{Rom}{16}{5}{Mt 18:20 1Co 16:19 Col 4:15 Phm 1:2}
\crossref{Rom}{16}{6}{16:12 Mt 27:55 1Ti 5:10}
\crossref{Rom}{16}{7}{16:11,\allowbreak21}
\crossref{Rom}{16}{8}{16:5 Php 4:1 1Jo 3:14}
\crossref{Rom}{16}{9}{16:2,\allowbreak3,\allowbreak21}
\crossref{Rom}{16}{10}{Ro 14:18 De 8:2 1Co 11:19 2Co 2:9; 8:22 Php 2:22 1Ti 3:10 1Pe 1:7}
\crossref{Rom}{16}{11}{}
\crossref{Rom}{16}{12}{Mt 9:38 1Co 15:10,\allowbreak58; 16:16 Col 1:29; 4:12 1Th 1:3; 5:12,\allowbreak13}
\crossref{Rom}{16}{13}{Mr 15:21}
\crossref{Rom}{16}{14}{Ro 8:29 Col 1:2 Heb 3:1 1Pe 1:22,\allowbreak23}
\crossref{Rom}{16}{15}{16:2; 1:7 Isa 60:21 Eph 1:1 1Pe 1:2}
\crossref{Rom}{16}{16}{Ac 20:37 1Co 16:20 2Co 13:12 1Th 5:26 1Pe 5:14}
\crossref{Rom}{16}{17}{Php 3:17 2Th 3:14,\allowbreak15}
\crossref{Rom}{16}{18}{Mt 6:24 Joh 12:26 Ga 1:10 Php 2:21 Col 3:24 Jas 1:1 Jude 1:1}
\crossref{Rom}{16}{19}{Ro 1:8 1Th 1:8,\allowbreak9}
\crossref{Rom}{16}{20}{Ro 15:33}
\crossref{Rom}{16}{21}{Ac 16:1-\allowbreak3; 17:14; 18:5; 19:22; 20:4 2Co 1:1,\allowbreak19 Col 1:1 Php 1:1}
\crossref{Rom}{16}{22}{Ga 6:11}
\crossref{Rom}{16}{23}{1Co 1:14 3Jo 1:1-\allowbreak6}
\crossref{Rom}{16}{24}{16:20 1Th 5:28}
\crossref{Rom}{16}{25}{Ro 14:4 Ac 20:32 Eph 3:20,\allowbreak21 1Th 3:13 2Th 2:16,\allowbreak17; 3:3 Heb 7:25}
\crossref{Rom}{16}{26}{Eph 1:9 Col 1:26 2Ti 1:10 Tit 1:2,\allowbreak3}
\crossref{Rom}{16}{27}{Ro 11:36 Ga 1:4,\allowbreak5 Eph 3:20,\allowbreak21 Php 4:20 1Ti 1:17; 6:16 2Ti 4:18}

% 1Cor
\crossref{1Cor}{1}{1}{Ro 1:1 Ga 2:7,\allowbreak8}
\crossref{1Cor}{1}{2}{Ac 18:1,\allowbreak8-\allowbreak11 2Co 1:1 Ga 1:2 1Th 1:1 2Th 1:1 1Ti 3:15}
\crossref{1Cor}{1}{3}{Ro 1:7 2Co 1:2 Eph 1:2 1Pe 1:2}
\crossref{1Cor}{1}{4}{Ro 1:8; 6:17 Ac 11:23; 21:20}
\crossref{1Cor}{1}{5}{1Co 4:7-\allowbreak10 Ro 11:12 2Co 9:11 Eph 2:7; 3:8}
\crossref{1Cor}{1}{6}{1Co 2:1,\allowbreak2 Ac 18:5; 20:21,\allowbreak24; 22:18; 23:11; 28:23 1Ti 2:6 2Ti 1:8}
\crossref{1Cor}{1}{7}{2Co 12:13}
\crossref{1Cor}{1}{8}{Ps 37:17,\allowbreak28 Ro 14:4; 16:25 2Co 1:21 1Th 3:13; 5:24 2Th 3:3}
\crossref{1Cor}{1}{9}{1Co 10:13 Nu 23:19 De 7:9; 32:4 Ps 89:33-\allowbreak35; 100:5 Isa 11:5; 25:1}
\crossref{1Cor}{1}{10}{1Co 4:16 Ro 12:1 2Co 5:20; 6:1; 10:1 Ga 4:12 Eph 4:1 Phm 1:9,\allowbreak10}
\crossref{1Cor}{1}{11}{1Co 11:18 Ge 27:42; 37:2 1Sa 25:14-\allowbreak17}
\crossref{1Cor}{1}{12}{1Co 7:29; 15:50 2Co 9:6 Ga 3:17}
\crossref{1Cor}{1}{13}{2Co 11:4 Ga 1:7 Eph 4:5}
\crossref{1Cor}{1}{14}{1:4; 14:18 2Co 2:14 Eph 5:20 Col 3:15,\allowbreak17 1Th 5:18 1Ti 1:12 Phm 1:4}
\crossref{1Cor}{1}{15}{Joh 3:28,\allowbreak29; 7:18 2Co 11:2}
\crossref{1Cor}{1}{16}{1Co 16:15,\allowbreak17 Ac 16:15,\allowbreak33}
\crossref{1Cor}{1}{17}{Joh 4:2 Ac 10:48; 26:17,\allowbreak18}
\crossref{1Cor}{1}{18}{1:23,\allowbreak24; 2:2 Ga 6:12-\allowbreak14}
\crossref{1Cor}{1}{19}{1Co 3:19 Job 5:12,\allowbreak13 Isa 19:3,\allowbreak11; 29:14 Jer 8:9}
\crossref{1Cor}{1}{20}{Isa 33:18; 53:1}
\crossref{1Cor}{1}{21}{1:24 Da 2:20 Ro 11:33 Eph 3:10}
\crossref{1Cor}{1}{22}{Mt 12:38,\allowbreak39; 16:1-\allowbreak4 Mr 8:11 Lu 11:16,\allowbreak20 Joh 2:18; 4:28}
\crossref{1Cor}{1}{23}{1:18; 2:2 Lu 24:46,\allowbreak47 Ac 7:32-\allowbreak35; 10:39-\allowbreak43 2Co 4:5 Ga 3:1; 6:14}
\crossref{1Cor}{1}{24}{1:2,\allowbreak9 Lu 7:35 Ro 8:28-\allowbreak30; 9:24}
\crossref{1Cor}{1}{25}{1:18,\allowbreak27-\allowbreak29 Ex 13:17; 14:2-\allowbreak4 Jos 6:2-\allowbreak5 Jud 7:2-\allowbreak8; 15:15,\allowbreak16}
\crossref{1Cor}{1}{26}{1:20; 2:3-\allowbreak6,\allowbreak13; 3:18-\allowbreak20 Zep 3:12 Mt 11:25,\allowbreak26 Lu 10:21 Joh 7:47-\allowbreak49}
\crossref{1Cor}{1}{27}{Ps 8:2 Isa 26:5,\allowbreak6; 29:14,\allowbreak19 Zep 3:12 Mt 4:18-\allowbreak22; 9:9; 11:25}
\crossref{1Cor}{1}{28}{Ro 4:17 2Co 12:11}
\crossref{1Cor}{1}{29}{1:31; 4:7; 5:6 Ps 49:6 Isa 10:15 Jer 9:23 Ro 3:19,\allowbreak27; 4:2; 15:17}
\crossref{1Cor}{1}{30}{1Co 12:18,\allowbreak27 Isa 45:17 Joh 15:1-\allowbreak6; 17:21-\allowbreak23 Ro 8:1; 12:5; 16:7,\allowbreak11}
\crossref{1Cor}{1}{31}{1Ch 16:10,\allowbreak35 Ps 105:3 Isa 41:16; 45:25 Jer 4:2; 9:23,\allowbreak24}
\crossref{1Cor}{2}{1}{Ac 18:1-\allowbreak4}
\crossref{1Cor}{2}{2}{1Co 1:22-\allowbreak25 Joh 17:3 Ga 3:1; 6:14 Php 3:8-\allowbreak10}
\crossref{1Cor}{2}{3}{1Co 4:10-\allowbreak13 Ac 17:1,\allowbreak6-\allowbreak12; 20:18,\allowbreak19 2Co 4:1,\allowbreak7-\allowbreak12,\allowbreak16; 6:4; 7:5; 10:1,\allowbreak10}
\crossref{1Cor}{2}{4}{Ac 20:27}
\crossref{1Cor}{2}{5}{1Co 1:17; 3:6 Ac 16:14 2Co 4:7; 6:7}
\crossref{1Cor}{2}{6}{1Co 14:20}
\crossref{1Cor}{2}{7}{Ps 78:2 Isa 48:6,\allowbreak7 Mt 11:25; 13:35 Ro 16:25,\allowbreak26 Eph 1:4; 3:4-\allowbreak9}
\crossref{1Cor}{2}{8}{2:6; 1:26-\allowbreak28 Mt 11:25 Joh 7:48}
\crossref{1Cor}{2}{9}{Ps 31:19 Mt 20:23; 25:34 Heb 11:16}
\crossref{1Cor}{2}{10}{1Co 14:30 Am 3:7 Mt 11:25-\allowbreak27; 13:11; 16:17 Lu 2:26; 10:21 Eph 3:3,\allowbreak5}
\crossref{1Cor}{2}{11}{Pr 14:10; 20:5,\allowbreak27 Jer 17:9}
\crossref{1Cor}{2}{12}{2:6 Ro 8:1,\allowbreak5,\allowbreak6 2Co 4:4 Eph 2:2 Jas 4:5 1Jo 4:4,\allowbreak5; 5:19 Re 12:9}
\crossref{1Cor}{2}{13}{2:4; 1:17 2Pe 1:16}
\crossref{1Cor}{2}{14}{Mt 13:11 etc.}
\crossref{1Cor}{2}{15}{1Co 3:1; 14:37 Ga 6:1 Col 1:9}
\crossref{1Cor}{2}{16}{Job 15:8; 22:2; 40:2 Isa 40:13,\allowbreak14 Jer 23:18 Ro 11:34}
\crossref{1Cor}{3}{1}{1Co 2:6,\allowbreak15 Ga 6:1}
\crossref{1Cor}{3}{2}{Heb 5:12-\allowbreak14 1Pe 2:2}
\crossref{1Cor}{3}{3}{1Co 1:11; 6:1-\allowbreak8; 11:18 2Co 12:20 Ga 5:15,\allowbreak19-\allowbreak21 Jas 3:16; 4:1,\allowbreak2}
\crossref{1Cor}{3}{4}{1Co 1:12; 4:6}
\crossref{1Cor}{3}{5}{3:7; 4:1,\allowbreak2 Lu 1:2 Ro 10:14,\allowbreak15 2Co 3:6; 4:5,\allowbreak7; 6:1,\allowbreak4; 11:23}
\crossref{1Cor}{3}{6}{3:9,\allowbreak10; 4:14,\allowbreak15; 9:1,\allowbreak7-\allowbreak11; 15:1-\allowbreak11 Ac 18:4-\allowbreak11 2Co 10:14,\allowbreak15}
\crossref{1Cor}{3}{7}{1Co 13:2 Ps 115:1 Isa 40:17; 41:29 Da 4:35 Joh 15:5 2Co 12:9 Ga 6:3}
\crossref{1Cor}{3}{8}{3:9; 4:6 Joh 4:36-\allowbreak38}
\crossref{1Cor}{3}{9}{3:6 Mt 9:37 Mr 16:20 2Co 6:1 3Jo 1:8}
\crossref{1Cor}{3}{10}{3:5; 15:10 Ro 1:5; 12:3; 15:15 Eph 3:2-\allowbreak8 Col 1:29 1Ti 1:11-\allowbreak14}
\crossref{1Cor}{3}{11}{Isa 28:16 Mt 16:18 Ac 4:11,\allowbreak12 2Co 11:2-\allowbreak4 Ga 1:7-\allowbreak9 Eph 2:20}
\crossref{1Cor}{3}{12}{Ps 19:10; 119:72 Pr 8:10; 16:16 Isa 60:17 1Ti 4:6 2Ti 2:20}
\crossref{1Cor}{3}{13}{3:14,\allowbreak15; 4:5 2Ti 3:9}
\crossref{1Cor}{3}{14}{3:8; 4:5 Da 12:3 Mt 24:45-\allowbreak47; 25:21-\allowbreak23 1Th 2:19 2Ti 4:7 1Pe 5:1,\allowbreak4}
\crossref{1Cor}{3}{15}{3:12,\allowbreak13 Re 3:18}
\crossref{1Cor}{3}{16}{1Co 5:6; 6:2,\allowbreak3,\allowbreak9,\allowbreak16,\allowbreak19; 9:13,\allowbreak24 Ro 6:3 Jas 4:4}
\crossref{1Cor}{3}{17}{1Co 6:18-\allowbreak20 Le 15:31; 20:3 Nu 19:20 Ps 74:3; 79:1 Eze 5:11; 7:22}
\crossref{1Cor}{3}{18}{1Co 6:9; 15:33 Pr 5:7 Isa 44:20 Jer 37:9 Lu 21:8 Ga 6:3,\allowbreak7 Eph 5:6}
\crossref{1Cor}{3}{19}{1Co 1:19,\allowbreak20; 2:6 Isa 19:11-\allowbreak14; 29:14-\allowbreak16; 44:25 Ro 1:21,\allowbreak22}
\crossref{1Cor}{3}{20}{Ps 94:11}
\crossref{1Cor}{3}{21}{3:4-\allowbreak7; 1:12 etc.}
\crossref{1Cor}{3}{22}{3:5-\allowbreak8; 9:19-\allowbreak22 2Co 4:5 Eph 4:11,\allowbreak12}
\crossref{1Cor}{3}{23}{1Co 6:19,\allowbreak20; 7:22; 15:23 Joh 17:9,\allowbreak10 Ro 14:8 2Co 10:7 Ga 3:29; 5:24}
\crossref{1Cor}{4}{1}{4:13 2Co 12:6}
\crossref{1Cor}{4}{2}{4:17; 7:25 Nu 12:7 Pr 13:17 Mt 25:21,\allowbreak23 Lu 12:42; 16:10-\allowbreak12 2Co 2:17}
\crossref{1Cor}{4}{3}{1Co 2:15 1Sa 16:7 Joh 7:24}
\crossref{1Cor}{4}{4}{Job 27:6 Ps 7:3-\allowbreak5 Joh 21:17 2Co 1:12 1Jo 3:20,\allowbreak21}
\crossref{1Cor}{4}{5}{Mt 7:1,\allowbreak2 Lu 6:37 Ro 2:1,\allowbreak16; 14:4,\allowbreak10-\allowbreak13 Jas 4:11}
\crossref{1Cor}{4}{6}{1Co 1:12; 3:4-\allowbreak7 2Co 10:7,\allowbreak12,\allowbreak15; 11:4,\allowbreak12-\allowbreak15}
\crossref{1Cor}{4}{7}{1Co 12:4-\allowbreak11; 15:10 Ro 9:16-\allowbreak18 Eph 3:3-\allowbreak5 2Th 2:12-\allowbreak14 1Ti 1:12-\allowbreak15}
\crossref{1Cor}{4}{8}{1Co 1:5; 3:1,\allowbreak2; 5:6 Pr 13:7; 25:14 Isa 5:21 Lu 1:51-\allowbreak53; 6:25 Ro 12:3}
\crossref{1Cor}{4}{9}{1Co 15:30-\allowbreak32 2Co 1:8-\allowbreak10; 4:8-\allowbreak12; 6:9 Php 1:29,\allowbreak30 1Th 3:3}
\crossref{1Cor}{4}{10}{1Co 1:1 etc.}
\crossref{1Cor}{4}{11}{1Co 9:4 2Co 4:8; 6:4,\allowbreak5; 11:26,\allowbreak27 Php 4:12}
\crossref{1Cor}{4}{12}{1Co 9:6 Ac 18:3; 20:34 1Th 2:9 2Th 3:8 1Ti 4:10}
\crossref{1Cor}{4}{13}{La 3:45 Ac 22:22}
\crossref{1Cor}{4}{14}{1Co 9:15 2Co 7:3; 12:19}
\crossref{1Cor}{4}{15}{2Ti 4:3}
\crossref{1Cor}{4}{16}{1Co 11:1 Joh 10:4,\allowbreak5 Php 3:17 1Th 1:6 2Th 3:9 Heb 13:7 1Pe 5:3}
\crossref{1Cor}{4}{17}{1Co 16:10 Ac 19:21,\allowbreak22 Php 2:19 1Th 3:2,\allowbreak3}
\crossref{1Cor}{4}{18}{4:6-\allowbreak8; 5:2}
\crossref{1Cor}{4}{19}{1Co 14:5 Ac 19:21 2Co 1:15,\allowbreak17,\allowbreak23; 2:1,\allowbreak2}
\crossref{1Cor}{4}{20}{1Co 1:24; 2:4 Ro 1:16; 14:17; 15:19 2Co 10:4,\allowbreak5 1Th 1:5}
\crossref{1Cor}{4}{21}{1Co 5:5 2Co 10:2,\allowbreak6,\allowbreak8; 12:20,\allowbreak21; 13:2; 3:10}
\crossref{1Cor}{5}{1}{1Co 1:11 Ge 37:2 1Sa 2:24}
\crossref{1Cor}{5}{2}{5:6; 4:6-\allowbreak8,\allowbreak18}
\crossref{1Cor}{5}{3}{2Co 10:1,\allowbreak11; 13:2 Col 2:5 1Th 2:17}
\crossref{1Cor}{5}{4}{Ac 3:6; 4:7-\allowbreak12,\allowbreak30; 16:18 Eph 5:20 Col 3:17}
\crossref{1Cor}{5}{5}{5:13 Job 2:6 Ps 109:6 2Co 2:6; 10:6; 13:10 Ac 26:18 1Ti 1:20}
\crossref{1Cor}{5}{6}{5:2; 3:21; 4:18,\allowbreak19 Jas 4:16}
\crossref{1Cor}{5}{7}{5:13 Ex 12:15; 13:6,\allowbreak7 Eph 4:22 Col 3:5-\allowbreak9}
\crossref{1Cor}{5}{8}{Ex 12:15; 13:6 Le 23:6 Nu 28:16,\allowbreak17 De 16:16 Isa 25:6}
\crossref{1Cor}{5}{9}{5:2,\allowbreak7 Ps 1:1,\allowbreak2 Pr 9:6 2Co 6:14,\allowbreak17 Eph 5:11 2Th 3:14}
\crossref{1Cor}{5}{10}{1Co 10:27}
\crossref{1Cor}{5}{11}{1Co 6:6; 7:12,\allowbreak15; 8:11 Mt 18:17 Ac 9:17 Ro 16:17 2Th 3:6,\allowbreak14 2Jo 1:10}
\crossref{1Cor}{5}{12}{Lu 12:14 Joh 18:36}
\crossref{1Cor}{5}{13}{Ps 50:6 Ac 17:31 Ro 2:16 Heb 13:4 2Pe 2:9}
\crossref{1Cor}{6}{1}{Mt 18:15-\allowbreak17 Ac 18:14,\allowbreak15; 19:38}
\crossref{1Cor}{6}{2}{Ps 49:14; 149:5-\allowbreak9 Da 7:18,\allowbreak22 Zec 14:5 Mt 19:28 Lu 22:30}
\crossref{1Cor}{6}{3}{Mt 25:41 2Pe 2:4 Jude 1:6}
\crossref{1Cor}{6}{4}{1Co 5:12}
\crossref{1Cor}{6}{5}{1Co 4:14; 11:14; 15:34}
\crossref{1Cor}{6}{6}{6:1,\allowbreak7 Ge 13:7-\allowbreak9; 45:24 Ne 5:8,\allowbreak9 Ps 133:1-\allowbreak3 Ac 7:26 Php 2:14,\allowbreak15}
\crossref{1Cor}{6}{7}{Pr 2:5,\allowbreak8-\allowbreak10 Ho 10:2 Jas 4:1-\allowbreak3}
\crossref{1Cor}{6}{8}{Le 19:13 Mic 2:2 Mal 3:5}
\crossref{1Cor}{6}{9}{6:2,\allowbreak3,\allowbreak15,\allowbreak16,\allowbreak19; 3:16; 9:24}
\crossref{1Cor}{6}{10}{Ps 50:17,\allowbreak18 Isa 1:23 Jer 7:11 Eze 22:13,\allowbreak27,\allowbreak29 Mt 21:19}
\crossref{1Cor}{6}{11}{1Co 12:2 Ro 6:17-\allowbreak19 Eph 2:1-\allowbreak3; 4:17-\allowbreak22; 5:8 Col 3:5-\allowbreak7 Tit 3:3-\allowbreak6}
\crossref{1Cor}{6}{12}{1Co 10:23 Ro 14:14}
\crossref{1Cor}{6}{13}{Mt 15:17,\allowbreak20 Mr 7:19 Ro 14:17}
\crossref{1Cor}{6}{14}{1Co 15:15-\allowbreak20 Ac 2:24; 17:31 Ro 6:4-\allowbreak8; 8:11 2Co 4:14 Php 3:10,\allowbreak11}
\crossref{1Cor}{6}{15}{6:19; 11:3; 12:27 Ro 12:5 Eph 1:22,\allowbreak23; 4:12,\allowbreak15,\allowbreak16; 5:23,\allowbreak30 Col 2:19}
\crossref{1Cor}{6}{16}{Ge 34:31; 38:15,\allowbreak24 Jud 16:1 Mt 21:31,\allowbreak32 Heb 11:31}
\crossref{1Cor}{6}{17}{1Co 12:13 Joh 3:6; 17:21-\allowbreak23 Eph 4:3,\allowbreak4; 5:30 Php 2:5}
\crossref{1Cor}{6}{18}{Ge 39:12-\allowbreak18 Pr 2:16-\allowbreak19; 5:3-\allowbreak15; 6:24-\allowbreak32; 7:5 etc.}
\crossref{1Cor}{6}{19}{6:15,\allowbreak16}
\crossref{1Cor}{6}{20}{1Co 7:23 Ac 20:28 Ga 3:13 Heb 9:12 1Pe 1:18 2Pe 2:1 Re 5:9}
\crossref{1Cor}{7}{1}{7:8,\allowbreak26,\allowbreak27,\allowbreak37,\allowbreak38 Mt 19:10,\allowbreak11}
\crossref{1Cor}{7}{2}{7:9; 6:18 Pr 5:18,\allowbreak19 1Ti 4:3}
\crossref{1Cor}{7}{3}{Ex 21:10 1Pe 3:7}
\crossref{1Cor}{7}{4}{Ho 3:3 Mt 19:9 Mr 10:11,\allowbreak12}
\crossref{1Cor}{7}{5}{Ex 19:15 1Sa 21:4,\allowbreak5 Joe 2:16 Zec 7:3; 12:12-\allowbreak14}
\crossref{1Cor}{7}{6}{7:12,\allowbreak25 2Co 8:8; 11:17}
\crossref{1Cor}{7}{7}{1Co 12:11 Mt 19:11,\allowbreak12}
\crossref{1Cor}{7}{8}{1Co 7:26,\allowbreak27,\allowbreak32,\allowbreak34,\allowbreak35}
\crossref{1Cor}{7}{9}{7:2,\allowbreak28,\allowbreak36,\allowbreak39 1Ti 5:11,\allowbreak14}
\crossref{1Cor}{7}{10}{7:12,\allowbreak25,\allowbreak40}
\crossref{1Cor}{7}{11}{Jud 19:2,\allowbreak3 Jer 3:1}
\crossref{1Cor}{7}{12}{7:6,\allowbreak25 2Co 11:17}
\crossref{1Cor}{7}{13}{}
\crossref{1Cor}{7}{14}{1Co 6:15-\allowbreak17 Ezr 9:1,\allowbreak2 1Ti 4:5 Tit 1:15}
\crossref{1Cor}{7}{15}{Mt 12:50 Jas 2:15}
\crossref{1Cor}{7}{16}{1Co 9:22 Pr 11:30 Lu 15:10 1Ti 4:16 Jas 5:19,\allowbreak20 1Pe 3:1,\allowbreak2}
\crossref{1Cor}{7}{17}{7:7 Mt 19:12 Ro 12:3-\allowbreak8 1Pe 4:10,\allowbreak11}
\crossref{1Cor}{7}{18}{Ac 15:1,\allowbreak5,\allowbreak19,\allowbreak24,\allowbreak28 Ga 5:1-\allowbreak3 Col 3:11}
\crossref{1Cor}{7}{19}{1Co 8:8 Ro 2:25-\allowbreak29; 3:30 Ga 5:6; 6:15}
\crossref{1Cor}{7}{20}{7:17,\allowbreak21-\allowbreak23 Pr 27:8 Lu 3:10-\allowbreak14 1Th 4:11 2Th 3:12}
\crossref{1Cor}{7}{21}{1Co 12:13 Ga 3:28 Col 3:11 1Ti 6:1-\allowbreak3 1Pe 2:18-\allowbreak24}
\crossref{1Cor}{7}{22}{Lu 1:74,\allowbreak75 Joh 8:32-\allowbreak36 Ro 6:18-\allowbreak22 Ga 5:1,\allowbreak13 Eph 6:5,\allowbreak6}
\crossref{1Cor}{7}{23}{1Co 6:20 Le 25:42 Ac 20:28 Tit 2:14 1Pe 1:18,\allowbreak19; 3:18 Re 5:9}
\crossref{1Cor}{7}{24}{7:17,\allowbreak20}
\crossref{1Cor}{7}{25}{7:28,\allowbreak34,\allowbreak36-\allowbreak38 Ps 78:63}
\crossref{1Cor}{7}{26}{7:1,\allowbreak8,\allowbreak28,\allowbreak35-\allowbreak38 Jer 16:2-\allowbreak4 Mt 24:19 Lu 21:23; 23:28,\allowbreak29 1Pe 4:17}
\crossref{1Cor}{7}{27}{7:12-\allowbreak14,\allowbreak20}
\crossref{1Cor}{7}{28}{7:36 Heb 13:4}
\crossref{1Cor}{7}{29}{Job 14:1,\allowbreak2 Ps 39:4-\allowbreak7; 90:5-\allowbreak10; 103:15,\allowbreak16 Ec 6:12; 9:10}
\crossref{1Cor}{7}{30}{Ps 30:5; 126:5,\allowbreak6 Ec 3:4 Isa 25:8; 30:19 Lu 6:21,\allowbreak25; 16:25}
\crossref{1Cor}{7}{31}{1Co 9:18 Ec 2:24,\allowbreak25; 3:12,\allowbreak13; 5:18-\allowbreak20; 9:7-\allowbreak10; 11:2,\allowbreak9,\allowbreak10}
\crossref{1Cor}{7}{32}{Ps 55:22 Mt 6:25-\allowbreak34; 13:22 Php 4:6}
\crossref{1Cor}{7}{33}{Ne 5:1-\allowbreak5 Lu 12:22 1Th 4:11,\allowbreak12 1Ti 5:8}
\crossref{1Cor}{7}{34}{Lu 2:36,\allowbreak37 2Co 7:11,\allowbreak12; 8:16; 11:28 1Ti 3:5 Tit 3:8}
\crossref{1Cor}{7}{35}{7:36 Eph 5:3 Php 4:8 1Ti 1:10 Tit 2:3}
\crossref{1Cor}{7}{36}{1Sa 2:33}
\crossref{1Cor}{7}{37}{}
\crossref{1Cor}{7}{38}{7:28}
\crossref{1Cor}{7}{39}{7:10,\allowbreak15 Ro 7:2,\allowbreak3}
\crossref{1Cor}{7}{40}{7:1,\allowbreak8,\allowbreak26,\allowbreak35}
\crossref{1Cor}{8}{1}{8:10; 10:19-\allowbreak22,\allowbreak28 Nu 25:2 Ac 15:10,\allowbreak19,\allowbreak20,\allowbreak29; 21:25 Re 2:14,\allowbreak20}
\crossref{1Cor}{8}{2}{Pr 26:12; 30:2-\allowbreak4 Ro 11:25 Ga 6:3 1Ti 1:5-\allowbreak7; 6:3,\allowbreak4}
\crossref{1Cor}{8}{3}{1Co 2:9 Ro 8:28 Jas 1:12; 2:5 1Pe 1:8 1Jo 4:19; 5:2,\allowbreak3}
\crossref{1Cor}{8}{4}{1Co 10:19,\allowbreak20 Ps 115:4-\allowbreak8 Isa 41:24; 44:8,\allowbreak9 Jer 10:14; 51:17,\allowbreak18}
\crossref{1Cor}{8}{5}{De 10:17 Jer 2:11,\allowbreak28; 11:13 Da 5:4 Joh 10:34,\allowbreak35 Ga 4:8 2Th 2:4}
\crossref{1Cor}{8}{6}{8:4 Jon 1:9 Mal 2:10 Joh 10:30; 14:9,\allowbreak10; 17:3; 20:17 Eph 1:3; 3:14}
\crossref{1Cor}{8}{7}{1Co 1:10,\allowbreak11}
\crossref{1Cor}{8}{8}{1Co 6:13 Ro 14:17 Col 2:20-\allowbreak23 Heb 13:9}
\crossref{1Cor}{8}{9}{8:10; 10:24,\allowbreak29 Mt 18:6,\allowbreak7,\allowbreak10 Lu 17:1,\allowbreak2 Ro 14:20,\allowbreak21 Ga 5:13 1Pe 2:16}
\crossref{1Cor}{8}{10}{8:1,\allowbreak2}
\crossref{1Cor}{8}{11}{8:13; 10:33; 11:1 Ro 14:15,\allowbreak20,\allowbreak21; 15:1-\allowbreak3}
\crossref{1Cor}{8}{12}{Ge 20:9; 42:22 Ex 32:21 1Sa 2:25; 19:4,\allowbreak5; 24:11 Mt 18:21}
\crossref{1Cor}{8}{13}{1Co 6:12; 9:12,\allowbreak19-\allowbreak23; 10:33; 11:1; 13:5 Ro 14:21 2Co 11:29 2Ti 3:8,\allowbreak9}
\crossref{1Cor}{9}{1}{9:2,\allowbreak3; 1:1; 15:8,\allowbreak9 Ac 9:15; 13:2; 14:4; 22:14,\allowbreak15; 16:17,\allowbreak18}
\crossref{1Cor}{9}{2}{Joh 6:27 2Co 3:1-\allowbreak3; 12:12}
\crossref{1Cor}{9}{3}{Ac 22:1; 25:16 Php 1:7,\allowbreak17 2Ti 4:16}
\crossref{1Cor}{9}{4}{9:7-\allowbreak14 Mt 10:10 Lu 10:7 Ga 6:6 1Th 2:6 2Th 3:8,\allowbreak9 1Ti 5:17,\allowbreak18}
\crossref{1Cor}{9}{5}{1Ti 3:2; 4:3 Tit 1:6 Heb 13:4}
\crossref{1Cor}{9}{6}{Ac 4:36; 11:22; 13:1,\allowbreak2,\allowbreak50; 14:12; 15:36,\allowbreak37}
\crossref{1Cor}{9}{7}{2Co 10:4,\allowbreak5 1Ti 1:18; 6:12 2Ti 2:3,\allowbreak4; 4:7}
\crossref{1Cor}{9}{8}{1Co 7:40 Ro 6:19 1Th 2:13; 4:8}
\crossref{1Cor}{9}{9}{De 25:4 1Ti 5:18}
\crossref{1Cor}{9}{10}{Mt 24:22 Ro 15:4 2Co 4:15}
\crossref{1Cor}{9}{11}{Mal 3:8,\allowbreak9 Mt 10:10 Ro 15:27 Ga 6:6}
\crossref{1Cor}{9}{12}{2Co 11:20}
\crossref{1Cor}{9}{13}{1Co 10:18 Le 6:16-\allowbreak18,\allowbreak26; 7:6-\allowbreak8 Nu 5:9,\allowbreak10; 18:8-\allowbreak20 De 10:9; 18:1-\allowbreak5}
\crossref{1Cor}{9}{14}{9:4 Mt 10:10 Lu 10:7 Ga 6:6 1Ti 5:17}
\crossref{1Cor}{9}{15}{9:12; 4:12 Ac 8:3; 20:34 1Th 2:9 2Th 3:8}
\crossref{1Cor}{9}{16}{Ro 4:2; 15:17}
\crossref{1Cor}{9}{17}{1Ch 28:9; 29:5,\allowbreak9,\allowbreak14 Ne 11:2 Isa 6:8 2Co 8:12 Phm 1:14 1Pe 5:2-\allowbreak4}
\crossref{1Cor}{9}{18}{9:6,\allowbreak7; 10:33 2Co 4:5; 11:7-\allowbreak9; 12:13-\allowbreak18 1Th 2:6 2Th 3:8,\allowbreak9}
\crossref{1Cor}{9}{19}{9:1 Ga 5:1}
\crossref{1Cor}{9}{20}{Ac 16:3; 17:2,\allowbreak3; 18:18; 21:20-\allowbreak26}
\crossref{1Cor}{9}{21}{Ac 15:28; 16:4; 21:25 Ro 2:12,\allowbreak14 Ga 2:3,\allowbreak4,\allowbreak12-\allowbreak14; 3:2}
\crossref{1Cor}{9}{22}{1Co 8:13 Ro 15:1 2Co 11:29 Ga 6:1}
\crossref{1Cor}{9}{23}{9:12 Mr 8:35 2Co 2:4 Ga 2:5 2Ti 2:10}
\crossref{1Cor}{9}{24}{Ho 12:10}
\crossref{1Cor}{9}{25}{Eph 6:12-\allowbreak18 1Ti 6:12 2Ti 2:5; 4:7 Heb 12:4}
\crossref{1Cor}{9}{26}{2Co 5:1,\allowbreak8 Php 1:21 2Ti 1:12; 2:5 Heb 4:1 1Pe 5:1 2Pe 1:10}
\crossref{1Cor}{9}{27}{9:25; 4:11,\allowbreak12; 6:12,\allowbreak13; 8:13 Ro 8:13 2Co 6:4,\allowbreak5; 11:27 Col 3:5 2Ti 2:22}
\crossref{1Cor}{10}{1}{1Co 12:1; 14:38 Ro 11:21}
\crossref{1Cor}{10}{2}{1Co 1:13-\allowbreak16 Ex 14:31 Joh 9:28,\allowbreak29 Heb 3:2,\allowbreak3}
\crossref{1Cor}{10}{3}{Ex 16:4,\allowbreak15,\allowbreak35 De 8:3 Ne 9:15,\allowbreak20 Ps 78:23-\allowbreak25; 105:40}
\crossref{1Cor}{10}{4}{Ex 17:6 Nu 20:11 Ps 78:15,\allowbreak20; 105:41 Isa 43:20; 48:21}
\crossref{1Cor}{10}{5}{Nu 14:11,\allowbreak12,\allowbreak28-\allowbreak35; 26:64,\allowbreak65 De 1:34,\allowbreak35; 2:15,\allowbreak16 Ps 78:32-\allowbreak34}
\crossref{1Cor}{10}{6}{10:11 Zep 3:6,\allowbreak7 Heb 4:11 2Pe 2:6 Jude 1:7}
\crossref{1Cor}{10}{7}{1Co 14:20-\allowbreak22; 5:11; 6:9; 8:7 De 9:12,\allowbreak16-\allowbreak21 Ps 106:19,\allowbreak20 1Jo 5:21}
\crossref{1Cor}{10}{8}{1Co 6:9,\allowbreak18 Nu 25:1-\allowbreak9 Ps 106:29 Re 2:14}
\crossref{1Cor}{10}{9}{Ex 17:2,\allowbreak7; 23:20,\allowbreak21 Nu 21:5 De 6:16 Ps 78:18,\allowbreak56; 95:9; 106:14}
\crossref{1Cor}{10}{10}{Ex 15:24; 16:2-\allowbreak9; 17:2,\allowbreak3 Nu 14:2,\allowbreak27-\allowbreak30; 16:41 Ps 106:25 Php 2:14}
\crossref{1Cor}{10}{11}{1Co 9:10 Ro 15:4}
\crossref{1Cor}{10}{12}{1Co 4:6-\allowbreak8; 8:2 Pr 16:18; 28:14 Mt 26:33,\allowbreak34,\allowbreak40,\allowbreak41 Ro 11:20 Re 3:17,\allowbreak18}
\crossref{1Cor}{10}{13}{Jer 12:5 Mt 24:21-\allowbreak24 Lu 11:4; 22:31,\allowbreak46 2Co 11:23-\allowbreak28 Eph 6:12,\allowbreak13}
\crossref{1Cor}{10}{14}{Ro 12:19 2Co 7:1; 11:11; 12:15,\allowbreak19 Php 4:1 Phm 1:1 1Pe 2:11}
\crossref{1Cor}{10}{15}{1Co 4:10; 6:5; 8:1; 11:13; 14:20 Job 34:2,\allowbreak3 1Th 5:21}
\crossref{1Cor}{10}{16}{10:21; 11:23-\allowbreak29 Mt 26:26-\allowbreak28 Mr 14:22-\allowbreak25 Lu 22:19,\allowbreak20}
\crossref{1Cor}{10}{17}{1Co 12:12,\allowbreak27 Ro 12:5 Ga 3:26-\allowbreak28 Eph 1:22,\allowbreak23; 2:15,\allowbreak16; 3:6; 4:12,\allowbreak13}
\crossref{1Cor}{10}{18}{Ro 4:1,\allowbreak12; 9:3-\allowbreak8 2Co 11:18-\allowbreak22 Ga 6:16 Eph 2:11,\allowbreak12 Php 3:3-\allowbreak5}
\crossref{1Cor}{10}{19}{1Co 1:28; 3:7; 8:4; 13:2 De 32:21 Isa 40:17; 41:29 2Co 12:11}
\crossref{1Cor}{10}{20}{Le 17:7 De 32:16,\allowbreak17 2Ch 11:15 Ps 106:37-\allowbreak39 2Co 4:4 Re 9:20}
\crossref{1Cor}{10}{21}{10:16; 8:10 De 32:37,\allowbreak38 1Ki 18:21 Mt 6:24 2Co 6:15-\allowbreak17}
\crossref{1Cor}{10}{22}{Ex 20:5; 34:14 De 4:24; 6:15; 32:16,\allowbreak21 Jos 24:19 Ps 78:58}
\crossref{1Cor}{10}{23}{1Co 6:12; 8:9 Ro 14:15,\allowbreak20}
\crossref{1Cor}{10}{24}{10:33; 9:19-\allowbreak23; 13:5 Php 2:4,\allowbreak5,\allowbreak21}
\crossref{1Cor}{10}{25}{Ro 14:14 1Ti 4:4 Tit 1:15}
\crossref{1Cor}{10}{26}{10:28 Ex 19:5 De 10:14 Job 41:11 Ps 24:1; 50:12 1Ti 6:17}
\crossref{1Cor}{10}{27}{1Co 5:9-\allowbreak11 Lu 5:29,\allowbreak30; 15:23; 19:7}
\crossref{1Cor}{10}{28}{1Co 8:10-\allowbreak13 Ro 14:15}
\crossref{1Cor}{10}{29}{10:32; 8:9-\allowbreak13 Ro 14:15-\allowbreak21}
\crossref{1Cor}{10}{30}{Ro 14:6 1Ti 4:3,\allowbreak4}
\crossref{1Cor}{10}{31}{1Co 7:34 De 12:7,\allowbreak12,\allowbreak18 Ne 8:16-\allowbreak18 Zec 7:5,\allowbreak6 Lu 11:41}
\crossref{1Cor}{10}{32}{10:33; 8:13 Ro 14:13 2Co 6:3 Php 1:10}
\crossref{1Cor}{10}{33}{10:24}
\crossref{1Cor}{11}{1}{1Co 4:16; 10:33 Php 3:17 1Th 1:6 2Th 3:9 Heb 6:12}
\crossref{1Cor}{11}{2}{11:17,\allowbreak22 Pr 31:28-\allowbreak31}
\crossref{1Cor}{11}{3}{Eph 1:22,\allowbreak23; 4:15; 5:23 Php 2:10,\allowbreak11 Col 1:18; 2:10,\allowbreak19}
\crossref{1Cor}{11}{4}{1Co 12:10,\allowbreak28; 14:1 etc.}
\crossref{1Cor}{11}{5}{Lu 2:36 Ac 2:17; 21:9}
\crossref{1Cor}{11}{6}{Nu 5:18 De 22:5}
\crossref{1Cor}{11}{7}{Ge 1:26,\allowbreak27; 5:1; 9:6 Ps 8:6 Jas 3:9}
\crossref{1Cor}{11}{8}{Ge 2:21,\allowbreak22 1Ti 2:13}
\crossref{1Cor}{11}{9}{Ge 2:18,\allowbreak20,\allowbreak23,\allowbreak24}
\crossref{1Cor}{11}{10}{Ec 5:6 Mt 18:10 Heb 1:14}
\crossref{1Cor}{11}{11}{1Co 7:10-\allowbreak14; 12:12-\allowbreak22 Ga 3:28}
\crossref{1Cor}{11}{12}{1Co 8:6 Pr 16:4 Ro 11:36 Heb 1:2,\allowbreak3}
\crossref{1Cor}{11}{13}{1Co 10:15 Lu 12:57 Joh 7:24}
\crossref{1Cor}{11}{14}{2Sa 14:26}
\crossref{1Cor}{11}{15}{11:15}
\crossref{1Cor}{11}{16}{1Ti 6:3,\allowbreak4}
\crossref{1Cor}{11}{17}{11:2,\allowbreak22 Le 19:17 Pr 27:5 Ro 13:3 1Pe 2:14}
\crossref{1Cor}{11}{18}{1Co 1:10-\allowbreak12; 3:3; 5:1; 6:1}
\crossref{1Cor}{11}{19}{Mt 18:7 Lu 17:1 Ac 20:30 1Ti 4:1,\allowbreak2 2Pe 2:1,\allowbreak2}
\crossref{1Cor}{11}{20}{11:20}
\crossref{1Cor}{11}{21}{11:23-\allowbreak25; 10:16-\allowbreak18}
\crossref{1Cor}{11}{22}{11:34}
\crossref{1Cor}{11}{23}{1Co 15:3 De 4:5 Mt 28:20 Ga 1:1,\allowbreak11,\allowbreak12 1Th 4:2}
\crossref{1Cor}{11}{24}{1Co 5:7,\allowbreak8 Ps 22:26,\allowbreak29 Pr 9:5 So 5:1 Isa 25:6; 55:1-\allowbreak3 Joh 6:53-\allowbreak58}
\crossref{1Cor}{11}{25}{11:27,\allowbreak28}
\crossref{1Cor}{11}{26}{1Co 4:5; 15:23 Joh 14:3; 21:22 Ac 1:11 1Th 4:16 2Th 1:10; 2:2,\allowbreak3}
\crossref{1Cor}{11}{27}{1Co 10:21 Le 10:1-\allowbreak3 Nu 9:10,\allowbreak13 2Ch 30:18-\allowbreak20 Mt 22:11}
\crossref{1Cor}{11}{28}{11:31 Ps 26:2-\allowbreak7 La 3:40 Hag 1:5,\allowbreak7 Zec 7:5-\allowbreak7 2Co 13:5 Ga 6:4}
\crossref{1Cor}{11}{29}{11:24,\allowbreak27 Ec 8:5 Heb 5:14}
\crossref{1Cor}{11}{30}{11:32 Ex 15:26 Nu 20:12,\allowbreak24; 21:6-\allowbreak9 2Sa 12:14-\allowbreak18 1Ki 13:21-\allowbreak24}
\crossref{1Cor}{11}{31}{11:28 Ps 32:3-\allowbreak5 Jer 31:18-\allowbreak20 Lu 15:18-\allowbreak20 1Jo 1:9 Re 2:5; 3:2,\allowbreak3}
\crossref{1Cor}{11}{32}{11:30 De 8:5 Job 5:17,\allowbreak18; 33:18-\allowbreak30; 34:31,\allowbreak32 Ps 94:12,\allowbreak13; 118:18}
\crossref{1Cor}{11}{33}{}
\crossref{1Cor}{11}{34}{11:21,\allowbreak22}
\crossref{1Cor}{12}{1}{12:4-\allowbreak11; 14:1-\allowbreak18,\allowbreak37 Eph 4:11}
\crossref{1Cor}{12}{2}{1Co 6:11 Ga 4:8 Eph 2:11,\allowbreak12; 4:17,\allowbreak18 1Th 1:9 Tit 3:3 1Pe 4:3}
\crossref{1Cor}{12}{3}{Mr 9:39 Joh 16:14,\allowbreak15 1Jo 4:2,\allowbreak3}
\crossref{1Cor}{12}{4}{12:8-\allowbreak11,\allowbreak28 Ro 12:4-\allowbreak6 Eph 4:4 Heb 2:4 1Pe 4:10}
\crossref{1Cor}{12}{5}{12:28,\allowbreak29 Ro 12:6-\allowbreak8 Eph 4:11,\allowbreak12}
\crossref{1Cor}{12}{6}{12:11; 3:7 Job 33:29 Joh 5:17 Eph 1:19-\allowbreak22 Col 1:29 Php 2:13 Heb 13:21}
\crossref{1Cor}{12}{7}{1Co 14:5,\allowbreak12,\allowbreak17,\allowbreak19,\allowbreak22-\allowbreak26 Mt 25:14 etc.}
\crossref{1Cor}{12}{8}{1Co 1:5,\allowbreak30; 2:6-\allowbreak10; 13:2,\allowbreak8 Ge 41:38,\allowbreak39 Ex 31:3 1Ki 3:5-\allowbreak12 Ne 9:20}
\crossref{1Cor}{12}{9}{1Co 13:2 Mt 17:19,\allowbreak20; 21:21 Mr 11:22,\allowbreak23 Lu 17:5,\allowbreak6 2Co 4:13 Eph 2:8}
\crossref{1Cor}{12}{10}{12:28,\allowbreak29 Mr 16:17,\allowbreak20 Lu 24:49 Joh 14:12 Ac 1:8 Ro 15:19 Ga 3:5}
\crossref{1Cor}{12}{11}{12:4; 7:7,\allowbreak17 Joh 3:27 Ro 12:6 2Co 10:13 Eph 4:7}
\crossref{1Cor}{12}{12}{1Co 10:17 Ro 12:4,\allowbreak5 Eph 1:23; 4:4,\allowbreak12,\allowbreak15,\allowbreak16; 5:23,\allowbreak30 Col 1:18,\allowbreak24; 2:19}
\crossref{1Cor}{12}{13}{1Co 10:2 Isa 44:3-\allowbreak5 Eze 36:25-\allowbreak27 Mt 3:11 Lu 3:16 Joh 1:16,\allowbreak33; 3:5}
\crossref{1Cor}{12}{14}{12:12,\allowbreak19,\allowbreak27,\allowbreak28 Eph 4:25}
\crossref{1Cor}{12}{15}{Jud 9:8-\allowbreak15 2Ki 14:9}
\crossref{1Cor}{12}{16}{12:16,\allowbreak22 Ro 12:3,\allowbreak10 Php 2:3}
\crossref{1Cor}{12}{17}{12:21,\allowbreak29 1Sa 9:9 Ps 94:9; 139:13-\allowbreak16 Pr 20:12}
\crossref{1Cor}{12}{18}{12:24,\allowbreak28}
\crossref{1Cor}{12}{19}{12:14}
\crossref{1Cor}{12}{20}{12:20}
\crossref{1Cor}{12}{21}{Nu 10:31,\allowbreak32 1Sa 25:32 Ezr 10:1-\allowbreak5 Ne 4:16-\allowbreak21 Job 29:11}
\crossref{1Cor}{12}{22}{Pr 14:28 Ec 4:9-\allowbreak12; 5:9; 9:14,\allowbreak15 2Co 1:11 Tit 2:9,\allowbreak10}
\crossref{1Cor}{12}{23}{Ge 3:7,\allowbreak21}
\crossref{1Cor}{12}{24}{Ge 2:25; 3:11}
\crossref{1Cor}{12}{25}{1Co 1:10-\allowbreak12; 3:3 Joh 17:21-\allowbreak26 2Co 13:11}
\crossref{1Cor}{12}{26}{Ro 12:15 2Co 11:28,\allowbreak29 Ga 6:2 Heb 13:3 1Pe 3:8}
\crossref{1Cor}{12}{27}{12:12,\allowbreak14-\allowbreak20 Ro 12:5 Eph 1:23; 4:12; 5:23,\allowbreak30 Col 1:24}
\crossref{1Cor}{12}{28}{12:7-\allowbreak11 Lu 6:14 Ac 13:1-\allowbreak3; 20:28 Ro 12:6-\allowbreak8 Eph 2:20; 3:5; 4:11-\allowbreak13}
\crossref{1Cor}{12}{29}{12:4-\allowbreak11,\allowbreak14-\allowbreak20}
\crossref{1Cor}{12}{30}{}
\crossref{1Cor}{12}{31}{1Co 8:1; 14:1,\allowbreak39 Mt 5:6 Lu 10:42}
\crossref{1Cor}{13}{1}{13:2,\allowbreak3; 12:8,\allowbreak16,\allowbreak29,\allowbreak30; 14:6 2Co 12:4 2Pe 2:18}
\crossref{1Cor}{13}{2}{1Co 12:8-\allowbreak10,\allowbreak28; 14:1,\allowbreak6-\allowbreak9 Nu 24:15-\allowbreak24 Mt 7:22,\allowbreak23}
\crossref{1Cor}{13}{3}{Mt 6:1-\allowbreak4; 23:5 Lu 18:22,\allowbreak28; 19:8; 21:3,\allowbreak4 Joh 12:43 Ga 5:26}
\crossref{1Cor}{13}{4}{Pr 10:12 2Co 6:6 Ga 5:22 Eph 4:2 Col 1:11; 3:12 2Ti 2:25; 3:10}
\crossref{1Cor}{13}{5}{1Co 7:36}
\crossref{1Cor}{13}{6}{1Sa 23:19-\allowbreak21 2Sa 4:10-\allowbreak12 Ps 10:3; 119:136 Pr 14:9 Jer 9:1; 13:17}
\crossref{1Cor}{13}{7}{13:4 Nu 11:12-\allowbreak14 De 1:9 Pr 10:12 So 8:6,\allowbreak7 Ro 15:1 Ga 6:2}
\crossref{1Cor}{13}{8}{13:10,\allowbreak13 Lu 22:32 Ga 5:6}
\crossref{1Cor}{13}{9}{13:12; 2:9; 8:2 Job 11:7,\allowbreak8; 26:14 Ps 40:5; 139:6 Pr 30:4 Mt 11:27}
\crossref{1Cor}{13}{10}{13:12 Isa 24:23; 60:19,\allowbreak20 2Co 5:7,\allowbreak8 Re 21:22,\allowbreak23; 22:4,\allowbreak5}
\crossref{1Cor}{13}{11}{1Co 3:1,\allowbreak2; 14:20 Ec 11:10 Ga 4:1}
\crossref{1Cor}{13}{12}{2Co 3:18; 5:7 Php 3:12 Jas 1:23}
\crossref{1Cor}{13}{13}{1Co 3:14 1Pe 1:21 1Jo 2:14,\allowbreak24; 3:9}
\crossref{1Cor}{14}{1}{Pr 15:9; 21:21 Isa 51:1 Ro 9:30; 14:19 1Ti 5:10; 6:11 Heb 12:14}
\crossref{1Cor}{14}{2}{14:9-\allowbreak11,\allowbreak16,\allowbreak21,\allowbreak22 Ge 11:7; 42:23 De 28:49 2Ki 18:26 Ac 2:4-\allowbreak11; 10:46}
\crossref{1Cor}{14}{3}{14:4,\allowbreak12,\allowbreak26; 8:1; 10:23 Ac 9:31 Ro 14:19; 15:2 Eph 4:12-\allowbreak16,\allowbreak29 1Th 5:11}
\crossref{1Cor}{14}{4}{14:14}
\crossref{1Cor}{14}{5}{1Co 12:28-\allowbreak30; 13:4 Nu 11:28,\allowbreak29}
\crossref{1Cor}{14}{6}{1Co 10:33; 12:7; 13:3 1Sa 12:21 Jer 16:19; 23:32 Mt 16:26 2Ti 2:14}
\crossref{1Cor}{14}{7}{1Co 13:1}
\crossref{1Cor}{14}{8}{Nu 10:9 Jos 6:4-\allowbreak20 Jud 7:16-\allowbreak18 Ne 4:18-\allowbreak21 Job 39:24,\allowbreak25}
\crossref{1Cor}{14}{9}{14:19}
\crossref{1Cor}{14}{10}{}
\crossref{1Cor}{14}{11}{14:21 Ac 28:2,\allowbreak4 Ro 1:14 Col 3:11}
\crossref{1Cor}{14}{12}{14:1; 12:7,\allowbreak31 Tit 2:14}
\crossref{1Cor}{14}{13}{14:27,\allowbreak28; 12:10,\allowbreak30 Mr 11:24 Joh 14:13,\allowbreak14 Ac 1:14; 4:29-\allowbreak31; 8:15}
\crossref{1Cor}{14}{14}{14:2,\allowbreak15,\allowbreak16,\allowbreak19}
\crossref{1Cor}{14}{15}{1Co 10:19 Ro 3:5; 8:31 Php 1:18}
\crossref{1Cor}{14}{16}{14:2,\allowbreak14}
\crossref{1Cor}{14}{17}{14:4,\allowbreak6}
\crossref{1Cor}{14}{18}{1Co 1:4-\allowbreak6; 4:7}
\crossref{1Cor}{14}{19}{14:4,\allowbreak21,\allowbreak22}
\crossref{1Cor}{14}{20}{1Co 3:1,\allowbreak2; 13:11 Ps 119:99 Isa 11:3 Ro 16:19 Eph 4:14,\allowbreak15 Php 1:9}
\crossref{1Cor}{14}{21}{De 28:49 Isa 28:11,\allowbreak12 Jer 5:15}
\crossref{1Cor}{14}{22}{Mr 16:17 Ac 2:6-\allowbreak12,\allowbreak32-\allowbreak36}
\crossref{1Cor}{14}{23}{1Co 11:18}
\crossref{1Cor}{14}{24}{1Co 2:15 Joh 1:47-\allowbreak49; 4:29 Ac 2:37 Heb 4:12,\allowbreak13}
\crossref{1Cor}{14}{25}{Ge 44:14 De 9:18 Ps 72:11 Isa 60:14 Lu 5:8; 8:28 Re 5:8; 19:4}
\crossref{1Cor}{14}{26}{14:6; 12:8-\allowbreak10}
\crossref{1Cor}{14}{27}{14:27}
\crossref{1Cor}{14}{28}{}
\crossref{1Cor}{14}{29}{14:39; 12:10 1Th 5:19-\allowbreak21 1Jo 4:1-\allowbreak3}
\crossref{1Cor}{14}{30}{14:6,\allowbreak26}
\crossref{1Cor}{14}{31}{14:3,\allowbreak19,\allowbreak35 Pr 1:5; 9:9 Eph 4:11,\allowbreak12}
\crossref{1Cor}{14}{32}{14:29,\allowbreak30 1Sa 10:10-\allowbreak13; 19:19-\allowbreak24 2Ki 2:3,\allowbreak5 Job 32:8-\allowbreak11 Jer 20:9}
\crossref{1Cor}{14}{33}{1Co 7:15 Lu 2:14 Ro 15:33 Ga 5:22 2Th 3:16 Heb 13:20 Jas 3:17,\allowbreak18}
\crossref{1Cor}{14}{34}{1Co 11:5 1Ti 2:11,\allowbreak12}
\crossref{1Cor}{14}{35}{Eph 5:25-\allowbreak27 1Pe 3:7}
\crossref{1Cor}{14}{36}{Isa 2:3 Mic 4:1,\allowbreak2 Zec 14:8 Ac 13:1-\allowbreak3; 15:35,\allowbreak36; 16:9,\allowbreak10}
\crossref{1Cor}{14}{37}{1Co 8:2; 13:1-\allowbreak3 Nu 24:3,\allowbreak4,\allowbreak16 Ro 12:3 2Co 10:7,\allowbreak12; 11:4,\allowbreak12-\allowbreak15 Ga 6:8}
\crossref{1Cor}{14}{38}{Ho 4:17 Mt 7:6; 15:14 1Ti 6:3-\allowbreak5 2Ti 4:3,\allowbreak4 Re 22:11,\allowbreak12}
\crossref{1Cor}{14}{39}{14:1,\allowbreak3,\allowbreak5,\allowbreak24,\allowbreak25; 12:31 1Th 5:20}
\crossref{1Cor}{14}{40}{14:26-\allowbreak33; 11:34 Ro 13:13}
\crossref{1Cor}{15}{1}{15:3-\allowbreak11; 1:23,\allowbreak24; 2:2-\allowbreak7 Ac 18:4,\allowbreak5 Ga 1:6-\allowbreak12}
\crossref{1Cor}{15}{2}{1Co 1:18,\allowbreak21 Ac 2:47}
\crossref{1Cor}{15}{3}{1Co 4:1,\allowbreak2; 11:2,\allowbreak23 Eze 3:17 Mt 20:18,\allowbreak19 Mr 16:15,\allowbreak16 Lu 24:46,\allowbreak47}
\crossref{1Cor}{15}{4}{Isa 53:9 Mt 27:57-\allowbreak60 Mr 15:43-\allowbreak46 Lu 23:50-\allowbreak53 Joh 19:38-\allowbreak42}
\crossref{1Cor}{15}{5}{Lu 24:34 etc.}
\crossref{1Cor}{15}{6}{Mt 28:10,\allowbreak16,\allowbreak17 Mr 16:7}
\crossref{1Cor}{15}{7}{Lu 24:50 Ac 1:2-\allowbreak12}
\crossref{1Cor}{15}{8}{1Co 9:1 Ac 9:3-\allowbreak5,\allowbreak17; 18:9; 22:14,\allowbreak18; 26:16 2Co 12:1-\allowbreak6}
\crossref{1Cor}{15}{9}{2Co 11:5; 12:11 Eph 3:7,\allowbreak8}
\crossref{1Cor}{15}{10}{1Co 4:7 Ro 11:1,\allowbreak5,\allowbreak6 Eph 2:7,\allowbreak8; 3:7,\allowbreak8 1Ti 1:15,\allowbreak16}
\crossref{1Cor}{15}{11}{15:3,\allowbreak4; 2:2}
\crossref{1Cor}{15}{12}{15:4}
\crossref{1Cor}{15}{13}{15:20 Joh 11:25,\allowbreak26 Ac 23:8 Ro 4:24,\allowbreak25; 8:11,\allowbreak23 2Co 4:10-\allowbreak14}
\crossref{1Cor}{15}{14}{15:2,\allowbreak17 Ps 73:13 Isa 49:4 Ge 8:8 Mt 15:9 Ac 17:31 Ga 2:2}
\crossref{1Cor}{15}{15}{Ex 23:3 Job 13:7-\allowbreak10 Ro 3:7,\allowbreak8}
\crossref{1Cor}{15}{16}{}
\crossref{1Cor}{15}{17}{15:2,\allowbreak14 Ro 4:25}
\crossref{1Cor}{15}{18}{15:6 1Th 4:13,\allowbreak14 Re 14:13}
\crossref{1Cor}{15}{19}{Ps 17:14 Ec 6:11; 9:9 Lu 8:14; 21:34 1Co 6:3,\allowbreak4 2Ti 2:4}
\crossref{1Cor}{15}{20}{15:4-\allowbreak8}
\crossref{1Cor}{15}{21}{15:22 Ro 5:12-\allowbreak17}
\crossref{1Cor}{15}{22}{15:45-\allowbreak49 Ge 2:17; 3:6,\allowbreak19 Joh 5:21-\allowbreak29 Ro 5:12-\allowbreak21}
\crossref{1Cor}{15}{23}{15:20 Isa 26:19 1Th 4:15-\allowbreak17}
\crossref{1Cor}{15}{24}{Da 12:4,\allowbreak9,\allowbreak13 Mt 10:22; 13:39,\allowbreak40; 24:13 1Pe 4:7}
\crossref{1Cor}{15}{25}{Ps 2:6-\allowbreak10; 45:3-\allowbreak6; 110:1 Mt 22:44 Mr 12:36 Lu 20:42,\allowbreak43 Ac 2:34}
\crossref{1Cor}{15}{26}{15:55 Isa 25:8 Ho 13:14 Lu 20:36 2Ti 1:10 Heb 2:14 Re 20:14; 21:4}
\crossref{1Cor}{15}{27}{Ps 8:6 Mt 11:27; 28:18 Joh 3:35; 13:3 Eph 1:20 Php 2:9-\allowbreak11}
\crossref{1Cor}{15}{28}{Ps 2:8,\allowbreak9; 18:39,\allowbreak47; 21:8,\allowbreak9 Da 2:34,\allowbreak35,\allowbreak40-\allowbreak45 Mt 13:41-\allowbreak43 Php 3:21}
\crossref{1Cor}{15}{29}{15:16,\allowbreak32 Ro 6:3,\allowbreak4 Mt 20:22}
\crossref{1Cor}{15}{30}{15:31 Ro 8:36-\allowbreak39 2Co 4:7-\allowbreak12; 6:9; 11:23-\allowbreak27 Ga 5:11}
\crossref{1Cor}{15}{31}{Ge 43:3 1Sa 8:9 Jer 11:7 Zec 3:6 Php 3:3}
\crossref{1Cor}{15}{32}{Ro 6:19 Ga 3:15}
\crossref{1Cor}{15}{33}{1Co 6:9 Mt 24:4,\allowbreak11,\allowbreak24 Ga 6:7 Eph 5:6 2Th 2:10 Re 12:9; 13:8-\allowbreak14}
\crossref{1Cor}{15}{34}{Joe 1:5 Jon 1:6 Ro 13:11 Eph 5:14}
\crossref{1Cor}{15}{35}{Job 11:12; 22:13 Ps 73:11 Ec 11:5 Eze 37:3,\allowbreak11 Joh 3:4,\allowbreak9; 9:10}
\crossref{1Cor}{15}{36}{Lu 12:20; 24:25 Ro 1:22 Eph 5:15}
\crossref{1Cor}{15}{37}{15:37}
\crossref{1Cor}{15}{38}{1Co 3:7 Ge 1:11,\allowbreak12 Ps 104:14 Isa 61:11 Mr 4:26-\allowbreak29}
\crossref{1Cor}{15}{39}{Ge 1:20-\allowbreak26}
\crossref{1Cor}{15}{40}{15:40}
\crossref{1Cor}{15}{41}{Ge 1:14 De 4:19 Job 31:26 Ps 8:3; 19:4-\allowbreak6; 148:3-\allowbreak5 Isa 24:23}
\crossref{1Cor}{15}{42}{15:50-\allowbreak54 Da 12:3 Mt 13:43 Php 3:20,\allowbreak21}
\crossref{1Cor}{15}{43}{Da 12:1 Mt 13:43 Php 3:20,\allowbreak21}
\crossref{1Cor}{15}{44}{Lu 24:31 Joh 20:19,\allowbreak26}
\crossref{1Cor}{15}{45}{15:47-\allowbreak49 Ge 2:7 Ro 5:12-\allowbreak14 Re 16:3}
\crossref{1Cor}{15}{46}{Ro 6:6 Eph 4:22-\allowbreak24 Col 3:9,\allowbreak10}
\crossref{1Cor}{15}{47}{15:45 Ge 2:7; 3:19 Joh 3:13,\allowbreak31 2Co 5:1}
\crossref{1Cor}{15}{48}{15:21,\allowbreak22 Ge 5:3 Job 14:4 Joh 3:6 Ro 5:12-\allowbreak21}
\crossref{1Cor}{15}{49}{Ge 5:3}
\crossref{1Cor}{15}{50}{1Co 1:12; 7:29 2Co 9:6 Ga 3:17; 5:16 Eph 4:17 Col 2:4}
\crossref{1Cor}{15}{51}{1Co 2:7; 4:1; 13:2 Eph 1:9; 3:3; 5:32}
\crossref{1Cor}{15}{52}{Ex 33:5 Nu 16:21,\allowbreak45 Ps 73:19 2Pe 3:10}
\crossref{1Cor}{15}{53}{Ro 13:12-\allowbreak14 2Co 5:2-\allowbreak4 Ga 3:27 Eph 4:24 1Jo 3:2}
\crossref{1Cor}{15}{54}{Ro 2:7; 6:12; 8:11 2Co 4:11 2Th 1:10}
\crossref{1Cor}{15}{55}{Ho 13:14}
\crossref{1Cor}{15}{56}{Ge 3:17-\allowbreak19 Ps 90:3-\allowbreak11 Pr 14:32 Joh 8:21,\allowbreak24 Ro 5:15,\allowbreak17; 6:23}
\crossref{1Cor}{15}{57}{Ac 27:35 Ro 7:25 2Co 1:11; 2:14; 9:15 Eph 5:20}
\crossref{1Cor}{15}{58}{2Co 7:1 2Pe 1:4-\allowbreak9; 3:14}
\crossref{1Cor}{16}{1}{Ac 11:28,\allowbreak30; 24:17 Ro 15:25,\allowbreak26 2Co 8:1-\allowbreak9:15 Ga 2:10}
\crossref{1Cor}{16}{2}{Lu 24:1 Joh 20:19,\allowbreak26 Ac 20:7 Re 1:10}
\crossref{1Cor}{16}{3}{1Co 4:19-\allowbreak21; 11:34}
\crossref{1Cor}{16}{4}{Ro 15:25 2Co 8:4,\allowbreak19}
\crossref{1Cor}{16}{5}{Ac 19:21; 20:1-\allowbreak3 2Co 1:15-\allowbreak17}
\crossref{1Cor}{16}{6}{Ac 27:12; 28:11 Tit 3:12}
\crossref{1Cor}{16}{7}{1Co 4:19 Pr 19:21 Jer 10:23 Ac 18:21 Ro 1:10 Jas 4:15}
\crossref{1Cor}{16}{8}{1Co 15:32}
\crossref{1Cor}{16}{9}{Ac 19:8 etc.}
\crossref{1Cor}{16}{10}{1Co 4:17 Ac 19:22}
\crossref{1Cor}{16}{11}{16:10 Lu 10:16 1Th 4:8 1Ti 4:12 Tit 2:15}
\crossref{1Cor}{16}{12}{1Co 1:12; 3:5,\allowbreak22 Ac 18:24-\allowbreak28; 19:1 Tit 3:4}
\crossref{1Cor}{16}{13}{Mt 24:42-\allowbreak44; 25:13; 26:41 Mr 13:33-\allowbreak37; 14:37,\allowbreak38 Lu 12:35-\allowbreak40; 21:36}
\crossref{1Cor}{16}{14}{1Co 8:1; 12:31; 13:1-\allowbreak13; 14:1 Joh 13:34,\allowbreak35; 15:17 Ro 13:8-\allowbreak10; 14:15}
\crossref{1Cor}{16}{15}{16:17; 1:16}
\crossref{1Cor}{16}{16}{Eph 5:21 Heb 13:17 1Pe 5:5}
\crossref{1Cor}{16}{17}{16:15}
\crossref{1Cor}{16}{18}{Pr 25:13,\allowbreak25 Ro 15:32 2Co 7:6,\allowbreak7,\allowbreak13 Php 2:28 Col 4:8 1Th 3:6,\allowbreak7}
\crossref{1Cor}{16}{19}{Ac 19:10 1Pe 1:1 Re 1:11}
\crossref{1Cor}{16}{20}{Ro 16:16,\allowbreak21,\allowbreak23 2Co 13:13 Php 4:22 Phm 1:23,\allowbreak24 Heb 13:24}
\crossref{1Cor}{16}{21}{Ga 6:11 Col 4:18 2Th 3:17}
\crossref{1Cor}{16}{22}{So 1:3,\allowbreak4,\allowbreak7; 3:1-\allowbreak3; 5:16 Isa 5:1 Mt 10:37; 25:40,\allowbreak45 Joh 8:42}
\crossref{1Cor}{16}{23}{Ro 16:20,\allowbreak24}
\crossref{1Cor}{16}{24}{16:14; 4:14,\allowbreak15 2Co 11:11; 12:15 Php 1:8 Re 3:19}

% 2Cor
\crossref{2Cor}{1}{1}{Ro 1:1-\allowbreak5 1Co 1:1 1Ti 1:1 2Ti 1:1}
\crossref{2Cor}{1}{2}{Ro 1:7 2Sa 15:20 1Ch 12:18 Da 4:1 1Co 1:3 Ga 6:16 Eph 6:23}
\crossref{2Cor}{1}{3}{Ge 14:20 1Ch 29:10 Ne 9:5 Job 1:21 Ps 18:46; 72:19 Da 4:34}
\crossref{2Cor}{1}{4}{2Co 7:6,\allowbreak7 Ps 86:17 Isa 12:1; 49:10; 51:3,\allowbreak12; 52:9; 66:12,\allowbreak13 Joh 14:16}
\crossref{2Cor}{1}{5}{2Co 4:10,\allowbreak11; 11:23-\allowbreak30 Ac 9:4 1Co 4:10-\allowbreak13 Php 1:20; 3:10 Col 1:24}
\crossref{2Cor}{1}{6}{1:4; 4:15-\allowbreak18 1Co 3:21-\allowbreak23 2Ti 2:10}
\crossref{2Cor}{1}{7}{1:14; 7:9; 12:20 Php 1:6,\allowbreak7 1Th 1:3,\allowbreak4}
\crossref{2Cor}{1}{8}{2Co 4:7-\allowbreak12 Ac 19:23-\allowbreak35 1Co 15:32; 16:9}
\crossref{2Cor}{1}{9}{2Co 3:5; 4:7; 12:7-\allowbreak10 Job 40:14 Ps 22:29; 44:5-\allowbreak7 Pr 28:26 Jer 9:23,\allowbreak24}
\crossref{2Cor}{1}{10}{1Sa 7:12; 17:37 Job 5:17-\allowbreak22 Ps 34:19 Isa 46:3 Ac 26:21 2Ti 4:17}
\crossref{2Cor}{1}{11}{2Co 9:14 Isa 37:4; 62:6,\allowbreak7 Ac 12:5 Ro 15:30-\allowbreak32 Eph 6:18,\allowbreak19 Php 1:19}
\crossref{2Cor}{1}{12}{Job 13:15; 23:10-\allowbreak12; 27:5,\allowbreak6; 31:1-\allowbreak40 Ps 7:3-\allowbreak5; 44:17-\allowbreak21 Isa 38:3}
\crossref{2Cor}{1}{13}{2Co 4:2; 5:11; 13:6 Phm 1:6}
\crossref{2Cor}{1}{14}{2Co 2:5 Ro 11:25 1Co 11:18}
\crossref{2Cor}{1}{15}{1Co 4:19; 11:34}
\crossref{2Cor}{1}{16}{Ac 19:21,\allowbreak22; 21:5 1Co 16:5-\allowbreak7}
\crossref{2Cor}{1}{17}{Jud 9:4 Jer 23:32 Zep 3:4}
\crossref{2Cor}{1}{18}{1:23; 11:31 Joh 7:28; 8:26 1Jo 5:20 Re 3:7,\allowbreak14}
\crossref{2Cor}{1}{19}{Ps 2:7 Mt 3:17; 16:16,\allowbreak17; 17:5; 26:63,\allowbreak64; 27:40,\allowbreak54 Mr 1:1 Lu 1:35}
\crossref{2Cor}{1}{20}{Ge 3:15; 22:18; 49:10 Ps 72:17 Isa 7:14; 9:6,\allowbreak7 Lu 1:68-\allowbreak74}
\crossref{2Cor}{1}{21}{2Co 5:5 Ps 37:23,\allowbreak24; 87:5; 89:4 Isa 9:7; 49:8; 62:7 Ro 16:25 Col 2:7}
\crossref{2Cor}{1}{22}{Joh 6:27 Ro 4:11 Eph 1:13,\allowbreak14; 4:30 2Ti 2:19 Re 2:17; 7:3; 9:4}
\crossref{2Cor}{1}{23}{1:18; 11:11,\allowbreak31 Ro 1:9; 9:1 Ga 1:20 Php 1:8 1Th 2:5}
\crossref{2Cor}{1}{24}{Mt 23:8-\allowbreak10; 24:49 1Co 3:5 2Ti 2:24-\allowbreak26 1Pe 5:3}
\crossref{2Cor}{2}{1}{2Co 1:15-\allowbreak17,\allowbreak Ac 11:29; 15:2,\allowbreak37 1Co 2:2; 5:3 Tit 3:12}
\crossref{2Cor}{2}{2}{2Co 1:14; 11:29 Ro 12:15 1Co 12:26}
\crossref{2Cor}{2}{3}{1Co 4:21; 5:1 etc.}
\crossref{2Cor}{2}{4}{Le 19:17,\allowbreak18 Ps 119:136 Pr 27:5,\allowbreak6 Jer 13:15-\allowbreak17 Lu 19:41-\allowbreak44}
\crossref{2Cor}{2}{5}{Pr 17:25 1Co 5:1-\allowbreak5,\allowbreak12,\allowbreak13 Ga 5:10}
\crossref{2Cor}{2}{6}{2Co 13:10 1Co 5:4,\allowbreak5 1Ti 5:20}
\crossref{2Cor}{2}{7}{Ga 6:1,\allowbreak2 Eph 4:32 Col 3:13 2Th 3:6,\allowbreak14,\allowbreak15 Heb 12:12-\allowbreak15}
\crossref{2Cor}{2}{8}{Ga 5:13; 6:1,\allowbreak2,\allowbreak10 Jude 1:22,\allowbreak23}
\crossref{2Cor}{2}{9}{2Co 7:12-\allowbreak15; 8:24 Ex 16:4 De 8:2,\allowbreak16; 13:3 Php 2:22}
\crossref{2Cor}{2}{10}{2Co 5:20 Mt 18:18 Joh 20:23 1Co 5:4}
\crossref{2Cor}{2}{11}{2Co 11:3,\allowbreak14 1Ch 21:1,\allowbreak2 Job 1:11; 2:3,\allowbreak5,\allowbreak9 Zec 3:1-\allowbreak4 Lu 22:31}
\crossref{2Cor}{2}{12}{Ac 16:8; 20:1-\allowbreak6,\allowbreak8}
\crossref{2Cor}{2}{13}{2Co 7:5,\allowbreak6}
\crossref{2Cor}{2}{14}{2Co 1:11; 8:16; 9:15 Eph 5:20 1Th 3:9 Re 7:12}
\crossref{2Cor}{2}{15}{Ge 8:21 Ex 29:18,\allowbreak25 Eze 20:41 Eph 5:2 Php 4:18}
\crossref{2Cor}{2}{16}{Lu 2:34 Joh 9:39 Ac 13:45-\allowbreak47; 20:26,\allowbreak27 1Pe 2:7,\allowbreak8}
\crossref{2Cor}{2}{17}{2Co 4:2; 11:13-\allowbreak15 Jer 5:31; 23:27-\allowbreak32 Mt 24:24 1Ti 1:19,\allowbreak20; 4:1-\allowbreak3}
\crossref{2Cor}{3}{1}{2Co 2:17; 5:12; 10:8,\allowbreak12; 12:11,\allowbreak19 1Co 3:10; 4:15; 10:33}
\crossref{2Cor}{3}{2}{1Co 3:10; 9:1,\allowbreak2}
\crossref{2Cor}{3}{3}{Ex 31:18 Re 2:1,\allowbreak8,\allowbreak12,\allowbreak18; 3:1,\allowbreak7,\allowbreak14,\allowbreak22}
\crossref{2Cor}{3}{4}{2Co 2:14 Php 1:6}
\crossref{2Cor}{3}{5}{2Co 2:16; 4:7 Ex 4:10 Joh 15:5}
\crossref{2Cor}{3}{6}{2Co 5:18-\allowbreak20 Mt 13:52 Ro 1:5 1Co 3:5,\allowbreak10; 12:28 Eph 3:7; 4:11,\allowbreak12}
\crossref{2Cor}{3}{7}{3:6,\allowbreak9 Ro 7:10}
\crossref{2Cor}{3}{8}{3:6,\allowbreak17; 11:4 Isa 11:2; 44:3; 59:21 Joe 2:28,\allowbreak29 Joh 1:17; 7:39 Ac 2:17,\allowbreak18}
\crossref{2Cor}{3}{9}{3:6,\allowbreak7 Ex 19:12-\allowbreak19; 20:18,\allowbreak19 Ro 1:18; 8:3,\allowbreak4 Ga 3:10 Heb 12:18-\allowbreak21}
\crossref{2Cor}{3}{10}{Job 25:5 Isa 24:23 Hag 2:3,\allowbreak7-\allowbreak9 Ac 26:13 Php 3:7-\allowbreak8 2Pe 1:17}
\crossref{2Cor}{3}{11}{3:7 Ro 5:20,\allowbreak21 Heb 7:21-\allowbreak25; 8:13; 12:25-\allowbreak29}
\crossref{2Cor}{3}{12}{2Co 4:2,\allowbreak3,\allowbreak13 Joh 10:24; 16:25,\allowbreak29 1Co 14:19 Col 4:4}
\crossref{2Cor}{3}{13}{Ex 34:33-\allowbreak35}
\crossref{2Cor}{3}{14}{2Co 4:3,\allowbreak4 Ps 69:23 Isa 6:10; 26:10-\allowbreak12; 42:18-\allowbreak20; 44:18; 56:10; 59:10}
\crossref{2Cor}{3}{15}{Ac 13:27-\allowbreak29}
\crossref{2Cor}{3}{16}{Ex 34:34 De 4:30; 30:10 La 3:40 Ho 3:4,\allowbreak5 Ro 11:25-\allowbreak27}
\crossref{2Cor}{3}{17}{3:6 Joh 6:63 1Co 15:45}
\crossref{2Cor}{3}{18}{3:13}
\crossref{2Cor}{4}{1}{2Co 3:6,\allowbreak12; 5:18 Eph 3:7,\allowbreak8}
\crossref{2Cor}{4}{2}{1Co 4:5}
\crossref{2Cor}{4}{3}{Ro 2:16 1Th 1:5 1Ti 1:11}
\crossref{2Cor}{4}{4}{Mt 4:8,\allowbreak9 Joh 12:31,\allowbreak40; 14:30; 16:11 1Co 10:20 Eph 2:2; 6:12}
\crossref{2Cor}{4}{5}{Mt 3:11 Joh 1:21-\allowbreak23; 3:27-\allowbreak31; 7:18 Ac 3:12,\allowbreak13; 8:9,\allowbreak10; 10:25,\allowbreak26}
\crossref{2Cor}{4}{6}{Ge 1:3,\allowbreak14,\allowbreak15 Ps 74:16; 136:7-\allowbreak9 Isa 45:7}
\crossref{2Cor}{4}{7}{4:1; 6:10 Mt 13:44,\allowbreak52 Eph 3:8 Col 1:27; 2:3}
\crossref{2Cor}{4}{8}{2Co 1:8-\allowbreak10; 6:4; 7:5; 11:23-\allowbreak30}
\crossref{2Cor}{4}{9}{Ps 9:10; 22:1; 37:25,\allowbreak28 Isa 62:4 Heb 13:5}
\crossref{2Cor}{4}{10}{2Co 1:5,\allowbreak9 Ro 8:17,\allowbreak18 Ga 6:17 Php 3:10,\allowbreak11 Col 1:24}
\crossref{2Cor}{4}{11}{Ps 44:22; 141:7 Ro 8:36 1Co 15:31,\allowbreak49}
\crossref{2Cor}{4}{12}{2Co 12:15; 13:9 Ac 20:24 1Co 4:10 Php 2:17,\allowbreak30 1Jo 3:16}
\crossref{2Cor}{4}{13}{Ac 15:11 Ro 1:12 1Co 12:9 Heb 11:1 etc.}
\crossref{2Cor}{4}{14}{2Co 5:1-\allowbreak4 Isa 26:19 Joh 11:25,\allowbreak26 Ro 8:11 1Co 6:14; 15:20-\allowbreak22}
\crossref{2Cor}{4}{15}{2Co 1:4-\allowbreak6 Ro 8:28 1Co 3:21-\allowbreak23 Col 1:24 2Ti 2:10}
\crossref{2Cor}{4}{16}{4:1 Ps 27:13; 119:81 Isa 40:29 1Co 15:58}
\crossref{2Cor}{4}{17}{2Co 11:23-\allowbreak28 Ps 30:5 Isa 54:8 Ac 20:23 Ro 8:18,\allowbreak34,\allowbreak37 1Pe 1:6; 4:7}
\crossref{2Cor}{4}{18}{2Co 5:7 Ro 8:24,\allowbreak25 Heb 11:1,\allowbreak25-\allowbreak27; 12:2,\allowbreak3}
\crossref{2Cor}{5}{1}{Job 19:25,\allowbreak26 Ps 56:9 2Ti 1:12 1Jo 3:2,\allowbreak14,\allowbreak19; 5:19,\allowbreak20}
\crossref{2Cor}{5}{2}{5:4 Ro 7:24; 8:23 1Pe 1:6,\allowbreak7}
\crossref{2Cor}{5}{3}{Ge 3:7-\allowbreak11 Ex 32:25 Re 3:18; 16:15}
\crossref{2Cor}{5}{4}{2Pe 1:13}
\crossref{2Cor}{5}{5}{2Co 4:17 Isa 29:23; 60:21; 61:3 Eph 2:10}
\crossref{2Cor}{5}{6}{5:8 Ps 27:3,\allowbreak4 Pr 14:26 Isa 30:15; 36:4 Heb 10:35 1Pe 5:1 Re 1:9}
\crossref{2Cor}{5}{7}{2Co 1:24; 4:18 De 12:9 Ro 8:24,\allowbreak25 1Co 13:12 Ga 2:20 Heb 10:38}
\crossref{2Cor}{5}{8}{5:6; 12:2,\allowbreak3 Lu 2:29 Ac 21:13 Php 1:20-\allowbreak24 2Ti 4:7,\allowbreak8 2Pe 1:14,\allowbreak15}
\crossref{2Cor}{5}{9}{Joh 6:27 Ro 15:20 1Co 9:26,\allowbreak27; 15:58 Col 1:29 1Th 4:11}
\crossref{2Cor}{5}{10}{Ge 18:25 1Sa 2:3,\allowbreak10 Ps 7:6-\allowbreak8; 9:7,\allowbreak8; 50:3-\allowbreak6; 96:10-\allowbreak13; 98:9}
\crossref{2Cor}{5}{11}{Ge 35:5 Job 6:4; 18:11; 31:23 Ps 73:19; 76:7; 88:15,\allowbreak16; 90:11}
\crossref{2Cor}{5}{12}{2Co 3:1; 6:4; 10:8,\allowbreak12,\allowbreak18; 12:11 Pr 27:2}
\crossref{2Cor}{5}{13}{2Co 11:1,\allowbreak16,\allowbreak17; 12:6,\allowbreak11}
\crossref{2Cor}{5}{14}{2Co 8:8,\allowbreak9 So 1:4; 8:6,\allowbreak7 Mt 10:37,\allowbreak38 Lu 7:42-\allowbreak47 Joh 14:21-\allowbreak23}
\crossref{2Cor}{5}{15}{2Co 3:6 Eze 16:6; 37:9,\allowbreak14 Hab 2:4 Zec 10:9 Joh 3:15,\allowbreak16; 5:24; 6:57}
\crossref{2Cor}{5}{16}{De 33:9 1Sa 2:29 Mt 10:37; 12:48-\allowbreak50 Mr 3:31-\allowbreak35 Joh 2:4; 15:14}
\crossref{2Cor}{5}{17}{5:19,\allowbreak21; 12:2 Isa 45:17,\allowbreak24,\allowbreak25 Joh 14:20; 15:2,\allowbreak5; 17:23 Ro 8:1,\allowbreak9; 16:7,\allowbreak11}
\crossref{2Cor}{5}{18}{Joh 3:16,\allowbreak27 Ro 11:36 1Co 1:30; 8:6; 12:6 Col 1:16,\allowbreak17 Jas 1:17}
\crossref{2Cor}{5}{19}{Mt 1:23 Joh 14:10,\allowbreak11,\allowbreak20; 17:23 1Ti 3:16}
\crossref{2Cor}{5}{20}{2Co 3:6 Job 33:23 Pr 13:17 Mal 2:7 Joh 20:21 Ac 26:17,\allowbreak18 Eph 6:20}
\crossref{2Cor}{5}{21}{Isa 53:4-\allowbreak6,\allowbreak9-\allowbreak12 Da 9:26 Zec 13:7 Ro 8:3 Ga 3:13 Eph 5:2}
\crossref{2Cor}{6}{1}{2Co 5:18-\allowbreak20 1Co 3:9}
\crossref{2Cor}{6}{2}{Isa 49:8; 61:2 Eze 16:8 Lu 4:19; 19:42-\allowbreak44 Heb 3:7,\allowbreak13; 4:7}
\crossref{2Cor}{6}{3}{2Co 1:12; 8:20 Mt 17:27; 18:6 Ro 14:13 1Co 8:9-\allowbreak13; 9:12,\allowbreak22}
\crossref{2Cor}{6}{4}{2Co 2:17; 7:11 Ac 2:22 Ro 14:18; 16:10 1Co 9:11 1Th 2:3-\allowbreak11 1Ti 2:15}
\crossref{2Cor}{6}{5}{2Co 11:23-\allowbreak25 De 25:3 Isa 53:5 Ac 16:23}
\crossref{2Cor}{6}{6}{2Co 7:2 1Th 2:10 1Ti 4:12; 5:2 Tit 2:7}
\crossref{2Cor}{6}{7}{2Co 1:18-\allowbreak20; 4:2; 7:14 Ps 119:43 Eph 1:13; 4:21 Col 1:5 2Ti 2:15}
\crossref{2Cor}{6}{8}{Ac 4:21; 5:13,\allowbreak40,\allowbreak41; 14:11-\allowbreak20; 16:20-\allowbreak22,\allowbreak39; 28:4-\allowbreak10 1Co 4:10-\allowbreak13}
\crossref{2Cor}{6}{9}{Ac 17:18; 21:37,\allowbreak38; 25:14,\allowbreak15,\allowbreak19,\allowbreak26 1Co 4:9}
\crossref{2Cor}{6}{10}{2Co 2:4; 7:3-\allowbreak10 Mt 5:4,\allowbreak12 Lu 6:21 Joh 16:22 Ac 5:41; 16:25 Ro 5:2,\allowbreak3}
\crossref{2Cor}{6}{11}{Ga 3:1 Php 4:15}
\crossref{2Cor}{6}{12}{Ec 6:9}
\crossref{2Cor}{6}{13}{1Co 4:14,\allowbreak15 Ga 4:19 1Th 2:11 Heb 12:5,\allowbreak6 1Jo 2:1,\allowbreak12-\allowbreak14; 3:7,\allowbreak18}
\crossref{2Cor}{6}{14}{Ex 34:16 Le 19:19 De 7:2,\allowbreak3; 22:9-\allowbreak11 Ezr 9:1,\allowbreak2,\allowbreak11,\allowbreak12; 10:19}
\crossref{2Cor}{6}{15}{1Sa 5:2-\allowbreak4 1Ki 18:21 1Co 10:20,\allowbreak21}
\crossref{2Cor}{6}{16}{Ex 20:3; 23:13; 34:14 De 4:23,\allowbreak24; 5:7; 6:14,\allowbreak15 Jos 24:14-\allowbreak24}
\crossref{2Cor}{6}{17}{2Co 7:1 Nu 16:21,\allowbreak26,\allowbreak45 Ezr 6:21; 10:11 Ps 1:1-\allowbreak3 Pr 9:6 Isa 52:11}
\crossref{2Cor}{6}{18}{Ps 22:30 Jer 3:19; 31:1,\allowbreak9 Ho 1:9,\allowbreak10 Joh 1:12 Ro 8:14-\allowbreak17,\allowbreak29}
\crossref{2Cor}{7}{1}{2Co 1:20; 6:17,\allowbreak18 Ro 5:20,\allowbreak21; 6:1 etc.}
\crossref{2Cor}{7}{2}{2Co 11:16 Mt 10:14,\allowbreak40 Lu 10:8 Php 2:29 Col 4:10 Phm 1:12,\allowbreak17 2Jo 1:10}
\crossref{2Cor}{7}{3}{7:12; 2:4,\allowbreak5; 13:10 1Co 4:14,\allowbreak15}
\crossref{2Cor}{7}{4}{2Co 3:12; 6:11; 10:1,\allowbreak2; 11:21 Eph 6:19,\allowbreak20 Php 1:20 1Th 2:2}
\crossref{2Cor}{7}{5}{2Co 1:16,\allowbreak17; 2:13 Ac 20:1 1Co 16:5}
\crossref{2Cor}{7}{6}{2Co 1:3,\allowbreak4; 2:14 Isa 12:1; 51:12; 57:15,\allowbreak18; 61:1,\allowbreak2 Jer 31:13 Mt 5:4}
\crossref{2Cor}{7}{7}{Ac 11:23 Ro 1:12 Col 2:5 1Th 3:8 2Jo 1:4}
\crossref{2Cor}{7}{8}{7:6,\allowbreak11; 2:2-\allowbreak11 La 3:32 Mt 26:21,\allowbreak22 Lu 22:61,\allowbreak62 Joh 16:6; 21:17}
\crossref{2Cor}{7}{9}{7:6,\allowbreak7,\allowbreak10 Ec 7:3 Jer 31:18-\allowbreak20 Zec 12:10 Lu 15:7,\allowbreak10,\allowbreak17-\allowbreak24,\allowbreak32}
\crossref{2Cor}{7}{10}{2Co 12:21 2Sa 12:13 1Ki 8:47-\allowbreak50 Job 33:27,\allowbreak28 Jer 31:9 Eze 7:16}
\crossref{2Cor}{7}{11}{7:9 Isa 66:2 Zec 12:10-\allowbreak14 1Co 5:2}
\crossref{2Cor}{7}{12}{2Co 2:9 1Co 5:1}
\crossref{2Cor}{7}{13}{2Co 2:3 Ro 12:15 1Co 12:26; 13:5-\allowbreak7 Php 2:28 1Pe 3:8}
\crossref{2Cor}{7}{14}{7:4; 8:24; 9:2-\allowbreak4}
\crossref{2Cor}{7}{15}{2Co 6:12 Ge 43:30 1Ki 3:26 So 5:4 Php 1:8 Col 3:12 1Jo 3:17}
\crossref{2Cor}{7}{16}{2Th 3:4 Phm 8:21}
\crossref{2Cor}{8}{1}{8:19}
\crossref{2Cor}{8}{2}{1Th 1:6; 2:14; 3:3,\allowbreak4}
\crossref{2Cor}{8}{3}{2Co 9:6,\allowbreak7 Mr 14:8 Ac 11:29 1Co 16:2 1Pe 4:11}
\crossref{2Cor}{8}{4}{8:18,\allowbreak19 Ge 33:10,\allowbreak11 2Ki 5:15,\allowbreak16 Ac 16:15 1Co 16:3,\allowbreak4}
\crossref{2Cor}{8}{5}{2Co 5:14,\allowbreak15 1Sa 1:28 2Ch 30:8 Isa 44:3-\allowbreak5 Jer 31:33 Zec 13:9}
\crossref{2Cor}{8}{6}{8:16,\allowbreak17; 12:18}
\crossref{2Cor}{8}{7}{Ro 15:14 1Co 1:5; 4:7; 12:13; 14:12 Re 3:17}
\crossref{2Cor}{8}{8}{8:10; 9:7 1Co 7:6,\allowbreak12,\allowbreak25}
\crossref{2Cor}{8}{9}{2Co 13:14 Joh 1:14,\allowbreak17 Ro 5:8,\allowbreak20,\allowbreak21 1Co 1:4 Eph 1:6-\allowbreak8; 2:7; 3:8,\allowbreak19}
\crossref{2Cor}{8}{10}{1Co 7:25,\allowbreak40}
\crossref{2Cor}{8}{11}{}
\crossref{2Cor}{8}{12}{2Co 9:7 Ex 25:2; 35:5,\allowbreak21,\allowbreak22,\allowbreak29 1Ch 29:3-\allowbreak18 2Ch 6:8 Pr 19:22}
\crossref{2Cor}{8}{13}{Ac 4:34 Ro 15:26,\allowbreak27}
\crossref{2Cor}{8}{14}{8:14}
\crossref{2Cor}{8}{15}{Ex 16:18 Lu 22:35}
\crossref{2Cor}{8}{16}{Ezr 7:27 Ne 2:12 Jer 31:31; 32:40 Col 3:17 Re 17:17}
\crossref{2Cor}{8}{17}{8:6 Heb 13:22}
\crossref{2Cor}{8}{18}{Ro 16:4}
\crossref{2Cor}{8}{19}{8:1-\allowbreak4 Ac 6:3-\allowbreak6; 15:22,\allowbreak25 1Co 16:3,\allowbreak4}
\crossref{2Cor}{8}{20}{2Co 11:12 Mt 10:16 Ro 14:16 1Co 16:3 Eph 5:15 1Th 5:22}
\crossref{2Cor}{8}{21}{Ro 12:17 Php 4:8 1Ti 5:14 Tit 2:5-\allowbreak8 1Pe 2:12}
\crossref{2Cor}{8}{22}{Php 2:20-\allowbreak22}
\crossref{2Cor}{8}{23}{8:6,\allowbreak16; 7:6; 12:18}
\crossref{2Cor}{8}{24}{8:8; 7:14; 9:2-\allowbreak4}
\crossref{2Cor}{9}{1}{Ge 27:42 1Sa 20:23 2Ki 22:18 Job 37:23 Ps 45:1 Mt 22:31}
\crossref{2Cor}{9}{2}{2Co 8:8,\allowbreak10,\allowbreak19 1Th 1:7}
\crossref{2Cor}{9}{3}{9:4; 7:14; 8:6,\allowbreak17-\allowbreak24}
\crossref{2Cor}{9}{4}{9:2; 8:1-\allowbreak5}
\crossref{2Cor}{9}{5}{2Co 8:6 1Co 16:2}
\crossref{2Cor}{9}{6}{1Co 1:12; 7:29; 15:20 Ga 3:17; 5:16 Eph 4:17 Col 2:4}
\crossref{2Cor}{9}{7}{De 15:7-\allowbreak11,\allowbreak14 Pr 23:6-\allowbreak8 Isa 32:5,\allowbreak8 Jas 5:9 1Pe 4:9}
\crossref{2Cor}{9}{8}{2Ch 25:9 Ps 84:11 Pr 3:9; 10:22; 11:24; 28:27 Hag 2:8 Mal 3:10}
\crossref{2Cor}{9}{9}{Ps 112:9}
\crossref{2Cor}{9}{10}{Ge 1:11,\allowbreak12; 47:19,\allowbreak23,\allowbreak24 Isa 55:10}
\crossref{2Cor}{9}{11}{2Co 8:2,\allowbreak3 1Ch 29:12-\allowbreak14 2Ch 31:10 Pr 3:9,\allowbreak10 Mal 3:10,\allowbreak11 1Ti 6:17,\allowbreak18}
\crossref{2Cor}{9}{12}{9:1; 8:4}
\crossref{2Cor}{9}{13}{Ps 50:23 Mt 5:16 Joh 15:8 Ac 4:21; 11:18; 21:19,\allowbreak20 Ga 1:24}
\crossref{2Cor}{9}{14}{2Co 1:11 Ezr 6:8-\allowbreak10 Ps 41:1,\allowbreak2 Pr 11:26 Lu 16:9 Php 4:18,\allowbreak19}
\crossref{2Cor}{9}{15}{9:11; 2:14 1Ch 16:8,\allowbreak35 Ps 30:4,\allowbreak12; 92:1 Lu 2:14,\allowbreak38 1Co 15:57 Eph 5:20}
\crossref{2Cor}{10}{1}{1Co 16:21,\allowbreak22 Ga 5:2 2Th 3:17 Phm 1:9 Re 1:9}
\crossref{2Cor}{10}{2}{2Co 12:20; 13:2,\allowbreak10 1Co 4:19-\allowbreak21}
\crossref{2Cor}{10}{3}{Ga 2:20 1Pe 4:1,\allowbreak2}
\crossref{2Cor}{10}{4}{2Co 6:7 Ro 6:13}
\crossref{2Cor}{10}{5}{Lu 1:51 Ac 4:25,\allowbreak26 Ro 1:21 1Co 1:19,\allowbreak27-\allowbreak29; 3:19}
\crossref{2Cor}{10}{6}{2Co 13:2,\allowbreak10 Nu 16:26-\allowbreak30 Ac 5:3-\allowbreak11; 13:10,\allowbreak11 1Co 4:21; 5:3-\allowbreak5 1Ti 1:20}
\crossref{2Cor}{10}{7}{10:1}
\crossref{2Cor}{10}{8}{2Co 1:24; 13:2,\allowbreak3,\allowbreak8,\allowbreak10 Ga 1:1}
\crossref{2Cor}{10}{9}{10:10 1Co 4:5,\allowbreak19-\allowbreak21}
\crossref{2Cor}{10}{10}{10:11}
\crossref{2Cor}{10}{11}{2Co 12:20; 13:2,\allowbreak3,\allowbreak10 1Co 4:19,\allowbreak20}
\crossref{2Cor}{10}{12}{2Co 3:1; 5:12 Job 12:2 Pr 25:27; 27:2 Lu 18:11 Ro 15:18}
\crossref{2Cor}{10}{13}{10:15 Pr 25:14}
\crossref{2Cor}{10}{14}{2Co 3:1-\allowbreak3 Ro 15:18,\allowbreak19 1Co 2:10; 3:5,\allowbreak10; 4:15; 9:1,\allowbreak2}
\crossref{2Cor}{10}{15}{10:13 Ro 15:20}
\crossref{2Cor}{10}{16}{Ro 15:24-\allowbreak28}
\crossref{2Cor}{10}{17}{Ps 105:3; 106:5 Isa 41:16; 45:25; 65:16 Jer 4:2; 9:23,\allowbreak24}
\crossref{2Cor}{10}{18}{10:12; 3:1; 5:12 Pr 21:2; 27:2 Lu 16:15; 18:10-\allowbreak14}
\crossref{2Cor}{11}{1}{Nu 11:29 Jos 7:7 2Ki 5:3 Ac 26:29 1Co 4:8}
\crossref{2Cor}{11}{2}{Ga 4:11,\allowbreak17-\allowbreak19 Php 1:8 1Th 2:11}
\crossref{2Cor}{11}{3}{11:29; 12:20,\allowbreak21 Ps 119:53 Ga 1:6; 3:1; 4:11 Php 3:18,\allowbreak19}
\crossref{2Cor}{11}{4}{Ac 4:12 1Ti 2:5}
\crossref{2Cor}{11}{5}{2Co 12:11,\allowbreak12 1Co 15:10 Ga 2:6-\allowbreak9}
\crossref{2Cor}{11}{6}{2Co 10:10 1Co 1:17,\allowbreak21; 2:1-\allowbreak3,\allowbreak13}
\crossref{2Cor}{11}{7}{2Co 10:1; 12:13 Ac 18:1-\allowbreak3; 20:34 1Co 4:10-\allowbreak12; 9:6,\allowbreak12,\allowbreak14-\allowbreak18 1Th 2:9}
\crossref{2Cor}{11}{8}{11:9 Php 4:14-\allowbreak16}
\crossref{2Cor}{11}{9}{2Co 6:4; 9:12 Php 2:25; 4:11-\allowbreak14 Heb 11:37}
\crossref{2Cor}{11}{10}{11:31; 1:23; 12:19 Ro 1:9; 9:1 Ga 1:20 1Th 2:5,\allowbreak10 1Ti 2:7}
\crossref{2Cor}{11}{11}{2Co 6:11,\allowbreak12; 7:3; 12:15}
\crossref{2Cor}{11}{12}{11:9; 1:17 Job 23:13}
\crossref{2Cor}{11}{13}{11:15; 2:17; 4:2 Mt 25:24 Ac 15:1,\allowbreak24; 20:30 Ro 16:18 Ga 1:7; 2:4; 4:17}
\crossref{2Cor}{11}{14}{11:3; 2:11 Ge 3:1-\allowbreak5 Mt 4:1-\allowbreak10 Ga 1:8 Re 12:9}
\crossref{2Cor}{11}{15}{2Ki 5:13 1Co 9:11}
\crossref{2Cor}{11}{16}{11:1}
\crossref{2Cor}{11}{17}{1Co 7:6,\allowbreak12}
\crossref{2Cor}{11}{18}{11:12,\allowbreak21-\allowbreak23; 10:12-\allowbreak18 Jer 9:23,\allowbreak24 1Co 4:10 1Pe 1:24}
\crossref{2Cor}{11}{19}{1Co 4:10; 8:1; 10:15 Re 3:17}
\crossref{2Cor}{11}{20}{2Co 1:24 Ga 2:4; 4:3,\allowbreak9,\allowbreak25; 5:1,\allowbreak10; 6:12}
\crossref{2Cor}{11}{21}{2Co 10:1,\allowbreak2,\allowbreak10; 13:10}
\crossref{2Cor}{11}{22}{Ex 3:18; 5:3; 7:16; 9:1,\allowbreak13; 10:3 Ac 22:3 Ro 11:1 Php 3:5}
\crossref{2Cor}{11}{23}{2Co 3:6; 6:4 1Co 3:5; 4:1 1Th 3:2 1Ti 4:6}
\crossref{2Cor}{11}{24}{De 25:2,\allowbreak3 Mt 10:17 Mr 13:9}
\crossref{2Cor}{11}{25}{Ac 16:22,\allowbreak23,\allowbreak33,\allowbreak37; 22:24}
\crossref{2Cor}{11}{26}{Ac 9:23,\allowbreak26-\allowbreak30; 11:25,\allowbreak26; 13:1-\allowbreak14:28; 15:2-\allowbreak4,\allowbreak40,\allowbreak41; 16:1-\allowbreak18:1; 18:18-\allowbreak23}
\crossref{2Cor}{11}{27}{11:23; 6:5 Ac 20:5-\allowbreak11,\allowbreak34,\allowbreak35 1Th 2:9 2Th 3:8}
\crossref{2Cor}{11}{28}{11:23-\allowbreak27}
\crossref{2Cor}{11}{29}{2Co 2:4,\allowbreak5; 7:5,\allowbreak6; 13:9 Ezr 9:1-\allowbreak3 Ro 12:15; 15:1 1Co 8:13; 9:22; 12:26}
\crossref{2Cor}{11}{30}{11:16-\allowbreak18; 12:1,\allowbreak11 Pr 25:27; 27:2 Jer 9:23,\allowbreak24}
\crossref{2Cor}{11}{31}{2Co 1:3,\allowbreak23 Joh 10:30; 20:17 Ro 1:9; 9:1 Eph 1:3; 3:14 Ga 1:2,\allowbreak3}
\crossref{2Cor}{11}{32}{11:26 Ac 9:24,\allowbreak25}
\crossref{2Cor}{11}{33}{Jos 2:18 1Sa 19:12}
\crossref{2Cor}{12}{1}{2Co 8:10 Joh 16:7; 18:14 1Co 6:12; 10:23}
\crossref{2Cor}{12}{2}{12:3,\allowbreak5}
\crossref{2Cor}{12}{3}{}
\crossref{2Cor}{12}{4}{Eze 31:9 Lu 23:43 Re 2:7}
\crossref{2Cor}{12}{5}{12:2-\allowbreak4}
\crossref{2Cor}{12}{6}{2Co 10:8; 11:16 1Co 3:5,\allowbreak9,\allowbreak10}
\crossref{2Cor}{12}{7}{2Co 10:5; 11:20 De 8:14; 17:20 2Ch 26:16; 32:25,\allowbreak26,\allowbreak31 Da 5:20 1Ti 3:6}
\crossref{2Cor}{12}{8}{De 3:23-\allowbreak27 1Sa 15:11 2Sa 12:16-\allowbreak18 Ps 77:2-\allowbreak11 Mt 20:21,\allowbreak22}
\crossref{2Cor}{12}{9}{12:10; 3:5,\allowbreak6 Ex 3:11,\allowbreak12; 4:10-\allowbreak15 De 33:25-\allowbreak27 Jos 1:9 Isa 43:2}
\crossref{2Cor}{12}{10}{2Co 1:4; 4:8-\allowbreak10,\allowbreak17; 7:4 Ac 5:41 Ro 5:3; 8:35-\allowbreak39 Php 1:29; 2:17,\allowbreak18}
\crossref{2Cor}{12}{11}{2Co 1:6; 11:1,\allowbreak16,\allowbreak17}
\crossref{2Cor}{12}{12}{2Co 4:2; 6:4-\allowbreak10; 11:4,\allowbreak6 Ro 15:18,\allowbreak19 1Co 1:5-\allowbreak7; 9:2; 14:18}
\crossref{2Cor}{12}{13}{12:14; 11:8,\allowbreak9 1Co 9:6,\allowbreak12,\allowbreak15-\allowbreak18}
\crossref{2Cor}{12}{14}{Pr 11:30 Ac 20:33 1Co 10:33 Php 4:1,\allowbreak17 1Th 2:5,\allowbreak6,\allowbreak8,\allowbreak19,\allowbreak20}
\crossref{2Cor}{12}{15}{12:9; 1:6,\allowbreak14; 2:3; 7:3 Joh 10:10,\allowbreak11 Ga 4:10 Php 2:17 Col 1:24 1Th 2:8}
\crossref{2Cor}{12}{16}{12:13; 11:9,\allowbreak10}
\crossref{2Cor}{12}{17}{12:18 2Ki 5:16,\allowbreak20-\allowbreak27 1Co 4:17; 16:10}
\crossref{2Cor}{12}{18}{2Co 2:12,\allowbreak13; 7:2,\allowbreak6}
\crossref{2Cor}{12}{19}{2Co 3:1; 5:12}
\crossref{2Cor}{12}{20}{12:21; 13:9}
\crossref{2Cor}{12}{21}{12:7; 8:24; 9:3,\allowbreak4}
\crossref{2Cor}{13}{1}{2Co 12:14}
\crossref{2Cor}{13}{2}{2Co 1:23; 10:1,\allowbreak2,\allowbreak8-\allowbreak11; 12:20 1Co 4:19-\allowbreak21; 5:5}
\crossref{2Cor}{13}{3}{2Co 10:8-\allowbreak10}
\crossref{2Cor}{13}{4}{Lu 22:43,\allowbreak44 Joh 10:18 1Co 15:43 Php 2:7,\allowbreak8 Heb 5:7 1Pe 3:18}
\crossref{2Cor}{13}{5}{Ps 17:3; 26:2; 119:59; 139:23,\allowbreak24 La 3:40 Eze 18:28 Hag 1:5,\allowbreak7}
\crossref{2Cor}{13}{6}{13:3,\allowbreak4,\allowbreak10; 12:20}
\crossref{2Cor}{13}{7}{13:9 1Ch 4:10 Mt 6:13 Joh 17:15 Php 1:9-\allowbreak11 1Th 5:23 2Ti 4:18}
\crossref{2Cor}{13}{8}{13:10; 10:8 Nu 16:28-\allowbreak35 1Ki 22:28 2Ki 1:9-\allowbreak13; 2:23-\allowbreak25 Pr 21:30}
\crossref{2Cor}{13}{9}{13:8; 11:30; 12:5-\allowbreak10 1Co 4:10}
\crossref{2Cor}{13}{10}{2Co 2:3; 10:2; 12:20,\allowbreak21 1Co 4:21}
\crossref{2Cor}{13}{11}{Lu 9:61 Ac 15:29; 18:21; 23:30 Php 4:4 1Th 5:16}
\crossref{2Cor}{13}{12}{Ro 16:16 1Co 16:20 1Th 5:26 1Pe 5:14}
\crossref{2Cor}{13}{13}{Ro 16:16,\allowbreak21-\allowbreak23 Php 4:21,\allowbreak22 Phm 1:23,\allowbreak24 Heb 13:24 1Pe 5:13}
\crossref{2Cor}{13}{14}{Nu 6:23-\allowbreak27 Mt 28:19 Joh 1:16,\allowbreak17}

% Gal
\crossref{Gal}{1}{1}{Ro 1:1 1Co 1:1}
\crossref{Gal}{1}{2}{Php 2:22; 4:21}
\crossref{Gal}{1}{3}{Ro 1:7 etc.}
\crossref{Gal}{1}{4}{Ga 2:20 Mt 20:28; 26:28 Mr 10:45 Lu 22:19 Joh 10:11,\allowbreak17,\allowbreak18}
\crossref{Gal}{1}{5}{1Ch 29:13 Ps 41:13; 72:19 Isa 24:15; 42:12 Mt 6:13 Lu 2:14}
\crossref{Gal}{1}{6}{Mr 6:6 Joh 9:30}
\crossref{Gal}{1}{7}{Ga 2:4; 4:17; 5:10,\allowbreak12; 6:12,\allowbreak13,\allowbreak17 Ac 15:1-\allowbreak5,\allowbreak24; 20:30 Ro 16:17,\allowbreak18}
\crossref{Gal}{1}{8}{1:9 1Co 16:22 2Co 11:13,\allowbreak14 1Ti 1:19,\allowbreak20 Tit 3:10 Re 22:18,\allowbreak19}
\crossref{Gal}{1}{9}{2Co 1:17; 13:1,\allowbreak2 Php 3:1; 4:4}
\crossref{Gal}{1}{10}{Ac 4:19,\allowbreak20; 5:29 2Co 5:9-\allowbreak11 1Th 2:4}
\crossref{Gal}{1}{11}{1:1 1Co 2:9,\allowbreak10; 11:23; 15:1-\allowbreak3 Eph 3:3-\allowbreak8}
\crossref{Gal}{1}{12}{}
\crossref{Gal}{1}{13}{Ac 22:3-\allowbreak5; 26:4,\allowbreak5}
\crossref{Gal}{1}{14}{Isa 29:13; 57:12}
\crossref{Gal}{1}{15}{De 7:7,\allowbreak8 1Sa 12:22 1Ch 28:4,\allowbreak5 Mt 11:26 Lu 10:21 1Co 1:1}
\crossref{Gal}{1}{16}{Mt 16:17 1Co 2:9-\allowbreak13 2Co 4:6 Eph 1:17,\allowbreak18; 3:5-\allowbreak10}
\crossref{Gal}{1}{17}{1:18 Ac 9:20-\allowbreak25}
\crossref{Gal}{1}{18}{Ac 9:26-\allowbreak29; 22:17,\allowbreak18}
\crossref{Gal}{1}{19}{Mt 10:3 Mr 3:18 Lu 6:15 Ac 1:13}
\crossref{Gal}{1}{20}{Ro 9:1 2Co 11:10,\allowbreak11,\allowbreak31}
\crossref{Gal}{1}{21}{Ac 9:30; 11:25,\allowbreak26; 13:1; 15:23,\allowbreak41; 18:18; 21:3}
\crossref{Gal}{1}{22}{Ac 9:31 1Th 2:14}
\crossref{Gal}{1}{23}{Ac 9:13,\allowbreak20,\allowbreak26 1Co 15:8-\allowbreak10 1Ti 1:13-\allowbreak16}
\crossref{Gal}{1}{24}{Nu 23:23 Lu 2:14; 7:16; 15:10,\allowbreak32 Ac 11:18; 21:19,\allowbreak20 2Co 9:13}
\crossref{Gal}{2}{1}{Ga 1:18}
\crossref{Gal}{2}{2}{Ac 16:9,\allowbreak10; 18:9; 23:11}
\crossref{Gal}{2}{3}{Ga 5:2-\allowbreak6 Ac 15:24; 16:3 1Co 9:20,\allowbreak21}
\crossref{Gal}{2}{4}{Ga 5:10,\allowbreak12 Ac 15:1,\allowbreak24; 20:30 2Co 11:13,\allowbreak17,\allowbreak26 1Jo 4:1}
\crossref{Gal}{2}{5}{Ga 3:1,\allowbreak2 Ac 15:2 Col 2:4-\allowbreak8 Jude 1:3}
\crossref{Gal}{2}{6}{2:2,\allowbreak9; 6:3 2Co 11:5,\allowbreak21-\allowbreak23; 12:11 Heb 13:7,\allowbreak17}
\crossref{Gal}{2}{7}{2:9 Ac 15:12,\allowbreak25,\allowbreak26 2Pe 3:15}
\crossref{Gal}{2}{8}{Ac 1:8; 2:14-\allowbreak41; 3:12-\allowbreak26; 4:4; 5:12-\allowbreak16; 8:17}
\crossref{Gal}{2}{9}{Ac 15:7,\allowbreak13,\allowbreak22-\allowbreak29}
\crossref{Gal}{2}{10}{Ac 11:29,\allowbreak30; 24:17 Ro 15:25-\allowbreak27 1Co 16:1,\allowbreak2 2Co 8:1-\allowbreak9:15}
\crossref{Gal}{2}{11}{Ac 15:30-\allowbreak35}
\crossref{Gal}{2}{12}{2:9 Ac 21:18-\allowbreak25}
\crossref{Gal}{2}{13}{Ge 12:11-\allowbreak13; 26:6,\allowbreak7; 27:24 Ec 7:20; 10:1 1Co 5:6; 8:9; 15:33}
\crossref{Gal}{2}{14}{Ps 15:2; 58:1; 84:11 Pr 2:7; 10:9}
\crossref{Gal}{2}{15}{Mt 3:7-\allowbreak9 Joh 8:39-\allowbreak41 Ro 4:16 Eph 2:3}
\crossref{Gal}{2}{16}{2:19; 3:10-\allowbreak12; 5:4 Job 9:2,\allowbreak3,\allowbreak29; 25:4 Ps 130:3,\allowbreak4 Lu 10:25-\allowbreak29}
\crossref{Gal}{2}{17}{Ro 9:30-\allowbreak33; 11:7}
\crossref{Gal}{2}{18}{2:4,\allowbreak5,\allowbreak12-\allowbreak16,\allowbreak21; 4:9-\allowbreak12; 5:11 Ro 14:15 1Co 8:11,\allowbreak12}
\crossref{Gal}{2}{19}{Ga 3:10,\allowbreak24 Ro 3:19,\allowbreak20; 4:15; 5:20; 7:7-\allowbreak11,\allowbreak14,\allowbreak22,\allowbreak23; 8:2; 10:4,\allowbreak5}
\crossref{Gal}{2}{20}{Ga 5:24; 6:14 Ro 6:4-\allowbreak6; 8:3,\allowbreak4 Col 2:11-\allowbreak14}
\crossref{Gal}{2}{21}{2:18 Ps 33:10 Mr 7:9}
\crossref{Gal}{3}{1}{3:3 De 32:6 1Sa 13:13 Mt 7:26 Lu 24:25 Eph 5:15 1Ti 6:4}
\crossref{Gal}{3}{2}{3:5,\allowbreak14 Ac 2:38; 8:15; 10:44-\allowbreak47; 11:15-\allowbreak18; 15:8; 19:2-\allowbreak6 1Co 12:7-\allowbreak13}
\crossref{Gal}{3}{3}{Ga 4:7-\allowbreak10; 5:4-\allowbreak8; 6:12-\allowbreak14 Heb 7:16-\allowbreak19; 9:2,\allowbreak9,\allowbreak10}
\crossref{Gal}{3}{4}{Eze 18:24 Heb 6:4-\allowbreak6; 10:32-\allowbreak39 2Pe 2:20-\allowbreak22 2Jo 1:8}
\crossref{Gal}{3}{5}{3:2 2Co 3:8}
\crossref{Gal}{3}{6}{3:9 Ge 15:6 Ro 4:3-\allowbreak6,\allowbreak9,\allowbreak10,\allowbreak21,\allowbreak22; 9:32,\allowbreak33 Jas 2:23}
\crossref{Gal}{3}{7}{Ps 100:3 Lu 21:31 Heb 13:23}
\crossref{Gal}{3}{8}{3:22; 4:30 Joh 7:38,\allowbreak42; 19:37 Ro 9:17 2Ti 3:15-\allowbreak17}
\crossref{Gal}{3}{9}{3:7,\allowbreak8,\allowbreak14,\allowbreak29; 4:28 Ro 4:11,\allowbreak16,\allowbreak24}
\crossref{Gal}{3}{10}{3:11}
\crossref{Gal}{3}{11}{Ga 2:16 1Ki 8:46 Job 9:3; 40:4; 42:6 Ps 19:12; 130:3,\allowbreak4; 143:2 Ec 7:20}
\crossref{Gal}{3}{12}{Ro 4:4,\allowbreak5,\allowbreak14,\allowbreak16; 9:30-\allowbreak32; 10:5,\allowbreak6; 11:6}
\crossref{Gal}{3}{13}{3:10; 4:5 Isa 55:5-\allowbreak7,\allowbreak10-\allowbreak12 Da 9:24,\allowbreak26 Zec 13:7 Mt 26:28}
\crossref{Gal}{3}{14}{3:6-\allowbreak9,\allowbreak29}
\crossref{Gal}{3}{15}{Ro 6:19 1Co 15:32}
\crossref{Gal}{3}{16}{3:8 Ge 12:3,\allowbreak7; 13:15,\allowbreak16; 15:5; 17:7,\allowbreak8; 21:12; 22:17,\allowbreak18; 26:3,\allowbreak4; 28:13}
\crossref{Gal}{3}{17}{Ga 5:16 1Co 1:12; 7:29; 10:19 2Co 9:6 Eph 4:17 Col 2:4}
\crossref{Gal}{3}{18}{3:10,\allowbreak12,\allowbreak26,\allowbreak29; 2:21 Ro 4:13-\allowbreak16; 8:17}
\crossref{Gal}{3}{19}{Ro 3:1,\allowbreak2; 7:7-\allowbreak13}
\crossref{Gal}{3}{20}{Job 9:33 Ac 12:20 1Ti 2:5}
\crossref{Gal}{3}{21}{Mt 5:17-\allowbreak20 Ro 3:31; 7:7-\allowbreak13}
\crossref{Gal}{3}{22}{3:8-\allowbreak10,\allowbreak23 Ps 143:2 Ro 3:9-\allowbreak20,\allowbreak23; 5:12,\allowbreak20; 11:32}
\crossref{Gal}{3}{23}{3:19,\allowbreak24,\allowbreak25; 4:1-\allowbreak4 Heb 12:2}
\crossref{Gal}{3}{24}{3:25; 2:19; 4:2,\allowbreak3 Mt 5:17,\allowbreak18 Ac 13:38,\allowbreak39 Ro 3:20-\allowbreak22; 7:7-\allowbreak9,\allowbreak24,\allowbreak25; 10:4}
\crossref{Gal}{3}{25}{3:23}
\crossref{Gal}{3}{26}{Ga 4:5,\allowbreak6 Joh 1:12,\allowbreak13; 20:17 Ro 8:14-\allowbreak17 2Co 6:18 Eph 1:5; 5:1}
\crossref{Gal}{3}{27}{Mt 28:19,\allowbreak20 Mr 16:15,\allowbreak16 Ac 2:38; 8:36-\allowbreak38; 9:18; 16:15,\allowbreak31-\allowbreak33}
\crossref{Gal}{3}{28}{Ga 5:6 Ro 1:16; 2:9,\allowbreak10; 3:29,\allowbreak30; 4:11,\allowbreak12; 9:24; 10:12-\allowbreak15}
\crossref{Gal}{3}{29}{Ga 5:24 1Co 3:23; 15:23 2Co 10:7}
\crossref{Gal}{4}{1}{4:23,\allowbreak29 Ge 24:2,\allowbreak3 2Ki 10:1,\allowbreak2; 11:12; 12:2}
\crossref{Gal}{4}{2}{}
\crossref{Gal}{4}{3}{Ga 3:19,\allowbreak24,\allowbreak25}
\crossref{Gal}{4}{4}{Ge 49:10 Da 9:24-\allowbreak26 Mal 3:1 Mr 1:15 Ac 1:7 Eph 1:10 Heb 9:10}
\crossref{Gal}{4}{5}{4:21; 3:13 Mt 20:28 Lu 1:68 Ac 20:28 Eph 1:7; 5:2 Col 1:13-\allowbreak20}
\crossref{Gal}{4}{6}{Lu 11:13 Joh 7:39; 14:16 Ro 5:5; 8:15-\allowbreak17 2Co 1:22 Eph 1:13; 4:30}
\crossref{Gal}{4}{7}{4:1,\allowbreak2,\allowbreak5,\allowbreak6,\allowbreak31; 5:1}
\crossref{Gal}{4}{8}{Ex 5:2 Jer 10:25 Joh 1:10 Ac 17:23,\allowbreak30 Ro 1:28 1Co 1:21}
\crossref{Gal}{4}{9}{1Ki 8:43 1Ch 28:9 Ps 9:10 Pr 2:5 Jer 31:34 Hab 2:14 Mt 11:27}
\crossref{Gal}{4}{10}{Le 23:1-\allowbreak44; 25:1,\allowbreak13 Nu 28:1-\allowbreak29:40 Ro 14:5 Col 2:16,\allowbreak17}
\crossref{Gal}{4}{11}{4:20 2Co 11:2,\allowbreak3; 12:20,\allowbreak21}
\crossref{Gal}{4}{12}{Ga 2:14; 6:14 Ge 34:15 1Ki 22:4 Ac 21:21 1Co 9:20-\allowbreak23 Php 3:7,\allowbreak8}
\crossref{Gal}{4}{13}{1Co 2:3 2Co 10:10; 11:6,\allowbreak30; 12:7-\allowbreak10; 13:4}
\crossref{Gal}{4}{14}{4:13 Job 12:5 Ps 119:141 Ec 9:16 Isa 53:2,\allowbreak3 1Co 1:28; 4:10}
\crossref{Gal}{4}{15}{Ga 3:14; 5:22; 6:4 Lu 8:13 Ro 4:6-\allowbreak9; 5:2; 15:13}
\crossref{Gal}{4}{16}{Ga 3:1-\allowbreak4 1Ki 18:17,\allowbreak18; 21:20; 22:8,\allowbreak27 2Ch 24:20-\allowbreak22; 25:16 Ps 141:5}
\crossref{Gal}{4}{17}{Ga 6:12,\allowbreak13 Mt 23:15 Ro 10:2; 16:18 1Co 11:2 2Co 11:3,\allowbreak13-\allowbreak15}
\crossref{Gal}{4}{18}{Nu 25:11-\allowbreak13 Ps 69:9; 119:139 Isa 59:17 Joh 2:17 1Co 15:58}
\crossref{Gal}{4}{19}{1Co 4:14 1Ti 1:2 Tit 1:4 Phm 1:10,\allowbreak19 Jas 1:18 1Jo 2:1,\allowbreak12; 5:21}
\crossref{Gal}{4}{20}{1Co 4:19-\allowbreak21 1Th 2:17,\allowbreak18; 3:9}
\crossref{Gal}{4}{21}{4:9; 3:10,\allowbreak23,\allowbreak24 Ro 6:14; 7:5,\allowbreak6; 9:30-\allowbreak32; 10:3-\allowbreak10}
\crossref{Gal}{4}{22}{Ge 16:2-\allowbreak4,\allowbreak15; 21:1,\allowbreak2,\allowbreak10}
\crossref{Gal}{4}{23}{Ro 9:7,\allowbreak8}
\crossref{Gal}{4}{24}{Eze 20:49 Ho 11:10 Mt 13:35 1Co 10:11}
\crossref{Gal}{4}{25}{4:24}
\crossref{Gal}{4}{26}{Ps 87:3-\allowbreak6 Isa 2:2,\allowbreak3; 52:9; 62:1,\allowbreak2; 65:18; 66:10 Joe 3:17 Mic 4:1,\allowbreak2}
\crossref{Gal}{4}{27}{Isa 54:1-\allowbreak5}
\crossref{Gal}{4}{28}{4:23; 3:29 Ac 3:25 Ro 4:13-\allowbreak18; 9:8,\allowbreak9}
\crossref{Gal}{4}{29}{Ge 21:9}
\crossref{Gal}{4}{30}{Ga 3:8,\allowbreak22 Ro 4:3; 11:2 Jas 4:5}
\crossref{Gal}{4}{31}{Ga 5:1,\allowbreak13 Joh 1:12,\allowbreak13; 8:36 Heb 2:14,\allowbreak15 1Jo 3:1,\allowbreak2}
\crossref{Gal}{5}{1}{Pr 23:23 1Co 15:58; 16:13 Eph 6:14 Php 1:27 1Th 3:8 2Th 2:15}
\crossref{Gal}{5}{2}{1Co 16:21 2Co 10:1 1Th 2:18 Phm 1:9}
\crossref{Gal}{5}{3}{De 8:19; 31:21 Ne 9:29,\allowbreak30,\allowbreak34 Lu 16:28 Ac 2:40; 20:21 Eph 4:17}
\crossref{Gal}{5}{4}{5:2; 2:21 Ro 9:31,\allowbreak32; 10:3-\allowbreak5}
\crossref{Gal}{5}{5}{Joh 16:8-\allowbreak15 Eph 2:18}
\crossref{Gal}{5}{6}{5:2,\allowbreak3; 3:28; 6:15 Ro 2:25-\allowbreak29; 3:29-\allowbreak31 1Co 7:19 Col 3:11}
\crossref{Gal}{5}{7}{Mt 13:21 1Co 9:24 Heb 12:1}
\crossref{Gal}{5}{8}{Ga 1:6}
\crossref{Gal}{5}{9}{Mt 23:33; 16:6-\allowbreak12 Mr 8:15 Lu 12:1; 13:21 1Co 5:6,\allowbreak7; 15:33}
\crossref{Gal}{5}{10}{Ga 4:11,\allowbreak20 2Co 1:15; 2:3; 7:16; 8:22 2Th 3:4 Phm 1:21}
\crossref{Gal}{5}{11}{Ga 2:3 Ac 16:3}
\crossref{Gal}{5}{12}{5:10; 1:8,\allowbreak9 Ge 17:14 Ex 12:15; 30:33 Le 22:3 Jos 7:12,\allowbreak25 Joh 9:34}
\crossref{Gal}{5}{13}{5:1; 4:5-\allowbreak7,\allowbreak22-\allowbreak31 Isa 61:1 Lu 4:18 Joh 8:32-\allowbreak36 Ro 6:18-\allowbreak22}
\crossref{Gal}{5}{14}{Mt 7:12; 19:18,\allowbreak19; 22:39,\allowbreak40 Ro 13:8-\allowbreak10 Jas 2:8-\allowbreak11}
\crossref{Gal}{5}{15}{5:26 2Sa 2:26,\allowbreak27 Isa 9:20,\allowbreak21; 11:5-\allowbreak9,\allowbreak13 1Co 3:3; 6:6-\allowbreak8}
\crossref{Gal}{5}{16}{Ga 3:17 1Co 7:29}
\crossref{Gal}{5}{17}{Ps 19:12,\allowbreak13; 51:1-\allowbreak5,\allowbreak10-\allowbreak12; 65:3; 119:5,\allowbreak20,\allowbreak24,\allowbreak25,\allowbreak32,\allowbreak35,\allowbreak40,\allowbreak133,\allowbreak159}
\crossref{Gal}{5}{18}{5:16,\allowbreak25; 4:6 Ps 25:4,\allowbreak5,\allowbreak8,\allowbreak9; 143:8-\allowbreak10 Pr 8:20 Isa 48:16-\allowbreak18 Eze 36:27}
\crossref{Gal}{5}{19}{5:13,\allowbreak17; 6:8 Ps 17:4 Joh 3:6 Ro 7:5,\allowbreak18,\allowbreak25; 8:3,\allowbreak5,\allowbreak9,\allowbreak13 1Co 3:3 1Pe 4:2}
\crossref{Gal}{5}{20}{Eze 22:18 De 18:10 1Sa 15:23 1Ch 10:13,\allowbreak14 2Ch 33:6}
\crossref{Gal}{5}{21}{De 21:20 Lu 21:34 Ro 13:13 1Co 5:11; 6:10 Eph 5:18 1Th 5:7}
\crossref{Gal}{5}{22}{5:16-\allowbreak18 Ps 1:3; 92:14 Ho 14:8 Mt 12:33 Lu 8:14,\allowbreak15; 13:9}
\crossref{Gal}{5}{23}{Ac 24:25 1Co 9:25 Tit 1:8; 2:2}
\crossref{Gal}{5}{24}{Ga 3:29 Ro 8:9 1Co 3:23; 15:23 2Co 10:7}
\crossref{Gal}{5}{25}{Joh 6:63 Ro 8:2,\allowbreak10 1Co 15:45 2Co 3:6 1Pe 4:6 Re 11:11}
\crossref{Gal}{5}{26}{Lu 14:10 1Co 3:7 Php 2:1-\allowbreak3 Jas 4:16}
\crossref{Gal}{6}{1}{Ga 2:11-\allowbreak13 Ge 9:20-\allowbreak24; 12:11-\allowbreak13 Nu 20:10-\allowbreak13 2Sa 11:2 etc.}
\crossref{Gal}{6}{2}{6:5; 5:13,\allowbreak14 Ex 23:5 Nu 11:11,\allowbreak12 De 1:12 Isa 58:6}
\crossref{Gal}{6}{3}{Ga 2:6 Pr 25:14; 26:12 Lu 18:11 Ro 12:3,\allowbreak16 1Co 3:18; 8:2}
\crossref{Gal}{6}{4}{Pr 14:14 1Co 4:3,\allowbreak4 2Co 1:12 1Jo 3:19-\allowbreak22}
\crossref{Gal}{6}{5}{Isa 3:10,\allowbreak11 Jer 17:10; 32:19 Eze 18:4 Mt 16:27 Ro 2:6-\allowbreak9}
\crossref{Gal}{6}{6}{De 12:19 Mt 10:10 Ro 15:27 1Co 9:9-\allowbreak14 1Ti 5:17,\allowbreak18}
\crossref{Gal}{6}{7}{6:3 Job 15:31 Jer 37:9 Ob 1:3 Lu 21:8 1Co 3:18; 6:9; 15:33}
\crossref{Gal}{6}{8}{Ro 6:13; 8:13; 13:14 Jas 3:18}
\crossref{Gal}{6}{9}{Mal 1:13 1Co 15:58 2Th 3:13 Heb 12:3}
\crossref{Gal}{6}{10}{Ec 9:10 Joh 9:4; 12:35 Eph 5:16 Php 4:10 Col 4:5}
\crossref{Gal}{6}{11}{Ro 16:22 1Co 16:21-\allowbreak23}
\crossref{Gal}{6}{12}{6:13 Mt 6:2,\allowbreak5,\allowbreak16; 23:5,\allowbreak28 Lu 16:15; 20:47 Joh 7:18 2Co 10:12}
\crossref{Gal}{6}{13}{Mt 23:3,\allowbreak15,\allowbreak23 Ro 2:17-\allowbreak24; 3:9-\allowbreak19 2Pe 2:19}
\crossref{Gal}{6}{14}{Ro 3:4-\allowbreak6 Php 3:3,\allowbreak7,\allowbreak8}
\crossref{Gal}{6}{15}{Ga 5:6 Ro 8:1 2Co 5:17}
\crossref{Gal}{6}{16}{Ga 5:16,\allowbreak25 Ps 125:4,\allowbreak5 Php 3:16}
\crossref{Gal}{6}{17}{Ga 1:7; 5:12 Jos 7:25 Ac 15:24 Heb 12:15}
\crossref{Gal}{6}{18}{Ro 16:20,\allowbreak24 2Co 13:14 2Ti 4:22 Phm 1:25 Re 22:21}

% Eph
\crossref{Eph}{1}{1}{Ro 1:1 1Co 1:1 Ga 1:1}
\crossref{Eph}{1}{2}{Ro 1:7 2Co 1:2 Ga 1:3 Tit 1:4}
\crossref{Eph}{1}{3}{Ge 14:20 1Ch 29:20 Ne 9:5 Ps 72:19 Da 4:34 Lu 2:28 2Co 1:3}
\crossref{Eph}{1}{4}{De 7:6,\allowbreak7 Ps 135:4 Isa 41:8,\allowbreak9; 42:1; 65:8-\allowbreak10 Mt 11:25,\allowbreak26; 24:22,\allowbreak24}
\crossref{Eph}{1}{5}{1:11 Ro 8:29,\allowbreak30}
\crossref{Eph}{1}{6}{1:7,\allowbreak8,\allowbreak12,\allowbreak14,\allowbreak18; 2:7; 3:10,\allowbreak11 Pr 16:4 Isa 43:21; 61:3,\allowbreak11 Jer 33:9 Lu 2:14}
\crossref{Eph}{1}{7}{Job 33:24 Ps 130:7 Da 9:24-\allowbreak26 Zec 9:11; 13:1,\allowbreak7 Mt 20:28; 26:28}
\crossref{Eph}{1}{8}{Ro 5:15,\allowbreak20,\allowbreak21}
\crossref{Eph}{1}{9}{1:17,\allowbreak18; 3:3-\allowbreak9 Mt 13:11 Ro 16:25-\allowbreak27 1Co 2:10-\allowbreak12 Ga 1:12,\allowbreak16 Col 1:26-\allowbreak28}
\crossref{Eph}{1}{10}{Isa 2:2-\allowbreak4 Da 2:44; 9:24-\allowbreak27 Am 9:11 Mic 4:1,\allowbreak2 Mal 3:1 1Co 10:11}
\crossref{Eph}{1}{11}{1:14 Ps 37:18 Ac 20:32; 26:18 Ro 8:17 Ga 3:18 Col 1:12; 3:24}
\crossref{Eph}{1}{12}{1:6,\allowbreak14; 2:7; 3:21 2Th 2:13}
\crossref{Eph}{1}{13}{Eph 2:11,\allowbreak12 Col 1:21-\allowbreak23 1Pe 2:10}
\crossref{Eph}{1}{14}{Ro 8:15-\allowbreak17,\allowbreak23 2Co 1:22; 5:5 Ga 4:6}
\crossref{Eph}{1}{15}{Col 1:3,\allowbreak4 Phm 1:5}
\crossref{Eph}{1}{16}{Ro 1:8,\allowbreak9 1Sa 7:8; 12:23 Php 1:3,\allowbreak4 Col 1:3 1Th 5:17 2Th 1:3}
\crossref{Eph}{1}{17}{1:3 Joh 20:17}
\crossref{Eph}{1}{18}{Eph 5:8 Ps 119:18 Isa 6:10; 29:10,\allowbreak18; 32:3; 42:7 Mt 13:15 Lu 24:45}
\crossref{Eph}{1}{19}{Eph 2:10; 3:7,\allowbreak20 Ps 110:2,\allowbreak3 Isa 53:1 Joh 3:6 Ac 26:18 Ro 1:16}
\crossref{Eph}{1}{20}{Eph 2:5,\allowbreak6 Ro 6:5-\allowbreak11 Php 3:10 1Pe 1:3}
\crossref{Eph}{1}{21}{Php 2:9,\allowbreak10 Col 2:10 Heb 1:4}
\crossref{Eph}{1}{22}{Ge 3:15 Ps 8:6-\allowbreak8; 91:13 1Co 15:25-\allowbreak27 Heb 2:8}
\crossref{Eph}{1}{23}{Eph 2:16; 4:4,\allowbreak12; 5:23-\allowbreak32 Ro 13:5 1Co 12:12-\allowbreak27 Col 1:18,\allowbreak24; 3:15}
\crossref{Eph}{2}{1}{2:5,\allowbreak6; 1:19,\allowbreak20 Joh 5:25; 10:10; 11:25,\allowbreak26; 14:6 Ro 8:2 1Co 15:45}
\crossref{Eph}{2}{2}{2:3; 4:22 Job 31:7 Ac 19:35 1Co 6:11 Col 1:21; 3:7 1Pe 4:3 1Jo 5:19}
\crossref{Eph}{2}{3}{Isa 53:6; 64:6,\allowbreak7 Da 9:5-\allowbreak9 Ro 3:9-\allowbreak19 1Co 6:9-\allowbreak11 Ga 2:15,\allowbreak16; 3:22}
\crossref{Eph}{2}{4}{2:7; 1:7; 3:8 Ex 33:19; 34:6,\allowbreak7 Ne 9:17 Ps 51:1; 86:5,\allowbreak15; 103:8-\allowbreak11; 145:8}
\crossref{Eph}{2}{5}{2:1 Ro 5:6,\allowbreak8,\allowbreak10}
\crossref{Eph}{2}{6}{Eph 1:19,\allowbreak20 Ro 6:4,\allowbreak5 Col 1:18; 2:12,\allowbreak13; 3:1-\allowbreak3}
\crossref{Eph}{2}{7}{Eph 3:5,\allowbreak21 Ps 41:13; 106:48 Isa 60:15 1Ti 1:17}
\crossref{Eph}{2}{8}{2:5 Ro 3:24 2Th 1:9}
\crossref{Eph}{2}{9}{Ro 3:20,\allowbreak27,\allowbreak28; 4:2; 9:11,\allowbreak16; 11:6 1Co 1:29-\allowbreak31 2Ti 1:9 Tit 3:3-\allowbreak5}
\crossref{Eph}{2}{10}{De 32:6 Ps 100:3; 138:8 Isa 19:25; 29:23; 43:21; 44:21; 60:21; 61:3}
\crossref{Eph}{2}{11}{Eph 5:8 De 5:15; 8:2; 9:7; 15:15; 16:12 Isa 51:1,\allowbreak2 Eze 16:61-\allowbreak63; 20:43}
\crossref{Eph}{2}{12}{Joh 10:16; 15:5 Col 1:21}
\crossref{Eph}{2}{13}{Ro 8:1 1Co 1:30 2Co 5:17 Ga 3:28}
\crossref{Eph}{2}{14}{Isa 9:6,\allowbreak7 Eze 34:24,\allowbreak25 Mic 5:5 Zec 6:13 Lu 1:79; 2:14 Joh 16:33}
\crossref{Eph}{2}{15}{Col 1:22 Heb 10:19-\allowbreak22}
\crossref{Eph}{2}{16}{Ro 5:10 2Co 5:18-\allowbreak21 Col 1:21-\allowbreak22}
\crossref{Eph}{2}{17}{Ps 85:10 Isa 27:5; 52:7; 57:19-\allowbreak21 Zec 9:10 Mt 10:13 Lu 2:14}
\crossref{Eph}{2}{18}{Eph 3:12 Joh 10:7,\allowbreak9; 14:6 Ro 5:2 Heb 4:15,\allowbreak16; 7:19; 10:19,\allowbreak20 1Pe 1:21}
\crossref{Eph}{2}{19}{2:12}
\crossref{Eph}{2}{20}{Eph 4:12 1Pe 2:4,\allowbreak5}
\crossref{Eph}{2}{21}{Eph 4:13-\allowbreak16 Eze 40:1-\allowbreak42:20 1Co 3:9 Heb 3:3,\allowbreak4}
\crossref{Eph}{2}{22}{Joh 14:17-\allowbreak23; 17:21-\allowbreak23 Ro 8:9-\allowbreak11 1Co 3:16; 6:19 1Pe 2:4,\allowbreak5}
\crossref{Eph}{3}{1}{2Co 10:1 Ga 5:2}
\crossref{Eph}{3}{2}{Eph 4:21 Ga 1:13 Col 1:4,\allowbreak6 2Ti 1:11}
\crossref{Eph}{3}{3}{Eph 1:17 Ac 22:17,\allowbreak21; 23:9; 26:15-\allowbreak19 1Co 2:9,\allowbreak10 Ga 1:12,\allowbreak16-\allowbreak19}
\crossref{Eph}{3}{4}{Mt 13:11 1Co 2:6,\allowbreak7; 13:2 2Co 11:6}
\crossref{Eph}{3}{5}{3:9 Mt 13:17 Lu 10:24 Ac 10:28 Ro 16:25 2Ti 1:10,\allowbreak11 Tit 1:1-\allowbreak3}
\crossref{Eph}{3}{6}{Eph 2:13-\allowbreak22 Ro 8:15-\allowbreak17 Ga 3:26-\allowbreak29; 4:5-\allowbreak7}
\crossref{Eph}{3}{7}{3:2 Ro 15:16 2Co 3:6; 4:1 Col 1:23-\allowbreak25}
\crossref{Eph}{3}{8}{Pr 30:2,\allowbreak3 Ro 12:10 1Co 15:9 Php 2:3 1Ti 1:13,\allowbreak15 1Pe 5:5,\allowbreak6}
\crossref{Eph}{3}{9}{Mt 10:27; 28:19 Mr 16:15,\allowbreak16 Lu 24:47 Ro 16:26 Col 1:23 2Ti 4:17}
\crossref{Eph}{3}{10}{Ex 25:17-\allowbreak22 Ps 103:20; 148:1,\allowbreak2 Isa 6:2-\allowbreak4 Eze 3:12 1Pe 1:12}
\crossref{Eph}{3}{11}{Eph 1:4,\allowbreak9,\allowbreak11 Isa 14:24-\allowbreak27; 46:10,\allowbreak11 Jer 51:29 Ro 8:28-\allowbreak30; 9:11}
\crossref{Eph}{3}{12}{Eph 2:18 Joh 14:6 Ro 5:2 Heb 4:14-\allowbreak16; 10:19-\allowbreak22}
\crossref{Eph}{3}{13}{De 20:3 Isa 40:30,\allowbreak31 Zep 3:16 Ac 14:22 Ga 6:9 2Th 3:13}
\crossref{Eph}{3}{14}{Eph 1:16-\allowbreak19 1Ki 8:54; 19:18 2Ch 6:13 Ezr 9:5 Ps 95:6 Isa 45:23}
\crossref{Eph}{3}{15}{Eph 1:10,\allowbreak21 Php 2:9-\allowbreak11 Col 1:20 Re 5:8-\allowbreak14; 7:4-\allowbreak12}
\crossref{Eph}{3}{16}{3:8; 1:7,\allowbreak18; 2:7 Ro 9:23 Php 4:19 Col 1:27}
\crossref{Eph}{3}{17}{Eph 2:21 Isa 57:15 Joh 6:56; 14:17,\allowbreak23; 17:23 Ro 8:9-\allowbreak11 2Co 6:16}
\crossref{Eph}{3}{18}{3:19; 1:18-\allowbreak23 Job 11:7-\allowbreak9 Ps 103:11,\allowbreak12,\allowbreak17; 139:6 Isa 55:9 Joh 15:13}
\crossref{Eph}{3}{19}{3:18; 5:2,\allowbreak25 Joh 17:3 2Co 5:14 Ga 2:20 Php 2:5-\allowbreak12 Col 1:10 2Pe 3:18}
\crossref{Eph}{3}{20}{Ge 17:1; 18:4 2Ch 25:9 Jer 32:17,\allowbreak27 Da 3:17; 6:20 Mt 3:9}
\crossref{Eph}{3}{21}{Eph 1:6 1Ch 29:11 Ps 29:1,\allowbreak2; 72:19; 115:1 Isa 6:3; 42:12 Mt 6:13}
\crossref{Eph}{4}{1}{Eph 3:1}
\crossref{Eph}{4}{2}{Nu 12:3 Ps 45:4; 138:6 Pr 3:34; 16:19 Isa 57:15; 61:1-\allowbreak3 Zep 2:3}
\crossref{Eph}{4}{3}{4:4 Joh 13:34; 17:21-\allowbreak23 Ro 14:17-\allowbreak19 1Co 1:10; 12:12,\allowbreak13 2Co 13:11}
\crossref{Eph}{4}{4}{Eph 2:16; 5:30 Ro 12:4,\allowbreak5 1Co 10:17; 12:12,\allowbreak13,\allowbreak20 Col 3:15}
\crossref{Eph}{4}{5}{Ac 2:36; 10:36 Ro 14:8,\allowbreak9 1Co 1:2,\allowbreak13; 8:6; 12:5 Php 2:11; 3:8}
\crossref{Eph}{4}{6}{Eph 6:23 Nu 16:22 Isa 63:16 Mal 2:10 Mt 6:9 Joh 20:17 1Co 8:6; 12:6}
\crossref{Eph}{4}{7}{4:8-\allowbreak14 Mt 25:15 Ro 12:6-\allowbreak8 1Co 12:8-\allowbreak11,\allowbreak28-\allowbreak30}
\crossref{Eph}{4}{8}{Ps 68:18}
\crossref{Eph}{4}{9}{Pr 30:4 Joh 3:13; 6:33,\allowbreak62; 20:17 Ac 2:34-\allowbreak36}
\crossref{Eph}{4}{10}{Eph 1:20-\allowbreak23 Ac 1:9,\allowbreak11 1Ti 3:16 Heb 4:14; 7:26; 8:1; 9:23,\allowbreak24}
\crossref{Eph}{4}{11}{4:8; 2:20; 3:5 Ro 10:14,\allowbreak15 1Co 12:28 Jude 1:17 Re 18:20; 21:14}
\crossref{Eph}{4}{12}{Lu 22:32 Joh 21:15-\allowbreak17 Ac 9:31; 11:23; 14:22,\allowbreak23; 20:28 Ro 15:14,\allowbreak29}
\crossref{Eph}{4}{13}{4:3,\allowbreak5 Jer 32:38,\allowbreak39 Eze 37:21,\allowbreak22 Zep 3:9 Zec 14:9 Joh 17:21}
\crossref{Eph}{4}{14}{Isa 28:9 Mt 18:3,\allowbreak4 1Co 3:1,\allowbreak2; 14:20 Heb 5:12-\allowbreak14}
\crossref{Eph}{4}{15}{4:25 Zec 8:16 2Co 4:2; 8:8}
\crossref{Eph}{4}{16}{4:12 Joh 15:5}
\crossref{Eph}{4}{17}{1Co 1:12; 15:50 2Co 9:6 Ga 3:17 Col 2:4}
\crossref{Eph}{4}{18}{Ps 74:20; 115:4-\allowbreak8 Isa 44:18-\allowbreak20; 46:5-\allowbreak8 Ac 17:30; 26:17,\allowbreak18}
\crossref{Eph}{4}{19}{1Ti 4:2}
\crossref{Eph}{4}{20}{Lu 24:47 Joh 6:45 Ro 6:1,\allowbreak2 2Co 5:14,\allowbreak15 Tit 2:11-\allowbreak14 1Jo 2:27}
\crossref{Eph}{4}{21}{Mt 17:5 Lu 10:16 Joh 10:27 Ac 3:22,\allowbreak23 Heb 3:7,\allowbreak8}
\crossref{Eph}{4}{22}{4:25 1Sa 1:14 Job 22:23 Eze 18:30-\allowbreak32 Col 2:11; 3:8,\allowbreak9 Heb 12:1}
\crossref{Eph}{4}{23}{Eph 2:10 Ps 51:10 Eze 11:19; 18:31; 36:26 Ro 12:2 Col 3:10 Tit 3:5}
\crossref{Eph}{4}{24}{Eph 6:11 Job 29:14 Isa 52:1; 59:17 Ro 13:12,\allowbreak14 1Co 15:53 Ga 3:27}
\crossref{Eph}{4}{25}{Le 19:11 1Ki 13:18 Ps 52:3; 119:29 Pr 6:17; 12:19,\allowbreak22; 21:6}
\crossref{Eph}{4}{26}{4:31,\allowbreak32 Ex 11:8; 32:21,\allowbreak22 Nu 20:10-\allowbreak13,\allowbreak24; 25:7-\allowbreak11 Ne 5:6-\allowbreak13}
\crossref{Eph}{4}{27}{Eph 6:11,\allowbreak16 Ac 5:3 2Co 2:10,\allowbreak11 Jas 4:7 1Pe 5:8}
\crossref{Eph}{4}{28}{Ex 20:15,\allowbreak17; 21:16 Pr 30:9 Jer 7:9 Ho 4:2 Zec 5:3 Joh 12:6}
\crossref{Eph}{4}{29}{Eph 5:3,\allowbreak4 Ps 5:9; 52:2; 73:7-\allowbreak9 Mt 12:34-\allowbreak37 Ro 3:13,\allowbreak14 1Co 15:32,\allowbreak33}
\crossref{Eph}{4}{30}{Ge 6:3,\allowbreak6 Jud 10:16 Ps 78:40; 95:10 Isa 7:13; 43:24; 63:10}
\crossref{Eph}{4}{31}{Ps 64:3 Ro 3:14 Col 3:8,\allowbreak19 Jas 3:14,\allowbreak15}
\crossref{Eph}{4}{32}{Ru 2:20 Ps 112:4,\allowbreak5,\allowbreak9 Pr 19:22 Isa 57:1}
\crossref{Eph}{5}{1}{Eph 4:32 Le 11:45 Mt 5:45,\allowbreak48 Lu 6:35,\allowbreak36 1Pe 1:15,\allowbreak16 1Jo 4:11}
\crossref{Eph}{5}{2}{Eph 3:17; 4:2,\allowbreak15 Joh 13:34 Ro 14:16 1Co 16:14 Col 3:14 1Th 4:9}
\crossref{Eph}{5}{3}{5:5; 4:19,\allowbreak20 Nu 25:1 De 23:17,\allowbreak18 Mt 15:19 Mr 7:21 Ac 15:20 Ro 1:29}
\crossref{Eph}{5}{4}{Eph 4:29 Pr 12:23; 15:2 Ec 10:13 Mt 12:34-\allowbreak37 Mr 7:22 Col 3:8}
\crossref{Eph}{5}{5}{1Co 6:9,\allowbreak10 Ga 5:19,\allowbreak21}
\crossref{Eph}{5}{6}{Jer 29:8,\allowbreak9,\allowbreak31 Eze 13:10-\allowbreak16 Mic 3:5 Mt 24:4,\allowbreak24 Mr 13:5,\allowbreak22}
\crossref{Eph}{5}{7}{5:11 Nu 16:26 Ps 50:18 Pr 1:10-\allowbreak17; 9:6; 13:20 1Ti 5:22 Re 18:4}
\crossref{Eph}{5}{8}{Eph 2:11,\allowbreak12; 4:18; 6:12 Ps 74:20 Isa 9:2; 42:16; 60:2 Jer 13:16}
\crossref{Eph}{5}{9}{Ga 5:22,\allowbreak23}
\crossref{Eph}{5}{10}{1Sa 17:39 Ro 12:1,\allowbreak2 Php 1:10 1Th 5:21}
\crossref{Eph}{5}{11}{5:7 Ge 49:5-\allowbreak7 Ps 1:1,\allowbreak2; 26:4,\allowbreak5; 94:20,\allowbreak21 Pr 4:14,\allowbreak15; 9:6 Jer 15:17}
\crossref{Eph}{5}{12}{5:3 Ro 1:24-\allowbreak27 1Pe 4:3}
\crossref{Eph}{5}{13}{La 2:14 Ho 2:10; 7:1}
\crossref{Eph}{5}{14}{Isa 51:17; 52:1; 60:1 Ro 13:11,\allowbreak12 1Co 15:34 1Th 5:6}
\crossref{Eph}{5}{15}{5:33 Mt 8:4; 27:4,\allowbreak24 1Th 5:15 Heb 12:25 1Pe 1:22 Re 19:10}
\crossref{Eph}{5}{16}{Ec 9:10 Ro 13:11 Ga 6:10 Col 4:5}
\crossref{Eph}{5}{17}{5:15 Col 4:5}
\crossref{Eph}{5}{18}{Ge 9:21; 19:32-\allowbreak35 De 21:20 Ps 69:12 Pr 20:1; 23:20,\allowbreak21,\allowbreak29-\allowbreak35}
\crossref{Eph}{5}{19}{Ac 16:25 1Co 14:26 Col 3:16 Jas 5:13}
\crossref{Eph}{5}{20}{5:4 Job 1:21 Ps 34:1 Isa 63:7 Ac 5:41 1Co 1:4 Php 1:3; 4:6}
\crossref{Eph}{5}{21}{5:22,\allowbreak24 Ge 16:9 1Ch 29:24 Ro 13:1-\allowbreak5 1Co 16:16 Php 2:3 1Ti 2:11}
\crossref{Eph}{5}{22}{5:24 Ge 3:16 Es 1:16-\allowbreak18,\allowbreak20 1Co 14:34 Col 3:18 etc.}
\crossref{Eph}{5}{23}{1Co 11:3-\allowbreak10}
\crossref{Eph}{5}{24}{5:33 Ex 23:13; 29:35 Col 3:20,\allowbreak22 Tit 2:7,\allowbreak9}
\crossref{Eph}{5}{25}{5:28 Ge 2:24; 24:67 2Sa 12:3 Pr 5:18,\allowbreak19 Col 3:19 1Pe 3:7}
\crossref{Eph}{5}{26}{Joh 17:17-\allowbreak19 Ac 26:18 1Co 6:11 Tit 2:14 Heb 9:14; 10:10 1Pe 1:2}
\crossref{Eph}{5}{27}{2Co 4:14; 11:2 Col 1:22,\allowbreak28 Jude 1:24}
\crossref{Eph}{5}{28}{5:31,\allowbreak33 Ge 2:21-\allowbreak24 Mt 19:5}
\crossref{Eph}{5}{29}{5:31 Pr 11:17 Ec 4:5 Ro 1:31}
\crossref{Eph}{5}{30}{Eph 1:23 Ge 2:23 Ro 12:5 1Co 6:15; 12:12-\allowbreak27 Col 2:19}
\crossref{Eph}{5}{31}{Ge 2:24 Mt 19:5 Mr 10:7,\allowbreak8 1Co 6:16}
\crossref{Eph}{5}{32}{Eph 6:19 Col 2:2 1Ti 3:8,\allowbreak16}
\crossref{Eph}{5}{33}{5:25,\allowbreak28,\allowbreak29 Col 3:19 1Pe 3:7}
\crossref{Eph}{6}{1}{Ge 28:7; 37:13 Le 19:3 De 21:18 1Sa 17:20 Es 2:20 Pr 1:8; 6:20}
\crossref{Eph}{6}{2}{Ex 20:12 De 27:16 Pr 20:20 Jer 35:18 Eze 22:7 Mal 1:6}
\crossref{Eph}{6}{3}{De 4:40; 5:16; 6:3,\allowbreak18; 12:25,\allowbreak28; 22:7 Ru 3:1 Ps 128:1,\allowbreak2 Isa 3:10}
\crossref{Eph}{6}{4}{Ge 31:14,\allowbreak15 1Sa 20:30-\allowbreak34 Col 3:21}
\crossref{Eph}{6}{5}{Ge 16:9 Ps 123:2 Mal 1:6 Mt 6:24; 8:9 Ac 10:7,\allowbreak8 Col 3:22}
\crossref{Eph}{6}{6}{Php 2:12 Col 3:22 1Th 2:4}
\crossref{Eph}{6}{7}{Ge 31:6,\allowbreak38-\allowbreak40 2Ki 5:2,\allowbreak3,\allowbreak13}
\crossref{Eph}{6}{8}{Pr 11:18; 23:18 Isa 3:11 Mt 5:12; 6:1,\allowbreak4; 10:41,\allowbreak42; 16:27 Lu 6:35}
\crossref{Eph}{6}{9}{Le 19:13; 25:39-\allowbreak46 De 15:11-\allowbreak16; 24:14,\allowbreak15 Ne 5:5,\allowbreak8,\allowbreak9 Job 24:10-\allowbreak12}
\crossref{Eph}{6}{10}{2Co 13:11 Php 3:1; 4:8 1Pe 3:8}
\crossref{Eph}{6}{11}{Eph 4:24 Ro 13:14 Col 3:10}
\crossref{Eph}{6}{12}{Lu 13:24 1Co 9:25-\allowbreak27 2Ti 2:5 Heb 12:1,\allowbreak4}
\crossref{Eph}{6}{13}{6:11-\allowbreak17 2Co 10:4}
\crossref{Eph}{6}{14}{Eph 5:9 Isa 11:5 Lu 12:35 2Co 6:7 1Pe 1:13}
\crossref{Eph}{6}{15}{De 33:25 So 7:1 Hab 3:19 Lu 15:22}
\crossref{Eph}{6}{16}{1Th 5:19}
\crossref{Eph}{6}{17}{1Sa 17:5,\allowbreak58 Isa 59:17 1Th 5:8}
\crossref{Eph}{6}{18}{Eph 1:16 Job 27:10 Ps 4:1; 6:9 Isa 26:16 Da 6:10 Lu 3:26,\allowbreak37; 18:1-\allowbreak7}
\crossref{Eph}{6}{19}{Ro 15:30 2Co 1:11 Php 1:19 Col 4:3 1Th 5:25 2Th 3:1 Phm 1:22}
\crossref{Eph}{6}{20}{Pr 13:17 Isa 33:7 2Co 5:20}
\crossref{Eph}{6}{21}{Php 1:12 Col 4:7}
\crossref{Eph}{6}{22}{Php 2:19,\allowbreak25 Col 4:7,\allowbreak8 1Th 3:2 2Th 2:17}
\crossref{Eph}{6}{23}{Ro 1:7 1Co 1:3 Ge 43:23 1Sa 25:6 Ps 122:6-\allowbreak9 Joh 14:27 Ga 6:16}
\crossref{Eph}{6}{24}{1Co 16:23 2Co 13:14 Col 4:18 2Ti 4:22 Tit 3:15 Heb 13:25}

% Phil
\crossref{Phil}{1}{1}{Ro 1:1 1Co 1:1}
\crossref{Phil}{1}{2}{Ro 1:7 2Co 1:2 1Pe 1:2}
\crossref{Phil}{1}{3}{Ro 1:8,\allowbreak9; 6:17 1Co 1:4}
\crossref{Phil}{1}{4}{1:9-\allowbreak11}
\crossref{Phil}{1}{5}{1:7; 4:14 Ac 16:15 Ro 11:17; 12:13; 15:26 1Co 1:9 2Co 8:1 Eph 2:19-\allowbreak22}
\crossref{Phil}{1}{6}{2Co 1:15; 2:3; 7:16; 9:4 Ga 5:10 2Th 3:4 Phm 1:21 Heb 10:35}
\crossref{Phil}{1}{7}{1Co 13:7 1Th 1:2-\allowbreak5; 5:5 Heb 6:9,\allowbreak10}
\crossref{Phil}{1}{8}{Ro 1:9; 9:1 Ga 1:20 1Th 2:5}
\crossref{Phil}{1}{9}{1:4}
\crossref{Phil}{1}{10}{Isa 7:15,\allowbreak16 Am 5:14,\allowbreak15 Mic 3:2 Joh 3:20 Ro 2:18; 7:16,\allowbreak22; 8:7}
\crossref{Phil}{1}{11}{Php 4:17 Ps 1:3; 92:12-\allowbreak14 Isa 5:2 Lu 13:6-\allowbreak9 Joh 15:2,\allowbreak8,\allowbreak16 Ro 6:22}
\crossref{Phil}{1}{12}{Ac 21:28 etc.}
\crossref{Phil}{1}{13}{Ac 20:23,\allowbreak24; 21:11-\allowbreak13; 26:29,\allowbreak31; 28:17,\allowbreak20 Eph 3:1; 4:1; 6:20}
\crossref{Phil}{1}{14}{Php 4:1 Col 4:7}
\crossref{Phil}{1}{15}{1:16,\allowbreak18 Ac 5:42; 8:5,\allowbreak35; 9:20; 10:36; 11:20 1Co 1:23 2Co 1:19; 4:5}
\crossref{Phil}{1}{16}{1:10 2Co 2:17; 4:1,\allowbreak2}
\crossref{Phil}{1}{17}{1:7 Ro 1:13-\allowbreak17 1Co 9:16,\allowbreak17 Ga 2:7,\allowbreak8 1Ti 2:7 2Ti 1:11,\allowbreak12; 4:6,\allowbreak7}
\crossref{Phil}{1}{18}{Ro 3:9; 6:15 1Co 10:19; 14:15}
\crossref{Phil}{1}{19}{Ro 8:28 1Co 4:17 1Pe 1:7-\allowbreak9}
\crossref{Phil}{1}{20}{Ps 62:5 Pr 10:28; 23:18 Ro 8:19}
\crossref{Phil}{1}{21}{1:20; 2:21 1Co 1:30 Ga 6:14 Col 3:4}
\crossref{Phil}{1}{22}{1:24 2Co 10:3 Ga 2:20 Col 2:1 1Pe 4:2}
\crossref{Phil}{1}{23}{2Sa 24:14 1Th 2:1,\allowbreak13 Lu 12:50 2Co 6:12}
\crossref{Phil}{1}{24}{1:22,\allowbreak25,\allowbreak26 Joh 16:7 Ac 20:29-\allowbreak31}
\crossref{Phil}{1}{25}{Php 2:24 Ac 20:25}
\crossref{Phil}{1}{26}{Php 2:16-\allowbreak18; 3:1,\allowbreak3; 4:4,\allowbreak10 So 5:1 Joh 16:22,\allowbreak24 2Co 1:14; 5:12; 7:6}
\crossref{Phil}{1}{27}{Php 3:18-\allowbreak21 Eph 4:1 Col 1:10 1Th 2:11,\allowbreak12; 4:1 Tit 2:10 2Pe 1:4-\allowbreak9}
\crossref{Phil}{1}{28}{Isa 51:7,\allowbreak12 Mt 10:28 Lu 12:4-\allowbreak7; 21:12-\allowbreak19 Ac 4:19-\allowbreak31; 5:40-\allowbreak42}
\crossref{Phil}{1}{29}{Ac 5:41 Ro 5:3 Jas 1:2 1Pe 4:13}
\crossref{Phil}{1}{30}{Joh 16:33 Ro 8:35-\allowbreak37 1Co 4:9-\allowbreak14; 15:30-\allowbreak32 Eph 6:11-\allowbreak18 Col 2:1}
\crossref{Phil}{2}{1}{Php 3:3 Lu 2:10,\allowbreak11,\allowbreak25 Joh 14:18,\allowbreak27; 15:11; 16:22-\allowbreak24; 17:13 Ro 5:1,\allowbreak2}
\crossref{Phil}{2}{2}{2:16; 1:4,\allowbreak26,\allowbreak27 Joh 3:29 2Co 2:3; 7:7 Col 2:5 1Th 2:19,\allowbreak20; 3:6-\allowbreak10}
\crossref{Phil}{2}{3}{2:14; 1:15,\allowbreak16 Pr 13:10 Ro 13:13 1Co 3:3 2Co 12:20 Ga 5:15,\allowbreak20,\allowbreak21,\allowbreak26}
\crossref{Phil}{2}{4}{Mt 18:6 Ro 12:15; 14:19-\allowbreak22; 15:1 1Co 8:9-\allowbreak13; 10:24,\allowbreak32,\allowbreak33; 12:22-\allowbreak26}
\crossref{Phil}{2}{5}{Mt 11:29; 20:26-\allowbreak28 Lu 22:27 Joh 13:14,\allowbreak15 Ac 10:38; 20:35}
\crossref{Phil}{2}{6}{Isa 7:14; 8:8; 9:6 Jer 23:6 Mic 5:2 Mt 1:23 Joh 1:1,\allowbreak2,\allowbreak18; 17:5}
\crossref{Phil}{2}{7}{Ps 22:6 Isa 49:7; 50:5,\allowbreak6; 52:14; 53:2,\allowbreak3 Da 9:26 Zec 9:9 Mr 9:12}
\crossref{Phil}{2}{8}{Mt 17:2 Mr 9:2,\allowbreak3 Lu 9:29}
\crossref{Phil}{2}{9}{Ge 3:15 Ps 2:6-\allowbreak12; 8:5-\allowbreak8; 45:6,\allowbreak7; 69:29,\allowbreak30; 72:17-\allowbreak19; 91:14; 110:1,\allowbreak5}
\crossref{Phil}{2}{10}{Ge 41:43 Isa 45:23-\allowbreak25 Mt 27:29; 28:18 Ro 11:4; 14:10,\allowbreak11 Eph 3:14}
\crossref{Phil}{2}{11}{Ps 18:49}
\crossref{Phil}{2}{12}{Php 4:1 1Co 4:14 1Pe 2:11}
\crossref{Phil}{2}{13}{2Ch 30:12 Isa 26:12 Jer 31:33; 32:38 Joh 3:27 Ac 11:21 2Co 3:5}
\crossref{Phil}{2}{14}{2:3 Ex 16:7,\allowbreak8 Nu 14:27 Ps 106:25 Mt 20:11 Mr 14:5 Ac 6:1}
\crossref{Phil}{2}{15}{Lu 1:6 1Co 1:8 Eph 5:27 1Th 5:23 1Ti 3:2,\allowbreak10; 5:7 Tit 1:6}
\crossref{Phil}{2}{16}{Php 1:27 Ps 40:9; 71:17 Mt 10:27 Lu 12:8 Ro 10:8-\allowbreak16 Re 22:17}
\crossref{Phil}{2}{17}{2:30; 1:20 Ac 20:24; 21:13 2Co 12:15 1Th 2:8 2Ti 4:6 1Jo 3:16}
\crossref{Phil}{2}{18}{Php 3:1; 4:4 Eph 3:13 Jas 1:2-\allowbreak4}
\crossref{Phil}{2}{19}{2:24 Jer 17:5 Mt 12:21 Ro 15:12 Eph 1:13 2Ti 1:12}
\crossref{Phil}{2}{20}{2:2,\allowbreak22 Ps 55:13 Pr 31:29 Joh 10:13; 12:6 1Co 1:10,\allowbreak11 Col 4:11}
\crossref{Phil}{2}{21}{2:4 Isa 56:11 Mal 1:10 Mt 16:24 Lu 9:57-\allowbreak62; 14:26 Ac 13:13; 15:38}
\crossref{Phil}{2}{22}{Ac 16:3-\allowbreak12 2Co 2:9; 8:8,\allowbreak22,\allowbreak24}
\crossref{Phil}{2}{23}{1Sa 22:3}
\crossref{Phil}{2}{24}{2:19; 1:25,\allowbreak26 Ro 15:28,\allowbreak29 Phm 1:22 2Jo 1:12 3Jo 1:14}
\crossref{Phil}{2}{25}{Php 4:18}
\crossref{Phil}{2}{26}{Php 1:3,\allowbreak8; 4:1 2Sa 13:39 Ro 1:11 2Co 9:14}
\crossref{Phil}{2}{27}{2:30 2Ki 20:1 Ps 107:18 Ec 9:1,\allowbreak2 Joh 11:3,\allowbreak4 Ac 9:37}
\crossref{Phil}{2}{28}{2:26 Ge 45:27,\allowbreak28; 46:29,\allowbreak30; 48:11 Joh 16:22 Ac 20:38 2Ti 1:4}
\crossref{Phil}{2}{29}{Mt 10:40,\allowbreak41 Lu 9:5 Joh 13:20 Ro 16:2 1Co 16:10 2Co 7:2}
\crossref{Phil}{2}{30}{1Co 15:53; 16:10}
\crossref{Phil}{3}{1}{Php 4:8 2Co 13:11 Eph 6:10 1Th 4:1}
\crossref{Phil}{3}{2}{Pr 26:11 Isa 56:10,\allowbreak11 Mt 7:6,\allowbreak15; 24:10 Ga 5:15 2Ti 4:14,\allowbreak15}
\crossref{Phil}{3}{3}{Ge 17:5-\allowbreak11 De 10:16; 30:6 Jer 4:4; 9:26 Ro 2:25-\allowbreak29; 4:11,\allowbreak12}
\crossref{Phil}{3}{4}{2Co 11:18-\allowbreak22}
\crossref{Phil}{3}{5}{Ge 17:12 Lu 2:21 Joh 7:21-\allowbreak24}
\crossref{Phil}{3}{6}{2Sa 21:2 2Ki 10:16 Ac 21:20 Ro 10:2 Ga 1:13,\allowbreak14}
\crossref{Phil}{3}{7}{3:4-\allowbreak6,\allowbreak8-\allowbreak10 Ge 19:17,\allowbreak26 Job 2:4 Pr 13:8; 23:23 Mt 13:44-\allowbreak46; 16:26}
\crossref{Phil}{3}{8}{Nu 14:30 Ps 126:6 Lu 11:20 1Co 9:10 1Jo 2:19}
\crossref{Phil}{3}{9}{Ge 7:23 De 19:3,\allowbreak4 Heb 6:18 1Pe 3:19,\allowbreak20}
\crossref{Phil}{3}{10}{3:8 1Jo 2:3,\allowbreak5}
\crossref{Phil}{3}{11}{Ps 49:7 Ac 27:12 Ro 11:14 1Co 9:22,\allowbreak27 2Co 11:3 1Th 3:5 2Th 2:3}
\crossref{Phil}{3}{12}{3:13,\allowbreak16 Ps 119:5,\allowbreak173-\allowbreak176 Ro 7:19-\allowbreak24 Ga 5:17 1Ti 6:12 Jas 3:2}
\crossref{Phil}{3}{13}{3:8,\allowbreak12; 1:18-\allowbreak21; 4:11-\allowbreak13}
\crossref{Phil}{3}{14}{Lu 16:16 2Co 4:17,\allowbreak18; 5:1 2Ti 4:7,\allowbreak8 Re 3:21}
\crossref{Phil}{3}{15}{Ro 15:1 1Co 2:6; 14:20 Col 1:28; 4:12 2Ti 3:17 Heb 5:14}
\crossref{Phil}{3}{16}{Ga 5:7 Heb 10:38,\allowbreak39 2Pe 2:10-\allowbreak20 Re 2:4,\allowbreak5; 3:3}
\crossref{Phil}{3}{17}{Php 4:9 1Co 4:16; 10:32,\allowbreak33; 11:1 1Th 1:6; 2:10-\allowbreak14 2Th 3:7,\allowbreak9 1Ti 4:12}
\crossref{Phil}{3}{18}{Isa 8:11 Da 4:37 Ga 2:14 Eph 4:17 2Th 3:11 2Pe 2:10 Jude 1:13}
\crossref{Phil}{3}{19}{Mt 25:41 Lu 12:45,\allowbreak46 2Co 11:15 2Th 2:8,\allowbreak12 Heb 6:6-\allowbreak8 2Pe 2:1,\allowbreak3}
\crossref{Phil}{3}{20}{Php 1:18-\allowbreak21 Ps 16:11; 17:15; 73:24-\allowbreak26 Pr 15:24 Mt 6:19-\allowbreak21; 19:21}
\crossref{Phil}{3}{21}{1Co 15:42-\allowbreak44,\allowbreak48-\allowbreak54}
\crossref{Phil}{4}{1}{Php 3:20,\allowbreak21 2Pe 3:11-\allowbreak14}
\crossref{Phil}{4}{2}{Php 2:2,\allowbreak3; 3:16 Ge 45:24 Ps 133:1-\allowbreak3 Mr 9:50 Ro 12:16-\allowbreak18 1Co 1:10}
\crossref{Phil}{4}{3}{4:2 Ro 12:1 Phm 1:8,\allowbreak9}
\crossref{Phil}{4}{4}{Php 3:1 Ro 12:12}
\crossref{Phil}{4}{5}{Mt 5:39-\allowbreak42; 6:25,\allowbreak34 Lu 6:29-\allowbreak35; 12:22-\allowbreak30; 21:34 1Co 6:7; 7:29-\allowbreak31}
\crossref{Phil}{4}{6}{Da 3:16 Mt 6:25-\allowbreak33; 10:19; 13:22 Lu 10:41; 12:29 1Co 7:21,\allowbreak32}
\crossref{Phil}{4}{7}{Php 1:2 Nu 6:26 Job 22:21; 34:29 Ps 29:11; 85:8 Isa 26:3,\allowbreak12; 45:7}
\crossref{Phil}{4}{8}{Php 3:1}
\crossref{Phil}{4}{9}{Php 3:17 1Co 10:31-\allowbreak33; 11:1 1Th 1:6; 2:2-\allowbreak12,\allowbreak14; 4:1-\allowbreak8 2Th 3:6-\allowbreak10}
\crossref{Phil}{4}{10}{Php 1:1,\allowbreak3 2Co 7:6,\allowbreak7}
\crossref{Phil}{4}{11}{1Co 4:11,\allowbreak12 2Co 6:10; 8:9; 11:27}
\crossref{Phil}{4}{12}{1Co 4:9-\allowbreak13 2Co 6:4-\allowbreak10; 10:1,\allowbreak10; 11:7,\allowbreak27; 12:7-\allowbreak10}
\crossref{Phil}{4}{13}{Joh 15:4,\allowbreak5,\allowbreak7 2Co 3:4,\allowbreak5}
\crossref{Phil}{4}{14}{1Ki 8:18 2Ch 6:8 Mt 25:21 3Jo 1:5-\allowbreak8}
\crossref{Phil}{4}{15}{2Ki 5:16,\allowbreak20 2Co 11:8-\allowbreak12; 12:11-\allowbreak15}
\crossref{Phil}{4}{16}{1Th 2:9}
\crossref{Phil}{4}{17}{4:11 Mal 1:10 Ac 20:33,\allowbreak34 1Co 9:12-\allowbreak15 2Co 11:16 1Th 2:5 1Ti 3:3}
\crossref{Phil}{4}{18}{4:12 2Th 1:3}
\crossref{Phil}{4}{19}{2Sa 22:7 2Ch 18:13 Ne 5:19 Da 6:22 Mic 7:7 Joh 20:17,\allowbreak27 Ro 1:8}
\crossref{Phil}{4}{20}{Php 1:11 Ps 72:19; 115:1 Mt 6:9,\allowbreak13 Ro 11:36; 16:27 Ga 1:5 Eph 3:21}
\crossref{Phil}{4}{21}{Ro 16:3-\allowbreak16}
\crossref{Phil}{4}{22}{Ro 16:16 2Co 13:13 Heb 13:24 1Pe 5:13 3Jo 1:14}
\crossref{Phil}{4}{23}{Ro 16:20,\allowbreak24 2Co 13:14}

% Col
\crossref{Col}{1}{1}{Ro 1:1 1Co 1:1 2Co 1:1 Eph 1:1}
\crossref{Col}{1}{2}{Ps 16:3 1Co 1:2 Ga 3:9 Eph 1:1}
\crossref{Col}{1}{3}{Ro 1:8,\allowbreak9 1Co 1:4 Eph 1:15 Php 1:3-\allowbreak5; 4:6 1Th 1:2}
\crossref{Col}{1}{4}{1:9 2Co 7:7 Eph 1:15 1Th 3:6 3Jo 1:3,\allowbreak4}
\crossref{Col}{1}{5}{1:23,\allowbreak27 Ac 23:6; 24:15; 26:6,\allowbreak7 1Co 13:13; 15:19 Ga 5:5 Eph 1:18,\allowbreak19}
\crossref{Col}{1}{6}{1:23 Ps 98:3 Mt 24:14; 28:19 Mr 16:15 Ro 10:18; 15:19; 16:26}
\crossref{Col}{1}{7}{Col 4:12 Phm 1:23}
\crossref{Col}{1}{8}{1:4 Ro 5:5; 15:30 Ga 5:22 2Ti 1:7 1Pe 1:22}
\crossref{Col}{1}{9}{1:3,\allowbreak4,\allowbreak6 Ro 1:8-\allowbreak10 Eph 1:15,\allowbreak16}
\crossref{Col}{1}{10}{Col 2:6; 4:5 Mic 4:5 Ro 4:12; 6:4 Eph 4:1; 5:2,\allowbreak15 Php 1:27 1Th 2:12}
\crossref{Col}{1}{11}{Isa 45:24 2Co 12:9 Eph 3:16; 6:10 Php 4:13}
\crossref{Col}{1}{12}{Col 3:15,\allowbreak17 1Ch 29:20 Ps 79:13; 107:21,\allowbreak22; 116:7 Da 2:23 Eph 5:4,\allowbreak20}
\crossref{Col}{1}{13}{Isa 49:24,\allowbreak25; 53:12 Mt 12:29,\allowbreak30 Ac 26:18 Heb 2:14}
\crossref{Col}{1}{14}{Mt 20:28 Ac 20:28 Ro 3:24,\allowbreak25 Ga 3:13 Eph 1:7; 5:2 1Ti 2:6}
\crossref{Col}{1}{15}{Ex 24:10 Nu 12:8 Eze 1:26-\allowbreak28 Joh 1:18; 14:9; 15:24 2Co 4:4,\allowbreak6}
\crossref{Col}{1}{16}{1:15 Ps 102:25-\allowbreak27 Isa 40:9-\allowbreak12; 44:24 Joh 1:3 1Co 8:6 Eph 3:9}
\crossref{Col}{1}{17}{1:15 Pr 8:22,\allowbreak23 Isa 43:11-\allowbreak13; 44:6 Mic 5:2 Joh 1:1-\allowbreak3; 8:58; 17:5}
\crossref{Col}{1}{18}{1:24; 2:10-\allowbreak14 1Co 11:3 Eph 1:10,\allowbreak22,\allowbreak23; 4:15,\allowbreak16; 5:23}
\crossref{Col}{1}{19}{Col 2:3,\allowbreak9; 3:11 Mt 11:25-\allowbreak27 Lu 10:21 Joh 1:16; 3:34 Eph 1:3,\allowbreak23; 4:10}
\crossref{Col}{1}{20}{1:21,\allowbreak22 Le 6:30 Ps 85:10,\allowbreak11 Isa 9:6,\allowbreak7 Eze 45:17-\allowbreak20 Da 9:24-\allowbreak26}
\crossref{Col}{1}{21}{Ro 1:30; 5:9,\allowbreak10; 8:7,\allowbreak8 1Co 6:9-\allowbreak11 Eph 2:1,\allowbreak2,\allowbreak12,\allowbreak19; 4:18 Tit 3:3-\allowbreak7}
\crossref{Col}{1}{22}{Ro 7:4 Eph 2:15,\allowbreak16 Heb 10:10,\allowbreak20}
\crossref{Col}{1}{23}{Ps 92:13,\allowbreak14; 125:5 Eze 18:26 Ho 6:3,\allowbreak4 Zep 1:6 Mt 24:13}
\crossref{Col}{1}{24}{Mt 5:11,\allowbreak12 Ac 5:41 Ro 5:3 2Co 7:4 Eph 3:1,\allowbreak13 Php 2:17,\allowbreak18}
\crossref{Col}{1}{25}{1:23 1Th 3:2 1Ti 4:6}
\crossref{Col}{1}{26}{Ro 16:25,\allowbreak26 1Co 2:7 Eph 3:3-\allowbreak10}
\crossref{Col}{1}{27}{1Co 2:12-\allowbreak14 2Co 2:14; 4:6 Ga 1:15,\allowbreak16}
\crossref{Col}{1}{28}{Ac 3:20; 5:42; 8:5,\allowbreak35; 9:20; 10:36; 11:20; 13:38; 17:3,\allowbreak18 Ro 16:25}
\crossref{Col}{1}{29}{Col 4:12 1Co 15:10 2Co 5:9; 6:5; 11:23 Php 2:16 1Th 2:9 2Th 3:8}
\crossref{Col}{2}{1}{Col 1:24,\allowbreak29; 4:12 Ge 30:8; 32:24-\allowbreak30 Ho 12:3,\allowbreak4 Lu 22:44 Ga 4:19}
\crossref{Col}{2}{2}{Col 4:8 Isa 40:1 Ro 15:13 2Co 1:4-\allowbreak6 1Th 3:2; 5:14 2Th 2:16,\allowbreak17}
\crossref{Col}{2}{3}{Col 1:9,\allowbreak19; 3:16 Ro 11:33 1Co 1:24,\allowbreak30; 2:6-\allowbreak8 Eph 1:8; 3:10}
\crossref{Col}{2}{4}{2:8,\allowbreak18 Mt 24:4,\allowbreak24 Mr 13:22 Ac 20:30 Ro 16:18,\allowbreak19 2Co 11:3,\allowbreak11-\allowbreak13}
\crossref{Col}{2}{5}{2:1 1Co 5:3,\allowbreak4 1Th 2:17}
\crossref{Col}{2}{6}{Mt 10:40 Joh 1:12,\allowbreak13; 13:20 1Co 1:30 Heb 3:14 1Jo 5:11,\allowbreak12,\allowbreak20}
\crossref{Col}{2}{7}{Col 1:23 Ps 1:3; 92:13 Isa 61:3 Jer 17:8 Eze 17:23,\allowbreak24 Ro 11:17,\allowbreak18}
\crossref{Col}{2}{8}{De 6:12 Mt 7:15; 10:17; 16:6 Php 3:2 2Pe 3:17}
\crossref{Col}{2}{9}{2:2,\allowbreak3; 1:19 Isa 7:14 Mt 1:23 Joh 10:30,\allowbreak38; 14:9,\allowbreak10,\allowbreak20; 17:21}
\crossref{Col}{2}{10}{Col 3:11 Joh 1:16 1Co 1:30,\allowbreak31 Ga 3:26-\allowbreak29 Heb 5:9}
\crossref{Col}{2}{11}{De 10:16; 30:6 Jer 4:4 Ro 2:29 Php 3:3}
\crossref{Col}{2}{12}{Ro 6:4,\allowbreak5}
\crossref{Col}{2}{13}{Eze 37:1-\allowbreak10 Lu 9:60; 15:24,\allowbreak32 Ro 6:13 2Co 5:14,\allowbreak15 Eph 2:1,\allowbreak5,\allowbreak6}
\crossref{Col}{2}{14}{Nu 5:23 Ne 4:5 Ps 51:1,\allowbreak9 Isa 43:25; 44:22 Ac 3:19}
\crossref{Col}{2}{15}{Ge 3:15 Ps 68:18 Isa 49:24,\allowbreak25; 53:12 Mt 12:29 Lu 10:18; 11:22}
\crossref{Col}{2}{16}{Ro 14:3,\allowbreak10,\allowbreak13 1Co 10:28-\allowbreak31 Ga 2:12,\allowbreak13 Jas 4:11}
\crossref{Col}{2}{17}{Joh 1:17 Heb 8:5; 9:9; 10:1}
\crossref{Col}{2}{18}{2:4,\allowbreak8 Ge 3:13 Nu 25:18 Mt 24:24 Ro 16:18 2Co 11:3 Eph 5:6}
\crossref{Col}{2}{19}{2:6-\allowbreak9; 1:18 Ga 1:6-\allowbreak9; 5:2-\allowbreak4 1Ti 2:4-\allowbreak6}
\crossref{Col}{2}{20}{Col 3:3 Ro 6:2-\allowbreak11; 7:4-\allowbreak6 Ga 2:19,\allowbreak20; 6:14 1Pe 4:1-\allowbreak3}
\crossref{Col}{2}{21}{Ge 3:3 Isa 52:11 2Co 6:17 1Ti 4:3}
\crossref{Col}{2}{22}{Mr 7:18,\allowbreak19 Joh 6:27 1Co 6:13}
\crossref{Col}{2}{23}{Ge 3:5,\allowbreak6 Mt 23:27,\allowbreak28 2Co 11:13-\allowbreak15 1Ti 4:3,\allowbreak8}
\crossref{Col}{3}{1}{Col 2:12,\allowbreak13,\allowbreak20 Ro 6:4,\allowbreak5,\allowbreak9-\allowbreak11 Ga 2:19,\allowbreak20 Eph 1:19,\allowbreak20; 2:5,\allowbreak6}
\crossref{Col}{3}{2}{3:1 1Ch 22:19; 29:3 Ps 62:10; 91:14; 119:36,\allowbreak37 Pr 23:5 Ec 7:14}
\crossref{Col}{3}{3}{Col 2:20 Ro 6:2 Ga 2:20}
\crossref{Col}{3}{4}{Joh 11:25; 14:6; 20:31 Ac 3:15 Ga 2:20 2Ti 1:1 1Jo 1:1,\allowbreak2; 5:12}
\crossref{Col}{3}{5}{Ro 6:6; 8:13 Ga 5:24 Eph 5:3-\allowbreak6}
\crossref{Col}{3}{6}{Ro 1:18 Eph 5:6 Re 22:15}
\crossref{Col}{3}{7}{Col 2:13 Ro 6:19,\allowbreak20; 7:5 1Co 6:11 Eph 2:2 Tit 3:3 1Pe 4:3,\allowbreak4}
\crossref{Col}{3}{8}{3:5,\allowbreak9 Eph 4:22 Heb 12:1 Jas 1:21 1Pe 2:1}
\crossref{Col}{3}{9}{Le 19:11 Isa 63:8 Jer 9:3-\allowbreak5 Zep 3:13 Zec 8:16 Joh 8:44}
\crossref{Col}{3}{10}{3:12,\allowbreak14 Job 29:14 Isa 52:1; 59:17 Ro 13:12,\allowbreak14 1Co 15:53,\allowbreak54}
\crossref{Col}{3}{11}{Ps 117:2 Isa 19:23-\allowbreak25; 49:6; 52:10; 66:18-\allowbreak22 Jer 16:19 Ho 2:23}
\crossref{Col}{3}{12}{3:10 Eph 4:24}
\crossref{Col}{3}{13}{Ro 15:1,\allowbreak2 2Co 6:6 Ga 6:2 Eph 4:2,\allowbreak32}
\crossref{Col}{3}{14}{Col 2:2 Joh 13:34; 15:12 Ro 13:8 1Co 13:1-\allowbreak13 Eph 5:2 1Th 4:9}
\crossref{Col}{3}{15}{Ps 29:11 Isa 26:3; 27:5; 57:15,\allowbreak19 Joh 14:27; 16:33 Ro 5:1; 14:17}
\crossref{Col}{3}{16}{Joh 5:39,\allowbreak40 2Ti 3:15 Heb 4:12,\allowbreak13 1Pe 1:11,\allowbreak12 Re 19:10}
\crossref{Col}{3}{17}{3:23 2Ch 31:20,\allowbreak21 Pr 3:6 Ro 14:6-\allowbreak8 1Co 10:31}
\crossref{Col}{3}{18}{Ge 3:16 Es 1:20 1Co 11:3; 14:34 Eph 5:22-\allowbreak24,\allowbreak33 1Ti 2:12}
\crossref{Col}{3}{19}{Ge 2:23,\allowbreak24; 24:67 Pr 5:18,\allowbreak19 Ec 9:9 Mal 2:14-\allowbreak16 Lu 14:26}
\crossref{Col}{3}{20}{Ge 28:7 Ex 20:12 Le 19:3 De 21:18-\allowbreak21; 27:16 Pr 6:20; 20:20}
\crossref{Col}{3}{21}{Ps 103:13 Pr 3:12; 4:1-\allowbreak4}
\crossref{Col}{3}{22}{3:20 Ps 123:2 Mal 1:6 Mt 8:9 Lu 6:46; 7:8 Eph 6:5-\allowbreak7 1Ti 6:1,\allowbreak2}
\crossref{Col}{3}{23}{3:17 2Ch 31:21 Ps 47:6,\allowbreak7; 103:1; 119:10,\allowbreak34,\allowbreak145 Ec 9:10 Jer 3:10}
\crossref{Col}{3}{24}{Col 2:18 Ge 15:1 Ru 2:12 Pr 11:18 Mt 5:12,\allowbreak46; 6:1,\allowbreak2,\allowbreak5,\allowbreak16; 10:41}
\crossref{Col}{3}{25}{1Co 6:7,\allowbreak8 1Th 4:6 Phm 1:18}
\crossref{Col}{4}{1}{Le 19:13; 25:39-\allowbreak43 De 15:12-\allowbreak15; 24:14,\allowbreak15 Ne 5:5-\allowbreak13 Job 24:11,\allowbreak12}
\crossref{Col}{4}{2}{4:12; 1:9 1Sa 12:23 Job 15:4; 27:8-\allowbreak10 Ps 55:16,\allowbreak17; 109:4}
\crossref{Col}{4}{3}{Ro 15:30-\allowbreak32 Eph 6:19 Php 1:19 1Th 5:25 Phm 1:22 Heb 13:18,\allowbreak19}
\crossref{Col}{4}{4}{Mt 10:26,\allowbreak27 Ac 4:29 2Co 3:12; 4:1-\allowbreak4}
\crossref{Col}{4}{5}{Col 3:16 Ps 90:12 Mt 10:16 Ro 16:19 1Co 14:19-\allowbreak25 Eph 5:15-\allowbreak17}
\crossref{Col}{4}{6}{Col 3:16 De 6:6,\allowbreak7; 11:19 1Ch 16:24 Ps 37:30,\allowbreak31; 40:9,\allowbreak10; 45:2; 66:16}
\crossref{Col}{4}{7}{Eph 6:21-\allowbreak23}
\crossref{Col}{4}{8}{1Co 4:17 2Co 12:18 Eph 6:22 Php 2:28 1Th 3:5}
\crossref{Col}{4}{9}{4:7 Phm 1:10-\allowbreak19}
\crossref{Col}{4}{10}{Ac 19:29; 20:4; 27:2 Phm 1:24}
\crossref{Col}{4}{11}{Ac 10:45; 11:2 Ro 4:12 Ga 2:7,\allowbreak8 Eph 2:11 Tit 1:10}
\crossref{Col}{4}{12}{Col 1:7 Phm 1:23}
\crossref{Col}{4}{13}{Ro 10:2 2Co 8:3}
\crossref{Col}{4}{14}{2Ti 4:11 Phm 1:24}
\crossref{Col}{4}{15}{4:13}
\crossref{Col}{4}{16}{1Th 5:27}
\crossref{Col}{4}{17}{Phm 1:2}
\crossref{Col}{4}{18}{1Co 16:21 2Th 3:17}

% 1Thes
\crossref{1Thes}{1}{1}{Ac 15:27,\allowbreak32,\allowbreak34,\allowbreak40; 16:19,\allowbreak25,\allowbreak29; 17:4,\allowbreak15; 18:5}
\crossref{1Thes}{1}{2}{Ro 1:8,\allowbreak9; 6:17 1Co 1:4 Eph 1:15,\allowbreak16 Php 1:3,\allowbreak4 Col 1:3 Phm 1:4}
\crossref{1Thes}{1}{3}{1Th 3:6 2Ti 1:3-\allowbreak5}
\crossref{1Thes}{1}{4}{1:3 Ro 8:28-\allowbreak30; 11:5-\allowbreak7 Eph 1:4 Php 1:6,\allowbreak7 1Pe 1:2 2Pe 1:10}
\crossref{1Thes}{1}{5}{Isa 55:11 Ro 2:16 2Co 4:3 Ga 1:8-\allowbreak12; 2:2 2Th 2:14 2Ti 2:8}
\crossref{1Thes}{1}{6}{1Th 2:14 1Co 4:16; 11:1 2Co 8:5 Php 3:17 2Th 3:9}
\crossref{1Thes}{1}{7}{1Th 4:10 1Ti 4:12 Tit 2:7 1Pe 5:3}
\crossref{1Thes}{1}{8}{Isa 2:3; 52:7; 66:19 Ro 10:14-\allowbreak18 1Co 14:36 2Th 3:1 Re 14:6; 22:17}
\crossref{1Thes}{1}{9}{1:5,\allowbreak6; 2:1,\allowbreak13}
\crossref{1Thes}{1}{10}{1Th 4:16,\allowbreak17 Ge 49:18 Job 19:25-\allowbreak27 Isa 25:8,\allowbreak9 Lu 2:25 Ac 1:11; 3:21}
\crossref{1Thes}{2}{1}{2:13; 1:3-\allowbreak10 2Th 3:1}
\crossref{1Thes}{2}{2}{Ac 5:41; 16:12,\allowbreak22-\allowbreak24,\allowbreak37 2Ti 1:12 Heb 11:36,\allowbreak37; 12:2,\allowbreak3}
\crossref{1Thes}{2}{3}{2:5,\allowbreak6,\allowbreak11; 4:1,\allowbreak2 Nu 16:15 1Sa 12:3 Ac 20:33,\allowbreak34 2Co 2:17; 4:2,\allowbreak5; 7:2}
\crossref{1Thes}{2}{4}{1Co 7:25 Eph 3:8 1Ti 1:11-\allowbreak13}
\crossref{1Thes}{2}{5}{Job 17:5; 32:21,\allowbreak22 Ps 12:2,\allowbreak3 Pr 20:19; 26:28; 28:23 Isa 30:10}
\crossref{1Thes}{2}{6}{Es 1:4; 5:11 Pr 25:27 Da 4:30 Joh 5:41,\allowbreak44; 7:18; 12:43}
\crossref{1Thes}{2}{7}{Ge 33:13,\allowbreak14 Isa 40:11 Eze 34:14-\allowbreak16 Mt 11:29,\allowbreak30 Joh 21:15-\allowbreak17}
\crossref{1Thes}{2}{8}{Jer 13:15-\allowbreak17 Ro 1:11,\allowbreak12; 9:1-\allowbreak3; 10:1; 15:29 2Co 6:1,\allowbreak11-\allowbreak13}
\crossref{1Thes}{2}{9}{1Th 1:3 Ac 18:3; 20:34,\allowbreak35 1Co 4:12; 9:6,\allowbreak15 2Co 6:5 2Th 3:7-\allowbreak9}
\crossref{1Thes}{2}{10}{1Th 1:5 1Sa 12:3-\allowbreak5 Ac 20:18,\allowbreak26,\allowbreak33,\allowbreak34 2Co 4:2; 5:11; 11:11,\allowbreak31}
\crossref{1Thes}{2}{11}{1Th 4:1; 5:11 Ac 20:2 2Th 3:12 1Ti 6:2 2Ti 4:2 Tit 2:6,\allowbreak9,\allowbreak15}
\crossref{1Thes}{2}{12}{1Th 4:1,\allowbreak12 Ga 5:16 Eph 4:1; 5:2,\allowbreak8 Php 1:27 Col 1:10; 2:6 1Pe 1:15,\allowbreak16}
\crossref{1Thes}{2}{13}{1Th 1:2,\allowbreak3 Ro 1:8,\allowbreak9}
\crossref{1Thes}{2}{14}{1Th 1:6}
\crossref{1Thes}{2}{15}{Mt 5:12; 21:35-\allowbreak39; 23:31-\allowbreak35,\allowbreak37; 27:25 Lu 11:48-\allowbreak51; 13:33,\allowbreak34}
\crossref{1Thes}{2}{16}{Ac 11:2,\allowbreak3,\allowbreak17,\allowbreak18; 13:50; 14:5,\allowbreak19; 17:5,\allowbreak6,\allowbreak13; 18:12,\allowbreak13; 19:9}
\crossref{1Thes}{2}{17}{2Ki 5:26 Ac 17:10 1Co 5:3 Col 2:5}
\crossref{1Thes}{2}{18}{1Co 16:21 Col 4:18 2Th 3:17 Phm 1:9}
\crossref{1Thes}{2}{19}{2Co 1:14 Php 2:16; 4:1}
\crossref{1Thes}{2}{20}{Pr 17:6 1Co 11:7}
\crossref{1Thes}{3}{1}{3:5; 2:17 Jer 20:9; 44:22 2Co 2:13; 11:29,\allowbreak30}
\crossref{1Thes}{3}{2}{Ac 16:1; 17:14,\allowbreak15; 18:5}
\crossref{1Thes}{3}{3}{Ps 112:6 Ac 2:25; 20:24; 21:13 Ro 5:3 1Co 15:58 Eph 3:13}
\crossref{1Thes}{3}{4}{Joh 16:1-\allowbreak3 Ac 20:24}
\crossref{1Thes}{3}{5}{3:1}
\crossref{1Thes}{3}{6}{Ac 18:1,\allowbreak5}
\crossref{1Thes}{3}{7}{3:8,\allowbreak9 2Co 1:4; 7:6,\allowbreak7,\allowbreak13 2Jo 1:4}
\crossref{1Thes}{3}{8}{1Sa 25:6}
\crossref{1Thes}{3}{9}{1Th 1:2,\allowbreak3 2Sa 7:18-\allowbreak20 Ne 9:5 Ps 71:14,\allowbreak15 2Co 2:14; 9:15}
\crossref{1Thes}{3}{10}{Lu 2:37 Ac 26:7 2Ti 1:3 Re 4:8; 7:15}
\crossref{1Thes}{3}{11}{3:13 Isa 63:16 Jer 31:9 Mal 1:6 Mt 6:4,\allowbreak6,\allowbreak8,\allowbreak9,\allowbreak14,\allowbreak18,\allowbreak26,\allowbreak32}
\crossref{1Thes}{3}{12}{1Th 4:10 Ps 115:4 Lu 17:5 2Co 9:10 Jas 1:17 2Pe 3:18}
\crossref{1Thes}{3}{13}{1Th 5:23 Ro 14:4; 16:25 1Co 1:8 Php 1:10 2Th 2:16,\allowbreak17}
\crossref{1Thes}{4}{1}{1Th 2:11 Ro 12:1 2Co 6:1; 10:1 Eph 4:1 Phm 1:9,\allowbreak10 Heb 13:22}
\crossref{1Thes}{4}{2}{Eze 3:17 Mt 28:20 1Co 9:21 2Th 3:6,\allowbreak10}
\crossref{1Thes}{4}{3}{1Th 5:18 Ps 40:8; 143:10 Mt 7:21; 12:50 Mr 3:35 Joh 4:34; 7:17}
\crossref{1Thes}{4}{4}{Ro 6:19; 12:1 1Co 6:15,\allowbreak18-\allowbreak20}
\crossref{1Thes}{4}{5}{Ro 1:24,\allowbreak26 Col 3:5}
\crossref{1Thes}{4}{6}{Ex 20:15,\allowbreak17 Le 19:11,\allowbreak13 De 24:7; 25:13-\allowbreak16 Pr 11:1; 16:11; 20:14}
\crossref{1Thes}{4}{7}{Le 11:44; 19:2 Ro 1:7; 8:29,\allowbreak30 1Co 1:2 Eph 1:4; 2:10; 4:1}
\crossref{1Thes}{4}{8}{1Sa 8:7; 10:19 Joh 12:48}
\crossref{1Thes}{4}{9}{Le 19:8 Ps 133:1 Joh 13:34,\allowbreak35; 15:12-\allowbreak17 Ac 4:32 Ro 12:10}
\crossref{1Thes}{4}{10}{1Th 1:7 2Co 8:1,\allowbreak2,\allowbreak8-\allowbreak10 Eph 1:15 Col 1:4 2Th 1:3 Phm 1:5-\allowbreak7}
\crossref{1Thes}{4}{11}{Pr 17:1 Ec 4:6 La 3:26 2Th 3:12 1Ti 2:2 1Pe 3:4}
\crossref{1Thes}{4}{12}{1Th 5:22 Ro 12:17; 13:13 2Co 8:20,\allowbreak21 Php 4:8 Tit 2:8-\allowbreak10}
\crossref{1Thes}{4}{13}{Ro 1:13 1Co 10:1; 12:1 2Co 1:8 2Pe 3:8}
\crossref{1Thes}{4}{14}{Isa 26:19 Ro 8:11 1Co 15:12-\allowbreak23 2Co 4:13,\allowbreak14 Re 1:18}
\crossref{1Thes}{4}{15}{1Ki 13:1,\allowbreak9,\allowbreak17,\allowbreak18,\allowbreak22; 20:35; 22:14}
\crossref{1Thes}{4}{16}{Isa 25:8,\allowbreak9 Mt 16:27; 24:30,\allowbreak31; 25:31; 26:64 Ac 1:11 2Th 1:7}
\crossref{1Thes}{4}{17}{4:15 1Co 15:52}
\crossref{1Thes}{4}{18}{1Th 5:11,\allowbreak14 Isa 40:1,\allowbreak2 Lu 21:28 Heb 12:12}
\crossref{1Thes}{5}{1}{Mt 24:3,\allowbreak36 Mr 13:30-\allowbreak32 Ac 1:7}
\crossref{1Thes}{5}{2}{Jer 23:20}
\crossref{1Thes}{5}{3}{De 29:19 Jud 18:27,\allowbreak28 Ps 10:11-\allowbreak13 Isa 21:4; 56:12 Da 5:3-\allowbreak6}
\crossref{1Thes}{5}{4}{Ro 13:11-\allowbreak13 Col 1:13 1Pe 2:9,\allowbreak10 1Jo 2:8}
\crossref{1Thes}{5}{5}{Lu 16:8 Joh 12:36 Ac 26:18 Eph 5:8}
\crossref{1Thes}{5}{6}{Pr 19:15 Isa 56:10 Jon 1:6 Mt 13:25; 25:5 Mr 14:37 Lu 22:46}
\crossref{1Thes}{5}{7}{Job 4:13; 33:15 Lu 21:34,\allowbreak35 Ro 13:13 1Co 15:34 Eph 5:14}
\crossref{1Thes}{5}{8}{5:5 Ro 13:13 Eph 5:8,\allowbreak9 1Pe 2:9 1Jo 1:7}
\crossref{1Thes}{5}{9}{1Th 1:10; 3:3 Ex 9:16 Pr 16:4 Eze 38:10-\allowbreak17 Mt 26:24 Ac 1:20,\allowbreak25}
\crossref{1Thes}{5}{10}{Mt 20:28 Joh 10:11,\allowbreak15,\allowbreak17; 15:13 Ro 5:6-\allowbreak8; 8:34; 14:8,\allowbreak9}
\crossref{1Thes}{5}{11}{1Th 4:18}
\crossref{1Thes}{5}{12}{1Co 16:18 Php 2:19 Heb 13:7,\allowbreak17}
\crossref{1Thes}{5}{13}{Mt 10:40 1Co 4:1,\allowbreak2; 9:7-\allowbreak11 Ga 4:14; 6:6}
\crossref{1Thes}{5}{14}{Ro 12:1}
\crossref{1Thes}{5}{15}{Ge 45:24 1Co 16:10 Eph 5:15,\allowbreak33 1Pe 1:22 Re 19:10; 22:9}
\crossref{1Thes}{5}{16}{2Co 6:10 Php 4:4 Mt 5:12 Lu 10:20 Ro 12:12}
\crossref{1Thes}{5}{17}{Lu 18:1; 21:36 Ro 12:12 Eph 6:18 Col 4:2 1Pe 4:7}
\crossref{1Thes}{5}{18}{Eph 5:20 Php 4:6 Col 3:17 Job 1:21 Ps 34:1 Heb 13:15}
\crossref{1Thes}{5}{19}{So 8:7 Eph 4:30; 6:16}
\crossref{1Thes}{5}{20}{1Th 4:8 Nu 11:25-\allowbreak29 1Sa 10:5,\allowbreak6,\allowbreak10-\allowbreak13; 19:20-\allowbreak24 Ac 19:6}
\crossref{1Thes}{5}{21}{Isa 8:20 Mt 7:15-\allowbreak20 Mr 7:14-\allowbreak16 Lu 12:57 Ac 17:11 Ro 12:2}
\crossref{1Thes}{5}{22}{1Th 4:12 Ex 23:7 Isa 33:15 Mt 17:26,\allowbreak27 Ro 12:17}
\crossref{1Thes}{5}{23}{Ro 15:5,\allowbreak13,\allowbreak33; 16:20 1Co 14:33 2Co 5:19 Php 4:9 2Th 3:16}
\crossref{1Thes}{5}{24}{De 7:9 Ps 36:5; 40:10; 86:15; 89:2; 92:2; 100:5; 138:2; 146:6}
\crossref{1Thes}{5}{25}{Ro 15:30 2Co 1:11 Eph 6:18-\allowbreak20 Php 1:19 Col 4:3 2Th 3:1-\allowbreak3}
\crossref{1Thes}{5}{26}{Ro 16:16 1Co 16:20}
\crossref{1Thes}{5}{27}{1Th 2:11 Nu 27:23 1Ki 22:16 2Ch 18:15 Mt 26:63 Mr 5:7 Ac 19:13}
\crossref{1Thes}{5}{28}{Ro 1:7; 16:20,\allowbreak24 2Th 3:18}

% 2Thes
\crossref{2Thes}{1}{1}{2Co 1:19 1Th 1:1 etc.}
\crossref{2Thes}{1}{2}{Ro 1:7 1Co 1:3,\allowbreak8}
\crossref{2Thes}{1}{3}{2Th 2:13}
\crossref{2Thes}{1}{4}{2Co 7:14; 9:2,\allowbreak4 1Th 2:19}
\crossref{2Thes}{1}{5}{1:6 Php 1:28 1Pe 4:14-\allowbreak18}
\crossref{2Thes}{1}{6}{De 32:41-\allowbreak43 Ps 74:22,\allowbreak23; 79:10-\allowbreak12; 94:20-\allowbreak23 Isa 49:26 Zec 2:8}
\crossref{2Thes}{1}{7}{Isa 57:2 Mt 5:10-\allowbreak12 Lu 16:25 Ro 8:17 2Co 4:17 2Ti 2:12}
\crossref{2Thes}{1}{8}{Ge 3:24 De 4:11; 5:5 Ps 21:8,\allowbreak9; 50:2-\allowbreak6 Da 7:10 Mt 25:41,\allowbreak46}
\crossref{2Thes}{1}{9}{Isa 33:14; 66:24 Da 12:2 Mt 25:41,\allowbreak46; 26:24 Mr 9:43-\allowbreak49}
\crossref{2Thes}{1}{10}{1:12 Nu 23:23 Ps 89:7 Isa 43:21; 44:23; 49:3; 60:21 Jer 33:9}
\crossref{2Thes}{1}{11}{Ro 1:9 Eph 1:16; 3:14-\allowbreak21 Php 1:9-\allowbreak11 Col 1:9-\allowbreak13 1Th 3:9-\allowbreak13}
\crossref{2Thes}{1}{12}{1:10 Joh 17:10 1Pe 4:14}
\crossref{2Thes}{2}{1}{Ro 12:1}
\crossref{2Thes}{2}{2}{Isa 7:2; 8:12,\allowbreak13; 26:3 Mt 24:6 Mr 13:7 Lu 21:9,\allowbreak19 Joh 14:1,\allowbreak27}
\crossref{2Thes}{2}{3}{Mt 24:4-\allowbreak6 1Co 6:9 Eph 5:6}
\crossref{2Thes}{2}{4}{Isa 14:13 Eze 28:2,\allowbreak6,\allowbreak9 Da 7:8,\allowbreak25; 8:9-\allowbreak11; 11:36 Re 13:6}
\crossref{2Thes}{2}{5}{Mt 16:9 Mr 8:18 Lu 24:6,\allowbreak7 Ac 20:31}
\crossref{2Thes}{2}{6}{2:7}
\crossref{2Thes}{2}{7}{1Ti 3:16 Re 17:5,\allowbreak7}
\crossref{2Thes}{2}{8}{2:3 Mt 13:19,\allowbreak38 1Jo 2:13; 3:12; 5:18}
\crossref{2Thes}{2}{9}{Joh 8:41,\allowbreak44 Ac 8:9-\allowbreak11; 13:10 2Co 4:4; 11:3,\allowbreak14 Eph 2:2 Re 9:11}
\crossref{2Thes}{2}{10}{Ro 16:18 2Co 2:17; 4:2; 11:13,\allowbreak15 Eph 4:14 2Pe 2:18 Heb 3:13}
\crossref{2Thes}{2}{11}{Ps 81:11,\allowbreak12; 109:17 Isa 29:9-\allowbreak14 Joh 12:39-\allowbreak43 Ro 1:21-\allowbreak25,\allowbreak28}
\crossref{2Thes}{2}{12}{De 32:35 Mr 16:16 Joh 3:36 1Th 5:9 2Pe 2:3 Jude 1:4,\allowbreak5}
\crossref{2Thes}{2}{13}{2Th 1:3}
\crossref{2Thes}{2}{14}{Ro 8:28-\allowbreak30 1Th 2:12 1Pe 5:10}
\crossref{2Thes}{2}{15}{1Co 15:58; 16:13 Php 4:1}
\crossref{2Thes}{2}{16}{2Th 1:1,\allowbreak2 Ro 1:7 1Th 3:11}
\crossref{2Thes}{2}{17}{2:16 Isa 51:3,\allowbreak12; 57:15; 61:1,\allowbreak2; 66:13 Ro 15:13 2Co 1:3-\allowbreak6}
\crossref{2Thes}{3}{1}{Mt 9:38 Lu 10:2 Ro 15:30 2Co 1:11 Eph 6:19,\allowbreak20 Col 4:3}
\crossref{2Thes}{3}{2}{Ro 15:31 1Co 15:32 2Co 1:8-\allowbreak10 1Th 2:18 2Ti 4:17}
\crossref{2Thes}{3}{3}{1Co 1:9; 10:13 1Th 5:24}
\crossref{2Thes}{3}{4}{Ro 15:14 2Co 2:3; 7:16; 8:22 Ga 5:10 Php 1:6 Phm 1:21}
\crossref{2Thes}{3}{5}{1Ki 8:58 1Ch 29:18 Ps 119:5,\allowbreak36 Pr 3:6 Jer 10:23 Jas 1:16-\allowbreak18}
\crossref{2Thes}{3}{6}{1Co 5:4 2Co 2:10 Eph 4:17 Col 3:17 1Th 4:1 1Ti 5:21; 6:13,\allowbreak14}
\crossref{2Thes}{3}{7}{3:9 1Co 4:16; 11:1 Php 3:17; 4:9 1Th 1:6,\allowbreak7 1Ti 4:12 Tit 2:7}
\crossref{2Thes}{3}{8}{3:12 Pr 31:27 Mt 6:11}
\crossref{2Thes}{3}{9}{Mt 10:10 1Co 9:4-\allowbreak14 Ga 6:6 1Th 2:6}
\crossref{2Thes}{3}{10}{Lu 24:44 Joh 16:4 Ac 20:18}
\crossref{2Thes}{3}{11}{3:6}
\crossref{2Thes}{3}{12}{3:6}
\crossref{2Thes}{3}{13}{Isa 40:30,\allowbreak31 Mal 1:13 Ro 2:7 1Co 15:28 Ga 6:9,\allowbreak10 Php 1:9}
\crossref{2Thes}{3}{14}{De 16:12 Pr 5:13 Zep 3:2 2Co 2:9; 7:15; 10:6 Php 2:12 1Th 4:8}
\crossref{2Thes}{3}{15}{Le 19:17,\allowbreak18 1Co 5:5 2Co 2:6-\allowbreak10; 10:8; 13:10 Ga 6:1 1Th 5:14}
\crossref{2Thes}{3}{16}{Ps 72:3,\allowbreak7 Isa 9:6,\allowbreak7 Zec 6:13 Lu 2:14 Joh 14:27 Ro 15:33; 16:20}
\crossref{2Thes}{3}{17}{1Co 16:21 Col 4:18}
\crossref{2Thes}{3}{18}{Ro 16:20,\allowbreak24}

% 1Tim
\crossref{1Tim}{1}{1}{Ro 1:1 1Co 1:1}
\crossref{1Tim}{1}{2}{Ac 16:1-\allowbreak3 1Th 3:2}
\crossref{1Tim}{1}{3}{Ac 19:1 etc.}
\crossref{1Tim}{1}{4}{1Ti 4:7; 6:4,\allowbreak20 2Ti 2:14,\allowbreak16-\allowbreak18; 4:4 Tit 1:14 2Pe 1:16}
\crossref{1Tim}{1}{5}{Ro 10:4; 13:8-\allowbreak10 Ga 5:13,\allowbreak14,\allowbreak22 1Jo 4:7-\allowbreak14}
\crossref{1Tim}{1}{6}{1Ti 5:15; 6:4,\allowbreak5,\allowbreak20 2Ti 2:23,\allowbreak24 Tit 1:10; 3:9}
\crossref{1Tim}{1}{7}{Ac 15:1 Ro 2:19-\allowbreak21 Ga 3:2,\allowbreak5; 4:21; 5:3,\allowbreak4 Tit 1:10,\allowbreak11}
\crossref{1Tim}{1}{8}{De 4:6-\allowbreak8 Ne 9:13 Ps 19:7-\allowbreak10; 119:96-\allowbreak105,\allowbreak127,\allowbreak128 Ro 7:12,\allowbreak13,\allowbreak16}
\crossref{1Tim}{1}{9}{Ro 4:13; 5:20; 6:14 Ga 3:10-\allowbreak14,\allowbreak19; 5:23}
\crossref{1Tim}{1}{10}{Mr 7:21,\allowbreak22 1Co 6:9,\allowbreak10 Ga 5:19-\allowbreak21 Eph 5:3-\allowbreak6 Heb 13:4}
\crossref{1Tim}{1}{11}{Ro 2:16}
\crossref{1Tim}{1}{12}{Joh 5:23 Php 2:11 Re 5:9-\allowbreak14; 7:10-\allowbreak12}
\crossref{1Tim}{1}{13}{Ac 8:3; 9:1,\allowbreak5,\allowbreak13; 22:4; 26:9-\allowbreak11 1Co 15:9 Ga 1:13 Php 3:6}
\crossref{1Tim}{1}{14}{Ac 15:11 Ro 16:20 2Co 8:9; 13:14 Re 22:21}
\crossref{1Tim}{1}{15}{1:19; 3:1; 4:9 2Ti 2:11 Tit 3:8 Re 21:5; 22:6}
\crossref{1Tim}{1}{16}{Nu 23:3 Ps 25:11 Isa 1:18; 43:25 Eph 1:6,\allowbreak12; 2:7 2Th 1:10}
\crossref{1Tim}{1}{17}{1Ti 6:15,\allowbreak16 Ps 10:16; 45:1,\allowbreak6; 47:6-\allowbreak8; 90:2; 145:13 Jer 10:10 Da 2:44}
\crossref{1Tim}{1}{18}{1:11,\allowbreak12; 4:14; 6:13,\allowbreak14,\allowbreak20 2Ti 2:2; 4:1-\allowbreak3}
\crossref{1Tim}{1}{19}{1:5; 3:9 Tit 1:9 Heb 3:14 1Pe 3:15,\allowbreak16 Re 3:3,\allowbreak8,\allowbreak10}
\crossref{1Tim}{1}{20}{2Ti 2:17}
\crossref{1Tim}{2}{1}{2Co 8:6 Eph 3:13 Heb 6:11}
\crossref{1Tim}{2}{2}{Ezr 6:10 Ne 1:11 Ps 20:1-\allowbreak4; 72:1 Jer 29:7}
\crossref{1Tim}{2}{3}{1Ti 5:4 Ro 12:1,\allowbreak2; 14:18 Eph 5:9,\allowbreak10 Php 1:11; 4:18 Col 1:10 1Th 4:1}
\crossref{1Tim}{2}{4}{Isa 45:22; 49:6; 55:1 Eze 18:23,\allowbreak32; 33:11 Lu 14:23 Joh 3:15-\allowbreak17}
\crossref{1Tim}{2}{5}{De 6:4 Isa 44:6 Mr 12:29-\allowbreak33 Joh 17:3 Ro 3:29,\allowbreak30; 10:12 1Co 8:6}
\crossref{1Tim}{2}{6}{Job 33:24 Isa 53:6 Mt 20:28 Mr 10:45 Joh 6:51; 10:15}
\crossref{1Tim}{2}{7}{1Ti 1:11,\allowbreak12}
\crossref{1Tim}{2}{8}{1Ti 5:14 1Co 7:7}
\crossref{1Tim}{2}{9}{1Pe 3:3-\allowbreak5}
\crossref{1Tim}{2}{10}{1Pe 3:3-\allowbreak5 2Pe 3:11}
\crossref{1Tim}{2}{11}{Ge 3:16 Es 1:20 1Co 11:3; 14:34,\allowbreak35 Eph 5:22-\allowbreak24 Col 3:18}
\crossref{1Tim}{2}{12}{2:12}
\crossref{1Tim}{2}{13}{Ge 1:27; 2:7,\allowbreak18,\allowbreak22 1Co 11:8,\allowbreak9}
\crossref{1Tim}{2}{14}{Ge 3:6,\allowbreak12 2Co 11:3}
\crossref{1Tim}{2}{15}{Ge 3:15 Isa 7:14; 9:6 Jer 31:22 Mt 1:21-\allowbreak25 Lu 2:7,\allowbreak10,\allowbreak11}
\crossref{1Tim}{3}{1}{1Ti 1:15; 4:9 2Ti 2:11 Tit 3:8}
\crossref{1Tim}{3}{2}{Tit 1:6-\allowbreak9}
\crossref{1Tim}{3}{3}{2Ti 2:24,\allowbreak25 Tit 1:7}
\crossref{1Tim}{3}{4}{3:12 Ge 18:19 Jos 24:15 Ps 101:2-\allowbreak8 Ac 10:2 Tit 1:6}
\crossref{1Tim}{3}{5}{1Sa 2:29,\allowbreak30; 3:13}
\crossref{1Tim}{3}{6}{1Co 3:1 Heb 5:12,\allowbreak13 1Pe 2:2}
\crossref{1Tim}{3}{7}{1Ti 5:24,\allowbreak25 1Sa 2:24 Ac 6:3; 10:22; 22:12 3Jo 1:12}
\crossref{1Tim}{3}{8}{Ac 6:3-\allowbreak6 Php 1:1}
\crossref{1Tim}{3}{9}{1Ti 1:5,\allowbreak19}
\crossref{1Tim}{3}{10}{3:6; 5:22 1Jo 4:1}
\crossref{1Tim}{3}{11}{Le 21:7,\allowbreak13-\allowbreak15 Eze 44:22 Lu 1:5-\allowbreak6 Tit 2:3}
\crossref{1Tim}{3}{12}{3:2,\allowbreak4,\allowbreak5}
\crossref{1Tim}{3}{13}{Mt 25:21 Lu 16:10-\allowbreak12; 19:17}
\crossref{1Tim}{3}{14}{1Ti 4:13 1Co 11:34; 16:5-\allowbreak7 2Co 1:15-\allowbreak17 1Th 2:18 Phm 1:22 Heb 13:23}
\crossref{1Tim}{3}{15}{3:2 De 31:23 1Ki 2:2,\allowbreak4 1Ch 22:13; 28:9-\allowbreak21 Ac 1:2}
\crossref{1Tim}{3}{16}{Heb 7:7}
\crossref{1Tim}{4}{1}{Joh 16:13 Ac 13:2; 28:25 1Co 12:11 1Jo 2:18 Re 2:7,\allowbreak11,\allowbreak17,\allowbreak29}
\crossref{1Tim}{4}{2}{1Ki 13:18; 22:22 Isa 9:15 Jer 5:21; 23:14,\allowbreak32 Da 8:23-\allowbreak25 Mt 7:15}
\crossref{1Tim}{4}{3}{Da 11:37 1Co 7:28,\allowbreak36-\allowbreak39 Heb 13:4}
\crossref{1Tim}{4}{4}{Ge 1:31 De 32:4}
\crossref{1Tim}{4}{5}{4:3 Lu 11:41 1Co 7:14 Tit 1:15}
\crossref{1Tim}{4}{6}{Ac 20:31,\allowbreak35 Ro 15:15 1Co 4:17 2Ti 1:6; 2:14 2Pe 1:12-\allowbreak15; 3:1,\allowbreak2}
\crossref{1Tim}{4}{7}{1Ti 1:4; 6:20 2Ti 2:16,\allowbreak23; 4:4 Tit 1:14; 3:9}
\crossref{1Tim}{4}{8}{1Sa 15:22 Ps 50:7-\allowbreak15 Isa 1:11-\allowbreak16; 58:3-\allowbreak5 Jer 6:20 Am 5:21-\allowbreak24}
\crossref{1Tim}{4}{9}{1Ti 1:15}
\crossref{1Tim}{4}{10}{1Co 4:9-\allowbreak13 2Co 4:8-\allowbreak10; 6:3-\allowbreak10; 11:23-\allowbreak27 2Ti 2:9,\allowbreak10; 3:10-\allowbreak12}
\crossref{1Tim}{4}{11}{1Ti 6:2 2Ti 4:2 Tit 2:15; 3:8}
\crossref{1Tim}{4}{12}{Mt 18:10 1Co 16:10,\allowbreak11 2Ti 2:7,\allowbreak15,\allowbreak22}
\crossref{1Tim}{4}{13}{1Ti 3:14,\allowbreak15}
\crossref{1Tim}{4}{14}{Mt 25:14-\allowbreak30 Lu 19:12-\allowbreak26 Ro 12:6-\allowbreak8 1Th 5:19 2Ti 1:6 1Pe 4:9-\allowbreak11}
\crossref{1Tim}{4}{15}{Jos 1:8 Ps 1:2; 19:14; 49:3; 63:6; 77:12; 104:34; 105:5; 119:15,\allowbreak23,\allowbreak48}
\crossref{1Tim}{4}{16}{1Ch 28:10 2Ch 19:6 Mr 13:9 Lu 21:34 Ac 20:28 1Co 3:10,\allowbreak11}
\crossref{1Tim}{5}{1}{5:19,\allowbreak20 Le 19:32 De 33:9 Ga 2:11-\allowbreak14}
\crossref{1Tim}{5}{2}{5:3 Mt 12:50 Joh 19:26,\allowbreak27}
\crossref{1Tim}{5}{3}{5:2,\allowbreak17 Ex 20:12 Mt 15:6 1Th 2:6 1Pe 2:17; 3:7}
\crossref{1Tim}{5}{4}{Jud 12:14}
\crossref{1Tim}{5}{5}{5:3 Ro 1:5,\allowbreak12,\allowbreak20,\allowbreak21 1Co 7:32}
\crossref{1Tim}{5}{6}{1Sa 25:6 Job 21:11-\allowbreak15 Ps 73:5-\allowbreak7 Isa 22:13 Am 6:5,\allowbreak6 Lu 12:19}
\crossref{1Tim}{5}{7}{1Ti 1:3; 4:11; 6:17 2Ti 4:1 Tit 1:13; 2:15}
\crossref{1Tim}{5}{8}{Ge 30:30 Isa 58:7 Mt 7:11 Lu 11:11-\allowbreak13 2Co 12:14 Ga 6:10}
\crossref{1Tim}{5}{9}{5:3,\allowbreak4}
\crossref{1Tim}{5}{10}{1Ti 3:7 Ac 6:3; 10:22; 22:12 3Jo 1:12}
\crossref{1Tim}{5}{11}{5:9,\allowbreak14}
\crossref{1Tim}{5}{12}{1Co 11:34 Jas 3:1 1Pe 4:17}
\crossref{1Tim}{5}{13}{Pr 31:27 2Th 3:6-\allowbreak11}
\crossref{1Tim}{5}{14}{1Ti 2:8}
\crossref{1Tim}{5}{15}{Php 3:18,\allowbreak19 2Ti 1:15; 2:18; 4:10 2Pe 2:2,\allowbreak20-\allowbreak22; 3:16 1Jo 2:19}
\crossref{1Tim}{5}{16}{5:4,\allowbreak8}
\crossref{1Tim}{5}{17}{5:1}
\crossref{1Tim}{5}{18}{Ro 4:3; 9:17; 10:11; 11:2 Ga 3:8 Jas 4:5}
\crossref{1Tim}{5}{19}{Joh 18:29 Ac 24:2-\allowbreak13; 25:16 Tit 1:6}
\crossref{1Tim}{5}{20}{Le 19:17 Ga 2:11-\allowbreak14 2Ti 4:2 Tit 1:13}
\crossref{1Tim}{5}{21}{1Ti 6:13 1Th 5:27 2Ti 2:14; 4:1}
\crossref{1Tim}{5}{22}{1Ti 4:14 Ac 6:6; 13:3 2Ti 1:6 Heb 6:2}
\crossref{1Tim}{5}{23}{1Ti 3:3; 4:4 Le 10:9-\allowbreak11 Ps 104:15 Pr 31:4-\allowbreak7 Eze 44:21 Eph 5:18}
\crossref{1Tim}{5}{24}{Jer 2:34 Ac 1:16-\allowbreak20; 5:1-\allowbreak11; 8:18 Ga 5:19-\allowbreak21 2Ti 4:10}
\crossref{1Tim}{5}{25}{1Ti 3:7 Mt 5:16 Ac 9:36; 10:22; 16:1-\allowbreak3; 22:12 Ga 5:22,\allowbreak23 Php 1:11}
\crossref{1Tim}{6}{1}{De 28:48 Isa 47:6; 58:6 Mt 11:9,\allowbreak30 Ac 15:10 1Co 7:21,\allowbreak22 Ga 5:1}
\crossref{1Tim}{6}{2}{Col 4:1 Phm 1:10-\allowbreak16}
\crossref{1Tim}{6}{3}{1Ti 1:3,\allowbreak6 Ro 16:17 Ga 1:6,\allowbreak7}
\crossref{1Tim}{6}{4}{1Ti 1:7; 3:6 Pr 13:7; 25:14; 26:12 Ac 8:9,\allowbreak21-\allowbreak23 Ro 12:16 1Co 3:18}
\crossref{1Tim}{6}{5}{1Ti 1:6 1Co 11:16}
\crossref{1Tim}{6}{6}{1Ti 4:8 Ps 37:16; 84:11 Pr 3:13-\allowbreak18; 8:18-\allowbreak21; 15:16; 16:8 Mt 6:32,\allowbreak33}
\crossref{1Tim}{6}{7}{Job 1:21 Pr 27:24 Ec 5:15,\allowbreak16}
\crossref{1Tim}{6}{8}{Ge 28:20; 48:15 De 2:7; 8:3,\allowbreak4 Pr 27:23-\allowbreak27; 30:8,\allowbreak9 Ec 2:24-\allowbreak26}
\crossref{1Tim}{6}{9}{Ge 13:10-\allowbreak13 Nu 22:17-\allowbreak19 Jos 7:11 2Ki 5:20-\allowbreak27 Pr 15:27; 20:21}
\crossref{1Tim}{6}{10}{Ge 34:23,\allowbreak24; 38:16 Ex 23:7,\allowbreak8 De 16:19; 23:4,\allowbreak5,\allowbreak18 Jud 17:10,\allowbreak11}
\crossref{1Tim}{6}{11}{2Ti 2:22}
\crossref{1Tim}{6}{12}{1Ti 1:18 Zec 10:5 1Co 9:25,\allowbreak26 2Co 6:7; 10:3-\allowbreak5 Eph 6:10-\allowbreak18 1Th 5:8,\allowbreak9}
\crossref{1Tim}{6}{13}{1Ti 5:21}
\crossref{1Tim}{6}{14}{6:20; 4:11-\allowbreak16 1Ch 28:9,\allowbreak10,\allowbreak20 Col 4:17}
\crossref{1Tim}{6}{15}{1Ti 1:11,\allowbreak17 Ps 47:2; 83:18 Jer 10:10; 46:18 Da 2:44-\allowbreak47; 4:34 Mt 6:13}
\crossref{1Tim}{6}{16}{1Ti 1:17 Ex 3:14 De 32:40 Ps 90:2 Isa 57:15 Joh 8:58 Heb 13:8}
\crossref{1Tim}{6}{17}{6:13; 1:3; 5:21}
\crossref{1Tim}{6}{18}{2Ch 24:16 Ps 37:3 Ec 3:12 Lu 6:33-\allowbreak35 Ac 10:38 Ga 6:10}
\crossref{1Tim}{6}{19}{Ps 17:14 Mt 6:19-\allowbreak21; 10:41,\allowbreak42; 19:21; 25:34-\allowbreak40 Lu 12:33; 16:9}
\crossref{1Tim}{6}{20}{6:11 2Ti 2:1}
\crossref{1Tim}{6}{21}{6:10; 1:6,\allowbreak19 2Ti 2:18 Heb 10:1-\allowbreak12:29}

% 2Tim
\crossref{2Tim}{1}{1}{Ro 1:1 2Co 1:1}
\crossref{2Tim}{1}{2}{1Ti 1:2 Ro 12:19 Php 4:1}
\crossref{2Tim}{1}{3}{Ro 1:8 Eph 1:16}
\crossref{2Tim}{1}{4}{2Ti 4:9,\allowbreak21 Ro 1:11; 15:30-\allowbreak32 Php 1:8; 2:26 1Th 2:17-\allowbreak20; 3:1}
\crossref{2Tim}{1}{5}{Ps 77:6}
\crossref{2Tim}{1}{6}{2Ti 2:14 Isa 43:26 1Ti 4:6 2Pe 1:12; 3:1 Jude 1:5}
\crossref{2Tim}{1}{7}{Ac 20:24; 21:13 Ro 8:15 Heb 2:15 1Jo 4:18}
\crossref{2Tim}{1}{8}{1:12 Ps 119:46 Isa 51:7 Mr 8:38 Lu 9:26 Ac 5:41 Ro 1:16; 9:33}
\crossref{2Tim}{1}{9}{Mt 1:21 Ac 2:47 1Co 1:18 Eph 2:5,\allowbreak8 1Ti 1:1 Tit 3:4,\allowbreak5}
\crossref{2Tim}{1}{10}{Isa 25:7; 60:2,\allowbreak3 Lu 2:31,\allowbreak32 Ro 16:26 Eph 1:9 Col 1:26,\allowbreak27}
\crossref{2Tim}{1}{11}{Ac 9:15 Eph 3:7,\allowbreak8}
\crossref{2Tim}{1}{12}{1:8; 2:9; 3:10-\allowbreak12; 4:16,\allowbreak17 Ac 9:16; 13:46,\allowbreak50; 14:5,\allowbreak6; 21:27-\allowbreak31; 22:21-\allowbreak24}
\crossref{2Tim}{1}{13}{1:14; 3:14 Pr 3:18,\allowbreak21; 4:4-\allowbreak8,\allowbreak13; 23:23 Php 1:27 1Th 5:21 Tit 1:9}
\crossref{2Tim}{1}{14}{2Ti 2:2 Lu 16:11 Ro 3:2 1Co 9:17 2Co 5:19,\allowbreak20 Ga 2:7 Col 4:11}
\crossref{2Tim}{1}{15}{Ac 16:6; 19:10,\allowbreak27,\allowbreak31; 20:16 1Co 16:19}
\crossref{2Tim}{1}{16}{1:18 Ne 5:19; 13:14,\allowbreak22,\allowbreak31 Ps 18:25; 37:26 Mt 5:7; 10:41,\allowbreak42}
\crossref{2Tim}{1}{17}{Ac 28:30,\allowbreak31}
\crossref{2Tim}{1}{18}{1:16 1Ki 17:20 Mt 25:34-\allowbreak40}
\crossref{2Tim}{2}{1}{2Ti 1:2}
\crossref{2Tim}{2}{2}{2Ti 1:13; 3:10,\allowbreak14}
\crossref{2Tim}{2}{3}{2:10; 1:8; 3:11; 4:5 1Co 13:7 2Co 1:6 Heb 6:15; 10:32; 11:27; 12:2,\allowbreak3}
\crossref{2Tim}{2}{4}{De 20:5-\allowbreak7 Lu 9:59-\allowbreak62}
\crossref{2Tim}{2}{5}{Lu 13:24 1Co 9:24-\allowbreak27 Php 1:15 Col 1:29 Heb 12:4}
\crossref{2Tim}{2}{6}{Isa 28:24-\allowbreak26 Mt 9:37,\allowbreak38; 20:1; 21:33-\allowbreak41 Lu 10:2 Joh 4:35-\allowbreak38}
\crossref{2Tim}{2}{7}{De 4:39; 32:29 Ps 64:9 Pr 24:32 Isa 1:3; 5:12 Lu 9:44 Php 4:8}
\crossref{2Tim}{2}{8}{Heb 12:2,\allowbreak3}
\crossref{2Tim}{2}{9}{2Ti 1:8,\allowbreak12,\allowbreak16 Ac 9:16}
\crossref{2Tim}{2}{10}{2:3 Eph 3:13 Col 1:24}
\crossref{2Tim}{2}{11}{1Ti 1:15; 3:1 Tit 3:8}
\crossref{2Tim}{2}{12}{Mt 19:28,\allowbreak29 Ac 14:22 Ro 8:17 Php 1:28 2Th 1:4-\allowbreak8 1Pe 4:13-\allowbreak16}
\crossref{2Tim}{2}{13}{Isa 25:1 Mt 24:35 Ro 3:3; 9:6 1Th 5:24 2Th 3:3}
\crossref{2Tim}{2}{14}{2Ti 1:6 2Pe 1:13}
\crossref{2Tim}{2}{15}{Heb 4:11 2Pe 1:10,\allowbreak15; 3:14}
\crossref{2Tim}{2}{16}{2:14 1Ti 4:7; 6:20 Tit 1:14; 3:9}
\crossref{2Tim}{2}{17}{Na 3:15 Jas 5:3}
\crossref{2Tim}{2}{18}{Mt 22:29 1Ti 1:19; 6:10,\allowbreak21 Heb 3:10 Jas 5:19}
\crossref{2Tim}{2}{19}{Pr 10:25 Isa 14:32; 28:16 Mt 7:25 Lu 6:48 1Co 3:10,\allowbreak11 Eph 2:20}
\crossref{2Tim}{2}{20}{1Co 3:9,\allowbreak16,\allowbreak17 Eph 2:22 1Ti 3:15 Heb 3:2-\allowbreak6 1Pe 2:5}
\crossref{2Tim}{2}{21}{Ps 119:9 Isa 1:25; 52:11 Jer 15:19 Mal 3:3 1Co 5:7 2Co 7:1}
\crossref{2Tim}{2}{22}{Pr 6:5 1Co 6:18; 10:14 1Ti 6:11}
\crossref{2Tim}{2}{23}{2:14,\allowbreak16 1Ti 1:4; 4:7; 6:4,\allowbreak5 Tit 3:9}
\crossref{2Tim}{2}{24}{De 34:5 Jos 1:1 2Ch 24:6 Da 6:20 1Ti 6:11 Tit 1:1; 3:2 Jas 1:1}
\crossref{2Tim}{2}{25}{Mt 11:29 Ga 6:1 1Ti 6:11 1Pe 3:15}
\crossref{2Tim}{2}{26}{Lu 15:17 1Co 15:34 Eph 5:14}
\crossref{2Tim}{3}{1}{2Ti 4:3 Ge 49:1 Isa 2:2 Jer 48:47; 49:39 Eze 38:16 Da 10:14 Ho 3:5}
\crossref{2Tim}{3}{2}{3:4 Ro 15:1-\allowbreak3 2Co 5:15 Php 2:21 Jas 2:8}
\crossref{2Tim}{3}{3}{Mt 10:21 Ro 1:31}
\crossref{2Tim}{3}{4}{2Pe 2:10 etc.}
\crossref{2Tim}{3}{5}{Isa 29:13; 48:1,\allowbreak2; 58:1-\allowbreak3 Eze 33:30-\allowbreak32 Mt 7:15; 23:27,\allowbreak28}
\crossref{2Tim}{3}{6}{Mt 23:14 Tit 1:11 Jude 1:4}
\crossref{2Tim}{3}{7}{2Ti 4:3,\allowbreak4 De 29:4 Pr 14:6 Isa 30:10,\allowbreak11 Eze 14:4-\allowbreak10 Mt 13:11}
\crossref{2Tim}{3}{8}{Ex 7:11,\allowbreak22; 8:7,\allowbreak18}
\crossref{2Tim}{3}{9}{3:8 Ex 7:12; 8:18,\allowbreak19; 9:11 1Ki 22:25 Ps 76:10 Jer 28:15-\allowbreak17}
\crossref{2Tim}{3}{10}{3:16,\allowbreak17; 4:3 Ac 2:42 Ro 16:17 Eph 4:14 1Ti 1:3; 4:12,\allowbreak13 Tit 2:7 Heb 13:9}
\crossref{2Tim}{3}{11}{Ac 9:16; 20:19,\allowbreak23,\allowbreak24 Ro 8:35-\allowbreak37 1Co 4:9-\allowbreak11 2Co 1:8-\allowbreak10; 4:8-\allowbreak11}
\crossref{2Tim}{3}{12}{2Co 1:12 1Ti 2:2; 3:16; 6:3 Tit 1:1; 2:12 2Pe 3:11}
\crossref{2Tim}{3}{13}{3:8; 2:16,\allowbreak17 2Th 2:6-\allowbreak10 1Ti 4:1 2Pe 2:20; 3:3 Re 12:9; 13:14; 18:23}
\crossref{2Tim}{3}{14}{2Ti 1:13; 2:2}
\crossref{2Tim}{3}{15}{2Ti 1:5 1Sa 2:18 2Ch 34:3 Ps 71:17 Pr 8:17; 22:6 Ec 12:1 Lu 1:15}
\crossref{2Tim}{3}{16}{2Sa 23:2 Mt 21:42; 22:31,\allowbreak32,\allowbreak43; 26:54,\allowbreak56 Mr 12:24,\allowbreak36 Joh 10:35}
\crossref{2Tim}{3}{17}{Ps 119:98-\allowbreak100 1Ti 6:11}
\crossref{2Tim}{4}{1}{2Ti 2:14}
\crossref{2Tim}{4}{2}{Ps 40:9 Isa 61:1-\allowbreak3 Jon 3:2 Lu 4:18,\allowbreak19; 9:60 Ro 10:15}
\crossref{2Tim}{4}{3}{2Ti 3:1-\allowbreak6 1Ti 4:1-\allowbreak3}
\crossref{2Tim}{4}{4}{2Ti 1:15 Pr 1:32 Zec 7:11 Ac 7:57 Heb 13:25}
\crossref{2Tim}{4}{5}{Isa 56:9,\allowbreak10; 62:6 Jer 6:17 Eze 3:17; 33:2,\allowbreak7 Mr 13:34,\allowbreak37 Lu 12:37}
\crossref{2Tim}{4}{6}{Php 2:17}
\crossref{2Tim}{4}{7}{1Ti 6:12}
\crossref{2Tim}{4}{8}{Ps 31:19 Mt 6:19,\allowbreak20 Col 1:5 1Ti 6:19}
\crossref{2Tim}{4}{9}{4:21; 1:4}
\crossref{2Tim}{4}{10}{Col 4:14,\allowbreak15 Phm 1:24}
\crossref{2Tim}{4}{11}{2Ti 1:15}
\crossref{2Tim}{4}{12}{Ac 20:4 Eph 6:21 Col 4:7 Tit 3:12}
\crossref{2Tim}{4}{13}{1Co 4:11 2Co 11:27}
\crossref{2Tim}{4}{14}{Ac 19:33,\allowbreak34 1Ti 1:20}
\crossref{2Tim}{4}{15}{Mt 10:16,\allowbreak17 Php 3:2}
\crossref{2Tim}{4}{16}{Ac 22:1; 25:16 1Co 9:3 2Co 7:11 Php 1:7,\allowbreak17 1Pe 3:15}
\crossref{2Tim}{4}{17}{Ps 37:39,\allowbreak40; 109:31 Jer 15:20,\allowbreak21; 20:10,\allowbreak11 Mt 10:19 Ac 18:9,\allowbreak10}
\crossref{2Tim}{4}{18}{Ge 48:16 1Sa 25:39 1Ch 4:10 Ps 121:7 Mt 6:13 Lu 11:4 Joh 17:15}
\crossref{2Tim}{4}{19}{Ac 18:2,\allowbreak18,\allowbreak26 Ro 16:3,\allowbreak4 1Co 16:19}
\crossref{2Tim}{4}{20}{Ac 19:22 Ro 16:23}
\crossref{2Tim}{4}{21}{4:9,\allowbreak13; 1:4}
\crossref{2Tim}{4}{22}{Mt 28:20 Ro 16:20 2Co 13:14 Ga 6:18 Phm 1:25}

% Titus
\crossref{Titus}{1}{1}{1Ch 6:49 Ro 1:1 Php 1:1}
\crossref{Titus}{1}{2}{Tit 2:7,\allowbreak13; 3:7 Joh 5:39; 6:68 Ro 2:7; 5:2,\allowbreak4 Col 1:27 1Th 5:8}
\crossref{Titus}{1}{3}{Da 8:23; 9:24-\allowbreak27; 10:1; 11:27 Hab 2:3 Ac 17:26 Ro 5:6 Ga 4:4}
\crossref{Titus}{1}{4}{2Co 2:13; 7:6,\allowbreak13,\allowbreak14; 8:6,\allowbreak16,\allowbreak23; 12:18 Ga 2:3}
\crossref{Titus}{1}{5}{1Ti 1:3}
\crossref{Titus}{1}{6}{1Ti 3:2-\allowbreak7}
\crossref{Titus}{1}{7}{1:5 Php 1:1 1Ti 3:1,\allowbreak2 etc.}
\crossref{Titus}{1}{8}{1Ti 3:2}
\crossref{Titus}{1}{9}{Job 2:3; 27:6 Pr 23:23 1Th 5:21 2Th 2:15 2Ti 1:13 Jude 1:3}
\crossref{Titus}{1}{10}{Ac 20:29 Ro 16:17-\allowbreak18 2Co 11:12-\allowbreak15 Eph 4:14 2Th 2:10-\allowbreak12}
\crossref{Titus}{1}{11}{1:9; 3:10 Ps 63:11; 107:42 Eze 16:63 Lu 20:40 Ro 3:19 2Co 11:10}
\crossref{Titus}{1}{12}{Ac 17:28}
\crossref{Titus}{1}{13}{Tit 2:15 Pr 27:5 2Co 13:10 1Ti 5:20 2Ti 4:2}
\crossref{Titus}{1}{14}{1Ti 1:4-\allowbreak7 2Ti 4:4}
\crossref{Titus}{1}{15}{Lu 11:39-\allowbreak41 Ac 10:15 Ro 14:14,\allowbreak20 1Co 6:12,\allowbreak13; 10:23,\allowbreak25,\allowbreak31}
\crossref{Titus}{1}{16}{Nu 24:16 Isa 29:13; 48:1; 58:2 Eze 33:31 Ho 8:2,\allowbreak3 Ro 2:18-\allowbreak24}
\crossref{Titus}{2}{1}{}
\crossref{Titus}{2}{2}{Le 19:32 Job 12:12 Ps 92:14 Pr 16:31 Isa 65:20}
\crossref{Titus}{2}{3}{Ro 16:2 Eph 5:3 1Ti 2:9,\allowbreak10; 3:11; 5:5-\allowbreak10 1Pe 3:3-\allowbreak5}
\crossref{Titus}{2}{4}{1Ti 5:2,\allowbreak11,\allowbreak14}
\crossref{Titus}{2}{5}{2:2}
\crossref{Titus}{2}{6}{Job 29:8 Ps 148:12 Ec 11:9; 12:1 Joe 2:28 1Pe 5:5 1Jo 2:13}
\crossref{Titus}{2}{7}{Ac 20:33-\allowbreak35 2Th 3:9 1Ti 4:12 1Pe 5:3}
\crossref{Titus}{2}{8}{Mr 12:17,\allowbreak28,\allowbreak32,\allowbreak34 1Ti 6:3}
\crossref{Titus}{2}{9}{Eph 6:5-\allowbreak8 Col 3:22-\allowbreak25 1Ti 6:1,\allowbreak2 1Pe 2:18-\allowbreak25}
\crossref{Titus}{2}{10}{2Ki 5:20-\allowbreak24 Lu 16:6-\allowbreak8 Joh 12:6 Ac 5:2,\allowbreak3}
\crossref{Titus}{2}{11}{Tit 3:4,\allowbreak5 Ps 84:11 Zec 4:7; 12:10 Joh 1:14,\allowbreak16,\allowbreak17 Ac 11:23}
\crossref{Titus}{2}{12}{Mt 28:20 Joh 6:25 1Th 4:9 Heb 8:11 1Jo 2:27}
\crossref{Titus}{2}{13}{1Co 1:7 Php 3:20,\allowbreak21 2Ti 4:8 2Pe 3:12-\allowbreak14}
\crossref{Titus}{2}{14}{Mt 20:28 Joh 6:51; 10:15 Ga 1:4; 2:20; 3:13 Eph 5:2,\allowbreak23-\allowbreak27}
\crossref{Titus}{2}{15}{Tit 1:13 2Ti 4:2}
\crossref{Titus}{3}{1}{Isa 43:26 1Ti 4:6 2Ti 1:6 2Pe 1:12; 3:1,\allowbreak2 Jude 1:5}
\crossref{Titus}{3}{2}{Ps 140:11 Pr 6:19 Ac 23:5 1Co 6:10 2Co 12:20 Eph 4:31}
\crossref{Titus}{3}{3}{Ro 3:9-\allowbreak20 1Co 6:9-\allowbreak11 Eph 2:1-\allowbreak3 Col 1:21; 3:7 1Pe 4:1-\allowbreak3}
\crossref{Titus}{3}{4}{Tit 2:11 Ro 5:20,\allowbreak21 Eph 2:4-\allowbreak10}
\crossref{Titus}{3}{5}{Job 9:20; 15:14; 25:4 Ps 143:2 Isa 57:12 Lu 10:27-\allowbreak29}
\crossref{Titus}{3}{6}{Pr 1:23 Isa 32:15; 44:3 Eze 36:25 Joe 2:28 Joh 1:16; 7:37}
\crossref{Titus}{3}{7}{Tit 2:11 Ro 3:24,\allowbreak28; 4:4,\allowbreak16; 5:1,\allowbreak2,\allowbreak15-\allowbreak21; 11:6 1Co 6:11 Ga 2:16}
\crossref{Titus}{3}{8}{Tit 1:9 1Ti 1:15}
\crossref{Titus}{3}{9}{Tit 1:14 1Ti 1:3-\allowbreak7; 4:7 2Ti 2:23}
\crossref{Titus}{3}{10}{1Co 11:19 Ga 5:20 2Pe 2:1}
\crossref{Titus}{3}{11}{Tit 1:11 Ac 15:24 1Ti 1:19,\allowbreak20 2Ti 2:14 Heb 10:26}
\crossref{Titus}{3}{12}{Ac 20:4 2Ti 4:12}
\crossref{Titus}{3}{13}{Mt 22:35 Lu 7:30; 10:25; 11:45,\allowbreak52; 14:3}
\crossref{Titus}{3}{14}{3:8}
\crossref{Titus}{3}{15}{Ro 16:21-\allowbreak24}

% Phlm
\crossref{Phlm}{1}{1}{1:9}
\crossref{Phlm}{1}{2}{Col 4:17}
\crossref{Phlm}{1}{3}{Ro 1:7 2Co 13:14 Eph 1:2}
\crossref{Phlm}{1}{4}{Ro 1:8 Eph 1:16 Php 1:3 Col 1:3 1Th 1:2 2Th 1:3 2Ti 1:3}
\crossref{Phlm}{1}{5}{Ga 5:6 Eph 1:15 Col 1:4}
\crossref{Phlm}{1}{6}{2Co 9:12-\allowbreak14 Php 1:9-\allowbreak11 Tit 3:14 Heb 6:10 Jas 2:14,\allowbreak17}
\crossref{Phlm}{1}{7}{1Th 1:3; 2:13,\allowbreak19; 3:9 2Jo 1:4 3Jo 1:3-\allowbreak6}
\crossref{Phlm}{1}{8}{2Co 3:12; 10:1,\allowbreak2; 11:21 1Th 2:2,\allowbreak6}
\crossref{Phlm}{1}{9}{Ro 12:1 2Co 5:20; 6:1 Eph 4:1 Heb 13:19 1Pe 2:11}
\crossref{Phlm}{1}{10}{2Sa 9:1-\allowbreak7; 18:5; 19:37,\allowbreak38 Mr 9:17 1Ti 1:2 Tit 1:4}
\crossref{Phlm}{1}{11}{Job 30:1,\allowbreak2 Mt 25:30 Lu 17:10 Ro 3:12 1Pe 2:10}
\crossref{Phlm}{1}{12}{Mt 6:14,\allowbreak15; 18:21-\allowbreak35 Mr 11:25 Eph 4:32}
\crossref{Phlm}{1}{13}{1Co 16:17 Php 2:30}
\crossref{Phlm}{1}{14}{1:8,\allowbreak9 2Co 1:24 1Pe 5:3}
\crossref{Phlm}{1}{15}{Ge 45:5-\allowbreak8; 50:20 Ps 76:10 Isa 20:6 Ac 4:28}
\crossref{Phlm}{1}{16}{Mt 23:8 Ac 9:17 Ga 4:28,\allowbreak29 1Ti 6:2 Heb 3:1 1Pe 1:22,\allowbreak23 1Jo 5:1}
\crossref{Phlm}{1}{17}{Ac 16:15 2Co 8:23 Eph 3:6 Php 1:7 1Ti 6:2 Heb 3:1,\allowbreak14 Jas 2:5}
\crossref{Phlm}{1}{18}{Isa 53:4-\allowbreak7}
\crossref{Phlm}{1}{19}{1Co 16:21,\allowbreak22 Ga 5:2; 6:11}
\crossref{Phlm}{1}{20}{2Co 2:2; 7:4-\allowbreak7,\allowbreak13 Php 2:2; 4:1 1Th 2:19,\allowbreak20; 3:7-\allowbreak9 Heb 13:17 3Jo 1:4}
\crossref{Phlm}{1}{21}{2Co 2:3; 7:16; 8:22 Ga 5:10 2Th 3:4}
\crossref{Phlm}{1}{22}{Ac 28:23}
\crossref{Phlm}{1}{23}{Col 1:7; 4:12}
\crossref{Phlm}{1}{24}{Ac 12:12,\allowbreak25; 13:13; 15:37-\allowbreak39 Col 4:10 2Ti 4:11}
\crossref{Phlm}{1}{25}{Ro 16:20,\allowbreak24}

% Heb
\crossref{Heb}{1}{1}{Ge 3:15; 6:3,\allowbreak13 etc.}
\crossref{Heb}{1}{2}{Ge 49:1 Nu 24:14 De 4:30; 18:15; 31:29 Isa 2:2 Jer 30:24; 48:47}
\crossref{Heb}{1}{3}{Joh 1:14; 14:9,\allowbreak10 2Co 4:6}
\crossref{Heb}{1}{4}{1:9; 2:9 Eph 1:21 Col 1:18; 2:10 2Th 1:7 1Pe 3:22 Re 5:11,\allowbreak12}
\crossref{Heb}{1}{5}{Heb 5:5 Ps 2:7 Ac 13:33}
\crossref{Heb}{1}{6}{De 32:43}
\crossref{Heb}{1}{7}{1:14 2Ki 2:11; 6:17 Ps 104:4 Isa 6:2}
\crossref{Heb}{1}{8}{Ps 45:6,\allowbreak7}
\crossref{Heb}{1}{9}{Heb 7:26 Ps 11:5; 33:5; 37:28; 40:8; 45:7 Isa 61:8}
\crossref{Heb}{1}{10}{Ps 102:25-\allowbreak27}
\crossref{Heb}{1}{11}{Heb 12:27 Isa 34:4; 65:17 Mt 24:35 Mr 13:31 Lu 21:33 2Pe 3:7-\allowbreak10}
\crossref{Heb}{1}{12}{Heb 13:8 Ex 3:14 Joh 8:58 Jas 1:17}
\crossref{Heb}{1}{13}{1:5}
\crossref{Heb}{1}{14}{Heb 8:6; 10:11 Ps 103:20,\allowbreak21 Da 3:28; 7:10 Mt 18:10 Lu 1:19,\allowbreak23; 2:9,\allowbreak13}
\crossref{Heb}{2}{1}{2:2-\allowbreak4; 1:1,\allowbreak2; 12:25,\allowbreak26}
\crossref{Heb}{2}{2}{De 32:2 Ps 68:17 Ac 7:53 Ga 3:19}
\crossref{Heb}{2}{3}{Heb 4:1,\allowbreak11; 10:28,\allowbreak29; 12:25 Isa 20:6 Eze 17:15,\allowbreak18 Mt 23:33 Ro 2:3}
\crossref{Heb}{2}{4}{Mr 16:20 Joh 15:26 Ac 2:32,\allowbreak33; 3:15,\allowbreak16; 4:10; 14:3; 19:11,\allowbreak12}
\crossref{Heb}{2}{5}{Heb 6:5 2Pe 3:13 Re 11:15}
\crossref{Heb}{2}{6}{Heb 4:4; 5:6 1Pe 1:11}
\crossref{Heb}{2}{7}{2:9}
\crossref{Heb}{2}{8}{2:5; 1:13 Ps 2:6 Da 7:14 Mt 28:18 Joh 3:35; 13:3 1Co 15:27}
\crossref{Heb}{2}{9}{Heb 8:3; 10:5 Ge 3:15 Isa 7:14; 11:1; 53:2-\allowbreak10 Ro 8:3 Ga 4:4 Php 2:7-\allowbreak9}
\crossref{Heb}{2}{10}{Heb 7:26 Ge 18:25 Lu 2:14; 24:26,\allowbreak46 Ro 3:25,\allowbreak26 Eph 1:6-\allowbreak8; 2:7; 3:10}
\crossref{Heb}{2}{11}{Heb 10:10,\allowbreak14; 13:12 Joh 17:19}
\crossref{Heb}{2}{12}{Ps 22:22,\allowbreak25}
\crossref{Heb}{2}{13}{2Sa 22:3 Ps 16:1; 18:2; 36:7,\allowbreak8; 91:2 Isa 12:2; 50:7-\allowbreak9 Mt 27:43}
\crossref{Heb}{2}{14}{2:10}
\crossref{Heb}{2}{15}{Job 33:21-\allowbreak28 Ps 33:19; 56:13; 89:48 Lu 1:74,\allowbreak75 2Co 1:10}
\crossref{Heb}{2}{16}{Heb 6:16; 12:10 Ro 2:25 1Pe 1:20}
\crossref{Heb}{2}{17}{2:11,\allowbreak14 Php 2:7,\allowbreak8}
\crossref{Heb}{2}{18}{Heb 4:15,\allowbreak16; 5:2,\allowbreak7-\allowbreak9 Mt 4:1-\allowbreak10; 26:37-\allowbreak39 Lu 22:53}
\crossref{Heb}{3}{1}{Col 1:22; 3:12 1Th 5:27 2Ti 1:9 1Pe 2:9; 3:5 2Pe 1:3-\allowbreak10 Re 18:20}
\crossref{Heb}{3}{2}{Heb 2:17 Joh 6:38-\allowbreak40; 7:18; 8:29; 15:10; 17:4}
\crossref{Heb}{3}{3}{3:6; 1:2-\allowbreak4; 2:9 Col 1:18}
\crossref{Heb}{3}{4}{3:3; 1:2 Es 2:10; 3:9}
\crossref{Heb}{3}{5}{3:2 Nu 12:7 Mt 24:45; 25:21 Lu 12:42; 16:10-\allowbreak12 1Co 4:2 1Ti 1:12}
\crossref{Heb}{3}{6}{Heb 1:2; 4:14 Ps 2:6,\allowbreak7,\allowbreak12 Isa 9:6,\allowbreak7 Joh 3:35,\allowbreak36 Re 2:18}
\crossref{Heb}{3}{7}{Heb 9:8 2Sa 23:2 Mt 22:43 Mr 12:36 Ac 1:16; 28:25 2Pe 1:21}
\crossref{Heb}{3}{8}{3:12,\allowbreak13 Ex 8:15 1Sa 6:6 2Ki 17:14 2Ch 30:8; 36:13 Ne 9:16}
\crossref{Heb}{3}{9}{Ex 19:4; 20:22 De 4:3,\allowbreak9; 11:7; 29:2 Jos 23:3; 24:7 Lu 7:22}
\crossref{Heb}{3}{10}{Ge 6:6 Jud 10:16 Ps 78:40 Isa 63:10 Mr 3:5 Eph 4:30}
\crossref{Heb}{3}{11}{3:18,\allowbreak19; 4:3 Nu 14:20-\allowbreak23,\allowbreak25,\allowbreak27-\allowbreak30,\allowbreak35; 32:10-\allowbreak13 De 1:34,\allowbreak35; 2:14}
\crossref{Heb}{3}{12}{Heb 2:1-\allowbreak3; 12:15 Mt 24:4 Mr 13:9,\allowbreak23,\allowbreak33 Lu 21:8 Ro 11:21 1Co 10:12}
\crossref{Heb}{3}{13}{Heb 10:24,\allowbreak25 Ac 11:23 1Th 2:11; 4:18; 5:11 2Ti 4:2}
\crossref{Heb}{3}{14}{3:1; 6:4; 12:10 Ro 11:17 1Co 1:30; 9:23; 10:17 Eph 3:6 1Ti 6:2}
\crossref{Heb}{3}{15}{3:7,\allowbreak8; 10:38,\allowbreak29}
\crossref{Heb}{3}{16}{3:9,\allowbreak10 Nu 14:2,\allowbreak4; 26:65 Ps 78:17}
\crossref{Heb}{3}{17}{3:10}
\crossref{Heb}{3}{18}{3:11 Nu 14:30 De 1:34,\allowbreak35}
\crossref{Heb}{3}{19}{Mr 16:16 Joh 3:18,\allowbreak36 2Th 2:12 1Jo 5:10 Jude 1:5}
\crossref{Heb}{4}{1}{4:11; 2:1-\allowbreak3; 12:15,\allowbreak25; 13:7 Pr 14:16; 28:14 Jer 32:40 Ro 11:20}
\crossref{Heb}{4}{2}{Ac 3:26; 13:46 Ga 3:8; 4:13 1Pe 1:12}
\crossref{Heb}{4}{3}{Heb 3:14 Isa 28:12 Jer 6:16 Mt 11:28,\allowbreak29 Ro 5:1,\allowbreak2}
\crossref{Heb}{4}{4}{Heb 2:6}
\crossref{Heb}{4}{5}{4:3; 3:11}
\crossref{Heb}{4}{6}{4:9 1Co 7:29}
\crossref{Heb}{4}{7}{Heb 3:7,\allowbreak8 2Sa 23:1,\allowbreak2 Mt 22:43 Mr 12:36 Lu 20:42 Ac 2:29,\allowbreak31; 28:25}
\crossref{Heb}{4}{8}{Ac 7:45}
\crossref{Heb}{4}{9}{4:1,\allowbreak3; 3:11 Isa 11:10; 57:2; 60:19,\allowbreak20 Re 7:14-\allowbreak17; 21:4}
\crossref{Heb}{4}{10}{Heb 1:3; 10:12 Re 14:13}
\crossref{Heb}{4}{11}{4:1; 6:11 Mt 7:13; 11:12,\allowbreak28-\allowbreak30 Lu 13:24; 16:16 Joh 6:27 Php 2:12}
\crossref{Heb}{4}{12}{Heb 13:7 Isa 49:2 Lu 8:11 Ac 4:31 2Co 2:17; 4:2 Re 20:4}
\crossref{Heb}{4}{13}{1Sa 16:7 1Ch 28:9 2Ch 6:30 Ps 7:9; 33:13-\allowbreak15; 44:21; 90:8; 139:11}
\crossref{Heb}{4}{14}{Heb 2:17; 3:1; 3:5,\allowbreak6}
\crossref{Heb}{4}{15}{Heb 5:2 Ex 23:9 Isa 53:4,\allowbreak5 Ho 11:8 Mt 8:16,\allowbreak17; 12:20 Php 2:7,\allowbreak8}
\crossref{Heb}{4}{16}{Heb 10:19-\allowbreak23; 13:6 Ro 8:15-\allowbreak17 Eph 2:18; 3:12}
\crossref{Heb}{5}{1}{Heb 10:11 Ex 28:1 etc.}
\crossref{Heb}{5}{2}{Heb 2:18; 4:15}
\crossref{Heb}{5}{3}{Heb 7:27; 9:7 Ex 29:12-\allowbreak19 Le 4:3-\allowbreak12; 8:14-\allowbreak21; 9:7; 16:6,\allowbreak15}
\crossref{Heb}{5}{4}{Ex 28:1 Le 8:2 Nu 3:3; 16:5,\allowbreak7,\allowbreak10,\allowbreak35,\allowbreak40,\allowbreak46-\allowbreak48; 17:3-\allowbreak11; 18:1-\allowbreak5}
\crossref{Heb}{5}{5}{Joh 7:18; 8:54}
\crossref{Heb}{5}{6}{5:10; 6:20; 7:3,\allowbreak15,\allowbreak17,\allowbreak21 Ps 110:4}
\crossref{Heb}{5}{7}{Heb 2:14 Joh 1:14 Ro 8:3 Ga 4:4 1Ti 3:16 1Jo 4:3 2Jo 1:7}
\crossref{Heb}{5}{8}{Heb 1:5,\allowbreak8; 3:6}
\crossref{Heb}{5}{9}{Heb 2:10; 11:40 Da 9:24 Lu 13:32 Joh 19:30}
\crossref{Heb}{5}{10}{5:5,\allowbreak6; 6:20}
\crossref{Heb}{5}{11}{1Ki 10:1 Joh 6:6; 16:12 2Pe 3:16}
\crossref{Heb}{5}{12}{Mt 17:17 Mr 9:19}
\crossref{Heb}{5}{13}{Ps 119:123 Ro 1:17,\allowbreak18; 10:5,\allowbreak6 2Co 3:9 2Ti 3:16}
\crossref{Heb}{5}{14}{Mt 5:48 1Co 2:6 Eph 4:13 Php 3:15 Jas 3:2}
\crossref{Heb}{6}{1}{Heb 5:12-\allowbreak14}
\crossref{Heb}{6}{2}{Heb 9:10 Mr 7:4,\allowbreak8 Lu 11:38}
\crossref{Heb}{6}{3}{Ac 18:21 Ro 15:32 1Co 4:19; 16:7 Jas 4:15}
\crossref{Heb}{6}{4}{Heb 10:26-\allowbreak29; 12:15-\allowbreak17 Mt 5:13; 12:31,\allowbreak32,\allowbreak45 Lu 11:24-\allowbreak26 Joh 15:6}
\crossref{Heb}{6}{5}{Mt 13:20,\allowbreak21 Mr 4:16,\allowbreak17; 6:20 Lu 8:13 1Pe 2:3 2Pe 2:20}
\crossref{Heb}{6}{6}{6:4 Ps 51:10 Isa 1:28 2Ti 2:25}
\crossref{Heb}{6}{7}{De 28:11,\allowbreak12 Ps 65:9-\allowbreak13; 104:11-\allowbreak13 Isa 55:10-\allowbreak13 Joe 2:21-\allowbreak26}
\crossref{Heb}{6}{8}{Heb 12:17 Ge 3:17,\allowbreak18; 4:11; 5:29 De 29:28 Job 31:40 Ps 107:34}
\crossref{Heb}{6}{9}{6:4-\allowbreak6,\allowbreak10; 10:34,\allowbreak39 Php 1:6,\allowbreak7 1Th 1:3,\allowbreak4}
\crossref{Heb}{6}{10}{Pr 14:31 Mt 10:42; 25:40 Joh 13:20}
\crossref{Heb}{6}{11}{Ro 12:8,\allowbreak11 1Co 15:58 Ga 6:9 Php 1:9-\allowbreak11; 3:15 1Th 4:10 2Th 3:13}
\crossref{Heb}{6}{12}{Heb 5:11}
\crossref{Heb}{6}{13}{6:16-\allowbreak18 Ge 22:15-\allowbreak18 Eze 32:13 Ps 105:9,\allowbreak10 Isa 45:23 Jer 22:5}
\crossref{Heb}{6}{14}{Ge 17:2; 48:4 Ex 32:13 De 1:10 Ne 9:23}
\crossref{Heb}{6}{15}{6:12 Ge 12:2,\allowbreak3; 15:2-\allowbreak6; 17:16,\allowbreak17; 21:2-\allowbreak7 Ex 1:7 Hab 2:2,\allowbreak3}
\crossref{Heb}{6}{16}{6:13 Ge 14:22; 21:23 Mt 23:20-\allowbreak22}
\crossref{Heb}{6}{17}{Ps 36:8 So 5:1 Isa 55:7 Joh 10:10 1Pe 1:3}
\crossref{Heb}{6}{18}{Heb 3:11; 7:21 Ps 110:4 Mt 24:35}
\crossref{Heb}{6}{19}{Ac 27:29,\allowbreak40}
\crossref{Heb}{6}{20}{Heb 2:10 Joh 14:2,\allowbreak3}
\crossref{Heb}{7}{1}{Heb 6:20 Ge 14:18-\allowbreak20}
\crossref{Heb}{7}{2}{Ge 28:22 Le 27:32 Nu 18:21 1Sa 8:15,\allowbreak17}
\crossref{Heb}{7}{3}{Ex 6:18,\allowbreak20-\allowbreak27 1Ch 6:1-\allowbreak3}
\crossref{Heb}{7}{4}{Ac 2:29; 7:8,\allowbreak9}
\crossref{Heb}{7}{5}{Heb 5:4 Ex 28:1 Nu 16:10,\allowbreak11; 17:3-\allowbreak10; 18:7,\allowbreak21-\allowbreak26}
\crossref{Heb}{7}{6}{7:3}
\crossref{Heb}{7}{7}{1Ti 3:16}
\crossref{Heb}{7}{8}{7:23; 9:27}
\crossref{Heb}{7}{9}{7:4 Ge 14:20 Ro 5:12}
\crossref{Heb}{7}{10}{7:5 Ge 35:11; 46:26 1Ki 8:19}
\crossref{Heb}{7}{11}{7:26-\allowbreak28}
\crossref{Heb}{7}{12}{Isa 66:21 Jer 31:31-\allowbreak34 Eze 16:61 Ac 6:13,\allowbreak14}
\crossref{Heb}{7}{13}{Nu 16:40; 17:5 2Ch 26:16-\allowbreak21}
\crossref{Heb}{7}{14}{Lu 1:43 Joh 20:13,\allowbreak28 Eph 1:3 Php 3:8}
\crossref{Heb}{7}{15}{7:3,\allowbreak11,\allowbreak17-\allowbreak21 Ps 110:4}
\crossref{Heb}{7}{16}{Heb 9:9,\allowbreak10; 10:1 Ga 4:3,\allowbreak9 Col 2:14,\allowbreak20}
\crossref{Heb}{7}{17}{7:15,\allowbreak21; 5:6,\allowbreak10; 6:20 Ps 110:4}
\crossref{Heb}{7}{18}{7:11,\allowbreak12; 8:7-\allowbreak13; 10:1-\allowbreak9 Ro 3:31 Ga 3:15,\allowbreak17}
\crossref{Heb}{7}{19}{7:11; 9:9 Ac 13:39 Ro 3:20,\allowbreak21; 8:3 Ga 2:16}
\crossref{Heb}{7}{20}{}
\crossref{Heb}{7}{21}{7:17 Ps 110:4}
\crossref{Heb}{7}{22}{Ge 43:9; 44:32 Pr 6:1; 20:16}
\crossref{Heb}{7}{23}{7:8 1Ch 6:3-\allowbreak14 Ne 12:10,\allowbreak11}
\crossref{Heb}{7}{24}{7:8-\allowbreak25,\allowbreak28; 13:8 Isa 9:6,\allowbreak7 Joh 12:34 Ro 6:9 Re 1:18}
\crossref{Heb}{7}{25}{Heb 2:18; 5:7 Isa 45:22; 63:1 Da 3:15,\allowbreak17,\allowbreak29; 6:20 Joh 5:37-\allowbreak40; 10:29}
\crossref{Heb}{7}{26}{7:11; 8:1; 9:23-\allowbreak26; 10:11-\allowbreak22}
\crossref{Heb}{7}{27}{Heb 10:11 Ex 29:36-\allowbreak42 Nu 28:2-\allowbreak10}
\crossref{Heb}{7}{28}{Heb 5:1,\allowbreak2 Ex 32:21,\allowbreak22 Le 4:3}
\crossref{Heb}{8}{1}{Heb 7:26-\allowbreak28}
\crossref{Heb}{8}{2}{Heb 9:8-\allowbreak12; 10:21 Ex 28:1,\allowbreak35 Lu 24:44 Ro 15:8}
\crossref{Heb}{8}{3}{Heb 5:1; 7:27}
\crossref{Heb}{8}{4}{Heb 7:11-\allowbreak15 Nu 16:40; 17:12,\allowbreak13; 18:5 2Ch 26:18,\allowbreak19}
\crossref{Heb}{8}{5}{Heb 9:9,\allowbreak23,\allowbreak24; 10:1 Col 2:17}
\crossref{Heb}{8}{6}{8:7-\allowbreak13 2Co 3:6-\allowbreak11}
\crossref{Heb}{8}{7}{8:6; 7:11,\allowbreak18 Ga 3:21}
\crossref{Heb}{8}{8}{Jer 31:31-\allowbreak34}
\crossref{Heb}{8}{9}{Heb 9:18-\allowbreak20 Ex 24:3-\allowbreak11; 34:10,\allowbreak27,\allowbreak28 De 5:2,\allowbreak3; 29:1,\allowbreak12 Ga 3:15-\allowbreak19}
\crossref{Heb}{8}{10}{Heb 10:16,\allowbreak17}
\crossref{Heb}{8}{11}{Isa 2:3; 54:13 Jer 31:34 Joh 6:45 1Jo 2:27}
\crossref{Heb}{8}{12}{Heb 10:16,\allowbreak17 Ps 25:7; 65:3 Isa 43:25; 44:22 Jer 33:8; 50:20 Mic 7:19}
\crossref{Heb}{8}{13}{8:8}
\crossref{Heb}{9}{1}{Heb 8:7,\allowbreak13}
\crossref{Heb}{9}{2}{Ex 26:1-\allowbreak30; 29:1,\allowbreak35; 36:8-\allowbreak38; 39:32-\allowbreak34; 40:2,\allowbreak18-\allowbreak20}
\crossref{Heb}{9}{3}{Heb 6:19; 10:20 Ex 26:31-\allowbreak33; 36:35-\allowbreak38; 40:3,\allowbreak21 2Ch 3:14 Isa 25:7}
\crossref{Heb}{9}{4}{Le 16:12 1Ki 7:50 Re 8:3}
\crossref{Heb}{9}{5}{Ex 25:17-\allowbreak22; 37:6-\allowbreak9 Le 16:2 Nu 7:89 1Sa 4:4 1Ki 8:6,\allowbreak7 2Ki 19:15}
\crossref{Heb}{9}{6}{Ex 27:21; 30:7,\allowbreak8 Nu 28:3 2Ch 26:16-\allowbreak19 Da 8:11 Lu 1:8-\allowbreak11}
\crossref{Heb}{9}{7}{9:24,\allowbreak25 Ex 30:10 Le 16:2-\allowbreak20,\allowbreak34}
\crossref{Heb}{9}{8}{Heb 3:7; 10:15 Isa 63:11 Ac 7:51,\allowbreak52; 28:25 Ga 3:8 2Pe 1:21}
\crossref{Heb}{9}{9}{9:24; 11:19 Ro 5:14 1Pe 3:21}
\crossref{Heb}{9}{10}{Heb 13:9 Le 11:2 etc.}
\crossref{Heb}{9}{11}{Ge 49:10 Ps 40:7 Isa 59:20 Mal 3:1 Mt 2:6; 11:3 Joh 4:25}
\crossref{Heb}{9}{12}{9:13; 10:4 Le 8:2; 9:15; 16:5-\allowbreak10}
\crossref{Heb}{9}{13}{Le 16:14,\allowbreak16}
\crossref{Heb}{9}{14}{De 31:27 2Sa 4:11 Job 15:16 Mt 7:11 Lu 12:24,\allowbreak28 Ro 11:12,\allowbreak24}
\crossref{Heb}{9}{15}{Heb 7:22; 8:6; 12:24 1Ti 2:5}
\crossref{Heb}{9}{16}{9:16}
\crossref{Heb}{9}{17}{Ge 48:21 Joh 14:27}
\crossref{Heb}{9}{18}{Heb 8:7-\allowbreak9 Ex 12:22; 24:3-\allowbreak8}
\crossref{Heb}{9}{19}{9:12; 10:4 Ex 24:5,\allowbreak6,\allowbreak8 etc.}
\crossref{Heb}{9}{20}{Heb 13:20 Zec 9:11 Mt 26:28}
\crossref{Heb}{9}{21}{Ex 29:12,\allowbreak20,\allowbreak36 Le 8:15,\allowbreak19; 9:8,\allowbreak9,\allowbreak18; 16:14-\allowbreak19 2Ch 29:19-\allowbreak22}
\crossref{Heb}{9}{22}{Le 14:6,\allowbreak14,\allowbreak25,\allowbreak51,\allowbreak52}
\crossref{Heb}{9}{23}{9:9,\allowbreak10,\allowbreak24; 8:5; 10:1 Col 2:17}
\crossref{Heb}{9}{24}{9:11 Mr 14:58 Joh 2:19-\allowbreak21}
\crossref{Heb}{9}{25}{9:7,\allowbreak14,\allowbreak26; 10:10}
\crossref{Heb}{9}{26}{Mt 25:34 Joh 17:24 1Pe 1:20 Re 13:8; 17:8}
\crossref{Heb}{9}{27}{Ge 3:19 2Sa 14:14 Job 14:5; 30:23 Ps 89:48 Ec 3:20; 9:5,\allowbreak10; 12:7}
\crossref{Heb}{9}{28}{9:25 Ro 6:10 1Pe 3:18 1Jo 3:5}
\crossref{Heb}{10}{1}{Heb 8:5; 9:9,\allowbreak11,\allowbreak23 Col 2:17}
\crossref{Heb}{10}{2}{10:17; 9:13,\allowbreak14 Ps 103:12 Isa 43:25; 44:22 Mic 7:19}
\crossref{Heb}{10}{3}{Heb 9:7 Ex 30:10 Le 16:6-\allowbreak11,\allowbreak21,\allowbreak22,\allowbreak29,\allowbreak30,\allowbreak34; 23:27,\allowbreak28 Nu 29:7-\allowbreak11}
\crossref{Heb}{10}{4}{10:8; 9:9,\allowbreak13 Ps 50:8-\allowbreak12; 51:16 Isa 1:11-\allowbreak15; 66:3 Jer 6:20; 7:21,\allowbreak22}
\crossref{Heb}{10}{5}{10:7; 1:6 Mt 11:3 Lu 7:19}
\crossref{Heb}{10}{6}{10:4 Le 1:1-\allowbreak6:7}
\crossref{Heb}{10}{7}{10:9,\allowbreak10 Pr 8:31 Joh 4:34; 5:30; 6:38}
\crossref{Heb}{10}{8}{}
\crossref{Heb}{10}{9}{Heb 9:11-\allowbreak14}
\crossref{Heb}{10}{10}{Heb 2:11; 13:12 Zec 13:1 Joh 17:19; 19:34 1Co 1:30; 6:11 1Jo 5:6}
\crossref{Heb}{10}{11}{Heb 7:27 Ex 29:38,\allowbreak39 Nu 28:3,\allowbreak24; 29:6 Eze 45:4 Da 8:11; 9:21,\allowbreak27}
\crossref{Heb}{10}{12}{Heb 1:3; 8:1; 9:12 Ac 2:33,\allowbreak34 Ro 8:34 Col 3:1}
\crossref{Heb}{10}{13}{Heb 1:13 Ps 110:1 Da 2:44 Mt 22:44 Mr 12:36 Lu 20:43 Ac 2:35}
\crossref{Heb}{10}{14}{10:1; 7:19,\allowbreak25; 9:10,\allowbreak14}
\crossref{Heb}{10}{15}{Heb 2:3,\allowbreak4; 3:7; 9:8 2Sa 23:2 Ne 9:30 Joh 15:26 Ac 28:25 1Pe 1:11,\allowbreak12}
\crossref{Heb}{10}{16}{Heb 8:8-\allowbreak12 Jer 31:33,\allowbreak34 Ro 11:27}
\crossref{Heb}{10}{17}{}
\crossref{Heb}{10}{18}{10:2,\allowbreak14}
\crossref{Heb}{10}{19}{Heb 4:16; 12:28 Ro 8:15 Ga 4:6,\allowbreak7 Eph 3:12 2Ti 1:7 1Jo 3:19-\allowbreak21; 4:17}
\crossref{Heb}{10}{20}{Joh 10:7,\allowbreak9; 14:6}
\crossref{Heb}{10}{21}{Heb 2:17; 3:1; 4:14-\allowbreak16; 6:20; 7:26; 8:1}
\crossref{Heb}{10}{22}{Heb 4:16; 7:19 Ps 73:28 Isa 29:13 Jer 30:21 Jas 4:8}
\crossref{Heb}{10}{23}{Heb 3:6,\allowbreak14; 4:14 Re 3:11}
\crossref{Heb}{10}{24}{Heb 13:3 Ps 41:1 Pr 29:7 Ac 11:29 Ro 12:15; 15:1,\allowbreak2 1Co 8:12,\allowbreak13; 9:22}
\crossref{Heb}{10}{25}{Mt 18:20 Joh 20:19-\allowbreak29 Ac 1:13,\allowbreak14; 2:1,\allowbreak42; 16:16; 20:7 1Co 5:4}
\crossref{Heb}{10}{26}{Heb 6:4-\allowbreak6 Le 4:2,\allowbreak13 Nu 15:28-\allowbreak31 De 17:12 Ps 19:12,\allowbreak13 Da 5:22,\allowbreak23}
\crossref{Heb}{10}{27}{Heb 2:3; 12:25 1Sa 28:19,\allowbreak20 Isa 33:14 Da 5:6 Ho 10:8 Mt 8:29}
\crossref{Heb}{10}{28}{Heb 2:2 Nu 15:30,\allowbreak31,\allowbreak36 De 13:6-\allowbreak10; 17:2-\allowbreak13 2Sa 12:9,\allowbreak13}
\crossref{Heb}{10}{29}{Heb 2:3; 12:25}
\crossref{Heb}{10}{30}{De 32:35 Ps 94:1 Isa 59:17; 61:2; 63:4 Na 1:2 Ro 12:19; 13:4}
\crossref{Heb}{10}{31}{10:27 Isa 33:14 Lu 21:11}
\crossref{Heb}{10}{32}{Ga 3:3,\allowbreak4 Php 3:16 2Jo 1:8 Re 2:5; 3:3}
\crossref{Heb}{10}{33}{Heb 11:36 Ps 71:7 Na 3:6 Zec 3:8 1Co 4:9}
\crossref{Heb}{10}{34}{Ac 21:33; 28:20 Eph 3:1; 4:1; 6:20 Php 1:7 2Ti 1:16; 2:9}
\crossref{Heb}{10}{35}{Heb 3:6,\allowbreak14; 4:14}
\crossref{Heb}{10}{36}{Heb 6:15; 12:1 Ps 37:7; 40:1 Mt 10:22; 24:13 Lu 8:15; 21:19 Ro 2:7}
\crossref{Heb}{10}{37}{Isa 26:20; 60:22 Hab 2:3,\allowbreak4 Lu 18:8 Jas 5:7-\allowbreak9 2Pe 3:8 Re 22:20}
\crossref{Heb}{10}{38}{Hab 2:4 Ro 1:17 Ga 3:11}
\crossref{Heb}{10}{39}{Heb 6:6-\allowbreak9 1Sa 15:11 Ps 44:18 Pr 1:32; 14:14 Lu 11:26 1Jo 5:16}
\crossref{Heb}{11}{1}{11:13; 10:22,\allowbreak39 Ac 20:21 1Co 13:13 Ga 5:6 Tit 1:1 1Pe 1:7 2Pe 1:1}
\crossref{Heb}{11}{2}{11:4-\allowbreak39}
\crossref{Heb}{11}{3}{Heb 1:2 Ge 1:1 etc.}
\crossref{Heb}{11}{4}{Ge 4:3-\allowbreak5,\allowbreak15,\allowbreak25 1Jo 3:11,\allowbreak12}
\crossref{Heb}{11}{5}{Ge 5:22-\allowbreak24 Lu 3:37 Jude 1:14}
\crossref{Heb}{11}{6}{Heb 3:12,\allowbreak18,\allowbreak19; 4:2,\allowbreak6 Nu 14:11; 20:12 Ps 78:22,\allowbreak32; 106:21,\allowbreak22,\allowbreak24}
\crossref{Heb}{11}{7}{Ge 6:13,\allowbreak22; 7:1,\allowbreak5 Mt 24:38 Lu 17:26}
\crossref{Heb}{11}{8}{Ge 11:31; 12:1-\allowbreak4 Jos 24:3 Ne 9:7,\allowbreak8 Isa 41:2; 51:2 Ac 7:2-\allowbreak4}
\crossref{Heb}{11}{9}{Ge 17:8; 23:4; 26:3; 35:27 Ac 7:5,\allowbreak6}
\crossref{Heb}{11}{10}{Heb 12:22,\allowbreak28; 13:14 Joh 14:2 Php 3:20}
\crossref{Heb}{11}{11}{Ge 17:17-\allowbreak19; 18:11-\allowbreak14; 21:1,\allowbreak2 Lu 1:36 1Pe 3:5,\allowbreak6}
\crossref{Heb}{11}{12}{Ro 4:19}
\crossref{Heb}{11}{13}{Ge 25:8; 27:2-\allowbreak4; 48:21; 49:18,\allowbreak28,\allowbreak33; 50:24}
\crossref{Heb}{11}{14}{11:16; 13:14 Ro 8:23-\allowbreak25 2Co 4:18; 5:1-\allowbreak7 Php 1:23}
\crossref{Heb}{11}{15}{Ge 11:31; 12:10; 24:6-\allowbreak8; 31:18; 32:9-\allowbreak11}
\crossref{Heb}{11}{16}{11:14; 12:22}
\crossref{Heb}{11}{17}{Ge 22:1-\allowbreak12 Jas 2:21-\allowbreak24}
\crossref{Heb}{11}{18}{Ge 17:19; 21:12 Ro 9:7}
\crossref{Heb}{11}{19}{Ge 22:5}
\crossref{Heb}{11}{20}{Ge 27:27-\allowbreak40; 28:2,\allowbreak3}
\crossref{Heb}{11}{21}{Ge 48:5-\allowbreak22}
\crossref{Heb}{11}{22}{Ge 50:24,\allowbreak25 Ex 13:19 Jos 24:32 Ac 7:16}
\crossref{Heb}{11}{23}{Ex 2:2 etc.}
\crossref{Heb}{11}{24}{Ex 2:10 Ac 7:21-\allowbreak24}
\crossref{Heb}{11}{25}{Heb 10:32 Job 36:21 Ps 84:10 Mt 5:10-\allowbreak12; 13:21 Ac 7:24,\allowbreak25; 20:23,\allowbreak24}
\crossref{Heb}{11}{26}{Heb 10:33; 13:13 Ps 69:7,\allowbreak20; 89:50,\allowbreak51 Isa 51:7 Ac 5:41 2Co 12:10}
\crossref{Heb}{11}{27}{Ex 10:28,\allowbreak29; 11:8; 12:11,\allowbreak37 etc.}
\crossref{Heb}{11}{28}{Ex 12:3-\allowbreak14,\allowbreak21-\allowbreak30}
\crossref{Heb}{11}{29}{Ex 14:13-\allowbreak31; 15:1-\allowbreak21 Jos 2:10 Ne 9:11 Ps 66:6; 78:13; 106:9-\allowbreak11}
\crossref{Heb}{11}{30}{Jos 6:3-\allowbreak20 2Co 10:4,\allowbreak5}
\crossref{Heb}{11}{31}{Jos 2:1-\allowbreak22; 6:22-\allowbreak25 Mt 1:1,\allowbreak5 Jas 2:25}
\crossref{Heb}{11}{32}{Ro 3:5; 4:1; 6:1; 7:7}
\crossref{Heb}{11}{33}{Jos 6:1-\allowbreak13:33 2Sa 5:4-\allowbreak25; 8:1-\allowbreak14 Ps 18:32-\allowbreak34; 44:2-\allowbreak6; 144:1,\allowbreak2,\allowbreak10}
\crossref{Heb}{11}{34}{Ps 66:12 Isa 43:2 Da 3:19-\allowbreak28 1Pe 4:12}
\crossref{Heb}{11}{35}{1Ki 17:22-\allowbreak24 2Ki 4:27-\allowbreak37 Lu 7:12-\allowbreak16 Joh 11:40-\allowbreak45 Ac 9:41}
\crossref{Heb}{11}{36}{Jud 16:25 2Ki 2:23 2Ch 30:10; 36:16 Jer 20:7 Mt 20:19 Mr 10:34}
\crossref{Heb}{11}{37}{1Ki 21:10,\allowbreak13-\allowbreak15 2Ch 24:21 Mt 21:35; 23:37 Lu 13:34 Joh 10:31-\allowbreak33}
\crossref{Heb}{11}{38}{1Ki 14:12,\allowbreak13 2Ki 23:25-\allowbreak29 Isa 57:1}
\crossref{Heb}{11}{39}{11:2,\allowbreak13 Lu 10:23,\allowbreak24 1Pe 1:12}
\crossref{Heb}{11}{40}{Heb 7:19,\allowbreak22; 8:6; 9:23; 12:24}
\crossref{Heb}{12}{1}{Heb 11:2-\allowbreak38}
\crossref{Heb}{12}{2}{12:3; 9:28 Isa 8:17; 31:1; 45:22 Mic 7:7 Zec 12:10 Joh 1:29; 6:40; 8:56}
\crossref{Heb}{12}{3}{12:2; 3:1 1Sa 12:24 2Ti 2:7,\allowbreak8}
\crossref{Heb}{12}{4}{12:2; 10:32-\allowbreak34 Mt 24:9 1Co 10:13 2Ti 4:6,\allowbreak7 Re 2:13; 6:9-\allowbreak11; 12:11; 17:6}
\crossref{Heb}{12}{5}{De 4:9,\allowbreak10 Ps 119:16,\allowbreak83,\allowbreak109 Pr 3:1; 4:5 Mt 16:9,\allowbreak10 Lu 24:6,\allowbreak8}
\crossref{Heb}{12}{6}{De 8:5 Ps 32:1-\allowbreak5; 73:14,\allowbreak15; 89:30-\allowbreak34; 119:71,\allowbreak75 Pr 3:12; 13:24}
\crossref{Heb}{12}{7}{Job 34:31,\allowbreak32 Pr 19:18; 22:15; 23:13,\allowbreak14; 29:15,\allowbreak17 Ac 14:22}
\crossref{Heb}{12}{8}{12:6 Ps 73:1,\allowbreak14,\allowbreak15 1Pe 5:9,\allowbreak10}
\crossref{Heb}{12}{9}{Joh 3:6 Ac 2:30 Ro 1:3; 9:3,\allowbreak5}
\crossref{Heb}{12}{10}{Le 11:44,\allowbreak45; 19:2 Ps 17:15 Eze 36:25-\allowbreak27 Eph 4:24; 5:26,\allowbreak27}
\crossref{Heb}{12}{11}{Ps 89:32; 118:18 Pr 15:10; 19:18}
\crossref{Heb}{12}{12}{12:3,\allowbreak5 Job 4:3,\allowbreak4 Isa 35:3 Eze 7:17; 21:7 Da 5:6 Na 2:10 1Th 5:14}
\crossref{Heb}{12}{13}{Pr 4:26,\allowbreak27 Isa 35:3,\allowbreak8-\allowbreak10; 40:3,\allowbreak4; 42:16; 58:12 Jer 18:15 Lu 3:5}
\crossref{Heb}{12}{14}{Ge 13:7-\allowbreak9 Ps 34:14; 38:20; 120:6; 133:1 Pr 15:1; 16:7; 17:14}
\crossref{Heb}{12}{15}{Heb 2:1,\allowbreak2; 3:12; 4:1,\allowbreak11; 6:11; 10:23-\allowbreak35 De 4:9 Pr 4:23 1Co 9:24-\allowbreak27}
\crossref{Heb}{12}{16}{Heb 13:4 Mr 7:21 Ac 15:20,\allowbreak29 1Co 5:1-\allowbreak6,\allowbreak9-\allowbreak11; 6:15-\allowbreak20; 10:8 2Co 12:21}
\crossref{Heb}{12}{17}{Ge 27:31-\allowbreak41}
\crossref{Heb}{12}{18}{Ex 19:12-\allowbreak19; 20:18; 24:17 De 4:11; 5:22-\allowbreak26 Ro 6:14; 8:15 2Ti 1:7}
\crossref{Heb}{12}{19}{Ex 19:16-\allowbreak19 1Co 15:52 1Th 4:16}
\crossref{Heb}{12}{20}{De 33:2 Ro 3:19,\allowbreak20 Ga 2:19; 3:10}
\crossref{Heb}{12}{21}{Ex 19:16,\allowbreak19 Ps 119:120 Isa 6:3-\allowbreak5 Da 10:8,\allowbreak17 Re 1:17}
\crossref{Heb}{12}{22}{Ps 2:6; 48:2; 132:13,\allowbreak14 Isa 12:6; 14:32; 28:16; 51:11,\allowbreak16; 59:20}
\crossref{Heb}{12}{23}{Ps 89:7; 111:1 Ac 20:28 Eph 1:22; 5:24-\allowbreak27 Col 1:24 1Ti 3:5}
\crossref{Heb}{12}{24}{Heb 7:22; 8:6,\allowbreak8 1Ti 2:5}
\crossref{Heb}{12}{25}{Heb 8:5 Ex 16:29 1Ki 12:16 Isa 48:6; 64:9 Mt 8:4 1Th 5:15 1Pe 1:22}
\crossref{Heb}{12}{26}{Ex 19:18 Ps 114:6,\allowbreak7 Hab 3:10}
\crossref{Heb}{12}{27}{Ps 102:26,\allowbreak27 Eze 21:27 Mt 24:35 2Pe 3:10,\allowbreak11 Re 11:15; 21:1}
\crossref{Heb}{12}{28}{Isa 9:7 Da 2:44; 7:14,\allowbreak27 Mt 25:34 Lu 1:33; 17:20,\allowbreak21 1Pe 1:4,\allowbreak5}
\crossref{Heb}{12}{29}{Heb 10:27 Ex 24:17 Nu 11:1; 16:35 De 4:24; 9:3 Ps 50:3; 97:3}
\crossref{Heb}{13}{1}{Heb 6:10,\allowbreak11; 10:24 Joh 13:34,\allowbreak35; 15:17 Ac 2:1,\allowbreak44-\allowbreak46; 4:32 Ro 12:9,\allowbreak10}
\crossref{Heb}{13}{2}{Le 19:34 De 10:18,\allowbreak19 1Ki 17:10-\allowbreak16 2Ki 4:8 Job 31:19,\allowbreak32}
\crossref{Heb}{13}{3}{Heb 10:34 Ge 40:14,\allowbreak15,\allowbreak23 Jer 38:7-\allowbreak13 Mt 25:36,\allowbreak43 Ac 16:29-\allowbreak34}
\crossref{Heb}{13}{4}{Ge 1:27,\allowbreak28; 2:21,\allowbreak24 Le 21:13-\allowbreak15 2Ki 22:14 Pr 5:15-\allowbreak23 Isa 8:3}
\crossref{Heb}{13}{5}{Ex 20:17 Jos 7:21 Ps 10:3; 119:36 Jer 6:13 Eze 33:31 Mr 7:22}
\crossref{Heb}{13}{6}{Heb 4:16; 10:19 Eph 3:12}
\crossref{Heb}{13}{7}{13:17,\allowbreak24 Mt 24:45 Lu 12:42 Ac 14:23 1Th 5:12,\allowbreak13 1Ti 3:5}
\crossref{Heb}{13}{8}{Heb 1:12 Ps 90:2,\allowbreak4; 102:27,\allowbreak28; 103:17 Isa 41:4; 44:6 Mal 3:6}
\crossref{Heb}{13}{9}{Mt 24:4,\allowbreak24 Ac 20:30 Ro 16:17,\allowbreak18 2Co 11:11-\allowbreak15 Ga 1:6-\allowbreak9 Eph 4:14}
\crossref{Heb}{13}{10}{1Co 5:7,\allowbreak8; 9:13; 10:17,\allowbreak20}
\crossref{Heb}{13}{11}{Ex 29:14 Le 4:5-\allowbreak7,\allowbreak11,\allowbreak12,\allowbreak16-\allowbreak21; 6:30; 9:9,\allowbreak11; 16:14-\allowbreak19,\allowbreak27 Nu 19:3}
\crossref{Heb}{13}{12}{Heb 2:11; 9:13,\allowbreak14,\allowbreak18,\allowbreak19; 10:29 Joh 17:19; 19:34 1Co 6:11 Eph 5:26}
\crossref{Heb}{13}{13}{Heb 11:26; 12:3 Mt 5:11; 10:24,\allowbreak25; 16:24; 27:32,\allowbreak39-\allowbreak44 Lu 6:22 Ac 5:41}
\crossref{Heb}{13}{14}{Heb 4:9; 11:9,\allowbreak10,\allowbreak12-\allowbreak16; 12:22 Mic 2:10 1Co 7:29 2Co 4:17,\allowbreak18; 5:1-\allowbreak8}
\crossref{Heb}{13}{15}{Heb 7:25 Joh 10:9; 14:6 Eph 2:18 Col 3:17 1Pe 2:5}
\crossref{Heb}{13}{16}{13:1,\allowbreak2 Ps 37:3 Mt 25:35-\allowbreak40 Lu 6:35,\allowbreak36 Ac 9:36; 10:38 Ga 6:10}
\crossref{Heb}{13}{17}{13:7 1Sa 8:19; 15:19,\allowbreak20 Pr 5:13 Php 2:12,\allowbreak29 1Th 5:12,\allowbreak13}
\crossref{Heb}{13}{18}{Ro 15:30 Eph 6:19,\allowbreak20 Col 4:3 1Th 5:25 2Th 3:1}
\crossref{Heb}{13}{19}{Ro 1:10-\allowbreak12; 15:31,\allowbreak32 Phm 1:22}
\crossref{Heb}{13}{20}{Ro 15:33; 16:20 1Co 14:33 2Co 13:11 Php 4:9 1Th 5:23 2Th 3:16}
\crossref{Heb}{13}{21}{Heb 12:23 De 32:4 Ps 138:8 Joh 17:23 Eph 3:16-\allowbreak19 Col 1:9-\allowbreak12; 4:12}
\crossref{Heb}{13}{22}{13:1-\allowbreak3,\allowbreak12-\allowbreak16; 2:1; 3:1,\allowbreak12,\allowbreak13; 4:1,\allowbreak11; 6:11,\allowbreak12; 10:19-\allowbreak39; 12:1,\allowbreak2,\allowbreak12-\allowbreak16,\allowbreak25-\allowbreak28}
\crossref{Heb}{13}{23}{Ac 16:1-\allowbreak3 1Th 3:2 Phm 1:1}
\crossref{Heb}{13}{24}{Ro 16:1-\allowbreak16}
\crossref{Heb}{13}{25}{Ro 1:7; 16:20,\allowbreak24 Eph 6:24 2Ti 4:22 Tit 3:15 Re 22:21}

% James
\crossref{James}{1}{1}{Mt 10:3; 13:55 Mr 3:18 Lu 6:15 Ac 1:13; 12:17; 15:13; 21:18}
\crossref{James}{1}{2}{1:12 Mt 5:10-\allowbreak12 Lu 6:22,\allowbreak23 Ac 5:41 Ro 8:17,\allowbreak18,\allowbreak35-\allowbreak37 2Co 12:9}
\crossref{James}{1}{3}{Ro 5:3,\allowbreak4; 8:28 2Co 4:17}
\crossref{James}{1}{4}{Jas 5:7-\allowbreak11 Job 17:9 Ps 37:7; 40:1 Hab 2:3 Mt 10:22 Lu 8:15; 21:19}
\crossref{James}{1}{5}{Ex 31:3,\allowbreak6; 36:1-\allowbreak4 1Ki 3:7-\allowbreak9,\allowbreak11,\allowbreak12 Job 28:12-\allowbreak28 Pr 3:5-\allowbreak7; 9:4-\allowbreak6}
\crossref{James}{1}{6}{Mt 21:22 Mr 11:22-\allowbreak24 1Ti 2:8 Heb 11:6}
\crossref{James}{1}{7}{Jas 4:3 Pr 15:8; 21:27 Isa 1:15; 58:3,\allowbreak4}
\crossref{James}{1}{8}{Jas 4:8 1Ki 18:21 2Ki 17:33,\allowbreak41 Isa 29:13 Ho 7:8-\allowbreak11; 10:2 Mt 6:22,\allowbreak24}
\crossref{James}{1}{9}{Jas 2:5,\allowbreak6 De 15:7,\allowbreak9,\allowbreak11 Ps 62:9 Pr 17:5; 19:1 Lu 1:52}
\crossref{James}{1}{10}{Isa 57:15; 66:2 Mt 5:3 Php 3:8 1Ti 6:17}
\crossref{James}{1}{11}{Isa 49:10 Jon 4:7,\allowbreak8 Mt 13:6 Mr 4:6}
\crossref{James}{1}{12}{1:2-\allowbreak4; 5:11 Job 5:17 Ps 94:12; 119:67,\allowbreak71,\allowbreak75 Pr 3:11,\allowbreak12 Heb 6:15}
\crossref{James}{1}{13}{1:2,\allowbreak12 Ge 3:12 Isa 63:17 Hab 2:12,\allowbreak13 Ro 9:19,\allowbreak20}
\crossref{James}{1}{14}{Jas 4:1,\allowbreak2 Ge 6:5; 8:21 Jos 7:21-\allowbreak24 2Sa 11:2,\allowbreak3 1Ki 21:2-\allowbreak4 Job 31:9}
\crossref{James}{1}{15}{Ge 3:6; 4:5-\allowbreak8 Job 15:35 Ps 7:14 Isa 59:4 Mic 2:1-\allowbreak3}
\crossref{James}{1}{16}{Mt 22:29 Mr 12:24,\allowbreak27 Ga 6:7 Col 2:4,\allowbreak8 2Ti 2:18}
\crossref{James}{1}{17}{1:5; 3:15,\allowbreak17 Ge 41:16,\allowbreak38,\allowbreak39 Ex 4:11,\allowbreak12; 31:3-\allowbreak6; 36:1,\allowbreak2 Nu 11:17,\allowbreak25}
\crossref{James}{1}{18}{Joh 1:13; 3:3-\allowbreak5 Ro 4:17; 8:29-\allowbreak31; 9:15-\allowbreak18 Eph 2:4,\allowbreak5 Col 1:20,\allowbreak21}
\crossref{James}{1}{19}{Ne 8:2,\allowbreak3,\allowbreak12-\allowbreak14,\allowbreak18; 9:3 Pr 8:32-\allowbreak35 Ec 5:1 Mr 2:2; 12:37 Lu 15:1}
\crossref{James}{1}{20}{Jas 3:17,\allowbreak18 Nu 20:11,\allowbreak12 2Ti 2:24,\allowbreak25}
\crossref{James}{1}{21}{Isa 2:20; 30:22 Eze 18:31 Ro 13:12,\allowbreak13 Eph 4:22 Col 3:5-\allowbreak8}
\crossref{James}{1}{22}{Jas 4:17 Mt 7:21-\allowbreak25; 12:50; 28:20 Lu 6:46-\allowbreak48; 11:28; 12:47,\allowbreak48}
\crossref{James}{1}{23}{Jas 2:14-\allowbreak26 Jer 44:16 Eze 33:31,\allowbreak32 Mt 7:26,\allowbreak27 Lu 6:47-\allowbreak49 etc.}
\crossref{James}{1}{24}{Jud 8:18 Mt 8:27 Lu 1:66; 7:39 1Th 1:5 2Pe 3:11}
\crossref{James}{1}{25}{Pr 14:15 Isa 8:20 2Co 13:5 Heb 12:15}
\crossref{James}{1}{26}{Pr 14:12; 16:25 Lu 8:18 1Co 3:18 Ga 2:6,\allowbreak9; 6:3}
\crossref{James}{1}{27}{Jas 3:17 Ps 119:1 Mt 5:8 Lu 1:6 1Ti 1:5; 5:4}
\crossref{James}{2}{1}{Ac 20:21; 24:24 Col 1:4 1Ti 1:19 Tit 1:1 2Pe 1:1 Re 14:12}
\crossref{James}{2}{2}{Es 3:10; 8:2 Lu 15:22}
\crossref{James}{2}{3}{Jude 1:16}
\crossref{James}{2}{4}{Jas 1:1-\allowbreak27 Job 34:19 Mal 2:9}
\crossref{James}{2}{5}{Jud 9:7 1Ki 22:28 Job 34:10; 38:14 Pr 7:24; 8:32 Mr 7:14 Ac 7:2}
\crossref{James}{2}{6}{2:3 Ps 14:6 Pr 14:31; 17:5 Ec 9:15,\allowbreak16 Isa 53:3 Joh 8:49 1Co 11:22}
\crossref{James}{2}{7}{Ps 73:7-\allowbreak9 Mt 12:24; 27:63 Lu 22:64,\allowbreak65 Ac 26:11 1Ti 1:13}
\crossref{James}{2}{8}{2:12; 1:25 1Pe 2:9}
\crossref{James}{2}{9}{2:1-\allowbreak4 Le 19:15}
\crossref{James}{2}{10}{De 27:26 Mt 5:18,\allowbreak19 Ga 3:10}
\crossref{James}{2}{11}{Ex 20:13,\allowbreak14 De 5:17,\allowbreak18 Mt 5:21-\allowbreak28; 19:18 Mr 10:19 Lu 18:20}
\crossref{James}{2}{12}{Php 4:8 Col 3:17 2Pe 1:4-\allowbreak8}
\crossref{James}{2}{13}{Jas 5:4 Ge 42:21 Jud 1:7 Job 22:6-\allowbreak10 Pr 21:13 Isa 27:11}
\crossref{James}{2}{14}{2:16 Jer 7:8 Ro 2:25 1Co 13:3 1Ti 4:8 Heb 13:9}
\crossref{James}{2}{15}{2:5 Job 31:16-\allowbreak21 Isa 58:7,\allowbreak10 Eze 18:7 Mt 25:35-\allowbreak40}
\crossref{James}{2}{16}{Job 22:7-\allowbreak9 Pr 3:27,\allowbreak28 Mt 14:15,\allowbreak16; 15:32; 25:42-\allowbreak45 Ro 12:9}
\crossref{James}{2}{17}{2:14,\allowbreak19,\allowbreak20,\allowbreak26 1Co 13:3,\allowbreak13 1Th 1:3 1Ti 1:5 2Pe 1:5-\allowbreak9}
\crossref{James}{2}{18}{2:14,\allowbreak22 Ro 14:23 1Co 13:2 Ga 5:6 Heb 11:6,\allowbreak31}
\crossref{James}{2}{19}{De 6:4 Isa 43:10; 44:6,\allowbreak8; 45:6,\allowbreak21,\allowbreak22; 46:9 Zec 14:9 Mr 12:29}
\crossref{James}{2}{20}{Jas 1:26 Job 11:11,\allowbreak12 Ps 94:8-\allowbreak11 Pr 12:11 Jer 2:5 Ro 1:21}
\crossref{James}{2}{21}{Jos 24:3 Isa 51:2 Mt 3:9 Lu 1:73; 16:24,\allowbreak30 Joh 8:39,\allowbreak53}
\crossref{James}{2}{22}{2:18 Ga 5:6 Heb 11:17-\allowbreak19}
\crossref{James}{2}{23}{Mr 12:10; 15:28 Lu 4:21 Ac 1:16 Ro 9:17; 11:2}
\crossref{James}{2}{24}{2:15-\allowbreak18,\allowbreak21,\allowbreak22 Ps 60:12}
\crossref{James}{2}{25}{Jos 2:1 Mt 1:5}
\crossref{James}{2}{26}{Job 34:14,\allowbreak15 Ps 104:29; 146:4 Ec 12:7 Isa 2:22 Lu 23:46}
\crossref{James}{3}{1}{Mal 2:12 Mt 9:11; 10:24; 23:8-\allowbreak10,\allowbreak14 Joh 3:10 Ac 13:1 Ro 2:20,\allowbreak21}
\crossref{James}{3}{2}{1Ki 8:46 2Ch 6:36 Pr 20:9 Ec 7:20 Isa 64:6 Ro 3:10; 7:21}
\crossref{James}{3}{3}{Jas 1:26 2Ki 19:28 Ps 32:9; 39:1 Isa 37:29}
\crossref{James}{3}{4}{Ps 107:25-\allowbreak27 Jon 1:4 Mt 8:24 Ac 27:14 etc.}
\crossref{James}{3}{5}{Ex 5:2; 15:9 2Ki 19:22-\allowbreak24 Job 21:14,\allowbreak15; 22:17 Ps 10:3; 12:2-\allowbreak4}
\crossref{James}{3}{6}{Jud 12:4-\allowbreak6 2Sa 19:43; 20:1 2Ch 10:13-\allowbreak16; 13:17 Ps 64:3; 140:3}
\crossref{James}{3}{7}{Mr 5:4}
\crossref{James}{3}{8}{3:6 Ps 55:21; 57:4; 59:7; 64:3,\allowbreak4}
\crossref{James}{3}{9}{Ps 16:9; 30:12; 35:28; 51:14; 57:8; 62:4; 71:24; 108:1 Ac 2:26}
\crossref{James}{3}{10}{Ps 50:16-\allowbreak20 Jer 7:4-\allowbreak10 Mic 3:11 Ro 12:14 1Pe 3:9}
\crossref{James}{3}{11}{3:11}
\crossref{James}{3}{12}{Isa 5:2-\allowbreak4 Jer 2:21 Mt 7:16-\allowbreak20; 12:33 Lu 6:43,\allowbreak44 Ro 11:16-\allowbreak18}
\crossref{James}{3}{13}{3:1 Ps 107:43 Ec 8:1,\allowbreak5 Jer 9:12,\allowbreak23 Mt 7:24 1Co 6:5 Ga 6:4}
\crossref{James}{3}{14}{3:16; 4:1-\allowbreak5 Ge 30:1,\allowbreak2; 37:11 Job 5:2 Pr 14:30; 27:4 Isa 11:13}
\crossref{James}{3}{15}{3:17; 1:5,\allowbreak17 Joh 3:17 1Co 3:3 Php 3:19}
\crossref{James}{3}{16}{3:14 1Co 3:3 Ga 5:20}
\crossref{James}{3}{17}{3:15; 1:5,\allowbreak17 Ge 41:38,\allowbreak39 Ex 36:2 1Ki 3:9,\allowbreak12,\allowbreak28 1Ch 22:12}
\crossref{James}{3}{18}{Jas 1:20 Pr 11:18,\allowbreak28,\allowbreak30 Isa 32:16,\allowbreak17 Ho 10:12 Mt 5:9 Joh 4:36}
\crossref{James}{4}{1}{Jas 3:14-\allowbreak18}
\crossref{James}{4}{2}{Jas 5:1-\allowbreak5 Pr 1:19 Ec 4:8 Hab 2:5 1Ti 6:9,\allowbreak10}
\crossref{James}{4}{3}{Jas 1:6,\allowbreak7 Job 27:8-\allowbreak10; 35:12}
\crossref{James}{4}{4}{Ps 50:18; 73:27 Isa 57:3 Jer 9:2 Ho 3:1 Mt 12:39; 16:4}
\crossref{James}{4}{5}{Joh 7:42; 10:35; 19:37 Ro 9:17 Ga 3:8}
\crossref{James}{4}{6}{Ex 10:3,\allowbreak4; 15:9,\allowbreak10; 18:11 1Sa 2:3 Job 22:29; 40:10-\allowbreak12}
\crossref{James}{4}{7}{1Sa 3:18 2Sa 15:26 2Ki 1:13-\allowbreak15 2Ch 30:8; 33:12,\allowbreak13}
\crossref{James}{4}{8}{Ge 18:23 1Ch 28:9 2Ch 15:2 Ps 73:28; 145:18 Isa 29:13; 55:6,\allowbreak7}
\crossref{James}{4}{9}{Jas 5:1,\allowbreak2 Ps 119:67,\allowbreak71,\allowbreak136; 126:5,\allowbreak6 Ec 7:2-\allowbreak5 Isa 22:12,\allowbreak13}
\crossref{James}{4}{10}{4:6,\allowbreak7}
\crossref{James}{4}{11}{Ps 140:11 Eph 4:31 1Ti 3:11 2Ti 3:3 Tit 2:3 1Pe 2:1}
\crossref{James}{4}{12}{Isa 33:22}
\crossref{James}{4}{13}{Jas 5:1 Ge 11:3,\allowbreak4,\allowbreak7 Ec 2:1 Isa 5:5}
\crossref{James}{4}{14}{Jas 1:10 Job 7:6,\allowbreak7; 9:25,\allowbreak26; 14:1,\allowbreak2 Ps 39:5; 89:47; 90:5-\allowbreak7; 102:3}
\crossref{James}{4}{15}{2Sa 15:25,\allowbreak26 Pr 19:21 La 3:37 Ac 18:21 Ro 1:10; 15:32}
\crossref{James}{4}{16}{Jas 3:14 Ps 52:1,\allowbreak7 Pr 25:14; 27:1 Isa 47:7,\allowbreak8,\allowbreak10}
\crossref{James}{4}{17}{Lu 12:47,\allowbreak48 Joh 9:41; 13:17; 15:22 Ro 1:20,\allowbreak21,\allowbreak32; 2:17-\allowbreak23; 7:13}
\crossref{James}{5}{1}{Jas 4:13}
\crossref{James}{5}{2}{Jer 17:11 Mt 6:19,\allowbreak20 Lu 12:33 1Pe 1:4}
\crossref{James}{5}{3}{2Ti 2:17}
\crossref{James}{5}{4}{Le 19:13 De 24:14,\allowbreak15 Job 24:10,\allowbreak11; 31:38,\allowbreak39 Isa 5:7 Jer 22:13}
\crossref{James}{5}{5}{1Sa 25:6,\allowbreak36 Job 21:11-\allowbreak15 Ps 17:14; 73:7 Ec 11:9}
\crossref{James}{5}{6}{Jas 2:6 Mt 21:38; 23:34,\allowbreak35; 27:20,\allowbreak24,\allowbreak25 Joh 16:2,\allowbreak3 Ac 2:22,\allowbreak23}
\crossref{James}{5}{7}{5:8,\allowbreak9 Mt 24:27,\allowbreak44 Lu 18:8; 21:27 1Co 1:7 1Th 2:19; 3:13 2Pe 3:4}
\crossref{James}{5}{8}{Ge 49:18 Ps 37:7; 40:1-\allowbreak3; 130:5 La 3:25,\allowbreak26 Mic 7:7 Hab 2:3}
\crossref{James}{5}{9}{Jas 4:11 Le 19:18 Ps 59:15 Mr 6:19}
\crossref{James}{5}{10}{Isa 39:8 Jer 23:22; 26:16 Ac 3:21 Heb 13:7}
\crossref{James}{5}{11}{Jas 1:12 Ps 94:12 Mt 5:10,\allowbreak11; 10:22 Heb 3:6,\allowbreak14; 10:39}
\crossref{James}{5}{12}{1Pe 4:8 3Jo 1:2}
\crossref{James}{5}{13}{2Ch 33:12,\allowbreak13 Job 33:26 Ps 18:6; 50:15; 91:15; 116:3-\allowbreak5; 118:5}
\crossref{James}{5}{14}{Ac 14:23; 15:4 Tit 1:5}
\crossref{James}{5}{15}{5:13,\allowbreak16; 1:6 Mt 17:20,\allowbreak21; 21:21,\allowbreak22 Mr 11:22-\allowbreak24; 16:17,\allowbreak18 1Co 12:28-\allowbreak30}
\crossref{James}{5}{16}{Ge 41:9,\allowbreak10 2Sa 19:19 Mt 3:6; 18:15-\allowbreak17 Lu 7:3,\allowbreak4 Ac 19:18}
\crossref{James}{5}{17}{1Ki 17:1}
\crossref{James}{5}{18}{1Ki 18:18,\allowbreak42-\allowbreak45 Jer 14:22 Ac 14:17}
\crossref{James}{5}{19}{Ps 119:21,\allowbreak118 Pr 19:27 Isa 3:12 1Ti 6:10,\allowbreak21 2Ti 2:18 2Pe 3:17}
\crossref{James}{5}{20}{5:19}

% 1Pet
\crossref{1Pet}{1}{1}{Mt 4:18; 10:2 Joh 1:41,\allowbreak42; 21:15-\allowbreak17}
\crossref{1Pet}{1}{2}{1Pe 2:9 De 7:6 Isa 65:9,\allowbreak22 Mt 24:22,\allowbreak24,\allowbreak31 Mr 13:20,\allowbreak22,\allowbreak27 Lu 18:7}
\crossref{1Pet}{1}{3}{1Ki 8:15 1Ch 29:10-\allowbreak13,\allowbreak20 Ps 41:13; 72:18,\allowbreak19 2Co 1:3 Eph 1:3,\allowbreak17}
\crossref{1Pet}{1}{4}{1Pe 3:9 Mt 25:34 Ac 20:32; 26:18 Ga 3:18 Eph 1:11,\allowbreak14,\allowbreak18 Col 1:12}
\crossref{1Pet}{1}{5}{1Sa 2:9 Ps 37:23,\allowbreak24,\allowbreak28; 103:17,\allowbreak18; 125:1,\allowbreak2 Pr 2:8 Isa 54:17}
\crossref{1Pet}{1}{6}{1:8; 4:13 1Sa 2:1 Ps 9:14; 35:10; 95:1 Isa 12:2,\allowbreak3; 61:3,\allowbreak10 Mt 5:12}
\crossref{1Pet}{1}{7}{1Pe 4:12 Job 23:10 Ps 66:10-\allowbreak12 Pr 17:3 Isa 48:10 Jer 9:7 Zec 13:9}
\crossref{1Pet}{1}{8}{Joh 20:29 2Co 4:18; 5:7 Heb 11:1,\allowbreak27 1Jo 4:20}
\crossref{1Pet}{1}{9}{Ro 6:22 Heb 11:13 Jas 1:21}
\crossref{1Pet}{1}{10}{Ge 49:10 Da 2:44 Hag 2:7 Zec 6:12 Mt 13:17 Lu 10:24; 24:25-\allowbreak27}
\crossref{1Pet}{1}{11}{1Pe 3:18,\allowbreak19 Ro 8:9 Ga 4:6 2Pe 1:21 Re 19:10}
\crossref{1Pet}{1}{12}{Isa 53:1 Da 2:19,\allowbreak22,\allowbreak28,\allowbreak29,\allowbreak47; 10:1 Am 3:7 Mt 11:25,\allowbreak27; 16:17}
\crossref{1Pet}{1}{13}{Ex 12:11 1Ki 18:46 2Ki 4:29 Job 38:3; 40:7 Isa 11:5 Jer 1:17}
\crossref{1Pet}{1}{14}{Eph 2:2; 5:6}
\crossref{1Pet}{1}{15}{1Pe 2:9; 5:10 Ro 8:28-\allowbreak30; 9:24 Php 3:14 1Th 2:12; 4:7 2Ti 1:9}
\crossref{1Pet}{1}{16}{Le 11:44; 19:2; 20:7 Am 3:3}
\crossref{1Pet}{1}{17}{Zep 3:9 Mt 6:9; 7:7-\allowbreak11 2Co 1:2 Eph 1:17; 3:14}
\crossref{1Pet}{1}{18}{Ps 49:7,\allowbreak8 1Co 6:20; 7:23}
\crossref{1Pet}{1}{19}{1Pe 2:22-\allowbreak24; 3:18 Da 9:24 Zec 13:7 Mt 20:28; 26:28 Ac 20:28}
\crossref{1Pet}{1}{20}{Ge 3:15 Pr 8:23 Mic 5:2 Ro 3:25; 16:25,\allowbreak26 Eph 1:4; 3:9,\allowbreak11}
\crossref{1Pet}{1}{21}{Joh 5:24; 12:44; 14:6 Heb 6:1; 7:25}
\crossref{1Pet}{1}{22}{Joh 15:3; 17:17,\allowbreak19 Ac 15:9 Ro 6:16,\allowbreak17 2Th 2:13 Jas 4:8}
\crossref{1Pet}{1}{23}{1:3 Joh 1:3; 3:5}
\crossref{1Pet}{1}{24}{2Ki 19:26 Ps 37:2; 90:5; 92:7; 102:4; 103:15; 129:6 Isa 40:6-\allowbreak8}
\crossref{1Pet}{1}{25}{1:23 Ps 102:12,\allowbreak26; 119:89 Isa 40:8 Mt 5:18 Lu 16:17}
\crossref{1Pet}{2}{1}{1Pe 1:18-\allowbreak25}
\crossref{1Pet}{2}{2}{1Pe 1:23 Mt 18:3 Mr 10:15 Ro 6:4 1Co 3:1; 14:20}
\crossref{1Pet}{2}{3}{Ps 9:10; 24:8; 63:5 So 2:3 Zec 9:17 Heb 6:5,\allowbreak6}
\crossref{1Pet}{2}{4}{Isa 55:3 Jer 3:22 Mt 11:28 Joh 5:40; 6:37}
\crossref{1Pet}{2}{5}{1Co 3:16; 6:19 2Co 6:16 Eph 2:20-\allowbreak22 Heb 3:6 Re 3:12}
\crossref{1Pet}{2}{6}{Da 10:21 Mr 12:10 Joh 7:38 Ac 1:16 2Ti 3:16 2Pe 1:20; 3:16}
\crossref{1Pet}{2}{7}{1Pe 1:8 So 5:9-\allowbreak16 Hag 2:7 Mt 13:44-\allowbreak46 Joh 4:42; 6:68,\allowbreak69 Php 3:7-\allowbreak10}
\crossref{1Pet}{2}{8}{Isa 8:14; 57:14 Lu 2:34 Ro 9:32,\allowbreak33 1Co 1:23 2Co 2:16}
\crossref{1Pet}{2}{9}{1Pe 1:2 De 10:15 Ps 22:30; 33:12; 73:15 Isa 41:8; 44:1}
\crossref{1Pet}{2}{10}{Ho 1:9,\allowbreak10 Ro 9:25,\allowbreak26}
\crossref{1Pet}{2}{11}{Ro 12:1 2Co 5:20; 6:1 Eph 4:1 Phm 1:9,\allowbreak10}
\crossref{1Pet}{2}{12}{1Pe 3:2 Ps 37:14; 50:23 2Co 1:12 Eph 2:3; 4:22 Php 1:27 1Ti 4:12}
\crossref{1Pet}{2}{13}{Pr 17:11; 24:21 Jer 29:7 Mt 22:21 Mr 12:17 Lu 20:25 Ro 13:1-\allowbreak7}
\crossref{1Pet}{2}{14}{Ro 13:3,\allowbreak4}
\crossref{1Pet}{2}{15}{1Pe 4:2 Eph 6:6,\allowbreak7 1Th 4:3; 5:18}
\crossref{1Pet}{2}{16}{Joh 8:32-\allowbreak36 Ro 6:18,\allowbreak22 1Co 7:22 Ga 5:1,\allowbreak13 Jas 1:25; 2:12}
\crossref{1Pet}{2}{17}{1Pe 5:5 Ex 20:12 Le 19:32 1Sa 15:30 Ro 12:10; 13:7 Php 2:3 1Ti 6:1}
\crossref{1Pet}{2}{18}{Eph 6:5-\allowbreak7 Col 3:22-\allowbreak25 1Ti 6:1-\allowbreak3 Tit 2:9,\allowbreak10}
\crossref{1Pet}{2}{19}{2:20 Lu 6:32}
\crossref{1Pet}{2}{20}{1Pe 3:14; 4:14-\allowbreak16 Mt 5:47}
\crossref{1Pet}{2}{21}{Mt 10:38; 16:24 Mr 8:34,\allowbreak35 Lu 9:23-\allowbreak25; 14:26,\allowbreak27 Joh 16:33}
\crossref{1Pet}{2}{22}{Isa 53:9 Mt 27:4,\allowbreak19,\allowbreak23,\allowbreak24 Lu 23:41,\allowbreak47 Joh 8:46 2Co 5:21}
\crossref{1Pet}{2}{23}{Ps 38:12-\allowbreak14 Isa 53:7 Mt 27:39-\allowbreak44 Mr 14:60,\allowbreak61; 15:29-\allowbreak32}
\crossref{1Pet}{2}{24}{Ex 28:38 Le 16:22; 22:9 Nu 18:22 Ps 38:4 Isa 53:4-\allowbreak6,\allowbreak11 Mt 8:17}
\crossref{1Pet}{2}{25}{Ps 119:176 Isa 53:6 Jer 23:2 Eze 34:6 Mt 9:36; 18:12 Lu 15:4-\allowbreak6}
\crossref{1Pet}{3}{1}{Ge 3:16 Es 1:16-\allowbreak20 Ro 7:2}
\crossref{1Pet}{3}{2}{3:16; 1:15; 2:12 Php 1:27; 3:20 1Ti 4:12 2Pe 3:11}
\crossref{1Pet}{3}{3}{1Ti 2:9,\allowbreak10 Tit 2:3 etc.}
\crossref{1Pet}{3}{4}{Ps 45:13; 51:6 Mt 23:26 Lu 11:40 Ro 2:29; 6:6; 7:22 2Co 4:16}
\crossref{1Pet}{3}{5}{Pr 31:10,\allowbreak30 Lu 8:2,\allowbreak3 Ac 1:14; 9:36 1Ti 2:10; 5:10 Tit 2:3,\allowbreak4}
\crossref{1Pet}{3}{6}{Ge 18:12}
\crossref{1Pet}{3}{7}{Ge 2:23,\allowbreak24 Pr 5:15-\allowbreak19 Mal 2:14-\allowbreak16 Mt 19:3-\allowbreak9 1Co 7:3 Col 3:19}
\crossref{1Pet}{3}{8}{Ac 2:1; 4:32 Ro 12:16; 15:5 1Co 1:10 Php 3:16}
\crossref{1Pet}{3}{9}{1Pe 2:20-\allowbreak23 Pr 17:13; 20:22 Mt 5:39,\allowbreak44 Lu 6:27-\allowbreak29 Ro 12:14,\allowbreak17,\allowbreak19-\allowbreak21}
\crossref{1Pet}{3}{10}{Ps 34:12-\allowbreak16}
\crossref{1Pet}{3}{11}{Job 1:1; 2:3; 28:28 Ps 34:14; 37:27 Pr 3:7; 16:6,\allowbreak17 Isa 1:16,\allowbreak17}
\crossref{1Pet}{3}{12}{De 11:12 2Ch 16:9 Ps 11:4 Pr 15:3 Zec 4:10}
\crossref{1Pet}{3}{13}{Pr 16:7 Ro 8:28; 13:3}
\crossref{1Pet}{3}{14}{1Pe 2:19,\allowbreak20; 4:13-\allowbreak16 Jer 15:15 Mt 5:10-\allowbreak12; 10:18-\allowbreak22,\allowbreak39; 16:25; 19:29}
\crossref{1Pet}{3}{15}{Nu 20:12; 27:14 Isa 5:16; 29:23}
\crossref{1Pet}{3}{16}{3:21; 2:19 Ac 24:16 Ro 9:1 2Co 1:12; 4:2 1Ti 1:5,\allowbreak19 2Ti 1:3}
\crossref{1Pet}{3}{17}{1Pe 4:19 Mt 26:39,\allowbreak42 Ac 21:14}
\crossref{1Pet}{3}{18}{1Pe 2:21-\allowbreak24; 4:1 Isa 53:4-\allowbreak6 Ro 5:6-\allowbreak8; 8:3 2Co 5:21 Ga 1:4; 3:13}
\crossref{1Pet}{3}{19}{1Pe 1:11,\allowbreak12; 4:6 Ne 9:30 Re 19:10}
\crossref{1Pet}{3}{20}{Ge 6:3,\allowbreak5,\allowbreak13}
\crossref{1Pet}{3}{21}{Ro 5:14 1Co 4:6 Heb 9:24}
\crossref{1Pet}{3}{22}{Mr 16:19 Ac 1:11; 2:34-\allowbreak36; 3:21 Heb 6:20; 8:1; 9:24}
\crossref{1Pet}{4}{1}{1Pe 3:18}
\crossref{1Pet}{4}{2}{1Pe 2:1,\allowbreak14 Ro 7:4; 14:7 Eph 4:17,\allowbreak22-\allowbreak24; 5:7,\allowbreak8 Col 3:7,\allowbreak8 Tit 3:3-\allowbreak8}
\crossref{1Pet}{4}{3}{Eze 44:6; 45:9 Ac 17:30 Ro 8:12,\allowbreak13 1Co 6:11}
\crossref{1Pet}{4}{4}{Mt 23:25 Lu 15:13 Ro 13:13 2Pe 2:22}
\crossref{1Pet}{4}{5}{Mal 3:13-\allowbreak15 Mt 12:36 Lu 16:2 Ro 14:12 Jude 1:14,\allowbreak15}
\crossref{1Pet}{4}{6}{1Pe 3:19 Joh 5:25,\allowbreak26}
\crossref{1Pet}{4}{7}{Ec 7:2 Jer 5:31 Eze 7:2,\allowbreak3,\allowbreak6 Mt 24:13,\allowbreak14 Ro 13:12 1Co 7:29}
\crossref{1Pet}{4}{8}{Col 3:14 Jas 5:12 3Jo 1:2}
\crossref{1Pet}{4}{9}{Ro 12:13; 16:23 1Ti 3:2 Tit 1:8 Heb 13:2,\allowbreak16}
\crossref{1Pet}{4}{10}{Mt 25:14,\allowbreak15 Lu 19:13 Ro 12:6-\allowbreak8 1Co 4:7; 12:4-\allowbreak11}
\crossref{1Pet}{4}{11}{Isa 8:20 Jer 23:22 Eph 4:29 Col 4:6 Jas 1:19,\allowbreak26; 3:1-\allowbreak6}
\crossref{1Pet}{4}{12}{4:4 Isa 28:21}
\crossref{1Pet}{4}{13}{1Pe 1:6 Mt 5:12 Lu 6:22,\allowbreak23 Ac 5:41; 16:25 Ro 5:3 2Co 4:17; 12:9,\allowbreak10}
\crossref{1Pet}{4}{14}{1Pe 2:19,\allowbreak20; 3:14,\allowbreak16}
\crossref{1Pet}{4}{15}{1Pe 2:20 Mt 5:11 2Ti 2:9}
\crossref{1Pet}{4}{16}{4:19; 3:17,\allowbreak18 Ac 11:26; 26:28 Eph 3:13-\allowbreak15}
\crossref{1Pet}{4}{17}{Isa 10:12 Jer 25:29; 49:12 Eze 9:6 Mal 3:5 Mt 3:9,\allowbreak10}
\crossref{1Pet}{4}{18}{1Pe 5:8 Pr 11:31 Jer 25:29 Eze 18:24 Zec 13:9 Mt 24:22-\allowbreak24}
\crossref{1Pet}{4}{19}{4:12-\allowbreak16; 3:17 Ac 21:11-\allowbreak14}
\crossref{1Pet}{5}{1}{Ac 11:30; 14:23; 15:4,\allowbreak6,\allowbreak22,\allowbreak23; 20:17,\allowbreak28}
\crossref{1Pet}{5}{2}{So 1:8 Isa 40:11 Eze 34:2,\allowbreak3,\allowbreak23 Mic 5:4; 7:14 Joh 21:15-\allowbreak17}
\crossref{1Pet}{5}{3}{Eze 34:4 Mt 20:25,\allowbreak26; 23:8-\allowbreak10 Mr 10:42-\allowbreak45 Lu 22:24-\allowbreak27}
\crossref{1Pet}{5}{4}{5:2; 2:25 Ps 23:1 Isa 40:11 Eze 34:23; 37:24 Zec 13:7 Joh 10:11}
\crossref{1Pet}{5}{5}{Le 19:32 Heb 13:17}
\crossref{1Pet}{5}{6}{Ex 10:3 Le 26:41 1Ki 21:29 2Ki 22:19 2Ch 12:6,\allowbreak7,\allowbreak12; 30:11; 32:26}
\crossref{1Pet}{5}{7}{1Sa 1:10-\allowbreak18; 30:6 Ps 27:13,\allowbreak14; 37:5; 55:22; 56:3,\allowbreak4 Mt 6:25,\allowbreak34}
\crossref{1Pet}{5}{8}{1Pe 1:13; 4:7 Mt 24:48-\allowbreak50 Lu 12:45,\allowbreak46; 21:34,\allowbreak36 Ro 13:11-\allowbreak13}
\crossref{1Pet}{5}{9}{Lu 4:3-\allowbreak12 Eph 4:27; 6:11-\allowbreak13 Jas 4:7}
\crossref{1Pet}{5}{10}{Ex 34:6,\allowbreak7 Ps 86:5,\allowbreak15 Mic 7:18,\allowbreak19 Ro 5:20,\allowbreak21; 15:5,\allowbreak13 2Co 13:11}
\crossref{1Pet}{5}{11}{1Pe 4:11 Re 1:6; 5:13}
\crossref{1Pet}{5}{12}{2Co 1:19 1Th 1:1 2Th 1:1}
\crossref{1Pet}{5}{13}{Ps 87:4 Re 17:5; 18:2}
\crossref{1Pet}{5}{14}{Ro 16:16 1Co 16:20 2Co 13:12 1Th 5:26}

% 2Pet
\crossref{2Pet}{1}{1}{Ac 15:14}
\crossref{2Pet}{1}{2}{Nu 6:24-\allowbreak26 Da 4:1; 6:25}
\crossref{2Pet}{1}{3}{Ps 110:3 Mt 28:18 Joh 17:2 2Co 12:9 Eph 1:19-\allowbreak21 Col 1:16}
\crossref{2Pet}{1}{4}{1:1 Eze 36:25-\allowbreak27 Ro 9:4 2Co 1:20; 6:17,\allowbreak18; 7:1 Ga 3:16}
\crossref{2Pet}{1}{5}{Lu 16:26; 24:21}
\crossref{2Pet}{1}{6}{Ac 24:25 1Co 9:25 Ga 5:23 Tit 1:8; 2:2}
\crossref{2Pet}{1}{7}{Joh 13:34,\allowbreak35 Ro 12:10 1Th 3:12; 4:9,\allowbreak10 Heb 13:1 1Pe 1:22; 2:17}
\crossref{2Pet}{1}{8}{Joh 5:42 2Co 9:14; 13:5 Php 2:5 Col 3:16 Phm 1:6}
\crossref{2Pet}{1}{9}{1:5-\allowbreak7 Mr 10:21 Lu 18:22 Ga 5:6,\allowbreak13 Jas 2:14-\allowbreak26}
\crossref{2Pet}{1}{10}{1:5; 3:17}
\crossref{2Pet}{1}{11}{Mt 25:34 2Co 5:1 2Ti 4:8 Re 3:21}
\crossref{2Pet}{1}{12}{1:13,\allowbreak15; 3:1 Ro 15:14,\allowbreak15 Php 3:1 1Ti 4:6 2Ti 1:6 Heb 10:32 Jude 1:3,\allowbreak17}
\crossref{2Pet}{1}{13}{1:14 2Co 5:1-\allowbreak4,\allowbreak8 Heb 13:3}
\crossref{2Pet}{1}{14}{De 4:21,\allowbreak22; 31:14 Jos 23:14 1Ki 2:2,\allowbreak3 Ac 20:25 2Ti 4:6}
\crossref{2Pet}{1}{15}{De 31:19-\allowbreak29 Jos 24:24-\allowbreak29 1Ch 29:1-\allowbreak20 Ps 71:18 2Ti 2:2 Heb 11:4}
\crossref{2Pet}{1}{16}{2Pe 3:3,\allowbreak4 1Co 1:17,\allowbreak23; 2:1,\allowbreak4 2Co 2:17; 4:2; 12:16,\allowbreak17 Eph 4:14 2Th 2:9}
\crossref{2Pet}{1}{17}{Mt 11:25-\allowbreak27; 28:19 Lu 10:22 Joh 3:35; 5:21-\allowbreak23,\allowbreak26,\allowbreak36,\allowbreak37}
\crossref{2Pet}{1}{18}{Mt 17:6}
\crossref{2Pet}{1}{19}{Ps 19:7-\allowbreak9 Isa 8:20; 41:21-\allowbreak23,\allowbreak26 Lu 16:29-\allowbreak31 Joh 5:39 Ac 17:11}
\crossref{2Pet}{1}{20}{2Pe 3:3 Ro 6:6; 13:11 1Ti 1:9 Jas 1:3}
\crossref{2Pet}{1}{21}{Lu 1:70 2Ti 3:16 1Pe 1:11}
\crossref{2Pet}{2}{1}{De 13:1-\allowbreak3 1Ki 18:19-\allowbreak22; 22:6 Ne 6:12-\allowbreak14 Isa 9:15; 56:10,\allowbreak11}
\crossref{2Pet}{2}{2}{Mt 24:10-\allowbreak13,\allowbreak24 Mr 13:22 1Jo 2:18,\allowbreak19 Re 12:9; 13:8,\allowbreak14}
\crossref{2Pet}{2}{3}{2:14,\allowbreak15 Isa 56:11 Jer 6:13; 8:10 Eze 13:19 Mic 3:11 Mal 1:10}
\crossref{2Pet}{2}{4}{2:5 De 29:20 Ps 78:50 Eze 5:11; 7:4,\allowbreak9 Ro 8:32; 11:21}
\crossref{2Pet}{2}{5}{Ge 6:1-\allowbreak8:22 Job 22:15,\allowbreak16 Mt 24:37-\allowbreak39 Lu 17:26,\allowbreak27 Heb 11:7}
\crossref{2Pet}{2}{6}{Ge 19:24,\allowbreak25,\allowbreak28 De 29:23 Isa 13:19 Jer 50:40 Eze 16:49-\allowbreak56}
\crossref{2Pet}{2}{7}{Ge 19:16,\allowbreak22,\allowbreak29 1Co 10:13}
\crossref{2Pet}{2}{8}{Pr 25:26; 28:12 1Ti 1:9 Jas 5:16}
\crossref{2Pet}{2}{9}{Job 5:19 Ps 34:15-\allowbreak19 1Co 10:13}
\crossref{2Pet}{2}{10}{Ro 8:1,\allowbreak4,\allowbreak5,\allowbreak12,\allowbreak13 2Co 10:3 Heb 13:4}
\crossref{2Pet}{2}{11}{Ps 103:20; 104:4 Da 6:22 2Th 1:7 Jude 1:9}
\crossref{2Pet}{2}{12}{Ps 49:10; 92:6; 94:8 Jer 4:22; 5:4; 10:8,\allowbreak21; 12:3 Eze 21:31 Jude 1:10}
\crossref{2Pet}{2}{13}{Isa 3:11 Ro 2:8,\allowbreak9 Php 3:19 2Ti 4:14 Heb 2:2,\allowbreak3 Jude 1:12 etc.}
\crossref{2Pet}{2}{14}{2Sa 11:2-\allowbreak4 Job 31:7,\allowbreak9 Pr 6:25 Mt 5:28 1Jo 2:16}
\crossref{2Pet}{2}{15}{1Sa 12:23 1Ki 18:18; 19:10 Eze 9:10 Pr 28:4 Ho 14:8 Ac 13:10}
\crossref{2Pet}{2}{16}{Nu 22:22-\allowbreak33}
\crossref{2Pet}{2}{17}{Job 6:14-\allowbreak17 Jer 14:3 Ho 6:4 Jude 1:12,\allowbreak13}
\crossref{2Pet}{2}{18}{Ps 52:1-\allowbreak3; 73:8,\allowbreak9 Da 4:30; 11:36 Ac 8:9 2Th 2:4 Jude 1:13,\allowbreak15,\allowbreak16}
\crossref{2Pet}{2}{19}{Ga 5:1,\allowbreak13 1Pe 2:16}
\crossref{2Pet}{2}{20}{Mt 12:43-\allowbreak45 Lu 11:24-\allowbreak26 Heb 6:4-\allowbreak8; 10:26,\allowbreak27}
\crossref{2Pet}{2}{21}{Mt 11:23,\allowbreak24 Lu 12:47 Joh 9:41; 15:22}
\crossref{2Pet}{2}{22}{Pr 26:11}
\crossref{2Pet}{3}{1}{2Co 13:2 1Pe 1:1,\allowbreak2}
\crossref{2Pet}{3}{2}{2Pe 1:19-\allowbreak21 Lu 1:70; 24:27,\allowbreak44 Ac 3:18,\allowbreak24-\allowbreak26; 10:43; 28:23 1Pe 1:10-\allowbreak12}
\crossref{2Pet}{3}{3}{1Ti 4:1,\allowbreak2 2Ti 3:1 1Jo 2:18 Jude 1:18}
\crossref{2Pet}{3}{4}{Ge 19:14 Ec 1:9; 8:11 Isa 5:18,\allowbreak19 Jer 5:12,\allowbreak13; 17:15}
\crossref{2Pet}{3}{5}{Pr 17:16 Joh 3:19,\allowbreak20 Ro 1:28 2Th 2:10-\allowbreak12}
\crossref{2Pet}{3}{6}{2Pe 2:5 Ge 7:10-\allowbreak23; 9:15 Job 12:15 Mt 24:38,\allowbreak39 Lu 17:27}
\crossref{2Pet}{3}{7}{3:10 Ps 50:3; 102:26 Isa 51:6 Zep 3:8 Mt 24:35; 25:41 2Th 1:8}
\crossref{2Pet}{3}{8}{Ro 11:25 1Co 10:1; 12:1}
\crossref{2Pet}{3}{9}{Isa 46:13 Hab 2:3 Lu 18:7,\allowbreak8 Heb 10:37}
\crossref{2Pet}{3}{10}{Isa 2:12 Joe 1:15; 2:1,\allowbreak31; 3:14 Mal 4:5 1Co 5:5 2Co 1:14 Jude 1:6}
\crossref{2Pet}{3}{11}{3:12 Ps 75:3 Isa 14:31; 24:19; 34:4}
\crossref{2Pet}{3}{12}{1Co 1:7 Tit 2:13 Jude 1:21}
\crossref{2Pet}{3}{13}{Isa 65:17; 66:22 Re 21:1,\allowbreak27}
\crossref{2Pet}{3}{14}{Php 3:20 Heb 9:28}
\crossref{2Pet}{3}{15}{3:9 Ro 2:4 1Ti 1:16 1Pe 3:20}
\crossref{2Pet}{3}{16}{1Pe 1:1}
\crossref{2Pet}{3}{17}{2Pe 1:12 Pr 1:17 Mt 24:24,\allowbreak25 Mr 13:23 Joh 16:4}
\crossref{2Pet}{3}{18}{Ps 92:12 Ho 14:5 Mal 4:2 Eph 4:15 Col 1:10 2Th 1:3 1Pe 2:2}

% 1John
\crossref{1John}{1}{1}{1Jo 2:13 Pr 8:22-\allowbreak31 Isa 41:4 Mic 5:2 Joh 1:1,\allowbreak2 etc.}
\crossref{1John}{1}{2}{1Jo 5:11,\allowbreak20 Joh 1:4; 11:25,\allowbreak26; 14:6}
\crossref{1John}{1}{3}{1:1 Ac 4:20}
\crossref{1John}{1}{4}{Isa 61:10 Hab 3:17,\allowbreak18 Joh 15:11; 16:24 2Co 1:24 Eph 3:19}
\crossref{1John}{1}{5}{1Jo 3:11 1Co 11:23}
\crossref{1John}{1}{6}{1:8,\allowbreak10; 2:4; 4:20 Mt 7:22 Jas 2:14,\allowbreak16,\allowbreak18 Re 3:17,\allowbreak18}
\crossref{1John}{1}{7}{1Jo 2:9,\allowbreak10 Ps 56:13; 89:15; 97:11 Isa 2:5 Joh 12:35 Ro 13:12}
\crossref{1John}{1}{8}{1:6,\allowbreak10; 3:5,\allowbreak6 1Ki 8:46 2Ch 6:36 Job 9:2; 14:4; 15:14; 25:4}
\crossref{1John}{1}{9}{Le 26:40-\allowbreak42 1Ki 8:47 2Ch 6:37,\allowbreak38 Ne 1:6; 9:2 etc.}
\crossref{1John}{1}{10}{1:8 Ps 130:3}
\crossref{1John}{2}{1}{2:12,\allowbreak13; 3:7,\allowbreak18; 4:4; 5:21 Joh 13:33; 21:5 1Co 4:14,\allowbreak15 Ga 4:19}
\crossref{1John}{2}{2}{1Jo 1:7; 4:10 Ro 3:25,\allowbreak26 1Pe 2:24; 3:18}
\crossref{1John}{2}{3}{2:4-\allowbreak6; 3:14,\allowbreak19; 4:13; 5:19}
\crossref{1John}{2}{4}{2:9; 1:6,\allowbreak8,\allowbreak10; 4:20 Jas 2:14-\allowbreak16}
\crossref{1John}{2}{5}{2:3,\allowbreak4 Ps 105:45; 106:3; 119:2,\allowbreak4,\allowbreak146 Pr 8:32; 28:7 Ec 8:5 Eze 36:27}
\crossref{1John}{2}{6}{2:4; 1:6}
\crossref{1John}{2}{7}{1Jo 3:11 Ac 17:19 2Jo 1:5}
\crossref{1John}{2}{8}{1Jo 4:21 Joh 13:34; 15:12}
\crossref{1John}{2}{9}{2:4}
\crossref{1John}{2}{10}{1Jo 3:14 Ho 6:3 Joh 8:31 Ro 14:13 2Pe 1:10}
\crossref{1John}{2}{11}{2:9 Joh 12:35 Tit 3:3}
\crossref{1John}{2}{12}{2:7,\allowbreak13,\allowbreak14,\allowbreak21; 1:4}
\crossref{1John}{2}{13}{2:14 1Ti 5:1}
\crossref{1John}{2}{14}{2:13}
\crossref{1John}{2}{15}{1Jo 4:5; 5:4,\allowbreak5,\allowbreak10 Joh 15:19 Ro 12:2 Ga 1:10 Eph 2:2 Col 3:1,\allowbreak2}
\crossref{1John}{2}{16}{Nu 11:4,\allowbreak34 Ps 78:18,\allowbreak30 Pr 6:25 Mt 5:28 Ro 13:14 1Co 10:6}
\crossref{1John}{2}{17}{Ps 39:6; 73:18-\allowbreak20; 90:9; 102:26 Isa 40:6-\allowbreak8 Mt 24:35 1Co 7:31}
\crossref{1John}{2}{18}{2:1 Joh 21:5}
\crossref{1John}{2}{19}{De 13:13 Ps 41:9 Mt 13:20,\allowbreak21 Mr 4:5,\allowbreak6,\allowbreak16,\allowbreak17 Lu 8:13 Joh 15:2}
\crossref{1John}{2}{20}{2:27; 4:13 Ps 23:5; 45:7; 92:10 Isa 61:1 Lu 4:18 Ac 10:38 2Co 1:21,\allowbreak22}
\crossref{1John}{2}{21}{Pr 1:5; 9:8,\allowbreak9 Ro 15:14,\allowbreak15 2Pe 1:12}
\crossref{1John}{2}{22}{2:4; 1:6; 4:20 Joh 8:44 Re 3:9}
\crossref{1John}{2}{23}{2:22; 4:15 Mt 11:27 Lu 10:22 Joh 5:23; 8:19; 10:30; 14:9,\allowbreak10; 15:23,\allowbreak24}
\crossref{1John}{2}{24}{Ps 119:11 Pr 23:23 Lu 9:44 Joh 15:7 Col 3:16 Heb 2:1; 3:14}
\crossref{1John}{2}{25}{1Jo 1:2; 5:11-\allowbreak13,\allowbreak20 Da 12:2 Lu 18:30 Joh 5:39; 6:27,\allowbreak47,\allowbreak54,\allowbreak68; 10:28}
\crossref{1John}{2}{26}{1Jo 3:7 Pr 12:26 Eze 13:10 Mr 13:22 Ac 20:29,\allowbreak30 2Co 11:13-\allowbreak15}
\crossref{1John}{2}{27}{2:20; 3:24 Joh 4:14 1Pe 1:23 2Jo 1:2}
\crossref{1John}{2}{28}{2:1}
\crossref{1John}{2}{29}{2:1; 3:5 Zec 9:9 Ac 3:14; 22:14 2Co 5:21 Heb 1:8,\allowbreak9; 7:2,\allowbreak26 1Pe 3:18}
\crossref{1John}{3}{1}{1Jo 4:9,\allowbreak10 2Sa 7:19 Ps 31:19; 36:7-\allowbreak9; 89:1,\allowbreak2 Joh 3:16 Ro 5:8; 8:32}
\crossref{1John}{3}{2}{3:1; 5:1 Isa 56:5 Ro 8:14,\allowbreak15,\allowbreak18 Ga 3:26; 4:6}
\crossref{1John}{3}{3}{Ro 5:4,\allowbreak5 Col 1:5 2Th 2:16 Tit 3:7 Heb 6:18}
\crossref{1John}{3}{4}{3:8,\allowbreak9 1Ki 8:47 1Ch 10:13 2Co 12:21 Jas 5:15}
\crossref{1John}{3}{5}{1Jo 1:2; 4:9-\allowbreak14 Joh 1:31 1Ti 3:16 1Pe 1:20}
\crossref{1John}{3}{6}{1Jo 2:28 Joh 15:4-\allowbreak7}
\crossref{1John}{3}{7}{1Jo 2:26,\allowbreak29 Ro 2:13 1Co 6:9 Ga 6:7,\allowbreak8 Eph 5:6 Jas 1:22; 2:19; 5:1-\allowbreak3}
\crossref{1John}{3}{8}{3:10; 5:19}
\crossref{1John}{3}{9}{1Jo 2:29; 4:7; 5:1,\allowbreak4,\allowbreak18 Joh 1:13}
\crossref{1John}{3}{10}{1Jo 5:2 Lu 6:35 Ro 8:16,\allowbreak17 Eph 5:1}
\crossref{1John}{3}{11}{1Jo 1:5; 2:7,\allowbreak8}
\crossref{1John}{3}{12}{Ge 4:4-\allowbreak15,\allowbreak25 Heb 11:4 Jude 1:11}
\crossref{1John}{3}{13}{Ec 5:8 Joh 3:7 Ac 3:12 Re 17:7}
\crossref{1John}{3}{14}{1Jo 2:3; 5:2,\allowbreak13,\allowbreak19,\allowbreak20 2Co 5:1}
\crossref{1John}{3}{15}{Ge 27:41 Le 19:16-\allowbreak18 2Sa 13:22-\allowbreak28 Pr 26:24-\allowbreak26 Mt 5:21,\allowbreak22,\allowbreak28}
\crossref{1John}{3}{16}{1Jo 4:9,\allowbreak10 Mt 20:28 Joh 3:16; 10:15; 15:13 Ac 20:28 Ro 5:8}
\crossref{1John}{3}{17}{De 15:7-\allowbreak11 Pr 19:17 Isa 58:7-\allowbreak10 Lu 3:11 2Co 8:9,\allowbreak14,\allowbreak15; 9:5-\allowbreak9}
\crossref{1John}{3}{18}{1Jo 2:1}
\crossref{1John}{3}{19}{3:14; 1:8 Joh 13:35; 18:37}
\crossref{1John}{3}{20}{Job 27:6 Joh 8:9 Ac 5:33 Ro 2:14,\allowbreak15 1Co 4:4; 14:24,\allowbreak25 Tit 3:11}
\crossref{1John}{3}{21}{1Jo 2:28; 4:17 Job 22:26; 27:6 Ps 7:3-\allowbreak5; 101:2 1Co 4:4 2Co 1:12}
\crossref{1John}{3}{22}{1Jo 5:14 Ps 10:17; 34:4,\allowbreak15-\allowbreak17; 50:15; 66:18,\allowbreak19; 145:18,\allowbreak19 Pr 15:29}
\crossref{1John}{3}{23}{De 18:15-\allowbreak19 Ps 2:12 Mr 9:7 Joh 6:29; 14:1; 17:3 Ac 16:31}
\crossref{1John}{3}{24}{3:22 Joh 14:21-\allowbreak23; 15:7-\allowbreak10}
\crossref{1John}{4}{1}{De 13:1-\allowbreak5 Pr 14:15 Jer 5:31; 29:8,\allowbreak9 Mt 7:15,\allowbreak16; 24:4,\allowbreak5 Ro 16:18}
\crossref{1John}{4}{2}{1Jo 5:1 Joh 16:13-\allowbreak15 1Co 12:3}
\crossref{1John}{4}{3}{1Jo 2:18,\allowbreak22 2Th 2:7,\allowbreak8 2Jo 1:7}
\crossref{1John}{4}{4}{4:6,\allowbreak16; 3:9,\allowbreak10; 5:19}
\crossref{1John}{4}{5}{Ps 17:4 Lu 16:8 Joh 3:31; 7:6,\allowbreak7; 8:23; 15:19,\allowbreak20; 17:14,\allowbreak16 Re 12:9}
\crossref{1John}{4}{6}{4:4 Mic 3:8 Ro 1:1 1Co 2:12-\allowbreak14 2Pe 3:2 Jude 1:17}
\crossref{1John}{4}{7}{4:20,\allowbreak21}
\crossref{1John}{4}{8}{1Jo 2:4,\allowbreak9; 3:6 Joh 8:54,\allowbreak55}
\crossref{1John}{4}{9}{1Jo 3:16 Joh 3:16 Ro 5:8-\allowbreak10; 8:32}
\crossref{1John}{4}{10}{4:8,\allowbreak9; 3:1}
\crossref{1John}{4}{11}{1Jo 3:16,\allowbreak17,\allowbreak23 Mt 18:32,\allowbreak33 Lu 10:37 Joh 13:34; 15:12,\allowbreak13 2Co 8:8,\allowbreak9}
\crossref{1John}{4}{12}{4:20 Ge 32:30 Ex 33:20 Nu 12:8 Joh 1:18 1Ti 1:17; 6:16 Heb 11:27}
\crossref{1John}{4}{13}{4:15,\allowbreak16}
\crossref{1John}{4}{14}{1Jo 1:1-\allowbreak3; 5:9 Joh 1:14; 3:11,\allowbreak32; 5:39; 15:26,\allowbreak27 Ac 18:5 1Pe 5:12}
\crossref{1John}{4}{15}{4:2; 5:1,\allowbreak5 Mt 10:32 Lu 12:8 Ro 10:9 Php 2:11 2Jo 1:7}
\crossref{1John}{4}{16}{4:9,\allowbreak10; 3:1,\allowbreak16 Ps 18:1-\allowbreak3; 31:19; 36:7-\allowbreak9 Isa 64:4 1Co 2:9}
\crossref{1John}{4}{17}{4:12; 2:5 Jas 2:22}
\crossref{1John}{4}{18}{Lu 1:74,\allowbreak75 Ro 8:15 2Ti 1:7 Heb 12:28}
\crossref{1John}{4}{19}{4:10 Lu 7:47 Joh 3:16; 15:16 2Co 5:14,\allowbreak15 Ga 5:22 Eph 2:3-\allowbreak5}
\crossref{1John}{4}{20}{1Jo 2:4; 3:17}
\crossref{1John}{4}{21}{4:11; 3:11,\allowbreak14,\allowbreak18,\allowbreak23 Le 19:18 Mt 22:37-\allowbreak39 Mr 12:29-\allowbreak33 Lu 10:37}
\crossref{1John}{5}{1}{1Jo 2:22,\allowbreak23; 4:2,\allowbreak14,\allowbreak15 Mt 16:16 Joh 1:12,\allowbreak13; 6:69 Ac 8:37 Ro 10:9,\allowbreak10}
\crossref{1John}{5}{2}{1Jo 3:22-\allowbreak24; 4:21 Joh 13:34,\allowbreak35; 15:17}
\crossref{1John}{5}{3}{Ex 20:6 De 5:10; 7:9; 10:12,\allowbreak13 Da 9:4 Mt 12:47-\allowbreak50 Joh 14:15}
\crossref{1John}{5}{4}{5:1; 3:9}
\crossref{1John}{5}{5}{5:1; 4:15}
\crossref{1John}{5}{6}{Joh 19:34,\allowbreak35}
\crossref{1John}{5}{7}{5:10,\allowbreak11 Joh 8:13,\allowbreak14}
\crossref{1John}{5}{8}{5:7}
\crossref{1John}{5}{9}{5:10 Joh 3:32,\allowbreak33; 5:31-\allowbreak36,\allowbreak39; 8:17-\allowbreak19; 10:38 Ac 5:32; 17:31}
\crossref{1John}{5}{10}{5:1 Joh 3:16}
\crossref{1John}{5}{11}{5:7,\allowbreak10 Joh 1:19,\allowbreak32-\allowbreak34; 8:13,\allowbreak14; 19:35 3Jo 1:12 Re 1:2}
\crossref{1John}{5}{12}{1Jo 2:23,\allowbreak24 Joh 1:12; 3:36; 5:24 1Co 1:30 Ga 2:20 Heb 3:14 2Jo 1:9}
\crossref{1John}{5}{13}{1Jo 1:4; 2:1,\allowbreak13,\allowbreak14,\allowbreak21,\allowbreak26 Joh 20:31; 21:24 1Pe 5:12}
\crossref{1John}{5}{14}{1Jo 3:21 Eph 3:12 Heb 3:6,\allowbreak14; 10:35}
\crossref{1John}{5}{15}{Pr 15:29 Jer 15:12,\allowbreak13}
\crossref{1John}{5}{16}{Ge 20:7,\allowbreak17 Ex 32:10-\allowbreak14,\allowbreak31,\allowbreak32; 34:9 Nu 12:13; 14:11-\allowbreak21 De 9:18-\allowbreak20}
\crossref{1John}{5}{17}{1Jo 3:4 De 5:32; 12:32}
\crossref{1John}{5}{18}{5:1,\allowbreak4; 2:29; 3:9; 4:6 Joh 1:13; 3:2-\allowbreak5 Jas 1:18 1Pe 1:23}
\crossref{1John}{5}{19}{5:10,\allowbreak13,\allowbreak20; 3:14,\allowbreak24; 4:4-\allowbreak6 Ro 8:16 2Co 1:12; 5:1 2Ti 1:12}
\crossref{1John}{5}{20}{5:1; 4:2,\allowbreak14}
\crossref{1John}{5}{21}{1Jo 2:1}

% 2John
\crossref{2John}{1}{1}{1Pe 5:1 3Jo 1:1}
\crossref{2John}{1}{2}{1Co 9:23 2Co 4:5}
\crossref{2John}{1}{3}{Ro 1:7 1Ti 1:2}
\crossref{2John}{1}{4}{Php 4:10 1Th 2:19,\allowbreak20; 3:6-\allowbreak10 3Jo 1:3,\allowbreak4}
\crossref{2John}{1}{5}{1Jo 2:7,\allowbreak8; 3:11}
\crossref{2John}{1}{6}{Joh 14:15,\allowbreak21; 15:10,\allowbreak14 Ro 13:8,\allowbreak9 Ga 5:13,\allowbreak14 1Jo 5:3,\allowbreak15}
\crossref{2John}{1}{7}{2Pe 2:1-\allowbreak3 1Jo 2:18-\allowbreak22; 4:1}
\crossref{2John}{1}{8}{Mt 24:4,\allowbreak24,\allowbreak25 Mr 13:5,\allowbreak6,\allowbreak9,\allowbreak23 Lu 21:8 Heb 12:15 Re 3:11}
\crossref{2John}{1}{9}{Joh 15:6 1Jo 2:22-\allowbreak24}
\crossref{2John}{1}{10}{1:11 Ro 16:17,\allowbreak18 1Co 5:11; 16:22 Ga 1:8,\allowbreak9 2Ti 3:5,\allowbreak6 Tit 3:10}
\crossref{2John}{1}{11}{Ps 50:18 Eph 5:11 1Ti 5:22 Re 18:4}
\crossref{2John}{1}{12}{Joh 16:12}
\crossref{2John}{1}{13}{1:1 1Pe 5:13}

% 3John
\crossref{3John}{1}{1}{2Jo 1:1}
\crossref{3John}{1}{2}{Jas 5:12 1Pe 4:8}
\crossref{3John}{1}{3}{1:4}
\crossref{3John}{1}{4}{Pr 23:24}
\crossref{3John}{1}{5}{Mt 24:45 Lu 12:42; 16:10-\allowbreak12 2Co 4:1-\allowbreak3 Col 3:17 1Pe 4:10,\allowbreak11}
\crossref{3John}{1}{6}{1:12 Phm 1:5-\allowbreak7}
\crossref{3John}{1}{7}{Ac 8:4; 9:16 2Co 4:5 Col 1:24 Re 2:3}
\crossref{3John}{1}{8}{1:10 Mt 10:14,\allowbreak40 Lu 11:7 2Co 7:2,\allowbreak3}
\crossref{3John}{1}{9}{1:8 Mt 10:40-\allowbreak42 Mr 9:37 Lu 9:48}
\crossref{3John}{1}{10}{1Co 5:1-\allowbreak5 2Co 10:1-\allowbreak11; 13:2}
\crossref{3John}{1}{11}{Ex 23:2 Ps 37:27 Pr 12:11 Isa 1:16,\allowbreak17 Joh 10:27; 12:26}
\crossref{3John}{1}{12}{Ac 10:22; 22:12 1Th 4:12 1Ti 3:7}
\crossref{3John}{1}{13}{2Jo 1:12}
\crossref{3John}{1}{14}{Ge 43:23 Da 4:1 Ga 5:16 Eph 6:23 1Pe 5:14}
\crossref{3John}{1}{15}{Ge 43:23 Da 4:1 Ga 5:16 Eph 6:23 1Pe 5:14}

% Jude
\crossref{Jude}{}{1}{Mt 10:3}
\crossref{Jude}{}{2}{Ro 1:7 1Pe 1:2 2Pe 1:2 Re 1:4-\allowbreak6}
\crossref{Jude}{}{3}{Ro 15:15,\allowbreak16 Ga 6:11 Heb 13:22 1Pe 5:12 2Pe 1:12-\allowbreak15; 3:1}
\crossref{Jude}{}{4}{Mt 13:25 Ac 15:24 Ga 2:4 Eph 4:14 2Ti 3:6 2Pe 2:1,\allowbreak2}
\crossref{Jude}{}{5}{Ro 15:15 2Pe 1:12,\allowbreak13; 3:1}
\crossref{Jude}{}{6}{Joh 8:44}
\crossref{Jude}{}{7}{Ge 13:13; 18:20; 19:24-\allowbreak26 De 29:23 Isa 1:9; 13:19}
\crossref{Jude}{}{8}{Jer 38:25-\allowbreak28}
\crossref{Jude}{}{9}{1Th 4:16}
\crossref{Jude}{}{10}{2Pe 2:12}
\crossref{Jude}{}{11}{Isa 3:9,\allowbreak11 Jer 13:27 Eze 13:3 Zec 11:17 Mt 11:21; 23:13-\allowbreak16}
\crossref{Jude}{}{12}{2Pe 2:13,\allowbreak14}
\crossref{Jude}{}{13}{Ps 65:7; 93:3,\allowbreak4 Isa 57:20 Jer 5:22,\allowbreak23}
\crossref{Jude}{}{14}{Ge 5:18,\allowbreak24 1Ch 1:1-\allowbreak3 Heb 11:5,\allowbreak6}
\crossref{Jude}{}{15}{Ps 9:7,\allowbreak8; 37:6; 50:1-\allowbreak6; 98:9; 149:9 Ec 11:9; 12:14 Joh 5:22,\allowbreak23,\allowbreak27}
\crossref{Jude}{}{16}{Nu 14:36; 16:11 De 1:27 Ps 106:25 Isa 29:24 Lu 5:30; 15:2; 19:7}
\crossref{Jude}{}{17}{Mal 4:4 Ac 20:35 Eph 2:20; 4:11 2Pe 3:2 1Jo 4:6}
\crossref{Jude}{}{18}{Ac 20:29 1Ti 4:1,\allowbreak2 2Ti 3:1-\allowbreak5,\allowbreak13; 4:3 2Pe 2:1; 3:3}
\crossref{Jude}{}{19}{Pr 18:1 Isa 65:5 Eze 14:7 Ho 4:14; 9:10 Heb 10:25}
\crossref{Jude}{}{20}{Ac 9:31 Ro 15:2 1Co 1:8; 10:23; 14:4,\allowbreak5,\allowbreak26 Eph 4:12,\allowbreak16,\allowbreak29}
\crossref{Jude}{}{21}{1:24 Joh 14:21; 15:9,\allowbreak10 Ac 11:23 1Jo 4:16; 5:18,\allowbreak21 Re 12:11}
\crossref{Jude}{}{22}{1:4-\allowbreak13 Eze 34:17 Ga 4:20; 6:1 Heb 6:4-\allowbreak8}
\crossref{Jude}{}{23}{Ro 11:14 1Co 5:3-\allowbreak5 2Co 7:10-\allowbreak12 1Ti 4:16}
\crossref{Jude}{}{24}{1:21 Joh 10:29,\allowbreak30 Ro 8:31; 14:4; 16:25-\allowbreak27 Eph 3:20 2Ti 4:18}
\crossref{Jude}{}{25}{Ps 104:24; 147:5 Ro 11:33; 16:27 Eph 1:8; 3:10 1Ti 1:17}

% Rev
\crossref{Rev}{1}{1}{Da 2:28,\allowbreak29 Am 3:7 Ro 16:25 Ga 1:12 Eph 3:3}
\crossref{Rev}{1}{2}{1:9; 6:9; 12:11,\allowbreak17 Joh 1:32; 12:17; 19:35; 21:24 1Co 1:6; 2:1}
\crossref{Rev}{1}{3}{Re 22:7 Pr 8:34 Da 12:12,\allowbreak13 Lu 11:28}
\crossref{Rev}{1}{4}{1:1}
\crossref{Rev}{1}{5}{Re 3:14 Ps 89:36,\allowbreak37 Isa 55:4 Joh 3:11,\allowbreak32; 8:14-\allowbreak16; 18:37 1Ti 6:13}
\crossref{Rev}{1}{6}{Re 5:10; 20:6 Ex 19:6 Isa 61:6 Ro 12:1 1Pe 2:5-\allowbreak9}
\crossref{Rev}{1}{7}{Re 14:14-\allowbreak16 Ps 97:2 Isa 19:1 Da 7:13 Na 1:3 Mt 24:30; 26:64}
\crossref{Rev}{1}{8}{1:11,\allowbreak17; 2:8; 21:6; 22:13 Isa 41:4; 43:10; 44:6; 48:12}
\crossref{Rev}{1}{9}{1:4}
\crossref{Rev}{1}{10}{Re 4:2; 17:3; 21:10 Mt 22:43 Ac 10:10 2Co 12:2-\allowbreak4}
\crossref{Rev}{1}{11}{1:8,\allowbreak17}
\crossref{Rev}{1}{12}{Eze 43:5,\allowbreak6 Mic 6:9}
\crossref{Rev}{1}{13}{Re 14:14 Eze 1:26-\allowbreak28 Da 7:9,\allowbreak13; 10:5,\allowbreak6,\allowbreak16 Php 2:7,\allowbreak8 Heb 2:14-\allowbreak17}
\crossref{Rev}{1}{14}{Da 7:9 Mt 28:3}
\crossref{Rev}{1}{15}{Re 2:18 Eze 1:7; 40:3 Da 10:6}
\crossref{Rev}{1}{16}{1:20; 2:1; 3:1; 12:1 Job 38:7 Da 8:10; 12:3}
\crossref{Rev}{1}{17}{Eze 1:28 Da 8:18; 10:8,\allowbreak9,\allowbreak17-\allowbreak19 Hab 3:16 Mt 17:2-\allowbreak6 Joh 13:23}
\crossref{Rev}{1}{18}{Job 19:25 Ps 18:46 Joh 14:19 Ro 6:9 2Co 13:4 Ga 2:20 Col 3:3}
\crossref{Rev}{1}{19}{1:11,\allowbreak12 etc.}
\crossref{Rev}{1}{20}{Mt 13:11 Lu 8:10}
\crossref{Rev}{2}{1}{2:8,\allowbreak12,\allowbreak18; 3:1,\allowbreak7,\allowbreak14}
\crossref{Rev}{2}{2}{2:9,\allowbreak13,\allowbreak19; 3:1,\allowbreak8,\allowbreak15 Ps 1:6 Mt 7:23 1Th 1:3 2Ti 2:19 Heb 6:10}
\crossref{Rev}{2}{3}{Ps 69:7 Mic 7:9 Mr 15:21 Lu 14:27 1Co 13:7 Ga 6:2 Heb 13:13}
\crossref{Rev}{2}{4}{2:14,\allowbreak20}
\crossref{Rev}{2}{5}{Re 3:3,\allowbreak19 Eze 16:61-\allowbreak63; 20:43; 36:31 2Pe 1:12,\allowbreak13}
\crossref{Rev}{2}{6}{2:14,\allowbreak15 2Ch 19:2 Ps 26:5; 101:3; 139:21,\allowbreak22 2Jo 1:9,\allowbreak10}
\crossref{Rev}{2}{7}{2:11,\allowbreak17,\allowbreak29; 3:6,\allowbreak13,\allowbreak22; 13:9 Mt 11:15; 13:9,\allowbreak43 Mr 7:16}
\crossref{Rev}{2}{8}{2:1}
\crossref{Rev}{2}{9}{2:2}
\crossref{Rev}{2}{10}{Mt 10:22}
\crossref{Rev}{2}{11}{2:7; 13:9}
\crossref{Rev}{2}{12}{2:1; 1:11}
\crossref{Rev}{2}{13}{2:2,\allowbreak9}
\crossref{Rev}{2}{14}{2:4,\allowbreak20}
\crossref{Rev}{2}{15}{2:6}
\crossref{Rev}{2}{16}{2:5,\allowbreak21,\allowbreak22; 3:19; 16:9 Ac 17:30,\allowbreak31}
\crossref{Rev}{2}{17}{2:7,\allowbreak11; 3:6,\allowbreak13,\allowbreak22}
\crossref{Rev}{2}{18}{2:1; 1:11}
\crossref{Rev}{2}{19}{2:2,\allowbreak9,\allowbreak13}
\crossref{Rev}{2}{20}{2:4,\allowbreak14}
\crossref{Rev}{2}{21}{Re 9:20,\allowbreak21 Jer 8:4-\allowbreak6 Ro 2:4,\allowbreak5; 9:22 1Pe 3:20 2Pe 3:9,\allowbreak15}
\crossref{Rev}{2}{22}{Re 17:2; 18:3,\allowbreak9; 19:18-\allowbreak21 Eze 16:37-\allowbreak41; 23:29,\allowbreak45-\allowbreak48}
\crossref{Rev}{2}{23}{Re 6:8}
\crossref{Rev}{2}{24}{Re 12:9; 13:14 2Co 2:11; 11:3,\allowbreak13-\allowbreak15 Eph 6:11,\allowbreak12 2Th 2:9-\allowbreak12}
\crossref{Rev}{2}{25}{Re 3:3,\allowbreak11 Ac 11:28 Ro 12:9 1Th 5:21 Heb 3:6; 4:14; 10:23}
\crossref{Rev}{2}{26}{2:7,\allowbreak11,\allowbreak17; 3:5,\allowbreak12,\allowbreak21; 21:7 Ro 8:37 1Jo 5:5}
\crossref{Rev}{2}{27}{Re 12:5; 19:15 Ps 2:8,\allowbreak9; 49:14; 149:5-\allowbreak9 Da 7:22}
\crossref{Rev}{2}{28}{Re 22:16 Lu 1:78,\allowbreak79 2Pe 1:19}
\crossref{Rev}{2}{29}{2:7}
\crossref{Rev}{3}{1}{Re 1:11,\allowbreak20}
\crossref{Rev}{3}{2}{Re 16:15 Isa 56:10; 62:6,\allowbreak7 Eze 34:8-\allowbreak10,\allowbreak16 Zec 11:16 Mt 24:42-\allowbreak51}
\crossref{Rev}{3}{3}{Re 2:5 Eze 16:61-\allowbreak63; 20:43; 36:31 Heb 2:1 2Pe 1:13; 3:1}
\crossref{Rev}{3}{4}{Re 11:13}
\crossref{Rev}{3}{5}{Re 2:7 1Sa 17:25}
\crossref{Rev}{3}{6}{Re 2:7}
\crossref{Rev}{3}{7}{Re 1:11; 2:1}
\crossref{Rev}{3}{8}{3:1,\allowbreak15}
\crossref{Rev}{3}{9}{Re 2:9}
\crossref{Rev}{3}{10}{Re 1:9; 13:10; 14:12}
\crossref{Rev}{3}{11}{Re 1:3; 22:7,\allowbreak12,\allowbreak20 Zep 1:14 Php 4:5 Jas 5:9}
\crossref{Rev}{3}{12}{Re 2:7; 17:14 1Jo 2:13,\allowbreak14; 4:4}
\crossref{Rev}{3}{13}{Re 2:7}
\crossref{Rev}{3}{14}{Re 1:11; 2:1}
\crossref{Rev}{3}{15}{3:1; 2:2}
\crossref{Rev}{3}{16}{Re 2:5 Jer 14:19; 15:1-\allowbreak4 Zec 11:8,\allowbreak9}
\crossref{Rev}{3}{17}{Re 2:9 Pr 13:7 Ho 12:8 Zec 11:5 Lu 1:53; 6:24; 18:11,\allowbreak12 Ro 11:20,\allowbreak25}
\crossref{Rev}{3}{18}{Ps 16:7; 32:8; 73:24; 107:11 Pr 1:25,\allowbreak30; 19:20 Ec 8:2}
\crossref{Rev}{3}{19}{De 8:5 2Sa 7:14 Job 5:17 Ps 6:1; 39:11; 94:10 Pr 3:11,\allowbreak12; 15:10}
\crossref{Rev}{3}{20}{So 5:2-\allowbreak4 Lu 12:36}
\crossref{Rev}{3}{21}{Re 2:7; 12:11 1Jo 5:4,\allowbreak5}
\crossref{Rev}{3}{22}{3:6,\allowbreak13}
\crossref{Rev}{4}{1}{Re 1:1-\allowbreak3:22}
\crossref{Rev}{4}{2}{Re 1:10; 17:3; 21:10 Eze 3:12-\allowbreak14}
\crossref{Rev}{4}{3}{Re 21:11,\allowbreak19,\allowbreak20 Ex 24:10 Eze 1:26; 28:13}
\crossref{Rev}{4}{4}{Re 11:16; 20:4 Mt 19:28 Lu 22:30}
\crossref{Rev}{4}{5}{Re 8:5; 11:19; 16:17,\allowbreak18 Ex 19:16; 20:18 Ps 18:13,\allowbreak14; 68:35 Joe 3:16}
\crossref{Rev}{4}{6}{Re 15:2 Ex 38:8 1Ki 7:23}
\crossref{Rev}{4}{7}{4:6 Ge 49:9 Nu 2:2 etc.}
\crossref{Rev}{4}{8}{Isa 6:2 etc.}
\crossref{Rev}{4}{9}{Re 5:13,\allowbreak14; 7:11,\allowbreak12}
\crossref{Rev}{4}{10}{Re 5:8,\allowbreak14; 19:4 Job 1:20 Ps 72:11 Mt 2:11}
\crossref{Rev}{4}{11}{Re 5:2,\allowbreak9,\allowbreak12 2Sa 22:4 Ps 18:3}
\crossref{Rev}{5}{1}{Re 4:3}
\crossref{Rev}{5}{2}{Ps 103:20}
\crossref{Rev}{5}{3}{5:13 Isa 40:13,\allowbreak14; 41:28 Ro 11:34}
\crossref{Rev}{5}{4}{Re 4:1 Da 12:8,\allowbreak9}
\crossref{Rev}{5}{5}{Re 4:4,\allowbreak10; 7:13}
\crossref{Rev}{5}{6}{Re 4:4-\allowbreak6}
\crossref{Rev}{5}{7}{5:1}
\crossref{Rev}{5}{8}{5:14}
\crossref{Rev}{5}{9}{Re 7:10-\allowbreak12; 14:3 Ps 33:3; 40:3; 96:1; 98:1; 144:9; 149:1 Isa 42:10}
\crossref{Rev}{5}{10}{Re 1:6; 20:6; 22:5 Ex 19:6 1Pe 2:5-\allowbreak9}
\crossref{Rev}{5}{11}{Re 7:11 1Ki 22:19 2Ki 6:16-\allowbreak18 Ps 103:20; 148:2}
\crossref{Rev}{5}{12}{5:9 Zec 13:7}
\crossref{Rev}{5}{13}{5:3; 7:9,\allowbreak10 Ps 96:11-\allowbreak13; 148:2-\allowbreak13 Lu 2:14 Php 2:10 Col 1:23}
\crossref{Rev}{5}{14}{Re 19:4}
\crossref{Rev}{6}{1}{Re 5:5-\allowbreak7}
\crossref{Rev}{6}{2}{Ps 45:3-\allowbreak5; 76:7}
\crossref{Rev}{6}{3}{6:1; 4:7}
\crossref{Rev}{6}{4}{Re 12:3; 17:3,\allowbreak6 Zec 1:8; 6:2}
\crossref{Rev}{6}{5}{6:1; 4:6,\allowbreak7; 5:5,\allowbreak9}
\crossref{Rev}{6}{6}{Re 9:4 Ps 76:10}
\crossref{Rev}{6}{7}{6:1,\allowbreak3,\allowbreak5; 4:7}
\crossref{Rev}{6}{8}{Zec 6:3}
\crossref{Rev}{6}{9}{Re 8:3; 9:13; 14:18 Le 4:7 Joh 16:2}
\crossref{Rev}{6}{10}{Ge 4:10 Ps 9:12 Lu 18:7,\allowbreak8 Heb 12:24}
\crossref{Rev}{6}{11}{Re 3:4,\allowbreak5; 7:9,\allowbreak14}
\crossref{Rev}{6}{12}{Re 8:5; 11:13; 16:18 1Ki 19:11-\allowbreak13 Isa 29:6 Am 1:1 Zec 14:5 Mt 24:7}
\crossref{Rev}{6}{13}{Re 8:10-\allowbreak12; 9:1 Eze 32:7 Da 8:10 Lu 21:25}
\crossref{Rev}{6}{14}{Ps 102:26 Isa 34:4 Heb 1:11-\allowbreak13 2Pe 3:10}
\crossref{Rev}{6}{15}{Re 18:9-\allowbreak11; 19:13-\allowbreak21 Job 34:19,\allowbreak20 Ps 2:10-\allowbreak12; 49:1,\allowbreak2; 76:12; 110:5,\allowbreak6}
\crossref{Rev}{6}{16}{Re 10:6 Jer 8:3 Ho 10:8 Lu 23:30}
\crossref{Rev}{6}{17}{Re 11:18; 16:14 Isa 13:6 etc.}
\crossref{Rev}{7}{1}{Re 4:1-\allowbreak6:17}
\crossref{Rev}{7}{2}{Re 8:3; 10:1 Mal 3:1; 4:2 Ac 7:30-\allowbreak32}
\crossref{Rev}{7}{3}{Re 6:6; 9:4 Isa 6:13; 27:8; 65:8 Mt 24:22,\allowbreak31}
\crossref{Rev}{7}{4}{Re 9:16}
\crossref{Rev}{7}{5}{Ex 1:2-\allowbreak4 Nu 1:4-\allowbreak15; 10:14-\allowbreak27; 13:4-\allowbreak16 1Co 2:1,\allowbreak2}
\crossref{Rev}{7}{6}{Lu 2:36}
\crossref{Rev}{7}{7}{Nu 1:22; 25:14; 26:14 Jos 19:1 Jud 1:3}
\crossref{Rev}{7}{8}{}
\crossref{Rev}{7}{9}{Ge 49:10 Ps 2:8; 22:27; 72:7-\allowbreak11; 76:4; 77:2; 98:3; 110:2,\allowbreak3; 117:1,\allowbreak2}
\crossref{Rev}{7}{10}{Zec 4:7}
\crossref{Rev}{7}{11}{Re 4:6; 5:11-\allowbreak13; 19:4-\allowbreak6 Ps 103:20,\allowbreak21; 148:1,\allowbreak2}
\crossref{Rev}{7}{12}{Re 1:18; 5:13,\allowbreak14; 19:4 Ps 41:13; 72:19; 89:52; 106:48 Mt 6:13 Jude 1:25}
\crossref{Rev}{7}{13}{Re 4:4,\allowbreak10; 5:5,\allowbreak11}
\crossref{Rev}{7}{14}{Ex 37:3}
\crossref{Rev}{7}{15}{Re 4:4; 14:3-\allowbreak5 Heb 8:1; 12:2}
\crossref{Rev}{7}{16}{Ps 42:2; 63:1; 143:6 Isa 41:17; 49:10; 65:13 Mt 5:6 Lu 1:53; 6:21}
\crossref{Rev}{7}{17}{Re 5:6}
\crossref{Rev}{8}{1}{Re 5:1,\allowbreak9; 6:1,\allowbreak3,\allowbreak5,\allowbreak7,\allowbreak9,\allowbreak12}
\crossref{Rev}{8}{2}{Re 15:1; 16:1 Mt 18:10 Lu 1:19}
\crossref{Rev}{8}{3}{Re 7:2; 10:1}
\crossref{Rev}{8}{4}{8:3; 15:8 Ex 30:1 Ps 141:2 Lu 1:10}
\crossref{Rev}{8}{5}{Re 16:1 etc.}
\crossref{Rev}{8}{6}{8:2}
\crossref{Rev}{8}{7}{Re 16:21 Ex 9:23-\allowbreak25,\allowbreak33 Jos 10:11 Ps 11:5,\allowbreak6; 18:12,\allowbreak13; 78:47,\allowbreak48}
\crossref{Rev}{8}{8}{Jer 51:25 Mr 11:23}
\crossref{Rev}{8}{9}{8:7,\allowbreak10,\allowbreak12; 16:3 Ex 7:21 Zec 13:8}
\crossref{Rev}{8}{10}{Re 1:20; 6:13; 9:1; 12:4 Isa 14:12 Lu 10:18 Jude 1:13}
\crossref{Rev}{8}{11}{De 29:18 Ru 1:20 Pr 5:4 Jer 9:15; 23:15 La 3:5,\allowbreak19 Am 5:7; 6:12}
\crossref{Rev}{8}{12}{Re 16:8,\allowbreak9 Isa 13:10; 24:23 Jer 4:23 Eze 32:7,\allowbreak8 Joe 2:10,\allowbreak31 Am 8:9}
\crossref{Rev}{8}{13}{Re 14:3,\allowbreak6; 19:17 Ps 103:20 Heb 1:14}
\crossref{Rev}{9}{1}{9:12,\allowbreak13; 8:6-\allowbreak8,\allowbreak10,\allowbreak12; 11:14,\allowbreak15}
\crossref{Rev}{9}{2}{9:17; 14:11 Ge 15:17; 19:28 Isa 14:31 Joe 2:30 Ac 2:19}
\crossref{Rev}{9}{3}{Ex 10:4-\allowbreak15 Jud 7:12 Isa 33:4 Joe 1:4; 2:25 Na 3:15,\allowbreak17}
\crossref{Rev}{9}{4}{Re 6:6; 7:3 Job 1:10,\allowbreak12 Ps 76:10 Mt 24:24 2Ti 3:8,\allowbreak9}
\crossref{Rev}{9}{5}{Re 13:5,\allowbreak7 Da 5:18-\allowbreak22; 7:6 Joh 19:11}
\crossref{Rev}{9}{6}{Re 6:16 2Sa 1:9 Job 3:20-\allowbreak22; 7:15,\allowbreak16 Isa 2:19 Jer 8:3 Ho 10:8}
\crossref{Rev}{9}{7}{Joe 2:4,\allowbreak5 Na 3:17}
\crossref{Rev}{9}{8}{2Ki 9:30 Isa 3:24 1Co 11:14,\allowbreak15 1Ti 2:9 1Pe 3:3}
\crossref{Rev}{9}{9}{Re 9:17 Job 40:18; 41:23-\allowbreak30 Joe 2:8}
\crossref{Rev}{9}{10}{9:3,\allowbreak5}
\crossref{Rev}{9}{11}{Re 12:9 Joh 12:31; 14:30; 16:11 2Co 4:4 Eph 2:2 1Jo 4:4; 5:19}
\crossref{Rev}{9}{12}{9:1,\allowbreak2}
\crossref{Rev}{9}{13}{9:1}
\crossref{Rev}{9}{14}{Re 8:2,\allowbreak6}
\crossref{Rev}{9}{15}{9:5,\allowbreak10}
\crossref{Rev}{9}{16}{Ps 68:17 Da 7:10}
\crossref{Rev}{9}{17}{Re 21:20}
\crossref{Rev}{9}{18}{9:15,\allowbreak17}
\crossref{Rev}{9}{19}{9:10 Isa 9:15 Eph 4:14}
\crossref{Rev}{9}{20}{9:21; 2:21,\allowbreak22; 16:8 De 31:29 2Ch 28:22 Jer 5:3; 8:4-\allowbreak6 Mt 21:32}
\crossref{Rev}{9}{21}{Re 11:7-\allowbreak9; 13:7,\allowbreak15; 16:6; 18:24 Da 7:21-\allowbreak25; 11:33}
\crossref{Rev}{10}{1}{10:5,\allowbreak6; 5:2; 7:1,\allowbreak2; 8:2-\allowbreak5,\allowbreak13; 9:13,\allowbreak14; 14:14,\allowbreak15}
\crossref{Rev}{10}{2}{10:10; 5:1-\allowbreak5; 6:1,\allowbreak3 Eze 2:9,\allowbreak10}
\crossref{Rev}{10}{3}{Pr 19:12 Isa 5:29; 31:4; 42:13 Jer 25:30 Joe 3:16 Am 1:2; 3:8}
\crossref{Rev}{10}{4}{Re 1:11; 2:1-\allowbreak3:22 Isa 8:1 Hab 2:2,\allowbreak3}
\crossref{Rev}{10}{5}{10:2}
\crossref{Rev}{10}{6}{}
\crossref{Rev}{10}{7}{Re 11:15-\allowbreak18}
\crossref{Rev}{10}{8}{10:4,\allowbreak5 Isa 30:21}
\crossref{Rev}{10}{9}{Job 23:12 Jer 15:16 Eze 2:8; 3:1-\allowbreak3,\allowbreak14 Col 3:6}
\crossref{Rev}{10}{10}{Ps 19:10; 104:34; 119:103 Pr 16:24 Eze 3:3}
\crossref{Rev}{10}{11}{Re 11:9; 14:6; 17:15 Jer 1:9,\allowbreak10; 25:15-\allowbreak30}
\crossref{Rev}{11}{1}{Re 21:15 Isa 28:17 Eze 40:3-\allowbreak5; 42:15-\allowbreak20 Zec 2:1,\allowbreak2 Ga 6:14-\allowbreak16}
\crossref{Rev}{11}{2}{Eze 40:17-\allowbreak20; 42:20}
\crossref{Rev}{11}{3}{Nu 11:26 De 17:6; 19:15 Mt 18:16 2Co 13:1}
\crossref{Rev}{11}{4}{Ps 52:8 Jer 11:16 Zec 4:2,\allowbreak3,\allowbreak11-\allowbreak14 Ro 11:17}
\crossref{Rev}{11}{5}{Nu 16:28-\allowbreak35 2Ki 1:10-\allowbreak12 Ps 18:8 Isa 11:4 Jer 1:10; 5:14}
\crossref{Rev}{11}{6}{1Ki 17:1 Lu 4:25 Jas 5:16-\allowbreak18}
\crossref{Rev}{11}{7}{11:3 Lu 13:32 Joh 17:4; 19:30 Ac 20:24 2Ti 4:7}
\crossref{Rev}{11}{8}{11:9 Ps 79:2,\allowbreak3 Jer 26:23 Eze 37:11}
\crossref{Rev}{11}{9}{Re 10:11; 13:7; 17:15}
\crossref{Rev}{11}{10}{Re 12:13; 13:8,\allowbreak14 Mt 10:22}
\crossref{Rev}{11}{11}{11:9}
\crossref{Rev}{11}{12}{Re 4:1 Ps 15:1; 24:3 Isa 40:31}
\crossref{Rev}{11}{13}{11:19; 6:12; 8:5; 16:18}
\crossref{Rev}{11}{14}{Re 8:13; 9:12; 15:1; 16:1 etc.}
\crossref{Rev}{11}{15}{Re 8:2-\allowbreak6,\allowbreak12; 9:1,\allowbreak13; 10:7}
\crossref{Rev}{11}{16}{Re 4:4,\allowbreak10; 5:5-\allowbreak8,\allowbreak14; 7:11; 19:4}
\crossref{Rev}{11}{17}{Re 4:9 Da 2:23; 6:10 Mt 11:25 Lu 10:21 Joh 11:41 2Co 2:14; 9:15}
\crossref{Rev}{11}{18}{11:2,\allowbreak9,\allowbreak10; 17:12-\allowbreak15; 19:19,\allowbreak20 Ps 2:1-\allowbreak3 Isa 34:1-\allowbreak10; 63:1-\allowbreak6}
\crossref{Rev}{11}{19}{Re 14:15-\allowbreak17; 15:5-\allowbreak8; 19:11 Isa 6:1-\allowbreak4}
\crossref{Rev}{12}{1}{12:3; 11:19; 15:1 2Ch 32:31 Mr 13:25 Ac 2:19}
\crossref{Rev}{12}{2}{12:4 Isa 53:11; 54:1; 66:7,\allowbreak8 Mic 5:3 Joh 16:21 Ga 4:19,\allowbreak27}
\crossref{Rev}{12}{3}{12:1}
\crossref{Rev}{12}{4}{Re 9:10,\allowbreak19 Da 8:9-\allowbreak12}
\crossref{Rev}{12}{5}{12:2 Isa 7:14 Jer 31:22 Mic 5:3 Mt 1:25}
\crossref{Rev}{12}{6}{12:4,\allowbreak14}
\crossref{Rev}{12}{7}{Re 13:7; 19:11-\allowbreak20 Isa 34:5 Eph 6:12}
\crossref{Rev}{12}{8}{12:11 Ps 13:4; 118:10-\allowbreak13; 129:2 Jer 1:19; 5:22 Mt 16:18 Ro 8:31-\allowbreak39}
\crossref{Rev}{12}{9}{12:3,\allowbreak7 Lu 10:18 Joh 12:31}
\crossref{Rev}{12}{10}{Re 11:15; 19:1-\allowbreak7}
\crossref{Rev}{12}{11}{Re 2:7,\allowbreak11,\allowbreak17,\allowbreak26; 3:5,\allowbreak12,\allowbreak21 Joh 16:33 Ro 8:33-\allowbreak39; 16:20 1Co 15:57}
\crossref{Rev}{12}{12}{Re 18:20; 19:1-\allowbreak7 Ps 96:11-\allowbreak13; 148:1-\allowbreak4 Isa 49:13; 55:12,\allowbreak13 Lu 2:14}
\crossref{Rev}{12}{13}{12:4,\allowbreak5 Ge 3:15 Ps 37:12-\allowbreak14 Joh 16:33}
\crossref{Rev}{12}{14}{Ex 19:4 De 32:11,\allowbreak12 Ps 55:6 Isa 40:31}
\crossref{Rev}{12}{15}{Re 17:15 Ps 18:4; 65:7; 93:3,\allowbreak4 Isa 8:7; 28:2; 59:19}
\crossref{Rev}{12}{16}{Ex 12:35,\allowbreak36 1Ki 17:6 2Ki 8:9}
\crossref{Rev}{12}{17}{12:12 Joh 8:44 1Pe 5:8}
\crossref{Rev}{12}{18}{}
\crossref{Rev}{13}{1}{Jer 5:22}
\crossref{Rev}{13}{2}{Jer 5:6; 13:23 Da 7:6,\allowbreak7 Ho 13:7 Hab 1:8}
\crossref{Rev}{13}{3}{13:1,\allowbreak2,\allowbreak14; 17:10}
\crossref{Rev}{13}{4}{13:2; 9:20 Ps 106:37,\allowbreak38 1Co 10:20-\allowbreak22 2Co 4:4}
\crossref{Rev}{13}{5}{Da 7:8,\allowbreak11,\allowbreak25; 11:36}
\crossref{Rev}{13}{6}{Job 3:1 Mt 12:34; 15:19 Ro 3:13}
\crossref{Rev}{13}{7}{Re 11:7; 12:17 Da 7:21,\allowbreak25; 8:24,\allowbreak25; 11:36-\allowbreak39; 12:1}
\crossref{Rev}{13}{8}{13:3,\allowbreak4,\allowbreak14,\allowbreak15}
\crossref{Rev}{13}{9}{Re 2:7,\allowbreak11,\allowbreak17,\allowbreak29}
\crossref{Rev}{13}{10}{Ex 21:23-\allowbreak25 Isa 14:2; 33:1 Mt 7:2}
\crossref{Rev}{13}{11}{13:1; 11:7; 17:8}
\crossref{Rev}{13}{12}{13:3,\allowbreak14-\allowbreak17; 17:10,\allowbreak11 2Th 2:4}
\crossref{Rev}{13}{13}{Re 16:14; 19:20 Ex 7:11,\allowbreak12,\allowbreak22; 8:7,\allowbreak18,\allowbreak19; 9:11 De 13:1-\allowbreak3 Mt 24:24}
\crossref{Rev}{13}{14}{Re 12:9; 18:23; 19:20; 20:3,\allowbreak10 2Ki 22:20}
\crossref{Rev}{13}{15}{Ge 2:7 Ps 135:17 Jer 10:14; 51:17 Hab 2:19 Jas 2:26}
\crossref{Rev}{13}{16}{Re 11:18; 19:5,\allowbreak18; 20:12 2Ch 15:13 Ps 115:13 Ac 26:22}
\crossref{Rev}{13}{17}{13:16}
\crossref{Rev}{13}{18}{Re 1:3; 17:9 Ps 107:43 Da 12:10 Ho 14:9 Mr 13:14}
\crossref{Rev}{14}{1}{14:14; 4:1; 6:8; 15:5 Jer 1:11 Eze 1:4; 2:9; 8:7; 10:1,\allowbreak9; 44:4 Da 12:5}
\crossref{Rev}{14}{2}{Re 10:4; 11:12,\allowbreak15; 19:1-\allowbreak7}
\crossref{Rev}{14}{3}{Re 5:9; 15:3 Ps 33:3; 40:3; 96:1; 98:1; 144:9; 149:1 Isa 42:10}
\crossref{Rev}{14}{4}{Ps 45:14 So 1:3; 6:8 Mt 25:1 1Co 7:25,\allowbreak26,\allowbreak28 2Co 11:2 1Ti 4:3}
\crossref{Rev}{14}{5}{Ps 32:2; 34:13; 55:11 Pr 8:8 Isa 53:9 Zep 3:13 Mt 12:34 Joh 1:47}
\crossref{Rev}{14}{6}{Re 8:13 Isa 6:2,\allowbreak6,\allowbreak7 Eze 1:14 Da 9:21}
\crossref{Rev}{14}{7}{Isa 40:3,\allowbreak6,\allowbreak9; 44:23; 52:7,\allowbreak8; 58:1 Ho 8:1}
\crossref{Rev}{14}{8}{Re 16:19; 17:5,\allowbreak18; 18:2,\allowbreak3,\allowbreak10,\allowbreak11,\allowbreak18-\allowbreak21 Isa 21:9 Jer 51:7,\allowbreak8,\allowbreak64}
\crossref{Rev}{14}{9}{14:6-\allowbreak8 Jer 44:4}
\crossref{Rev}{14}{10}{Re 16:19; 18:3 Job 21:20 Ps 11:6; 60:3; 75:8 Isa 29:9; 51:17,\allowbreak21,\allowbreak22}
\crossref{Rev}{14}{11}{Re 18:18; 19:3 Ge 19:28 Isa 33:14; 34:10 Joe 2:30 Lu 16:23}
\crossref{Rev}{14}{12}{Re 13:10}
\crossref{Rev}{14}{13}{Re 11:15,\allowbreak19; 16:17 Mt 3:17}
\crossref{Rev}{14}{14}{14:15,\allowbreak16; 1:7; 10:1; 20:11 Ps 97:2 Isa 19:1 Mt 17:5 Lu 21:27}
\crossref{Rev}{14}{15}{Re 16:17}
\crossref{Rev}{14}{16}{14:14 Mt 16:27 Joh 5:22,\allowbreak23}
\crossref{Rev}{14}{17}{14:14,\allowbreak15,\allowbreak18; 15:5,\allowbreak6; 16:1}
\crossref{Rev}{14}{18}{Re 6:9,\allowbreak10}
\crossref{Rev}{14}{19}{Re 19:15-\allowbreak21 De 32:32,\allowbreak33}
\crossref{Rev}{14}{20}{Isa 63:1-\allowbreak3 La 1:15}
\crossref{Rev}{15}{1}{Re 12:1-\allowbreak3 Da 4:2,\allowbreak3; 6:27}
\crossref{Rev}{15}{2}{Re 4:6; 21:18}
\crossref{Rev}{15}{3}{Ex 15:1-\allowbreak18 De 31:30; 32:1-\allowbreak43}
\crossref{Rev}{15}{4}{Ex 15:14-\allowbreak16 Ps 89:7 Isa 60:5 Jer 5:22; 10:7 Ho 3:5 Lu 12:4,\allowbreak5}
\crossref{Rev}{15}{5}{Re 11:19 Ex 25:21 Nu 1:50,\allowbreak53 Mt 27:51}
\crossref{Rev}{15}{6}{15:1}
\crossref{Rev}{15}{7}{Re 4:6-\allowbreak9}
\crossref{Rev}{15}{8}{Ex 40:34 1Ki 8:10 2Ch 5:14 Ps 18:8-\allowbreak14 Isa 6:4}
\crossref{Rev}{16}{1}{Re 14:15,\allowbreak18; 15:5-\allowbreak8}
\crossref{Rev}{16}{2}{Re 8:7; 14:16}
\crossref{Rev}{16}{3}{Re 8:8; 10:2; 13:1}
\crossref{Rev}{16}{4}{Re 8:10,\allowbreak11}
\crossref{Rev}{16}{5}{16:4}
\crossref{Rev}{16}{6}{Re 6:10,\allowbreak11; 13:10,\allowbreak15; 17:6,\allowbreak7; 18:24; 19:2 De 32:42,\allowbreak43 2Ki 24:4}
\crossref{Rev}{16}{7}{Re 6:9; 8:3-\allowbreak5; 14:18 Isa 6:6 Eze 10:2,\allowbreak7}
\crossref{Rev}{16}{8}{Re 6:12; 8:12; 9:2 Isa 24:23 Lu 21:25 Ac 2:20}
\crossref{Rev}{16}{9}{16:10,\allowbreak11,\allowbreak21 2Ki 6:33 2Ch 28:22 Isa 1:5; 8:21 Jer 5:3; 6:29,\allowbreak30}
\crossref{Rev}{16}{10}{Re 11:2,\allowbreak8; 13:2-\allowbreak4; 17:9,\allowbreak17; 18:2,\allowbreak21,\allowbreak23}
\crossref{Rev}{16}{11}{16:9,\allowbreak21}
\crossref{Rev}{16}{12}{Re 9:14; 11:14 Isa 8:7}
\crossref{Rev}{16}{13}{16:14 2Th 2:9-\allowbreak11 1Ti 4:1-\allowbreak3 2Ti 3:1-\allowbreak6 2Pe 2:1-\allowbreak3 1Jo 4:1-\allowbreak3}
\crossref{Rev}{16}{14}{Re 12:9 1Ki 22:19-\allowbreak23 2Ch 18:18-\allowbreak22 Eze 14:9 Joh 8:44 2Co 11:13-\allowbreak15}
\crossref{Rev}{16}{15}{Re 3:3 Mt 24:43 1Th 5:2,\allowbreak3 2Pe 3:10}
\crossref{Rev}{16}{16}{Re 17:14; 19:17-\allowbreak21 Jud 4:7 Joe 3:9-\allowbreak14 Zec 14:2,\allowbreak3}
\crossref{Rev}{16}{17}{Re 20:1-\allowbreak3 Eph 2:2; 6:12}
\crossref{Rev}{16}{18}{Re 4:5; 8:5; 11:19}
\crossref{Rev}{16}{19}{Re 14:8; 17:18; 18:2,\allowbreak10,\allowbreak16-\allowbreak19,\allowbreak21}
\crossref{Rev}{16}{20}{Re 6:14; 20:11 Isa 2:14-\allowbreak17 Jer 4:23-\allowbreak25}
\crossref{Rev}{16}{21}{Re 8:7; 11:19 Ex 9:23-\allowbreak26 Jos 10:11 Isa 30:30 Eze 13:11,\allowbreak13; 38:21,\allowbreak22}
\crossref{Rev}{17}{1}{Re 15:1,\allowbreak6; 17:1-\allowbreak17; 21:9}
\crossref{Rev}{17}{2}{17:13,\allowbreak17; 14:8; 18:3,\allowbreak9,\allowbreak23 Jer 51:7}
\crossref{Rev}{17}{3}{Re 1:10; 4:2; 21:10 1Ki 18:12 2Ki 2:16 Eze 3:12; 8:3; 11:24 Ac 8:39}
\crossref{Rev}{17}{4}{Re 18:7,\allowbreak12,\allowbreak16}
\crossref{Rev}{17}{5}{Re 7:3 Isa 3:9 Php 3:19}
\crossref{Rev}{17}{6}{Re 13:7,\allowbreak15; 16:6; 18:20-\allowbreak24 Da 7:21,\allowbreak25}
\crossref{Rev}{17}{7}{17:1-\allowbreak6,\allowbreak8}
\crossref{Rev}{17}{8}{17:11; 14:8-\allowbreak20; 16:1-\allowbreak18:24; 19:15-\allowbreak21; 20:10 Da 7:11,\allowbreak26; 11:45 2Th 2:3-\allowbreak8}
\crossref{Rev}{17}{9}{Re 13:18 Da 12:4,\allowbreak8-\allowbreak10 Ho 14:9 Mt 13:11; 24:15}
\crossref{Rev}{17}{10}{}
\crossref{Rev}{17}{11}{17:8}
\crossref{Rev}{17}{12}{Re 12:3; 13:1 Da 2:40-\allowbreak43; 7:7,\allowbreak8,\allowbreak20,\allowbreak24 Zec 1:18-\allowbreak21}
\crossref{Rev}{17}{13}{Php 1:27; 2:2}
\crossref{Rev}{17}{14}{Re 11:7; 8:6,\allowbreak7; 16:14; 19:15-\allowbreak21 Da 7:21,\allowbreak25; 11:9-\allowbreak12,\allowbreak24,\allowbreak25 Zec 2:8}
\crossref{Rev}{17}{15}{17:1 Ps 18:4; 65:7; 93:3-\allowbreak4 Isa 8:7,\allowbreak8 Jer 51:13,\allowbreak42,\allowbreak55}
\crossref{Rev}{17}{16}{17:2,\allowbreak10,\allowbreak12}
\crossref{Rev}{17}{17}{17:13 Ac 4:27,\allowbreak28}
\crossref{Rev}{17}{18}{Re 16:19; 18:2 Da 2:40,\allowbreak41; 7:23 Lu 2:1}
\crossref{Rev}{18}{1}{Re 17:1}
\crossref{Rev}{18}{2}{Re 1:15; 5:2; 10:3; 14:15 Jer 25:30 Joe 3:16}
\crossref{Rev}{18}{3}{18:9; 14:8; 17:2 Jer 51:7}
\crossref{Rev}{18}{4}{Ge 19:12,\allowbreak13 Nu 16:26,\allowbreak27 Isa 48:20; 52:11 Jer 50:8; 51:6,\allowbreak45,\allowbreak50}
\crossref{Rev}{18}{5}{Ge 18:20,\allowbreak21 2Ch 28:9 Ezr 9:6 Jer 51:9 Jon 1:2}
\crossref{Rev}{18}{6}{Re 13:10; 16:5,\allowbreak6 Ex 21:23-\allowbreak25 Ps 137:8 Jer 50:15,\allowbreak29; 51:24,\allowbreak49}
\crossref{Rev}{18}{7}{Isa 22:12-\allowbreak14; 47:1,\allowbreak2,\allowbreak7-\allowbreak9 Eze 28:2-\allowbreak10 Zep 2:15 2Th 2:4-\allowbreak8}
\crossref{Rev}{18}{8}{18:10,\allowbreak17,\allowbreak19 Isa 47:9-\allowbreak11 Jer 51:6}
\crossref{Rev}{18}{9}{18:3,\allowbreak7; 17:2,\allowbreak12,\allowbreak13}
\crossref{Rev}{18}{10}{Nu 16:34}
\crossref{Rev}{18}{11}{18:3,\allowbreak9,\allowbreak15,\allowbreak20,\allowbreak23; 13:16,\allowbreak17 Isa 23:1-\allowbreak15; 47:15}
\crossref{Rev}{18}{12}{Re 17:4 1Ki 10:11,\allowbreak12 Pr 8:10,\allowbreak11 Eze 27:5-\allowbreak25}
\crossref{Rev}{18}{13}{1Ki 10:10,\allowbreak15,\allowbreak25 2Ch 9:9 Pr 7:17 So 1:3; 4:13,\allowbreak14; 5:5 Am 6:6}
\crossref{Rev}{18}{14}{Nu 11:4,\allowbreak34 Ps 78:18; 106:14 1Co 10:6 Jas 4:2 1Jo 2:16,\allowbreak17}
\crossref{Rev}{18}{15}{18:3,\allowbreak11 Ho 12:7,\allowbreak8 Zec 11:5 Mr 11:17 Ac 16:19; 19:24-\allowbreak27}
\crossref{Rev}{18}{16}{18:10,\allowbreak11; 17:4 Lu 16:19 etc.}
\crossref{Rev}{18}{17}{18:10 Isa 47:9 Jer 51:8 La 4:6}
\crossref{Rev}{18}{18}{18:9}
\crossref{Rev}{18}{19}{Jos 7:6 1Sa 4:12 2Sa 13:19 Ne 9:1 Job 2:12 Eze 27:30}
\crossref{Rev}{18}{20}{Re 19:1-\allowbreak3 Jud 5:31 Ps 48:11; 58:10; 96:11-\allowbreak13; 107:42; 109:28}
\crossref{Rev}{18}{21}{Ex 15:5 Ne 9:11 Jer 51:63,\allowbreak64}
\crossref{Rev}{18}{22}{Isa 24:8,\allowbreak9 Jer 7:34; 16:9; 25:10; 33:11 Eze 26:13}
\crossref{Rev}{18}{23}{Re 22:5 Job 21:17 Pr 4:18,\allowbreak19; 24:20}
\crossref{Rev}{18}{24}{Re 11:7; 16:6; 17:6; 19:2 Jer 2:34 Eze 22:9,\allowbreak12,\allowbreak27 Da 7:21 Mt 23:27}
\crossref{Rev}{19}{1}{Re 18:1-\allowbreak24}
\crossref{Rev}{19}{2}{Re 15:3; 16:5-\allowbreak7 De 32:4 Ps 19:9 Isa 25:1}
\crossref{Rev}{19}{3}{19:1}
\crossref{Rev}{19}{4}{Re 4:4-\allowbreak10; 5:8-\allowbreak11,\allowbreak14; 11:15,\allowbreak16; 15:7}
\crossref{Rev}{19}{5}{Re 7:15; 11:19; 16:17}
\crossref{Rev}{19}{6}{Re 1:15; 14:2 Eze 1:24; 43:2}
\crossref{Rev}{19}{7}{De 32:43 1Sa 2:1 Ps 9:14; 48:11; 95:1-\allowbreak3; 100:1,\allowbreak2; 107:42 Pr 29:2}
\crossref{Rev}{19}{8}{Re 3:4,\allowbreak5,\allowbreak18 Ps 45:13,\allowbreak14 Isa 61:10 Eze 16:10 Mt 22:12 Ro 3:22}
\crossref{Rev}{19}{9}{Re 1:19; 2:1,\allowbreak8,\allowbreak12,\allowbreak18; 3:1,\allowbreak7,\allowbreak14; 10:4; 14:13 Isa 8:1 Hab 2:2}
\crossref{Rev}{19}{10}{Re 22:8,\allowbreak9 Mr 5:22; 7:25 Ac 10:25,\allowbreak26; 14:11-\allowbreak15 1Jo 5:21}
\crossref{Rev}{19}{11}{Re 4:1; 11:19; 15:5}
\crossref{Rev}{19}{12}{Re 1:14; 2:18}
\crossref{Rev}{19}{13}{Re 14:20 Ps 58:10 Isa 9:5; 34:3-\allowbreak8; 63:1-\allowbreak6}
\crossref{Rev}{19}{14}{Re 14:1,\allowbreak20; 17:14 Ps 68:17; 149:6-\allowbreak9 Zec 14:5 Mt 26:53 2Th 1:7}
\crossref{Rev}{19}{15}{19:21; 1:16; 2:12,\allowbreak16 Isa 11:4; 30:33 2Th 2:8}
\crossref{Rev}{19}{16}{19:12,\allowbreak13}
\crossref{Rev}{19}{17}{Re 8:13; 14:6 Isa 34:1-\allowbreak8}
\crossref{Rev}{19}{18}{De 28:26 1Sa 17:44,\allowbreak46 Ps 110:5,\allowbreak6 Jer 7:33; 16:4; 19:7; 34:20}
\crossref{Rev}{19}{19}{Re 13:1-\allowbreak10; 14:9; 16:14,\allowbreak16; 17:12-\allowbreak14; 18:9 Eze 38:8-\allowbreak18 Da 7:21-\allowbreak26}
\crossref{Rev}{19}{20}{19:19; 13:1-\allowbreak8,\allowbreak18; 17:3-\allowbreak8,\allowbreak12 Da 2:40-\allowbreak45; 7:7,\allowbreak12-\allowbreak14,\allowbreak19-\allowbreak21,\allowbreak23}
\crossref{Rev}{19}{21}{19:11-\allowbreak15; 1:16}
\crossref{Rev}{20}{1}{Re 10:1; 18:1}
\crossref{Rev}{20}{2}{Ge 3:15 Isa 27:1; 49:24,\allowbreak25 Mt 8:29; 19:29 Mr 5:7 Lu 11:20-\allowbreak22}
\crossref{Rev}{20}{3}{20:1; 17:8}
\crossref{Rev}{20}{4}{Da 7:9,\allowbreak18,\allowbreak22,\allowbreak27 Mt 19:28 Lu 22:30 1Co 6:2,\allowbreak3}
\crossref{Rev}{20}{5}{20:8,\allowbreak9; 19:20,\allowbreak21}
\crossref{Rev}{20}{6}{20:5; 14:13; 22:7 Isa 4:3 Da 12:12 Lu 14:15}
\crossref{Rev}{20}{7}{20:2}
\crossref{Rev}{20}{8}{20:3,\allowbreak10}
\crossref{Rev}{20}{9}{Isa 8:7,\allowbreak8 Eze 38:9,\allowbreak16 Hab 1:6}
\crossref{Rev}{20}{10}{20:2,\allowbreak3,\allowbreak8}
\crossref{Rev}{20}{11}{20:2; 19:11 Ge 18:25 Ps 9:7,\allowbreak8; 14:6,\allowbreak7; 47:8; 89:14; 97:2 Mt 25:31}
\crossref{Rev}{20}{12}{20:11 Da 12:2 Joh 5:28,\allowbreak29; 11:25,\allowbreak26 Ac 24:15 1Co 15:21-\allowbreak23}
\crossref{Rev}{20}{13}{Joh 5:28,\allowbreak29}
\crossref{Rev}{20}{14}{Re 19:20 Ho 13:14 1Co 15:26,\allowbreak53}
\crossref{Rev}{20}{15}{Mr 16:16 Joh 3:18,\allowbreak19,\allowbreak36; 14:6 Ac 4:12 Heb 2:3; 12:25 1Jo 5:11,\allowbreak12}
\crossref{Rev}{21}{1}{21:5 Isa 65:17-\allowbreak19; 66:22 2Pe 3:13}
\crossref{Rev}{21}{2}{Re 1:1,\allowbreak4,\allowbreak9}
\crossref{Rev}{21}{3}{Re 10:4,\allowbreak8; 12:10}
\crossref{Rev}{21}{4}{Re 7:17 Isa 25:8}
\crossref{Rev}{21}{5}{Re 4:2,\allowbreak9; 5:1; 20:11}
\crossref{Rev}{21}{6}{Re 16:17}
\crossref{Rev}{21}{7}{Re 2:11,\allowbreak17,\allowbreak25}
\crossref{Rev}{21}{8}{De 20:8 Jud 7:3 Isa 51:12; 57:11 Mt 8:26; 10:28 Lu 12:4-\allowbreak9}
\crossref{Rev}{21}{9}{Re 15:1-\allowbreak7; 16:1-\allowbreak17}
\crossref{Rev}{21}{10}{Re 1:10; 4:2; 17:3 1Ki 18:12 2Ki 2:16 Eze 3:14; 8:3; 11:1,\allowbreak24; 40:1-\allowbreak3}
\crossref{Rev}{21}{11}{21:22,\allowbreak23; 22:5 Isa 4:5; 60:1,\allowbreak2,\allowbreak19,\allowbreak20 Eze 48:35}
\crossref{Rev}{21}{12}{21:17-\allowbreak20 Ezr 9:9 Ne 12:27 Ps 51:18; 122:7}
\crossref{Rev}{21}{13}{Eze 48:31-\allowbreak34}
\crossref{Rev}{21}{14}{21:19-\allowbreak21 Isa 54:11 Heb 11:10}
\crossref{Rev}{21}{15}{Re 11:1,\allowbreak2 Ex 40:3-\allowbreak5 Eze 41:1 etc.}
\crossref{Rev}{21}{16}{Eze 48:8-\allowbreak19}
\crossref{Rev}{21}{17}{Re 7:4; 14:3}
\crossref{Rev}{21}{18}{21:11,\allowbreak19}
\crossref{Rev}{21}{19}{Job 28:16-\allowbreak19 Pr 3:15 Isa 54:11,\allowbreak12}
\crossref{Rev}{21}{20}{}
\crossref{Rev}{21}{21}{21:12; 17:4 Mt 13:45,\allowbreak46}
\crossref{Rev}{21}{22}{21:4,\allowbreak5 1Ki 8:27 2Ch 2:6; 6:18 Isa 66:1 Joh 4:23}
\crossref{Rev}{21}{23}{21:11; 22:5 Isa 24:23; 60:19,\allowbreak20}
\crossref{Rev}{21}{24}{Re 22:2 De 32:43 Ps 22:27 Isa 2:2; 52:15; 55:5,\allowbreak10; 66:12,\allowbreak18 Jer 4:2}
\crossref{Rev}{21}{25}{Isa 60:11}
\crossref{Rev}{21}{26}{21:24}
\crossref{Rev}{21}{27}{Le 13:46 Nu 5:3; 12:15 Ps 101:8 Isa 35:8; 52:1; 60:21 Joe 3:17}
\crossref{Rev}{22}{1}{Ps 36:8; 46:4 Isa 41:18; 48:18; 66:12 Eze 47:1-\allowbreak9 Zec 14:8}
\crossref{Rev}{22}{2}{22:1; 21:21 Eze 47:1,\allowbreak12}
\crossref{Rev}{22}{3}{Re 21:4 De 27:26 Zec 14:11 Mt 25:41 Ge 3:10-\allowbreak13}
\crossref{Rev}{22}{4}{Eze 33:18-\allowbreak20,\allowbreak23 Job 33:26 Ps 4:6 Isa 33:17; 35:2; 40:5 Mt 5:8}
\crossref{Rev}{22}{5}{Re 18:23}
\crossref{Rev}{22}{6}{Re 19:9; 21:5}
\crossref{Rev}{22}{7}{22:10,\allowbreak12,\allowbreak20}
\crossref{Rev}{22}{8}{Re 19:10,\allowbreak19}
\crossref{Rev}{22}{9}{Re 19:10 De 4:19 Col 2:18,\allowbreak19 1Jo 5:20}
\crossref{Rev}{22}{10}{22:12,\allowbreak13,\allowbreak16,\allowbreak20}
\crossref{Rev}{22}{11}{Re 16:8-\allowbreak11,\allowbreak21 Ps 81:12 Pr 1:24-\allowbreak33; 14:32 Ec 11:3 Eze 3:27 Da 12:10}
\crossref{Rev}{22}{12}{22:7 Zep 1:14}
\crossref{Rev}{22}{13}{Re 1:8,\allowbreak11; 21:6 Isa 41:4; 44:6; 48:12}
\crossref{Rev}{22}{14}{22:7 Ps 106:3-\allowbreak5; 112:1; 119:1-\allowbreak6 Isa 56:1,\allowbreak2 Da 12:12 Mt 7:21-\allowbreak27}
\crossref{Rev}{22}{15}{Re 9:20,\allowbreak21; 21:8,\allowbreak27 1Co 6:9,\allowbreak10 Ga 5:19-\allowbreak21 Eph 5:3-\allowbreak6 Col 3:6}
\crossref{Rev}{22}{16}{22:6; 1:1}
\crossref{Rev}{22}{17}{22:16 Isa 55:1-\allowbreak3 Joh 16:7-\allowbreak15}
\crossref{Rev}{22}{18}{22:16; 3:14 Eph 4:17 1Th 4:6}
\crossref{Rev}{22}{19}{Re 22:18 Lu 11:52}
\crossref{Rev}{22}{20}{22:18}
\crossref{Rev}{22}{21}{Re 1:18 So 8:14 Isa 25:9 Joh 21:25 2Ti 4:8 Heb 9:28 2Pe 3:12-\allowbreak14}


\begin{document}

\setcounter{tocdepth}{3}
\tableofcontents*
\newpage

\preresetbooktab
\def\thetestament{Old Testament}
\addcontentsline{toc}{part}{Old Testament}
\addcontentsline{toc}{chapter}{The Law}
\bookheader{Genesis}
\labelbook{Gen}

\bookpretitle{The First Book of the Law called}
\booktitle{Genesis}

\labelchapt{1}
\passage{The Creation}

\chapt{1}
\v{1}In the beginning, God created the universe.\fnote{\fbackref{1:1} Lit. \fbib{the heavens and the earth}; i.e. space and matter} \v{2}When the earth\fnote{\fbackref{1:2} Or \fbib{\v{1}When God began to create the universe,} \fbib{\v{2}the earth}} was as yet unformed and desolate, with the surface of the ocean depths shrouded\fnote{\fbackref{1:2} The Heb. lacks \fbib{shrouded}} in darkness, and while the Spirit of God was hovering\fnote{\fbackref{1:2} Or \fbib{brooding}} over the surface of the waters, \v{3}God said, ``Let there be light!'' So there was light.

\v{4}God saw that the light was beautiful.\fnote{\fbackref{1:4} Or \fbib{good}} He\fnote{\fbackref{1:5} Lit. \fbib{God}} separated the light from the darkness, \v{5}calling the light ``day,'' and the darkness\fnote{\fbackref{1:5} Lit. \fbib{darkness he called}} ``night.'' The twilight and the dawn were day one.

\v{6}Then God said, ``Let there be a canopy\fnote{\fbackref{1:6} Or \fbib{an expanse}} between bodies of water,\fnote{\fbackref{1:6} Lit. \fbib{between waters}} separating bodies of water\fnote{\fbackref{1:6} Lit. \fbib{separating waters}} from bodies of water!''\fnote{\fbackref{1:6} Lit. \fbib{from waters}} \v{7}So God made a canopy\fnote{\fbackref{1:7} Or \fbib{made an expanse}} that separated the water beneath the canopy\fnote{\fbackref{1:7} Or \fbib{expanse}} from the water above it.\fnote{\fbackref{1:7} Lit. \fbib{above the canopy}} And that is what happened:\fnote{\fbackref{1:7} Lit. \fbib{And so it was}} \v{8}God called the canopy\fnote{\fbackref{1:8} Or \fbib{expanse}} ``sky.''\fnote{\fbackref{1:8} Or \fbib{Heaven}} The twilight and the dawn were the second day.

\v{9}Then God said, ``Let the water beneath the sky come together into one area, and let dry ground appear!'' And that is what happened:\fnote{\fbackref{1:9} Lit. \fbib{And so it was}} \v{10}God called the dry ground ``land,''\fnote{\fbackref{1:10} Or \fbib{Earth}} and he called the water that had come together ``oceans.'' And God saw how good it was.

\v{11}Then God said, ``Let vegetation sprout all over the earth, including\fnote{\fbackref{1:11} The Heb. lacks \fbib{including}} seed-bearing plants and fruit trees, each kind containing its own seed!'' And that is what happened:\fnote{\fbackref{1:11} Lit. \fbib{And so it was}} \v{12}Vegetation sprouted all over the earth, including seed-bearing plants and fruit trees, each kind containing its own seed. And God saw that it was good. \v{13}The twilight and the dawn were the third day.

\v{14}Then God said, ``Let there be lights across\fnote{\fbackref{1:14} Lit. \fbib{lights in the expanse of}} the sky to distinguish day from night, to act as signs for seasons, days, and years, \v{15}to serve as lights in\fnote{\fbackref{1:15} Lit. \fbib{lights in the expanse of}} the sky, and to shine on the earth!'' And that is what happened:\fnote{\fbackref{1:15} Lit. \fbib{And so it was}} \v{16}God fashioned two great lights---the larger light to shine during\fnote{\fbackref{1:16} Lit. \fbib{to govern}} the day and the smaller light to shine during\fnote{\fbackref{1:16} Lit. \fbib{to govern}} the night---as well as stars. \v{17}God placed them in space\fnote{\fbackref{1:17} Lit. \fbib{the expanse of the sky}} to shine on the earth, \v{18}to differentiate between\fnote{\fbackref{1:18} Lit. \fbib{to govern}} day and night, and to distinguish\fnote{\fbackref{1:18} Or \fbib{separate}} light from darkness. And God saw how good it was. \v{19}The twilight and the dawn were the fourth day.

\v{20}Then God said, ``Let the oceans swarm\fnote{\fbackref{1:20} Lit. \fbib{swarm with a swarm}} with living creatures, and let flying creatures soar above the earth throughout\fnote{\fbackref{1:20} Lit. \fbib{earth in the expanse of}} the sky!'' \v{21}So God created every kind of magnificent marine creature, every kind of living marine crawler\fnote{\fbackref{1:21} Lit. \fbib{living thing that crawls}} with which the waters swarmed, and every kind of flying creature.\fnote{\fbackref{1:21} Lit. \fbib{winged bird}} And God saw how good it was. \v{22}God blessed them by saying, ``Be fruitful, multiply, and fill the oceans. Let the birds multiply throughout the earth!'' \v{23}The twilight and the dawn were the fifth day.

\v{24}Then God said, ``Let the earth bring forth each kind of living creature, each kind of livestock and crawling thing, and each kind of earth's animals!''\fnote{\fbackref{1:24} I.e., non-domesticated animals, as opposed to domesticated livestock; and so through 2:25} And that is what happened:\fnote{\fbackref{1:24} Lit. \fbib{And so it was}} \v{25}God made each kind of the earth's animals, along with every kind of livestock and crawling thing.\fnote{\fbackref{1:25} Lit. \fbib{thing of the earth}} And God saw how good it was.

\v{26}Then God said, ``Let us make mankind in our image, to be like us.\fnote{\fbackref{1:26} Lit. \fbib{image, according to our likeness}} Let them be masters over the fish in the ocean, the birds that fly,\fnote{\fbackref{1:26} Lit. \fbib{birds of the sky}; and so through 2:25} the livestock, everything that crawls on the earth, and over the earth itself!''

\begin{poetry}
\poeml \v{27}So God created mankind in his own image; \\
\poemll    in his own image God created them;\fnote{\fbackref{1:27} Lit. \fbib{him}} \\
\poemlll       he created them male and female.
\end{poetry}

\v{28}God blessed the humans by saying to them, ``Be fruitful, multiply, fill the earth, and subdue it! Be masters over the fish in the ocean, the birds that fly, and every living thing that crawls on the earth!''

\v{29}God also told them,\fnote{\fbackref{1:29} The Heb. lacks \fbib{them}} ``Look! I have given you every seed-bearing plant that grows throughout\fnote{\fbackref{1:29} Lit. \fbib{plant that is on the surface of}} the earth, along with every tree that grows seed-bearing fruit. They will produce your food. \v{30}I have given all green plants as food for every wild animal\fnote{\fbackref{1:30} I.e., non-domesticated animals, as opposed to domesticated livestock} of the earth, every bird that flies, and to every living thing\fnote{\fbackref{1:30} Lit. \fbib{soul}} that crawls on the earth.'' And that is what happened.\fnote{\fbackref{1:30} Lit. \fbib{And so it was}}

\v{31}Now God saw all that he had made, and indeed, it was very good! The twilight and the dawn were the sixth day.
\labelchapt{2}
\passage{The Seventh Day}

\chapt{2}
\v{1}With this, the universe\fnote{\fbackref{2:1} Lit. \fbib{the heavens and the earth}; i.e. space and matter} was\fnote{\fbackref{2:1} Lit. \fbib{were}} completed, including all of its vast array.\fnote{\fbackref{2:1} Lit. \fbib{of their hosts}; i.e. armies of sentient beings (or stars, if referring to the night sky)} \v{2}By the seventh day, God had completed the work he had been doing, so on the seventh day he stopped working on\fnote{\fbackref{2:2} Or \fbib{he rested from}} everything that he had done. \v{3}Then God blessed the seventh day and made it holy, because on it God stopped working on\fnote{\fbackref{2:3} Or \fbib{God rested from}} everything that he had been creating.
\passage{Humans in the Garden}

\v{4}These are the records of the universe at its\fnote{\fbackref{2:4} Lit. \fbib{the heavens and the earth at their}} creation. On the day that the \divine{Lord} God made the universe,\fnote{\fbackref{2:4} Lit. \fbib{the earth and the heavens}; or \fbib{the earth and space}} \v{5}no shrubs had yet grown in the meadows of the earth and no vegetation had sprouted,\fnote{\fbackref{2:5} Lit. \fbib{sprouted in the fields}} because the \divine{Lord} God had not sent rain on the earth and there were no human beings\fnote{\fbackref{2:5} Lit. \fbib{there was no man}} to work the ground. \v{6}Instead, an underground stream\fnote{\fbackref{2:6} Or \fbib{mist}} would arise out of the earth and water the surface of the ground. \v{7}So the \divine{Lord} God formed the man from the dust of the ground, breathed life into his lungs,\fnote{\fbackref{2:7} Lit. \fbib{nostrils}} and the man became a living being.

\v{8}The \divine{Lord} God planted a garden in Eden, toward\fnote{\fbackref{2:8} Lit. \fbib{in}} the east, where he placed the man whom he had formed. \v{9}The \divine{Lord} God caused every tree that is both beautiful\fnote{\fbackref{2:9} Lit. \fbib{is pleasing to the eyes}} and suitable for food to spring up out of the ground. The tree of life was also in the middle of the garden, along with the tree of the knowledge of good and evil. \v{10}A river flows from Eden to water the garden, and from there it divides, becoming four branches. \v{11}The name of the first one is Pishon---it winds through the entire land of Havilah,\fnote{\fbackref{2:11} Possibly a reference to Nubia, a source of gold for ancient Egypt} where there is gold. \v{12}The gold of that land is pure;\fnote{\fbackref{2:12} Lit. \fbib{good}} bdellium\fnote{\fbackref{2:12} I.e. a gum resin; or \fbib{pearl}} and onyx are also found\fnote{\fbackref{2:12} The Heb. lacks \fbib{also found}} there. \v{13}The name of the second river is Gihon---\fnote{\fbackref{2:13} Possibly an ancient reference to one of the branches of the Nile River} it winds through the entire land of Cush.\fnote{\fbackref{2:13} Possibly a portion of northeast Africa} \v{14}The third river is named the Tigris--- it flows to the east of Assyria. The fourth river is the Euphrates.

\v{15}The \divine{Lord} God took the man and placed him in the Garden of Eden in order to have him work it and guard\fnote{\fbackref{2:15} Or \fbib{and watch over}} it. \v{16}The \divine{Lord} God commanded the man: ``You may freely eat from every tree of the garden, \v{17}but you are not to eat from the tree of the knowledge of good and evil, because you will certainly die during the day that you eat from it.''
\passage{The Creation of the Woman}

\v{18}Later, the \divine{Lord} God said, ``It is not good for the man to be alone. I will make the woman\fnote{\fbackref{2:18} The Heb. lacks \fbib{the woman}} to be an authority\fnote{\fbackref{2:18} Or \fbib{make a strength}; or \fbib{make a power}} corresponding\fnote{\fbackref{2:18} Or \fbib{equal}} to him.'' \v{19}After the \divine{Lord} God formed from the ground every wild animal\fnote{\fbackref{2:19} Lit. \fbib{every animal of the field}; i.e., non-domesticated animals, as opposed to domesticated livestock} and every bird that flies, he brought each of them\fnote{\fbackref{2:19} The Heb. lacks \fbib{each of them}} to the man to see what he would call it. Whatever the man called each living creature became its name. \v{20}The man gave names to all the livestock, to the birds that fly, and to each of earth's animals,\fnote{\fbackref{2:20} I.e., non-domesticated animals, as opposed to domesticated livestock} but there was not found a strength\fnote{\fbackref{2:20} Or \fbib{found an authority} or \fbib{found a power}} corresponding\fnote{\fbackref{2:20} Or \fbib{equal}} to him, \v{21}so the \divine{Lord} God caused a deep sleep to overshadow the man.

When the man\fnote{\fbackref{2:21} Lit. \fbib{When he}} was asleep, he removed one of the man's\fnote{\fbackref{2:21} Lit. \fbib{of his}} ribs and closed up the flesh where it had been. \v{22}Then the \divine{Lord} God formed the rib that he had taken from the man into a woman and brought her to the man. \v{23}So the man exclaimed,

\begin{poetry}
\poeml ``At last! This is \\
\poemll    bone from my bones \\
\poemlll       and flesh from my flesh. \\
\poeml This one will be called `Woman,' \\
\poemll    because she was taken from Man.''\fnote{\fbackref{2:23} The Heb. roots for \fbib{Man} and \fbib{Woman} are identical.}
\end{poetry}

\v{24}(Therefore a man will leave his father and his mother and cling to his wife, and they will become one flesh.) \v{25}Even though both the man and his wife were naked, they were not ashamed about it.\fnote{\fbackref{2:25} The Heb. lacks \fbib{about it}}
\labelchapt{3}
\passage{The Temptation and Fall}

\chapt{3}
\v{1}Now the Shining One\fnote{\fbackref{3:1} The Heb. word \fbib{Ha-Nachash} means \fbib{the Shining One}; or \fbib{the Diviner}; i.e. one who falsely claims to reveal God's word; or \fbib{the Serpent}; and so through 3:14; cf. Isa 14:12; Eze 28:13-14} was more clever than any animal of the field that the \divine{Lord} God had made. It\fnote{\fbackref{3:1} Lit. \fbib{And it}} asked the woman, ``Did God actually say, `You are not to eat from any tree of the garden'?''

\v{2}``We may eat from the trees of the garden,'' the woman answered the Shining One, \v{3}``but as for the fruit of the tree that is in the middle of the garden, God has said, `You are not to eat from it, nor are you to touch it, or you will die.'\,''

\v{4}``You certainly will not die!'' the Shining One told the woman. \v{5}``Even God knows that on the day you eat from it, your eyes will be opened and you'll become like God,\fnote{\fbackref{3:5} Or \fbib{gods}} knowing good and evil.''

\v{6}When the woman saw that the tree produced good food, was attractive in appearance,\fnote{\fbackref{3:6} Lit. \fbib{was pleasing to the eyes}} and was desirable for making one wise, she took some of its fruit and ate it.\fnote{\fbackref{3:6} The Heb. lacks \fbib{it}} Then she also gave some to her husband who was with her, and he ate some, too.\fnote{\fbackref{3:6} The Heb. lacks \fbib{some, too}} \v{7}As a result, they both understood what they had done,\fnote{\fbackref{3:7} Lit. \fbib{the eyes of both of them were opened}} and they became aware that they were naked. So they sewed fig leaves together and made loincloths for themselves.

\v{8}When they heard the voice of the \divine{Lord} God as he was walking in the garden during the breeze of the day, the man and his wife concealed themselves from the presence of the \divine{Lord} God among the trees of the garden. \v{9}So the \divine{Lord} God called out to the man, asking him, ``Where are you?''

\v{10}``I heard your voice in the garden,'' the man\fnote{\fbackref{3:10} Lit. \fbib{he}} answered, ``and I was afraid because I was naked, so I hid from you.''\fnote{\fbackref{3:10} The Heb. lacks \fbib{from you}}

\v{11}``Who told you that you are naked?'' God\fnote{\fbackref{3:11} Lit. \fbib{he}} asked. ``Did you eat fruit\fnote{\fbackref{3:11} The Heb. lacks \fbib{fruit}} from the tree that I commanded you not to eat?''

\v{12}The man answered, ``The woman whom you provided for\fnote{\fbackref{3:12} Or \fbib{you gave}} me gave me fruit\fnote{\fbackref{3:12} The Heb. lacks \fbib{fruit}} from the tree, and I ate some of it.''\fnote{\fbackref{3:12} The Heb. lacks \fbib{some of it}}

\v{13}Then the \divine{Lord} God asked the woman, ``What did you do?''\fnote{\fbackref{3:13} Lit. \fbib{What is this you did?}}

``The Shining One misled me,'' the woman answered, ``so I ate.''
\passage{The Penalty of Sin}

\v{14}The \divine{Lord} God told the Shining One,

\begin{poetry}
\poeml ``Because you have done this, \\
\poemll    you are more cursed than all the livestock, \\
\poemlll       and more than all the earth's animals,\fnote{\fbackref{3:14} I.e., non-domesticated animals, as opposed to domesticated livestock} \\
\poeml You'll crawl on your belly \\
\poemll    and eat dust \\
\poemlll       as long as you live. \\
\poeml \v{15}``I'll place hostility between you and the woman, \\
\poemll    between your offspring and her offspring. \\
\poeml He'll strike you on the head, \\
\poemll    and you'll strike him on the heel.''
\end{poetry}

\v{16}He told the woman,

\begin{poetry}
\poeml ``I'll greatly increase the pain of your labor during childbirth. \\
\poemll    It will be painful for you to bear children, \\
\poeml ``since your trust is turning\fnote{\fbackref{3:16} Or \fbib{Your desire is}} toward your husband, \\
\poemll    and he will dominate you.''
\end{poetry}

\v{17}He told the man,

\begin{poetry}
\poeml ``Because you have listened to what your wife said,\fnote{\fbackref{3:17} Lit. \fbib{to the voice of your wife}} \\
\poemll    and have eaten from the tree about which I commanded you,\fnote{\fbackref{3:17} Lit. \fbib{you when I said}} \\
\poemlll       `You are not to not eat from it,' \\
\poeml cursed is the ground because of you. \\
\poemll    You'll eat from it through pain-filled labor \\
\poemlll       for the rest of your life. \\
\poeml \v{18}It will produce thorns and thistles for you, \\
\poemll    and you'll eat the plants from the meadows. \\
\poeml \v{19}You will eat food by the sweat of your brow \\
\poemll    until you're buried in\fnote{\fbackref{3:19} Lit. \fbib{you return to}} the ground, \\
\poemlll       because you were taken from it. \\
\poeml You're made from dust \\
\poemll    and you'll return to dust.''
\end{poetry}

\v{20}Now Adam\fnote{\fbackref{3:20} Or \fbib{the man}} had named his wife ``Eve,''\fnote{\fbackref{3:20} The Heb. name \fbib{Hawwa} (\fbib{Eve}) means \fbib{life}.} because she was to become the mother of everyone who was living. \v{21}The \divine{Lord} God fashioned garments from animal skins for Adam and his wife, and clothed them.

\v{22}Later, the \divine{Lord} God said, ``Look! The man has become like one of us in knowing good and evil. Now, so he won't reach out, also take from the tree of life, eat, and then live forever---'' \v{23}therefore the \divine{Lord} God expelled the man\fnote{\fbackref{3:23} Lit. \fbib{expelled him}} from the garden of Eden so he would work the ground from which he had been taken. \v{24}After he had expelled the man, the \divine{Lord} God\fnote{\fbackref{3:24} Lit. \fbib{man, he}} placed winged angels\fnote{\fbackref{3:24} MT reads \fbib{placed cherubim}} at the eastern end of the garden of Eden, along with a fiery, turning sword, to prevent access to\fnote{\fbackref{3:24} Or \fbib{to watch over}} the tree of life.
\labelchapt{4}
\passage{Cain and Abel}

\chapt{4}
\v{1}Later, Adam\fnote{\fbackref{4:1} Or \fbib{the man}} had sexual relations with\fnote{\fbackref{4:1} Lit. \fbib{Adam knew}} his wife Eve. She became pregnant and gave birth to Cain. She said, ``I have given birth to\fnote{\fbackref{4:1} Or \fbib{have acquired}; the Heb. verb resembles the word for \fbib{Cain}} a male child---the \divine{Lord}.''\fnote{\fbackref{4:1} Or \fbib{child with the \divine{Lord}}; the Heb. lacks \fbib{with}} \v{2}And she did it again, giving birth to his brother Abel. Abel shepherded flocks and Cain became a farmer.\fnote{\fbackref{4:2} Lit. \fbib{a worker of the ground}}

\v{3}Later, after a while, Cain brought an offering to the \divine{Lord} from the fruit that he had harvested,\fnote{\fbackref{4:3} Lit. \fbib{fruit of the ground}} \v{4}while Abel brought the best parts\fnote{\fbackref{4:4} Lit. \fbib{the fatty portions}} of some of the firstborn from his flock. The \divine{Lord} looked favorably upon Abel and his offering, \v{5}but he did not look favorably upon Cain and his offering.

When Cain became very upset and\fnote{\fbackref{4:5} Lit. \fbib{and his face was}} depressed, \v{6}the \divine{Lord} asked Cain, ``Why are you so upset? Why are you\fnote{\fbackref{4:6} Lit. \fbib{Why is your face}} depressed? \v{7}If you do what is appropriate,\fnote{\fbackref{4:7} Or \fbib{good}} you'll be accepted, won't you? But if you don't do what is appropriate,\fnote{\fbackref{4:7} Or \fbib{good}} sin is crouching near your doorway, turning toward you. Now as for you, will you take dominion over it?''\fnote{\fbackref{4:7} Or \fbib{However, you must take dominion over it.}}

\v{8}Instead, Cain told his brother Abel, ``Let's go out to the wilderness.''\fnote{\fbackref{4:8} So with SP, LXX, Vg, and Syr; the Heb. lacks \fbib{Let's go out to the wilderness.}} When they were outside in the fields, Cain attacked his brother Abel and killed him.

\v{9}Later, the \divine{Lord} asked Cain, ``Where's your brother Abel?''

``I don't know,'' he answered. ``Am I my brother's guardian?''

\v{10}``What did you do?'' God\fnote{\fbackref{4:10} Lit. \fbib{he}} asked. ``Your brother's blood cries out to me from the ground. \v{11}Now you're more cursed than the ground, which has opened\fnote{\fbackref{4:11} Lit. \fbib{opened its mouth}} to receive your brother's blood from your hand. \v{12}Whenever you work the ground, it will no longer yield its produce to you, and you'll wander throughout the earth as a fugitive.''

\v{13}``My punishment is too great to bear,'' Cain told the \divine{Lord}. \v{14}``You're driving me from the soil\fnote{\fbackref{4:14} Lit. \fbib{the face of the ground}} today. I'll be hidden from you, and I'll wander throughout the earth as a fugitive. In the future,\fnote{\fbackref{4:14} Lit. \fbib{So it will be that}} whoever finds me will kill me.''

\v{15}The \divine{Lord} told him, ``This won't happen, because whoever kills you\fnote{\fbackref{4:15} Lit. \fbib{Cain}} will suffer seven times the vengeance.'' Then the \divine{Lord} placed a sign on Cain so that no one finding him would kill him. \v{16}After this, Cain left the presence of the \divine{Lord} and settled in the land of Nod, east of Eden.
\passage{From Cain to Lamech}

\v{17}Later, Cain had sexual relations with\fnote{\fbackref{4:17} Lit. \fbib{Cain knew}} his wife. She became pregnant and gave birth to Enoch. Cain\fnote{\fbackref{4:17} Lit. \fbib{He}} founded a city and named it after\fnote{\fbackref{4:17} Lit. \fbib{called its name after the name of}} his son Enoch. \v{18}Irad was born to Enoch. Irad fathered Mehujael, and Mehujael fathered Methushael, and Methushael fathered Lamech. \v{19}Later, Lamech married two wives. One was named Adah and the other was named\fnote{\fbackref{4:19} Lit. \fbib{the name of the second was}} Zillah. \v{20}Adah gave birth to Jabal, who became the ancestor of those who live in tents and herd\fnote{\fbackref{4:20} The Heb. lacks \fbib{herd}} livestock. \v{21}His brother was named Jubal; he became the ancestor of all those who play the lyre and the flute. \v{22}Zillah gave birth to Tubal-cain, who became a forger of bronze and iron work. Tubal-cain's sister was Naamah. \v{23}Lamech told his wives,

\begin{poetry}
\poeml ``Adah and Zillah, listen to what I have to say: \\
\poemll    You wives of Lamech, hear what I'm announcing! \\
\poeml I've killed a man for wounding me, \\
\poemll    a young man for bruising me. \\
\poeml \v{24}For if Cain is being avenged seven times, \\
\poemll    then Lamech will be avenged\fnote{\fbackref{4:24} The Heb. lacks \fbib{will be avenged}} 77 times.''
\end{poetry}

\v{25}Later on, after Adam had sexual relations with\fnote{\fbackref{4:25} Lit. \fbib{Adam knew}} his wife, she gave birth to a son and named him\fnote{\fbackref{4:25} Lit. \fbib{called his name}} Seth, because

\begin{poetry}
\poeml ``God granted\fnote{\fbackref{4:25} The Heb. verb \fbib{granted} resembles the word \fbib{Seth}} me another offspring to replace Abel, \\
\poemll    since Cain murdered him.''
\end{poetry}

\v{26}Seth also fathered a son, whom he named Enosh. At that time, profaning\fnote{\fbackref{4:26} Or \fbib{invoking;} lit. \fbib{calling on}} the name of the \divine{Lord} began.
\labelchapt{5}
\passage{From Adam to Noah}

\chapt{5}
\v{1}This is the historical record\fnote{\fbackref{5:1} Or \fbib{the generations scroll}} of Adam's\fnote{\fbackref{5:1} Or \fbib{mankind's}} generations.

\begin{poetry}
\poeml When\fnote{\fbackref{5:1} Lit. \fbib{On the day that}} God created mankind,\fnote{\fbackref{5:1} Lit. \fbib{Adam}} \\
\poemll    he made them in his own likeness.\fnote{\fbackref{5:1} Lit. \fbib{in the likeness of God}} \\
\poeml \v{2}Creating them male and female, \\
\poemll    he blessed them \\
\poeml and called them humans\fnote{\fbackref{5:2} Lit. \fbib{called their name Adam}} \\
\poemll    when\fnote{\fbackref{5:2} Lit. \fbib{on the day he created them}} he created them.
\end{poetry}

\v{3}After Adam had lived 130 years, he fathered a son just like him,\fnote{\fbackref{5:3} Lit. \fbib{son in his likeness}} that is,\fnote{\fbackref{5:3} The Heb. lacks \fbib{that is}} according to his own likeness, and named him Seth. \v{4}Adam lived another 800 years, fathering other\fnote{\fbackref{5:4} The Heb. lacks \fbib{other}; and so throughout the chapter} sons and daughters after he had fathered Seth. \v{5}Adam lived a total\fnote{\fbackref{5:5} Lit. \fbib{all the days}; and so throughout the chapter} of 930 years, and then died.

\v{6}When Seth had lived 105 years, he fathered Enosh. \v{7}After he fathered Enosh, Seth lived 807 years, fathering other sons and daughters. \v{8}Seth lived a total of 912 years, and then died.

\v{9}When Enosh had lived 90 years, he fathered Kenan. \v{10}After he fathered Kenan, Enosh lived 815 years, fathering other sons and daughters. \v{11}Enosh lived a total of 905 years, and then died.

\v{12}When Kenan had lived 70 years, he fathered Mahalalel. \v{13}After he fathered Mahalalel, Kenan lived 840 years, fathering other sons and daughters. \v{14}Kenan lived a total of 910 years, and then died.

\v{15}When Mahalalel had lived 65 years, he fathered Jared. \v{16}After he fathered Jared, Mahalalel lived 830 years, fathering other sons and daughters. \v{17}Mahalalel lived a total of 895 years, and then died.

\v{18}When Jared had lived 162 years, he fathered Enoch. \v{19}After he fathered Enoch, Jared lived 800 years, fathering other sons and daughters. \v{20}Jared lived a total of 962 years, and then died.

\v{21}When Enoch had lived 65 years, he fathered Methuselah. \v{22}After he fathered Methuselah, Enoch communed\fnote{\fbackref{5:22} Lit. \fbib{walked}} with God for 300 years and fathered other sons and daughters. \v{23}Enoch lived a total of 365 years, \v{24}communing\fnote{\fbackref{5:24} Lit. \fbib{walking}} with God---and then he was there no longer, because God had taken him.

\v{25}When Methuselah had lived 187 years, he fathered Lamech. \v{26}After he fathered Lamech, Methuselah lived 782 years, fathering other sons and daughters. \v{27}Methuselah lived a total of 969 years, and then died.

\v{28}When Lamech had lived 182 years, he fathered a son, \v{29}whom he named Noah,\fnote{\fbackref{5:29} The Heb. name \fbib{Noah} sounds like the word \fbib{comfort}} because he said, ``May this one comfort us from our work, from pain that is caused by our manual labor, and from the ground that the \divine{Lord} has cursed.'' \v{30}After he fathered Noah, Lamech lived 595 years, fathering other sons and daughters. \v{31}Lamech lived a total of 777 years, and then died.

\v{32}After Noah had lived 500 years, he fathered Shem, Ham, and Japheth.
\labelchapt{6}
\passage{Human Corruption}

\chapt{6}
\v{1}Now after the population of human beings had increased throughout the\fnote{\fbackref{6:1} Lit. \fbib{increase on the surface of the}} earth, and daughters had been born to them, \v{2}some divine beings\fnote{\fbackref{6:2} Lit. \fbib{them, \v{2}the sons of God}} noticed how attractive human women\fnote{\fbackref{6:2} Lit. \fbib{attractive daughters of Adam}} were, so they took wives for themselves from a selection that pleased them.\fnote{\fbackref{6:2} Lit. \fbib{from all whom they had selected}} \v{3}So the \divine{Lord} said, ``My Spirit won't remain\fnote{\fbackref{6:3} Or \fbib{contend}} with human beings forever, because they're truly mortal.\fnote{\fbackref{6:3} Lit. \fbib{flesh}} Their lifespan\fnote{\fbackref{6:3} Lit. \fbib{days}} will be 120 years.''

\v{4}The Nephilim\fnote{\fbackref{6:4} MT reads \fbib{The Fallen Ones}; LXX and Aram. read \fbib{Giants}; cf. Num 13:33} were on the earth at that time\fnote{\fbackref{6:4} Lit. \fbib{earth in those days}} (and also immediately afterward), when those divine beings\fnote{\fbackref{6:4} Or \fbib{after, the sons of God}} were having sexual relations with\fnote{\fbackref{6:4} Lit. \fbib{beings went in to}} those human women,\fnote{\fbackref{6:4} Lit. \fbib{with Adam's daughters}} who gave birth to children for them. These children\fnote{\fbackref{6:4} The Heb. lacks \fbib{children}} became the heroes and legendary figures of ancient times.\fnote{\fbackref{6:4} Lit. \fbib{heroes of ancient times, men of renown}}
\passage{God Decides to Destroy the World}

\v{5}The \divine{Lord} saw that human evil was growing more and more throughout the earth, with every inclination of people's thoughts\fnote{\fbackref{6:5} Lit. \fbib{hearts}} becoming only evil on a continuous basis. \v{6}Then the \divine{Lord} regretted that he had made human beings on the earth, and he was deeply grieved about that.\fnote{\fbackref{6:6} Lit. \fbib{was grieved to the heart}} \v{7}So the \divine{Lord} said, ``I will annihilate these human beings whom I've created from the\fnote{\fbackref{6:7} Lit. \fbib{the surface of the}} earth, including people, animals, crawling things, and flying creatures, because I'm grieving that I made them.'' \v{8}However, the \divine{Lord} was pleased with Noah.
\passage{Noah Obeys God}

\v{9}These are the family records\fnote{\fbackref{6:9} Or \fbib{the generations}} of Noah: Noah was a righteous man. Blameless during his times,\fnote{\fbackref{6:9} Or \fbib{generations}} Noah communed\fnote{\fbackref{6:9} Lit. \fbib{lifetime, Noah walked}} with God. \v{10}Noah fathered three sons: Shem, Ham, and Japheth. \v{11}By this time, the earth had become ruined in God's opinion\fnote{\fbackref{6:11} Lit. \fbib{sight}} and filled with violence. \v{12}God looked at the earth, observing how corrupt its population had become, because the entire human race\fnote{\fbackref{6:12} Lit. \fbib{all the flesh on the earth}} had corrupted itself.\fnote{\fbackref{6:12} Lit. \fbib{corrupted their ways}} \v{13}So God announced to Noah, ``I've decided to destroy every living thing on earth,\fnote{\fbackref{6:13} Lit. \fbib{The end of all flesh has come before me}} because it has become filled with violence due to them. Look! I'm about to annihilate them, along with the earth. \v{14}So make yourself an ark out of cedar,\fnote{\fbackref{6:14} Or \fbib{cypress}} constructing compartments in it, and cover it inside and out with tar. \v{15}Make the ark like this: 300 cubits\fnote{\fbackref{6:15} I.e. about 450 feet} long, 50 cubits\fnote{\fbackref{6:15} I.e. about 75 feet} wide, and 30 cubits\fnote{\fbackref{6:15} I.e. about 45 feet} high. \v{16}Make a roof\fnote{\fbackref{6:16} Or \fbib{cupola}} for the ark, and finish the walls\fnote{\fbackref{6:16} The Heb. lacks \fbib{the walls}} to within one cubit\fnote{\fbackref{6:16} I.e. about one and a half feet} from the top.\fnote{\fbackref{6:16} I.e. for a skylight} Place the entrance in the side of the ark, and build a lower, a middle, and an upper deck.

\v{17}``For my part, I'm about to flood the earth with water and destroy every living thing\fnote{\fbackref{6:17} Lit. \fbib{thing under heaven}} that breathes. Everything on earth will die. \v{18}However, I will establish my own covenant with you, and you are to enter the ark---you, your sons, your wife, and your sons' wives. \v{19}You are to bring two of every living thing\fnote{\fbackref{6:19} Lit. \fbib{every kind of flesh}} into the ark so they may remain alive with you. They are to be male and female. \v{20}From birds according to their species,\fnote{\fbackref{6:20} Lit. \fbib{kind}} from domestic animals according to their species,\fnote{\fbackref{6:20} Lit. \fbib{kind}} and from everything that crawls on the ground according to their species\fnote{\fbackref{6:20} Lit. \fbib{kind}}---two of everything will come to you so they may remain alive. \v{21}For your part, take some of the edible food and store it away---these stores will be food for you and the animals.''\fnote{\fbackref{6:21} Lit. \fbib{and them}}

\v{22}Noah did all of this, precisely as\fnote{\fbackref{6:22} Lit. \fbib{this, everything that}} God had commanded.
\labelchapt{7}
\passage{Entering the Ark}

\chapt{7}
\v{1}Then the \divine{Lord} told Noah, ``Come---you and all your household---into the ark, because I've seen that you alone are righteous\fnote{\fbackref{7:1} Lit. \fbib{righteous before me}} in this generation. \v{2}You are to take with you seven pairs\fnote{\fbackref{7:2} Lit. \fbib{seven seven}} of every clean animal, a male and its mate, and two of the unclean animals, a male and its mate; \v{3}along with seven pairs\fnote{\fbackref{7:3} Lit. \fbib{seven seven}} of the flying birds, male and female, in order to keep their offspring alive on the surface of all the earth. \v{4}Seven days from now I'll send rain on the earth for 40 days and 40 nights, and I'll destroy every living creature that I've made.''

\v{5}Noah did everything that the \divine{Lord} commanded.
\passage{The Flood Begins}

\v{6}Noah was 600 years old when water began to flood the earth. \v{7}Noah, his sons, his wife, and his sons' wives entered the ark with him before the flood waters arrived.\fnote{\fbackref{7:7} The Heb. lacks \fbib{arrived}} \v{8}From both clean and unclean animals, from birds, and from everything that crawls on the ground, \v{9}two by two, male and female, they entered the ark to join Noah, just as God had commanded.

\v{10}Seven days later, the flooding started. \v{11}On the seventeenth day of the second month, when Noah was 600 years old, all the springs of the great deep burst open, the floodgates of the heavens were opened, \v{12}and it rained throughout the earth for 40 days and 40 nights. \v{13}On that very day, Noah entered the ark with his\fnote{\fbackref{7:13} Lit. \fbib{Noah's}} sons Shem, Ham, and Japheth, Noah's wife, his sons' three wives with them, \v{14}along with every species of wild animal,\fnote{\fbackref{7:14} I.e., non-domesticated animals, as opposed to domesticated livestock} livestock, crawling creature, bird, and every creature that has wings. \v{15}Two of each living creature\fnote{\fbackref{7:15} Lit. \fbib{each of all flesh in which there was life}} entered the ark with Noah. \v{16}The males and females of each living creature\fnote{\fbackref{7:16} Lit. \fbib{of all flesh}} entered the ark,\fnote{\fbackref{7:16} The Heb. lacks \fbib{the ark}} just as God had commanded. Then the \divine{Lord} sealed them inside.

\v{17}The flood continued throughout the earth for 40 days, while the flood waters increased, lifting the ark so that it rose above the surface of the\fnote{\fbackref{7:17} The Heb. lacks \fbib{surface of the}} earth. \v{18}The flood waters continued to surge, increasing throughout the earth, while the ark floated on the surface of the flood water. \v{19}The flood water surged even higher throughout the earth, until all the highest mountains under the sky were covered. \v{20}The flood waters rose 15 cubits\fnote{\fbackref{7:20} I.e. about 22 and a half feet} above the mountains. \v{21}Every living thing\fnote{\fbackref{7:21} Lit. \fbib{flesh that moves}} on earth died---flying creatures, livestock, wildlife, all creatures that swarm over the earth, and all human beings. \v{22}Everything that breathed\fnote{\fbackref{7:22} Lit. \fbib{that had breath in its nostrils}} and everything that had lived on dry land died. \v{23}All existing creatures that had lived on the surface of the ground were annihilated, from humans to livestock, from crawling creatures to birds of the sky. They were wiped off the earth. Only Noah remained, along with those who were with him in the ark. \v{24}The flood waters surged over the earth for 150 days.
\labelchapt{8}
\passage{The Waters Recede}

\chapt{8}
\v{1}God kept Noah in mind, along with all the wildlife\fnote{\fbackref{8:1} I.e., non-domesticated animals, as opposed to domesticated livestock} and livestock that were with him in the ark. God's Spirit\fnote{\fbackref{8:1} Or \fbib{wind}} moved throughout the earth, causing the flood waters to subside. \v{2}The water sources from the ocean depths were blocked and the floodgates of the heavens were closed. \v{3}Then the flood waters steadily receded,\fnote{\fbackref{8:3} Lit. \fbib{receded from the dry land}} diminishing completely by the end of the 150 days. \v{4}The ark came to rest on the mountains of Ararat\fnote{\fbackref{8:4} I.e. ancient Urartu} on the seventeenth day of the seventh month. \v{5}The flood water continued to recede until the tenth month, when, on the first of that month, the tops of the mountains could be seen.

\v{6}After 40 days, Noah opened the window of the ark that he had built \v{7}and sent out a raven. It went back and forth as the flood water continued to evaporate throughout the earth. \v{8}Later, he sent a dove out from the ark\fnote{\fbackref{8:8} Lit. \fbib{from his presence}} to see whether the water that covered the land's surface had completely\fnote{\fbackref{8:8} The Heb. lacks \fbib{completely}} receded, \v{9}but the dove could not yet find a place to rest,\fnote{\fbackref{8:9} Lit. \fbib{rest for its foot}} so it returned to Noah\fnote{\fbackref{8:9} Lit. \fbib{him}} on the ark, since water still covered the land. Noah reached out his hand and took the dove back\fnote{\fbackref{8:9} Lit. \fbib{took it}} into the ark with him.

\v{10}Noah\fnote{\fbackref{8:10} Lit. \fbib{He}} waited another seven days and sent the dove out from the ark again. \v{11}The dove returned to him in the evening, but in its beak there was an olive leaf that it had plucked! So Noah knew that the flood waters had decreased on the land. \v{12}He waited seven more days and sent the dove out again, but it did not return to him anymore.

\v{13}In the six hundred and first year of Noah's life,\fnote{\fbackref{8:13} The Heb. lacks \fbib{of Noah's life}} during the first month, the flood water began to evaporate from the land. Noah then removed the ark's cover and saw that the surface of the land was drying. \v{14}By the twenty-seventh day of the second month, the ground was dry.
\passage{The \divine{Lord}'s Covenant}

\v{15}God spoke to Noah, \v{16}``It's time for you, your wife, your sons, and your sons' wives who are with you to leave the ark. \v{17}Bring out with you every living creature---including the flying creatures, animals, and everything that crawls on the ground---so they may disperse throughout the land, be fruitful, and multiply throughout the earth.'' \v{18}So Noah, his sons, his wife, and his sons' wives emerged. \v{19}Every animal, every crawling thing, every flying creature, and everything that moves on the earth emerged from the ark by groups.\fnote{\fbackref{8:19} Lit. \fbib{by their groups}}

\v{20}Then Noah built an altar to the \divine{Lord} and offered burnt offerings on it\fnote{\fbackref{8:20} Lit. \fbib{on the altar}} from every clean animal and every clean bird. \v{21}When the \divine{Lord} smelled the pleasing aroma, he told himself, ``I will never again curse the land because of human beings---even though human inclinations remain evil from youth---nor will I destroy every living being ever again, as I've done.

\begin{poetry}
\poeml \v{22}``Never\fnote{\fbackref{8:22} The Heb. lacks \fbib{Never}} again, as long as the earth exists, \\
\poemll    will sowing and harvest, \\
\poeml cold and heat, \\
\poemll    summer and winter, \\
\poemlll       and day and night ever cease.''
\end{poetry}
\labelchapt{9}
\passage{The Covenant with Noah}

\chapt{9}
\v{1}God blessed Noah and his sons and ordered them, ``Be productive, multiply, and fill the earth. \v{2}All the living creatures of the earth will be filled with fear and terror of you from now on, including all the creatures that fly in the sky, everything that crawls on the ground, and all the fish of the ocean. They've been assigned to live under your dominion.\fnote{\fbackref{9:2} Lit. \fbib{your hand}}

\v{3}``Every living, moving creature will be food for you. Just as I gave you green plants before, so now you have everything. \v{4}However, you are not to eat meat with its life---that is, its blood---in it! \v{5}Also, I will certainly demand an accounting regarding bloodshed, from every animal and from every human being. I'll demand an accounting from every human being for the life of another human being.

\begin{poetry}
\poeml \v{6}``Whoever sheds human blood, \\
\poemll    by a human his own blood is to be shed; \\
\poeml because God made human beings \\
\poemll    in his own image. \\
\poeml \v{7}Now as for you, be productive \\
\poemll    and multiply; \\
\poeml spread out over the land \\
\poemll    and multiply throughout it.''
\end{poetry}

\v{8}Later, God told Noah and his sons, \v{9}``Pay attention! I'm establishing my covenant with you and with your descendants after you, \v{10}and with every living creature that is with you---the flying creatures, the livestock, and all the wildlife of the earth that are with you---all the earth's animals that came out of the ark. \v{11}I will establish my covenant with you: No living beings will ever be cut off again by flood waters, and there will never again be a flood that destroys the earth.''
\passage{The Sign of God's Covenant}

\v{12}God also said, ``Here's the symbol that represents the covenant that I'm making between me and you and every living being with you, for all future generations: \v{13}I've set my rainbow in the sky\fnote{\fbackref{9:13} Lit. \fbib{cloud}} to symbolize the covenant between me and the earth. \v{14}Whenever I bring clouds over the earth and the rainbow becomes visible in the clouds, \v{15}I'll remember my covenant between me and you and every living creature, so that water will never again become a flood to destroy all living beings. \v{16}When the rainbow is in the clouds, I will observe it and remember the everlasting covenant between God and all living beings on the earth.''

\v{17}God also told Noah, ``This is the symbol of the covenant that I've established between me and everything\fnote{\fbackref{9:17} Lit. \fbib{all flesh}} that lives on the earth.''
\passage{Noah and His Family}

\v{18}Noah's sons who came out of the ark were Shem, Ham, and Japheth. (Ham later fathered Canaan.) \v{19}These three were Noah's sons, and from these men the whole earth was repopulated.

\v{20}Noah, a man of the soil, was the first to plant and farm a vineyard. \v{21}He drank some of the wine, got drunk, and lay down naked\fnote{\fbackref{9:21} Or \fbib{and exposed himself}} right in the middle of his tent. \v{22}Ham, who fathered Canaan, saw his father's genitals and told his two brothers outside. \v{23}Then Shem and Japheth took their father's\fnote{\fbackref{9:23} Lit. \fbib{took the}} cloak, laid it across both their shoulders, and walking backwards, they both covered their father's genitals. Their faces were turned away, and they did not see their father's genitals. \v{24}When Noah sobered up and learned what his youngest son had done to him, \v{25}he said,

\begin{poetry}
\poeml ``Canaan is cursed! \\
\poemll    He will be the lowest of slaves to his relatives.''
\end{poetry}

\v{26}He also said,

\begin{poetry}
\poeml ``Blessed be the \divine{Lord} God of Shem, \\
\poemll    and may Canaan be his slave. \\
\poeml \v{27}May God make room for\fnote{\fbackref{9:27} Or \fbib{God extend}; the Heb. verb sounds like the name \fbib{Japheth}} Japheth; \\
\poemll    may God\fnote{\fbackref{9:27} Lit. \fbib{he}} live in Shem's tents, \\
\poemlll       and may Canaan serve him.''
\end{poetry}

\v{28}Noah lived 350 years after the flood. \v{29}After Noah had lived a total of 950 years, he died.
\labelchapt{10}
\passage{Descendants and Nations from Noah}

\chapt{10}
\v{1}These are the records\fnote{\fbackref{10:1} Or \fbib{generations}} of Noah's sons, Shem, Ham, and Japheth, to whom descendants\fnote{\fbackref{10:1} Lit. \fbib{sons}, and so throughout the chapter} were born after the flood.

\v{2}Japheth's descendants included\fnote{\fbackref{10:2} The Heb. lacks \fbib{included}; and so throughout the chapter} Gomer, Magog, Madai, Javan, Tubal, Meshech, and Tiras.

\v{3}Gomer's descendants included Ashkenaz, Riphath, and Togarmah.

\v{4}Javan's descendants included Elisha, Tarshish, Kittim, and Dodanim,\fnote{\fbackref{10:4} So MT; LXX and a Heb. mss. read \fbib{Rodanim;} Cf. 1Chr 1:7} \v{5}from whom the coastal nations\fnote{\fbackref{10:5} Lit. \fbib{peoples}} spread into their own lands and nations, each with their own language and family groups.
\passage{Ham's Descendants}

\v{6}Ham's descendants included Cush, Egypt, Put, and Canaan.

\v{7}Cush's descendants included Seba, Havilah, Sabtah, Raamah, and Sabteca.

Raamah's descendants included Sheba and Dedan.

\v{8}Cush fathered Nimrod, who became the first fearless\fnote{\fbackref{10:8} Or \fbib{valiant}} leader throughout the land. \v{9}He became a fearless\fnote{\fbackref{10:9} Or \fbib{valiant}} hunter in defiance of\fnote{\fbackref{10:9} Lit. \fbib{hunter before}} the \divine{Lord}. That is why it is said, ``Like Nimrod, a fearless hunter in defiance of\fnote{\fbackref{10:9} Lit. \fbib{hunter before}} the \divine{Lord}.'' \v{10}His kingdom began in the region\fnote{\fbackref{10:10} Lit. \fbib{land}} of Shinar\fnote{\fbackref{0:10} I.e. southern Mesopotamia or Babylonia} with the cities of\fnote{\fbackref{10:10} The Heb. lacks \fbib{the cities of}} Babylon, Erech,\fnote{\fbackref{10:10} Or \fbib{Uruk}} Akkad, and Calneh. \v{11}From there\fnote{\fbackref{10:11} Lit. \fbib{from that land}} he went north\fnote{\fbackref{10:11} The Heb. lacks \fbib{north}} to Assyria and built Nineveh, Rehoboth-ir, and Calah, \v{12}along with Resen, which was located between Nineveh and the great city of Calah.

\v{13}Egypt fathered the Ludites, the Anamites, the Lehabites, the Naphtuhites, \v{14}the Pathrusites, the Casluhites (from which came the Philistines), and the Caphtorites.

\v{15}Canaan fathered Sidon his firstborn, along with the Hittites, \v{16}the Jebusites, the Amorites, the Girgashites, \v{17}the Hivites, the Arkites, the Sinites, \v{18}the Arvadites, the Zemarites, and the Hamathites.

Later, the Canaanite families were widely scattered. \v{19}The Canaanite border extended south\fnote{\fbackref{10:19} The Heb. lacks \fbib{south}} from Sidon toward Gerar as far as Gaza, and east\fnote{\fbackref{10:19} The Heb. lacks \fbib{east}} toward Sodom, Gomorrah, Admah, and Zeboiim, as far as Lasha.

\v{20}These are Ham's descendants, listed by their families, each with their own lands, language, and family groups.
\passage{Shem's Descendants}

\v{21}Shem, Japheth's older brother, also had descendants.\fnote{\fbackref{10:21} Lit. \fbib{sons}} Shem was the father of the descendants of Eber. \v{22}Shem's sons included Elam, Asshur, Arpachshad, Lud, and Aram.

\v{23}Aram's descendants included Uz, Hul, Gether, and Mash.

\v{24}Arpachshad fathered Cainan, Cainan fathered Shelah, and Shelah fathered Eber.\fnote{\fbackref{10:24} So with LXX (cf. Gen. 11:12-13 \& Luke 3:35-36); the Heb. lacks \fbib{Cainan, Cainan fathered.}} \v{25}To Eber were born two sons. One was named Peleg,\fnote{\fbackref{10:25} The Heb. name \fbib{Peleg} sounds like the Heb. verb \fbib{divided}} because the earth was divided during his lifetime. His brother was named Joktan.

\v{26}Joktan fathered Almodad, Sheleph, Hazarmaveth, Jerah, \v{27}Hadoram, Uzal, Diklah, \v{28}Obal, Abimael, Sheba, \v{29}Ophir, Havilah, and Jobab. All these were Joktan's descendants. \v{30}Their settlements extended from Mesha towards Sephar, the eastern hill country.

\v{31}These are Shem's descendants, listed by their families, each with their own lands, language, and family groups.

\v{32}These are the families of Noah's sons, according to their records, by their nations. From these people, the nations on the earth spread out after the flood.
\labelchapt{11}
\passage{The Tower in Babylon}

\chapt{11}
\v{1}There was a time when the entire earth spoke a common language with an identical vocabulary. \v{2}As people\fnote{\fbackref{11:2} Lit. \fbib{they}} migrated westward,\fnote{\fbackref{11:2} Lit. \fbib{migrated from the east}; i.e. from the mountains of Ararat} they came across a plain in the region of Shinar\fnote{\fbackref{11:2} I.e. Babylonia or ancient Sumer} and settled there. \v{3}They told each other, ``Come on! Let's burn bricks thoroughly.'' They used bricks for stone and tar for mortar. \v{4}Then they said, ``Come on! Let's build ourselves a city and a tower, with its summit in the heavens, and let's make a name for ourselves\fnote{\fbackref{11:4} The Heb. lacks \fbib{for ourselves}} so we won't be scattered over the surface of the whole earth.''

\v{5}However, the \divine{Lord} descended to look over the city and the tower that the humans were building. \v{6}The \divine{Lord} said, ``Look! They are one people with the same language for all of them, and this is only the beginning of what they will do.\fnote{\fbackref{11:6} The Heb. lacks \fbib{of what they will do}} Nothing that they have a mind to do will be impossible for them! \v{7}Come on! Let's go down there and confuse their language, so that they won't understand each other's speech.''

\v{8}So the \divine{Lord} scattered them abroad from there over the surface of the whole earth, so that they had to stop building the city. \v{9}Therefore it was called Babylon,\fnote{\fbackref{11:9} The Heb. name \fbib{Babel} means \fbib{confusion}} because there the \divine{Lord} confused the language of all the earth, and from there the \divine{Lord} scattered them over the surface of the entire earth.
\passage{Descendants of Shem}

\v{10}These are the family records\fnote{\fbackref{11:10} Or \fbib{generations}} of Shem. When Shem had lived 100 years, he fathered Arpachshad two years after the flood. \v{11}Shem lived 500 years after he fathered Arpachshad and had other\fnote{\fbackref{11:11} The Heb. lacks \fbib{other}} sons and daughters.

\v{12}When Arpachshad had lived 35 years, he fathered Cainan. \v{13}After he fathered Cainan, Arpachshad lived 430 years and had other\fnote{\fbackref{11:13} The Heb. lacks \fbib{other}} sons and daughters, and then died.

Cainan lived 130 years and fathered Shelah. After he fathered Shelah, Cainan lived 330 years and had other\fnote{\fbackref{11:13} The Heb. lacks \fbib{other}} sons and daughters, and then died.\fnote{\fbackref{11:12-13} So with LXX (cf. Gen. 10:24 \& Luke 3:35-36). MT reads \fbib{Arpachshad lived 403 years after fathering Shelah, and had sons and daughters.}}

\v{14}When Shelah had lived 30 years, he fathered Eber. \v{15}After he fathered Eber, Shelah lived 403 years and had other\fnote{\fbackref{11:15} The Heb. lacks \fbib{other}} sons and daughters.

\v{16}When Eber had lived 34 years, he fathered Peleg. \v{17}After he fathered Peleg, Eber lived 430 years and had other\fnote{\fbackref{11:17} The Heb. lacks \fbib{other}} sons and daughters.

\v{18}When Peleg had lived 30 years, he fathered Reu. \v{19}After he fathered Reu, Peleg lived 209 years and had other\fnote{\fbackref{11:19} The Heb. lacks \fbib{other}} sons and daughters.

\v{20}When Reu had lived 32 years, he fathered Serug. \v{21}After he fathered Serug, Reu lived 207 years and had other\fnote{\fbackref{11:21} The Heb. lacks \fbib{other}} sons and daughters.

\v{22}When Serug had lived 30 years, he fathered Nahor. \v{23}After he fathered Nahor, Serug lived 200 years and had other\fnote{\fbackref{11:23} The Heb. lacks \fbib{other}} sons and daughters.

\v{24}When Nahor had lived 29 years, he fathered Terah. \v{25}After he fathered Terah, Nahor lived 119 years and had other\fnote{\fbackref{11:25} The Heb. lacks \fbib{other}} sons and daughters.

\v{26}When Terah had lived 70 years, he fathered Abram, Nahor, and Haran.
\passage{Descendants of Terah}

\v{27}Now these are the family records\fnote{\fbackref{11:27} Or \fbib{generations}} of Terah: Terah fathered Abram, Nahor, and Haran; and Haran fathered Lot. \v{28}Haran died during his father's lifetime in the land of his birth, that is, in Ur of the Chaldeans. \v{29}Abram and Nahor took wives for themselves. The name of Abram's wife was Sarai, and the name of Nahor's wife was Milcah. She was the daughter of Haran, who was the father of Milcah and Iscah. \v{30}Sarai was barren, so she had not borne children.

\v{31}Terah took his son Abram, his grandson Lot (Haran's son), and his daughter-in-law Sarai, his son Abram's wife, and they journeyed together from Ur of the Chaldeans to go to the land of Canaan. But when they had gone as far as Haran, they settled there, \v{32}where Terah died at the age of 205 years.
\labelchapt{12}
\passage{God Calls Abram}

\chapt{12}
\v{1}The \divine{Lord} told Abram, ``You are to leave your land, your relatives, and your father's house and go to the land that I'm going to show you. \v{2}I'll make a great nation of your descendants, I'll bless you, and I'll make your reputation great, so that you will be a blessing. \v{3}I'll bless those who bless you, but I'll curse the one who curses you, and through you all the people\fnote{\fbackref{12:3} Lit. \fbib{families}} of the earth will be blessed.''

\v{4}So Abram left there, as the \divine{Lord} had directed him, and Lot accompanied him. Abram was 75 years old when he left Haran. \v{5}Abram took his wife Sarai, his nephew Lot, all the possessions they had accumulated, and the servants\fnote{\fbackref{12:5} Lit. \fbib{the living beings}} he had acquired while living\fnote{\fbackref{12:5} The Heb. lacks \fbib{while living}} in Haran. Then they set out to go to the land of Canaan. When they arrived in the land of Canaan, \v{6}Abram traveled through the land to the place called Shechem, as far as the oak of Moreh. At that time the Canaanites were in the land.

\v{7}Then the \divine{Lord} appeared to Abram and said, ``I'll give this land to your descendants.''\fnote{\fbackref{12:7} Lit. \fbib{seed}} So Abram\fnote{\fbackref{12:7} Lit. \fbib{he}} built an altar to the \divine{Lord}, who had appeared to him. \v{8}From there Abram\fnote{\fbackref{12:8} Lit. \fbib{he}} traveled on to the hill country east of Bethel and set up his tent, with Bethel on the west and Ai on the east. There he built an altar to the \divine{Lord} and called on the name of the \divine{Lord}. \v{9}Then Abram traveled on, continuing into the Negev.\fnote{\fbackref{12:9} I.e. the southern regions of the Sinai peninsula; cf. Josh 10:40}
\passage{Abram and Sarai in Egypt}

\v{10}There was a famine in the land, so Abram went down to Egypt to live because the famine was so severe. \v{11}When he was about to enter Egypt, he told his wife Sarai, ``Look, I'm aware that you're a beautiful woman. \v{12}When the Egyptians see you, they will say, `She is his wife.' Then they'll kill me, but allow you to live. \v{13}Please say that you are my sister, so things will go well for me for your sake. That way, you'll be saving my life.''

\v{14}As Abram was entering Egypt, the Egyptians noticed how beautiful Sarai\fnote{\fbackref{12:14} Lit. \fbib{that the woman}} was. \v{15}When Pharaoh's officials saw her, they brought her to the attention of Pharaoh and took the woman to Pharaoh's palace. \v{16}He treated Abram well because of her, so Abram acquired sheep, oxen, male and female donkeys, male and female servants, and camels. \v{17}But the \divine{Lord} afflicted Pharaoh and his household with severe plagues because of Sarai, Abram's wife. \v{18}Pharaoh summoned Abram and asked, ``What have you done to me! Why didn't you tell me that she was your wife? \v{19}Why did you say, `She is my sister,' so that I took her as a wife for myself? Now, here is your wife! Take her and get out!''

\v{20}So Pharaoh assigned men to Abram,\fnote{\fbackref{12:20} Lit. \fbib{him}} and they escorted him, his wife, and all that he had out of the country.\fnote{\fbackref{12:20} The Heb. lacks \fbib{out of the country}}
\labelchapt{13}
\passage{Abram and Lot Part Ways}

\chapt{13}
\v{1}Abram traveled from Egypt, along with his wife and everyone who belonged to his household\fnote{\fbackref{13:1} Lit. \fbib{who pertained to him}}---including Lot---to the Negev.\fnote{\fbackref{13:1} I.e. the southern regions of the Sinai peninsula; cf. Josh 10:40}

\v{2}Now Abram had become quite wealthy in livestock, silver, and gold. \v{3}He journeyed by stages from the Negev\fnote{\fbackref{13:3} I.e. the southern regions of the Sinai peninsula; cf. Josh 10:40} to Bethel, the place where his tent had formerly been, between Bethel and Ai, \v{4}where he had first built an altar. There Abram called on the name of the \divine{Lord}.

\v{5}Lot, who was traveling with Abram, also had flocks of sheep, herds, and tents. \v{6}But the land could not support them living together, because they had so many livestock that they could not stay together. \v{7}There was strife between the herdsmen in charge of Abram's livestock and the herdsmen in charge of Lot's livestock. Also, at that time the Canaanites and the Perizzites were living in the land.

\v{8}So Abram told Lot, ``Please, let's not have strife between you and me, or between your herdsmen and my herdsmen, since we are relatives.\fnote{\fbackref{13:8} Lit. \fbib{brothers}} \v{9}Isn't the whole land available to you? Let's separate: If you go\fnote{\fbackref{13:9} The Heb. lacks \fbib{you go}} to the left, then I will go to the right; if you go\fnote{\fbackref{13:9} The Heb. lacks \fbib{you go}} to the right, then I will go to the left.''

\v{10}Lot looked around and noticed that the whole Jordan plain as far as Zoar was well-watered like the garden of the \divine{Lord} or like the land of Egypt. (This was before the \divine{Lord} destroyed Sodom and Gomorrah.) \v{11}So Lot chose for himself all the Jordan plain. Then Lot traveled eastward, and they separated from each other.

\v{12}So Abram lived in the land of Canaan, while Lot settled in the cities of the plain, setting up his tent in the vicinity of Sodom. \v{13}Now the men of Sodom were particularly evil and sinful in their defiance of\fnote{\fbackref{13:13} Lit. \fbib{sinful before}} the \divine{Lord}.

\v{14}After Lot had separated from Abram, the \divine{Lord} told Abram, ``Look off to the north, south\fnote{\fbackref{13:14} Lit. \fbib{the Negev}}, east, and west\fnote{\fbackref{13:14} Lit. \fbib{the sea}} from where you're living, \v{15}because I'm going to give you and your descendants all of the land that you see---forever! \v{16}I'll make your descendants as plentiful as\fnote{\fbackref{13:16} The Heb. lacks \fbib{plentiful as}} the specks of\fnote{\fbackref{13:16} The Heb. lacks \fbib{the specks of}} dust of the earth, so that if one could count the specks of\fnote{\fbackref{13:16} The Heb. lacks \fbib{the specks of}} dust of the earth, then your descendants could also be counted. \v{17}Get up! Walk throughout the length and breadth of the land, because I'm going to give it to you.''

\v{18}So Abram moved his tent and settled beside the oaks of Mamre that are by Hebron, where he built an altar to the \divine{Lord}.
\labelchapt{14}
\passage{Abram Battles Kings for Lot}

\chapt{14}
\v{1}At the time when Amraphel was king of Shinar, Arioch was king of Ellasar, Chedorlaomer was king of Elam, and Tidal was king of the Goiim, \v{2}they engaged in war against King Bera of Sodom, King Birsha of Gomorrah, King Shinab of Admah, King Shemeber of Zeboiim, along with the king of Bela (which was also known as Zoar). \v{3}All of this latter group of kings\fnote{\fbackref{14:3} Lit. \fbib{All of these}} allied together in the Valley of Siddim (that is, the Salt Sea\fnote{\fbackref{14:3} I.e. the Dead Sea}). \v{4}They were subject to Chedorlaomer for twelve years, but they rebelled in the thirteenth year.

\v{5}In the fourteenth year, Chedorlaomer and the kings with him came and defeated the Rephaim in Ashteroth-karnaim, the Zuzites in Ham, the Emites in Shaveh-kiriathaim, \v{6}and the Horites in the hill country of Seir, near El-paran by the desert. \v{7}Next they turned back and came to En-mishpat (which was also known as Kadesh) and conquered all the territory of the Amalekites, along with the Amorites who lived in Hazazon-tamar.

\v{8}Then the kings of Sodom, Gomorrah, Admah, Zeboiim, and Bela (which was also known as Zoar) prepared for battle in the Valley of Siddim \v{9}against King Chedorlaomer of Elam, King Tidal of Goiim, King Amraphel of Shinar, and King Arioch of Ellasar---four kings against five.

\v{10}Now the Valley of Siddim was full of tar pits, so when the kings of Sodom and Gomorrah fled, some of their people\fnote{\fbackref{14:10} The Heb. lacks \fbib{of their people}} fell into them, while the rest fled to the hill country. \v{11}The conquerors\fnote{\fbackref{14:11} Lit. \fbib{They}} captured all the possessions of Sodom and Gomorrah, including their entire food supply, and then left. \v{12}They also took Abram's nephew Lot captive, and confiscated\fnote{\fbackref{14:12} The Heb. lacks \fbib{confiscated}} his possessions, since he was living in Sodom.

\v{13}Someone escaped, arrived, and reported what had happened\fnote{\fbackref{14:13} The Heb. lacks \fbib{what had happened}} to Abram the Hebrew, who was living by the oaks belonging to Mamre the Amorite, whose brothers Eshcol and Aner were allied with Abram. \v{14}When Abram heard that his nephew\fnote{\fbackref{14:14} Lit. \fbib{brother}} had been taken prisoner, he gathered together 318 of his trained men, who had been born in his household, and they went out in pursuit as far as Dan. \v{15}During the night, Abram\fnote{\fbackref{14:15} Lit. \fbib{he}} and his servants divided his forces,\fnote{\fbackref{14:15} The Heb. lacks \fbib{his forces}} conquered his enemies,\fnote{\fbackref{14:15} Lit. \fbib{conquered them}} and pursued them as far as Hobah, north of Damascus. \v{16}He recovered all the goods and brought back his nephew Lot, together with his possessions, the women, and the other\fnote{\fbackref{14:16} The Heb. lacks \fbib{other}} people.
\passage{The Blessing of Melchizedek}

\v{17}After Abram's return\fnote{\fbackref{14:17} Lit. \fbib{his return}} from defeating Chedorlaomer and the kings who were with them, the king of Sodom went out to meet with him in the Shaveh Valley (that is, the King's Valley). \v{18}King Melchizedek of Salem brought out bread and wine, since he was serving as\fnote{\fbackref{14:18} The Heb. lacks \fbib{serving as}} the priest of God Most High. \v{19}Melchizedek\fnote{\fbackref{14:19} Lit. \fbib{He}} blessed Abram\fnote{\fbackref{14:19} Lit. \fbib{him}} and said,

\begin{poetry}
\poeml ``Abram is blessed by God Most High, \\
\poemll    Creator of heaven and earth, \\
\poeml \v{20}and blessed be God Most High, \\
\poemll    who has delivered your enemies \\
\poemlll       into your control.''
\end{poetry}

Then Abram gave him a tenth of everything.
\passage{Conversation with the King of Sodom}

\v{21}The king of Sodom told Abram, ``Return the people to me, and you take the possessions for yourself.''

\v{22}But Abram answered the king of Sodom, ``I have made an oath to the \divine{Lord} God Most High, Creator of heaven and earth, \v{23}that I will not take a thread or a sandal strap or anything that belongs to you, so you won't be able to say, `I made Abram rich.' \v{24}I will take nothing except what my warriors have eaten. But as for what belongs to the men who were allied\fnote{\fbackref{14:24} Lit. \fbib{who came}} with me, including Aner, Eshcol, and Mamre, let them take their share.''
\labelchapt{15}
\passage{The Abrahamic Covenant}

\chapt{15}
\v{1}Some time later, a message came from the \divine{Lord} to Abram in a vision. ``Stop being afraid, Abram.'' he said. ``I myself---your shield---am your very great reward.''

\v{2}But Abram replied, ``Lord \divine{God}, what can you give me since I continue to be childless, and the heir of my household is Eliezer from Damascus? \v{3}Look!'' Abram said, ``You haven't given me any offspring, so a servant born in\fnote{\fbackref{15:3} Lit. \fbib{a son of}} my house is going to be my heir.''

\v{4}A message came from the \divine{Lord} to him again: ``This one will not be your heir. Instead, the child who will be born to you\fnote{\fbackref{15:4} Lit. \fbib{the one who will come from your loins}} will be your heir.'' \v{5}Then the \divine{Lord}\fnote{\fbackref{15:5} Lit. \fbib{he}} took him outside. ``Look up at the sky and count the stars---if you can!'' he said. ``Your descendants will be that numerous.''\fnote{\fbackref{15:5} Lit. \fbib{will be so}} \v{6}Abram believed the \divine{Lord}, and it was credited to him as righteousness.

\v{7}The \divine{Lord}\fnote{\fbackref{15:6} Lit. \fbib{He}} spoke to him, ``I am the \divine{Lord}, who brought you from Ur of the Chaldeans, to give you this land as an inheritance.''

\v{8}But he replied, ``Lord \divine{God}, how will I know that I will inherit it?''

\v{9}The \divine{Lord} responded, ``Bring me a three-year-old cow, a three-year-old female goat, a three-year-old ram, a turtledove, and a young pigeon.''

\v{10}So Abram brought him all these animals and cut each of them in half, down the middle, placing the pieces opposite each other, but he did not cut the birds in half. \v{11}When birds of prey swooped down on the carcasses, Abram drove them away. \v{12}As the sun began to set, Abram was overcome with deep sleep, and suddenly a frightening and terrifying darkness descended on him.

\v{13}Then the \divine{Lord} told Abram, ``You can be certain about this: Your descendants will be foreigners in a land that isn't theirs. They will be slaves there and will be oppressed for 400 years. \v{14}However, I will judge the nation that they serve, and later they will leave there with many possessions. \v{15}Now as for you, you'll die peacefully, join your ancestors, and be buried at a good old age. \v{16}Your descendants\fnote{\fbackref{15:16} Lit. \fbib{They}} will return here in the fourth generation, since the iniquity of the Amorites has not yet run its course.''

\v{17}When the sun had fully set and it was dark, a smoking fire pot and a fiery torch passed between the animal pieces.\fnote{\fbackref{15:17} Lit. \fbib{these pieces}} \v{18}That very day the \divine{Lord} made this covenant with Abram: ``I'm giving\fnote{\fbackref{15:18} Or \fbib{have given}} this land to your descendants, from the river of Egypt to the great Euphrates River--- \v{19}including the land of the Kenites, the Kenizzites, the Kadmonites, \v{20}the Hittites, the Perizzites, the Rephaim, \v{21}the Amorites, the Canaanites, the Girgashites, and the Jebusites.''
\labelchapt{16}
\passage{Sarai, Hagar, and Ishmael}

\chapt{16}
\v{1}Now Abram's wife Sarai had not borne a child for him. She had an Egyptian servant girl whose name was Hagar. \v{2}So Sarai told Abram, ``You are well aware that the \divine{Lord} has prevented me from giving birth to a child. Go have sex with my servant, so that I may possibly bear a son\fnote{\fbackref{16:2} Lit. \fbib{possibly be built up}} through her.''

Abram listened to Sarai's suggestion, \v{3}so Abram's wife Sarai took her Egyptian servant, Hagar, and gave her as a wife to her husband Abram. This took place\fnote{\fbackref{16:3} The Heb. lacks \fbib{This took place}} ten years after Abram had settled in the land of Canaan. \v{4}He had sex with Hagar, and she became pregnant. When she realized that she was pregnant, she looked with contempt on her mistress.

\v{5}Then Sarai told Abram, ``My suffering is your fault! I gave you my servant so you could have sex with her\fnote{\fbackref{16:5} Lit. \fbib{my servant your bosom}}, and when she discovered that she was pregnant, she looked on me with contempt. May the \divine{Lord} judge between you and me!''

\v{6}Abram answered Sarai, ``Look, your servant is under your control, so do to her as you wish.''\fnote{\fbackref{16:6} Lit. \fbib{her what is good in your eyes}} So Sarai dealt so harshly with Hagar\fnote{\fbackref{16:6} Lit. \fbib{her}} that she ran away from Sarai.\fnote{\fbackref{16:6} Lit. \fbib{her}}

\v{7}The angel of the \divine{Lord} found her by a spring of water in the desert on the road to Shur. \v{8}``Hagar, servant of Sarai,'' he asked, ``Where are you coming from and where are you going?''

She answered, ``I am running away from my mistress Sarai.''

\v{9}The angel of the \divine{Lord} told her, ``You must go back to your mistress and submit to her authority.'' \v{10}The angel of the \divine{Lord} also told her, ``I will greatly multiply your offspring, who will be too many to count.

\v{11}``Look, you are pregnant and will give birth to a son,'' the angel of the \divine{Lord} continued to say to her. ``You will name him Ishmael,\fnote{\fbackref{16:11} I.e. God hears} because the \divine{Lord} has heard your cry of\fnote{\fbackref{16:11} The Heb. lacks \fbib{cry of}} misery. \v{12}He'll be a wild donkey of a man. He'll\fnote{\fbackref{16:12} Lit. \fbib{His hand}} be against everyone, and everyone will be against him.\fnote{\fbackref{16:12} Lit. \fbib{against his hand}} He will live in conflict with\fnote{\fbackref{16:12} Lit. \fbib{in the face of}} all of his relatives.''

\v{13}So she called the name of the \divine{Lord} who spoke to her, ``You are `God who sees,' because I have truly seen the one who looks after me.''

\v{14}That's why the spring was called, ``The Well of the Living One who Looks after Me.'' It was between Kadesh and Bered.

\v{15}Hagar eventually gave birth to Abram's son. Abram named his son whom Hagar bore Ishmael. \v{16}Abram was 86 years old when Hagar gave birth to Ishmael for Abram.
\labelchapt{17}
\passage{God Appears to Abram}

\chapt{17}
\v{1}When Abram was 99 years old, the \divine{Lord} appeared to Abram and announced, ``I am God Almighty. Live in constant awareness that I'm always with you,\fnote{\fbackref{17:1} Lit. \fbib{in my presence}} and be blameless. \v{2}I'll establish my covenant between me and you, and I'll greatly increase your numbers.'' \v{3}Then Abram fell to the ground\fnote{\fbackref{17:3} Lit. \fbib{fell on his face}} as God continued speaking to him. \v{4}``Look, I've made a covenant with you. You will be the father of many nations. \v{5}Your name is no longer to be Abram.\fnote{\fbackref{17:5} The Heb. name means \fbib{exalted father}} Instead your name will be Abraham,\fnote{\fbackref{17:5} The Heb. name means \fbib{father of many}} since I'll make you the father of many nations. \v{6}I'm going to cause you to have many descendants, and I'll bring nations from you. Kings will come from you. \v{7}I'm establishing my covenant between me and you, and with your descendants who come after you, generation after generation, as an eternal covenant, to be your God and your descendants' God after you. \v{8}I'll give to you and to your descendants the land to which you have traveled---all the land of Canaan---as an eternal possession. I will be their God.''
\passage{The Sign of the Covenant}

\v{9}God continued to speak to Abraham, ``You and your descendants who are born in the future are to keep my covenant---that is, you and your descendants, generation after generation. \v{10}Here is my covenant that you are to observe, between me and you and your descendants: Every male among you is to be circumcised. \v{11}You are all to be circumcised in the flesh of your foreskin, and this is to be the sign of the covenant between me and you. \v{12}Generation after generation, every male among you is to be circumcised on the eighth day after his birth,\fnote{\fbackref{17:12} The Heb. lacks \fbib{after his birth}} including the servant born in your house or the one purchased from a foreigner, who is not of your offspring. \v{13}The servant born in your house or the one purchased with money is to be circumcised. My covenant is to remain in your flesh as an eternal covenant. \v{14}Any uncircumcised male who does not have the foreskin of his flesh circumcised on the eighth day\fnote{\fbackref{17:14} So LXX, SP, Jubilees; the Heb. lacks \fbib{on the eighth day}} after his birth\fnote{\fbackref{17:14} The Heb. lacks \fbib{after his birth}} is to be eliminated from his people because he has broken my covenant.''
\passage{Sarah's Pregnancy Foretold}

\v{15}God told Abraham, ``As for Sarai your wife, you are not to call her Sarai any longer,\fnote{\fbackref{17:15} The Heb. lacks \fbib{any longer}} because her name is to be Sarah.\fnote{\fbackref{17:15} The Heb. name means \fbib{princess}} \v{16}I will bless her. Furthermore, I will give you a son from her. I will bless her, so that nations, kings, and people will come from her.''

\v{17}Abraham fell to the ground,\fnote{\fbackref{17:17} Lit. \fbib{fell on his face}} laughed, and told himself, ``Can a child be born to a 100-year-old man? Can a 90-year-old Sarah give birth?'' \v{18}So Abraham responded to God, ``If only Ishmael would live in constant awareness that you're always with him!''\fnote{\fbackref{17:18} Lit. \fbib{in your presence}}

\v{19}But God replied, ``No, but your wife Sarah will give birth to your son, and you are to name him Isaac.\fnote{\fbackref{17:19} The Heb. name means \fbib{laughter}} I'll confirm my covenant with him as an eternal covenant for his descendants. \v{20}And as for Ishmael, I've heard you. I'll bless him, and he'll have many descendants.\fnote{\fbackref{17:20} Lit. \fbib{he'll be fruitful}} I will multiply him greatly, he'll father twelve tribal leaders, and I'll cause his descendants\fnote{\fbackref{17:20} Lit. \fbib{cause him}} to become a great nation. \v{21}Now as to Isaac, I'll confirm my covenant with him, to whom Sarah will give birth as your son at this time next year.'' \v{22}With that, God finished talking to Abraham, and ascended, leaving him.

\v{23}Abraham took his son Ishmael and all the servants born in his house or purchased with his money---every male among the men of his household---and circumcised them\fnote{\fbackref{17:23} Lit. \fbib{them in the flesh of their foreskins}} that very day, just as God had spoken to him. \v{24}Abraham was 99 years old when he was circumcised,\fnote{\fbackref{17:24} Lit. \fbib{circumcised in the flesh of his foreskin}} \v{25}and his son Ishmael was thirteen years old when he was circumcised.\fnote{\fbackref{17:25} Lit. \fbib{circumcised in the flesh of his foreskin}} \v{26}Both Abraham and his son Ishmael were circumcised on that very day. \v{27}Every man born in his household---as well as those who had been purchased with money from a foreigner---was circumcised with him.
\labelchapt{18}
\passage{Abraham's Three Visitors}

\chapt{18}
\v{1}Later, the \divine{Lord} appeared to Abraham\fnote{\fbackref{18:1} Lit. \fbib{him}} by the oaks\fnote{\fbackref{18:1} Or \fbib{terebinths}} belonging to Mamre. As Abraham\fnote{\fbackref{18:1} Lit. \fbib{he}} was sitting near the entrance to his tent during the hottest part of the day, \v{2}he glanced up and saw three men standing there, not far from him. As soon as he noticed them, Abraham\fnote{\fbackref{18:2} Lit. \fbib{he}} ran from the tent entrance to greet them and bowed low to the ground. \v{3}``My lords,'' he told them, ``if I have found favor with you,\fnote{\fbackref{18:3} Lit. \fbib{favor in your eyes}} please don't leave your servant. \v{4}I'll have some water brought to wash your feet while you rest under the tree. \v{5}I'll bring some food for you,\fnote{\fbackref{18:5} Lit. \fbib{you, that you may nourish yourselves}} and after that you may continue your journey, since you have come to visit your servant.''

So they replied, ``Very well! Do what you've proposed.''

\v{6}Abraham hurried into the tent and told Sarah, ``Quick! Take three measures\fnote{\fbackref{18:6} Lit. \fbib{seahs}} of the best flour, knead it, and make some flat bread.''

\v{7}Next, Abraham ran to the herd, found a choice and tender calf, and gave it to the young men, who went off in a hurry to prepare it. \v{8}Then he took curds, milk, and the calf that had been prepared, placed the food in front of them, and stood near them under the tree while they ate.
\passage{Sarah Laughs at the Promise}

\v{9}The men asked him, ``Where is your wife Sarah?''

``There, in the tent,'' he replied.

\v{10}Then one of them said, ``I will certainly return to you in about a year's time.\fnote{\fbackref{18:10} Lit. \fbib{you according to the time of life}} By then, your wife Sarah will have borne a son.''

Now Sarah was listening at the tent entrance behind him. \v{11}Abraham and Sarah were old---really old\fnote{\fbackref{18:11} Lit. \fbib{well advanced in days}}---and Sarah was beyond the age of childbearing.\fnote{\fbackref{18:11} Lit. \fbib{The way of women has ceased for Sarah}} \v{12}That's why Sarah laughed to herself, thinking, ``After I'm so old and my husband is old, too, am I going to have sex?''\fnote{\fbackref{18:12} Lit. \fbib{pleasure}}

\v{13}The \divine{Lord} asked Abraham, ``Why did Sarah laugh and think, `Am I really going to bear a child, since I'm so old?' \v{14}Is anything impossible\fnote{\fbackref{18:14} Lit. \fbib{wonderful}} for the \divine{Lord}? At the time set for it, I will return to you---about a year from now---and Sarah will have a son.''

\v{15}But Sarah denied it. ``I didn't laugh,'' she claimed, because she was afraid.

The \divine{Lord}\fnote{\fbackref{18:15} Lit. \fbib{He}} replied, ``No! You did laugh!''
\passage{God Reveals His Plans to Abraham}

\v{16}After this, the men set out from there and looked out over Sodom. Abraham went with them to send them off.

\v{17}``Should I hide from Abraham what I'm about to do,'' the \divine{Lord} asked, \v{18}``since Abraham's descendants will become a great and powerful nation, and all the nations of the earth will be blessed through him? \v{19}Indeed, I've made myself known to him in order that he may encourage his sons and his household that is born after him to keep the way of the \divine{Lord}, and to do what is right and just, so that the \divine{Lord} may bring about for Abraham what he has promised.'' \v{20}The \divine{Lord} also said, ``How great is the disapproval of Sodom and Gomorrah! Their sin is so very serious! \v{21}I'm going down to see whether they've acted according to the protests that have reached me. If not, I wish to know.''

\v{22}Then two of\fnote{\fbackref{18:22} The Heb. lacks \fbib{two of}} the men turned away from there and walked toward Sodom, while Abraham remained standing in the presence of the \divine{Lord}.
\passage{Abraham Negotiates with God}

\v{23}Abraham approached and asked, ``Will you actually destroy the righteous along with the wicked? \v{24}Perhaps there are 50 righteous ones within the city. Will you actually destroy it and not forgive the place for the sake of the 50 righteous that are found there? \v{25}Far be it from you to do such a thing---to kill the righteous along with the wicked, so that the righteous and the wicked are treated alike! The Judge of all the earth will do what is right, won't he?''

\v{26}The \divine{Lord} said, ``If I find 50 righteous people within Sodom, I'll forgive the whole place for their sake.''

\v{27}Abraham answered, ``Look, even though I am only dust and ashes, I've ventured to speak to my \divine{Lord}. \v{28}What if there are five less than 50 righteous ones? Will you bring destruction upon the city because of those five?''

The \divine{Lord}\fnote{\fbackref{18:28} Lit. \fbib{He}} said, ``I won't destroy it if I find 45 there.''

\v{29}Abraham\fnote{\fbackref{18:29} Lit. \fbib{He}} continued to speak to him, asking, ``What if 40 are found there?''

The \divine{Lord}\fnote{\fbackref{18:29} Lit. \fbib{He}} replied, ``I won't do it for the sake of those 40.''

\v{30}Abraham\fnote{\fbackref{18:30} Lit. \fbib{He}} then asked, ``I hope my \divine{Lord} will not be angry if I speak. What if 30 are found there?''

The \divine{Lord}\fnote{\fbackref{18:30} Lit. \fbib{He}} answered, ``I won't do it for the sake of those 30.''

\v{31}``Look,'' Abraham\fnote{\fbackref{18:31} Lit. \fbib{He}} said, ``I've presumed to speak to my \divine{Lord}{\ldots} so what if 20 are found there?''

``For the sake of those 20,'' the \divine{Lord}\fnote{\fbackref{18:31} Lit. \fbib{He}} responded, ``I won't destroy it.''

\v{32}Finally, Abraham\fnote{\fbackref{18:32} Lit. \fbib{He}} inquired, ``I hope my \divine{Lord} will not be angry if I speak only once more. What if ten are found there?''

He replied, ``For the sake of those ten I won't destroy it.''

\v{33}As soon as he finished talking to Abraham, the \divine{Lord} left and Abraham returned to where he had been sitting.\fnote{\fbackref{18:33} Lit. \fbib{to his place}}
\labelchapt{19}
\passage{Sodom's Depravity}

\chapt{19}
\v{1}The two angels entered Sodom at sunset while Lot was sitting in the gate area of the city.\fnote{\fbackref{19:1} Lit. \fbib{of Sodom}} When Lot saw them,\fnote{\fbackref{19:1} The Heb. lacks \fbib{them}} he got up, greeted them, bowed low with his face to the ground, \v{2}and said, ``Look, my lords! Please come inside your servant's house, wash your feet, and spend the night. Then you can get up early and be on your way.''

But they responded, ``No, we would rather spend the night in the town square.''

\v{3}But Lot\fnote{\fbackref{19:3} Lit. \fbib{he}} kept urging them strongly, so they turned aside and entered his house. He prepared a festival and baked unleavened flat bread for them, and they ate.

\v{4}Before they could lie down, all the men of Sodom and its outskirts, both young and old, surrounded the house. \v{5}They called out to Lot and asked, ``Where are the men who came to visit\fnote{\fbackref{19:5} The Heb. lacks \fbib{visit}} you tonight? Bring them out to us so we can have sex with\fnote{\fbackref{19:5} Lit. \fbib{can know}} them!''

\v{6}Lot went outside to them, shut the door behind him, \v{7}and said, ``I urge you, my brothers, don't do such a wicked thing. \v{8}Look here, I have two daughters who are virgins.\fnote{\fbackref{19:8} Lit. \fbib{have not known a man}} Let me bring them out to you, and you may do to them whatever you wish,\fnote{\fbackref{19:8} Lit. \fbib{seems good in your eyes}} only don't do anything to these men, because they're here under my protection.''\fnote{\fbackref{19:8} Lit. \fbib{under the shadow of my roof}}

\v{9}But they replied, ``Get out of the way! This man came here as a foreigner, and now he's acting like a judge! So we're going to deal more harshly with you than with them.'' Then they pushed hard against the man (that is, against Lot), intending to break down the door.

\v{10}But the angels\fnote{\fbackref{19:10} Lit. \fbib{men}} inside reached out, dragged Lot back into the house with them, shut the door, \v{11}and blinded the men who were at the entrance of the house, from the least important to the greatest, so they were unable to find the doorway.
\passage{Lot Negotiates with the Angels}

\v{12}``Do you have anyone else here in the city?'' the angels\fnote{\fbackref{19:12} Lit. \fbib{men}} asked Lot. ``A son-in-law? Sons? Daughters? Get them out of this place, \v{13}because we're going to destroy it. The \divine{Lord} knows how their behavior stinks,\fnote{\fbackref{19:13} Lit. \fbib{how great is their outcry}} so he\fnote{\fbackref{19:13} Lit. \fbib{so the \divine{Lord}}} sent us here to destroy it!''

\v{14}Lot then went out and told his sons-in-law (they had married his daughters), ``Get out of here! The \divine{Lord} is going to destroy this city!'' But his sons-in-law thought he was joking.

\v{15}As dawn was breaking, the angels pressured Lot. ``Get going!'' they told him. ``Take your wife and your two daughters who are here, or you will be engulfed by the devastation that's coming to this city.''

\v{16}But Lot kept lingering in the city,\fnote{\fbackref{19:16} The Heb. lacks \fbib{in the city}} so the men\fnote{\fbackref{19:16} Or \fbib{angels}} grabbed his hand and the hands of his wife and two daughters (because of the \divine{Lord}'s compassion for him!), brought them out of the city, and left them outside. \v{17}Then one of them said, ``Flee for your lives! Don't look back or stop anywhere on the plain. Escape to the hills, or you'll be swept away!''

\v{18}``No! Please, my lords!'' Lot pleaded with them. \v{19}``Your servant has found favor in your sight, and you have shown me your gracious love in how you have dealt with me by keeping me alive. I cannot escape to the hills, because I'm afraid the disaster will overtake me, and I'll die. \v{20}Look, there is a town nearby where I can flee, and it's a small one. Let me escape there! It's a small one, isn't it? That way I'll stay alive!''

\v{21}``All right,'' the angel replied to Lot,\fnote{\fbackref{19:21} Lit. \fbib{him}} ``I'll agree with your request!\fnote{\fbackref{19:21} Lit. \fbib{I'll lift up your face in this matter as well!}} I won't overthrow the town that you mentioned. \v{22}Hurry up and flee there, because I cannot do anything until you get to that town.'' Therefore the name of the town was called Zoar.\fnote{\fbackref{19:22} The Heb. name \fbib{Zoar} means \fbib{small}}
\passage{Lot's Wife Becomes a Pillar of Salt}

\v{23}The sun had risen over the land about the time Lot reached Zoar. \v{24}Then the \divine{Lord} rained sulfur and fire out of the sky from the \divine{Lord} on Sodom and Gomorrah, \v{25}overthrowing those cities, all of the plain, and everyone who lived in the cities. He also destroyed the plants that grew out of the ground. \v{26}But Lot's\fnote{\fbackref{19:26} Lit. \fbib{his}} wife looked back as she lingered behind him, and she became a pillar of salt.

\v{27}Abraham went early in the morning to the place where he had stood before the \divine{Lord} earlier. \v{28}He looked off toward Sodom, Gomorrah, and the entire\fnote{\fbackref{19:28} Lit. \fbib{entire land of the}} plain, and he saw smoke rising from the land like smoke from a furnace. \v{29}And so it was that, when God destroyed the cities of the plain, he remembered Abraham and brought Lot out from the midst of the destruction when he overthrew the cities where Lot had lived.
\passage{The Origin of Moab and Ammon}

\v{30}Later on, Lot and his two daughters abandoned Zoar and settled in the hills because Lot was afraid to live in Zoar. He lived there in a cave, along with his two daughters. \v{31}One day the firstborn told the younger one, ``Our father is old, and there's no man in the land to have sex with us,\fnote{\fbackref{19:31} Lit. \fbib{to come in to us}} as everybody else throughout all the earth does. \v{32}Come on! Let's make our father drink wine, and then we'll have sex with him so we can preserve our father's lineage.''

\v{33}So they had their father drink wine that night, and the older one had sexual relations with her father, but he was not aware when she lay down or when she got up. \v{34}The next day the firstborn told the younger one, ``Look! I had sex with my father last night. Let's make him drink wine tonight again as well. Then you have sex with him, too. That way we'll preserve our father's lineage.'' \v{35}So they made their father drink wine that night as well, so he was not aware when she lay down or when she got up.

\v{36}That's how both of Lot's daughters became pregnant by their father. \v{37}The firstborn gave birth to a son and named him Moab,\fnote{\fbackref{19:37} The Heb. name \fbib{Moab} means \fbib{from my father}} and he is the ancestor of the Moabites to this day. \v{38}The younger daughter also gave birth to a son and named him Ben-ammi,\fnote{\fbackref{19:38} The Heb. name \fbib{Ben-ammi} means \fbib{son of my people}} and he is the ancestor of the Ammonites to this day.
\labelchapt{20}
\passage{Abraham and Abimelech}

\chapt{20}
\v{1}Abraham traveled from there to the Negev\fnote{\fbackref{20:1} I.e. the southern regions of the Sinai peninsula; cf. Josh 10:40} and settled between Kadesh and Shur. While he was living in Gerar as an outsider, \v{2}because Abraham kept saying about his wife Sarah, ``She is my sister,'' King Abimelech of Gerar summoned them and took Sarah into his household.\fnote{\fbackref{20:2} The Heb. lacks \fbib{into his household}}

\v{3}But God came to Abimelech in a dream during the night and spoke to him, ``Pay attention! You're about to die, because the woman you have taken is a man's wife!''

\v{4}Now Abimelech had not yet come near her, so he asked, ``\divine{Lord}, will you destroy an innocent nation? \v{5}Didn't he say to me, `She's my sister'? And she also said, `He's my brother.' I did this with pure intentions and clean hands.''

\v{6}Then God replied to him in the dream, ``I know that you did this with pure intentions, and it was I who kept you from sinning against me. Therefore, I didn't allow you to touch her. \v{7}Now then, return the man's wife. As a matter of fact, he's a prophet and can intercede for you so you'll live. But if you don't return her, be aware that you and all who are yours will certainly die.''

\v{8}So Abimelech got up early the next morning, summoned all his servants, and told them all these things. The men became terrified.

\v{9}Then Abimelech called Abraham and asked him, ``What have you done to us? How have I sinned against you, that you have brought such great sin against me and my kingdom? You've done things to me that ought not to have been done.''

\v{10}Abimelech also asked Abraham, ``What could you have been thinking when you did this?''

\v{11}``I thought that there's no fear of God in this place,'' Abraham replied, ``and that they would kill me because of my wife. \v{12}Besides, she really is my sister---she's my father's daughter, but not my mother's daughter---so she could become my wife. \v{13}When God caused me to journey from my father's house, I asked her to do me this favor and say,\fnote{\fbackref{20:13} Lit. \fbib{say about me}} `He's my brother.'\,''

\v{14}So Abimelech took some sheep and oxen, and some male and female servants, gave them to Abraham, returned his wife Sarah to him, \v{15}and said, ``Look! My land is available to you, so settle wherever you please.''

\v{16}Abimelech also told Sarah, ``Look! I am giving your brother 1,000 pieces of silver to vindicate\fnote{\fbackref{20:16} Lit. \fbib{to serve as a cloak for}} you in the eyes of all who are with you. As a result, you will be completely vindicated.''

\v{17}Then Abraham interceded with God, and God healed Abimelech, his wife, and his female servants so they could bear children, \v{18}since the \divine{Lord} had made all the women barren\fnote{\fbackref{20:18} Lit. \fbib{had closed all the wombs}} in Abimelech's household on account of Abraham's wife Sarah.
\labelchapt{21}
\passage{Isaac is Born}

\chapt{21}
\v{1}The \divine{Lord} came to Sarah, just as he had said, and the \divine{Lord} did for Sarah what he had promised. \v{2}Sarah conceived and gave birth to a son for Abraham in his old age, at the very time that God had told him.

\v{3}Abraham named his son who was born to him Isaac---the very one whom Sarah bore for him! \v{4}On the eighth day after his son Isaac had been born,\fnote{\fbackref{21:4} Lit. \fbib{Isaac was a son of eight days when}} Abraham circumcised him, just as God had commanded him. \v{5}Abraham was 100 years old when his son Isaac was born to him.

\v{6}Now Sarah had said, ``God has caused me to laugh,\fnote{\fbackref{21:6} The Heb. name \fbib{Isaac} means \fbib{laughter}} and all who hear about it\fnote{\fbackref{21:6} The Heb. lacks \fbib{about it}} will laugh with me.'' \v{7}She also said, ``Who would have told Abraham that Sarah would nurse sons? Yet I have given birth to a son in my husband's\fnote{\fbackref{21:7} Lit. \fbib{in his}} old age!''
\passage{Hagar and Ishmael Leave}

\v{8}The child grew and eventually was weaned, so Abraham threw a tremendous banquet on the very day Isaac was weaned. \v{9}Nevertheless, when Sarah saw the son of Hagar the Egyptian---whom Hagar had borne to Abraham---making fun of Isaac,\fnote{\fbackref{21:9} The Heb. lacks \fbib{of Isaac}} \v{10}she told Abraham, ``Throw out this slave girl, along with her son, because this slave's son will never be a co-heir with my son Isaac!''

\v{11}Abraham was very troubled about what was being said about his son, \v{12}but God told Abraham, ``Don't be troubled about the youth and your slave girl. Pay attention to Sarah in everything she tells you, because your offspring are to be named through Isaac. \v{13}Nevertheless, I will make the slave girl's son into a nation, since he, too, is your offspring.''

\v{14}So early the next morning, Abraham got up, took bread and a leather bottle of water, gave them to Hagar, and placed them on her shoulder. He then sent her away, along with the child. She went off and roamed in the Beer-sheba wilderness. \v{15}Eventually, the water in the leather bottle ran out, so she placed the child under one of the bushes. \v{16}Then she went and sat by herself about a distance of a bowshot away, because she kept saying to herself, ``I can't bear to watch the child die!'' That's why she sat a short distance away, crying aloud and weeping.
\passage{The \divine{Lord} Rescues Hagar and Ishmael}

\v{17}God heard the boy's voice, and the angel of God called to Hagar from heaven. He asked her, ``What's wrong with you, Hagar? Don't be afraid, because God has heard the voice of the youth where he is. \v{18}Get up! Pick up the youth and grab his hand, because I will make a great nation of his descendants.''\fnote{\fbackref{21:18} Lit. \fbib{of him}} \v{19}Then God opened her eyes, and she saw a well of water. So she went, filled the skin with water, and gave the boy a drink. \v{20}God was with the boy as he grew up. He settled in the wilderness and became an expert archer. \v{21}Later he settled in the desert area of Paran, and his mother chose a wife for him from the land of Egypt.
\passage{A Covenant with Abimelech}

\v{22}About that time, Abimelech and Phicol, the commander of his army, told Abraham, ``God is with you in everything that you're doing. \v{23}Therefore swear an oath here by God that you won't deal falsely with me, my sons, or my descendants. Just as I've dealt graciously with you, won't you do so with me and with the land in which you live as a foreigner?''

\v{24}And Abraham replied, ``I agree!'' \v{25}But then Abraham complained to Abimelech about a well of water that Abimelech's servants had seized.

\v{26}``I don't know who did this thing,'' Abimelech replied. ``You didn't report this to me, and I didn't hear about it until today.''

\v{27}So Abraham took sheep and oxen and presented them to Abimelech, and the two of them made a covenant. \v{28}Then Abraham set aside seven ewe lambs, \v{29}so Abimelech asked Abraham, ``What is the meaning of these seven ewe lambs that you have set aside?''

\v{30}He replied, ``You are to accept from me these seven ewe lambs as a witness that I have dug this well.'' \v{31}Therefore that place was called Beer-sheba, because the two of them swore an oath.\fnote{\fbackref{21:31} The name \fbib{Beer-sheba} in Heb. means \fbib{well of the seven-fold oath}} \v{32}So after they had made a covenant in Beer-sheba, Abimelech and Phicol, the commander of his army, left and returned to Philistine territory.

\v{33}Abraham\fnote{\fbackref{21:33} Lit. \fbib{He}} planted a tamarisk tree in Beer-sheba, and there he called on the name of the \divine{Lord} God Everlasting. \v{34}After this, Abraham resided as a foreigner in Philistine territory for a long period of time.
\labelchapt{22}
\passage{The Command to Offer Isaac}

\chapt{22}
\v{1}Sometime later, God tested Abraham. He called out to him, ``Abraham!''

``Here I am!'' he answered.

\v{2}God\fnote{\fbackref{22:2} Lit. \fbib{He}} said, ``Please take your son, your unique son whom you love---Isaac---and go to the land of Moriah. Offer him as a burnt offering there on one of the mountains that I will point out to you.''

\v{3}So Abraham got up early in the morning, saddled his donkey, and took two of his male servants\fnote{\fbackref{22:3} Or \fbib{young men}} with him, along with his son Isaac. He cut the wood for the burnt offering and set out to go to the place about which God had spoken to him. \v{4}On the third day he looked ahead and saw the place from a distance.

\v{5}Abraham ordered his two servants,\fnote{\fbackref{22:5} Or \fbib{young men}} ``Both of you are to stay here with the donkey. Now as for the youth and me, we'll go up there, we'll worship, and then we'll return to you.'' \v{6}Then Abraham took the wood for the burnt offering and placed it on his son Isaac. Abraham\fnote{\fbackref{22:6} Lit. \fbib{He}} carried the fire and the knife. And so the two of them went on together.
\passage{Abraham Answers Isaac's Question}

\v{7}Isaac addressed his father Abraham: ``My father!''

``I'm here, my son,'' Abraham replied.

Isaac asked, ``The fire and the wood are here, but where's the lamb for the burnt offering?''

\v{8}Abraham answered, ``God will provide\fnote{\fbackref{22:8} Or \fbib{will see to}} himself the lamb for the burnt offering, my son.''

The two of them went on together \v{9}and came to the place about which God had spoken. Abraham built an altar there, arranged the wood, tied up his son Isaac, and placed him on the altar on top of the wood. \v{10}Then he stretched out his hand and grabbed the knife to slaughter his son.
\passage{The Angel of the \divine{Lord} Intervenes}

\v{11}Just then, an angel of the \divine{Lord} called out to him from heaven and said, ``Abraham! Abraham!''

``Here I am,'' he answered.

\v{12}``Don't lay your hand on the youth!'' he said. ``Don't do anything to him, because I've just demonstrated\fnote{\fbackref{22:12} Lit. \fbib{because now I know}} that you fear God, since you have not withheld your son, your only unique one, from me.''

\v{13}Then Abraham looked up and behind him to see a ram caught by its horns in the thicket. So Abraham went over, grabbed the ram, and offered it as a burnt offering in place of his son. \v{14}Abraham named that place, ``The \divine{Lord} Will Provide,''\fnote{\fbackref{22:14} Or \fbib{Will See To It}} as it is told this day, ``On the \divine{Lord}'s mountain, he will provide.''\fnote{\fbackref{22:14} Or \fbib{will see to it}}

\v{15}The angel of the \divine{Lord} called to Abraham a second time from heaven \v{16}and said, ``I have taken an oath to swear by myself,'' declares the \divine{Lord}, ``that since you have carried this out and have not withheld your only unique\fnote{\fbackref{22:16} Or \fbib{only}} son, \v{17}I will certainly bless you and make your descendants as numerous as the stars in heaven and as the sand on the seashore. Your descendants will take possession of the gates\fnote{\fbackref{22:17} I.e. the centers of power in their cities} of their enemies. \v{18}Furthermore, through your descendants all the nations of the earth will be blessed,\fnote{\fbackref{22:18} Or \fbib{gain blessing for themselves}} because you have obeyed my command.''

\v{19}After this, Abraham returned to his servants\fnote{\fbackref{22:19} Or \fbib{young men}} and they set out together for Beer-sheba, where Abraham settled.
\passage{Nahor's Children}

\v{20}Now after these things somebody told Abraham, ``Look, Milcah has given birth to sons for your brother Nahor. \v{21}Uz is his firstborn, Buz is his brother, and Kemuel is the father of Aram, \v{22}Chesed, Hazo, Pildash, Jidlaph, and Bethuel.'' \v{23}Bethuel fathered Rebekah. Milcah bore these eight sons to Nahor, Abraham's brother. \v{24}Also, his concubine Reumah gave birth to Tebah, Gaham, Tahash, and Maacah.
\labelchapt{23}
\passage{A Burial Place for Sarah}

\chapt{23}
\v{1}Sarah lived for 127 years. That's how long Sarah's life was. \v{2}She died in Kiriath-arba (that is, in Hebron) in the land of Canaan. Abraham went in\fnote{\fbackref{23:2} I.e. into Sarah's tent} to mourn for Sarah and to weep for her. \v{3}Then Abraham stood up from beside his dead wife\fnote{\fbackref{23:3} The Heb. lacks \fbib{wife}} and addressed the Hittites. He said, \v{4}``I am an alien and an outsider among you. Give me a cemetery among you where I can bury my dead away from my presence.''

\v{5}The Hittites responded to Abraham, \v{6}``Listen to us, sir.\fnote{\fbackref{23:6} Lit. \fbib{us, my lord}} You are a mighty prince\fnote{\fbackref{23:6} MT reads \fbib{a prince of God;} LXX reads \fbib{a king of God}} among us. Bury your dead in the choicest of our burial tombs. None of us would refuse you his tomb for burying your dead.''

\v{7}Abraham rose and bowed before the Hittites, the people of the land, \v{8}and addressed them, ``If you are willing that I should bury my dead out of my sight, listen to me and make a request of Zohar's son Ephron on my behalf. \v{9}Give me the cave of Machpelah that belongs to him, at the end of his field. He should sell\fnote{\fbackref{23:9} Lit. \fbib{give}} it to me in your presence at full price for a burial site.''

\v{10}Now since Ephron the Hittite had taken a seat there among the Hittites, he responded publicly to Abraham where the Hittites and everyone who was entering the gate of his city could hear him: \v{11}``No, sir.\fnote{\fbackref{23:11} Lit. \fbib{No, my lord}} Listen to me! I'll give you the field, and I'll give you the cave that's in it. I give it to you publicly, in the sight of my people. Bury your dead.''

\v{12}Abraham bowed before the people of the land \v{13}and then addressed Ephron so all the people of the land could hear him: ``Please listen to me! I'm willing to pay the price of the field. Accept it from me, so I may bury my dead there.''

\v{14}So Ephron answered Abraham, \v{15}``Sir,\fnote{\fbackref{23:15} Lit. \fbib{My lord}} listen to me! The land is worth 400 shekels of silver, but what's that between us? You may bury your dead.''

\v{16}Abraham agreed with Ephron, so he\fnote{\fbackref{23:16} Lit. \fbib{Abraham}} weighed out to Ephron the money to which he had agreed publicly while the Hittites were listening: 400 shekels of silver at the current merchant rate.

\v{17}That's how Ephron's field in Machpelah, east of\fnote{\fbackref{23:17} Lit. \fbib{which faces} or \fbib{is before}} Mamre---the field, the cave that was in it, and all the trees that were within the boundaries of\fnote{\fbackref{23:17} Lit. \fbib{within the area around}} the field---came to be deeded \v{18}to Abraham in the presence of all the Hittites and everyone who was entering the city gate. \v{19}After this, Abraham buried his wife Sarah in the cave at the field of Machpelah, east of Mamre (that is, in Hebron) in the land of Canaan. \v{20}And so the field with its cave was deeded by the Hittites to Abraham as a burial site.
\labelchapt{24}
\passage{Finding a Bride for Isaac}

\chapt{24}
\v{1}Now Abraham had grown old, was well advanced in age, and the \divine{Lord} had blessed Abraham in every way. \v{2}So Abraham instructed his servant, who was the oldest member of his household and in charge of everything he owned, ``Make this solemn oath to me\fnote{\fbackref{24:2} Lit. \fbib{Place your hand under my thigh}; i.e., to make a solemn promise based on the sanctity of the family and commitment to the family line} \v{3}as a promise to the \divine{Lord}, the God of heaven and earth, that you won't acquire a wife for my son from the Canaanite women among whom I'm living. \v{4}Instead, you are to go to my country and to my family and acquire a wife for my son Isaac.''

\v{5}``What if the woman doesn't want to come back with me to this land?'' the servant asked. ``Shouldn't I have your son go to the land from which you came?''

\v{6}``Make sure not to take my son there,'' Abraham replied. \v{7}``The \divine{Lord} God of heaven, who brought me from my father's house and from my family's land, who spoke to me and promised me `I will give this land to your descendants,' will send his angel ahead of you, and you are to acquire a wife for my son from there. \v{8}If the woman isn't willing to follow you, then you'll be free from this oath to me. Just don't take my son back there!'' \v{9}So the servant made a solemn oath\fnote{\fbackref{24:9} Lit. \fbib{servant placed his hand under Abraham's thigh}; i.e., to make a solemn promise based on the sanctity of the family and commitment to the family line} to his master Abraham regarding this matter.
\passage{The Servant Encounters Rebekah}

\v{10}Then Abraham's servant took ten camels from his master's herd of\fnote{\fbackref{24:10} The Heb. lacks \fbib{herd of}} camels and left on his journey with all kinds of gifts from his master's inventory. Eventually, he traveled as far as Aram-naharaim, Nahor's home town. \v{11}As evening approached, he had the camels kneel outside the town at the water well, right about the time when women customarily went out to draw water.

\v{12}That's when he prayed, ``\divine{Lord} God of my master Abraham, help me to succeed today. Please show your gracious love to my master Abraham. \v{13}I've stationed myself here by the spring as the women of the town come to draw water. \v{14}May it be that the young woman to whom I ask, `Please, lower your jug so that I may drink,' responds, `Have a drink, and I'll water your camels as well.' May she be the one whom you have chosen for your servant Isaac. This is how I'll know that you have shown your gracious love to my master.''

\v{15}Before he had finished speaking, Rebekah appeared. She was a daughter of Milcah's son Bethuel. (Milcah was the wife of Abraham's brother Nahor.) She approached the well, carrying a jug on her shoulder. \v{16}The woman was very beautiful, young, and had not had sexual relations with a man. Going down to the spring, she filled her jug and turned for home. \v{17}Then Abraham's servant ran to meet her and asked her, ``Please, let me have a sip of water from your jug.''

\v{18}``Drink, sir!'' she replied as she quickly lowered her jug on her arm to offer him a drink. \v{19}When she had finished giving him a drink, she also said, ``I'll also draw water\fnote{\fbackref{24:19} The Heb. lacks \fbib{water}} for your camels until they've had enough to drink.''

\v{20}She quickly emptied her jug into the trough and ran to the well to draw again until she had drawn enough water\fnote{\fbackref{24:20} The Heb. lacks \fbib{enough water}} for all ten of the servant's\fnote{\fbackref{24:20} Lit. \fbib{of his}} camels. \v{21}The man stared at her in silence, waiting to see whether or not the \divine{Lord} had made his journey successful. \v{22}When the camels had finished drinking, the man took out a gold nose ring weighing a half shekel and two bracelets for her wrists, weighing 10 shekels and presented them to her.\fnote{\fbackref{24:22} The Heb. lacks \fbib{and presented them to her}}

\v{23}He asked her, ``Whose daughter are you? Please tell me, is there room in your father's house for us to spend the night?''

\v{24}``I am the daughter of Bethuel,'' she answered. ``He's the son of Milcah and Nahor. \v{25}And yes,'' she continued, ``we have plenty of straw and feed, as well as a place to spend the night.''

\v{26}At this, the man bowed down and worshipped the \divine{Lord}. \v{27}``Blessed be the \divine{Lord} God of my master Abraham, who hasn't held back his gracious love and faithfulness from my master! The \divine{Lord} has led me to the house of my master's relatives!''

\v{28}The young woman then ran ahead and informed her mother's household what had happened.
\passage{Rebekah's Brother Laban}

\v{29}Now Rebekah had a brother named Laban, who ran out to the man and met him\fnote{\fbackref{24:29} The Heb. lacks \fbib{and met him}} at the spring. \v{30}And so it was, as soon as he saw the nose ring and bracelets on his sister's wrists, and as soon as he heard what his sister Rebekah was saying about what the man had spoken to her,\fnote{\fbackref{24:30} Lit. \fbib{saying, ``This is what the man spoke to me!''}} he went out to the man who was still standing by the camels at the spring! \v{31}``Come on,'' Laban\fnote{\fbackref{24:31} Lit. \fbib{he}} said. ``The \divine{Lord} has blessed you! So why are you standing out here when I've prepared some space in the house and a place for the camels?''

\v{32}So the servant went to the house and unbridled the camels. They provided straw and feed for the camels and water for washing his feet and those of the men with him. \v{33}But when they had prepared a meal and set it in front of him, he said, ``I'm not eating until I've spoken.''

``Speak up!'' Laban\fnote{\fbackref{24:33} Lit. \fbib{He}} exclaimed.
\passage{The Servant Relates His Adventures}

\v{34}``I'm Abraham's servant,'' he said. \v{35}``The \divine{Lord} has greatly blessed my master, so that he has become wealthy. He has provided him sheep and cattle, silver and gold, male and female servants, camels and donkeys. \v{36}My master's wife Sarah gave birth to my master's son in her old age, and Abraham\fnote{\fbackref{24:36} Lit. \fbib{he}} has given him everything that belongs to him. \v{37}My master made me swear this oath: `You are not to select a wife for my son from among the daughters of the Canaanites in this land where I live. \v{38}Instead, you are to go to my father's household, to my relatives, and choose a wife for my son there.'

\v{39}``So I asked my master, `What if the woman won't come back with me?'

\v{40}``Abraham\fnote{\fbackref{24:40} Lit. \fbib{He}} told me, `The \divine{Lord}, who is with me wherever I go, will send his angel with you to make your journey successful. So you are to choose a wife for my son from my family, from my father's household. \v{41}Only then will you be released from fulfilling\fnote{\fbackref{24:41} The Heb. lacks \fbib{fulfilling}} my oath. However, when you come to my family, if they don't give her to you, you'll be released from fulfilling\fnote{\fbackref{24:41} The Heb. lacks \fbib{fulfilling}} my oath.'

\v{42}``So today I arrived at the spring and prayed, `\divine{Lord} God of my master Abraham, if you wish to make the journey that I have traveled successful, \v{43}here I am standing by the spring. May it be that the young woman who comes out to draw water, from whom I request a little water from her\fnote{\fbackref{24:43} Lit. \fbib{your}} jug to drink, \v{44}if she tells me to drink and also draws water for the\fnote{\fbackref{24:44} Lit. \fbib{your}} camels, may she be the woman that the \divine{Lord} has chosen for my master's son.'

\v{45}``Before I had finished praying, along came Rebekah with her jug on her shoulder! She went to the spring and drew some water. I asked her to please let me have a drink. \v{46}She quickly lowered her jug from her shoulder\fnote{\fbackref{24:46} The Heb. lacks \fbib{shoulder}} and told me, `Have a drink while I also water your camels.' So I drank, and she also gave my camels water\fnote{\fbackref{24:46} The Heb. lacks \fbib{water}} to drink.

\v{47}``That's when I asked, `Whose daughter are you?'

``She replied, `I'm the daughter of Bethuel, Nahor's son, whom Milcah bore for him.'

``So I gave her a ring for her nose and bracelets for her wrists. \v{48}I bowed down and worshipped the \divine{Lord}, and I praised the \divine{Lord} God of my master Abraham, who led me on the true way to request\fnote{\fbackref{24:48} Lit. \fbib{to take}} the daughter of my master's brother for his son. \v{49}So now, if you wish to show gracious love and truth toward my master, tell me so. But if not, tell me, so that I may go elsewhere.''\fnote{\fbackref{24:49} Lit. \fbib{turn to the right or the left}}
\passage{Laban and Bethuel Acquiesce}

\v{50}``Since this has come from the \divine{Lord},'' Laban and Bethuel both replied, ``we cannot speak one way or another.\fnote{\fbackref{24:50} Lit. \fbib{speak bad or good}} \v{51}So here's Rebekah---she's right in front of you. Take her and go, so she can become a wife for your master's son, just as the \divine{Lord} has decreed.''

\v{52}When Abraham's servant heard what they had said, he bowed down to the ground before the \divine{Lord}. \v{53}Then the servant brought out some silver and gold items, along with some clothing, and gave them to Rebekah. He also gave gifts to her brother and to her mother. \v{54}He and the men with him ate and drank, and then they spent the night.
\passage{The Servant Prepares to Leave}

When they got up the next morning, the servant\fnote{\fbackref{24:54} Lit. \fbib{he}} requested, ``Send me off to my master.''

\v{55}But her brother and mother said, ``Let the young lady stay with us a few days---at least ten---and after that she may go.''

\v{56}``Please don't delay me,'' the servant\fnote{\fbackref{24:56} Lit. \fbib{he}} answered them. ``The \divine{Lord} has made my journey successful. Send me off so I can return to my master.''

\v{57}But they said, ``We'll call the young lady and see what she has to say about this.''\fnote{\fbackref{24:57} The Heb. lacks \fbib{about this}}

\v{58}So they called Rebekah and asked her, ``Do you want to go with this man?''

``I will go,'' she replied.

\v{59}So they sent off their sister Rebekah, along with her personal assistant,\fnote{\fbackref{24:59} Lit. \fbib{nurse}; or \fbib{cook}} Abraham's servant, and his men. \v{60}As they were leaving, they all blessed Rebekah by\fnote{\fbackref{24:59} The Heb. lacks \fbib{by}} saying,

\begin{poetry}
\poeml ``Our sister, may you become the mother of tens of millions!\fnote{\fbackref{24:60} Lit. \fbib{of thousands upon ten thousands}} \\
\poemll    May your descendants take over \\
\poemlll       the city gates\fnote{\fbackref{24:60} I.e. the centers of power in their cities} of those who hate them.''\fnote{\fbackref{24:60} Lit. \fbib{him}}
\end{poetry}

\v{61}Then Rebekah and her young servant women got up, mounted their camels, and followed Abraham's servant, who took Rebekah and went on his way.
\passage{Isaac Marries Rebekah}

\v{62}Later on, as Isaac was returning one evening from Beer-lahai-roi\fnote{\fbackref{24:62} Lit. \fbib{The Well of the Living One Who Looks After Me,} cf. Gen. 16:13-14} (he had been living in the Negev\fnote{\fbackref{24:62} I.e. the southern regions of the Sinai peninsula; cf. Josh 10:40}), \v{63}Isaac\fnote{\fbackref{24:63} Lit. \fbib{he}} went out walking\fnote{\fbackref{24:63} Or \fbib{meditating}} in a field. He looked up, and all of a sudden there were some camels coming. \v{64}Rebekah looked up, and when she saw Isaac, she quickly dismounted from her camel \v{65}and asked the servant, ``Who is that man coming in the field to meet us?''

``That's my master,'' the servant told her. So she reached for a veil and covered herself. \v{66}Then the servant informed Isaac about everything he had done. \v{67}Later, Isaac brought Rebekah into the tent that had belonged to his mother Sarah and married her. Isaac loved her, and that's how he was comforted following the loss of\fnote{\fbackref{24:67} The Heb. lacks \fbib{the loss of}} his mother.
\labelchapt{25}
\passage{Abraham Names Isaac to be His Heir}

\chapt{25}
\v{1}Abraham had taken another wife whose name was Keturah. \v{2}She bore him Zimran, Jokshan, Medan, Midian, Ishbak, and Shuah. \v{3}Jokshan was the father of Sheba and Dedan. Dedan's sons were the Asshurites, Letushites, and Leummites. \v{4}Midian's sons were Ephah, Epher, Hanoch, Abida, and Eldaah. All of these were Keturah's descendants.

\v{5}Abraham gave everything he owned to Isaac. \v{6}While he was still alive, Abraham gave gifts to his concubines\fnote{\fbackref{25:6} Lit. \fbib{concubines whom Abraham had.}} and sent them to the east country in order to keep them away from his son Isaac.

\v{7}Abraham lived for 175 years,\fnote{\fbackref{25:7} Lit. \fbib{These are the days of Abraham's years: 175 years}} \v{8}then passed away, dying at a ripe old age, having lived a full life, and joined his ancestors.\fnote{\fbackref{25:8} Lit. \fbib{and he was gathered to his people}} \v{9}His sons Isaac and Ishmael buried him in the cave of Machpelah near Mamre, in the field that used to belong to Zohar the Hittite's son Ephron. \v{10}This was the same field that Abraham had bought from the son of Heth, where Abraham and his wife Sarah were buried. \v{11}After Abraham's death, God blessed his son Isaac, who continued to live near Beer-lahai-roi.
\passage{A Summary of Ishmael's Life}

\v{12}Now this is what happened to Ishmael, whom Sarah's Egyptian servant Hagar bore for Abraham. \v{13}Here's a list of the names of Ishmael's sons, recorded by their names and descendants: Nebaioth was the firstborn, followed by\fnote{\fbackref{25:13} The Heb. lacks \fbib{followed by}} Kedar, Adbeel, Mibsam, \v{14}Mishma, Dumah, Massa, \v{15}Hadad, Tema, Jetur, Naphish, and Kedemah. \v{16}These were Ishmael's children, listed by their names according to their villages and their camps. There were a total of twelve tribal chiefs, according to their clans. \v{17}Ishmael lived\fnote{\fbackref{25:17} Lit. \fbib{These are the years of Ishmael's life}} for 137 years, then he took his last breath, died, and joined his ancestors.\fnote{\fbackref{25:17} Lit. \fbib{and he was gathered to his people}} \v{18}His descendants\fnote{\fbackref{25:18} Lit. \fbib{They}} settled from Havilah to Shur (that's near Egypt), all the way to Assyria, in defiance\fnote{\fbackref{25:18} Lit. \fbib{in the face of}} of all of his relatives.
\passage{The Births of Esau and Jacob}

\v{19}This is the account of Isaac, Abraham's son. Abraham fathered Isaac. \v{20}Isaac was forty years old when he married\fnote{\fbackref{25:20} Lit. \fbib{took}} Rebekah, the daughter of Bethuel, the Aramean\fnote{\fbackref{25:20} In later centuries this region would be called Syria} from Paddan-aram\fnote{\fbackref{25:20} Paddan-aram was located in northwest Mesopotamia} and sister of Laban the Aramean.\fnote{\fbackref{25:20} In later centuries this region would be called Syria} \v{21}Later, Isaac prayed to the \divine{Lord} on behalf of his wife, since she was unable to conceive children, and the \divine{Lord} responded to him---his wife Rebekah became pregnant.

\v{22}But when the infants\fnote{\fbackref{25:22} Lit. \fbib{sons}} kept on wrestling each other inside her womb,\fnote{\fbackref{25:22} Or \fbib{within her}} she asked herself, ``Why is this happening?''\fnote{\fbackref{25:22} Lit. \fbib{If so . . . why this I}?} So she asked the \divine{Lord} for an explanation.\fnote{\fbackref{25:22} The Heb. lacks \fbib{for an explanation}}

\v{23}``Two nations\fnote{\fbackref{25:23} Or \fbib{two infants}} are in your womb,'' the \divine{Lord} responded, ``and two separate people will emerge. One people will be the stronger, and the older one will serve the younger.''

\v{24}Sure enough, when her due date arrived, she delivered twin sons.\fnote{\fbackref{25:24} Lit. \fbib{twins from her womb}} \v{25}The first son came out reddish---his entire body was covered with hair---so they named him Esau.\fnote{\fbackref{25:25} The Heb. name \fbib{Esau} means \fbib{hairy}} \v{26}After that, his brother came out with his hand clutching Esau's heel, so they named him Jacob.\fnote{\fbackref{25:26} The Heb. name \fbib{Jacob} means \fbib{heel grabber}} Isaac was 60 years old when they were born.

\v{27}As the boys were growing up, Esau became skilled at hunting and was a man of the outdoors, but Jacob was the quiet type who tended to stay indoors. \v{28}Isaac loved Esau, because he loved to hunt, while Rebekah loved Jacob. \v{29}One day, while Jacob was cooking some stew, Esau happened to come in from being outdoors, and he was feeling famished.

\v{30}Esau told Jacob, ``Let me gobble down some of this red stuff, since I'm starving.'' (That's how Esau got his nickname ``Edom''.)\fnote{\fbackref{25:30} The Heb. name \fbib{Edom} means \fbib{red}}

\v{31}But Jacob responded, ``Sell me your birthright. Do it now.''\fnote{\fbackref{25:31} Lit. \fbib{today}}

\v{32}``Look! I'm about to die,'' Esau replied. ``What good is this birthright to me?''

\v{33}But Jacob insisted, ``Swear it by an oath right now.'' So he swore an oath to him and sold his birthright to Jacob. \v{34}Then Jacob gave Esau some of his food, along with some boiled stew. So Esau ate, drank, got up, and left, after having belittled his own birthright.
\labelchapt{26}
\passage{Isaac Lives in Philistia for a While}

\chapt{26}
\v{1}Later on, a famine swept through the land. This famine was different from the previous famine that had occurred earlier, during Abraham's lifetime. So Isaac went to Abimelech, king of the Philistines, at Gerar.

\v{2}That's when the \divine{Lord} appeared to Isaac.\fnote{\fbackref{26:2} Lit. \fbib{him}} ``You are not to go down to Egypt,'' he said. ``Instead, you are to settle down in an area within this land where I'll tell you. \v{3}Remain in this land, and I'll be with and bless you by giving all these lands to you and to your descendants in fulfillment of my solemn promise that I made to your father Abraham. \v{4}I'll cause you to have as many descendants as the stars of the heavens, and I'll certainly give all these lands to your descendants. Later on, through your descendants all the nations of the earth will bless one another. \v{5}I'm going to do this because Abraham did what I told him to do. He kept my instructions, commands, statutes, and laws.''

\v{6}So Isaac lived in Gerar.
\passage{Isaac Lies about His Wife}

\v{7}Later on, the men of that place asked about his wife, so he replied, ``She's my sister,'' because he was afraid to call her ``my wife.'' He kept thinking, ``{\ldots}otherwise, the men around here will kill me on account of Rebekah, since she's very beautiful.''

\v{8}After he had been there awhile, Abimelech, king of the Philistines, looked out through a window and saw Isaac caressing\fnote{\fbackref{26:8} Or \fbib{fondling}; the Heb. verb is a word play on the name \fbib{Isaac} and sounds like it.} his wife Rebekah.

\v{9}So Abimelech called Isaac and confronted him. ``She is definitely your wife!'' he accused him, ``So why did you claim, `She's my sister?'\,''

Isaac responded, ``Because I had thought `{\ldots}otherwise, I'll die on account of her.'\,''

\v{10}``What have you done to us?'' Abimelech asked. ``Any minute now, one of the people could have had sex with your wife and you would have caused all of us to be guilty.'' \v{11}So he issued this order to everyone: ``Whoever touches this man or his wife is to be executed.''
\passage{Isaac Grows Wealthy}

\v{12}Isaac received a 100-fold return on what he planted that year in the land he received,\fnote{\fbackref{26:12} Lit. \fbib{found}} because the \divine{Lord} blessed him. \v{13}He\fnote{\fbackref{26:13} Lit. \fbib{The man}} became very wealthy and lived a life of wealth,\fnote{\fbackref{26:13} Lit. \fbib{and walked}} becoming more and more wealthy. \v{14}He owned so many sheep, cattle, and servants that the Philistines eventually became envious of him. \v{15}They\fnote{\fbackref{26:15} Lit. \fbib{The Philistines}.} filled in with sand all of the wells that Isaac's\fnote{\fbackref{26:15} Lit. \fbib{his}} father Abraham's servants had dug during his lifetime. \v{16}Then Abimelech ordered Isaac, ``Move away from us! You've become more powerful than we are.'' \v{17}So Isaac moved from there and encamped in the Gerar Valley, where he settled.
\passage{Disputes over Water Rights}

\v{18}Isaac re-excavated some wells that his father had first dug during his lifetime, because the Philistines had filled them with sand\fnote{\fbackref{26:18} The Heb. lacks \fbib{with sand}} after Abraham's death. Isaac\fnote{\fbackref{26:18} Lit. \fbib{He}} renamed those wells with the same names that his father had called them.

\v{19}While Isaac's servants were digging in the valley, they discovered a well with flowing water. \v{20}But the herdsmen who lived in Gerar quarreled with Isaac's herdsmen. ``The water is ours,'' they said. As a result, Isaac named the well Esek,\fnote{\fbackref{26:20} The Heb. name \fbib{Esek} means \fbib{disputed}} for they had fiercely disputed with him about it. \v{21}When his workers started digging another well, those herdsmen\fnote{\fbackref{26:21} Lit. \fbib{well, they}} quarreled about that one, too, so Isaac\fnote{\fbackref{26:21} Lit. \fbib{he}} named it Sitnah.\fnote{\fbackref{26:21} The Heb. name \fbib{Sitnah} means \fbib{strife}} \v{22}Then he left that area and dug still another well. Because they did not quarrel over that one, Isaac\fnote{\fbackref{26:22} Lit. \fbib{he}} named it Rehoboth,\fnote{\fbackref{26:22} The Heb. name \fbib{Rehoboth} means \fbib{wide places}} because he used to say, ``The \divine{Lord} has enlarged the territory\fnote{\fbackref{26:22} The Heb. lacks \fbib{the territory}} for us. We will prosper in the land.''
\passage{God Renews His Promise to Isaac}

\v{23}Later on, he left there and went to Beer-sheba, \v{24}where one night the \divine{Lord} appeared to him. ``I am the God of your father Abraham,'' he told him. ``Don't be afraid, because I'm with you. I'm going to bless you and multiply your descendants on account of my servant Abraham.'' \v{25}In response, Isaac built an altar there and called on the name of the \divine{Lord}. He also pitched his tents there and his servants dug a well.
\passage{Abimelech Requests a Covenant}

\v{26}Later, Abimelech traveled from Gerar to visit Isaac\fnote{\fbackref{26:26} Lit. \fbib{him}}. He arrived with Ahuzzath, his staff advisor, and Phicol, the commanding officer of his army.

\v{27}``Why have you come to see me,'' Isaac asked them, ``since you hate me so much that you sent me away from you?''

\v{28}``We've seen that the \divine{Lord} is with you,'' they responded, ``so we're proposing an agreement\fnote{\fbackref{26:28} Lit. \fbib{oath}} between us---between us and you. Allow us to make a treaty with you \v{29}by which you'll agree not to do us any harm, just as we haven't harmed\fnote{\fbackref{26:29} Lit. \fbib{touched}} you, since we've done nothing but good for you after we sent you away in peace. As a result, you've been tremendously blessed by the \divine{Lord}.'' \v{30}So Isaac\fnote{\fbackref{26:30} Lit. \fbib{he}} held a festival for them, and they ate and drank. \v{31}They woke up early the next morning and made the treaty.\fnote{\fbackref{26:31} Lit. \fbib{and swore an oath one to another.}} After this, Isaac sent them off and they left on peaceful terms.

\v{32}That very same day, Isaac's servants arrived and reported to him about a well that they had just completed digging. ``We've found water!'' they said. \v{33}So Isaac\fnote{\fbackref{26:33} Lit. \fbib{he}} named the well Shebah,\fnote{\fbackref{26:33} The Heb. name \fbib{Shebah} means \fbib{oath}} which is why the city is named Beer-sheba\fnote{\fbackref{26:33} The Heb. name \fbib{Beer-sheba} means \fbib{Well of the Oath}} to this day.
\passage{Esau Causes Trouble for Isaac}

\v{34}When Esau was 40 years old, he married\fnote{\fbackref{26:34} Lit. \fbib{he took as a wife}} Judith, the daughter of Beeri the Hittite and Basemath, the daughter of Elon the Hittite. \v{35}This brought extreme grief to Isaac and Rebekah.
\labelchapt{27}
\passage{The Theft of Esau's Blessing}

\chapt{27}
\v{1}Eventually, Isaac grew so old that he could not see.\fnote{\fbackref{27:1} Lit. \fbib{that his eyes were dim}} One day, he called his eldest son Esau. ``My son,'' he called out to him. \v{2}``Look how old I am! I could die any day now,\fnote{\fbackref{27:2} Lit. \fbib{I don't know the day of my death}} \v{3}so go find your weapons, take your bow and arrows, go outside, and hunt some game for me. \v{4}Then prepare some food, just the way I like it, and bring it to me so that I can eat and bless you before I die.''

\v{5}Now Rebekah overheard Isaac while he was speaking to his son Esau. When Esau had gone out to the field to hunt and bring in some game, \v{6}Rebekah gave these instructions to her son Jacob: ``Quick! Pay attention!'' she said. ``I heard your father talking to your brother Esau. He told him, \v{7}`Bring me some game and then prepare some food for me so I can eat and bless you in the presence of the \divine{Lord} before I die.' \v{8}So now, my son, listen to what I have to say and pay attention to what I'm about to tell you. \v{9}Go to the flock and bring me two healthy young goats. I'll prepare some delicious food for your father, just the way he loves it. \v{10}Then you are to take it to your father so that he can eat and bless you before he dies.''

\v{11}``But look!'' Jacob pointed out to his mother Rebekah, ``My brother Esau is a hairy man, but I'm smooth skinned. \v{12}My father might touch me and he'll realize that I'm deceiving him. Then, I'll bring a curse on myself instead of a blessing.''

\v{13}``My son,'' she replied, ``let any curse against you fall on me. Just listen to me, then go and get them for me.'' \v{14}So out he went, got them, and brought them to his mother, who then prepared some delicious food, just the way his father liked it.
\passage{Rebekah and Jacob Deceive Isaac}

\v{15}Then Rebekah took some garments that belonged to her elder son Esau---the best ones available---and put them on her younger son Jacob. \v{16}She put some goat skins over his hands and on the smooth part of his neck. \v{17}Then she handed the delicious food and bread that she had prepared to her son Jacob, \v{18}who went to his father and said, ``My father{\ldots}''

``It's me!'' he replied. ``Which one are you, my son?''

\v{19}``I'm Esau, your firstborn!'' Jacob told his father. ``I've done what you asked, so please sit up and eat what I caught, so you can bless me.''

\v{20}``How did you get it so quickly, my son?'' Isaac asked.

Jacob\fnote{\fbackref{27:20} Lit. \fbib{He}} responded, ``{\ldots}because the \divine{Lord} your God made me successful.''

\v{21}So Isaac told Jacob, ``Come here, my son, so I can feel you and know for sure whether or not you're my son Esau.''

\v{22}So Jacob approached his father, who felt him and said, ``It's Jacob's voice, but Esau's hands.'' \v{23}He didn't recognize Jacob,\fnote{\fbackref{27:23} Lit. \fbib{him}} because his hands were hairy like those of his brother Esau, so Isaac\fnote{\fbackref{27:23} Lit. \fbib{he}} blessed him.

\v{24}He asked, ``Are you really my son Esau?''

``I am,'' Jacob\fnote{\fbackref{27:24} Lit. \fbib{he}} replied.

\v{25}``Come closer to me,'' Isaac replied, ``so I can eat some of the game, my son, and then bless you.'' So Jacob came closer, and Isaac ate. Jacob also brought wine so his father\fnote{\fbackref{27:25} Lit. \fbib{so he}} could drink. \v{26}After this, Jacob's father Isaac told him, ``Come closer and kiss me, my son.'' \v{27}So Jacob\fnote{\fbackref{27:27} Lit. \fbib{he}} drew closer to kiss him. When Isaac\fnote{\fbackref{27:27} Lit. \fbib{he}} smelled the scent of his son's\fnote{\fbackref{27:27} The Heb. lacks \fbib{son's}} clothes, he blessed him and said,

\begin{poetry}
\poeml ``How my son's scent is the fragrance of the field \\
\poemll    that the \divine{Lord} has blessed. \\
\poeml \v{28}May the \divine{Lord} grant you dew from the skies,\fnote{\fbackref{27:28} Or \fbib{from heaven}} \\
\poemll    and from the fertile land; \\
\poeml may he grant you\fnote{\fbackref{27:28} The Heb. lacks \fbib{may he grant you}} \\
\poemll    abundant grain and fresh wine. \\
\poeml \v{29}May people serve and bow before you; \\
\poemll    may you be master over your brothers; \\
\poeml may your mother's sons bow before you; \\
\poemll    may anyone who curses you be cursed; \\
\poemlll       and may anyone who blesses you be blessed.''
\end{poetry}
\passage{Esau Learns of Isaac's Deception}

\v{30}Just after Isaac had finished blessing Jacob and Jacob had left his father Isaac, Jacob's\fnote{\fbackref{27:30} Lit. \fbib{his}} brother Esau returned from hunting, \v{31}prepared some delicious food, brought it to his father, and told him, ``Can you get up now, father, so you may eat some of your son's game and then bless me?''

\v{32}But his father Isaac asked him, ``Who are you?''

``I'm Esau, your firstborn son,'' he answered

\v{33}At this, Isaac began to tremble violently. ``Who then,'' he asked, ``hunted some game and brought it to me to eat before you arrived, so that I've blessed him? Indeed, he is blessed.''

\v{34}When Esau realized\fnote{\fbackref{27:34} Lit. \fbib{heard}} what his father Isaac was saying, he began to wail out loud bitterly. ``Bless me,'' he cried, ``even me, too, my father!''

\v{35}Isaac\fnote{\fbackref{27:35} Lit. \fbib{Then he}} replied, ``Your brother came here deceitfully and stole your blessing.''

\v{36}Then he said, ``Isn't his name rightly called Jacob?''\fnote{\fbackref{27:36} The Heb. name \fbib{Jacob} means \fbib{heel grabber}} Esau asked. ``He has circumvented me this second time. First,\fnote{\fbackref{27:36} The Heb. lacks \fbib{First}} he took away my birthright, and now, look how he also stole my blessing.'' Then he added, ``Haven't you reserved a blessing for me?''

\v{37}In response, Isaac told Esau, ``Look! I've predicted that he's going\fnote{\fbackref{27:37} Lit. \fbib{I've set him}} to become your master, and I've assigned all his brothers to be his servants. What then can I do for you, my son?''

\v{38}Then Esau implored his father, ``Don't you have even one blessing for me, my father? Bless me, even me too, my father!'' Then Esau lifted his voice and wept bitterly.

\v{39}At this, his father Isaac replied to him,

\begin{poetry}
\poeml ``Look! Away from the fertile land will be your dwellings; \\
\poemll    away from the dew of the skies above. \\
\poeml \v{40}By your sword you'll live; \\
\poemll    but you'll serve your brother. \\
\poeml But when you've become restless, \\
\poemll    you'll break off his yoke from your neck.''
\end{poetry}

\v{41}So Esau harbored animosity toward Jacob because of the way his father had blessed him. Esau kept saying to himself,\fnote{\fbackref{27:41} Lit. \fbib{saying in his heart}} ``The time\fnote{\fbackref{27:41} Lit. \fbib{days}} to mourn for my father is very near. That's when I'm going to kill my brother Jacob.''

\v{42}Eventually, what Rebekah's older son Esau had been saying was reported to her, so she sent for her younger son Jacob and warned him, ``Look! Your brother is planning to get even by killing you.\fnote{\fbackref{27:42} Lit. \fbib{is comforting himself concerning you to kill you}} \v{43}Son, you'd better do what I say! Get up, run off to my brother Laban in Haran, \v{44}and stay there with him a few days until your brother's fury subsides.\fnote{\fbackref{27:44} Lit. \fbib{turns back}} \v{45}After that happens\fnote{\fbackref{27:45} Lit. \fbib{After your brother's anger subsides}} and he has forgotten what you've done to him, I'll send for you so you can return from there. Why should I be bereaved of you both in one day?''

\v{46}Rebekah also told herself,\fnote{\fbackref{27:46} The Heb. lacks \fbib{herself}} ``Heth's daughters are making me tired of living. If Jacob marries one of Heth's daughters, and she turns out to be just like these other local women,\fnote{\fbackref{27:46} Lit. \fbib{these daughters}} what kind of life would there be left for me?''
\labelchapt{28}
\passage{Isaac Sends Jacob to Paddan-aram}

\chapt{28}
\v{1}Later, Isaac called Jacob and blessed him, instructing him, ``Don't marry a wife from the local Canaanite women. \v{2}Instead, get up, travel to Paddan-aram,\fnote{\fbackref{28:2} Paddan-aram was located in northwest Mesopotamia} and visit the household of Bethuel, your mother's father. Marry one of Laban's daughters, since he's your mother's brother. \v{3}May God Almighty bless you and make you fruitful so that your descendants\fnote{\fbackref{28:3} Lit. \fbib{that you}} become a whole group of people. \v{4}May he give you and your descendants the blessings that he gave Abraham. May you possess the land where you have lived\fnote{\fbackref{28:4} Lit. \fbib{land of your journeying}} that God gave to Abraham.''

\v{5}So Isaac sent Jacob off toward Paddan-aram\fnote{\fbackref{28:5} Paddan-aram was located in northwest Mesopotamia} to visit Bethuel's son Laban, the Aramean\fnote{\fbackref{28:5} In later centuries this region would be called Syria} and brother of Rebekah, the mother of Jacob and Esau.
\passage{Esau Marries a Canaanite Woman}

\v{6}Esau noticed that after Isaac had blessed Jacob as he was sending him off to Paddan-aram\fnote{\fbackref{28:6} Paddan-aram was located in northwest Mesopotamia} to marry a wife from there, he had instructed Jacob,\fnote{\fbackref{28:6} Lit. \fbib{him}} ``Don't marry a Canaanite woman.'' \v{7}After Jacob had obeyed his father and mother's instructions to set out for Paddan-aram,\fnote{\fbackref{28:7} Paddan-aram was located in northwest Mesopotamia} \v{8}Esau realized\fnote{\fbackref{28:8} Lit. \fbib{saw}} that Canaan women didn't please his father Isaac, \v{9}so he went to Abraham's son Ishmael and married Ishmael's daughter Mahalath, who was the sister of Nebaioth.
\passage{God Visits Jacob in a Dream}

\v{10}Meanwhile, Jacob had left\fnote{\fbackref{28:10} Lit. \fbib{went out from}} Beer-sheba and was on his way to Haran. \v{11}He reached a certain place and spent the night there, because the sun was setting. He found a stone there, used it for a pillow,\fnote{\fbackref{28:11} Lit. \fbib{for his head.}} and slept there for the night, \v{12}when he had a dream! He saw a raised highway that had been built with its ending point on earth and its beginning point in heaven. God's angels were ascending and descending on it. \v{13}And there was the \divine{Lord}, standing above it and telling Jacob, ``I am the \divine{Lord} God of your grandfather Abraham. I'm Isaac's God, too. I'm giving you and your descendants the ground on which you're sleeping. \v{14}Your descendants are going to become like the dust of the earth and spread out to the west, east, north, and south. All the families of the earth\fnote{\fbackref{28:14} Or \fbib{land}} will be blessed through you and your descendants. \v{15}Now pay attention! I'm here with you, and I'm going to be watching over you wherever you go. I'm going to bring you back to this land, because I won't ever leave you until I've accomplished what I've promised about you.''
\passage{Jacob Worships God in Bethel}

\v{16}Then Jacob woke up during the night\fnote{\fbackref{28:16} Lit. \fbib{woke from his sleep}} and told himself,\fnote{\fbackref{28:16} The Heb. lacks \fbib{to himself}} ``Surely, the \divine{Lord} is in this place and I never knew it!'' \v{17}In mounting terror, he cried out, ``How scary this place is! This is nothing less than God's house and the gateway to heaven!'' \v{18}When Jacob got up early the next morning, he took the stone that he had used for his pillow,\fnote{\fbackref{28:18} Lit. \fbib{for his head}} set it up as a pillar, drenched it with oil, \v{19}and named\fnote{\fbackref{28:19} Lit. \fbib{called the name of}} the place Beth-el, although previously\fnote{\fbackref{28:19} Lit. \fbib{at the first}} the city had been named Luz.

\v{20}Then he made this solemn vow:\fnote{\fbackref{28:20} Lit. \fbib{vowed a vow}} ``If God remains with me, watches over me throughout this journey that I'm taking, gives me food to eat and clothes to wear, \v{21}and returns me safely to my father's house, then the \divine{Lord} will be my God, \v{22}this stone that I've erected in the form of a pillar will be God's house, and I'll give you a tenth of everything that you give to me.''
\labelchapt{29}
\passage{Jacob Meets Rachel}

\chapt{29}
\v{1}Jacob journeyed on and reached the territory that belonged to the people who lived in the east.\fnote{\fbackref{29:1} Lit. \fbib{sons of the east}} \v{2}As he was observing a well that had been dug out on the open range, all of a sudden he noticed three flocks of sheep lying there, because shepherds watered their flocks from that well. There was a very large stone that covered the opening of the well, \v{3}and when all the flocks had been gathered there, they would roll away the stone from the opening of the well, water their flocks, and then return the stone to its place covering the opening of the well.

\v{4}Jacob asked them, ``My brothers, where are you from?''

``We're from Haran,'' they answered.

\v{5}``Do you happen to know Nahor's son Laban?'' he inquired.

``We do,'' they replied.

\v{6}So he asked them, ``How's he doing?''

``Very well,'' they answered. ``As a matter of fact, look over there! That's his daughter Rachel, coming here with his sheep.''

\v{7}``Look!'' Jacob replied. ``The sun\fnote{\fbackref{29:7} Lit. \fbib{day}} is still high. It's not yet time for the flocks to be gathered. Let's water the sheep, then let them graze.''

\v{8}But they responded, ``We can't do that until all the sheep have been gathered and the stone has been rolled away from the opening of the well. Only then can we water the flock.'' \v{9}While he was still talking with them, Rachel arrived with her father's sheep, since she was a shepherdess.

\v{10}When Jacob saw Rachel, the daughter of Laban, his mother's brother, accompanied by Laban's sheep, Jacob approached the well, rolled the stone from the opening of the well, and then watered his mother's brother Laban's flock. \v{11}Then Jacob kissed Rachel and began to cry out loud. \v{12}Jacob told Rachel that he was related to her father, since he was Rebekah's son, so she ran and told her father.

\v{13}When Laban heard the news about his sister's son Jacob, he ran out to meet him. He embraced him, kissed him, and brought him back to his house. Then Jacob told Laban about everything that had happened. \v{14}Laban responded, ``You certainly are my flesh and blood!''\fnote{\fbackref{29:14} Lit. \fbib{bones}} So Jacob\fnote{\fbackref{29:14} Lit. \fbib{he}} stayed with him for about a month.\fnote{\fbackref{29:14} Lit. \fbib{for days of a new month}}
\passage{Jacob Agrees to Work in Order to Marry Rachel}

\v{15}Later, Laban asked Jacob, ``Should you serve me for free, just because you're my nephew?\fnote{\fbackref{29:15} Lit. \fbib{brother}} Let's talk about what your wages should be.''

\v{16}Now Laban happened to have two daughters. The older one was named Leah and the younger was named Rachel. \v{17}Leah looked rather plain,\fnote{\fbackref{29:17} Or \fbib{Leah had weak eyes}} but Rachel was lovely in form and appearance. \v{18}Jacob loved Rachel, so he made this offer to Laban: ``I'll serve you for seven years for Rachel, your younger daughter.''

\v{19}``It's better that I give her to you than to another man,'' Laban replied, ``so stay with me.'' \v{20}Jacob served seven years for Rachel, but it seemed like only a few days because of his love for her.

\v{21}Eventually, Jacob told Laban, ``Bring me my wife, now that my time of service\fnote{\fbackref{29:21} The Heb. lacks \fbib{of service}} has been completed, so I can go be with her.'' \v{22}So Laban gathered all the men who lived in that place and held a wedding festival.
\passage{Laban Deceives Jacob}

\v{23}That night Laban took his daughter Leah and brought her to Jacob.\fnote{\fbackref{29:23} Lit. \fbib{him}} He had marital relations with her. \v{24}Laban also gave his servant woman Zilpah to Leah to be her maidservant. \v{25}The next morning, Jacob\fnote{\fbackref{29:25} Lit. \fbib{he}} realized that it was Leah! ``What have you done to me?'' he demanded of Laban. ``Didn't I serve you for seven years in order to marry Rachel? Why did you deceive me?''

\v{26}But Laban responded, ``It's not the practice of our place to give the younger one in marriage\fnote{\fbackref{29:26} The Heb. lacks \fbib{in marriage}} before the firstborn. \v{27}Fulfill the week for this daughter,\fnote{\fbackref{29:27} Lit. \fbib{one}} then we'll give you the other one in exchange for serving me another seven years.''

\v{28}So Jacob completed another seven years' work, and then Laban\fnote{\fbackref{29:28} Lit. \fbib{he}} gave him his daughter Rachel to be his wife. \v{29}Laban also gave his woman servant Bilhah to his daughter Rachel to be her maidservant. \v{30}Jacob\fnote{\fbackref{29:30} Lit. \fbib{he}} also married Rachel, since he loved her. He served Laban another full seven years' work for Rachel.
\passage{Leah's Children}

\v{31}Later, the \divine{Lord} noticed that Leah was being neglected,\fnote{\fbackref{29:31} Lit. \fbib{hated}} so he made her fertile, while Rachel remained childless. \v{32}Leah conceived, bore a son, and named him Reuben,\fnote{\fbackref{29:32} The Heb. name \fbib{Reuben} means \fbib{See, a son}} because she was saying, ``The \divine{Lord} had looked on my torture, so now my husband will love me.''

\v{33}Later, she conceived again, bore a son, and declared, ``Because the \divine{Lord} heard that I'm neglected, he gave me this one, too.'' So she named him Simeon.\fnote{\fbackref{29:33} The Heb. name \fbib{Simeon} means \fbib{heard}}

\v{34}Later, she conceived again and said, ``This time my husband will become attached to me, now that I've borne him three sons.'' So he named him Levi.\fnote{\fbackref{29:34} The Heb. name \fbib{Levi} means \fbib{joined}}

\v{35}Then she conceived yet again, bore a son, and said, ``This time I'll praise the \divine{Lord}.'' So she named him Judah.\fnote{\fbackref{29:35} The Heb. name \fbib{Judah} means \fbib{praise}}

Then she stopped bearing children.
\labelchapt{30}
\passage{Rachel's Children by Bilhah}

\chapt{30}
\v{1}Rachel noticed that she was not bearing children for Jacob, so because she envied her sister Leah, she told Jacob, ``If you don't give me sons, I'm going to die!''

\v{2}That made Jacob angry with Rachel, so he asked her, ``Can I take God's place, who has not allowed you to conceive?''\fnote{\fbackref{30:2} Lit. \fbib{has withheld from you fruit of the womb}}

\v{3}Rachel\fnote{\fbackref{30:3} Lit. \fbib{She}} responded, ``Here's my handmaid Bilhah. Go have sex with her. She can bear children\fnote{\fbackref{30:3} Lit. \fbib{them}} on my knees so I can have children through her.''

\v{4}So Rachel\fnote{\fbackref{30:4} Lit. \fbib{she}} gave Jacob\fnote{\fbackref{30:4} Lit. \fbib{him}} her woman servant Bilhah to be his wife, and Jacob had sex with her. \v{5}Bilhah conceived and bore a son for Jacob. \v{6}Then Rachel said, ``God has vindicated me! He has heard my voice and has given me a son.'' Therefore, she named him Dan.\fnote{\fbackref{30:6} The Heb. name \fbib{Dan} means \fbib{judge}}

\v{7}Rachel's servant conceived again and bore a second son for Jacob, \v{8}so Rachel said, ``I've been through a mighty struggle with my sister and won.'' She named him Naphtali.\fnote{\fbackref{30:8} The Heb. name \fbib{Naphtali} means \fbib{my struggle}}

\v{9}When Leah saw that she had stopped bearing children, she took her woman servant Zilpah and gave her to Jacob as a wife. \v{10}Leah's servant Zilpah bore a son to Jacob, \v{11}and Leah exclaimed, ``How fortunate!'' So she named him Gad.\fnote{\fbackref{30:11} The Heb. name \fbib{Gad} means \fbib{lucky}}

\v{12}Later, Leah's servant Zilpah bore a second son for Jacob. \v{13}She said, ``How happy I am, because women will call me happy!'' So she named him Asher.\fnote{\fbackref{30:13} The Heb. name \fbib{Asher} means \fbib{happy}}
\passage{Jacob and the Mandrakes}

\v{14}Some time later, during the wheat harvest season, Reuben went out and found some mandrakes\fnote{\fbackref{30:14} I.e. a plant native to Canaan thought to facilitate procreation} in the field and brought them back for his mother Leah. Then Rachel\fnote{\fbackref{30:14} Lit. \fbib{she}} told Leah, ``Please give me your son's mandrakes.''

\v{15}In response, Leah asked her, ``Wasn't it enough that you've taken away my husband? Now you also want to take my son's mandrakes!''

But Rachel replied, ``Very well, let's let Jacob sleep with you tonight in exchange for your son's mandrakes.''

\v{16}When Jacob came in from the field that evening, Leah went to meet him and told him, ``You're having sex with me tonight. I traded my son's mandrakes for you!'' So he slept with her that night.

\v{17}God heard what Leah had said, so she conceived and bore a fifth son for Jacob. \v{18}Then Leah said, ``God has paid me for giving my servant to my husband as his wife.'' So she named him Issachar.\fnote{\fbackref{30:18} The Heb. name \fbib{Issachar} means \fbib{wages}}

\v{19}Later, Leah conceived again and bore a sixth son for Jacob. \v{20}Then Leah said, ``God has given me a good gift. This time my husband will exalt me, because I've borne him six sons.'' So she named him Zebulun.\fnote{\fbackref{30:20} The Heb. name \fbib{Zebulun} means \fbib{exalted}}

\v{21}After that, Leah conceived, bore a daughter, and named her Dinah.
\passage{Rachel's Son Joseph is Born}

\v{22}Then God remembered Rachel. He listened to her and opened her womb, \v{23}so she conceived, bore a son, and remarked, ``God has removed my shame.'' \v{24}Because she had been asking, ``May God give me another son,'' she named him Joseph.\fnote{\fbackref{30:24} The Heb. name \fbib{Joseph} means \fbib{added}}
\passage{Jacob and Laban's Livestock}

\v{25}After Rachel had given birth to Joseph, Jacob told Laban, ``Send me off so that I can go back to my place and country. \v{26}Give me my wives and children for whom I've served you. Then I'll leave, since you're aware of my service to you.''

\v{27}Then Laban responded, ``If I've found favor in your sight, please stay with me, because I've learned through divination that the \divine{Lord} has blessed me because of you. \v{28}Name your wage, and I'll give it to you.''

\v{29}But Jacob replied to Laban, ``You know how I've served you and how your cattle thrived under my care. \v{30}What you had previously was only a few head, but the herd has now multiplied, because the \divine{Lord} has blessed you through my efforts.\fnote{\fbackref{30:30} Lit. \fbib{my foot}} But now, when am I going to be able to provide for my own household?''

\v{31}``What do I have to give you?'' Laban asked.

Jacob responded, ``You don't have to give me anything. Just do this for me: Let me tend your flock again and watch over it. \v{32}Let me walk among your flocks today and remove every speckled or spotted sheep, along with every black lamb, and let me do the same with the speckled and spotted goats. These will be my wages. \v{33}In the future, you'll be able to verify my honesty because, when you come to check\fnote{\fbackref{30:33} The Heb. lacks \fbib{check}} what I've earned, if you find a goat that's not speckled or spotted or a sheep that's not black, then it will have been stolen by me.''

\v{34}``Very well,'' Laban replied. ``We'll do it the way you've asked.'' \v{35}That very day, Laban\fnote{\fbackref{30:35} The Heb. lacks \fbib{Laban}} removed the male goats that were striped or spotted, all the female goats that were speckled or spotted---that is, every one that had white on them---and all the black lambs and placed them into the care\fnote{\fbackref{30:35} Lit. \fbib{hand}} of his sons. \v{36}He sent them as far away from Jacob as a three days' journey could take them.

Meanwhile, Jacob kept tending the rest of Laban's flock. \v{37}Jacob took branches\fnote{\fbackref{30:37} The Heb. has \fbib{rod}} from white poplar trees, freshly cut almond trees, and some other trees,\fnote{\fbackref{30:37} Or \fbib{and plane trees}; i.e. a species of trees that could readily be stripped of their bark} stripped off their bark to make white streaks, and uncovered the white part inside the branches. \v{38}Then he placed the branches that he had stripped bare in all the watering troughs where the flocks came to drink. He placed the branches in front of the flock, and they went into heat as they came to drink. \v{39}When the flocks mated in front of the branches, they would bear offspring\fnote{\fbackref{30:39} The Heb. lacks \fbib{offspring}} that were striped, speckled, or spotted.

\v{40}Jacob kept the lambs separate, facing the striped and entirely black ones that belonged to Laban's flock. He set his own herd by itself and would not let them be with Laban's flock. \v{41}Whenever the more vigorous of the flock came into heat, Jacob would place the branches in the troughs in front of the flock to make them mate by the branches.

\v{42}But he didn't put the branches in front of any of the feeble members of the flock. As a result, the feeble ones belonged to Laban, but the stronger ones belonged to Jacob. \v{43}Therefore the man Jacob\fnote{\fbackref{30:43} The Heb. lacks \fbib{Jacob}} prospered so much that he had large flocks, female and male servants, as well as camels and donkeys.
\labelchapt{31}
\passage{Jacob Decides to Leave Laban}

\chapt{31}
\v{1}Now Jacob\fnote{\fbackref{31:1} Lit. \fbib{He}} used to listen while Laban's sons kept on complaining,\fnote{\fbackref{31:1} Lit. \fbib{saying}} ``Jacob has taken over everything our father owns! He made himself wealthy from what belongs to our father!'' \v{2}Jacob also noticed that the way\fnote{\fbackref{31:2} Lit. \fbib{face}} Laban had been looking at him wasn't as nice as it had been just two days earlier.\fnote{\fbackref{31:2} Lit. \fbib{been the day before yesterday}}

\v{3}Then the \divine{Lord} ordered Jacob, ``Go back to your father's territory and to your relatives. I'll be with you.''

\v{4}Jacob sent for Rachel and Leah to come out to the field where his flock was \v{5}and informed them, ``I've noticed that the way\fnote{\fbackref{31:5} Lit. \fbib{the face of}} your father has been looking at us hasn't been as nice as it was just two days ago.\fnote{\fbackref{31:5} Lit. \fbib{was the day before yesterday}} But my father's God has been with me. \v{6}You know I've been serving your father with all my heart. \v{7}Even so, your father has cheated me. He broke our wage agreement ten times. However, God didn't allow him to harm me.

\v{8}``When Laban said, `The speckled ones will be your wages,' then all the flock gave birth to speckled ones. Then when he said, `The streaked ones will be your wages,' all the flock gave birth to streaked offspring.

\v{9}``So God has taken away your father's livestock and has given them to me. \v{10}As it was, when it was time for the livestock to breed, I once looked up in a dream, and the male goats that were mating\fnote{\fbackref{31:10} Lit. \fbib{climbing up}} with the flock were producing streaked, speckled, and spotted offspring.

\v{11}``Later, the angel of God spoke to me in a dream, `Jacob.'

```Here I am,' I replied

\v{12}```Look around!' he said. `Go ahead, look! All the male goats have been mating with the flock, producing offspring that are streaked, speckled, and spotted, because I've been watching everything that Laban has done to you. \v{13}I am the God of Bethel, the place where you consecrated that stone and made a vow to me. Now get up, leave this territory, and return to your native land.'\,''\fnote{\fbackref{31:13} Lit. \fbib{to the land of your birth}}
\passage{Rachel and Leah Consent to Leave}

\v{14}Then Rachel and Leah asked him, ``Do we have anything left of inheritance\fnote{\fbackref{31:14} Lit. \fbib{portion and inheritance}} remaining in our father's house? \v{15}He's treating us like foreigners. He sold us and spent all of the money\fnote{\fbackref{31:15} Lit. \fbib{silver}} that rightfully belonged to us. \v{16}Furthermore, all of the wealth that God has stripped away from our father belongs to us now and to our children. So do everything that God tells you to do.'' \v{17}So Jacob got up, seated his children and wives on camels, \v{18}and drove all his livestock ahead of him, with everything that belonged to him, including the livestock that he had bought and accumulated in Paddan-aram,\fnote{\fbackref{31:18} Paddan-aram was located in northwest Mesopotamia} intending to deliver them to his father Isaac in the land of Canaan.
\passage{Laban Pursues Jacob}

\v{19}Meanwhile, Laban had been out shearing his sheep. While he was away, Rachel stole her father's personal idols.\fnote{\fbackref{31:19} Lit. \fbib{father's teraphim}; i.e. personal idols typically stored inside a small household shrine} \v{20}Moreover, Jacob had deceived\fnote{\fbackref{31:20} Lit. \fbib{had stolen away the heart of}} Laban the Aramean,\fnote{\fbackref{31:20} In later centuries this region would be called Syria} because he had never told him that he was intending to leave. \v{21}Jacob fled, taking everything that he owned. He got up, crossed the river,\fnote{\fbackref{31:21} I.e. possibly the Euphrates River} and headed to the hill country of Gilead. \v{22}Three days later, somebody reported to Laban that Jacob had left, \v{23}so he took his relatives with him and pursued Jacob. Laban\fnote{\fbackref{31:23} Lit. \fbib{He}} was on the road for seven days when he finally caught up with Jacob\fnote{\fbackref{31:23} Lit. \fbib{him}} in the hill country of Gilead.
\passage{God Warns Laban}

\v{24}That night, God appeared to Laban the Aramean\fnote{\fbackref{31:24} In later centuries this region would be called Syria} in a dream and warned him, ``Be careful what you say to Jacob, whether it's one word good or bad.'' \v{25}Meanwhile, Jacob had pitched his tent on the mountain, where Laban had caught up with him.\fnote{\fbackref{31:25} Lit. \fbib{Jacob}} Laban and his relatives encamped on that same mountain in the hill country of Gilead, too.

\v{26}Then Laban asked Jacob, ``What did you do? You deceived me,\fnote{\fbackref{31:26} Lit. \fbib{You stole my heart}} carried off my daughters like you would war captives,\fnote{\fbackref{31:26} Lit. \fbib{captives of the sword}} \v{27}ran away from me secretly,\fnote{\fbackref{31:27} Lit. \fbib{me, hiding yourself}} and stole from me by not keeping me informed. Otherwise, I could have sent you off with a party and singing, accompanied by a band playing tambourines and harps. \v{28}As it is, you didn't even allow me to kiss my grandchildren\fnote{\fbackref{31:28} Lit. \fbib{sons}} and daughters goodbye! You've acted foolishly. \v{29}It's actually in my power to do some serious\fnote{\fbackref{31:29} The Heb. lacks \fbib{some serious}} evil to you, but last night the God of your father told me, `Be careful what you say to Jacob whether good or evil.' \v{30}Now, you can go if you must go, because you certainly are longing to go to your father's house. But why did you steal my gods?''
\passage{Laban Searches for His Idols}

\v{31}``I was afraid,'' Jacob replied. ``I thought you might take your daughters from me. \v{32}Now as to your gods, if you find someone has them in their possession, he's a dead man.\fnote{\fbackref{31:32} Lit. \fbib{he is not to live}} Take our relatives as witnesses, search through our belongings, and take whatever belongs to you that's in my possession.'' But Jacob didn't know that Rachel had stolen the idols.\fnote{\fbackref{31:32} Lit. \fbib{them}} \v{33}So Laban entered Jacob's tent, Leah's tent, and the tents of the two maid servants, but he didn't find them.\fnote{\fbackref{31:33} The Heb. lacks \fbib{them}} Then he left Leah's tent and entered Rachel's tent.

\v{34}Meanwhile, Rachel had taken the idols,\fnote{\fbackref{31:34} Lit. \fbib{teraphim}; i.e. personal idols typically stored inside a small household shrine} placed them inside the saddle of her camel, and sat on them. Laban searched through the whole tent, but found nothing. \v{35}Then Rachel told her father, ``Sir, please don't be angry that I cannot stand up in your presence. It's that time of the month.''\fnote{\fbackref{31:35} Lit. \fbib{that manner of women for me}} So Laban\fnote{\fbackref{31:35} Lit. \fbib{he}} searched for the idols,\fnote{\fbackref{31:35} Lit. \fbib{teraphim}; i.e. personal idols typically stored inside a small household shrine} but never did find them.\fnote{\fbackref{31:35} The Heb. lacks \fbib{them}}
\passage{Jacob Rebukes Laban}

\v{36}Then Jacob got angry and started an argument with Laban. ``What have I done?'' he demanded. ``What's my crime that would cause you to come pursue me so violently? \v{37}Now that you've searched all my belongings, what did you find that belongs to your house? Set it here in front of our relatives\fnote{\fbackref{31:37} Lit. \fbib{my relatives and your relatives}} and we'll let them judge between us! \v{38}Meanwhile, these past 20 years that I've been with you, your sheep and goats never had miscarriages, I never once ate any of the rams from your flock, \v{39}and whatever was torn by beasts, I never bothered to bring to you. Instead, I bore the losses myself. Even so, you demanded that I provide restitution for anything that was stolen, whether during the day or the night. \v{40}As it was, I was attacked by drought during the day and by cold at night. I never got any decent rest. \v{41}I've lived in your house these 20 years---serving fourteen years for your two daughters and another six years for your flocks. During all that time you changed\fnote{\fbackref{31:41} Lit. \fbib{you cut through}} my wages ten times. \v{42}If the God of my father---the God of Abraham, the God whom Isaac feared---had not been with me, you would have sent me away empty handed. But God saw my misery and how hard I've worked with my own hands---and he rebuked you last night.''

\v{43}But Laban answered Jacob, ``These women are my daughters. These children are my children. The flocks are mine. In fact, everything that you see belongs to me. But what would I do today to my daughters and the children they have borne? \v{44}Come, let's make a covenant just between you and me. And let it serve as a witness between you and me.''

\v{45}So Jacob took a stone and raised it as a pillar. \v{46}Then Jacob told his relatives, ``Go gather some stones.'' So they picked up stones and stacked them one on top of the other. Then they had a meal together there by the stack of stones. \v{47}Laban named the place Jegar-sahadutha,\fnote{\fbackref{31:47} The Aram. name \fbib{Jegar-sahadutha} means \fbib{stack of witness}} but Jacob named it Galeed.\fnote{\fbackref{31:47} The Heb. name \fbib{Galeed} means \fbib{stack of witness}}

\v{48}Then Laban said, ``This stack will serve as a witness between you and me today.'' That's how the place came to be named Galeed. \v{49}It was also called Mizpah,\fnote{\fbackref{31:49} The Heb. word \fbib{Mizpah} means \fbib{watchtower}} because Laban\fnote{\fbackref{31:49} Lit. \fbib{he}} said, ``May the \divine{Lord} watch between you and me, when we are estranged\fnote{\fbackref{31:49} Or \fbib{concealed}} from each other. \v{50}If you mistreat my daughters or if you take other wives besides them, though no one is watching\fnote{\fbackref{31:50} Lit. \fbib{with}} us, keep in mind that God stands as a witness between you and me.''

\v{51}``Look!'' Laban added, ``Here is the stack of stones and here is the pillar that I've set up between you and me. \v{52}This stack is a witness, and so is this pillar, reminding me not to cross beyond this stack of stones, and reminding you not to pass by this stack in my direction, intending to cause harm. \v{53}May Abraham's God and Nahor's god judge between us.''

So Jacob made an oath by his father's Fear,\fnote{\fbackref{31:53} I.e. the \fbib{}\divine{Lord}} \v{54}offered sacrifices there on the mountain, and called on his relatives to eat some food. So they ate the food and spent the night on the mountain. \v{55}\fnote{\fbackref{31:55} This v. is 32:1 in MT}Early the next morning, Laban woke up, kissed his grandchildren and daughters, blessed them, and then left for home.\fnote{\fbackref{31:55} Lit. \fbib{for his place}}
\labelchapt{32}
\passage{Jacob Prepares to Meet Esau}

\chapt{32}
\v{1}\fnote{\fbackref{32:1} This v. is 32:2 in MT, and so through v. 33}As Jacob went on his way, angels from God met him. \v{2}As he was watching them, Jacob said, ``This must be God's camp,'' so he named that place Mahanaim.\fnote{\fbackref{32:2} The Heb. name \fbib{Mahanaim} means \fbib{two camps}}

\v{3}Then Jacob sent messengers ahead of him into the land of Seir (that is, into the territory of Edom) to meet his brother Esau. \v{4}He instructed them, ``This is what you are to say to my master Esau: `Your servant Jacob told me to tell you, ``I've journeyed to stay with Laban and I've remained there until now. \v{5}I now have cattle, donkeys, flocks, and male and female servants. I'm sending this message to you, sir,\fnote{\fbackref{32:5} Lit. \fbib{to my lord}} so that you'll show favor to me.''\,'\,''

\v{6}Later, the messengers returned to Jacob and reported, ``We went to your brother Esau. He's now coming to meet you---and he has 400 men with him!''

\v{7}Feeling mounting terror and distress, Jacob divided the people who were with him into two groups, doing the same with the flocks, the cattle, and the camels. \v{8}Jacob was thinking, ``If Esau comes to one group and attacks it, then the remaining group may escape.''

\v{9}Then Jacob prayed,\fnote{\fbackref{32:9} Lit. \fbib{said}} ``O God of my father Abraham, O God of my father Isaac, O \divine{Lord}, you who told me, `Return to your country and to your relatives and I'll cause things to go well for you.' \v{10}I'm unworthy of all your gracious love, your faithfulness, and everything that you've done for your servant. When I first crossed over this river, I had only my staff. But now I've become two groups. \v{11}Deliver me from my brother Esau's control, because I'm terrified of him, and I'm afraid that he's coming to attack me, the mothers, and their children. \v{12}Now, you promised me that `I'm certainly going to cause things to go well with you, and I'm going to make your offspring\fnote{\fbackref{32:12} Lit. \fbib{seed}} as numerous as the sand of the sea, which cannot be counted.'\,''

\v{13}Jacob spent the night there. Out of everything that he had brought with him, he chose a gift for his brother Esau--- \v{14}200 female goats, 20 male goats, 200 ewes, 20 rams, \v{15}30 milking camels with their young, 40 cows with ten bulls, and 20 female donkeys with ten male donkeys. \v{16}He entrusted them into the care of his servants, one herd at a time.\fnote{\fbackref{32:16} Lit. \fbib{herd by herd}} Then he told his servants, ``Go in front of me, making sure there's plenty of space between herds.''

\v{17}To the first group he said, ``When you meet my brother Esau, if he asks, `To whom do you belong? Where are you going? And to whom do these herds\fnote{\fbackref{32:17} The Heb. lacks \fbib{herds}} belong?' \v{18}then you are to reply, `We're from\fnote{\fbackref{32:18} Lit. \fbib{To}} your servant Jacob. The herds\fnote{\fbackref{32:18} Lit. \fbib{They}} are a gift. He's sending them to my master, Esau. Look! There he is, coming along behind us.'\,''

\v{19}He issued similar instructions to the second and third group, as well as to all the others who drove the herds that followed: ``This is how you are to speak to Esau when you find him. \v{20}You are to tell him, `Look! Your servant Jacob is coming along behind us.'\,''

Jacob was thinking, ``I'll pacify him with the presents that are being sent ahead of me. Then, when I meet him,\fnote{\fbackref{32:20} Lit. \fbib{I see his face}} perhaps he'll accept me.''\fnote{\fbackref{32:20} Lit. \fbib{he'll lift my face}} \v{21}So the presents went\fnote{\fbackref{32:21} Lit. \fbib{passed}} ahead of him, while he spent that night in the camp. \v{22}Later that night, he woke up, quickly took his two wives, his\fnote{\fbackref{32:22} The Heb. lacks \fbib{his}} two women servants, and his eleven children, and forded the river at Jabbok. \v{23}He took them across the river, along with all his possessions.
\passage{Jacob Struggles with God}

\v{24}And so Jacob was left alone, and he struggled with a man until daybreak. \v{25}When the man realized that he hadn't yet won the struggle, he injured the socket\fnote{\fbackref{32:25} Or \fbib{hollow} and so throughout the chapter} of Jacob's thigh, dislocating it as he wrestled with him, \v{26}and said, ``Let me go, because the dawn has come.''\fnote{\fbackref{32:26} Lit. \fbib{has ascended.}}

``I won't let you go,'' Jacob\fnote{\fbackref{32:26} Lit. \fbib{he}} replied, ``unless you bless me.''

\v{27}Then the man\fnote{\fbackref{32:27} Lit. \fbib{Then he}} asked him, ``What's your name?''

``Jacob,'' he responded

\v{28}``Your name won't be\fnote{\fbackref{32:28} Lit. \fbib{be called}} Jacob anymore,'' the man\fnote{\fbackref{32:28} Lit. \fbib{anymore,'' he}} replied, ``but Israel, because you exerted yourself against both God and men, and you've emerged victorious.''

\v{29}``Please,'' Jacob inquired, ``Tell me your name.''

But he asked, ``Why are you asking about my name?'' And he blessed Jacob\fnote{\fbackref{32:29} Lit. \fbib{him}} there.

\v{30}Jacob would later call that place Peniel,\fnote{\fbackref{32:30} The Heb. name means \fbib{facing God}} because ``I saw God face to face, but my life was spared.''

\v{31}The sun was rising above Jacob\fnote{\fbackref{32:31} Lit. \fbib{him}} as he crossed over from Peniel, limping due to his wounded thigh. \v{32}Therefore, to this day the Israelis do not eat the hip tendon that connects to the thigh socket, because he had injured the socket of the thigh where the tendon connected to Jacob's hip.
\labelchapt{33}
\passage{Jacob Meets Esau}

\chapt{33}
\v{1}When Jacob looked off in the distance, there was Esau coming toward him, accompanied by 400 men! So Jacob divided Leah's children, Rachel, and the children of the two servants into separate groups.\fnote{\fbackref{33:1} The Heb. lacks \fbib{into separate groups}} \v{2}Then he positioned the women servants and their children first, then Leah and her children next, and then Rachel and Joseph after them. \v{3}Then he went out to meet Esau,\fnote{\fbackref{33:3} The Heb. lacks \fbib{went out to meet Esau}} passing in front of all of them, and bowed low to the ground seven times as he approached his brother.

\v{4}Esau ran to meet Jacob and embraced him. Then he fell on his neck and kissed him. And they wept.

\v{5}When Esau eventually looked around, he saw the women and the children. ``Who are these people\fnote{\fbackref{33:5} The Heb. lacks \fbib{people}} with you?'' he asked.

``The children, whom God has graciously given\fnote{\fbackref{33:5} The Heb. lacks \fbib{given}} your servant,'' he answered. \v{6}Then the women servants approached, accompanied by their children, and bowed low. \v{7}Leah also approached, and she and her children bowed low. After this, Joseph and Rachel approached and bowed low.

\v{8}Then Esau asked, ``What are all these livestock for?''

``To solicit favor from you,\fnote{\fbackref{33:8} Lit. \fbib{from your eyes}} sir,''\fnote{\fbackref{33:8} Lit. \fbib{you, my lord}} Jacob answered.

\v{9}But Esau replied, ``I already have so much, my brother, so keep what belongs to you.''

\v{10}``Please,'' Jacob implored him, ``don't refuse. If I'm to receive favor from you, then receive this gift from me, because seeing your face is like seeing the face of God, since you have favorably accepted me. \v{11}So receive my blessing, which has been sent to you, since God has been gracious to me. Besides, I have enough.'' Because Jacob kept pressing him, Esau accepted the gifts.

\v{12}Then Esau suggested, ``Let's set out and travel together, but let me go in front of you.''

\v{13}``Sir, you know\fnote{\fbackref{33:13} Lit. \fbib{My lord knows}} that the children are frail,'' Jacob suggested, ``and the ewes and cows with me are still nursing their young. If they're driven even for a day, the entire flock will die. \v{14}So allow yourself to\fnote{\fbackref{33:14} Lit. \fbib{So let my lord}} go ahead of your servant while I travel more slowly, letting the herds set their own pace\fnote{\fbackref{33:14} Lit. \fbib{feet}} with the children until I arrive to see my lord in Seir.''

\v{15}Esau said, ``Let me leave with you some of the people who are with me.''

``Why do that?'' Jacob asked. ``I've already found favor in your sight, sir.''\fnote{\fbackref{33:15} Lit. \fbib{sight, my lord}} \v{16}So Esau set out that very day back on his way to Seir, \v{17}but Jacob set out for Succoth, built a house there, and constructed some cattle shelters. He named the place Succoth.\fnote{\fbackref{33:17} The Heb. name \fbib{Succoth} means \fbib{shelters}}
\passage{Jacob Buys Land in Shechem}

\v{18}After Jacob had arrived safely from Paddan-aram,\fnote{\fbackref{33:18} Paddan-aram was located in northwest Mesopotamia} he entered the city of Shechem, which was located in the territory of Canaan, and encamped facing that city. \v{19}Then he bought a parcel of land for 100 pieces of silver from the descendants of Hamor, Shechem's father. He pitched his tent there, \v{20}set up an altar, and named it El-elohe-israel.\fnote{\fbackref{33:20} The Heb. name \fbib{El-elohe-israel} means \fbib{God, the God of Israel}}
\labelchapt{34}
\passage{Jacob's Daughter Dinah is Raped}

\chapt{34}
\v{1}Some time later, Dinah, Leah's daughter whom she had borne to Jacob, went out to visit the women\fnote{\fbackref{34:1} Lit. \fbib{daughters}} of the land. \v{2}When Hamor the Hivite's son Shechem, the regional leader, saw her, he grabbed her and raped her, humiliating her. \v{3}He was attached to\fnote{\fbackref{34:3} Lit. \fbib{His soul clung}} Dinah, Jacob's daughter, since he loved the young woman and spoke tenderly to her.\fnote{\fbackref{34:3} Lit. \fbib{to the heart of the young lady}} \v{4}Then Shechem told his father Hamor, ``Get this young woman\fnote{\fbackref{34:4} Or \fbib{girl}} for me to be my wife.''

\v{5}Because Jacob learned that Shechem had dishonored his daughter Dinah while his sons were still out with their cattle on the open range, he remained silent until they returned. \v{6}Meanwhile, Shechem's father Hamor arrived to talk to Jacob. \v{7}Just then Jacob's sons arrived from the field. When they heard what had happened, they were distraught with grief and livid with anger toward Shechem,\fnote{\fbackref{34:7} Lit. \fbib{toward the man}} because he had committed a disgraceful deed in Israel by forcing Jacob's daughter to have sex, an act that never should have happened.

\v{8}But Hamor said this: ``My son is deeply attracted to your daughter. Please give her to him as his wife. \v{9}Intermarry with us. Give your daughters to us and take our sons for yourselves. \v{10}Live with us anywhere you want.\fnote{\fbackref{34:10} Lit. \fbib{us, since the land lays open before you}} Live, trade, and grow rich in it.''

\v{11}Shechem also addressed Dinah's\fnote{\fbackref{34:11} Lit. \fbib{her}} father and brothers. He told them, ``If you'll just approve me, I'll give whatever you ask of me. \v{12}No matter how big or how extensive your demands are for a dowry and wedding presents from me, I'll provide whatever you ask. Only give me the young lady to be my wife.''
\passage{Jacob's Sons Plot Revenge}

\v{13}But Jacob's sons answered Shechem and his father Hamor deceptively, because Shechem had dishonored their sister Dinah. \v{14}They told them, ``We can't do this. We can't give our sister to a man who isn't circumcised, because that would be insulting to us. \v{15}But we'll agree to your request, only if you will become like us by circumcising every male among you. \v{16}Then we'll give our daughters to you and take your daughters for ourselves, live among you, and be as a united people. \v{17}But if you won't listen to us, then we're going to take our daughter and leave.'' \v{18}What they said pleased Hamor and his son Shechem, \v{19}so the young man did not delay the matter any further, since he was delighted with Jacob's daughter.

Now Shechem was the most important person in his father's household. \v{20}So Hamor and his son Shechem entered the gate of their city and addressed the men of their city. \v{21}``These men are at peace with us,'' they announced. ``Therefore, let them live in the land and trade in it. Look! The land is large enough for them. Let's take their daughters as wives for ourselves and let's give our sons to them.

\v{22}``However,'' they added, ``only on this condition will the men consent to live with us and be united as a single people with us: every male among us will have to be circumcised just as they are. \v{23}Shouldn't all their cattle, acquisitions, and animals belong to us? So, let's give our consent to them, and then they'll live with us.''
\passage{Simeon and Levi Attack Shechem}

\v{24}All of the males who heard Hamor and his son Shechem, who had gone out to the city gate, were circumcised. \v{25}Three days later, while they were still in pain, Jacob's sons Simeon and Levi, two of Dinah's brothers, each grabbed a sword and entered the city unannounced, intending to kill all the males. \v{26}They killed Hamor and his son Shechem with their swords, took back Dinah from Shechem's house, and left. \v{27}Jacob's other sons came along afterward and plundered the city where their sister had been defiled, \v{28}seizing all of their flocks, herds, donkeys, and whatever else was in the city or had been left out in the field. \v{29}They carried off all their wealth, their children, and their wives as captives, plundering everything that remained in the houses.

\v{30}Then Jacob told Simeon and Levi, ``You have certainly stirred up trouble for me! You've made me despised by\fnote{\fbackref{34:30} Lit. \fbib{me stink in the eyes of}} the Canaanites and the Perizzites who live in this territory. Because I have only a few men with me, they're going to gather themselves together and attack me until I am totally destroyed, along with my entire household!''

\v{31}``Should he have treated our sister like a whore?'' they asked in response.
\labelchapt{35}
\passage{Jacob Moves to Bethel}

\chapt{35}
\v{1}Later, God told Jacob, ``Get up, move to Bethel, and live there. Build an altar to the God who appeared to you when you were fleeing from your brother Esau.''

\v{2}Jacob announced to his household and to everyone with him, ``Throw away the foreign gods that you've kept among you, purify yourselves, and change your clothes. \v{3}Then let's get up and go to Bethel, where I'll build an altar to the God who answered me when I was in distress and who was with me on the road, wherever I went.''

\v{4}So they handed over to Jacob all their foreign gods on which they had been depending,\fnote{\fbackref{35:4} Lit. \fbib{gods that were in their hands}} along with the rings that they were wearing on their ears. Jacob buried them under the oak that grew near Shechem. \v{5}As they set out on their journey, because the people who lived in the\fnote{\fbackref{35:5} The Heb. lacks \fbib{people who lived in the}} cities around them feared God, they did not pursue Jacob's sons.

\v{6}Eventually, Jacob and everyone with him arrived at Luz (also called Beth-el) in the territory of Canaan. \v{7}He built an altar there to God and named the place El Beth-el, because God had revealed himself there when he was fleeing from his brother. \v{8}Rebekah's nurse Deborah died and was buried there, under the oak tree that was below Beth-el. That's why the place was named Allon-bacuth.\fnote{\fbackref{35:8} The Heb. name \fbib{Allon-bacuth} means \fbib{Weeping Oak}}
\passage{God Appears Again to Jacob}

\v{9}God appeared again to Jacob after he had arrived from Paddan-aram\fnote{\fbackref{35:9} Paddan-aram was located in northwest Mesopotamia} and blessed him. \v{10}Then God told him,

\begin{poetry}
\poeml ``Your name is Jacob. \\
\poemll    No longer are you to be called Jacob. \\
\poemlll       Instead, your name will be Israel.''
\end{poetry}

So God called his name Israel \v{11}and also told him,

\begin{poetry}
\poeml ``I am God Almighty. \\
\poemll    You are to be fruitful \\
\poemlll       and multiply. \\
\poeml You will become a nation--- \\
\poemll    in fact, an assembly of nations! \\
\poeml Kings will come from you--- \\
\poemll    they'll emerge from your own loins! \\
\poeml \v{12}Now as for the land \\
\poemll    that I gave to Abraham and Isaac, \\
\poeml I'm giving it to you \\
\poemll    and to your descendants who come after you. \\
\poeml I'm giving the land to you!''
\end{poetry}

\v{13}After this, God ascended from the place where he had been speaking to him. \v{14}Jacob erected a pillar of stone at that very place where God had spoken to him. He poured a drink offering over it, anointed it with oil, \v{15}and named the place where God had spoken to him Beth-el.
\passage{Rachel Dies in Childbirth}

\v{16}Later, they set out from Beth-el. While still a long way\fnote{\fbackref{35:16} Lit. \fbib{a distance of land}} from Ephrathah, Rachel started to have trouble giving birth. \v{17}While she was suffering due to her difficult labor, the midwife told her, ``Don't fear! You're going to have another son.''

\v{18}Just before she died,\fnote{\fbackref{35:18} Lit. \fbib{As her soul was departing while she was dying}} Rachel called her son's\fnote{\fbackref{35:18} Lit. \fbib{called his}} name Ben-oni,\fnote{\fbackref{35:18} The Heb. name \fbib{Ben-oni} means \fbib{child of my pain}} but his father Jacob\fnote{\fbackref{35:18} The Heb. lacks \fbib{Jacob}} named him Benjamin.\fnote{\fbackref{35:18} The Heb. name \fbib{Benjamin} means \fbib{child of my right hand}} \v{19}So Rachel died and was buried on the way to Ephrathah, also known as Bethlehem. \v{20}Jacob erected a pillar over her grave, and that pillar stands over Rachel's grave to this day.
\passage{Jacob Settles Near Migdal Eder}

\v{21}Jacob continued his travels, and eventually pitched his tent facing Migdal Eder. \v{22}But while Israel lived in that land, Reuben went inside his father's tent\fnote{\fbackref{35:22} The Heb. lacks \fbib{his father's tent}} and had sexual relations with his father's concubine Bilhah, and Israel heard about it. Now Jacob had twelve sons. \v{23}Leah's sons were Reuben (Jacob's first-born), Simeon, Levi, Judah, Issachar, and Zebulun. \v{24}Rachel's sons were Joseph and Benjamin. \v{25}Rachel's servant Bilhah's sons were Dan and Naphtali. \v{26}Leah's servant Zilpah's sons were Gad and Asher. These were Jacob's sons who were born to him while he lived in Paddan-aram.\fnote{\fbackref{35:26} Paddan-aram was located in northwest Mesopotamia}
\passage{The Death of Isaac}

\v{27}So Jacob reached his father Isaac at Mamre, in Kiriath-arba (also known as Hebron), where Abraham and Isaac had lived. \v{28}Isaac had lived a total of 180 years \v{29}when he died and joined his ancestors at a ripe old age. Then his sons Esau and Jacob buried him.
\labelchapt{36}
\passage{Esau's Genealogies}

\chapt{36}
\v{1}This is a record of Esau's genealogy, that is, of Edom. \v{2}Esau had married Canaanite women, including Elon the Hittite's daughter Adah, Oholibamah, the daughter of Anah (who was Zibeon the Hivite's daughter), and \v{3}Ishamael's daughter Basemath (who was Nebaioth's sister). \v{4}Adah bore Eliphaz to Esau, Basemath bore Reuel, and \v{5}Oholibamah bore Jeush, Jalam, and Korah. These were Esau's sons, who were born to him in the territory of Canaan.

\v{6}Later, Esau took his wives, his children, everyone in his household, his livestock, all his animals, and all his possessions that he had acquired in the territory of Canaan and moved far away from his brother Jacob, \v{7}because their holdings were too vast to allow them to stay together, since the land where they had settled was not able to support all of their livestock. \v{8}So Esau lived in Mount Seir.\fnote{\fbackref{36:8} This mountain, the modern \fbib{Jebel esh-sher\'{a}}, is located in the mountain range that extends south of the Dead Sea toward the Gulf of Aqaba, and is bordered by the Arabah Valley to the west.} (Esau was also known as Edom.)

\v{9}This is a record of the family history of Esau, the ancestor of the Edomites of Mount Seir. \v{10}The names of Esau's sons were Eliphaz (the son of Esau's wife Adah) and Reuel (the son of Esau's wife Basemath).

\v{11}Eliphaz's sons were Teman, Omar, Zepho, Gatam, and Kenaz. \v{12}Timnah was a concubine of Esau's son Eliphaz. She bore Amalek to Eliphaz.

\v{13}Reuel's sons were Nahath, Zerah, Shammah, and Mizzah. These were the sons of Esau's wife Basemath.

\v{14}These were the sons of Esau's wife Oholibamah, the daughter of Anah, who was the daughter of Zibeon. She bore Jeush, Jalam, and Korah for Esau.
\passage{Leaders of Esau's Descendants}

\v{15}These were the tribal leaders of Esau's descendants; that is, the children of Eliphaz, who was Esau's firstborn: tribal leaders\fnote{\fbackref{36:15} This term precedes each name listed through v. 18} Teman, Omar, Zepho, Kenaz, \v{16}Korah, Gatam, and Amalek. These were the tribal leaders who descended\fnote{\fbackref{36:16} The Heb. lacks \fbib{who descended}.} from Eliphaz in the territory of Edom. These were Adah's sons.

\v{17}These were the descendants of Esau's son Reuel: tribal leaders Nahath, Zerah, Shammah, and Mizzah. These were the tribal leaders who descended from Reuel in the territory of Edom. These were the sons of Esau's wife Basemath.

\v{18}These were the descendants of Esau's wife Oholibamah: tribal leaders Jeush, Jalam, and Korah. These tribal leaders descended from Esau's wife Oholibamah, Anah's daughter. \v{19}These were the descendants of Esau (also known as Edom) and their tribal leaders.
\passage{Leaders of Seir's Descendants}

\v{20}These were the descendants of Seir the Horite, who lived in the territory: Lotan, Shobal, Zibeon, Anah, \v{21}Dishon, Ezer, and Dishan. These were the tribal leaders who descended from the Horites, the descendants of Seir in the territory of Edom.

\v{22}Lotan's children were Hori and Hemam. Lotan's sister was Timna.

\v{23}Shobal's children were Alvan, Manahath, Ebal, Shepho, and Onam.

\v{24}Zibeon's children were Aiah and Anah. Anah discovered the hot springs in the wilderness while grazing his father Zibeon's donkeys.

\v{25}Anah's children were Dishon and Anah's daughter Oholibamah.

\v{26}Dishon's children were Hemdan, Eshban, Ithran, and Keran.

\v{27}Ezer's children were Bilhan, Zaavan, and Akan.

\v{28}Dishan's children were Uz and Aran.

\v{29}These were the tribal leaders who descended from the Horites: tribal leaders Lotan, Shobal, Zibeon, Anah, \v{30}Dishon, Ezer, and Dishan. These were the tribal leaders who descended from the Horites, according to their tribal leaders in the territory of Seir.

\v{31}This is a list of the kings who ruled the territory of Edom before any king reigned over the Israelis. \v{32}Beor's son Bela ruled over Edom. His city's name was Dinhabah.

\v{33}After Bela died, Zerah's son Jobab from Bozrah ruled in his place.

\v{34}After Jobab died, Husham from the territory of the Temanites ruled in his place.

\v{35}After Husham died, Bedad's son Hadad, who killed Midian in the field of Moab, ruled in his place. His city's name was Avith.

\v{36}After Hadad died, Samlah from Masrekah ruled in his place.

\v{37}After Samlah died, Shaul from Rehoboth by the river ruled in his place.

\v{38}After Shaul died, Achbor's son Baal-hanan ruled in his place.

\v{39}After Achbor's son Baal-hanan died, Hadar ruled in his place. His city's name was Pau. And his wife's name was Mehetabel, who was the daughter of Matred, and granddaughter of Me-zahab.

\v{40}These were the names of the chiefs who descended from Esau according to their clans, territories, and names: tribal leaders Timna, Alvah, Jetheth, \v{41}Oholibamah, Elah, Pinon, \v{42}Kenaz, Teman, Mibzar, \v{43}Magdiel, and Iram. These were the chiefs who descended from Edom, according to their territories in their own land.\fnote{\fbackref{36:43} Or \fbib{land of their possession}} This was the dynasty of Esau, who was the ancestor of the Edomites.
\labelchapt{37}
\passage{Joseph's Life before His Captivity}

\chapt{37}
\v{1}Jacob continued to live in the land they were occupying, where his father had journeyed in the territory of Canaan. \v{2}This is a record of Jacob's descendants.

When Joseph was seventeen years old, he was helping his brothers tend their flocks. He was a young man at that time, as were the children of Bilhah and Zilpah, his father's wives. But Joseph would come back and tell his father that his brothers were doing bad things. \v{3}Now Israel loved Joseph more than all his brothers, since he was born to him in his old age, so he had made a richly-embroidered\fnote{\fbackref{37:3} Or \fbib{long-sleeved}; LXX reads \fbib{multi-colored}} tunic for him. \v{4}When Joseph's\fnote{\fbackref{37:4} Lit. \fbib{his}} brothers realized that their father loved him more than all of his brothers, they hated him so much that they were unable to speak politely to him.
\passage{Joseph's Dreams}

\v{5}Right about this time, Joseph had a dream and then told it to his brothers. As a result, his brothers hated him all the more! \v{6}``Let me tell you about this dream that I had!'' he said. \v{7}``We were tying sheaves together out in the middle of the fields, when all of a sudden, my sheaf stood up erect! And then your sheaves gathered around it and bowed down to my sheaf!''

\v{8}At this, his brothers replied, ``Do you really think you're going to rule us or lord it over us?'' So they hated him even more because of his dreams and his interpretations of them.

\v{9}But then he had another dream, and he proceeded to tell his brothers about that one, too. ``I had another dream,'' he said. ``The sun, moon, and eleven of the stars were bowing down before me!''

\v{10}When Joseph told his father about this, his father rebuked him and asked him, ``What kind of dream is that? Will I, your mother, and your brothers really come to you and bow down to the ground in front of you?'' \v{11}As a result, his brothers became more envious of him. But his father kept thinking about all of this.
\passage{Joseph is Sent to Visit His Brothers}

\v{12}Some time later, his brothers left to tend their father's flock in Shechem. \v{13}And Israel instructed Joseph, ``Your brothers are tending the flock in Shechem. Come here, because I'm going to send you to them.''

``Here I am!'' he responded.

\v{14}``Go and see how things are with your brothers,'' Israel\fnote{\fbackref{37:14} Lit. \fbib{he}} ordered him. ``And see how things are with the flock. Bring back a report for me.'' Then he sent Joseph\fnote{\fbackref{37:14} Lit. \fbib{him}} from the valley of Hebron.

When Joseph reached Shechem, \v{15}a man found him wandering around in a field. So the man asked him, ``What are you looking for?''

\v{16}``I'm searching for my brothers,'' he responded. ``Tell me, where are they tending the flock?''\fnote{\fbackref{37:16} The Heb. lacks \fbib{the flock}}

\v{17}``They've already left,'' the man answered. ``I heard them saying that they were headed to Dothan.'' So Joseph followed his brothers to Dothan and found them there.
\passage{Joseph's Brothers Plot to Kill Him}

\v{18}Now as soon as they saw him approaching from a distance, before he arrived they plotted together to kill him. \v{19}``Look!'' they said. ``Here comes the Dream Master! \v{20}Come on! Let's kill him and toss him into one of the cisterns. Then we'll report that some wild animal devoured him and wait to see what becomes of his dreams!''

\v{21}When Reuben heard about it, he tried to save Joseph\fnote{\fbackref{37:21} Lit. \fbib{him}} from their plot. ``Let's not do any killing,''\fnote{\fbackref{37:21} Lit. \fbib{Let's not kill a soul}} \v{22}Reuben told them. ``And no blood shedding, either. Instead, let's toss him into this cistern that's way out here in the wilderness. But don't lay a hand on him.'' (Reuben\fnote{\fbackref{37:22} Lit. \fbib{He}} intended to free Joseph\fnote{\fbackref{37:22} Lit. \fbib{him from their control}} and return him to his father.)
\passage{Joseph is Sold into Slavery}

\v{23}As it was, when Joseph arrived where his brothers were, they stripped off the tunic that Jacob had given him---that is, the richly-embroidered\fnote{\fbackref{37:23} Or \fbib{long-sleeved}; LXX reads \fbib{multi-colored}} tunic that he was wearing. \v{24}They grabbed him and tossed him into the cistern, but the cistern was empty. (There was no water in it.) \v{25}After this, while they were seated, eating their food, they looked around and saw a caravan of Ishmaelites coming from Gilead with camels carrying spices, balm, and myrrh for sale down in Egypt.

\v{26}Then Judah suggested to his brothers, ``Where's the profit in just killing our brother and shedding his blood? \v{27}Come on! Let's sell him to the Ishmaelites! That way, we won't have laid our hands on him. After all, he's our brother, our own flesh.''

So Judah's\fnote{\fbackref{37:27} Lit. \fbib{his}} brothers listened to him. \v{28}As the Midianite merchants were passing through, they extracted Joseph from the cistern and sold Joseph for 20 pieces of silver to the Ishmaelites, who then took Joseph down to Egypt.

\v{29}Later, when Reuben returned to the cistern, Joseph wasn't there! In mounting panic, he tore his clothes, \v{30}returned to his brothers, and shouted, ``He's\fnote{\fbackref{37:30} Lit. \fbib{The young man is}} not there! Now what? Where am I to go?''

\v{31}So they took Joseph's coat, slaughtered a young goat, and dipped the coat in the blood. \v{32}Then they stretched out the richly-embroidered\fnote{\fbackref{37:32} Or \fbib{long-sleeved}; LXX reads \fbib{multi-colored}} tunic to dry,\fnote{\fbackref{37:32} The Heb. lacks \fbib{to dry}} and brought it to their father.

``We've found this,'' they reported. ``Look at it and see if this is or isn't your son's tunic.''

\v{33}Examining it, he cried out, ``It's my son's tunic! A wild animal has no doubt torn Joseph to pieces.''

\v{34}So Jacob tore his clothes, dressed himself in sackcloth, and then mourned many days for his son. \v{35}All his sons and daughters showed\fnote{\fbackref{37:35} Lit. \fbib{rose}} up to comfort him, but he refused to be comforted. He kept saying, ``Leave me alone! I'll go down to the next world,\fnote{\fbackref{37:35} Lit. \fbib{to Sheol}; i.e. the realm of the dead} still mourning for my son.'' So Joseph's father wept for him.
\passage{Joseph is Enslaved to Potiphar}

\v{36}Meanwhile, down in Egypt, the Midianites sold Joseph\fnote{\fbackref{37:36} Lit. \fbib{him}} to Potiphar, one of Pharaoh's court officials, who was also Commander-in-Chief of the imperial guards.
\labelchapt{38}
\passage{Judah's Life among the Adullamites}

\chapt{38}
\v{1}Right about then, Judah left his brothers and went to live with an Adullamite man named Hirah. \v{2}There Judah met\fnote{\fbackref{38:2} Lit. \fbib{saw}} the daughter of a Canaanite man named Shua. He married\fnote{\fbackref{38:2} Lit. \fbib{took}} her, had sexual relations with her, \v{3}and she conceived, bore a son, and named him Er. \v{4}Later, she conceived again, bore another son, and named him Onan. \v{5}Then she bore yet another son and named him Shelah. Judah was living in Kezib when she bore him.

\v{6}Judah found a wife for his oldest son Er. Her name was Tamar. \v{7}But the \divine{Lord} considered Er, Judah's oldest son, to be wicked---so he put him to death. \v{8}So Judah instructed Onan, ``You are to have sexual relations with your dead brother's wife, performing the duty of a brother-in-law with her, and have offspring for your brother.''

\v{9}But Onan knew that the offspring wouldn't be his own heir, so whenever he had sexual relations with his brother's wife, he would spill his semen on the ground to avoid fathering offspring for his brother. \v{10}The \divine{Lord} considered what Onan was doing to be evil, so he put him to death, too.

\v{11}After this, Judah told his daughter-in-law Tamar, ``Go live as a widow in your father's house until my son Shelah grows up.'' But he was really thinking, ``{\ldots}otherwise, Shelah\fnote{\fbackref{38:11} Lit. \fbib{he}} might die like his brothers.'' So Tamar left and lived in her father's house. \v{12}Some years later, Shua's daughter (that is, Judah's wife) died. As Judah was grieving, he visited the shearers of his flock in Timnah, accompanied by his Adullamite friend Hirah.
\passage{Tamar Avenges Judah's Treachery}

\v{13}``Look!'' somebody reported to Tamar, ``Your father-in-law is going to Timnah to shear his sheep.'' \v{14}So she took off her mourning apparel, covered herself with a shawl, and concealed her outward appearance. Then she went out and sat at the entrance of Enaim, which is on the way to Timnah, because she knew that even though Shelah had grown up, she wasn't being given to him as his wife.

\v{15}When Judah saw her, he thought she was a prostitute, since she had concealed her face. \v{16}So on the way, he turned aside, approached her, and said, ``Come on! Let's have some sex!'' But he didn't realize that he was talking to his own daughter-in-law.

``What will you give me,'' she asked, ``in order to have sex with me?''

\v{17}``I'll send you a young goat from the flock,'' he responded.

But she pressed him, asking, ``What security will you put up until you've sent it?''

\v{18}Then he asked, ``What pledge do you want me to give you?''

``Your signet ring, cord, and the staff in your hand,'' she suggested. So he gave them to her, had sex with her, and she became pregnant by him. \v{19}Then she got up and left. Later, she took off her shawl and put on her mourning clothes.

\v{20}Later on, Judah sent his Adullamite friend to take her a young goat, intending to retrieve what he had put up as security from the woman, but he could not find her. \v{21}He asked the men who lived in that area, ``Where's that temple prostitute who was sitting alongside the road at Enaim?''

But they replied, ``There's been no temple prostitute here.''

\v{22}So he returned to Judah and said, ``I haven't found her. Also, the men who are from there said, `There's been no prostitute here.'\,''

\v{23}Then Judah said, ``Let her have those things.\fnote{\fbackref{38:23} Lit. \fbib{it}} Otherwise, we'll become contemptible. I sent this young goat, but you didn't find her.''
\passage{Tamar's Pregnancy Rebukes Judah}

\v{24}Three months later, it was reported to Judah, ``Your daughter-in-law Tamar has turned to prostitution!\fnote{\fbackref{38:24} Lit. \fbib{has been acting like a whore}} And look! She's pregnant because of it!''

``Bring her out,'' Judah responded. ``Let's burn her to death!''

\v{25}While they were bringing her out, she sent this message to her father-in-law: ``I am pregnant by the man to whom these things belong. Furthermore,'' she added, ``tell me to whom this signet ring, cord, and staff belongs.''

\v{26}When Judah recognized them, he admitted, ``She is more upright than I, because I never did give her my son Shelah.'' And he never had sex with her again.

\v{27}Later, when it was time for Tamar\fnote{\fbackref{38:27} Lit. \fbib{her}} to give birth, she was carrying twins in her womb! \v{28}While she was giving birth, one of them put out his hand, so the midwife grabbed it and tied something scarlet around his hand, observing, ``This one came out first.''

\v{29}As it was, he withdrew his hand, and then his brother was born. Amazed, the midwife\fnote{\fbackref{38:29} Lit. \fbib{Amazed, she}} cried out loud, ``What's this? A breach birth?'' So that boy\fnote{\fbackref{38:29} Lit. \fbib{So he}} was named Perez.\fnote{\fbackref{38:29} The Heb. name \fbib{Perez} means \fbib{breach}} \v{30}Afterwards, his brother came out, and around his hand was the scarlet. So they named him Zerah.\fnote{\fbackref{38:30} The Heb. name \fbib{Zerah} means \fbib{rising}}
\labelchapt{39}
\passage{Joseph is Delivered to Potiphar}

\chapt{39}
\v{1}Meanwhile, Joseph had been delivered to Egypt and turned over to Potiphar, one of Pharaoh's court officials and the Commander-in-Chief of the imperial guards. An Egyptian, he bought Joseph from the Ishmaelites, who had brought him down there.

\v{2}But the \divine{Lord} was with Joseph. He became a very prosperous man while in the house of his Egyptian master, \v{3}who could see that the \divine{Lord} was with Joseph,\fnote{\fbackref{39:3} Lit. \fbib{him}} because the \divine{Lord} made everything prosper that Joseph\fnote{\fbackref{39:3} Lit. \fbib{him}} did. \v{4}That's how Joseph pleased Potiphar\fnote{\fbackref{39:4} Lit. \fbib{Joseph found favor in his sight}} as he served him. Eventually, Potiphar appointed Joseph as overseer of his entire household. Moreover, he entrusted everything that he owned into his care.\fnote{\fbackref{39:4} Lit. \fbib{hand} and so throughout the chapter} \v{5}From the time he appointed Joseph to be overseer over his entire household and everything that he owned, the \divine{Lord} blessed the household of the Egyptian because of Joseph. The \divine{Lord}'s blessing rested on Joseph,\fnote{\fbackref{39:5} Lit. \fbib{him}} whether in Potiphar's household or in Potiphar's fields. \v{6}Everything that he owned, he entrusted into Joseph's care. He never concerned himself about anything, except for the food he ate.
\passage{Potiphar's Wife Accuses Joseph}

Now Joseph was well built and good looking. \v{7}That's why, sometime later, Joseph's master's wife looked straight at Joseph and propositioned him: ``Come on! Let's have a little sex!''\fnote{\fbackref{39:7} Lit. \fbib{Lie down with me}.}

\v{8}But he refused, telling his master's wife, ``Look! My master doesn't have to worry about anything in the house with me in charge, and he has entrusted everything into my care. \v{9}No one has more authority in this house than I do. He has withheld nothing from me, except you, and that's because you're his wife. So how can I commit such a horrible evil? How can I sin against God?''

\v{10}She kept on talking to him like this day after day, but he wouldn't listen to her. Not only would he refuse to have sex with her, he refused even to stay around her. \v{11}One day, though,\fnote{\fbackref{39:11} Lit. \fbib{About this time}} he went into the house to do his work. None of the household servants\fnote{\fbackref{39:11} Lit. \fbib{men}} were inside, \v{12}so she grabbed Joseph\fnote{\fbackref{39:12} Lit. \fbib{him}} by his outer garment and demanded ``Let's have some sex!''

Instead, Joseph ran outside, leaving his outer garment still in her hand. \v{13}When she realized that he had left his outer garment right there in her hand, she ran outside \v{14}and yelled for her household servants. ``Look!'' she cried out. ``My husband\fnote{\fbackref{39:14} Lit. \fbib{He}} brought in a Hebrew man to humiliate us. He came in here to have sex with me, but I screamed out loud! \v{15}When he heard me starting to scream, he left his outer garment with me and fled outside.'' \v{16}She kept his outer garment by her side until Joseph's master came home, \v{17}and then this is what she told him: ``That Hebrew slave whom you brought to us came in here to rape\fnote{\fbackref{39:17} Or \fbib{humiliate}} me. \v{18}But when I started to scream, he left his outer garment with me and ran outside.''
\passage{Joseph is Locked in Prison}

\v{19}When Joseph's master heard his wife's claim to the effect that ``This is how your servant treated me,'' he flew into a rage, \v{20}arrested Joseph, and locked him up in the same prison where the king's prisoners were confined. So Joseph remained there in prison.

\v{21}But the \divine{Lord} was with Joseph. He extended gracious love to him, causing the prison warden to be pleased with Joseph.\fnote{\fbackref{39:21} Lit. \fbib{him}} \v{22}So the prison warden entrusted into Joseph's care all the prisoners who were confined in prison. Whatever they did, Joseph was in charge of the work detail.\fnote{\fbackref{39:22} Lit. \fbib{was the one who did it}} \v{23}The prison warden did not have to worry about anything under Joseph's care, because the \divine{Lord} was with him. That's why Joseph prospered in everything he did.
\labelchapt{40}
\passage{Pharaoh's Two Servants}

\chapt{40}
\v{1}Some time later, both the senior security advisor\fnote{\fbackref{40:1} Lit. \fbib{the cupbearer}; a servant who tested food and beverages for poison; and so throughout the chapter; cf. Neh 1:11} to the king of Egypt and his head chef\fnote{\fbackref{40:1} Lit. \fbib{baker}; and so throughout the chapter} offended their master, Egypt's king. \v{2}Pharaoh was so angry with his two officers---his senior security advisor and his head chef--- \v{3}that he locked them up in the prison dungeon operated by the captain of the guard, the very place where Joseph was imprisoned. \v{4}The captain of the guard entrusted them to Joseph's custody, who took care of them, since they were to remain there in custody for a number of days.

\v{5}Then the two of them each had a dream. They both had their dreams the same night, and there were separate interpretations for each dream---the senior security advisor and the head chef to the king of Egypt, who had confined them in prison. \v{6}When Joseph came to see them in the morning, he noticed how downcast they looked! They were both very sad. \v{7}So he asked Pharaoh's officers, who were with him in prison in his master's house, ``Why are you so sad today?''

\v{8}``We had a dream,'' they replied, ``but there's no one to interpret it.''

``Interpretations belong to God,'' Joseph told them, ``so please tell me your stories.''
\passage{The Security Advisor's Dream}

\v{9}So the senior security advisor related his dream to Joseph. ``In my dream,'' he said, ``all of a sudden there was a vine in front of me! \v{10}On the vine were three branches that budded. Blossoms shot out, and clusters grew up that produced ripe grapes. \v{11}Then, with Pharaoh's cup in my hand, I took the grapes, squeezed them into Pharaoh's cup, then handed the cup directly to Pharaoh.''

\v{12}Then Joseph told him, ``This is what your dream means:\fnote{\fbackref{40:12} Lit. \fbib{is its interpretation}} The three branches are three days. \v{13}Within three days, Pharaoh will encourage you\fnote{\fbackref{40:13} Lit. \fbib{will lift up your head}} and return you to your responsibilities. You'll attend to Pharaoh's personal wine cup, just as you did when you were his senior security advisor. \v{14}But keep me in mind when things go well for you. Be sure to extend kindness to me by remembering me to Pharaoh. Bring me out of this prison,\fnote{\fbackref{40:14} Lit. \fbib{house}} \v{15}because I was kidnapped from the land of the Hebrews. Not only that, I haven't done anything that deserves me being confined to this pit.''
\passage{The Head Chef's Dream}

\v{16}When the head chef heard that the interpretation was good, he told Joseph, ``I was also in my dream. All of a sudden, there were three baskets with white bread stacked on top of my head. \v{17}There was all kinds of food in the basket that was on top, including baked food for Pharaoh. The birds were eating them from the basket on my head.''

\v{18}Joseph replied, ``This is what your dream means:\fnote{\fbackref{40:18} Lit. \fbib{is its interpretation}} The three baskets are also three days. \v{19}Within three more days, Pharaoh will behead you and hang you on gallows,\fnote{\fbackref{40:19} Lit. \fbib{a tree}} where birds will eat your flesh from you.''
\passage{The Dreams are Fulfilled}

\v{20}On the third day, which just happened to be Pharaoh's birthday, he threw a party for all his servants. He lifted the head of both his senior security advisor and of his head chef in front of his servants--- \v{21}that is, he restored his senior security advisor to his former responsibilities, including attending to Pharaoh's personal wine cup, \v{22}but he beheaded and\fnote{\fbackref{40:22} The Heb. lacks \fbib{beheaded and}} hanged the head chef, just as Joseph had interpreted for them. \v{23}Despite all of this, the senior security advisor not only didn't remember Joseph, he deliberately forgot him.
\labelchapt{41}
\passage{Pharaoh's Dream}

\chapt{41}
\v{1}Two years later---to the day---Pharaoh dreamed that he was standing by the Nile River,\fnote{\fbackref{41:1} The Heb. lacks \fbib{River}, and so throughout the chapter} \v{2}when all of a sudden seven healthy, plump cows emerged from the Nile to graze in the grass that grew in the reeds that lined the bank.\fnote{\fbackref{41:2} The Heb. lacks \fbib{that lined the bank}} \v{3}Right after that, seven more cows came up out of the Nile. Ugly and gaunt, they stood next to the other cows on the bank of the Nile River. \v{4}But all of a sudden they ate up the seven healthy, plump cows! Then Pharaoh woke up.

\v{5}After he had fallen back to sleep, he had a second dream, in which seven ears of plump, fruit-filled grain grew up on a single stalk. \v{6}Suddenly seven thin ears of grain that had been scorched by an east wind sprouted up right after them \v{7}and ate up the seven plump, fruit-filled ears. Then Pharaoh woke up a second time,\fnote{\fbackref{41:7} The Heb. lacks \fbib{a second time}} and it had been a very vivid\fnote{\fbackref{41:7} Lit. \fbib{and behold, it was a}} dream!
\passage{Pharaoh Seeks an Interpretation}

\v{8}The very next morning, he\fnote{\fbackref{41:8} Lit. \fbib{morning, his spirit}} was frustrated\fnote{\fbackref{41:8} Or \fbib{troubled}} about the dream, so he sent word to summon all the magicians and wise men of Egypt. Pharaoh told them what he had dreamed, but no one could interpret them.\fnote{\fbackref{41:8} Lit. \fbib{interpret the dreams for Pharaoh}}

\v{9}Then Pharaoh's senior security advisor\fnote{\fbackref{41:9} Lit. \fbib{Pharaoh's cupbearer}; a servant who tested the Pharaoh's food and beverages for poison; cf. Neh 1:11} spoke up. ``Maybe I should make a confession. \v{10}When Pharaoh was angry with some of his servants, he incarcerated me in custody of the captain of the bodyguard, along with Pharaoh's head chef.\fnote{\fbackref{41:10} Lit. \fbib{baker}} \v{11}We each had a dream on the same night, and each dream had its own meaning. \v{12}There was a Hebrew young man incarcerated with us, who was also working as a servant to the captain of the bodyguard.

``We each related our dreams,\fnote{\fbackref{41:12} The Heb. lacks \fbib{our dreams}} and then he interpreted them for us. He provided specific meanings for each of our dreams. \v{13}And what he interpreted for each of us came true! Pharaoh\fnote{\fbackref{41:13} Lit. \fbib{He}} restored me to my responsibilities, but he executed\fnote{\fbackref{41:13} Lit. \fbib{hanged}} the other man.''
\passage{Pharaoh Tells Joseph His Dream}

\v{14}Pharaoh sent word to summon Joseph quickly from the dungeon, so they shaved his beard, changed his clothes, and then sent him straight to Pharaoh. \v{15}``I've had a dream,'' Pharaoh told Joseph, ``but nobody can interpret it. I've heard that you can interpret dreams.''

\v{16}``I can't do that,'' Joseph replied, ``but God is concerned about Pharaoh's well-being.''

\v{17}So Pharaoh told Joseph, ``In my dream, I was standing on the bank of the Nile River, \v{18}and all of a sudden seven healthy, plump, beautiful cows emerged from the Nile and began to graze among the reeds that line the bank.\fnote{\fbackref{41:18} The Heb. lacks \fbib{that lined the bank}} \v{19}Just then, seven other cows emerged after them, poor, ugly, and appearing very gaunt in their flesh. I've never seen anything as ugly as those cows anywhere in the entire land of Egypt! \v{20}But those thin, gaunt cows gobbled up the first seven healthy cows! \v{21}Not only that,'' Pharaoh continued,\fnote{\fbackref{41:21} The Heb. lacks \fbib{Pharaoh continued}} ``after they had finished devouring the cows, nobody could tell that they had gobbled them up, because they were just as ugly as before. Then I woke up. \v{22}Later, I also dreamed about seven plump, fruit-filled ears of grain\fnote{\fbackref{41:22} The Heb. lacks \fbib{of grain}} that grew up out of a single stalk. \v{23}All of a sudden, seven thin, withered ears of grain,\fnote{\fbackref{41:23} The Heb. lacks \fbib{of grain}} scorched by the east wind, sprouted up after them. \v{24}But the thin ears gobbled up the seven good ears. I told all this to my advisors, but nobody was able to explain it to me.''
\passage{Joseph Interprets Pharaoh's Dream}

\v{25}``Pharaoh's dreams are identical,'' Joseph replied. ``God has told Pharaoh what he is getting ready to do. \v{26}The seven healthy cows represent seven years, as do the seven healthy ears. The dreams are identical. \v{27}The seven gaunt cows that arose after the healthy cows\fnote{\fbackref{41:27} Lit. \fbib{after them}} are seven years, as are the seven gaunt ears scorched by the east wind. There will be seven years of famine. \v{28}So the message that I have for Pharaoh is that God is telling Pharaoh what he is getting ready to do. \v{29}Be advised that seven years of phenomenal abundance are coming throughout all the land of Egypt, \v{30}but after them seven years of famine are ahead, during which all of the abundance will be forgotten throughout the land of Egypt. The famine will ravage the land so severely that\fnote{\fbackref{41:30} The Heb. lacks \fbib{so severely that}} \v{31}there will be no surplus in the land due to the coming famine, because it will be very severe.

\v{32}``Now since Pharaoh had that dream twice, it means that this event has been scheduled by God, and God will bring it to pass very soon. \v{33}Therefore let Pharaoh select a wise, discerning person to place in charge over the land of Egypt. \v{34}Also, let Pharaoh immediately proceed to appoint supervisors over the land of Egypt, who will collect one fifth of its agricultural production\fnote{\fbackref{41:34} Lit. \fbib{of the land}} during the coming seven years of abundance. \v{35}Let them collect all the food during the coming fruitful years, store up the grain in cities governed by Pharaoh's authority,\fnote{\fbackref{41:35} Lit. \fbib{cities in Pharaoh's hand}} and place it under guard. \v{36}Let the food be kept in reserve to feed\fnote{\fbackref{41:36} Lit. \fbib{reserve for}} the land for the seven years of famine that will occur throughout Egypt, so the people don't\fnote{\fbackref{41:36} Lit. \fbib{land doesn't}} die during the famine.''
\passage{Pharaoh Appoints Joseph as Regent}

\v{37}What Joseph proposed pleased Pharaoh and all of his advisors, \v{38}so Pharaoh asked his servants, ``Can we find anyone else like this---someone in whom the Spirit of God lives? \v{39}Since God has revealed all of this to you,'' Pharaoh told Joseph, ``there is no one so wise and discerning as you. \v{40}So you are to be appointed in charge over my palace, and all of my people are to do whatever you command them to do. Only the throne will have greater authority than you.''

\v{41}``Look!'' Pharaoh confirmed\fnote{\fbackref{41:41} Lit. \fbib{said}} to Joseph, ``I've put you in charge of the entire land of Egypt!''

\v{42}Then Pharaoh\fnote{\fbackref{41:42} Lit. \fbib{he}} removed his signet ring from his hand, placed it on Joseph's hand, had him clothed in fine linen garments, and placed a gold chain around his neck. \v{43}Then he provided him with a chariot as his second-in-command, outfitted with a group of people who shouted out in front of him, ``Bow your knees!'' And that's how Pharaoh set Joseph over the entire land of Egypt.
\passage{Pharaoh Rewards Joseph}

\v{44}Pharaoh also told Joseph, ``I'm still Pharaoh, but without your permission nobody in all of the land of Egypt will so much as lift up their hands or take a step!'' \v{45}Pharaoh also changed Joseph's name to Zaphenath-paneah\fnote{\fbackref{41:45} The Heb. name means \fbib{the God who speaks and lives}} and gave Asenath, daughter of Potiphera, the priest of On, to him as his wife. And that's how Joseph gained authority over the land of Egypt.
\passage{Joseph Begins Gathering Grain}

\v{46}Joseph was 30 years old when he began to serve Pharaoh, king of Egypt, by traveling throughout the land of Egypt, independent from Pharaoh's oversight.\fnote{\fbackref{41:46} Lit. \fbib{presence}} \v{47}While bumper crops grew during the seven abundant years, \v{48}Joseph\fnote{\fbackref{41:48} Lit. \fbib{he}} collected the surplus food throughout the land of Egypt, storing food in cities; that is, he gathered the food from fields that surrounded every city and stored it there. \v{49}Joseph stored up so much grain---like sand on the seashore in so much abundance!---that he stopped keeping records because it was proving to be impossible to measure how much they were gathering.
\passage{Joseph's Children are Born}

\v{50}Before the years of famine arrived, Joseph fathered two sons with Asenath, the daughter of Potiphera, the priest of On. \v{51}Joseph named his firstborn son\fnote{\fbackref{41:51} The Heb. lacks \fbib{son}} Manasseh because, he said, ``God has made me forget all of my hard life and my father's house.'' \v{52}He named his second son Ephraim because, he said, ``God has made me fruitful in the land of my troubles.''
\passage{The Famine Begins}

\v{53}As soon as the seven years of abundance throughout the land of Egypt ended, \v{54}the seven years of famine started, just as Joseph had predicted.\fnote{\fbackref{41:54} Lit. \fbib{said}} It was an international famine, but there was food everywhere throughout the land of Egypt. \v{55}Eventually, the land of Egypt began to feel the effects of the famine, so the people\fnote{\fbackref{41:55} Lit. \fbib{so they}} cried out to Pharaoh for food. ``Go see Joseph,'' Pharaoh announced to all the Egyptians, ``and do whatever he tells you to do.''

\v{56}Joseph opened all of the storehouses and sold grain to the Egyptians, because the famine was beginning to be severe throughout the land of Egypt. \v{57}In addition, all of the surrounding nations\fnote{\fbackref{41:57} Lit. \fbib{the world}} came to Joseph to buy grain from Egypt, because the famine had become severe throughout the world.
\labelchapt{42}
\passage{Joseph's Brothers Visit Egypt}

\chapt{42}
\v{1}Eventually, Jacob observed that there was grain in Egypt, so he asked his sons, ``Why do you keep on staring at one another? \v{2}Pay attention now! I've heard that there is grain in Egypt, so go down there and buy some grain for us, so we can live, instead of dying.''

\v{3}So ten of Joseph's brothers left to buy grain from Egypt. \v{4}Jacob would not send Joseph's brother Benjamin to accompany them, because he was saying, ``I'm afraid that he'll come to some kind of harm.'' \v{5}Israel's sons went in a caravan that included others who were going to Egypt to buy grain, because the famine pervaded the land of Canaan, too.
\passage{Joseph's Brothers Encounter Joseph}

\v{6}Meanwhile, Joseph continued to be ruler over the land, in charge of selling to everyone in the land. Joseph's brothers appeared and bowed down to him, face down.\fnote{\fbackref{42:6} Lit. \fbib{faces to the ground}} \v{7}As soon as Joseph saw his brothers, he knew who they were, but he remained disguised and asked them gruffly, ``Where are you from?''

``From the land of Canaan,'' they replied. ``We're here\fnote{\fbackref{42:7} The Heb. lacks \fbib{We're here}} to buy food.''

\v{8}But Joseph had already recognized his brothers, even though they had not recognized him. \v{9}Furthermore, Joseph remembered the dreams that he had about them. So he accused them, ``You're spies! You've come here to spy on our undefended territories!''\fnote{\fbackref{42:9} Lit. \fbib{to scout the nakedness of the land}}

\v{10}``No, your majesty,'' they replied. ``Your servants have come here to buy food. \v{11}We're all sons of a common father. We're honest men, your majesty. We're\fnote{\fbackref{42:11} Lit. \fbib{Your servants are}} not spies!''

\v{12}But Joseph\fnote{\fbackref{42:12} Lit. \fbib{he}} kept insisting, ``It's just as I've said---you've come here to spy on our unguarded\fnote{\fbackref{42:12} Lit. \fbib{naked}} territories!''

\v{13}``But your majesty,'' they pleaded, ``your servants include twelve brothers, the sons of a common father back in the land of Canaan. Please! Our youngest brother\fnote{\fbackref{42:13} The Heb. lacks \fbib{brother}} remains with our father, and the other one\fnote{\fbackref{42:13} The Heb. lacks \fbib{one}} is no longer alive.''

\v{14}``I'm right!'' Joseph insisted. ``Just as I said, you're spies! \v{15}So here's how we'll test you. You can bet the life of Pharaoh that you're not leaving here until your youngest brother comes here! \v{16}One of you is to be sent back so he can get your brother while the rest of\fnote{\fbackref{42:16} The Heb. lacks \fbib{the rest of}} you remain in custody. That way, we'll test whether or not you're telling the truth. If you're not, as surely as the Pharaoh lives, you're spies!''

\v{17}Then Joseph locked them all together in prison for three days. \v{18}Three days later, Joseph told them, ``I fear God, so do this and you'll live. \v{19}If you're honest men, leave one of your brothers here in custody, then the rest of\fnote{\fbackref{42:19} The Heb. lacks \fbib{the rest of}} you can leave and take some grain with you\fnote{\fbackref{42:19} The Heb. lacks \fbib{with you}} to alleviate the famine that's affecting your households. \v{20}Just be sure to bring your youngest brother back to me so what you've claimed can be verified. That way, you won't die.''
\passage{Joseph's Brothers Mull over Their Predicament}

\v{21}``We're all guilty because of what we did to\fnote{\fbackref{42:21} The Heb. lacks \fbib{what we did to}} our brother!'' they told each other. ``We kept on watching his suffering while he pleaded with us! We're in this mess because we wouldn't listen!''

\v{22}``Didn't I tell you!'' Reuben replied. ```Don't wrong the kid!' I said, but would you listen? No! Now it's payback time!''

\v{23}Meanwhile, they had no idea that Joseph could understand them, since he was talking to them through an interpreter. \v{24}He turned away from them and began to weep.
\passage{Joseph Arrests Simeon}

When he returned, he spoke with them, but then he took Simeon away from them and had him placed under arrest\fnote{\fbackref{42:24} Lit. \fbib{him bound}} right in front of them. \v{25}After this, Joseph gave orders to fill up their sacks with grain, to return each man's money to his own sack, and to supply each of them with provisions for their return journey. All of this was done for them.
\passage{Joseph's Brothers Leave for Canaan}

\v{26}Then they each mounted up, their donkeys having been loaded with grain, and left from there. \v{27}Later on, one of them opened up his sack to give his donkey some fodder after they had stopped at the place where they intended to lodge for the night. There, in the mouth of his sack, was all of his money! \v{28}He reported to his brothers, ``My money has been returned! It's right here in my sack!''

Trembling with mounting consternation, each of them asked one another, ``What is God doing to us?''
\passage{Jacob Learns What Happened in Egypt}

\v{29}As soon as they had returned to their father Jacob in the land of Canaan, they told him everything that had happened to them. \v{30}``The man who was in charge\fnote{\fbackref{42:30} Lit. \fbib{was lord}; and so in v. 33} of the land spoke harshly to us,'' they said. ``He accused us of being spies!\fnote{\fbackref{42:30} Lit. \fbib{spies of the land}} \v{31}But we told him, `No! We're honest men! We're not spies! \v{32}Our father has twelve sons, but one of us isn't alive anymore, and our youngest brother is with our father today back home in\fnote{\fbackref{42:32} Lit. \fbib{today in the land of}} Canaan.' \v{33}But the man who was in charge of the land responded, `I'm going to test your honesty. Leave one of your brothers with me, take some grain for the famine that's afflicting your households, and leave. \v{34}But bring your youngest brother back to me so I can be sure that you're honest men, and not spies. Then I'll return your brother to you, and you'll be allowed to trade anywhere in the land.'\,''

\v{35}Later on, as they went about unloading their sacks, each man's bundle of money was found in each man's sack. When they and their father saw their bundles of money, they were greatly distressed. \v{36}Their father Jacob told them, ``You're causing me to lose my children! Joseph is gone. Now Simeon is gone, and you're planning to take Benjamin, too. Everything's going against me!''

\v{37}``Feel free to put my own two sons to death,'' Reuben responded to his father, ``if I don't bring him back to you. Trust me---I'll bring him back to you.''

\v{38}But Jacob replied, ``My son isn't going back with you, since his brother is dead and he's the only one left. If something should harm him as you travel, then it'll be death for me and my sad, gray hair!''\fnote{\fbackref{42:38} Lit. \fbib{then you'll send me and my gray hair to Sheol}; i.e. to the realm of the dead}
\labelchapt{43}
\passage{Preparing to Return to Egypt}

\chapt{43}
\v{1}Meanwhile, the famine remained severe throughout the region. \v{2}As a result, when Jacob's family\fnote{\fbackref{43:2} Lit. \fbib{As they}} was beginning to eat the last of the grain that they had brought back from Egypt, their father Jacob\fnote{\fbackref{43:2} The Heb. lacks \fbib{Jacob}} told his sons, ``Go back to Egypt and buy us some food.''

\v{3}But Judah reminded him, ``The man distinctly warned us: `You'll never see my face unless your brother comes with you.' \v{4}So if you send our brother with us, we'll go down and buy some food. \v{5}But if you don't send him, we're not going, because the man told us, `You'll never see my face unless your brother is with you.'\,''

\v{6}Israel replied, ``Why did you make all this trouble by telling the man that you have another brother?''

\v{7}``The man specifically asked about us and our relatives,'' they responded. ``He asked us, `Is your father still alive?' and `Do you have another brother?' So we answered his questions. How could we have known that he would tell us to bring our brother back with us?''

\v{8}``Send the young man with me,'' Judah told his father Israel, ``and we'll get up and go so we can survive and not die---and that includes all of us, you and our families.\fnote{\fbackref{43:8} Lit \fbib{our defenseless ones}; i.e. their wives and children} \v{9}I'll even offer myself to guarantee that I'll be responsible for him. If I don't bring him back and present him to you, I'll personally bear the consequences forever. \v{10}After all, if we hadn't delayed, we could have been there and back\fnote{\fbackref{43:10} Lit. \fbib{have returned}} twice by now!''
\passage{Jacob Gives Instructions for the Trip}

\v{11}``If that's the way it has to be,'' their father Israel replied, ``then do this: take some of the best produce of the land in your containers and take them to the man as a gift---some resin ointment, some honey, fragrant resins, myrrh, pistachios, and almonds. \v{12}Also take twice as much money with you so you can return the money that had been replaced in the mouth of your sacks. Maybe it was an accounting\fnote{\fbackref{43:12} The Heb. lacks \fbib{accounting}} mistake on his part. \v{13}And be sure to take your brother, too. So get up, return to the man, \v{14}and may God Almighty cause the man to show compassion toward you. May he send all of you back, including your other brother and Benjamin. Now as for me, if I lose my children, I lose them.''

\v{15}So the men took their gift and twice as much money, got up, took Benjamin with them, and set out for Egypt. Eventually they appeared before Joseph.
\passage{Joseph Sees Benjamin}

\v{16}As soon as Joseph noticed that Benjamin had come with them, he ordered his palace manager, ``Bring the men into the palace.\fnote{\fbackref{43:16} Lit. \fbib{house}, and so through v. 26} Slaughter an animal and prepare it, because these men will be dining with me for lunch.''\fnote{\fbackref{43:16} Or \fbib{me at midday}; i.e. at noon} \v{17}So the man did what Joseph had ordered, and brought the men to Joseph's palace.

\v{18}The men were terrified as they were being taken to Joseph's palace. ``It's because of that money that was returned to our sacks the first time we were brought to him,'' they reasoned. ``He's seeking an excuse to attack us, enslave us, and confiscate our donkeys!''

\v{19}So they approached Joseph's palace manager and talked with him at the palace entrance. \v{20}``Your Excellency,'' they said, ``The first time we came here to buy food, \v{21}when we arrived at our overnight lodging place, we opened our sacks and discovered each man's money was still in the mouth of his sack. All of our money was there! We've brought it back with us in full. \v{22}We've also brought along some more money to buy supplies, but we don't know who put our money back into our sacks.''

\v{23}``Relax,'' the manager said. ``You can stop being afraid, now. Your God, the God of your father, has placed hidden treasure within those sacks for you. I've been paid in full.'' Then he brought Simeon out to them, \v{24}ushered the men into Joseph's palace, gave them water to wash their feet, and provided\fnote{\fbackref{43:24} The Heb. lacks \fbib{provided}} fodder for their donkeys. \v{25}Then off he went to prepare the honorary meal that was to be made ready for Joseph's arrival at noon, since they had been informed that they were going to be eating there.
\passage{Joseph Inquires about His Family}

\v{26}When Joseph arrived at his palace, his brothers\fnote{\fbackref{43:26} Lit. \fbib{palace, they}} brought to him their gifts that they had carried with them and bowed to the ground in front of him.

\v{27}Joseph asked them how they had been doing. ``Is your father well, the older gentleman about whom you spoke?'' he inquired. ``Is he still alive?''

\v{28}``Your servant, our father, is doing well,'' they replied. ``He is still alive.'' Then they bowed down in humility.

\v{29}As Joseph looked up and recognized his brother Benjamin, his own mother's son, he asked, ``Is this your youngest brother about whom you spoke to me?'' And he addressed him directly, ``May God be gracious to you, my son.''\fnote{\fbackref{43:29} Or \fbib{you, Benny}; i.e., perhaps a nickname for Joseph's brother \fbib{Benjamin}}

\v{30}At this, Joseph hurried out, deeply moved because of his brother, and looked for a place to weep by himself. He entered his personal quarters, wept there awhile,\fnote{\fbackref{43:30} The Heb. lacks \fbib{awhile}} \v{31}then washed his face and came out. Barely controlling himself, he ordered his staff to serve the meal.

\v{32}Joseph's staff\fnote{\fbackref{43:32} Lit. \fbib{They}} served him by himself, his brothers\fnote{\fbackref{43:32} Lit. \fbib{and them}} separately, and the Egyptian staff members by themselves, because the Egyptians wouldn't take their meal with the Hebrews, since doing so was detestable for the Egyptians. \v{33}Meanwhile, the brothers\fnote{\fbackref{43:33} Lit. \fbib{they}} were seated in front of Joseph in birth order, from firstborn to youngest. The men stared at one another in astonishment. \v{34}Joseph\fnote{\fbackref{43:34} Lit. \fbib{He}} himself brought portions to them from his own table, except that he provided to Benjamin five times as much as he did for each of the others. So they feasted together and drank freely with Joseph.\fnote{\fbackref{43:34} Lit. \fbib{him}}
\labelchapt{44}
\passage{The Brothers Leave for Canaan}

\chapt{44}
\v{1}Later, Joseph\fnote{\fbackref{44:1} Lit. \fbib{he}} commanded his palace manager, ``Fill the men's sacks to full capacity with food and replace each man's money at the top of the sack. \v{2}Then place my cup---the silver one---in the top of the sack belonging to the youngest one, along with the money he brought to buy\fnote{\fbackref{44:2} The Heb. lacks \fbib{he brought to buy}} grain.'' So the manager\fnote{\fbackref{44:2} Lit. \fbib{So he}} did precisely what Joseph told him to do.

\v{3}Early the next morning, the men were sent on their way, along with their donkeys. \v{4}They had not traveled far from the city when Joseph ordered his palace manager, ``Get up, follow those men, and when you've caught up with them, ask them, `Why did you repay evil for good? \v{5}Don't you have\fnote{\fbackref{44:5} Lit. \fbib{Isn't this}} the cup that my master uses to drink from and also uses to practice divination? You're wrong to have done this.'\,'' \v{6}So he went after them and made that accusation.

\v{7}``Your Excellency,'' they replied, ``Why do you speak like this? Far be it from your servants to act like this. \v{8}Look, we brought back to you from the land of Canaan the money that we found at the top of our sacks. How, then, could we have stolen silver or gold from your master's palace? \v{9}Go ahead and execute whichever one of your servants is discovered to have it, and we'll remain as your master's slaves.''

\v{10}``Agreed,'' he responded. ``Just as you've said, the one who is found to have it in his possession will become my slave, and the rest of\fnote{\fbackref{44:10} The Heb. lacks \fbib{the rest of}} you will be innocent.''

\v{11}So they quickly dismounted, unloaded their sacks onto the ground, and each one of them opened his own sack. \v{12}The palace manager\fnote{\fbackref{44:12} Lit. \fbib{Then he}} searched for the cup, beginning with the oldest brother's sack and ending with the youngest brother's sack, and there it was!---in Benjamin's sack. \v{13}At this, they all tore their clothes,\fnote{\fbackref{44:13} I.e., a response of despair} reloaded their donkeys, and returned to the city.
\passage{Joseph Confronts His Brothers}

\v{14}Joseph was waiting for them back at his palace when his brothers returned. They fell to the ground in front of him, \v{15}and Joseph asked them, ``Why did you do this? Don't you know that I'm an expert at divination?''
\passage{Judah Explains Their Predicament}

\v{16}``What can we say, Your Excellency?'' Judah replied. ``How can we explain this or justify ourselves? God has discovered the sin of your servants, and now we've become slaves to you, Your Excellency, both we and the one in whose possession the cup has been discovered.''

\v{17}``Far be it from me to do this,'' Joseph\fnote{\fbackref{44:17} Lit. \fbib{he}} responded. ``The man in whose possession the cup was discovered will be my slave, but the rest of you may leave in peace to be with your father.''

\v{18}But Judah approached him and begged him, ``Your Excellency, please allow your servant to speak to you privately.\fnote{\fbackref{44:18} Lit. \fbib{speak a word in your ears}} Please don't be angry with your servant, since you are equal to Pharaoh. \v{19}Your Excellency asked his servants, `Do you have a father or brother?' \v{20}and we answered Your Excellency, `We have an aged father and a younger child who was born when he was old. His brother is now dead, so he's the only surviving son of his mother. His father loves him.'

\v{21}``But then you ordered your servants, `Bring him here to me so I can see him for myself.' \v{22}So we told Your Excellency, `The young man cannot leave his father, because if he were to do so, his father would die.' \v{23}But then you told your servants, `Unless your youngest brother comes back with you, you won't see my face again.' \v{24}Later on, after we had gone back to your servant, my father, we told him what Your Excellency had said.

\v{25}```Go back,' our father ordered, `and buy us a little food.'

\v{26}``But we told him, `We can't go back there. If our youngest brother accompanies us, we'll go back, but we cannot see the man's face again unless our youngest brother accompanies us.'

\v{27}``Then your servant, our father, told us, `You know my wife bore me two sons. \v{28}One of them left me, so I concluded ``I'm certain that he has been torn to pieces,'' and I haven't seen him since then. \v{29}If you take this one from me, too, and then something harmful happens to him, then it will be death for me and my sad, gray hair!'\fnote{\fbackref{44:29} Lit. \fbib{then you'll send me and my gray hair to Sheol}; i.e. to the realm of the dead}

\v{30}``So when I go back to your servant, my father, and the young man isn't with us, since he's constantly living life focused on his son,\fnote{\fbackref{44:30} Lit. \fbib{since his soul is bound to his son's soul}} \v{31}when he notices that the young man hasn't come back with us, he'll die, and your servants really will have brought death to your servant, our father,\fnote{\fbackref{44:31} Lit. \fbib{have brought your servant, our father, to Sheol}; i.e. to the realm of the dead} along with his sad, gray hair! \v{32}Also, your servant pledged his own life as\fnote{\fbackref{44:32} The Heb. lacks \fbib{his own life as}} a guarantee of the young man's safety. I told my father, `If I don't bring him back to you, you can blame me forever.' \v{33}Therefore, please allow your servant to remain as a slave to Your Excellency, instead of the young man, and let the young man go back home with his brothers. \v{34}After all, how can I go back to my father if the young man doesn't accompany me? I'm afraid of what might happen to my father.''
\labelchapt{45}
\passage{Joseph Reveals Himself}

\chapt{45}
\v{1}At this point, Joseph could not control his emotions any longer, so he cried out to everyone who was standing nearby, ``Everybody! Leave me!'' As a result, none of his staff\fnote{\fbackref{45:1} Lit. \fbib{result, no man}} was anywhere near\fnote{\fbackref{45:1} Lit. \fbib{was standing nearby}} him when he revealed himself to his brothers. \v{2}He cried so loudly that the Egyptians heard him, including Pharaoh's household.

\v{3}Joseph blurted out, ``I'm Joseph! Is my father really alive?'' But his brothers could not answer him, because they had become terrified\fnote{\fbackref{45:3} Or \fbib{dismayed}} to be in his presence.

\v{4}Joseph implored his brothers, ``Please come close to me.'' So they did.

``I'm your brother Joseph, whom you sold into slavery in\fnote{\fbackref{45:4} The Heb. lacks \fbib{slavery in}} Egypt!'' he told them. \v{5}``But\fnote{\fbackref{45:5} Or \fbib{So}} don't be distressed or angry at yourselves because you sold me here, because God sent me ahead of you all in order to deliver us.\fnote{\fbackref{45:5} The Heb. lacks \fbib{us}} \v{6}That's because this famine has been going on for two years now in this region, and there are still five years left, during which there won't be any plowing or harvesting. \v{7}God sent me ahead of you to keep you alive on the earth, and to save you all in a magnificent way. \v{8}As a result, it wasn't you who sent me here, but God himself! He established me as a father-figure to Pharaoh himself! I'm in charge of his entire palace and ruler over the entire land of Egypt. \v{9}So hurry up, go back to my father, and tell him that his son Joseph tells him, `God has made me master of all of Egypt. Hurry up! Come live with me!' \v{10}You are to live in the land of Goshen, near where I am---you, your children, your grandchildren, your flocks, your herds, and everything that you own. \v{11}I'll provide for you there, since there are still five years of famine left to go, and you, your households, and everything you own would have otherwise become impoverished.

\v{12}``Look, now! All of you can see me! And my own brother Benjamin can tell that it's really me\fnote{\fbackref{45:12} Lit. \fbib{it's my mouth}} speaking to you! \v{13}So go tell my father about all of my splendor in Egypt. Tell him about everything that you've seen. Be quick about it, and bring my father down here!''

\v{14}Then he threw his arms around Benjamin\fnote{\fbackref{45:14} Lit. \fbib{he collapsed on Benjamin's neck}} and wept as they embraced.\fnote{\fbackref{45:14} Lit. \fbib{as Benjamin wept on his neck}} \v{15}He kissed all of his brothers and wept with them, too, and then his brothers were able to talk with him.
\passage{Pharaoh is Pleased}

\v{16}As soon as the news reached Pharaoh's palace that Joseph's brothers had arrived, Pharaoh and his servants were ecstatic. \v{17}Pharaoh told Joseph, ``Be sure to tell your brothers, `Do this: load up your livestock, go back to the land of Canaan, \v{18}get your father and your households, and come back to me. I'll give you the best of the land of Egypt and you can live off the abundance of the land.' \v{19}In addition,'' Pharaoh ordered, ``Do this: take some transport wagons from the land of Egypt for your little ones to ride in, along with your wives, and bring your father and come! \v{20}Don't worry about your household goods, because the best of all the land of Egypt is yours.''
\passage{Joseph's Brothers Go Back Home}

\v{21}So Israel's sons did what they were asked to do, and Joseph provided wagons for them, as Pharaoh had commanded. He also gave them provisions for the journey. \v{22}He gave each of them some changes of clothes, but he also gave Benjamin 300 pieces of silver and five changes of clothes. \v{23}He sent his father ten male donkeys loaded with the best of Egyptian goods and ten female donkeys loaded with grain, bread, and provisions for his father during the journey. \v{24}Then Joseph\fnote{\fbackref{45:24} Lit. \fbib{he}} sent his brothers away, and they left for home.\fnote{\fbackref{45:24} The Heb. lacks \fbib{for home}} As they were leaving, Joseph admonished them, ``Don't quarrel on the way back!''

\v{25}So Joseph's brothers\fnote{\fbackref{45:25} Lit. \fbib{So they}} left Egypt and returned to the land of Canaan and to their father Jacob, \v{26}where they informed their father, ``Joseph is still alive! As a matter of fact, he's ruling the entire land of Egypt.'' But Jacob didn't believe them, because he had become cynical.\fnote{\fbackref{45:26} Lit. \fbib{because his heart had become numb}} \v{27}However, as soon as his sons\fnote{\fbackref{45:27} Lit. \fbib{as they}} had told him everything Joseph had said, and after he saw the wagons that Joseph had sent along to carry him, their father Jacob's spirit was encouraged.

\v{28}``It's enough,'' Israel replied. ``My son Joseph is still alive. I'm going to go see him before I die!''
\labelchapt{46}
\passage{The Move to Egypt}

\chapt{46}
\v{1}Later, Israel began his journey, taking along everything that he owned, and arrived at Beer-sheba, where he offered sacrifices to the God of his father Isaac. \v{2}God spoke to Israel through night visions, addressing him, ``Jacob! Jacob!''

``Here I am!'' Jacob\fnote{\fbackref{46:2} Lit. \fbib{he}} replied.

\v{3}``I'm God, your father's God. Don't be afraid to move down to Egypt, because I'm going to turn you into a mighty nation there. \v{4}I'm going down with you to Egypt, and I'm certainly going to bring you back again. And Joseph himself will be with you when you die.''\fnote{\fbackref{46:4} Lit. \fbib{will place his hand over your eyes}} \v{5}So Jacob got up and left Beer-sheba, and Israel's sons carried their father Jacob, their little ones, and their wives in the transport wagons that Pharaoh had sent to carry them. \v{6}They took their livestock and their household property that they had acquired in the land of Canaan and traveled to Egypt. Jacob and all of his descendants went with him--- \v{7}including his sons, his grandsons, his daughters, and his granddaughters---every one of his descendants accompanied him to Egypt.
\passage{List of Those who Went to Egypt}
\passageinfo{(Ex 1:1--4; Num 26:4, 5; 1Chron 2:1ff)}

\v{8}Here's a list of the names of Israel's sons, that is, of Jacob and his sons who moved to Egypt: Reuben, Jacob's firstborn; \v{9}Reuben's sons Hanoch, Pallu, Hezron, and Carmi; \v{10}Simeon's sons Jemuel,\fnote{\fbackref{46:10} Cf. Num 26:12 and 1Chr 4:24, where his name is spelled \fbib{Nemuel.}} Jamin, Ohad, Jachin,\fnote{\fbackref{46:10} Cf. 1Chr 4:24, where his name is spelled \fbib{Jarib.}} Zohar,\fnote{\fbackref{46:10} Cf. Num 26:13 and 1Chr 4:24, where his name is spelled \fbib{Zerah.}} and Shaul, who was the son of a Canaanite woman; \v{11}Levi's sons Gershon,\fnote{\fbackref{46:11} Cf. 1Chr 6:16, where his name is spelled \fbib{Gershom.}} Kohath, and Merari; \v{12}and Judah's sons Er, Onan, Shelah, Perez, and Zerah. (Technically,\fnote{\fbackref{46:12} Lit. \fbib{but}} Er and Onan had died in the land of Canaan.) Perez's sons were Hezron and Hamul. \v{13}Also included were Issachar's sons Tola, Puvvah,\fnote{\fbackref{46:13} Cf. Num 26:23, where his name is spelled \fbib{Puvah}, and 1Chr 7:1, where his name is spelled \fbib{Puah.}} Job,\fnote{\fbackref{46:13} Cf. Num 26:24 and 1Chr 7:1, where his name is spelled \fbib{Jashub.}} and Shimron; \v{14}along with Zebulun's sons Sered, Elon, and Jahleel. \v{15}These were all sons from Leah, whom she bore for Jacob in Paddan-aram,\fnote{\fbackref{46:15} Paddan-aram was located in northwest Mesopotamia} along with his daughter Dinah. He had 33 sons and daughters.

\v{16}Also included were Gad's sons Ziphion, Haggi, Shuni, Ezbon, Eri, Arodi, and Areli; \v{17}Asher's sons Imnah, Ishvah, Ishvi, Beriah, and their sister Serah. Beriah's sons Heber and Malchiel were also included.\fnote{\fbackref{46:17} The Heb. lacks \fbib{were also included}} \v{18}These were all sons from Zilpah, whom Laban had given to his daughter Leah. She bore these sixteen children for Jacob.

\v{19}Jacob's wife Rachel's sons were Joseph and Benjamin.

\v{20}Joseph's sons born in the land of Egypt were Manasseh and Ephraim, whom Asenath, daughter of Potiphera, the priest of On, bore for him. \v{21}Benjamin's sons included Bela, Becher, Ashbel, Gera, Naaman, Ehi, Rosh, Muppim, Huppim, and Ard. \v{22}These were all the sons of Rachel, who were born for Jacob---fourteen in all.

\v{23}Also included were Dan's son Hushim; \v{24}Naphtali's sons Jahzeel, Guni, Jezer, and Shillem. \v{25}These were sons of Bilhah, whom Laban had given to his daughter Rachel. She bore these children for Jacob---seven in all.

\v{26}All of these people, who belonged to Jacob's family, traveled to Egypt. All of Jacob's\fnote{\fbackref{46:26} Lit. \fbib{his}} direct descendants, not including his sons' wives, numbered 66 persons in all. \v{27}Joseph had two sons born to him in Egypt, and all of Jacob's household who went to Egypt numbered 70.
\passage{Jacob Arrives in Goshen}

\v{28}Jacob\fnote{\fbackref{46:28} Lit. \fbib{He}} sent Judah ahead of them to meet with Joseph, who would be guiding them to Goshen, and so they arrived. \v{29}Joseph prepared his chariot and went to meet his father Israel in Goshen. As soon as Jacob\fnote{\fbackref{46:29} Lit. \fbib{he}} appeared in his presence, he embraced him\fnote{\fbackref{46:29} Lit. \fbib{he fell on his neck}} and wept for a long time as he held on to him.\fnote{\fbackref{46:29} Lit. \fbib{to his neck}} \v{30}``Now let me die,'' Israel told Joseph, ``since I've seen your face and confirmed that you're still alive!''

\v{31}But Joseph addressed his brothers and his father's household and told them, ``I'll go up and tell Pharaoh that my brothers and my father's household have arrived from Canaan to be with me. \v{32}I'll mention that\fnote{\fbackref{46:32} The Heb. lacks \fbib{I'll mention that}} the men are shepherds. Because they've been taking care of livestock, they brought along their flocks, their herds, and everything else that they own. \v{33}When Pharaoh calls for you and asks you `What's your occupation?' \v{34}you are to tell him, `Your servants have been taking care of livestock since we were youths. We and our ancestors have taken care of livestock.' That way, you'll be able to live in the Goshen territory, since shepherds are detestable to the Egyptians.''
\labelchapt{47}
\passage{Joseph's Family Settles in Goshen}

\chapt{47}
\v{1}After this, Joseph went to inform Pharaoh. ``My father and brothers have come here from Canaan,''\fnote{\fbackref{47:1} Lit. \fbib{from the land of Canaan}, and so throughout the chapter} he said, ``and they've come with their flocks, herds, and everything else they have. I settled them in the Goshen territory!'' \v{2}He brought along five of his brothers to present before Pharaoh.

\v{3}Pharaoh asked his brothers, ``What are your occupations?''

``Your servants are shepherds,'' they replied, ``both we and our ancestors. \v{4}We've come to live for a while\fnote{\fbackref{47:4} The Heb. lacks \fbib{for a while}} in this region, since there is no pasture back in Canaan\fnote{\fbackref{47:4} The Heb. lacks \fbib{back in Canaan}} for your servants' flocks. May your servants please live in the Goshen territory?''

\v{5}Then Pharaoh replied to Joseph, ``Now that your father and your brothers have come to you, \v{6}Egypt\fnote{\fbackref{47:6} Lit. \fbib{from the land of Egypt}, and so throughout the chapter} is at your disposal,\fnote{\fbackref{47:6} Lit. \fbib{is before you}} so settle your father and brothers in the best part of the land! Let them live in the Goshen territory. If you learn that any of them are especially skilled, put them in charge of my livestock.''

\v{7}Later, Joseph brought his father Jacob to Pharaoh and introduced him. Jacob blessed Pharaoh. \v{8}``How old are you?''\fnote{\fbackref{47:8} Lit. \fbib{How many years have you lived?}} Pharaoh asked Jacob.

\v{9}``I'm 130 years old,'' Jacob replied. ``My years have turned out to be few and unpleasant, but I haven't yet reached the age my ancestors did during their travels on earth.''\fnote{\fbackref{47:9} The Heb. lacks \fbib{on earth}} \v{10}Then Jacob blessed Pharaoh and then left the throne room.\fnote{\fbackref{47:10} Lit. \fbib{left his presence}}

\v{11}Joseph settled his father and brothers, assigning them their own land in the best part of Egypt (in the territory of Rameses), just as Pharaoh had ordered. \v{12}Joseph provided food for his father, his brothers, and all of his father's household, proportionate to the number of young children.
\passage{The Famine Continues}

\v{13}Meanwhile, there continued to be no food throughout the land, because the famine remained very severe. As a result, both Egypt and Canaan languished under the effects of the famine. \v{14}So Joseph kept on accumulating all the money that was to be found throughout Egypt and Canaan in exchange for the grain that was being purchased. He stored the money in Pharaoh's palace.

\v{15}After all the money had been spent throughout Egypt and Canaan, all the Egyptians came to Joseph and demanded, ``Give us food! Why should we die right in front of you? Our money is spent!''

\v{16}``You can surrender your livestock,'' Joseph replied. ``I'll feed them in exchange, since your money is gone.''

\v{17}So they brought their livestock to Joseph, and Joseph traded food in exchange for horses, various flocks and herds, and donkeys. He fed them with food in exchange for their livestock during that year.

\v{18}The following year, they came to him and reminded him, ``We won't hide from you, your Excellency, that we've spent all of our money, and that our livestock all belong to you. There's nothing left to trade with you, your Excellency, except our bodies and our territories. \v{19}So why should we and our land die right in front of you? Buy us and our land in exchange for food, and we and our land will be slaves to Pharaoh. Give us seed, so we can survive and not die, and so the land won't stay desolate.''
\passage{Pharaoh Gains Control of All of Egypt}

\v{20}So Joseph purchased all of the Egyptian territory for Pharaoh. Every Egyptian sold his field, because the famine's effect was so severe. That's how Pharaoh came to own the land. \v{21}Then Joseph transported the people to cities from one end of Egypt to the other. \v{22}However, he did not purchase land belonging to the priests, because the priests held an allotment, previously provided to them by Pharaoh, from which they lived. That's why they did not sell their land.

\v{23}After this, Joseph addressed the people. ``Pay attention,'' he said. ``I've bought you and your land for Pharaoh today, in exchange for seed for you. Now go sow the land. \v{24}When harvest season arrives, you are to provide a fifth of the harvest to Pharaoh. The remaining four fifths are to be for your use, for seed, and to feed you, your households, and your little ones.''

\v{25}``You've saved our lives,'' they replied. ``If it pleases you, your Excellency, we'll be Pharaoh's slaves.''

\v{26}So Joseph crafted a statute concerning Egypt that remains valid to this day that Pharaoh should own a fifth of the produce, excluding the land belonging to the priests, which remained outside of Pharaoh's control.

\v{27}Israel remained in Egypt's Goshen territory, acquired land there, became prosperous, and his descendants\fnote{\fbackref{47:27} The Heb. lacks \fbib{his descendants}} grew very numerous. \v{28}He lived for seventeen more years in Egypt, until he was 147 years old. \v{29}As the time approached for Israel to die, he called for his son Joseph and addressed him. ``Please,'' he asked, ``if you're happy with me, make a solemn promise\fnote{\fbackref{47:29} Lit. \fbib{me, place your hand under my thigh}; i.e., make a solemn promise based on the sanctity of the family and commitment to the family line} that you'll treat me fairly and kindly by not burying me in Egypt. \v{30}Instead, when I've died, as my ancestors have, you are to carry me out of Egypt and bury me in their tomb.''\fnote{\fbackref{47:30} Lit. \fbib{place}}

``I'll do what you've asked,'' Joseph\fnote{\fbackref{47:30} Lit. \fbib{he}} replied.

\v{31}``Promise me,'' Israel\fnote{\fbackref{47:31} Lit. \fbib{he}} insisted. So Joseph promised. Then Israel collapsed\fnote{\fbackref{47:31} Lit. \fbib{Israel bent low}} on his bed.
\labelchapt{48}
\passage{Joseph Visits His Ill Father}

\chapt{48}
\v{1}Some time later, somebody informed Joseph, ``Your father is ill!'' So he took his two sons Manasseh and Ephraim with him to visit Jacob.\fnote{\fbackref{48:1} The Heb. lacks \fbib{to visit Jacob}}

\v{2}As soon as Jacob was informed, ``Look! Your son Joseph has come to visit you,'' Israel rallied his strength and sat up in bed.

\v{3}Jacob reminded Joseph, ``God Almighty revealed himself to me at Luz in Canaan and blessed me. \v{4}He told me, `Pay attention! I'm going to make you fruitful and numerous. I'm going to build you into a vast nation of people and then I'll give this land to your descendants\fnote{\fbackref{48:4} Lit. \fbib{descendants who come after you}} for an eternal possession.' \v{5}You have two sons who were born to you in Egypt before I came to be with you, whom I now take as my own. Ephraim and Manasseh are mine, just as Reuben and Simeon are. \v{6}Your descendants\fnote{\fbackref{48:6} Lit. \fbib{descendants who come after you}} are to be reckoned as yours, but are to be referred to among the names of their brothers in their respective\fnote{\fbackref{48:6} The Heb. lacks \fbib{respective}} inheritances.

\v{7}``Now as for me, Rachel died after I arrived in Canaan from Paddan, much to my sorrow. While I was on my journey to Ephrathah (also known as Bethlehem), I buried her there.''
\passage{Joseph Seeks Blessings for His Sons}

\v{8}Just then, Israel saw Joseph's sons and asked, ``Who are these?''

\v{9}``These are my sons,'' Joseph replied.\fnote{\fbackref{48:9} Lit. \fbib{replied to his father}} ``God gave them to me here in Egypt.''\fnote{\fbackref{48:9} The Heb. lacks \fbib{in Egypt}}

``Please bring them close to me,'' Jacob\fnote{\fbackref{48:9} Lit. \fbib{he}} said, ``so I can bless them.''

\v{10}Now Israel's eyesight had become poor\fnote{\fbackref{48:10} Lit. \fbib{dim}} from age. Because he couldn't see well, Joseph brought them close to him, and Israel\fnote{\fbackref{48:10} Lit. \fbib{he}} kissed them both and embraced them. \v{11}Then he told Joseph, ``I never thought I'd see you again, and now God has allowed me to see your children as well!''

\v{12}Joseph took them off his knees and then bowed low with his face to the ground. \v{13}Then he brought them both close to his father,\fnote{\fbackref{48:13} The Heb. lacks \fbib{to his father}} placing Ephraim with his right hand toward Israel's left and Manasseh with his left hand toward Israel's right. \v{14}But Israel stretched out his right hand, laying it on Ephraim's head (he was the younger son) and laying his left hand on Manasseh's head (even though Manasseh was the firstborn).
\passage{Israel Blesses Joseph's Sons}

\v{15}Then Israel blessed Joseph by saying:

\begin{poetry}
\poeml ``May the God in whose presence \\
\poemll    my ancestors Abraham and Isaac walked, \\
\poeml the God who has continued shepherding me \\
\poemll    my whole life even until today, \\
\poeml \v{16}the angel who has been rescuing\fnote{\fbackref{48:16} Or \fbib{redeeming}} me \\
\poemll    from all sorts of evil, \\
\poemlll       bless these young men. \\
\poeml May my name continue to live on within them, \\
\poemll    including the names \\
\poemlll       of my ancestors Abraham and Isaac, \\
\poeml and may they grow into a vast multitude \\
\poemll    throughout the earth.''
\end{poetry}

\v{17}But Joseph observed that his father had laid his right hand on Ephraim's head. That displeased him, so he grabbed his father's hand and started to move it from Ephraim's head to Manasseh's head. \v{18}``No, father, this one is the firstborn. Place your right hand on his head.''

\v{19}But his father refused. ``I know,'' he said. ``I know. He's going to produce a large nation, and he's going to be very great. However, his younger brother will become even greater than he, and his descendants will become a multitude of nations.''

\v{20}That very day, Jacob\fnote{\fbackref{48:20} Lit. \fbib{he}} blessed them with this blessing:\fnote{\fbackref{48:20} The Heb. lacks \fbib{with this blessing}}

\begin{poetry}
\poeml ``By you Israel will extend this blessing: \\
\poemll    `May God make you like Ephraim and Manasseh!'\,''
\end{poetry}

By doing this, he placed Ephraim before Manasseh. \v{21}Then Israel told Joseph, ``Pay attention! I'm about to die, but God will be with you. He'll bring you back to the land that belongs to your ancestors. \v{22}I'm assigning you one portion more than your brothers from the land that I confiscated from the control\fnote{\fbackref{48:22} Lit. \fbib{hand}} of the Amorites in battle.''\fnote{\fbackref{48:22}Lit. \fbib{Amorites with my sword and my bow}}
\labelchapt{49}
\passage{Jacob's Final Blessings}

\chapt{49}
\v{1}After this, Jacob called his sons together and told them, ``Assemble yourselves around me\fnote{\fbackref{49:1} The Heb. lacks \fbib{around me}} so I can tell you all what is going to happen to you in the last days.\fnote{\fbackref{49:1} Or \fbib{in days to come}}

\begin{poetry}
\poeml \v{2}``Gather together and listen, \\
\poemll    you children of Jacob. \\
\poemlll       Listen to your father Israel.''
\passage{On the Future of Reuben}
\poeml \v{3}``Reuben, you're my firstborn, \\
\poemll    my strength, \\
\poemlll       and the first fruit of my vitality. \\
\poeml You excel in rank \\
\poemll    and excel in power. \\
\poeml \v{4}But you're as undisciplined as a roaring river, \\
\poemll    so eventually you won't succeed, \\
\poeml because you got in your father's bed,\fnote{\fbackref{49:4} Cf. Gen 35:22} \\
\poemll    defiled it, and then approached my couch.''
\passage{On the Future of Simeon and Levi}
\poeml \v{5}``Simeon and Levi are brothers; \\
\poemll    their swords are violent weapons. \\
\poeml \v{6}I'll\fnote{\fbackref{49:6} Lit. \fbib{Let my soul}} never join their council; \\
\poemll    I'll never enter their assembly. \\
\poeml In their anger they committed murder \\
\poemll    and lamed cattle just for fun. \\
\poeml \v{7}Their anger is cursed, \\
\poemll    because it is so fierce, \\
\poeml as is their vehemence, \\
\poemll    because it is so cruel. \\
\poeml I will separate them throughout Jacob's territory\fnote{\fbackref{49:7} The Heb. lacks `\fbib{s territory}} \\
\poemll    and disperse them throughout Israel.''
\passage{On the Future of Judah}
\poeml \v{8}``Your brothers will praise you, Judah.\fnote{\fbackref{49:8} The Heb. verb \fbib{praise} is a word play on the name \fbib{Judah}} \\
\poemll    Your hand will be at the throat of your enemies, \\
\poeml and your father's children will bow down to you. \\
\poeml \v{9}Judah is a lion cub. \\
\poemll    My son, you have gone up from the prey. \\
\poeml Crouching like a lion, \\
\poemll    he lies down, \\
\poeml Like a lioness, \\
\poemll    who would dare rouse him? \\
\poeml \v{10}The scepter will never depart from Judah, \\
\poemll    nor a ruler's staff from between his feet, \\
\poeml until the One\fnote{\fbackref{49:10} Or \fbib{until Shiloh}} comes, who owns them both,\fnote{\fbackref{49:10} Lit. \fbib{comes to whom it belongs}; i.e. the authority represented by the scepter and ruler's staff} \\
\poemll    and to him will belong the allegiance\fnote{\fbackref{49:10} Or \fbib{obedience}} of nations. \\
\poeml \v{11}Binding his donkey to the vine \\
\poemll    and his mare's foal to its thick tendrils, \\
\poeml he will wash his garments in wine \\
\poemll    and his robe in the juice of grapes. \\
\poeml \v{12}His eyes are darker than wine \\
\poemll    and his teeth whiter than milk.''
\passage{On the Future of Zebulun}
\poeml \v{13}``Zebulun will settle down near the sea shore \\
\poemll    and become a safe haven for shipping, \\
\poemlll       bordering Sidon.''
\passage{On the Future of Issachar}
\poeml \v{14}``Issachar is a strong donkey, \\
\poemll    resting between sheepfolds. \\
\poeml \v{15}He observed that his resting place was excellent, \\
\poemll    and that the land was pleasant; \\
\poeml he bent down, \\
\poemll    picked up his burdens, \\
\poemlll       and became a slave at forced labor.''
\passage{On the Future of Dan}
\poeml \v{16}``Dan will judge\fnote{\fbackref{49:16} The Heb. name \fbib{Dan} means \fbib{judge}} his people \\
\poemll    as one of Israel's tribes. \\
\poeml \v{17}Dan will be a snake on the path, \\
\poemll    a viper on the road \\
\poeml that snaps at the heels of horses, \\
\poemll    causing their riders to fall off. \\
\poeml \v{18}``\divine{Lord}, I'm waiting for your salvation.''
\passage{On the Future of Gad}
\poeml \v{19}``Bandits will raid Gad, \\
\poemll    but Gad will raid them back.''\fnote{\fbackref{49:19} Lit. \fbib{raid the heel}}
\passage{On the Future of Asher}
\poeml \v{20}``Asher's food will be delicious; \\
\poemll    he will be a provider of delicacies fit for royalty.''
\passage{On the Future of Naphtali}
\poeml \v{21}``Naphtali is a free running deer \\
\poemll    who produces eloquent literature.''
\passage{On the Future of Joseph}
\poeml \v{22}``Joseph is descended from a fruitful vine, \\
\poemll    a fruitful vine planted near springs of water. \\
\poemlll       His branches climb over walls. \\
\poeml \v{23}Even though enemies\fnote{\fbackref{49:23} The Heb. lacks \fbib{enemies}} attacked him, \\
\poemll    shooting at him \\
\poemlll       and pursuing him viciously, \\
\poeml \v{24}nevertheless his bow remained steady \\
\poemll    and his arms kept in shape \\
\poemlll       by the strength of Jacob's Mighty One, \\
\poeml in the name of the Shepherd, \\
\poemll    Israel's Rock, \\
\poeml \v{25}by your father's God \\
\poemll    who helps you, \\
\poeml by the Almighty \\
\poemll    who will keep on blessing you \\
\poeml with blessings from heaven above, \\
\poemll    with blessings from the deepest ocean, \\
\poeml with blessing from the breasts and the womb. \\
\poeml \v{26}Your father's blessings will prove to be stronger \\
\poemll    than blessings from the eternal mountains \\
\poemlll       or bounties from the everlasting hills. \\
\poeml May they come to rest on Joseph's head, \\
\poemll    May they be set upon the brow of the one \\
\poemlll       who was separated from his own brothers.''
\passage{On the Future of Benjamin}
\poeml \v{27}``Benjamin is vicious like a wolf; \\
\poemll    what he kills in the morning \\
\poemlll       he devours in the evening.''
\end{poetry}
\passage{Jacob Dies and is Buried}

\v{28}That's how Israel blessed these\fnote{\fbackref{49:28} Lit. \fbib{All these are the}} twelve tribes of Israel, and this is what their father told them when he pronounced his blessing for them, blessing each one with a blessing suitable for them. \v{29}In his last words, Jacob\fnote{\fbackref{49:29} Lit. \fbib{he}} issued this set of instructions to them all: ``I'm about to join\fnote{\fbackref{49:29} Lit. \fbib{to be gathered to}} our ancestors. Bury me alongside my ancestors in the cave in the field that used to belong to Ephron the Hittite. \v{30}It's the cave in the field near Mamre at Machpelah in the land of Canaan that Abraham bought to serve as a cemetery. \v{31}It's where Abraham and his wife Sarah were buried, where Isaac and his wife Rebekah were buried, and where I buried Leah. \v{32}Both the field and the cave that's in it were purchased from the Hittites.''

\v{33}After concluding this set of instructions to his sons, Jacob\fnote{\fbackref{49:33} Lit. \fbib{he}} tucked his feet up into bed, quit breathing, and was gathered to his ancestors.
\labelchapt{50}
\passage{Joseph Mourns for His Father}

\chapt{50}
\v{1}Then Joseph embraced his father,\fnote{\fbackref{50:1} Lit. \fbib{Joseph fell on his father's face}} cried over him, and kissed him. \v{2}After this, he issued orders to his physician servants to embalm his father. So they embalmed Israel. \v{3}It took 40 days to complete the process, the normal period required for embalming. Meanwhile, the Egyptians mourned for him for 70 days. \v{4}At the conclusion of the mourning period, Joseph addressed Pharaoh's household. ``If you're satisfied with me, would you please take this message to Pharaoh for me? Tell him, \v{5}`My father told me, ``Look! I'm about to die. Bury me in my grave that I dug for myself in the land of Canaan.'' So please let me travel to bury my father. I'll be right back.'\,''

\v{6}``Please go,'' Pharaoh replied. ``Bury your father, as he asked you to do.''
\passage{Joseph Mourns in Canaan}

\v{7}So Joseph got up and went to bury his father, accompanied by all of Pharaoh's servants, all of the elders of Egypt, \v{8}all of Joseph's household, his brothers, and his father's household. They left behind in the territory of Goshen only their youngest children, their flocks, and their herds. \v{9}Chariots and horsemen also accompanied Joseph,\fnote{\fbackref{50:9} Lit. \fbib{him}} so there were a lot of people. \v{10}When they arrived at Atad's threshing floor, which is located beyond the Jordan River,\fnote{\fbackref{50:10} The Heb. lacks \fbib{River}} they held a great and mournful memorial service, during which Joseph\fnote{\fbackref{50:10} Lit. \fbib{he}} spent seven days mourning for his father. \v{11}As soon as the Canaanites who lived in the land observed the mourning going on at Atad's threshing floor, they commented ``This is a significant time of mourning for the Egyptians.'' That's why the place, which is located beyond the Jordan River,\fnote{\fbackref{50:11} The Heb. lacks \fbib{River}} became known as Abel-mizraim.\fnote{\fbackref{50:11} The Heb. name \fbib{Abel-mizraim} means \fbib{Mourning of the Egyptians}}
\passage{The Burial at Machpelah}

\v{12}And so Israel's\fnote{\fbackref{50:12} Lit. \fbib{so his}} sons did what he had instructed them to do: \v{13}they carried him to the territory of Canaan and buried him in the cave in Machpelah field near Mamre that Abraham had purchased\fnote{\fbackref{50:13} Lit. \fbib{purchased along with the field}} as a cemetery from Ephron the Hittite. \v{14}After he had buried his father, Joseph and his brothers returned to Egypt, along with everyone who had gone with him to attend the burial.

\v{15}Later, after Joseph's brothers faced the reality of their father's death, they asked themselves, ``What happens if Joseph decides to hold a grudge against us? What if he pays us back in full for all the wrong things we did to him?''

\v{16}So they sent this message to Joseph: \v{17}``Before he died, your father left some instructions. He told us, `Tell Joseph, ``Please forgive your brothers' offenses. I beg you, forgive their sins, because they wronged you.''\,' So please forgive the transgression of the servants of your father's God.''

Joseph wept when they talked to him. \v{18}So Joseph's\fnote{\fbackref{50:18} Lit. \fbib{his}} brothers went to visit him, fell prostrate in front of him, and declared, ``Look! We're your servants.''

\v{19}``Don't be afraid,'' Joseph responded. ``Am I sitting in God's place? \v{20}As far as you're concerned, you were planning evil against me, but God intended it for good, planning to bring about the present result so that many people would be preserved alive. \v{21}So don't be afraid! I'll take care of you and your little ones.'' So Joseph\fnote{\fbackref{50:21} Lit. \fbib{he}} kept on comforting them, speaking to the needs of\fnote{\fbackref{50:21} The Heb. lacks \fbib{the needs of}} their hearts.
\passage{Joseph's Death and Burial}

\v{22}Joseph continued to live in Egypt, along with his father's household, until he was 110 years old. \v{23}Joseph saw the third generation of Ephraim's children, as well as the children who had been born to Manasseh's son Machir, whom he adopted as his own.\fnote{\fbackref{50:23} Lit. \fbib{Machir, who were born on Joseph's knees}; i.e. they were placed in a special position of inheritance rights} \v{24}Later, Joseph told his brothers, ``I'm going to die soon, but God will certainly provide for you and bring you up from this land to the land that he promised with an oath to give\fnote{\fbackref{50:24} The Heb. lacks \fbib{to give}} to Abraham, Isaac, and Jacob.'' \v{25}So Joseph made all of Israel's other\fnote{\fbackref{50:25} The Heb. lacks \fbib{other}} children make this promise: ``Because God is certainly going to take care of you, you are to carry my bones up from here.''

\v{26}Some time later, Joseph died at the age of 110 years, and he was embalmed and placed in a coffin in Egypt.

\bookheader{Exodus}
\labelbook{Exod}

\bookpretitle{The Second Book of the Law called}
\booktitle{Exodus}

\labelchapt{1}
\passage{The Israelis Prosper in Egypt}

\chapt{1}
\v{1}These are the names of the Israelis\fnote{\fbackref{1:1} Lit. \fbib{the sons of Israel} and so throughout the book} who entered Egypt with Jacob, each one having come with his family:\fnote{\fbackref{1:1} Or \fbib{household}} \v{2}Reuben, Simeon, Levi, Judah, \v{3}Issacar, Zebulun, Benjamin, \v{4}Dan, Naphtali, Gad, and Asher. \v{5}All those who descended from\fnote{\fbackref{1:5} Lit. \fbib{came out of the loins of}} Jacob totaled 75 persons.\fnote{\fbackref{1:5} So with DSS and LXX. MT reads \fbib{70}} Now Joseph was already\fnote{\fbackref{1:5} The Heb. lacks \fbib{already}} in Egypt. \v{6}Then Joseph, all his brothers, and that entire generation died. \v{7}But the Israelis were fruitful and increased abundantly.\fnote{\fbackref{1:7} Lit. \fbib{swarmed}} They multiplied in numbers and became very, very strong. As a result, the land was filled with them.
\passage{The Israelis Become Slaves}

\v{8}Eventually a new king who was unacquainted with Joseph came to power in\fnote{\fbackref{1:8} Lit. \fbib{arose over}} Egypt. \v{9}He told his people, ``Look, the Israeli people are more numerous and more powerful than we are. \v{10}Come on, let's be careful how we treat them, so that when they grow numerous, if a war breaks out they won't join our enemies, fight against us, and leave our land.'' \v{11}So the Egyptians\fnote{\fbackref{1:11} Lit. \fbib{they}} placed supervisors over them, oppressing them with heavy burdens. The Israelis\fnote{\fbackref{1:11} Lit. \fbib{They}} built the supply cities of Pithom and Rameses for Pharaoh. \v{12}But the more the Egyptians afflicted the Israelis,\fnote{\fbackref{1:12} Lit. \fbib{them}} the more they multiplied and flourished, so that the Egyptians\fnote{\fbackref{1:12} Lit. \fbib{they}} became terrified of\fnote{\fbackref{1:12} Or \fbib{came to loathe}} the Israelis. \v{13}The Egyptians ruthlessly forced the Israelis to serve them, \v{14}making their lives bitter through hard labor with mortar, bricks, and all kinds of outdoor labor. They ruthlessly imposed all this\fnote{\fbackref{1:14} Lit. \fbib{their}} work on them.
\passage{Pharaoh Orders Male Children Killed}

\v{15}Later, the king of Egypt spoke to the Hebrew midwives, one of whom was named Shiphrah and the other Puah. \v{16}``When you help the Hebrew women give birth,'' he said, ``watch them as they deliver.\fnote{\fbackref{1:16} Lit. \fbib{them on the birth stool}} If it's a son, kill him; but if it's a daughter, let her live.'' \v{17}But the midwives feared God and didn't do what the king of Egypt told them. Instead,\fnote{\fbackref{1:17} The Heb. lacks \fbib{Instead}} they let the boys live.

\v{18}When the king of Egypt called for the midwives, he asked them, ``Why have you done this\fnote{\fbackref{1:18} Lit. \fbib{this thing}} and allowed the boys to live?''

\v{19}``Hebrew women aren't like Egyptian women,'' the midwives replied to Pharaoh. ``They're so healthy that they give birth before the midwives arrive to help\fnote{\fbackref{1:19} The Heb. lacks \fbib{to help}} them.''

\v{20}God was pleased with the midwives, and the people multiplied and became very strong. \v{21}Because the midwives feared God, he provided families\fnote{\fbackref{1:21} Or \fbib{households}; lit. \fbib{houses}} for them. \v{22}Meanwhile, Pharaoh continued commanding all of his people, ``You're to throw every Hebrew\fnote{\fbackref{1:22} The Heb. lacks \fbib{Hebrew}} son who is born into the Nile River,\fnote{\fbackref{1:22} The Heb. lacks \fbib{River}} but you're to allow every Hebrew\fnote{\fbackref{1:22} The Heb. lacks \fbib{Hebrew}} daughter to live.''
\labelchapt{2}
\passage{Moses is Born}

\chapt{2}
\v{1}A man of the family of Levi married the daughter of a descendant of Levi. \v{2}Later, the woman became pregnant and gave birth to a son. She saw that he was a beautiful\fnote{\fbackref{2:2} Or \fbib{good}} child, and hid him for three months. \v{3}But when she was no longer able to hide him, she took a papyrus container, coated it with asphalt and pitch, placed the child in it, and put it among the reeds along the bank of the Nile. \v{4}Then his sister positioned herself some distance away in order to find out what would happen to him.
\passage{Pharaoh's Daughter Adopts Moses}

\v{5}Then Pharaoh's daughter came down to the Nile River\fnote{\fbackref{2:5} The Heb. lacks \fbib{River}} to bathe while her maids walked along the river bank. She saw the container among the reeds and sent a servant girl to get it. \v{6}When she opened it and saw the child, the little boy suddenly began crying. Filled with compassion for him, she exclaimed, ``This is one of the Hebrew children!''

\v{7}Then his sister asked Pharaoh's daughter, ``Shall I go and call one of the nursing Hebrew women so she can nurse the child for you?''

\v{8}Pharaoh's daughter told her, ``Go,'' so the young girl went and called the child's mother. \v{9}Pharaoh's daughter instructed her, ``Take this child and nurse him for me, and I'll pay you a salary.'' So the woman took the child and nursed him. \v{10}After the child had grown older,\fnote{\fbackref{2:10} The Heb. lacks \fbib{older}} she brought him to Pharaoh's daughter, and he became her son. She named him Moses,\fnote{\fbackref{2:10} The Heb. name \fbib{Moses} sounds like the Heb. verb \fbib{draw out}} because she said, ``I drew him out of the water.''
\passage{Moses Kills an Egyptian}

\v{11}Years later, after\fnote{\fbackref{2:11} Lit. \fbib{It happened in those days that}} Moses had grown up, he went out to his own people,\fnote{\fbackref{2:11} Lit. \fbib{brothers}} and took notice of their heavy burdens. He saw an Egyptian beating up a Hebrew, one of his own people.\fnote{\fbackref{2:11} Lit. \fbib{brothers}} \v{12}Looking around and seeing no one else, he killed\fnote{\fbackref{2:12} Lit. \fbib{struck}} the Egyptian and hid him in the sand. \v{13}Going out the next day, Moses noticed\fnote{\fbackref{2:13} The Heb. lacks \fbib{noticed}} two Hebrew men fighting right in front of him. He told the one who was at fault, ``Why did you strike your companion?''

\v{14}The man\fnote{\fbackref{2:14} Lit. \fbib{He}} replied, ``Who appointed you to be an official judge over us? Are you planning\fnote{\fbackref{2:14} Lit. \fbib{saying}} to kill me like you killed the Egyptian?''

Then Moses became terrified and told himself,\fnote{\fbackref{2:14} The Heb. lacks \fbib{to himself}} ``Certainly this event has become known!''
\passage{Moses Flees to Midian}

\v{15}When Pharaoh heard about this matter, he tried to kill Moses. So Moses fled from Pharaoh, settled in the land of Midian, and sat down by a well. \v{16}Meanwhile, the seven daughters of a certain Midianite priest would come to draw water in order to fill water troughs for their father's sheep. \v{17}Some shepherds came to drive them away, but Moses got up, came to their rescue, and watered their sheep. \v{18}When they returned to their father Reuel,\fnote{\fbackref{2:18} I.e. another name for Jethro} he asked, ``Why have you returned so quickly today?''

\v{19}``An Egyptian rescued us from the shepherds,''\fnote{\fbackref{2:19} Lit. \fbib{the hand of the shepherds}} they replied, ``and he even drew water for us and watered the sheep!''

\v{20}``Then where is he?'' He asked his daughters. ``Why did you leave the man behind? Go invite him to have something to eat.''\fnote{\fbackref{2:20} Lit. \fbib{to eat bread}}

\v{21}Moses agreed to stay with the man, and he gave his daughter Zipporah to Moses in marriage.\fnote{\fbackref{2:21} The Heb. lacks \fbib{in marriage}} \v{22}Later she gave birth to a son, and Moses\fnote{\fbackref{2:22} Lit. \fbib{he}} named him Gershom,\fnote{\fbackref{2:22} Gershom sounds like Heb. for \fbib{alien}} because he used to say, ``I became an alien in a foreign land.''
\passage{The Israelis Cry Out to God}

\v{23}The king of Egypt eventually\fnote{\fbackref{2:23} Lit. \fbib{It happened after those many days that the king of Egypt}} died, and the Israelis groaned because of the bondage. They cried out, and their cry for deliverance from slavery ascended to God. \v{24}God heard their groaning and remembered his covenant with Abraham, Isaac, and Jacob. \v{25}God watched the Israelis and took notice of them.
\labelchapt{3}
\passage{God Calls Moses}

\chapt{3}
\v{1}Meanwhile, Moses continued tending the sheep that belonged to his father-in-law Jethro, the priest of Midian. He led the sheep to the western\fnote{\fbackref{3:1} Or \fbib{the back part of the}} desert and came to Horeb, God's mountain, where\fnote{\fbackref{3:1} The Heb. lacks \fbib{where}} \v{2}the angel of the \divine{Lord} appeared to him in flaming fire from the center of a bush. As Moses\fnote{\fbackref{3:2} Lit. \fbib{He}} continued to watch, amazingly the bush kept on burning, but was not consumed. \v{3}Then Moses told himself,\fnote{\fbackref{3:3} The Heb. lacks \fbib{to himself}} ``I'll go over and see this remarkable\fnote{\fbackref{3:3} Or \fbib{great}} sight. Why isn't the bush burning up?''

\v{4}When the \divine{Lord} saw that he had gone over to look, God called to him from the center of the bush, ``Moses! Moses!''

He said, ``Here I am.''

\v{5}``Do not come any closer,'' God\fnote{\fbackref{3:5} Lit. \fbib{he}} said. ``Remove your sandals from your feet, because the place where you are standing is holy ground.'' \v{6}Then he said, ``I am the God of your ancestors, the God of Abraham, the God of Isaac, and the God of Jacob.'' At this, Moses hid his face, because he was afraid to look at God.

\v{7}The \divine{Lord} said, ``I have certainly seen the affliction of my people who are in Egypt, and I have heard their cry caused by their slave masters. I really do understand their pain, \v{8}so I have come down to deliver them from their domination by\fnote{\fbackref{3:8} Lit. \fbib{from the hand of}} the Egyptians and to bring them out of that land to a good and spacious land, a land flowing with milk and honey, to the territory\fnote{\fbackref{3:8} Lit. \fbib{place}} of the Canaanites, the Hittites, the Amorites, the Perizzites, the Hivites, and the Jebusites. \v{9}Now, listen carefully! The cry of the Israelis has come to my attention about how severely the Egyptians have been oppressing them. \v{10}So go! I am sending you to Pharaoh. Bring my people the Israelis out of Egypt.''

\v{11}But Moses told God, ``Who am I? How can I go to Pharaoh and bring the Israelis out of Egypt?''

\v{12}Then God\fnote{\fbackref{3:12} Lit. \fbib{he}} said, ``I certainly will be with you. And this will be the sign for you that it is I who sent you: When you have brought the people out of Egypt, all of you will serve God on this mountain.''

\v{13}Moses told God, ``Look! When I go to the Israelis and tell them, `The God of your ancestors sent me to you,' they'll say to me, `What is his name?' What should I say to them?''

\v{14}God replied to Moses, ``I AM WHO I AM,''\fnote{\fbackref{3:14} Or \fbib{I WILL BE WHO I WILL BE} or \fbib{I AM THE ONE WHO IS}} and then said, ``Tell the Israelis: `I AM sent me to you.'\,''

\v{15}God also told Moses, ``Tell the Israelis, `The \divine{Lord}, the God of your ancestors, the God of Abraham, the God of Isaac, and the God of Jacob sent me to you.' This is my name forever, and this is how I am to be remembered from generation to generation.

\v{16}``Go and gather the elders of Israel. Tell them, `The \divine{Lord} God of your ancestors, appeared to me---the God of Abraham, Isaac, and Jacob---and he said, ``I have paid close attention to you and to what has been done to you in Egypt. \v{17}I have said that I will bring you out of the affliction of Egypt to the land of the Canaanites, the Hittites, the Amorites, the Perizzites, the Hivites, and the Jebusites---to a land flowing with milk and honey.''\,'

\v{18}``The elders of Israel\fnote{\fbackref{3:18} Lit. \fbib{They}} will listen to you,\fnote{\fbackref{3:18} Lit. \fbib{to your voice}} and then you and they\fnote{\fbackref{3:18} Lit. \fbib{and the elders of Israel}} are to go to the king of Egypt and say to him, `The \divine{Lord} God of the Hebrews has met with us. Now, let us take a three-day journey into the desert to sacrifice to the \divine{Lord} our God.' \v{19}I know that the king of Egypt won't allow you to go unless compelled to do so by force,\fnote{\fbackref{3:19} Lit. \fbib{with a strong hand}} \v{20}so I will stretch out my hand and strike Egypt with all my wonders that I will do there. After that he will release you. \v{21}I will grant this people public favor with the Egyptians, and as a result, when you leave you won't go empty-handed. \v{22}Each woman is to ask her neighbor or any foreign\fnote{\fbackref{3:22} Lit. \fbib{resident alien}} woman in her house for articles of gold and for clothing, and use them to clothe your sons and daughters. You will plunder the Egyptians.''
\labelchapt{4}
\passage{Moses Argues with God}

\chapt{4}
\v{1}Then Moses answered, ``Look, they won't believe me and they won't listen to me.\fnote{\fbackref{4:1} Lit. \fbib{to my voice.} And so through the passage} Instead, they'll say, `The \divine{Lord} didn't appear to you.'\,''

\v{2}``What's that in your hand?'' the \divine{Lord} asked him.

Moses\fnote{\fbackref{4:2} Lit. \fbib{he}} answered, ``A staff.''\fnote{\fbackref{4:2} Or \fbib{rod}}

\v{3}Then God\fnote{\fbackref{4:3} Lit. \fbib{he}} said, ``Throw it to the ground.'' He threw it to the ground and it became a snake. Moses ran away from it.

\v{4}Then God told Moses, ``Reach out\fnote{\fbackref{4:4} Lit. \fbib{Stretch out your hand}} and grab its tail.'' So he reached out, grabbed it, and it became a staff\fnote{\fbackref{4:4} Or \fbib{rod}} in his hand. \v{5}God said, ``I've done this\fnote{\fbackref{4:5} The Heb. lacks \fbib{God said, ``I have done this}} so that they may believe that the \divine{Lord} God of their ancestors---the God of Abraham, the God of Isaac, and the God of Jacob---has appeared to you.''

\v{6}Again the \divine{Lord} told him, ``Put your hand into your bosom.''\fnote{\fbackref{4:6} I.e. under the folds of the garment at the chest} He put his hand into his bosom and as soon as he brought it out it was leprous, like snow.\fnote{\fbackref{4:6} I.e. his hand was white} \v{7}Then God\fnote{\fbackref{4:7} Lit. \fbib{He}} said, ``Put your hand back into your bosom.'' He returned it\fnote{\fbackref{4:7} Lit. \fbib{his hand}} to his bosom and as soon as he brought it out,\fnote{\fbackref{4:7} Lit. \fbib{out from his bosom}} it was restored like the rest of\fnote{\fbackref{4:7} The Heb. lacks \fbib{the rest of}} his skin.\fnote{\fbackref{4:7} Lit. \fbib{flesh}}

\v{8}``Then if they don't believe you and respond to the first sign, they may respond to the second\fnote{\fbackref{4:8} Lit. \fbib{latter}} sign. \v{9}But if they don't believe even these two signs, and won't listen to you, then take some water out of the Nile River\fnote{\fbackref{4:9} The Heb. lacks \fbib{River}} and pour it on the dry ground. The water you took from the Nile River\fnote{\fbackref{4:9} The Heb. lacks \fbib{River}} will turn into blood on the dry ground.''

\v{10}Then Moses told the \divine{Lord}, ``Please, \divine{Lord}, I'm not eloquent.\fnote{\fbackref{4:10} Lit. \fbib{a man of words}} I never was in the past\fnote{\fbackref{4:10} Lit. \fbib{either yesterday or the day before}} nor am I now since you spoke to your servant. In fact, I talk too slowly\fnote{\fbackref{4:10} Lit. \fbib{heavy of mouth}} and I have a speech impediment.''\fnote{\fbackref{4:10} Lit. \fbib{heavy}}

\v{11}Then God asked him, ``Who gives a person a mouth? Who makes him unable to speak, or deaf, or able to see, or blind, or lame? Is it not I, the \divine{Lord}? \v{12}Now, go! I myself will help you with your speech,\fnote{\fbackref{4:12} Lit. \fbib{will be with your mouth}} and I'll teach you what you are to say.''

\v{13}Moses said, ``Please, \divine{Lord}, send somebody else.''\fnote{\fbackref{4:13} Lit. \fbib{by a hand send}; i.e. \fbib{by someone else's hand send}}

\v{14}Then the \divine{Lord} was angry with Moses and said, ``There is your brother Aaron, a descendant of Levi, isn't there? I know that he certainly is eloquent.\fnote{\fbackref{4:14} Lit. \fbib{he certainly speaks}} Right now he's coming to meet you and he will be pleased to see you. \v{15}You're to speak to him and tell him what to say.\fnote{\fbackref{4:15} Lit. \fbib{put the words in his mouth}} I'll help both you and him with your speech,\fnote{\fbackref{4:15} Lit. \fbib{I'll be with your mouth and with his mouth}} and I'll teach both of you what you are to do. \v{16}He is to speak to the people for you as your spokesman\fnote{\fbackref{4:16} Lit. \fbib{be your mouth}} and you are to act in the role of\fnote{\fbackref{4:16} Lit. \fbib{be}} God for him. \v{17}Now pick up that staff with your hand. You'll use it to perform the signs.''
\passage{Moses Decides to Return to Egypt}

\v{18}Moses left and returned to his father-in-law Jethro. Moses\fnote{\fbackref{4:18} Lit. \fbib{he}} told him, ``Please let me go and return to my own people\fnote{\fbackref{4:18} Lit. \fbib{my brothers}} in Egypt so I can see whether they're still alive.''

Jethro told Moses, ``Go in peace.''

\v{19}The \divine{Lord} told Moses in Midian, ``Go back to Egypt, because all the men who wanted to kill you are dead.'' \v{20}So Moses took his wife and son, put them on donkeys, and headed back to the land of Egypt. Moses took the staff of God in his hand.

\v{21}Then the \divine{Lord} told Moses, ``When you set out to return to Egypt, keep in mind\fnote{\fbackref{4:21} Lit. \fbib{see, watch}} all the wonders that I've put in your power,\fnote{\fbackref{4:21} Lit. \fbib{hand}} so that you may do them before Pharaoh. But I'll harden his heart so that he won't let the people go. \v{22}You are to say to Pharaoh, `This is what the \divine{Lord} says: ``Israel is my firstborn son. \v{23}And I say to you, `Let my son go so he may serve me. If you refuse to let him go, then I will kill your firstborn son.'\,''\,'\,''
\passage{Zipporah Circumcises Moses' Son}

\v{24}But later on, at the lodging place along the way, the \divine{Lord} met Moses\fnote{\fbackref{4:24} Lit. \fbib{him}} and was about to kill him. \v{25}Zipporah took a flint knife, cut off her son's foreskin, and touched Moses'\fnote{\fbackref{4:25} Lit. \fbib{his}} feet with it, saying while doing so,\fnote{\fbackref{4:25} Lit. \fbib{touched to his feet}} ``{\ldots}because you are a bridegroom of blood to me.'' \v{26}Then the \divine{Lord}\fnote{\fbackref{4:26} Lit. \fbib{Then he}} withdrew from him, and she said, ``{\ldots}a bridegroom of blood because of circumcision.''
\passage{Moses and Aaron Meet and Return to Egypt}

\v{27}The \divine{Lord} told Aaron, ``Go meet Moses in the desert.'' So Aaron\fnote{\fbackref{4:27} Lit. \fbib{he}} went, found\fnote{\fbackref{4:27} Lit. \fbib{encountered}} him at the mountain of God, and embraced\fnote{\fbackref{4:27} Lit. \fbib{kissed}} him. \v{28}Moses told Aaron all of the \divine{Lord}'s messages that he had sent with Moses, and all of the signs that he commanded him to do.\fnote{\fbackref{4:28} The Heb. lacks \fbib{to do}} \v{29}Later, Moses and Aaron brought together all the elders of the Israelis. \v{30}Aaron spoke everything that the \divine{Lord} had spoken to Moses, and Moses\fnote{\fbackref{4:30} Lit. \fbib{he}} performed the miracles\fnote{\fbackref{4:30} Lit. \fbib{signs}} before the very eyes of the people. \v{31}The people believed and understood\fnote{\fbackref{4:31} Or \fbib{they heard}} that the \divine{Lord} had paid attention to the Israelis and had seen their affliction. They bowed their heads and prostrated themselves in worship.
\labelchapt{5}
\passage{Pharaoh Refuses to Let the People Go}

\chapt{5}
\v{1}After Moses and Aaron arrived, they told Pharaoh, ``This is what the \divine{Lord} God of Israel says: `Let my people go so they may make a pilgrimage for me in the desert.'\,''

\v{2}Pharaoh said, ``Who is the \divine{Lord} that I should listen to\fnote{\fbackref{5:2} Or \fbib{obey}} him and let Israel go? I don't know about\fnote{\fbackref{5:2} The Heb. lacks \fbib{about}} the \divine{Lord}, nor will I let Israel go!''

\v{3}Then they said, ``The God of the Hebrews has met with us. Please let us go a three-day journey into the desert to offer sacrifices to the \divine{Lord} our God so he does not strike us with pestilence or sword.''\fnote{\fbackref{5:3} I.e. invasions by foreign armies}

\v{4}The king of Egypt replied to them, ``Moses and Aaron, why are you keeping the people from their labor? Go back to your work!''\fnote{\fbackref{5:4} Lit. \fbib{burdens}} \v{5}Then Pharaoh said, ``Look, the people in the land are now numerous, and you are stopping them from working.''\fnote{\fbackref{5:5} Lit. \fbib{from their burdens}}
\passage{Pharaoh Increases the Israelis' Work}

\v{6}That day Pharaoh ordered the taskmasters of the people and their officials, \v{7}``You're no longer to give the people straw for making bricks, as in the past.\fnote{\fbackref{5:7} Lit. \fbib{like yesterday and the day before}} They must gather straw for themselves. \v{8}But you're to impose the previous quota\fnote{\fbackref{5:8} Lit. \fbib{as yesterday and the day before}} of bricks that they're making. You're not to reduce it! It is because they're lazy that they're crying out, `Let's go offer sacrifices to our God.' \v{9}So increase the work load on the people,\fnote{\fbackref{5:9} Or \fbib{men}} and let them do it so they don't pay attention to deceptive speeches.''

\v{10}Then the taskmasters of the people and their officials went out and told the people, ``This is Pharaoh's response: `I'll no longer give you any\fnote{\fbackref{5:10} The Heb. lacks \fbib{any}} straw. \v{11}Go get straw for yourselves wherever you can find it, but your work quotas won't be reduced at all.'\,''\fnote{\fbackref{5:11} Lit. \fbib{from your labor}} \v{12}So the people scattered throughout the entire land of Egypt to collect stubble\fnote{\fbackref{5:12} I.e. the stalks left in the field after grain is harvested} for straw.

\v{13}The taskmasters pressured them by saying, ``Finish your work---each day's quota\fnote{\fbackref{5:13} Lit. \fbib{matter}}---just as when you were given straw.''\fnote{\fbackref{5:13} Lit. \fbib{when there was straw.}}

\v{14}The Israeli supervisors whom Pharaoh's taskmasters had appointed over them were beaten and told,\fnote{\fbackref{5:14} Lit. \fbib{saying.}} ``Why didn't you, both yesterday and today, fulfill\fnote{\fbackref{5:14} Lit. \fbib{complete}} your quota\fnote{\fbackref{5:14} Lit. \fbib{prescribed amount}} for making bricks as before?''
\passage{The Israelis' Appeal Rejected by Pharaoh}

\v{15}The Israeli supervisors came and cried out to Pharaoh, ``Why are you doing this to us?\fnote{\fbackref{5:15} Lit. \fbib{your servants}; and so throughout the book} \v{16}No straw is being given to us, yet they're saying to us, `Make bricks!' Look, we are being beaten. It's wrong how you are treating your people!''

\v{17}Then Pharaoh\fnote{\fbackref{5:17} Lit. \fbib{he}} said, ``You are lazy, lazy! That's why\fnote{\fbackref{5:17} Lit. \fbib{therefore}} you're saying, `Let's go offer sacrifices to the \divine{Lord}.' \v{18}Now, go! Get to work! And straw won't be given to you, but you are to deliver the same\fnote{\fbackref{5:18} The Heb. lacks \fbib{same}} number of bricks!'' \v{19}The Israeli supervisors realized they were in trouble when he said,\fnote{\fbackref{5:19} Lit. \fbib{saying}} ``You won't reduce each day's quota of bricks!''\fnote{\fbackref{5:19} Lit. \fbib{your bricks}}
\passage{The Israelis Blame Moses and Moses Complains to God}

\v{20}As they left Pharaoh's presence,\fnote{\fbackref{5:20} Lit. \fbib{from with}} they met Moses and Aaron standing there.\fnote{\fbackref{5:20} The Heb. lacks \fbib{there}} \v{21}The supervisors\fnote{\fbackref{5:21} Lit. \fbib{they}} told them, ``May the \divine{Lord} look on you and judge you!\fnote{\fbackref{5:21} The Heb. lacks \fbib{you}} You have made us repulsive to\fnote{\fbackref{5:21} Lit. \fbib{made our odor stink in the eyes of}} Pharaoh and his servants. You have put\fnote{\fbackref{5:21} Lit. \fbib{servants to give}} a sword in their hands to kill us.''

\v{22}So Moses returned to the \divine{Lord} and asked him, ``\divine{Lord}, why have you caused trouble for this people? Why have you sent me here? \v{23}Ever since I came to Pharaoh to speak in your name, he has caused trouble for this people, and you have done nothing to deliver your people.''
\labelchapt{6}
\passage{God Promises to Deliver Israel}

\chapt{6}
\v{1}The \divine{Lord} told Moses, ``Now you're about to see what I'll do to Pharaoh. Indeed, he'll send them out under compulsion\fnote{\fbackref{6:1} Lit. \fbib{out by a strong hand}} and he'll drive them out of his land violently.''\fnote{\fbackref{6:1} Lit. \fbib{land by a strong hand}}

\v{2}Later, God told Moses, ``I am the \divine{Lord}. \v{3}I appeared to Abraham, to Isaac, and to Jacob as God Almighty,\fnote{\fbackref{6:3} Heb. \fbib{El Shaddai}} and did I not reveal to them my name `\divine{Lord}'? \v{4}I also established my covenant with them to give them the land of Canaan, the land where they lived as resident aliens for a time. \v{5}Also, I've heard the groaning of the Israelis whom the Egyptians have forced to labor for them, and I've remembered my covenant. \v{6}Therefore, tell the Israelis, `I am the \divine{Lord}. I'll bring you out from under the burdens of the Egyptians, and I'll deliver you from their bondage. I'll redeem you with an outstretched arm and with great acts of judgment.\fnote{\fbackref{6:6} Lit. \fbib{great judgments}} \v{7}I'll take you for my own people,\fnote{\fbackref{6:7} Lit. \fbib{for Myself for a people}} and I'll be your God. Then you will know that I am the \divine{Lord} your God, who brings you out from under the burdens of the Egyptians. \v{8}I'll bring you to the land that I swore\fnote{\fbackref{6:8} Lit. \fbib{I lifted my hand}} to give to Abraham, to Isaac, and to Jacob. I'll give it to you as a possession. I am the \divine{Lord}.'\,''

\v{9}Then Moses reported this to the Israelis, but they did not listen to Moses due to their irritation and impatience because there was no deliverance\fnote{\fbackref{6:9} Lit. \fbib{due to shortness of spirit}} and because of the cruel bondage.

\v{10}Then the \divine{Lord} told Moses, \v{11}``Go, speak to Pharaoh, king of Egypt, that he should let the Israelis go out of his land.''

\v{12}Then Moses said right in front of the \divine{Lord}, ``Look, the Israelis didn't listen to me, so how will Pharaoh? I'm not a persuasive speaker.''\fnote{\fbackref{6:12} Lit. \fbib{uncircumcised of lip}; i.e. an unrefined speaker} \v{13}Then the \divine{Lord} spoke to Moses and Aaron, issuing orders to them regarding the Israelis for delivery to Pharaoh, king of Egypt; that is, to bring the Israelis out of the land of Egypt.
\passage{Genealogies of Moses and Aaron}

\v{14}These are the heads of their ancestors' households: the sons of Reuben, the firstborn of Israel: Hanoch and Pallu; Hezron and Carmi.

These are the families of Reuben, including \v{15}Simeon's sons Jemuel, Jamin, Ohad, Jachin, Zohar, and Shaul, the Canaanite woman's son. These are the families of Simeon.

\v{16}These are the names of Levi's sons according to their genealogies: Gershon, Kohath, and Merari. Levi lived\fnote{\fbackref{6:16} Lit. \fbib{Now the years of Levi's life were}} 137 years. \v{17}Gershon's sons were Libni and Shimei, according to their families. \v{18}Kohath's descendants included Amram, Izhar, Hebron, and Uzziel. Now Kohath lived for 133 years. \v{19}The sons of Merari were Mahli and Mushi. These are the families of the descendants of Levi, according to their genealogies.

\v{20}Amram married Jochebed, his father's sister, and she bore him Aaron and Moses. Amram lived for 137 years. \v{21}The sons of Izhar were Korah, Nepheg, and Zichri. \v{22}The sons of Uzziel were Mishael, Elzaphan, and Sithri.

\v{23}Then Aaron married Elisheba daughter of Amminadab, sister of Nahshon. She bore him Nadab, Abihu, Eleazar, and Ithamar. \v{24}The sons of Korah were Assir, Elkanah, and Abiasaph. These were the families of the descendants of Korah. \v{25}Aaron's son Eleazar married one of Putiel's daughters, and she bore him Phineas. These are the heads of the ancestors of the descendants of Levi, according to their families.

\v{26}This is the same Aaron and Moses to whom the \divine{Lord} said, ``Bring the Israelis out of the land of Egypt by their tribal divisions.'' \v{27}They were the ones speaking to Pharaoh, king of Egypt, to bring the Israelis out of Egypt; this is that same Moses and Aaron.
\passage{Moses Doubts that Pharaoh will Listen}

\v{28}And it happened when the \divine{Lord} spoke to Moses in the land of Egypt \v{29}that the \divine{Lord} told Moses, ``I am the \divine{Lord}. Tell Pharaoh, king of Egypt, everything that I'm saying to you.''

\v{30}Moses said in the presence of the \divine{Lord}, ``Look, I'm not a persuasive speaker,\fnote{\fbackref{6:30} Lit. \fbib{I'm uncircumcised of lips}} so how will Pharaoh listen to me?''
\labelchapt{7}
\passage{God Appoints Aaron to Assist Moses}

\chapt{7}
\v{1}The \divine{Lord} told Moses, ``Listen! I've positioned you as God\fnote{\fbackref{7:1} Or \fbib{as a god}} to Pharaoh, and your brother Aaron will be your prophet. \v{2}You are to speak everything that I've commanded you, and then your brother Aaron will speak to Pharaoh, telling him to let the Israelis go out of his land. \v{3}I'll harden Pharaoh's heart and I'll add more and more of my signs and wonders in the land of Egypt. \v{4}When Pharaoh won't listen to you, I'll let loose my power\fnote{\fbackref{7:4} Or \fbib{I'll put my hand}} upon Egypt. I'll bring out my tribal divisions---my people the Israelis---from the land of Egypt with great acts of judgment.\fnote{\fbackref{7:4} Lit. \fbib{great judgments}} \v{5}The Egyptians will know that I am the \divine{Lord} when I stretch out my hand over Egypt to bring the Israelis out from among them.'' \v{6}Moses and Aaron did what the \divine{Lord} commanded them. \v{7}Moses was 80 years old and Aaron was 83 when they spoke to Pharaoh.
\passage{Moses' Staff Becomes a Snake}

\v{8}Then the \divine{Lord} told Moses and Aaron, \v{9}``When Pharaoh says to you, `Perform a miraculous sign,' then you are to say to Aaron, `Take your staff and throw it in front of Pharaoh.' It will become a serpent.''

\v{10}So Moses and Aaron went in to Pharaoh and did what the \divine{Lord} had commanded them. Aaron threw his staff in front of Pharaoh and his officials, and it became a serpent. \v{11}Then Pharaoh also called for the wise men and sorcerers, and they---along with the Egyptian magicians---did the same thing with their secret arts. \v{12}So each one threw down his staff and it became a serpent, but Aaron's staff swallowed up their staffs. \v{13}Yet Pharaoh's heart was stubborn\fnote{\fbackref{7:13} Lit. \fbib{strong}} and he did not listen to them, just as the \divine{Lord} had said would happen.
\passage{Water is Turned into Blood}

\v{14}Then the \divine{Lord} told Moses, ``Pharaoh's heart is hard. He has refused to let the people go. \v{15}Go to Pharaoh in the morning as he's going down to the water. Stand on the bank of the Nile River\fnote{\fbackref{7:15} The Heb. lacks \fbib{River}} and meet him. Be sure to take with you\fnote{\fbackref{7:15} Lit. \fbib{in your hand}} the staff that was turned into a snake. \v{16}Then say to him, `The \divine{Lord} God of the Hebrews, has sent me to you. He says, ``Let my people go so they may serve\fnote{\fbackref{7:16} Or \fbib{worship}} me in the desert, but until now you haven't obeyed.''\,'\fnote{\fbackref{7:16} Or \fbib{listened}}

\v{17}```This is what the \divine{Lord} says: ``This is how you'll know that I am the \divine{Lord}: Right now I'm going to strike the water of the Nile River\fnote{\fbackref{7:17} The Heb. lacks \fbib{River}} with the staff that's in my hand, and it will be turned to blood. \v{18}The fish in the Nile River\fnote{\fbackref{7:18} The Heb. lacks \fbib{River}} will die and the river\fnote{\fbackref{7:18} Or \fbib{the Nile}} will stink. The Egyptians will be unable\fnote{\fbackref{7:18} Or \fbib{weary themselves}} to drink water from the Nile River.\fnote{\fbackref{7:18} The Heb. lacks \fbib{River}}''\,'\,''

\v{19}The \divine{Lord} also told Moses, ``Tell Aaron, `Take your staff and stretch out your hand over the waters of Egypt, over their rivers, over their Nile River\fnote{\fbackref{7:19} The Heb. lacks \fbib{River}}, over their ponds, and over their reservoirs,\fnote{\fbackref{7:19} Lit. \fbib{every collection of their waters}} and they'll become blood. There will be blood throughout the land of Egypt, even in their\fnote{\fbackref{7:19} The Heb. lacks \fbib{their}} wood and stone containers.'\,''\fnote{\fbackref{7:19} The Heb. lacks \fbib{containers}}

\v{20}Moses and Aaron did just what the \divine{Lord} had commanded. Aaron\fnote{\fbackref{7:20} Lit. \fbib{He}} raised his staff and struck the water in the Nile River\fnote{\fbackref{7:20} The Heb. lacks \fbib{River}} in front of\fnote{\fbackref{7:20} Lit. \fbib{before the eyes of}} Pharaoh and his\fnote{\fbackref{7:20} Lit. \fbib{before the eyes of his}} officials,\fnote{\fbackref{7:20} Or \fbib{servants}} and all the water in the Nile River\fnote{\fbackref{7:20} The Heb. lacks \fbib{River}} turned to blood. \v{21}The fish in the Nile River\fnote{\fbackref{7:21} The Heb. lacks \fbib{River}} died and the river\fnote{\fbackref{7:21} Or \fbib{the Nile}} stank. The Egyptians were not able to drink water from the Nile River,\fnote{\fbackref{7:21} The Heb. lacks \fbib{River}} and blood was throughout the land of Egypt. \v{22}But the Egyptian magicians did the same thing\fnote{\fbackref{7:22} Lit. \fbib{did thus}} with their secret arts. Pharaoh's heart was stubborn,\fnote{\fbackref{7:22} Lit. \fbib{strong}} and he did not listen to them, just as the \divine{Lord} had said. \v{23}Then Pharaoh turned away, went to his palace, and paid no attention to any of this. \v{24}All the Egyptians dug around the Nile River\fnote{\fbackref{7:24} The Heb. lacks \fbib{River}} for water to drink because they could not drink from the water in the Nile River.\fnote{\fbackref{7:24} The Heb. lacks \fbib{River}}
\labelchapt{8}
\passage{The Plague of Frogs}

\v{25}Seven days after\fnote{\fbackref{7:25} Lit. \fbib{days were filled after}} the \divine{Lord} had struck the Nile River,\fnote{\fbackref{7:25} The Heb. lacks \fbib{River}}\chapt{8}
\v{1}\fnote{\fbackref{8:1} This verse is 7:26 in MT}he told Moses, ``Go to Pharaoh and tell him, `This is what the \divine{Lord} says: ``Let my people go so they may serve\fnote{\fbackref{8:1} Or \fbib{worship}} me. \v{2}And if you refuse to let them go, then I'm going to strike all your territory with frogs. \v{3}The Nile will swarm with frogs. They'll come up and enter your house, your bedroom, your bed, and your servants' houses. They'll jump on your people, into your ovens, and into your kneading troughs. \v{4}The frogs will be all over you and your servants.''\,'\,''

\v{5}\fnote{\fbackref{8:5} This verse is 8:1 in MT}Then the \divine{Lord} told Moses, ``Tell Aaron, `Stretch out your hand with your staff over the rivers, over the Nile River,\fnote{\fbackref{8:5} The Heb. lacks \fbib{River}} and over the ponds, and bring up frogs over the land of Egypt.'\,'' \v{6}So Aaron stretched his hand over the waters of Egypt, and the frogs came up and covered the land of Egypt. \v{7}But the magicians did the same thing\fnote{\fbackref{8:7} Lit. \fbib{thus}} with their secret arts, and they brought up frogs on the land of Egypt.

\v{8}Then Pharaoh called to Moses and Aaron and said, ``Plead with the \divine{Lord} so that he may remove the frogs from me and my people. I'll let the people go so they can offer sacrifices to the \divine{Lord}.''

\v{9}Moses told Pharaoh, ``You decide\fnote{\fbackref{8:9} Lit. \fbib{you have honor over me.} i.e. I'll defer to your decision} when I should plead for you, your servants, and your people to remove\fnote{\fbackref{8:9} Lit. \fbib{cut off}} the frogs from you and your household. They'll remain only in the Nile River.\fnote{\fbackref{8:9} The Heb. lacks \fbib{River}}''

\v{10}Pharaoh\fnote{\fbackref{8:10} Lit. \fbib{he}} said, ``Tomorrow.''

Moses\fnote{\fbackref{8:10} Lit. \fbib{he}} said, ``It will be just as you say,\fnote{\fbackref{8:10} Lit. \fbib{according to your word}} so that you may know that there is no one like the \divine{Lord} our God. \v{11}The frogs will leave you, your house, your officials,\fnote{\fbackref{8:11} Or \fbib{servants}} and your people. They'll remain only in the Nile River.\fnote{\fbackref{8:11} The Heb. lacks \fbib{River}}''

\v{12}Then Moses and Aaron left Pharaoh's presence, and Moses cried out to the \divine{Lord} about the frogs which he had sent\fnote{\fbackref{8:12} Lit. \fbib{put}} on Pharaoh. \v{13}The \divine{Lord} did just as Moses asked,\fnote{\fbackref{8:13} Lit. \fbib{according to the word of}} and the frogs died in the houses, in the courtyards, and in the fields. \v{14}They gathered them up into large piles and the land smelled terrible. \v{15}But when Pharaoh saw that there was relief, he hardened his heart and did not listen to them, just as the \divine{Lord} had predicted.
\passage{The Plague of Gnats}

\v{16}Then the \divine{Lord} told Moses, ``Tell Aaron, `Stretch out your staff, strike the dust of the ground, and the dust\fnote{\fbackref{8:16} Lit. \fbib{it}} will become gnats throughout the land of Egypt.'\,'' \v{17}They did this.\fnote{\fbackref{8:17} Lit. \fbib{thus}} Aaron stretched his hand out with his staff, struck the dust of the land, and gnats came on people and animals---all the dust of the ground became gnats throughout the land of Egypt. \v{18}The magicians tried\fnote{\fbackref{8:18} Lit. \fbib{they did}} to do the same thing\fnote{\fbackref{8:18} Lit. \fbib{thus}} with their secret arts, but they were unable to bring out the gnats. The gnats were on the people and the animals.

\v{19}The magicians told Pharaoh, ``It is the finger of God!''\fnote{\fbackref{8:19} I.e. an act of God} But Pharaoh's heart was stubborn\fnote{\fbackref{8:19} Lit. \fbib{strong}} and he did not listen to them, just as the \divine{Lord} had predicted.

\v{20}The \divine{Lord} told Moses, ``Get up early in the morning and stand before Pharaoh as he's going down to the water. You are to say to him, `This is what the \divine{Lord} says: ``Let my people go so they can serve\fnote{\fbackref{8:20} Or \fbib{worship}} me. \v{21}But if you don't let my people go, I'll send swarms of insects upon you, your servants, your people, and your households. The houses of Egypt---and even the ground on which they stand---will be filled with swarms of insects. \v{22}On that day I'll treat the land of Goshen where my people live\fnote{\fbackref{8:22} Lit. \fbib{are standing}} differently so that swarms of insects won't be there. As a result, you will know that I the \divine{Lord} am in the midst of the land. \v{23}I'll make a distinction between my people and your people, and this sign will occur tomorrow.''\,'\,''

\v{24}The \divine{Lord} did this, and dense swarms of insects came into the house of Pharaoh and into the houses of his servants. The land was ruined throughout\fnote{\fbackref{8:24} The Heb. lacks \fbib{throughout}} Egypt because of the swarms of insects. \v{25}Then Pharaoh summoned Moses and Aaron and said, ``Go, offer sacrifices to your God in the land.''

\v{26}``It wouldn't be right to sacrifice in this way,''\fnote{\fbackref{8:26} Lit. \fbib{thus}} Moses replied, ``because if we do,\fnote{\fbackref{8:26} The Heb. lacks \fbib{if we do}} we will sacrifice to the \divine{Lord} our God what is offensive to the Egyptians.\fnote{\fbackref{8:26} Lit. \fbib{an abomination to the Egyptians}} If we offer sacrifices that are offensive to the Egyptians\fnote{\fbackref{8:26} Lit. \fbib{an abomination to the Egyptians}} in front of them, they'll stone us, won't they? \v{27}We must go a three-day journey into the desert, and we'll offer sacrifices to the \divine{Lord} our God just as he has told us.''

\v{28}Then Pharaoh said, ``I'll let you go so you can offer sacrifices to the \divine{Lord} your God in the desert. But you must not go very far away. Pray for me.''

\v{29}Moses said, ``Right now I'm going to leave you, and I'll pray to the \divine{Lord} that the swarms of insects may depart from Pharaoh, from his officials, and from his people tomorrow. But Pharaoh, don't continue lying by not letting the people go to offer sacrifices to the \divine{Lord}.''

\v{30}Then Moses left Pharaoh's presence and prayed to the \divine{Lord}. \v{31}The \divine{Lord} did what Moses asked,\fnote{\fbackref{8:31} Lit. \fbib{did according to the word of Moses}} and the swarms of insects departed from Pharaoh, his officials, and his people. Not one remained. \v{32}But this time also Pharaoh hardened his heart, and he did not let the people go.
\labelchapt{9}
\passage{The Plague on the Egyptian Cattle}

\chapt{9}
\v{1}Then the \divine{Lord} told Moses, ``Go to Pharaoh and say to him, `This is what the \divine{Lord} God of the Hebrews says: ``Let my people go so they may serve\fnote{\fbackref{9:1} Or \fbib{worship}} me. \v{2}But if you refuse to let them go and continue to hold them, \v{3}then the hand of the \divine{Lord} will come\fnote{\fbackref{9:3} Lit. \fbib{be}} with a very severe plague on your livestock in the fields, on horses, on donkeys, on camels, on cattle, and on sheep. \v{4}The \divine{Lord} will distinguish between the livestock of Israel and the livestock of the Egyptians, so that nothing that belongs to the Israelis will die.''\,'\,''

\v{5}The \divine{Lord} set the time: ``Tomorrow the \divine{Lord} will do this thing in the land.'' \v{6}The \divine{Lord} did this thing the next day, and all the livestock of the Egyptians died. But not one of the livestock died that belonged to the Israelis. \v{7}Then Pharaoh inquired and discovered\fnote{\fbackref{9:7} Lit. \fbib{sent and behold}} that not a single one of the livestock of Israel had died, but Pharaoh's heart was stubborn\fnote{\fbackref{9:7} Lit. \fbib{strong}} and he would not let the people go.
\passage{The Plague of Boils}

\v{8}Then the \divine{Lord} told Moses and Aaron, ``Take handfuls of soot from a kiln, and let Moses throw it into the air\fnote{\fbackref{9:8} Lit. \fbib{toward heaven}} in front of Pharaoh. \v{9}The soot\fnote{\fbackref{9:9} Lit. \fbib{it}} will become dust over the entire land of Egypt, and it will become boils erupting into sores on people and animals throughout the land of Egypt.''

\v{10}So they took soot from the kiln and stood before Pharaoh. Then Moses threw it into the air,\fnote{\fbackref{9:10} Lit. \fbib{toward heaven}} and it became boils producing running sores on people and animals. \v{11}The magicians were not able to stand before Moses because of the boils, because the boils were on the magicians and on all the Egyptians. \v{12}The \divine{Lord} made Pharaoh's heart stubborn\fnote{\fbackref{9:12} Lit. \fbib{strong}; i.e. determined} so that he would not listen to them, just as the \divine{Lord} had told Moses.
\passage{The Plague of Hail}

\v{13}Then the \divine{Lord} told Moses, ``Get up early in the morning, present yourself to Pharaoh, and say to him, `This is what the \divine{Lord} God of the Hebrews says: ``Let my people go so they may serve\fnote{\fbackref{9:13} Or \fbib{worship}} me. \v{14}Indeed, this time I'm sending all my plagues against you\fnote{\fbackref{9:14} Lit. \fbib{to your heart}}, your officials,\fnote{\fbackref{9:14} Or \fbib{servants}} and your people, so you may know that there is no one like me in all the earth. \v{15}Indeed, by now I could have sent forth my hand and struck you and your people with a plague, and you would have been destroyed from the earth. \v{16}However, I've kept you standing\fnote{\fbackref{9:16} Or \fbib{allowed you to live}; Lit. \fbib{caused you to stand}} in order to show you my power and to declare my name in all the earth. \v{17}You are still acting arrogantly against my people by not letting them go. \v{18}Look! About this time tomorrow, I'll send a severe hail storm, such as has not happened in Egypt from the day it was founded until now. \v{19}So send for your livestock and everything that belongs to you that's out in the field, because\fnote{\fbackref{9:19} Lit. \fbib{and}} every person and animal found in the field that has not been brought inside to shelters will die when the hail comes down on them.''\,'\,''

\v{20}Whoever feared the message from the \divine{Lord} among Pharaoh's officials\fnote{\fbackref{9:20} Or \fbib{servants}} made his servants and livestock flee into shelters. \v{21}But whoever did not pay attention\fnote{\fbackref{9:21} Lit. \fbib{set his heart}} to the message from the \divine{Lord} left his servants and his livestock outside in the fields.

\v{22}Then the \divine{Lord} told Moses, ``Stretch out your hand toward heaven, and there will be hail in all the land of Egypt, on people, animals, and all the vegetation of the field throughout the land of Egypt.'' \v{23}When Moses stretched out his staff toward heaven, the \divine{Lord} sent thunder and hail, and lightning struck the earth. The \divine{Lord} rained hail on the land of Egypt.

\v{24}There was very heavy hail, and lightning was flashing continuously in the midst of the hail. There had not been anything like it in the land of Egypt since it had become a nation. \v{25}The hail struck everything, including people and animals, outside in the fields throughout the land of Egypt. The hail struck all the vegetation of the fields and shattered all the trees in the orchards. \v{26}Only in the land of Goshen, where the Israelis were, was there no hail.

\v{27}Pharaoh sent word\fnote{\fbackref{9:27} The Heb. lacks \fbib{word}} and called for Moses and Aaron. ``I've sinned this time,'' he told them. ``The \divine{Lord} is righteous, but I and my people are wicked. \v{28}Pray to the \divine{Lord}! There has been enough of God's thunder and hail! I'll let you go, and you need not stay any longer.''

\v{29}Moses told him, ``When I leave the city I'll spread out my hands to the \divine{Lord}. The thunder will cease and the hail won't continue, so that you may know that the earth belongs to the \divine{Lord}. \v{30}But as for you and your officials,\fnote{\fbackref{9:30} Or \fbib{servants}} I know that you don't yet fear the \divine{Lord} God.'' \v{31}(Now the flax and the barley were ruined because the barley was in ear and the flax was in bud. \v{32}The wheat and the wild grain\fnote{\fbackref{9:32} Or \fbib{spelt}} were not ruined because they were late crops.)

\v{33}Then Moses went out of the city from Pharaoh and spread out his hands to the \divine{Lord}. The thunder and hail stopped, and the rain no longer poured out on the land. \v{34}When Pharaoh saw that the rain, hail, and thunder had stopped, he continued to sin. He, along with his officials,\fnote{\fbackref{9:34} Or \fbib{servants}} hardened his heart. \v{35}Pharaoh's heart was stubborn,\fnote{\fbackref{9:35} Lit. \fbib{strong}} and he did not let the Israelis go, just as the \divine{Lord} had said through Moses.
\labelchapt{10}
\passage{The Plague of Locusts}

\chapt{10}
\v{1}Then the \divine{Lord} told Moses, ``Go to Pharaoh, for I've hardened his heart and the hearts of his officials\fnote{\fbackref{10:1} Or \fbib{servants}} in order to perform\fnote{\fbackref{10:1} Lit. \fbib{put}} these signs of mine among them,\fnote{\fbackref{10:1} Lit. \fbib{him}} \v{2}so you may tell\fnote{\fbackref{10:2} Lit. \fbib{declare in the ears of}} your children and your grandchildren how I toyed with the Egyptians and about my miraculous signs that I performed among them, so all of you\fnote{\fbackref{10:2} Lit. \fbib{you} (pl.)} may know that I am the \divine{Lord}.

\v{3}Moses and Aaron went to Pharaoh and told him, ``This is what the \divine{Lord} God of the Hebrews says: `How long will you refuse to humble yourself before me? Let my people go, so they may serve\fnote{\fbackref{10:3} Or \fbib{worship}} me. \v{4}But if you refuse to let my people go, tomorrow I'm going to bring locusts into your territory. \v{5}They'll cover the surface of the land so a person\fnote{\fbackref{10:5} Lit. \fbib{he}} cannot see the ground, and they'll eat what is left for you of the residue from the hail. They'll also eat all your trees that grow in the orchards. \v{6}Your houses will be filled, along with the houses of all your officials\fnote{\fbackref{10:6} Or \fbib{servants}} and the houses of all the Egyptians---something that neither your fathers nor your ancestors ever saw from the time they were on earth until now.'\,'' Then Moses\fnote{\fbackref{10:6} Lit. \fbib{he}} turned and left Pharaoh's presence.

\v{7}Then the officials\fnote{\fbackref{10:7} Or \fbib{servants}} of Pharaoh told him, ``How long will this man be a snare to us? Let the people go so they may serve the \divine{Lord} their God! Don't you realize yet that Egypt is about to be destroyed?''

\v{8}Moses and Aaron were brought back to Pharaoh and he told them, ``Go, serve\fnote{\fbackref{10:8} Or \fbib{worship}} the \divine{Lord} your God. But exactly who\fnote{\fbackref{10:8} Lit. \fbib{who and who}} will go?''

\v{9}Moses said, ``We will go with our young and with our old. We will go with our sons and our daughters, with our sheep and our cattle, because it's a festival to the \divine{Lord} for us.''

\v{10}Then Pharaoh\fnote{\fbackref{10:10} Lit. \fbib{he}} told them, ``The \divine{Lord} will certainly\fnote{\fbackref{10:10} Lit. \fbib{it will be thus that}} be with you if I let you and your little ones go. I know\fnote{\fbackref{10:10} Lit. \fbib{See!}} some evil plan is in your mind.\fnote{\fbackref{10:10} Lit. \fbib{is before you}} \v{11}No! Let the men go and serve\fnote{\fbackref{10:11} Or \fbib{worship}} the \divine{Lord}, for that is what you were seeking.'' Then they were driven out from the presence of Pharaoh.

\v{12}The \divine{Lord} told Moses, ``Stretch out your hand over the land of Egypt to bring\fnote{\fbackref{10:12} Lit. \fbib{for}} the locusts, and they'll come up over the land of Egypt and eat all the vegetation of the land, everything that the hail left.'' \v{13}Moses stretched out his staff over the land of Egypt, and the \divine{Lord} sent an east wind into the land all that day and throughout\fnote{\fbackref{10:13} Lit. \fbib{all}} the night. When morning came, the east wind brought the locusts.

\v{14}The locusts came up over all the land of Egypt and settled on all the territory of Egypt in great swarms.\fnote{\fbackref{10:14} Lit. \fbib{it was very heavy}} There had never been locusts like this before nor would there ever be again. \v{15}They covered the surface of the entire land so that it\fnote{\fbackref{10:15} Lit. \fbib{the land}} was dark. They ate all the vegetation of the land and the fruit from the trees that the hail left. Nothing green was left on the trees or on the vegetation in all the land of Egypt.

\v{16}Pharaoh quickly called Moses and Aaron and said, ``I've sinned against the \divine{Lord} your God and against you. \v{17}Now, please forgive my sin only this time, and pray to the \divine{Lord} your God that he would at least remove this\fnote{\fbackref{10:17} Lit. \fbib{this death}} from me.''

\v{18}Moses left Pharaoh and prayed to the \divine{Lord}. \v{19}Then the \divine{Lord} brought\fnote{\fbackref{10:19} Lit. \fbib{turned}} a very strong west wind that took the locusts and drove them into the Reed\fnote{\fbackref{10:19} So MT; LXX reads \fbib{Red}} Sea. Not one locust remained in all the territory of Egypt. \v{20}But the \divine{Lord} made Pharaoh's heart stubborn\fnote{\fbackref{10:20} Lit. \fbib{strong}} and he would not let the Israelis go.
\passage{The Plague of Darkness}

\v{21}Then the \divine{Lord} told Moses, ``Stretch your hand toward the sky and there will be darkness over the land of Egypt, a darkness that one can feel.'' \v{22}So Moses stretched his hand toward the sky, and there was thick darkness in all the land of Egypt for three days. \v{23}No one could see anyone else, nor could anyone get up from his place for three days. But there was light for all the Israelis in their dwellings.

\v{24}Pharaoh called Moses and said, ``Go serve\fnote{\fbackref{10:24} Or \fbib{worship}} the \divine{Lord}, but your flocks and your cattle are to remain. Even your little ones can go with you!''

\v{25}Moses said, ``You must let us have\fnote{\fbackref{10:25} Lit. \fbib{give into our hand}} sacrifices and burnt offerings to offer to the \divine{Lord} our God. \v{26}And even our livestock must go with us. Not a hoof will be left behind because we will use\fnote{\fbackref{10:26} Lit. \fbib{take}} some of them to serve the \divine{Lord} our God, and until we get there we won't know what we need to serve\fnote{\fbackref{10:26} Lit. \fbib{what (or how) we will serve}} the \divine{Lord}.''

\v{27}The \divine{Lord} made Pharaoh's heart stubborn,\fnote{\fbackref{10:27} Lit. \fbib{strong}} and he did not want to let them go. \v{28}Then Pharaoh told him, ``Get away from me! Watch out that you never see my face again, because on the day you see my face, you will die!''

\v{29}Moses said, ``Just as you have said, I won't see your face again!''
\labelchapt{11}
\passage{Warning of the Death of the Firstborn}

\chapt{11}
\v{1}Then the \divine{Lord} told Moses, ``I'll bring one more plague on Pharaoh and Egypt. After that he'll let you leave from here, and when he lets you go, he will certainly drive you out from here. \v{2}Tell\fnote{\fbackref{11:2} Lit. \fbib{Say in the ears of}} the people that each man is to ask his neighbor and each woman her neighbor for articles of silver and gold.''

\v{3}The \divine{Lord} made the Egyptians look on the people with favor. Also the man Moses was highly regarded\fnote{\fbackref{11:3} Lit. \fbib{very great}} in the land of Egypt, both in the opinion\fnote{\fbackref{11:3} Lit. \fbib{sight}} of Pharaoh's officials\fnote{\fbackref{11:3} Or \fbib{servants}} and in the opinion\fnote{\fbackref{11:3} Lit. \fbib{sight}} of the people.

\v{4}So Moses announced to Pharaoh,\fnote{\fbackref{11:4} The Heb. lacks \fbib{to Pharaoh}} ``This is what the \divine{Lord} says: `About midnight I'm going throughout Egypt, \v{5}and all the firstborn in the land of Egypt will die, from the firstborn of Pharaoh who sits on his throne to the firstborn of the slave girl who operates\fnote{\fbackref{11:5} Lit. \fbib{is behind}} the hand mill, along with the firstborn of the animals. \v{6}There will be a great cry throughout the land of Egypt, like there has never been and never will be again. \v{7}But among the Israelis, from people to animals, not even a dog will bark,\fnote{\fbackref{11:7} Lit. \fbib{will sharpen its tongue}} so you may know that the \divine{Lord} is distinguishing between the Egyptians and the Israelis.' \v{8}All these officials\fnote{\fbackref{11:8} Or \fbib{servants}} of yours will come down to me, prostrate themselves to me, and say, `Get out, you and all the people following\fnote{\fbackref{11:8} Lit. \fbib{at your feet}} you!' After that I'll go out.'' Then Moses\fnote{\fbackref{11:8} Lit. \fbib{he}} angrily left Pharaoh.

\v{9}The \divine{Lord} told Moses, ``Pharaoh won't listen to you. As a result, my wonders will increase throughout the land of Egypt.'' \v{10}Moses and Aaron did all these wonders in front of Pharaoh, but the \divine{Lord} made Pharaoh's heart stubborn,\fnote{\fbackref{11:10} Lit. \fbib{strong}} and he would not let the Israelis go out from his land.
\labelchapt{12}
\passage{The Passover is Instituted}

\chapt{12}
\v{1}The \divine{Lord} told Moses and Aaron in the land of Egypt, \v{2}``This month will mark the beginning of months for you. It will be the first month of the year for you. \v{3}Tell the entire congregation of Israel, `On the tenth of this month they're each to take a lamb for themselves, according to their ancestors' households, one lamb for each household. \v{4}If a household is too small for a lamb, then it and its closest neighbor are to obtain one based on the number of individuals---dividing\fnote{\fbackref{12:4} Lit. \fbib{calculate}} the lamb based on what each person can eat. \v{5}Your lamb is to be a year old male without blemish. You may take it from the sheep or from the goats. \v{6}It is to remain under your care until the fourteenth day of this month, and then the entire assembly of the congregation of Israel is to slaughter it at twilight. \v{7}They're to take some of the blood and put it on the two doorposts and on the lintel of the houses where they eat the lamb.\fnote{\fbackref{12:7} Lit. \fbib{it}} \v{8}That very night they're to eat the meat, roasted over the fire, with unleavened bread and bitter herbs. \v{9}Don't eat any of it raw or boiled in water. Instead, roast it over the fire, with its head, legs, and internal organs. \v{10}Don't leave any of it until morning, and whatever does remain of it until morning you are to burn in the fire.

\v{11}```This is how you are to eat it: with your cloak tucked into your belt, your sandals on your feet, and your staff in your hand. You are to eat it hurriedly---it's the \divine{Lord}'s Passover. \v{12}I'll pass through the land of Egypt that night and strike every firstborn in the land of Egypt, both people and animals. I'll execute judgments on all the gods of Egypt. I am the \divine{Lord}. \v{13}The blood will be a sign for you on the houses where you are. I'll see the blood and pass over you. There will be no plague to destroy you when I strike the land of Egypt.

\v{14}```This day is to be a memorial for you, and you are to celebrate it as a festival to the \divine{Lord}. You are to celebrate it as a perpetual ordinance from generation to generation. \v{15}You are to eat unleavened bread for seven days. On the first day be sure to remove all the leaven from your houses, because any person who eats anything leavened from the first day until the seventh will be cut off from Israel. \v{16}Also, on the first day you're to hold a holy assembly, and on the seventh day you're to hold a holy assembly. No work is to be done during those days, except for preparing what is to be eaten by each person.

\v{17}```You are to observe the Festival of Unleavened Bread, since on this very day I brought your tribal divisions from the land of Egypt. You are to observe this day from generation to generation as a perpetual ordinance. \v{18}In the first month, from the evening of the fourteenth day of the month until the evening of the twenty-first day of the month, you are to eat unleavened bread. \v{19}For seven days leaven is not to be found in your houses. Indeed, any person who eats anything leavened, is to be cut off from the congregation of Israel, whether an alien or a native of the land. \v{20}You are not to eat what is leavened. You are to eat unleavened bread in all your settlements.'\,''

\v{21}Then Moses summoned all the elders of Israel and told them, ``Choose sheep for your families, and slaughter the Passover lamb. \v{22}Take a bundle of hyssop and dip it in the blood that is in the basin, and apply some of the blood in the basin to the lintel and the two doorposts. None of you is to go out of the doorway of his house until morning, \v{23}because the \divine{Lord} will pass through to strike down the Egyptians, and when he sees the blood on the lintel and the two doorposts, the \divine{Lord} will pass over the doorway, and won't allow the destroyer to enter your houses to strike you down. \v{24}You are to observe this event as a perpetual ordinance for you and your children forever. \v{25}When you enter the land that the \divine{Lord} will give you, just as he promised, you are to observe this ritual. \v{26}And when your children say to you, `What does this ritual mean?'\fnote{\fbackref{12:26} Lit. \fbib{is . . . to you?}} \v{27}you are to say, `It is the Passover sacrifice to the \divine{Lord}, who passed over the houses of the Israelis in Egypt when he struck down the Egyptians but spared our houses.'\,'' Then the people bowed down and worshipped. \v{28}The Israelis did this. Moses and Aaron did just what the \divine{Lord} had commanded.
\passage{The Death of the Firstborn in Egypt}

\v{29}And so at midnight the \divine{Lord} struck down every firstborn in the land of Egypt, from the firstborn of Pharaoh who sat on his throne to the firstborn of the prisoner who was in the dungeon, and all the firstborn of the livestock. \v{30}Pharaoh got up during the night, he, all his officials,\fnote{\fbackref{12:30} Or \fbib{servants}} and all the Egyptians, and there was loud wailing in Egypt, because there was not a house without someone dead in it. \v{31}Then he summoned Moses and Aaron during the night and told them: ``Get up, go out from among my people, both you and the Israelis! Go, serve\fnote{\fbackref{12:31} Or \fbib{worship}} the \divine{Lord} as you have said. \v{32}Take both your sheep and your cattle, just as you demanded\fnote{\fbackref{12:32} Lit. \fbib{said}} and go! And bless me too!''

\v{33}The Egyptian officials\fnote{\fbackref{12:33} The Heb. lacks \fbib{officials}} urged the people to send them out of the land quickly, because they were saying, ``We'll all be dead!'' \v{34}So the people took their dough before it was leavened, with their kneading bowls wrapped up in their cloaks on their shoulders. \v{35}Meanwhile, the Israelis had done as Moses said;\fnote{\fbackref{12:35} Lit. \fbib{according to the word of Moses}} they had asked the Egyptians for objects of silver and objects of gold, and for clothes. \v{36}The \divine{Lord} had given the people favor in the eyes of the Egyptians, so that they gave them what they requested. As a result, they plundered the Egyptians.
\passage{The Exodus Begins}

\v{37}About 600,000 Israeli men traveled from Rameses to Succoth on foot, not counting children. \v{38}A mixed multitude also went up with them, along with a very large number of livestock, including sheep and cattle. \v{39}They baked the dough that they brought out of Egypt into thin cakes of unleavened bread. It had not been leavened because they were driven out of Egypt and could not wait, nor had they prepared provisions for themselves.

\v{40}Now the time that the Israelis lived in Egypt was 430 years. \v{41}At the end of 430 years, to the very day, all the tribal divisions of the \divine{Lord} went out from the land of Egypt. \v{42}That was for the \divine{Lord} a night of vigil\fnote{\fbackref{12:42} Or \fbib{watching, guarding}} to bring them out of the land of Egypt. This same night belongs to the \divine{Lord}, and is to be a vigil for all the Israelis from generation to generation.
\passage{Instructions for the Passover}

\v{43}The \divine{Lord} told Moses and Aaron, ``These are the regulations for the Passover: No foreigner is to eat it, \v{44}though any slave\fnote{\fbackref{12:44} Lit. \fbib{of a man}} purchased with money may eat it after you have circumcised him. \v{45}But no temporary resident or a hired servant is to eat it. \v{46}It is to be eaten in one house, and you are not to take any of the meat outside the house, nor are you to break any of its bones. \v{47}The whole congregation of Israel is to observe it. \v{48}If an alien who resides with you wants to observe the Passover to the \divine{Lord}, every male in his household\fnote{\fbackref{12:48} Lit. \fbib{belonging to him}} must be circumcised, and then he may come near to observe it. He is to be like a native of the land, but no uncircumcised person is to eat it. \v{49}A single law exists for the native and the alien who resides among you.''

\v{50}All the Israelis did this. They did exactly as the \divine{Lord} commanded Moses and Aaron. \v{51}And on that very day, the \divine{Lord} brought the Israelis out of the land of Egypt by their tribal divisions.
\labelchapt{13}
\passage{Consecration of the Firstborn}

\chapt{13}
\v{1}The \divine{Lord} spoke to Moses, \v{2}``Consecrate to me every firstborn male. Whatever is the first to open the womb among the Israelis, both of humans and of animals, belongs to me.''
\passage{The Festival of Unleavened Bread}

\v{3}Then Moses told the people, ``Remember this day on which you came out of Egypt, from the house of bondage, because the \divine{Lord} brought you out from this place with a strong show of force.\fnote{\fbackref{13:3} Lit. \fbib{strong hand}} Moreover, nothing leavened is to be eaten. \v{4}Today, in the month of Abib, you are going out. \v{5}When the \divine{Lord} brings you to the land of the Canaanite, the Hittite, the Amorite, the Hivite, and the Jebusite, which he swore to your ancestors to give you---a land flowing with milk and honey---you are to observe this ritual in this month. \v{6}You are to eat unleavened bread for seven days, and on the seventh day there is to be a festival to the \divine{Lord}. \v{7}Unleavened bread is to be eaten for seven days, and nothing leavened is to be seen among you, nor is leaven to be seen among you throughout your territory. \v{8}And you are to tell your child on that day, `This is because of what the \divine{Lord} did for me when I came out of Egypt.' \v{9}It is to be a sign for you on your hand and a reminder on your forehead,\fnote{\fbackref{13:9} Lit. \fbib{between your eyes}} so that you may speak about the instruction\fnote{\fbackref{13:9} Or \fbib{Law}} of the \divine{Lord}; for the \divine{Lord} brought you out of Egypt with a strong show of force.\fnote{\fbackref{13:9} Lit. \fbib{strong hand}} \v{10}You are to keep this ordinance at its appointed time from year to year.''
\passage{The Redemption of the Firstborn}

\v{11}``When the \divine{Lord} brings you to the land of the Canaanite and gives it to you, just as he promised you and your ancestors, \v{12}you are to dedicate to the \divine{Lord} everything that first opens the womb. All the firstborn males\fnote{\fbackref{13:12} Lit. \fbib{Whatever first opens the womb}} of your livestock belong to the \divine{Lord}. \v{13}You are to redeem every firstborn donkey\fnote{\fbackref{13:13} Lit. \fbib{Whatever first opens the womb}} with a lamb, and if you don't redeem it, you are to break its neck. You are to redeem every firstborn\fnote{\fbackref{13:13} Lit. \fbib{firstborn of man}} among your sons. \v{14}Then when your child asks you in the future, `What is this?', you are to say to him, `The \divine{Lord} brought us out of Egypt, from the house of bondage with a strong show of force.\fnote{\fbackref{13:14} Lit. \fbib{strong hand}} \v{15}And when Pharaoh stubbornly refused to let us go, the \divine{Lord} killed every firstborn in the land of Egypt, from the firstborn of humans to the firstborn of animals. Therefore, I sacrifice to the \divine{Lord} every male that first opens the womb, but I redeem every firstborn of my sons. \v{16}It is to be a sign on your hand and an emblem\fnote{\fbackref{13:16} Or \fbib{phylacteries}} on your forehead,\fnote{\fbackref{13:16} Lit. \fbib{between your eyes}} because the \divine{Lord} brought us out of Egypt with a strong show of force.'\,''\fnote{\fbackref{13:16} Lit. \fbib{strong hand}}
\passage{God Guides the People in the Desert}

\v{17}When Pharaoh let the people go, God did not lead them along the road through the land of the Philistines, even though it was nearer, because God had said, ``If the people face war, they may change their minds and return to Egypt.'' \v{18}So God led the people the roundabout way of the desert toward the Reed\fnote{\fbackref{13:18} So MT; LXX reads \fbib{Red}} Sea. The Israelis went up from the land of Egypt in military formation.\fnote{\fbackref{13:18} Or \fbib{prepared for battle}} \v{19}Moses took the bones of Joseph with him, because Joseph\fnote{\fbackref{13:19} Lit. \fbib{he}} had made the Israelis take this solemn oath: ``God will certainly take notice of you, and then you must carry my bones up with you from here.'' \v{20}They left Succoth and camped in Etham at the edge of the desert. \v{21}The \divine{Lord} went in front of them by day in a pillar of cloud to lead them along the way, and by night in a pillar of fire to give them light, so they could travel both day and night. \v{22}Neither the pillar of cloud by day nor the pillar of fire by night left its place in front of the people.
\labelchapt{14}
\passage{Crossing the Reed Sea}

\chapt{14}
\v{1}The \divine{Lord} told Moses, \v{2}``Tell the Israelis that they are to turn back and camp in front of Pi-hahiroth, between Migdol and the sea. You are to camp in front of Baal-zephon, opposite it by the sea. \v{3}Pharaoh will say about the Israelis, `They're wandering aimlessly in the land, and the desert has closed in on them.' \v{4}I've made Pharaoh's heart stubborn\fnote{\fbackref{14:4} Lit. \fbib{strong}} so he will pursue them. But I'll receive honor by means of\fnote{\fbackref{14:4} Or \fbib{over}} Pharaoh and his army, so that the Egyptians will know that I am the \divine{Lord}.'' So this is what the Israelis\fnote{\fbackref{14:4} Lit. \fbib{they}} did.

\v{5}When the king of Egypt was told that the people had fled, the minds\fnote{\fbackref{14:5} Lit. \fbib{heart}} of Pharaoh and his officials\fnote{\fbackref{14:5} Or \fbib{servants}} changed toward the people, and they said, ``What have we done in releasing Israel from serving us?'' \v{6}So Pharaoh\fnote{\fbackref{14:6} Lit. \fbib{he}} had his chariot prepared and took his troops\fnote{\fbackref{14:6} Or \fbib{people}} with him.

\v{7}He took 600 of the best chariots, and all the other\fnote{\fbackref{14:7} The Heb. lacks \fbib{other}} chariots of Egypt with officers in charge of each one. \v{8}The \divine{Lord} made the heart of Pharaoh, king of Egypt, stubborn,\fnote{\fbackref{14:8} Lit. \fbib{strong}} and he defiantly\fnote{\fbackref{14:8} Lit. \fbib{with a high hand}} pursued the Israelis as they were leaving. \v{9}The Egyptians pursued them---all the chariot-horses of Pharaoh, along with his horsemen and army---and they overtook them camped by the sea, near Pi-hahiroth, in front of Baal Zephon.

\v{10}As Pharaoh approached, the Israelis looked up, and there were the Egyptians bearing down on them! Extremely frightened, the Israelis cried out to the \divine{Lord}. \v{11}They also\fnote{\fbackref{14:11} The Heb. lacks \fbib{also}} told Moses, ``Was it because there were no graves in Egypt that you took us out to die in the desert? What have you done to us, by bringing us out of Egypt? \v{12}Is this not what we told you in Egypt, when we said, `Leave us alone!'\fnote{\fbackref{14:12} Lit. \fbib{cease from us}} and `Let us serve the Egyptians!'? Indeed, it would have been better for us to serve the Egyptians than to die in the desert!''

\v{13}Moses told the people, ``Don't be afraid! Stand still and watch how the \divine{Lord} will deliver you today, because you will never again see the Egyptians whom you're looking at today. \v{14}The \divine{Lord} will fight for you while you keep still.''

\v{15}Then the \divine{Lord} told Moses, ``Why are you crying out to me? Tell the Israelis to move out! \v{16}You are to raise your staff, stretch out your hand over the sea, and divide it, so the Israelis may go into the middle of the sea on dry land. \v{17}Even now I'm hardening the heart of the Egyptians so they'll go after the Israelis.\fnote{\fbackref{14:17} Lit. \fbib{them}} Then I'll receive honor by means of\fnote{\fbackref{14:17} Or \fbib{over}} Pharaoh and all his army, his chariots, and his horsemen. \v{18}Then the Egyptians will know that I am the \divine{Lord} when I receive honor by means of\fnote{\fbackref{14:18} Or \fbib{over}} Pharaoh, his chariots, and his horsemen.''

\v{19}Then the angel of God, who was going in front of the camp of Israel, moved behind them. The pillar of cloud also\fnote{\fbackref{14:19} The Heb. lacks \fbib{also}} moved from in front of them and stood behind them, \v{20}coming between the camp of the Egyptians and the camp of Israel. The cloud remained there even\fnote{\fbackref{14:20} The Heb. lacks \fbib{even}} in the darkness,\fnote{\fbackref{14:20} Lit. \fbib{and the darkness}} illuminating the night, so that the one side did not come near the other all night.

\v{21}Then Moses stretched out his hand over the sea, and the \divine{Lord} caused the water to retreat by a strong east wind all night, turning the sea into dry land. As the waters were divided, \v{22}the Israelis went into the middle of the sea on dry land, and the waters formed a wall for them on their right and on their left.

\v{23}The Egyptians pursued---all the horses of Pharaoh, his chariots, and his horsemen---and they went into the middle of the sea after them. \v{24}In the morning watch, the \divine{Lord} looked down on the Egyptian camp through the pillar of fire and cloud, and he threw the Egyptian camp into confusion. \v{25}He made the wheels of their chariots wobble\fnote{\fbackref{14:25} Or \fbib{fall off}} so that they drove them with difficulty. The Egyptians said, ``Let's flee from Israel because the \divine{Lord} is fighting for them and against us.''\fnote{\fbackref{14:25} Lit. \fbib{for them against the Egyptians}}
\passage{The Egyptians Drown in the Sea}

\v{26}Then the \divine{Lord} told Moses, ``Stretch out your hand over the sea and the water will come back over the Egyptians, over their chariots, and over their horsemen.'' \v{27}Moses stretched out his hand over the sea, and the water returned to its normal depth at daybreak. The Egyptians tried to retreat in front of the advancing water,\fnote{\fbackref{14:27} Lit. \fbib{of it}} but the \divine{Lord} destroyed\fnote{\fbackref{14:27} Lit. \fbib{shook off}} the Egyptians in the middle of the sea. \v{28}The water returned, covering the chariots and the horsemen of Pharaoh's entire army that had pursued the Israelis into the sea. Not a single one of them remained. \v{29}But the Israelis walked through the middle of the sea on dry land, and the water stood like a wall for them on their right and on their left.

\v{30}On that day the \divine{Lord} delivered Israel from the hand of the Egyptians, and Israel saw the Egyptians dead along the seashore. \v{31}When Israel saw the great force\fnote{\fbackref{14:31} Lit. \fbib{hand}} by which the \divine{Lord} had acted against the Egyptians, the people feared the \divine{Lord}, and they believed the \divine{Lord} and Moses his servant.
\labelchapt{15}
\passage{The Song of Moses}

\chapt{15}
\v{1}Then Moses and the Israelis sang this song to the \divine{Lord}:

\begin{poetry}
\poeml ``I'll sing to the \divine{Lord}, \\
\poemll    for he is highly exalted. \\
\poeml The horse and its rider \\
\poemll    he has thrown into the sea. \\
\poeml \v{2}The \divine{Lord} is my strength and song,\fnote{\fbackref{15:2} Some mss. read \fbib{my song}} \\
\poemll    and he has become my salvation. \\
\poeml This is my God and I'll praise him, \\
\poemll    the God of my father and I'll exalt him. \\
\poeml \v{3}The \divine{Lord} is a man of war, \\
\poemll    the \divine{Lord} is his name! \\
\poeml \v{4}``Pharaoh's chariots and his army \\
\poemll    he has hurled into the sea; \\
\poemlll       his best officers sank in the Reed\fnote{\fbackref{15:4} So MT; LXX reads \fbib{Red}} Sea. \\
\poeml \v{5}The deep covered them, \\
\poemll    they went down into the depths like a rock. \\
\poeml \v{6}Your right hand, \divine{Lord}, was majestic in strength, \\
\poemll    your right hand, \divine{Lord}, shattered the enemy. \\
\poeml \v{7}In the greatness of your majesty \\
\poemll    you broke down your enemies. \\
\poeml You sent forth your anger, \\
\poemll    it consumed them like stubble. \\
\poeml \v{8}By the breath\fnote{\fbackref{15:8} Or \fbib{wind}} of your nostrils \\
\poemll    the waters were piled up, \\
\poeml the flowing waters stood up like a hill, \\
\poemll    the deep waters congealed in the heart of the sea. \\
\poeml \v{9}``The enemy said, `I'll pursue them,\fnote{\fbackref{15:9} The Heb. lacks \fbib{them}} I'll overtake them,\fnote{\fbackref{15:9} The Heb. lacks \fbib{them}} \\
\poemll    I'll divide the spoil. \\
\poeml I'll satisfy my anger\fnote{\fbackref{15:9} Lit. \fbib{my soul}} on them, \\
\poemll    I'll draw my sword, \\
\poemlll       and my hand will bring them to ruin.' \\
\poeml \v{10}``You blew with your breath,\fnote{\fbackref{15:10} Or \fbib{wind}} \\
\poemll    and the sea covered them; \\
\poemlll       they sank like lead in the mighty water. \\
\poeml \v{11}``Who is like you among the gods, \divine{Lord}? \\
\poemll    Who is like you, majestic in holiness, \\
\poemlll       awesome in splendor,\fnote{\fbackref{15:11} I.e. in acts deserving of praise} and working wonders? \\
\poeml \v{12}You stretched out your right hand, \\
\poemll    and the earth swallowed them. \\
\poeml \v{13}``You have led with your gracious love \\
\poemll    this people whom you redeemed. \\
\poeml You have guided them with your strength \\
\poemll    to your holy dwelling. \\
\poeml \v{14}``The nations\fnote{\fbackref{15:14} Lit. \fbib{peoples}} heard and they quaked, \\
\poemll    anguish\fnote{\fbackref{15:14} Lit. \fbib{writhing}} seized the inhabitants of Philistia. \\
\poeml \v{15}Then the chiefs of Edom were terrified, \\
\poemll    the nobles of Moab trembled uncontrollably, \\
\poemlll       and all the inhabitants of Canaan melted away. \\
\poeml \v{16}Dread and fear have fallen on them, \\
\poemll    because of the strength\fnote{\fbackref{15:16} Lit. \fbib{greatness}} of your arm. \\
\poeml They have become silent as a stone, \\
\poemll    until your people pass by, \divine{Lord}, \\
\poemlll       until this people you acquired pass by. \\
\poeml \v{17}``You will bring them in and plant them \\
\poemll    on the mountain of your inheritance. \\
\poeml You have made a place where you will reside, \divine{Lord}. \\
\poemll    Your own hands have established a sanctuary, \divine{Lord}. \\
\poeml \v{18}The \divine{Lord} will reign forever and ever.''
\end{poetry}

\v{19}When the horses of Pharaoh, his chariots, and his horsemen went into the sea, the \divine{Lord} caused the waters of the sea to come back over them, but the Israelis walked through the middle of the sea on dry land.
\passage{The Song of Miriam}

\v{20}Then Miriam the prophetess, Aaron's sister, took a tambourine in her hand and went out with all the women behind her with tambourines and dancing. \v{21}Miriam sang to them,

\begin{poetry}
\poeml ``Sing to the \divine{Lord}, for he is highly exalted! \\
\poemll    The horse and its rider \\
\poemlll       he has thrown into the sea.''
\end{poetry}
\passage{God Provides Water for the People}

\v{22}Then Moses led Israel from the Reed\fnote{\fbackref{15:22} So MT; LXX reads \fbib{Red}} Sea and they went to the desert of Shur. They traveled into the desert for three days and did not find water. \v{23}When they came to Marah, they could not drink the water at Marah because it was bitter. (That is why it's called\fnote{\fbackref{15:23} Lit. \fbib{why one calls its name}} Marah.)\fnote{\fbackref{15:23} \fbib{Marah} means \fbib{bitter} in Heb.} \v{24}Then the people complained against Moses: ``What are we to drink?'' \v{25}Moses\fnote{\fbackref{15:25} Lit. \fbib{He}} cried out to the \divine{Lord}, and the \divine{Lord} showed him a tree, which he threw into the water, and the water became sweet.

There the \divine{Lord}\fnote{\fbackref{15:25} Lit. \fbib{he}} presented to them a statute and an ordinance, and there he tested them. \v{26}He said, ``If you will carefully obey the \divine{Lord} your God, do what he sees to be right, listen to his commandments, and keep all his statutes, then I won't inflict on you all the diseases that I inflicted on the Egyptians, because I am the \divine{Lord} your healer.'' \v{27}Then they came to Elim where there were twelve springs of water and 70 palm trees, and they camped there by the water.
\labelchapt{16}
\passage{Manna and Quail Provided}

\chapt{16}
\v{1}Later, they left Elim, and the whole congregation of the Israelis came to the desert\fnote{\fbackref{16:1} Or \fbib{wilderness}} of Sin, which lay between Elim and Sinai, on the fifteenth day of the second month after their departure from the land of Egypt. \v{2}The whole congregation of the Israelis complained against Moses and Aaron in the desert. \v{3}The Israelis told them, ``If only we had died by the \divine{Lord}'s hand in the land of Egypt when we sat by the cooking pots,\fnote{\fbackref{16:3} Lit. \fbib{pots for cooking meat}} when we ate bread until we were filled---because you brought us to this desert to kill this entire congregation with hunger.''

\v{4}The \divine{Lord} told Moses, ``Listen very carefully! I'll cause food to rain down for you from heaven, and the people are to go out and gather each day's portion on that day. In this way I'll test them to demonstrate whether or not they'll live according to my instructions. \v{5}On the sixth day, when they prepare what they bring in, it will be double what they gather on other days.''\fnote{\fbackref{16:5} Lit. \fbib{gather daily}}

\v{6}So Moses and Aaron addressed the entire congregation of the Israelis: ``This evening you will know that the \divine{Lord} has brought you out of the land of Egypt, \v{7}and in the morning you will see the glory of the \divine{Lord}, because he has heard your complaints against him.\fnote{\fbackref{16:7} Lit. \fbib{against the \divine{Lord}}} After all, who are we that you complain against us?'' \v{8}Moses also said, ``When the \divine{Lord} gives you meat to eat in the evening, and bread in the morning to satisfy you, the \divine{Lord} will hear your complaints directed\fnote{\fbackref{16:8} Lit. \fbib{complained}} against him. Who are we? Your complaints aren't against us, but rather against the \divine{Lord}.''

\v{9}Then Moses instructed Aaron, ``Say to the whole congregation of the Israelis, `Come near into the \divine{Lord}'s presence, because he has heard your complaints.'\,''

\v{10}While Aaron was speaking to all the congregation of the Israelis, they turned toward the desert, and there the glory of the \divine{Lord} was seen in the cloud. \v{11}The \divine{Lord} told Moses, \v{12}``I've heard the complaints of the Israelis. Tell them, `At twilight you are to eat meat and in the morning you are to be filled with bread, so you may know that I am the \divine{Lord} your God.'\,''

\v{13}Later that evening quail came up and covered the camp, and then in the morning there was a layer of dew around the camp. \v{14}When the layer of dew evaporated,\fnote{\fbackref{16:14} Lit. \fbib{went up}} on the surface of the desert a fine flaky substance, as fine as frost, appeared on the ground. \v{15}When the Israelis saw it, they asked one another, ``What is it?'',\fnote{\fbackref{16:15} Heb. \fbib{man hu;} cf. vs. 31} because they did not know what it was.

Moses told them, ``It's the food that the \divine{Lord} has given you to eat. \v{16}This is what the \divine{Lord} has commanded: `You are to gather from it what each person is to eat,\fnote{\fbackref{16:16} Lit. \fbib{each according to his eating}} about one omer\fnote{\fbackref{16:16} I.e. about two quarts} per person according to the number of your people, and one person is to gather for everyone in his tent.'\,''

\v{17}The Israelis did this, some gathering much, some little. \v{18}When they measured it with a vessel the capacity of which was one omer,\fnote{\fbackref{16:18} I.e. a vessel with a dry capacity of about two quarts} the one who gathered much did not have an excess, while the one who gathered little did not lack. They gathered exactly what each needed to eat.\fnote{\fbackref{16:18} Lit. \fbib{each according to his eating}}

\v{19}Then Moses told them, ``No one is to leave any of it until morning.'' \v{20}But they did not listen to Moses---some people left part of it until morning, and it produced maggots and smelled bad, so Moses got angry at them. \v{21}Every morning they gathered it, according to what each needed to eat; and when the sun became hot, it melted.

\v{22}On the sixth day they gathered twice as much bread, about two omers\fnote{\fbackref{16:22} I.e. about four quarts} per person. Then all the leaders of the congregation came and reported to Moses, \v{23}and he told them, ``This is what the \divine{Lord} said: `Tomorrow is a Sabbath observance, a holy Sabbath to the \divine{Lord}. Bake what you want to bake and boil what you want to boil, and put aside whatever remains to be kept for yourselves until morning.'\,'' \v{24}So they put it away until morning, as Moses commanded, and it did not smell bad, and there were no maggots in it. \v{25}Moses said, ``Eat it today, since today is a Sabbath to the \divine{Lord}, and today you won't find it in the field. \v{26}For six days you are to gather it, but on the seventh day, the Sabbath, there won't be any.''\fnote{\fbackref{16:26} Lit. \fbib{any on it}}

\v{27}Nevertheless, that seventh day some of the people went out to gather, but they did not find any. \v{28}Then the \divine{Lord} asked Moses, ``How long will you people\fnote{\fbackref{16:28} Lit. \fbib{you} (pl.); the Heb. lacks \fbib{people}} refuse to keep my commandments and my instructions?\fnote{\fbackref{16:28} Or \fbib{laws}} \v{29}You see that the \divine{Lord} has given you the Sabbath, and so on the sixth day he gives you food for two days. Let each person stay where he is; let no one leave his place on the seventh day.'' \v{30}So the people rested on the seventh day.

\v{31}The Israelis named it\fnote{\fbackref{16:31} Lit. \fbib{called its name}} ``manna''.\fnote{\fbackref{16:31} Manna sounds like the Heb. term \fbib{What is it?}; cf. vs. 15} It was white like coriander seed, and tasted like a wafer made with honey. \v{32}Moses said, ``This is what the \divine{Lord} has commanded: `Set aside one omer\fnote{\fbackref{16:32} I.e. about two quarts} of it for future generations, so that they may see the food with which I fed you in the desert when I brought you out of the land of Egypt.'\,''

\v{33}Then Moses told Aaron, ``Take a jar, fill it with about one omer\fnote{\fbackref{16:33} I.e. about two quarts} of manna, and place it in the \divine{Lord}'s presence, to be preserved throughout future generations.'' \v{34}So Aaron placed it before the Testimony\fnote{\fbackref{16:34} I.e. the tablets on which the ten commandments were written and which were placed in the Ark of the Covenant; cf. Exod 25:16 and 31:18} to be kept, just as the \divine{Lord} had commanded Moses. \v{35}The Israelis ate manna for 40 years until they came to a land where they could settle.\fnote{\fbackref{16:35} Or \fbib{an inhabited land}} They ate manna until they came to the border of the land of Canaan. \v{36}Now one omer\fnote{\fbackref{16:36} I.e. about two quarts} is a tenth of an ephah.\fnote{\fbackref{16:36} An ephah was about one half bushel}
\labelchapt{17}
\passage{God Provides Water from a Rock}
\passageinfo{(Numbers 20:1-13)}

\chapt{17}
\v{1}The whole congregation of the Israelis set out from the desert\fnote{\fbackref{17:1} Or \fbib{wilderness}} of Sin, traveling from place to place according to the command\fnote{\fbackref{17:1} Lit. \fbib{mouth}} of the \divine{Lord}. They camped at Rephidim, but there was no water for the people to drink.

\v{2}The people quarreled with Moses: ``Give us water to drink.''

Moses told them, ``Why are you quarreling with me? Why are you testing the \divine{Lord}?''

\v{3}But the people were thirsty there for water, so they\fnote{\fbackref{17:3} Lit. \fbib{the people}} complained against Moses: ``Why did you bring us up from Egypt to kill us, our children, and our livestock with thirst?''

\v{4}So Moses cried out to the \divine{Lord}: ``What am I to do with these people? Just a little more and they'll stone me.''

\v{5}Then the \divine{Lord} told Moses, ``Go over in front of the people and take some of the elders of Israel with you. Take in your hand the staff with which you struck the Nile, and go. \v{6}I'll be standing there in front of you on the rock at Horeb. You are to strike the rock and water will come out of it, so the people can drink.'' Moses did this in front of the elders of Israel.

\v{7}He named the place Massah\fnote{\fbackref{17:7} The Heb. name \fbib{Massah} means \fbib{Testing}} and Meribah,\fnote{\fbackref{17:7} The Heb. name \fbib{Meribah} means \fbib{Quarreling}} because the Israelis quarreled and tested the \divine{Lord} by saying: ``Is the \divine{Lord} really among us or not?''
\passage{The Amalekites Fight the Israelis}

\v{8}After this, the Amalekites came and fought with the Israelis at Rephidim. \v{9}Moses told Joshua, ``Choose some men for us and go out to fight against the Amalekites. Tomorrow I'll stand on top of the hill with the staff of God in my hand.'' \v{10}So Joshua did as Moses told him and fought against the Amalekites, while Moses, Aaron, and Hur went up to the top of the hill. \v{11}Whenever Moses raised his hand, the Israelis prevailed, but when his hand remained at his side,\fnote{\fbackref{17:11} Lit. \fbib{rested}} then the Amalekites prevailed. \v{12}When Moses' hands became heavy, they took a stone and put it under him, and he sat on it. Aaron and Hur supported his hands, one on one side and one on the other, and so his hands were steady until the sun went down. \v{13}Joshua defeated\fnote{\fbackref{17:13} Or \fbib{weakened}} Amalek and his army using swords.

\v{14}Then the \divine{Lord} told Moses, ``Write this in a book as a memorial and recite it to\fnote{\fbackref{17:14} Lit. \fbib{put it in the ear of}} Joshua: `I'll certainly wipe out the memory of the Amalekites from under heaven.'\,'' \v{15}Moses built an altar and named it ``The \divine{Lord} is My Banner.'' \v{16}``Because,'' he said, ``a fist has been raised in defiance\fnote{\fbackref{17:16} The Heb. lacks \fbib{in defiance}} against the throne of the \divine{Lord}, the \divine{Lord} will wage war against Amalek from generation to generation.''
\labelchapt{18}
\passage{Jethro Visits Moses}

\chapt{18}
\v{1}Jethro, the priest of Midian, Moses' father-in-law, heard all that God had done for Moses and for his people Israel, and how the \divine{Lord} had brought Israel out of Egypt. \v{2}Now Jethro, Moses' father-in-law, had taken back Moses' wife Zipporah after she had been sent away, \v{3}along with her two sons. The name of the one was Gershom, because he used to say, ``I was an alien\fnote{\fbackref{18:3} The Heb. word for alien (\fbib{ger} ) sounds like Gershom} in a foreign land,'' \v{4}while the name of the other was Eliezer,\fnote{\fbackref{18:4} The Heb. name Eliezer means \fbib{My God helps}} because he used to say,\fnote{\fbackref{18:4} The Heb. lacks \fbib{he used to say}} ``My father's God helped me and delivered me from Pharaoh's sword.''

\v{5}Moses' father-in-law Jethro, together with Moses' two sons and his wife, came to Moses in the desert where he was camped at the mountain of God.\fnote{\fbackref{18:5} I.e. Mount Sinai} \v{6}He told Moses, ``I, your father-in-law Jethro, am coming to you along with your wife and her two sons.'' \v{7}When Moses went out to meet his father-in-law, he bowed low and kissed him, and they greeted one another. Then they went into the tent.

\v{8}Moses told his father-in-law all that the \divine{Lord} had done to Pharaoh and to the Egyptians on Israel's behalf, all the hardships that they had encountered along the way, and how the \divine{Lord} had delivered them. \v{9}Jethro rejoiced over all the good that the \divine{Lord} had done for Israel in delivering them from the hand of the Egyptians. \v{10}Jethro said, ``Blessed be the \divine{Lord}, who delivered you from the hand of the Egyptians and from the hand of Pharaoh, and who delivered the people from the oppression\fnote{\fbackref{18:10} Lit. \fbib{from under the hand of}} of the Egyptians. \v{11}Now I know that the \divine{Lord} is greater than all other gods,\fnote{\fbackref{18:11} Lit. \fbib{all the gods}} because of what happened to\fnote{\fbackref{18:11} Lit. \fbib{the matter in which}} the Egyptians when\fnote{\fbackref{18:11} The Heb. lacks \fbib{when}} they acted arrogantly against Israel.'' \v{12}Jethro, Moses' father-in-law, brought a burnt offering and sacrifices for God, and Aaron and all the elders of Israel came to dine with Moses' father-in-law in the presence of God.
\passage{Jethro Advises Moses to Appoint Judges}

\v{13}The next day Moses sat down to judge the people, and the people stood around Moses from morning until evening. \v{14}When Moses' father-in-law saw all that he was doing for the people, he said, ``What is this that you are doing for the people? Why do you alone sit as judge,\fnote{\fbackref{18:14} The Heb. lacks \fbib{as judge}} with all the people standing around you from morning until evening?''

\v{15}Moses told his father-in-law, ``Because the people come to me to seek God's will.\fnote{\fbackref{18:15} Lit. \fbib{to inquire of God}} \v{16}When they have a dispute, it comes to me and I decide between a person and his neighbor, and make known the statutes of God and his instructions.''

\v{17}Moses' father-in-law told him, ``What you are doing is not good. \v{18}You will certainly wear yourself out, both you and these people who are with you, because the task is too heavy for you. You cannot do it by yourself. \v{19}Now listen to me. I'll advise you, and may God be with you. You are to represent the people before God and bring the disputes to God. \v{20}You are to teach them the statutes and instructions and make known to them the way they're to go and the things they're to do. \v{21}You are to look for capable men among the people, men who fear God, men of integrity who hate dishonest gain. You are to set these men over them as officials over thousands, hundreds,\fnote{\fbackref{18:21} Lit. \fbib{officials over hundreds}} fifties,\fnote{\fbackref{18:21} Lit. \fbib{officials over fifties}} and tens.\fnote{\fbackref{18:21} Lit. \fbib{officials over tens}} \v{22}They are to judge the people at all times. Let them bring every major matter to you, but let them judge every minor matter. It will lighten your burden, and they'll bear it with you. \v{23}If you do this,\fnote{\fbackref{18:23} Lit. \fbib{this thing}} and God so commands you, you will be able to stand the strain,\fnote{\fbackref{18:23} Lit. \fbib{stand}} and all these people will also go to their homes in peace.''

\v{24}Moses listened to his father-in-law and did everything he said. \v{25}Moses chose capable men from all Israel and appointed them as heads over the people, as officials over thousands, hundreds,\fnote{\fbackref{18:25} Lit. \fbib{officials over hundreds}} fifties,\fnote{\fbackref{18:25} Lit. \fbib{officials over fifties}} and tens.\fnote{\fbackref{18:25} Lit. \fbib{officials over tens}} \v{26}They judged the people at all times; the difficult matters they brought to Moses, but every minor matter they judged. \v{27}Moses sent his father-in-law on his way, and he went to his own land.
\labelchapt{19}
\passage{The Israelis Reach Mount Sinai}

\chapt{19}
\v{1}On the third New Moon after the Israelis went out of the land of Egypt, on that very day,\fnote{\fbackref{19:1} Lit. \fbib{on this day}} they came to the desert of Sinai. \v{2}They had set out from Rephidim and arrived at the desert of Sinai where they camped in the desert. Israel camped there in front of the mountain. \v{3}Then Moses went up to God, and the \divine{Lord} called to him from the mountain: ``This is what you are to say to the house of Jacob and declare to the sons of Israel, \v{4}`You saw what I did to the Egyptians, and how I carried you on eagles' wings and brought you to myself. \v{5}And now if you carefully obey me and keep my covenant, you are to be my special possession out of all the nations,\fnote{\fbackref{19:5} Lit. \fbib{peoples}} because the whole earth belongs to me, \v{6}but you are to be a kingdom of priests and a holy nation to me.' These are the words you are to declare to the Israelis.''

\v{7}When Moses came, he summoned the elders of the people and told them everything that the \divine{Lord} had commanded him. \v{8}All the people answered together: ``We'll do everything that the \divine{Lord} has said!''

Then Moses reported all the words of the people back to the \divine{Lord}. \v{9}The \divine{Lord} told Moses, ``Look, I'm coming to you in a thick cloud, so that the people may listen when I speak with you and always believe you.'' Moses reported the words of the people to the \divine{Lord}.
\passage{Preparation for the Covenant}

\v{10}The \divine{Lord} told Moses, ``Go to the people and consecrate them today and tomorrow. They must wash their clothes, \v{11}and be ready for the third day, for on the third day the \divine{Lord} will come down on Mount Sinai in the sight of all the people. \v{12}You are to set boundaries for the people all around: `Be very careful that you don't go up on the mountain or touch the side of it. Anyone who touches the mountain is certainly to be put to death. \v{13}No hand is to touch that person,\fnote{\fbackref{19:13} Lit. \fbib{him}} but he is certainly to be stoned or shot;\fnote{\fbackref{19:13} I.e. shot with arrows} whether animal or person, he is not to live.' They are to approach\fnote{\fbackref{19:13} Or \fbib{go up to}} the mountain only when the ram's horn sounds a long blast.''

\v{14}When Moses went down from the mountain to the people, he consecrated the people, and they washed their clothes. \v{15}He told the people, ``Be ready for the third day; don't go near a woman.''\fnote{\fbackref{19:15} I.e. to have sexual relations}
\passage{The \divine{Lord} Appears on Mount Sinai}

\v{16}When morning came on the third day, there was thunder and lightning, with a heavy cloud over the mountain, and the very loud sound of a ram's horn. All the people in the camp trembled. \v{17}Moses brought the people from the camp to meet God, and they stood at the base of the mountain. \v{18}Mount Sinai was completely enveloped in smoke because the \divine{Lord} had come down in fire on it. Smoke went up from it like smoke from a kiln, and the whole mountain shook violently. \v{19}As the sound of the ram's horn grew louder and louder, Moses would speak and God would answer with thunder.\fnote{\fbackref{19:19} Or \fbib{in a voice}} \v{20}When the \divine{Lord} came down on Mount Sinai to the top of the mountain, he\fnote{\fbackref{19:20} Lit. \fbib{the \divine{Lord}}} summoned Moses to the top of the mountain, and Moses went up.

\v{21}The \divine{Lord} told Moses, ``Go down and warn the people so they don't break through to look at the \divine{Lord}, and many of them perish.\fnote{\fbackref{19:21} Lit. \fbib{fall}} \v{22}Even the priests who approach the \divine{Lord} must consecrate themselves. Otherwise, the \divine{Lord} will attack them.''

\v{23}Moses told the \divine{Lord}, ``The people cannot come up to Mount Sinai because you warned us: `Set boundaries around the mountain and consecrate it.'\,''\fnote{\fbackref{19:23} I.e. set it apart as holy}

\v{24}The \divine{Lord} told him, ``Go down, and come back up with Aaron, but the priests and the people must not break through to go up to the \divine{Lord}. Otherwise, he will attack them.'' \v{25}So Moses went down to the people and spoke to them.
\labelchapt{20}
\passage{The Ten Commandments}
\passageinfo{(Deuteronomy 5:1-21)}

\chapt{20}
\v{1}Then God spoke all these words:
\begin{bulletlist}
\itemb{\heb{'}\fnote{\fbackref{20:2-17} The Heb. letters to the left denote numbers 1-10}} \v{2}``I am the \divine{Lord} your God, who brought you out of the land of Egypt--- from the house of slavery. \v{3}You are to have no other gods as a substitute for me.\fnote{\fbackref{20:3} Lit. \fbib{gods besides me}}
\itemb{\heb{b}} \v{4}``You are not to craft for yourselves an idol or anything resembling what is in the skies above, or on earth beneath, or in the water sources under the earth. \v{5}You are not to bow down to them in worship or serve them, because I, the \divine{Lord} your God, am a jealous God, visiting the guilt of parents\fnote{\fbackref{20:5} Lit. \fbib{fathers}} on children, to the third and fourth generation\fnote{\fbackref{20:5} So LXX. The Heb. lacks \fbib{generation}} of those who hate me, \v{6}but showing gracious love to the thousands of those who love me and keep my commandments.
\itemb{\heb{g}} \v{7}``You are not to misuse the name of the \divine{Lord} your God,\fnote{\fbackref{20:7} Lit. \fbib{to take in vain the name of the \divine{Lord} your God}; i.e. for a worthless purpose} because the \divine{Lord} will not leave unpunished the one who misuses his name.\fnote{\fbackref{20:7} Lit. \fbib{who takes his name in vain} i.e. for a worthless purpose}
\itemb{\heb{d}} \v{8}``Remember the Sabbath day, maintaining its holiness.\fnote{\fbackref{20:8} Lit. \fbib{day as holy;} i.e. to set apart the day as holy} \v{9}Six days you are to labor and do all your work, \v{10}but the seventh day is a Sabbath to the \divine{Lord} your God. You are not to do any work---neither you, nor your son, nor your daughter, nor your male or female servant, nor your livestock, nor any foreigner who lives among you---\fnote{\fbackref{20:11} Lit. \fbib{lives within your gates}} \v{11}because the \divine{Lord} made the heavens, the earth, the sea, and everything that is in them in six days. Then he rested on the seventh day. Therefore, the \divine{Lord} blessed the Sabbath day and made it holy.
\itemb{\heb{h}} \v{12}``Honor your father and your mother, so that you may live long in the land that the \divine{Lord} your God is giving you.
\itemb{\heb{w}} \v{13}``You are not to commit murder.
\itemb{\heb{z}} \v{14}``You are not to commit adultery.
\itemb{\heb{.h}} \v{15}``You are not to steal.
\itemb{\heb{.t}} \v{16}``You are not to give false testimony against your neighbor.
\itemb{\heb{y}} \v{17}``You are not to desire\fnote{\fbackref{20:17} Lit. \fbib{to covet}; i.e. to set your heart on} your neighbor's house,\fnote{\fbackref{20:17} Or \fbib{neighbor's family dynasty}} nor your neighbor's wife, his male or female servant, his ox, his donkey, nor anything else that pertains to your neighbor.''
\end{bulletlist}
\passage{The People are Terrified in God's Presence}

\v{18}All the people experienced the thunder and lightning, the sound of the ram's horn, and the smoking mountain. And as the people experienced it, they trembled and stood at a distance. \v{19}They told Moses, ``You speak to us and we will listen, but don't let God speak with us, or we may die.

\v{20}Moses told the people, ``Don't be afraid, for God has come to test you, so that you may fear him in order that you don't sin.'' \v{21}Then the people stood at a distance, and Moses approached the thick cloud where God was.
\passage{Instruction about Idols and Altars}

\v{22}The \divine{Lord} told Moses, ``This is what you are to say to the Israelis, `You have seen for yourselves that I spoke to you from heaven. \v{23}You are not to make gods of silver alongside me, nor are you to make for yourselves gods of gold. \v{24}You are to make an altar of earth for me, and you are to sacrifice on it your burnt offerings and peace offerings, your sheep, and your cattle. Everywhere I cause my name to be remembered, I'll come to you and bless you. \v{25}If you make an altar of stone for me, you must not build it of cut stones, because if you strike it with your chisel, you will profane it. \v{26}You are not to ascend to my altar on steps, so that your nakedness may not be exposed on it.'\,''
\labelchapt{21}
\passage{Laws Concerning Servants}

\chapt{21}
\v{1}``These are the ordinances that you are to set before them.

\v{2}``When you acquire a Hebrew servant, he is to serve for six years, and in the seventh he is to go out a free man without paying anything. \v{3}If he came in by himself,\fnote{\fbackref{21:3} Lit. \fbib{with his body}; i.e. single, and so throughout the chapter} he is to go out by himself. If he was married, his wife is to go out with him. \v{4}If his master gives him a wife and she bears him sons or daughters, the wife and children belong to her master, and he is to go out by himself. \v{5}But if the servant, in fact, says, `I love my master, my wife, and my children, and I won't go out a free man,' \v{6}then his master is to bring him before the judges\fnote{\fbackref{21:6} Or \fbib{before God}} and he is to bring him to the door or to the doorpost. His master is to pierce his ear with an awl, and he is to serve him permanently.

\v{7}``When a man sells his daughter as a servant, she won't go out as the male servants do.\fnote{\fbackref{21:7} The Heb. lacks \fbib{as the male servants do}} \v{8}If she's displeasing to\fnote{\fbackref{21:8} Lit. \fbib{bad in the eyes of}} her master who selected her for himself,\fnote{\fbackref{21:8} I.e. as a secondary wife also called a mistress or concubine} he must let her be redeemed. He does not have the right to sell her to foreign people, because he has dealt unfairly\fnote{\fbackref{21:8} Or \fbib{treacherously}} with her. \v{9}If he has selected her for his son,\fnote{\fbackref{21:9} I.e. as a secondary wife also called a mistress or concubine} he is to treat her according to the ordinance for daughters. \v{10}If he takes another woman for himself, he may not withhold from the first\fnote{\fbackref{21:10} The Heb. lacks \fbib{from the first}} her food, her clothing, or her marital rights. \v{11}If he does not do these three things for her, she may go out without paying anything at all.''\fnote{\fbackref{21:11} The Heb. lacks \fbib{at all}}
\passage{Laws Concerning Personal Injury and Homicide}

\v{12}``Whoever strikes a man so that he dies is certainly to be put to death. \v{13}If he didn't lie in wait, but God let him fall into his reach,\fnote{\fbackref{21:13} Lit. \fbib{hand}; i.e. \fbib{the event was not premeditated by the accused}} then I'll appoint for you a place to which he may flee. \v{14}If a man acts deliberately against his neighbor, to kill him by treachery, you are to take him to die even if he's at\fnote{\fbackref{21:14} Lit. \fbib{even from}} my altar.

\v{15}``Whoever strikes his father or his mother is certainly to be put to death.

\v{16}``Whoever kidnaps a person, whether he has sold him or whether the victim\fnote{\fbackref{21:16} Lit. \fbib{he}} is still in his possession, is certainly to be put to death.

\v{17}``Whoever curses his father or his mother is certainly to be put to death.

\v{18}``If people quarrel and one strikes the other with a rock or his fist, and he does not die but ends up\fnote{\fbackref{21:18} Lit. \fbib{falls}} in bed, \v{19}and the injured person\fnote{\fbackref{21:19} Lit. \fbib{he}} then gets up and walks around outside with the help of his staff,\fnote{\fbackref{21:19} Lit. \fbib{with his staff}} the one who struck him is not liable, except that he is to compensate him for his loss of time\fnote{\fbackref{21:19} Lit. \fbib{his rest}} and take care of his complete recovery.

\v{20}``If a man strikes his male or female servant with a stick and he or she dies as a direct result,\fnote{\fbackref{21:20} Lit. \fbib{under his hand}} the master must be punished.\fnote{\fbackref{21:20} Lit. \fbib{suffer vengeance}} \v{21}But if the servant\fnote{\fbackref{21:21} Lit. \fbib{he}} survives a day or two, the master\fnote{\fbackref{21:21} Lit. \fbib{he}} is not to be punished because the servant\fnote{\fbackref{21:21} Lit. \fbib{he}} is his property.

\v{22}``If two men are fighting and they strike a pregnant woman and her children are born prematurely,\fnote{\fbackref{21:22} Lit. \fbib{children come out}} but there is no harm, he is certainly to be fined as the husband of the woman demands of him, and he will pay as the court decides.\fnote{\fbackref{21:22} Or \fbib{according to the assessment}} \v{23}If there is harm, then you are to require\fnote{\fbackref{21:23} Lit. \fbib{give}} life for life, \v{24}eye for eye, tooth for tooth, hand for hand, foot for foot, \v{25}burn for burn, wound for wound, and bruise for bruise.

\v{26}``If a man strikes the eye of his male or female servant and destroys it, he is to release him as a free man in exchange for his eye. \v{27}If he knocks out the tooth of his male\fnote{\fbackref{21:27} Lit. \fbib{male servant}} or female servant,\fnote{\fbackref{21:27} Lit. \fbib{tooth of his female servant}} he is to release him as a free man in exchange for his tooth.

\v{28}``If an ox gores a man or woman so that they die, the ox is certainly to be stoned and its flesh may not be eaten, but the owner of the ox is free from liability. \v{29}But if the ox has gored previously, and its owner has been warned about it but didn't restrain it, and it kills a man or woman, the ox is to be stoned and its owner also is to be put to death. \v{30}If a fine is imposed on him, he may pay all that was imposed on him as a ransom for his life. \v{31}This same ordinance applies\fnote{\fbackref{21:31} Lit. \fbib{It shall be done to him according to this ordinance}} if it gores a son or daughter.

\v{32}``If the ox gores a male or female servant, the owner is to give 30 shekels\fnote{\fbackref{21:32} I.e., a unit of weight equal to about 16 barley grains; about 0.025 ounces or 0.5 grams; cf. Num 3:47; Num 18:16} of silver to the servant's\fnote{\fbackref{21:32} Lit. \fbib{his}} master, and the ox is to be stoned. \v{33}If a man opens a pit or digs a pit and does not cover it, and an ox or donkey falls into it,\fnote{\fbackref{21:33} Lit. \fbib{there}} \v{34}the owner of the pit is to make restitution. He is to pay money to its owner, and the dead animal will become his.

\v{35}``If a man's ox strikes his neighbor's ox and it dies, they are to sell the live ox and divide the money. They also are to divide the dead animal. \v{36}But if it was known that the ox had gored previously, and its owner didn't restrain it, he is certainly to repay ox for ox, and the dead ox is to become his.''
\labelchapt{22}
\passage{Laws Concerning Theft of Personal Property}

\chapt{22}
\v{1}\fnote{\fbackref{22:1} This verse is 21:37 in MT}``If a man steals an ox or sheep and slaughters it or sells it, he is to repay five oxen\fnote{\fbackref{22:1} Or \fbib{cattle}} for the ox and four sheep for the sheep.

\v{2}``If a thief is found while breaking into a house,\fnote{\fbackref{22:2} Lit. \fbib{while breaking in}} and is struck down and dies, it is not a capital crime\fnote{\fbackref{22:2} Lit. \fbib{dies, there is no bloodguilt}} in that case,\fnote{\fbackref{22:2} Lit. \fbib{for him} or \fbib{for it}} \v{3}but if the sun has risen on him, then it is a capital crime\fnote{\fbackref{22:3} Lit. \fbib{then there is bloodguilt}} in that case.\fnote{\fbackref{22:3} Lit. \fbib{for him} or \fbib{for it}} A thief\fnote{\fbackref{22:3} Lit. \fbib{He}} is certainly to make restitution, but if he has nothing, he is to be sold\fnote{\fbackref{22:3} I.e. sold into slavery} for his theft. \v{4}If what was stolen is actually found alive in his possession, whether an ox, a donkey or a sheep, he is to repay double.

\v{5}``When a man lets a field or vineyard be grazed over or releases his livestock so that they graze in another man's field, he is to make restitution from the best of his field or vineyard.\fnote{\fbackref{22:5} Lit. \fbib{or the best of his vineyard}}

\v{6}``When a fire breaks out and spreads into thorn bushes and consumes stacked grain or standing grain or the field, the one who started the fire certainly is to make restitution.

\v{7}``When a man gives his neighbor money or goods for safekeeping and it's stolen from the neighbor's house, the thief, if found, is to repay double. \v{8}If the thief is not found, the owner of the house is to appear before the judges\fnote{\fbackref{22:8} Or \fbib{God}} to see\fnote{\fbackref{22:8} The Heb. lacks \fbib{to see}} whether or not the thief took\fnote{\fbackref{22:8} Lit. \fbib{not he laid his hands on}} his neighbor's property.

\v{9}``In every ownership dispute\fnote{\fbackref{22:9} Lit. \fbib{matter of transgression}} involving an ox, donkey, sheep, garment, or anything that is lost where a person says, `This is mine,'\fnote{\fbackref{22:9} Lit. \fbib{This is it}} the case between the two of them is to come before the judges,\fnote{\fbackref{22:9} Or \fbib{God}} and the one that the judges\fnote{\fbackref{22:9} Or \fbib{God}} declare guilty is to repay double to his neighbor.

\v{10}``When a man gives a donkey, ox, sheep, or any animal to his neighbor for safe keeping, and it dies or is injured or is driven away when no one is looking, \v{11}the two of them are to take an oath in the \divine{Lord}'s presence that the accused\fnote{\fbackref{22:11} Lit. \fbib{that he}} has not taken\fnote{\fbackref{22:11} Lit. \fbib{not laid his hands on}} his neighbor's property. Its owner is to accept this, and the neighbor\fnote{\fbackref{22:11} Lit. \fbib{he}} is not to make restitution. \v{12}But if it was actually stolen from him, the neighbor\fnote{\fbackref{22:12} Lit. \fbib{he}} is to make restitution to its owner. \v{13}If it was torn to pieces, let the neighbor\fnote{\fbackref{22:13} Lit. \fbib{him}} bring the remains\fnote{\fbackref{22:13} Lit. \fbib{bring it}} as evidence, and he is not to make restitution for what was torn apart.

\v{14}``When a man borrows\fnote{\fbackref{22:14} Lit. \fbib{asks}} an animal from his neighbor, and it's injured or dies while its owner was not with it, he is certainly to make restitution. \v{15}If its owner was with it, he is not to make restitution. If it was hired, its fee covers the loss.''\fnote{\fbackref{22:15} Lit. \fbib{its fee comes}; i.e. the fee compensates the owner for the loss}
\passage{Various Other Laws}

\v{16}``When a man seduces a virgin who is not engaged to be married and has sexual relations with her, he must pay her bride price, and she is to become his wife. \v{17}If her father absolutely refuses to give her to him, he is to pay an amount\fnote{\fbackref{22:17} Lit. \fbib{silver}} equal to the bride price for virgins.

\v{18}``You are not to allow a sorceress to live.

\v{19}``Whoever has sexual relations with an animal is certainly to be put to death.

\v{20}``Anyone who sacrifices to a god, except the \divine{Lord} alone, is to be utterly destroyed.

\v{21}``You are not to wrong or oppress an alien, because you were aliens in the land of Egypt.

\v{22}``You are not to mistreat any widow or orphan. \v{23}If you do mistreat them, they'll certainly cry out to me, and I'll immediately hear their cry. \v{24}And I'll be angry and will kill you with swords,\fnote{\fbackref{22:24} I.e. using invasions by foreign armies} and your wives will become widows and your children orphans.

\v{25}``If you loan money to my people, to the poor among you, don't be like a creditor to them and don't impose interest on them. \v{26}If you take your neighbor's coat as collateral, you are to return it to him by sunset, \v{27}for it's his only covering; it's his outer garment,\fnote{\fbackref{22:27} Lit. \fbib{his coat for his skin}} for what else can he sleep in? And when he cries out to me, I'll hear him, for I am gracious.

\v{28}``You are not to blaspheme God or curse a ruler of your people.

\v{29}``You are not to hold back the fullness of your harvest\fnote{\fbackref{22:29} Lit. \fbib{your fullness}} and the outflow of your wine presses.\fnote{\fbackref{22:29} Lit. \fbib{your outflow}} You are to give to me the firstborn of your sons. \v{30}You are to do the same with your oxen and your sheep. They are to be with their mother for seven days and then on the eighth day you are to give them to me.

\v{31}``You are to be people set apart\fnote{\fbackref{22:31} Lit. \fbib{holy}} for me. You are not to eat flesh torn apart in the field; you are to throw it to the dogs.''
\labelchapt{23}
\passage{Laws about Truthful Testimony}

\chapt{23}
\v{1}``You are not to spread a false report, nor are you to join forces\fnote{\fbackref{23:1} Lit. \fbib{set your hand}} with the wicked to be a malicious witness. \v{2}You are not to follow the majority\fnote{\fbackref{23:2} Or \fbib{many}} in doing wrong, and you are not to testify in a lawsuit so as to follow the majority and pervert justice. \v{3}You are not to show partiality to a poor man in his lawsuit.

\v{4}``If you come across your enemy's ox or donkey wandering off, you are to certainly return it to him. \v{5}If you see your enemy's donkey lying helpless under its load, you must not abandon it; rather, you are certainly to return it to him.\fnote{\fbackref{23:5} Lit. \fbib{leave it with him}}

\v{6}``You are not to pervert justice for the poor among you\fnote{\fbackref{23:6} Lit. \fbib{your poor}} in their lawsuits.\fnote{\fbackref{23:6} Lit. \fbib{in his lawsuit}} \v{7}Stay far away from a false charge, and don't kill the innocent or the righteous, because I won't acquit the guilty. \v{8}You are not to take a bribe because a bribe blinds the clear-sighted and distorts the words of the righteous. \v{9}You are not to oppress the resident alien, because you were aliens in the land of Egypt.''
\passage{Instructions for Sabbaths and Sabbatical Years}

\v{10}``You are to sow your land and gather its crops for six years, \v{11}but you are to let it rest the seventh year, leaving it unplanted. The poor of your people may eat from it,\fnote{\fbackref{23:11} Lit. \fbib{shall eat}} and the wild animals may eat what they leave. You are to do the same with your vineyards and olive groves. \v{12}You are to do your work for six days, but on the seventh day you are to refrain from work so that your ox and donkey\fnote{\fbackref{23:12} Lit. \fbib{your donkey}} may rest, and so the son of your maidservant and the alien may be refreshed.

\v{13}``Be careful about everything I've told you, and don't mention the name of other gods. Don't let them be heard in your mouth!''
\passage{The Three Major Festivals}

\v{14}``Three times a year you are to celebrate a festival for me. \v{15}You are to observe the Festival of Unleavened Bread. As I commanded you, you are to eat unleavened bread for seven days at the appointed time in the month Abib, because in it you came out of Egypt. No one is to appear before me empty handed. \v{16}You are to observe\fnote{\fbackref{23:16} The Heb. lacks \fbib{You are to observe}} the Festival of Harvest,\fnote{\fbackref{23:16} I.e. the Festival of Weeks or Pentecost} celebrating\fnote{\fbackref{23:16} Lit. \fbib{of}} the first fruits of your work in planting the field, and the Festival of Ingathering\fnote{\fbackref{23:16} Also known as the Festival of Tents} at the end of the year, when you gather the fruit of your work from the field. \v{17}Three times a year all your males are to appear in the presence of the Lord \divine{God}.''
\passage{Various Laws}

\v{18}``You are not to offer the blood of my sacrifice with anything leavened, and you are not to let the fat portion of my sacrifice remain overnight until morning.

\v{19}``You are to bring the best of the first fruits of your soil to the house of the \divine{Lord} your God.

``You are not to boil a young goat in its mother's milk.''
\passage{God Promises Help as the Israelis Enter Canaan}

\v{20}``Look, I'm sending an angel in front of you to guard you on the way and to bring you to the place I've prepared. \v{21}Be careful! Be sure to obey him. Don't rebel against him, because he won't forgive your transgression, since my Name is in him. \v{22}Indeed, if you carefully obey him and do everything that I say, then I'll be an enemy to your enemies and an adversary to your adversaries, \v{23}because my angel will go ahead of you and will bring you to the Amorites, the Hittites, the Perizzites, the Canaanites, the Hivites, and the Jebusites, and I'll annihilate them. \v{24}You are not to bow down to their gods or serve them. You are not to follow their practices,\fnote{\fbackref{23:24} Lit. \fbib{do their deeds}} but you are to overthrow them completely and smash their sacred stones\fnote{\fbackref{23:24} Or \fbib{pillars}} to pieces. \v{25}You are to serve the \divine{Lord} your God, and he will bless your food\fnote{\fbackref{23:25} Or \fbib{bread}} and water, and I'll remove sickness from you. \v{26}No woman will miscarry or be barren in your land, and I'll make every day of your life complete.\fnote{\fbackref{23:26} Lit. \fbib{make the number of your days full}}

\v{27}``I'll go ahead of you and terrorize all the people to whom you are coming. I'll confuse your enemies and make them turn their backs on you and run away.\fnote{\fbackref{23:27} The Heb. lacks \fbib{and run away}} \v{28}I'll send hornets ahead of you and they'll drive out the Hivites, the Canaanites, and the Hittites before you. \v{29}I won't drive them out before you in a single year, so that the land does not become desolate and so that wild animals do not overrun you. \v{30}I'll drive them out ahead of you little by little until you increase in numbers\fnote{\fbackref{23:30} Lit. \fbib{you are fruitful}} and possess the land.

\v{31}``I'll set your borders from the Reed\fnote{\fbackref{23:31} So MT; LXX reads \fbib{Red}} Sea to the Sea of the Philistines,\fnote{\fbackref{23:31} I.e. Mediterranean Sea} and from the desert to the River,\fnote{\fbackref{23:31} MT does not identify the river} bringing\fnote{\fbackref{23:31} Lit. \fbib{giving}} the inhabitants of the land under your control,\fnote{\fbackref{23:31} Lit. \fbib{into your hand}} and you are to drive them out ahead of you. \v{32}You are not to make a covenant with them or with their gods. \v{33}They are not to live in your land. Otherwise they will cause you to sin against me. If you worship their gods, it will become a snare for you.''
\labelchapt{24}
\passage{The Covenant is Sealed with Blood}

\chapt{24}
\v{1}The \divine{Lord}\fnote{\fbackref{24:1} Lit. \fbib{He}} told Moses, ``Come up to the \divine{Lord}, you and Aaron, Nadab and Abihu, and 70 of the elders of Israel, and worship\fnote{\fbackref{24:1} Or \fbib{bow down in worship}} at a distance. \v{2}Only Moses is to approach the \divine{Lord}, but the others\fnote{\fbackref{24:2} Lit. \fbib{but they}} are not to approach; the people are not to come up with him.''

\v{3}Then Moses came and reported all the words of the \divine{Lord} and all the statutes to the people, and they all\fnote{\fbackref{24:3} Lit. \fbib{all the people}} answered with one voice, ``We will do everything that the \divine{Lord} has decreed.''

\v{4}So Moses wrote down all the words of the \divine{Lord}. He got up early in the morning and built an altar with twelve pillars for the twelve tribes of Israel at the base of the mountain. \v{5}He sent young Israeli men to offer up burnt offerings and sacrifice bulls as peace offerings to the \divine{Lord}. \v{6}Moses took half the blood and put it in bowls, while he sprinkled the other half\fnote{\fbackref{24:6} Lit. \fbib{half of the blood}} on the altar. \v{7}He took the Book of the Covenant and read it to\fnote{\fbackref{24:7} Lit. \fbib{in the ears of}} the people. They said, ``We will put into practice and obey everything that the \divine{Lord} has decreed.''

\v{8}Moses took the blood, sprinkled it on the people, and said, ``This is the blood of the covenant that the \divine{Lord} made with you based on all these words.''

\v{9}Then Moses and Aaron, Nadab and Abihu, and 70 of the elders of Israel went up \v{10}and saw the God of Israel. Under his feet was something like a pavement made of sapphire, as clear as the sky. \v{11}Because\fnote{\fbackref{24:11} Lit. \fbib{But}} God\fnote{\fbackref{24:11} Lit. \fbib{He}} did not punish\fnote{\fbackref{24:11} Lit. \fbib{not send forth his hand against}} the Israeli leaders, they looked at God, yet lived\fnote{\fbackref{24:11} The Heb. lacks \fbib{lived}} to eat and drink.
\passage{Moses Receives the Law on the Mountain}

\v{12}Then the \divine{Lord} told Moses, ``Go up to me on the mountain and stay\fnote{\fbackref{24:12} Lit. \fbib{be}} there. I'll give you stone tablets with the instruction and law that I've written to teach the people.''\fnote{\fbackref{24:12} Lit. \fbib{them}}

\v{13}So Moses got up, along with Joshua his servant, and went up on the mountain of God. \v{14}He told the elders, ``Wait here for us until we return to you. Look, Aaron and Hur are with you, and whoever has a dispute, let him come to them.''

\v{15}When Moses went up on the mountain, the cloud covered it. \v{16}The glory of the \divine{Lord} settled on Mount Sinai, and the cloud covered it for six days. Then on the seventh day he called to Moses from within the cloud. \v{17}To the Israelis\fnote{\fbackref{24:17} Lit. \fbib{in the sight of}} the appearance of the glory of the \divine{Lord} was like a consuming fire on top of the mountain. \v{18}When Moses went up on the mountain, he went into the center of the cloud and was on the mountain for 40 days and 40 nights.
\labelchapt{25}
\passage{An Offering for the Sanctuary}

\chapt{25}
\v{1}The \divine{Lord} told Moses, \v{2}``Tell the Israelis to take an offering for me, and you are to accept my offering from every person whose heart moves him to give.\fnote{\fbackref{25:2} Lit. \fbib{moves him}} \v{3}This is the offering that you are to accept from them: gold, silver, and bronze; \v{4}blue, purple, and scarlet material;\fnote{\fbackref{25:4} The Heb. lacks \fbib{material}; and so throughout the book} fine linen and goat hair; \v{5}ram skins dyed\fnote{\fbackref{25:5} Or \fbib{tanned}} red, dolphin\fnote{\fbackref{25:5} Or \fbib{dugong,} a marine animal similar to a walrus or manatee} skins, and acacia wood; \v{6}oil for lighting, spices for the anointing oil and for aromatic incense; \v{7}and onyx stones, stones for setting on the ephod and the breast piece.\fnote{\fbackref{25:7} Or \fbib{the pouch on the breast piece}} \v{8}Let them make a sanctuary for me so I may live among them. \v{9}This is how you are to make it: according to all that I'm showing you, according to the pattern for the tent and the pattern for all its furnishings.''
\passage{The Ark of the Covenant}

\v{10}``They are to make an ark of acacia wood, two and a half cubits\fnote{\fbackref{25:10} I.e. about 45 inches} long, one and a half cubits\fnote{\fbackref{25:10} I.e. about 27 inches} wide, and one and a half cubits\fnote{\fbackref{25:10} I.e. about 27 inches} high. \v{11}You are to overlay it with pure gold---you are to overlay it inside and outside---and you are to make a gold molding around it. \v{12}You are to cast four rings for it and put them on its four feet, two rings on one side of it and two rings on its other side. \v{13}You are to make poles of acacia wood and overlay them with gold. \v{14}You are to put the poles into the rings on the sides of the ark with which to carry it.\fnote{\fbackref{25:14} The Heb. lacks \fbib{it}} \v{15}The poles are to remain in the rings of the ark and are not to be removed from it. \v{16}You are to put the Testimony\fnote{\fbackref{25:16} I.e. the tablets on which the ten commandments were written and which were placed in the Ark of the Covenant; and so through chapter 31.} that I will give you into the ark.

\v{17}``You are to make a Mercy Seat\fnote{\fbackref{25:17} Or \fbib{atonement cover}; and so through chapter 31} of pure gold, two and a half cubits\fnote{\fbackref{25:17} I.e. about 45 inches} long and one and a half cubits\fnote{\fbackref{25:17} I.e. about 27 inches} wide. \v{18}You are to make two cherubim\fnote{\fbackref{25:18} I.e. representations of certain angelic beings} of gold; you are to make them of hammered work at the two ends of the Mercy Seat. \v{19}Place one cherub at one end and one cherub at the other end. You are to make the cherubim at the two ends of the Mercy Seat, and of one piece with it. \v{20}The cherubim are to spread their wings upward, covering the Mercy Seat with their wings and facing each other. The faces of the cherubim is to be turned toward the Mercy Seat. \v{21}You are to put the Mercy Seat on top of the ark, and put the Testimony that I'll give you into the ark. \v{22}I'll meet with you there, and I'll tell you all my commandments\fnote{\fbackref{25:22} Lit. \fbib{all that I have commanded you}} for the Israelis from above the Mercy Seat, from between the two cherubim that are on the Ark of the Testimony.''
\passage{The Table of Showbread}

\v{23}``You are to make a table of acacia wood, two cubits\fnote{\fbackref{25:23} I.e. about 36 inches} long, a cubit\fnote{\fbackref{25:23} I.e. about 18 inches} wide, and one and a half cubits\fnote{\fbackref{25:23} I.e. about 45 inches} high. \v{24}You are to overlay it with pure gold, and put a gold molding around it. \v{25}You are to make a rim one handbreadth in width\fnote{\fbackref{25:25} I.e. about 4 inches} around it, and you are to make a gold molding around the rim. \v{26}You are to make four gold rings for it, and put the rings on the four corners where its four feet are. \v{27}The rings are to be close to the rim as holders for the poles to carry the table. \v{28}You are to make the poles of acacia wood, and overlay them with gold so the table can be carried with them. \v{29}You are to make its plates, dishes, jars, and bowls from which libations will be poured, and you are to make them of pure gold. \v{30}You are to put the bread of the Presence on the table before me continuously.''
\passage{The Lamp Stand}
\passageinfo{(Numbers 3:1-10)}

\v{31}``You are to make a lamp stand of pure gold: the lamp stand and its base and stem are to be of hammered work, and its cups, calyxes,\fnote{\fbackref{25:31} Or \fbib{buds}; i.e. the round base at the bottom of a flower; and so through chapter 31} and flowers are to be of one piece with it. \v{32}Six branches are to extend from its sides, three branches of the lamp stand from one side of it and three branches of the lamp stand from its other side. \v{33}Three cups shaped like almond blossoms with calyxes and flowers are to be on one branch and three cups shaped like almond blossoms with calyxes and flowers are to be on the other branch, and so for the six branches extending from the lamp stand.

\v{34}``On the lamp stand itself there are to be four cups shaped like almond blossoms with their calyxes and flowers. \v{35}A calyx\fnote{\fbackref{25:35} Or \fbib{bud}; i.e. the round base at the bottom of a flower; and so through chapter 31} is to be under the two branches that extend out of the stem;\fnote{\fbackref{25:35} Lit. \fbib{out of it}} a calyx is to be under the next pair of branches\fnote{\fbackref{25:35} Lit. \fbib{under the two branches}} that extend out of the stem;\fnote{\fbackref{25:35} Lit. \fbib{out of it}} and a calyx is to be under the last pair of branches\fnote{\fbackref{25:35} Lit. \fbib{under the two branches}} that extend out of the stem,\fnote{\fbackref{25:35} Lit. \fbib{out of it}} and so for the six\fnote{\fbackref{25:35} The Heb. lacks \fbib{six}} branches extending from the lamp stand. \v{36}Their calyxes and their branches are to be of one piece with it; all of it is to be made of one piece of hammered work of pure gold.

\v{37}``You are to make seven lamps for it, and its lamps are to be mounted so as to give light in front of it. \v{38}Its tongs and trays are to be of pure gold. \v{39}The lamp stand\fnote{\fbackref{25:39} Lit. \fbib{It}}---together with all its furnishings---is to be made from a talent\fnote{\fbackref{25:39} I.e. about 75 pounds} of pure gold. \v{40}Now see that you make them according to the pattern for them which you are being shown on the mountain.''
\labelchapt{26}
\passage{The Tent}

\chapt{26}
\v{1}``You are to make the tent with ten curtains of fine woven\fnote{\fbackref{26:1} Or \fbib{twisted}; and so through chapter 31} linen and with blue, purple, and scarlet material. You are to make them with cherubim skillfully worked into them. \v{2}The length of each curtain is to be 28 cubits,\fnote{\fbackref{26:2} I.e. about 42 feet} the width of each curtain four cubits,\fnote{\fbackref{26:2} I.e. about six feet} and all the curtains are to have the same measurements.\fnote{\fbackref{26:2} Lit. \fbib{and the measure of one shall be for every curtain}}

\v{3}``Five of the curtains are to be joined together, and the other five curtains are to be joined together. \v{4}You are to make loops of blue material along the edge of the outermost curtain in the first set, and likewise you are to make loops along the edge of the outermost curtain in the second set. \v{5}You are to make 50 loops in the one curtain, and you are to make 50 loops along the edge of the curtain that is in the second set, with the loops opposite each other. \v{6}Then you are to make 50 gold clasps, and join the curtains to each other with the clasps so that the tent will be one piece.

\v{7}``You are to make curtains of goat hair for a tent over the tent. You are to make eleven curtains. \v{8}The length of each curtain is to be 30 cubits,\fnote{\fbackref{26:8} I.e. about 45 feet} and the width of each curtain two cubits;\fnote{\fbackref{26:8} I.e. about six feet} the measurements of each of the eleven curtains is to be the same.\fnote{\fbackref{26:8} Lit. \fbib{and the measure of one shall be for the eleven curtains}} \v{9}You are to join five curtains by themselves, and six curtains by themselves, and you are to double over the sixth curtain at the front of the tent. \v{10}You are to make 50 loops along the edge of the outermost curtain in the first set, and 50 loops along the edge of the curtain of the other set. \v{11}You are to make 50 bronze clasps, put the clasps into the loops, and join the tent together so that it will be one piece. \v{12}As for the excess that remains of the curtains of the tent---the half curtain that remains---is to hang over the back of the tent. \v{13}The half cubit\fnote{\fbackref{26:13} I.e. about nine inches} that remain on either end of the length of the curtains of the tent is to hang over each side of the tent to cover it.

\v{14}``You are to make a cover for the tent of ram skins dyed red\fnote{\fbackref{26:14} Or \fbib{tanned}} and a covering of dolphin\fnote{\fbackref{26:14} Or \fbib{dugong}, a marine animal resembling a walrus or manatee} skins above that.

\v{15}``You are to make upright boards of acacia wood for the tent. \v{16}Each board is to be ten cubits\fnote{\fbackref{26:16} I.e. about 15 feet} long and one and a half cubits\fnote{\fbackref{26:16} I.e. about 27 inches} wide. \v{17}Each board is to have two pegs joined to one another, and you are to do this for all the boards of the tent. \v{18}You are to make the boards for the tent: 20 boards for the south side.\fnote{\fbackref{26:18} Lit. \fbib{toward the Negev (south), toward Teman (a city to the south)}} \v{19}And you are to make 40 silver sockets\fnote{\fbackref{26:19} Or \fbib{bases}; and so through chapter 27} under the 20 boards: two sockets under the one board for its two pegs and two sockets under the next\fnote{\fbackref{26:19} Lit. \fbib{the one}; and so through chapter 27} board for its two pegs.

\v{20}``For the second side of the tent to the north you are to make\fnote{\fbackref{26:20} The Heb. lacks \fbib{you are to make}} 20 boards \v{21}and 40 silver sockets for them, two sockets under one board and two sockets under the next board. \v{22}On the west you are to make six boards for the rear of the tent, \v{23}and you are to make two boards for the rear corners of the tent. \v{24}They are to be interlocked together\fnote{\fbackref{26:24} Lit. \fbib{twins;} perhaps designed with interlocking pieces} at the bottom and connected\fnote{\fbackref{26:24} Lit. \fbib{complete}; perhaps the tops were joined together by a metal ring} on top by one ring. Do this for the two of them, and they are to be the two corners. \v{25}There is to be eight boards with their sixteen silver sockets, two sockets under one board and two sockets under the next board.

\v{26}``You are to make bars of acacia wood, five for the boards on one side of the tent, \v{27}five bars for the boards on the second side of the tent, and five bars for the boards on the back side of the tent to the west. \v{28}The center bar in the middle of the boards is to pass through from end to end. \v{29}You are to overlay the boards with gold, and you are to make gold rings for them as holders for the bars, and you are to overlay the bars with gold. \v{30}You are to erect the tent according to the plan for it that was shown you on the mountain.

\v{31}``You are to make a curtain of blue, purple, and scarlet material, and fine woven linen. You are to make it with cherubim skillfully worked into it. \v{32}You are to hang it on four pillars of acacia overlaid with gold, which have hooks of gold, and are set on four sockets of silver. \v{33}You are to hang the curtain from\fnote{\fbackref{26:33} Or \fbib{under}} the clasps and bring the Ark of the Testimony there inside the curtain. The curtain is to separate for you the Holy Place from the Most Holy Place.

\v{34}``You are to put the Mercy Seat on the Ark of the Testimony in the Most Holy Place. \v{35}You are to put the table outside the curtain. You are to put the table on the north side with the lamp stand opposite the table on the south side of the tent. \v{36}For the doorway of the tent you are to make a screen of blue, purple, and scarlet material, and with fine woven linen, the work of an embroiderer. \v{37}You are to make five pillars of acacia for the screens and overlay them with gold. Their hooks are to be of gold, and you are to cast five bronze sockets for them.''
\labelchapt{27}
\passage{The Altar}

\chapt{27}
\v{1}``You are to make the altar of acacia wood. It is to be five cubits\fnote{\fbackref{27:1} I.e. about seven and a half feet} long and five cubits\fnote{\fbackref{27:1} I.e. about seven and a half feet} wide; the altar is to be a square, and it is to be three cubits\fnote{\fbackref{27:1} I.e. about four and a half feet} high. \v{2}You are to make horns\fnote{\fbackref{27:2} Lit. \fbib{its horns}} on its four corners. Its corners are to be of one piece with it, and you are to overlay it with bronze. \v{3}You are to make pans for removing its ashes, shovels, bowls, forks, and fire-pans for it, and you are to make all its utensils of bronze. \v{4}You are to make a lattice, a netting of bronze for it, and you are to make four bronze rings on the netting at its four corners. \v{5}You are to put it under the ledge of the altar, so that the netting extends halfway up the altar. \v{6}You are to make poles for the altar, poles of acacia wood, and you are to overlay them with bronze. \v{7}The poles for it are to be put through the rings, so the poles are on the two sides of the altar when it's carried. \v{8}You are to make it hollow out of boards---just as it was shown you on the mountain, that's how they are to make it.''
\passage{The Court of the Tent}

\v{9}``You are to make the court of the tent. On the south\fnote{\fbackref{27:9} Lit. \fbib{toward the Negev, southward}} side there is to be hangings of fine woven linen for the court, 100 cubits\fnote{\fbackref{27:9} I.e. about 150 feet} long on one side. \v{10}It is to have 20 pillars, with 20 bronze sockets, and the hooks of the pillars and their bands\fnote{\fbackref{27:10} Perhaps a kind of connecting rod joining the pillars together} are to be made of silver. \v{11}Likewise for the length of the north side there are to be hangings 100 cubits\fnote{\fbackref{27:11} Lit. \fbib{a hundred}; i.e. about 150 feet; the Heb. lacks \fbib{cubits}} long, and it is to have 20 pillars with 20 bronze sockets, and the hooks of the pillars and their bands\fnote{\fbackref{27:11} Perhaps a kind of connecting rod joining the pillars together} are to be made of silver.

\v{12}``The width of the court on the west side is to have hangings 50 cubits\fnote{\fbackref{27:12} I.e. about 75 feet} long with ten pillars and ten sockets. \v{13}The width of the court on the east side\fnote{\fbackref{27:13} Lit. \fbib{on the east side toward the rising (of the sun)}} is to be 50 cubits.\fnote{\fbackref{27:13} I.e. about 75 feet} \v{14}The hangings for the one section\fnote{\fbackref{27:14} Lit. \fbib{the shoulder}} are to be fifteen cubits long,\fnote{\fbackref{27:14} I.e. about 22 and a half feet} with their three pillars and three sockets.

\v{15}``For the second section there are to be hangings of fifteen cubits,\fnote{\fbackref{27:15} I.e. about 22 and a half feet} with their three pillars and three sockets. \v{16}There is to be a screen of 20 cubits\fnote{\fbackref{27:16} I.e. about 30 feet} of blue, purple, and scarlet material and fine woven linen, the work of an embroiderer, for the gate of the court, and it is to have four pillars and four sockets. \v{17}All the pillars around the court are to be banded with silver. Their hooks are to be made of silver and their sockets made of bronze. \v{18}The length of the court is to be 100 cubits,\fnote{\fbackref{27:18} I.e. about 150 feet} the width 50 cubits,\fnote{\fbackref{27:18} Lit. \fbib{the width 50 by 50} (I.e. 50 cubits on the east side and 50 cubits on the west side)} and the height five cubits,\fnote{\fbackref{27:18} I.e. about seven and a half feet} with the hangings\fnote{\fbackref{27:18} The Heb. lacks \fbib{with the hangings}} of fine woven linen, and the sockets of bronze. \v{19}All the utensils of the tent for its service, all its pegs, and all the pegs for the court are to be made of bronze.''
\passage{The Oil for the Lamp}

\v{20}``And you are to command the Israelis to bring you pure olive oil, extracted by hand,\fnote{\fbackref{27:20} Lit. \fbib{beaten}; i.e. the olives were crushed in a mortar rather than pressed in an olive press} for the light in order to keep the lamp burning\fnote{\fbackref{27:20} Lit. \fbib{going up}} continuously. \v{21}In the Tent of Meeting, outside the curtain that is before the Testimony, Aaron and his sons are to maintain\fnote{\fbackref{27:21} Lit. \fbib{arrange}} the lamp stand\fnote{\fbackref{27:21} Lit. \fbib{it}} from evening until morning in the \divine{Lord}'s presence. It is to be a perpetual ordinance from generation to generation among the Israelis.''
\labelchapt{28}
\passage{The Garments for the Priests}

\chapt{28}
\v{1}``You are to bring your brother Aaron, along with his sons, from among the Israelis so they can serve as priests for me: that is, Aaron and his sons\fnote{\fbackref{28:1} Lit. \fbib{Aaron's sons}} Nadab, Abihu, Eleazar, and Ithamar. \v{2}You are to make holy garments for Aaron your brother, for dignity and beauty. \v{3}You are to speak to all who are skilled,\fnote{\fbackref{28:3} Lit. \fbib{wise (or skilled) of heart}} whom I've endowed\fnote{\fbackref{28:3} Lit. \fbib{filled}} with talent,\fnote{\fbackref{28:3} Lit. \fbib{a spirit of wisdom (or skill)}} that they should make Aaron's garments for consecrating him to serve me as priest. \v{4}These are the garments that they are to make: a breast piece, an ephod, a robe, a checkered tunic, a turban, and a sash. They are to make holy garments for Aaron your brother and for his sons to serve me as priests. \v{5}They are to use\fnote{\fbackref{28:5} Lit. \fbib{take}} gold, blue, purple, and scarlet material, as well as fine linen.''
\passage{The Ephod}

\v{6}``They are to make the ephod from gold, along with blue, purple, and scarlet material and fine woven linen, all of it\fnote{\fbackref{28:6} The Heb. lacks \fbib{all of it}} skillfully worked. \v{7}It is to have two shoulder-pieces attached to its two edges so it can be joined together. \v{8}The skillfully woven band that is on it is to be made like it, that is, of one piece with it, of gold, blue, purple, and scarlet material, and fine woven linen. \v{9}You are to take two onyx stones and engrave the names of the sons of Israel on them, \v{10}six of their names on one stone, and the six remaining names on the other stone. Engrave them\fnote{\fbackref{28:10} The Heb. lacks \fbib{Engrave them}} according to their order of birth. \v{11}With work like a jeweler engraves on a signet,\fnote{\fbackref{28:11} I.e. a type of seal used to indicate ownership} you are to inscribe the two stones with the names of the sons of Israel, and you are to mount them in settings of gold filigree. \v{12}You are to put the two stones on the shoulder pieces of the ephod as stones of remembrance representing the sons of Israel, and Aaron is to carry their names into the \divine{Lord}'s presence on his two shoulders as a memorial. \v{13}You are to make settings of gold filigree, \v{14}and you are to make two chains of pure gold twisted like cords, and then fasten the twisted chains to the filigree settings.''
\passage{The Breast Piece}

\v{15}``You are to make a breast piece to be worn by the high priest when he makes legal decisions.\fnote{\fbackref{28:15} Lit. \fbib{breast piece of judgment}} It is to be skillfully worked, made like the work of the ephod from gold, blue, purple, and scarlet material, and from fine woven linen. \v{16}It is to be square when folded double, one hand span\fnote{\fbackref{28:16} I.e. about the distance between the outstretched thumb and little finger, or about nine inches} long and one hand span wide.\fnote{\fbackref{28:16} I.e. about the distance between the outstretched thumb and little finger, or about nine inches} \v{17}You are to mount on it a setting for four rows of stones. The first row is to contain carnelian,\fnote{\fbackref{28:17} The meaning of MT is uncertain.} topaz, and emerald; \v{18}the second row ruby,\fnote{\fbackref{28:18} Or \fbib{turquoise}} sapphire, and crystal; \v{19}the third row jacinth, agate, and amethyst; \v{20}the fourth row beryl, onyx, and jasper, and they are to be set in gold filigree. \v{21}The stones are to correspond to the names of the sons of Israel, twelve stones\fnote{\fbackref{28:21} The Heb. lacks \fbib{stones}} corresponding to their names. They are to be engraved like a signet,\fnote{\fbackref{28:21} Lit. \fbib{the engravings of a seal (} or \fbib{signet ring)}} each with the name of one of the twelve tribes.

\v{22}``You are to make chains of pure gold, twisted like cords, for the breast piece. \v{23}You are to make two gold rings for the breast piece, and put the two rings on the two edges of the breast piece. \v{24}You are to put the two gold cords on the two gold rings at the edges of the breast piece, \v{25}and you are to attach the other two ends of the two cords to the filigree settings and attach them to the shoulder pieces of the ephod in front.

\v{26}``You are to make two gold rings and attach them to the two edges of the breast piece, on the side of it that is toward the inner side of the ephod. \v{27}You are to make two gold rings and attach them in front on the lower part of the two shoulder pieces of the ephod close to the place where it's joined, above the skillfully woven band of the ephod. \v{28}They are to fasten the rings on the breast piece to the rings on the ephod with a blue cord so it will rest\fnote{\fbackref{28:28} Lit. \fbib{be}} on the skillfully woven band of the ephod and so the breast piece won't come loose from the ephod.

\v{29}``Aaron is to carry the names of Israel's sons on his heart on the breast piece to be worn by the high priest when he makes legal decisions,\fnote{\fbackref{28:29} Lit. \fbib{breast piece of judgment}} that is, whenever he goes into the Holy Place in order to remember them continuously in the \divine{Lord}'s presence. \v{30}You are to put the Urim and Thummim\fnote{\fbackref{28:30} I.e. the jewel-encrusted breastplate worn by the high priest by which the will of God could be revealed; cf. Ezra 2:63, Neh 7:65} into the breast piece of judgment, and they are to be on Aaron's heart when he goes into the \divine{Lord}'s presence. He is to carry the breast piece of decision\fnote{\fbackref{28:30} Lit. \fbib{judgment}} that depicts Israel's sons\fnote{\fbackref{28:30} Lit. \fbib{of judgment of Israel's sons}} on his heart in the \divine{Lord}'s presence continuously.''
\passage{Other Garments for the Priests}

\v{31}``You are to make the robe of the ephod entirely of blue. \v{32}There is to be an opening at its top, in the middle, with a woven binding around the opening like the opening of a coat of mail so that it cannot be torn. \v{33}On its hem you are to make blue and purple and scarlet pomegranates, all around the skirt, with gold bells between them all the way\fnote{\fbackref{28:33} The Heb. lacks \fbib{the way}} around. \v{34}You are to have a gold bell and a pomegranate, then\fnote{\fbackref{28:34} The Heb. lacks \fbib{then}} a gold bell and a pomegranate, on the hem of the robe all the way\fnote{\fbackref{28:34} The Heb. lacks \fbib{the way}} around it. \v{35}Aaron is to wear the robe when he ministers\fnote{\fbackref{28:35} Lit. \fbib{for ministering}} so its sound may be heard when he enters and leaves the Holy Place in the \divine{Lord}'s presence, so that he won't die.

\v{36}``You are to make a medallion\fnote{\fbackref{28:36} Or \fbib{plate}} of pure gold, and engrave on it `Holy to the \divine{Lord},' like the engravings of a signet. \v{37}You are to put it on a blue cord and place it on the turban. It is to be on the front of the turban \v{38}and worn on Aaron's forehead in order to take away any guilt contained in the holy things which the Israelis consecrate as holy gifts. It is to remain on his forehead continuously, so they may be accepted in the \divine{Lord}'s presence.

\v{39}``You are to weave the checkered tunic of fine linen, you are to make a turban of fine linen, and you are to make an embroidered sash. \v{40}``You are to make tunics for the sons of Aaron, you are to make sashes for them, and you are to make head coverings for them for dignity and beauty. \v{41}You are to put them on Aaron your brother, and on his sons with him, and you are to anoint them, ordain them,\fnote{\fbackref{28:41} Lit. \fbib{fill their hand}} and consecrate them to serve as my priests.

\v{42}``You are to make linen undergarments for them to cover their naked flesh, and they are to reach\fnote{\fbackref{28:42} Lit. \fbib{be}} from the loins to the thighs. \v{43}They are to be on Aaron and his sons when they enter the Tent of Meeting or when they approach the altar to minister in the Holy Place so they don't incur guilt and die. This is to be a perpetual ordinance for him and for his descendants\fnote{\fbackref{28:43} Lit. \fbib{seed}} after him.''
\labelchapt{29}
\passage{The Consecration of the Priests}

\chapt{29}
\v{1}``This is what you are to do to them in order to consecrate them to serve me as priests: Take a young bull, two rams without blemish, \v{2}unleavened bread, unleavened cakes mixed with oil, and unleavened wafers spread\fnote{\fbackref{29:2} Or \fbib{anointed}} with oil, which you are to make from fine wheat flour. \v{3}You are to put them\fnote{\fbackref{29:3} I.e. the bread, cakes, and wafers} in one basket and bring them in the basket along with the bull and the two rams. \v{4}You are to bring Aaron and his sons to the doorway of the Tent of Meeting, and wash them with water. \v{5}Take the garments and clothe Aaron with the tunic, the robe of the ephod, the ephod, and the breast piece, and then gird him with the skillfully woven band of the ephod. \v{6}Then put the turban on his head, and place the holy crown on the turban. \v{7}You are to take the anointing oil, pour it on his head, and anoint him. \v{8}Then you are to bring his sons and clothe them with tunics. \v{9}You are to gird Aaron and his sons with sashes and tie headdresses on them. The priesthood is to belong to them by perpetual ordinance, and you are to ordain\fnote{\fbackref{29:9} Lit. \fbib{fill the hand of}} Aaron and his sons.

\v{10}``You are to bring the bull in front of the Tent of Meeting, and Aaron and his sons are to lay their hands on the head of the bull. \v{11}Then you are to slaughter the bull in the \divine{Lord}'s presence at the doorway of the Tent of Meeting. \v{12}Take some of the blood of the bull, put it on the horns of the altar with your finger, and pour out the rest\fnote{\fbackref{29:12} Lit. \fbib{all}} of the blood at the base of the altar. \v{13}You are to take all the fat that covers the entrails, the lobe of the liver, the two kidneys, and the fat that is on them and send them up in smoke on the altar. \v{14}You are to burn the flesh of the bull, its hide, and its refuse with fire outside the camp. It is a sin offering.

\v{15}``You are to take one of the rams, and Aaron and his sons are to lay their hands on its\fnote{\fbackref{29:15} Lit. \fbib{the head of the ram}} head. \v{16}Then you are to slaughter the ram, take its blood, and scatter it around the altar. \v{17}You are to cut the ram into pieces,\fnote{\fbackref{29:17} Lit. \fbib{its pieces}} wash its entrails and legs, put them on the altar along\fnote{\fbackref{29:17} The Heb. lacks \fbib{on the altar along}} with the pieces\fnote{\fbackref{29:17} Lit. \fbib{its pieces}} and its head, \v{18}and send up the whole ram in smoke on the altar. It is a burnt offering to the \divine{Lord}; it's a soothing aroma, an offering by fire to the \divine{Lord}.

\v{19}``You are to take the other ram, and Aaron and his sons are to lay their hands on the head of the ram. \v{20}You are to slaughter the ram, take some of its blood, and put it on the lobe of Aaron's right ear, the lobe of his sons' right ears, the thumbs of their right hands, and the big toes of their right feet. Then you are to scatter the rest of the blood around the altar. \v{21}You are to take some of the blood which is on the altar, along with some of the anointing oil, and sprinkle it on Aaron and his garments, and on his sons and their\fnote{\fbackref{29:21} Lit. \fbib{on his sons' garments}} garments. He is to be consecrated with his garments, along with his sons and their garments

\v{22}``You are to take the fat from the ram, the fat tail, the fat that covers the entrails, the lobe of the liver, the two kidneys and the fat that is on them, the right thigh (for it's a ram of ordination), \v{23}and one loaf of bread, one cake of bread mixed with oil, and one wafer out of the basket of unleavened bread that is in the \divine{Lord}'s presence. \v{24}You are to put all of these in the hands of Aaron and in the hands of his sons, and present them as a wave offering in the \divine{Lord}'s presence. \v{25}Then you are to take them from their hands and send them up in smoke on the altar on top of the burnt offering for a soothing aroma in the \divine{Lord}'s presence. It is an offering by fire to the \divine{Lord}.

\v{26}``You are to take the breast of the ram of Aaron's ordination, and present it as a wave offering in the \divine{Lord}'s presence, and it is to be your portion. \v{27}You are to consecrate the portion of the ram of ordination that belongs to Aaron and his sons:\fnote{\fbackref{29:27} Lit. \fbib{from what was for Aaron and from what was for his sons}} the breast of the wave offering that was waved and the thigh of the presented offering that was presented.\fnote{\fbackref{29:27} Or \fbib{lifted up}} \v{28}These offerings\fnote{\fbackref{29:28} Lit. \fbib{it}} from the Israelis are to be a perpetual ordinance for Aaron and his sons. They are presented offerings, and they are to be presented offerings from the Israelis out of their peace offerings. They are presented offerings to the \divine{Lord}.

\v{29}``The holy garments of Aaron are to be for his sons after him\fnote{\fbackref{29:29} I.e. \fbib{descendants}} so that they may be anointed in them and ordained in them. \v{30}Aaron's son, who is priest in his place, is to wear them for seven days when he comes into the Tent of Meeting to minister in the Holy Place.

\v{31}``You are to take the ram of ordination and boil its flesh in a Holy Place. \v{32}Then Aaron and his sons are to eat the flesh of the ram along with the bread that is in the basket at the doorway of the Tent of Meeting. \v{33}They are to eat these things by which atonement was made at their ordination to consecrate them, but an unqualified person\fnote{\fbackref{29:33} Lit. \fbib{a stranger}; i.e. one not qualified to serve as a priest} is not to eat because these things are holy. \v{34}If any of the flesh of the ordination ram\fnote{\fbackref{29:34} The Heb. lacks \fbib{ram}} or any of the bread is left until morning, you are to burn what is left with fire. Because it's holy, what remains is not to be eaten. \v{35}You are to do this for Aaron and his sons, just as I've commanded you. You are to ordain them for seven days, \v{36}and every day you are to offer a bull as a sin offering for atonement. Offer the sin offering on the altar when you make atonement for it and anoint the altar to consecrate it. \v{37}You are to make atonement for the altar for seven days and consecrate it. It will be most holy, and whatever touches it will be holy.''
\passage{The Altar for Burnt Offering}
\passageinfo{(Numbers 28:1-8)}

\v{38}``This is what you are to offer on the altar continually: two one year old lambs each day. \v{39}``You are to offer one lamb in the morning and the other\fnote{\fbackref{29:39} I.e. about one quart} at twilight, \v{40}and there is to be a tenth measure of choice flour mixed with one fourth of a hin\fnote{\fbackref{29:40} I.e. about one quart} of oil extracted by hand,\fnote{\fbackref{29:40} Lit. \fbib{beaten}; i.e. the olives were crushed in a mortar rather than pressed in an olive press} and one fourth of a hin\fnote{\fbackref{29:40} I.e. about one quart} of wine as a drink offering for one lamb. \v{41}You are to offer the other lamb at twilight with the same grain offering and drink offering as in the morning. You are to offer it as a soothing aroma, an offering by fire to the \divine{Lord}. \v{42}It is to be a regular burnt offering throughout your generations at the doorway to the Tent of Meeting in the \divine{Lord}'s presence, where I'll meet with you to speak to you there. \v{43}I'll meet there with the Israelis, and it is to be consecrated by my glory. \v{44}I'll consecrate the Tent of Meeting and the altar, and I'll consecrate Aaron and his sons to serve as my priests. \v{45}I'll live among the Israelis, and I'll be their God. \v{46}They are to know that I am the \divine{Lord} their God, who brought them out of Egypt so that I may live among them. I am the \divine{Lord} your God.''
\labelchapt{30}
\passage{The Altar of Incense}

\chapt{30}
\v{1}``You are to make an altar for burning incense. You are to make it of acacia wood. \v{2}It is to be a square, one cubit\fnote{\fbackref{30:2} I.e. about one and a half feet} long and one cubit\fnote{\fbackref{30:2} I.e. about one and a half feet} wide, and it is to be two cubits\fnote{\fbackref{30:2} I.e. about three feet} high, with its horns of one piece with it. \v{3}You are to overlay it with pure gold, its top, its sides all around, and its horns, and you are to make a molding of gold all around it.

\v{4}``You are to make two gold rings for it under its molding. You are to make them on its two opposite sides, and they are to be holders for poles by which to carry it. \v{5}You are to make the poles of acacia wood and overlay them with gold. \v{6}You are to put the altar\fnote{\fbackref{30:6} Lit. \fbib{it}} in front of the curtain that is over the Ark of the Testimony, in front of the Mercy Seat\fnote{\fbackref{30:6} Or \fbib{atonement place}, and so throughout the book} that is over the Testimony where I'll meet with you. \v{7}Aaron is to offer fragrant incense on it. Every morning when he trims the lamps he is to offer it, \v{8}and when Aaron sets up the lamps at twilight, he is to offer it as a continual incense offering in the \divine{Lord}'s presence throughout your generations. \v{9}You are not to offer strange incense, a burnt offering, or a grain offering on it, nor are you to pour out a libation on it. \v{10}Each year Aaron is to make atonement on its horns with the blood of the sin offering of atonement. He is to make atonement on it each year throughout your generations. It is most holy to the \divine{Lord}.''
\passage{Offerings for the Tent}

\v{11}The \divine{Lord} told Moses, \v{12}``When you take a census of the Israelis to register them, each is to give a ransom for himself\fnote{\fbackref{30:12} Or \fbib{his life}} to the \divine{Lord} when they're registered so there won't be a plague among them when they're registered. \v{13}This is what everyone who is registered\fnote{\fbackref{30:13} Lit. \fbib{the one who passes over to those who have been registered}} is to give: half a shekel\fnote{\fbackref{30:13} I.e., a unit of weight measurement equal to about 16 barley grains; about 0.025 ounces or 0.5 grams; cf. Num 3:47; Num 18:16} according to the shekel\fnote{\fbackref{30:13} I.e., a unit of weight measurement equal to about 16 barley grains; about 0.025 ounces or 0.5 grams; cf. Num 3:47; Num 18:16} of the sanctuary (the shekel\fnote{\fbackref{30:13} I.e., a unit of weight measurement equal to about 16 barley grains; about 0.025 ounces or 0.5 grams; cf. Num 3:47; Num 18:16} weighs 20 gerahs), half a shekel\fnote{\fbackref{30:13} I.e., a unit of weight measurement equal to about 16 barley grains; about 0.025 ounces or 0.5 grams; cf. Num 3:47; Num 18:16} as a contribution to the \divine{Lord}. \v{14}All who are registered, 20 years of age and older, are to give a contribution to the \divine{Lord}. \v{15}The rich person is not to give more,\fnote{\fbackref{30:15} Lit. \fbib{increase from}} nor is the poor person to give less\fnote{\fbackref{30:15} Lit. \fbib{decrease from}} than the half shekel,\fnote{\fbackref{30:15} I.e., a unit of weight measurement equal to about 16 barley grains; about 0.025 ounces or 0.5 grams; cf. Num 3:47; Num 18:16} when you give a contribution to the \divine{Lord} to make atonement for yourselves.\fnote{\fbackref{30:15} Or \fbib{for your lives}} \v{16}You are to take the atonement money from the Israelis and give it for the service of the Tent of Meeting, and it is to be a memorial for the Israelis in the \divine{Lord}'s presence to make atonement for yourselves.''\fnote{\fbackref{30:16} Or \fbib{for your lives}}
\passage{The Bronze Basin}

\v{17}The \divine{Lord} told Moses, \v{18}``You are to make a bronze basin with a bronze base for washing. You are to pace it between the Tent of Meeting and the altar, put water in it,\fnote{\fbackref{30:18} Lit. \fbib{there}} \v{19}and Aaron and his sons are to wash their hands and their feet from it. \v{20}When they enter the Tent of Meeting or when they approach the altar to minister to make an offering by fire to the \divine{Lord}, they are to wash with water so they don't die. \v{21}They are to wash their hands and their feet so that they don't die, and it is to be for them a perpetual ordinance for Aaron\fnote{\fbackref{30:21} Lit. \fbib{for him}} and his seed from generation to generation.''
\passage{The Anointing Oil}

\v{22}The \divine{Lord} told Moses, \v{23}``You are to take for yourself the finest spices: 500 shekels\fnote{\fbackref{30:23} The Heb. lacks \fbib{shekels}; Five hundred shekels is about 12 {\textonehalf} pounds}by weight of liquid myrrh, half as much fragrant cinnamon (250 shekels), 250 shekels of fragrant reeds, \v{24}500 shekels of cassia---all according to the shekel of the sanctuary---and a hin\fnote{\fbackref{30:24} I.e. about a quart} of olive oil. \v{25}You are to make them into a holy anointing oil, a perfume mixture made by a perfumer. It is to be a holy anointing oil. \v{26}You are to use it to anoint the Tent of Meeting, the Ark of the Testimony, \v{27}the table and all its utensils, the lamp stand and its utensils, the altar of incense, \v{28}the altar for burnt offerings and all its utensils, and the basin and its base. \v{29}You are to consecrate them and they are to be most holy. Whatever touches them is to be holy. \v{30}You are to anoint Aaron and his sons, and you are to consecrate them to serve as my priests. \v{31}You are to address the Israelis and tell them, `This is to be holy anointing oil for me from generation to generation. \v{32}It is not to be poured out on a person's body,\fnote{\fbackref{30:32} I.e. used for ordinary anointing purposes} nor are you to make anything like it with similar formulations. It is holy, and it is to be holy to you. \v{33}Anyone who mixes anything like it or who puts any of it on an unqualified person\fnote{\fbackref{30:33} Lit. \fbib{a stranger}; i.e. a person not qualified to serve as a priest} is to be cut off from his people.'\,''
\passage{The Incense}

\v{34}The \divine{Lord} told Moses, ``Take for yourself spices: stacte, onycha, galbanum, and spices with pure frankincense, all in equal amounts. \v{35}You are to make it into a fragrant incense, expertly\fnote{\fbackref{30:35} Lit. \fbib{the work of a perfumer}} blended,\fnote{\fbackref{30:35} Or \fbib{salted}} pure, and holy. \v{36}You are to grind some of it fine, and put some before the Testimony in the Tent of Meeting where I will meet with you. It is to be most holy to you. \v{37}You are not to make the incense that you make in this formulation for your own use. It is to be holy to the \divine{Lord} for you. \v{38}Anyone who makes anything like it to use it as perfume is to be cut off from his people.''
\labelchapt{31}
\passage{Craftsmen for the Tent}

\chapt{31}
\v{1}The \divine{Lord} told Moses, \v{2}``Look, I've called\fnote{\fbackref{31:2} Lit. \fbib{called by name}} Uri's son Bezalel, grandson of Hur from Judah's tribe \v{3}and I've filled him with the Spirit of God, with wisdom, understanding, knowledge, and all kinds of craftsmanship \v{4}to create plans\fnote{\fbackref{31:4} Lit. \fbib{to devise devices}} for work in gold, silver, and bronze, \v{5}and for cutting stones to set them, for carving wood, and for doing all kinds of craftsmanship. \v{6}Along with him I'm appointing Ahisamach's son Oholiab from the tribe of Dan, and I've given wisdom\fnote{\fbackref{31:6} Or \fbib{ability}} to all who are skilled\fnote{\fbackref{31:6} Lit. \fbib{wise of heart}} so they can make everything that I've commanded you, \v{7}including the Tent of Meeting, the Ark of the Testimony, the Mercy Seat\fnote{\fbackref{31:7} Or \fbib{atonement cover}} that is on it, all the furnishings of the tent--- \v{8}the table and its furnishings, the lamp stand of pure gold,\fnote{\fbackref{31:8} Lit. \fbib{the pure lamp stand}} all its furnishings, the altar of incense, \v{9}the altar for burnt offerings, its furnishings, the basin, its base, \v{10}the woven garments, the holy garments of Aaron the priest, the garments of his sons as they serve as priests, \v{11}the anointing oil, and the fragrant incense for the Holy Place. They are to make them in accordance with everything that I commanded you.''

\v{12}The \divine{Lord} told Moses, \v{13}``You are to tell the Israelis: `You are to certainly observe my Sabbaths because it's a sign between me and you from generation to generation, so you may know that I am the \divine{Lord} who sanctifies you. \v{14}You are to observe the Sabbath, because it's holy for you. Whoever profanes it is certainly to die; indeed, whoever does work on it is to be cut off from among his people. \v{15}Work may be done for six days, but the seventh day is a Sabbath of complete rest, holy to the \divine{Lord}. Whoever does work on the Sabbath is certainly to die. \v{16}The Israelis are to keep the Sabbath to make the Sabbath observance a perpetual covenant from generation to generation. \v{17}It is a sign forever between me and the Israelis, because the \divine{Lord} made the heavens and the earth in six days, but on the seventh day he rested and was refreshed.'\,''

\v{18}When he finished speaking with Moses\fnote{\fbackref{31:18} Lit. \fbib{him}} on Mount Sinai, he gave him\fnote{\fbackref{31:18} Lit. \fbib{to Moses}} the two Tablets of the Testimony, tablets of stone written by the finger of God.
\labelchapt{32}
\passage{Aaron Makes the Golden Calf}

\chapt{32}
\v{1}When the people saw that Moses took a long time to come down the mountain, they gathered around Aaron and told him, ``Come here and make us a god\fnote{\fbackref{32:1} Or \fbib{gods}; and so throughout the chapter} who will go before us, because, as for this fellow Moses who led us out of the land of Egypt, we don't know what has become of him.''

\v{2}Aaron told them, ``Tear off the gold rings which are in the ears of your wives, your sons, and your daughters and bring them to me.''

\v{3}All the people tore off the gold rings that were in their ears and brought them to him. \v{4}He took them from them\fnote{\fbackref{32:4} Lit. \fbib{from their hand}} and, using a tool, fashioned them into a molten calf.\fnote{\fbackref{32:4} I.e. an image made by pouring hot, liquid metal into a mold} The people\fnote{\fbackref{32:4} Lit. \fbib{They}} said, ``This, Israel, is your god who brought you out of the land of Egypt.''

\v{5}When Aaron saw this, he built an altar in front of it, and then he proclaimed, ``Tomorrow is to be a festival to the \divine{Lord}.'' \v{6}They got up early the next day and offered burnt offerings and brought peace offerings. Then the people sat down to eat and drink, and then they got up to play.\fnote{\fbackref{32:6} I.e. to engage in sexual immorality}
\passage{Moses Intercedes for Israel}

\v{7}The \divine{Lord} told Moses, ``Go down immediately,\fnote{\fbackref{32:7} Lit. \fbib{Go, go down}} because your people whom you led out of Egypt have behaved corruptly. \v{8}They have been quick to turn aside from the way I commanded them, and they have made for themselves a molten calf. They have bowed down to it in worship, they have offered sacrifices to it, and they have said, `This, Israel, is your god who brought you out of the land of Egypt.'\,''

\v{9}Then the \divine{Lord} told Moses, ``I've seen these people and indeed they're obstinate.\fnote{\fbackref{32:9} Lit. \fbib{stiff-necked}} \v{10}Now let me alone so that my anger may burn against them and that I may consume them, but I'll make a great nation of you.''

\v{11}But Moses implored the \divine{Lord} his God: ``\divine{Lord}, why are you angry with your people whom you brought out of the land of Egypt with great power and a show of force?\fnote{\fbackref{32:11} Lit. \fbib{a mighty hand}} \v{12}Why should the Egyptians say, `He brought them out with an evil intention to kill them in the mountains and to destroy them from the face of the earth'? Turn from your anger and change your mind about the calamity against your people. \v{13}Remember Abraham, Isaac, and Israel, your servants to whom you swore by yourself as you told them, `I'll increase the number of your descendants like the stars of the heavens, I'll give your descendants all of this land about which I have spoken, and they are to possess\fnote{\fbackref{32:13} Or \fbib{inherit}} it forever.'\,''

\v{14}So the \divine{Lord} changed his mind about the calamity he had said he would bring on his people.
\passage{Moses Destroys the Golden Calf and the Tablets of the Law}

\v{15}Then Moses turned and went down the mountain with the two Tablets of the Testimony in his hand, tablets which were written on both sides. They were written on one side and the other. \v{16}The tablets were the work of God and the writing was God's writing, inscribed on the tablets. \v{17}When Joshua heard the sound of the people as they shouted, he told Moses, ``The sound of war is coming from\fnote{\fbackref{32:17} Lit. \fbib{is in}} the camp.''

\v{18}Moses\fnote{\fbackref{32:18} Lit. \fbib{He}} said,

\begin{poetry}
\poeml ``It is not the sound of a victory shout, \\
\poemll    and it's not the sound of a shout of defeat, \\
\poemlll       but it's the sound of singing that I hear.''
\end{poetry}

\v{19}As Moses approached the camp and saw the calf and the dancing, he became angry. He threw the tablets from his hands and shattered them at the base of the mountain. \v{20}He took the calf that they had made, burned it with fire, and ground it into powder. He scattered it on the water and made the Israelis drink it. \v{21}Then Moses asked Aaron, ``What did this people do to you that you brought such great sin upon them?''

\v{22}Aaron said, ``Sir,\fnote{\fbackref{32:22} Lit. \fbib{My lord}} don't be angry. You know the people---that they're intent on evil. \v{23}They told me, `Make a god for us who will go before us because, as for this fellow Moses who brought us out of the land of Egypt, we don't know what has become of him.' \v{24}So I told them, `Whoever has gold ornaments, tear them off.' When they gave it to me, I threw it into the fire, and out came this calf.''
\passage{The Descendants of Levi Punish the Guilty Israelis}

\v{25}When Moses saw that the people were out of control---since Aaron had let them get out of control, something that incited ridicule from their enemies\fnote{\fbackref{32:25} Lit. \fbib{for ridicule among their enemies}}---\v{26}he stood in the gate of the camp and called out: ``Whoever is for the \divine{Lord} come over\fnote{\fbackref{32:26} The Heb. lacks \fbib{come over}} to me,'' and all the sons of Levi gathered around him. \v{27}He told them, ``This is what the \divine{Lord}, the God of Israel, says, `Every man put his sword on his thigh, and go back and forth from gate to gate in the camp, and each of you kill his brother and friend and neighbor.'\,''

\v{28}The descendants of Levi did just as Moses told them,\fnote{\fbackref{32:28} Lit. \fbib{according to the word of Moses}} and about 3,000 people died that day. \v{29}Moses said, ``You have been ordained\fnote{\fbackref{32:29} Or \fbib{Consecrate yourselves}} to serve the \divine{Lord}\fnote{\fbackref{32:29} Lit. \fbib{ordained for the \divine{Lord}}} today, and you have brought a blessing on yourselves today because every man opposed his son or brother.''\fnote{\fbackref{32:29} Or \fbib{today at the cost of his son or brother}}
\passage{Moses Again Intercedes for the People}

\v{30}The next day Moses told the people, ``You committed a great sin, and now I'll go up to the \divine{Lord}, and perhaps I can make atonement for your sin.''

\v{31}Moses returned to the \divine{Lord} and said, ``Please, \divine{Lord}, this people committed a great sin by making a god of gold for themselves. \v{32}Now, if you will, forgive their sin---but if not, blot me out of your book which you have written.''

\v{33}The \divine{Lord} told Moses, ``Whoever sins against me, I'll blot him out of my book. \v{34}Now, go, and lead the people where I told you, and now my angel will go before you, but on the day when I do punish, I'll punish them for their sin.'' \v{35}Then the \divine{Lord} sent a plague on the people because they made the calf (the one Aaron made).
\labelchapt{33}
\passage{The \divine{Lord} Instructs Israel to Leave}

\chapt{33}
\v{1}The \divine{Lord} told Moses, ``Go up\fnote{\fbackref{33:1} Lit. \fbib{go, go up}} from here, you and the people whom you brought out of Egypt, to the land about which I swore to Abraham, Isaac, and Jacob saying, `I'll give it to your descendants.'\fnote{\fbackref{33:1} Lit. \fbib{your seed}} \v{2}I'll send an angel in front of you and I'll drive out the Canaanites, the Amorites, the Hittites, the Perizzites, the Hivites, and the Jebusites. \v{3}Go up to a land flowing with milk and honey, but I won't go up among you, because you are an obstinate\fnote{\fbackref{33:3} Lit. \fbib{stiff-necked}} people, and otherwise I might consume you along the way.''

\v{4}When the people heard this troubling word, they mourned, and no one put on his ornaments. \v{5}The \divine{Lord} had told Moses, ``Say to the Israelis, `You are an obstinate people,\fnote{\fbackref{33:5} Lit. \fbib{stiff-necked}} and if for one moment I went up among you, I would put an end to you. Now take off your ornaments so I may decide\fnote{\fbackref{33:5} Lit. \fbib{know}} what to do with you.'\,'' \v{6}So the Israelis did not wear\fnote{\fbackref{33:6} Lit. \fbib{stripped themselves of}} their ornaments from Mount Horeb onward.
\passage{God's Presence at the Tent of Meeting}

\v{7}Moses used to take the tent and set it up outside the camp at a distance from the camp, and he called it the Tent of Meeting. When anyone sought the \divine{Lord}, he would go out to the Tent of Meeting which was outside the camp. \v{8}When Moses would go out to the tent, all the people would get up, and each would stand in the doorway of his tent, watching Moses until he entered the tent. \v{9}When Moses entered the tent, the pillar of cloud would come down and stand at the doorway of the tent while God\fnote{\fbackref{33:9} Lit. \fbib{he}} spoke with Moses. \v{10}When all the people saw the pillar of cloud standing at the doorway of the tent, all of them\fnote{\fbackref{33:10} Lit. \fbib{all the people}} would get up and prostrate themselves in worship, each one at the doorway of his tent. \v{11}The \divine{Lord} would speak to Moses face to face just as a man speaks with his friend. When Moses\fnote{\fbackref{33:11} Lit. \fbib{he}} returned to the camp, Nun's son Joshua, his young servant, would not leave the tent.
\passage{The Promise of God's Presence on the Journey}

\v{12}Moses told the \divine{Lord}, ``Look, you have told me, `Bring up this people,' but you haven't let me know whom you will send with me. Yet you have said, `I know you by name,' and also, `You have found favor in my sight.' \v{13}Now, if I've found favor in your sight, please show me your ways so I may know you in order to find favor in your sight. And remember,\fnote{\fbackref{33:13} Or \fbib{consider}; Lit. \fbib{see}} this nation is your people.''

\v{14}He said, ``My presence will go with you, and I'll give you rest.'' \v{15}Then Moses\fnote{\fbackref{33:15} Lit. \fbib{he}} told the \divine{Lord},\fnote{\fbackref{33:15} Lit. \fbib{to him}} ``If your presence does not go with us,\fnote{\fbackref{33:15} Lit. \fbib{does not go}} don't bring us up from here. \v{16}Otherwise,\fnote{\fbackref{33:16} Lit. \fbib{For}} how shall it be known that your people and I have received favor from you, unless you go with us and that we, your people and I, are distinguished from all the people on the surface of the earth?''
\passage{Moses Sees God's Glory}

\v{17}The \divine{Lord} told Moses, ``I'll do the very\fnote{\fbackref{33:17} Lit. \fbib{this}} thing that you have said, because you have found favor in my sight and I know you by name.''

\v{18}Then Moses\fnote{\fbackref{33:18} Lit. \fbib{he}} said, ``Please show me your glory.''

\v{19}God\fnote{\fbackref{33:19} Lit. \fbib{He}} said, ``I'll cause all my goodness to pass before you, and I'll proclaim the name `the \divine{Lord}' before you. I'll be gracious to whom I'll be gracious, and I'll show compassion on whom I'll show compassion. \v{20}But,'' he said, ``You cannot see my face, because a man cannot see me and live.''

\v{21}The \divine{Lord} said, ``Look, there is a place near\fnote{\fbackref{33:21} Or \fbib{with}} me where you can stand on the rock; \v{22}and as my glory passes by, I'll put you in a crevice in the rock, and cover you with my hand until I've passed by. \v{23}Then I'll remove my hand so you may see my back, but my face must not be seen.''
\labelchapt{34}
\passage{The Tablets of the Law Replaced}

\chapt{34}
\v{1}The \divine{Lord} told Moses, ``Cut out for yourself two stone tablets like the first ones, and I'll write on the tablets the words which were on the first tablets that you broke. \v{2}Be ready in the morning, and come up in the morning on Mount Sinai, where you are to present yourself to me there on the top of the mountain. \v{3}No one is to come up with you, nor is anyone to be seen anywhere on the mountain. Also, the sheep and cattle are not to graze in front of that mountain.''

\v{4}So Moses\fnote{\fbackref{34:4} Lit. \fbib{He}} carved out two stone tablets like the first ones, got up early in the morning, and climbed Mount Sinai, just as the \divine{Lord} had commanded him. He took with him the two stone tablets. \v{5}The \divine{Lord} came down in a cloud and stood there with him and proclaimed the name of the \divine{Lord}.\fnote{\fbackref{34:5} Or \fbib{and he called on the name of the \divine{Lord}}} \v{6}The \divine{Lord} passed in front of him and proclaimed,

\begin{poetry}
\poeml ``The \divine{Lord}, the \divine{Lord} God, \\
\poemll    compassionate and gracious, \\
\poeml slow to anger, \\
\poemll    and filled with\fnote{\fbackref{34:6} Or \fbib{and abundant in}} gracious love and truth. \\
\poeml \v{7}He graciously loves thousands, \\
\poemll    and forgives iniquity, transgression, and sin. \\
\poeml But he does not leave the guilty unpunished, \\
\poemll    visiting the iniquity of the ancestors on their children, \\
\poeml and on their children's children \\
\poemll    to the third and fourth generation.''
\end{poetry}

\v{8}Moses quickly bowed to the ground and prostrated himself in worship. \v{9}He said, ``If I've found favor in your sight, \divine{Lord}, please, \divine{Lord}, walk among us. Certainly this is an obstinate people, but pardon our iniquity and our sin, and take us for your own inheritance.''
\passage{The Covenant Promises Repeated}

\v{10}Then the \divine{Lord} said, ``I'm now going to make a covenant. I'll do miraculous deeds in full view of your people that haven't been done\fnote{\fbackref{34:10} Lit. \fbib{created}} in all the earth or in any nation. All the people among whom you live will see the work of the \divine{Lord}, because it's an awesome thing that I'll do with you. \v{11}Obey\fnote{\fbackref{34:11} Lit. \fbib{keep}} what I am commanding you today and I'll drive out from before you the Amorites, the Canaanites, the Hittites, the Perizzites, the Hivites, and the Jebusites.

\v{12}``Be very careful not to make a covenant with the inhabitants of the land to which you are going, so they won't be a snare among you. \v{13}Rather, you are to tear down their altars, you are to smash their sacred pillars, and you are to cut down their sacred poles\fnote{\fbackref{34:13} Heb. \fbib{Asherim}; wooden symbols of the chief female Canaanite deity}---\v{14}indeed, you are not to bow down in worship to any other god, because the \divine{Lord}'s name is Jealous---he's a jealous God---\v{15}Otherwise, you may make a covenant with the inhabitants of the land and when they prostitute themselves with their gods and offer sacrifices to their gods, someone may invite you and then you may eat some of their sacrifices.

\v{16}``You are not to take any of their daughters for your sons. Otherwise, when their daughters prostitute themselves with their gods, they may cause your sons to prostitute themselves with their gods.

\v{17}``You are not to make molten gods for yourselves.

\v{18}``You are to observe the Festival of Unleavened Bread. For seven days, at the appointed time in the month Abib, you are to eat unleavened bread as I commanded you, for in the month Abib you came out of Egypt.

\v{19}``Everything firstborn\fnote{\fbackref{34:19} Lit. \fbib{Everything that first opens the womb}} belongs to me: all the males of your herds, the firstborn of both cattle and sheep. \v{20}You are to redeem the firstborn of a donkey with a sheep, and if you don't redeem it, you are to break its neck. You are to redeem every firstborn of your sons, and no one is to appear before me empty-handed.

\v{21}``For six days you are to work, but on the seventh day you are to rest; even during plowing time and harvest you are to rest.

\v{22}``You are to observe the Festival of Weeks, the first fruits of the wheat harvest, and the Festival of Ingathering at the turn of the year. \v{23}Three times during the year all your males are to appear in the presence of the \divine{Lord} God of Israel, \v{24}since I'm going to drive out nations before you, and enlarge your borders, and no one will covet your land, when you go up to appear in the presence of the \divine{Lord} your God three times a year.

\v{25}``You are not to offer the blood of my sacrifice with anything leavened, nor are you to allow the sacrifice of the Festival of Passover to remain until morning.

\v{26}``You are to bring the best\fnote{\fbackref{34:26} Or \fbib{the first}} of the first fruits of the ground to the house of the \divine{Lord} your God.

``You are not to boil a young goat in its mother's milk.''

\v{27}Then the \divine{Lord} told Moses, ``Write down these words, because I'm making a covenant with you and with Israel according to these words.''

\v{28}While Moses\fnote{\fbackref{34:28} Lit. he} was there with the \divine{Lord} for 40 days and 40 nights, he did not eat or drink.\fnote{\fbackref{34:28} Lit. \fbib{eat bread or drink water}} He wrote the Ten Commandments, the words of the covenant, on the tablets.
\passage{Moses' Face Shines}

\v{29}When Moses came down from Mount Sinai, he had the two tablets in his hand,\fnote{\fbackref{34:29} Lit. \fbib{hand as he came down from the mountain}} and he did not know that the skin of his face was ablaze with light because he had been speaking with God.\fnote{\fbackref{34:29} Lit. \fbib{him}} \v{30}Aaron and all the Israelis saw Moses and immediately noticed that the skin of his face was shining, and they were afraid to come near him. \v{31}When Moses called to them, Aaron and the leaders of the congregation returned to him, and he spoke to them. \v{32}Afterwards all the Israelis came near and he gave them everything the \divine{Lord} told him on Mount Sinai as commandments. \v{33}When Moses finished speaking with them he put a veil over his face, \v{34}and then whenever Moses would come in the \divine{Lord}'s presence to speak with him, he would remove the veil until he left the \divine{Lord}'s presence.\fnote{\fbackref{34:34} The Heb. lacks \fbib{the \divine{Lord}'s presence}} When he went out, he would tell the Israelis what he had been commanded. \v{35}The Israelis would see the face of Moses and that the skin of his face shone; then Moses would put the veil back over his face until he went in to speak with God.\fnote{\fbackref{34:35} Lit. \fbib{him}}
\labelchapt{35}
\passage{The Israelis Collect Material for the Tent}

\chapt{35}
\v{1}Moses assembled the entire congregation of the Israelis and told them, ``These are the things that the \divine{Lord} has commanded you to do:\fnote{\fbackref{35:1} Lit. \fbib{to do them}} \v{2}For six days work is to be done, but on the seventh day you are to have a holy day, a Sabbath of complete rest in dedication to the \divine{Lord}. Anyone who does work on that day is to be executed. \v{3}You are not to light a fire in any of your dwellings on the Sabbath.''

\v{4}Then Moses told the entire congregation of the Israelis, ``This is what the \divine{Lord} has commanded, \v{5}`Take from among yourselves an offering for the \divine{Lord}. Everyone whose heart is willing is to bring as an offering for the \divine{Lord}: gold, silver, and bronze; \v{6}blue, purple, and scarlet material;\fnote{\fbackref{35:6} The Heb. lacks \fbib{material}} fine linen and goat hair; \v{7}ram skins dyed red,\fnote{\fbackref{35:7} Or \fbib{tanned}} dolphin\fnote{\fbackref{35:7} Or \fbib{dugong}; i.e. a marine animal similar to a walrus or manatee} skins, acacia wood, \v{8}oil for lighting, spices for the anointing oil and for aromatic incense, \v{9}onyx stones, and stones for setting in the ephod and the breast piece.

\v{10}```Let everyone who is skilled\fnote{\fbackref{35:10} Lit. \fbib{wise of heart}} among you come and make everything that the \divine{Lord} has commanded: \v{11}the tent, its tent, its covering, its clasps, its boards, its bars, its pillars, and its sockets, \v{12}the ark, its poles, the Mercy Seat, the curtain,\fnote{\fbackref{35:12} I.e. the one that separated the Holy Place from the Most Holy Place} \v{13}the table, its poles, all its furnishings, and the bread of the presence, \v{14}the lamp stand for light, its furnishings, its lamps, and oil for the light, \v{15}the altar of incense, its poles, the anointing oil, the aromatic incense, and the screen for the doorway at the entrance to the tent, \v{16}the altar for burnt offerings, the bronze lattice for it, its poles, and all its furnishings, the basin and its base, \v{17}the hangings for the court, its pillars, its sockets,\fnote{\fbackref{35:17} Or \fbib{its bases}} the screen for the gate of the court, \v{18}the pegs for the tent, the pegs for the court, and their cords, \v{19}the woven garments for ministering in the Holy Place, the holy garments of Aaron the priest and the garments of his sons for serving as priests.'\,''

\v{20}Then the entire congregation of the Israelis withdrew from Moses' presence, \v{21}and every person whose heart moved him and all whose spirits prompted them, brought an offering to the \divine{Lord} for constructing\fnote{\fbackref{35:21} Lit. \fbib{for the work of}} the Tent of Meeting, for all its service, and for the holy garments. \v{22}Both the men and women came---all whose hearts prompted them---and brought brooches, earrings, rings, pendants, and all kinds of gold jewelry. Every person presented a wave offering of gold to the \divine{Lord}.

\v{23}Everyone who had blue, purple, and scarlet material, fine linen, goat hair, ram skins dyed red,\fnote{\fbackref{35:23} Or \fbib{tanned}} and dolphin\fnote{\fbackref{35:23} Or \fbib{dugong}, a marine animal similar to a walrus or manatee} skins brought them. \v{24}Everyone who could give an offering of silver and bronze brought it as a contribution for the \divine{Lord}. Also all who had acacia wood for any use in the work\fnote{\fbackref{35:24} Lit. \fbib{work of the service}} brought it.

\v{25}Every skilled\fnote{\fbackref{35:25} Lit. \fbib{wise of heart}} woman spun with her hands, and brought what she had spun: blue, purple, and scarlet material, and fine linen. \v{26}All the women who were skilled artisans\fnote{\fbackref{35:26} Lit. \fbib{whose hearts stirred them with skill} (or \fbib{wisdom})} spun the goat hair.

\v{27}The leaders brought onyx stones and stones to be set in the ephod and the breast piece, \v{28}spices and oil for the light and for the anointing oil and the aromatic incense. \v{29}Each Israeli man and woman whose heart was prompted brought something\fnote{\fbackref{35:29} The Heb. lacks \fbib{something}} as a freewill offering to the \divine{Lord} for all the work that the \divine{Lord} had commanded them to do through\fnote{\fbackref{35:29} Lit. \fbib{by the hand of}} Moses.
\passage{Craftsmen for Building the Tent}

\v{30}Moses told the Israelis, ``Look, the \divine{Lord} has called\fnote{\fbackref{35:30} Lit. \fbib{called by name}} Uri's son Bezalel, grandson of Hur, from the tribe of Judah, \v{31}and he has filled him with the Spirit of God, with wisdom, with understanding, and with knowledge of all kinds of work, \v{32}to make artistic designs, to work in gold, silver, and bronze, \v{33}to cut stones for setting, to carve wood, and to engage in all kinds of artistic work. \v{34}And he has given both him and Ahisamach's son Oholiab from the tribe of Dan the ability to teach. \v{35}He has equipped them\fnote{\fbackref{35:35} Lit. \fbib{has given them wisdom of heart}} to do all kinds of work done by an engraver, designer, embroider in blue, purple and scarlet material and in fine linen, or as a weaver. They were able to do\fnote{\fbackref{35:35} Lit. \fbib{doers of}} all kinds of work and were skilled designers.\chapt{36}
\v{1}Bezalel and Oholiab and all the skilled craftsmen to whom the \divine{Lord} gave wisdom and understanding to know how to do all the work in constructing\fnote{\fbackref{36:1} Lit. \fbib{for the service of}} the sanctuary are to do everything that the \divine{Lord} has commanded.''
\labelchapt{36}
\passage{Contributions for Building the Tent}

\v{2}Then Moses summoned Bezalel, Oholiab, and all the skilled\fnote{\fbackref{36:2} Lit. \fbib{wise of heart}} people to whom the \divine{Lord} had given ability,\fnote{\fbackref{36:2} Lit. \fbib{wisdom in his heart}} including everyone whose hearts stirred them to come forward to do the work. \v{3}They received from Moses all the offerings that the Israelis had brought for doing the work of constructing\fnote{\fbackref{36:3} Lit. \fbib{for the service of}} the sanctuary, and the people\fnote{\fbackref{36:3} Lit. \fbib{they}} continued to bring freewill offerings every morning. \v{4}All the craftsmen who were doing all the work on the sanctuary left the work they were doing \v{5}and told Moses, ``The people are bringing much more than enough for the work that the \divine{Lord} has commanded us to do.'' \v{6}Then Moses issued an order, and the message was taken throughout the camp, ``Men and women, don't bring any more offerings for the sanctuary.'' The people were restrained from bringing any more,\fnote{\fbackref{36:6} The Heb. lacks \fbib{any more}} \v{7}since the material was more than sufficient for doing all the work.

\v{8}All the skilled craftsmen among the workers made the tent with ten curtains of fine woven\fnote{\fbackref{36:8} Or \fbib{twisted}} linen, blue, purple, and scarlet material.\fnote{\fbackref{36:8} The Heb. lacks \fbib{material}} He\fnote{\fbackref{36:8} Perhaps Bezalel as the head of the skilled workers; and so through the rest of the book} made them with cherubim skillfully worked into them. \v{9}The length of each curtain was 28 cubits,\fnote{\fbackref{36:9} I.e. about 42 feet} and the width of each curtain two cubits.\fnote{\fbackref{36:9} I.e. about six feet} All the curtains had the same measurements.\fnote{\fbackref{36:9} Lit. \fbib{the measure of one for every curtain}} \v{10}He joined five of the curtains together, and the other five curtains he joined together. \v{11}He made loops of blue material\fnote{\fbackref{36:11} The Heb. lacks \fbib{material}} along the edge of the outermost curtain in the first set, and likewise, he made loops along the edge of the outermost curtain in the second set. \v{12}He made 50 loops in the one curtain, and he made 50 loops along the edge of the curtain that is in the second set, with the loops opposite each other. \v{13}Then he made 50 gold clasps, and joined the curtains to each other with the clasps so the tent was one piece.

\v{14}He made curtains of goat hair for a tent over the tent; he made 11 curtains. \v{15}The length of each curtain was 30 cubits,\fnote{\fbackref{36:15} I.e. about 45 feet} and the width of each curtain was two cubits;\fnote{\fbackref{36:15} I.e. about six feet} the measurements of each of the eleven curtains was the same.\fnote{\fbackref{36:15} Lit. \fbib{the measure of one for the eleven curtains}} \v{16}He joined five curtains by themselves, and six curtains by themselves. \v{17}He made 50 loops along the edge of the outermost curtain in the first set, and 50 loops along the edge of the curtain of the other set. \v{18}He made 50 bronze clasps to join the tent together so it would be one piece. \v{19}Then he made a cover for the tent of ram skins dyed red\fnote{\fbackref{36:19} Or \fbib{tanned}} and a covering of dolphin\fnote{\fbackref{36:19} Or \fbib{dugong}, a marine animal resembling a walrus or manatee} skins above that.

\v{20}Then he made upright boards of acacia wood for the tent. \v{21}Each\fnote{\fbackref{36:21} Lit. \fbib{the one}} board was ten cubits\fnote{\fbackref{36:21} I.e. about 15 feet} long, and one and a half cubits\fnote{\fbackref{36:21} I.e. about 27 inches} wide. \v{22}Each board had two pegs, joined to one another, and he did this for all the boards of the tent. \v{23}He made the boards for the tent: 20 boards for the south side.\fnote{\fbackref{36:23} Lit. \fbib{toward the Negev (south), toward Teman (a city to the south)}} \v{24}He made 40 silver sockets under the 20 boards: two sockets under one board for its two pegs and two sockets\fnote{\fbackref{36:24} Or \fbib{bases}} under the next\fnote{\fbackref{36:24} Lit. \fbib{the one}} board for its two pegs. \v{25}For the second side of the tent to the north he made 20 boards,\fnote{\fbackref{36:25} The Heb. lacks \fbib{he made}} \v{26}and 40 silver sockets for them, two sockets under one board and two sockets under the next\fnote{\fbackref{36:26} Lit. \fbib{the one}} board. \v{27}For the rear of the tent on the west he made six boards, \v{28}and he made two boards for the rear corners of the tent. \v{29}They were joined together\fnote{\fbackref{36:29} Lit. \fbib{twins}; perhaps designed with interlocking pieces} at the bottom and they were connected\fnote{\fbackref{36:29} Lit. \fbib{complete}; Perhaps the tops were joined together by a metal ring.} on top, by one ring. He did this for the two of them, and they were the two corners. \v{30}There were eight boards with their sixteen silver sockets, two sockets under each board.

\v{31}Then he made bars of acacia wood, five for the boards on one side of the tent, \v{32}five bars for the boards on the second side of the tent, and five bars for the boards on the back side of the tent to the west. \v{33}He made the middle bar in the center of the boards pass through from end to end. \v{34}He overlaid the boards with gold, and made gold rings for them as holders for the bars, and he overlaid the bars with gold.

\v{35}He made a curtain of blue, purple, and scarlet material, and fine woven linen. He made it with cherubim skillfully worked into it. \v{36}He made four pillars of acacia for it and overlaid them with gold, along with their gold hooks, and he cast four silver sockets for them. \v{37}For the doorway of the tent, he made a screen of blue, purple, and scarlet material and fine woven linen, the work of an embroiderer, \v{38}and five pillars of acacia along with their hooks. He overlaid their tops and their bands\fnote{\fbackref{36:38} Perhaps a kind of connecting rod joining the pillars together} with gold. Their five sockets were made of bronze.
\labelchapt{37}
\passage{The Ark of the Covenant}

\chapt{37}
\v{1}Bezalel made the ark of acacia wood two and a half cubits\fnote{\fbackref{37:1} I.e. about 45 inches} long, one and a half cubits\fnote{\fbackref{37:1} I.e. about 27 inches} wide, and one and a half cubits\fnote{\fbackref{37:1} I.e. about 27 inches} high. \v{2}He overlaid it with pure gold, inside and outside, and made a gold molding around it. \v{3}He cast four rings for it on its four feet, two rings on one side of it and two rings on its other side. \v{4}He made poles of acacia wood and overlaid them with gold. \v{5}He put the poles into the rings on the sides of the ark to carry\fnote{\fbackref{37:5} Lit. \fbib{with which to carry}} it.

\v{6}He made a Mercy Seat of pure gold two and a half cubits\fnote{\fbackref{37:6} I.e. about 45 inches} long and one and a half cubits\fnote{\fbackref{37:6} I.e. about 27 inches} wide. \v{7}He made two cherubim of gold; he made them of hammered work at the two ends of the Mercy Seat. \v{8}One cherub was at one end and one cherub at the other end. He made the cherubim at the two ends of the Mercy Seat and of one piece with it. \v{9}The cherubim had their wings spread upward, covering the Mercy Seat with their wings and facing each other. The faces of the cherubim were turned toward the Mercy Seat.
\passage{The Table of Showbread}

\v{10}Then he made a table of acacia wood two cubits\fnote{\fbackref{37:10} I.e. about six feet} long, one cubit\fnote{\fbackref{37:10} I.e. about one and a half feet} wide, and one and a half cubits\fnote{\fbackref{37:10} I.e. about 27 inches} high. \v{11}He overlaid it with pure gold and put a gold molding around it. \v{12}He made a rim one handbreadth\fnote{\fbackref{37:12} I.e. about five inches} wide around it, and made a gold molding around the rim. \v{13}He cast four gold rings for it and put the rings on the four corners where its four feet were. \v{14}The rings were close to the rim as holders for the poles to carry the table. \v{15}He made the poles of acacia wood and overlaid them with gold to carry the table. \v{16}He made the utensils which were on the table, its plates, dishes, bowls, and jars out of which libations are poured. He made them of pure gold.
\passage{The Lamp Stand}

\v{17}He made the lamp stand of pure gold. He made the lamp stand, its base, and stem of hammered work and its cups, calyxes, and flowers were of one piece with it. \v{18}Six branches extended from its sides, three branches of the lamp stand from one side of it, and three branches of the lamp stand from its other side. \v{19}Three cups shaped like almond blossoms with calyxes and flowers were on one branch and three cups shaped like almond blossoms with calyxes and flowers were on the other branch, and so on for the six branches extending from the lamp stand. \v{20}On the lamp stand itself there were four cups shaped like almond blossoms each with their calyxes and flowers. \v{21}A calyx was under the two branches that extended out of the stem;\fnote{\fbackref{37:21} Lit. \fbib{out of it}} a calyx was under the next pair of\fnote{\fbackref{37:21} Lit. \fbib{under two}} branches that extended out of the stem;\fnote{\fbackref{37:21} Lit. \fbib{out of it}} and a calyx was under the last pair of\fnote{\fbackref{37:21} Lit. \fbib{under two}} branches that extended out of the stem,\fnote{\fbackref{37:21} Lit. \fbib{out of it}} and so on for the six branches extending from the lamp stand. \v{22}Their calyxes and their branches were of one piece with it, all of it was of one piece of hammered work of pure gold. \v{23}He made its seven lamps, its tongs, and its trays from pure gold. \v{24}He made it and all of its furnishings from a talent\fnote{\fbackref{37:24} I.e. about 75 pounds} of pure gold.
\passage{The Altar for Incense}

\v{25}He made the altar for burning incense of acacia wood, a square, one cubit\fnote{\fbackref{37:25} I.e. about one and a half feet} long, one cubit\fnote{\fbackref{37:25} I.e. about one and a half feet} wide, and two cubits\fnote{\fbackref{37:25} I.e. about six feet} high, with its horns of one piece with it. \v{26}He overlaid it with pure gold---its top, its sides all around, and its horns---and he made a gold molding around it. \v{27}He made two gold rings for it under its molding, on its two opposite sides, as holders for poles by which to carry it. \v{28}He made the poles of acacia wood and overlaid them with gold. \v{29}And he made the holy anointing oil and the pure aromatic incense, the work of a perfumer.
\labelchapt{38}
\passage{The Altar for Burnt Offerings}

\chapt{38}
\v{1}Then he made the altar for burnt offerings of acacia wood. It was a square, five cubits\fnote{\fbackref{38:1} I.e. about seven and a half feet} long and five cubits\fnote{\fbackref{38:1} I.e. about seven and a half feet} wide, and it was three cubits\fnote{\fbackref{38:1} Ie. About four and a half feet} high. \v{2}He made horns\fnote{\fbackref{38:2} Lit. \fbib{its horns}} on its four corners. Its horns were of one piece with it, and he overlaid it with bronze. \v{3}He made all the utensils for the altar---the pans, the shovels, the bowls, the forks, and the fire-pans---and he made all its utensils of bronze. \v{4}He made a lattice, a netting of bronze, for the altar. It was under its ledge, extending halfway up. \v{5}He cast four rings on the four ends of the bronze lattice as holders for the poles. \v{6}He made poles of acacia wood and overlaid them with bronze. \v{7}And he put the poles through rings on the sides of the altar to carry it.\fnote{\fbackref{38:7} Lit. \fbib{by which to carry it}} He made it hollow, out of boards.
\passage{The Bronze Basin}

\v{8}He made the bronze basin and its bronze base from\fnote{\fbackref{38:8} Lit. \fbib{with}} mirrors contributed by the women who served at the entrance to the Tent of Meeting.
\passage{The Court of the Tent}

\v{9}Then he made the court. On the south\fnote{\fbackref{38:9} Lit. \fbib{toward the Negev, southward}} side the hangings for the court were made of fine woven linen, 100 cubits\fnote{\fbackref{38:9} I.e. about 150 feet} long.\fnote{\fbackref{38:9} The Heb. lacks \fbib{long}} \v{10}He made their 20 pillars\fnote{\fbackref{38:10} The Heb. lacks \fbib{20 pillars}} and their 20 sockets of bronze, while the hooks of the pillars and their bands\fnote{\fbackref{38:10} Perhaps a kind of connecting rod joining the pillars together} were made of silver. \v{11}The north side was 100 cubits\fnote{\fbackref{38:11} I.e. about 150 feet} long,\fnote{\fbackref{38:11} The Heb. lacks \fbib{long}} and its\fnote{\fbackref{38:11} Lit. \fbib{their}} 20 pillars\fnote{\fbackref{38:11} The Heb. lacks \fbib{20 pillars}} and 20 sockets were made of bronze, and the hooks of the pillars and their bands\fnote{\fbackref{38:11} Perhaps a kind of connecting rod joining the pillars together} were made of silver. \v{12}For the west side there were hangings 50 cubits\fnote{\fbackref{38:12} I.e. about 75 feet} long with their ten pillars and ten sockets. The hooks of the pillars and their bands were made of silver. \v{13}The east side\fnote{\fbackref{38:13} Lit. \fbib{on the east side toward the rising (of the sun)}} was 50 cubits\fnote{\fbackref{38:13} I.e. about 75 feet} long.\fnote{\fbackref{38:13} The Heb. lacks \fbib{long}} \v{14}The hangings for one section\fnote{\fbackref{38:14} Lit. \fbib{the shoulder}} were fifteen cubits\fnote{\fbackref{38:14} I.e. about 22 and a half feet} long, with their three pillars and three sockets, \v{15}and also for the second section. On either side of the gate of the court were hangings of fifteen cubits\fnote{\fbackref{38:15} I.e. about 22 and a half feet} long with their three pillars and three sockets. \v{16}All the hangings around the court were made of fine woven linen. \v{17}The sockets for the pillars were made of bronze and the hooks of the pillars and their bands\fnote{\fbackref{38:17} Perhaps a kind of connecting rod joining the pillars together} were made of silver. Their tops were overlaid with silver, and all the pillars of the court were banded with silver. \v{18}The screen of the gate of the court was the work of an embroiderer of blue, purple, and scarlet material, and fine woven linen. The length was 20 cubits\fnote{\fbackref{38:18} I.e. about 30 feet} and it was five cubits\fnote{\fbackref{38:18} I.e. about seven and a half feet} high along its width, corresponding to the hangings of the court. \v{19}Their four pillars and their four sockets were made of bronze, and their hooks were made of silver. Their tops were overlaid with silver and their bands were made of silver. \v{20}All the pegs for the tent and for all around the court were made of bronze.
\passage{The Record of Materials}

\v{21}Here is a summary of materials for the Tent of Meeting that was compiled at Moses' direction, the work of the descendants of Levi under the direction of Aaron the priest's son Ithamar. \v{22}Now Uri's son Bezalel, grandson of Hur from the tribe of Judah, made everything that the \divine{Lord} had ordered Moses to build.\fnote{\fbackref{38:22} The Heb. lacks \fbib{to build}} \v{23}With him was Ahisamach's son Oholiab from the tribe of Dan, an engraver, designer, and embroiderer in blue, purple, and scarlet material, and of fine linen.

\v{24}All the gold that was used in the work, in all the work on the sanctuary, including\fnote{\fbackref{38:24} Lit. \fbib{it was}} the gold from the wave offering, totaled\fnote{\fbackref{38:24} Lit. \fbib{was}} 29 talents,\fnote{\fbackref{38:24} I.e. 2,175 pounds; a talent weighed about 75 pounds} 730 shekels,\fnote{\fbackref{38:24} 3,000 shekels made one talent.} according to the standard used in\fnote{\fbackref{38:24} Lit. \fbib{the shekel of the}} the sanctuary. \v{25}The silver from those of the congregation who were recorded\fnote{\fbackref{38:25} Or \fbib{numbered}} totaled\fnote{\fbackref{38:25} Lit. \fbib{was}} 100 talents\fnote{\fbackref{38:25} I.e. 7,500 pounds; a talent weighed about 75 pounds} and 1,775 shekels, according to the standard used in\fnote{\fbackref{38:25} Lit. \fbib{the shekel of the}} the sanctuary; \v{26}a beka a head (a beka is half a shekel, according to the standard used in\fnote{\fbackref{38:26} Lit. \fbib{the shekel of the}} the sanctuary) for everyone who went through the registration\fnote{\fbackref{38:26} Or \fbib{who were numbered}} process\fnote{\fbackref{38:26} Lit. \fbib{who passed over to those who were registered}} from 20 years old and older. The total numbered 603,550 bekas.

\v{27}One hundred talents\fnote{\fbackref{38:27} I.e. 7,500 pounds; a talent weighed about 75 pounds} of silver were used to cast the sockets for the sanctuary and the sockets for the curtain, 100 sockets for 100 talents,\fnote{\fbackref{38:27} I.e. 7,500 pounds; a talent weighed about 75 pounds} a talent\fnote{\fbackref{38:27} I.e. 75 pounds; a talent weighed about 75 pounds} per socket. \v{28}And with 1,775 talents\fnote{\fbackref{38:28} The Heb. lacks \fbib{talents}} he made hooks for the pillars, overlaid their tops, and made bands for them.

\v{29}The bronze from the wave offering totaled\fnote{\fbackref{38:29} Lit. \fbib{was}} 70 talents\fnote{\fbackref{38:29} I.e. 5,250 pounds; a talent weighed about 75 pounds} and 2,400 shekels. \v{30}With it he made the sockets for the doorway to the Tent of Meeting, the bronze altar, the bronze lattice for it, all the furnishings\fnote{\fbackref{38:30} Or \fbib{utensils}} for the altar, \v{31}the sockets for all around the court, the sockets for the gate to the court, all the pegs for the sanctuary, and all the pegs for all around the court.
\labelchapt{39}
\passage{The Priestly Garments}

\chapt{39}
\v{1}From the blue, purple, and scarlet material they made finely woven garments for ministering in the Holy Place, and they made the holy garments for Aaron, just as the \divine{Lord} commanded Moses.
\passage{The Ephod}

\v{2}He made the ephod out of gold, blue, purple, and scarlet material and fine woven linen. \v{3}They hammered out gold sheets and cut off threads to work into the blue, purple, and scarlet material and into the fine linen, a work of skillful design. \v{4}They made connecting shoulder pieces for the ephod\fnote{\fbackref{39:4} Lit. \fbib{for it}} and attached them to its two edges. \v{5}The skillfully woven band that was on it was made like it, of one piece with it: of gold, blue, purple, and scarlet material and fine woven linen, just as the \divine{Lord} commanded Moses. \v{6}They prepared the onyx stones, engraved with the names of the sons of Israel like the engraving on a signet,\fnote{\fbackref{39:6} I.e. a type of seal used to indicate ownership} and mounted them in settings of gold filigree. \v{7}He put them on the shoulder pieces of the ephod as stones of remembrance for the sons of Israel, just as the \divine{Lord} commanded Moses.
\passage{The Breast Piece}

\v{8}He made a breast piece, skillfully worked, like the work of the ephod: of gold, blue, purple, and scarlet material and fine woven linen. \v{9}They made the breast piece square when folded double: one span\fnote{\fbackref{39:9} I.e. about nine inches} in length and one span\fnote{\fbackref{39:9} I.e. about nine inches} in width when folded double. \v{10}They mounted on it four rows of stones. The first row was a row of carnelian, topaz, and emerald; \v{11}the second row ruby,\fnote{\fbackref{39:11} Or \fbib{turquoise}} sapphire, and crystal; \v{12}the third row jacinth, agate, and amethyst; \v{13}and the fourth row beryl, onyx, and jasper. They were set in gold filigree when they were mounted. \v{14}The stones corresponded to the names of the sons of Israel, twelve stones\fnote{\fbackref{39:14} The Heb. lacks \fbib{stones}} corresponding to their names, with the engraving of a signet,\fnote{\fbackref{39:14} I.e. A type of seal used to indicate ownership} each with the name of one of the twelve tribes.

\v{15}They made chains of pure gold twisted like cords for the breast piece. \v{16}They made two settings of gold filigree and two gold rings, and they put the two rings on the two edges of the breast piece. \v{17}They put the two gold cords on the two gold rings at the edges of the breast piece, \v{18}and they attached the other two ends of the two cords to the filigree settings, and then attached them to the shoulder pieces of the ephod in front. \v{19}They made two gold rings and attached them to the two edges of the breast piece, on the side of it which is toward the inner side of the ephod. \v{20}They made two gold rings and attached them in front, on the lower part of the two shoulder pieces of the ephod, close to the place where it's joined, above the skillfully woven band of the ephod. \v{21}They tied the breast piece by its rings to the rings of the ephod with a blue cord so it would rest on the skillfully woven band of the ephod and so the breast piece would not come loose from the ephod.
\passage{The Robe of the Ephod}

\v{22}He made the robe of the ephod of woven work, entirely of blue. \v{23}The opening of the robe was in the middle, like the opening of a coat of mail, with a binding around the opening so it would not be torn. \v{24}On the hem of the robe, they placed pomegranates made of blue, purple, and scarlet material and woven linen. \v{25}They made bells of pure gold, and put the bells between\fnote{\fbackref{39:25} Lit. \fbib{among}} the pomegranates, on the hem of the robe, all around between\fnote{\fbackref{39:25} Lit. \fbib{among}} the pomegranates. \v{26}There was a bell and a pomegranate, then\fnote{\fbackref{39:26} The Heb. lacks \fbib{then}} a bell and a pomegranate, all around the hem of the robe for when the High Priest ministered,\fnote{\fbackref{39:26} Lit. \fbib{for ministering}} just as the \divine{Lord} commanded Moses.
\passage{The Other Priestly Garments}

\v{27}They made tunics for Aaron and his sons, woven from fine linen, \v{28}the turban of fine linen, decorated head coverings of fine linen, linen undergarments of fine woven linen, \v{29}and the sash of fine woven linen, woven of blue, purple, and scarlet material, just as the \divine{Lord} had commanded Moses. \v{30}They made the medallion\fnote{\fbackref{39:30} Or \fbib{plate}} for the holy crown of pure gold, and they wrote on it an inscription like the engraving on a seal: ``Holy to the \divine{Lord}.'' \v{31}They fastened a blue cord to it in order to fasten it on the turban above, as the \divine{Lord} had commanded Moses.
\passage{Moses Inspects the Completed Work}

\v{32}All the work on the Tent of Meeting was completed, and the Israelis had crafted it according to everything that the \divine{Lord} had commanded Moses, as they should have.\fnote{\fbackref{39:32} Lit. \fbib{Moses. So they had done.}} \v{33}They brought to Moses the tent, all its furnishings, its clasps, its boards, its bars, its pillars, its sockets, \v{34}the covering of ram skins dyed red,\fnote{\fbackref{39:34} Or \fbib{of tanned ram skins}} the covering of dolphin\fnote{\fbackref{39:34} Or \fbib{dugong}; i.e. a marine animal similar to a walrus or manatee} skins, the curtain,\fnote{\fbackref{39:34} I.e. the one that separates the Holy Place from the Most Holy Place} \v{35}the Ark of the Testimony and its poles, the Mercy Seat, \v{36}the table and all its utensils, the bread of the presence, \v{37}the lamp stand of pure gold,\fnote{\fbackref{39:37} Lit. \fbib{the pure lamp stand}} its lamps (with the lamps in order), its furnishings, its oil for lighting, \v{38}the altar of gold, anointing oil, aromatic incense, the screen for the doorway to the tent, \v{39}the bronze altar and the bronze lattice for it, its poles, all its furnishings, the basin and its base, \v{40}the hangings for the court, its pillars, its sockets, the screen for the gate of the court, its cords, its pegs, all the furnishings for the service of the tent, for the Tent of Meeting, \v{41}the woven garments for Aaron the priest for ministering in the Holy Place, and the garments for his sons for serving as priests. \v{42}The Israelis had done all the work according to all that the \divine{Lord} had commanded Moses. \v{43}Moses blessed them when he saw all the work and that they had completed it. They had done it just as the \divine{Lord} had commanded.
\labelchapt{40}
\passage{The \divine{Lord}'s Instructions for Setting up the Tent}

\chapt{40}
\v{1}The \divine{Lord} spoke to Moses: \v{2}``On the first day of the first month you are to set up the tent of the Tent of Meeting. \v{3}You are to put the Ark of the Testimony there, and screen off the ark with the curtain. \v{4}You are to bring in the table and properly arrange what goes on it.\fnote{\fbackref{40:4} Lit. \fbib{arrange its arrangement}} Then you are to bring in the lamp stand and set up its lamps.

\v{5}``You are to put the golden altar for incense in front of the Ark of the Testimony and then set up the screen for the doorway to the tent. \v{6}You are to put the altar for burnt offerings in front of the doorway of the tent of the Tent of Meeting. \v{7}You are to put the basin between the Tent of Meeting and the altar and put water in it.\fnote{\fbackref{40:7} Lit. \fbib{there}} \v{8}You are to set up the court all around, and hang up the screen for the gate of the court. \v{9}You are to take the anointing oil and anoint the tent and all that is in it. You are to consecrate it and all its furnishings and it will be holy.

\v{10}``You are to anoint the altar for burnt offerings and all its utensils. You are to consecrate the altar and the altar will be most holy. \v{11}You are to anoint the basin and its base and consecrate it. \v{12}Then you are to bring Aaron and his sons to the doorway of the Tent of Meeting, and wash them with water. \v{13}You are to clothe Aaron with the holy garments, you are to anoint him, and consecrate him so he may serve me as priest. \v{14}You are to bring his sons and clothe them with tunics. \v{15}You are to anoint them just as you anointed their father so they may serve me as priests. Their anointing is to qualify them\fnote{\fbackref{40:15} Lit. \fbib{shall be to them}} to belong to a perpetual priesthood from generation to generation.''
\passage{Moses Obeys God's Instructions}

\v{16}Moses did everything that the \divine{Lord} had commanded him, so he did. \v{17}And so in the first month of the second year, on the first day of the month, the tent was set up. \v{18}Moses set up the tent. He installed its sockets and set its boards in place. He inserted its bars and set up its pillars. \v{19}He spread the tent over the tent and put the covering of the tent on top of it, just as the \divine{Lord} had commanded him.\fnote{\fbackref{40:19} Lit. \fbib{Moses}} \v{20}Then he took the Testimony, put it into the ark, and placed the poles on the ark. He then put the Mercy Seat on top of the ark. \v{21}He brought the ark into the tent, set up the curtain, and screened off the Ark of the Testimony, just as the \divine{Lord} had commanded him.\fnote{\fbackref{40:21} Lit. \fbib{Moses}} \v{22}He put the table in the Tent of Meeting, on the north side of the tent, outside the curtain, \v{23}and properly arranged the bread on it in the \divine{Lord}'s presence, just as the \divine{Lord} had commanded him.\fnote{\fbackref{40:23} Lit. \fbib{Moses}}

\v{24}Then he put the lamp stand in the Tent of Meeting, opposite the table on the south side of the tent, \v{25}and set up the lamps in the \divine{Lord}'s presence, just as the \divine{Lord} had commanded him.\fnote{\fbackref{40:25} Lit. \fbib{Moses}} \v{26}He put the golden altar in the Tent of Meeting in front of the curtain \v{27}and burned aromatic incense on it, just as the \divine{Lord} had commanded Moses.

\v{28}He set up the screen for the doorway of the tent. \v{29}He put the altar for burnt offerings at the doorway of the tent of the Tent of Meeting, and offered the burnt offering and the grain offering on it, just as the \divine{Lord} had commanded him.\fnote{\fbackref{40:29} Lit. \fbib{Moses}} \v{30}He put the basin between the Tent of Meeting and the altar, and put water in it for washing. \v{31}Moses, Aaron, and his sons washed their hands and feet from it. \v{32}When they entered the Tent of Meeting and approached the altar, they washed, just as the \divine{Lord} had commanded him.\fnote{\fbackref{40:32} Lit. \fbib{Moses}} \v{33}He set up the court all around the tent and the altar, and hung up the screen for the gate of the court. And so Moses finished the work.
\passage{The Glory of the \divine{Lord} Fills the Completed Tent}

\v{34}The cloud covered the Tent of Meeting, and the glory of the \divine{Lord} filled the tent. \v{35}Moses was not able to enter the Tent of Meeting because the cloud had settled on it, and the glory of the \divine{Lord} filled the tent. \v{36}Whenever the cloud was lifted up from the tent, the Israelis would set out on their journey, \v{37}but if the cloud was not lifted up, they would not set out until\fnote{\fbackref{40:37} Lit. \fbib{until the time when}} it was lifted up, \v{38}since the cloud of the \divine{Lord} was over the tent by day, and the fire was in it by night, in the sight of all the house of Israel in all their journeys.

\bookheader{Leviticus}
\labelbook{Lev}

\bookpretitle{The Third Book of the Law called}
\booktitle{Leviticus}

\labelchapt{1}
\passage{Burnt Offerings}

\chapt{1}
\v{1}The \divine{Lord} told Moses from the middle of the Tent of Meeting, \v{2}``Speak to the Israelis and tell them that when any person\fnote{Lit. \fbib{man}} brings an offering to the \divine{Lord} from among you, whether he brings on offering of animals from either cattle or flock, \v{3}if his offering is a burnt offering from the herd, he is to bring a male without any defect. He is to present it at the entrance to the Tent of Meeting. At the appointed time, it is to be presented in the presence of the \divine{Lord} so that he may be accepted. \v{4}He is to lay his hand on the head of the burnt offering, and it will be accepted for him as an atonement on his behalf. \v{5}Then he is to slaughter the young bull\fnote{Or \fbib{calf}} in the \divine{Lord}'s presence.''
\passage{General Instructions}

``Aaron's sons, the priests, are to bring the blood and sprinkle it\fnote{Lit. \fbib{the blood}} around the altar that stands at the entrance to the Tent of Meeting. \v{6}He is to skin the burnt offering and cut it into pieces. \v{7}Aaron's sons, the priests, are to build a fire on the altar and arrange the wood over the fire. \v{8}They\fnote{Lit. \fbib{Then Aaron's sons, the priests,}} are to arrange the pieces of meat---including the head and the fat---on the wood over the fire that burns on the altar. \v{9}Then he is to wash its entrails and legs with water. After this, the priest is to offer all of it on the altar---a burnt offering by fire, an aroma that will be pleasing to the \divine{Lord}.''
\passage{Burnt Offerings of Livestock}

\v{10}``If his offering is a burnt offering from the flock, whether lamb or goat, he is to bring a male without any defect \v{11}and slaughter it at the north side of the altar in the \divine{Lord}'s presence. Then Aaron's sons, the priests, are to sprinkle its blood around the altar. \v{12}He is to cut up its head and fat into separate pieces arrange them in rows on the wood over the fire that burns on the altar, \v{13}wash its entrails and legs with water, and then offer all of it on the altar---a burnt offering by fire, an aroma that will be pleasing to the \divine{Lord}.''
\passage{Burnt Offerings of Birds}

\v{14}``If his offering is a burnt offering of birds to the \divine{Lord}, he is to bring turtledoves or young doves. \v{15}The priest is to bring it to the altar to offer it up in smoke. He is to decapitate it and drain its blood on the side\fnote{Lit. \fbib{wall}} of the altar, \v{16}and then he is to eviscerate it and throw the viscera and the feathers to the east side of the altar, where the fatty ashes are located. \v{17}He is then to tear it open by its wings, but not divide it completely into two parts. The priest is then to offer all of it on the wood over the fire as a burnt offering by fire, an aroma pleasing to the \divine{Lord}.''
\labelchapt{2}
\passage{Grain Offerings}

\chapt{2}
\v{1}``When a person brings an offering---that is, a grain offering---to the \divine{Lord}, his offering is to consist of fine flour. He is to pour olive oil mixed with frankincense over it. \v{2}Then he is to bring it to Aaron's sons, the priests. He is to take a handful of fine flour, the olive oil, and all of the frankincense. Then the priest is to offer a memorial offering by fire, an aroma pleasing to the \divine{Lord}. \v{3}The remnants from the grain offering is for Aaron and his sons---the holiest\fnote{Or \fbib{most holy}} of the offerings made by fire to the \divine{Lord}.''
\passage{Burnt Offerings of Grain}

\v{4}``When you bring an offering---that is, a grain offering baked in an oven---it is to consist of fine flour baked into unleavened bread mixed with olive oil or of wafers made of unleavened bread and smeared with olive oil.

\v{5}``If your grain offering has been prepared on\fnote{The Heb. lacks \fbib{has been prepared on}} a griddle, then it is to consist of fine flour mixed with olive oil. \v{6}Crumble it into morsels of bread and then pour olive oil on it. It's a grain offering.

\v{7}``When your grain offering has been prepared in\fnote{The Heb. lacks \fbib{has been prepared in}} a stew pan, it is to consist of fine flour mixed with olive oil. \v{8}Bring the grain offering that you prepared from these ingredients to the \divine{Lord}. Present it to the priest, who will bring it to the altar. \v{9}Then the priest will dedicate\fnote{Lit. \fbib{exalt}} some of the grain offering as a memorial offering and offer it in smoke on the altar, an offering by fire that will be a pleasing aroma to the \divine{Lord}. \v{10}The remainder from the memorial offering is for Aaron and his sons---the holiest\fnote{Or \fbib{most holy}} of the offerings made by fire to the \divine{Lord}.''
\passage{Prohibitions Regarding Yeast}

\v{11}``Any grain offering that you bring to the \divine{Lord} is not to be prepared with yeast, because anything with leaven and honey may not be offered in smoke as an offering by fire to the \divine{Lord}. \v{12}You may bring them to the \divine{Lord} as an offering of first fruits, but they are not to be offered on the altar for a pleasing aroma.''
\passage{Requirements for Salt}

\v{13}``Also, be sure to rub every offering from your grain offering with salt. You are not ever to remove the salt of the covenant of your God from your grain offering. Present all your offerings with salt.''
\passage{First Fruit Offerings}

\v{14}``Whenever you bring a grain offering of first fruits to the \divine{Lord}, bring fresh\fnote{Lit. \fbib{bring young ears of}} barley roasted\fnote{Or \fbib{parched}} in fire, young kernels crushed into bits. Bring the grain offering with your first fruits \v{15}and then pour olive oil and frankincense over it as a grain offering. \v{16}The priest is to offer the memorial offering in smoke---its crushed bits, olive oil, and frankincense---as an offering by fire to the \divine{Lord}.''
\labelchapt{3}
\passage{Peace Offerings}

\chapt{3}
\v{1}``If someone's\fnote{Lit. \fbib{his}} offering is a peace offering\fnote{Or \fbib{sacrifice of peace}} from the cattle, the presenter\fnote{The Heb. lacks \fbib{presenter}, and so throughout the chapter} is to offer it without defect, whether the animal\fnote{Lit. \fbib{whether it}} is male or female. They are to be brought to the \divine{Lord}. \v{2}Then the presenter is to lay his hand on the head of the offering and slaughter it at the entrance to the Tent of Meeting. After this, Aaron's sons, the priests, are to sprinkle the blood on and around the altar.

\v{3}``The presenter is then to bring a gift from the peace offering, an offering made by fire to the \divine{Lord}. He is to remove the fat that covers the internal organs,\fnote{Or \fbib{inward parts}} all of the fat that is inside the internal organs, \v{4}the two kidneys with the fat on them by the loins, and the fatty mass\fnote{Or \fbib{appendage}} that surrounds the liver and kidneys. \v{5}Then Aaron's sons are to burn them on the altar, over the burnt offering that has been placed on the wood over the fire, as an offering made by fire, an aroma pleasing to the \divine{Lord}.

\v{6}``If his offering to the \divine{Lord} is a peace offering from the flock, whether male or female, he is to bring it without defect. \v{7}If the offering that he is bringing is a lamb, then he is to bring it to the \divine{Lord}. \v{8}He is to lay his hand on the head of his offering and slaughter it at the entrance to the Tent of Meeting. Then Aaron's sons are to sprinkle the blood on and around the altar.

\v{9}``The presenter is then to bring a gift from the peace offering as an offering made by fire to the \divine{Lord}. He is to remove the fat---the entire fat tail near the spine, the fat that covers the internal organs, all of the fat that is inside the internal organs, \v{10}the two kidneys with the fat on them by the loins, and the fatty mass\fnote{Or \fbib{appendage}} that surrounds the liver and kidneys. \v{11}Then the priest is to burn them on the altar as a food offering made by fire to the \divine{Lord}.

\v{12}``If his offering is a goat, then he is to bring it to the \divine{Lord}, \v{13}lay his hand over its head, then slaughter it at the entrance to the Tent of Meeting. After this, Aaron's sons are to sprinkle the blood on and around the altar.

\v{14}``The presenter is then to present the gift as an offering made by fire to the \divine{Lord}, that is, the fat that covers the internal organs, all the fat that is inside the internal organs, \v{15}the two kidneys with the fat on them by the loins, and the fatty mass\fnote{Or \fbib{appendage}} that surrounds the liver and kidneys. \v{16}The priest is to burn it on the altar, a food offering made by fire, a pleasing aroma. All the fat belongs to the \divine{Lord}.

\v{17}``This is to be a lasting statute for all your generations, wherever you live. You are not to eat any fat or blood.''
\labelchapt{4}
\passage{Personal Sin Offerings}

\chapt{4}
\v{1}The \divine{Lord} told Moses, \v{2}``Speak to the Israelis and tell them that if a person inadvertently sins with respect to any of the \divine{Lord}'s commands that should not be violated, but nevertheless he disobeys one of them, \v{3}or if the anointed priest sins, thereby bringing guilt on the people, let him bring a young bull\fnote{Lit. \fbib{a bull, a son of a bull}} without defect as a sin offering to the \divine{Lord} for his sin that he had committed.

\v{4}``He is to bring the bull to the entrance to the Tent of Meeting, into the \divine{Lord}'s presence, where he is to lay his hand on the head of the bull and slaughter it in the \divine{Lord}'s presence. \v{5}The anointed priest is to take\fnote{The Heb. lacks the word \fbib{takes} . . . \fbib{some}, and so with vss. 5, 16, 30, 34} blood from the bull to the Tent of Meeting. \v{6}The priest is to dip his finger in the blood and sprinkle some of the blood seven times in the \divine{Lord}'s presence in front of the curtain of the sanctuary.

\v{7}``The priest is then to put some blood on the horn of the altar that is near the Tent of Meeting as an incense of pleasing aroma in the \divine{Lord}'s presence. He is to pour the rest of the bull's blood\fnote{Lit. \fbib{all of the blood}} for a burnt offering at the base of the altar that is at the entrance to the Tent of Meeting. \v{8}Then he is to remove all the fat from the bull for a sin offering---that is, the fat that covers the internal organs,\fnote{Or \fbib{inward parts}} all of the fat that is inside the internal organs, \v{9}the two kidneys with the fat on them by the loins, and the fatty mass\fnote{Or \fbib{appendage}} surrounding the liver and kidneys--- \v{10}just as it is taken from the bull for a peace offering. Then the priest is to burn it on the altar for burnt offerings.

\v{11}``Now as for the bull's hide, its flesh, its head, its legs, its internal organs, and its dung, \v{12}along with the rest of the bull, he is to bring it outside the camp to a clean place, where fat ashes are to be poured over it and then it is to be thoroughly burned over wood with fire. It is to be burned where the fat ashes are poured out.''
\passage{National Sin Offerings}

\v{13}``If the whole congregation of Israel goes astray, and if the sin is hidden from the eyes of the assembly, and if they go astray from one of the \divine{Lord}'s commands that should not be violated, then they will stand guilty. \v{14}When the sin that they have committed becomes known, the entire congregation is to bring a young bull as a sin offering to the Tent of Meeting, \v{15}where the elders of the community are to lay their hands on the head of the bull in the \divine{Lord}'s presence and slaughter it.\fnote{Lit. \fbib{the bull in the \divine{Lord}'s presence}} \v{16}The anointed priest is to take blood from the bull and bring it to the Tent of Meeting. \v{17}Then the priest is to dip his finger in the blood, sprinkle some of the blood seven times in front of the curtain in the \divine{Lord}'s presence, \v{18}then put blood on the horn of the altar near the Tent of Meeting in the \divine{Lord}'s presence. He is to pour the rest of the blood as a burnt offering at the base of the altar that is at the entrance to the Tent of Meeting. \v{19}Then he is to remove all the fat from the bull for a sin offering and burn it on the altar. \v{20}He is to do to this bull what he did to the bull for the sin offering. He is to do it this way so that the priest will make atonement for them and they will be forgiven. \v{21}Then he is to bring the rest of the bull outside the camp and burn it just as he had burned the first bull. This is the sin offering for the congregation.''
\passage{Sin Offerings for Rulers}

\v{22}``When a ruler inadvertently sins, disobeying any one of the commands of the \divine{Lord} his God that should not be violated, he will be guilty. \v{23}When the sin that he had committed is disclosed to him, he is to bring his offering: a male goat without defect. \v{24}He is then to lay his hand on the head of the goat and slaughter it at the place where the burnt offering is slaughtered---in the \divine{Lord}'s presence---as a sin offering. \v{25}Then the priest is to take blood from the sin offering with his finger, put it on the horn of the altar that is used for burnt offerings, and then pour the rest of the blood at the base of the altar that is used for burnt offerings. \v{26}He is to burn all the fat on the altar as is done for the fat for the sacrifice of a peace offering. This is how the priest will make atonement for him concerning his sin. It will be forgiven him.''
\passage{Sin Offerings for the People}

\v{27}``If any\fnote{Lit. \fbib{soul}} of the common people of the land inadvertently sins by disobeying one of the \divine{Lord}'s commands that should not be violated, he will be guilty. \v{28}When the sin that he committed is disclosed to him, he is to bring his offering for his sin that he had committed: a female goat without defect. \v{29}He is to lay his hand on the head of the sin offering and slaughter it\fnote{Lit. \fbib{the sin offering}} at the place for burnt offering. \v{30}Then the priest is to take blood with his finger, put it on the horn of the altar that is used for burnt offerings, and then pour the rest of the blood at the base of the altar. \v{31}He is to remove all the fat, just as the fat was removed from the sacrifice for the peace offering. Then the priest is to burn it on the altar as a pleasing aroma to the \divine{Lord}. This is how the priest will make atonement for him. It will be forgiven him.

\v{32}``If he brings a lamb for his offering, he is to bring a female without defect. \v{33}He is to lay his hand on the head of the offering and slaughter it for a sin offering at the place where the burnt offering is slaughtered. \v{34}Then the priest is to take blood with his finger and put it on the horn of the altar for burnt offering. Then he is to pour the rest of the blood at the base of the altar. \v{35}Then the presenter is to remove all its fat, just as the fat was removed from the sacrifice of the peace offering. The priest is to burn it on the altar over the offerings made by fire to the \divine{Lord}. This is how the priest will make atonement for him concerning the sin that he had committed. It will be forgiven him.''
\labelchapt{5}
\passage{Laws of Public Testimony}

\chapt{5}
\v{1}``If someone sins because he has failed to testify after receiving notice\fnote{Lit. \fbib{after having heard}} to testify as a witness regarding what he has observed or learned, he is to be held responsible.''\fnote{Lit. \fbib{guilty}}
\passage{Offerings for Uncleanness}

\v{2}``When a person has touched a ceremonially unclean thing inadvertently,\fnote{Lit. \fbib{thing and it was hidden from him}; and so throughout the chapter} such as the carcass of an unclean animal, or some unclean creeping thing, he will be unclean and guilty nevertheless. \v{3}When he inadvertently touches the uncleanness of a human being, whatever his uncleanness that made him unclean may be, when he himself comes to know about it, he will be guilty. \v{4}When a person has sworn inadvertently by what he has said, whether for evil or good, whatever it was that the person spoke, when he comes to understand what he said, he will incur guilt by one of these things. \v{5}When a person is guilty of one of these things, then he is to confess\fnote{Or \fbib{acknowledge}} whatever sin it was \v{6}and bring compensation to the \divine{Lord} for the guilt that he committed: a female from the flock---whether a lamb or goat---for a sin offering. Then the priest is to make atonement for him.''
\passage{Inexpensive Offering Alternatives}

\v{7}``If he can't afford a goat, then he is to bring to the \divine{Lord} for his sin offering two turtledoves or two young doves:\fnote{Lit. \fbib{or offspring of a dove}} one for a sin offering and the other for a burnt offering. \v{8}He is to bring them to the priest, who will offer a sin offering first. He is to wring off its head without separating it. \v{9}Then he is to sprinkle some of the blood from the sin offering on the sidewall of the altar. Now as to the remainder of the blood, he is to pour it out at the base of the altar for a sin offering. \v{10}With respect to the second offering, he is to prepare it as a burnt offering, according to the approved procedure.\fnote{Lit. \fbib{judgment}} The priest is to make atonement for him on account of his sin that he had committed. Then it will be forgiven him.

\v{11}``If he can't afford\fnote{Lit. \fbib{if his hands cannot reach}} two turtledoves or two young doves, then he is to bring as his offering a tenth of an ephah\fnote{I.e., an ephah was equal to from \footfract{2}{3} to \footfract{3}{4} of a bushel} of fine flour as a sin offering for what he has committed. He is to put no olive oil or frankincense on it, since it's a sin offering. \v{12}He is to bring it to the priest. The priest is to take a handful as a memorial and burn it on the altar as an offering made by fire to the \divine{Lord}. It's a sin offering. \v{13}The priest will make atonement for him, on account of the sin that he had committed in any of these things and it will be forgiven him. As far as the priest is concerned, it will be a meal offering.''
\passage{Offerings for Inadvertent Sins}

\v{14}The \divine{Lord} told Moses, \v{15}``When a person commits a truly treacherous act and sins inadvertently concerning the sacred things of the \divine{Lord}, then he is to bring a trespass offering to the \divine{Lord} from the flock as compensation for his guilt. It is to be a ram without defect, estimated as to its value in silver shekels, according to the sanctuary shekel. \v{16}He is to compensate for whatever sin he had committed concerning the sacred things of the \divine{Lord}, add a fifth part to it, and give it to the priest. The priest is to make atonement for him with the ram as a sin offering and he'll be forgiven.

\v{17}``If a person sins and does what the \divine{Lord} commanded is not to be done, and if he didn't know that he had sinned, then he will be guilty nevertheless.\fnote{Lit. \fbib{he will bear his sin}} \v{18}He is to bring from the flock to the priest a ram without defect, estimated as to its value in silver shekels, as a guilt offering. Then the priest is to make atonement for him concerning his inadvertent act that he committed through ignorance, and it will be forgiven him. \v{19}It's a sin offering for his guilt in the \divine{Lord}'s presence.''
\labelchapt{6}
\passage{Restitution Offerings}

\chapt{6}
\v{1}\fnote{This vs. is 5:20 in MT, and so through vs. 7}The \divine{Lord} told Moses, \v{2}``A person sins against the \divine{Lord} by acting treacherously toward his neighbor regarding something entrusted to his care, regarding security for a loan, robbery, if he has oppressed his neighbor, \v{3}if he has found something that had been lost and then lied about it, or if he makes a false oath about any of these things, thus committing a sin with respect to these things. \v{4}If that person has sinned and has been found guilty, then he is to return the stolen thing that he took or obtained by oppression, or the security that had been entrusted to him, or the lost thing that he had found, \v{5}or the thing about which he had given a false oath. He is to restore it in full, add a fifth to it, then give it to whom it belongs the very day he's found guilty. \v{6}Now as to his guilt offering, he is to bring to the \divine{Lord} a ram without defect from the flock, estimated as to its value, to the priest. \v{7}Then the priest is to make atonement for him in the \divine{Lord}'s presence, and it will be forgiven him regarding whatever he did.''

\v{8}\fnote{This vs. is 6:1 in MT, and so through vs. 30}The \divine{Lord} told Moses, \v{9}``Deliver these orders to Aaron and his sons concerning the regulations for burnt offerings: The burnt offering is to remain on the hearth of the altar throughout the entire night until morning, and the fire on the altar is to be kept burning along with it. \v{10}The priest is to clothe himself with a linen robe and undergarments.\fnote{Lit. \fbib{underclothes over his body}} Then he is to take the ashes of the burnt offering on the altar that had been consumed by the fire and set them beside the altar. \v{11}Then he is to change his clothes, dressing himself with a different set of clothes, and take the ashes to a clean place outside the camp. \v{12}The fire on the altar is to be kept burning continuously without being extinguished. The priest is to burn wood on it every morning, arrange burnt offerings over it, and then burn the fat contained in the peace offerings over it. \v{13}The fire is to continue to burn on the altar and is never to be extinguished.''
\passage{Grain Offerings}

\v{14}``This is the law concerning grain offerings: Aaron's sons are to offer them in the \divine{Lord}'s presence, in front of the altar. \v{15}He is to take a handful of fine flour for a grain offering, some olive oil, and all of the frankincense for the grain offering, and make a sacrifice of smoke on the altar as a memorial portion, a pleasing aroma to the \divine{Lord}. \v{16}Aaron and his sons are to eat what remains of the unleavened offering at this sacred place---the court of the Tent of Meeting. \v{17}It is not to be baked with leaven. I've given it as their portion out of my offerings made by fire. It's a most holy thing, like the sin and guilt offerings. \v{18}Every male of Aaron's sons is to eat it as a portion continually allotted for your generations from the offerings made by fire to the Lord. Anyone who touches them is to be holy.''
\passage{Offerings by the Priests}

\v{19}Then the \divine{Lord} told Moses, \v{20}``This is the offering that Aaron and his sons are to offer to the \divine{Lord} the day he is anointed: a tenth of an ephah\fnote{I.e., an ephah was equal to from \footfract{2}{3} to \footfract{3}{4} of a bushel} of flour is to be offered throughout the day, half in the morning and half in the evening. \v{21}It is to be prepared with olive oil on a griddle. Once it has been mixed thoroughly, bake it, bring it in pieces, and offer it like a grain offering of broken pieces, a pleasing aroma to the \divine{Lord}. \v{22}The anointed priest who succeeds him from among his sons is to offer\fnote{Lit. \fbib{do}} it. As a permanent statute, it is to be offered whole and made to smoke in the \divine{Lord}'s presence. \v{23}Every grain offering from a priest is to be burned\fnote{The Heb. lacks \fbib{burned}} whole. It is not to be eaten.''
\passage{Sin Offerings}

\v{24}Then the \divine{Lord} told Moses, \v{25}``Tell Aaron and his sons that this is the regulation concerning sin offerings: Slaughter the sin offering in the same place where the whole burnt offering is slaughtered---in the \divine{Lord}'s presence. It's a most holy thing. \v{26}The priest who offers it as a sin offering is to eat it at a sacred place in the court of the Tent of Meeting. \v{27}Whoever touches its meat will be holy. If some of its blood sprinkles on a garment, wash where it was sprinkled in a sacred place. \v{28}The earthen vessel in which it was boiled is to be broken, unless it was boiled in a bronze vessel, in which case it is to be polished very well and rinsed in water. \v{29}Every male among the priests is to eat it. It's a most sacred thing. \v{30}Any sin offering from which its blood was brought to the Tent of Meeting to make atonement in the sacred place is not to be eaten. Instead, it is to be incinerated.''
\labelchapt{7}
\passage{Guilt Offerings}

\chapt{7}
\v{1}``This is the regulation concerning guilt offerings. They are most holy. \v{2}The guilt offering is to be offered in the same place where the burnt offering is slaughtered. The priest\fnote{Lit. \fbib{he}} is to sprinkle some of its blood on the altar and around it. \v{3}As to all its fat---that is, the fat on the tail and the fat covering the internal organs---the one presenting the sacrifice\fnote{Lit. \fbib{he}} is to offer it. \v{4}But the two kidneys, the fat over them by the loins, and the appendage on the liver are to be taken away, along with the kidneys. \v{5}Then the priest is to offer them on the altar, incinerating them with fire as a guilt offering to the \divine{Lord}. \v{6}Any male among the priests may eat it, provided that it is eaten at a sacred place as a most holy thing. \v{7}The law for the sin offering is the same as the guilt offering. It belongs to the priest who made atonement with it. \v{8}The hide from the burnt offering brought by the offeror\fnote{Lit. \fbib{by a man}} is to belong to the priest. \v{9}Every grain offering that's baked in the oven and everything that's prepared\fnote{Lit. \fbib{made}} in a stew pan or in the frying pan belongs to the priest who offered it. \v{10}Furthermore, every grain offering that's mixed with olive oil or that's dry will be for Aaron's sons, each one like the other.''\fnote{Lit. \fbib{a man like his brother}}
\passage{Peace Offerings}

\v{11}``This is the law concerning the sacrifice for peace offerings that are to be brought to the \divine{Lord}: \v{12}If someone\fnote{Lit. \fbib{he}} brings it to demonstrate thanksgiving, then he is to present along with the thanksgiving offering unleavened cakes mixed with olive oil, unleavened wafers spread\fnote{Lit. \fbib{anointed}} with olive oil, and cakes of mixed fine flour with olive oil. \v{13}Along with the cakes of unleavened bread, he is to bring his thanksgiving offering with his peace offerings. \v{14}He is to present one from each grain offering,\fnote{The Heb. lacks \fbib{grain}} a separate offering to the \divine{Lord}. It will belong to the priest who sprinkles the blood of the peace offering. \v{15}As to the meat\fnote{Lit. \fbib{flesh}} contained in his peace offerings, it is to be eaten on the day it is offered.\fnote{Lit. \fbib{of its offering}} Nothing of it is to remain until morning.''
\passage{Voluntary Offerings}

\v{16}``If his sacrifice accompanies a fulfilled vow or is a voluntary offering, it is to be eaten on the day the offeror\fnote{Lit. \fbib{day he}} brings the sacrifice. Anything left over is to be eaten the next day,\fnote{Lit. \fbib{in the morrow}} \v{17}but whatever remains uneaten from the meat of the sacrifice by the third day is to be incinerated. \v{18}If any of the meat of his sacrifice of peace offerings is eaten on the third day, it won't be accepted for the one who brought it. It is to be considered as refuse, and whoever eats it will bear the punishment of his iniquity.''
\passage{Distinguishing the Clean and Unclean}

\v{19}``Meat that comes in contact with a ceremonially unclean thing is not to be eaten. Incinerate it instead. As for ceremonially clean\fnote{The Heb. lacks \fbib{ceremonially clean}} meat, anyone who is clean may eat it.\fnote{Lit. \fbib{eat the flesh}} \v{20}But the person who eats meat from the sacrifice that belongs to the \divine{Lord}, while still affected by his uncleanness, is to be eliminated from contact with\fnote{The Heb. lacks \fbib{contact with}} his people. \v{21}Any person who touches a ceremonially unclean thing---whether the uncleanness pertains to human beings, animals, or to creeping things---and then eats from the meat of peace offerings that belongs to the \divine{Lord} is to be eliminated from contact with\fnote{The Heb. lacks \fbib{contact with}} his people.''
\passage{Prohibited Consumption}

\v{22}The \divine{Lord} told Moses, \v{23}``Tell the Israelis, `You are not to eat the fat of an ox, a lamb, or a goat. \v{24}The carcass of an animal that died of its own and an animal torn by wild beast may be used for any purpose except for eating. \v{25}Anyone who eats the fat of an animal that has been offered by fire to the \divine{Lord} is to be eliminated from contact with\fnote{The Heb. lacks \fbib{contact with}} his people. \v{26}You are not to eat any form of blood in any of your dwellings, whether it's from birds or animals. \v{27}Any person who eats any form of blood is to be eliminated from contact with\fnote{The Heb. lacks \fbib{contact with}} his people.'\,''
\passage{The Priests' Portions}

\v{28}The \divine{Lord} told Moses, \v{29}``Tell the Israelis that whoever brings a peace offering sacrifice to the \divine{Lord} is to bring his offering to the \divine{Lord} from the sacrifice of his peace offerings. \v{30}He is to bring the offering made by fire with his own hands to the \divine{Lord}. He is to bring the fat with the breast, since the breast is to be waved as a raised offering to the \divine{Lord}. \v{31}The priest will burn the fat on the altar, but the breast belongs to Aaron and his sons. \v{32}From the sacrifices of your peace offerings give the right thigh to the priest as a raised offering to the \divine{Lord}. \v{33}The descendant of Aaron's sons who brings the blood and the fat from the peace offering is to keep the right thigh for his own portion, \v{34}since I've taken the breast and the thigh as raised offerings from the sacrifices of peace offerings of the Israelis and have given them to Aaron the priest and his sons as their perpetual portion from the Israelis.''

\v{35}This is the consecrated portion for Aaron and his descendants from the offerings made by fire to the \divine{Lord}, the day they were presented to be priests to the \divine{Lord}. \v{36}This is what the \divine{Lord} had commanded to give them the day he anointed them from among the Israelis---a perpetual portion for their generations.
\passage{Summary of Gifts}

\v{37}This is the regulation concerning burnt, grain, sin, guilt, and installation offerings, along with the sacrifice for peace offerings. \v{38}This is what the \divine{Lord} had commanded Moses on Mount Sinai on the day he commanded the Israelis to bring their offerings to the \divine{Lord} in the Sinai wilderness.
\labelchapt{8}
\passage{Ordination of the Priesthood}
\passageinfo{(Exodus 29:1-37)}

\chapt{8}
\v{1}The \divine{Lord} told Moses, \v{2}``Take Aaron, his sons with him, the clothing, the anointing oil, the bull for sin offering, two rams, and a basket of unleavened bread \v{3}and then assemble the entire congregation at the entrance to the Tent of Meeting.''

\v{4}So Moses did just as the \divine{Lord} had commanded him. He assembled the congregation at the entrance to the Tent of Meeting. \v{5}Moses told the congregation, ``This is what the \divine{Lord} commanded to be done.''

\v{6}Moses brought Aaron and his sons and washed them with water. \v{7}Then he clothed Aaron with the tunic, girded him with the band\fnote{Or \fbib{girdle}} for priests, clothed him with the robe, placed the ephod on him, girded him with the skillfully woven band of the ephod, and bound it on him. \v{8}He set the breastplate on him, placed the Urim and Thummim\fnote{I.e. the jewel-encrusted breastplate worn by the high priest by which the will of God could be revealed; cf. Ezra 2:63, Neh 7:65} on top of the breastplate, \v{9}then he set the turban on his head. On the turban at the front he set the golden plate, the sacred crown that the \divine{Lord} had commanded. \v{10}After this, Moses took the anointing oil and anointed the tent, consecrating everything that was in it. \v{11}He sprinkled some on the altar seven times, and then anointed the altar, all its vessels, the basin, and its base to consecrate them. \v{12}After doing this, he poured the oil of anointing on Aaron's head to anoint and consecrate him. \v{13}Then Moses brought Aaron's sons, clothed them with the tunics, girded them with the bands, and bound turbans on them, just as the \divine{Lord} had commanded him.\fnote{Lit. \fbib{Moses}}
\passage{Moses' Sin and Whole Offerings}

\v{14}Next, he brought the bull for a sin offering. Aaron and his sons laid their hands on the bull's head for a sin offering. \v{15}So Moses slaughtered it, took the blood, and applied some of it at the horns of the altar and around it with his fingers, thus purifying the altar. Then he poured the blood at the base of the altar, thereby sanctifying it as a means to make atonement with it. \v{16}Moses burned on the altar all the fat on the internal organs, the appendage on the liver, the two kidneys, and the fat. \v{17}As to the bull and its fat, skin, and offal, he incinerated them outside the camp, just as the \divine{Lord} had commanded him.\fnote{Lit. \fbib{Moses}} \v{18}Next, he brought the ram for the whole burnt offering. Aaron and his sons laid their hands on the head of the ram, \v{19}and Moses slaughtered it and poured its blood over and around the altar. \v{20}As to the ram, he cut it into parts at the joints, burned the head, the internal organs, and the fat, \v{21}washed the internal organs and the thigh with water, and then burned the entire ram on the altar as a whole burnt offering, a pleasing aroma of an offering made by fire to the \divine{Lord}, just as the \divine{Lord} had commanded him.\fnote{Lit. \fbib{Moses}}
\passage{Moses' Consecration Offerings}

\v{22}Moses brought the ram---that is, the second of the rams---for consecration. Aaron and his sons laid their hands on the head of the ram. \v{23}Moses then slaughtered it, took some of its blood, and put it on Aaron's right earlobe, right thumb, and right great toe. \v{24}Then Moses brought Aaron's sons, took some of the ram's blood, put it on their right earlobes, on their right thumbs, and on their right great toes, and then poured the blood on the altar and all around it. \v{25}Then he took the fat from the tail, all the fat on the internal organs, the appendage of the liver, the two kidneys with the fat, and the right thigh. \v{26}From the basket of unleavened bread that is in the \divine{Lord}'s presence he took one piece of unleavened bread, one cake spread with olive oil, and one wafer, which he placed over the fat and the right thigh. \v{27}He put all of these things in the hands of Aaron and his sons, and they all waved them in a raised offering to the \divine{Lord}. \v{28}After this, Moses took those things from their hands and burned them on the altar over the whole burnt offering for consecration. They served as a pleasing aroma, an offering made by fire to the \divine{Lord}. \v{29}Moses took the breast and waved it as a raised offering in the \divine{Lord}'s presence as the portion that belonged to Moses from the ram of consecration, just as the \divine{Lord} had commanded him.\fnote{Lit. \fbib{Moses}}
\passage{Moses' Oil of Anointing}

\v{30}Moses took some anointing oil and blood that was on the altar and sprinkled it on Aaron, on his clothes, on his sons, and on their clothes, consecrating Aaron, his clothes, his sons, and their clothes. \v{31}Then he told Aaron and his sons, ``Boil the meat at the entrance to the Tent of Meeting. You may eat it there, along with the bread that is in the basket for consecration, just as I've commanded when I told him, `Aaron and his sons may eat of it, \v{32}but the leftover meat and bread is to be incinerated.' \v{33}Furthermore, you are not to go out past the entrance to the Tent of Meeting until the days of your ordination have been completed, since it will take seven days to ordain you. \v{34}What has been done today\fnote{Lit. \fbib{as has been done today}} has been commanded by the \divine{Lord} to make atonement for you. \v{35}Stay seven days and nights at the entrance to the Tent of Meeting and attend to the service of the \divine{Lord}, so that you won't die, because this is what I've commanded.''

\v{36}So Aaron and his sons did everything that the \divine{Lord} had commanded through\fnote{Lit. \fbib{commanded through the hand of}} Moses.
\labelchapt{9}
\passage{Aaron's Ministry Commences}

\chapt{9}
\v{1}Eight days later, Moses called Aaron, his sons, and the elders of Israel. \v{2}He told Aaron, ``Take a young calf for a sin offering and a ram without defect for a whole burnt offering and bring them into the \divine{Lord}'s presence.''

\v{3}He also told the Israelis, ``Bring a male goat for a sin offering, a calf, a year old lamb without defect for a whole burnt offering, \v{4}an ox, a ram for a peace offering to sacrifice in the \divine{Lord}'s presence, and a grain offering with olive oil, because on that day the \divine{Lord} will appear to you.'' \v{5}So they brought what Moses had commanded to the entrance to the Tent of Meeting. The entire congregation drew near and stood in the \divine{Lord}'s presence.

\v{6}Then Moses said, ``This is what the \divine{Lord} commanded you to do so that the glory of the \divine{Lord} may be revealed to you.''

\v{7}Moses then told Aaron, ``Approach the altar and bring your sin and whole burnt offerings. Make atonement for yourself and the people. Then bring the people's offering and make atonement for them, as the \divine{Lord} commanded.''

\v{8}So Aaron drew near to the altar and slaughtered the calf for a sin offering on behalf of himself. \v{9}Next, Aaron's sons brought the blood to him and he dipped his fingers in the blood and placed it on the horns of the altar. As to the rest of the\fnote{The Heb. lacks \fbib{rest of the}} blood, he poured it at the base of the altar. \v{10}He incinerated the fat, the kidneys, and the appendage from the liver of the sin offering, just as the \divine{Lord} had commanded Moses. \v{11}He also incinerated the meat and skin outside the camp. \v{12}And so the burnt offering was slaughtered, and Aaron's sons secured for him the blood, which he poured on the altar and around it.
\passage{Aaron's Burnt Offering}

\v{13}As for the burnt offering, they delivered it to Aaron\fnote{Lit. \fbib{him}} piece by piece, and he burned the head on the altar, \v{14}washed the internal organs and thighs, and incinerated them on the altar, along with the whole burnt offering. \v{15}He brought the people's offering, presenting a goat for a sin offering on behalf of the people. He slaughtered it and offered it as the first sin offering. \v{16}Then he brought the whole burnt offering and offered it according to procedure.

\v{17}Next, he brought the grain offering, filled his hand with it, and burned it on the altar next to the burnt offering for that morning. \v{18}He slaughtered the ox and ram for the peace offering sacrifice on behalf of the people. Aaron's sons delivered the blood to him, which he poured on the altar and around it. \v{19}As to the fat from the ox and the ram---the tail, the fat covering the kidneys, and the appendage of the liver--- \v{20}they placed the fat on the breast and burned the fat on the altar. \v{21}Aaron waved the breast and the right thigh as a raised offering in the \divine{Lord}'s presence, just as Moses had commanded. \v{22}Aaron raised his hand toward the people and blessed them. Then he came down from the altar after\fnote{The Heb. lacks \fbib{the altar after}} offering the sin, whole burnt, and peace offerings.

\v{23}Moses and Aaron entered the Tent of Meeting. When they came out, they blessed the people and the glory of the \divine{Lord} appeared to all the people. \v{24}A fire came down from the \divine{Lord}'s presence and consumed the burnt offering on the altar as well as the fat. When the people saw it, they shouted and fell on their faces.
\labelchapt{10}
\passage{Nadab and Abihu}
\passageinfo{(Numbers 3:1-10)}

\chapt{10}
\v{1}Aaron's sons Nadab and Abihu each took his own censer, placed fire in it, covered it with incense, and brought it into the \divine{Lord}'s presence as unauthorized fire that he had never prescribed for them. \v{2}As a result, fire came out from the \divine{Lord}'s presence and incinerated them. They died while in the \divine{Lord}'s presence. \v{3}Moses spoke with Aaron about what the \divine{Lord} had said: ``Among those who are near me, I'll show myself holy so that I'll be glorified before all people.'' So Aaron remained silent.
\passage{After the Deaths of Nadab and Abihu}

\v{4}Then Moses called on Mishael and Elzaphan, the sons of Uzziel, Aaron's uncle, and said, ``Come here and carry your brothers away from the sanctuary, outside the camp.'' \v{5}So they approached to carry them in their tunics outside the camp, just as Moses had commanded.

\v{6}Then Moses told Aaron and his sons Eleazar and Ithamar, ``You are not to loosen the hair of your head and you are not to rend your clothes. That way, you won't die and wrath won't come on the entire congregation. Your brothers and the assembly\fnote{Lit. \fbib{house}} of Israel will mourn because of the fire that the \divine{Lord} kindled. \v{7}Also, you are not to leave the entrance to the Tent of Meeting. Otherwise, you'll die, since the \divine{Lord}'s anointing oil remains on you.'' So they followed Moses' instructions.
\passage{Prohibitions against Drinking Wine}

\v{8}Then the \divine{Lord} told Aaron, \v{9}``You and your sons with you are not to drink wine---that is, any intoxicating drink---when you enter the Tent of Meeting. That way, you won't die. This is to be a perpetual statute throughout your generations. \v{10}You are to differentiate between what's sacred and common and between what's unclean and clean. \v{11}You are to teach the Israelis all the statutes that the \divine{Lord} commanded you by the authority of Moses.''
\passage{Additional Orders for Offerings}

\v{12}Then Moses told Aaron and his sons Eleazar and Ithamar, ``Take the leftovers from the grain offering and the offerings made by fire and eat the unleavened bread beside the altar, because it is most holy to the \divine{Lord}. \v{13}Eat at a sacred place, because it's your and your sons' prescribed portions. It's from the offering made by fire to the \divine{Lord}, since I've commanded it. \v{14}As to the breast and thigh raised offerings, you and your sons and daughters with you may eat them\fnote{The Heb. lacks \fbib{them}} at a clean place, because they belong to you and are your sons' prescribed portions and were taken from the sacrifices of peace offering presented by the Israelis. \v{15}They are to bring the thigh offering, the breast raised offering, and the offerings made by fire from the fat to wave as a raised offering in the \divine{Lord}'s presence. It will be a perpetual portion for you and your sons with you, just as the \divine{Lord} commanded.''
\passage{Confusion Occurs, but is Resolved}

\v{16}Now Moses diligently sought for the goat that had been offered as a sin offering, but it had already been incinerated, so he was angry with Aaron's sons who remained. He asked Eleazar and Ithamar, \v{17}``Why didn't you eat the sin offering at the sacred place? It's most holy and he has given it to you so that you may bear the punishment for the iniquity of the entire congregation and make atonement for them in the \divine{Lord}'s presence. \v{18}Look! Its blood wasn't brought inside the sanctuary. You were to have eaten it in the sanctuary, just as I commanded.''

\v{19}But Aaron replied to Moses, ``Today they've offered their sin and whole burnt offerings in the \divine{Lord}'s presence. Yet things such as these have happened to me. Had I eaten the sin offering today, would that have pleased the \divine{Lord}?''\fnote{Lit. \fbib{have been pleasing in the \divine{Lord}'s sight?}}

\v{20}When Moses heard that explanation, he was pleased.
\labelchapt{11}
\passage{Clean and Unclean Animals}
\passageinfo{(Deuteronomy 14:3-21)}

\chapt{11}
\v{1}The \divine{Lord} told Moses and Aaron,\fnote{Lit. \fbib{to them}} \v{2}``Tell the Israelis that these are the living creatures that you may eat among the animals of the earth: \v{3}You may eat any animal that has divided hooves with cloven feet and that ruminates its cud, \v{4}except you are not to eat the following animals that have divided hooves or ruminate their cud: the camel (because it chews the cud but doesn't have divided hooves, it is to be unclean for you), \v{5}the rock badger (because it chews its cud but its hooves aren't divided, it is to be unclean for you), \v{6}the hare (because it chews its cud, but its hooves aren't divided, it is to be unclean for you), \v{7}and the pig (because it has divided hooves and is therefore cloven-footed, but it doesn't ruminate its cud, it is to be unclean for you). \v{8}You are not to eat their flesh or even touch their carcasses. They are to be unclean for you.''
\passage{Clean and Unclean Seafood}

\v{9}``You may eat anything that lives in the water---that is, you may eat anything that has fins and scales either from the seas or from the rivers. \v{10}But anything that doesn't have fins or scales---whether from the seas or the rivers---any of the swarming creatures and living creatures in the waters are detestable for you. \v{11}They are to remain detestable for you. You are not to eat of their meat and you are to detest their carcasses. \v{12}Anything that doesn't have fins or scales in the waters is a detestable thing for you.''
\passage{Clean and Unclean Winged Creatures}

\v{13}``These are detestable things for you among winged creatures that you are not to eat, because they are detestable for you: the eagle, vulture, osprey, \v{14}red kite, falcons of any kind, \v{15}every kind of raven, \v{16}ostrich, nighthawk, seagull, hawks of every kind, \v{17}owls, cormorants, the ibis, \v{18}water-hens, pelicans, carrion, \v{19}storks, herons of every kind, the hoopoe, bata, \v{20}and any winged insect that crawls on four legs is detestable for you. \v{21}However, you may eat winged creatures that crawl on four legs that extend over its head and by which it hops on the ground. \v{22}These creatures that you may eat include the locust of any kind, the bald locust of any kind, the cricket of any kind, and the grasshopper of any kind. \v{23}But any other winged insect that has four legs is detestable for you \v{24}and is unclean. Anyone who touches their carcasses becomes unclean until evening. \v{25}And anyone who carries their carcasses is to wash his clothes, since he will remain unclean until evening.''
\passage{Summary of Clean and Unclean}

\v{26}``Any animal that has divided hooves and is cloven-footed but doesn't chew the cud is unclean for you. Anyone who touches them is unclean. \v{27}Among the animals, anything that walks on their paws and on four legs is unclean for you. Anyone who touches their carcasses becomes unclean until evening. \v{28}Whoever carries their carcass is to wash their clothes, because they've become unclean until evening. They're unclean for you.

\v{29}``These are unclean for you among the swarming creatures that crawl over the land: the rat,\fnote{Or \fbib{weasel}} mouse, lizards of every kind, \v{30}the gecko, crocodile, lizard, sand lizard, and chameleon. \v{31}These are unclean for you among the swarming creatures, so anyone who touches them when they're dead becomes unclean until evening. \v{32}Furthermore, anything on which they fall when they're dead becomes unclean, whether on an article of wood, clothing, skin, or a sack. And any vessel used for any work is to be washed in water, because it has become unclean until evening. \v{33}Any earthen vessel into which any of these things fall becomes unclean, along with everything in it. You are to destroy it, along with all its contents.''
\passage{Clean and Unclean Vessels}

\v{34}``Any food that may be eaten, but into which water has soaked, becomes unclean. Any drink that may be drunk in any of these vessels becomes unclean, \v{35}and anything into which their carcass falls becomes unclean. An oven or stove is to be broken in pieces. They're unclean and therefore unclean for you.

\v{36}``A spring or a cistern that holds water is clean, but whoever touches the carcass of an unclean animal will be unclean. \v{37}If their carcass falls on a seed, which is for sowing, what is to be sown is clean. \v{38}But if water is put on the seed and part of their carcass falls on it, then it has become unclean for you.

\v{39}``If any of the animals that you may eat dies, the one who touches its carcass becomes unclean until evening. \v{40}The one who eats from its carcass is to wash his clothes, because he has become unclean until evening. Even the one who carries the carcass is to wash his clothes, because he has become unclean until evening.''
\passage{Unclean Swarming Animals}

\v{41}``Every swarming thing that swarms the land is detestable for you. It is not to be eaten. \v{42}You are not to eat anything that crawls on its belly, anything that walks on four legs, anything that has many legs, or any of the swarming creatures that swarm the land, because they're detestable. \v{43}You are not to make yourselves detestable on account of any swarming creature that swarms the land, and you are not to defile yourselves and become unclean due to them, \v{44}because I, the \divine{Lord}, am your God. Set yourselves apart and be holy, because I am holy. You are not to defile yourselves with any of the swarming creatures that swarm the earth. \v{45}I am the \divine{Lord}, who brought you out of the land of Egypt to be your God. You are to be holy, because I am holy. \v{46}This is the law concerning animals, every living creature that moves on the waters or swarms\fnote{Lit. \fbib{every living creature}} on land. \v{47}You are to differentiate between the clean and unclean, between the living creature that can be eaten and the living creature that is not to be eaten.''
\labelchapt{12}
\passage{Post-Natal Purification}

\chapt{12}
\v{1}The \divine{Lord} told Moses, \v{2}``Tell the Israelis that a woman who conceives and bears a son is unclean for seven days. Just like the days of her menstruation,\fnote{Lit. \fbib{days of her impurity, she is ill}} she is unclean. \v{3}On the eighth day, the flesh of the baby's foreskin is to be circumcised. \v{4}For 33 days after this, she is to remain in purification due to her blood loss.\fnote{The Heb. lacks \fbib{loss}} She is not to touch any sacred thing or enter the sanctuary until the days of her purification have been completed.

\v{5}``If she gives birth to a female, then she is to remain unclean for two weeks, just like her menstruation. She is to remain in purification for 66 days due to her blood loss.\fnote{The Heb. lacks \fbib{loss}} \v{6}When the days of her purification have been completed, whether for her son or daughter, she is to bring to the priest at the entrance to the Tent of Meeting a one year old lamb for a whole burnt offering or a young dove for a sin offering. \v{7}He is to offer it in the \divine{Lord}'s presence and make atonement for her so that she becomes clean from her blood loss. This is the law concerning the bearing of a male or female child. \v{8}If she cannot afford a goat, then two turtledoves or two young doves---one for a burnt offering and the other for a sin offering---will serve for him to make atonement for her, so that she becomes clean.''
\labelchapt{13}
\passage{Diagnosing Skin Diseases}

\chapt{13}
\v{1}The \divine{Lord} said this to Moses and Aaron: \v{2}``When a person\fnote{Lit. \fbib{man}} has a swelling or a scab in the skin on his body\fnote{Lit. \fbib{flesh}, and so throughout the chapter} that turns white in appearance and appears to be more extensive than skin deep, he is to be brought to Aaron the priest or to one of his sons among the priests. \v{3}The priest is to examine the skin rash on the body. If the hair on the skin rash has turned white and its appearance is deeper than the skin of his body, it's an infectious skin disease. When the priest has examined it, then he is to declare him unclean.

\v{4}``If the light spot in the skin of his body is white but the appearance of the skin rash isn't deeper than the skin of his body and its hair has not become white, then the priest is to isolate\fnote{I.e. in medical confinement} the one who is infected for seven days. \v{5}On the seventh day, the priest is to examine him again. If, in his opinion, the skin rash remained the same and it\fnote{Lit. \fbib{and the skin rash in his skin}} did not spread, then he is to isolate\fnote{I.e. in medical confinement} him for another seven days.

\v{6}``On the next\fnote{Lit. \fbib{the second}} seventh day, the priest is to examine him again. If the skin rash didn't become dull and it\fnote{Lit. \fbib{and the skin rash}} didn't spread in the skin, then the priest is to pronounce him clean: it's a scab. He is to wash his clothes and be clean. \v{7}But if the scab did spread in the skin after he presented himself to the priest for cleansing, then he is to show himself a second time to the priest. \v{8}When the priest examines him and determines that the scab did, in fact, spread in his skin, then the priest is to pronounce him unclean, since it's an infectious skin disease.''
\passage{Infectious Skin Diseases}

\v{9}``When a person has a skin rash that's infectious, he is to be brought to the priest. \v{10}The priest is to examine it. If it is, indeed, a white swelling in the skin that has turned the hair white, and yet it sustains live flesh on the swelling, \v{11}it's a festering skin disease in his body. The priest is to declare him unclean. The man need not be confined, since he's already unclean. \v{12}If the infectious skin disease spreads in the skin so that it covers his entire body from head to foot (as the priest examines it), \v{13}when the priest's examination reveals that the infectious skin disease has covered his entire body, then he is to declare him clean, even though he still has the skin infection. He has turned entirely white, so he's clean. \v{14}But if, one day, infected flesh appears again in him, he is unclean. \v{15}The priest is to examine the infected flesh and declare him unclean. The raw flesh is unclean; it's an infectious skin disease. \v{16}If the raw flesh recurs and turns white, then he is to go to the priest. \v{17}When the priest examines him and finds that the skin rash has indeed turned white, then the priest is to declare the one with the skin rash clean, and he will be clean.''
\passage{On Boils}

\v{18}``When someone is infected with a boil, but after it's healed, \v{19}in place of the boil there remains a white swelling or a bright, white-reddish spot, he is to present himself to the priest. \v{20}When the priest undertakes his examination and finds that it appears more extensive than skin deep and that its hair has turned white, then the priest is to declare him unclean, since an infectious skin disease has flourished in the boil. \v{21}If the priest undertakes an examination, but there's no white hair in it and it's not more extensive than skin deep, but it's dull, then the priest is to isolate\fnote{I.e. in medical confinement} him for seven days. \v{22}But if the infection has spread in the skin, then the priest is to declare him unclean. It's a skin rash. \v{23}If the scab remains in place and doesn't spread, then it's the scab from the boil. The priest is to declare him clean.''
\passage{Burn Scars}

\v{24}``When a person has a burn scar in the skin that turns bright, white-reddish, or white, \v{25}if the priest examines it and indeed the hair has turned white with a white spot appearing more extensive than skin deep, it's an infectious skin disease with a burn scar that has spread. The priest is to declare him unclean. It's an infectious skin disease. \v{26}But if the priest examines it and discovers that there's no bright area or white hair, or if he discovers that\fnote{The Heb. lacks \fbib{if he discovers that}} it's not more extensive than skin deep and it's dull, then the priest is to isolate\fnote{I.e. in medical confinement} him for seven days. \v{27}When the priest examines it on the seventh day and finds that it has indeed spread on the skin, then the priest is to declare him unclean. It's an infectious skin disease. \v{28}But if the bright spot remains in place, doesn't spread in the skin, and it's dull, it's the swelling of the burned area. The priest is to declare him clean, since it's the scar from a burn.''
\passage{Rashes}

\v{29}``Now when a man or a woman has a skin rash on the head or the man develops a skin rash under his beard,\fnote{The Heb. lacks \fbib{the man develops a skin rash under his}} \v{30}if when the priest examines the skin rash and indeed it appears more extensive than skin deep, and it's accompanied by fine, yellowish hair, then the priest is to declare him unclean. The scales on the head or the beard are an infectious skin disease. \v{31}But when the priest examines the scales of the skin rash and it doesn't appear more extensive than skin deep and there's no black hair in it, then the priest is to isolate\fnote{I.e. in medical confinement} him for seven days. \v{32}When the priest examines the skin rash on the seventh day and finds that indeed the scab did not spread, there's no yellowish hair on it, and the scales don't appear more extensive than skin deep, \v{33}then he is to be shaven, but the scab is not to be shaved off. The priest is to isolate\fnote{I.e. in medical confinement} him a second time for seven days. \v{34}The priest is to examine the scab on the seventh day. If, indeed, the scab hasn't spread on the skin and it doesn't appear more extensive than skin deep, then the priest is to declare him clean. He is to wash his garments and be clean.

\v{35}``But if the scales spread on the skin after his cleansing, \v{36}and the priest examines it and finds the scale to have spread on the skin, the priest need not look for yellowish hair, since he is clean. \v{37}If, in his opinion, the scab remained the same and a black hair grew in it, then the scab has healed. He's clean. The priest is to declare him clean. \v{38}If a man or a woman has a light or whitish spot in the skin of their body, \v{39}when the priest examines it and finds that there is a light or dull white patch of skin on the body, it's a harmless skin eruption that has spread on the skin. The person is clean.''
\passage{Baldness vs. Head Rashes}

\v{40}``When a man's head becomes bare, he's bald, but he's clean. \v{41}When his head becomes bare on the side corner of his face, he has a bald forehead, but he's clean. \v{42}But when in the baldness of his head or his forehead there develops a skin rash that's white or reddish, it's an infectious skin disease that has spread to his bald head or forehead. \v{43}When the priest examines it and finds that the swelling of the skin rash is white or reddish on his bald head or forehead, similar in appearance to an infectious disease in the skin of the body, \v{44}he's a man with an infectious skin disease. He's unclean. The priest is to declare him unclean on account of the skin rash in his head. \v{45}The person with the infectious skin disease is to tear his garments and loosen his hair.\fnote{Lit. \fbib{head}} He is to cover his mustache and shout out, `Unclean! Unclean!' \v{46}The whole time that the skin rash infects him, he will be unclean. He is to live by himself in a home outside the encampment.''
\passage{Infected Clothing}

\v{47}``When clothing becomes infected with a contagion---whether the clothing is wool or linen--- \v{48}in woven or knitted material, in leather, or with any article containing leather, \v{49}if the contagion is greenish or reddish in the clothing, leather, woven material, knitted material, or with any article containing leather, it's a fungal infection and is to be shown to the priest.

\v{50}``The priest is to examine the contagion and isolate\fnote{I.e. in medical confinement} the clothing\fnote{Lit. \fbib{isolate it}} for seven days. \v{51}The priest is to examine the contagion on the seventh day. If the infection has spread on the clothing, in the woven material, the knitted material, or in the leather, no matter the purpose for which the leather material had been manufactured, the contagion is a chronic fungal infection. It's unclean.

\v{52}``Incinerate the clothing, the woven material, the knitted material (whether wool or linen), or any of the leather articles on which the contagion is found, because it's a chronic fungal infection. It is to be incinerated.

\v{53}``But if the priest examines it and the infection did not spread on the clothing, either in the woven or knitted material or on anything made of leather, \v{54}then the priest is to command that they wash whatever has the contagion and then isolate\fnote{I.e. in medical confinement} it for seven days a second time. \v{55}Then the priest is to examine it after the contagion has been washed. If the contagion hasn't changed in appearance,\fnote{Lit. \fbib{eye}} even though the contagion hasn't spread, it's unclean. Incinerate it. It's a fungal infection, especially if the infection is on its exposed side.

\v{56}``If the priest examines the item and determines that the contagion has become dull after it has been washed, tear it away from the garment, leather, woven material, or knitted material. \v{57}But if it recurs on the clothing (whether woven or knitted material) or on any article made of leather, it's a breakout, so incinerate it with fire wherever the contagion is found. \v{58}Then the clothing (whether it is woven or knitted material) or any article made of leather that you've washed, if the contagion has been removed from it and it's washed a second time, then it's clean.

\v{59}``This is the law concerning fungal contagions on clothing of wool or linen (whether woven or knitted material) or in any of the articles made of leather, for determining whether it is clean or unclean.''
\labelchapt{14}
\passage{Purification Requirements}

\chapt{14}
\v{1}The \divine{Lord} told Moses, \v{2}``This is the law concerning those who have infectious skin diseases, after they have been cleansed: \v{3}The priest is to go outside the camp and examine the infectious skin disease to confirm that the person has been healed. \v{4}If he has been healed, then the priest is to command that two live and clean birds, some cedar\fnote{I.e. a genus of coniferous evergreen in the family \fbib{Pinaceae}; and so throughout the book} wood, some crimson thread, and hyssop be brought for the one cleansed. \v{5}Then the priest is to command that one bird be slaughtered on an earthen vessel over flowing water. \v{6}He is to take the live bird, the cedar wood, the crimson thread, and the hyssop, and dip them together in the blood of the bird that had been slaughtered over the flowing water. \v{7}He is to sprinkle the blood\fnote{Lit. \fbib{it}} seven times on the person with the infectious skin disease and then pronounce him clean. Then he is to release the live bird into the open fields. \v{8}The person who is clean is to wash his clothes, shave all his hair, and bathe in water, after which he is to be declared clean. Then he can be brought back to the camp, but he is to remain outside his tent for seven days. \v{9}On the seventh day, he is to shave the hair on his head, chin, back, and eyebrows. After he has shaved all his hair, washed his clothes, and bathed himself with water, then he will be clean.''
\passage{Reconsecration after Infections}

\v{10}``On the eighth day, he is to take two lambs without defect, a one year old ewe lamb without defect, one third of a measure of\fnote{The unit of measurement is not specified in MT} fine flour mixed with olive oil for a meal offering, and one log\fnote{Lit. \fbib{log}; i.e., a liquid measure equal to one twelfth of a hin or about \footfract{2}{3} pint; a \fbib{hin} held about one gallon} of oil. \v{11}The priest who will pronounce him clean is to present the person to be cleansed and these offerings\fnote{The Heb. lacks \fbib{offerings}} in the \divine{Lord}'s presence at the entrance to the Tent of Meeting. \v{12}The priest is to take one of the lambs and present it as a guilt offering, along with one log\fnote{Lit. \fbib{log}; i.e., a liquid measure equal to one twelfth of a hin or about \footfract{2}{3} pint; a \fbib{hin} held about one gallon} of olive oil, which he is to wave as a raised offering in the \divine{Lord}'s presence. \v{13}Then he is to slaughter the lamb in the place where he slaughtered the sin and burnt offerings---that is, at a place in the sanctuary. Just as the sin offering is for the priest, so also is the guilt offering. It's a most holy thing.

\v{14}``Then the priest is to take some of the blood from the guilt offering and place it on the right earlobe of the person to be cleansed, on his right thumb, and on his right great toe. \v{15}Then the priest is to take some of the log\fnote{Lit. \fbib{log}; i.e., a liquid measure equal to one twelfth of a hin or about \footfract{2}{3} pint; a \fbib{hin} held about one gallon} of olive oil and pour it into his own left hand. \v{16}The priest is to dip his right finger in the olive oil that is in his left palm and sprinkle some of the olive oil with his finger seven times in the \divine{Lord}'s presence.

\v{17}``As to the remainder of the olive oil in his palm, he is to place some on the right earlobe of the person to be cleansed, on his right thumb, on his right great toe, and on the blood of the guilt offering. \v{18}Then he is to place the rest of the oil in his palm on the head of the person to be cleansed, thus making atonement for him in the \divine{Lord}'s presence. \v{19}This is how\fnote{Lit. \fbib{If he}} the priest is to present the sin offering to make atonement for the person being cleansed of his impurity. After this, he is to slaughter the whole burnt offering. \v{20}The priest is to offer both the whole burnt and the grain offerings on the altar. After the priest makes atonement for him, he will be clean.''
\passage{Alternate Offerings}

\v{21}``If the offeror\fnote{The Heb. lacks \fbib{person}} is poor and cannot afford the regular offering,\fnote{Lit. \fbib{and his hand can't reach}; and so throughout the chapter} then he is to take one lamb for a guilt offering that will be presented in the form of a wave offering to atone for him, one tenth of a measure of\fnote{The unit of measurement is not specified in MT, but cf. Lev. 5:11, 6:20.} fine flour mixed with olive oil for a grain offering, one log\fnote{Lit. \fbib{log}; i.e., a liquid measure equal to one twelfth of a hin or about \footfract{2}{3} pint; a \fbib{hin} held about one gallon} of olive oil, \v{22}and two turtledoves or two young pigeons, whichever he can afford. One is for a sin offering and the other is for a whole burnt offering.

\v{23}``On the eighth day, he is to bring them for cleansing to the priest in the \divine{Lord}'s presence at the entrance to the Tent of Meeting. \v{24}The priest is to take the lamb for a guilt offering and the olive oil and wave them as a raised offering in the \divine{Lord}'s presence. \v{25}Then he\fnote{Lit. \fbib{the priest}} is to take the lamb for the guilt offering and place some blood from the guilt offering on the right earlobe of the person to be cleansed, on his right thumb, and on his right great toe. \v{26}Then the priest is to pour olive oil into his left palm \v{27}and use his right finger to sprinkle oil from his left palm seven times in the \divine{Lord}'s presence. \v{28}The priest is to place oil from his palm on the right earlobe of the person being cleansed, on his right thumb, on his right great toe, and where the blood for the guilt offering is poured.

\v{29}``As to the remainder of the oil in his palm, the priest is to use it to anoint the head of the person to be cleansed, in order to make atonement for him in the \divine{Lord}'s presence. \v{30}Then he is to offer one of the turtledoves or the young pigeons, whichever he can afford. \v{31}Based on what he can afford, one is for a sin offering and the other is for a whole burnt offering. Along with the grain offering, the priest is to make atonement for the person to be cleansed in the \divine{Lord}'s presence. \v{32}This is the regulation concerning one who has an infectious skin disease but who cannot afford his cleansing.''
\passage{Infected Dwellings}

\v{33}The \divine{Lord} spoke to Moses and Aaron: \v{34}``When you enter the land of Canaan that I'm about to give you as your own possession, and if I put a contagion in a house in the land that you possess, \v{35}then the owner of the house is to approach the priest and tell him, `There appears to be a contagion in the house.'

\v{36}``The priest is to command that the house be cleared before he\fnote{Lit. \fbib{priest}} comes to examine the contagion so that not everything in the house becomes unclean. After this,\fnote{The Heb. lacks \fbib{after this}} the priest is to enter the house and examine it. \v{37}He is to determine if the contagion is indeed on the walls of the house, with greenish or reddish streaks, and to determine if it appears to be deeper than the surface of the wall. \v{38}The priest is to leave through the entrance to the house and seal the house for seven days. \v{39}He is to return after seven days to examine it. If the contagion has spread to the walls of the house, \v{40}then the priest is to command that they take out the contaminated stones and discard them in an unclean place outside the city.

\v{41}``Now as for the house, they are to scrape off inside and outside the house and then discard the torn out plaster in an unclean place outside the city. \v{42}They are then to take other stones and bring them to replace those stones. Lastly, they are to replaster the house.''
\passage{Destruction of Infected Dwellings}

\v{43}``If the contagion returns and spreads throughout the house after the stones have been removed, after the house has been scraped out, and after it has been re-coated, \v{44}and the priest comes, undertakes an examination, and determines that the contagion has spread in the house, it's a chronic fungal infection in the house. It's unclean. \v{45}He is to pull down the house, its stones, its lumber, and all the plaster on the house, and discard them in an unclean place outside the city. \v{46}Moreover, whoever enters the house during the time it was isolated is to be considered unclean until the evening. \v{47}Whoever has slept in the house is to wash his clothes, along with whoever has eaten in the house.

\v{48}``But if the priest comes in to conduct an examination and determines that the contagion has not spread throughout the house after the house has been repaired, then the priest may declare the house clean, because the contagion has been cleansed. \v{49}In order to cleanse the house, he is to take two birds, some cedar wood, two crimson threads, and some hyssop. \v{50}Then he is to slaughter one bird on an earthen vessel over flowing water. \v{51}He is to take the cedar wood, the hyssop, the two crimson threads, and the live bird, and dip them in the blood of the slaughtered bird over flowing water. Then he is to sprinkle the house seven times. \v{52}He is to clean the house with the blood of the bird over flowing water, including cleansing\fnote{The Heb. lacks \fbib{including cleansing}} the live bird, the cedar wood, the hyssop, and the crimson thread. \v{53}Then he is to send the bird away, outside the city, facing the fields, to make atonement for the house. Then it is to be considered clean.

\v{54}``This is the law for every contagion of infectious skin disease and scabs, \v{55}for fungal infections on clothing or in a house, \v{56}and for swelling of the skin, scabs, and bright spots, \v{57}to distinguish when\fnote{Lit. \fbib{in the day}} it's unclean and clean. This is the law for infectious skin diseases.''
\labelchapt{15}
\passage{Regulations Concerning Discharges}

\chapt{15}
\v{1}The \divine{Lord} told Moses and Aaron, \v{2}``Tell the Israelis that when a man has a discharge from his body, his discharge is unclean, \v{3}and this is the cause of his uncleanness---his discharge. Whether his body is releasing the discharge or his body has stopped the discharge, he's unclean. \v{4}Every bed on which he lies down with the discharge is to be considered unclean, and every object on which he sits becomes unclean. \v{5}Any person\fnote{Lit. \fbib{ma.}} who touches his bed is to wash his garments and bathe with water, and he will remain unclean until evening. \v{6}Whoever sits on any object on which the one with the discharge has sat is to wash his clothes and bathe with water, and he will remain unclean until evening.

\v{7}``Whoever touches the body of someone with a discharge is to wash his clothes and bathe with water, and he will remain unclean until evening. \v{8}Whoever has a discharge and spits on someone who is clean, then he is to wash his clothes and bathe with water, and he will remain unclean until evening.

\v{9}``Any saddle that anyone with a discharge rides on will become unclean. \v{10}Whoever touches anything that was under him will be unclean until evening. Whoever carries these things is to wash his clothes and bathe with water, and he will remain unclean until evening.

\v{11}``Anyone whom the one with the discharge touches without rinsing his hands with water is to wash his clothes and bathe with water, and he will remain unclean until evening. \v{12}The earthen vessel that the person with the discharge touches is to be broken in pieces, and every wooden vessel is to be rinsed with water.''
\passage{On Cleansing from Discharges}

\v{13}``When the one with the discharge is cleansed from his discharge, then he is to set aside for himself seven days for his cleansing. He is to wash his clothes and bathe with flowing water. Then he will be clean. \v{14}On the eighth day, he is to take for himself two turtledoves or two young doves, bring them to the \divine{Lord} at the entrance to the Tent of Meeting, and give them to the priest. \v{15}Then the priest is to offer them---one for a sin offering and the other for a whole burnt offering. That's how the priest will make atonement for him in the \divine{Lord}'s presence regarding his discharge.''
\passage{On Seminal Emissions}

\v{16}``If a man has a seminal emission, he is to bathe his entire body with water and remain unclean until evening. \v{17}Every garment (including leather) on which the semen is found is to be washed with water, and it will remain unclean until evening. \v{18}When a man has sexual relations with a woman and the man releases semen, both are to bathe with water, and they will remain unclean until evening.''
\passage{On Menstrual Discharges}

\v{19}``When a woman has a discharge,\fnote{Or \fbib{flow}} and the blood is her monthly menstrual discharge\fnote{The Heb. lacks \fbib{monthly menstrual}} from her body, then for seven days she is to remain in her menstrual uncleanness. Whoever touches her will remain unclean until evening. \v{20}Everything that she sleeps on during her uncleanness will be unclean. Moreover, everything that she sits on will become unclean. \v{21}Anyone who touches her bed is to wash his clothes and bathe with water, and he will remain unclean until evening. \v{22}Anyone who touches any of the objects on which she has sat is to wash his clothes and bathe with water, and he will remain unclean until evening. \v{23}Any bed or other object on which she sat that he touches will make him unclean until evening. \v{24}When a man has sexual relations with her and her menstrual uncleanness touches him, he will be unclean for seven days. Every bed where he sleeps will remain unclean.

\v{25}``When a woman has a continuous discharge of blood many days beyond the time of her menstrual uncleanness, or if she has a discharge that lasts beyond the days of her menstrual uncleanness, her uncleanness is to be treated like the days of her menstruation---she's unclean. \v{26}Every bed on which she sleeps the whole time she has the discharge will be her own unclean bed, so that every object on which she sits becomes unclean like her menstrual uncleanness. \v{27}Whoever touches them will become unclean. He is to wash his clothes and bathe with water and he will remain unclean until evening.

\v{28}``If she becomes clean with her discharge, then she is to count for herself seven days, after which she becomes clean. \v{29}On the eighth day, she is to take for herself two turtledoves or two young pigeons and bring them to the priest at the entrance to the Tent of Meeting. \v{30}Then the priest is to offer one for a sin offering and the other for a whole burnt offering. This is how the priest will make atonement in the \divine{Lord}'s presence for her regarding her unclean discharge.

\v{31}``So separate the Israelis from their uncleanness so that they won't die in their uncleanness if they defile my tent that is in their midst. \v{32}These are the regulations for one whose discharge of semen causes him to become unclean because of it, \v{33}for her whose menstruation causes her to become ill,\fnote{Lit. \fbib{who is unwell due to menstrual uncleanness}} for anyone who has a discharge (whether male or female), and for the man who has sexual relations\fnote{Lit. \fbib{who sleeps}; or \fbib{who lays down}} with one who is unclean.''
\labelchapt{16}
\passage{The Day of Atonement}

\chapt{16}
\v{1}The \divine{Lord} spoke to Moses after the death of Aaron's two sons when they had approached the \divine{Lord} and died. \v{2}The \divine{Lord} told Moses, ``Remind\fnote{Lit. \fbib{Tell}} your brother Aaron that at no time is he to enter the sacred place from the room that contains the curtain into the presence of the Mercy Seat\fnote{Lit. \fbib{atonement place}; and so throughout the book} on top of the ark. Otherwise, he'll die, because I will appear in a cloud at the Mercy Seat. \v{3}Aaron is to enter the sacred place with a young bull for a sin offering and a ram for a whole burnt offering. \v{4}He is to wear a sacred linen tunic and linen undergarments that will cover his genitals. He is to clothe himself with a sash and wrap his head with a linen turban. Because they are sacred garments, he is to wash himself with water before putting them on.''
\passage{The Atonement and Scapegoat Lots}

\v{5}``He is to take two male goats for a sin offering and one ram for a whole burnt offering from the assembly of the Israelis. \v{6}Then Aaron is to bring the bull as a sin offering for himself and make atonement for himself and his household. \v{7}Then he is to take the two male goats and present them in the \divine{Lord}'s presence at the entrance to the Tent of Meeting. \v{8}Aaron is to cast lots over the two male goats---one lot for the \divine{Lord} and the other one for the scapegoat.\fnote{So with LXX; MT reads \fbib{for Azazel}; i.e. the goat that will be sent away} \v{9}Aaron is then to bring the male goat on which the lot fell for the \divine{Lord} and offer it as a sin offering. \v{10}The male goat on which the lot fell for the scapegoat is to be brought alive into the \divine{Lord}'s presence to make atonement for himself. Then he is to send it into the wilderness.''
\passage{The Sin Offering}

\v{11}``Aaron is then to bring the bull for a sin offering for himself, thus making atonement for himself and his household. He is to slaughter the ox for himself. \v{12}Then he is to take a censer and fill it with coals from the fire on the altar in the \divine{Lord}'s presence. With his hands full of spiced and refined incense, he is to bring it beyond the curtain.

\v{13}``Then he is to place the incense over the fire in the \divine{Lord}'s presence, ensuring that the smoke\fnote{Lit. \fbib{cloud}} from the incense covers the Mercy Seat, according to regulation, so he won't die. \v{14}He is to take blood from the ox and sprinkle it with his forefinger toward the surface of the Mercy Seat. Then he is to sprinkle the blood on the surface of the Mercy Seat with his forefinger seven times.

\v{15}``He is to slaughter the male goat as a sin offering for the people and bring its blood beyond the curtain and do with its blood as he did with the blood of the bull: He is to sprinkle it on the Mercy Seat---that is, over the surface of the Mercy Seat. \v{16}Then he is to make atonement on the sacred\fnote{Or \fbib{holy}} place on account of the uncleanness of the Israelis, their transgressions, and all their sins. This is how he is to act in the Tent of Meeting, which will remain with them in the middle of their uncleanness.

\v{17}``No person\fnote{Lit. \fbib{man}} is to be there when he enters the Tent of Meeting to make atonement in the sacred place, until he comes out and has made atonement on account of himself, his household, and the entire assembly of Israel. \v{18}When he goes to the altar in the \divine{Lord}'s presence to make atonement for himself, he is to take some of the blood from the bull and the male goat, place it around the horns of the altar, \v{19}and sprinkle it with the blood on his forefinger seven times, cleansing and sanctifying it from Israel's sins.''
\passage{The Scapegoat Offering}

\v{20}``When he has completed making atonement at the sacred place, the Tent of Meeting, and the altar, then he is to present the live male goat. \v{21}Aaron is to lay his two hands upon the head of the male goat and confess over it the sins of Israel, all their transgressions, and all their sins, thus placing them on the head of the male goat that he'll then send out to the wilderness by the hand of a man capable of carrying out this task.\fnote{The Heb. lacks \fbib{of carrying out this task}} \v{22}The male goat will bear on itself all their sins to a solitary land as Aaron sends the goat out to the wilderness.

\v{23}``Then Aaron is to enter the Tent of Meeting, take off his white linen clothes that he had put on when he entered the sacred place, and leave them there. \v{24}He is to wash his body with water at the sacred place and put on his clothes. Then he is to go out and offer a whole burnt offering for himself and a whole burnt offering for the people, thereby making atonement on account of himself and on account of the people.

\v{25}``As to the fat from the burnt offering, he is to incinerate it on the altar. \v{26}The one who sent away the male goat as a scapegoat\fnote{So with LXX; MT reads \fbib{for Azazel}; i.e. the goat that will be sent away} is to wash his clothes and bathe his body with water. After doing so, he may enter the camp.

\v{27}``The bull for the sin offering and the male goat for the sin offering, whose blood was brought into the sacred place, are to be taken outside the camp. Their skin, meat, and offal are to be incinerated. \v{28}The one who burns them is to wash his clothes and bathe his body with water. After doing so, he may enter the camp.''
\passage{The Perpetual Statute}

\v{29}``This is to be a perpetual statute for you: On the tenth day of the seventh month, you (including both the native born and the resident alien) are to humble yourselves by not doing any work, \v{30}because on that day, atonement will be made\fnote{Lit. \fbib{day, he will make atonement}} for you to cleanse you from all your sins. You are to be clean in the \divine{Lord}'s presence. \v{31}It's the Sabbath of all Sabbaths for you, so humble yourselves. This is to be a perpetual statute. \v{32}The priest who has been anointed and consecrated to be priest after his father is to make the atonement. He is to put on the sacred linen clothing \v{33}and make atonement for the sacred sanctuary, the Tent of Meeting, and the altar where atonement is carried out. He is also to make atonement for the priests and the people of the entire assembly. \v{34}This will be a perpetual statute for you as you make atonement once a year for the Israelis on account of all their sins.''

So Moses did just as the \divine{Lord} had commanded him.
\labelchapt{17}
\passage{Ritual Animal Slaughter}

\chapt{17}
\v{1}The \divine{Lord} told Moses, \v{2}``Speak to Aaron, his sons, and all the Israelis and tell them that this is what the \divine{Lord} has commanded: \v{3}When a person from the house of Israel slaughters an ox, a lamb, or a goat (whether in the camp or outside the camp), \v{4}but fails to bring it to the entrance to the Tent of Meeting as an offering in the presence of the tent of the \divine{Lord}, that person will incur bloodguilt. Because he has shed blood, that person is to be eliminated from contact with\fnote{The Heb. lacks \fbib{from contact with}} his people.''
\passage{Centralized Sacrificial Slaughter}

\v{5}``This statute is required so that\fnote{Lit. \fbib{For the sake of}} the Israelis may bring their sacrifices that they have been sacrificing to the \divine{Lord} in the open field to the priest at the entrance to the Tent of Meeting, where they are to slaughter their peace offering to the \divine{Lord}. \v{6}The priest is to sprinkle the blood on the \divine{Lord}'s altar at the entrance to the Tent of Meeting and incinerate the fat, making a pleasing aroma to the \divine{Lord}. \v{7}They are no longer to slaughter their sacrifices to the goat demons, with whom they have been committing prostitution. This will be a perpetual statute for you throughout your generations. \v{8}Tell them that if a person from the house of Israel or a resident alien who lives among you brings a whole burnt offering or a sacrifice \v{9}to the entrance to the Tent of Meeting, but fails to bring it to offer\fnote{Lit. \fbib{to do}} it to the \divine{Lord}, that person\fnote{Lit. \fbib{man}} is to be eliminated from contact with\fnote{The Heb. lacks \fbib{from contact with}} his people.''
\passage{Prohibitions against Eating Blood}

\v{10}``If anyone from the house of Israel or a resident alien who lives among you eats any form of blood, I'll oppose\fnote{Lit. \fbib{I'll set my face against}} that person who ate the blood and eliminate him from his people, \v{11}because the life of the flesh is in the blood itself, and I myself have given it to you all so that atonement may be made for your souls on the altar, since the blood itself makes atonement through the life that is in it. \v{12}This is why I've told the Israelis that no person\fnote{Lit. \fbib{soul}} among you is to eat blood. Even the resident alien who lives among you is not to eat blood.

\v{13}``If a person from the house of Israel or a resident alien who lives among you has hunted live game or a bird that may be eaten, he is to extract its blood and cover it with soil, \v{14}because the life of any flesh is the blood itself. Therefore, I'm saying to the Israelis that the blood of any flesh is not to be eaten, because the life of any flesh is in its blood. Anyone who eats of it is to be eliminated from contact with his people.\fnote{The Heb. lacks \fbib{from contact with his people}}

\v{15}``Any person who eats a carcass or an animal that was torn by beasts (whether that person is native born or is a resident alien), is to wash his clothes and bathe himself with water, and he will remain unclean until evening, and then he'll become clean. \v{16}But if he doesn't wash or bathe his body, then he is to bear the punishment of his iniquity.''
\labelchapt{18}
\passage{Sexual Relations with Relatives Prohibited}

\chapt{18}
\v{1}The \divine{Lord} told Moses, \v{2}``Tell the Israelis that I am the \divine{Lord} your God. \v{3}You are not to do what you used to do in the land of Egypt where you lived. You are not to do what Canaan does, where I'm about to bring you, so that you live according to their statutes. \v{4}Obey\fnote{Lit. \fbib{do}} my ordinances and keep my statutes by living by them. I am the \divine{Lord} your God. \v{5}Keep my statutes and my ordinances, which a person\fnote{Lit. \fbib{man}} is to obey in order to live in them. I am the \divine{Lord}.

\v{6}``A person is not to approach a near blood relative for sexual relations.\fnote{Lit. \fbib{relative to expose nakedness}, and so throughout the chapter} I am the \divine{Lord}.

\v{7}``Neither your father's nakedness nor your mother's nakedness is to be exposed. She's your mother, so you are not to have sexual relations with her.

\v{8}``You are not to have sexual relations with your father's wife. It's your own father's nakedness.

\v{9}``You are not to have sexual relations with your sister, whether she's your father's daughter or your mother's daughter, whether she's born in your home or outside your home. You are not to have sexual relations with her.

\v{10}``You are not to have sexual relations with your son's daughter or your daughter's daughter. You are not to have sexual relations with them, because their nakedness is your own nakedness.

\v{11}``You are not to have sexual relations with the daughter of your father's wife. Born of your father, she's your sister, so you are not to have sexual relations with her.

\v{12}``You are not to have sexual relations with your father's sister. She's your father's near blood relative.

\v{13}``You are not to have sexual relations with your mother's sister. She's your mother's near blood relative.

\v{14}``You are not to expose the nakedness of your father's brother by having sexual relations with his wife. She's your aunt.

\v{15}``You are not to expose the nakedness of your daughter-in-law. She's the wife of your son. You are not to have sexual relations with her.

\v{16}``You are not to have sexual relations with your brother's wife. She's the nakedness of your brother.

\v{17}``You are not to have sexual relations with a woman and her daughter.

``You are not to have sexual relations with her son's daughter or her daughter's daughter. They're near blood relatives. It's wickedness.

\v{18}``You are not to marry a woman and then have sexual relations with her sister as a rival when your wife\fnote{Lit. \fbib{when she}} is still alive.

\v{19}``You are not to approach a menstruating woman to have sexual relations with her.\fnote{The Heb. lacks \fbib{to have sexual relations with her}}

\v{20}``You are not to have sexual relations with your neighbor's wife and thereby become ceremonially unclean with her.''
\passage{Child Sacrifice Prohibited}

\v{21}``You are not to present any of your children to Molech as a sacrifice.\fnote{Lit. \fbib{to Molech to pass through}; i.e. to incinerate an infant as a fire sacrifice} That way, you won't defile the name of your God.''
\passage{Same Sex Unions Prohibited}

``I am the \divine{Lord}. \v{22}You are not to have sexual relations\fnote{Lit. \fbib{to lie down}, and so throughout the chapter} with a male as you would with a woman. It's detestable.''
\passage{Sexual Relations with Animals Prohibited}

\v{23}``You are not to present yourself to an animal in order to have sexual relations with it and by doing so to defile yourself. A woman is not to present herself to an animal to have sexual relations with it. It's detestable.

\v{24}``You are not to defile yourselves by doing any of these things, since all of these nations that I'm casting out before you have defiled themselves this way. \v{25}The land has been defiled, so I brought the punishment of its iniquity to it. As a result, the land is vomiting out its inhabitants.

\v{26}``Therefore, keep my statutes and ordinances. You are not to do any of these detestable things---this applies to the native born and the resident alien who lives among you--- \v{27}because the inhabitants\fnote{Lit. \fbib{men}} of the land did all of these detestable things and by doing so defiled the land before you. \v{28}So you are not to let the land vomit you up because of your uncleanness as it is vomiting the nations that were here before you. \v{29}Anyone who does any of these detestable things---whoever the person\fnote{Lit. \fbib{souls}} may be---is to be eliminated from contact with his people.\fnote{The Heb. lacks \fbib{from contact with his people}} \v{30}Therefore, keep my injunctions so that you won't practice these detestable things that have been done before you, and so that you won't be defiled in them. I am the \divine{Lord}.''
\labelchapt{19}
\passage{Ritual Purity}

\chapt{19}
\v{1}The \divine{Lord} spoke to Moses, \v{2}``Tell the entire assembly of Israel that they are to be holy, since I, the \divine{Lord} your God, am holy.

\v{3}``Each of you is to fear his mother and father.

``Observe my Sabbaths. I am the \divine{Lord} your God.

\v{4}``You are not to turn to their idols or cast gods out of melted metal for yourselves. I am the \divine{Lord} your God.

\v{5}``When you offer a peace offering to the \divine{Lord}, you are to offer it for your acceptance. \v{6}Your sacrifice is to be eaten on that day and the next day. Anything that remains to the third day is to be incinerated. \v{7}If it is eaten on the third day, it's unclean. It won't be accepted. \v{8}Anyone who eats it will bear the punishment of his sin, since he will have defiled himself regarding the \divine{Lord}'s holy things. That person\fnote{Lit. \fbib{soul}} is to be eliminated from contact with his people.''\fnote{The Heb. lacks \fbib{from contact with his people}}
\passage{Harvesting and Gleaning}

\v{9}``When you reap the harvest of your land, you are not to completely finish harvesting the corners of the field---that is, you are not to pick what remains after you have reaped your harvest. \v{10}You are not to gather your vineyard or pick up the fallen grapes of your vineyard. Leave something for the poor and the resident alien who lives among you. I am the \divine{Lord} your God.''
\passage{Just Dealings}

\v{11}``You are not to steal or lie or deal falsely with your neighbor.

\v{12}``You are not to use my name to deceive, thereby defiling the name of your God. I am the \divine{Lord}.

\v{13}``You are not to oppress your neighbor or rob him.\fnote{The Heb. lacks \fbib{him}}

``The wages of a hired laborer are not to remain in your possession until morning.

\v{14}``You are not to curse a deaf person or put a stumbling block before the blind.

``You are to fear God. I am the \divine{Lord}.

\v{15}``You are not to be unjust in deciding a case. You are not to show partiality to the poor or honor the great. Instead, decide the case of your neighbor with righteousness.''
\passage{Social Responsibility}

\v{16}``You are not to go around slandering your people.

``You are not to stand idle\fnote{The Heb. lacks \fbib{idle}} when your neighbor's life is at stake.\fnote{Lit. \fbib{stand on the blood of your neighbor}} I am the \divine{Lord}.

\v{17}``You are not to hate your relative in your heart. Rebuke your neighbor if you must, but you are not to incur guilt on account of him.

\v{18}``You are not to seek vengeance or hold a grudge against the descendants of your people. Instead, love your neighbor as yourself. I am the \divine{Lord}.''
\passage{On Preserving Distinctiveness}

\v{19}``Observe my statutes.

``You are not to let your cattle breed with a different species.\fnote{Lit. \fbib{breed within two kinds}}

``You are not to sow your fields with two different kinds of seeds.\fnote{The Heb. lacks \fbib{of seeds}}

``You are not to wear clothing made from two different kinds of material.

\v{20}``When a person has sexual relations\fnote{Lit. \fbib{a lying of seed}} with a woman servant who is engaged to another man, but she has not been completely redeemed nor has her freedom been granted to her, there is to be an inquiry, but they won't be put to death, since she has not been freed. \v{21}The perpetrator\fnote{Lit. \fbib{He}} is to bring his guilt offering to the \divine{Lord} at the entrance to the Tent of Meeting---that is, a ram as a guilt offering. \v{22}Then the priest is to make atonement for him with the ram as a guilt offering in the \divine{Lord}'s presence on account of his sin which he has committed, but which will be forgiven him.''
\passage{Restrictions on Initial Harvests}

\v{23}``When you have entered the land and planted all sorts of trees for food, regard its fruit as uncircumcised for the first three years for you. It is not to be eaten. \v{24}During the fourth year, all its fruit is to be offered as a holy token of praise to the \divine{Lord}. \v{25}But on the fifth year, you may eat its fruits to increase its produce for you.''
\passage{Prohibited Practices}

\v{26}``You are not to eat anything containing blood, engage in occult practices,\fnote{I.e. divination} or practice fortune telling.\fnote{Or \fbib{practice witchcraft}}

\v{27}``You are not to cut your hair in ritualistic patterns\fnote{Lit. \fbib{cut the sides of your hair}; i.e. as a sign of affiliation} on your head or deface the edges of your beard.

\v{28}``You are not to make incisions in your flesh on account of the dead nor submit to cuts or tattoos. I am the \divine{Lord}.

\v{29}``You are not to defile your daughter by engaging her in prostitution so the land won't become filled with wickedness.

\v{30}``Observe my Sabbath and stand in awe of my sanctuary. I am the \divine{Lord}.

\v{31}``You are to consult neither mediums nor familiar spirits. You are never to seek them---you'll just be defiled by them. I am the \divine{Lord} your God.

\v{32}``Rise in the presence of the aged\fnote{Lit. \fbib{of the grey head}} and honor the elderly face-to-face.

``Fear your God. I am the \divine{Lord}.

\v{33}``If a resident alien lives with you in your land, you are not to mistreat him. \v{34}You are to treat the resident alien the same way you treat the native born among you---love him like yourself, since you were foreigners in the land of Egypt.

\v{35}``You are not to act unjustly in deciding a case\fnote{Lit. \fbib{in judgment}} or when measuring weight and quantity. \v{36}You are to maintain just balances and reliable standards for weights, dry volumes, and liquid volumes.\fnote{Lit. \fbib{and honest weight, ephah, and hin}} I am the \divine{Lord} your God, who brought you out of the land of Egypt. \v{37}Observe all my statutes and all my ordinances in order to practice them. I am the \divine{Lord}.''
\labelchapt{20}
\passage{Prohibiting Child Sacrifice}

\chapt{20}
\v{1}The \divine{Lord} spoke to Moses, \v{2}``Tell the Israelis that when an Israeli or a resident alien\fnote{Or \fbib{foreigner who lives with you}} who lives in Israel offers\fnote{Or \fbib{gives}} his child to Molech, he is certainly to be put to death.\fnote{Lit. \fbib{to die, he'll die}} The people who live in the land are to stone him with stones. \v{3}As for me, I'll oppose that man. I'll eliminate him from contact with his people\fnote{The Heb. lacks \fbib{from contact with his people}} for sacrificing his children to Molech, thereby defiling my sanctuary and profaning my holy name. \v{4}If the people avoid dealing\fnote{Lit. \fbib{people conceal their face from}} with that man when he offers his child to Molech---that is, if they fail to execute him--- \v{5}then I'll oppose that man and his family and eliminate him from contact with his people,\fnote{The Heb. lacks \fbib{from contact with his people}} along with all the prostitutes who accompany him and who have committed prostitution with Molech.''
\passage{Consulting the Dead Prohibited}

\v{6}``I'll oppose and eliminate from contact with his people\fnote{The Heb. lacks \fbib{contact with his people}} whoever consults mediums or familiar spirits, thereby committing spiritual prostitution with them. \v{7}Therefore, separate yourselves and be holy, because I am the \divine{Lord} your God. \v{8}Keep my statutes and observe them. I am the \divine{Lord}, who has set you apart.''
\passage{Honoring Parents}

\v{9}``Anyone who curses his father or mother is certainly to be put to death.\fnote{Lit. \fbib{to die, he'll die}} He has cursed his father or mother, so his guilt will remain his responsibility.''
\passage{Honoring the Seventh Commandment}

\v{10}``If anyone commits adultery with another man's wife, including when someone commits adultery with his neighbor's wife, both the adulterer and the adulteress are to die.

\v{11}``If a man has sexual relations with his father's wife, he has exposed his father's nakedness, so both of them are to be put to death. Their guilt will remain their responsibility.

\v{12}``If a man has sexual relations with his daughter-in-law, the two are to be put to death. They've committed a repulsive act. Their guilt\fnote{Lit. \fbib{blood}} will remain their responsibility.

\v{13}``If a man has sexual relations with another male as he would with a woman, both have committed a repulsive act. They are certainly to be put to death.

\v{14}``If a man takes a wife along with her mother, that's wickedness. They are to be burned with fire---that is, both him and them, so that there will be no wickedness in your midst.

\v{15}``If a man has sexual relations with an animal, he is to be put to death, and you are also to kill the animal.

\v{16}``If a woman approaches any animal to have sexual relations with it, both the woman and the animal are to be put to death. Their guilt\fnote{Lit. \fbib{blood}} will remain their responsibility.

\v{17}``If a man takes his sister, his father's daughter, or his mother's daughter, so that he exposes her nakedness and she exposes his nakedness, it's a shameful thing. They are to be eliminated from contact with their people\fnote{The Heb. lacks \fbib{from contact with their people}} in front of their people's children. He has exposed his sister's nakedness. He'll continue to bear responsibility for\fnote{The Heb. lacks \fbib{responsibility for}} his iniquity.

\v{18}``If a man has sexual relations with a menstruating woman, he has exposed her nakedness, laying bare her fountain. He has exposed the source of her blood. Both are to be eliminated from contact with their people.\fnote{The Heb. lacks \fbib{from contact with their people}}

\v{19}``You are not to have sexual relations with your mother's sister or your father's sister, because that is laying bare the nakedness of his close relative. They'll continue to bear responsibility for\fnote{The Heb. lacks \fbib{responsibility for}} their iniquity.

\v{20}``If a man has sexual relations with his uncle's wife, he has exposed his uncle's nakedness. They are to bear responsibility for\fnote{The Heb. lacks \fbib{responsibility for}} punishment of their sin. They'll die childless.

\v{21}``If a man takes his brother's wife, it's immoral.\fnote{Lit. \fbib{an impurity}} He has exposed his brother's nakedness. They'll be childless.''
\passage{Living Distinctively in Holiness}

\v{22}``Be sure to keep all my statutes and observe all my ordinances, so that the land where I'm about to bring you to live won't vomit you out. \v{23}You are not to live\fnote{Lit. \fbib{walk}} by the customs of the nations, whom I've cast away right in front of you. Because they did all of these things, I detested them. \v{24}But I've promised\fnote{Lit. \fbib{said}} you that you'll inherit the land that I'm about to give you as your permanent possession\fnote{Lit. \fbib{you to inherit}}---a land flowing with milk and honey.

``I am your God. I've separated you from the people. \v{25}You are to differentiate between the clean animal and the unclean and between the unclean bird and the clean. You are not to make yourselves detestable on account of any animal, bird, or any creeping creature of the ground that I've separated for you as unclean.

\v{26}``You are to be holy toward me, because I, the \divine{Lord}, am holy. I've separated you from among the people to be mine.

\v{27}``Moreover, a man or a woman who has a ritual spirit or a familiar spirit is certainly to die. They are to be stoned to death with boulders. They will continue to bear responsibility for their guilt.''\fnote{Lit. \fbib{blood}}
\labelchapt{21}
\passage{Priestly Holiness}

\chapt{21}
\v{1}The \divine{Lord} told Moses, ``Speak to the priests, Aaron's sons, and tell them that no priest is to defile himself on account of the dead among his people, \v{2}except his close relatives---his mother, father, son, daughter, brother, or \v{3}virgin sister (who is a near relative of him and did not have a husband---\fnote{Lit. \fbib{hasn't had a man}} he may defile himself for her). \v{4}Because he is a husband among his people, he is not to defile himself, thereby polluting himself.

\v{5}``They are not to cut their hair in ritualistic patterns\fnote{Lit. \fbib{cut the sides of their hair}; i.e. as a sign of affiliation} on their heads, deface the edges of their beards, or make incisions in their flesh. \v{6}They are to be holy to their God. They are not to defile the name of their God, because they're the ones who bring the offerings of the \divine{Lord} made by fire---the food of their God---so they are to be holy.

\v{7}``They are not to marry\fnote{Or \fbib{take}} a prostitute or a woman who has been dishonored or who was divorced from her husband, because the priest\fnote{Lit. \fbib{he}} is holy to his God. \v{8}Consecrate him, because he's the one who offers the food of your God. He is to be holy for you, because I, the \divine{Lord}, the one who sanctifies you, am holy.

\v{9}``Now if the daughter of any priest defiles herself by being a prostitute, she defiles her father. She is to be incinerated.

\v{10}``The high priest among his relatives---whose head has been anointed with oil and who has been consecrated to put on the priestly clothing---is not to let his hair hang loose or to tear his clothes. \v{11}He is not to come near any dead body---whether the deceased\fnote{The Heb. lacks \fbib{the deceased}} is his father or his mother---so as not to defile himself. \v{12}He is not to go out of the sanctuary or defile the sanctuary of his God, because his God's consecrating oil of anointing rests on him. I am the \divine{Lord}.

\v{13}``Furthermore, he is to marry a true virgin.\fnote{Lit. \fbib{a wife in her virginity}} \v{14}He is not to marry a widow or one who has been divorced, has been defiled, or has been a prostitute. Instead, he is to take a virgin from among his people as his wife.

\v{15}``He is not to defile his children\fnote{Or \fbib{offspring}} among his people, because I am the \divine{Lord}, who sets him apart.''
\passage{On Physical Defects}

\v{16}The \divine{Lord} told Moses, \v{17}``Tell Aaron that whoever of your descendants throughout their generations has a bodily defect is not to approach to offer the food of his God. \v{18}Indeed, any person who has a defect is not to approach the Tent of Meeting---\fnote{The Heb. lacks \fbib{the Tent of Meeting}} the blind, the lame, one who is mutilated in the face or who has a very long limb, \v{19}or a person who has a fractured foot or hand, \v{20}has scoliosis,\fnote{Or \fbib{has a crooked back}} is a dwarf, or has an eye defect, an itching disease, scabs, or a crushed testicle. \v{21}None of the descendants of Aaron the priest who has a defect is to approach to bring offerings of the \divine{Lord} made by fire, since he has a defect. He is not to approach to offer the food of his God. \v{22}However, he may eat the food of his God, including the most holy and the holy offerings, \v{23}but he is not to enter through the curtain nor approach the altar, since he has a defect. That way, he won't defile my sanctuary, since I am the \divine{Lord}, who sets you apart.''

\v{24}Moses told all of this\fnote{The Heb. lacks \fbib{all of this}} to Aaron, to his sons, and to all the Israelis.
\labelchapt{22}
\passage{Holy Offerings}

\chapt{22}
\v{1}Later on, the \divine{Lord} told Moses, \v{2}``Tell Aaron and his sons that they are to separate themselves for the sacred things of the Israelis and that they are not to defile my holy name. I am the \divine{Lord}. \v{3}Tell them that whoever among your descendants throughout your generations approaches the sacred things that the Israelis had consecrated to the \divine{Lord} while still remaining unclean is to be eliminated from my presence. I am the \divine{Lord}. \v{4}If one of Aaron's descendants has an infectious skin disease or a discharge, he is not to eat anything sacred until he has been cleansed. Anyone who touches an unclean thing on account of the dead, or who has a seminal discharge, \v{5}or who becomes unclean by touching a creeping creature or another human being, whatever the uncleanness may be---\v{6}such a person\fnote{Lit. \fbib{soul}} who comes in contact with anything like this will become unclean until evening. As a result, he is not to eat the sacred things unless he has bathed himself\fnote{Lit. \fbib{his body}} with water. \v{7}When the sun has gone down and he has been cleansed, he may eat of the sacred things, since that's his food. \v{8}He is not to eat the carcass of an animal that was torn by animals,\fnote{The Heb. lacks \fbib{by animals}} thereby defiling himself with it. I am the \divine{Lord}. \v{9}They are to keep my charge. By doing so, they won't bear the punishment of sin because of it and therefore die if they've been defiled by it. I am the \divine{Lord}, who sets them apart.''
\passage{Other Prohibitions}

\v{10}``No resident alien is to eat anything sacred. Neither the visitor\fnote{Lit. \fbib{sojourner}} of the priest nor a hired laborer is to eat anything sacred. \v{11}If a priest acquires a slave as property with his own money, he may eat with him. Those who were born in his house may eat his food. \v{12}If a priest's daughter marries a resident alien, she is not to eat the sacred raised offerings. \v{13}If the priest's daughter is a widow, or is divorced and childless,\fnote{Lit. \fbib{There's no offspring to her}} so that she has to return to her father's house as in her younger days,\fnote{Lit. \fbib{early life}} she may eat her father's food, but no resident alien may eat it. \v{14}If a person eats anything sacred inadvertently, he is to add a fifth part to it and then give the sacred thing to the priest. \v{15}They are not to defile the sacred things of the Israelis that they have offered\fnote{Lit. \fbib{to rise}} to the \divine{Lord}, \v{16}thereby causing them to bear the punishment of their iniquity for wrongdoing when they eat their sacred things, because I am the \divine{Lord}, who sets them apart.''
\passage{Acceptable Offerings}

\v{17}The \divine{Lord} told Moses, \v{18}``Tell Aaron, his sons, and all the Israelis that when a person from the house of Israel or from the resident aliens living in Israel brings his offering to the \divine{Lord} as a whole burnt offering (whether in fulfillment of a promise or a free will offerings), \v{19}so that he'll be sure to be accepted,\fnote{Lit. \fbib{for your acceptance}} he is to offer\fnote{The Heb. lacks \fbib{he is to offer}} a male without defect from the bulls, the lambs, and the goats. \v{20}However, whatever has a defect is not to be offered, because it won't be acceptable for you.

\v{21}``If a person brings a peace offering sacrifice to the \divine{Lord} to fulfill a vow or a free will offering from the herd or the flock, it is to be sound in order to be accepted, without any defect in it. \v{22}You are not to bring to the \divine{Lord} an offering that is blind, fractured, mutilated, or infected with ulcers, scurvy, or scales. You are not to present any of them as an offering made by fire on the altar for the \divine{Lord}.

\v{23}``You may offer a bull or lamb that has one limb longer than the other or that is stunted as a free will offering, but it's not acceptable in fulfillment of a promise. \v{24}You are not to bring to the \divine{Lord} an animal\fnote{The Heb. lacks \fbib{animal}} that has been emasculated, crushed, torn, or cut apart. You are not to practice this in your land. \v{25}A resident alien is not to offer as food to your God any of these items, because they are afflicted with ritual corruption due to their defects. They're not acceptable for you.''

\v{26}The \divine{Lord} told Moses, \v{27}``Whenever a bull, a sheep, or a goat is born, it is to remain for seven days under the care of its mother. But on the eighth day onwards, it may be accepted as an offering made by fire to the \divine{Lord}. \v{28}However, you are not to slaughter a bull or a ewe along with its offspring on the same day. \v{29}When you offer\fnote{Lit. \fbib{sacrifice}} a sacrifice of thanksgiving to the \divine{Lord}, bring it so that it's acceptable for you. \v{30}It is to be eaten that same day. You are not to leave any of it until morning. I am the \divine{Lord}.

\v{31}``Keep my commands and observe them. I am the \divine{Lord}.

\v{32}``You are not to defile my sacred name, because I've been set apart in the midst of the Israelis. Furthermore, I am the \divine{Lord}, who sets you apart--- \v{33}who brought you out of the land of Egypt to be your God. I am the \divine{Lord}.''
\labelchapt{23}
\passage{Scheduled Festivals}

\chapt{23}
\v{1}The \divine{Lord} told Moses, \v{2}``Tell the Israelis that these are my festival times appointed by the \divine{Lord}\fnote{Lit. \fbib{appointed times for festivals,} and so throughout the chapter} that you are to declare as sacred assemblies: \v{3}Six days you may work, but the seventh day is a Sabbath of rest, a sacred assembly. You are not to do any work. It's a Sabbath to the \divine{Lord} wherever you live.\fnote{Lit. \fbib{\divine{Lord} in all your dwelling places}, and so throughout the chapter} \v{4}These are the \divine{Lord}'s appointed festivals and sacred assemblies that you are to declare at their appointed time.

\v{5}``The \divine{Lord}'s Passover is to begin on the fourteenth day of the first month at twilight.\fnote{Lit. \fbib{between evenings}} \v{6}On the fifteenth day of that month is the Festival of Unleavened Bread to the \divine{Lord}. For seven days you are to eat unleavened bread. \v{7}On the first day that you hold the sacred assembly, you are to do no servile work. \v{8}Instead, you are to bring an offering made by fire to the \divine{Lord} daily for seven days. On the seventh day, you are also to hold a sacred assembly during which you are to do no servile work.''
\passage{First Fruit Offerings}

\v{9}The \divine{Lord} told Moses, \v{10}``Tell the Israelis that when you enter the land that I'm about to give you and gather its produce, you are to bring a sheaf from the first portion of your harvest to the priest, \v{11}who will offer the sheaf in the \divine{Lord}'s presence for your acceptance. The priest is to wave it on the day after the Sabbath. \v{12}On the day you wave the sheaf, you are to offer a one year old male lamb without defect for a burnt offering in the \divine{Lord}'s presence. \v{13}Also present a meal offering of two tenths of a measure of\fnote{The unit of measurement is not specified in MT, but cf. Lev. 5:11, 6:20.} fine flour mixed with olive oil as an offering made by fire to the \divine{Lord}, a pleasing aroma. Now as to a drink offering, you are to present a fourth of a hin\fnote{I.e. about one quart; the \fbib{hin} was equivalent to about one gallon} of wine. \v{14}You are not to eat bread, parched grain, or fresh grain until that day\fnote{Lit. \fbib{grain until the bone of this day}} when you've brought the offering of your God. This is to be\fnote{The Heb. lacks \fbib{This is to be}} an eternal ordinance throughout your generations, wherever you live.''
\passage{New Meal Offerings}

\v{15}``Starting the day after the Sabbath, count for yourselves seven weeks from the day you brought the sheaf of the wave offering. They are to be complete. \v{16}Count fifty days to the day after the seventh Sabbath, then bring a new meal offering to the \divine{Lord}. \v{17}Bring two loaves\fnote{The Heb. lacks \fbib{loaves}} of bread from home as wave offerings made from two tenths of fine flour baked with leaven as first fruits to the \divine{Lord}. \v{18}Along with the loaves of bread, bring seven lambs (each of them\fnote{The Heb. lacks \fbib{each of them}} one year old and without defect), one young bull as an offering, and two rams as offerings to the \divine{Lord}---along with your gift and drink offerings---and present them as an offering made by fire, a pleasing aroma to the \divine{Lord}. \v{19}Prepare one male goat for a sin offering and two one year old rams for peace offerings. \v{20}Then the priest is to wave them---the two lambs with the bread of first fruits---as raised offerings in the \divine{Lord}'s presence. They'll be sacred to the \divine{Lord} on account of the priest.

\v{21}``On the same day, proclaim a sacred assembly for yourselves. You are not to do any servile work---and this is to be an eternal ordinance wherever you live throughout your generations. \v{22}Furthermore, when you harvest the produce of your land, you are not to harvest all the way to the corners of your field or gather the gleanings of your harvest. Leave them for the poor and resident alien. I am the \divine{Lord} your God.''
\passage{Offerings in the Seventh Month}
\passageinfo{(Numbers 29:1-6)}

\v{23}The \divine{Lord} told Moses, \v{24}``Tell the Israelis that on the first day of the seventh month you are to have a Sabbath of rest for you---a memorial announced by a loud blast of trumpets. It is to be a sacred assembly. \v{25}You are not to do any servile work. Instead, bring an offering made by fire to the \divine{Lord}.''
\passage{Day of Atonement}

\v{26}The \divine{Lord} spoke to Moses, \v{27}``However, on the tenth day of this seventh month is the Day of Atonement. It's a sacred assembly for you. Humble yourselves\fnote{Lit. \fbib{your souls}} and bring an offering made by fire to the \divine{Lord}. \v{28}You are not to do any work that same day. It's the Day of Atonement, because your atonement is made in the presence of the \divine{Lord} your God. \v{29}Anyone who doesn't humble himself that same day is to be eliminated from contact with his people.\fnote{The Heb. lacks \fbib{from contact with his people}} \v{30}I'll eliminate anyone who does work that day from among his people. \v{31}You are not to do any work. This is to be an eternal ordinance throughout your generations, wherever you live. \v{32}It's a Sabbath of rest\fnote{Lit. \fbib{Sabbath of all Sabbath}} for you on which you are to humble yourselves starting the evening of the ninth day of the month. You are to observe your Sabbath from evening to evening.''
\passage{Festival of Tents}

\v{33}The \divine{Lord} spoke to Moses, \v{34}``Tell the Israelis that starting the fifteenth day of this seventh month is the week-long Festival of Tents to the \divine{Lord}. \v{35}On the first day, you are to hold a sacred assembly, when you are not to do any servile work. \v{36}For seven days, bring offerings made by fire to the \divine{Lord}. The eighth day is also to be a sacred assembly for you. Bring offerings made by fire to the \divine{Lord}. It's a sacred assembly. You are not to do any servile work.

\v{37}``These are the \divine{Lord}'s appointed festivals that you are to proclaim as sacred assemblies. Bring offerings made by fire to the \divine{Lord}---a whole burnt offering, a meal offering, a sacrifice, and drink offerings. Do this every day on its assigned date \v{38}in addition to the \divine{Lord}'s Sabbath---regarding your gifts, your offerings in fulfillment of vows, and your freely given offerings that you will bring to the \divine{Lord}.

\v{39}``On the fifteenth day of the seventh month, when you've harvested the produce of the land, you are to observe the festival of the \divine{Lord} for seven days. The first day is to be a Sabbath rest, and the eighth day also is to be a Sabbath rest.

\v{40}``On the first day, take branches from impressive fruit trees,\fnote{Lit. \fbib{from fruit from impressive trees}} branches from palm trees, boughs from thick trees, and poplars from the brooks. Then you are to rejoice in the presence of the \divine{Lord} your God for seven days. \v{41}Observe it as a pilgrimage festival in the presence of the \divine{Lord} for seven days of the year. This is to be an eternal ordinance throughout your generations. Observe the festival during the seventh month. \v{42}You are to live in tents for seven days. Every native born of Israel is to live in tents \v{43}in order for your future\fnote{The Heb. lacks \fbib{future}} generations to know that the Israelis lived in tents when I brought them out of the land of Egypt. I am the \divine{Lord} your God.''

\v{44}This is what Moses spoke about to the Israelis regarding the \divine{Lord}'s appointed festivals.
\labelchapt{24}
\passage{The Lamp}

\chapt{24}
\v{1}The \divine{Lord} spoke to Moses, \v{2}``Tell the Israelis that they are to bring to you pure oil made from beaten olives in order to keep the lamp burning continuously. \v{3}Outside the Canopy of the Testimony in the Tent of Meeting, Aaron is to arrange it continually in the \divine{Lord}'s presence from evening until morning as an eternal ordinance throughout your generations. \v{4}He is to arrange the lamps so that they burn continuously on a ceremonially pure lamp stand in the \divine{Lord}'s presence. \v{5}Take fine flour and bake twelve cakes using two tenths of a measure\fnote{The unit of measurement is not specified in MT, but cf. Lev. 5:11, 6:20.} for each cake. \v{6}Arrange them in two rows---six in each row---on a ceremonially pure table in the \divine{Lord}'s presence. \v{7}Put pure frankincense on each row for a memorial offering. It will serve as an offering made by fire to the \divine{Lord}. \v{8}They are to be arranged every Sabbath day\fnote{Lit. \fbib{in the day of the Sabbath, in the day of the Sabbath}} in the \divine{Lord}'s presence as a gift\fnote{The Heb. lacks \fbib{as a gift}} from the Israelis---an eternal covenant. \v{9}This gift\fnote{The Heb. lacks \fbib{This gift}} will belong to Aaron and his sons, and they are to eat it in a sacred place, because it's the most holy thing for him of all the offerings made by fire to the \divine{Lord}. This is to be an eternal ordinance.''
\passage{A Case History of Blasphemy}

\v{10}Now a son of an Israeli woman and an Egyptian man\fnote{Lit. \fbib{woman the son of an Egyptian man}} went out among the Israelis. The Israeli woman's son got into a fight with an Israeli man in the camp. \v{11}Then the Israeli woman's son blasphemed the Name and cursed, so they brought him to Moses. His mother's name was Shelomith, the daughter of Dibri, from the tribe of Dan. \v{12}They placed him in custody until a decision would be made\fnote{The Heb. lacks \fbib{would be made}} to them according to the word\fnote{Lit. \fbib{mouth}} of the \divine{Lord}. \v{13}Then the \divine{Lord} spoke to Moses, \v{14}``Take the one who cursed outside the camp. Everyone who heard him is to lay their hands on his head. Then the entire congregation is to stone him to death. \v{15}Moreover, tell the Israelis that anyone who curses his God will bear the consequences of his own sin, \v{16}because the one who blasphemes the name of the \divine{Lord} is certainly to be put to death. The entire congregation is to stone him to death. As it is for the resident alien, so it is to be with the native born: when he blasphemes the Name, he is to be put to death.

\v{17}``If a man beats a human being\fnote{Lit. \fbib{soul of mankind}} to death,\fnote{The Heb. lacks \fbib{to death}} he is certainly to be executed, \v{18}but whoever beats an animal to death is to replace it---life for life. \v{19}If a man disfigures his fellow, whatever he did is to be done to him also. \v{20}Fracture for fracture, eye for eye, tooth for tooth---just as he had caused a disfigurement against another man, so it is to be done against him. \v{21}Whoever beats an animal to death is to replace it, but whoever beats a human being to death\fnote{The Heb. lacks \fbib{to death}} is to be put to death. \v{22}You are to have for yourselves consistent\fnote{Lit. \fbib{one}} procedures in deciding a case. As it is for the resident alien, so it is for the native born. I am the \divine{Lord} your God.''

\v{23}So Moses spoke to the Israelis and they brought the one who cursed outside the camp and stoned him to death with boulders. The Israelis did just as the \divine{Lord} had commanded Moses.
\labelchapt{25}
\passage{Sabbatical Years}

\chapt{25}
\v{1}The \divine{Lord} told Moses on Mount Sinai, \v{2}``Tell the Israelis that when you enter the land that I'm about to give you, you are to let the land observe a Sabbath to the \divine{Lord}. \v{3}For six years you may plant your fields, and for six years you may prune your vineyard and gather its produce. \v{4}But the seventh year is to be a Sabbath of rest for the land---a Sabbath for the \divine{Lord}. You are not to plant your field or prune your vineyard. \v{5}You are not to gather what grows from the spilled kernels of your crops. You are not to pick the grapes of your untrimmed vines. Let it be a year of Sabbath for the land. \v{6}You may take the Sabbath produce\fnote{The Heb. lacks \fbib{produce}} of the land for your food---you, your male and maid servants, your hired laborers, and the resident alien with you. \v{7}The cattle and the wild animals in your land---everything it produces---are for your food.

\v{8}``Count for yourselves seven years of Sabbaths---seven times seven years. This set of seven weeks of years total 49 years for you. \v{9}Sound a horn on the tenth day of the seventh month of this fiftieth year.\fnote{The Heb. lacks \fbib{of this fiftieth year}} Likewise, on the Day of Atonement, sound the horn throughout your land. \v{10}Set aside and consecrate the fiftieth year to declare liberty throughout the land for all of its inhabitants. It is to be a jubilee for you. Every person\fnote{Lit. \fbib{man}} is to return to his own land that he has inherited. Likewise, every person is to return to his tribe. \v{11}The fiftieth year is to be a year of jubilee for you. You are not to sow or harvest the spilled kernels that grow of themselves or pick grapes from the untrimmed vines \v{12}because it's the jubilee---it's sacred for you. But you may eat its produce from the field.

\v{13}``During this year of jubilee, each person is to return to his own land that he has inherited. \v{14}So if you had sold property\fnote{Lit. \fbib{sold a ware}} to a neighbor or had acquired land from your neighbor, you are not to cheat one another. \v{15}According to the number of years after the jubilee, you may buy from your neighbor. And according to the number of years with crops, he may sell to you. \v{16}If the number of years after the jubilee\fnote{The Heb. lacks \fbib{after the jubilee}} is more, increase the selling price. If the number of years after the jubilee\fnote{The Heb. lacks \fbib{after the jubilee}} is few, decrease its selling price, because he's selling to you according to the potential production volume\fnote{Lit. \fbib{the number}} of the land.\fnote{The Heb. lacks \fbib{of the land}} \v{17}No one is to cheat his neighbor. Instead, you are to fear your God, because I am the \divine{Lord} your God.

\v{18}``Observe my statutes and keep my ordinances. Do them so that you may live securely in the land. \v{19}Then the land will yield its fruit and you'll eat to your satisfaction and live securely.

\v{20}``Now if you ask, `What will we eat during the seventh year? After all, we may not plant or even gather our produce!' \v{21}I'll command my blessing on you during the sixth year so that it will yield produce for three years! \v{22}That way, you are to sow in the eighth year, eating the produce from the old harvest. Until the ninth year when its produce comes in, you'll eat from the old harvest.''
\passage{Land Redemption}

\v{23}``The land is not to be sold with any finality, because the land belongs to me. You're sojourners and travelers\fnote{Lit. \fbib{you are travelers with me}} with me. \v{24}So throughout all of your land inheritance,\fnote{Or \fbib{possession}} grant the right of redemption for the land.

\v{25}``If your brother becomes so poor that he has to a sell portion of his inheritance, then his nearest kinsman redeemer is to come and redeem what his brother has sold. \v{26}If a person\fnote{Lit. \fbib{man}} doesn't have a kinsman redeemer, but has become rich\fnote{Lit. \fbib{but his hands had overtaken with blessings}} and found sufficient means for his redemption, \v{27}then let him account for the years for which it was sold, return the excess to the person to whom it was sold, and then return to his property. \v{28}If he's not able to redeem it back for himself,\fnote{Lit. \fbib{If his hand can't acquire it back for himself}} then what he sold is to remain in the hand of the buyer until the year of jubilee. In the jubilee, it is to be returned so he may return to his property.

\v{29}``If a person sells a residential house in a walled city, he is to redeem it within the year in which it was sold. He may have right to its redemption for a full year. \v{30}But if it's not redeemed by the end of a full year, then the house next to which is a wall is to belong in perpetuity to the one who bought it throughout his generations. It is not to be returned in the jubilee. \v{31}However, the houses in the villages that don't have walls around them are to be categorized along with the fields of the land---they may be redeemed and returned in the jubilee. \v{32}Nevertheless, the cities that belong to the descendants of Levi---that is, the houses in the cities that belong to them---are to belong to the descendants of Levi perpetually as part of their\fnote{The Heb. lacks \fbib{as part of their}} right of redemption. \v{33}If someone from the descendants of Levi redeems the houses in the cities that they own, they are to be returned in the jubilee, because the houses of the cities of the descendants of Levi are to remain their property among the Israelis. \v{34}Also, the open land of their cities is not to be sold, because it is to remain their perpetual inheritance.''
\passage{Treatment of Poor Israelis}

\v{35}``If your relative becomes so poor that he is indebted to you,\fnote{Lit. \fbib{his hand fails with you}} then you are to support him. You are to let him live with you just like the resident alien and the traveler. \v{36}You are not to take interest or profit from him. Instead, you are to fear your God and let your relative live with you. \v{37}You are not to loan him money with interest or sell him your food at a profit. \v{38}I am the \divine{Lord} your God, who brought you out of the land of Egypt to give you the land of Canaan and to be your God.

\v{39}``If your brother with you becomes so poor that he sells himself to you, you are not to make him serve like a bond slave.\fnote{Lit. \fbib{slave of slaves}} \v{40}Instead, he is to serve with you like a hired servant or a traveler who lives with you, until the year of jubilee. \v{41}Then he and his children with him may leave\fnote{Lit. \fbib{may go out from you}} to return to his family and his ancestor's inheritance. \v{42}Since they're my servants whom I've brought out of the land of Egypt, they are not to be sold as slaves. \v{43}You are not to rule over them with harshness. You are to fear your God.''
\passage{Release of Slaves}

\v{44}``As for your male and maid slaves who will be with you, you may buy male and female slaves from among the nations. \v{45}You may also buy from resident aliens who live among you and their families who are with you, whom they fathered in your land. They may become your property. \v{46}You may give them as inherited property to your children\fnote{Lit. \fbib{sons}} after you, to own as properties in perpetuity. You may make bond slaves of them, but no one is to rule over his fellow Israeli with harshness.

\v{47}``If a resident alien or traveler becomes rich,\fnote{Lit. \fbib{his hand overtakes}} but your relative who lives next to him is so poor that he sells himself to that resident alien or traveler among you or to a member of the resident alien's family, \v{48}he has the right to be redeemed after he sells himself. One of his brothers may redeem him. \v{49}His uncle or his uncle's son may redeem him or any blood\fnote{Lit. \fbib{flesh}} relative from his tribe may redeem him. If\fnote{So LXX and Syriac} he becomes rich,\fnote{Lit. \fbib{his hand overtakes}} then he may redeem himself.

\v{50}``He is to bring an accounting to the one who bought him, starting from the year he had sold himself until the year of jubilee. The price of his sale is to correspond to the number of years comparable to the time a hired servant stays with him. \v{51}If there are still many years left, he is to refund the cost\fnote{Or \fbib{price-money}} of his redemption. \v{52}But if only a few years are left until the year of jubilee, he is to bring an accounting of the years that he is to refund for his redemption. \v{53}Like a hired servant, he is to remain with him year after year, but he is not to rule over him with what you see as severity. \v{54}If he isn't redeemed by these, then he is to be set free in the year of jubilee---he and his children\fnote{Lit. \fbib{his sons}} with him--- \v{55}because the Israelis are my servants. They're my servants, since I brought them out of the land of Egypt. I am the \divine{Lord} your God.''
\labelchapt{26}
\passage{Rewards for Obedience}

\chapt{26}
\v{1}``You are not to make worthless idols, images, or pillars for yourselves, nor set up for yourselves carved images to bow down to them in the land, because I am the \divine{Lord} your God.

\v{2}``You are to keep my Sabbath and fear my sanctuary. I am the \divine{Lord}.

\v{3}``If you live\fnote{Lit. \fbib{walk}} by my statutes, obey my commands, and observe them, \v{4}then I'll send\fnote{Lit. \fbib{give}} your rain in its season so that the land will yield its produce and the trees of the field will yield their fruit. \v{5}Threshing will extend to the time of vintage and the vintage will extend to the time of sowing, so that you'll eat your bread to your satisfaction and live securely in your land. \v{6}I'll give peace in the land so that you'll lie down without fear. I'll remove wild\fnote{Lit. \fbib{evil}} beasts from the land, and not even war will come to\fnote{Lit. \fbib{sword won't pass through}} your land. \v{7}Instead, you'll pursue your enemies and they'll die\fnote{Lit. \fbib{fall}} by the sword before you. \v{8}Five of you will chase a hundred, a hundred of you will chase ten thousand, and your enemies will fall by the sword before you.

\v{9}``I'll look after you, ensuring that you'll be fruitful. I'll increase your number\fnote{Lit. \fbib{multiply you}} and keep\fnote{Lit. \fbib{raise} or \fbib{establish}} my covenant with you. \v{10}When you have consumed what was stored of the old, then you'll take out the old and replace it with what's new. \v{11}I'll set up my tent in your midst and I\fnote{Lit. \fbib{my soul}} won't loathe you. \v{12}I'll walk among you. I will be your God, and you'll be my people. \v{13}I am the \divine{Lord} your God, who brought you out of the land of Egypt so that you will no longer be their slaves, since I've broken their oppressive yoke upon you to make you walk upright.''
\passage{Cascading Consequences}

\v{14}``But if you won't listen to me and obey all these commands, \v{15}and if you refuse my statutes, loathe my ordinances, and fail to carry out all of my commands, thereby breaching my covenant, \v{16}then I will certainly do this to you: I'll appoint sudden terror to infect you like tuberculosis and fever. Your eyes will fail and your life will waste away. You'll plant in vain, because your enemies will consume what you plant. \v{17}I'll set my face against you so that you'll be defeated before your enemies. Those who hate you will have dominion over you and you'll keep fleeing even when no one is pursuing you.

\v{18}``If, despite all of this, you still don't listen to me, then I'll punish you seven times more on account of your sins. \v{19}I'll break your mighty pride.\fnote{Lit. \fbib{pride of your strength}} I'll make the heavens to be like iron and the ground like bronze. \v{20}Your strength will be spent in vain, because your land won't yield its produce and the trees of the land won't yield their fruit.

\v{21}``If you live life contrary to me and remain unwilling to listen to me, then I'll add to your wounds seven times more on account of your sins. \v{22}I'll send wild beasts against you from the open country to deprive you of your children, destroy your cattle, and decrease your number\fnote{The Heb. lacks \fbib{your number}} so that your roads become desolate.

\v{23}``If, despite these things, you still won't return to me, but live life contrary to me, \v{24}then I'll certainly oppose\fnote{Lit. \fbib{walk against}} you. I'll take vengeance against you seven fold on account of your sins. \v{25}I'll bring the sword against you to execute the vengeance of my covenant. When you gather in your cities, I'll send a pestilence. As a result, you'll be delivered into the control of your enemies. \v{26}When I destroy the source of your bread, ten women will bake bread in one oven. Then they'll return back your bread by weight. You'll eat but won't be satisfied.

\v{27}``If, after all of this time, you don't listen to me, but instead live life contrary to me, \v{28}I'll oppose\fnote{Lit. \fbib{walk against}} you with vicious rage. Indeed, I myself will punish you seven fold on account of your sins. \v{29}At that time, you'll eat the flesh of your sons and you'll eat the flesh of your daughters. \v{30}I'll destroy your high places and cut down your sun pillars. Then I'll cast your dead bodies on top of the bodies of your idols. I'll loathe you. \v{31}I'll lay your cities to waste and destroy your sanctuaries so I don't have to smell the scent of your soothing odors. \v{32}I'll make the land so desolate that your enemies who live in it will be astonished.''
\passage{Captivity among the Nations}

\v{33}``I'll scatter you among the nations and draw the sword after you so that your land becomes desolate and your towns become ruins. \v{34}Then the land will finally be pleased with its Sabbaths as long as it lies desolate while you are in the land of your enemies. At that time, the land will rest and take its Sabbaths. \v{35}As long as it lies desolate, it will have rest that it will not have had during your Sabbaths when you were living in it.

\v{36}``As for the remnants among you, I'll bring despair in their hearts in the land of their enemies so that even the sound of a blown leaf will chase them and they flee as though pursued by the sword and fall when no one is pursuing. \v{37}They'll stumble over each other as though fleeing before the sword, even though no one is pursuing.

``You won't have power to resist your enemies. \v{38}You'll perish among the nations and the land of your enemies will consume you. \v{39}The remnants among you will waste away in the land of your enemies due to their iniquity. Indeed, they'll also waste away on account of the iniquities of their ancestors with them.''
\passage{Return from Captivity}

\v{40}``Nevertheless, when they confess their iniquity, the iniquity of their ancestors, and their unfaithfulness by which they acted unfaithfully against me by living life contrary to me--- \v{41}causing me to oppose them and take them to the land of their enemies so that the uncircumcised foreskin of their hearts can be humbled and so that they accept the punishment of their iniquity--- \v{42}then I'll remember my covenant with Jacob, my covenant with Isaac, and my covenant with Abraham. I'll also remember the land. \v{43}They will leave the land so it can rest while it lies desolate without them. That's when they'll receive the punishment of their iniquity, because indeed they will have rejected my ordinances and despised my statutes. \v{44}Yet, despite all of these things, when they're in the land of their enemies, I won't reject or despise them so as to completely destroy them and by doing so violate my covenant with them, because I am the \divine{Lord} their God. \v{45}Instead, on account of them, I'll remember my covenant with their ancestors when I brought them out of the land of Egypt right before the eyes of the nations, so that I could be their God. I am the \divine{Lord}.''

\v{46}These are the statutes, ordinances, and laws that the \divine{Lord} made between himself and the Israelis on Mount Sinai, as recorded by the hand of Moses.
\labelchapt{27}
\passage{Special Offerings}

\chapt{27}
\v{1}The \divine{Lord} told Moses, \v{2}``Tell the Israelis that when a person\fnote{Lit. \fbib{man}, and so throughout the chapter} makes a special vow based on the appropriate value of people who belong to the \divine{Lord}, \v{3}if your valuation of the vow\fnote{The Heb. lacks \fbib{of the vow}} is for a male from 20 to 60 years old, the valuation is to be 50 shekels of silver, according to the shekel of the sanctuary. \v{4}If she is a female from 20 to 60 years old, then your valuation is to be 30 shekels, according to the shekel of the sanctuary. \v{5}If a person\fnote{Lit. \fbib{son of}} is from five to 20 years, then your valuation for a male is to be 20 shekels and for a female ten shekels. \v{6}If a person is from one month to five years old, then your valuation for a male is to be five shekels of silver, and for a female your valuation is to be three shekels of silver. \v{7}If a person is 60 or more years old, then your valuation for a male is to be fifteen shekels and for a female ten shekels. \v{8}But if he is too poor to be valuated, then cause him to stand before the priest and let the priest set a value on him according to the ability\fnote{Lit. \fbib{according to what the hand can reach}} of the one making the vow.

\v{9}``If it's an animal from which they make an offering to the \divine{Lord}, everything that he gives to the \divine{Lord} from it will be holy. \v{10}He is not to substitute it or exchange it---the good with the bad or the bad with the good. If he ever makes an exchange of an animal for an animal, then it and what's being exchanged is holy. \v{11}If any animal is unclean, which cannot be brought to the \divine{Lord} as an offering, make the animal stand in the presence of the priest, \v{12}then the priest will evaluate it as to whether it is good or bad. According to your---that is, the priest's---valuation, so it is to be. \v{13}If a kinsman redeemer decides to redeem it, then he is to add a fifth to your valuation.''
\passage{Gifts of Residences}

\v{14}``If a person consecrates his house to be holy to the \divine{Lord}, then the priest is to set a value for it as to its worth, whether good or bad. As the priest sets value on it, so it will stand. \v{15}And if he that consecrated it wishes to redeem his house, he is to add one fifth to your valuation, after which it is to belong to him.

\v{16}``If a person consecrates to the \divine{Lord} a portion of the field from his inheritance, then your valuation is to be based on its capacity for yielding a harvest.\fnote{Lit. \fbib{valuation according to seed for sowing}} Each omer\fnote{I.e. about two quarts} of barley is to be valued at 50 shekels of silver. \v{17}If he consecrates his field in the year of jubilee, it is to be based on your valuation. \v{18}If he consecrates his field after the jubilee, then the priest is to account to him the silver according to the years that remain until the year of jubilee, with a deduction corresponding to your valuation.

\v{19}``If the one who consecrated the field intends to redeem it, then he is to add one fifth of your valuation to it in silver, then it is to be established as his. \v{20}But if he won't redeem the field, but instead sells it to another person,\fnote{Lit. \fbib{man}} then it is not to be redeemed anymore. \v{21}When the field is released in the jubilee, it will be holy to the \divine{Lord}. As a field that's devoted, it is to belong to the priest as his inheritance. \v{22}If he consecrates a field that he had bought and that isn't part of his inheritance, \v{23}then the priest is to account to him the evaluated worth until the year of jubilee. Then he is to give the amount of valuation on that day as a holy gift to the \divine{Lord}. \v{24}During the year of jubilee, the field is to be returned by the one who originally sold it---that is, to the owner of the land. \v{25}Every valuation is to be according to the shekel of the sanctuary, evaluated at 20 gerahs to the shekel.

\v{26}``No person is to consecrate the firstborn, because the firstborn of the animals already belongs to the \divine{Lord}. Whether ox or goat, it belongs to the \divine{Lord}. \v{27}If it's an unclean animal, then he is to ransom it according to your valuation, adding a fifth to it. If it's not redeemed, then it is to be sold according to your valuation. \v{28}However, any devoted thing that a person consecrates to the \divine{Lord} from what he owns---whether man, animals, or inherited fields---is not to be sold or redeemed. Any devoted thing is most sacred. It belongs to the \divine{Lord}. \v{29}But anyone who is completely devoted from among human beings is not to be ransomed. He is certainly to be put to death.

\v{30}``Any tithes of the land---from grain grown on the land or from fruit grown on the trees---belong to the \divine{Lord}. They are sacred to the \divine{Lord}. \v{31}But if a person wishes to redeem his tithe, he is to add a fifth to it. \v{32}All the tithes from cattle and flocks that pass under the measuring rod are sacred to the \divine{Lord}. \v{33}He is not to examine it to see if it's good or bad or even exchange it. If he does exchange it, what has been exchanged as well as its substitute\fnote{The Heb. lacks \fbib{substitute}} is sacred. It is not to be redeemed.''

\v{34}These are the commands that the \divine{Lord} commanded Moses to deliver\fnote{The Heb. lacks \fbib{deliver}} to the Israelis on Mount Sinai.

\bookheader{Numbers}
\labelbook{Num}

\bookpretitle{The Fourth Book of the Law called}
\booktitle{Numbers}

\labelchapt{1}
\passage{A Census of Israel is Taken}
\passageinfo{(2 Samuel 24:1-9; 1 Chronicles 21:1-6)}

\chapt{1}
\v{1}In the Sinai desert, the \divine{Lord} spoke to Moses inside the Tent of Meeting on the first day of the second month of the second year after they had left the land of Egypt. He said, \v{2}``Take a census of the entire\fnote{Lit. \fbib{census of the head of all the}} Israeli community, numbering them by their tribes\fnote{Or \fbib{families}; and so throughout the book} and by ancestral houses. List the names of every male one-by-one, \v{3}from 20 years and upward. You and Aaron are to register everyone in Israel who is able to go to war, company by company. \v{4}One man from each tribe is to accompany you, each man being the leader of his ancestral house.

\v{5}``Here is a list of names of the men who are to assist\fnote{Lit. \fbib{to stand with}} you:

``From Reuben: Shedeur's son Elizur. \v{6}From Simeon: Zurishaddai's son Shelumiel. \v{7}From Judah: Amminadab's son Nahshon. \v{8}From Issachar: Zuar's son Nethanel. \v{9}From Zebulun: Helon's son Eliab.

\v{10}``From Joseph's descendants through Ephraim: Ammihud's son Elishama. From Manasseh: Pedahzur's son Gamaliel. \v{11}From Benjamin: Gideoni's son Abidan. \v{12}From Dan: Ammishaddai's son Ahiezer. \v{13}From Asher: Ochran's son Pagiel. \v{14}From Gad: Deuel's son Eliasaph. \v{15}From Naphtali: Enan's son Ahira.''

\v{16}These men were appointed from within their communities, since they were leaders of their ancestral houses and heads of the tribes of Israel.

\v{17}Moses and Aaron gathered these men who had been mentioned by name. \v{18}They assembled the entire community together during the second month. Then they recorded their ancestries,\fnote{Or \fbib{genealogies}; and so throughout the book} according to their tribes and ancestral houses, as well as the names of the men\fnote{Or \fbib{sons of Israel}; and so throughout the book} 20 years old and above individually,\fnote{Lit. \fbib{according to their heads;} and so throughout the book} \v{19}just as the \divine{Lord} had commanded Moses. He numbered them in the Sinai desert.
\passage{Numbering the Tribes}

\v{20}The genealogies of the descendants of Reuben, the firstborn of Israel, were recorded individually, according to their tribes and ancestral houses, as were the names of all the men 20 years and above who could serve in the army. \v{21}Those registered with the tribe of Reuben numbered 46,500.

\v{22}The genealogies of Simeon's descendants were recorded individually, according to their tribes and ancestral houses, as were the names of all the men 20 years and above who could serve in the army. \v{23}Those registered with the tribe of Simeon numbered 59,300.

\v{24}The genealogies of Gad's descendants were recorded individually, according to their tribes and ancestral houses, as were the names of all the men 20 years and above who could serve in the army. \v{25}Those registered with the tribe of Gad numbered 45,650.

\v{26}The genealogies of Judah's descendants were recorded individually, according to their tribes and ancestral houses, as were the names of all the men 20 years and above who could serve in the army. \v{27}Those registered with the tribe of Judah numbered 74,600.

\v{28}The genealogies of Issachar's descendants were recorded individually, according to their tribes and ancestral houses, as were the names of all the men 20 years and above who could serve in the army. \v{29}Those registered with the tribe of Issachar numbered 54,400.

\v{30}The genealogies of Zebulun's descendants were recorded individually, according to their tribes and ancestral houses, as were the names of all the men 20 years and above who could serve in the army. \v{31}Those registered with the tribe of Zebulun numbered 57,400.

\v{32}The genealogies of Joseph's descendants were recorded individually, according to their tribes and ancestral houses, as were the names of all the men 20 years and above who could serve in the army. \v{33}Those registered with the tribe of Joseph numbered 40,500.

\v{34}The genealogies of Manasseh's descendants were recorded individually, according to their tribes and ancestral houses, as were the names of all the men 20 years and above who could serve in the army. \v{35}Those registered with the tribe of Manasseh numbered 32,200.

\v{36}The genealogies of Benjamin's descendants were recorded individually, according to their tribes and ancestral houses, as were the names of all the men 20 years and above who could serve in the army. \v{37}Those registered with the tribe of Benjamin numbered 35,400.

\v{38}The genealogies of Dan's descendants were recorded individually, according to their tribes and ancestral houses, as were the names of all the men 20 years and above who could serve in the army. \v{39}Those registered with the tribe of Dan numbered 62,700.

\v{40}The genealogies of Asher's descendants were recorded individually, according to their tribes and ancestral houses, as were the names of all the men 20 years and above who could serve in the army. \v{41}Those registered with the tribe of Asher numbered 41,500.

\v{42}The genealogies of Naphtali's descendants were recorded individually, according to their tribes and ancestral houses, as were the names of all the men 20 years and above who could serve in the army. \v{43}Those registered with the family of Naphtali numbered 53,400.

\v{44}These individuals were the ones whom Moses and Aaron registered from the twelve leaders of Israel, each person from his ancestral house. \v{45}Everyone was numbered from the descendants of Israel, from their ancestral houses, from all the men who were 20 years and above and who could serve in the army. \v{46}The total of all those who were numbered was 603,550.
\passage{Exemption of the Tribe of Levi from the Census}

\v{47}The descendants of Levi were not counted according to their ancestral houses \v{48}because the \divine{Lord} had ordered Moses: \v{49}``Be sure not to number or count the tribe of Levi with the rest of the Israelis. \v{50}Instead, appoint the descendants of Levi over the Tent of Meeting, all the vessels, and everything in it. They are to carry the tent and all the vessels in it. They are to attend to it and camp around it. \v{51}Whenever the tent is ready for travel, the descendants of Levi are to take it down. When it's time to encamp, the descendants of Levi are to set it up. Any unauthorized person\fnote{Lit. \fbib{stranger}} who approaches it is to be executed. \v{52}Then the Israelis are to encamp around the tent,\fnote{The Heb. lacks \fbib{around the tent}} arranged according to their company and the standard of their army. \v{53}But the descendants of Levi are to encamp on all sides of the Tent of Meeting so that divine wrath won't fall on the congregation of Israel.\fnote{Lit. \fbib{sons of Israel}} The descendants of Levi are to take care of the Tent of Meeting.''

\v{54}The Israelis observed everything that the \divine{Lord} had commanded Moses, doing exactly what they were told.
\labelchapt{2}
\passage{Encampment Orders}

\chapt{2}
\v{1}Later, the \divine{Lord} told Moses and Aaron, \v{2}``Every single Israeli\fnote{Lit. \fbib{Each man of the Israelis}} is to encamp beneath his standard with the emblem of his ancestral house. The Israelis are to encamp in front of and surrounding the Tent of Meeting.''
\passage{Eastern Encampment Order}

\v{3}``The encampment of Judah is to settle east toward the sunrise\fnote{Lit. \fbib{east}} under their standard. The leader of Judah is to be Amminadab's son Nahshon. \v{4}Those in his division number 74,600.\fnote{Cf. Num 1:27}

\v{5}``The tribe of Issachar is to encamp beside Judah.\fnote{Lit. \fbib{him}} The leader of Issachar is to be Zuar's son Nethanel. \v{6}Those in his division number 54,400.\fnote{Cf. Num 1:29}

\v{7}``Next is to be\fnote{Lit. \fbib{Then}} the tribe of Zebulun. The leader of Zebulun is to be Helon's son Eliab. \v{8}Those in his division number 57,400. \v{9}All those numbered by division in the camp of Judah total 186,400. They are to be the first to travel.''
\passage{Southern Encampment Order}

\v{10}``Toward the south is to be the division of the camp of Reuben under their standard. The leader of Reuben is to be Shedeur's son Elizur. \v{11}Those in his division number 46,500.

\v{12}``The tribe of Simeon is to camp beside Reuben.\fnote{Lit. \fbib{him}} The leader of Simeon is to be Zurishaddai's son Shelumiel. \v{13}Those in his division number 59,300.

\v{14}``Next is to be\fnote{Lit. \fbib{Then}} the tribe of Gad. The leader of Gad is to be Deuel's son Eliasaph. \v{15}Those in his division number 45,650. \v{16}All those numbered by division in the camp of Reuben total 151,450. They are to be the second to travel.''
\passage{Tribe at the Center}

\v{17}``Then the Tent of Meeting is to travel with the camp of the descendants of Levi in the middle of the camps. They are to travel just as they have camped, each as designated\fnote{Lit. \fbib{each upon his hand}} under his standard.''
\passage{Western Encampment Order}

\v{18}``Toward the west\fnote{Lit. \fbib{the sea}} is to be the division of the camp of Ephraim under their standard. The leader of Ephraim is to be Ammihud's son Elishama. \v{19}Those in his division number 40,500.

\v{20}``The tribe of Manasseh is to encamp beside them.\fnote{Lit. \fbib{him}} The leader of Manasseh is to be Pedahzur's son Gamaliel. \v{21}Those in his division number 32,200.

\v{22}``Next is to be\fnote{Lit. \fbib{Then}} the tribe of Benjamin. The leader of Benjamin is to be Gideoni's son Abidan. \v{23}Those in his division number 35,400. \v{24}All those numbered by division in the camp Ephraim total 108,100. They are to be the third to travel.''
\passage{Northern Encampment Order}

\v{25}``Toward the north is to be the division of the camp of Dan under their standard. The leader of Dan is to be Ammishaddai's son Ahiezer. \v{26}Those in his division number 62,700.

\v{27}``The tribe of Asher is to encamp beside them.\fnote{Lit. \fbib{him}} The leader of Asher is to be Ochran's son Pagiel. \v{28}Those in his division number 41,500.

\v{29}``Next is to be\fnote{Lit. \fbib{Then}} the tribe of Naphtali. The leader of Naphtali is to be Enan's son Ahira. \v{30}Those in his division number 53,400. \v{31}All those numbered by division in the camp of Dan total 157,600. They are to be the last to travel under their standards.''
\passage{Summary of the Encampment}

\v{32}Here is a summary of the census of the Israelis according to the tribes of their ancestral houses: All the divisions in the camps numbered 603,550, \v{33}but the descendants of Levi were not numbered along with the other Israelis, just as the \divine{Lord} had commanded Moses. \v{34}So the Israelis did everything just as the \divine{Lord} had commanded Moses; that is, they encamped under their standard as each person traveled with his own tribe and ancestral house.
\labelchapt{3}
\passage{Aaron's Descendants}
\passageinfo{(Leviticus 10:1-7)}

\chapt{3}
\v{1}This is a record of the genealogies\fnote{Lit. \fbib{generations}} of Aaron and Moses current as of\fnote{The Heb. lacks \fbib{current as of}} the day on which the \divine{Lord} addressed Moses on Mount Sinai. \v{2}The\fnote{Lit. \fbib{These are the names of the}} sons of Aaron were Nadab the first-born, Abihu, Eleazar, and Ithamar \v{3}who\fnote{Lit. \fbib{These are the names of the sons of Aaron who}} were anointed priests and whom he consecrated\fnote{Lit. \fbib{filled their hands}} as priests. \v{4}Nadab and Abihu died in the \divine{Lord}'s presence when they offered unauthorized\fnote{Lit. \fbib{strange}} fire before him\fnote{Lit. \fbib{the \divine{Lord}}} in the Sinai wilderness. Since they didn't have their own children, Eleazar and Ithamar ministered as priests under the authority of\fnote{Lit. \fbib{priest before}} Aaron their father.
\passage{Appointment of the Descendants of Levi as Priests}

\v{5}The \divine{Lord} told Moses, \v{6}``Bring the tribe of Levi near and present them to Aaron the priest so they may serve him. \v{7}They are to take care of his needs and the needs of the whole congregation at the Tent of Meeting by performing duties\fnote{Or \fbib{work}} at the tent. \v{8}They are to take charge of the utensils at the Tent of Meeting and meet the needs of the Israelis by performing duties on behalf of the tent. \v{9}Assign\fnote{Lit. \fbib{Give}} the descendants of Levi to Aaron and his sons from among the Israelis. \v{10}Appoint Aaron and his sons so that they are to take responsibility for their priesthood. Any unauthorized\fnote{Or \fbib{undesignated}} person who approaches it is to be put to death.''
\passage{The Descendants of Levi as Substitutes for the First-born}

\v{11}Later, the \divine{Lord} told Moses, \v{12}``I'm taking the descendants of Levi for myself from among the Israelis in place of every first-born who opens the womb.\fnote{Lit. \fbib{womb from among the Israelis}} The descendants of Levi belong to me \v{13}because all the first-born belong to me. When\fnote{Or \fbib{the day}} I destroyed all the firstborn in the land of Egypt, I consecrated all the first-born in Israel for myself---from human beings to livestock. They belong to me, since\fnote{The Heb. lacks \fbib{since}} I am the \divine{Lord}.''
\passage{Numbering the Descendants of Levi}

\v{14}The \divine{Lord} also told Moses in the Sinai wilderness, \v{15}``Number the descendants of Levi according to their ancestral houses and tribes, numbering every male from a month old and above.''

\v{16}So Moses numbered them according to the instruction\fnote{Lit. \fbib{mouth}} of the \divine{Lord}, as he had been commanded. \v{17}These are Levi's descendants by name: Gershon, Kohath, and Merari. \v{18}These are names of Gershon's descendants according to their families: Libni and Shimei. \v{19}These are the names of Kohath's descendants according to their families: Amram, Izhar, Hebron, and Uzziel. \v{20}Merari's descendants according to their families were Mahli and Mushi. These are the families of the descendants of Levi according to their ancestral house.
\passage{The Descendants of Gershon}

\v{21}The families of Libni and Shimei were descendants of Gershon. As families of the descendants of Gershon, \v{22}all the males a month old and above numbered 7,500. \v{23}The families of the descendants of Gershon encamped behind the tent toward the west.\fnote{Lit. \fbib{sea}} \v{24}The leader of the tribe and family of Gershon was Lael's son Eliasaph. \v{25}The duties of the descendants of Gershon at the Tent of Meeting pertained to the tent, the tent covering, the curtain\fnote{Or \fbib{screen}} to the entrance to the Tent of Meeting, \v{26}the hangings at the courtyard, the curtain at the entrance of the courtyard that surrounded the tent, the altar, and all of the tent cords in use.
\passage{The Descendants of Kohath and Their Duties}

\v{27}The families of Amram, Izhar, Hebron, and Uzziel were descendants of Kohath. As families of the descendants of Kohath, \v{28}all the males a month old and above numbered 8,600.\fnote{So MT; LXX reads \fbib{8,300}} They were tasked to the care of the sanctuary. \v{29}The descendants of Kohath encamped beside the tent toward the south. \v{30}The leader of the tribe and family of Kohath was Uzziel's son Elizaphan. \v{31}Their duties pertained to the ark, the table, the lamp stand, the altars, the utensils of the sanctuary with which they ministered, and all the curtains in use. \v{32}The chief of all the leaders of the descendants of Levi was Aaron the priest's son Eleazar. He was assigned to oversee those who were in charge of the services of the sanctuary.
\passage{The Descendants of Merari and Their Duties}

\v{33}The families of Mahli and Mushi were descendants of Merari. As families of Merari, \v{34}all the males a month old and above numbered 6,200. \v{35}The leader of the tribe and family of Merari was Abihail's son Zuriel. The descendants of Merari encamped beside the tent toward the north. \v{36}The duties of the caretakers from the descendants of Merari included the boards of the tent, its bars, crossbars, sockets, all its utensils for their services, \v{37}the pillars around the courtyard, their sockets, pegs, and tent cords.
\passage{The Encampment of Moses and Aaron}

\v{38}In front of the tent and east of the Tent of Meeting, Moses, Aaron, and Aaron's\fnote{Lit \fbib{his}} sons encamped facing the east. They were tasked to perform the duties of the sanctuary and care for the needs of the Israelis. Any unauthorized\fnote{Or \fbib{undesignated}} person who approached was to be executed. \v{39}As the \divine{Lord} had instructed, everyone counted by Moses and Aaron from the descendants of Levi, according to their tribe, all males from a month old and above numbered 22,000.
\passage{Numbering Israel's First-born}

\v{40}Later the \divine{Lord} instructed Moses: ``Number all the first-born males of Israel from a month old and above and list their names. \v{41}Separate\fnote{Lit. \fbib{Take}} the descendants of Levi for me---since\fnote{The Heb. lacks \fbib{since}} I am the \divine{Lord}---in place of all the first-born sons of Israel. Also separate\fnote{The Heb. lacks \fbib{separate}} the livestock of the descendants of Levi in place of all the firstborn of the livestock of Israel.'' \v{42}So Moses numbered all the firstborn from the sons of Israel just as the \divine{Lord} commanded him. \v{43}All the first-born males according to the list of their names from a month old and above numbered 22,273.
\passage{Creation of the Levite Ministry}

\v{44}Then the \divine{Lord} told Moses, \v{45}``Separate the descendants of Levi in place of all the firstborn sons of Israel and the livestock of the descendants of Levi in place of their livestock. The descendants of Levi belong to me, since\fnote{The Heb. lacks \fbib{since}} I am the \divine{Lord}. \v{46}You are to pay a ransom for the 273 first-born Israelis who exceed the census number of the descendants of Levi, \v{47}so collect five shekels for each individual,\fnote{Lit. \fbib{head}} denominated in shekels of the sanctuary, that is, the shekel that weighs 20 gerahs.\fnote{I.e., a unit of weight measurement equal to about 16 barley grains; about 0.025 ounces or 0.5 grams; cf. Exod 30:13; Num 18:16} \v{48}Then give the money meant for ransom of their excess to Aaron and his sons.''

\v{49}So Moses took the ransom money to account for the difference in the total number\fnote{Lit. \fbib{the excess}} of those redeemed by the descendants of Levi. \v{50}From the firstborn of the Israelis, Moses took money amounting to 1,365 shekels according to the shekel of the sanctuary. \v{51}Moses gave the ransom money to Aaron and his sons according to the \divine{Lord}'s instructions, just as the \divine{Lord} had commanded Moses.
\labelchapt{4}
\passage{The Duties of the Descendants of Kohath}

\chapt{4}
\v{1}The \divine{Lord} told Moses and Aaron, \v{2}``Take a census\fnote{Lit. \fbib{Lift the head}} of the descendants of Kohath from among the descendants of Levi according to their tribes and ancestral houses \v{3}from 30 years and older through the age of 50 years, from everyone who can enter the service to perform work at the Tent of Meeting.

\v{4}``Here's what the descendants of Kohath are to do regarding the Tent of Meeting and what's inside the Most Holy Place: \v{5}When the camp is about to travel, Aaron and his sons are to come and take down the veil of the curtain and cover the Ark of the Testimony with it. \v{6}They are to set a leather-dyed\fnote{Or \fbib{porpoise}; or \fbib{fine leather}} skin covering over it, cover it with a pure blue cloth, and then insert its poles.

\v{7}``They are to spread a blue cloth over the table of the Presence and on top of it the dishes, pans, bowls, pitchers for drink offerings,\fnote{Or \fbib{libation}} and the bread of presence are to be on it continually. \v{8}They are to spread over them a scarlet cloth and a leather-dyed\fnote{Or \fbib{porpoise}; or \fbib{fine leather}} skin covering and then insert its poles.

\v{9}``They are to take a blue cloth and cover the lamp stand for the light with its lamp, lamp-snuffers, censer, and all the utensils for its oil with which they minister. \v{10}Then they are to put them with all the other\fnote{The Heb. lacks \fbib{other}} utensils on the leather-dyed\fnote{Or \fbib{porpoise}; or \fbib{fine leather}} skin covering and set them on the beams for transport.\fnote{Or \fbib{poles for carrying stuff}}

\v{11}``On the golden altar, they are to spread a blue cloth, cover it with a leather-dyed\fnote{Or \fbib{porpoise}; or \fbib{fine leather}} skin covering, and then insert its poles. \v{12}Then they are to take all the utensils for service with which they minister at the sanctuary, set them on the blue cloth, cover them with the leather-dyed\fnote{Or \fbib{porpoise}; or \fbib{fine leather}} skin covering, and then set them on the beams for transport. \v{13}They are also to remove the ashes on the altar and spread a purple cloth over it. \v{14}Then they are to put all the instruments with which they minister there---trays, forks, shovels, bowls, and all the utensils of the altar. They are to spread over it a leather-dyed\fnote{Or \fbib{porpoise}; or \fbib{fine leather}} skin covering and then insert its poles.

\v{15}``When Aaron and his sons have finished covering the sanctuary and all the utensils of the sanctuary, and the camp is about to travel, then the descendants of Kohath are to come and carry them, but they are not to touch the most sacred objects, so they won't die. These are the duties of the descendants of Kohath at the Tent of Meeting.''

\v{16}``Now the duty of Eleazar, the son of Aaron the priest is to maintain the oil for the light, the spiced incense, the daily offerings, and the oil for anointing, to carry out all the duties of the tent and the sanctuary, and to maintain\fnote{The Heb. lacks \fbib{to maintain}} its utensils.''

\passage{Protecting the Descendants of Kohath}

\v{17}Then the \divine{Lord} told Moses and Aaron, \v{18}``You are not to eliminate the tribe of the families of the descendants of Kohath from the descendants of Levi. \v{19}But do this for them so that they may live and not die when they approach the Most Holy Place: Aaron and his sons are to go in and set specific responsibilities for each of them to carry out.\fnote{Lit. \fbib{responsibilities according to his service and to his burden}} \v{20}But they are not to go in to see the sanctuary as it is being covered,\fnote{I.e. in preparation for travel} so they won't die.''
\passage{Eleazar's Duties}

\v{21}Then the \divine{Lord} told Moses, \v{22}``Take a census\fnote{Lit. \fbib{Lift the head}} of the descendants of Gershon according to their ancestral house and tribes. \v{23}Count their number from between 30 to 50 years old, including everyone who can enter the service to perform work at the Tent of Meeting.''
\passage{Gershonite Responsibilities}

\v{24}``These are the responsibilities that the descendants of Gershon are to have: \v{25}They are to carry the curtain of the tent, the covering of the Tent of Meeting, the dyed leather covering that goes over it, the curtain for the entrance to the Tent of Meeting, \v{26}the hangings for the courtyard, the curtain for the entrance to the gate of the courtyard that surrounds the tent, the altar, the ropes, all the service utensils, and everything made for them. This is to be their service area. \v{27}The descendants of Gershon are to carry out the instructions of Aaron and his sons. You are to assign them their responsibilities to carry out. \v{28}This is the work of the tribes of Gershon at the Tent of Meeting---their duties under the supervision of\fnote{Lit. \fbib{the hand of}} Ithamar, the son of Aaron the priest.
\passage{Merarite Responsibilities}

\v{29}``For the descendants of Merari, number them according to their tribes and ancestral houses \v{30}from 30 to 50 years old as you count them, including everyone who can enter service and perform work at the Tent of Meeting. \v{31}This is to be their area of responsibility to carry out with respect to their service at the Tent of Meeting: the board of the tent, its bars, its crossbars, its sockets, \v{32}the pillars around the courtyard, their sockets, their pegs, their ropes, and all the utensils for all their services. Assign the utensils by name to each person whose responsibility it will be to carry them. \v{33}This is the work of the tribes of the descendants of Merari with reference to their service at the Tent of Meeting under the supervision of Aaron the priest's son Ithamar.''
\passage{Responsibilities are Assigned}

\v{34}Moses, Aaron, and the congregational leaders numbered the descendants of Kohath according to their tribes and ancestral houses \v{35}from 30 to 50 years old---that is, everyone who entered the service to perform work at the Tent of Meeting. \v{36}The total according to their tribe numbered 2,750 \v{37}from the tribe of the descendants of Kohath, everyone who would be serving at the Tent of Meeting, whom Moses and Aaron numbered according to what the \divine{Lord} had said, under the supervision of Moses.

\v{38}The tribes and the ancestral houses of the descendants of Gershon were numbered \v{39}from 30 to 50 years old; that is, everyone who entered the service to perform work at the Tent of Meeting. \v{40}The total according to their tribes and ancestral house numbered 2,630 \v{41}from the tribes of the descendants of Gershon, everyone who would be serving at the Tent of Meeting, whom Moses and Aaron numbered according to what the \divine{Lord} had said.

\v{42}The tribes and ancestral house of Merari were numbered \v{43}from 30 to 50 years old; that is, everyone who entered the service to perform work at the Tent of Meeting.\v{44}The total according to their tribes numbered 3,200 \v{45}from the tribes of the descendants of Merari, whom Moses and Aaron numbered according to what the \divine{Lord} had said, under the supervision of Moses.

\v{46}The total of those who were numbered from the descendants of Levi by Moses and Aaron; that is, from the leaders of Israel counted according to their tribes and ancestral houses \v{47}from 30 to 50 years old, who entered the service for work at the Tent of Meeting \v{48}was 8,580. \v{49}They were numbered under the supervision of Moses according to what the \divine{Lord} had said. Each person was assigned a responsibility to carry out, just as the \divine{Lord} had commanded Moses.
\labelchapt{5}
\passage{On Unclean Persons}

\chapt{5}
\v{1}The \divine{Lord} told Moses, \v{2}``Command the Israelis to send outside the encampment every leper, everyone who has a discharge, and whoever is ritually defiled by contact with a corpse.\fnote{Lit. \fbib{soul}} \v{3}Whether male or female, send them outside the camp so that they won't defile their camp, because I live among them.'' \v{4}So the Israelis sent them outside the camp. The Israelis did just what the \divine{Lord} had told Moses.
\passage{On Restitution for Offenses}

\v{5}The \divine{Lord} told Moses, \v{6}``Instruct the Israelis that whenever a man or woman does something contained in the list\fnote{The Heb. lacks \fbib{something contained in the list}} of the sins of man, thereby acting treacherously against the \divine{Lord}, then that person stands guilty. \v{7}He\fnote{Lit. \fbib{they}} is to confess the sin that he had committed, pay its full compensation, add one fifth to it, and give the compensation to whomever he offended. \v{8}But if the person has no related redeemer to whom compensation may be made, the payment is to be brought to the \divine{Lord} and given to the priest, in addition to a ram for atonement with which he is to be atoned. \v{9}Every offering from all the most sacred things of the Israelis that they bring to the priest is to belong to him. \v{10}Furthermore, everyone's sacred things belong to him, as well as whatever a person gives to the priest.''
\passage{The Test for Marital Unfaithfulness}

\v{11}Then the \divine{Lord} told Moses, \v{12}``Instruct the Israelis what to do if a man's wife turns astray so that she unfaithfully acts against him, \v{13}a man has sexual relations\fnote{Lit. \fbib{lies down with her}} with her and she conceals it from her husband,\fnote{Or \fbib{man}} keeping it secret although she has defiled herself with there being no witnesses against her, but she was caught anyway. \v{14}If an attitude of jealousy overcomes him so that he becomes jealous at his wife when she is defiled, or if an attitude of jealousy overcomes him and he becomes jealous of his wife even though she isn't defiled, \v{15}then that man is to bring his wife to the priest along with an offering for her consisting of a tenth of an ephah\fnote{I.e., an ephah was equal to from \footfract{2}{3} to \footfract{3}{4} of a bushel} of barley flour. He is not to pour oil or set frankincense over it, because it's to be a jealousy offering, a memorial offering that will serve as a reminder of iniquity. \v{16}Then the priest is to bring it and make her stand in the \divine{Lord}'s presence. \v{17}The priest is to put some holy water into an earthen vessel, take some dust from the floor of the tent, and put it into the water. \v{18}The priest is to have the woman stand in the \divine{Lord}'s presence, uncover her head,\fnote{Lit. \fbib{head of the woman}} and put the grain offering as a memorial, a reminder of jealousy, into her hands. The priest is also to have in his hand the contaminated\fnote{Lit. \fbib{bitter}, and so throughout the chapter} water that carries a curse.

\v{19}``The priest is to administer this oath to the woman: `If indeed another man didn't have sexual relations\fnote{Or \fbib{lie with a man}} with you and you didn't become unfaithful to your husband,\fnote{Or \fbib{man}} then may you be free from these waters that bring a curse. \v{20}But if you have become unfaithful to your husband and have become defiled because a man who isn't your husband has had sexual relations with you{\ldots}' \v{21}then the priest is to have the woman commit to an oath by saying to the woman, `May the \divine{Lord} make you a curse and a curse among your people. When the \divine{Lord} makes your thigh waste away and your abdomen swell \v{22}and this water that brings a curse enters your abdomen, making it swell and your thigh waste away.'

``Then the woman is to say `Amen.'

\v{23}``Then the priest is to write all of these words in a document and wipe it off with the contaminated water. \v{24}The woman is to drink the bitter water that brings a curse and the water that brings a curse is to be considered contaminated. \v{25}The priest is to take the offering of jealousy from the woman's hand, wave the offering in the \divine{Lord}'s presence, and have her approach the altar. \v{26}The priest is to take a handful of grain from the memorial and offer a sacrifice on the altar, after which he is to have the woman drink the water. \v{27}When he has had her drink the water, if she was defiled and had acted unfaithfully toward her husband, then the contaminated water that brings a curse will enter her and infect her, causing her abdomen to swell and her thigh to waste away. Then she is to be a cursed woman among her people. \v{28}But if the woman isn't defiled, then she is to be freed and will be able to bear children.\fnote{Lit. \fbib{and sow seed}} \v{29}This is the law in cases of jealousy when a woman defiles herself while under her husband's authority: \v{30}When a man becomes under the control of an attitude\fnote{Lit. \fbib{spirit}} of jealousy regarding his wife, he is to present her to the Lord, and the priest is to apply this entire statute to her. \v{31}The husband\fnote{Or \fbib{man}} will be free from guilt, but the wife is to bear the punishment of her iniquity.''
\labelchapt{6}
\passage{Nazirites}

\chapt{6}
\v{1}Then the \divine{Lord} told Moses, \v{2}``Tell the Israelis that a man or woman who commits to the vow of the Nazirite, is to be separated to the \divine{Lord}, \v{3}then is to remain separate from wine and strong drink. He is not to drink vinegar or strong drink made from wine. He is not to drink grape juice or eat grapes, whether fresh or dried. \v{4}During the entire time of his dedication, he is not to eat any product from the grapevine, from the seed to the skin. \v{5}During the entire time of his dedication, he is not to allow a razor to pass over his head until the days of his holy consecration to the \divine{Lord} have been fulfilled. He is to let the locks on his head grow long.

\v{6}``During the entire time of his dedication, he is not to come near a dead body.\fnote{Lit. \fbib{soul}} \v{7}He is not to defile himself on account of his father, mother, brother, and sister when they die, because the crown of his consecration to God is on his head. \v{8}During the entire time of his dedication, he is set apart to God. \v{9}When someone suddenly dies beside him, so that his consecrated head is defiled, then he is to shave his head on the day of his purification. Seven days later he is to shave it again. \v{10}On the eighth day, he is to bring two turtledoves or two pigeons to the priest at the entrance to the Tent of Meeting. \v{11}Then the priest is to offer one for a sin offering and the other for a burnt offering to make atonement for him because of the guilt he incurred on account of his contact with the dead body. Then he is to consecrate his head on that day. \v{12}He is to dedicate to the \divine{Lord} the days of his consecration by bringing a year old male lamb as his offering. The previous time will have failed because his consecration became defiled.

\v{13}``This is the law of the Nazirite: When the days of his consecration are completed, he is to come to the entrance at the Tent of Meeting. \v{14}He is to bring an offering to the \divine{Lord}, a year old male lamb, and a year old ewe female lamb, both without blemish, for a sin offering and a ram without blemish for a peace offering, \v{15}a basket of unleavened bread made\fnote{The Heb. lacks \fbib{made}} from choice flour, cakes mixed with oil, a wafer of unleavened bread smeared with oil, along with grain and drink offerings. \v{16}The priest is to come into the \divine{Lord}'s presence and present his sin and burnt offerings. \v{17}He is to offer the ram, a sacrifice of peace offering to the \divine{Lord}, along with the basket of unleavened bread. Then the priest is to present his grain and drink offerings. \v{18}The Nazirite is then to shave his head of consecration at the entrance to the Tent of Meeting. He is to take the lock of his head of consecration and set it over the fire where the peace offering for sacrifice is. \v{19}Then the priest is to take the boiled shoulder of the ram, one cake of unleavened bread from the basket, and one wafer of unleavened bread. He is to place them in the hands of the Nazirite, after he himself has shaved his symbol of consecration. \v{20}The priest is to wave the offerings, that is, the breast and the thigh offering in the \divine{Lord}'s presence. Then the Nazirite may drink wine afterward. \v{21}This is to be the law of the Nazirite when he commits his offering to the \divine{Lord} on account of his consecration, over and beyond what he owns alone plus whatever he can provide,\fnote{Lit. \fbib{his hand can reach}} based on the vow from his own mouth that he vows to fulfill on account of the law of his consecration.''
\passage{On Blessing the Israelis}

\v{22}Later, the \divine{Lord} told Moses, \v{23}``Teach Aaron and his sons to bless the Israelis:

\begin{poetry}
\poeml \v{24}May the \divine{Lord} bless you \\
\poemll    and guard you. \\
\poeml \v{25}May the \divine{Lord}'s face enlighten you \\
\poemll    and bestow favor on you. \\
\poeml \v{26}May the \divine{Lord} turn to face you, \\
\poemll    lavishing peace on you!
\end{poetry}

\v{27}They are to pour out my name to the Israelis while I continue to bless them.''
\labelchapt{7}
\passage{Offerings by Leaders}

\chapt{7}
\v{1}The same day that Moses finished setting up, anointing, and consecrating the tent and its utensils, he also anointed and consecrated the altar and its utensils. \v{2}Then the presiding leaders of Israel, as heads of the ancestral houses, brought an offering. They were the leaders of the tribes who supervised the census. \v{3}They brought their offering into the \divine{Lord}'s presence, consisting of\fnote{The Heb. lacks \fbib{consisting of}} six covered carts and twelve oxen---one cart each from two leaders and an ox from each one. After they presented them in front of the tent, \v{4}the \divine{Lord} told Moses, \v{5}``Take these gifts from them and use them in service at the Tent of Meeting. Present them to the descendants of Levi, distributing them to each person according to his work.''

\v{6}So Moses took the carts and the oxen and presented them to the descendants of Levi. \v{7}Two carts and four oxen were given to the descendants of Gershon for their work. \v{8}Four carts and eight oxen were given to the descendants of Merari for their work. \v{9}But he gave none of them to the descendants of Kohath, because their responsibility was to carry the holy things on their shoulders. \v{10}The leaders brought the offerings for the dedication of the altar the same day that it was anointed. After the leaders brought their offering to the altar, \v{11}the \divine{Lord} told Moses, ``They are to present their offerings, one leader per day,\fnote{Lit. \fbib{one leader for the day, one leader for the day}} for the dedication of the altar.''
\passage{Day One: Nahshon's Offering}

\v{12}On the first day Amminadab's son Nahshon, from the tribe of Judah, presented \v{13}as his offering a silver dish weighing 130 shekels and a silver bowl weighing 70 shekels (calculated according to the shekel of the sanctuary), both\fnote{Lit. \fbib{the two of them,} and so throughout the chapter} filled with choice flour mixed with oil for a grain offering; \v{14}one gold pan weighing ten shekels,\fnote{Lit. \fbib{gold}, and so throughout the chapter} full of incense; \v{15}one young bull, one ram, and a one year old male lamb for a burnt offering; \v{16}and one male goat for a sin offering. \v{17}Their sacrifice for a peace offering consisted of\fnote{The Heb. lacks \fbib{consisted of}, and so throughout the chapter} two bulls, five rams, five male goats, and five one year old lambs. These were the offerings presented by Amminadab's son Nahshon.
\passage{Day Two: Nathaniel's Offering}

\v{18}On the second day, Zuar's son Nethanel, leader of the descendants of Issachar, presented \v{19}as his offering a silver dish weighing 130 shekels and a silver bowl weighing 70 shekels (calculated according to the shekel of the sanctuary), both filled with choice flour mixed with oil for a grain offering; \v{20}one gold pan weighing ten shekels, full of incense; \v{21}one young bull, one ram, and a one year old male lamb for a burnt offering; \v{22}and one male goat for a sin offering. \v{23}Their sacrifice for a peace offering consisted of two bulls, five rams, five male goats, and five one year old lambs. These were the offerings presented by Zuar's son Nathaniel.
\passage{Day Three: Eliab's Offering}

\v{24}On the third day, Helon's son Eliab, leader of the descendants of Zebulun presented \v{25}as his offering a silver dish weighing 130 shekels and a silver bowl weighing 70 shekels (calculated according to the shekel of the sanctuary), both filled with choice flour mixed with oil for a grain offering; \v{26}one gold pan weighing ten shekels, full of incense; \v{27}one young bull, one ram, and a one year old male lamb for a burnt offering; \v{28}and one male goat for a sin offering. \v{29}Their sacrifice for a peace offering consisted of two bulls, five rams, five male goats, and five one year old lambs. These were the offerings presented by Helon's son Eliab.
\passage{Day Four: Elizur's Offering}

\v{30}On the fourth day, Shedeur's son Elizur, leader of the descendants of Reuben presented \v{31}as his offering a silver dish weighing 130 shekels and a silver bowl weighing 70 shekels (calculated according to the shekel of the sanctuary), both filled with choice flour mixed with oil for a grain offering; \v{32}one gold pan weighing ten shekels, full of incense; \v{33}one young bull, one ram, and a one year old male lamb for a burnt offering; \v{34}and one male goat for a sin offering. \v{35}Their sacrifice for a peace offering, two bulls, five rams, five male goats, and five one year old lambs. These were the offerings presented by Shedeur's son Elizur.
\passage{Day Five: Shelumiel's Offering}

\v{36}On the fifth day, Zurishaddai's son Shelumiel, leader of the descendants of Simeon, presented \v{37}as his offering a silver dish weighing 130 shekels and a silver bowl weighing 70 shekels (calculated according to the shekel of the sanctuary), both filled with choice flour mixed with oil for grain offering; \v{38}one gold pan weighing ten shekels, full of incense; \v{39}one young bull, one ram, and a one year old male lamb for a burnt offering; \v{40}and one male goat for a sin offering. \v{41}Their sacrifice for a peace offering consisted of two bulls, five rams, five male goats, and five one year old lambs. These were the offerings presented by Zurishaddai's son Shelumiel.
\passage{Day Six: Eliasaph's Offering}

\v{42}On the sixth day, Deuel's son Eliasaph, leader of the descendants of Gad, presented \v{43}as his offering a silver dish weighing 130 shekels and a silver bowl weighing 70 shekels (calculated according to the shekel of the sanctuary), both filled with choice flour mixed with oil for a grain offering; \v{44}one gold pan weighing ten shekels full of incense; \v{45}one young bull, one ram, and a one year old male lamb for a burnt offering; \v{46}and one male goat for a sin offering. \v{47}Their sacrifice for a peace offering consisted of two bulls, five rams, five male goats, and five one year old lambs. These were the offerings presented by Deuel's son Eliasaph.
\passage{Day Seven: Elishama's Offering}

\v{48}On the seventh day, Ammihud's son Elishama, leader of the descendants of Ephraim, presented \v{49}as his offering a silver dish weighing 130 shekels and a silver bowl weighing 70 shekels (calculated according to the shekel of the sanctuary), both filled with choice flour mixed with oil for a grain offering; \v{50}one gold pan weighing ten shekels, full of incense; \v{51}one young bull, one ram, and a one year old male lamb for a burnt offering; \v{52}and one male goat for a sin offering. \v{53}Their sacrifice for a peace offering consisted of two bulls, five rams, five male goats, and five one year old lambs. These were the offerings presented by Ammihud's son Elishama.
\passage{Day Eight: Gamaliel's Offering}

\v{54}On the eighth day, Pedahzur's son Gamaliel, leader of the descendants of Manasseh, presented \v{55}as his offering a silver dish weighing 130 shekels and a silver bowl weighing 70 shekels (calculated according to the shekel of the sanctuary), both filled with choice flour mixed with oil for grain offering; \v{56}one gold pan weighing ten shekels full of incense; \v{57}one young bull, one ram, and a one year old male lamb for a burnt offering; \v{58}and one male goat for a sin offering. \v{59}Their sacrifice for a peace offering consisted of two bulls, five rams, five male goats, and five one year old lambs. These were the offerings presented by Pedahzur's son Gamaliel.
\passage{Day Nine: Abidan's Offering}

\v{60}On the ninth day, Gideoni's son Abidan, leader of the descendants of Benjamin, presented \v{61}as his offering a silver dish weighing 130 shekels and a silver bowl weighing 70 shekels (calculated according to the shekel of the sanctuary), both filled with choice flour mixed with oil for grain offering; \v{62}one gold pan weighing ten shekels, full of incense; \v{63}one young bull, one ram, and a one year old male lamb for a burnt offering; \v{64}and one male goat for a sin offering. \v{65}Their sacrifice for a peace offering consisted of two bulls, five rams, five male goats, and five one year old lambs. These were the offerings presented by Gideoni's son Abidan.
\passage{Day Ten: Ahiezer's Offering}

\v{66}On the tenth day, Ammishaddai's son Ahiezer, leader of the descendants of Dan, presented \v{67}as his offering a silver dish weighing 130 shekels and a silver bowl weighing 70 shekels (calculated according to the shekel of the sanctuary), both filled with choice flour mixed with oil for a grain offering; \v{68}one gold pan weighing ten shekels full of incense; \v{69}one young bull, one ram, and a one year old male lamb for a burnt offering; \v{70}and one male goat for a sin offering. \v{71}Their sacrifice for a peace offering consisted of two bulls, five rams, five male goats, and five one year old lambs. These were the offerings presented by Ammishaddai's son Ahiezer.
\passage{Day Eleven: Pagiel's Offering}

\v{72}On the eleventh day, Ochran's son Pagiel, leader of the descendants of Asher, presented \v{73}as his offering a silver dish weighing 130 shekels and a silver bowl weighing 70 shekels (calculated according to the shekel of the sanctuary), both filled with choice flour mixed with oil for a grain offering; \v{74}one gold pan weighing ten shekels, full of incense; \v{75}one young bull, one ram, and a one year old male lamb for a burnt offering; \v{76}and one male goat for a sin offering. \v{77}Their sacrifice for a peace offering consisted of two bulls, five rams, five male goats, and five one year old lambs. These were the offerings presented by Ochran's son Pagiel.
\passage{Day Twelve: Ahira's Offering}

\v{78}On the twelfth day, Enan's son Ahira, leader of the descendants of Naphtali, presented \v{79}as his offering a silver dish weighing 130 shekels and a silver bowl weighing 70 shekels (calculated according to the shekel of the sanctuary), both filled with choice flour mixed with oil for grain offering; \v{80}one gold pan weighing ten shekels, full of incense; \v{81}one young bull, one ram, and a one year old male lamb for a burnt offering; \v{82}and one male goat for a sin offering. \v{83}Their sacrifice for a peace offering consisted of two bulls, five rams, five male goats, and five one year old lambs. These were the offerings presented by Enan's son Ahira.
\passage{Summary of Offerings}

\v{84}This was what was presented at\fnote{The Heb. lacks \fbib{what was presented at}} the dedication of the altar from the leaders of Israel on the same day that it was anointed: twelve silver bowls, twelve silver basins, twelve gold ladles. \v{85}Each bowl weighed 130 silver shekels and each basin weighed 70 shekels. All the silver vessels weighed a total of 2,400 shekels, calculated according to the\fnote{The Heb. lacks \fbib{calculated according to the}} shekel of the sanctuary. \v{86}Also, twelve gold ladles filled with incense were presented,\fnote{The Heb. lacks \fbib{were presented}} each ladle weighing ten shekels (calculated according to the shekel\fnote{The Heb. lacks \fbib{calculated according to the shekel}} of the sanctuary). All of the gold of the ladles weighed 120 shekels. \v{87}All the livestock for burnt offerings totaled twelve bulls, twelve rams, twelve sheep in their first year with corresponding meal offerings, and twelve male goats for sin offerings. \v{88}All the livestock for peace offerings totaled 24 bulls, 60 rams, 60 male goats, and 60 one year old lambs---all this was for the altar's dedication after it was anointed.
\passage{God Speaks above the Mercy Seat}

\v{89}When Moses entered the Tent of Meeting to speak with the \divine{Lord},\fnote{Lit. \fbib{with him}} he heard a voice speaking to him above the Mercy Seat\fnote{Or \fbib{atonement place}, and so throughout the book} over the Ark of the Testimony. He spoke to him from between the two cherubim.
\labelchapt{8}
\passage{The Seven Lamps}
\passageinfo{(Exodus 25:31-40)}

\chapt{8}
\v{1}The \divine{Lord} told Moses, \v{2}``Tell Aaron, `When you set up the lamps, the seven lamps will illuminate the area in\fnote{The Heb. lacks \fbib{the area in}} front of the lamp stand.'\,''\fnote{Or \fbib{menorah}} \v{3}So Aaron did so, setting up the lamps to illuminate the area in\fnote{The Heb. lacks \fbib{the area in}} front of the lamp stand, just as the \divine{Lord} had commanded Moses. \v{4}This was how the lamp stand was crafted from hammered gold: From its base to its flowers, it was made of hammered gold. Moses crafted the lamp stand just as the \divine{Lord} had showed him.\fnote{Lit. \fbib{Moses}}
\passage{Purifying the Descendants of Levi}

\v{5}Then the \divine{Lord} told Moses, \v{6}``Take the descendants of Levi from the Israelis and purify them. \v{7}This is what you are to do for them in order to purify them: Sprinkle purifying water over them, have them shave their skin, and then have them wash their garments, and they will be purified. \v{8}They are to take a young bull along with its meal offering made of flour mixed with oil. Then you are to take a second young bull as a sin offering. \v{9}Assemble the descendants of Levi in front of the appointed place of meeting, and assemble the whole congregation of Israel, too. \v{10}Bring the descendants of Levi into the \divine{Lord}'s presence and have the Israelis lay their hands on the descendants of Levi.

\v{11}``Then Aaron is to present the descendants of Levi as a wave offering before the \divine{Lord} from the Israelis, because they are to work in the service of the \divine{Lord}. \v{12}The descendants of Levi are then to lay their hands on the head of the bulls, offering one for a sin offering and the other one for a burnt offering to the \divine{Lord} to atone for the descendants of Levi. \v{13}You are to make the descendants of Levi stand in the presence of Aaron and his sons. Then you are to wave them as wave offerings to the \divine{Lord}. \v{14}This is how you are to separate the descendants of Levi from among the Israelis. The descendants of Levi belong to me.

\v{15}``After this, the descendants of Levi are to come to serve at the appointed place of meeting, after you have purified them and presented them as wave offerings, \v{16}since they've been set apart for me from among the Israelis. I've taken them for myself instead of the first to open the womb---every firstborn of the Israelis, \v{17}since every firstborn of Israel belongs to me, from human beings to livestock. On the same day that I destroyed all the firstborn in the land of Egypt, I consecrated them to myself, \v{18}taking the descendants of Levi instead of every firstborn of the Israelis. \v{19}I've set the descendants of Levi apart from the Israelis so that Aaron and his sons would work in service at the appointed place of meeting, making atonement on behalf of the Israelis so that there won't be a plague among the Israelis whenever they approach the sanctuary.''

\v{20}So Moses and Aaron and the Israelis did this on behalf of the descendants of Levi. The Israelis did everything that the \divine{Lord} commanded concerning the descendants of Levi. \v{21}The descendants of Levi therefore purified themselves, washed their clothes, and then Aaron presented them as wave offerings to the \divine{Lord}. Aaron provided atonement for them to purify them. \v{22}After this, the descendants of Levi entered into their work of service at the appointed place, in the presence of Aaron and his sons. They did everything that the \divine{Lord} commanded Moses concerning the descendants of Levi.
\passage{Age Restrictions for the Descendants of Levi}

\v{23}Later, the \divine{Lord} told Moses, \v{24}``Now regarding a descendant of Levi who is 25 years and above, he is to enter work in the service at the appointed place of meeting, \v{25}but starting at 50 years of age, he is to retire from service and is no longer to work. \v{26}He may minister to his brothers at the Tent of Meeting by keeping watch, but he is not to engage in service. This is how you are to act with respect to the obligations of the descendants of Levi.''
\labelchapt{9}
\passage{The Passover at Sinai}
\passageinfo{(Exodus 12:1-20)}

\chapt{9}
\v{1}The \divine{Lord} spoke to Moses in the Wilderness of Sinai during the first month of the second year that they had left Egypt, \v{2}``The Israelis are to observe the Passover at its appointed time \v{3}on the fourteenth day of this month. You are to observe it at this appointed time between the evenings. You are to observe it according to all its decrees and laws.''

\v{4}So Moses instructed the Israelis to observe the Passover. \v{5}They observed the Passover on the fourteenth day of the first month at twilight, in the Wilderness of Sinai. The Israelis did everything that the \divine{Lord} had commanded through Moses.
\passage{Special Passover Rules}

\v{6}But there were men who couldn't observe the Passover that day because they had come in contact with a corpse. That very day, they approached Moses and Aaron \v{7}and asked, ``Why can't we bring an offering to the \divine{Lord} at the appointed time among the Israelis, even though we are unclean because we came in contact with a corpse?''

\v{8}``Wait while I hear what the \divine{Lord} has to say about you,'' Moses replied.

\v{9}Then the \divine{Lord} told Moses, \v{10}``Instruct\fnote{Or \fbib{speak}} the Israelis that when any of you or your descendants becomes unclean due to contact with a corpse, or if he is on a long journey, he nevertheless is to observe the \divine{Lord}'s Passover. \v{11}On the fourteenth day of the second month at twilight, they are to eat it with unleavened bread and bitter herbs. \v{12}They are not to leave any of it to remain until morning nor are they to break any of its bones. They are to observe it according to all the statutes of the Passover. \v{13}Now as to the person\fnote{Lit. \fbib{man}} who is clean and isn't traveling, but fails to observe the Passover, that person\fnote{Or \fbib{soul}} is to be eliminated from his people, because he didn't bring an offering to the \divine{Lord} at its appointed time. That person is to bear his sin. \v{14}If a resident alien lives with you and wants to observe the \divine{Lord}'s Passover, let him observe it according to the statutes and laws of the Passover. You are to maintain the same statute\fnote{Lit. \fbib{one decree shall be for you}} for the resident alien as you do for the native of the land.''
\passage{The Fire Cloud over the Tent}

\v{15}On the same morning\fnote{Lit. \fbib{day}} that the tent was set up, a cloud covered the tent, that is, the Tent of Testimony, and in the evening fire appeared over the tent until morning. \v{16}It was so continuously---there was a cloud covering by day, and a fire cloud appeared at night. \v{17}Whenever the cloud above the tent ascended, the Israelis would travel and encamp in the place where the cloud settled. \v{18}According to whatever the \divine{Lord} said,\fnote{Lit. \fbib{to the mouth of the Lord}} the Israelis would travel. According to whatever the \divine{Lord} said, they would camp as long as the cloud remained over the Tent of Meeting.

\v{19}When the cloud over the tent remained for a longer time, the Israelis did what the \divine{Lord} had instructed and didn't travel. \v{20}There were times when the cloud remained over the tent for a number of days. They camped in accordance with the \divine{Lord}'s instructions and they traveled in accordance with the \divine{Lord}'s instructions. \v{21}There were times when the cloud remained from evening until morning, but when the cloud ascended in the morning, they would journey. Whether by day or by night, they would travel whenever the cloud ascended. \v{22}Whether for two days, a month, or for longer periods, whenever the cloud would remain above the tent, the Israelis would remain in camp, not traveling. But whenever it ascended, then they would travel. \v{23}According to what the \divine{Lord} said, they would remain in camp, and according to what the \divine{Lord} said, they would travel. They kept the commands that the \divine{Lord} had given through Moses.
\labelchapt{10}
\passage{Silver Trumpets}

\chapt{10}
\v{1}The \divine{Lord} also told Moses, \v{2}``Make two trumpets, crafting them from beaten silver, for use in calling the congregation together and for notifying the camps to set out for travel. \v{3}Sound them when the whole assembly is to gather together at the entrance to the appointed place of meeting. \v{4}When one trumpet is blown, the elders and the heads of the thousands of the Israelis are to gather to you. \v{5}When you sound an alarm, the ones encamped on the east side are to begin to travel. \v{6}When you sound the alarm the second time, those encamped on the south are to begin to travel. Alarms are to be sounded for their travels. \v{7}But when you blow the trumpet to assemble the whole congregation, don't use the same sound as you do for sounding an alarm.\fnote{The Heb. lacks \fbib{as you do for sounding an alarm}} \v{8}The descendants of Aaron the priest are to blow the trumpets. Have them do this for you permanently throughout your generations to come.''
\passage{Sounding the Trumpet in Battle}

\v{9}``When you wage war in your land against an enemy who is hostile to you, you are to sound an alarm with the trumpets. Then you will be remembered before the face of the \divine{Lord} your God and you will be delivered from your enemies. \v{10}At the beginning of the month, during your time of rejoicing at the appointed place, sound the trumpet over your burnt offering, then sacrifice your peace offering, since they are to be your memorial before the \divine{Lord} your God. I am the \divine{Lord} your God.''
\passage{Order of Travel in the Wilderness}

\v{11}On the twentieth day of the second month in the second year, the cloud was lifted up from the Tent of Meeting, \v{12}so the Israelis set out from the Sinai Wilderness until the cloud settled in the Paran Wilderness, \v{13}doing what the \divine{Lord} had said through Moses.

\v{14}The standard of the camp of Judah was the first to travel, accompanied by its army with Amminadab's son Nahshon in charge. \v{15}Zuar's son Nethanel was in charge of the camp of Issachar. \v{16}Helon's son Eliab was in charge of the camp of Zebulun. \v{17}The tent was taken down, and the descendants of Gershon and Merari carried the tent.

\v{18}Then the standard of the camp of Reuben set out, accompanied by its army with Shedeur's son Elizur in charge. \v{19}Zurishaddai's son Shelumiel was in charge of the tribe of Simeon. \v{20}Deuel's son Eliasaph was in charge of the tribe of Gad. \v{21}Then the descendants of Kohath, carrying the sanctuary, set out, since the tent was to be set up before they arrive.

\v{22}After this, the standard of the camp of Ephraim set out, accompanied by its army with Ammihud's son Elishama in charge. \v{23}Pedazzur's son Gamaliel was in charge of the tribe of Manasseh. \v{24}Gideoni's son Abidan was in charge of the army of the tribe of Benjamin.

\v{25}Then the standard of the camp of Dan set out, functioning as the rear guard for all the encampments, accompanied by its army with Ammishaddai's son Ahiezer. \v{26}Ochran's son Pagiel was in charge of the tribe of Asher. \v{27}Enan's son Ahira was in charge of the tribe of Naphtali.

\v{28}This was the travel order for the Israelis, whenever their companies traveled.
\passage{Moses invites His Father-in-Law to Accompany Israel}

\v{29}Then Moses told Reuel's son Hobab, Moses' relative by marriage\fnote{The Heb. word can connote any family relationship established through marriage, including \fbib{father-in-law} or \fbib{brother-in-law}; cf. Judg 4:11; Exod 2:18 3:1, 18.} from Midian, ``We are traveling to the place about which the \divine{Lord} said `I will give it to you.' So come with us and we'll be good to you, because the \divine{Lord} has spoken good things about Israel.''

\v{30}But he said, ``I won't go with you because I'm returning to my land and to my own family.''

\v{31}Then Moses\fnote{Lit. \fbib{he}} responded, ``Please don't leave us now, since you know where we can camp in the wilderness. You could be our guide.\fnote{Lit. \fbib{be eyes for us}} \v{32}And when you come with us, the good things that the \divine{Lord} will grant us, we'll give you as well.''\fnote{Lit. \fbib{we'll cause to be good to you}}

\v{33}So they traveled from the mountain of the \divine{Lord}, a three-day trip, with the Ark of the Covenant of the \divine{Lord} traveling in front of them---a three day trip to explore a place for them to rest. \v{34}Moreover, the cloud of the \divine{Lord} protected them during the day when they left their camp. \v{35}Whenever the ark was ready to travel, Moses would say:

\begin{poetry}
\poeml ``Arise, \divine{Lord}, \\
\poemll    to scatter your enemies, \\
\poeml so that whoever hates you \\
\poemll    will flee from your presence.''
\end{poetry}

\v{36}Whenever the ark was being readied to rest, he would say:

\begin{poetry}
\poeml ``Return, \divine{Lord}, \\
\poemll    to the countless thousands of Israel.''
\end{poetry}
\labelchapt{11}
\passage{Israel Complains}

\chapt{11}
\v{1}Eventually, the people began complaining about their distress, and the \divine{Lord} heard them. When the \divine{Lord} heard, his anger flared up and the \divine{Lord}'s fire incinerated some of them within the outskirts of the camp. \v{2}When the people cried out to Moses, he\fnote{Lit. \fbib{Moses}} prayed to the \divine{Lord} and the fire stopped. \v{3}He then named that place Taberah,\fnote{The Heb. name \fbib{Taberah} means ``burning''} because the \divine{Lord}'s fire had incinerated some of them.

\v{4}Meanwhile, certain riff-raff among the people\fnote{Lit. \fbib{among them}} had an insatiable appetite\fnote{Lit. \fbib{craved for a craving}} for food. As a result, they wept and turned back, and the Israelis cried out, ``If only somebody would feed us some meat! \v{5}How we remember the fish that we used to eat in Egypt for free! And the cucumbers, melons, leeks, onions, and garlic! \v{6}But now we can't stand it anymore,\fnote{Lit. \fbib{now our strength is dried up}} because there's nothing in front of us except this manna.''

\v{7}Now manna was reminiscent of coriander seed, with an appearance similar to amber.\fnote{Lit. \fbib{bdellium}; i.e. a clear gum resin} \v{8}People would go out to gather it, then they would grind it in mills or pound it in mortars, and then they would boil it in pots or make cakes out of it that tasted like butter cakes. \v{9}When the dew fell in the camp, the manna came with it.
\passage{Moses Responds}

\v{10}Moses heard the people weeping throughout their entire families. Everyone gathered at the entrance to their tents so that the \divine{Lord} was very angry. Moses thought the situation was bad, \v{11}so he\fnote{Lit. \fbib{Moses}} asked the \divine{Lord}, ``Why did you bring all this trouble to your servant? Why haven't I found favor in your eyes? After all, you're putting the burden of this entire people on me! \v{12}Did I conceive this people or give birth to them, so that you would tell me to carry them near my heart like a wet nurse carries a suckling baby to the land that you promised to their forefathers? \v{13}Where am I going to get meat to give this people? After all, they're crying in front of me, `Give us meat to eat!' \v{14}I cannot carry this whole nation! The burden is too heavy for me! \v{15}If this is how you treat me, please kill me right now, if I've found favor in your eyes, because I don't want to keep staring at all of this\fnote{Lit. \fbib{at my}} misery!''
\passage{The Appointment of 70 Elders}

\v{16}Then the \divine{Lord} told Moses, ``Gather together for me 70 men who are elders of Israel, men whom you know to be elders of the people and officers over them. Then bring them to the Tent of Meeting and let them stand there with you. \v{17}Then I'll come down and speak with you. I'll take some of the spirit that rests on you and apportion it among them, so that they may help you bear the burden of the people. That way, you won't bear it by yourself.''
\passage{God Threatens to Provide Meat}

\v{18}``But give this command to the people: `You are to consecrate yourselves, because tomorrow you're going to eat meat, since you've complained where the \divine{Lord} can hear it, ``Who can give us meat to eat? After all, life was better with us in Egypt.'' Therefore, the \divine{Lord} is going to give you meat and you'll eat--- \v{19}not only for a day, or for two days, or for five days, or for ten days, or for 20 days, \v{20}but for a whole month---until it comes out your nostrils and makes you vomit. This is because you've despised the \divine{Lord}, who is among you, and you cried out in his presence by complaining, ``Why did we ever leave Egypt?''\,'\,''
\passage{Moses Doubts God's Ability}

\v{21}Moses responded, ``I'm with 600,000 people on foot and you're saying I am to give them enough\fnote{The Heb. lacks \fbib{enough}} meat to eat for a whole month? \v{22}What if we were to slaughter our entire inventory of\fnote{The Heb. lacks \fbib{our entire inventory}} flocks and herds for them? Would that be enough? What if we could gather all the fish in the sea in nets for them? Would that be enough, either?''
\passage{God Rebukes Moses}

\v{23}But the \divine{Lord} responded to Moses, ``Is the \divine{Lord} short on power?\fnote{Lit. \fbib{hand}} You're now going to witness whether what I say will come to pass or not.''

\v{24}So Moses went out and told the people what the \divine{Lord} had said. He gathered 70 men from the elders of the people and stationed them around the tent. \v{25}The \divine{Lord} came down in a cloud, spoke to Moses,\fnote{Lit. \fbib{him}} and made an apportionment from the spirit who rested on him to the 70 elders. When the spirit rested on them, they prophesied, but that was it.\fnote{Lit. \fbib{prophesied, and not again}}

\v{26}Now two men had remained in camp. One was named Eldad and the other was named Medad. When the spirit rested on them, since they were among those who were listed but had not gone out to the tent, they stayed behind\fnote{The Heb. lacks \fbib{stayed behind and}} and prophesied in the camp. \v{27}A young man ran and reported to Moses, ``Eldad and Medad are prophesying in the camp!''

\v{28}In response, Nun's son Joshua, Moses' attendant and one of his choice men, exclaimed, ``My master Moses! Stop them!''

\v{29}``Are you jealous on account of me?'' Moses asked in reply. ``I wish all of the \divine{Lord}'s people were prophets and that the \divine{Lord} would put his spirit upon them!'' \v{30}Then Moses---that is, he and the elders of Israel---returned to the camp.
\passage{Quails Come to the Camp}

\v{31}Just then, a wind burst forth from the \divine{Lord}, who brought quails from the sea and spread them all around the camp, about a day's journey in each direction, completely encircling the camp about two cubits\fnote{I.e. about three feet; a cubit was about eighteen inches} deep on top of the ground! \v{32}The people stayed up all that day, all that night, and all through the next day, gathering quails. The one who gathered least gathered enough to fill ten omers,\fnote{I.e. in dry capacity about two and a half gallons by volume} as they spread out all around the camp. \v{33}But even as they were chewing the meat and before they had swallowed it, the \divine{Lord} became very angry with the people and struck them with a disastrous plague. \v{34}That's why the place was named Kibroth-hattaavah,\fnote{The Heb. name means \fbib{Graves of Desire}} because they buried the people there who had an insatiable appetite for meat.\fnote{Lit. \fbib{who had great cravings}} \v{35}Later, the people left Kibroth-hattaavah for Hazeroth and camped there.
\labelchapt{12}
\passage{Aaron and Miriam Rebel}

\chapt{12}
\v{1}Miriam and Aaron rebelled against Moses on account of the Cushite woman that he had married. \v{2}They asked, ``Has the \divine{Lord} spoken only through Moses? Hasn't he also spoken through us?''

But the \divine{Lord} heard it.

\v{3}Now the man Moses was very humble---more than any person on earth. \v{4}All of a sudden, the \divine{Lord} told Moses, Aaron, and Miriam, ``The three of you are to come out to the Tent of Meeting.'' So the three of them went out. \v{5}Then the \divine{Lord} came down in a pillar of cloud, stood at the entrance to the Tent of Meeting, and summoned Aaron and Miriam. So both of them went forward.

\v{6}Then he told the two of them: ``Pay attention to what I have to say! When there is a prophet among you, won't I, the \divine{Lord}, reveal myself to him in a vision? Won't I speak with him in a dream? \v{7}But that's not how it is with my servant Moses, since he has been entrusted with my entire household! \v{8}I speak to him audibly\fnote{Lit. \fbib{mouth to mouth}} and in visions, not in mysteries.\fnote{Lit. \fbib{dark speeches}} If he can gaze at the image of the \divine{Lord}, why aren't you afraid to speak against my servant Moses?'' \v{9}Because the \divine{Lord} was very angry with them, he left, \v{10}but when the cloud ascended from the tent, Miriam had become leprous, as white as snow! Aaron turned toward Miriam, and she had leprosy!

\v{11}Aaron begged Moses, ``I pray my lord, please don't hold this sin against us, since we've acted foolishly and sinned in doing so. \v{12}Please don't let her be like one of the living dead, who is born with a congenital skin disease.''\fnote{Lit. \fbib{with half his skin consumed}}

\v{13}So Moses prayed to the \divine{Lord}: ``O \divine{Lord}, please heal her.''

\v{14}But the \divine{Lord} told Moses, ``If her father had merely spit in her face, wouldn't she be humiliated? She is to be placed in isolation for seven days. After that, she may be brought in.'' \v{15}So Miriam was isolated outside the camp for seven days and the people didn't travel until Miriam was brought in. \v{16}After that, the people traveled from Hazeroth and encamped in the Wilderness of Paran.
\labelchapt{13}
\passage{The Twelve Explorers}
\passageinfo{(Deuteronomy 1:19-33)}

\chapt{13}
\v{1}Later, the \divine{Lord} told Moses, \v{2}``Send men to explore the land of Canaan that I'm about to give to the Israelis. Send one man to represent each of his ancestor's tribes, every one of them a distinguished leader\fnote{Lit. \fbib{them one lifted up}} among them.''

\v{3}So that's just what Moses did, sending them from the Wilderness of Paran according to the \divine{Lord}'s instructions. All of the men were Israeli leaders. \v{4}These were their names: From Reuben's tribe, Zaccur's son Shammua; \v{5}From Simeon's tribe, Hori's son Shaphat; \v{6}From Judah's tribe, Jephunneh's son Caleb; \v{7}from Issachar's tribe, Joseph's son Igal; \v{8}From Ephraim's tribe, Nun's son Hoshea; \v{9}From Benjamin's tribe, Raphu's son Palti; \v{10}from Zebulun's tribe, Sodi's son Gaddiel; \v{11}from Joseph's tribe of Manasseh, Susi's son Gaddi; \v{12}From Dan's tribe, Gemalli's son Ammiel; \v{13}from Asher's tribe, Michael's son Sethur; \v{14}from Naphtali's tribe, Vophsi's son Nahbi; \v{15}and from Gad's tribe, Machi's son Geuel. \v{16}These are the names of the men sent by Moses to explore the land.
\passage{Moses Issues Orders to the Explorers}

Moses renamed Nun's son Hoshea to Joshua. \v{17}Then he\fnote{Lit. \fbib{Moses}} sent them out to explore the land of Canaan. He instructed them, ``Go up from here through the Negev,\fnote{I.e. the southern regions of the Sinai peninsula; cf. Josh 10:40} then ascend to the hill country. \v{18}See what the land is like. Observe whether the people who live there are strong or weak, or whether they're few or numerous. \v{19}Look to see whether the land where they live is good or bad, and whether the cities in which they live are merely tents or if they're fortified. \v{20}Examine the farmland,\fnote{Lit. \fbib{land}} whether it's fertile or barren, and see if there are fruit-bearing trees in it or not. Be very courageous, and bring back some samples of the fruit of the land.''

As it was, that time of year\fnote{The Heb. lacks \fbib{of year}} was the season for the first fruits of the grape harvest. \v{21}So they went to explore the land from the Wilderness of Zin to Rehob, and as far as the outskirts of Hamath. \v{22}They went through the Negev\fnote{I.e. the southern regions of the Sinai peninsula; cf. Josh 10:40} and reached Hebron, where Ahiman, Sheshai, and Talmai, the descendants of Anak lived. (Hebron had been constructed seven years before Zoan in Egypt had been built).\fnote{The Heb. lacks \fbib{had been built}} \v{23}Soon they arrived in the valley of Eshcol, where they cut a single branch of grapes and carried it on a pole between two men,\fnote{The Heb. lacks \fbib{men}} along with some pomegranates and figs. \v{24}The entire place was called the Eshcol Valley on account of the cluster of grapes that the men of Israel had taken from there.
\passage{The Explorers Return}

\v{25}At the end of 40 days, they all returned from exploring the land, \v{26}came in to Moses and Aaron, and delivered their report to the entire congregation of Israel in the Wilderness of Paran at Kadesh. They brought back their report to the entire congregation and showed them the fruit of the land. \v{27}``We arrived at the place where you've sent us,'' they reported, ``and it certainly does flow with milk and honey. Furthermore, this is its fruit, \v{28}except that the people who have settled in the land are strong, and their cities are greatly fortified. We also saw the descendants of Anak. \v{29}Amalek lives throughout the Negev,\fnote{I.e. the southern regions of the Sinai peninsula; cf. Josh 10:40} while the Hittites, Jebusites, and Amorites live in the hill country. The Canaanites live by the sea and on the bank of the Jordan.''

\v{30}Caleb silenced the people on Moses' behalf and responded, ``Let's go up and take control, because we can definitely conquer it.''

\v{31}``We can't attack those people,'' the men who were with him said, ``because they're too strong compared to us.''

\v{32}So they put out this false report to the Israelis about the land that they had explored: ``The land that we've explored is one\fnote{Lit. \fbib{is a land}} that devours its inhabitants. All the people whom we observed were giants.\fnote{Lit. \fbib{observed are men of measurements}} \v{33}We also saw the Nephilim,\fnote{Cf. Gen 6:4} the descendants of Anak. Compared to the Nephilim, as we see things, we're like grasshoppers, and that's their opinion of us!''
\labelchapt{14}
\passage{The People Rebel}

\chapt{14}
\v{1}At this, the entire assembly\fnote{Or \fbib{congregation}} complained, started to shout, and cried through the rest of that night. \v{2}All the Israelis complained against Moses and Aaron. Then the entire assembly responded, ``We wish that we had died in Egypt or\fnote{Lit. \fbib{that we have died}} in this wilderness. \v{3}What's the point in the \divine{Lord} bringing us to this land? To die by the sword so our wives and children would become war victims? Wouldn't it be better for us to return to Egypt?''

\v{4}Then they told each other, ``Let's assign a leader and go back to Egypt.''

\v{5}Moses and Aaron fell on their faces in front of the entire assembly of the congregation of Israel. \v{6}Nun's son Joshua and Jephunneh's son Caleb, who had accompanied the others who also had explored the land, tore their clothes \v{7}and attempted to reason with the entire congregation of Israel. They told them, ``The land that we went through and explored is very, very good. \v{8}If the \divine{Lord} is pleased with us, he'll bring us into this land and give it to us. It flows with milk and honey. \v{9}However, don't rebel against the \divine{Lord} or be afraid of the people who live in the land, because we'll gobble them right up.\fnote{Lit. \fbib{because they are bread for us}} Their defenses will collapse, because the \divine{Lord} is with us. You are not to be afraid of them.''

\v{10}But the entire congregation was talking about stoning them to death.
\passage{God Rebukes Unbelieving Israel}

Suddenly, the glory of the \divine{Lord} appeared at the Tent of Meeting to all of the Israelis. \v{11}``How long will this people keep on spurning me and refusing to trust me, despite all the miracles\fnote{Or \fbib{signs}} that I've done among them?'' the \divine{Lord} asked Moses. \v{12}``That's why I'm going to attack them with pestilence and disinherit them. Instead, I'll make you a great nation---even mightier than they are!''
\passage{Moses Intercedes for Israel}

\v{13}But Moses responded to the \divine{Lord}, ``When Egypt hears that you've brought this people out from among them with a mighty demonstration of power,\fnote{The Heb. lacks \fbib{demonstration of}} \v{14}they'll also proclaim to the inhabitants of this land that they've heard you're among this people, \divine{Lord}, whom they've seen face to face,\fnote{Lit. \fbib{seen eye to eye}} since your cloud stands guard over them. You've guided them with a pillar of cloud by day and with a pillar of fire by night. \v{15}But if you slaughter this people all at the same time,\fnote{Lit. \fbib{as a man}} then the nations who heard about your fame\fnote{Lit. \fbib{report}} will say, \v{16}`The \divine{Lord} slaughtered this people in the wilderness because he wasn't able to bring them to the land that he promised them.'

\v{17}``Now, let the power of the \divine{Lord} be magnified, just as you promised when you said, \v{18}`The \divine{Lord} is slow to anger and abundant in faithful love, forgiving iniquity and transgression, but he won't acquit the guilty. He recalls the iniquity of fathers to the third and fourth generation.'\fnote{The Heb. lacks \fbib{generation}}

\v{19}``Forgive, please, the iniquity of this people, according to your great, faithful love, in the same way that you've carried this people from Egypt to this place.''
\passage{God Responds to Moses}

\v{20}The \divine{Lord} responded, ``I've forgiven them based on what you've said. \v{21}But just as I live, and just as the whole earth will be filled with the \divine{Lord}'s glory, \v{22}none of those men who saw my glory and watched my miracles that I did in Egypt and in the wilderness---even though they've tested me these ten times and never listened to my voice--- \v{23}will ever see the land that I promised to their ancestors. Those who spurned me won't see it. \v{24}Now as to my servant Caleb, because a different spirit is within him and he has remained true to me, I'm going to bring him into the land that he explored,\fnote{Lit. \fbib{entered}} and his descendants are to inherit it. \v{25}Now the Amalekite and the Canaanite live in the valley. Tomorrow, turn and then travel to the wilderness in the direction of the Reed\fnote{So MT; LXX reads \fbib{Red}} Sea.''

\v{26}Then the \divine{Lord} told Moses and Aaron, \v{27}``How long will this wicked assembly keep complaining about me? I've heard the complaints of the Israelis that they've been murmuring against me. \v{28}So tell them that as long as I live---consider this to be an oracle from the \divine{Lord}---as certainly as you've spoken right into my ears, that's how I'm going to treat you. \v{29}Your corpses will fall in this wilderness---every single one of you who has been counted among you, according to your number from 20 years and above, who complained against me. \v{30}You will certainly never enter the land about which I made an oath with my uplifted hand to settle you in it, except for Jephunneh's son Caleb and Nun's son Joshua. \v{31}However, I'll bring your little ones---the ones whom you claimed would become war victims---into the land so that they'll know by experience the land that you've rejected.

\v{32}``Now as for you, your corpses will fall in this wilderness \v{33}and your children will wander throughout the wilderness for 40 years. They'll bear the consequences of your idolatries\fnote{Lit. \fbib{fornications}} until your bodies are entirely consumed in the wilderness. \v{34}Just as you explored the land for 40 days, you'll bear the consequences of your iniquities for 40 years---one year for each day---as you experience my hostility. \v{35}I, the \divine{Lord}, have spoken. I will indeed do this to this evil congregation, who gathered together against me. They'll be eliminated in this wilderness and will surely die.''
\passage{God Kills the Unbelieving Explorers}

\v{36}After this, the men whom Moses sent out to explore the land, who returned and made the whole congregation complain against him by bringing an evil report concerning the land, \v{37}and who produced an evil report about the land, died of pestilence in the \divine{Lord}'s presence. \v{38}However, Nun's son Joshua and Jephunneh's son Caleb, who had explored the land, remained alive.
\passage{Rebellion against God's Punishment}

\v{39}After Moses had told all of this to the Israelis, the people deeply mourned. \v{40}So they got up early the next morning and traveled to the top of the mountain, telling themselves, ``Look, we're here and we're going to go up to the place that the \divine{Lord} had spoken about, even though we've sinned.''

\v{41}But Moses asked them, ``Why do you continue to sin against what the \divine{Lord} said? Don't you know that you can never succeed? \v{42}Don't go up, since you know that the \divine{Lord} is no longer with you.\fnote{Lit. \fbib{longer in your midst}} You'll be attacked right in front of your own enemies. \v{43}The Amalekites and Canaanites are there waiting for you. You'll die\fnote{Or \fbib{fall}} violently,\fnote{Lit. \fbib{die by the sword}} since you've turned your back and have stopped following the \divine{Lord}. The \divine{Lord} won't be with you.''

\v{44}But they presumed to go up to the top of the mountain, even though the Ark of the Covenant of the \divine{Lord} and Moses didn't leave the camp. \v{45}The Amalekites came down, accompanied by some Canaanites who lived in the mountains. They attacked and defeated them even while the Israelis were retreating\fnote{The Heb. lacks \fbib{even while the Israelis were retreating}} to Hormah.
\labelchapt{15}
\passage{Offerings by the Israelis}

\chapt{15}
\v{1}Later, the \divine{Lord} instructed\fnote{Or \fbib{spoke}} Moses, \v{2}``Tell the Israelis that \v{3}when you enter the land where you'll be living that I'm about to give you, you are to make an offering by fire to the \divine{Lord}, either a burnt offering, a sacrificial offering to fulfill a vow, or a voluntary offering at the appointed time, to make a pleasing aroma to the \divine{Lord} either from your cattle or from your flocks. \v{4}The offeror is to bring the oblation to the \divine{Lord}, containing one tenth of an ephah\fnote{The Heb. lacks the unit of measurement} of fine flour as a grain offering, mixed with one fourth of a hin\fnote{I.e. about one quart; the \fbib{hin} was equivalent to about one gallon} of olive oil. \v{5}Also prepare one fourth of a hin\fnote{I.e. about one quart; the \fbib{hin} was equivalent to about one gallon} of wine for a drink offering or for the sacrifice of each lamb.

\v{6}``For a ram, prepare a grain offering consisting of two tenths of an ephah\fnote{The Heb. lacks the unit of measurement} of fine flour mixed with one third of a hin\fnote{I.e. about one third of a gallon; the \fbib{hin} was equivalent to about one gallon} of olive oil. \v{7}Now as for your drink offering, offer one third of a hin\fnote{I.e. about one third of a gallon; the \fbib{hin} was equivalent to about one gallon} of wine as a pleasing aroma to the \divine{Lord}.

\v{8}``When you prepare a bull as a burnt offering, or as a sacrifice to fulfill a vow, or for peace offerings to the \divine{Lord}, \v{9}then the bullock is to be presented accompanied by a meal offering of three tenths of an ephah\fnote{The Heb. lacks the unit of measurement} of fine flour mixed with half a hin\fnote{I.e. about two quarts; the \fbib{hin} was equivalent to about one gallon} of oil.

\v{10}``As for drink offerings, offer half a hin\fnote{I.e. about two quarts; the \fbib{hin} was equivalent to about one gallon} of wine, for an offering made by fire is a pleasing aroma to the \divine{Lord}. \v{11}Do this for each bullock, ram, male lamb, or goat. \v{12}Depending on the number of offerings\fnote{The Heb. lacks \fbib{of offerings}} that you prepare, do for each one according to their number. \v{13}Every native born person is to do these things, bringing an offering made by fire as a pleasing aroma to the \divine{Lord}.''
\passage{Offerings by Resident Aliens}

\v{14}``Now, if a resident alien\fnote{Or \fbib{foreigner}} lives\fnote{Lit. \fbib{sojourn}} with you, or whoever else is with you throughout your generations, let him make an offering made by fire, a pleasing aroma to the \divine{Lord}. Just as you do, so is he to do. \v{15}There is to be a single standard for your community, one statute for you and the resident alien who lives with you, a long lasting statute throughout your generations. Just as you do, so is the resident alien to do in the presence of the \divine{Lord}. \v{16}There is to be one law and one ordinance for you and for the resident alien who lives with you.''
\passage{Offerings on Entering the Land}

\v{17}Then the \divine{Lord} instructed Moses: \v{18}``Tell the Israelis that when they enter the land that I'm about to bring you to, \v{19}when you have eaten some of the bread that the land produces, you are to offer a raised offering to the \divine{Lord}. \v{20}You are to offer a cake made from the first of your bread dough as a raised offering to the \divine{Lord}. Offer it as a raised offering right off your threshing floor. \v{21}From then on, throughout your generations give the first of your bread dough to the \divine{Lord}.''
\passage{Offerings for Inadvertent National Sin}

\v{22}``Here's what you are to do\fnote{The Heb. lacks \fbib{Here's what you are to do}} when you all\fnote{Lit. \fbib{you} (pl.)} go astray and fail to observe all these commands that the \divine{Lord} had spoken to Moses, \v{23}including anything that the \divine{Lord} commanded you by the authority\fnote{Lit. \fbib{by the hand}} of Moses, starting from the day the \divine{Lord} commanded Moses and continuing through your generations. \v{24}When anything is done without the knowledge\fnote{Lit. \fbib{is hidden from the eyes}} of the congregation, the entire community is to offer one young bull for a burnt offering, a pleasing aroma to the \divine{Lord}, along with its meal and drink offerings offered according to procedure, and one male goat for a sin offering. \v{25}Then the priest is to make atonement for the entire community of the Israelis, and they will be forgiven\fnote{Or \fbib{it are to be forgiven them}} for inadvertent sins. They are to bring their offering, an offering made by fire to the \divine{Lord}, as well as their sin offering, into the \divine{Lord}'s presence on account of their error. \v{26}Then the entire community of Israel will be forgiven, along with the resident alien who lives among them, since all the people will have sinned inadvertently.''
\passage{Offerings for Inadvertent Personal Sin}

\v{27}``Now when one person\fnote{Lit. \fbib{soul}} sins inadvertently, then he is to bring a one year old female goat for a sin offering. \v{28}Then, in the \divine{Lord}'s presence, the priest is to make atonement for the person who sinned inadvertently, that is, to make atonement on his behalf so he may be forgiven. \v{29}You are to have a single law for the one who does things inadvertently, whether for the native-born Israeli or for the resident alien who lives among you.''
\passage{On Willful Sin}

\v{30}``But if some person acts with a high hand, whether a native-born or a resident alien, he blasphemes God, and that person is to be eliminated from among his people. \v{31}Because he has despised the law of the \divine{Lord} and has broken his commands, that person is certainly to be eliminated. His iniquity will remain on him.''

\v{32}As it was when the Israelis were in the wilderness, they found a man who was gathering wood on the Sabbath day. \v{33}The ones who found him gathering wood brought him to Moses, Aaron, and all the people. \v{34}Then they confined him until it could be declared what should be done to him. \v{35}Then the \divine{Lord} told Moses, ``The man is certainly to die. The entire community is to stone him to death outside the camp.'' \v{36}So the whole community brought him outside the camp and stoned him with stones so that he died, just as the \divine{Lord} had commanded Moses.
\passage{On Garments and Reminders}

\v{37}Later, the \divine{Lord} instructed Moses, \v{38}``Tell the Israelis that they are to make tassels at the edges of their garments throughout their generations and that they are to put a violet cord on the tassels at the edges of their garments. \v{39}That way, when you see the tassel, you'll remember all the commands of the \divine{Lord} and you'll observe them. Then you won't seek your own interests and desires\fnote{Lit. \fbib{heart and your own eyes}} that lead you to be unfaithful. \v{40}Therefore, remember to observe all my commands and to be holy in the presence of your God. \v{41}I am the \divine{Lord} your God, who brought you out of the land of Egypt to be your God. I am the \divine{Lord} your God.''
\labelchapt{16}
\passage{The Rebellion of Korah, Dathan, and Abiram}

\chapt{16}
\v{1}Now Izhar's son Korah, the grandson of Kohath, a descendant of Levi, along with Eliab's sons Dathan and Abiram, and Peleth's son On, a descendant of Reuben, took charge \v{2}of a rebellion against Moses, along with 250 community leaders, Israelis who were famous men and representatives from the assembly. \v{3}They gathered together against Moses and Aaron and told them, ``You have appropriated too much for yourselves from the entire congregation, since all of them are holy, and the \divine{Lord} is among them, too. Why do you exalt yourselves over the \divine{Lord}'s assembly?''

\v{4}When Moses heard this, he fell on his face. \v{5}Then he addressed Korah and his entire company, ``In the morning, may the \divine{Lord} reveal who belongs to him and who is holy. May he cause that person\fnote{Lit. \fbib{him}} to approach him. May he cause to approach him the one whom he has chosen. \v{6}Korah, you and your entire company are to bring censers \v{7}and put fire and incense in them in the \divine{Lord}'s presence tomorrow. It will be that the man whom the \divine{Lord} chooses will be holy. You're taking too much for yourselves, you descendants of Levi.''

\v{8}Moses also told Korah, ``Listen now, you descendants of Levi! \v{9}Is it such an insignificant thing to you that the God of Israel has separated you from the Israelis to draw you to himself, appointing you to do the work of the tent of the \divine{Lord} and to stand before the community to minister to them? \v{10}He brought you near, along with all of your relatives, the descendants of Levi. Are you also seeking the priesthood? \v{11}Therefore you and your group have conspired against the \divine{Lord} and Aaron. What is it that causes you to complain against him?''

\v{12}So Moses sent for Eliab's sons Dathan and Abiram, but they responded, ``We're not coming. \v{13}Is it such an insignificant thing that you brought us out of a land flowing with milk and honey, to kill us in the wilderness? Now you're trying to make yourself be a prince and rule over us, aren't you? \v{14}You still haven't brought us into a land flowing with milk and honey, nor have you given us an inheritance of fields and vineyards. Do you really think that you can make these men look the other way?\fnote{Lit. \fbib{men blind}} We won't go up.''

\v{15}Moses was very angry, so he told the \divine{Lord}, ``Please don't accept their offering. I haven't taken even one donkey from them nor have I hurt even one of them.''

\v{16}Then Moses told Korah, ``You and your entire company are to present yourselves in the \divine{Lord}'s presence tomorrow---you, they, and Aaron. \v{17}Each man is to take a censer, put incense on it, and bring it into the \divine{Lord}'s presence, each man with his censer, for a total of 250 censers. You and Aaron are each to bring his own censer.''

\v{18}So each man took his censer, put fire coals inside of it, placed incense in it, and then stood at the entrance to the Tent of Meeting, accompanied by Moses and Aaron. \v{19}When Korah had assembled the entire community in opposition to Moses and Aaron\fnote{Lit. \fbib{to them}} at the entrance to the Tent of Meeting, the glory of the \divine{Lord} appeared to the entire community.
\passage{God Vindicates Moses and Aaron}

\v{20}Then the \divine{Lord} told Moses and Aaron, \v{21}``Separate yourselves from among this community, and I'll destroy them in a moment.''

\v{22}Then they fell on their faces and said, ``God, the God of the spirits of all flesh, will you be angry at the entire congregation on account of one man's sin?''

\v{23}Then the \divine{Lord} instructed Moses, \v{24}``Tell the community to move away from where Korah, Dathan, and Abiram are living.''

\v{25}So Moses got up and went to Dathan and Abiram, and the elders of Israel followed him. \v{26}Then he told the community, ``Move away from the camps of these wicked men and don't touch anything that belongs to them. That way you won't be destroyed along with all their sins.'' \v{27}So they all moved away from the entire area where Korah, Dathan, and Abiram were living.

Now Korah, Dathan, and Abiram stood at the entrance to their tents with their wives, sons, and little children. \v{28}Then Moses said, ``This is how you'll know that the \divine{Lord} has sent me to do all these awesome works---they're not coming merely from me.\fnote{Lit. \fbib{not of my heart}} \v{29}If these people die a death similar to all other human beings, or if they are punished with a punishment common to other men, then the \divine{Lord} didn't send me. \v{30}But if the \divine{Lord} creates something new,\fnote{Lit. \fbib{creates a creation}} so that the ground opens its mouth and swallows them and everything that belongs to them and they all descend directly to Sheol\fnote{I.e. the realm of the dead} while still alive, then you'll know that these men have spurned the \divine{Lord}.''
\passage{God Executes Korah's Families}

\v{31}Just as he finished saying all these things, the ground under them split open. \v{32}The earth opened its mouth and swallowed them, all their households, everyone who was affiliated with Korah, and all of their property. \v{33}So they and all that belonged to them descended alive to Sheol.\fnote{I.e. the realm of the dead} Then the earth closed over them. That's how they were annihilated from the assembly.

\v{34}Then all of the Israelis who were around them ran away when they heard them crying, ``{\ldots}so the ground won't swallow us up, too.'' \v{35}After this, fire came from the \divine{Lord} and incinerated the 250 men who offered the incense.
\passage{The Censers Used for the Altar}

\v{36}\fnote{This v. is 17:1 in MT, 16:37 is 17:2, and so through 16:50.} Then the \divine{Lord} instructed Moses, \v{37}``Tell Aaron's son Eleazar the priest to take out the censers out of the flames\fnote{Lit. from \fbib{between burning}} and scatter the coals far away, since they are holy. \v{38}As for the censers of those rebels who died, fasten them into beaten plates to line the altar. Since they brought them into the \divine{Lord}'s presence, they're holy. They are to become a reminder\fnote{Lit. \fbib{sign}} to the Israelis.''

\v{39}So Eleazar the priest took the bronze censers that had been burned and beat them into metal plates for the altar, \v{40}to serve as a memorial to the Israelis, a reminder that no unauthorized person, who isn't a descendant of Aaron, is to attempt\fnote{Lit. \fbib{to come near}} to burn\fnote{Lit. to \fbib{sacrifice}} incense in the \divine{Lord}'s presence, so that he may not become like Korah and his group, just as the \divine{Lord} had spoken by the authority\fnote{Lit. \fbib{by the hand}} of Moses.
\passage{The Israelis Continue to Complain}

\v{41}Nevertheless, the very next day, the whole congregation of Israel complained against Moses and Aaron, ``You've killed the \divine{Lord}'s people!''

\v{42}When the community gathered together against Moses and Aaron, they turned toward the Tent of Meeting. All of a sudden, a cloud covered it and the glory of the \divine{Lord} appeared. \v{43}Then Moses and Aaron entered the Tent of Meeting.

\v{44}The \divine{Lord} told Moses, \v{45}``Leave this community, so I can annihilate them in a moment.''

But they fell upon their faces. \v{46}Then Moses told Aaron. ``Take the censer, put fire on it from the altar, and burn some incense. Then walk quickly to the congregation and atone for them, because wrath has already come out from the \divine{Lord}---the plague has begun.''

\v{47}So Aaron took the censer,\fnote{The Heb. lacks \fbib{the censer}} just as Moses had spoken, and ran out to the center of the assembly, where a plague had begun among the people. He set the incense on fire and atoned for the people. \v{48}He stood between the dead and the living and restrained the plague. \v{49}Those who died due to the plague numbered 14,700, not counting those who died due to the matter with Korah.

\v{50}Then Aaron returned to Moses at the entrance to the Tent of Meeting after the slaughter had been restrained.
\labelchapt{17}
\passage{The Budding of Aaron's Rod}

\chapt{17}
\v{1}\fnote{This v. is 17:16 in MT}The \divine{Lord} instructed Moses, \v{2}``Tell the Israelis to take a rod---one from each ancestral house, that is, one from every leader, for a total of twelve rods. Write each tribal name on his rod. \v{3}You are also to write Aaron's name on the tribe of Levi, since there is to be one rod for every leader of their ancestral tribes.

\v{4}``Then lay them there in the Tent of Meeting in front of the Ark of the Covenant\fnote{Lit. \fbib{testimony}} where I'll meet with you. \v{5}The rod that belongs to the man whom I'll choose will burst into bloom. That's how I'll put a stop to the complaints of the Israelis, who are complaining against you.''

\v{6}So Moses spoke to the Israelis, and each of the tribe leaders gave him a rod, one for each leader, according to their ancestral tribes, for a total of twelve rods. Aaron's rod was one of them. \v{7}Then Moses laid out the rods in the \divine{Lord}'s presence, inside the Tent of Testimony. \v{8}The next morning, Moses went to the Tent of Testimony and the rod of Aaron of the tribe of Levi had burst into bloom! It sprouted buds, bloomed blossoms, and produced fully ripe almonds! \v{9}Then Moses took out all the rods from the \divine{Lord}'s presence to show\fnote{The Heb. lacks \fbib{show}} all the Israelis. Everybody looked, and then each man took his rod.

\v{10}Then the Lord instructed Moses, ``Return Aaron's rod before the testimony\fnote{I.e. the Ark of the Covenant} to be kept for a reminder\fnote{Lit. \fbib{sign}} against the rebels\fnote{Or \fbib{sons of rebellion}} so that you may put an end to their complaints against me and so that they won't die.''

\v{11}So Moses did exactly what the \divine{Lord} had commanded him to do. \v{12}Then the Israelis told Moses, ``We're sure to die! We're all going to perish---all of us! \v{13}Anyone who comes near or approaches the \divine{Lord}'s tent is to die. Are all of us going to die?''
\labelchapt{18}
\passage{Responsibilities for Priests and Descendants of Levi}

\chapt{18}
\v{1}Later, the \divine{Lord} told Aaron, ``You, your sons, and your father's tribe with you are to bear the iniquity of the sanctuary. Also, you and your sons with you are to bear the iniquity of your priesthood. \v{2}Moreover, bring your brothers from your father's tribe of Levi with you, so they may join you and minister to you while you and your sons with you stand in the presence of the Tent of Testimony. \v{3}They are to take care of your concerns and all the responsibilities involved with the tent. But they're not to approach the holy vessels or the altar. That way, neither you nor they will die. \v{4}They are to join you to maintain services related to the Tent of Meeting, for all the responsibilities involved with the tent. But no unauthorized person\fnote{Lit. \fbib{stranger}} is to approach you. \v{5}Take care of the sanctuary and the services of the altar so that there won't be any more wrath on the Israelis. \v{6}Notice that I've taken your brothers, the descendants of Levi, from among the Israelis, giving them to you as a gift from the \divine{Lord} to perform the service of the Tent of Meeting. \v{7}Now you and your sons with you are to maintain your priestly duties and all matters that concern the altar and what is housed within the veil. You are to perform these services. I'm giving you the priesthood as a gift of service, but any unauthorized person\fnote{Lit. \fbib{stranger}} who approaches is to be put to death.''
\passage{Ownership for Offerings}

\v{8}Then the \divine{Lord} told Aaron, ``Look! I am indeed placing you in charge of my raised offerings and the holy things concerning the Israelis. Because of your anointing, I'm giving you and your sons a prescribed portion forever. \v{9}This is what is to belong to you from consecrated offerings spared\fnote{The Heb. lacks \fbib{spared}} from the fire: all of their offerings, grain offerings, sin offerings, and trespass offerings that they render to me. They're to be considered most sacred to you and your sons. \v{10}You may eat them as consecrated gifts. Every male may eat them. They're sacred for you. \v{11}The raised offering and wave offerings presented by the Israelis are yours, too. I've given them to you, to your sons, and to your daughters as a prescribed apportionment forever. Everyone who is clean in your household may eat it. \v{12}All the best\fnote{Lit. \fbib{fat}} oil, wine, grain, and first fruits that they give to the \divine{Lord} are to belong to you. Everyone who is clean in your household may eat it.

\v{13}``The first ripe fruits of everything that the land produces and that they bring to the \divine{Lord} are yours, too. Everyone who is clean in your household may eat it. \v{14}Every devoted thing in Israel is yours, too. \v{15}Everything that opens the womb, any living thing that they bring to the \divine{Lord}---whether from human beings or animals---are for you. Just be sure that you redeem the firstborn of people and the firstborn of unclean animals. \v{16}Those that can be redeemed, you are to redeem at the age of one month, based on your estimate---for five shekels of silver, according to the shekel of the sanctuary, that is, for 20 gerahs. \v{17}But you are not to redeem the firstborn of a cow, sheep, or a female goat. They are holy. You are to sprinkle their blood on the altar and burn their fat for an offering made by fire, a pleasing aroma to the \divine{Lord}. \v{18}Their meat is to belong to you, just as the breast wave offering and the right thigh is yours. \v{19}I'm giving you, your sons, and your daughters as a prescribed portion forever all the raised offerings of the consecrated things that the Israelis offer to the \divine{Lord}. It's a salt covenant forever before the \divine{Lord} with you and your descendants with you.''
\passage{Land Prohibited to Descendants of Levi}

\v{20}Then the \divine{Lord} instructed Aaron, ``You are not to have any inheritance in the land, nor are you to have any portion among the people.\fnote{Lit. \fbib{among them}} I am your portion and your inheritance among the Israelis. \v{21}As to the descendants of Levi, certainly I've given all the tithes in Israel as their inheritance in return for their services that they perform at the Tent of Meeting. \v{22}Therefore, the Israelis need no longer come to the Tent of Meeting, so they won't suffer the consequences of their sin and die. \v{23}The descendants of Levi are to perform the service of the Tent of Meeting and they are to bear their iniquity. This is to be a statute forever, throughout your generations, that they are not to receive an inheritance among the Israelis, \v{24}because I've given to the descendants of Levi the tithes that the Israelis bring to the \divine{Lord} as raised offering. Therefore I told them that, unlike the Israelis, they won't receive an inheritance.''
\passage{Offerings Given to the Descendants of Levi}

\v{25}Then the \divine{Lord} instructed Moses, \v{26}``Tell the descendants of Levi that when they receive tithes from the Israelis (the tithes that I've given you from them as an inheritance), you are to offer a tenth of it\fnote{The Heb. lacks \fbib{a tenth of it}} as a raised offering for the \divine{Lord}. \v{27}Your raised offerings are to be accounted for you as though it were grain from threshing floors and full produce from wine vats. \v{28}You are to offer a raised offering to the \divine{Lord} from all your tithes that you receive from the Israelis. Give Aaron the priest the raised offering of the \divine{Lord} \v{29}out of all the most consecrated offerings that you receive, that is, all the raised offerings of the \divine{Lord}, with all its best and the most holy parts of it. \v{30}Tell them that when they bring the best from it, as far as the descendants of Levi are concerned, it is to be considered like produce from the threshing floors and wine vats. \v{31}You and your household may eat it anywhere, because it's a reward to you in return for your services at the Tent of Meeting. \v{32}You won't sin by offering the best of it, and you are not to profane the sacred things of the Israelis, so that you won't die.''
\labelchapt{19}
\passage{The Red Heifer}

\chapt{19}
\v{1}The \divine{Lord} told Moses and Aaron, \v{2}``This is the ordinance of the law that the \divine{Lord} commanded that the Israelis be told: They are to bring you a spotless red heifer, without physical defect, that has never been fitted with a yoke. \v{3}They are to deliver it to Eleazar the priest, and it is to be brought outside the camp and slaughtered in his presence. \v{4}Then Eleazar the priest is to take blood from it with his finger and sprinkle the blood in front of the Tent of Meeting. \v{5}The entire heifer is to be incinerated in his presence, including its skin, its flesh, its blood, and its dung. \v{6}Then the priest is to take some cedar\fnote{I.e. a genus of coniferous evergreen in the family \fbib{Pinaceae}; and so throughout the book} wood, hyssop, and scarlet material and throw it into the middle of the burning heifer. \v{7}The priest is to wash his clothes and bathe himself\fnote{Lit. \fbib{bathe his flesh}} with water, after which he may enter the camp, but he is to remain unclean until evening. \v{8}Whoever takes part in the burning is to wash his clothes and bathe himself\fnote{Lit. \fbib{bathe his flesh}} in water and is to remain unclean until the evening. \v{9}Then someone\fnote{Lit. \fbib{man}} who is unclean is to gather the ashes of the heifer and lay them outside the camp in a clean place. This is to be done for the community of Israel to use for water of purification from sin. \v{10}Whoever gathers the ashes of the heifer is to wash his clothes and is to remain unclean until the evening. This ordinance is to remain for the benefit of both the Israelis and the resident aliens who live among them.''
\passage{Purification for Contact with the Dead}

\v{11}``Whoever comes in contact with the body of a dead person is to remain unclean for seven days. \v{12}He is to purify himself on the third day and he will be clean on the seventh day. But if he can't purify himself on the third day then he can't be clean on the seventh day. \v{13}Anyone who comes in contact with a dead person (that is, with the corpse\fnote{Lit. \fbib{soul}} of a human being\fnote{Lit. \fbib{a man}} who has died), but who does not purify himself, defiles the \divine{Lord}'s tent. That person is to be eliminated from Israel, because the water of impurity wasn't sprinkled on him. He remains unclean and his uncleanness will remain with him.

\v{14}``This is the procedure to follow\fnote{Lit. \fbib{the law}} when a man dies in his tent: Everyone who enters the tent and everyone in it is to remain unclean for seven days. \v{15}Every open vessel that has no covering fastened around it is to be considered unclean. \v{16}Whoever is out in an open field and touches the body of\fnote{The Heb. lacks \fbib{the body of}} someone who was killed by a sword, or a dead body, or someone's bones, or a grave, he is to be considered unclean for seven days.

\v{17}``Now as for the unclean, they are to take ashes from the burning sin offering, and pour running water on it inside a vessel. \v{18}A clean person is to take some hyssop, dip it in water, and then sprinkle it on the tent, on every vessel, and on whoever\fnote{Lit. \fbib{souls}} was there (that is, on whoever touched the bones, the killed person, or the dead body, including whoever dug the grave). \v{19}The clean person is to sprinkle the unclean person on the third day and seventh day and then he is to purify himself on the seventh day, wash his clothes, and bathe with water. He is to be considered clean at evening.

\v{20}``The person\fnote{Lit. \fbib{man}} who is unclean but who doesn't purify himself is to be eliminated from contact with the assembly, since he has defiled the \divine{Lord}'s sanctuary and the water of impurity wasn't sprinkled on him. He is to be considered unclean \v{21}as a continuing\fnote{Or \fbib{eternal}} reminder to them. Whoever sprinkles the water of impurity is to wash his clothes, and whoever comes in contact with the water of impurity is to remain unclean until evening. \v{22}Furthermore, anything that the unclean person touches is to be considered unclean and the person who touches him is to be considered unclean until the evening.''
\labelchapt{20}
\passage{The Meribah Springs}
\passageinfo{(Exodus 17:1-7)}

\chapt{20}
\v{1}The entire community of the Israelis entered the Zin wilderness during the first month. The people stayed in Kadesh. Miriam died and was buried there.

\v{2}But there was no water for the community, so they gathered together against Moses and Aaron. \v{3}As the people argued with Moses, they told him, ``We wish that we had died when our relatives died in the \divine{Lord}'s presence! \v{4}Why did you bring the assembly of the \divine{Lord} into this wilderness? So we and our cattle could die here? \v{5}Why did you take us out of Egypt and bring us to this terrible place? There's no place to plant seeds, fig trees, vines, or pomegranates! And there's no water to drink!''

\v{6}Then Moses and Aaron went into the presence of the community at the entrance to the Tent of Meeting and fell on their faces. Then the glory of the \divine{Lord} appeared to them.

\v{7}The \divine{Lord} told Moses, \v{8}``Take the rod, gather the community together, and then you and your brother Aaron are to speak to the rock right before their eyes. It will release water. As you bring water to them from the rock, the community and the cattle will be able to drink.'' \v{9}So Moses took the rod in the \divine{Lord}'s presence, just as he had commanded.

\v{10}Then Moses and Aaron gathered the community together in front of the rock. ``Pay attention, you rebels!'' Moses told them. ``Are we to bring you water from this rock?'' \v{11}Then Moses raised his hand and struck the rock twice with his rod. Lots of water gushed out, and both the community and their cattle were able to drink.
\passage{The \divine{Lord} Disciplines Moses}

\v{12}But the \divine{Lord} rebuked Moses and Aaron, telling Moses: ``Because you both\fnote{Lit. \fbib{you} (pl.)} didn't believe me, because you didn't consecrate me as holy\fnote{Lit. to \fbib{set apart}} in the presence\fnote{Lit. \fbib{eyes}} of the Israelis, you won't be the ones to bring this congregation into the land that I'm about to give them.'' \v{13}Because the Israelis argued with the \divine{Lord} and he was set apart among them, this place was called the Meribah Springs.\fnote{The Heb. Name \fbib{Meribah} means \fbib{Place of Strife}}
\passage{The Israelis Approach Edom}

\v{14}Later, Moses sent messengers from Kadesh to the king of Edom with this message: ``This is what your relative Israel says: `You know all the hardships we've encountered.\fnote{Lit. \fbib{hardships that found us}} \v{15}Our ancestors went down to Egypt, where we lived for many\fnote{The Heb. lacks \fbib{many}} years. But the Egyptians treated us and our ancestors viciously. \v{16}Then we cried to the \divine{Lord} and he heard our voice, sending us a messenger who brought us out of Egypt. Now look! We've arrived in Kadesh, a city at the extreme end of your territory. \v{17}Permit us now to pass through your land. We won't pass through your fields or vineyards, and we won't drink water\fnote{Lit. \fbib{waters}} from your wells. We'll keep to the King's Highway without turning either right or left until we have passed through your territory.'\,''

\v{18}But Edom replied, ``You are not to pass through my land.\fnote{Lit. \fbib{through me}} If you do, I'll come out and start a war with you.''

\v{19}Then the Israelis replied, ``Permit us to travel on the highway. If we and our cattle drink your water, we'll pay the price you ask. Only please let us walk through, and nothing more.''\fnote{Lit. \fbib{through without anything}}

\v{20}But still he replied, ``No. You're not to pass through.'' Then Edom went out to meet Moses with a vast army and a lot of military might.\fnote{Lit. \fbib{a mighty hand}} \v{21}That's how Edom refused Israel passage through their territory. So Israel turned away from there.\fnote{Lit. \fbib{him}}
\passage{The Death of Aaron}

\v{22}They traveled from Kadesh, and then the entire community of the Israelis arrived at Mount Hor. \v{23}Then the \divine{Lord} told Moses and Aaron at Mount Hor, near the territory of Edom, \v{24}``Aaron is to be gathered to his people, since he is not to enter the land that I'm about to give the Israelis. After all, you both rebelled against my command\fnote{Lit. \fbib{my mouth}} at the Meribah Springs. \v{25}So take Aaron and his son Eleazar and ascend Mount Hor. \v{26}Remove Aaron's vestments and place them on his son Eleazar, because Aaron is to be gathered to his people\fnote{The Heb. lacks \fbib{to his people}} and die there.''

\v{27}So Moses did just what the \divine{Lord} had commanded. They ascended Mount Hor right in front of the entire community. \v{28}As Moses was stripping Aaron's garments from him and clothing Aaron's son Eleazar with them, Aaron died there on top of the mountain. Afterwards, Moses and Eleazar came down from the mountain. \v{29}When the entire community saw that Aaron had died, they mourned in memory of Aaron for 30 days.
\labelchapt{21}
\passage{The Destruction of Hormah}

\chapt{21}
\v{1}When the Canaanite king of Arad, who lived in the Negev,\fnote{I.e. the southern regions of the Sinai peninsula; cf. Josh 10:40} heard that Israel was coming along the Atharim caravan route, he fought against Israel and took some of them captive. \v{2}Then Israel\fnote{I.e. the Israelis personified as a nation} made this vow in the \divine{Lord}'s presence: ``If you give these people into our control,\fnote{Lit. \fbib{hand}} we intend to devote their cities to total destruction.'' \v{3}When the \divine{Lord} heard what Israel had decided to do,\fnote{Lit. \fbib{heard the voice of Israel}} he delivered the Canaanites to them, and Israel\fnote{Lit. \fbib{he}} exterminated them and their cities. They named the place Hormah.\fnote{The Heb. name \fbib{Hormah} sounds like the Heb. verb \fbib{devoted}}
\passage{The Bronze Serpent}

\v{4}After this, they traveled from Mount Hor along the caravan route by way of the Sea of Reeds and went around the land of Edom. But when the people got impatient because it was a long route, \v{5}the people complained against the \divine{Lord} and Moses. ``Why did you bring us out of Egypt to die in the wilderness?'' they asked. ``There's no food\fnote{Lit. \fbib{bread}} and water, and we're tired of this worthless bread.''\fnote{Or \fbib{light bread}; i.e. the manna}

\v{6}In response, the \divine{Lord} sent poisonous\fnote{Lit. \fbib{fiery}} serpents among the people to bite them. As a result, many people of Israel died. \v{7}Then the people approached Moses and admitted, ``We've sinned by speaking against the \divine{Lord} and you. Pray to the \divine{Lord}, that he'll remove\fnote{Lit. \fbib{turn away}} the serpents from us.'' So Moses prayed in behalf of the people.

\v{8}Then the \divine{Lord} instructed Moses, ``Make a poisonous serpent out of brass and fasten it to a pole. Anyone who has been bitten and who looks at it will live.'' \v{9}So Moses made a bronze serpent and fastened it to a pole. If a person who had been bitten by a poisonous serpent looked to the serpent,\fnote{Lit. \fbib{to it}} he lived.
\passage{Travels in the Wilderness}

\v{10}After this, the Israelis traveled and encamped at Oboth. \v{11}Then they traveled from Oboth and encamped at Iye-abarim, in the wilderness that is in the vicinity of Moab's eastern border. \v{12}From there, they traveled and encamped in the valley of Zered. \v{13}Then they traveled to the other side of Arnon and camped in the wilderness that borders the territory of the Amorites. (Arnon borders Moab between Moab and the Amorites, \v{14}which is why the Book of the Wars of the \divine{Lord}\fnote{I.e. a book chronicling ancient Israel's history, now lost to history} reads, ``Waheb and Suphah and the wadis\fnote{I.e. seasonal rivers that are dry in the summer} of the Arnon, \v{15}and the slope of the valleys, that extends to the dwelling places of Ar and the borders of Moab.'')

\v{16}From there they traveled\fnote{The Heb. lacks \fbib{they traveled}} to the Well of Beer, where the \divine{Lord} had instructed Moses, ``Gather the people together and I'll give you water.'' \v{17}That's also where Israel sang this song:

\begin{poetry}
\poeml Rise up, well! \\
\poemll    Sing to it! \\
\poeml \v{18}It's the well that the leaders dug, \\
\poemll    the one carved out by the nobles of the people \\
\poemlll       with their scepters and staffs.
\end{poetry}

Then they moved on in the wilderness from there to Mattanah, \v{19}then from Mattanah to Nahaliel, from Nahaliel to Bamoth, \v{20}and from Bamoth to the valley of Moab where their fields are, and from there to the top of Mount Pisgah, that looks down toward the open desert.
\passage{Israel Conquers the Amorites}

\v{21}Later, Israel sent messengers to Sihon, king of the Amorites, who conveyed this request:\fnote{The Heb. lacks \fbib{who conveyed this request}} \v{22}``Permit us to pass through your land. We won't trespass in your fields or vineyards. We won't drink water from any well, and we'll only travel along the King's Highway until we've passed through your territory.''

\v{23}Instead of letting Israel pass through his territory, Sihon mustered his entire army and marched out to meet them in the wilderness. He arrived at Jahaz and attacked Israel. \v{24}But Israel defeated\fnote{Lit. \fbib{smote}} him in battle\fnote{Lit. \fbib{him with the edge of the sword}} and took possession of all his lands from Arnon to Jabbok, including the Ammonites, even though the border of the Ammonites was strong. \v{25}So Israel captured all of those cities, occupied\fnote{Lit. \fbib{lived}} all the Amorite cities in Heshbon, and all its towns.\fnote{Lit. \fbib{in all its daughters}} \v{26}Heshbon was the capital city of Sihon, king of the Amorites, who fought against the previous king of Moab and captured all his land from his capital city\fnote{Lit. \fbib{his hand}} to Arnon. \v{27}Therefore the ones who speak in proverbs say:

\begin{poetry}
\poeml Come to Heshbon \\
\poemll    and let it be built! \\
\poemlll       Let the city of Sihon be established! \\
\poeml \v{28}A fire has gone out from Heshbon, \\
\poemll    and a flame from the city of Sihon. \\
\poeml It consumed Ar of Moab \\
\poemll    and the lords of the high places who lived in Arnon. \\
\poeml \v{29}Woe to you, Moab! \\
\poemll    You are destroyed, you people of Chemosh! \\
\poeml He has given up his sons as fugitives \\
\poemll    and his daughters have gone into captivity \\
\poemlll       to Sihon, king of the Amorites. \\
\poeml \v{30}We've fired at them. \\
\poemll    Heshbon has perished as far as Dibon. \\
\poeml We've destroyed them as far as Nophah \\
\poemll    even as far as Medeba.
\end{poetry}

\v{31}So Israel lived in Amorite territory.
\passage{Israel Conquers Bashan}

\v{32}Then Moses sent out explorers to scout Jazer. They captured its towns\fnote{Lit. \fbib{daughters}} and drove out the Amorites who were there. \v{33}Then they turned toward Bashan. However, Og, the king of Bashan, mustered his army and went out to attack them at Edrei. \v{34}The \divine{Lord} told Moses, ``You are not to fear him, because I'm going to deliver him, his entire army, and his land into your control. Do to him just what you've done to Sihon, king of the Amorites, who used to live in Heshbon.'' \v{35}So they attacked him, his sons, and his entire army, until there wasn't even a single survivor left. Then they took possession of his land.
\labelchapt{22}
\passage{Balak Summons Balaam}

\chapt{22}
\v{1}The Israelis continued their travels, eventually\fnote{The Heb. lacks \fbib{eventually}} encamping on the plains of Moab beside the Jordan River\fnote{The Heb. lacks \fbib{River}} opposite Jericho. \v{2}Zippor's son Balak saw everything that Israel had done to the Amorites. \v{3}As a result, Moab greatly feared the people, because they were so numerous. Because a sense of impending doom was afflicting the Moabites as they faced the Israelis, \v{4}the Moabites told the elders of Midian, ``This horde of people is about to lick up everything around us, like an ox licks up the green ground.''

At that time, Zippor's son Balak was the king of Moab. \v{5}He sent messengers to Beor's son Balaam in Pethor, near the Euphrates\fnote{The Heb. lacks \fbib{Euphrates}} River, the land where the descendants of his people originated,\fnote{Or \fbib{the river of the people of Amaw}; LXX reads \fbib{the river of the land}} to summon his aid. He said, ``Look! A group of\fnote{The Heb. lacks \fbib{group of}} people have escaped from Egypt. They cover the surface of the whole earth, and are sitting here right in front of me. \v{6}So come right now and curse this people for me, because there are too many of them for me to handle.\fnote{The Heb. lacks \fbib{to handle}} Perhaps I'll be able to strike them down and drive them out of the land, since I know that whomever you bless is blessed and whomever you curse is cursed.''

\v{7}So the elders of Moab and Midian left to visit Balaam, bringing an honorarium with them,\fnote{Lit. \fbib{bringing divinations in their hand}} and communicated Balak's concerns to him. \v{8}In answer, Balaam\fnote{Lit. \fbib{he}} told them, ``Stay here for the night and I'll bring back a message\fnote{Lit. \fbib{word}} to you, depending on what the \divine{Lord} says to me.'' So the officers of Moab stayed with Balaam overnight.
\passage{God Forbids Balaam to Cooperate}

\v{9}God visited Balaam and asked him, ``Who are these men with you?''

\v{10}Then Balaam told God, ``Zippor's son Balak, king of Moab, sent them to me and said, \v{11}`Look! A group of\fnote{The Heb. lacks \fbib{group of}} people have escaped from Egypt. They cover the surface of the whole earth! So come right now and curse them for me. Perhaps I'll be able to fight against them and drive them out.'\,''

\v{12}But God told Balaam, ``Don't go with them. Don't curse the people, because they're blessed.''

\v{13}So Balaam got up the next morning and told Balak's officials, ``Go back to your homeland, because the \divine{Lord} has refused me permission to go with you.''

\v{14}So Balak's officials got up, returned to Balak and reported, ``Balaam refused to come with us.''

\v{15}In response, Balak sent more officers---higher ranking ones, at that!--- \v{16}who approached Balaam with this message: ``This is what Zippor's son Balak says: `Don't let anything get in the way of your coming to me. \v{17}I'm determined to reward you generously, and I'll do everything you tell me to do. So come right away and curse this people for me.'\,''

\v{18}Balaam responded to Balak's entourage by saying, ``Even if Balak were to give me his house full of silver and gold, I won't double-cross the command of the \divine{Lord} my God in even the slightest way.\fnote{Lit. \fbib{God to do anything whether insignificant or great}} \v{19}Meanwhile, stay here overnight so I may learn what the \divine{Lord} might say to me.''

\v{20}God came to visit Balaam that same night and told him, ``If the men come to call on you, get up and go with them, but be sure to do only what I tell you to do.'' \v{21}The next morning, Balaam got up, saddled his donkey, and started to leave, accompanied by the Moabite officials.
\passage{Balaam's Donkey Rebukes its Owner}

\v{22}At this, the anger of the \divine{Lord} flared up against Balaam, because he was leaving. So the angel of the \divine{Lord} stood in the way to oppose him. As Balaam\fnote{Lit. \fbib{he}} was riding his donkey, accompanied by two of his servants, \v{23}all of a sudden the donkey saw the angel of the \divine{Lord} standing in the way, with an unsheathed sword in his hand! The donkey turned off the road and went into an open field. Balaam started beating the donkey in order to turn her back to the road, \v{24}but the angel of the \divine{Lord} stood on a narrow path that crossed the vineyards. It had walls on both sides of the path. \v{25}When the donkey saw the angel of the \divine{Lord}, she squeezed herself so close to the wall that Balaam's foot was pressed to the wall. So he beat her again!

\v{26}Then the angel of the \divine{Lord} went along a little further and stood in a much narrower space, where it was impossible\fnote{Lit. \fbib{there's no way}} to turn either right or left. \v{27}When the donkey saw the angel of the \divine{Lord}, she crouched down under Balaam. As a result, Balaam got so angry that he started to whip\fnote{Lit. \fbib{struck}} the donkey with his staff.

\v{28}That's when the \divine{Lord} enabled the donkey to speak.\fnote{Lit. \fbib{\divine{Lord} opened the donkey's mouth}} She asked Balaam, ``What did I do to you that you would beat me in the space of only\fnote{The Heb. lacks \fbib{only}} three footsteps?''

\v{29}``Because you're playing a dirty trick on me,'' Balaam answered the donkey. ``If only I had a sword in my hand! I'd kill you right now!''

\v{30}But in response, the donkey asked Balaam, ``I'm your donkey that you've ridden on in the past without incident,\fnote{The Heb. lacks \fbib{without incident}} am I not, and I'm the same donkey you're riding on right now, am I not? Am I in the habit of treating you like this?''

``No,'' he admitted.

\v{31}Then the \divine{Lord} enabled Balaam to see, so he observed the angel of the \divine{Lord} standing in the way, with an unsheathed sword in his hand. So he bowed down and prostrated himself on his face.

\v{32}Then the angel of the \divine{Lord} asked him, ``Why did you beat your donkey in the space of only\fnote{The Heb. lacks \fbib{only}} three footsteps? I've come to oppose you, because I say that what you're doing is perverted. \v{33}The donkey saw me and turned in front of me in the space of those three footsteps. \v{34}If she hadn't turned away from me, I would have killed you by now and left her alive!''

At this, Balaam replied to the angel of the \divine{Lord}, ``I've sinned! I didn't know that you were standing to meet me on the road. So now, since it displeases you, let me go back.''\fnote{Lit. \fbib{let me return to me}}

\v{35}But the angel of the \divine{Lord} told Balaam, ``Go with the men, but deliver only the message that I'm going to give you.'' So Balaam went with Balak's officials.

\v{36}When Balak heard that Balaam had arrived, he went out to meet him in the city of Moab on the border of Arnon at the extreme end of his territory. \v{37}Balak asked Balaam, ``Didn't I repeatedly send for you to summon you? Why didn't you come to me? I can pay you well,\fnote{Lit. \fbib{can honor you}} can't I?''

\v{38}Balaam answered Balak, ``Well, I'm here now. I've come to you, but I can't just say anything, can I? I'll speak only what God puts in my mouth to say.'' \v{39}So accompanied by Balaam and Balak's officials, Balak traveled to Kiriath-huzoth, \v{40}where he sacrificed oxen and sheep. \v{41}The next day, Balak brought Balaam up to Bamoth-baal, where he could see part of the community of Israel.
\labelchapt{23}
\passage{Balaam's First Sacrifice}

\chapt{23}
\v{1}Balaam told Balak, ``Build for me here seven altars and prepare here for me seven bulls and seven rams.''

\v{2}So Balak did just as Balaam instructed. Balak and Balaam offered a bull and a ram on each altar. \v{3}Then Balaam instructed Balak, ``Stand by your offering and leave me alone by myself. Perhaps the \divine{Lord} will come to meet me. I'll tell you whatever he reveals to me.''

And so he went to a high place, \v{4}where the \divine{Lord} met with Balaam, who told him, ``I've prepared seven altars and offered bulls and rams on an altar.''

\v{5}Then the \divine{Lord} gave Balaam this message. ``Return to Balak and speak to him.''

\v{6}So Balaam returned to where Balak had been standing, that is, next to his offerings, accompanied by all the Moabite officials.
\passage{Balaam's First Prophecy}

\v{7}Then Balaam uttered this prophetic statement:

\begin{poetry}
\poeml ``King Balak of Moab brought me from Aram, \\
\poemll    from the eastern mountains, \\
\poemlll       and told me, \\
\poeml `Come and curse Jacob for me. \\
\poemll    Come and curse Israel.' \\
\poeml \v{8}But how can I curse those whom God hasn't cursed? \\
\poemll    How can I denounce \\
\poemlll       those whom the \divine{Lord} hasn't denounced? \\
\poeml \v{9}I saw them from the top of the rocks. \\
\poemll    I watched them from the hills. \\
\poeml Truly this is a people that lives by itself \\
\poemll    and doesn't matter\fnote{Lit. \fbib{count}} among the nations. \\
\poeml \v{10}Who can count the dust of Jacob? \\
\poemll    Who can number the dust of Israel? \\
\poeml Let me die the death of the righteous, \\
\poemll    and may I end up like him.''
\end{poetry}

\v{11}``What are you doing to me?'' Balak asked Balaam. ``I brought you to curse my enemies, not pronounce a blessing!''

\v{12}But in response, Balaam asked, ``Shouldn't I be careful to communicate only what the \divine{Lord} puts in my mouth?''
\passage{Balaam's Second Sacrifice}

\v{13}``Come with me to another place where you can see them,'' Balak replied. ``You'll only see a portion of them, because you won't be able to see them completely. Come and curse them from there for me.''

\v{14}So Balak\fnote{Lit. \fbib{he}} took him to the field of Zophim, and from there to the top of Mount\fnote{The Heb. lacks \fbib{Mount}} Pisgah, where he built seven altars and then offered a bull and a ram on each altar. \v{15}Then he told Balak, ``Stand by your offering while I go alone to meet the \divine{Lord}.''\fnote{The Heb. lacks \fbib{with the \divine{Lord}}}

\v{16}Then the \divine{Lord} met with Balaam and gave a message to him. ``Now go back to Balak and speak to him.'' \v{17}So Balaam returned to where Balak had been standing, that is, next to his offerings, accompanied by the Moabite officials.

``What did the \divine{Lord} say?'' Balak asked him.
\passage{Balaam's Second Prophecy}

\v{18}In response, Balaam uttered this prophetic statement:

\begin{poetry}
\poeml ``Stand up, Balak, and pay attention! \\
\poemll    Listen to me, you son of Zippor! \\
\poeml \v{19}God is not a human male--- \\
\poemll    he doesn't lie, \\
\poeml nor is he a human being--- \\
\poemll    he never vacillates. \\
\poeml Once he speaks up, \\
\poemll    he's going to act, isn't he? \\
\poeml Once he makes a promise, \\
\poemll    he'll fulfill it, won't he? \\
\poeml \v{20}Look! I've received a blessing, \\
\poemll    and so I will bless. \\
\poemlll       I won't withdraw it. \\
\poeml \v{21}He has not responded to iniquity in Jacob \\
\poemll    or gazed at mischief in Israel. \\
\poeml The \divine{Lord} his God is with them, \\
\poemll    and the triumphant cry of a king is among them. \\
\poeml \v{22}From Egypt God brought them--- \\
\poemll    his strength was like a wild ox! \\
\poeml \v{23}No Satanic plan against Jacob \\
\poemll    nor divination against Israel \\
\poemlll       can ever prevail. \\
\poeml When the time is right, \\
\poemll    it is to be asked about Jacob and Israel, \\
\poemlll       `What has God accomplished?' \\
\poeml \v{24}Look! The people are like lions. \\
\poemll    Like the lion, he rises up! \\
\poeml He does not lie down again \\
\poemll    until he has consumed his prey \\
\poemlll       and drunk the blood of the slain.''
\end{poetry}

\v{25}Then Balak told Balaam, ``Don't curse them or bless them!''

\v{26}``Didn't I tell you,'' Balaam responded to Balak, ``that I'll say whatever the \divine{Lord} tells me to say?''
\passage{Balaam's Third Sacrifice}

\v{27}So Balak exhorted Balaam, ``Let's go right now! I'll take you to another place. Maybe God will agree to have you curse them for me from there.'' \v{28}So Balak took Balaam to the top of Mount\fnote{The Heb. lacks \fbib{Mount}} Peor, which overlooks the open wilderness.\fnote{Lit. \fbib{the Jeshimon}; a desert wasteland not suitable for agriculture or human habitation}

\v{29}Balaam told Balak, ``Build seven altars for me right here. Then prepare seven bulls and seven rams.'' \v{30}Balak did just what Balaam had instructed---he offered a bull and a ram on each altar.
\labelchapt{24}
\passage{Balaam's Third Prophecy}

\chapt{24}
\v{1}When Balaam noticed that the \divine{Lord} was pleased that Balaam was blessing Israel, he didn't behave as he had time after time before, that is, to practice divination. Instead, he turned with his face to the wilderness, \v{2}looked up, and saw Israel encamped in their respective tribal order. Just then, the spirit of God came upon him. \v{3}Balaam uttered this prophetic statement:

\begin{poetry}
\poeml ``A declaration by Beor's son Balaam, \\
\poemll    a declaration by the strong, blind man.\fnote{Lit. \fbib{strong man with a closed eye}} \\
\poeml \v{4}A declaration from one who hears what God has to say, \\
\poemll    who saw the vision that the Almighty revealed, \\
\poeml who keeps stumbling \\
\poemll    with open eyes. \\
\poeml \v{5}Jacob, your tents are so fine, \\
\poemll    as well as your dwelling places,\fnote{Or \fbib{your tents}} O Israel! \\
\poeml \v{6}They're spread out like valleys, \\
\poemll    like gardens along river banks, \\
\poeml like aloe planted by the \divine{Lord}, \\
\poemll    or like cedars beside water. \\
\poeml \v{7}He will pour water from his buckets, \\
\poemll    and his descendants will stream forth like abundant water. \\
\poeml His king will be more exalted than Agag \\
\poemll    when he exalts his own kingdom. \\
\poeml \v{8}God is bringing them\fnote{Lit. \fbib{him}; i.e. national Israel personified as an individual} out of Egypt \\
\poemll    with the strength of an ox. \\
\poeml He'll devour enemy nations, \\
\poemll    break their bones, \\
\poemlll       and impale them with arrows. \\
\poeml \v{9}He crouches, laying low like a lion. \\
\poemll    Who would awaken him? \\
\poeml Those who bless you are blessed, \\
\poemll    and those who curse you are cursed.''
\end{poetry}

\v{10}Balak flew into a rage and he started hitting his fists together. ``I called you to curse my enemies,'' he yelled at Balaam. ``But look here! You've blessed them three times! \v{11}Now get out of here! I had promised you that I would definitely honor you, but now the \divine{Lord} has kept me from doing that!''

\v{12}But Balaam replied to Balak, ``I told your messengers, \v{13}`Even if Balak gives me his palace\fnote{Or \fbib{house}} full of silver and gold, I won't double-cross the command of the \divine{Lord} and do anything---whether good or evil---on my own initiative, because I'm going to say whatever the \divine{Lord} says.' \v{14}Meanwhile, since I have to return to my people, come and listen while I tell you what this people will be doing to your people in the last days.''
\passage{Balaam's Final Prophecies}

\v{15}Then Balaam\fnote{Lit. \fbib{he}} uttered this prophetic statement:

\begin{poetry}
\poeml ``The declaration by Beor's son Balaam, \\
\poemll    a declaration by the strong, blind man. \\
\poeml \v{16}A declaration from one who hears what God has to say, \\
\poemll    who knows what the Most High knows, \\
\poeml who saw the vision that the Almighty revealed, \\
\poemll    who keeps stumbling with open eyes. \\
\poeml \v{17}I can see him, \\
\poemll    but not right now. \\
\poeml I observe him, \\
\poemll    but from a distance.\fnote{Lit. \fbib{but not nearby}} \\
\poeml A star streams forth from Jacob; \\
\poemll    a scepter arises from Israel. \\
\poeml He will crush Moab's forehead, \\
\poemll    along with all of Seth's descendants. \\
\poeml \v{18}Edom will be a conquered nation \\
\poemll    and Seir will be Israel's\fnote{Lit. \fbib{his}} defeated foe, \\
\poemlll       while Israel performs valiantly. \\
\poeml \v{19}He will rule over Jacob, \\
\poemll    annihilating those who survive in the city.''
\end{poetry}

\v{20}Next, Balaam\fnote{Lit. \fbib{he}} looked directly at Amalek and then uttered this prophetic statement:

\begin{poetry}
\poeml ``Even though Amalek is an international leader, \\
\poemll    his future is permanent destruction.''
\end{poetry}

\v{21}Balaam also uttered this prophetic statement about the Kenites:\fnote{I.e. gentile Midianites}

\begin{poetry}
\poeml ``Your dwelling places are stable, \\
\poemll    because your nest is carved in solid rock. \\
\poeml \v{22}Nevertheless, Kain will be incinerated. \\
\poemll    How long will it take until Asshur\fnote{I.e. ancient Assyria} takes you hostage?''
\end{poetry}

\v{23}Finally, he uttered this prophetic statement:

\begin{poetry}
\poeml ``Ah, who can live, \\
\poemll    unless God makes it happen? \\
\poeml \v{24}Ships under control of Kittim will devastate Asshur and Eber, \\
\poemll    until they are permanently destroyed.''
\end{poetry}

\v{25}Then Balaam got up, returned to his country, and Balak went on his way.
\labelchapt{25}
\passage{Worship of Baal of Peor}

\chapt{25}
\v{1}While Israel remained encamped in Shittim, the people began to commit sexual immorality with Moabite women, \v{2}who also invited the people to the sacrifices of their gods. So the people ate what they had sacrificed\fnote{The Heb. lacks \fbib{what they had sacrificed}} and then worshipped their gods. \v{3}The people joined the Baal-peor cult.\fnote{Lit. \fbib{joined themselves to Baal-Peor}; and so throughout the chapter} As a result, the anger of the \divine{Lord} flared up against Israel, \v{4}so the \divine{Lord} told Moses, ``Take all the leaders of the people and execute\fnote{Or \fbib{hang}} them in broad daylight for the \divine{Lord}, so the \divine{Lord}'s burning anger may be withdrawn from Israel.''

\v{5}Then Moses ordered the judges of Israel, ``Each one of you is to execute the men in his own tribe\fnote{The Heb. lacks \fbib{in his own tribe}} who joined the Baal-peor cult.''

\v{6}That very moment, one of the Israelis arrived, bringing to his brothers one of the Midianite women, right in front of Moses and the entire community of Israel, while they were weeping at the entrance to the Tent of Meeting! \v{7}When Eleazar's son Phinehas, grandson of Aaron the priest saw this, he jumped up from the middle of the community, grabbed a javelin in his hand, \v{8}followed the Israeli man inside his tent,\fnote{Or \fbib{inner part of the tent}} and impaled the two of them---the Israeli man and the woman---right through both of them and into her abdomen. Then the plague infecting the Israelis was brought to a halt. Nevertheless, \v{9}24,000 people died because of the plague.
\passage{God Commends Phinehas}

\v{10}The \divine{Lord} told Moses, \v{11}``Eleazar's son Phinehas, grandson of Aaron the priest, has turned my wrath away from Israel. Because his zealousness reflected my own zeal for them, I didn't consume Israel in my jealousy. \v{12}Therefore, I'm certainly going to be giving him my covenant of peace, \v{13}for him and for his descendants after him, too, a covenant of perpetual priesthood, because he was zealous for his God and made atonement for the Israelis.''

\v{14}Now the name of the Israeli man who was slain, along with the Midianite woman, was Salu's son Zimri, a leader from the tribe of Simeon. \v{15}The woman who was slain, that is, the Midianite woman, was named Cozbi. She was the daughter of Zur, a leader\fnote{Lit. \fbib{head}} of one of the ancestral houses of Midian.
\passage{God Orders the Destruction of Midian}

\v{16}Later, the \divine{Lord} ordered Moses, \v{17}``Attack the Midianites and execute them, \v{18}because they've acted deceitfully, bringing trouble to you in this incident at Peor with Cozbi, daughter of a prince from Midian, who was killed during the plague that came about because of the incident at Peor.''
\labelchapt{26}
\passage{The Second Census of Israel}

\chapt{26}
\v{1}After the plague was over, the \divine{Lord} told Moses and Aaron the priest's son Eleazar, \v{2}``Take a census\fnote{Lit. \fbib{Lift the head}} of the entire community of Israel from the age of 20 years and above, according to each ancestral tribe, counting everyone who is able to go out to war in Israel.''

\v{3}Moses and Eleazar the priest spoke to them in the plains of Moab, by the Jordan River\fnote{The Heb. lacks \fbib{River}} in Jericho. \v{4}They counted every male Israeli who had come out of Egypt and who was 20 years old and above, just as the \divine{Lord} had commanded Moses.

\v{5}From Reuben, Israel's firstborn, the descendants of Reuben included from Hanoch, the family of the descendants of Hanoch; from Pallu, the family of the descendants of Pallu; \v{6}from Hezron, the family of the descendants of Hezron; and from Carmi, the family of the descendants of Carmi. \v{7}These families of the descendants of Reuben numbered 43,730.

\v{8}Now Pallu's son was Eliab. \v{9}The descendants of Eliab were Nemuel, Dathan, and Abiram. Dathan and Abiram were removed from the community because they joined the rebellion against Moses and Aaron, as did Korah's company when they rebelled against the \divine{Lord}. \v{10}The ground had opened its mouth and swallowed them up, along with Korah. Also, that group died when the fire devoured 250 men as a warning sign, \v{11}but Korah's direct descendants didn't die.

\v{12}The descendants of Simeon, listed according to their families, included: From Nemuel, the family of the descendants of Nemuel; from Jamin, the family of the descendants of Jamin; from Jachin, the family of the descendants of Jachin; \v{13}from Zerah, the family of the descendants of Zerah; and from Shaul, the family of the descendants of Shaul. \v{14}These families of the descendants of Simeon numbered 22,200.

\v{15}The descendants of Gad, listed according to their families, included: From Zephon, the family of the descendants of Zephon; from Haggi, the family of the descendants of Haggi; from Shuni, the family of the descendants of Shuni; \v{16}from Ozni, the family of the descendants of Ozni; from Eri, the family of the descendants of Eri; \v{17}from Arod, the family of the descendants of Arod; and from Areli, the family of the descendants of Areli. \v{18}These families of the descendants of Gad numbered 40,500.

\v{19}The descendants of Judah originally included Er and Onan, though Er and Onan died in the land of Canaan. \v{20}The descendants of Judah, listed according to their families, included: From Shelah, the family of the descendants of Shelah; from Perez, the family of the descendants of Perez; and from Zerah, the family of the descendants of Zerah. \v{21}The descendants of Perez included: From Hezron, the family of the descendants of Hezron; and from Hamul, the family of the descendants of Hamul. \v{22}These families of Judah numbered 76,500.

\v{23}The tribe of Issachar, listed according to their families, included: From Tola, the family of the descendants of Tola; from Puvah, the family of the descendants of Puvah; \v{24}from Jashub, the family of the descendants of Jashub; and from Shimron, the family of the descendants of Shimron. \v{25}These families of Issachar numbered 64,300.

\v{26}The tribe of Zebulun, listed according to their families, included: From Sered, the family of the descendants of Sered; from Elon, the family of the descendants of Elon; and from Jahleel, the family of the descendants of Jahleel. \v{27}These families of the descendants of Zebulun numbered 60,500.

\v{28}The tribe of Joseph, listed according to their families, included Manasseh and Ephraim. \v{29}The descendants of Manasseh included: From Machir, the family of the descendants of Machir. (Machir was the father of Gilead.) From Gilead, the family of the Gileadites \v{30}included: From Iezer, the family of the descendants of Iezer; from Helek, the family of the descendants of Helek; \v{31}from Asriel, the family of the descendants of Asriel; from Shechem, the family of the descendants of Shechem; \v{32}from Shemida, the family of the descendants of Shemida; and from Hepher, the family of the descendants of Hepher. \v{33}Hepher's son Zelophehad had no sons, but the names of Zelophehad's daughters were Mahlah, Noah, Hoglah, Milcah, and Tirzah. \v{34}These families of Manasseh numbered 52,700.

\v{35}The descendants of Ephraim, listed according to their families, included: From Shuthelah, the family of the descendants of Shuthelah; from Becher, the family of the descendants of Becher; and from Tahan, the family of the descendants of Tahan. \v{36}The descendants of Shuthelah included from Eran, the family of the descendants of Eran. \v{37}These families of Ephraim numbered 32,500. These were the descendants of Joseph, listed according to their families.

\v{38}The tribe of Benjamin, listed according to their families, included: From Bela, the family of the descendants of Bela; from Ashbel, the family of the descendants of Ashbel; from Ahiram, the family of the descendants of Ahiram; \v{39}from Shephupham, the family of the descendants of Shephupham; and from Hupham, the family of the descendants of Hupham. \v{40}The descendants of Bela were Ard and Naaman: From Ard, the family of the descendants of Ard; and from Naaman, the family of the descendants of Naaman. \v{41}These descendants of Benjamin's families numbered 45,600.

\v{42}The tribe of Dan, listed according to their families, included the families of the descendants of Shuham. \v{43}All the families of the Shuhamites numbered 64,400.

\v{44}The tribe of Asher, listed according to their families, included: From Imnah, the family of the descendants of Imnah; from Ishvi, the family of the descendants of Ishvi; and from Beriah, the family of the descendants of Beriah. \v{45}The descendants of Beriah included: From Heber, the family of the descendants of Heber; and from Malchiel, the family of the descendants of Malchiel. \v{46}(The name of Asher's daughter was Serah.) \v{47}These descendants of Asher numbered 53,400.

\v{48}The tribe of Naphtali, listed according to their families, included: From Jahzeel, the family of the descendants of Jahzeel; from Guni, the family of the descendants of Guni; \v{49}from Jezer, the family of the descendants of Jezer; and from Shillem, the family of the descendants of Shillem. \v{50}These families of Naphtali numbered 45,400.

\v{51}The total\fnote{The Heb. lacks \fbib{total}} of those numbered among the Israelis was 601,730.
\passage{Instructions on Dividing the Land}

\v{52}Then the \divine{Lord} told Moses, \v{53}``The land is to be divided for an inheritance according to the total number of these names. \v{54}The more there are in number,\fnote{The Heb. lacks \fbib{number}} you are to increase their inheritance, and the less there are in number, you are to decrease their inheritance. You are to provide an inheritance to each based on the size of their family, \v{55}but the land is to be divided by lot, inheriting according to the names of their ancestor's tribe. \v{56}Depending on the lot, the portion of their inheritance is to be divided between those with more members\fnote{The Heb. lacks \fbib{members}} and those with fewer members.''\fnote{The Heb. lacks \fbib{members}}
\passage{Levitical Genealogies}

\v{57}Those who were numbered from the descendants of Levi, listed according to their families, included: From Gershon, the family of the descendants of Gershon; from Kohath, the family of the descendants of Kohath; and from Merari, the family of the descendants of Merari. \v{58}These were the families of Levi: The family of the descendants of Libni, the family of the descendants of Hebron, the family of the descendants of Mahli, the family of the descendants of Musha, and the family of the descendants of Korah.

Now Kohath had a son named Amram. \v{59}Amram's wife was Levi's daughter Jochebed, who was born to Levi in Egypt. She gave birth to Aaron, Moses, and their sister Miriam.

\v{60}To Aaron were born Nadab, Abihu, Eleazar, and Ithamar. \v{61}But Nadab and Abihu died when they offered unauthorized fire in the \divine{Lord}'s presence. \v{62}All of those individuals numbered 23,000. No male from the age of a month and above was numbered among the Israelis because no inheritance was to be assigned to them among the Israelis.

\v{63}So this has been a list of those who were registered\fnote{Or \fbib{numbered}} by Moses and Eleazar the priest when they numbered the Israelis in the plains of Moab by the Jordan at Jericho. \v{64}But none of these men among these numbered by Moses and by Aaron the priest (that is, when they numbered the Israelis in the wilderness of Sinai) survived to enter the land, \v{65}because the \divine{Lord} had said about them, ``They'll certainly die in the wilderness. No man will survive from them except Jephunneh's son Caleb and Nun's son Joshua.''
\labelchapt{27}
\passage{Zelophehad's Daughters}
\passageinfo{(Numbers 36:1-12)}

\chapt{27}
\v{1}Now the daughters of Hepher's son Zelophehad, Gilead's grandson, who had been fathered by Machir, who had been fathered by Manasseh, from the tribe of Manasseh, the direct son of Joseph, were named Mahlah, Noah, Hoglah, Milcah, and Tirzah. They approached \v{2}Moses, Eleazar the priest, the elders, and the entire community at the entrance to the Tent of Meeting, stood before them, and said, \v{3}``Our father died in the wilderness, but he wasn't with the company of those who gathered against the \divine{Lord} along with the company of Korah. He died in his own sin, and he had no sons. \v{4}Why are you going to eliminate the name of our father from his family, just because he had no son? Give us a possession from among our father's relatives.''

\v{5}So Moses brought the family into the \divine{Lord}'s presence, \v{6}and the \divine{Lord} told Moses, \v{7}``The daughters of Zelophehad are telling the truth. You are certainly to give to them a possession for an inheritance among their father's relatives. You are to pass on the inheritance of their father to them. \v{8}Tell the Israelis that when a man dies without a son, you are to pass his inheritance to his daughter. \v{9}If he doesn't have a daughter, give his inheritance to his brothers. \v{10}If he doesn't have brothers, give his inheritance to his father's brothers. \v{11}If his father doesn't have brothers, then give his inheritance to a relative who is nearest to him from the family and he'll take possession of it. This is to be a permanent ordinance\fnote{Lit. \fbib{a statute, an ordinance}} for the Israelis, just as the \divine{Lord} commanded Moses.''
\passage{Preparations for a Successor to Moses}

\v{12}Then the \divine{Lord} told Moses, ``You are to climb these Abarim mountains and look over the land that I'm going to give the Israelis. \v{13}After you've seen it, you'll be taken to be with your people just as your brother Aaron was gathered to them,\fnote{The Heb. lacks \fbib{to them}} \v{14}because in the wilderness of Zin, when the community rebelled, you rebelled against my command to treat me as holy before their eyes in regards to the Meribah Springs in Kadesh in the wilderness of Zin.''

\v{15}Moses responded to the \divine{Lord}, \v{16}``May the \divine{Lord} God of the spirits of all living creatures appoint a man over the community \v{17}who will go in and out before them, and who will lead them out and bring them in so that the \divine{Lord}'s community won't be like a flock without a shepherd.''
\passage{God Appoints Joshua}

\v{18}``Select Nun's son Joshua. The Spirit is in that man,'' the \divine{Lord} answered Moses. ``You are to lay your hand on him \v{19}and make him stand in front of Eleazar the priest and the entire community. Then you are to set him in charge right before their eyes, \v{20}turning over your authority\fnote{Or \fbib{power}} to him so that the entire community of Israel knows to\fnote{The Heb. lacks \fbib{knows to}} obey him. \v{21}You are to make him stand in the presence of Eleazar the priest, who is to inquire on his behalf using the Urim\fnote{I.e. a part of the priest's breast piece by which God provided revelation; cf. 1Sam 28:6} in the presence of the \divine{Lord} regarding a decision of judgment, because by his command\fnote{Lit. \fbib{mouth}} he and all the Israelis with him will go out or come in.''

\v{22}So Moses did what the \divine{Lord} had commanded him. He took Joshua, made him stand in the presence of Eleazar the priest and the entire community, \v{23}laid his hands on him, and charged him, just as the \divine{Lord} had commanded, using Moses' authority.\fnote{Lit. \fbib{hand}}
\labelchapt{28}
\passage{Daily Offerings}
\passageinfo{(Exodus 29:38-46)}

\chapt{28}
\v{1}The \divine{Lord} told Moses, \v{2}``You are to command the Israelis about my offerings that they are to be sure to bring edible offerings to me, presented by fire, and a pleasing aroma to me, at their appointed time.\v{3}Tell them that this is the offering, presented by fire, that you are to offer to the \divine{Lord}: two one year old lambs, offered daily every day. \v{4}Offer the first lamb in the morning and the second toward the evening,\fnote{Lit. \fbib{between the evenings}; i.e. between the beginning of sunset and the sun's disappearance over the horizon; and so through chapter 29} \v{5}accompanied by one tenth of an ephah\fnote{I.e., an ephah was equal to from \footfract{2}{3} to \footfract{3}{4} of a bushel} of fine flour for grain offering, mixed with one fourth of a hin\fnote{.5 I.e. about one quart; the \fbib{hin} was equivalent to about one gallon} of pure olive oil. \v{6}This burnt offering, which was prescribed at Mount Sinai, is to be offered every day\fnote{Lit. \fbib{offered continuously}} as a pleasing aroma made by fire to the \divine{Lord}.

\v{7}``The drink offering is to be one fourth of a hin\fnote{.7 I.e. about one quart; the \fbib{hin} was equivalent to about one gallon} for each\fnote{Lit. \fbib{the one}} lamb. You are to pour out a drink offering of strong wine to the \divine{Lord} in the Holy Place. \v{8}You are also to offer the second lamb toward the evening. Just like the morning sacrifice,\fnote{The Heb. lacks \fbib{sacrifice}} you are to present the grain offering, accompanied by its corresponding drink offering, as a presentation made by fire, a pleasing aroma to the \divine{Lord}.''
\passage{Sabbath Offerings}

\v{9}``Every Sabbath day, you are to offer two one year old lambs without any defects\fnote{Or \fbib{blemish}} with two tenths of an ephah\fnote{The Heb. lacks the unit of measurement; \fbib{ephah} is assumed through chapter 29; an ephah was equal to from \footfract{2}{3} to \footfract{3}{4} of a bushel} of fine flour for grain offering, mixed with olive oil, along with their corresponding drink offering. \v{10}This burnt offering is to be presented every Sabbath, as well as the regular burnt offering, along with its corresponding drink offering.''
\passage{Monthly Offerings}

\v{11}``On the first day of each month,\fnote{Lit. \fbib{of your months}} you are to offer a burnt offering to the \divine{Lord} consisting of two young bulls, one ram, and seven one year old lambs, all of them without any defects, \v{12}along with three tenths of an ephah of fine flour for a grain offering, mixed with olive oil, for each bull, two tenths of an ephah of fine flour for a grain offering, mixed with olive oil, for the one ram, \v{13}and one tenth of an ephah of fine flour mixed with olive oil as a grain offering for each lamb. This burnt offering will be a pleasing aroma, incinerated as an offering to the \divine{Lord}. \v{14}Their drink offerings are to be half a hin\fnote{.14 I.e. about two quarts; the \fbib{hin} was equivalent to about one gallon} of wine for each bull, one third of a hin\fnote{.14 I.e. about one third of a gallon; the \fbib{hin} was equivalent to about one gallon} for the ram, and one fourth of a hin\fnote{.14 I.e. about one quarts the \fbib{hin} was equivalent to about one gallon} for each lamb. This burnt offering is to be presented each and every month throughout the year. \v{15}One goat is to be offered at regular intervals as a sin offering to the \divine{Lord}, accompanied by its corresponding drink offering.''
\passage{Annual Offerings}

\v{16}``The \divine{Lord}'s Passover is to take place on the fourteenth day of the first month. \v{17}You are to hold a festival on the fifteenth day of this month for seven days, during which time unleavened bread is to be eaten.''
\passage{A Week of Post-Passover Offerings}

\v{18}``On the first day, you are to hold a sacred assembly. No servile work is to be done. \v{19}Bring an offering that is to be incinerated in the \divine{Lord}'s presence, consisting of two young bulls, a ram, and seven one year old lambs, all without any defects, \v{20}along with their grain offering of fine flour mixed with olive oil. Offer three tenths of an ephah for each bull, two tenths of an ephah for the ram, \v{21}and one tenth of an ephah for each of the seven lambs. \v{22}Then present one goat for a sin offering to make atonement for you, \v{23}apart from the burnt offering in the morning, which you are to continue offering. \v{24}Do this every day for seven days, as an edible sacrifice to the \divine{Lord} made by fire, a pleasing aroma. It is to be offered apart from the regular burnt offering and its corresponding drink offering. \v{25}On the seventh day you are to hold another sacred assembly for your benefit, on which no servile work is to be done.''
\passage{First Fruit Offerings}

\v{26}``On the first day of your harvest season, you are to hold a sacred assembly when you present your first fruits during the Festival of\fnote{The Heb. lacks \fbib{Festival of}} Weeks. No servile work is to be done. \v{27}You are to offer this burnt offering as a pleasing aroma to the \divine{Lord}: two young bulls, one ram, and seven one year old lambs, \v{28}along with their corresponding grain offerings of fine flour mixed with olive oil; specifically, three tenths of an ephah for each bull, two tenths of an ephah for the one ram, \v{29}one tenth of an ephah for each of the seven lambs, \v{30}and one goat to make atonement for you. \v{31}Offer them in addition to the regular burnt offering, accompanied by its grain offering and its corresponding drink offerings.''
\labelchapt{29}
\passage{Offerings for the Festival of Trumpets}
\passageinfo{(Leviticus 23:23-25)}

\chapt{29}
\v{1}``You are to hold a sacred assembly on the first day of the seventh month of each year. No servile work is to be done. It's a day of blowing trumpets\fnote{The Heb. lacks \fbib{trumpets}} for you.

\v{2}``You are to bring these burnt offerings as a pleasing aroma to the \divine{Lord}: a one year old young bull, one ram, and seven one year old lambs, all without any defects, \v{3}along with their corresponding grain offering of fine flour mixed with olive oil---three tenths of an ephah for the young bull, two tenths of an ephah for the ram, \v{4}and one tenth of an ephah for each lamb of the seven lambs, \v{5}accompanied by one goat for a sin offering to make atonement for you. \v{6}This is to be separate and apart from the burnt offering for the New Moon, with its corresponding grain offering, the regular burnt offering with its corresponding grain offering, and their drink offerings, according to their respective ordinances, as a pleasing aroma, an incinerated offering made to the \divine{Lord}.

\v{7}``You are to hold a sacred assembly on the tenth day of this same\fnote{The Heb. lacks \fbib{same}} seventh month. You are to humble yourselves,\fnote{Lit. \fbib{afflict your souls}} and no servile work is to be done. \v{8}You are to bring these burnt offerings to the \divine{Lord} for a pleasing aroma: one young bull, one ram, and seven one year old lambs, all without any defects, for you, \v{9}along with these corresponding grain offerings of fine flour mixed with olive oil: three tenths for the bull, two tenths for the one ram, \v{10}and one tenth for each of the seven lambs, \v{11}then one male goat for a sin offering, in addition to the sin offering, to make atonement, along with the regular burnt offering and its corresponding grain and drink offerings.''
\passage{Eight Days of Celebration: Day One}

\v{12}``You are to hold a sacred assembly on the fifteenth day of the same\fnote{The Heb. lacks \fbib{same}} seventh month. No servile work is to be done. You are to celebrate a festival to the \divine{Lord} for seven days by \v{13}bringing these burnt offerings made by fire as a pleasing aroma to the \divine{Lord}: Thirteen young bulls, two rams, and fourteen one year old lambs, all without any defects, \v{14}along with their grain offering of fine flour mixed with olive oil---three tenths for each of the thirteen bulls, two tenths for each of the two rams, \v{15}and one tenth for each of the fourteen lambs, \v{16}accompanied by one goat for a sin offering, in addition to the regular burnt offering, with its corresponding grain and drink offerings.''
\passage{Eight Days of Celebration: Day Two}

\v{17}``On the second day, you are to present twelve young bulls, two rams, and fourteen one year old lambs, all without defects, \v{18}along with corresponding grain and drink offerings for the bulls, rams, and lambs, according to their number, based on\fnote{The Heb. lacks \fbib{based on}} the ordinances, \v{19}and accompanied by one goat for a sin offering, in addition to the regular burnt offering, with its corresponding grain and drink offerings.''
\passage{Eight Days of Celebration: Day Three}

\v{20}``On the third day, you are to present eleven bulls, two rams, and fourteen one year old lambs, all without defects, \v{21}along with corresponding grain and drink offerings for the bulls, rams, and lambs, according to their number, based on the ordinances, \v{22}and accompanied by one goat for a sin offering, in addition to the regular burnt offering, with its corresponding grain and drink offerings.''
\passage{Eight Days of Celebration: Day Four}

\v{23}``On the fourth day, you are to present ten bulls, two rams, and fourteen one year old lambs, all without defects, \v{24}along with corresponding grain and drink offerings for the bulls, rams, and lambs, according to their number, based on the ordinances, \v{25}and accompanied by one goat for a sin offering, in addition to the regular burnt offering, with its corresponding grain and drink offerings.''
\passage{Eight Days of Celebration: Day Five}

\v{26}``On the fifth day, you are to present nine bulls, two rams, and fourteen one year old lambs, all without defects, \v{27}along with corresponding grain and drink offerings for the bulls, rams, and lambs, according to their number, based on the ordinances, \v{28}and accompanied by one goat for a sin offering, in addition to the regular burnt offering, with its corresponding grain and drink offerings.''
\passage{Eight Days of Celebration: Day Six}

\v{29}``On the sixth day, you are to present eight bulls, two rams, and fourteen one year old lambs, all without defects, \v{30}along with corresponding grain and drink offerings for the bulls, rams, and lambs, according to their number, based on the ordinances, \v{31}and accompanied by one goat for a sin offering, in addition to the regular burnt offering, with its corresponding grain and drink offerings.''
\passage{Eight Days of Celebration: Day Seven}

\v{32}``On the seventh day, you are to present seven bulls, two rams, and fourteen one year old lambs, all without defects, \v{33}along with corresponding grain and drink offerings for the bulls, rams, and lambs, according to their number, based on the ordinances, \v{34}and accompanied by one goat for a sin offering, in addition to the regular burnt offering, with corresponding grain and drink offerings.''
\passage{Eight Days of Celebration: Day Eight}

\v{35}``On the eighth day, you are to call a sacred assembly. No servile work is to be done. \v{36}You are to offer these burnt offerings by fire, a pleasing aroma to the \divine{Lord}: one bull, one ram, and seven one year old lambs, all without defects, \v{37}along with corresponding grain and drink offerings for the bull, ram, and lambs, according to their number, based on the ordinances, \v{38}and accompanied by one goat for a sin offering, in addition to the regular burnt offering, with corresponding grain and drink offerings.

\v{39}``Present these to the \divine{Lord} at your appointed festival, in addition to your offerings in fulfillment of vows, free will offerings, burnt offerings, grain offerings, drink offerings, and peace offerings.''

\v{40}\fnote{This v. is 30:1 in MT}Moses instructed the Israelis regarding everything that the \divine{Lord} had commanded him.\fnote{Lit. \fbib{Moses}}
\labelchapt{30}
\passage{Regulations Concerning Vows}

\chapt{30}
\v{1}\fnote{This v. is 30:2 in MT}Later, Moses told the elders of the Israeli tribes, ``This is what the \divine{Lord} has commanded: \v{2}`When a man makes a vow to the \divine{Lord}, or swears an oath---an obligation that is binding to himself---he is not to break his word. Instead, he is to fulfill whatever promise\fnote{Lit. \fbib{words}} came out of his mouth.'\,''
\passage{Vows by Unmarried Women}

\v{3}``When a young woman makes a vow to the \divine{Lord} or pledges\fnote{Lit. \fbib{binds}; and so throughout the chapter} herself\fnote{Lit. \fbib{soul}} to an obligation while she still lives in her father's house, \v{4}and her father hears her vow and the obligations that she had pledged\fnote{Or \fbib{bonded}; and so throughout the chapter} herself to fulfill, yet her father keeps silent about it, then all her vows and every obligation she pledged herself to are to stand. \v{5}But if her father disallows her on the same day that he hears what she has said, then all her vows and every obligation she had pledged herself to fulfill are not to stand. The \divine{Lord} will forgive her, because her father has forbidden her.''
\passage{Vows by Married Women}

\v{6}``If she has a husband and she makes a vow that is binding on herself, or if she makes a hasty vow with her mouth that she pledges herself\fnote{Lit. \fbib{soul}} to fulfill, \v{7}and her husband hears her vow, yet remains silent on the day that he hears it, then her vows are to stand and the obligation to which she had pledged herself is to stand. \v{8}But if, on the same day her husband hears and disallows her, then he has revoked her vows that she made for herself, along with any hasty vows that she spoke and to which she pledged herself to fulfill. The \divine{Lord} will forgive her.''
\passage{Vows by Widows or the Divorced}

\v{9}``Everything that a widow or a divorced woman pledges herself to fulfill are to be binding on her.\fnote{Lit. \fbib{are to stand against her}} \v{10}If, while she had been living in her late or former\fnote{The Heb. lacks \fbib{late or former}} husband's house, she makes a vow or a promise that binds her with an oath, \v{11}and her husband hears it but remains silent, not disallowing it, then all her vows are to stand, along with every obligation that she has pledged to fulfill. \v{12}But if her husband disallowed them the very day that he heard her, everything that she spoke relating to her vows and her obligation to herself are not to stand, because her husband revoked them. The \divine{Lord} will forgive her. \v{13}Her husband may confirm\fnote{Lit. \fbib{make it stand}} or revoke every vow and binding obligation that afflicts her. \v{14}But if her husband remains silent about her from day to day, then he has affirmed all her vows or obligations that she has obligated herself to fulfill.\fnote{The Heb. lacks \fbib{that she has obligated herself to fulfill}} He has affirmed them because he remained silent from the day he heard her vows.\fnote{The Heb. lacks \fbib{her vows}} \v{15}But if he nullified them after he had heard, then he will be responsible for any resulting iniquity.''

\v{16}These are the statutes that the \divine{Lord} commanded Moses concerning a man and his wife and concerning a father and his young daughter while she still lives in her father's house.
\labelchapt{31}
\passage{War against Midian}

\chapt{31}
\v{1}Later, the \divine{Lord} told Moses, \v{2}``Be sure to exact vengeance on behalf of the Israelis from the Midianites, after which you'll be taken home\fnote{Lit. \fbib{be gathered}} to your people.''

\v{3}So Moses instructed the people, ``Muster your men of war to attack the Midianites and deliver the \divine{Lord}'s vengeance against Midian. \v{4}Send 1,000 men to war from every tribe throughout all of Israel.'' \v{5}So 1,000 men from every tribe---12,000 from the thousands of Israel---were mustered and equipped for war.

\v{6}Moses sent 1,000 men from every tribe to fight against them, along with Eleazar's son Phinehas, in whose hands were the articles of the sanctuary and trumpets to sound battle alarms. \v{7}They fought against the Midianites\fnote{Lit. \fbib{Midian}} just as the \divine{Lord} had commanded Moses, killing every man. \v{8}They executed the five kings of Midian, including Evi, Rekem, Zur, Hur, and Reba. They also executed Beor's son Balaam with a sword. \v{9}After this, the Israelis took captive the Midianite women and children\fnote{Or \fbib{little ones}} and confiscated as spoils of war all their cattle, livestock, and their goods. \v{10}They burned every town where they had lived and incinerated all of their encampments. \v{11}They took all the booty and plunder, including both humans and animals. \v{12}Then they brought the captives, booty, and plunder to Moses, to Eleazar the priest, and to the entire community of Israel at the camp on the plains of Moab, by the Jordan River in Jericho. \v{13}Moses and Eleazar the priest and all the leaders of the community went out to meet them outside the camp.
\passage{Commands Concerning War Captives}

\v{14}But Moses became livid with anger at the officers of the army, the captains of thousands, and the captains of hundreds who had returned from servicing in the battle. \v{15}``Did you keep all the women alive?'' Moses asked them. \v{16}``Look! These women were the same ones who were counseled by Balaam to cause the Israelis to commit a grievous sin against the \divine{Lord} at Peor. As a result, that plague infected the \divine{Lord}'s community. \v{17}You are to kill every male child\fnote{Lit. \fbib{every male among the little ones}} and every woman who has had sexual relations with a man.\fnote{Lit. \fbib{every woman who has known a man by lying with him}} \v{18}You are to allow the young women who haven't yet had sexual relations with a man\fnote{Lit. \fbib{little ones among the women, who had not known a man by lying with him}} to live for yourselves.''
\passage{Purification after the Battle}

\v{19}``Now you are to stay outside the camp for seven days, after which any of you who has killed a person\fnote{Lit. \fbib{soul}} or touched someone who was killed may purify yourselves on the third day. You and your captives will be pure on the seventh day. \v{20}Furthermore, you are to purify every garment---that is, everything made of leather, goat's hair, or containing wood.''

\v{21}Eleazar the priest told the soldiers who had gone to battle, ``This is the ordinance of the law that the \divine{Lord} commanded Moses \v{22}concerning anything containing gold, silver, brass, iron, tin, lead, \v{23}or anything else that can survive a refiner's fire: You are to pass it through fire, after which it will be clean. Then it is to be purified with the water of impurity. Everything that cannot survive a refiner's fire is to be washed in water. \v{24}Wash your clothes on the seventh day, after which you will be clean. Then you may enter the camp.''
\passage{Offerings from War Booty}

\v{25}Then the \divine{Lord} told Moses, \v{26}``Take an inventory of the booty that was taken in the battle,\fnote{The Heb. lacks \fbib{in the battle}} both of humans and of animals. Then you, Eleazar the priest, and the leaders of the fathers of the community \v{27}are to divide the booty between the warriors who went to war and the rest of the community.

\v{28}``After this, you are to exact a tribute for the \divine{Lord} from the soldiers who went to war, consisting of the tribute earned by one person out of every 500, whether from people, cattle, donkeys, or flocks. \v{29}You are to take half their share and give it to Eleazar the priest as a raised offering to the \divine{Lord}. \v{30}Then take half the share of the Israelis, one drawn out of every 50 people, cattle, donkeys, flocks, and from every animal, then give to the descendants of Levi who maintain the service of the \divine{Lord}'s tent.''

\v{31}So Moses and Eleazar the priest did what the \divine{Lord} had commanded Moses. \v{32}The goods confiscated in excess of the war implements\fnote{Lit. \fbib{booty}} that the warriors had gathered included 675,000 sheep, \v{33}72,000 cattle, \v{34}61,000 donkeys, and \v{35}32,000 women who had not had sexual relations with a man.
\passage{God's Portion of the War Booty}

\v{36}Now half of the share of those who went to war numbered 337,500 sheep, \v{37}so the \divine{Lord}'s tribute from the sheep totaled 675. \v{38}The cattle numbered 36,000, so the \divine{Lord}'s tribute totaled 72. \v{39}The donkeys numbered 30,500, so the \divine{Lord}'s tribute totaled 61. \v{40}The people\fnote{MT as \fbib{soul of man}} numbered 16,000, so the \divine{Lord}'s tribute totaled 32 people. \v{41}Then Moses gave the tribute, a raised offering to the \divine{Lord}, to Eleazar the priest, just as the \divine{Lord} had commanded Moses. \v{42}From half of the share of the Israelis that Moses had set aside from the soldiers, \v{43}there were 337,500 sheep for the community, \v{44}36,000 cattle, \v{45}30,500 donkeys, \v{46}and 16,000 people.

\v{47}Moses took a portion drawn from every 50 Israelis, including from both human and animals, and gave them to the descendants of Levi who maintained the \divine{Lord}'s tent, just as the \divine{Lord} had commanded him.\fnote{Lit. \fbib{Moses}} \v{48}Then the officers in charge of thousands of soldiers, the captains of thousands, and the captains of hundreds approached Moses \v{49}and told him,\fnote{Lit. \fbib{Moses}} ``Your servants took a count of the soldiers who were under our authority. We didn't miss a single man. \v{50}We've brought offerings to the \divine{Lord} from whatever each man found---jewel-encrusted gold, anklets, bracelets, signet rings, earrings, and necklaces---to make atonement for ourselves\fnote{Or \fbib{our soul}} in the \divine{Lord}'s presence.''

\v{51}Then Moses and Eleazar the priest took the gold from them and everything that was fashioned into jewels. \v{52}The gold for the raised offering that they brought to the \divine{Lord} totaled 16,750 shekels, \v{53}because every soldier had confiscated war booty for his own use. \v{54}Moses and Eleazar took the gold from the captains of thousands and hundreds and brought it to the Tent of Meeting, to serve as a memorial to the Israelis in the \divine{Lord}'s presence.
\labelchapt{32}
\passage{Reuben and Gad Present a Proposal}
\passageinfo{(Deuteronomy 3:12-22)}

\chapt{32}
\v{1}Now, the descendants of Reuben and descendants of Gad happened to be joint owners of a very large herd of cattle. When they observed that Jazer and Gilead were good grazing lands\fnote{The Heb. lacks \fbib{grazing lands}} for cattle, \v{2}the descendants of Gad and descendants of Reuben approached Moses, Eleazar the priest, and the leaders of the community and said, \v{3}``Ataroth, Dibon, Jazer, Nimrah, Heshbon, Elealeh, Sebam, Nebo, and Beon---\v{4}the land that the \divine{Lord} defeated in the sight of the community of Israel---is perfect for cattle and your servants have cattle. \v{5}If we've found favor in your sight, let this land be given to your servants as our possession instead of us crossing the Jordan River.''\fnote{The Heb. lacks \fbib{River}; and so throughout the chapter}

\v{6}``Will your relatives have to go to war while you remain here?'' Moses asked the descendants of Gad and descendants of Reuben in response. \v{7}``Why would you discourage\fnote{Lit. \fbib{discourage the heart}} the Israelis from crossing over to the land that the \divine{Lord} has given them? \v{8}That's what\fnote{Lit. \fbib{thus}} your ancestors did when I sent them from Kadesh-barnea to explore\fnote{Or \fbib{spy}} the land. \v{9}When they arrived in the Eshcol Valley and saw the land, they discouraged\fnote{Lit. \fbib{discouraged the heart}} the Israelis from entering the land that the \divine{Lord} had given them. \v{10}That's why the \divine{Lord}'s anger flared up that day and he promised by an oath that \v{11}`Not one of the men who went up from Egypt, from 20 years old and above, will see the land that I promised to give to their ancestors, that is, to Abraham, Isaac, and Jacob, because none of them followed me wholeheartedly,\fnote{Lit. \fbib{fully}} \v{12}except Jephunneh's son Caleb, the Kenizzite, and Nun's son Joshua. They've wholeheartedly followed the \divine{Lord}.'

\v{13}``The \divine{Lord}'s anger had flared up against Israel so that he made them wander in the wilderness for 40 years until that whole generation, who committed evil in the eyes of the \divine{Lord}, had died. \v{14}And now, look! You're acting just like\fnote{Lit. \fbib{standing in place of}} your ancestors, like a brood\fnote{Or an \fbib{increase}} of sinful men, who are provoking the fierce anger of the \divine{Lord} against the Israelis one step at a time. \v{15}If you stop following him, he will once again abandon them in the wilderness. You'll end up destroying this entire people.''
\passage{A Compromise is Offered}

\v{16}Then they approached him and said, ``Here's where we're going to build corrals for our cattle and cities for our families,\fnote{Or \fbib{little ones}} \v{17}but we will keep ourselves armed and stay ready to go with the Israelis until we've brought them to their own places. Our families intend to live in fortified cities in the presence of the inhabitants of the land, \v{18}but we won't return to our homes until every Israeli has taken possession of each of their inheritances, \v{19}since our inheritance will not be with them across the Jordan River and beyond. Instead, our inheritance is on this side of the Jordan River, facing eastward.''
\passage{The Offer is Accepted}

\v{20}``If you do this,'' Moses replied to them, ``that is, if you equip yourselves for war in the \divine{Lord}'s presence \v{21}and every one of your armed soldiers crosses over the Jordan River in the presence of the \divine{Lord} until he has dispossessed his enemies ahead of him \v{22}and subjugated the land before him,\fnote{Lit. \fbib{before the \divine{Lord}}} then afterwards when you return, you'll be able to stand blameless before the \divine{Lord} and before Israel. This land will then be your possession before the \divine{Lord}. \v{23}``But if you won't do so, look out! You will be sinning against the \divine{Lord}. Be certain of this, that your sin will catch up to you! \v{24}So after you've built cities for your families and corrals for your cattle, be sure to keep your promises.''
\passage{Moses Assigns the Territory}

\v{25}Then the descendants of Gad and descendants of Reuben spoke up. ``Your servants will do exactly what our master has commanded.'' They said. \v{26}``Our children, wives, flocks, and all our cattle will be settled in the cities of Gilead, \v{27}but every soldier that we've equipped for battle will cross the Jordan River\fnote{The Heb. lacks \fbib{the Jordan River}} in the presence of the \divine{Lord}, as our master has spoken.''

\v{28}So Moses instructed Eleazar the priest and Nun's son Joshua, and the officers of the ancestral tribes of the Israelis, \v{29}telling them, ``If the descendants of Gad and descendants of Reuben cross over the Jordan River with you, that is, all of their soldiers who've been equipped for battle in the \divine{Lord}'s presence, so that the land is subjugated right before your eyes, then you are to give them the land of Gilead as their possession. \v{30}But if the armed men don't cross over with you, then they won't have any possession in the land of Canaan.''

\v{31}``We'll do just what the \divine{Lord} told your servants,'' the descendants of Gad and the descendants of Reuben responded. \v{32}``We are to cross over in battle array\fnote{Lit. \fbib{over as armed men}} in the \divine{Lord}'s presence into the land of Canaan, and afterwards the possession of our inheritance will be on this side of the Jordan River.''

\v{33}So Moses gave to the descendants of Gad, to the descendants of Reuben, and to the half-tribe of Joseph's son Manasseh the kingdom of Sihon, the king of the Amorites, and the kingdom of Og, the king of Bashan, the whole land with its cities, and even the territories surrounding it.
\passage{Gad and Reuben Rebuild Their Cities}

\v{34}The descendants of Gad rebuilt Dibon, Ataroth, Aroer, \v{35}Atrothshophan, Jazer, Jogbehah, \v{36}Beth-nimrah, and Beth-haran as fortified cities with corrals for sheep. \v{37}The descendants of Reuben rebuilt Heshbon, Elealeh, Kiriathaim, \v{38}Nebo, Baal-meon (after having changed their names), and Sibmah. The cities that they rebuilt were renamed. \v{39}The descendants of Manasseh's son Machir attacked Gilead and then captured and dispossessed the Amorites who were there. \v{40}That's why Moses gave Gilead to Manasseh's son Machir, who lived there at the time. \v{41}Manasseh's son Jair captured\fnote{Lit. \fbib{went and took}} their towns and renamed them Havvoth-jair. \v{42}Nobah captured Kenath and its towns and renamed it Nobah after himself.
\labelchapt{33}
\passage{Stages of Israel's Journey from Egypt}

\chapt{33}
\v{1}Here's the travel itinerary\fnote{Lit. \fbib{travel in stages}} for the Israelis after they left the land of Egypt in groups under the authority of Moses and Aaron. \v{2}Moses recorded their departures in their travels after being commanded\fnote{Lit. \fbib{mouth}} to do so by the \divine{Lord}. Here's a list of their travels based on\fnote{Lit. \fbib{according to}} their departures:

\v{3}They departed from Rameses in the first month, on the fifteenth day of that first month. The day\fnote{Lit. \fbib{morrow}} after the Passover, the Israelis came out confidently,\fnote{Lit. \fbib{with a high hand}} and all the Egyptians watched them leave, \v{4}while they were burying their firstborn, whom the \divine{Lord} had killed among them. The \divine{Lord} also executed justice against their gods.

\v{5}Then the Israelis traveled from Rameses and rested\fnote{Or \fbib{encamped} and so throughout the chapter} in Succoth.

\v{6}They traveled from Succoth, then rested in Etham, which is at the outskirts of the wilderness.

\v{7}They traveled from Etham but turned back to Pi-hahiroth, which is outside of\fnote{Lit. \fbib{before}; and so throughout the chapter} Baal-zephon.

They rested outside of Migdol. \v{8}They traveled from Hahiroth and passed through the midst of the sea to the wilderness. They were on the road three days in the wilderness of Etham, then rested in Marah.

\v{9}They traveled from Marah and arrived at Elim. In Elim there were twelve wells\fnote{Or \fbib{springs}} of water and 70 palm trees, so they rested there.

\v{10}They traveled from Elim, then rested by the Reed\fnote{So MT; LXX reads \fbib{Red}} Sea.

\v{11}They traveled from the Reed\fnote{So MT; LXX reads \fbib{Red}} Sea, then rested in the Wilderness of Zin.

\v{12}They traveled from the Wilderness of Zin, then rested in Dophkah.

\v{13}They traveled from Dophkah, then rested in Alush.

\v{14}They traveled from Alush, then rested in Rephidim, but there was no water there for the people to drink.

\v{15}They traveled from Rephidim, then rested in the Wilderness of Sinai.

\v{16}They traveled from the Wilderness of Sinai, then rested in Kibroth-hattaavah.

\v{17}They traveled from Kibroth-hattaavah, then rested in Hazeroth.

\v{18}They traveled from Hazeroth, then rested in Rithmah.

\v{19}They traveled from Rithmah, then rested in Rimmon-perez.

\v{20}They traveled from Rimmon-perez, then rested in Libnah.

\v{21}They traveled from Libnah, then rested in Rissah.

\v{22}They traveled from Rissah, then rested in Kehelathah.

\v{23}They traveled from Kehelathah, then rested at Mount Shepher.

\v{24}They traveled from Mount Shepher, then rested in Haradah.

\v{25}They traveled from Haradah, then rested in Makheloth.

\v{26}They traveled from Makheloth, then rested in Tahath.

\v{27}They traveled from Tahath, then rested in Terah.

\v{28}They traveled from Terah, then rested in Mithkah.

\v{29}They traveled from Mithkah, then rested in Hashmonah.

\v{30}They traveled from Hashmonah, then rested in Moseroth.

\v{31}They traveled from Moseroth, then rested in Bene-jaakan.

\v{32}They traveled from Bene-jaakan, then rested in Hor-haggidgad.

\v{33}They traveled from Hor-haggidgad, then rested in Jotbathah.

\v{34}They traveled from Jotbathah, then rested in Abronah.

\v{35}They traveled from Abronah, then rested in Ezion-geber.

\v{36}They traveled from Ezion-geber, then rested in the Wilderness of Zin, which is also known as Kadesh.

\v{37}They traveled from Kadesh, then rested in Mount Hor at the outskirts of the land of Edom.

\v{38}Then Aaron the priest ascended Mount Hor in obedience to the \divine{Lord}'s command and died there, in the fortieth year after the Israelis had come out of the land of Egypt, on the first day of the fifth month. \v{39}Aaron was 123 years old when he died on Mount Hor.

\v{40}Meanwhile, the Canaanite king of Arad, who lived in the Negev\fnote{I.e. the southern regions of the Sinai peninsula; cf. Josh 10:40} in the land of Canaan, heard of the approach of the Israelis, \v{41}who had traveled from Mount Hor and then rested in Zalmonah.

\v{42}They traveled from Zalmonah, then rested in Punon.

\v{43}They traveled from Punon, then rested in Oboth.

\v{44}They traveled from Oboth, then rested in Iye-abarim at the outskirts of Moab.

\v{45}They traveled from Iyim, then rested in Dibon-gad.

\v{46}They traveled from Dibon-gad, then rested in Almon-diblathaim.

\v{47}They traveled from Almon-diblathaim, then rested in the mountains of Abarim, facing Nebo.

\v{48}They traveled from the mountains of Abarim, then rested in the plains of Moab by the Jordan River, across from Jericho.

\v{49}They rested by the Jordan River in the area from Beth-jeshimoth to Abel-shittim in the plains of Moab.

\v{50}Then the \divine{Lord} told Moses in the plains of Moab by the Jordan River, across from Jericho, \v{51}``Tell the Israelis that when they have crossed the Jordan River to the land of Canaan, \v{52}they are to drive out all the inhabitants of the land and destroy all their idols and their molten images. You are to demolish all their high places, \v{53}take possession of the land, and live in it, because I've given you the land to inherit. \v{54}You are to divide the land among yourselves by lot according to your families. The larger the families are in number,\fnote{The Heb. lacks \fbib{the families are in number}} the larger their inheritance is to be. The fewer the families are in number,\fnote{The Heb. lacks \fbib{the families are in number}} the lesser their inheritance is to be. To whomever the lot falls, that inheritance goes to him. Divide it according to your ancestral tribes. \v{55}But if you fail to drive out the inhabitants of the land before you, their survivors will become irritants in your eyes and thorns in your sides, to prick your sides and afflict you in the very land in which you'll be living. \v{56}Then, what I had planned to do to them, I'll start to do to you.''
\labelchapt{34}
\passage{Boundaries of the Land}

\chapt{34}
\v{1}The \divine{Lord} told Moses, \v{2}``Issue these orders to the Israelis: `You're about to enter the land of Canaan. This territory has been apportioned to you as your inheritance: the entire land of Canaan, all the way to its borders.'\,''
\passage{The Southern Border of Israel}

\v{3}```To your south is the Wilderness of Zin, bordering Edom. Your southern border is to extend east toward the far end of the Dead\fnote{Lit. \fbib{Salt}; and so in 34:12} Sea, \v{4}then it is to turn southward to the ascent of Akrabbim, cross Zin, and then run south of Kadesh-barnea and proceed from there to Hazar-addar and across to Azmon. \v{5}Then the border is to turn from Azmon toward the wadi\fnote{I.e. a seasonal stream or river that channels water during rain seasons but is dry at other times} of Egypt and from there to the Mediterranean\fnote{The Heb. lacks \fbib{Mediterranean}} Sea.'\,''
\passage{The Western Border of Israel}

\v{6}```The western\fnote{Lit. \fbib{sea}} border is to be the Mediterranean\fnote{Lit. \fbib{Great}; and so throughout the chapter} Sea. This is to be the western border.'\,''
\passage{The Northern Border of Israel}

\v{7}```Your northern border is to extend from the Mediterranean Sea to Mount Hor. \v{8}From Mount Hor, you are to mark out the entrance to Hammath, with the border running through Zedad, \v{9}then through Ziphron, and then to Hazar-enan. This is to be the northern border.'\,''
\passage{The Eastern Border of Israel}

\v{10}```You are to mark the border on the east from Hazar-enan to Shepham. \v{11}The border is then to extend from Shepham to Riblah, on the east side of Ain, then to the Sea of Chinnereth\fnote{I.e. the Sea of Galilee} on the east. \v{12}The border is to continue along the Jordan River all the way to the Dead Sea. This is to be your land, as measured by its boundaries.'\,''
\passage{Assigning Tribal Responsibilities}

\v{13}Moses commanded the Israelis, ``You are to inherit this land by lot, just as the \divine{Lord} commanded to give it to the remaining\fnote{The Heb. lacks \fbib{remaining}} nine and a half tribes. \v{14}The tribes of Reuben, Gad, and half the tribe of Manasseh, as defined by their ancestral houses, have received their inheritance. \v{15}These two and a half tribes received their inheritance this side of the Jordan River, east of Jericho, facing the rising sun.''

\v{16}Then the \divine{Lord} told Moses, \v{17}``These are the names of the men who are to divide the land for your inheritance: Eleazar the priest and Nun's son Joshua. \v{18}You are to appoint a leader from each tribe to divide the land for inheritance. \v{19}These are the names of the men: Appoint Jephunneh's son Caleb from the tribe of Judah, \v{20}Ammihud's son Shemuel from the tribe of Simeon, \v{21}Chislon's son Elidad from the tribe of Benjamin, \v{22}and Jogli's son Bukki is to be leader of the tribe of Dan. \v{23}From the tribe of Joseph, you are to appoint Ephod's son Hanniel to be leader of the half tribe of Manasseh, \v{24}Shiphtan's son Kemuel to be leader of the half tribe of Ephraim, \v{25}Parnach's son Elizaphan to be leader of the tribe of Zebulun, \v{26}Azzan's son Paltiel to be leader of the tribe of Issachar, \v{27}Shelomi's son Ahihud to be leader of the tribe of Asher, \v{28}and Ammihud's son Pedahel to be leader of the tribe of Naphtali.''

\v{29}These are the ones whom the \divine{Lord} commanded to divide the inheritance of the Israelis in the land of Canaan.
\labelchapt{35}
\passage{Levitical Cities}

\chapt{35}
\v{1}The \divine{Lord} told Moses in the wilderness of Moab, beside the Jordan River near\fnote{Lit. \fbib{up against}} Jericho, \v{2}``Instruct the Israelis to set aside a portion of their inheritance for the descendants of Levi to live in, along with grazing land surrounding their towns. \v{3}The towns are to be reserved for their dwelling places and the grazing lands\fnote{Or \fbib{suburbs}} are to be reserved for their cattle, livestock, and all their animals. \v{4}The grazing lands that you are to reserve for use by the descendants of Levi are to extend 1,000 cubits\fnote{I.e. about 1,500 feet; the cubit was about eighteen inches} from the walls of the town. \v{5}You are to measure from outside the wall of the town on the east side 2,000 cubits,\fnote{I.e. about 3,000 feet; the cubit was about eighteen inches} on the south side 2,000 cubits,\fnote{I.e. about 3,000 feet; the cubit was about eighteen inches} on the west side 2,000 cubits,\fnote{I.e. about 3,000 feet; the cubit was about eighteen inches} and on the north side 2,000 cubits,\fnote{I.e. about 3,000 feet; the cubit was about eighteen inches} with the town placed at the center. This reserved area is to serve as grazing\fnote{Or \fbib{open land}} land for their towns. \v{6}You are to set aside six towns of refuge from the towns that you will be giving to the descendants of Levi, where someone who kills a human being may run for shelter. In addition, give them 42 other towns. \v{7}The total number of towns that you are to give to the descendants of Levi is to be 48 towns, including grazing lands surrounding these towns. \v{8}You are to apportion the towns that you will be giving the Israelis according to the relative size of the tribe. Take a larger portion from those larger in number and a lesser portion from those fewer in number. Each is to set aside towns for the descendants of Levi proportional to the size of their inheritance that they receive.''
\passage{Appointment of Cities of Refuge}

\v{9}Then the \divine{Lord} told Moses, \v{10}``Tell the Israelis that when they have crossed the Jordan River into the land of Canaan, \v{11}they are to designate some towns of refuge so that anyone who kills someone inadvertently may flee there. \v{12}They are to serve as cities of refuge from a blood avenger\fnote{Or \fbib{related redeemer}} in order to keep the inadvertent killer from dying until he has stood trial in the presence of the community. \v{13}You are to set aside six towns of refuge. \v{14}Appoint three towns this side of the Jordan River and three towns in the land of Canaan to serve as the towns of refuge, \v{15}that is, places\fnote{The Heb. lacks \fbib{places}} of refuge for the Israelis, the resident alien,\fnote{Lit. the \fbib{foreigner}} and any travelers among them. Anyone who kills a person inadvertently may flee there.''
\passage{Exceptions to Eligibility}

\v{16}``Whoever uses an iron implement to kill someone is to be adjudged\fnote{The Heb. lacks \fbib{to be adjudged}} a murderer, and that murderer is certainly to be put to death. \v{17}Furthermore, whoever uses a stone implement to kill someone is to be adjudged\fnote{The Heb. lacks \fbib{to be adjudged}} a murderer, and that murderer is certainly to be put to death. \v{18}Also, whoever uses a wooden implement to kill someone with it is to be adjudged\fnote{The Heb. lacks \fbib{to be adjudged}} a murderer, and that murderer is certainly to be put to death. \v{19}The blood avenger himself is to execute the murderer. When he meets him, the blood avenger\fnote{Lit. \fbib{him, he}} is to put him to death. \v{20}If the killer\fnote{Lit. \fbib{If he}} shoved his victim\fnote{Lit. \fbib{shoved him}} out of hatred, or hurled something\fnote{The Heb. lacks \fbib{something}} at him while waiting in ambush so that he died, \v{21}or if he struck him with his hand out of hatred so that he died, then the killer is certainly to be put to death for murder. The avenger of blood is to put him to death when he meets him.''
\passage{Case Examples for Eligibility}

\v{22}``But if he pushed him suddenly without hatred, or if he hurled something\fnote{The Heb. lacks \fbib{something}} in his direction without waiting in ambush, \v{23}or if he hit him\fnote{The Heb. lacks \fbib{or had he hit him}} with a stone carelessly\fnote{Lit. \fbib{stone without seeing it}} so that he was fatally injured, though he isn't his enemy and he wasn't seeking to commit evil against him, \v{24}then the community is to judge between the inadvertent killer and the blood avenger, following these ordinances. \v{25}The community is to release\fnote{Lit. \fbib{deliver}} the inadvertent killer from the blood avenger and return him to the town of refuge where he had fled. He is to live there until the High Priest dies, who will have anointed him with holy oil. \v{26}But if the inadvertent killer leaves the town of refuge where he had fled \v{27}and the blood avenger finds him outside the town of refuge where he had fled and kills him, the blood avenger is not to be found guilty of murder. \v{28}The inadvertent killer\fnote{Lit. \fbib{He}} is to live in the town of refuge until the High Priest dies. After the death of the High Priest, the inadvertent killer is to return to the land of his inheritance. \v{29}These are to be the statutes and ordinances for you throughout all your generations, regardless of where you live.''\fnote{Or \fbib{in all your dwelling places}}
\passage{Capital Cases Require Multiple Witnesses}

\v{30}``Every murderer of a human being\fnote{Or \fbib{soul}} is to be executed only according to testimony\fnote{Lit. \fbib{by the mouth}} given by multiple witnesses. A single witness is not to result in a death sentence.\fnote{Or \fbib{soul}} \v{31}You are to receive no ransom for the life\fnote{Or \fbib{soul}} of a killer who is guilty of murder; instead, he is to die. \v{32}You are not to receive payment of a\fnote{The Heb. lacks \fbib{payment of a}} ransom for someone who had fled to a town of refuge but then left to live in his homeland before the death of the high priest. \v{33}You are not to pollute the land where you live, because blood defiles the land, and the land cannot atone for blood that has been spilled on it, except through the blood of the one who spilled it. \v{34}You are not to defile the land where you will be living, because I'm living among you. I am the \divine{Lord}, who lives in Israel.''
\labelchapt{36}
\passage{The Daughters of Zelophehad}
\passageinfo{(Numbers 27:1-11)}

\chapt{36}
\v{1}The leaders of the ancestral families of the descendants of Gilead, who were descendants of Machir, and descendants of Manasseh, from Joseph's tribe, approached and spoke to Moses and the leaders of the ancestral houses\fnote{Lit. \fbib{the fathers}} of the Israelis. \v{2}``The \divine{Lord} commanded my master\fnote{Or \fbib{lord}} to apportion the land as an inheritance by lot to the Israelis,'' they said. ``Now my master was ordered by the \divine{Lord} to give the inheritance of our brother Zelophehad to his daughters. \v{3}But when they get married to one of the descendants of the tribes of Israel, their inheritances are to be withdrawn from our father's inheritance and added to the inheritance of the tribe to which they are to belong. Consequently, it is to be withdrawn from the portion of our inheritance. \v{4}Then, when the Jubilee Year of the Israelis comes, their inheritance will be added to the inheritance of the tribe to which they have come to belong. Their inheritance will thus be taken away from the inheritance of our father's tribe!''

\v{5}So Moses issued the Israelis these orders based on what the \divine{Lord} said: ``The tribe of the descendants of Joseph has spoken. \v{6}This is what the \divine{Lord} is commanding the daughters of Zelophehad: If they decide it's a good idea in their opinion\fnote{Lit. \fbib{eyes}} to get married only within the family of their father's tribe, then let them get married \v{7}so that the inheritance of the Israelis won't be turned over\fnote{Lit. \fbib{turned aside}} from one tribe to another. Each one has an inheritance from his own father's tribe that the Israelis are to maintain. \v{8}Every daughter who is in possession of an inheritance from the Israelis is to marry someone from the families within her father's tribe so the Israelis can retain possession of their ancestral inheritance. \v{9}That way, their inheritance won't be turned over from one tribe to another, because the Israelis are each to maintain their ancestral inheritances.''

\v{10}Zelophehad's daughters did just what the \divine{Lord} had commanded Moses \v{11}for Mahlah, Tirzah, Hoglah, Milcah, and Noah: Zelophehad's daughters married their uncle's sons. \v{12}They married\fnote{Lit. \fbib{became wives}} into families of the descendants of Manasseh, that is, Joseph's descendants, so that their inheritance remained within the tribe of their ancestor's family.

\v{13}These were the commands and the ordinances that the \divine{Lord} issued to the Israelis through Moses in the plains of Moab by the Jordan River in Jericho.

\bookheader{Deuteronomy}
\labelbook{Deut}

\bookpretitle{The Fifth Book of the Law called}
\booktitle{Deuteronomy}

\labelchapt{1}
\passage{The Setting of the Covenant}

\chapt{1}
\v{1}These are the words that Moses spoke to the assembly of\fnote{Lit. \fbib{to all}} Israel east\fnote{Lit. \fbib{Israel on the other side}; and so throughout the book} of the Jordan River,\fnote{The Heb. lacks \fbib{River}; and so throughout the book} in the Arabah desert, opposite Suph between Paran, Tophel, Laban, Hazeroth, and Di-zahab. \v{2}It takes eleven days to travel\fnote{The Heb. lacks \fbib{to travel}} from Horeb to Kadesh-barnea via Mount Seir.\fnote{This mountain, the modern \fbib{Jebel esh-sher\'{a}}, is located in the mountain range that extends south of the Dead Sea toward the Gulf of Aqaba, and is bordered by the Arabah Valley to the west.} \v{3}On the first day of the eleventh month,\fnote{I.e. the month of Shebat in the Hebrew calendar} in the fortieth year, Moses spoke to the Israelis about everything that the \divine{Lord} had commanded him concerning them. \v{4}This took place\fnote{The Heb. lacks \fbib{This took place}} after he defeated Sihon, king of the Amorites, who lived in Heshbon and Og, king of Bashan, who lived in Ashtaroth at Edrei.
\passage{Moses Reviews God's Instructions}

\v{5}East of the Jordan River, in the land of Moab, Moses began to expound this Law: \v{6}``The \divine{Lord} our God spoke to us in Horeb. He said, `You have been at this mountain long enough. \v{7}Break camp,\fnote{Lit. \fbib{Turn}} get going, and proceed to the hill country of the Amorites and all the nearby places in the Arabah desert, the highlands, the foothills, the Negev,\fnote{I.e. the southern regions of the Sinai peninsula; cf. Josh 10:40} the coastal plains, all of the land of the Canaanites, and Lebanon as far as the great river, the Euphrates. \v{8}Look! I've given you the land that lies ahead. Go in and possess the land that I, the \divine{Lord}, promised to give to your ancestors Abraham, Isaac, and Jacob, as well as to their descendants.'\,''
\passage{Moses Reviews the Selected Officials}

\v{9}``I also told you at that time that I won't be able to sustain you on my own. \v{10}The \divine{Lord} your God greatly multiplied your numbers, and today you are like the stars in the sky. \v{11}May the \divine{Lord}, the God of your ancestors, increase your numbers a thousand times more, and may he bless you, as he promised you. \v{12}How can I bear the burden of you and your bickering all by myself? \v{13}Choose for yourselves wise and discerning men, known to your tribes, and appoint them as your leaders. \v{14}You answered by saying that this plan is a good thing. \v{15}So I chose leaders from your tribes, wise and respected men, and I appointed them over you---commanders of thousands, hundreds, fifties, and tens. \v{16}I charged your judges at that time, `When you hold a hearing between brothers, judge fairly between a man and his brother or between foreigners. \v{17}When you hold a hearing, don't be partial\fnote{Lit. \fbib{don't recognize faces}} in judgment toward the least important or toward the great. Never fear men, because judgment belongs to God. If the matter is difficult for you, bring it to me for a hearing.' \v{18}I charged you at that time that you must do all of these things.''
\passage{Moses Reviews the Sending of the Scouts}
\passageinfo{(Numbers 13:1-15)}

\v{19}``Then we set out from Horeb and walked through that vast and dreadful desert, where you observed the road to the Amorite hill country. Just as the \divine{Lord} our God ordained for us, we finally arrived at Kadesh-barnea. \v{20}I told you at that time, `You have reached the hill country of the Amorites, which the \divine{Lord} our God is about to give us. \v{21}Look! The \divine{Lord} your God has given the land that lies\fnote{The Heb. lacks \fbib{that lies}} before you. Go and possess it, just as the \divine{Lord} God of your ancestors commanded you. Don't be afraid or discouraged.'

\v{22}``Then all of you approached me and said: `Let's send out men in advance of us so they can survey the land and bring back a report to us on how we'll go up to their cities.' \v{23}Because this suggestion\fnote{Lit. \fbib{word}} seemed good to me, I chose twelve men from among you, one from each tribe. \v{24}Then these men set out,\fnote{Lit. \fbib{Then they turned}} went up to the hill county, reached the Eshcol Valley, and surveyed it. \v{25}They hand-picked some of the fruit of the land, brought it down to us, and gave a report that said, `The land which the \divine{Lord} is about to give us is good.'\,''
\passage{Israel Rebels}

\v{26}``However, your ancestors didn't go up. Instead, they rebelled against the command\fnote{Lit. \fbib{mouth}} of the \divine{Lord} your God. \v{27}You murmured in your tents, `The \divine{Lord} hates us. He brought us out of the land of Egypt in order to deliver us to\fnote{Lit. \fbib{to give us into the hands of}} the Amorites so he could destroy us. \v{28}Where can we go? Our brothers discouraged us when they said that the people are bigger and taller than we are. Their cities are tall and fortified to the sky, and we also saw the Anakim\fnote{I.e. a race of giants that formerly populated Canaan; cf. Num 13:22, 33; Deut 9:2} there.'

\v{29}``Then I told you, `Don't be terrified or afraid of them. \v{30}The \divine{Lord} your God is the One who will be going ahead of you. He'll fight for you just as he did in Egypt before your eyes. \v{31}In the desert you saw that the \divine{Lord} carried you like a man carries his son, on every road you traveled until you reached this place.' \v{32}But despite this, you didn't trust in the \divine{Lord} your God, \v{33}who walked ahead of you along the way to scout a place for you to pitch camp---by fire at night and cloud by day---to lead you on the way you should go.''
\passage{Entrance is Denied}

\v{34}``When the \divine{Lord} heard your complaints, he became angry and declared, \v{35}`I swear that not one man of this evil generation will see the good land that I promised to give to your ancestors, \v{36}except Jephunneh's son Caleb. He will see it and I will give to him and to his descendants the land on which he has walked because he wholeheartedly followed the \divine{Lord}.'

\v{37}``The \divine{Lord} was also furious with me because of you. He said: `You will not enter the land.\fnote{Lit. \fbib{land there}} \v{38}However, Nun's son Joshua, your assistant, will go there. Encourage him, for he will cause Israel to take possession of it. \v{39}Your little ones---whom you said would be taken captive---and your children who do not yet\fnote{Lit. \fbib{this day}} know right from wrong will enter the land.\fnote{Lit. \fbib{land there}} I will give it to them and they themselves will possess it. \v{40}But as for you, prepare to set out for the desert on the way to the Reed\fnote{So MT; LXX reads \fbib{Red}} Sea.'

\v{41}``You responded to me and said, `We have sinned against the \divine{Lord}. We will now go up and fight according to what the \divine{Lord} our God commanded.' So each man put on his weapon for battle and recklessly started out for the hill country.''
\passage{The Amorites Defeat Israel}

\v{42}``Then the \divine{Lord} told me: `Tell them not to go up and fight because I will not be in their midst, or else you will be defeated before your enemies.'

\v{43}``I spoke to you but you didn't listen. Instead you rebelled against the command\fnote{Lit. \fbib{mouth}} of the \divine{Lord} and went up to the hill country. \v{44}The Amorites who lived in the hill country came out to engage you in battle. They pursued you like bees do and crushed you from Seir to Hormah. \v{45}You returned and cried out in the \divine{Lord}'s presence, but the \divine{Lord} didn't hear your voice or listen to you. \v{46}You remained in Kadesh for many days. It was a long time, indeed.''
\labelchapt{2}
\passage{Israel Passes through Edomite Territory}

\chapt{2}
\v{1}``We turned and set out for the desert on the road to the Reed\fnote{So MT; LXX reads \fbib{Red}} Sea, just as the \divine{Lord} had directed me. We traveled around Mount Seir for many days. \v{2}Then the \divine{Lord} told me, \v{3}`You've walked around this mountain long enough. Turn northward \v{4}and command this people, ``You are about to pass through the territory of your relatives, the descendants of Esau who live around Seir. They will be afraid of you so be very careful. \v{5}Don't fight them, because I won't give you any part of their land, not even the size of a footprint.\fnote{Lit. \fbib{the treading of the calf of your foot}} I have given Mount Seir to Esau as their property. \v{6}You may buy food to eat and water to drink from them, paying\fnote{The Heb. lacks \fbib{paying}} with cash.''\,' \v{7}Indeed, the \divine{Lord} your God blessed all the works of your hands. He knows about your travels through this vast desert. The \divine{Lord} your God was with you these past 40 years, so that you didn't lack anything. \v{8}So we bypassed our relatives, the descendants of Esau who live in Seir. We turned through the Arabah desert from Elath, and from Ezion-geber we traveled the desert road to Moab.''
\passage{Israel Passes through Moabite Territory}

\v{9}``Then the \divine{Lord} told me, `Don't harass Moab or provoke them to war, because I won't give you any part of their land. I have given Ar to the descendants of Lot as their property. \v{10}(The Emites, a people as powerful, numerous, and tall as the Anakim,\fnote{I.e. a race of giants that formerly populated Canaan; cf. Num 13:22, 33; Deut 9:2} lived there before. \v{11}Like the Anakim,\fnote{I.e. a race of giants that formerly populated Canaan; cf. Num 13:22, 33; Deut 9:2} they were thought of as Rephaim,\fnote{I.e. a race of giants that formerly populated Canaan; cf. Num 13:22, 33} but the Moabites called them Emites. \v{12}The Horites used to live in Seir before the descendants of Esau dispossessed them, exterminated them, and settled there instead, just as Israel will do in the land of its possession, which the \divine{Lord} gave them.) \v{13}Now get going and cross the Wadi\fnote{I.e. a seasonal stream or river that channels water during rain seasons but is dry at other times} Zered.' And so we crossed the Wadi\fnote{I.e. a seasonal stream or river that channels water during rain seasons but is dry at other times} Zered. \v{14}Now from the time we left Kadesh-barnea until we crossed the Wadi\fnote{I.e. a seasonal stream or river that channels water during rain seasons but is dry at other times} Zered was 38 years. All of that generation, the soldiers in the camp, were destroyed just as the \divine{Lord} swore they would be. \v{15}Indeed, the hand of the \divine{Lord} was against them to root them out from the camp until they were utterly destroyed.''
\passage{Israel Passes through Ammonite Territory}

\v{16}``And so all the soldiers among the people died. \v{17}Then the \divine{Lord} spoke to me, \v{18}`Today, you are about to cross the border of Moab at Ar. \v{19}When you come to the Ammonites, don't harass or provoke them to war, for I won't give any part of Ammonite land to you, since I have given it to the descendants of Lot as their property.

\v{20}```(Indeed, it was considered Rephaim\fnote{I.e. a race of giants that formerly populated Canaan} territory, since the Rephaim\fnote{I.e. a race of giants that formerly populated Canaan} used to lived there. The Ammonites called them Zamzummites, \v{21}a great people, numerous, and tall as the Anakim.\fnote{I.e. a race of giants that formerly populated Canaan; cf. Num 13:22, 33; Deut 9:2} But the \divine{Lord} destroyed the Rephaim,\fnote{I.e. a race of giants that formerly populated Canaan} so that the Ammonites dispossessed them and settled there instead. \v{22}This is what he did for the descendants of Esau who live in Seir, when he destroyed the Horites before them. So they dispossessed them and settled there in their place, where they live\fnote{The Heb. lacks \fbib{where they live}} to this day. \v{23}It was the same for the Avvites who lived in villages as far as Gaza. The Caphtorites, who came from Crete,\fnote{Lit. \fbib{Caphtor}} destroyed them and settled there in their place.) \v{24}Get ready and set out for the Wadi\fnote{I.e. a seasonal stream or river that channels water during rain seasons but is dry at other times} Arnon. Look! I've given into your control Sihon the Amorite, king of Heshbon, along with his land. Prepare to take possession by provoking him to war. \v{25}Starting today I will begin to instill fear and terror of you on the part of every nation under heaven who hears reports about you. They'll tremble in anguish before you.'\,''
\passage{Israel Defeats Sihon, King of Heshbon}

\v{26}``I sent messengers from the desert of Kedemoth to King Sihon of Heshbon with this message of peace: \v{27}`Let me pass through your territory. I'll stay on the main road. I won't turn to the right or left. \v{28}Sell me food for cash,\fnote{Lit. \fbib{silver}} so I can eat and give me water for cash,\fnote{Lit. \fbib{silver}} so I can drink. Just let me pass through on foot, \v{29}as the descendants of Esau who live in Seir did for me, as did the Moabites who live in Ar. I'll pass through,\fnote{The Heb. lacks \fbib{I'll pass through}} until I will have crossed the Jordan into the land that the \divine{Lord} our God is about to give us.' \v{30}But King Sihon of Heshbon did not allow us to pass through, because the \divine{Lord} your God had hardened his spirit and made him arrogant,\fnote{Lit. \fbib{and emboldened his heart}} in order to deliver him into your control today.

\v{31}``Then the \divine{Lord} told me, `See, I've begun to deliver Sihon and his territory over to you. Prepare to take possession of his land.'

\v{32}``Sihon came out to meet us, including his entire army, at the battle of Jahaz. \v{33}The \divine{Lord} our God delivered him to us, so we attacked him, his son, and his whole army. \v{34}We captured all his towns at that time. We utterly destroyed every town---the men, the women, and the children---leaving no survivors. \v{35}We only appropriated the livestock for our use, along with plunder from the cities that we captured. \v{36}From Aroer on the edge of Arnon Valley and from the town all the way to Gilead, there was no city that was too strong for us---the \divine{Lord} our God delivered them all to us. \v{37}You did not encroach onto Ammonite land, the banks of the Wadi\fnote{I.e. a seasonal stream or river that channels water during rain seasons but is dry at other times} Jabbok, the towns in the hill country, and all the other places that were forbidden\fnote{Lit. \fbib{commanded}} by the \divine{Lord} our God.''
\labelchapt{3}
\passage{Israel Defeats the King of Bashan}

\chapt{3}
\v{1}``We set out and went up along the road to Bashan. Then King Og of Bashan came out to meet us---he and his whole army---for a battle at Edrei. \v{2}Then the \divine{Lord} told me, `Don't fear him, because I've delivered him, his army, and his territory into your control. Do to him just as you have done to Sihon, king of the Amorites, who lived in Heshbon.'

\v{3}``So the \divine{Lord} our God also delivered into our control King Og of Bashan, along with his whole army. We attacked him until there were no survivors.\fnote{Lit. \fbib{survivors left to him}} \v{4}Then we captured all his cities at that time. There was not a city left that we didn't capture from them---60 cities in all from the region of Argob, which is part of the kingdom of Og in Bashan. \v{5}All of these cities were fortified with high walls, gates, and bars. Furthermore, there were very many unwalled regions. \v{6}We utterly destroyed them, just as we did King Sihon of Heshbon, attacking them in every city---the men, women, and children. \v{7}But we kept for ourselves all of the livestock and plunder from the towns.

\v{8}``So at that time, we took control from the two Amorite kings the territory east of the Jordan from Wadi\fnote{I.e. a seasonal stream or river that channels water during rain seasons but is dry at other times} Arnon to Mount Hermon. \v{9}(The Sidonians called Hermon Sirion, but the Amorites called it Senir.) \v{10}We took control of\fnote{The Heb. lacks \fbib{We took control of}} all the cities of the plain, all of Gilead and Bashan as far as Salecah and Edrei, cities of the kingdom of Og in Bashan. \v{11}Only King Og of Bashan remained from the remnants of the Rephaim.\fnote{I.e. a race of giants that formerly populated Canaan; cf. Num 13:22, 33} In fact, his bed was made of iron. It's in Rabbah of the Ammonites, isn't it? It was nine cubits\fnote{I.e. about thirteen and a half feet long} long and four cubits\fnote{I.e. about six feet} wide.''
\passage{Moses Allots Land East of the Jordan}
\passageinfo{(Numbers 32:1-15)}

\v{12}``Of the land that we captured at that time, I've given its towns to the descendants of Reuben and the descendants of Gad from Aroer near the Wadi\fnote{I.e. a seasonal stream or river that channels water during rain seasons but is dry at other times} Arnon to half of the hill country of Gilead. \v{13}The remainder of Gilead and Bashan of the kingdom of Og, I've given to the half-tribe of Manasseh. (The whole region of Argob---that is, all of Bashan---is called the land of the Rephaim.) \v{14}Manasseh's son Jair captured all the Argob region as far as the territory of the descendants of Geshur and the descendants of Maacath. Bashan was named after him; that's why it is called Havvoth-jair to this day. \v{15}Furthermore, I've given Gilead to Machir. \v{16}And I've given Gilead to the descendants of Reuben and the descendants of Gad as far as the Arnon Valley, designating the middle of the valley as its boundary, including up to the Jabbok River as a boundary with the Ammonites. \v{17}The Arabah and the Jordan River are also a boundary from Chinnereth\fnote{I.e. the Sea of Galilee} to the Sea of the Arabah (that is, the Salt Sea),\fnote{I.e. the Dead Sea} below the slopes of Pisgah on the east.''
\passage{Moses Instructs the Men of War}

\v{18}``Then I commanded you at that time, `The Lord your God gave you this land as a possession. Those equipped for battle---every man a warrior---will cross before your fellow Israelis. \v{19}However, your women, children, and livestock---and I know you have many---may reside in your towns that I gave you \v{20}until the \divine{Lord} grants rest to your fellow Israelis like you. When they take possession of the territory that the \divine{Lord} your God is about to give them on the other side of the Jordan River, then each of you may return to the territory that I've allotted for you.'

\v{21}``I also charged Joshua at that time, `You witnessed everything that the \divine{Lord} your God did to the two kings. Indeed, the \divine{Lord} will do this to all the kingdoms which you are about to enter. \v{22}You are not to fear them, because the \divine{Lord} your God will fight for you.'\,''
\passage{Moses Pleads with God}

\v{23}``I pleaded with the \divine{Lord} at that time, \v{24}`\divine{Lord} God, you've begun to show your greatness and your strong power to your servant. For what god in heaven or on earth can equal your works and mighty deeds? \v{25}Let me cross over that I may see the good land on the other side of the Jordan River---the good hill country---as well as Lebanon.'

\v{26}``However, the \divine{Lord} was furious with me because of you. He did not listen to me. Instead, the \divine{Lord} said, `You are not to speak to me about this matter again! \v{27}Go up to the top of Pisgah and lift your eyes toward the west, north, south, and east. Look with your own eyes, since you won't be able to cross this Jordan River. \v{28}Therefore charge Joshua to be doubly strong, because he will lead this people\fnote{Lit. \fbib{He will cross over before this people}.} and cause them to inherit the land that you'll see.' \v{29}We then encamped in the valley opposite Beth-peor.''
\labelchapt{4}
\passage{Moses Presents the Privileges of the Covenant}

\chapt{4}
\v{1}``Now, Israel, listen to the statutes and the ordinances that I'm teaching you to observe so you may live and go in to take possession of the land that the \divine{Lord}, the God of your ancestors, is about to give you. \v{2}Do not add or subtract a thing to what I'm commanding you. Observe the commands of the \divine{Lord} your God.\fnote{Lit. \fbib{God that I'm commanding you}} \v{3}You saw with your own eyes what he did in Baal Peor. The \divine{Lord} your God exterminated from among you every man who followed Baal of Peor. \v{4}But all of you who are clinging to the \divine{Lord} your God are alive today. \v{5}See! I taught you the statutes and the ordinances, just as the \divine{Lord} God commanded. Therefore, observe them\fnote{The Heb. lacks \fbib{them}} when you enter the land you are about to possess. \v{6}Observe them carefully, because this will show your wisdom and discernment in the eyes of people who'll listen to all these decrees. Then they'll say: `Surely this great nation is a wise and discerning people.' \v{7}For what great nation has a god so near like the \divine{Lord} our God whenever we call on him? \v{8}And what great nation has all the decrees and righteous ordinances like all this teaching that I'm giving you today? \v{9}Only guard yourselves carefully so you won't forget the things that you saw and let them slip from your mind for the rest of your life. Tell them to your children and to your grandchildren. \v{10}The day you stood in the presence of the \divine{Lord} your God in Horeb, the \divine{Lord} told me, `Gather the people before me so they may hear my words, learn to revere me the whole time that they live in the land, and teach them\fnote{The Heb. lacks \fbib{them}} to their children.'\,''
\passage{Moses Warns against Idolatry}

\v{11}``When you approached and stood at the foot of the mountain---a mountain that was blazing with fire at its core\fnote{Lit. \fbib{heart}} while the sky was covered with thick, dark clouds--- \v{12}the \divine{Lord} your God spoke from the midst of the fire. You heard the sound of words, but you saw no form; there was only a voice. \v{13}He declared to you his covenant, which he commanded you to observe---the Ten Commandments that he wrote on two stone tablets. \v{14}The \divine{Lord} commanded me at that time to teach you to observe the statutes and ordinances in the land after you cross over to take possession of it.

\v{15}``Therefore, for your own sake, be very careful, since you did not see any form on the day that the \divine{Lord} your God spoke to you in Horeb from the midst of the fire. \v{16}Be careful!\fnote{The Heb. lacks \fbib{be careful}} Otherwise, you will be destroyed when you make carved images for yourself---all sorts of images in the form of man, woman, \v{17}any animal on earth, any winged bird that flies in the sky, \v{18}any creeping thing on the ground, or any fish in the sea.\fnote{Lit. \fbib{in the waters below ground}} \v{19}Do not gaze toward the heavens and observe the sun, the moon, the stars---the entire array of the sky---with the intent\fnote{The Heb. lacks \fbib{with the intent}} to worship and serve what the \divine{Lord} your God gave every nation.\fnote{I.e. as lights in the night sky; cf. Gen 1:14-18} \v{20}For the \divine{Lord} took you and brought you out of the iron-smelting furnace---out of Egypt---to be the people of his inheritance, as you are today.

\v{21}``But the \divine{Lord} was angry with me because of you. So he swore that I'll never cross the Jordan River to enter the good land that the \divine{Lord} your God is about to give you as an inheritance. \v{22}I'm going to die in this land and I won't cross the Jordan River, but you're about to cross over to possess that good land. \v{23}Be careful! Otherwise, you will forget the covenant of the \divine{Lord} your God, who established that covenant with you. Don't make carved images of any likeness in violation of everything that you were commanded by the \divine{Lord} your God. \v{24}Indeed, the \divine{Lord} your God is a consuming\fnote{So MT; LXX reads \fbib{is an all-consuming}} fire. He is a jealous God.''
\passage{Warnings against Angering God}

\v{25}``After you've borne children and grandchildren, have been there for a long time in the land, have become so corrupted that you make images of any form, and have done evil in the eyes of the \divine{Lord} your God, you will provoke him to anger. \v{26}Heaven and earth will testify against what has occurred\fnote{Lit. \fbib{I invoke heaven and earth against you}} today: you'll surely and swiftly be destroyed from the land that you are about to possess by crossing the Jordan River. You won't live long in it, because you'll certainly be exterminated. \v{27}Moreover, the \divine{Lord} will scatter you among the nations, and you'll be fewer in number in the nations where the \divine{Lord} your God will drive you. \v{28}There you'll serve gods made by human hands, serving\fnote{The Heb. lacks \fbib{serving}} trees and stones that cannot see, hear, eat, nor smell. \v{29}If from there you will seek the \divine{Lord} your God, then you will find him if you seek him with all your heart and soul. \v{30}In your distress, when all these things happen to you in days to come and you return to the \divine{Lord} your God, then you will hear his voice. \v{31}For God is compassionate. The \divine{Lord} your God won't fail you. He won't destroy you or forget the covenant that he confirmed with your ancestors.''
\passage{Who is Like the \divine{Lord}?}

\v{32}``Indeed, ask from one end of the heavens to the other about days of old, before your time, when God created mankind on the earth. Did we ever have anything as great as this, or ever hear of anything like it? \v{33}Has any people heard the voice of God speaking from the middle of a fire just as you did,\fnote{Lit. \fbib{heard}} and survived it? \v{34}Or has any god ever taken for himself one nation out from another nation with testings, signs, wonders, wars, awesome power,\fnote{Lit. \fbib{wars, a mighty hand and an outstretched arm,}} and magnificent, terrifying deeds\fnote{The Heb. lacks \fbib{deeds}} as the \divine{Lord} your God did in Egypt before your eyes?

\v{35}``You have been shown this in order to know that `the \divine{Lord} is God' and there is no one like him. \v{36}You have been made to hear his voice from heaven so you may be instructed.\fnote{Or \fbib{disciplined}} And he showed you his great fire here on earth, and you heard his voice from the middle of that fire. \v{37}Moreover, he loved your ancestors, chose their descendants after them, and brought you out of Egypt, accompanied by his presence and great power, \v{38}in order to drive out nations that are stronger and more powerful than you, to bring you into this land,\fnote{The Heb. lacks \fbib{this land}} and to give you their land as an inheritance, as it is today.

\v{39}``May you acknowledge and take to heart this day that the \divine{Lord} is God in the heavens above and over the earth below---there is no other God.\fnote{The Heb. lacks \fbib{God}} \v{40}May you observe his statutes and keep his commands that I'm giving you today, so that life may go well for you and for your descendants after you. That way, you'll live a long life in the land that the \divine{Lord} your God is about to give you permanently.''\fnote{Lit. \fbib{all the days}}
\passage{Cities of Refuge}

\v{41}Then Moses designated three cities on the east side of the Jordan, \v{42}where a person who accidentally killed someone could flee, if he killed his neighbor without having enmity toward him in the past. He may flee to one of these cities and live: \v{43}Bezer in the desert plain for the descendants of Reuben, Ramoth in Gilead for the descendants of Gad, and Golan in Bashan for the descendants of Manasseh.
\passage{Moses Reviews the Law}

\v{44}This is the Law that Moses reviewed in the presence of the Israelis. \v{45}These are the instructions, decrees, and ordinances that Moses declared to the Israelis when they came out of Egypt. \v{46}He did this\fnote{The Heb. lacks \fbib{He did this}} east of the Jordan, in the valley opposite Beth Peor, in the land of Sihon, king of the Amorites, who lived in Heshbon, and whom Moses and the Israelis defeated after leaving Egypt. \v{47}So they took possession of his land, as well as the land of King Og of Bashan. Both Amorite kings lived east of the Jordan---\v{48}from Aroer on the edge of the Wadi\fnote{I.e. a seasonal stream or river that channels water during rain seasons but is dry at other times} Arnon as far as Mount Sirion,\fnote{MT reads \fbib{Sion}; cf. Deut 3:9} which is also called Hermon, \v{49}and all the Arabah east of the Jordan as far as the Dead Sea\fnote{Lit. \fbib{the Sea of the Arabah}} below the slopes of Pisgah.
\labelchapt{5}
\passage{The Ten Commandments}
\passageinfo{(Exodus 20:1-17)}

\chapt{5}
\v{1}Moses called all of Israel together and told them: ``Listen, Israel! Today I'm going to announce God's laws and regulations so that you will learn them and take care to obey them. \v{2}When the \divine{Lord} our God made a covenant with us in Horeb, \v{3}it was not with our ancestors that the \divine{Lord} made this covenant, but with us---we who are here today---all of us who are now living. \v{4}The \divine{Lord} spoke to you face to face on the mountain from the fire. \v{5}I stood at that time as mediator\fnote{The Heb. lacks \fbib{as mediator}} between the \divine{Lord} and you to declare his\fnote{Lit. \fbib{the \divine{Lord}'s}} message to you, because you were afraid of the fire and would not go up the mountain. He said:
\begin{bulletlist}
\itemb{\heb{'}\fnote{The Heb. letters to the left denote numbers 1-10}} \v{6}```I am the \divine{Lord} your God, who brought you out of the land of Egypt---from the house of slavery. \v{7}You are to have no other gods as a substitute for me.\fnote{Lit. \fbib{gods besides me}}
\itemb{\heb{b}} \v{8}```You are not to craft for yourselves an idol resembling what is in the skies above, or on earth beneath, or in the water sources under the earth. \v{9}You are not to bow down to them in worship or serve them, because I, the \divine{Lord} your God, am a jealous God, visiting the guilt of parents\fnote{Lit. \fbib{fathers}} on children, to the third and fourth generation\fnote{So LXX. The Heb. lacks \fbib{generation}} of those who hate me, \v{10}but showing gracious love to the thousands of those who love me and keep my\fnote{The MT is written \fbib{his} but is to be read \fbib{my}} commandments.
\itemb{\heb{g}} \v{11}```You are not to misuse the name of the \divine{Lord} your God,\fnote{Lit. \fbib{to take in vain the name of the \divine{Lord} your God}; i.e. for a worthless purpose} because the \divine{Lord} will not leave unpunished the one who misuses his name.\fnote{Lit. \fbib{who takes his name in vain} i.e. for a worthless purpose}
\itemb{\heb{d}} \v{12}```Observe the Sabbath day, maintaining its holiness,\fnote{Lit. \fbib{day as holy;} i.e. to set apart the day as holy} just as the \divine{Lord} your God commanded. \v{13}Six days you are to labor and do all your work, \v{14}but the seventh day is a Sabbath to the \divine{Lord} your God. You are not to do any work---neither you, your son, nor your daughter,\fnote{Lit. \fbib{your sons and your daughters}} your male and female servants, your oxen and donkeys, nor any of your livestock, nor any foreigner who lives among you---\fnote{Lit. \fbib{lives within your gates}} so that your male and female servants may rest as you do. \v{15}You are to remember that you were a slave in the land of Egypt, but the \divine{Lord} your God brought you out from there with great power and a show of force.\fnote{Lit. \fbib{with a mighty hand and an outstretched arm}} Therefore, the \divine{Lord} your God has commanded you to observe the Sabbath day.
\itemb{\heb{h}} \v{16}```Honor your father and your mother, just as the \divine{Lord} your God commanded you, so that you will live long and things will go well for you in the land that the \divine{Lord} your God is giving you.
\itemb{\heb{w}} \v{17}```You are not to commit murder.
\itemb{\heb{z}} \v{18}```You are not to commit adultery.
\itemb{\heb{.h}} \v{19}```You are not to steal.
\itemb{\heb{.t}} \v{20}```You are not to give false testimony against your neighbor.
\itemb{\heb{y}} \v{21}```You are not to desire\fnote{Lit. \fbib{to covet}; i.e. to set your heart on} your neighbor's wife nor crave your neighbor's house,\fnote{Or \fbib{neighbor's family dynasty}} his fields, his male and female servants, his ox, his donkey, nor anything else that pertains to your neighbor.'\,''
\end{bulletlist}
\passage{Moses Recalls God's Warnings}

\v{22}``The \divine{Lord} declared these commands in a loud voice to your entire assembly on the mountain from out of the fire\fnote{LXX Sam Pentateuch read \fbib{dark}; cf. Deut 4:11} and dark clouds,\fnote{Lit. \fbib{cloud and thick darkness}} and nothing more was added. He inscribed them on two tablets of stone and gave them to me. \v{23}When you heard the voice from the darkness while the mountain was blazing, all the leaders and elders of your tribes came to me and said: \v{24}`The \divine{Lord} our God truly has displayed his glory and power, for we heard him\fnote{The Heb. lacks \fbib{him}} from out of the fire today. We have witnessed how God spoke to human beings, yet they lived. \v{25}Now therefore, why should we die? This great fire will consume us. If we continue to listen to the voice of the \divine{Lord} our God any longer, we'll die. \v{26}For what mortal man\fnote{Lit. \fbib{For who among all flesh}} has heard the voice of the living God speaking out of the fire like we did, and lived? \v{27}As for you, go near and listen to everything that the \divine{Lord} our God will say to you, then repeat it\fnote{Lit. \fbib{Then tell everything that the \divine{Lord} our God will speak}} to us, and we'll listen and obey.'

\v{28}``The \divine{Lord} heard what you said. He told me: `I've heard what this people said. Everything they said was good. \v{29}If only they would commit\fnote{Lit. \fbib{only their heart would incline}} to fear me and keep all my commands, then it will go well with them and their children forever.\v{30}Go and tell them to return to their tents, \v{31}but you stand here with me and I'll speak to you all the commands, decrees, and laws that you must teach them to observe in the land that I'm giving you to possess. \v{32}You must be careful to do what the \divine{Lord} your God commanded you, turning neither to the left nor to the right. \v{33}You are to walk in every pathway that the \divine{Lord} your God commanded you, so that life\fnote{Lit. \fbib{it}} may go well for you, and so that you will prolong your days in the land that you will possess.'\,''
\labelchapt{6}
\passage{The Covenant of Love}

\chapt{6}
\v{1}``Now these are the commands, decrees, and ordinances that the \divine{Lord} commanded me\fnote{The Heb. lacks \fbib{me}} to teach you. Obey them in the land you are entering to possess, \v{2}so that you, your children, and your grandchildren may fear the \divine{Lord} your God. Keep all his decrees and commandments that I'm giving you every day of your life, so you may live a long time. \v{3}Listen, Israel! Be careful to obey, so that life\fnote{Lit. \fbib{it}} may go well for you and that you may increase greatly. Just as the \divine{Lord} God of your ancestors told you, you'll have a land flowing with milk and honey.

\v{4}``Listen, Israel! The \divine{Lord} is our God, the \divine{Lord} alone.\fnote{Or \fbib{The \divine{Lord} our God, the \divine{Lord} is one.}} \v{5}You are to love the \divine{Lord} your God with all your heart, all your soul, and all your strength. \v{6}Let these words that I'm commanding you today be always\fnote{The Heb. lacks \fbib{always}} on your heart. \v{7}Teach them repeatedly to your children. Talk about them while sitting in your house or walking on the road, and as you lie down or get up. \v{8}Tie them as reminders\fnote{Lit. \fbib{signs}} on your forearm, bind them on your forehead,\fnote{Lit. \fbib{them as frontlets between your eyes}} \v{9}and write them on the door frames of your house and on your gates.''
\passage{Serve the \divine{Lord} Only}

\v{10}``When the \divine{Lord} your God brings you to the land that he promised to your ancestors Abraham, Isaac, and Jacob, he will give you large and beautiful cities that you didn't build, \v{11}houses filled with every good thing that you didn't supply, wells that you didn't dig, and vineyards and olive groves that you didn't plant. When you eat and are satisfied, \v{12}be careful not to forget the \divine{Lord} your God, who brought you out of the land of Egypt and slavery.\fnote{Lit. \fbib{Egypt, out of the house of slavery}} \v{13}Fear the \divine{Lord} your God, serve him, and make your oaths in his name. \v{14}Do not follow other gods from the gods of the nations\fnote{Lit. \fbib{peoples}} around you, \v{15}because the \divine{Lord} your God who is among you is a jealous God. He will turn his anger against you and destroy you from the surface of the land.''
\passage{Do What is Right}

\v{16}``Don't test the \divine{Lord} your God like you did in Massah. \v{17}Be sure to observe the commands of the \divine{Lord} your God, his testimonies, and his decrees that he gave you. \v{18}Do what is good and right in the \divine{Lord}'s sight so it may go well with you. Then you'll enter and possess the good land that the \divine{Lord} your God promised to your ancestors, \v{19}expelling all your enemies before you, as the \divine{Lord} said.''
\passage{Remember What the \divine{Lord} has Done}

\v{20}``When your son asks you in the future, `What is the meaning of the instructions, decrees, and ordinances that the \divine{Lord} our God commanded you?' \v{21}tell him, `We were slaves to Pharaoh in Egypt, but the \divine{Lord} brought us out of Egypt with great power. \v{22}Before our very eyes, the \divine{Lord} did great and terrible signs and wonders in Egypt---to Pharaoh and to his entire household. \v{23}But as for us, he brought us out from there to bring us into the land and give it to us, as he promised our ancestors. \v{24}Then the \divine{Lord} commanded us to observe all these decrees and to fear the \divine{Lord} our God for our own good, so that he may keep us alive as we are today. \v{25}It will be credited as\fnote{The Heb. lacks \fbib{credited as}} righteousness for us if we're careful to obey the entire Law in the presence of the \divine{Lord} our God, as he commanded.'\,''
\labelchapt{7}
\passage{Instructions Regarding the Tribal Nations}

\chapt{7}
\v{1}``When the \divine{Lord} your God brings you into the land that you are entering to possess, he will drive out many nations before you: the Hittites, Girgashites, Amorites, Canaanites, Perizzites, Hivites, and Jebusites--- seven nations who are more numerous and stronger than you. \v{2}So when the \divine{Lord} your God delivers them to you and you have defeated them, then utterly destroy them. You are not to make any covenant with them nor be gracious to them. \v{3}You are not to intermarry with them. You are not to give your daughters to their sons nor take their daughters for your sons, \v{4}because they will turn your children from me to serve other gods so that the \divine{Lord}'s anger blazes against you and swiftly destroys you by fire. \v{5}This is what you are to do to them: tear down their altars, break their pillars, cut down their ritual pillars, and burn their carved idols in fire, \v{6}because you are a holy people to the \divine{Lord} your God. The \divine{Lord} your God chose you to be his people, his treasured possession from all the nations\fnote{Lit. \fbib{peoples}} on the face of the earth.''
\passage{The \divine{Lord} Keeps His Covenant}

\v{7}``It wasn't because you were more numerous than other nations\fnote{Lit. \fbib{peoples}} of the earth that the \divine{Lord} committed himself to you and chose you. In fact, you were the least numerous of all the nations.\fnote{Lit. \fbib{peoples}} \v{8}But the \divine{Lord} loved you and kept his oath that he made to your ancestors. The \divine{Lord} brought you out with great power from slavery,\fnote{Lit. \fbib{from the house of slaveries}} from the control of Pharaoh, king of Egypt. \v{9}Know that the \divine{Lord} your God is God, the trusted God who faithfully keeps his covenant to the thousandth generation of those who love him and obey his commands. \v{10}But for the one who hates him, he will repay him by destroying him. He will not delay dealing with someone who hates him. \v{11}Therefore, keep the commands, decrees, and ordinances that I am instructing you to obey today.''
\passage{The \divine{Lord} Blesses Obedience}

\v{12}``If you pay attention to these laws and obey them, then the \divine{Lord} your God will continue his covenant of gracious love with you that he promised with an oath to your ancestors. \v{13}He'll love you and increase your numbers. He'll bless the fruit of your womb, the fruit of your land (the grain, new wine, and oil), the offspring of your herds, and the lambs of your flock in the land that the \divine{Lord} promised your ancestors he would give you. \v{14}You'll be blessed among all the nations. There'll be no infertility among you, not even\fnote{Lit. \fbib{you, neither infertility}} among your herds. \v{15}The \divine{Lord} will turn aside every disease from you. He won't inflict on you the terrible diseases you knew in Egypt, but will inflict them instead on all who hate you. \v{16}You are to utterly destroy everyone whom the \divine{Lord} your God will deliver to you. Don't have pity on them nor serve their gods. Otherwise, they will become a snare for you.''
\passage{The \divine{Lord} will Fight for You}

\v{17}``You may say to yourselves, `These nations are more numerous than we are. How can we dispossess them?' \v{18}But you mustn't fear them. Be sure to remember what the \divine{Lord} your God did to Pharaoh and all of Egypt. \v{19}Your eyes saw the great trials, the signs and wonders, and the awesome power with which\fnote{Lit. \fbib{wonders, with a mighty hand and an outstretched arm}} the \divine{Lord} your God brought you out. The \divine{Lord} your God will do the same to all the people whom you fear. \v{20}He'll\fnote{Lit. \fbib{The \divine{Lord} your God will}} send plagues against them until the survivors who hide from you have perished. \v{21}Don't tremble before them, because the \divine{Lord} your God, who is among you, is a great and awesome God. \v{22}He\fnote{Lit. \fbib{The \divine{Lord} your God}} slowly will dislodge these nations before you, but he won't destroy them quickly, so the wild animals\fnote{Lit. \fbib{the beasts of the field}} won't multiply around you. \v{23}But the \divine{Lord} your God will deliver them over to you, throwing them into great confusion, until they are destroyed. \v{24}He will deliver kings into your control and you are to wipe out the memory of them\fnote{Lit. \fbib{will cause their names to perish}} from under heaven. No one will be able to stand before you. You are utterly to destroy them. \v{25}Burn the images of their gods in the fire. Desire neither the silver nor the gold that adorns them, nor take them for yourselves, so you won't be ensnared by them, because the gold and silver\fnote{Lit. \fbib{because it}} are detestable to the \divine{Lord} your God. \v{26}Don't bring any detestable thing to your house, because you yourself will be utterly destroyed along with these detestable things. You must absolutely abhor and detest all of\fnote{The Heb. lacks \fbib{all of}} it, because it has been devoted to destruction.''
\labelchapt{8}
\passage{Remember the \divine{Lord}'s Provisions}

\chapt{8}
\v{1}``Be careful to observe every command that I'm instructing you today, in order that you may live, increase, and enter and take possession of the land that the \divine{Lord} promised by an oath to your ancestors. \v{2}Remember how the \divine{Lord} your God led you all the way these 40 years in the desert to humble\fnote{Or \fbib{afflict}} and test you in order to make known what was in your heart---whether or not you would keep his commands. \v{3}He humbled\fnote{Or \fbib{afflicted}} you, causing you to be hungry, yet he fed you with manna that neither you nor your ancestors had known, in order to teach you that human beings are not to live by food alone---instead human beings are to live by every word that proceeds from the mouth of the \divine{Lord}.

\v{4}``The clothes you wore\fnote{Lit. \fbib{clothes from on you}} did not wear out, nor did your feet blister during these 40 years. \v{5}Be convinced in your heart that as a father disciplines his son, so the \divine{Lord} your God disciplines you. \v{6}Observe the commands of the \divine{Lord} your God by walking in his ways and by fearing\fnote{Or \fbib{revering}} him, \v{7}because the \divine{Lord} your God is bringing you to a good land---a land with rivers and deep springs flowing to the valleys and hills. \v{8}It's a land filled\fnote{The Heb. lacks \fbib{filled}} with wheat, barley, vines, fig trees, and pomegranates. It's a land filled\fnote{The Heb. lacks \fbib{filled}} with olive oil and honey--- \v{9}a land without scarcity. You'll eat food in it and lack nothing. It's a land where its rocks are iron and you can dig copper from its mountains.''
\passage{Remember the Source of Blessings}

\v{10}``When you have eaten and are satisfied, bless the \divine{Lord} your God for the good land that he has given you. \v{11}Be careful! Otherwise, you will forget the \divine{Lord} your God by failing to keep his commands, ordinances, and statutes that I'm commanding you this day. \v{12}Otherwise, when you eat and are satisfied, when you have built beautiful houses and lived in them, \v{13}when your cattle and oxen have multiplied, and when your silver and gold have increased, \v{14}then you will become arrogant. You'll neglect the \divine{Lord} your God, \v{15}who brought you out of the land of Egypt---from the house of slavery---and who led you through the vast and dangerous desert---that parched land without water---with its poisonous snakes and scorpions. He brought water out of solid rock for you \v{16}and fed you in the desert with manna that neither you nor your ancestors had known to humble and test you so that things may go well with you later. \v{17}You may say to yourselves, `I have become wealthy by my own strength and by my own ability.'\fnote{Lit. \fbib{by the power of my hand}} \v{18}But remember the \divine{Lord} your God, because he is the one who gives you the ability to produce wealth, in order to confirm his covenant that he promised by an oath to your ancestors, as is the case today. \v{19}If you neglect the \divine{Lord} your God, follow other gods, and serve and worship them, I testify to you today that you will certainly be destroyed. \v{20}Just like the nations whom the \divine{Lord} destroyed before you, so will you be destroyed, because you did not listen to the voice of the \divine{Lord} your God.''
\labelchapt{9}
\passage{When the \divine{Lord} Fulfills His Promise}

\chapt{9}
\v{1}``Listen, Israel! Today you are about to cross the Jordan to enter and dispossess greater and mightier nations than you, who live in\fnote{The Heb. lacks \fbib{who live in}} large cities that are fortified to the sky. \v{2}The Anakim\fnote{Or \fbib{giants}; cf. Num 13:22, 33} are strong and tall, and you know them. You've heard it said, `Who can stand up against the Anakim?'\fnote{Or \fbib{giants}; cf. Num 13:22, 33} \v{3}But know today that the \divine{Lord} your God is going ahead of you as a consuming fire. He will destroy and subdue them before you. He will dispossess and destroy them quickly, just as the \divine{Lord} told you. \v{4}After the \divine{Lord} has expelled them before you, you are not to say to yourselves, `The \divine{Lord} caused me to enter and possess this land because of my righteousness.' \v{5}On the contrary, it is because of the wickedness of these nations that the \divine{Lord} is dispossessing them before you to confirm what the \divine{Lord} promised by an oath to your ancestors Abraham, Isaac, and Jacob. \v{6}Know that it is not because of your righteousness that the \divine{Lord} your God is giving to you this good land to inherit, since you are a stubborn people.''
\passage{Israel Broke the Covenant}

\v{7}``Remember---and don't ever forget---how you provoked the \divine{Lord} your God in the desert. From the day that you came out of the land of Egypt until you came to this place, you have been rebelling against the \divine{Lord}. \v{8}At Horeb you continually rebelled against the \divine{Lord} so that he\fnote{Lit. \fbib{the \divine{Lord}}} was angry enough to destroy you. \v{9}Then I went up to the mountain to receive the two stone Tablets of the Covenant that the \divine{Lord} had established with you. I stayed on the mountain for 40 days and nights without eating food or drinking water. \v{10}Then the \divine{Lord} gave me the two stone tablets on which God inscribed with his own finger all the words that the \divine{Lord} spoke to you on the mountain from the middle of the fire that day when you were all assembled together. \v{11}At the end of 40 days and nights, the \divine{Lord} gave to me the two stone Tablets of the Covenant.

\v{12}``Then the \divine{Lord} told me, `Get going! Go down from here at once! Your people whom you brought out of Egypt have become corrupt. They have turned quickly from the way that I commanded them and have cast an idol for their use.'

\v{13}``Then the \divine{Lord} told me, `I have examined this people, and they\fnote{Lit. \fbib{and the people}} are stubborn indeed. \v{14}Let me alone! I will destroy them, blot out their name from heaven, and then I'll make you into a nation that will be mighty and more numerous than they are.'

\v{15}``So I turned and went down from the mountain while the mountain was on fire. The two Tablets of the Covenant were in both of my hands. \v{16}Then I saw how you had really sinned against the \divine{Lord} your God! You had made for yourselves a calf---a cast idol. You had turned aside quickly from the way that the \divine{Lord} your God had commanded. \v{17}So I grabbed the two tablets and threw them out of my hands, breaking them before your eyes. \v{18}I fell down in the \divine{Lord}'s presence, just as I had the first 40 days and nights. I didn't eat food or drink water because of your sin. You had sinned by committing this evil in full view of the \divine{Lord}, thereby provoking him to anger. \v{19}I feared the anger and wrath of the \divine{Lord} against you, because he was irate enough to destroy you. But the \divine{Lord} also listened to me at that time. \v{20}It was as had been the case with Aaron---the \divine{Lord} was very angry and about to destroy him, but I prayed for Aaron at that time. \v{21}So when you made the calf that made you sin, I grabbed it, burned it with fire, crushed it, and ground it thoroughly until it was pulverized to powder. Then I threw the powder into the river that was flowing from the mountain.''
\passage{Moses Interceded for Israel}

\v{22}``You provoked the \divine{Lord} again at Taberah, Massah, and Kibroth-hattaavah. \v{23}When the \divine{Lord} sent you from Kadesh-barnea and told you, `Go possess the land that I gave you,' instead you disobeyed what the \divine{Lord} your God said. You didn't trust him or listen to his voice. \v{24}You have been rebelling against the \divine{Lord} since the day I knew you. \v{25}I fell down in the \divine{Lord}'s presence for 40 days and nights, because the \divine{Lord} said he was ready to destroy you. \v{26}So I prayed to the \divine{Lord} and said, `Oh \divine{Lord} my God, don't destroy your people and your inheritance whom you redeemed by your power.\fnote{Lit. \fbib{redeemed in your greatness}} You brought them out from Egypt in a powerful way. \v{27}Remember your servants Abraham, Isaac, and Jacob. Don't pay attention to the stubbornness, wickedness, and sinfulness of this people. \v{28}Otherwise, the people of the land from which you brought us will say, ``The \divine{Lord} wasn't able to bring them into the land that he had promised them. So he brought them out to kill them in the desert because he hated them.'' \v{29}But they are your people and inheritance, whom you brought out by your mighty strength\fnote{Lit. \fbib{arm}} and awesome power.'\,''
\labelchapt{10}
\passage{A Copy of the Ten Commandments}

\chapt{10}
\v{1}``At that time, the \divine{Lord} told me, `Chisel two tablets of stone for yourself just like the first ones, and then come up to me on the mountain. Also make for yourself a wooden chest. \v{2}I'll write on the tablets what was\fnote{Lit. \fbib{tablets of words that were}} on the first tablets that you broke, and then you are to place them into the wooden chest.' \v{3}So I made a chest out of acacia wood and chiseled two tablets of stone just like the first ones. Then I went up the mountain with the two tablets in my hands. \v{4}Then the \divine{Lord}\fnote{Lit. \fbib{he}} inscribed on the tablets what he wrote before---that is, the Ten Commandments that the \divine{Lord} declared to you on the mountain from the middle of the fire during the day of the assembly. And the \divine{Lord} gave them to me. \v{5}Then I turned, went down the mountain, and placed the tablets in the chest that I had made. They are there now, just as the \divine{Lord} commanded me.''
\passage{Aaron Dies and the Descendants of Levi are Appointed}

\v{6}``The Israelis traveled from the wells of the descendants of Jaakan to Moserah. Aaron died and was buried there. His son Eleazar succeeded him as priest. \v{7}From there they moved on to Gudgodah and from Gudgodah to Jotbathah, a land with flowing streams. \v{8}At that time the \divine{Lord} set apart the tribe of Levi to carry the Ark of the Covenant of the \divine{Lord}, to stand in the \divine{Lord}'s presence, to serve, and to bless his name until this day. \v{9}That is why the descendants of Levi do not have a portion and an inheritance among their relatives. As for\fnote{The Heb. lacks \fbib{As for}} the \divine{Lord}, he is their inheritance, just as the \divine{Lord} your God told them. \v{10}When I stood on the mountain for 40 days and 40 nights as I did the first time, the \divine{Lord} listened to me once again. The \divine{Lord} was not willing to destroy you. \v{11}So the \divine{Lord} told me, `Get up and proceed to lead\fnote{Lit. \fbib{to a journey before}} the people, so they may enter and take possession of the land that I promised to give their ancestors by an oath.'\,''
\passage{Love the \divine{Lord}}

\v{12}``Now Israel, what does the \divine{Lord} your God desire from you? Only this: fear him,\fnote{Lit. \fbib{the \divine{Lord} your God}} walk in all his ways, love him, serve him\fnote{Lit. \fbib{the \divine{Lord} your God}} with all your heart and in all your life,\fnote{Or \fbib{soul}} \v{13}and observe his\fnote{Lit. \fbib{observe the \divine{Lord}'s}} commands and statutes that I'm commanding you today for your own good. \v{14}You see, heaven---even the highest heavens---belongs to the \divine{Lord}, along with the earth and all that is in it, \v{15}yet the \divine{Lord} committed himself to love your ancestors---and did so! He chose you---their descendants after them---from all the nations, as it is\fnote{The Heb. lacks \fbib{it is}} today. \v{16}Therefore, circumcise your heart and stop being stubborn. \v{17}For the \divine{Lord} your God is the God of all gods, the \divine{Lord} of all lords, the great God, mighty and awesome, who does not show favoritism or take bribes. \v{18}He executes justice for the orphan and the widows, loves the foreigner, and gives them food and clothing.''
\passage{Love Others}

\v{19}``You are to love the foreigner, because you were foreigners in the land of Egypt. \v{20}You are to fear the \divine{Lord} your God and serve him. Cling to him and swear by his name. \v{21}He is the one you are to praise, because he is\fnote{Lit. \fbib{He is your praise,}} your God who carried out those great and awesome things for you that you witnessed. \v{22}Your ancestors went down to Egypt with 70 people, but the \divine{Lord} your God has now made you as numerous as the stars in the sky.''
\labelchapt{11}
\passage{Remember God's Power}

\chapt{11}
\v{1}``Therefore love the \divine{Lord} your God and be very careful to keep his injunctions, statutes, ordinances, and commands all the time.\fnote{Lit. \fbib{days}} \v{2}Keep in mind today that I am not speaking to your children, who neither were aware of nor did they witness the discipline of the \divine{Lord} your God---that is, his great and far-reaching power, \v{3}including: the signs and works that he did within Egypt to Pharaoh, king of Egypt, and to all his land; \v{4}what he did to the Egyptian army, to its horses, and to its chariots, when he caused the water of the Reed\fnote{So MT; LXX reads \fbib{Red}} Sea to engulf them as they pursued you; how the \divine{Lord} destroyed them, even to this day; \v{5}what he did for you in the desert until you came to this place; \v{6}and what he did to Eliab's sons Dathan and Abiram, descendants of Reuben, when the ground opened up and swallowed them, their households, their tents, and every living thing belonging to them in the full sight\fnote{Lit. \fbib{the middle}} of Israel. \v{7}Your very own eyes saw all the great things that the \divine{Lord} did.''
\passage{Possessing a Fertile Land}

\v{8}``Keep all the commands that I'm giving\fnote{Lit. \fbib{commanding}} you today, so you can be strong enough to enter and possess the land that you are crossing over to inherit \v{9}and so you'll live long in the land that the \divine{Lord} your God promised by an oath to give your ancestors and their descendants---a land flowing with milk and honey, \v{10}since the land that you are about to enter to inherit isn't like the land of Egypt that you just left, where you plant a seed and irrigate it with your feet like a vegetable garden. \v{11}Instead, the land that you are crossing over to inherit is a land of hills and valleys that drinks water supplied by rain from heaven, \v{12}a land about which the \divine{Lord} your God is always concerned, because the eyes of the \divine{Lord} are continuously on it throughout the entire year.''\fnote{Lit. \fbib{on it from the beginning of the year until the end of the year}}
\passage{Delights of a Bountiful Land}

\v{13}``If you carefully observe the commands that I'm giving you today---that is, to love the \divine{Lord} your God and serve him with all your heart and soul--- \v{14}then he\fnote{So with LXX, SP, V; MT reads \fbib{I}} will send rain on the land in its season (the early and latter rains)\fnote{I.e. winter and spring rains} and you'll gather grain, new wine, and oil. \v{15}He\fnote{So with LXX, SP, V; MT reads \fbib{I}} will provide grass in the fields for your livestock, and you'll eat and be satisfied. \v{16}Be careful! Otherwise, your hearts will deceive you and you will turn away to serve other gods and worship them. \v{17}The wrath of God will burn against you so that he will restrain the heavens and it won't rain. The ground won't yield its produce and you'll be swiftly destroyed from the good land that the \divine{Lord} is about to give you. \v{18}Take these commands to heart and keep them in mind, tying them as reminders on your arm and as bands on your forehead. \v{19}Teach them to your children, talking about them while sitting in your house, walking on the road, or when you are about to lie down or get up. \v{20}Also write them upon the doorposts of your house and gates,\fnote{cf. Deut 6:7-8} \v{21}so that you and your children may live a long time in the land that the \divine{Lord} promised to give your ancestors---as long as the sky remains above the earth.''
\passage{Boundaries of the Land}

\v{22}``If you carefully observe all of these commands that I'm giving you to do---to love the \divine{Lord} your God, to walk in all his ways, and to cling to him--- \v{23}then the \divine{Lord} will dispossess all these nations before you and you'll dispossess nations that are even greater and stronger than you. \v{24}Every place upon which the soles of your feet tread will be yours as boundaries---from the desert to Lebanon and from the River (that is, from the Euphrates) to the Mediterranean\fnote{Lit. \fbib{Western}} Sea. \v{25}No one will be able to stand against you. The \divine{Lord} your God will instill terror and fear of you throughout the entire land wherever you go, just as he promised you. \v{26}Look! I'm about to grant you a blessing and a curse--- \v{27}a blessing if you obey the commands of the \divine{Lord} your God that I'm giving you today, \v{28}or a curse if you don't obey the commands of the \divine{Lord} your God, by turning from the way that I'm commanding you today and following other gods whom you have not known.''
\passage{Declaration of the Blessings and Curses}

\v{29}``When the \divine{Lord} brings you to the land that you are about to enter to inherit, repeat the blessings on Mount Gerizim and the curses on Mount Ebal. \v{30}They're across the Jordan River to the west in the land of the Canaanites who live in the Arabah opposite Gilgal near the Oak of Moreh, aren't they? \v{31}For you are about to cross the Jordan River to go in and possess the land that the \divine{Lord} your God is about to give you to inherit and live in. \v{32}Be careful to obey all the statutes and ordinances that I'm placing before you today.''
\labelchapt{12}
\passage{Destroying Altars to False Gods}

\chapt{12}
\v{1}``These are the statutes and ordinances that you must carefully observe in the land that the \divine{Lord} God of your ancestors has given you to possess every day that you live on the earth. \v{2}Be sure you destroy there all the places where the nations that you're going to dispossess serve their gods---upon the high mountains and hills and under every leafy tree. \v{3}Tear down their altars, then cut down their sacred poles\fnote{Lit. \fbib{their Ashram}; i.e. cultic pillars} and burn them. Cut down the carved images of their gods in order to destroy their names from that place.''
\passage{Sacrifice at the Central Sanctuary}

\v{4}``You must not act like this with respect to the \divine{Lord} your God. \v{5}Instead, you must seek to enter only the place that the \divine{Lord} your God will choose among your tribes. There he will establish his name and live. \v{6}Bring your burnt offerings there, along with your sacrifices, your tithes, your hand-carried gifts, your offerings in fulfillment of promises, your freely-given offerings, and the firstborn of your herds and flocks. \v{7}Then you and your household will eat in the presence of the \divine{Lord} your God and rejoice with all the things you have made by your own effort\fnote{Lit. \fbib{hand}} and with which he\fnote{Lit. \fbib{the \divine{Lord}}} blessed you.

\v{8}``You must not act as we have been doing here today, where everyone acts as they see fit, \v{9}since you haven't arrived yet to your allotted place\fnote{Lit. \fbib{allotted resting place and inheritance}} that the \divine{Lord} your God is about to give you. \v{10}But after you have crossed the Jordan River and settled in the land that the \divine{Lord} your God is giving you to inherit, and after you have received relief from the enemies around you and are living securely, \v{11}then bring to the place that the \divine{Lord} your God will choose as a dwelling place---where he will establish his name---everything that I'm commanding you: your burnt offerings, your sacrifices, your tithes, your hand-carried gifts, and all your best offerings in fulfillment of promises that you pledged to the \divine{Lord}.

\v{12}``Rejoice in the presence of the \divine{Lord} your God---you, your sons and daughters, your male and female servants, and the descendant of Levi who is in your city---because there is no territorial allotment\fnote{Lit. \fbib{no portion and possession}} for him as you have. \v{13}Be careful not to offer burnt offerings at any location you happen to see\fnote{Lit. \fbib{you see}} \v{14}instead of at the place the \divine{Lord} will choose in one of the tribal areas. There you may offer burnt offerings, and there you may do everything that I'm commanding you.''
\passage{Instructions Pertaining to Food}

\v{15}``You may slaughter and eat as much meat as you desire,\fnote{Lit. \fbib{with all the desire of your soul}} according to the blessing of the \divine{Lord} your God, when he provides for you in all your cities.\fnote{Lit. \fbib{gates}} Both ritually unqualified and qualified people\fnote{Lit. \fbib{unclean and clean}; and so throughout the book} may eat it as they would gazelle and deer. \v{16}However, you are not to consume the blood;\fnote{Cf. Acts 15:20, 29} instead, you are to pour it out on the ground as you would water.

\v{17}``You won't be allowed to eat your tithe of grain, new wine, or oil, the firstborn of your herd and flock, your voluntary offerings that you pledged, your free-will offerings, or the works of your hands in your own cities. \v{18}You'll eat only in the presence of the \divine{Lord} your God at the place that he\fnote{Lit. \fbib{that the \divine{Lord} your God}} will choose---you, your sons and your daughters, your male and female servants, and the descendant of Levi who is in your cities.\fnote{Lit. \fbib{gates}} Rejoice in the presence of the \divine{Lord} your God in everything you undertake.\fnote{Lit. \fbib{in every work of your hand}} \v{19}Be careful not to forget the descendant of Levi while you live\fnote{Lit. \fbib{all your days in the land}} in the land. \v{20}When the \divine{Lord} your God enlarges your territory---just as he told you---and you say `I want to eat meat' since you desire to eat it,\fnote{Lit. \fbib{meat}} you may do so as much as you please.\fnote{Lit. \fbib{may eat flesh with all the desire of your soul}}

\v{21}``If the place where the \divine{Lord} your God chooses to establish his name is distant from you, then you may slaughter from your herd and your flock what the \divine{Lord} has provided for you, as he instructed you. You may consume them in your cities\fnote{Lit. \fbib{gates}} as much as you please. \v{22}You may eat them just as you would gazelle and deer. Ritually unqualified and qualified people may eat them. \v{23}Only be sure to refrain from eating blood, because blood is the source of\fnote{The Heb. lacks \fbib{source of}} life and you are not to consume blood with the meat. \v{24}You are not to eat it; instead, you are to pour it on the ground as you would water. \v{25}You are not to eat it, so that life may go well for you and for your children after you. Then you'll do what is right in the eyes of the \divine{Lord}.

\v{26}``You may carry and bring only your consecrated gifts and offerings in fulfillment of promises to the place that the \divine{Lord} will choose. \v{27}You must offer your burnt offerings---both the meat and the blood---on the altar of the \divine{Lord} your God. You are to offer the blood by pouring it on the altar of the \divine{Lord} your God while you consume the meat. \v{28}Be sure to observe all these words that I'm commanding you, in order that life may go well for you and your children after you forever, for this is good and right in the eyes of the \divine{Lord} your God.''
\passage{Don't Become Ensnared}

\v{29}``When the \divine{Lord} your God eliminates the nations that you are about to dispossess so you can live in their land, \v{30}after they have been destroyed in your sight, be careful not to be ensnared as they were. Otherwise, you will seek their gods and ask yourselves, `How do these nations serve their gods? I will do likewise.' \v{31}You must not do the same to the \divine{Lord} your God, because they practiced in the presence of their gods every sort of abomination that the \divine{Lord} hates. Moreover, they sacrificed\fnote{Lit. \fbib{they burned in fire}} their sons and daughters to their gods. \v{32}\fnote{This v. is 13:1 in MT}Now as to everything I'm commanding you, you must be careful to observe it. Don't add to or subtract from it.''
\labelchapt{13}
\passage{Dealing with False Prophets}

\chapt{13}
\v{1}\fnote{This v. is 13:2 in MT}``A prophet or a diviner of dreams may arise among you, give you an omen or a miracle \v{2}that takes place, and then he may tell you, `Let's follow other gods (whom you have not known) and let's serve them.' Even though the sign or portent comes to pass, \v{3}you must not listen to the words of that prophet or that diviner of dreams. For the \divine{Lord} your God is testing you, to make known whether or not you'll continue to love the \divine{Lord} your God with all your heart and soul. \v{4}You must follow the \divine{Lord} your God, fear him, observe his commandments, listen to his voice, serve him, and cling to him. \v{5}That prophet or diviner of dreams must be executed, because he advocated rebellion against the \divine{Lord} your God, who brought you from the land of Egypt and redeemed you from the house of slavery, and because he lured you from the way in which the \divine{Lord} your God instructed you to live. Purge the evil from among you.''
\passage{Dealing with Idolaters}

\v{6}``Your own blood brother,\fnote{Lit. \fbib{your brother, the son of your mother}} your son, your daughter, your beloved wife, or your friend who is like your soul mate may entice you quietly. He may tell you, `Let's go and serve other gods' (whom neither you nor your ancestors have known \v{7}from the gods of the people that surround you---whether near or far from you---from one end of the earth to the other). \v{8}You are not to yield to him, listen to him, look with pity on him, show compassion to him, or even cover up for him. \v{9}Instead, you are surely to execute him. You must be the first to put him to death with your own hand, and then the hands of the whole community. \v{10}Stone him to death, because he sought to lure you from the \divine{Lord} your God, who brought you from the land of Egypt, from the land of slavery. \v{11}Then all Israel will hear about it, be afraid, and won't do this evil thing again among you.

\v{12}``You may hear in one of your towns that the \divine{Lord} your God is giving you to inhabit \v{13}that worthless men\fnote{Lit. \fbib{that men, sons of Belial}} have come from among you to entice those who live in the towns. They may say, `Let's go and serve other gods that you haven't known.' \v{14}You must thoroughly investigate and inquire if it is true that this detestable thing exists among you. If it is so,\fnote{The Heb. lacks \fbib{if it is so}} \v{15}then put the inhabitants of the town to death by the sword. Devote everything in it to divine destruction---even its livestock---by the sword. \v{16}Gather whatever you've taken as spoils at the public square of the town, then burn the town, along with whatever you've taken, as an offering to the \divine{Lord} your God. It will remain a permanent mound of ruins, never to be rebuilt again. \v{17}Moreover, you must never take any item from those condemned things, so the \divine{Lord} may yet relent from his burning anger and extend compassion, have mercy, and cause you to increase in number---as he promised by an oath to your ancestors--- \v{18}if you obey the voice of the \divine{Lord} your God by observing all his commands that I'm commanding you today. Do what is right in the sight of the \divine{Lord} your God.''
\labelchapt{14}
\passage{Refrain from Cutting Yourselves}

\chapt{14}
\v{1}``You are children of the \divine{Lord} your God. You must not lacerate yourselves or shave your foreheads on account of the dead, \v{2}because you are a holy people to the \divine{Lord} your God, and the \divine{Lord} chose to make you his precious possession from among all the nations\fnote{Lit. \fbib{peoples}} of the earth.''
\passage{Refrain from Unclean Food}

\v{3}``You must not eat any detestable food. \v{4}These are the animals that you may eat: ox, sheep, goat, \v{5}deer, gazelle, roebuck, wild goat, ibex, antelope, and mountain sheep. \v{6}You may eat every animal with a divided hoof---those with split cloven hooves---that chews the cud. \v{7}However, you must not eat these animals that chew the cud or have a divided hoof: the camel, hare, and rock badger. Even though they chew the cud, their hooves are not divided. Therefore, they are unclean for you. \v{8}And also the pig, because even though its hoof is divided, it does not chew the cud. It is therefore unclean for you. You must not eat their meat or even touch their carcasses.

\v{9}``You may choose to eat from these creatures in the water: you may eat anything with fin and scale, \v{10}but you may not eat anything without fin and scale, since it is unclean to you.

\v{11}``You may eat all clean birds, \v{12}but you must not eat any of these: the eagle, vulture, osprey, \v{13}buzzard, any kind of kite, \v{14}any kind of raven, \v{15}the ostrich, night hawk, seagull, any kind of falcon, \v{16}the little owl, great owl, horned owl, \v{17}pelican, carrion vulture, cormorant, \v{18}stork, any kind of heron, the hoopoe, and the bat. \v{19}Any winged, swarming insect is unclean to you---they are not to be eaten. \v{20}You may eat every bird that is clean.

\v{21}``You must not eat any carcass,\fnote{I.e., that dies of itself or in the wild} but you may give it to the alien in your cities\fnote{Lit. \fbib{in your gates}} so he may either consume it or sell it to a foreigner, since you are a people that is holy to the \divine{Lord} your God.

``You must not cook a young goat in its mother's milk.''
\passage{Remember to Tithe}

\v{22}``Be sure to tithe annually from everything you plant that yields a harvest in the field. \v{23}Then in the presence of the \divine{Lord} your God, in the place where he'll choose to establish his name, you may consume the tithe of your grain, your new wine, your oil, and the firstborn of your livestock and flock, so that you'll learn to revere the \divine{Lord} your God all your life. \v{24}Now the way may be distant from you, so that you are unable to transport your tithe because you have been blessed by the \divine{Lord} your God and the place where the \divine{Lord} your God chooses to establish his name may be distant from you. \v{25}In that case, convert it into cash, secure the money,\fnote{Lit. \fbib{bind the money with your hand}} and then bring it to the place where the \divine{Lord} will choose. \v{26}You may spend the money to your heart's content to buy livestock, flocks, wine, strong drink, and whatever you desire. You and your household may eat there and rejoice in the presence of the \divine{Lord} your God.''
\passage{The Levitical Tithe}

\v{27}``But you must not forget the descendant of Levi in your town,\fnote{Lit. \fbib{gates}} because there is no tribal allotment\fnote{Lit. \fbib{a portion and inheritance}} for him as there is for you. \v{28}Every third year, bring all the tithes of your produce of that year and store them in your cities \v{29}so the descendants of Levi---who have no tribal allotment as you do---foreigners, orphans, and widows who live in your cities may come, eat, and be satisfied. That way, the \divine{Lord} your God will bless you in everything you do.''\fnote{. Lit. \fbib{in the work of your hand that you do}}
\labelchapt{15}
\passage{The \divine{Lord}'s Remission}

\chapt{15}
\v{1}``You must cancel your debts at the end of every seventh year. \v{2}This is the way to conduct remission: every creditor must cancel the loan that his friend borrowed, and he must not pressure his friend or brother to repay it,\fnote{The Heb. lacks \fbib{to repay it}} because remission to the \divine{Lord} will be proclaimed. \v{3}You may exact payment from a foreigner, but cancel whatever your brother owes you. \v{4}Moreover, there will be no poor person among you, for the \divine{Lord} will surely bless you in the land that he\fnote{Lit. \fbib{the \divine{Lord}}} is about to give you to possess. \v{5}Only be certain to obey the voice of the \divine{Lord} your God. Carefully observe all of these commands that I'm commanding you today, \v{6}because the \divine{Lord} your God will bless you just as he promised. You are to lend to many nations, but not to borrow. Also, you will rule over many nations, but they will not rule over you.''
\passage{Care for the Poor}

\v{7}``If there should be a poor man among your relatives\fnote{Lit. \fbib{brothers}} in one of the cities of the land that the \divine{Lord} your God is about to give you, don't be hard-hearted or tight-fisted toward your poor relative.\fnote{Lit. \fbib{brother}} \v{8}Instead, be sure to open your hand to him and lend him enough to lessen his need. \v{9}Be careful not to think this wicked thought to yourselves: `The seventh year, the year of remission, is drawing near{\ldots}' and you show ill will\fnote{Lit. \fbib{and your eyes are evil}} toward your poor relative\fnote{Lit. \fbib{brother}} and not give to him. He may then call to the \divine{Lord} on account of you, and you will be guilty of sin. \v{10}You must certainly give to him and not feel regret for doing so.\fnote{Lit. \fbib{for giving to him}} Because of this, the \divine{Lord} your God will bless all your works and everything you do. \v{11}Since poor people won't cease to exist in the land, I'm commanding you: Be sure to display generosity\fnote{Lit. \fbib{to open your hand}} to your poor and needy relatives in your land.''
\passage{Releasing Slaves}

\v{12}``When a fellow Hebrew male or female slave is sold to you and serves you for six years, then in the seventh year you are to set them\fnote{Lit. \fbib{him}; and so throughout the chapter} free. \v{13}But when you set them free, don't send them away empty-handed. \v{14}Provide for them liberally from your flock, threshing floor, and wine vat. As the \divine{Lord} your God has blessed you, so give to them. \v{15}Don't ever forget that you were a slave in the land of Egypt, yet the \divine{Lord} your God redeemed you. Therefore, I'm giving you these commands today.

\v{16}``If that slave\fnote{Lit. \fbib{he}} should say to you, `I won't leave you,' because he loves you and your household, and it was good for him to be with you, \v{17}then take an awl and pierce through his earlobe into the door. He then will be your slave forever. You are to do the same for your female slaves. \v{18}Don't view this as a hardship for yourself when you set him free, for he will have served you for six years---twice the time of a paid worker. Then the \divine{Lord} will bless you in all that you do.''
\passage{Offering the Firstborn Male Animals}

\v{19}``Set apart for the \divine{Lord} your God every firstborn male among your herd and flock. You are not to put the firstborn of your ox to work or shear the firstborn of your flock. \v{20}Instead, in the presence of the \divine{Lord} your God, you and your household are to eat them every year at the place the \divine{Lord} will choose. \v{21}If it has a blemish---lameness, blindness, or any kind of defect---you must not sacrifice it to the \divine{Lord} your God. \v{22}In your cities,\fnote{Lit. \fbib{gates}} both the unclean and the clean together are to eat it together,\fnote{Or \fbib{completely}} as the gazelle and the deer, \v{23}but you are not to eat its blood. Pour it on the ground like water.''
\labelchapt{16}
\passage{Celebrate the Passover}

\chapt{16}
\v{1}``Observe the month of Abib, keeping the Passover to the \divine{Lord} your God, because the \divine{Lord} your God brought you out of Egypt during the night in the month of Abib. \v{2}Then sacrifice sheep and cattle for the Passover to the \divine{Lord} your God at the place where the \divine{Lord} your God will choose to establish his name. \v{3}You must not eat any yeast with it. Instead, for seven days eat bread without yeast---the bread of affliction---because you left the land of Egypt in haste. Remember the day you went out of the land of Egypt for the rest of your lives. \v{4}Yeast is not to be seen in any of your territories for seven days. The meat is not to remain from the evening of the first day until morning.

\v{5}``You must not sacrifice the Passover in just any of your cities\fnote{Lit. \fbib{gates}} that the \divine{Lord} your God is about to give you. \v{6}Instead, you are to sacrifice the Passover in the evening at dusk---at the time of day you left Egypt---at the place where your God will choose to establish his name. \v{7}Boil and eat the Passover meal\fnote{The Heb. lacks \fbib{the Passover meal}} at the place that the \divine{Lord} your God will choose. In the morning you may go back to your tents. \v{8}Eat bread without yeast for six days. Then on the seventh day, hold an assembly to the \divine{Lord} your God. Don't do any work.''
\passage{Celebrate the Festival of Weeks}

\v{9}``Count off seven weeks from when the sickle is first put to standing grain. \v{10}Then observe the Festival of Weeks in the presence of the \divine{Lord} your God by giving your tribute and the freewill offering of your hands in proportion to the manner in which the \divine{Lord} your God blessed you. \v{11}Rejoice in the presence of the \divine{Lord} your God with your son, daughter, male and female slaves, the descendant of Levi who is in your city,\fnote{Lit. \fbib{gate}} the stranger, the orphan, and the widow among you, at the place where the \divine{Lord} your God will choose to establish his name. \v{12}Remember that you were slaves in Egypt, so keep and observe these statutes.''
\passage{Celebrate the Festival of Tents}

\v{13}``Celebrate the Festival of Tents\fnote{Or \fbib{Tents}} for seven days after you harvest from your threshing floor and your wine press. \v{14}Rejoice in your festival---you, your son, your daughter, your male and female slaves, and the descendants of Levi, foreigners, orphans, and widows, who live in your cities.\fnote{Lit. \fbib{gates}} \v{15}For seven days you are to celebrate in the presence of the \divine{Lord} your God at the place where the \divine{Lord} will choose, because the \divine{Lord} your God will bless you in all your harvest and in everything you do, and your joy will be complete.

\v{16}``Every male must appear in the presence of the \divine{Lord} your God three times a year at the place where he will choose: for the Festival of Unleavened Bread, the Festival of Seven Weeks, and the Festival of Tents.\fnote{Or \fbib{Tents}} He must not appear in the \divine{Lord}'s presence empty-handed, \v{17}but each one must appear\fnote{The Heb. lacks \fbib{must appear}} with his own gift, proportional to the blessing that the \divine{Lord} your God has given you.''
\passage{Pursue Justice}

\v{18}``Appoint judges and civil servants according to your tribes in all your cities\fnote{Lit. \fbib{gates}} that the \divine{Lord} your God is about to give you, so they may judge the people impartially.\fnote{Lit. \fbib{people with righteous judgment}} \v{19}You must not twist justice, show favoritism, or take bribes, because a bribe blinds the eyes of the wise and subverts the speech of the righteous. \v{20}You are to pursue justice---and only justice---so you may live and possess the land that the \divine{Lord} your God is about to give you.''
\passage{Prohibited Practices}

\v{21}``You are not to set up a sacred pole\fnote{Lit. \fbib{Asherah}; i.e. a cultic pillar} beside the altar of the \divine{Lord} your God that you will build. \v{22}Furthermore, you are not to erect for yourselves a sacred stone pillar, because the \divine{Lord} your God detests these things.\chapt{17}
\v{1}You are not to sacrifice to the \divine{Lord} your God an ox or a sheep that has a defect or any flaw in it, because that is detestable to the \divine{Lord} your God.''
\labelchapt{17}
\passage{Death to the Idolater}

\v{2}``You may discover that a man or woman living in one of your cities that the \divine{Lord} your God is about to give you has done evil in the eyes of the \divine{Lord} your God by transgressing his covenant. \v{3}He may be following and serving other gods by bowing down to them---that is, to the sun, the moon, or to any of the heavenly host\fnote{Or \fbib{any of the stars or planets}, if referring to astronomical bodies; or \fbib{supernatural beings}, if referring to fallen or unfallen angelic armies} (something I did not command). \v{4}When it is reported to you or you hear of it, you are to investigate it thoroughly. When the truth has been established that this detestable thing has been done in Israel, \v{5}summon the man or the woman who did this evil thing to your city gates, and then stone the man or the woman to death. \v{6}Based on the testimony\fnote{Lit. \fbib{mouth}} of two or three witnesses, they must surely die, but they are not to die based on the testimony of one person. \v{7}Let the witnesses\fnote{Lit. \fbib{the hands of the witnesses}} be the first to begin executing them, then the rest of\fnote{Lit. \fbib{the hand of all}} the people are to follow. By doing this you will purge evil from among you.''
\passage{Deciding Difficult Cases}

\v{8}``If a case is too difficult for you to decide with respect to bloodshed,\fnote{Lit. \fbib{blood versus blood}} civil claims,\fnote{Lit. \fbib{justice versus justice}} assault and battery,\fnote{Lit. \fbib{wound versus wound}} or other matters of dispute within your courts,\fnote{Lit. \fbib{gates}} bring\fnote{Lit. \fbib{stand and go up}} it to the place that the \divine{Lord} your God will choose. \v{9}Present the case\fnote{The Heb. lacks \fbib{Present the case}} to the Levitical priest or the judge at that time. When you have inquired and they have announced the verdict, \v{10}carry out the verdict that was declared to you at the place that the \divine{Lord} will choose. Carefully observe all of their instructions to you \v{11}in accordance with what the Law says and in accordance with the verdict that will be handed to you. You must not deviate from the verdict that they declare to you either to the right or to the left. \v{12}If a man presumptuously disregards the priest who is serving the \divine{Lord} your God there, or the judge, that person must die so you will purge evil from Israel. \v{13}Then all the people who hear will be afraid and will not act presumptuously again.''
\passage{Duties of the Future King}

\v{14}``When you have come to the land that the \divine{Lord} your God is about to give you, and you have taken possession of it and have settled in it, then you will say, `I will appoint a king over me like all the nations around me.' \v{15}You will certainly set a king over you, whom the \divine{Lord} your God will choose from among your relatives, but you must not place a foreign king over you who is not from your relatives. \v{16}He must not amass horses for himself or cause the people to return to Egypt to obtain more horses, because the \divine{Lord} said you must never return that way again. \v{17}Also, he must not accumulate wives for himself (otherwise, his affection will become diverted), nor accumulate for himself excessive quantities of\fnote{The Heb. lacks \fbib{quantities of}} silver and gold. \v{18}When he occupies his royal throne, he must make a copy of this Law for himself from a scroll used by the Levitical priests. \v{19}It is to remain with him the rest of his life so he may learn to fear the \divine{Lord} his God and observe all the words of this Law and these statutes, in order to fulfill them. \v{20}He is not to exalt himself over his relatives, nor turn aside from the commandment---neither to the right nor to the left---so that he and his sons may reign long in Israel.''
\labelchapt{18}
\passage{Provision for the Descendants of Levi}

\chapt{18}
\v{1}``The Levitical priests---the whole tribe of Levi---will not have a portion or an inheritance within Israel. Instead, they will eat the burnt offerings of the \divine{Lord}, because that is their inheritance. \v{2}But they will not have an inheritance among their relatives, because the \divine{Lord} alone is their inheritance---as he promised them.''
\passage{Provision for the Priests}

\v{3}``A portion of what the people offer in sacrifice, whether cattle or sheep, is to be due the priests. They must set aside the shoulder, jowls, and stomach for the priest. \v{4}Give them the first gatherings of your grain, wine, and oil, as well as wool from the shearing of your flock. \v{5}For the \divine{Lord} your God has chosen them and their descendants\fnote{Lit. \fbib{sons}} from among your tribes to stand and serve in the name of the \divine{Lord} all their lives.''\fnote{Lit. \fbib{days}}
\passage{Provision for the Itinerant Levite}

\v{6}``Any descendant of Levi who wishes to do so may come from any city or part of Israel where he resides to the place that the \divine{Lord} will choose. \v{7}There he may serve in the name of the \divine{Lord} his God. Like his fellow descendants of Levi who stand there in the \divine{Lord}'s presence, \v{8}he may eat the same share as they do regardless of what he receives from his ancestral estate.''
\passage{Detestable Practices}

\v{9}``When you enter the land that the \divine{Lord} your God is about to give you, don't learn the detestable practices of those nations there. \v{10}There must never be found among you anyone who sacrifices\fnote{Lit. \fbib{passes}} his son or daughter in fire, practices divination, interprets omens, practices sorcery, \v{11}casts spells, or who is a medium, an occultist, or a necromancer. \v{12}Whoever practices these things is detestable to the \divine{Lord}, and the \divine{Lord} your God will expel them before you because of these things. \v{13}You must be completely faithful to the \divine{Lord} your God, \v{14}because those nations that you are about to dispossess listen to those who practice witchcraft and divination. But the \divine{Lord} does not allow you to act this way.''
\passage{Discerning the True Prophet}

\v{15}``The \divine{Lord} your God will raise up a prophet like me for you from among your relatives. You must listen to him, \v{16}because this is what you asked from the \divine{Lord} your God at Horeb when you were assembled together: `Don't let us\fnote{Lit. \fbib{me}} hear the voice of the \divine{Lord} our God again, or even see this great fire---otherwise, we\fnote{Lit. \fbib{I}} will die.'

\v{17}``Then the \divine{Lord} told me: `What they have suggested is good. \v{18}I will raise up a prophet like you from among their relatives, and I will place my words in his mouth so that he may expound everything that I have commanded to them. \v{19}But if someone will not listen to those words that the prophet\fnote{Lit. \fbib{he}} speaks in my name, I will hold him accountable. \v{20}Even then, if the prophet speaks presumptuously in my name, which I didn't authorize him to speak, or if he speaks in the name of other gods, that prophet must die.' \v{21}Now you may ask yourselves, `How will we be able to discern that the \divine{Lord} has not spoken?' \v{22}Whenever a prophet speaks in the name of the \divine{Lord} and the oracle does not come about or the word is not fulfilled, then the \divine{Lord} has not spoken it. The prophet will have spoken presumptuously, so you need not fear him.''
\labelchapt{19}
\passage{Cities of Refuge}

\chapt{19}
\v{1}``When the \divine{Lord} your God destroys those nations whose lands he\fnote{Lit. \fbib{the \divine{Lord} your God}} is about to give you, you must dispossess them and live in their cities and houses. \v{2}You must reserve\fnote{Or \fbib{set apart}} three cities within the land that the \divine{Lord} your God is about to give you to possess. \v{3}Build roads throughout the land that the \divine{Lord} your God is providing as an inheritance, and then divide it into three districts so that any killer may flee there.

\v{4}``Now this is the situation for any killer who flees there to live: suppose he strikes his friend unwittingly, not having hated him previously. \v{5}For instance,\fnote{The Heb. lacks \fbib{for instance}} he may have accompanied his friend to go to a forest to cut trees. Then he swung his axe to cut some wood, but the ax head flew off the handle\fnote{Lit. \fbib{tree}} and hit\fnote{Lit. \fbib{found}} his friend, so that he died. The killer\fnote{The Heb. lacks \fbib{the killer}} may flee to one of these cities to live. \v{6}Since the distance may be great, the angry avenger may overtake the killer he is pursuing and kill him, in which case there will be no justice in his death, because he did not hate his friend\fnote{Lit. \fbib{hate him}} previously. \v{7}Therefore I am commanding you to reserve\fnote{Or \fbib{set apart}} three cities.''
\passage{Increase the Cities of Refuge}

\v{8}``Now if the \divine{Lord} enlarges your territories just as he promised your ancestors and gives you all the land that he promised,\fnote{Lit. \fbib{promised to give your ancestors}} \v{9}and if you are careful to observe all these commands that I am commanding you today---to love the \divine{Lord} your God and to walk daily in his ways---then add three more cities in addition to these three cities. \v{10}You must not shed innocent blood on your land that the \divine{Lord} your God is about to give you as an inheritance. Otherwise, you'll be guilty of murder.''
\passage{Refuse Cold-Blooded Murderers}

\v{11}``However, if a person hates his neighbor, lies in wait for him, rises up against him, and attacks him so that he dies, and then he flees to one of those cities, \v{12}then the elders of his own city are to send for him, remove him from there, and deliver him to the related avenger for execution. \v{13}Have no pity on him, but totally purge the shedding of innocent blood from Israel so that life may go well with you.''
\passage{Boundary Markers}

\v{14}``When you inherit the land that the \divine{Lord} your God is about to give you, don't move your neighbor's boundary marker from where it was placed long ago.''
\passage{Laws about Witnesses}

\v{15}``The testimony of one person alone is not to suffice to convict anyone of any iniquity, sin, or guilt. But the matter will stand on the testimony of two or three witnesses. \v{16}When a malicious witness takes the stand\fnote{Lit. \fbib{witness stands}} against a man and accuses him, \v{17}then both must stand with their dispute in the \divine{Lord}'s presence, the priests, and the judges at that time. \v{18}The judges will investigate thoroughly. If the false witness lies in testifying against his relative, \v{19}do to him just as he intended to do to his relative. By doing this you will purge evil from your midst. \v{20}When others hear of this, they will be afraid and will not do such an evil deed again in your midst. \v{21}Your eyes must not show pity---life for life, eye for eye, tooth for tooth, hand for hand, and foot for foot.''
\labelchapt{20}
\passage{Rules of War}

\chapt{20}
\v{1}``When you go to war against your enemies and observe more horses, chariots, and soldiers\fnote{Lit. \fbib{people}} than you have, don't be afraid of them, for the \divine{Lord} your God who brought you out of the land of Egypt is with you. \v{2}As you draw near for battle, let the priest approach and speak to the army.\fnote{Lit. \fbib{people}; and so throughout the chapter} \v{3}He will say to them, `Listen, Israel! You're about to go into battle today against your enemies. Don't be faint-hearted. Don't be afraid, don't panic, and don't be terrified to face them. \v{4}For the \divine{Lord} your God will be with you, fighting on your behalf against your enemies in order to grant you victory.'

\v{5}``Furthermore, let the officials ask the army, `Is there a man here\fnote{The Heb. lacks \fbib{here}} who has built a new house but has not yet dedicated it? Let him go back home. Otherwise, he may die in battle and another man dedicate it. \v{6}And is there a man here\fnote{The Heb. lacks \fbib{here}} who has planted a vineyard and not yet benefited from it? Let him go home. Otherwise, he may die in battle and another man use it. \v{7}And is there a man here\fnote{The Heb. lacks \fbib{here}} who is engaged to a woman and has not yet married her? Let him go back home. Otherwise, he may die in battle and another man marry her.'

\v{8}``Let the officials also speak to the army, `Is there a man here\fnote{The Heb. lacks \fbib{here}} who is afraid and faint-hearted? Let him go back home. Otherwise, he may demoralize his fellow soldier.'\fnote{Lit. \fbib{his brother}}

\v{9}``When the officials have finished speaking to the army, they must appoint officers to lead the troops.''
\passage{Rules of Peace}

\v{10}``When you approach a city to wage war against it, extend terms of peace. \v{11}If it agrees to peace and welcomes you, then all the people found in it will serve you as forced laborers. \v{12}But if they refuse to make peace with you and instead choose war, then attack it. \v{13}The \divine{Lord} your God will deliver it into your control, and you must execute every male. \v{14}The women, children, all the livestock in the city, and all of the spoil and plunder will belong to you. Appropriate the spoil of your enemies, which the \divine{Lord} your God will give you. \v{15}Do this to all the cities that are distant from you---that is, to those cities that are not in neighboring nations.''
\passage{Destruction of the Canaanites}

\v{16}``You are not to leave even one person alive in the cities of these nations that the \divine{Lord} your God is about to give you as an inheritance. \v{17}You must completely destroy the Hittites, the Amorites, the Canaanites, the Perizzites, the Hivites, and the Jebusites, just as the \divine{Lord} your God commanded you, \v{18}so they won't teach you to do all the detestable things that they do for their gods. If you do what they teach you, you will sin against the \divine{Lord} your God.''
\passage{Preservation of Fruit Trees}

\v{19}``When you attack a city and have to fight against it for many days, don't destroy its trees by cutting them down with an ax. You may eat from them, but you must not cut them down. Are the trees of the field human beings, that you would come and attack them? \v{20}However, you may cut down the trees whose fruit\fnote{The Heb. lacks \fbib{whose fruit}} you know isn't edible, in order to build siege works against the city that waged war with you, until it falls.''
\labelchapt{21}
\passage{Atonement for Unsolved Murder}

\chapt{21}
\v{1}``If a murder victim is found fallen in the open country of the land that the \divine{Lord} your God is about to give you to possess, and it is not known who killed him, \v{2}then let your elders and judges go out and measure the distance from the dead body to the neighboring cities. \v{3}Then the elders of the city nearest the body are to take a heifer that hasn't been put to work or hasn't pulled a yoke \v{4}and\fnote{Lit. \fbib{The elders of the city}} are to lead the heifer to a flowing stream in a valley that has never been tilled or planted. They are to break the heifer's neck there. \v{5}Then the priests of the sons of Levi are to step forward, because the \divine{Lord} your God chose them to serve and pronounce blessings in his name.\fnote{Lit. \fbib{in the name of the \divine{Lord} your God}} Every case of dispute and assault is to be subject to their ruling. \v{6}All the elders of the city nearest the dead body are to wash their hands over the heifer whose neck was broken in the valley, \v{7}and they are to make this declaration: `Our hands didn't shed this blood, nor were we witnesses to the crime. \v{8}Make atonement for your people Israel, whom you have redeemed, \divine{Lord}, and don't charge the blood of an innocent man against them.'\fnote{Lit. \fbib{against your people Israel}} Then the blood that has been shed will be atoned for. \v{9}This is how you will remove the guilt of innocent blood from among you, for you must do what is right in the sight of the \divine{Lord}.''
\passage{Marriage to a War Captive}

\v{10}``If you go to battle against your enemies, and the \divine{Lord} your God delivers them into your control, you may take some prisoners captive. \v{11}If you see among the prisoners a beautiful woman and you desire her, then you may take her as your wife. \v{12}Bring her to your house, but shave her head and trim her nails. \v{13}Remove her prisoner's clothing and let her remain for a month in your house, mourning her parents. After that, you may\fnote{Lit. \fbib{may go in to her,}} become her husband and she is to become your wife. \v{14}If you aren't pleased with her and you send her away, you must not sell her for money or mistreat her, since you will have dishonored her.''
\passage{Preferential Treatment Prohibited}

\v{15}``If a man has two wives where one is loved but the other is unloved, and both\fnote{Lit. \fbib{the one who is loved and who is not loved}} of them bear him sons, but the firstborn is the son of the unloved wife, \v{16}then when he bequeaths his possessions to his sons, he must not give preference to the firstborn of the beloved wife over the firstborn of the unloved wife. \v{17}Instead, he must acknowledge the firstborn of the unloved wife by giving him double of everything he owns, because he is really the first fruit of his father's\fnote{The Heb. lacks \fbib{father's}} strength. The right of the firstborn belongs to him.''
\passage{Death to a Rebellious Son}

\v{18}``If a man has a stubborn son who does not obey his parents,\fnote{Lit. \fbib{obey the voice of his father or the voice of his mother}} and although they try to discipline him, he still refuses to pay attention to them, \v{19}then his parents\fnote{Lit. \fbib{his father and his mother}} are to seize him and bring him before the elders at the gate of his city. \v{20}Then they are to declare to the elders of their city: `Our son is stubborn and rebellious. He does not obey us. He lives wildly and is a drunkard.' \v{21}Then all the men of his city are to stone him with stones so that he dies. This is how you will remove this evil from among you. Then all Israel will hear of it and will be afraid.''
\passage{Burial of the Executed}

\v{22}``If a man is guilty of a capital offense, is executed, and then is impaled on a tree, \v{23}his body must not remain overnight on the tree. You must bury him that same day, because cursed of God is the one who has been hanged on a tree. Don't defile your land that the \divine{Lord} is about to give you as your inheritance.''
\labelchapt{22}
\passage{Hospitality to Neighbors}

\chapt{22}
\v{1}``When you see the ox or sheep of your fellow countryman\fnote{Lit. \fbib{brother's} and so throughout the chapter} straying, don't go away and leave it. Instead, be sure to return it to him.\fnote{Lit. \fbib{brother} and so throughout the chapter} \v{2}If your fellow countryman doesn't live near you or you don't know who he is, bring the animal\fnote{Lit. \fbib{bring it}} to your house and let it remain with you until he\fnote{Lit. \fbib{brother}} claims it. Then return it to him. \v{3}Do the same for his donkey, his garment, and for anything lost that belongs to your fellow countryman. When you find it, you must not ignore it. \v{4}When you see the donkey or the ox of your fellow countryman fallen on the road, don't ignore them. Instead be sure to help them get up.''
\passage{Miscellaneous Laws}

\v{5}``A woman is not to wear what is appropriate to a man, nor is a man to put on a woman's garment, because anyone who does this is detestable to the \divine{Lord} your God.

\v{6}``When you encounter a bird's nest along the road, whether in a tree or on the ground, and the mother bird is sitting on its chicks\fnote{Lit. \fbib{on the young}} or eggs, don't take the mother along with its young.\fnote{Lit. \fbib{sons}} \v{7}You may take the young, but be sure to release the mother, so that life will go well for you and that you may have a long life.

\v{8}``When you build a new house, install a parapet along your roof so that if someone falls from the roof, you won't bring guilt of bloodshed on your house.''
\passage{Principles of Distinction}

\v{9}``Don't plant two kinds of seeds in your vineyard. Otherwise, the entire crop will have to be forfeited, both the seed that you have sown and the produce from it.

\v{10}``Don't plow with an ox and a donkey yoked together.

\v{11}``Don't wear material made from wool and linen mixed together.

\v{12}``Sew tassels for yourself on the four corners of the garment with which you cover yourself.''
\passage{Integrity in Marriage}

\v{13}``Suppose a man marries a wife, but after having sexual relations with her, he despises her, \v{14}invents charges against her, and defames her by saying, `I have married this woman, but when I had sexual relations with her I found that she wasn't a virgin.' \v{15}Then the father of the young lady, along with her mother, is to bring evidence of the young lady's virginity to the elders at the gate. \v{16}The father of the young lady is to then say to the elders: `I have given my daughter to this man as a wife, but he despises her. \v{17}Now look, he has invented charges against her by saying, ``I haven't found your daughter to be a virgin.'' But here is the proof of my daughter's virginity.' Then they are to spread the cloth before the elders of the city. \v{18}The elders of that city will then take the man, punish him, \v{19}fine him 100 shekels of silver, and then give them to the young lady's father, because he defamed a virgin of Israel. She is to remain his wife and he can't divorce her as long as he lives. \v{20}But if this charge is true, and the evidence of the young lady's virginity wasn't found, \v{21}they are to bring her to the door of her father's house. Then the men of the city are to stone her with boulders until she dies for doing a detestable thing in Israel---acting like a prostitute while in her father's house. By doing this, you will remove this evil from among you.

\v{22}``If a man is caught having sexual relations with a married woman, then both of them must die---the man who had sex with the woman and the woman herself---so that this evil will be removed from Israel.

\v{23}``If a man meets a young virgin lady in the city who is engaged to be married and has sexual relations with her, \v{24}then the two must be brought to the city gate and there they must be stoned to death---the girl because she was in a city but did not cry out for help, and the man who abused a woman who was engaged to another man. By doing this you are to remove this evil from among you.

\v{25}``If a man meets a girl in the country who is engaged to be married and then rapes\fnote{Lit. \fbib{overwhelms}} her, the man alone---the one who had sexual relations with her---must die. \v{26}As for the young lady, don't do anything to her. The young lady did nothing worthy of death. This case is similar to when a man attacks his countryman and kills him. \v{27}Since he found her in the country, the engaged girl may have cried out, but there was no one to rescue her.

\v{28}``However, if a man meets a girl who isn't engaged to be married, and he seizes her, rapes her, and is later found out, \v{29}then the man who raped her must give 50 shekels of silver to the girl's father. Furthermore, he must marry her. Because he violated her, he is to not divorce her as long as he lives.

\v{30}\fnote{This v. is 23:1 in MT}``A man must not marry his father's wife, so that he will not dishonor his father's memory.''\fnote{Lit. \fbib{wing}; or \fbib{skirt}}
\labelchapt{23}
\passage{Qualifications for Assembling}

\chapt{23}
\v{1}\fnote{This v. is 2 in MT, and so throughout the chapter.}``No man whose testicles have been crushed\fnote{Or \fbib{wounded}} or whose penis has been cut off may participate in the assembly of the \divine{Lord}. \v{2}Furthermore, no one born due to an illicit sexual relationship may participate in the assembly of the \divine{Lord}, including his descendants to the tenth generation. \v{3}``No Ammonite or Moabite may participate in the assembly of the \divine{Lord}, and none of their descendants is to be admitted to the assembly of the \divine{Lord}, to the tenth generation, \v{4}because they didn't come to meet you with food and water along the way as you were coming out of Egypt. Instead, they hired Beor's son Balaam from Pethor in Aram-naharaim\fnote{I.e. Mesopotamia} to curse you. \v{5}However, the \divine{Lord} your God didn't listen to Balaam. The \divine{Lord} your God turned Balaam's\fnote{Lit. \fbib{his}} curse into a blessing, because the \divine{Lord} your God loves you. \v{6}Don't seek a peace treaty with them as long as you live. \v{7}Don't detest Edomites, since they are related to you. Don't detest Egyptians, either, because you were strangers in their land. \v{8}Their grandchildren\fnote{Lit. \fbib{Children born in the third generation}} may participate in the assembly of the \divine{Lord}.''
\passage{Community Sanitation}

\v{9}``When you are encamped for battle against your enemies, be on guard against every form of impropriety. \v{10}If someone among you becomes unclean due to nocturnal emissions, he must leave the camp and stay outside. \v{11}As evening approaches he must wash himself with water. Then at sunset, he may return to the camp.

\v{12}``Choose a place outside the camp for a latrine. \v{13}Include a spade among your equipment so that when you squat to relieve yourself, you can dig a hole and then cover your excrement. \v{14}For the \divine{Lord} your God is on the move within your camp to deliver you and to hand your enemies over to you. Therefore your camp must be holy so that he will not see anything indecent among you and turn away from you.''
\passage{Treatment of Slaves}

\v{15}``Don't hand over a slave who escaped from his master when he runs to you. \v{16}Let him live among you wherever he chooses in any of your cities that he likes. Don't mistreat him.''
\passage{Cultic Prostitution Prohibited}

\v{17}``There are to be no cultic prostitutes among the daughters or the sons of Israel. \v{18}Don't bring the earnings of a female prostitute nor the income of a male prostitute into the house of the \divine{Lord} your God as payment for any vow. Both of these are detestable to the \divine{Lord} your God.''
\passage{Fair Dealings}

\v{19}``Don't charge interest to your relatives, whether for money, food, or for anything that has been loaned at interest. \v{20}You may charge interest to a foreigner, but don't charge interest to your relatives, so the \divine{Lord} your God may bless you in everything you undertake in the land that you are about to enter and possess.

\v{21}``When you make a vow to the \divine{Lord} your God, don't delay paying it, because the \divine{Lord} your God will certainly demand payment from you, and then you will be guilty of sin. \v{22}But if you refrain from making a vow, then you won't be guilty. \v{23}Be sure you do whatever you promise, because you have given your word voluntarily to the \divine{Lord} your God.

\v{24}``When you enter your countrymen's vineyard, you may eat the grapes to your satisfaction, but don't take any in a basket. \v{25}When you enter your countrymen's grain fields, you may pluck the grain with your hand, but don't put a sickle to his standing grain.''
\labelchapt{24}
\passage{Various Laws}

\chapt{24}
\v{1}``If a man chooses to enter into marriage with a woman, but she finds herself displeasing to him because he has found something objectionable\fnote{Lit. \fbib{naked}; i.e. \fbib{indecent}} about her, he must draw up divorce papers, hand them to her, and then send her out of his house. \v{2}If she goes out from his house, becomes the wife of another man, \v{3}and this second husband\fnote{Lit. \fbib{this other man}} dislikes her, he, also, must draw up divorce papers, hand them to her, and then send her away from his house. Should the second husband die, \v{4}her first husband who married her and divorced her earlier must not remarry her,\fnote{Lit. \fbib{not take her to live with him as wife}} because she was defiled, since this is detestable to the \divine{Lord}. Don't defile the land that the \divine{Lord} your God is about to give you as a possession.

\v{5}``When a man is newly married, he must not be sent out to war or have a related duty placed on him. Let him stay home for one year and be happy with his wife whom he has married.

\v{6}``Don't take a pair of millstones, especially the upper millstone, as collateral for a loan, because this means taking a man's\fnote{Lit. \fbib{taking his}} livelihood.

\v{7}``If a man is found kidnapping his relative, a fellow Israeli, and mistreats or sells him, that kidnapper must die. By doing this, you will remove this evil from among you.

\v{8}``In cases of leprosy, be very careful to observe exactly what the Levitical priests instructed you. Carefully follow what I have commanded them. \v{9}Remember what the \divine{Lord} your God did to Miriam along the way as you were coming out of Egypt.''
\passage{Respecting the Poor}

\v{10}``When you loan something to your neighbor, don't enter his house to seize what he offered as collateral. \v{11}Stay outside and let the man to whom you made the loan bring it\fnote{Lit. \fbib{the collateral}} out to you. \v{12}If he is a poor man, don't go to sleep with his collateral in your possession.\fnote{The Heb. lacks \fbib{in your possession}} \v{13}Be sure to return his garment\fnote{Lit. \fbib{collateral}} to him at sunset so that he may sleep with it, and he will bless you. It will be a righteous deed in the presence of the \divine{Lord} your God. \v{14}Don't take advantage of a hired person who is poor and needy, whether he's your fellow citizen or a foreigner who lives in your city. \v{15}Pay his wages that same day before the sun sets, because he is poor and his livelihood\fnote{Lit. \fbib{life}} depends on it. Otherwise, he may cry out to the \divine{Lord} against you, and you will incur guilt.''
\passage{Practicing Justice}

\v{16}``Fathers are not to be put to death on account of their children's sin; nor are children to die on account of their fathers' sin. Each person is to be put to death for his own sin.

\v{17}``Don't deny justice to a foreigner or to an orphan, nor take a widow's garment as collateral for a loan. \v{18}Remember to observe this because you were slaves in Egypt, and the \divine{Lord} your God redeemed you from there. That is why I am commanding you to do this.

\v{19}``When you are reaping in the field, and you overlook a sheaf, don't return to get it. Let it remain for the foreigner, the orphan, or the widow, in order that the \divine{Lord} your God may bless everything you undertake. \v{20}When you harvest the olives from your trees, don't go back to the branches a second time. What remains is for the foreigner, the orphan, or the widow. \v{21}When you harvest the grapes in your vineyard, don't go back a second time. What remains are for the foreigner, the orphan, or the widow. \v{22}Remember to do this because you were slaves in the land of Egypt. That is why I'm commanding you to do this.''
\labelchapt{25}
\passage{Limitations on Punishment}

\chapt{25}
\v{1}``When there is a conflict between individuals, let them come to court to judge the case, decide who is innocent, and condemn the guilty person. \v{2}If the guilty person deserves a beating, the judge will make him lie down and be beaten in his presence with the number of lashes fit for his crime. \v{3}But he must not be beaten more than 40 lashes, because if he receives more than 40 lashes, your brother will be humiliated in your eyes.

\v{4}``Don't muzzle an ox while it is threshing grain.''
\passage{Levirate Marriage}

\v{5}``When two brothers are living together and one of them dies without leaving a son, his widow must not be married outside the family to a foreigner. Instead, the brother-in-law must go to her, take her as his wife, and by doing so perform the duty of a brother-in-law. \v{6}The firstborn whom she will bear will continue the name of the dead brother, so his name will not be erased from Israel. \v{7}But if the man does not want to marry his brother's widow, then she\fnote{Lit. \fbib{the brother's wife}} must go to the elders at the city gate and declare, `My husband's brother refuses to perform the duty of a brother-in-law in order to preserve the name of his brother in Israel. He is not willing to perform the duty of a brother-in-law.' \v{8}Then the elders of the city are to summon him and speak with him. If he insists on saying, `I don't want to marry her,' \v{9}then she is to approach her brother-in-law in the presence of the elders, remove his sandal, spit in his face, and say in response, `May this be done to the man who does not preserve the lineage\fnote{Lit. \fbib{house}} of his brother.' \v{10}Then his family name in Israel will be known `as the family of the one whose sandal was removed.'\,''
\passage{Limiting a Wife's Defense}

\v{11}``If two men are fighting together, and the wife of one comes to rescue her husband from the grasp of his assailant, and she reaches out and seizes his genitals, \v{12}you are to cut off her hand. Don't show any pity.''
\passage{Honest Weights}

\v{13}``Don't have different weights in your bag---one heavy and one light. \v{14}Don't have different measuring devices in your house---one large and one small. \v{15}You must have honest weights and measuring devices,\fnote{Lit. \fbib{and an honest and fair device}} so you may live long in the land that the \divine{Lord} your God is about to give you, \v{16}for anyone who does these things---anyone who deals dishonestly---is detestable to the \divine{Lord} your God.''
\passage{Annihilation of the Amalekites}

\v{17}``Remember what the Amalekites did to you along the road while you were coming out of Egypt, \v{18}how when you were very tired and weary, they lay in wait for you on the road and eliminated everyone who was lagging behind. They had no fear of God. \v{19}Therefore, when the \divine{Lord} your God has given you rest from all your enemies who surround you in the land that he\fnote{Lit. \fbib{that the \divine{Lord} your God}} is about to give you to possess as an inheritance, you must completely erase the memory of the Amalekites from under heaven. Don't forget!''
\labelchapt{26}
\passage{Gift of the First Produce}

\chapt{26}
\v{1}``When you arrive in the land that the \divine{Lord} your God is about to give you as an inheritance, take possession of it and settle in it. \v{2}Gather all the first produce of the ground that you harvest from your land that the \divine{Lord} your God is about to give you, place it in a basket, and bring it to the place where the \divine{Lord} your God will choose to establish his name. \v{3}Approach the priest who is in charge at that time and say to him, `I acknowledge today to the \divine{Lord} your God that I've arrived in the land that the \divine{Lord} promised our ancestors to give us.' \v{4}Then the priest will take the basket from you\fnote{Lit. \fbib{your hand}} and place it in front of the altar of the \divine{Lord} your God. \v{5}Then you are to affirm and declare in the presence of the \divine{Lord} your God:

\begin{poetry}
\poeml `A wandering Aramean was my ancestor, who went down to Egypt and traveled there with very few family members,\fnote{The Heb. lacks \fbib{family members}} yet there he became a great, powerful, and populous nation. \v{6}But the Egyptians oppressed us, afflicted us, and assigned us to hard labor. \v{7}So we cried out to the \divine{Lord} God of our ancestors, and he\fnote{Lit. \fbib{the \divine{Lord}}} heard our cries and observed our affliction, trouble, and oppression. \v{8}The \divine{Lord} brought us out of Egypt with his awesome power,\fnote{Lit. \fbib{his mighty hand and outstretched arm}} with great terror, signs, and wonders. \v{9}And then we arrived at this place, and he gave this land to us, flowing with milk and honey. \v{10}Now, look---I brought the first produce of the land that you, \divine{Lord}, have given me.'
\end{poetry}

Then set it in the presence of the \divine{Lord} your God and worship him.\fnote{Lit. \fbib{worship before the \divine{Lord} your God}} \v{11}Rejoice with the descendants of Levi and the foreigner among you at all the good things that the \divine{Lord} your God has given you and your family.''
\passage{Levitical Tithes}

\v{12}``When you have finished your harvest, reserve the tithe in the third year (the year of the tithe), and give the entire tithe to the descendants of Levi, to the foreigners, to the orphans, and to the widows, so they may eat and be satisfied in your cities. \v{13}Then declare in the presence of the \divine{Lord} your God:

\begin{poetry}
\poeml `I've removed the holy offering from my house and given it to the descendants of Levi, to the foreigners, to the orphans, and to the widows just as you have commanded me. I haven't violated or forgotten your commands. \v{14}I haven't eaten any part of it while mourning, nor removed any part of it while unclean, nor offered any of it to the dead. I've obeyed the voice of the \divine{Lord} my God and did all that he commanded me. \v{15}Look down from your holy habitation in heaven and bless your people Israel and the land that you have given us, just as you promised our ancestors---a land flowing with milk and honey.'\,''
\end{poetry}
\passage{Living for the Glory of God}

\v{16}``The \divine{Lord} your God is commanding you this very day to observe these statutes and judgments. Be careful to obey them with all your heart and soul. \v{17}You have declared this very day that the \divine{Lord} will be your God. You are to walk in his ways, keep his statutes, commands, and judgments, and obey his voice. \v{18}The \divine{Lord} affirmed this day that you are his prized possession. Therefore observe his commands, \v{19}so he may elevate you far above all the nations that he has made. Then you will live to the praise, fame, and glory of God,\fnote{The Heb. lacks \fbib{of God}} and so be a nation that is holy to the \divine{Lord} your God, as he has promised.''
\labelchapt{27}
\passage{Stone Memorials}

\chapt{27}
\v{1}Moses and the elders of Israel gave these orders to the people: ``Observe all of the commandments\fnote{So LXX. MT reads \fbib{commandment}} that I'm giving\fnote{Lit. \fbib{commanding}} you today. \v{2}On the day you cross over the Jordan River to the land that the \divine{Lord} your God is about to give you, set up large stones and coat them with plaster. \v{3}Then inscribe on them all the words of this law when you've crossed over into the land that the \divine{Lord} your God is about to give you---a land flowing with milk and honey---just as the \divine{Lord} God of your ancestors promised you.

\v{4}``When you have crossed the Jordan River, set up these stones about which I'm commanding you today on Mount Ebal, and coat them with plaster. \v{5}Then build an altar there to the \divine{Lord} your God, an altar of stones that hasn't been worked with iron tools. \v{6}Build the altar to the \divine{Lord} your God with uncut stones, then offer a burnt offering to him.\fnote{Lit. \fbib{the \divine{Lord} your God}} \v{7}Offer a burnt offering there, then eat and rejoice in the presence of the \divine{Lord} your God. \v{8}Inscribe on the stones plainly and distinctly\fnote{Lit. \fbib{and make good}} all the words of this Law.''

\v{9}Then Moses and the Levitical priests spoke to Israel. They said, ``Be quiet and listen, Israel! Today you have become the people of the \divine{Lord} your God. \v{10}Listen to his voice\fnote{Lit. \fbib{voice of the \divine{Lord} your God}} and carry out his commands and statutes that I'm giving\fnote{Lit. \fbib{commanding}} you today.''
\passage{Penalties for Disobedience}

\v{11}Moses gave the people these commands that day:

\begin{poetry}
\poeml \v{12}``When you cross the Jordan River, these tribes\fnote{The Heb. lacks \fbib{tribes}} are to stand on Mount Gerizim to bless the people---Simeon, Levi, Judah, Issachar, Joseph, and Benjamin. \v{13}The tribes of\fnote{The Heb. lacks \fbib{the tribes of}} Reuben, Gad, Asher, Zebulun, Dan, and Naphtali are to stand on Mount Ebal to pronounce the curse. \v{14}The descendants of Levi are to declare in a loud voice to every Israeli: \\
\poeml \v{15}```Cursed is the one\fnote{Lit. \fbib{man}; and so through v. 26} who makes a sculptured or cast image---a detestable thing to the \divine{Lord}, the work of a craftsman---and sets it up secretly.' \\
\poeml ``Then all the people are to respond by saying, `Amen!' \\
\poeml \v{16}``Cursed is the one who treats his father and mother with dishonor.' \\
\poeml ``Then all the people are to respond by saying, `Amen!' \\
\poeml \v{17}```Cursed is the one who moves his neighbor's boundary stone.' \\
\poeml ``Then all the people are to respond by saying, `Amen!' \\
\poeml \v{18}```Cursed is the one who misleads a blind person on the road. \\
\poeml ``Then all the people are to respond by saying, `Amen!' \\
\poeml \v{19}```Cursed is the one who perverts justice due the foreigner, the orphan, or the widow.' \\
\poeml ``Then all the people are to respond by saying, `Amen!' \\
\poeml \v{20}``Cursed is the one who has sexual relations with his father's wife, because he has disgraced his father.\fnote{Lit. \fbib{has uncovered his father's garment}} \\
\poeml ``Then all the people are to respond by saying, `Amen!' \\
\poeml \v{21}```Cursed is the one who has sexual relations with any animal. \\
\poeml ``Then all the people are to respond by saying, `Amen!' \\
\poeml \v{22}```Cursed is the one who has sexual relations with his sister, the daughter of his father or mother. \\
\poeml ``Then all the people are to respond by saying, `Amen!' \\
\poeml \v{23}```Cursed is the one who has sexual relations with his mother-in-law. \\
\poeml ``Then all the people are to respond by saying, `Amen!' \\
\poeml \v{24}```Cursed is one who strikes his neighbor secretly. \\
\poeml ``Then all the people are to respond by saying, `Amen!' \\
\poeml \v{25}```Cursed is one who accepts a bribe to kill an innocent person. \\
\poeml ``Then all the people are to respond by saying, `Amen!' \\
\poeml \v{26}```Cursed is the one who doesn't uphold the words of this Law and observe them. \\
\poeml ``Then all the people are to respond by saying, `Amen!'\,''
\end{poetry}
\labelchapt{28}
\passage{Rewards for Obedience}

\chapt{28}
\v{1}``Indeed, if you diligently obey\fnote{Or \fbib{listen to}} the \divine{Lord} your God to carry out all his commands that I'm giving you today, then the \divine{Lord} your God will set you high above all the nations of the earth. \v{2}Moreover, all these blessings will come upon you in abundance,\fnote{Lit. \fbib{and will overtake you}} if you obey the \divine{Lord} your God:

\v{3}``Blessed will you be in the city and blessed will you be in the country.

\v{4}``Blessed will your children\fnote{Lit. \fbib{shall the fruit of your womb}} be, as well as the produce of your land, the offspring of your beasts and cattle, and the offspring of your flock.

\v{5}``Blessed will be your grain\fnote{The Heb. lacks \fbib{grain}} basket and your kneading bowl.

\v{6}``Blessed will you be in your comings and goings.''

\v{7}``The \divine{Lord} will make your enemies, who rise against you and attack from one direction, to flee from you in seven directions.

\v{8}``The \divine{Lord} will send blessings for you with regard to your barns and everything you undertake. Indeed, he will bless you in the land that the \divine{Lord} your God is about to give you.

\v{9}``The \divine{Lord} will assign you to be a holy people\fnote{Or \fbib{nation}} for himself, just as he promised you, as long as you keep his\fnote{Lit. \fbib{of the Lord your God}} commands and walk in his ways.

\v{10}``Then all the people of the earth will observe that the name of the \divine{Lord} is proclaimed\fnote{Lit. \fbib{called}} among you, and they will fear you.

\v{11}``The \divine{Lord} will show his abundant goodness with respect to your children,\fnote{Lit. \fbib{the fruit of your womb}} the offspring of your animals, and the produce of your farmland that he\fnote{Lit. \fbib{the \divine{Lord}}} promised your ancestors he would give you.

\v{12}``The \divine{Lord} will open his rich\fnote{Or \fbib{good}} treasury, the heavens, to release rain upon your land in season and bless everything you undertake so that you'll lend to many nations but won't borrow.

\v{13}``The \divine{Lord} your God will make you the head and not the tail---placing you above and not beneath---if you obey the commands of the \divine{Lord} your God that I'm giving you today to keep and observe. \v{14}Do not deviate from any of his commands that I'm giving you today---neither to the right nor the left---to follow and serve other gods.''
\passage{Reversal of Blessings}

\v{15}``But if you don't obey the \divine{Lord} your God and faithfully carry out all his commands and statutes that I'm giving you today, then all these curses will come upon you and overwhelm you.

\v{16}``Cursed will you be in the city and cursed will you be in the country.

\v{17}``Cursed will be your grain\fnote{The Heb. lacks \fbib{grain}} basket and your kneading bowl.

\v{18}``Cursed will your children\fnote{Lit. \fbib{shall the fruit of your womb}} be, as well as the produce of your land, the offspring of your beasts and cattle, and the offspring of your flock.

\v{19}``Cursed will you be in your comings and goings.''
\passage{Diseases and Drought}

\v{20}``The \divine{Lord} will send the curse among you, will confuse you, and will rebuke you in everything you undertake until you are destroyed and perish quickly because of your evil deeds, since you will have forsaken him.\fnote{Lit. \fbib{me}} \v{21}The \divine{Lord} will cause you to be ill with long-lasting diseases until you are wiped out from the land that you are entering to possess. \v{22}The \divine{Lord} will afflict you with tuberculosis, fever, inflammation, high fever, drought, blight, and mildew. These will attack you until you are completely destroyed. \v{23}The sky above your head will become bronze while the ground beneath you will become iron. \v{24}The \divine{Lord} will change the rain on your land to powder and dust. It will come down from the sky until you are exterminated.''
\passage{From Defeat to Exile}

\v{25}``The \divine{Lord} will cause you to be defeated\fnote{Lit. \fbib{be struck down}} by your enemies. You'll go out against them in one direction, but you'll flee from them in seven directions. Consequently, you'll be in a state of great terror throughout all the kingdoms of the earth. \v{26}Your dead bodies will be food for the birds of the sky and the wild animals of the earth, with no one to chase them away.

\v{27}``The \divine{Lord} will afflict you with the boils of Egypt, with tumors, skin disease, and festering rashes, and none of them will be curable. \v{28}The \divine{Lord} will afflict you with insanity, blindness, and mental confusion.\fnote{Lit. \fbib{and confusion of the heart}} \v{29}As a result, you'll wander aimlessly in broad daylight just as a blind person wanders in darkness. You won't prosper in life.\fnote{Lit. \fbib{in your ways}} Instead you'll be oppressed and plundered all day long, with no deliverer.

\v{30}You'll be engaged to a woman, but another man will rape\fnote{Lit. \fbib{violate}} her. You'll build a house but you won't live in it. You'll plant a vineyard but you won't harvest\fnote{Or \fbib{enjoy}} it. \v{31}Your ox will be slaughtered in front of you, and you won't be able to eat it. Your donkey will be stolen from you while you watch and won't be returned to you. Your flock of sheep will be handed to your enemies and there will be no deliverer. \v{32}Your sons and daughters will be given to another people while you watch, and you won't be able to approach them at all,\fnote{Lit. \fbib{all the day}} and you'll be powerless to help.\fnote{Lit. \fbib{and there will be no power in your hand}}

\v{33}``A people whom you don't know will devour what your land and labor produces. You'll be only oppressed and discouraged continuously \v{34}until you are driven insane from what your eyes will see.

\v{35}``The \divine{Lord} will inflict you with incurable boils on your knees and legs, and from the sole of your foot to the top of your head.

\v{36}``The \divine{Lord} will banish you and your king whom you will appoint over you to go to a nation that neither you nor your ancestors have known, and there you'll serve other gods of wood and stone. \v{37}You'll become a desolation and a proverb, and you'll be mocked among the people where the \divine{Lord} will drive you.''
\passage{Complete Reversal}

\v{38}``You'll plant many seeds in a field, but your harvest will be small because the locust will consume it. \v{39}You'll plant a vineyard, but you won't drink wine or harvest any grapes, because worms will consume it. \v{40}You'll have olive trees throughout your territory, but you won't be able to anoint yourself with oil, because the olives will drop off the trees. \v{41}You'll bear sons and daughters, but they won't belong to you, because they'll go into captivity. \v{42}Whirling locusts will consume every tree and the produce of your land. \v{43}The foreigner in your midst will be elevated higher and higher over you, while you are brought low little by little. \v{44}He will lend to you, but you won't lend to him. He'll be the head, but you'll be the tail. \v{45}All these curses will come upon you and will overwhelm you until you are exterminated, because you didn't obey\fnote{Lit. \fbib{listen to the voice}} the \divine{Lord} your God to keep his commands and statutes, which he had commanded you. \v{46}These curses\fnote{Heb. lacks \fbib{curses}} will serve as a sign and wonder for you and your descendants\fnote{Lit. \fbib{seed}} as long as you live.''\fnote{Lit. \fbib{until eternity}}
\passage{Servitude and Bondage}

\v{47}``Because you didn't serve the \divine{Lord} your God joyfully and wholeheartedly,\fnote{Lit. \fbib{and with gladness of heart}} despite the abundance of everything you have, \v{48}you'll serve your enemies whom the \divine{Lord} your God will send against you. You will serve in famine and in drought,\fnote{Or \fbib{in hunger and thirst}} in nakedness, and in lack of everything. They'll\fnote{Lit. \fbib{he}} set a yoke of iron upon your neck until they\fnote{Lit. \fbib{he}} have exterminated you.

\v{49}``The \divine{Lord} will raise a distant nation against you from the other side of the earth. Swooping down like a vulture, \v{50}it will be a nation whose language you don't understand, whose\fnote{Lit. \fbib{a nation}} stern appearance\fnote{Or \fbib{face}} neither shows regard\fnote{Lit. \fbib{who does not lift faces}} nor extends grace to anyone whether old or young. \v{51}Its army\fnote{Heb. lacks \fbib{army}} will consume the offspring of your animals and the produce of your soil until you are exterminated. They\fnote{Lit. \fbib{it}} will leave you without your grain, wine, oil, the increase of your cattle, and the lamb of your flock, until you are completely destroyed. \v{52}They'll\fnote{Lit. \fbib{it}} besiege all your cities until your high and fortified walls in which you have trusted collapse throughout the land. Indeed, they will besiege all your cities, which the \divine{Lord} your God gave you.''
\passage{Cannibalism}

\v{53}``You'll eat your own children\fnote{Lit. \fbib{eat the fruit of your womb}}---the flesh of your sons and daughters, whom the Lord your God gave you---on account of the siege and the distress with which your enemy will oppress you. \v{54}Even the compassionate man among you---the very sensitive one---will look with evil in his eyes toward his brother, his beloved wife, and his surviving sons, whom he spared. \v{55}He will withhold from each of them the flesh of his sons that he is eating---since there will be nothing left---on account of the siege and distress with which your enemy will oppress you in all your cities. \v{56}The most tender and sensitive lady among you, who doesn't venture to touch the soles of her feet to the ground on account of her daintiness, will look with hostility in her eyes against her beloved husband, her sons, and her daughters. \v{57}She will eat her afterbirth\fnote{Lit. \fbib{will begrudge that which comes out from between her feet}} and her newborn children\fnote{Lit. \fbib{sons whom she will bear}} secretly---since there will be nothing left---on account of the siege and distress with which your enemy will oppress you in your cities.''
\passage{Reduction in Population}

\v{58}``If you aren't careful to observe all the words of this Law that have been written in this book, instructing you\fnote{The Heb. lacks \fbib{instructing you}} to fear this glorious and awesome name of the \divine{Lord} your God, \v{59}then he\fnote{Lit. the \fbib{\divine{Lord}}} will inflict extraordinary plagues on you and your children, great and lasting plagues, and severe and lasting illnesses. \v{60}He will inflict\fnote{Lit. \fbib{will return}} on you all the diseases of Egypt that you dreaded, and they won't be curable.\fnote{Lit. \fbib{they will cling to you}} \v{61}Moreover, the \divine{Lord} will inflict you with illnesses and plagues that were not written in this Book of the Law, until you are exterminated. \v{62}Because you will not have obeyed\fnote{Lit. \fbib{listen to the voice of}} the \divine{Lord} your God, very few of you will be left---instead of you being as numerous as the stars in the heavens. \v{63}Just as the \divine{Lord} delighted to prosper and increase you, so now the \divine{Lord} will delight to destroy, exterminate, and banish you from the land that you are about to enter to possess.''
\passage{Scattering among the Nations}

\v{64}``He'll\fnote{Lit. \fbib{the \divine{Lord}}} scatter you among the nations\fnote{Lit. \fbib{peoples}} from one end of the earth to the other,\fnote{Lit. \fbib{end of the earth}} and there you'll serve other gods made of wood and stones, which neither you nor your ancestors have known. \v{65}Among those nations you'll have no rest. There'll be no resting place for the soles of your feet. Instead, the \divine{Lord} will give you an anxious heart, failing eyesight, and a despairing spirit. \v{66}You'll cling to life, being fearful by both night and day, with no assurance of survival. \v{67}In the morning you'll say, `I wish it were evening.' Yet in the evening you'll say, ``I wish it were morning,'' on account of what you'll dread\fnote{Lit. \fbib{the dread of your heart that you will dread}} and what you'll see.\fnote{Lit. \fbib{the vision of your eyes that you will see}} \v{68}Finally, the \divine{Lord} will bring you back to Egypt by ship, a place that I said you'll never see again. There you'll try to sell yourselves to your enemies as male and female slaves, but no one will buy you.''
\labelchapt{29}
\passage{Remembering the Exodus and Conquest}

\chapt{29}
\v{1}\fnote{This v. is 28:69 in MT}These are the terms of the covenant that the \divine{Lord} commanded Moses to make with the Israelis in the land of Moab in addition to the covenant that he made with them in Horeb. \v{2}\fnote{This v. is 29:1 in MT, and so throughout the chapter}Moses called all Israel together and addressed them: ``You saw everything that the \divine{Lord} did before your eyes in the land of Egypt to Pharaoh, to all his servants,\fnote{Lit. \fbib{his slaves}} and to his whole country. \v{3}Those great feats that you saw with your own eyes are signs and great wonders. \v{4}Yet to this day, the \divine{Lord} hasn't given you a heart that understands, eyes that perceive, and ears that discern. \v{5}Though I've led you for 40 years in the desert, neither your clothes nor your shoes have worn out. \v{6}You didn't have bread to eat or wine or anything intoxicating to drink, so that you would learn\fnote{Or \fbib{know}} that I am the \divine{Lord} your God. \v{7}Then you reached this place, where King Sihon of Heshbon and King Og of Bashan had come out to meet and fight with us, but we defeated them. \v{8}We captured their land and handed it as an inheritance to the descendants of Reuben, the descendants of Gad, and half the tribe of Manasseh. \v{9}Therefore, keep the terms of this covenant, carrying them out so that you'll be wise in everything you do.''
\passage{Entering into a Covenant Relationship}

\v{10}``All of you are standing today in the presence of the \divine{Lord} your God---the heads of your tribes, your elders, your magistrates, all the men of Israel, \v{11}along with your children, your wives, even the foreigner in your camp, including the woodchopper and the water drawer---\v{12}to enter into a covenant with the \divine{Lord} your God and into the oath that he\fnote{Lit. \fbib{the Lord your God}} is about to make with you today, \v{13}so that he will elevate you to be a people for him. And he will be God to you, just as he promised you and swore to your ancestors Abraham, Isaac, and Jacob. \v{14}Now, I'm not making this covenant and oath with you alone, \v{15}but with whoever is here with us standing in the presence of the \divine{Lord} our God today, as well as with those who aren't here with us today.''
\passage{Incurring the Judgment of God}

\v{16}``Now, you know how we lived in the land of Egypt and how we traveled through the territory of\fnote{The Heb. lacks \fbib{the territory of}} other nations. \v{17}You have seen their detestable practices, their idols of wood, stone, silver, and gold that they had with them. \v{18}Be alert so there is no man, woman, family, or a tribe whose heart is turning away from the \divine{Lord} your God to go and serve the gods of those nations. Be alert so there will be no root among you that produces poisonous and bitter fruit,\fnote{Lit. \fbib{wormwood}; i.e. bitter things} \v{19}because when such a person\fnote{Lit. \fbib{he}} hears the words of this oath, he will bless himself and say:

\begin{poetry}
\poeml `I will have a peaceful life, even though I'm determined to be stubborn.'\fnote{The quotation possibly ends here.} By doing this he will be sweeping away both watered and parched ground alike.'
\end{poetry}

\v{20}``The \divine{Lord} won't forgive such a person.\fnote{Lit. \fbib{him}} Instead, the zealous anger of the \divine{Lord} will blaze against him. All the curses that were written in this book will fall on him. Then the \divine{Lord} will wipe out his memory\fnote{Lit. \fbib{name}} from under heaven. \v{21}The \divine{Lord} will set him apart from all the tribes of Israel for destruction,\fnote{Lit. \fbib{evil}} according to the curses of the covenant that were written in this Book of the Law.''
\passage{A Reminder of Sodom and Gomorrah}

\v{22}``Then the generation to come---your descendants after you and the foreigners who come from afar---will see plagues and illnesses infecting the land that the \divine{Lord} will inflict on it. \v{23}The whole land will be covered\fnote{The Heb. lacks \fbib{will be covered}} with salt pits and burning sulfur, with nothing planted, nothing sprouting, and producing no vegetation---overthrown like Sodom, Gomorrah, Admah, and Zeboiim, when the \divine{Lord} overthrew them in his raging fury. \v{24}All the nations will ask, `Why did the \divine{Lord} do this to this land? What is the meaning of this fierce and great anger?' \v{25}Then they will answer themselves,\fnote{The Heb. lacks \fbib{themselves}}

\begin{poetry}
\poeml `Because they've abandoned the covenant of their \divine{Lord}, the God of their ancestors that he had made with them when he brought them out of Egypt. \v{26}They followed and worshipped other gods whom they had not known and whom he did not assign to them. \v{27}For this reason, the anger of the \divine{Lord} raged against this land, to bring upon it all the curses that were written in this book. \v{28}The \divine{Lord} uprooted them from the land in his anger, wrath, and great fury, deporting them to another land, and that's the way things are today.'
\end{poetry}

\v{29}``The secret things belong to the \divine{Lord} our God, but what has been revealed belongs to us and to our children forever, so that we might observe the words of this Law.''
\labelchapt{30}
\passage{Restoration after the Exile}

\chapt{30}
\v{1}``When all these things happen to you---both the blessings and the curses that I've presented to you---and you take them seriously\fnote{Lit. \fbib{you cause them to return to your heart}} in all the nations where the \divine{Lord} your God will deport you, \v{2}and when you---you and your descendants, that is---will have returned to him\fnote{Lit. \fbib{the \divine{Lord} your God}} and obeyed all the commands that I'm giving you today with all your heart and soul, \v{3}then the \divine{Lord} your God will restore your fortunes and will show compassion to you. He will gather you from among the nations\fnote{Lit. \fbib{peoples}} where he\fnote{Lit. \fbib{the \divine{Lord} your God}} had scattered you. \v{4}Even if the \divine{Lord} had banished you to the ends of the heavens, the \divine{Lord} your God will gather you from there \v{5}and he'll\fnote{Lit. \fbib{\divine{Lord} your God}} bring you to the land that your ancestors inherited. You'll possess it, you'll prosper, and you'll greatly multiply more than your ancestors did. \v{6}Then the \divine{Lord} your God will circumcise both your hearts and those\fnote{Lit. \fbib{and the heart}} of your descendants so that you can love him\fnote{Lit. \fbib{the \divine{Lord} your God}} with your heart and with your soul and therefore live. \v{7}Then the \divine{Lord} your God will inflict all these curses on your enemies and on those who hate and persecute you.''
\passage{Prosperity in Obedience}

\v{8}``So now, return and obey the \divine{Lord} your God and observe all his commands that I'm giving you today, \v{9}and the \divine{Lord} your God will prosper you abundantly in all that you do, along with your children,\fnote{Lit. \fbib{fruit of your womb}} your livestock, and the produce of your fields, because the \divine{Lord} your God will again be delighted with you for good, just as he was delighted with your ancestors, \v{10}if you obey him\fnote{Lit. \fbib{the \divine{Lord} your God}} and keep his commands and statutes that are written in this Book of the Law, and if you return to him\fnote{Lit. \fbib{the \divine{Lord} your God}} with all your heart and soul. \v{11}Indeed, these commands that I'm giving you today are neither confusing nor unattainable for you. \v{12}They aren't in the heavens, so you have to ask, `Who'll go up to the heavens for us and get it for us so we can hear it and act on it?' \v{13}And they aren't beyond the seas either, so you have to ask, `Who'll cross the sea and get it for us so we can hear it and act on it?' \v{14}No, the word is very near you---it's within your mouth and heart for you to attain.''
\passage{Destruction in Disobedience}

\v{15}``Look! Today I have set before you life and what is good, along with death and what is evil. \v{16}That's why I'm commanding you today to love the \divine{Lord} your God by walking in his ways and by observing his commands, statutes, and ordinances, so that you may live long, increase, and so that the \divine{Lord} your God may bless you in the land that you are about to enter to possess. \v{17}But if you turn your heart away, and do not obey, but instead if you stray away to worship and serve other gods, \v{18}I'm declaring to you today that you will surely be destroyed. You won't live long\fnote{Lit. \fbib{your days will not be long}} in the land that you are crossing the Jordan River to enter and possess. \v{19}I call heaven and earth to testify against you today! I've set life and death before you today: both blessings and curses. Choose life, that it may be well with you---you and your children. \v{20}Love the \divine{Lord} your God, obey his voice, and cling to him, because he is your life---even your long life---so that you may live in the land that the \divine{Lord} promised to give Abraham, Isaac, and Jacob.''
\labelchapt{31}
\passage{Moses Commissions Joshua}

\chapt{31}
\v{1}Moses went and explained these things to everyone in Israel. \v{2}Then he concluded, ``I'm now 120 years old. I'm not able to get around anymore, \v{3}and the \divine{Lord} told me, `You won't be crossing the Jordan River.' But the \divine{Lord} your God is crossing over before you. He will destroy these nations in front of you and you will dispossess them. As for Joshua, he will cross over before you, just as the \divine{Lord} promised. \v{4}The \divine{Lord} will do to them just as he did to Sihon and Og, the kings of the Amorites, and to their land when he destroyed them. \v{5}The \divine{Lord} will hand them over to you, so you can do to them what I've instructed you to do. \v{6}Be strong and courageous. Don't fear or tremble before them, because the \divine{Lord} your God will be the one who keeps on walking with you---he won't leave you or abandon you.''

\v{7}Then Moses called on Joshua and told him in the presence of everyone in Israel, ``Be strong and courageous, because you'll bring this people to the land that the \divine{Lord} your God had promised to give your ancestors. You will be the one who causes them to possess it. \v{8}Indeed, the \divine{Lord} is the one who will keep on walking in front of you. He'll be with you and won't leave you or abandon you, so never be afraid and never be dismayed.''
\passage{Moses Entrusts the Law to the Levitical Priests}

\v{9}Then Moses wrote down this Law and gave it to the Levitical priests who carry the Ark of the Covenant of the \divine{Lord} and to all of Israel's leaders.\fnote{Lit. \fbib{elders}} \v{10}Then he\fnote{Lit. \fbib{Moses}} gave these\fnote{The Heb. lacks \fbib{these}} orders: ``At the end of seven years, the year designated for release,\fnote{Or \fbib{remission}} during the Festival of Tents,\fnote{Or \fbib{Tents}} \v{11}when all of Israel comes to appear in the presence of the \divine{Lord} your God at the place that he'll choose, read this Law aloud to them.\fnote{Lit. \fbib{to all Israel}} \v{12}Gather the people---the men, women, children, and the foreigners that live in your cities---so they may hear and fear the \divine{Lord} your God, and so they may be careful to obey the words contained in this Law. \v{13}Their children who don't know will hear and learn to fear the \divine{Lord} your God as long as you live in the land that you are crossing the Jordan River to possess.''
\passage{Moses and Joshua Present Themselves to the \divine{Lord}}

\v{14}Then the \divine{Lord} told Moses: ``Look! Because your time to die is approaching, call Joshua, present yourselves at the Tent of Meeting, and then I will commission him.'' Moses and Joshua complied\fnote{Lit. \fbib{walked}} and presented\fnote{Or \fbib{stood}} themselves at the Tent of Meeting. \v{15}So the \divine{Lord} appeared at the tent in a pillar of cloud that stood above the entrance.\fnote{Lit. \fbib{above the Tent of Meeting}}

\v{16}Then the \divine{Lord} told Moses, ``Look! You are about to join your ancestors.\fnote{Lit. \fbib{to lie down with}} Afterwards, this people will rebel\fnote{Heb. \fbib{rise up}} and commit prostitution with the foreign gods of the land that they are about to enter to possess. They will abandon me and break my covenant that I made with them.\fnote{Lit. \fbib{him}} \v{17}When that happens,\fnote{Lit. \fbib{On that day}} my anger will burn against them,\fnote{Lit. \fbib{him}} because they will have abandoned me. I'll hide my face from them, they will be consumed, and many evils and distresses will find them. When this happens,\fnote{\fbib{On that day}} they\fnote{Lit. \fbib{he}} will say, `These troubles have happened to us because God isn't among us.' \v{18}I'll surely hide my face in that day on account of the evil that they\fnote{Lit. \fbib{he}} will have done for they\fnote{Lit. \fbib{he}} turned to other gods.''
\passage{Moses Instructed to Teach a Song}

\v{19}``Now write this song and teach it to the Israelis. Put this song in their very mouths, so that it will be a witness for me against the Israelis, \v{20}because after I've brought them to the land flowing with milk and honey that I promised to their ancestors by an oath, they'll eat, grow fat, and then they'll turn to other gods and serve them, while despising me and breaking my covenant. \v{21}Then, when many evils and troubles will have come upon them, this song will serve as a witness against them, since their descendants won't fail to sing it. I know the plan that they are devising even before I bring them into the land that I promised them\fnote{The Heb. lacks \fbib{them}} by an oath.''

\v{22}So Moses wrote the song that very day and taught it to the Israelis. \v{23}Then the \divine{Lord} charged Nun's son Joshua, ``Be strong and courageous, because you'll bring the Israelis to the land that I promised to them by an oath. I'll be with you.''

\v{24}When Moses had finished writing the words of this Law in a book, \v{25}he\fnote{Lit. \fbib{Moses}} gave this\fnote{The Heb. lacks \fbib{this}} charge to the descendants of Levi who carried the Ark of the Covenant of the \divine{Lord}: \v{26}``Take the book of this Law and set it beside the Ark of the Covenant of the \divine{Lord} your God. Let it remain there with you as witness against you, \v{27}because indeed I know your rebellion and stubbornness. Note that even while I'm still alive, you've been rebelling against the \divine{Lord}---how much more so after my death! \v{28}Gather together the leaders\fnote{Lit. \fbib{elders}} of your tribes and your foremen so I can speak these words in their hearing and call heaven and the earth as witnesses against them, \v{29}because I know that after my death, you'll surely act wickedly and turn from the road that I've instructed you. As a result, evil will fall on you in days to come, because you'll act wickedly in the sight of the \divine{Lord}, causing him to become angry due to your behavior.'' \v{30}So Moses spoke the words of this song---to the very end---in front of the entire assembly of Israel.
\labelchapt{32}
\passage{The Song of Moses}

\begin{poetry}
\poeml \chapt{32}
\v{1}Hear, heavens, and I will speak! \\
\poemll    Listen, earth, to the words of my mouth! \\
\poeml \v{2}May my instructions descend like rain \\
\poemll    and may my words flow like dew, \\
\poeml as light rain upon the grass, \\
\poemll    and as showers upon new plants. \\
\poeml \v{3}For I'll proclaim the name of our \divine{Lord}. \\
\poemll    Ascribe greatness to our God! \\
\poeml \v{4}Flawless is the work of the Rock, \\
\poemll    because all his ways are just. \\
\poeml A faithful God---never unjust--- \\
\poemll    righteous and upright is he. \\
\poeml \v{5}But those who are not his children \\
\poemll    acted corruptly against him; \\
\poemlll       they are a defective and perverted generation. \\
\poeml \v{6}This is not the way to repay the \divine{Lord}, is it, \\
\poemll    you foolish and witless people? \\
\poeml Is he not your father, \\
\poemll    who bought you, formed you, and established you?
\passage{An Exhortation to Remember God's Work}
\poeml \v{7}Remember the days of old, \\
\poemll    reflect on the years of previous generations. \\
\poeml Ask your father, \\
\poemll    and he'll tell you; \\
\poemlll       your elders will inform you. \\
\poeml \v{8}When the Most High gave nations as their inheritance, \\
\poemll    when he separated the human race, \\
\poeml he set boundaries for the people \\
\poemll    according to the number of the children of God.\fnote{So with LXX and DSS 4QDeut. MT reads \fbib{the Israelis}} \\
\poeml \v{9}For the \divine{Lord}'s portion is his people; \\
\poemll    Jacob is his allotted portion.
\passage{The \divine{Lord}'s Work on Behalf of Israel}
\poeml \v{10}The \divine{Lord}\fnote{Lit. \fbib{He}} found him\fnote{I.e. Jacob as a personification of national Israel; and so throughout the song} in a desert land, \\
\poemll    in a barren, eerie\fnote{Lit. \fbib{howling}} wilderness. \\
\poeml He surrounded, cared for, and guarded him \\
\poemll    as the pupil of his eye. \\
\poeml \v{11}Like an eagle stirs its nest, \\
\poemll    hovering near its young, \\
\poeml spreading out his wings to take him \\
\poemll    and carry him on his pinions, \\
\poeml \v{12}the \divine{Lord} alone guided him. \\
\poemll    There was no foreign god with him. \\
\poeml \v{13}He mounted him on a high place above the earth, \\
\poemll    feeding him from the produce of the field. \\
\poeml He nourished\fnote{Or \fbib{nursed}} him with honey from the rock \\
\poemll    and with oil from the flint rock, \\
\poeml \v{14}with curds from cattle and with milk from sheep, \\
\poemll    with the fat of lambs, with rams from Bashan, \\
\poeml with the fat of goats, with the finest\fnote{Lit. \fbib{kernel}} of wheat--- \\
\poemll    and from the juice of grapes you drank wine.
\passage{Israel's Rebellion}
\poeml \v{15}Jacob dined until satisfied;\fnote{So DSS Q Sam and LXX; the Heb. lacks \fbib{Jacob dined until satisfied}} \\
\poemll    Jeshurun\fnote{I.e. a poetic term for national Israel; the Heb. name means \fbib{Upright One}} grew fat and kicked. \\
\poeml He\fnote{Lit. \fbib{You}} grew fat, coarse, and gross, \\
\poemll    so that he abandoned the God who made him \\
\poemlll       and spurned the Rock that was his salvation. \\
\poeml \v{16}They provoked him to jealousy over foreigners \\
\poemll    and to anger over detestable things. \\
\poeml \v{17}They sacrificed to demons--- \\
\poemll    not to the real God--- \\
\poeml gods whom they didn't know, \\
\poemll    new neighbors who had recently appeared, \\
\poemlll       whom your ancestors never feared. \\
\poeml \v{18}You\fnote{I.e. the nation of Israel personified in the second person sing. pronoun, and so throughout the verse} neglected the Rock that fathered you; \\
\poemll    you abandoned God, who was awaiting your birth.\fnote{Or \fbib{who was giving birth to you}}
\passage{The \divine{Lord}'s Response}
\poeml \v{19}The \divine{Lord} saw it and became jealous,\fnote{So DSS, LXX. MT reads \fbib{and was repulsed}} \\
\poemll    provoked by his sons and daughters. \\
\poeml \v{20}So he said: \\
\poeml ``Let me hide my face from them. \\
\poemll    I will observe what their end will be, \\
\poeml because they are a perverted generation, \\
\poemll    children within whom there is no loyalty. \\
\poeml \v{21}They provoked me to jealousy over non-gods, \\
\poemll    and to be angry over their vanity. \\
\poeml Now I'll provoke them to jealousy over a non-people; \\
\poemll    and over a foolish nation I'll provoke them to anger. \\
\poeml \v{22}For a fire breaks out in my anger--- \\
\poemll    burning to the deepest part of\fnote{The Heb. lacks \fbib{part of}} the afterlife,\fnote{Lit. \fbib{Sheol}} \\
\poeml consuming the earth and its produce \\
\poemll    and igniting the foundations of the mountains. \\
\poeml \v{23}I'll bury them in misfortunes \\
\poemll    and bring them to an end with my arrows. \\
\poeml \v{24}Emaciated from famine, \\
\poemll    feverish from plague, \\
\poemll    and destroyed by bitterness, \\
\poeml I'll send fanged beasts against them, \\
\poemll    along with poisonous snakes that glide through the dust. \\
\poeml \v{25}Outside, the sword will cause bereavement; \\
\poemll    within,\fnote{Lit. \fbib{within the room}} there will be terror \\
\poemlll       for the young man and virgin alike, \\
\poemlll       also for the nursing infant and the aged man.''\fnote{Lit. \fbib{and a man of gray hair}} \\
\poeml \v{26}``I said, \\
\poeml `I will scatter them,\fnote{Or \fbib{will break them to pieces}} \\
\poemll    erasing their memory from the human race,\fnote{Lit. \fbib{from among men}} \\
\poeml \v{27}if it weren't for dreading the taunting of their enemies--- \\
\poemll    otherwise, their adversary might misinterpret and say, \\
\poeml ``Our power is great. \\
\poemll    It isn't the \divine{Lord} who made all of this happen.''\,'\,''
\passage{Moses Warns Israel}
\poeml \v{28}They are a nation devoid of purpose \\
\poemll    and without insight. \\
\poeml \v{29}O, that they were wise to understand this \\
\poemll    and consider their future!\fnote{Lit. \fbib{end}} \\
\poeml \v{30}How can one person\fnote{The Heb. lacks \fbib{person}} chase a thousand of them \\
\poemll    and two put a myriad\fnote{Or \fbib{put countless ones}; Lit. \fbib{put ten thousand}} to flight, \\
\poeml unless their Rock delivers them \\
\poemll    and the \divine{Lord} gives them up? \\
\poeml \v{31}For their rock isn't like our Rock, \\
\poemll    as even\fnote{Lit. \fbib{and}} our enemies admit.\fnote{Or \fbib{concede}} \\
\poeml \v{32}Instead,\fnote{Lit. \fbib{Because}} their vine is from the vines of Sodom \\
\poemll    and the vineyards of Gomorrah. \\
\poeml Their grapes are poisonous, \\
\poemll    their clusters bitter. \\
\poeml \v{33}Their wine is the venom of serpents, \\
\poemll    a poisonous cobra.
\passage{The \divine{Lord}'s Response}
\poeml \v{34}``Is this not kept in reserve, \\
\poemll    sealed up with me in my treasury? \\
\poeml \v{35}To me belong vengeance and recompense. \\
\poemll    In due time their feet will slip, \\
\poeml because their time of calamity is near \\
\poemll    and the things prepared for them draw near. \\
\poeml \v{36}For the \divine{Lord} will vindicate his people \\
\poemll    and bring comfort to his servants, \\
\poeml because he will observe that their power\fnote{Lit. \fbib{hand}} has waned, \\
\poemll    when neither prisoner\fnote{Or \fbib{slave}} nor free person remain. \\
\poeml \v{37}``He will say, `Where are their gods, \\
\poemll    the rock in which they took refuge? \\
\poeml \v{38}Who ate the fat of their offerings \\
\poemll    and drank the wine that was their drink offering? \\
\poeml Let them rise and help you \\
\poemll    and be your hiding place!' \\
\poeml \v{39}``Look now! I AM,\fnote{So LXX; MT reads \fbib{I, I myself, am he}} \\
\poemll    and there is no other god besides me. \\
\poeml I myself cause death \\
\poemll    and I sustain life; \\
\poeml I wound severely \\
\poemll    and I also heal; \\
\poemlll       from my power\fnote{Lit. \fbib{hand}} no one can deliver. \\
\poeml \v{40}``I solemnly swear\fnote{Lit. \fbib{raise my hand}} to heaven--- \\
\poemll    I say `As certainly as I'm alive and living forever, \\
\poeml \v{41}I'll whet my shining sword, \\
\poemll    with my hands in firm grasp of judgment. \\
\poeml I'll show vengeance on my adversary \\
\poemll    and repay those who keep on hating me. \\
\poeml \v{42}I'll make my arrows drunk with blood. \\
\poemll    My sword will devour flesh, \\
\poemlll       along with the blood of the slain, \\
\poemll    and I'll take their enemy leaders captive.' \\
\poeml \v{43}``Sing for joy, nations! \\
\poemll    Sing for joy,\fnote{The Heb. lacks \fbib{Sing for joy}} people who belong to him! \\
\poeml For he'll avenge the blood of his servants, \\
\poemll    turn on his adversary, \\
\poemlll       and cleanse both his land and his people.''
\end{poetry}
\passage{Moses' Final Counsel}

\v{44}So Moses and Nun's son Joshua came and recited all the words of this song while the people were assembled. \v{45}When Moses had finished addressing all of these words to all Israel, \v{46}he told them, ``Take to heart my entire testimony against you today. Command your children to observe carefully every word of this Law, \v{47}because they're not just empty words for you---they are your very life. Through these instructions you will live long in the land that you are about to cross over the Jordan River to possess.''
\passage{Moses Forbidden to Enter Canaan}

\v{48}Later that day, the \divine{Lord} told Moses, \v{49}``Ascend this Abarim mountain range\fnote{The Heb. lacks \fbib{range}} toward Mount Nebo in the land of Moab across from Jericho, and look out over the land of Canaan that I'm about to give to the Israelis as a possession. \v{50}You will die on the mountain that you are about to ascend and be taken to be with your ancestors, just as your brother Aaron died on Mount Hor and was taken to be with his ancestors. \v{51}Both of you acted unfaithfully against me among the Israelis at Meribah-kadesh in the desert of Zin, when you failed to uphold my holiness among the Israelis. \v{52}You'll see the land from a distance, but you won't be able to enter the land that I am about to give to the Israelis.''
\labelchapt{33}
\passage{Moses Reviews the Tribes of Israel}

\chapt{33}
\v{1}This is the blessing with which Moses, the man of God, blessed the Israelis before his death. \v{2}He said:

\begin{poetry}
\poeml ``The \divine{Lord} came from Sinai. \\
\poemll    Rising from Seir upon us,\fnote{So LXX. MT reads \fbib{them}} \\
\poeml he shone forth from Mount Paran, \\
\poemll    accompanied\fnote{Lit. \fbib{came} or \fbib{brought}} by a myriad\fnote{Or \fbib{by countless}; Lit. \fbib{by ten thousands}} of his holy ones, \\
\poemlll       with flaming fire from his right hand for them. \\
\poeml \v{3}Indeed, lover of people, \\
\poemll    all of his holy ones are in your control.\fnote{Lit. \fbib{hand}} \\
\poeml They gather at your feet \\
\poemll    to do as you have instructed.\fnote{Lit. \fbib{to do your word}} \\
\poeml \v{4}Moses commanded with the Law, \\
\poemll    an inheritance for the community of Jacob. \\
\poeml \v{5}The \divine{Lord}\fnote{The Heb. lacks \fbib{The \divine{Lord}}} was king of Jeshurun\fnote{I.e. a poetic term for national Israel; the Heb. name means \fbib{Upright One}} \\
\poemll    when the leaders of the people--- \\
\poemlll       all the tribes of Israel---gathered together.''
\passage{Reuben}
\poeml \v{6}``May Reuben live and not die, \\
\poemll    though his numbers are few.''
\end{poetry}
\passage{Judah}

\v{7}He declared this about Judah:

\begin{poetry}
\poeml ``Hear, \divine{Lord}, the voice of Judah \\
\poemll    and return him to his people. \\
\poeml With his own strength he fights for himself, \\
\poemll    and you will be of assistance\fnote{So LXX. MT reads \fbib{be his helper}} against\fnote{Lit. \fbib{from}} his enemies.''
\end{poetry}
\passage{Levi}

\v{8}About Levi he said:

\begin{poetry}
\poeml ``Let your Thummim and Urim\fnote{I.e. the jewel-encrusted breastplate worn by the high priest by which the will of God could be revealed; cf. Ezra 2:63, Neh 7:65} be with the man \\
\poemll    to whom you showed gracious love, \\
\poeml whom you tested at Massah \\
\poemll    and with whom you struggled \\
\poemlll       at the waters of Meribah, \\
\poeml \v{9}the one who told his mother and father, \\
\poemll    `I don't know\fnote{Lit. \fbib{see}} them,' \\
\poeml and who would neither acknowledge his brothers \\
\poemll    nor know his own children. \\
\poeml For they kept your word \\
\poemll    and guarded your covenant. \\
\poeml \v{10}They will teach your ordinances to Jacob, \\
\poemll    and your Law to Israel. \\
\poeml They will offer incense as a pleasant aroma to you\fnote{Lit. \fbib{for your nose}} \\
\poemll    and a whole burnt offering upon your altar. \\
\poeml \v{11}\divine{Lord}, bless his substance \\
\poemll    and approve the work that he undertakes.\fnote{Lit. \fbib{work of his hands}} \\
\poeml Shatter the legs\fnote{Or \fbib{loins}} of those who oppose against him; \\
\poemll    may those who hate him stand no more.''
\end{poetry}
\passage{Benjamin}

\v{12}About Benjamin he said:

\begin{poetry}
\poeml ``The beloved of the \divine{Lord} will live confidently, \\
\poemll    the Most High protecting\fnote{Or \fbib{shading}} him all day long, \\
\poemlll       and resting in his bosom.\fnote{Lit. \fbib{between his shoulders}}''
\end{poetry}
\passage{Joseph}

\v{13}About Joseph he said:

\begin{poetry}
\poeml ``May the blessing of the \divine{Lord} be on his land: \\
\poemll    dew from the choicest of the heavens, \\
\poemlll       and from the depths beneath; \\
\poeml \v{14}from the choicest products of the sun, \\
\poemll    the rich fruit of the harvest moon,\fnote{Lit. \fbib{the month}} \\
\poeml \v{15}the choicest portion\fnote{The Heb. lacks \fbib{portion}} of the eternal mountains, \\
\poemll    and the best of the everlasting hills; \\
\poeml \v{16}from the choicest of the earth and its fullness, \\
\poemll    and the favor of the one who lived in the burning\fnote{The Heb. lacks \fbib{burning}} bush. \\
\poeml May blessing\fnote{Lit. \fbib{it}} rest on Joseph's head, \\
\poemll    and on the crown of the head \\
\poemlll       of the one set apart from his brothers. \\
\poeml \v{17}May the firstborn of his bull be honorable to him, \\
\poemll    and may his horns be those of a wild ox. \\
\poeml With them may he push people all together, \\
\poemll    to the ends of the earth. \\
\poeml These are the myriads\fnote{Or \fbib{the countless ones}; Lit. \fbib{the ten thousands}} of Ephraim \\
\poemll    and the thousands of Manasseh.''
\end{poetry}
\passage{Zebulun and Issachar}

\v{18}About Zebulun he said:

\begin{poetry}
\poeml ``Zebulun, rejoice as you go out \\
\poemll    and Issachar, in being inside your tents. \\
\poeml \v{19}They will call the nations\fnote{Lit. \fbib{peoples}} to the mountain, \\
\poemll    and there they will offer righteous sacrifices, \\
\poeml for they'll draw from the abundance of the sea \\
\poemll    and from the hidden treasures of the sand.''
\end{poetry}
\passage{Gad}

\v{20}About Gad he said:

\begin{poetry}
\poeml ``Blessed be the one who enlarges Gad! \\
\poemll    Like a roaring lion, he crouches, \\
\poemlll       tearing arm and scalp. \\
\poeml \v{21}He chose the best part for himself, \\
\poemll    when the leader's portion was assigned. \\
\poeml He came at the head of the people, \\
\poemll    carrying out the \divine{Lord}'s justice \\
\poemlll       and his ordinances concerning Israel.''
\end{poetry}
\passage{Dan}

\v{22}About Dan he said:

\begin{poetry}
\poeml ``Dan is a lion's cub, \\
\poemll    leaping forth from Bashan.''
\end{poetry}
\passage{Naphtali}

\v{23}About Naphtali he said:

\begin{poetry}
\poeml ``Naphtali, full of favor and the \divine{Lord}'s blessing, \\
\poemll    take possession of the west\fnote{Or \fbib{sea}} and south.''
\end{poetry}
\passage{Asher}

\v{24}About Asher he said:

\begin{poetry}
\poeml ``May Asher be blessed, along with his descendants, \\
\poemll    may his brothers be pleased with him, \\
\poemlll       may he dip his feet in oil, \\
\poeml \v{25}may your bolts be made of iron and bronze, \\
\poemll    and may your strength be sufficient for each day you live.''
\passage{Israel's Defender}
\poeml \v{26}``There is no one like the God of Jeshurun,\fnote{I.e. a poetic term for national Israel; the Heb. name means \fbib{Upright One}} \\
\poemll    who rides through the heavens \\
\poemlll       with its lofty clouds to help you. \\
\poeml \v{27}The God of old is a dwelling place, \\
\poemll    with everlasting arms underneath. \\
\poeml He drove out your enemies before you \\
\poemll    and said: `Destroy them!' \\
\poeml \v{28}So Israel lives in confidence, \\
\poemll    isolated as the fountain of Jacob \\
\poeml in a land of grain and new wine, \\
\poemll    where the heavens rain down dew. \\
\poeml \v{29}How blessed are you, Israel! \\
\poemll    Who can be like you, \\
\poeml a people delivered by the \divine{Lord}, \\
\poemll    your shield of help and \\
\poemlll       your finely crafted sword. \\
\poeml May your enemies cower before you. \\
\poemll    You will tread down their high places.''
\end{poetry}
\labelchapt{34}
\passage{Moses Ascends Pisgah}

\chapt{34}
\v{1}Moses ascended from the desert plain of Moab toward Mount Nebo, to the top of Pisgah, across from Jericho. There\fnote{The Heb. lacks \fbib{There}} the \divine{Lord} showed him the entire land, from Gilgal as far as Dan, \v{2}all of Naphtali, the territories\fnote{Lit. \fbib{land}} of Ephraim and Manasseh, and the entire territory\fnote{Lit. \fbib{land}} of Judah all the way to out over the sea,\fnote{I.e. the Mediterranean Sea} \v{3}including the Negev,\fnote{I.e. the southern regions of the Sinai peninsula; cf. Josh 10:40} the Arabah, the valley of Jericho, and the city of the palm trees as far as Zoar. \v{4}Then the \divine{Lord} told him: ``This is the land that I promised to Abraham, Isaac, and Jacob by an oath when I said, `I'll give it to your descendants.' I'll let you see it with your eyes, but you won't cross over there.''
\passage{Moses Dies}

\v{5}So Moses, the servant of the \divine{Lord}, died there in the land of Moab, just as the \divine{Lord} had said.\fnote{Lit. \fbib{Moab, according to the word of the \divine{Lord}}} \v{6}He was buried in the valley opposite Beth Peor, in the land of Moab, but no one knows to this day where his burial place is. \v{7}Moses was 120 years old when he died. His eyesight wasn't impaired and he was still vigorous and strong. \v{8}The Israelis mourned for Moses at the desert plain of Moab for 30 days, after which the period of mourning for Moses was completed.
\passage{The Epitaph for Moses}

\v{9}Now Nun's son Joshua was full of the spirit of wisdom, because Moses had placed his hands on him, so Israelis listened to him and did what the \divine{Lord} had commanded Moses. \v{10}No prophet ever rose again in Israel like Moses, whom the \divine{Lord} knew with such great intimacy.\fnote{Lit. \fbib{knew face to face}}

\v{11}What signs and wonders the \divine{Lord} sent him to do throughout the land of Egypt, to Pharaoh, and to all of his servants who lived in the whole land!

\v{12}What great power\fnote{Lit. \fbib{What a mighty hand}} and great terror Moses displayed on behalf of all Israel!

\addcontentsline{toc}{chapter}{History Before the Exile}
\bookheader{Joshua}
\labelbook{Josh}

\bookpretitle{The Book of}
\booktitle{Joshua}

\labelchapt{1}
\passage{God's Instructions to Joshua}

\chapt{1}
\v{1}After Moses, the servant of the \divine{Lord}, had died, the \divine{Lord} spoke to Nun's son Joshua, announcing to him, \v{2}``My servant Moses is dead. Now get ready to cross the Jordan River\fnote{\fbackref{1:2} The Heb. lacks \fbib{River}, and so throughout the book}---you and all the people---to the land that I'm giving the Israelis. \v{3}I'm giving you every place where the sole of your foot falls, just as I promised Moses. \v{4}Your territorial border will extend from the wilderness to the Lebanon Mountains,\fnote{\fbackref{1:4} The Heb. lacks \fbib{Mountains}} to the river---that great River Euphrates---all the land of the Hittites---as far as the Mediterranean\fnote{\fbackref{1:4} Lit. \fbib{Great} and so throughout the book} Sea where the sun sets. \v{5}No one will be victorious\fnote{\fbackref{1:5} Lit. \fbib{will stand up}} against you for the rest of your life. I'll be with you just like I was with Moses---I'll neither fail you nor abandon you.

\v{6}``Be strong and courageous, because you'll be leading this people to inherit the land that I promised to give their ancestors. \v{7}Only be strong and very courageous to ensure that you obey all the instructions\fnote{\fbackref{1:7} Or \fbib{Law}} that my servant Moses gave you---turn neither to the right nor to the left from it---so that you may succeed wherever you go. \v{8}This set of instructions\fnote{\fbackref{1:8} Or \fbib{This Book of the Law}} is not to cease being a part of your conversations.\fnote{\fbackref{1:8} Lit. \fbib{cease from your lips}} Meditate on it day and night, so that you may be careful to carry out everything that's written in it, for then you'll prosper and succeed. \v{9}I've commanded you, haven't I? Be strong and courageous. Don't be fearful or discouraged, because the \divine{Lord} your God is with you wherever you go.''
\passage{Joshua Gives Orders to His Leaders}

\v{10}Then Joshua gave orders to the officials of the people. \v{11}``Go through the camp,'' he said, ``and command the people, `Prepare provisions for yourselves, because within three days you'll be crossing the Jordan River to take possession of the land that the \divine{Lord} your God is giving you---so go get it!'\,''

\v{12}Joshua told the descendants of Reuben, the descendants of Gad, and the half-tribe of Manasseh, \v{13}``Remember what\fnote{\fbackref{1:13} Lit. \fbib{Remember the word}} Moses commanded you when he said, `The \divine{Lord} your God will provide you rest, as well as this land.' \v{14}Your wives, your young children, and your livestock will remain in the land that Moses gave you on this side of the Jordan River, but you and all your warriors will cross, ready for battle, in full view of your relatives, and you will help them \v{15}until the \divine{Lord} gives relief to your relatives, as he did to you. Then they'll take the land that the \divine{Lord} your God is giving them as their inheritance. You'll return to the land of your heritage and receive the inheritance that Moses the servant of the \divine{Lord} gave you on the east side of the Jordan River, in the direction of the sunrise.''
\passage{The People Reaffirm Their Commitment}

\v{16}``We'll do everything that you commanded,'' they replied. ``We'll go wherever you send us. \v{17}We'll listen and obey you in everything, just like we did with Moses. Only may the \divine{Lord} your God be with you, just as he was with Moses. \v{18}Anyone who rebels against what you say and doesn't listen to your words regarding everything that you command will be executed. Only be strong and courageous.''
\labelchapt{2}
\passage{Rahab Receives Two Scouts}

\chapt{2}
\v{1}After this, Nun's son Joshua sent two men from the Acacia groves\fnote{\fbackref{2:1} Or \fbib{from Shittim}; and so throughout the book} as undercover scouts. He told them, ``Go and look over the land. Pay special attention to Jericho.'' So they went out, came to the house of a prostitute named Rahab, and lodged there.

\v{2}Then the king of Jericho was told, ``Look! Israeli men arrived tonight to scout out the land.''

\v{3}So the king of Jericho sent for Rahab and ordered her, ``Bring out the men who came to visit you and lodged in your house, because they've come to scout out the entire land.''

\v{4}Now the woman had taken the two men and hid them. So she replied, ``The men really did come to me, but I didn't know from where they came. \v{5}At dusk, when it was time to close the city gates, the men left. I don't know where the men went. Go after them quickly, and\fnote{\fbackref{2:5} Lit. \fbib{for}} you might overtake them.''

\v{6}But she had taken them up to the roof and had hidden them among stalks of flax that she had laid out in order on the roof. \v{7}So the men pursued them along the road that leads to the fords of the Jordan River. As soon as the search party had left, they shut the city gate after them.
\passage{Rahab Seeks Protection}

\v{8}Before the scouts\fnote{\fbackref{2:8} Lit. \fbib{Before they}} had lain down, she went up to them on the roof \v{9}``I'm really convinced that the \divine{Lord} has given you the land,'' she said,\fnote{\fbackref{2:9} Lit. \fbib{told the men}} ``because we're overwhelmed with fear of you. All the other inhabitants of the land are demoralized at your presence, \v{10}because we heard how the \divine{Lord} dried up the water of the Reed\fnote{\fbackref{2:10} So MT; LXX reads \fbib{Red}} Sea right in front of you as you were coming out of Egypt, and what you did to the two kings of the Amorites who were on the other\fnote{\fbackref{2:10} Lit. \fbib{east}} side of the Jordan River---to Sihon and Og---whom you completely destroyed. \v{11}When we heard these reports,\fnote{\fbackref{2:11} The Heb. lacks \fbib{these reports}} we all became terrified and discouraged\fnote{\fbackref{2:11} Lit. \fbib{and no courage remained in any man}} because of you, for the \divine{Lord} your God is God in heaven above and on the earth beneath. \v{12}Now therefore, since I've treated you so kindly,\fnote{\fbackref{2:12} Lit. \fbib{you with gracious love}} please swear in the name of\fnote{\fbackref{2:12} Lit. \fbib{swear to}} the \divine{Lord} that you'll also be kind\fnote{\fbackref{2:12} Lit. \fbib{also show grace}} to my father's household by giving me this\fnote{\fbackref{2:12} Lit. \fbib{a}} sure sign: \v{13}Spare my father, my mother, and my brothers and sisters, along with everyone who belongs with them so we won't be killed.''
\passage{A Promise of Protection}

\v{14}So the men told her, ``Our life for yours---even to death---if you don't betray this mission of ours. Then when the \divine{Lord} gives us this land, we'll treat you graciously and faithfully.''

\v{15}So she let them down by a rope through the window, since her house was built into the town wall where she lived. \v{16}She told them, ``Go out to the hill country, so the search party won't find you, and hide for three days. After that, you may go on your own way.''

\v{17}The men replied, ``We'll be free from our commitment to you to which you've obligated us \v{18}when we invade the land, if you don't tie this rope made with red cords in the window through which you let us down, and if you don't gather your father, your mother, your brothers, and all of the rest of your father's household into your house. \v{19}Everyone who leaves through the doors of your house into the street will be responsible for his own death, but we'll be responsible for anyone who remains with you in the house if even so much as a hand is laid on him. \v{20}But if you report this incident, we'll be free from the oath to which you've made us swear.''

\v{21}``Since you put it that way,''\fnote{\fbackref{2:21} Lit. ``\fbib{According to your word,''}} she replied, ``I agree.''\fnote{\fbackref{2:21} Lit. \fbib{replied, ``may it be.''}} After she sent them on their way and they had left, she tied the red cord in the window.
\passage{The Scouts Report to Joshua}

\v{22}The scouts\fnote{\fbackref{2:22} Lit. \fbib{They}} left for the hill country and remained there for three days until the search party returned. The search party searched the entire road, but was unable to find them. \v{23}Later, the two men returned from the hill country, crossed over the Jordan River,\fnote{\fbackref{2:23} The Heb. lacks \fbib{the Jordan River}} approached Nun's son Joshua, and told him everything that had happened to them. \v{24}They reported to Joshua, ``The \divine{Lord} really has given the entire land into our control. The inhabitants of the land have melted away right in front of us!''
\labelchapt{3}
\passage{Joshua Prepares to Conquer Jericho}

\chapt{3}
\v{1}Joshua got up early the next morning. Accompanied by all the Israelis, he set out from the Acacia groves and arrived at the Jordan River, where they encamped before crossing it. \v{2}Three days later, the officers went throughout the camp, \v{3}giving orders to the people. They said, ``When you see the Ark of the Covenant of the \divine{Lord} your God being carried by the Levitical priests, then get up, leave where you are, and follow it. \v{4}Be sure to keep a distance of about 2,000 cubits\fnote{\fbackref{3:4} I.e. about 1,000 yards} between you and it. Don't come near it, so you can be certain where you're going, since you haven't passed this way before.''

\v{5}Then Joshua addressed the people: ``Consecrate yourselves, because tomorrow the \divine{Lord} will do marvelous things among you.''

\v{6}After this, Joshua\fnote{\fbackref{3:6} Lit. \fbib{He}} instructed the priests, ``Take up the Ark of the Covenant and cross over ahead of the people.'' So they took up the Ark of the Covenant and went on ahead of the people.
\passage{The \divine{Lord} Addresses Joshua}

\v{7}At this point, the \divine{Lord} told Joshua, ``Today I'm going to exalt you in the sight of all Israel, so they'll be sure that I'm going to be with you just as I was with Moses. \v{8}Give this command to the priests who are carrying the Ark of the Covenant: `When you arrive at the water of the Jordan River, stand still in the Jordan.'\,''
\passage{Joshua Addresses Israel}

\v{9}So Joshua told the Israelis, ``Come here and listen to what the \divine{Lord} your God has to say.'' \v{10}Joshua continued, ``This is how you'll know that the living God really is among you: he's going to remove the Canaanites, the Hittites, the Hivites, the Perizzites, the Girgashites, the Amorites, and the Jebusites right in front of you. \v{11}Look! The Ark of the Covenant of the Lord of whole the earth is crossing ahead of you into the Jordan River. \v{12}So take for yourselves twelve men from the tribes of Israel, one man from each tribe. \v{13}When the soles of the feet of the priests who carry the ark of the \divine{Lord}, the Lord of the whole earth, touch the water in the Jordan River, the water that feeds the Jordan will be cut off from above and they'll stand still in a single location.''
\passage{The Jordan River Stops Flowing}

\v{14}So the people set out from their tents to cross the Jordan River, with the priests carrying the Ark of the Covenant in full view of the people. \v{15}When the priests who carried the ark entered the Jordan River, as their feet touched the water's edge (The Jordan River overflows all of its banks daily during the harvest season.), \v{16}the water flowing downstream from above stood still in a single location, a great distance away at Adam, a city near Zarethan. The water that flowed south toward the sea in the Arabah (that is, the Dead\fnote{\fbackref{3:16} Lit. \fbib{Salt}; and so throughout the book} Sea) was completely cut off. So the people crossed opposite Jericho. \v{17}The priests who were carrying the Ark of the Covenant of the \divine{Lord} stood firm on dry ground in the middle of the Jordan River, while all Israel crossed on dry ground until the entire nation had finished crossing the Jordan River.
\labelchapt{4}
\passage{The Jordan River Memorial}

\chapt{4}
\v{1}As soon as the entire nation had completed its crossing of the Jordan, the \divine{Lord} spoke to Joshua. He said, \v{2}``Gather together twelve men from the people---one man from each tribe--- \v{3}and tell them, `Pick up twelve stones from the middle of the Jordan where the priests' feet were standing, bring them along with you, and put them down where you camp tonight.'\,''

\v{4}So Joshua called the twelve men whom he had chosen from the people of Israel, one man from each tribe. \v{5}Joshua told them, ``Cross over again in front of the ark of the \divine{Lord} your God into the middle of the Jordan River. Then each of you pick up a stone on his shoulder with which to build a memorial,\fnote{\fbackref{4:5} The Heb. lacks \fbib{with which to build a memorial}} one for each of the tribes of Israel. \v{6}Let this serve as\fnote{\fbackref{4:6} Lit. \fbib{this be}} a sign among you, so that when your children ask in times to come, `What do these stones mean to you,' \v{7}then you'll say to them, `Because the waters of the Jordan River were cut off in front of the Ark of the Covenant of the \divine{Lord}. When it crossed the Jordan River, the waters of the Jordan were cut off.' So these stones will become a memorial to the Israelis forever.''

\v{8}The Israelis did just as Joshua commanded. They took up twelve stones from the middle of the Jordan River---just as the \divine{Lord} had spoken to Joshua---according to the number of the tribes of the Israelis, and they carried them over to where they would be pitching camp, and they put them down there. \v{9}Then Joshua set up twelve stones in the middle of the Jordan River at the location where the feet of the priests who carried the Ark of the Covenant had been standing, and they remain there to this day.
\passage{Crossing the Jordan River}

\v{10}The priests who were carrying the ark stood in the middle of the Jordan River until everything had been done in accordance with what the \divine{Lord} had commanded Joshua to speak to the people and with everything that Moses had commanded Joshua. So the people hurried and crossed over. \v{11}When all of the people had completed their crossing, the ark of the \divine{Lord} and the priests crossed over in full view of the people. \v{12}Just as Moses had directed, the descendants of Reuben, the descendants of Gad, and the half-tribe of Manasseh crossed over, dressed in battle regalia, in full view of the other\fnote{\fbackref{4:12} The Heb. lacks \fbib{other}} Israelis. \v{13}About 40,000 soldiers equipped to do battle in the \divine{Lord}'s presence crossed over to the desert plains of Jericho.

\v{14}That day, the \divine{Lord} exalted Joshua in the presence of all Israel so that they revered him just as they had revered Moses throughout his life.

\v{15}Now the \divine{Lord} had told Joshua, \v{16}``Command the priests who carry the Ark of the Testimony to come up from the Jordan River.''

\v{17}So Joshua ordered the priests, ``Come up from the Jordan River.''

\v{18}As soon as the priests who were carrying the Ark of the Covenant of the \divine{Lord} had come up from the middle of the Jordan River, and the soles of the priests' feet came up to dry ground, the water of the Jordan River returned to normal,\fnote{\fbackref{4:18} Lit. \fbib{to its place}} covering its banks as it had done so before.
\passage{Why Joshua Set up the Memorial}

\v{19}The people came up from the Jordan River on the tenth day\fnote{\fbackref{4:19} The Heb. lacks \fbib{day}} of the first month and camped at Gilgal on the eastern outskirts of Jericho. \v{20}Joshua set up the twelve stones that they had removed from the Jordan River at Gilgal. \v{21}Then he told the Israelis, ``When your descendants ask their parents in years to come, `What is the meaning of these stones?' \v{22}you are to tell your descendants: `Israel crossed this Jordan River on dry ground \v{23}because the \divine{Lord} your God dried up the water of the Jordan River right in front of you, until you had crossed over, just as the \divine{Lord} your God had done to the Reed\fnote{\fbackref{4:23} So MT; LXX reads \fbib{Red}} Sea---which he had dried up in front of us until we had crossed it also.' \v{24}Do this\fnote{\fbackref{4:24} The Heb. lacks \fbib{Do this}} so that all of the people of the earth may know how strong the power\fnote{\fbackref{4:24} Lit. \fbib{hand}} of the \divine{Lord} is, and so that you may fear the \divine{Lord} your God every day.''
\labelchapt{5}
\passage{Israel's Enemies Become Discouraged}

\chapt{5}
\v{1}All the Amorite kings who lived across the Jordan River to the west and all the Canaanite kings by the Mediterranean\fnote{\fbackref{5:1} The Heb. lacks \fbib{Mediterranean}} Sea became discouraged as soon as they heard that the \divine{Lord} had dried up the water of the Jordan River for the people of Israel until they had crossed it. They no longer had a will to fight\fnote{\fbackref{5:1} Lit. \fbib{a spirit in them}} because of the people of Israel.
\passage{A New Generation is Circumcised}

\v{2}At that time the \divine{Lord} told Joshua, ``Make for yourselves some flint knives and circumcise the Israelis who haven't been circumcised yet.''\fnote{\fbackref{5:2} Lit. \fbib{Israelis a second time}}

\v{3}So Joshua made some flint knives and circumcised the Israelis at Gibeath-haaraloth.\fnote{\fbackref{5:3} The Heb. name \fbib{Gibeath-haaraloth} means \fbib{Foreskin Hill}} \v{4}Joshua circumcised them because all of the males among the people who came out of Egypt---that is, all the warriors---had died during their journey through the wilderness following their departure from Egypt. \v{5}Although everyone who had left Egypt had been circumcised, nevertheless all the people born during the journey after their departure from Egypt had not been circumcised. \v{6}The Israelis traveled 40 years in the wilderness until the entire nation---that is, the warriors who had departed from Egypt---had perished because they hadn't listened to the voice of the \divine{Lord}. The \divine{Lord} had promised them that he would not let them see the land that he had sworn to give us, a land that flows with milk and honey. \v{7}As a result, it was their descendants, whom he raised up to take their place, that Joshua circumcised. They had remained uncircumcised, because they had not been circumcised during their journey. \v{8}When the circumcision of the entire nation was complete, they remained in their places within the camp until they were healed.

\v{9}Then the \divine{Lord} told Joshua, ``Today I have rolled the disgrace of Egypt away from you.'' That's why that place is called ``Gilgal''\fnote{\fbackref{5:9} The Heb. word \fbib{Gilgal} means \fbib{to roll}} to this day.
\passage{The Manna Ceases}

\v{10}While the Israelis remained encamped at Gilgal on the plains of Jericho, they observed the Passover during the evening of the fourteenth day of the month. \v{11}On the day following Passover---on that exact day---they ate the produce of the land, unleavened cakes and parched grain. \v{12}The manna ceased on the day they ate the produce of the land. Since the Israelis no longer received manna, they ate crops from the land of Canaan that year.
\passage{Joshua is Visited by the \divine{Lord}}

\v{13}Now it happened that while Joshua was near Jericho, he looked up and much to his amazement, he saw a man standing in front of him, holding a drawn sword in his hand! Joshua approached him and asked him, ``Are you one of us, or are you with our enemies?''

\v{14}``Neither,'' he answered. ``I have come as commander of the \divine{Lord}'s Army.''

Joshua immediately fell on his face to the earth and worshipped, saying to him, ``Lord, what do you have for your servant by way of command?''

\v{15}The commander of the \divine{Lord}'s Army replied to Joshua, ``Remove your sandals from your feet, because the place where you're standing is holy.'' So Joshua did so.
\labelchapt{6}
\passage{Instructions for Joshua}

\chapt{6}
\v{1}Meanwhile, Jericho was fortified inside and out because of the Israelis. Nobody could leave or enter.

\v{2}The \divine{Lord} told Joshua, ``Look! I have given Jericho over to your control,\fnote{\fbackref{6:2} Lit. \fbib{hand}} along with its kings and valiant soldiers. \v{3}March around the city, all the soldiers circling the city once. Do this for six days, \v{4}with seven priests carrying in front of the ark seven trumpets made from rams' horns. On the seventh day march around the city seven times while the priests blow their trumpets. \v{5}When they sound a long blast with the ram's horn, as soon as you hear the sound of the trumpet, then the entire army is to cry out loud, the city wall will collapse, and then all of the soldiers are to charge straight ahead.''
\passage{The Destruction of Jericho}

\v{6}So Nun's son Joshua called for the priests. ``Pick up the Ark of the Covenant,'' he told them, ``and have seven priests carry seven trumpets made from rams' horns in front of the ark of the \divine{Lord}.''

\v{7}He told the army, ``Go out and encircle the city. Have the armed men march out in front of the ark of the \divine{Lord}.''

\v{8}And so, just as Joshua had commanded, seven of the priests went forward, carrying the seven trumpets made of rams' horns in the \divine{Lord}'s presence, blowing the trumpets while the Ark of the Covenant of the \divine{Lord} followed them. \v{9}Armed men preceded the priests who were blowing the trumpets, and a rear guard followed the ark, while the trumpets continued to blow.

\v{10}Joshua issued orders to the army: ``You are not to shout or even let your voice be heard. Don't utter a word until I tell you to shout. Then shout!'' \v{11}So the ark of the \divine{Lord} was taken once around the city, then they went back to camp and spent the night there.\fnote{\fbackref{6:11} Lit. \fbib{night in the camp}}

\v{12}Joshua got up early the next morning, and the priests picked up the ark of the \divine{Lord}. \v{13}The seven priests who carried the seven trumpets made from rams' horns preceded the ark of the \divine{Lord}, blowing their trumpets constantly. The armed men preceded them, and the rear guard followed the ark of the \divine{Lord}, while the trumpets continued to blow. \v{14}On the second day they marched around the city once and then went back to camp. They did this for six days. \v{15}They rose early at dawn on the seventh day and marched around the city seven times, just as they had before, except that on that day only they marched around the city seven times.

\v{16}As they completed the seventh time, after the priests had blown the trumpets, Joshua told the army, ``Shout, because the \divine{Lord} has given you the city! \v{17}The city---along with everything in it---is to be turned over to the \divine{Lord} for destruction. Only Rahab the prostitute and everyone who is with her in her house may live, because she hid the scouts we sent. \v{18}Now as for you, everything has been turned over for destruction. Don't covet or take any of these things. Otherwise, you'll make the camp of Israel itself an object worthy of destruction, and bring trouble on it. \v{19}But everything made of silver and gold, and vessels made of bronze and iron are set apart to the \divine{Lord}. They are to go into the treasury of the \divine{Lord}.''

\v{20}So the army shouted and the trumpets were blown again. As soon as the army heard the sound of the trumpets, they shouted loudly and the wall collapsed. The army charged straight ahead into the city and captured it. \v{21}They turned over everyone in the city for destruction and executed them,\fnote{\fbackref{6:21} Lit. \fbib{by the edge of the}} including both men and women, young and old, and oxen, sheep, and donkeys.

\v{22}Joshua told the two men who had scouted the land, ``Go into the prostitute's home and bring her out of it, along with everyone who is with her, just as you promised her.'' \v{23}So the young men who had been scouts went in and brought Rahab out, along with her father, her mother, her brothers, and everyone else who was with her. They brought her entire family out and set them outside the camp of Israel. \v{24}Then the army set fire to the city and to everything in it, except that they reserved the silver, gold, and vessels of bronze and iron for the treasury of the \divine{Lord}. \v{25}But Joshua spared Rahab the prostitute, along with her family and everyone who was with her. Her family\fnote{\fbackref{6:25} Lit. \fbib{She}} has lived in Israel ever since, because she hid the scouts whom Joshua sent to observe Jericho.
\passage{Joshua Curses the Rebuilding of Jericho}

\v{26}Then Joshua made everyone\fnote{\fbackref{6:26} Lit. \fbib{them}} take the following oath at that time. He said:

\begin{poetry}
\poeml ``Cursed in the presence of the \divine{Lord} is the man \\
\poemll    who restores and rebuilds this city of Jericho! \\
\poeml He will lay its foundation at the cost of\fnote{\fbackref{6:26} The Heb. lacks \fbib{At the cost of}} his firstborn, \\
\poemll    and at the cost of\fnote{\fbackref{6:26} The Heb. lacks \fbib{at the cost of}} his youngest he will set up its gates.''
\end{poetry}

\v{27}So the \divine{Lord} was with Joshua, and as a result, Joshua's\fnote{\fbackref{6:27} Lit. \fbib{his}} reputation spread throughout the land.
\labelchapt{7}
\passage{Israel is Defeated at Ai}

\chapt{7}
\v{1}Later, the Israelis broke their promise regarding the things that had been turned over to destruction. Carmi's son Achan, grandson of Zabdi and great-grandson of Zerah from the tribe of Judah, appropriated some of the things that had been turned over to destruction. As a result, the \divine{Lord} became angry with the Israelis.

\v{2}Meanwhile, Joshua had sent some soldiers from Jericho to Ai, which was near Beth-aven, east of Bethel. He ordered them, ``Go up and scout the land.'' So the soldiers went up and scouted Ai and \v{3}returned to Joshua.

``Not all of the people need to go up,'' they reported. ``Only about two or three thousand men should attack Ai. Since they are so few, don't make all of the army work hard up there.''

\v{4}So about three thousand went up there, but they ran away from the men of Ai. \v{5}The men of Ai killed about 36 of them, pursuing them outside the city gates as far as Shebarim, killing them as they descended. As a result, the army became terrified and lost their confidence.\fnote{\fbackref{7:5} Lit. \fbib{the hearts of the people melted and turned to water}} \v{6}At this, Joshua tore his clothes, fell down to the ground on his face before the ark of the \divine{Lord} until evening---he and the leaders of Israel---and they covered their heads with dust. \v{7}``Lord \divine{God},'' Joshua asked, ``Why have you brought this people across the Jordan River? To hand us over to the Amorites so we'll be destroyed? Wouldn't it have been better for us to be content to settle on the other side of the Jordan? \v{8}Lord, what am I to say, now that Israel has run\fnote{\fbackref{7:8} Lit. \fbib{turned}} away from its enemies? \v{9}The Canaanites and all the inhabitants of the land will hear of this, will surround us, and eliminate us\fnote{\fbackref{7:9} Lit. \fbib{eliminate our name}} from the earth! Then what will you do about your great reputation?''\fnote{\fbackref{7:9} Lit. \fbib{name}}
\passage{The \divine{Lord} Rebukes Joshua}

\v{10}``Get up!'' the \divine{Lord} replied to Joshua. ``Why have you fallen on your face? \v{11}Israel has sinned. They broke my covenant that I commanded them by taking some of the things that had been turned over to destruction. They have stolen, have been deceitful, and have stored what they stole\fnote{\fbackref{7:11} The Heb. lacks \fbib{what they stole}} among their own belongings. \v{12}The Israelis have been unable to stand before their enemies. They're turning their backs and running from\fnote{\fbackref{7:12} The Heb. lacks \fbib{and running from}} their enemies because they themselves have been turned over to destruction. I will not be with you anymore unless you destroy these things that have been turned over to destruction. \v{13}So get up and sanctify the people. Tell them, `Sanctify yourselves in preparation for tomorrow, because this is what the \divine{Lord} God of Israel, says: ``There are things turned over to destruction among you, Israel. You won't be able to defeat your enemies until you remove what has been turned over to destruction. \v{14}Tomorrow morning you are to come forward tribe by tribe. The tribe that the \divine{Lord} selects\fnote{\fbackref{7:14} Lit. \fbib{selects by lottery}; and so through v 18} is to come forward by tribes, the tribe that the \divine{Lord} selects is to come forward by households, and the household that the \divine{Lord} selects is to come forward one by one. \v{15}The one selected as having taken what has been turned over to destruction is to be incinerated, along with everything that pertains to him, because he has transgressed against the covenant of the \divine{Lord} and committed an outrageous thing in Israel.''\,'\,''
\passage{Achan's Sin Revealed}

\v{16}So Joshua got up early that morning, brought Israel near tribe by tribe, and the tribe of Judah was selected. \v{17}He brought near the tribes of Judah, and the Zerahite tribe was selected. Then he brought near the Zerahite tribe family by family, and the household of Zabdi was selected. \v{18}Next, he brought near his household one by one, and Carmi's son Achan, grandson of Zabdi and great-grandson of Zerah, was selected from the tribe of Judah.

\v{19}Joshua then spoke to Achan, ``My son, give glory and praise\fnote{\fbackref{7:19} Lit. \fbib{thanks}} to the \divine{Lord} God of Israel.\fnote{\fbackref{7:19} I.e. by telling the truth} Tell me right now what you did. Don't hide anything.''

\v{20}Achan answered Joshua, ``It's true. I'm the one who sinned against the \divine{Lord} God of Israel. \v{21}I noticed among the war spoils a beautiful mantle from Shinar,\fnote{\fbackref{7:21} I.e. Babylon} 200 shekels of silver, and a bar of gold weighing 50 shekels. Because I wanted them, I took them, and they're buried in the ground inside my tent. The silver is underneath.''

\v{22}So Joshua sent some messengers, who ran to the tent. And there it was, hidden in the tent with the silver underneath. \v{23}They took the things from the tent that had been turned over to destruction,\fnote{\fbackref{7:23} Lit. \fbib{took them}} brought them to Joshua and all of the Israelis, and laid them out in the presence of the \divine{Lord}. \v{24}Then Joshua, with all Israel accompanying him, took Zerah's son Achan, along with the silver, the mantle, the gold, his sons, his daughters, his oxen, his donkeys, his sheep, his tent, and everything that belonged to him to the Valley of Achor.

\v{25}Joshua announced, ``Why did you bring trouble to us? Today the \divine{Lord} is bringing trouble to you!'' So all Israel stoned him to death, incinerated them, and buried them with stones, \v{26}piling up a large mound of boulders that remains to this day. After this, the \divine{Lord} turned his burning anger away, and that is why that place is called ``the Valley of Achor''\fnote{\fbackref{7:26} The Heb. name \fbib{Achor} means \fbib{Trouble}} to this day.
\labelchapt{8}
\passage{The Destruction of Ai}

\chapt{8}
\v{1}The \divine{Lord} then told Joshua, ``Don't be afraid or lose heart! Take all the fighting men with you, and go up right now to Ai. Take note that I have handed over the king of Ai into your control, along with his people, his city, and his land. \v{2}Do to Ai and its king as you did to Jericho and its king, but take its spoil and its livestock as war booty for yourselves. Set an ambush around the city.''

\v{3}So Joshua and all of the fighting men prepared to go out against Ai. Joshua selected 30,000 valiant warriors and sent them out by night, \v{4}telling them, ``Pay attention now! You are to set up an ambush around the city. Don't go very far from the city, and all of you remain on alert. \v{5}I and all of the army with me will advance upon the city. When they come out after us like they did before, we'll run away from them. \v{6}They'll come after us until we've drawn them away from the city, because they'll say, `They're running away from us just like they did before.' While we're running away from them, \v{7}you get up from the ambush and seize the city, because the \divine{Lord} your God will give it into your control. \v{8}When you've taken the city, set it on fire, just as the \divine{Lord} ordered. Look! These are your orders!''\fnote{\fbackref{8:8} Lit. \fbib{Look! I have commanded you!}} \v{9}So Joshua sent them out, and they set up an ambush between Bethel and Ai, to the west of Ai.

Joshua spent that night in the camp\fnote{\fbackref{8:9} The Heb. lacks \fbib{in the camp}} among the army. \v{10}In the morning, Joshua got up early, mustered his army, and set off for Ai, accompanied by the elders of Israel in full view of the army. \v{11}The entire fighting force with him attacked, approaching the city, and camped on the north side of Ai, with a ravine between them and Ai. \v{12}Taking about 5,000 men, he set them in ambush between Bethel and Ai to the west of the city, \v{13}stationing their forces with its main encampment north of the city and its rear guard to the west. Joshua spent that night in the valley.

\v{14}When the king of Ai saw what had happened,\fnote{\fbackref{8:14} The Heb. lacks \fbib{what had happened}} he and his army quickly got up early and went out to meet Israel in battle. He and all his people met at the place adjacent to the desert plain. But he didn't know about the ambush that had been set for him on the other side of the city. \v{15}Because Joshua and the entire fighting force of\fnote{\fbackref{8:15} The Heb. lacks \fbib{fighting force of}} Israel pretended to lose the battle by running away in front of them toward the wilderness, \v{16}everyone in the city followed after them. As they pursued Joshua, they were drawn away from the town. \v{17}There wasn't a single man left in Ai or Bethel who didn't run out after Israel. They left the city open and pursued Israel.

\v{18}Then the \divine{Lord} told Joshua, ``Stretch out the battle lance\fnote{\fbackref{8:18} Or \fbib{the javelin}} that's in your hand toward Ai, because I will give it into your control.'' So Joshua stretched out the battle lance\fnote{\fbackref{8:18} Or \fbib{the javelin}} that was in his hand toward the city. \v{19}As soon as he stretched out his hand, the troops in ambush quickly got up from their place of hiding\fnote{\fbackref{8:19} The Heb. lacks \fbib{of hiding}} and attacked. They entered the city, seized it, and immediately set it\fnote{\fbackref{8:19} Lit. \fbib{set the city}} on fire.

\v{20}Then the men of Ai looked back behind them---and all of a sudden!---smoke from the city was rising into the sky. They were unable to run in any direction, because the Israelis\fnote{\fbackref{8:20} Lit. \fbib{people}} who had fled toward the wilderness had turned around to attack their pursuers. \v{21}When Joshua and the entire fighting force of\fnote{\fbackref{8:21} The Heb. lacks \fbib{fighting force of}} Israel observed that the men who had been in ambush had seized the city and that the smoke from the city was rising, they turned around and attacked the men of Ai. \v{22}Then the others came out from the city against them, so the men of Ai\fnote{\fbackref{8:22} Lit. \fbib{so they}} were surrounded by the Israelis, some on one side and some on the other. Israel attacked them until no one was left to survive or escape. \v{23}But the king of Ai was taken alive and brought to Joshua.

\v{24}When Israel had completed executing all of the residents of Ai in the open wilderness where they had chased them, and after all of them---to the very last of them---had been killed by swords, the entire fighting force of\fnote{\fbackref{8:24} The Heb. lacks \fbib{fighting force of}} Israel returned to Ai and attacked it with swords. \v{25}The total of all who fell that day, including men and women, was 12,000---the entire population of Ai. \v{26}Joshua did not cease his attack\fnote{\fbackref{8:26} Lit. \fbib{his hand with which he had stretched out the battle lance}} until he had completely destroyed every inhabitant of Ai. \v{27}Israel took only the livestock and the spoil of that city as their war booty, in accordance with what the \divine{Lord} had commanded to Joshua. \v{28}Joshua burned Ai, turning it into a permanent mound of ruins, and it remains so to this day. \v{29}He hanged the king of Ai on a tree until dusk, and at sunset Joshua ordered his body brought down from the tree and laid at the entrance to the gate of the town. There he raised over it a large mound of stones, which stands there to this day.\fnote{\fbackref{8:29} I.e. c. 1100 -- 1000 BC}
\passage{Joshua Renews the Covenant}

\v{30}Then Joshua built an altar to the \divine{Lord} God of Israel, on Mount Ebal, \v{31}just the way Moses the servant of the \divine{Lord} had commanded the Israelis in the Book of the Law of Moses: ``{\ldots}an altar of uncut\fnote{\fbackref{8:31} Or \fbib{whole}} stones that hasn't been worked with iron tools{\ldots}''\fnote{\fbackref{8:31} Cf. Deut. 27:5b} and they offered burnt offerings to the \divine{Lord} on it, along with peace offerings.

\v{32}There Joshua\fnote{\fbackref{8:32} Lit. \fbib{he}} inscribed on stones a copy of the Law of Moses that Moses had presented to\fnote{\fbackref{8:32} Lit. \fbib{that he had written in the presence of}} the Israelis. \v{33}All Israel, both foreigners and citizens, together with their elders, officers, and judges, stood on opposite sides of the Ark of the Covenant of the \divine{Lord}. Half stood in front of Mount Gerizim and half stood in front of Mount Ebal, just as Moses, the \divine{Lord}'s servant had commanded at the first, so that they could bless the people of Israel.\fnote{\fbackref{8:33} Cf. Deut 28:1-14} \v{34}Afterwards, Joshua\fnote{\fbackref{8:34} Lit. \fbib{he}} read all the words of the Law---both the blessings and the curses---according to everything written in the Book of the Law.\fnote{\fbackref{8:34} Cf. Deut 27:1-28:68} \v{35}There wasn't one word of everything Moses had commanded that Joshua did not read in front of the entire assembly of Israel, including the women, their little ones, and the foreigners who lived among them.
\labelchapt{9}
\passage{Trickery by the Gibeonites}

\chapt{9}
\v{1}Eventually all the kings who reigned in the hill country across the Jordan River and in the low-lying coastlands of the Mediterranean Sea facing Lebanon heard about this. So the Hittites, the Amorites, the Canaanites, the Perizzites, the Hivites, and the Jebusites \v{2}united together as one to fight against both Joshua and Israel.

\v{3}But when the inhabitants of Gibeon heard what Joshua had done to Jericho and Ai, \v{4}they took the initiative by preparing their provisions shrewdly: they took tattered sacks for their donkeys, worn-out, torn, and mended wineskins, \v{5}worn-out, patched sandals for their feet, and worn-out clothes. All of their food was dried out and covered in mold. \v{6}Then they approached Joshua in the camp at Gilgal and addressed him and the Israelis, ``We've arrived from a distant country, so please make a treaty with us right now.''

\v{7}But the Israelis responded to the Hivites, ``Perhaps you live in our midst. If this is so,\fnote{\fbackref{9:7} The Heb. lacks \fbib{If this is so}} how can we make a treaty with you?''

\v{8}So they responded to Joshua, ``We are your servants.''

Joshua asked them, ``Who are you? And where did you come from?''

\v{9}They answered, ``Your servants have arrived from a very distant land, because of the reputation\fnote{\fbackref{9:9} Lit. \fbib{name}} of the \divine{Lord} your God, because we've heard a report about all that he did in Egypt, \v{10}along with all of what he did to the two Amorite kings who were beyond the Jordan River---that is, to King Sihon of Heshbon and to King Og of Bashan, who lived in Ashtaroth. \v{11}So our leaders and all of the inhabitants of our country told us, `Take provisions along with you for your journey, go to meet them, and tell them, ``We are your servants. Come now and make a treaty with us.''\,' \v{12}Look at\fnote{\fbackref{9:12} Lit. \fbib{Here is}} our bread: it was still warm when we took it from our houses as our food for our journey on the very day we set out to come to you. But now, look how it's dry and moldy. \v{13}And these wineskins were new when we filled them, but look---now they're cracked. And our clothes and sandals are worn out from our very long journey.''

\v{14}So the leaders of Israel\fnote{\fbackref{9:14} The Heb. lacks \fbib{of Israel}} sampled their provisions, but did not ask the \divine{Lord} about it. \v{15}They made a treaty with them, guaranteeing their lives with a covenant, and the leaders of the congregation confirmed it with an oath to them.

\v{16}But three days after they had made the treaty with them, they learned that they were their neighbors and were living in their midst. \v{17}So the Israelis set out for their cities and three days later they reached their cities of Gibeon, Chephirah, Beeroth, and Kiriath-jearim. \v{18}The Israelis did not attack them, because the leaders of the congregation had made an oath with them in the name of\fnote{\fbackref{9:18} Lit. \fbib{them by}} the \divine{Lord}, the God of Israel. Nevertheless, the entire congregation grumbled against their leaders.

\v{19}Then all of the leaders spoke to the entire congregation, ``We have sworn to them in the name of\fnote{\fbackref{9:19} Lit. \fbib{them by}} the \divine{Lord}, the God of Israel, and we cannot touch them. \v{20}So this is what we'll do to them: we'll let them live, so that wrath won't come upon us because of the oath that we swore to them.''

\v{21}The leaders told them, ``Let them live.'' So they became wood cutters and water carriers for the entire congregation, which is what the leaders had decided concerning them.

\v{22}Joshua summoned the Gibeonites\fnote{\fbackref{9:22} Lit. \fbib{summoned them}} and asked them, ``Why did you deceive us by saying `We live far away from you,' even though you were, in fact, living in our midst? \v{23}Now therefore you are under a curse. Some of you will always be slaves, wood cutters, and water carriers for the house of my God.''

\v{24}They replied to Joshua, ``Because your servants had been informed that the \divine{Lord} your God had certainly commanded his servant Moses to give you the entire land and to destroy all of the inhabitants of the land before you. So we were terrified for our lives because of you. That's why we did this. \v{25}Now we're under your control: do to us as it seems good and right in your opinion.''

\v{26}So this is what Joshua\fnote{\fbackref{9:26} Lit. \fbib{he}} did for them: he saved them from the Israelis, and they did not kill them. \v{27}However, on that very day Joshua made them become wood cutters and water carriers for the congregation and for the \divine{Lord}'s altar in the place that he should choose, and this tradition continues\fnote{\fbackref{9:27} The Heb. lacks \fbib{and this tradition continues}} to this day.
\labelchapt{10}
\passage{The Sun Stands Still}

\chapt{10}
\v{1}King Adoni-zedek of Jerusalem eventually heard how Joshua had conquered Ai, utterly destroying it, doing to Ai and its king the same thing that he had done to Jericho and its king, and how the inhabitants of Gibeon had made peace with Israel and were now living among them. \v{2}So they\fnote{\fbackref{10:2} I.e. the inhabitants of Jerusalem} were terrified, since Gibeon was a large city, comparable to one of the royal cities, was larger than Ai, and all of its men had been warriors.

\v{3}So King Adoni-zedek of Jerusalem sent word to King Hoham of Hebron, King Piram of Jarmuth, King Japhia of Lachish, and King Debir of Eglon. He told them, \v{4}``Come over and help me, and let's attack Gibeon, because it made a peace treaty with Joshua and the Israelis.'' \v{5}So the five kings of the Amorites---the king of Jerusalem, the king of Hebron, the king of Jarmuth, the king of Lachish, and the king of Eglon---gathered their armies together and advanced with all of their armies toward Gideon, camped there, and laid siege to it.

\v{6}The Gibeonites sent word to Joshua at his camp in Gilgal: ``Don't abandon your servants. Come quickly, save us, and help us, because all of the kings of the Amorites who live in the hill country have attacked us.'' \v{7}So Joshua went up from Gilgal, along with his entire fighting force of mighty warriors with him.

\v{8}The \divine{Lord} told Joshua, ``Don't fear them, because I have handed them over to you. Not one of them will withstand you.'' \v{9}So after an all-night march from Gilgal, Joshua attacked them by surprise. \v{10}The \divine{Lord} threw the Amorites\fnote{\fbackref{10:10} Lit. \fbib{threw them}} into a panic right in front of the army\fnote{\fbackref{10:10} The Heb. lacks \fbib{the army}} of Israel, which then slaughtered many of them at Gibeon. The Israeli army\fnote{\fbackref{10:10} Lit. \fbib{They}} chased them along the road that goes up to Beth-horon, striking them down as far as Azekah and Makkedah. \v{11}While they were fleeing in front of Israel and descending the slope of Beth-horon, the \divine{Lord} rained down huge hailstones on them as far as Azekah, and they died. More died because of the hailstones than were killed by the Israelis in battle.\fnote{\fbackref{10:11} Lit. \fbib{Israelis by the sword}} \v{12}Later that day, Joshua spoke to the \divine{Lord} while the \divine{Lord} was delivering the Amorites to the Israelis. This is what he said in the presence of Israel:

\begin{poetry}
\poeml ``Sun, be still over Gibeon \\
\poemll    Moon, stand in place\fnote{\fbackref{10:12} The Heb. lacks \fbib{stand in place}} in the Aijalon Valley'' \\
\poeml \v{13}So the sun remained still \\
\poemll    and the moon stood in place \\
\poemlll       until the nation settled their score with their enemies.
\end{poetry}

This is recorded, is it not, in the book of Jashar?\fnote{\fbackref{10:13} Lit. \fbib{the Book of the Upright}; i.e. an ancient chronicle of Israel, apparently now lost. The first half of v. 13 rather than the quatrain following may be the citation.}

\begin{poetry}
\poeml The sun stood in place \\
\poemll    in the middle of the sky \\
\poeml and seemed not to be in a hurry \\
\poemll    to set for nearly an entire day.
\end{poetry}

\v{14}There has never been a day like it before or since, when the \divine{Lord} listened to the voice of a man, because the \divine{Lord} was fighting on behalf of Israel.

\v{15}After this, Joshua returned to the camp at Gilgal with the entire fighting force of\fnote{\fbackref{10:15} The Heb. lacks \fbib{fighting force of}} Israel.
\passage{Defeat of the Five Kings}

\v{16}Meanwhile, the five kings had fled and hidden themselves inside a cave at Makkedah. \v{17}Joshua was informed, ``The five kings have been discovered hiding in the cave at Makkedah.''

\v{18}So Joshua gave an order, ``Roll large stones up against the mouth of the cave and assign men to stand guard there, \v{19}but don't stay there yourselves. Instead, pursue your enemies and attack them from behind. Don't allow them to enter their cities, because the \divine{Lord} your God has delivered them into your control.''

\v{20}Now it came about that after Joshua and the Israelis had finished the battle,\fnote{\fbackref{10:20} Lit. \fbib{slaughter}} destroying and scattering their survivors, who retreated into their fortified cities, \v{21}the entire army returned safely to Joshua's encampment at Makkedah. No one could speak so much as a single word against any of the Israelis.

\v{22}Then Joshua gave this order: ``Unseal the mouth of the cave and bring out these five kings to me from the cave.''

\v{23}So they did. They brought out these five kings to him from within the cave: the king of Jerusalem, the king of Hebron, the king of Jarmuth, the king of Lachish, and the king of Eglon. \v{24}When they had brought these kings out to Joshua, Joshua called for all the men of Israel and spoke to the leaders of the men who had gone out to war along with him, ``Come close and put your feet on the necks of these kings.'' So they came near and put their feet on their necks.

\v{25}Joshua told the army,\fnote{\fbackref{10:25} Lit. \fbib{to them}} ``Don't fear or be dismayed! Be strong and courageous, because this is how the \divine{Lord} will treat all of your enemies whom you fight.''

\v{26}After this, Joshua struck those kings\fnote{\fbackref{10:26} Lit. \fbib{struck them}} down, executing them, and hanged them on five gallows\fnote{\fbackref{10:26} Or \fbib{trees}} until sunset. \v{27}When evening had come, Joshua gave a command to remove the bodies\fnote{\fbackref{10};27 Lit. \fbib{remove them}} from the gallows\fnote{\fbackref{10:27} Or \fbib{trees}} and bury them in the cave where they had hidden. The army\fnote{\fbackref{10:27} Lit. \fbib{They}} sealed the mouth of the cave with large stones that remain there to this very day.
\passage{The Southern Campaign}

\v{28}Joshua captured Makkedah that very day, and attacked both it and its king with swords, utterly destroying it along with every person in it, leaving no survivors. He dealt with the king of Makkedah the same way he had dealt with the king of Jericho.

\v{29}Afterward, Joshua and all of Israel passed on from Makkedah to Libnah, where they fought against Libnah. \v{30}The \divine{Lord} gave both it and its king into the control of Israel, and Joshua\fnote{\fbackref{10:30} Lit. \fbib{he}} executed both its king\fnote{\fbackref{10:30} Lit. \fbib{both him}} and every person in it with swords, leaving no survivors. He dealt with the king the same way he had dealt with the king of Jericho.

\v{31}Then Joshua and all of Israel passed from Libnah to Lachish, camped near it, and attacked it. \v{32}The \divine{Lord} gave Lachish into the control of Israel, and Joshua captured it the next day. He declared war on the city and executed\fnote{\fbackref{10:32} Lit. \fbib{He struck it with the edge of the sword and}} everyone in it, the same way he had treated Libnah.

\v{33}Then Horam king of Gezer appeared to help Lachish. So Joshua attacked him and his army, until he left no one remaining. \v{34}After this, Joshua, accompanied by all of Israel, proceeded from Lachish to Eglon, laid siege to it, and attacked it. \v{35}They captured it on that day, attacking it in battle. Then Joshua completely destroyed it that day, the same way he had dealt with Lachish.

\v{36}Then Joshua, accompanied by all of Israel, left Eglon for Hebron, where they attacked it, \v{37}captured it, and executed its inhabitants---its king, all of its cities, and every person in it, leaving no one remaining, the same way he had dealt with Eglon. He completely destroyed it, along with everyone in it.

\v{38}Then Joshua returned, accompanied by the entire fighting force of\fnote{\fbackref{10:38} The Heb. lacks \fbib{fighting force of}} Israel, to Debir, where they attacked it, \v{39}captured it, its king, and all of its villages. They executed them, totally destroying it and everyone in it, leaving no one remaining. He dealt with Debir and its king just as he had dealt with Hebron, treating them the same way he had dealt with Libnah and its king.

\v{40}So Joshua conquered the entire land, the hill country, the Negev,\fnote{\fbackref{10:40} I.e. the southern regions of the Sinai peninsula} the Shephelah,\fnote{\fbackref{10:40} I.e. the verdant central lowlands of Israel; and so throughout the book} and the wilderness highlands, along with all of their kings. He left none of them remaining, but completely destroyed every living person, just as the \divine{Lord} God of Israel had commanded. \v{41}Joshua conquered them from Kadesh-barnea to Gaza, including the entire territory of Goshen as far as Gibeon. \v{42}Joshua conquered all of these kings and their territories in one campaign, because the \divine{Lord} God of Israel fought for Israel. \v{43}Then Joshua returned to the camp at Gilgal, along with the entire fighting force of\fnote{\fbackref{10:43} The Heb. lacks \fbib{fighting force of}} Israel.
\labelchapt{11}
\passage{The Northern Campaign}

\chapt{11}
\v{1}When King Jabin of Hazor heard all of this,\fnote{\fbackref{11:1} The Heb. lacks \fbib{all of this}} he sent word\fnote{\fbackref{11:1} The Heb. lacks \fbib{word}} to Jobab king of Madon, to the king of Shimron, to the king of Achshaph, \v{2}and to the kings in the north, in the hill country, in the plain south of Chinnereth, in the Shephelah, and in the hills of Dor toward the west, \v{3}to the eastern and western Canaanites---the Amorites, the Hittites, the Perizzites, the Jebusites in the hill country, and the Hivites below Hermon in the territory of Mizpah. \v{4}So they went out, they and all of their armies with them---a multitude as numerous as the sand on the seashore---accompanied by many horses and chariots. \v{5}After all these kings had gathered together, they went out and camped together at the waters of Merom to fight Israel.

\v{6}But the \divine{Lord} told Joshua, ``Don't be afraid of them, because tomorrow about this time I am giving them all to you---dead---in the presence of Israel. Hamstring their horses and incinerate their chariots.''

\v{7}So Joshua and his entire fighting force approached them suddenly by the waters of Merom and attacked them. \v{8}The \divine{Lord} handed them over to the control of Israel, who defeated them and chased them as far as Greater Sidon and east as far as the Mizpah Valley. They attacked them until none remained. \v{9}Joshua dealt with them just as the \divine{Lord} had told him: he hamstrung their horses and incinerated their chariots.

\v{10}Joshua then turned back and captured Hazor, executing its king, because Hazor used to be the head of all of those kingdoms. \v{11}They executed all of the people who lived in it, completely destroying it and leaving no one alive. Then he burned Hazor in fire.

\v{12}So Joshua captured and annihilated all of these cities, along with their kings, completely destroying them, just as Moses the servant of the \divine{Lord} had commanded. \v{13}However, Israel did not burn any of the cities that had been built on mounds of ruins,\fnote{\fbackref{11:13} Lit. \fbib{on tels}} except for Hazor only, which Joshua burned. \v{14}The Israelis took the spoils of war from these cities, along with their livestock, but they executed every human being until they had completely destroyed them, leaving no one alive. \v{15}Joshua did just what the \divine{Lord} had commanded his servant Moses and just what Moses had commanded him, leaving nothing unfinished.
\passage{Summary of Joshua's Victory}

\v{16}So Joshua conquered all of these territories: the hill country, all of the Negev,\fnote{\fbackref{11:16} I.e. the southern regions of the Sinai peninsula; cf. Josh 10:40} the entire land of Goshen with its foothills, the plains of Jordan, and the mountains of Israel with its foothills \v{17}from Mount Halak and the ascent toward Seir, including as far as Baal-gad in the Lebanon Valley that lies at the foot of Mount Hermon. Joshua captured all of their kings, struck them down, and put them to death. \v{18}Joshua fought an extended campaign against all those kings. \v{19}There wasn't a single\fnote{\fbackref{11:19} The Heb. lacks \fbib{single}} city that made a peace accord with the Israelis, except the Hivites who lived in Gibeon. The Israelis\fnote{\fbackref{11:19} Lit. \fbib{They}} captured all the rest\fnote{\fbackref{11:19} The Heb. lacks \fbib{the rest}} in battle, \v{20}because the \divine{Lord} had hardened their hearts so they would fight Israel in war, be completely destroyed without mercy, and be completely wiped out, as the \divine{Lord} had commanded Moses.

\v{21}At that time Joshua came and annihilated the Anakim\fnote{\fbackref{11:21} I.e. a race of giants that formerly populated Canaan; cf. Num 13:22, 33; Deut 9:2} from the hill country, that is, from Hebron, Debir, and Anab, as well as from all the hill country of Judah and Israel. Joshua completely destroyed them along with their cities. \v{22}None of the Anakim\fnote{\fbackref{11:22} I.e. a race of giants that formerly populated Canaan; cf. Num 13:22, 33; Deut 9:2} remained in the land belonging to the Israelis---they remained only in Gaza, in Gath, and in Ashdod. \v{23}Joshua conquered the entire land, in accordance with everything that the \divine{Lord} had told Moses. Joshua presented it as an inheritance to Israel, dividing it according to tribal allotments. Then the land enjoyed rest from war.
\labelchapt{12}
\passage{Kingdoms Conquered by Israel}

\chapt{12}
\v{1}This is a list of the kings who ruled the land that the Israelis conquered, and whose territories they took on the other side of the Jordan River toward the east, from the Arnon River to Mount Hermon, along with the entire eastern Jordan plain.\fnote{\fbackref{12:1} Lit. \fbib{Arabah}} \v{2}Sihon king of the Amorites lived in Heshbon and ruled from Aroer, which is located on the edge of the Arnon River\fnote{\fbackref{12:2} The Heb. lacks \fbib{River}} from the middle of the valley, including half of Gilead as far as Wadi\fnote{\fbackref{12:2} I.e. a seasonal stream or river that channels water during rain seasons but is dry at other times} Jabbok, the border of the Ammonites, \v{3}and toward the Arabah as far as the Sea of Galilee\fnote{\fbackref{12:3} Lit. \fbib{Chinnereth}} to the east, as far as the Arabah Sea (that is, the Dead Sea) to the east as one travels in the direction\fnote{\fbackref{12:3} Lit. \fbib{east in the road}} of Beth-jeshimoth, and to the south as far as the foothills of Pisgah.\fnote{\fbackref{12:3} Lit. \fbib{Ashdoth-pisgah}; perhaps including Mount Nebo} \v{4}The territory of Og king of Bashan was conquered. He was\fnote{\fbackref{12:4} The Heb. lacks \fbib{was conquered. He was}} one of the last of the Rephaim,\fnote{\fbackref{12:4} I.e. a race of giants that formerly populated Canaan; cf. Num 13:22, 33} and lived at Ashtaroth and Edrei, \v{5}ruling over Mount Hermon, Salecah, and all of Bashan as far as the border of the descendants of Geshur, the descendants of Maacath, and half of Gilead to the border of Sihon king of Heshbon.

\v{6}Moses, the servant of the \divine{Lord}, and the Israelis defeated them. Then Moses, the servant of the \divine{Lord}, gave it to the descendants of Reuben, the descendants of Gad, and the half-tribe of Manasseh as their inheritance.\fnote{\fbackref{12:6} Or \fbib{possession}} \v{7}This is a list of the kings of the land whom Joshua and the Israelis defeated beyond the Jordan River toward the west, from Baal-gad in the Lebanon valley as far as Mount Halak, which rises in the direction of Seir. Joshua gave it to Israel, distributing it according to their tribal divisions as their inheritance, \v{8}in the mountain regions, in the Arabah, on the foothills, in the wilderness, in the Negev;\fnote{\fbackref{12:8} I.e. the southern regions of the Sinai peninsula; cf. Josh 10:40} that is, the Hittites, the Amorites, the Canaanites, the Perizzites, the Hivites, and the Jebusites:

\v{9}The king of Jericho: 1

The king of Ai, which is near Bethel: 1

\v{10}The king of Jerusalem: 1

The king of Hebron: 1

\v{11}The king of Jarmuth: 1

The king of Lachish: 1

\v{12}The king of Eglon: 1

The king of Gezer: 1

\v{13}The king of Debir: 1

The king of Geder: 1

\v{14}The king of Hormah: 1

The king of Arad: 1

\v{15}The king of Libnah: 1

The king of Adullam: 1

\v{16}The king of Makkedah: 1

The king of Bethel: 1

\v{17}The king of Tappuach: 1

The king of Hepher: 1

\v{18}The king of Aphek: 1

The king of Lasharon: 1

\v{19}The king of Madon: 1

The king of Hazor: 1

\v{20}The king of Shimron-meron: 1

The king of Achshaph: 1

\v{21}The king of Taanach: 1

The king of Megiddo: 1

\v{22}The king of Kedesh: 1

The king of Jokneam in Carmel: 1

\v{23}The king of Dor in the Dor heights: 1

The king of various\fnote{\fbackref{12:23} The Heb. lacks \fbib{various}} gentiles in Gilgal:\fnote{\fbackref{12:23} So MT; LXX reads \fbib{of Goyim in Galilee}} 1

\v{24}The king of Tirzah: 1

Total number of all kings: 31
\labelchapt{13}
\passage{Territories Yet to be Conquered}

\chapt{13}
\v{1}When Joshua had grown old, having lived many years, the \divine{Lord} told him, ``You are old and have lived many years, but much of the land still remains to be possessed. \v{2}This territory remains: all of the Philistine regions, including all Geshurite holdings\fnote{\fbackref{13:2} The Heb. lacks \fbib{holdings}} \v{3}from the Shihor east of Egypt as far as the border of Ekron on the north (which is considered part of Canaan). This includes the five rulers of the Philistines, the Gazites, the Ashdodites, the Ashkelonites, the Gittites, the Ekronites, and the Avvites.

\v{4}``To the south, there remains to be conquered\fnote{\fbackref{13:4} The Heb. lacks \fbib{there remains to be conquered}} all the territory held by the Canaanites, Mearah that belongs to the Sidonians, as far as Aphek, to the border of the Amorites, \v{5}including the territory of the Gebalites and all of Lebanon facing the east from Baal-gad at the foot of Mount Hermon as far as Lebo-hamath, \v{6}and all the inhabitants of the hill country from Lebanon to Misrephoth-maim, including all the Sidonians. I myself will drive them out in the presence of the Israelis. \v{7}You only have to allocate the land as an inheritance, just as I commanded you.''
\passage{Summary of Allocations to Reuben, Gad, and Manasseh}

\v{8}The descendants of Reuben and descendants of Gad, along with the other half-tribe of Manasseh, received their inherited portion that Moses the servant of the \divine{Lord} had given them to the east beyond the Jordan River. \v{9}Specifically included was from Aroer on the banks of the Wadi\fnote{\fbackref{13:9} I.e. a seasonal stream or river that channels water during rain seasons but is dry at other times} Arnon, and the town that lies in the middle of the valley, including all the plains from Medeba to Dibon, \v{10}all the cities pertaining to King Sihon of the Amorites, who reigned in Heshbon, as far as the boundary of the Ammonite territory,\fnote{\fbackref{13:10} The Heb. lacks \fbib{territory}} \v{11}Gilead and the region belonging to the descendants of Geshur and Maacath, including all of Mount Hermon, and all of Bashan as far as Salecah. \v{12}Also included was\fnote{\fbackref{13:12} The Heb. lacks \fbib{Also included was}} the entire kingdom of Og in Bashan, who reigned in Ashtaroth and Edrei. (He was the sole survivor left of the Rephaim.)\fnote{\fbackref{13:12} I.e. a race of giants that formerly populated Canaan; cf. Num 13:22, 33} Although Moses had defeated these people and driven them out, \v{13}the Israelis did not drive out the descendants of Geshur or the descendants of Maacath---Geshur and Maacath live within the territory of Israel to this day.
\passage{Allocations to Levi}

\v{14}Moses allotted no inheritance solely to the tribe of Levi. As he had mentioned to them, the offerings by fire to the \divine{Lord} God of Israel are their inheritance.
\passage{Allocations to Reuben}

\v{15}Moses allocated territory\fnote{\fbackref{13:15} The Heb. lacks \fbib{territory}} to the tribe of the descendants of Reuben according to their tribes. \v{16}Their allocation was from the border of Aroer on the edge of the Arnon valley (including the city that is located in the valley, as well as the entire plain next to Medeba), \v{17}Heshbon and all of its cities that are on the plain, including Dibon, Bamoth-baal, Beth-baal-meon, \v{18}Jahaz, Kedemoth, Mephaath, \v{19}Kiriathaim, Sibmah, and Zereth-shahar on the hill in the valley, \v{20}Beth-peor, the slopes of Pisgah, Beth-jeshimoth, \v{21}all of the cities of the plain, the entire kingdom of King Sihon of the Amorites, who used to reign in Heshbon and whom Moses attacked, along with the chiefs of Midian, Evi, Rekem, Zur, Hur, and Reba, nobles of Sihon who lived in the land. \v{22}The Israelis also killed Beor's son Balaam, the occult practitioner, executing him with a sword as one of those killed. \v{23}The border of the descendants of Reuben was the Jordan River and its banks. This was the inheritance belonging to the descendants of Reuben, divided according to their families, cities, and villages.
\passage{Allocations to Gad}

\v{24}Moses also allocated territory\fnote{\fbackref{13:24} The Heb. lacks \fbib{territory}} to the tribe of Gad, that is, to the descendants of Gad, according to their families. \v{25}Their territory included Jazer, all the cities of Gilead, half the land of the Ammonites as far as Aroer which is located near Rabbah, \v{26}from Heshbon as far as Ramath-mizpeh and Betonim, from Mahanaim as far as the border of Debir, \v{27}the valley containing Beth-haram, Beth-nimrah, Succoth, and Zaphon, the rest of the kingdom of King Sihon of Heshbon, with the Jordan River as its border as far as the southern\fnote{\fbackref{13:27} The Heb. lacks \fbib{southern}} end of the Sea of Galilee\fnote{\fbackref{13:27} Lit. \fbib{Chinnereth}} beyond the Jordan River to the east. \v{28}This was the inheritance belonging to the descendants of Gad according to their tribes, cities, and villages.
\passage{Allocations to Manasseh}

\v{29}Moses also allocated territory\fnote{\fbackref{13:29} The Heb. lacks \fbib{territory}} to the half-tribe of Manasseh, that is, for the half-tribe of the descendants of Manasseh according to their tribes. \v{30}Their territory extended from Mahanaim to include\fnote{\fbackref{13:30} The Heb. lacks \fbib{to include}} all of Bashan, all of the kingdom of King Og of Bashan, all of the 60 towns of Jair there in Bashan, \v{31}half of Gilead, including Ashtaroth and Edrei. The cities of the kingdom of Og in Bashan went to half of the descendants of Manasseh's son Machir, according to their tribes. \v{32}These were the allotments\fnote{\fbackref{13:32} The Heb. lacks \fbib{the allotments}} that Moses apportioned for an inheritance in the plains of Moab beyond the Jordan River east of Jericho.
\passage{Allocations to Levi}

\v{33}Moses allotted no inheritance to the tribe of Levi. The \divine{Lord} God of Israel is their inheritance, as he promised them.
\labelchapt{14}
\passage{Summary of Allocations}

\chapt{14}
\v{1}This is what the Israelis inherited in the land of Canaan, which Eleazar the priest, Nun's son Joshua, and the heads of the families of the Israelis allotted to them as an inheritance \v{2}by lot, just as the \divine{Lord} commanded through Moses for the nine tribes and the half-tribe, \v{3}since Moses had given the inheritance of the two tribes and the half-tribe across the Jordan River. However, he did not give an inheritance to the descendants of Levi who lived among them, \v{4}since the descendants of Joseph constituted two tribes---Manasseh and Ephraim. They did not allot a portion to the descendants of Levi in the land, since they were given\fnote{\fbackref{14:4} The Heb. lacks \fbib{they were given}} cities to live in, along with pastures for their livestock and property. \v{5}So the Israelis did just as the \divine{Lord} had commanded Moses---they divided the land.
\passage{Caleb's Request}
\passageinfo{(Judges 1:20)}

\v{6}After this, the descendants of Judah approached Joshua in Gilgal. Jephunneh the Kenizzite's son Caleb told him, ``You know the promise that the \divine{Lord} gave Moses the man of God concerning the two of us in Kadesh-barnea. \v{7}I was 40 years old when Moses the servant of the \divine{Lord} sent me from Kadesh-barnea to scout the land. I brought back an honest report\fnote{\fbackref{14:7} Lit. \fbib{a report with my heart}} to him. \v{8}As it happened, my fellow Israelis who went up with me terrified the people, but I fully followed the \divine{Lord} my God. \v{9}Moses made a promise to me on that day when he said, `The land that you covered on foot will certainly be your inheritance. It will belong to your descendants forever, because you have fully followed the \divine{Lord} my God.'

\v{10}``Look how\fnote{\fbackref{14:10} The Heb. lacks \fbib{how}} the \divine{Lord} has let me survive, as you can see, these 45 years since the time when the \divine{Lord} said this through Moses, while Israel was wandering through the wilderness. And look! I'm here today---my eighty-fifth birthday! \v{11}I'm still as strong today as I was the day Moses commissioned me. I'm as strong today as I was then, and I can go out to battle and come back successful. \v{12}Now then, give me that hill country about which the \divine{Lord} spoke back on that day, because you yourself heard back then that the Anakim\fnote{\fbackref{14:12} 2 I.e. a race of giants that formerly populated Canaan; cf. Num 13:22, 33; Deut 9:2} were there, with great reinforced cities. Perhaps the \divine{Lord} will be with me and I will expel them, just as the \divine{Lord} said.''

\v{13}So Joshua blessed him and gave Hebron to Jephunneh's son Caleb for his inheritance. \v{14}Therefore Hebron became the inheritance of Jephunneh the Kenizzite's son Caleb, and it remains so today, because he fully followed the \divine{Lord} God of Israel. \v{15}Hebron used to be known as Kiriath-arba, after the greatest man among the Anakim.\fnote{\fbackref{14:15} I.e. a race of giants that formerly populated Canaan; cf. Num 13:22, 33; Deut 9:2} After all of this, the land enjoyed rest from war.
\labelchapt{15}
\passage{Allotments to Judah}

\chapt{15}
\v{1}Joshua said,\fnote{\fbackref{15:1} The Heb. lacks \fbib{Joshua said}} ``Now the allotment for the tribe of the descendants of Judah, allocated\fnote{\fbackref{15:1} The Heb. lacks \fbib{allocated}} according to their families, will extend to the border of Edom, southward to the wilderness of Zin until land's end, \v{2}then from the southern end of the Dead Sea, that is, from the bay that orients toward the Negev,\fnote{\fbackref{15:2} I.e. the southern regions of the Sinai peninsula; cf. Josh 10:40} \v{3}proceeding south to the ascent of Akrabbim, then continuing to Zin, and from there up along the south of Kadesh-barnea to Hezron, and from there up to Addar and then to Karka, \v{4}passing along to Azmon toward the Wadi\fnote{\fbackref{15:4} I.e. a seasonal stream or river that channels water during rain seasons but is dry at other times} of Egypt and ending at the sea. This will be your southern border.''

\v{5}The eastern border was the Dead Sea as far as the mouth of the Jordan River. The border of the north side extended from the bay of the sea at the mouth of the Jordan River \v{6}toward Beth-hoglah, and continuing on the north of Beth-arabah. The border ascended up to the boundary marker set up by Reuben's son Bohan.

\v{7}The boundary then went up to Debir from the Achor valley and turned north toward Gilgal opposite the ascent of Adummim in the southern part of the valley. Then the border continued to the waters of En-shemesh and terminated at En-rogel. \v{8}Then the border proceeded up the valley of Ben-hinnom to the southern ascent of the Jebusites (that is, to Jerusalem), and from there to the top of the mountain that faces the valley of Hinnom to the west at the end of the valley of Rephaim\fnote{\fbackref{15:8} Lit. \fbib{Valley of the Giants}; the Rephaim were a race of giants that formerly populated Canaan; cf. Num 13:22, 33; Deut 9:2} toward the north.

\v{9}The border proceeded from the top of the mountain to the spring of the waters of Nephtoah, then to the cities of Mount Ephron, and then the border curved toward Baalah (also known as Kiriath-jearim). \v{10}The border turned west from Baalah to Mount Seir,\fnote{\fbackref{15:10} This mountain, the modern \fbib{Jebel esh-sher\'{a}}, is located in the mountain range that extends south of the Dead Sea toward the Gulf of Aqaba, and is bordered by the Arabah Valley to the west.} continuing to the top of Mount Jearim on the north (also known as Chesalon), and then proceeded to Beth-shemesh, continuing through Timnah.

\v{11}The border proceeded north to the edge of Ekron, then curved to Shikkeron and on to Mount Baalah, proceeding then to Jabneel, where the boundary ended at the sea. \v{12}The western border was at the Mediterranean Sea coastline. This is the border that surrounded the territory of\fnote{\fbackref{15:12} The Heb. lacks \fbib{the territory of}} the descendants of Judah, according to their families.
\passage{Caleb's Conquests}
\passageinfo{(Judges 1:11-15)}

\v{13}Now Joshua\fnote{\fbackref{15:13} Lit. \fbib{he}} gave an allotment among the descendants of Judah to Jephunneh's son Caleb, just as God told Joshua, Kiriath-arba, which was named after the\fnote{\fbackref{15:13} The Heb. lacks \fbib{which was named after the}} ancestor of Anak (that is, Hebron). \v{14}From there Caleb drove the three descendants of Anak, Sheshai, Ahiman, and Talmai---the Anakim.\fnote{\fbackref{15:14} I.e. a race of giants that formerly populated Canaan; cf. Num 13:22, 33; Deut 9:2} \v{15}Then he went up from there to attack the inhabitants of Debir. (Debir was formerly known as Kiriath-sepher.)

\v{16}Then Caleb announced, ``I will give my daughter Achsah in marriage to the one who attacks Kiriath-sepher and captures it.'' \v{17}Othniel, the son of Caleb's brother Kenaz, captured it, so Caleb gave him his daughter Achsah as his wife. \v{18}Sometime later, she came to Othniel\fnote{\fbackref{15:18} The Heb. lacks \fbib{to Othniel}} and persuaded him to ask her father for a field.

As she dismounted from her donkey, Caleb asked her, ``What do you want?''

\v{19}She replied, ``Give me a blessing. Since you have given me the land of the Negev,\fnote{\fbackref{15:19} I.e. the southern regions of the Sinai peninsula; cf. Josh 10:40} give me also some springs of water.'' So he gave her the upper and lower springs.
\passage{City Allotments for Judah}

\v{20}Here's a list of cities allotted for the tribe of the descendants of Judah according to their families: \v{21}The cities to the far south of the tribe of the descendants of Judah (toward the border of Edom in the south) included Kabzeel, Eder, Jagur, \v{22}Kinah, Dimonah, Adadah, \v{23}Kedesh, Hazor, Ithnan, \v{24}Ziph, Telem, Bealoth, \v{25}Hazor-hadattah, Kerioth-hezron (also known as Hazor), \v{26}Amam, Shema, Moladah, \v{27}Hazar-gaddah, Heshmon, Beth-pelet, \v{28}Hazar-shual, Beer-sheba, Biziothiah, \v{29}Baalah, Iim, Ezem, \v{30}Eltolad, Chesil, Hormah, \v{31}Ziklag, Madmannah, Sansannah, \v{32}Lebaoth, Shilhim, Ain, and Rimmon, for a total of 29 cities and villages.

\v{33}The lowland cities included Eshtaol, Zorah, Ashnah, \v{34}Zanoah, En-gannim, Tappuach, Enam, \v{35}Jarmuth, Adullam, Socoh, Azekah, \v{36}Shaaraim, Adithaim, Gederah, and Gederothaim, for a total of fourteen cities and villages.

\v{37}Also included were\fnote{\fbackref{15:37} The Heb. lacks \fbib{Also included were}; and so throughout the chapter} Zenan, Hadashah, Migdal-gad, \v{38}Dilan, Mizpeh, Joktheel, \v{39}Lachish, Bozkath, Eglon, \v{40}Cabbon, Lahmam, Chitlish, \v{41}Gederoth, Beth-dagon, Naamah, and Makkedah, for a total of sixteen cities and villages.

\v{42}Also included were Libnah, Ether, Ashan, \v{43}Iphtah, Ashnah, Nezib, \v{44}Keilah, Achzib, and Mareshah, for a total of nine cities and villages.

\v{45}Also included were Ekron, with its towns and villages, \v{46}from Ekron to the Mediterranean\fnote{\fbackref{15:46} The Heb. lacks \fbib{Mediterranean}} Sea, including everything by the edge of Ashdod, along with their villages, \v{47}Ashdod and its towns and villages, Gaza and its towns and villages as far as the River of Egypt, and the coastline of the Mediterranean Sea.

\v{48}The hill country included Shamir, Jattir, Socoh, \v{49}Dannah, Kiriath-sannah (also known as Debir), \v{50}Anab, Eshtemoh, Anim, \v{51}Goshen, Holon, Giloh, for a total of eleven cities and villages. \v{52}Also included were Arab, Dumah, Eshan, \v{53}Janum, Beth-tappuach, Aphekah, \v{54}Humtah, Kiriath-arba (also known as Hebron), and Zior, for a total of nine cities and villages. \v{55}Also included were Maon, Carmel, Ziph, Juttah, \v{56}Jezreel, Jokdeam, Zanoah, \v{57}Kain, Gibeah, and Timnah, for a total of ten cities and villages. \v{58}Also included were Halhul, Beth-zur, Gedor, \v{59}Maarath, Beth-anoth, and Eltekon, for a total of six cities and villages. \v{60}Also included were Kiriath-baal (also known as Kiriath-jearim) and Rabbah, for a total of two cities and villages.

\v{61}The wilderness included Beth-arabah, Middin, Secacah, \v{62}Nibshan, Salt City, and En-gedi, for a total of six cities and villages.

\v{63}Now as for the Jebusites who lived in Jerusalem, the descendants of Judah could not expel them, so Jebusites live with the descendants of Judah in Jerusalem to this day.
\labelchapt{16}
\passage{Ephraim's Allocation}

\chapt{16}
\v{1}The territorial allotment for the descendants of Joseph proceeded from the Jordan River by Jericho eastward of the Jericho waters into the wilderness, proceeding from Jericho through the hill country of Bethel \v{2}and from Bethel to Luz, continuing to the border of the Archites at Ataroth. \v{3}It proceeded westward to the territory of the Japhletites as far as the territory of lower Beth-horon, then toward Gezer, ending at the Mediterranean\fnote{\fbackref{16:3} The Heb. lacks \fbib{Mediterranean}} Sea.

\v{4}Manasseh and Ephraim, the descendants of Joseph, received their inheritance. \v{5}This was the territory allocated to the descendants of Ephraim according to their families: the border of their inheritance on the east was Ataroth-addar as far as upper Beth-horon. \v{6}Then the border proceeded west from Michmethath on the north, then turned east toward Taanath-shiloh, continuing to the east of Janoah. \v{7}It proceeded from Janoah to Ataroth, then to Naarah, then proceeded to Jericho and ended at the Jordan River. \v{8}From Tappuach, the border proceeded west to the Kanah brook, ending at the Mediterranean Sea. This is the inheritance of the tribe of the descendants of Ephraim according to their families, \v{9}along with the cities that had been set aside for the descendants of Ephraim within the allotment of the descendants of Manasseh, including all of the cities and villages. \v{10}However, they did not drive out the Canaanites who lived in Gezer, so the Canaanites live within the territory of\fnote{\fbackref{16:10} The Heb. lacks \fbib{the territory of}} Ephraim to this day, but they serve as forced laborers.
\labelchapt{17}
\passage{Manasseh's Allocation}

\chapt{17}
\v{1}The territorial allotment for the tribe of Manasseh, the firstborn of Joseph, was allocated first\fnote{\fbackref{17:1} The Heb. lacks \fbib{was allocated first}} to Machir the firstborn of Manasseh and father of Gilead. Since he had been a man of war, Gilead and Bashan were allocated to him.\fnote{\fbackref{17:1} The Heb. lacks \fbib{were allocated to him}}

\v{2}Now allotments were made\fnote{\fbackref{17:2} The Heb. lacks \fbib{allotments were made}} with respect to the remaining descendants of Manasseh according to their families: for the descendants of Abiezer, the descendants of Helek, the descendants of Asriel, the descendants of Shechem, the descendants of Hepher, and the descendants of Shemida---the male descendants of Joseph's son Manasseh, according to their families.

\v{3}Hepher's son Zelophehad, grandson of Gilead and great-grandson of Manasseh's son Machir had no sons, only daughters. These are the names of his daughters: Mahlah, Noah, Hoglah, Milcah, and Tirzah. \v{4}They appeared before Eleazar the priest and Nun's son Joshua and declared, ``The \divine{Lord} commanded Moses to give us an inheritance among our relatives.'' So in keeping what the \divine{Lord} had commanded, he gave them an inheritance among their ancestor's relatives. \v{5}That is why ten allotments fell to Manasseh, besides the land of Gilead and Bashan beyond the Jordan River, \v{6}since the granddaughters of Manasseh received an inheritance along with his sons. (The land of Gilead belonged to the rest of the descendants of Manasseh.)

\v{7}The border of Manasseh proceeded from Asher to Michmethath east of Shechem, then turned south to include the inhabitants of En-tappuach. \v{8}(The territory of Tappuach belonged to Manasseh, but Tappuach itself,\fnote{\fbackref{17:8} The Heb. lacks \fbib{itself}} on the border of Manasseh, was allocated\fnote{\fbackref{17:8} The Heb. lacks \fbib{was allocated}} to the descendants of Ephraim.) \v{9}The border proceeded to the Kanah brook and proceeded south. These cities belonged to Ephraim among the cities of Manasseh, with the border of Manasseh on the north of the brook, terminating at the Mediterranean\fnote{\fbackref{17:9} The Heb. lacks \fbib{Mediterranean}} Sea.

\v{10}The southern area was allocated to Ephraim and the northern area to Manasseh. The Mediterranean\fnote{\fbackref{17:10} The Heb. lacks \fbib{Mediterranean}} Sea was the border, extending to Asher on the North and to Issachar on the east. \v{11}In Issachar and Asher, Manasseh held Beth-shean and its towns, Ibleam and its towns, the inhabitants of En-dor and its towns, the inhabitants of Taanach and its towns, and the inhabitants of Megiddo and its towns, and the three coastal districts.\fnote{\fbackref{17:11} Or \fbib{the third is Napheth}} \v{12}The descendants of Manasseh did not take possession of these cities, because the Canaanites predominated in that territory. \v{13}Later on, when the Israelis had become strong, they forced the Canaanites to work for them, but they never did expel them completely.
\passage{Protests by the Tribe of Joseph}

\v{14}At that time, the descendants of Joseph asked Joshua, ``Why did you give us\fnote{\fbackref{17:14} Lit. \fbib{me}} only one allotment and portion for an inheritance, since we're numerous and the \divine{Lord} has blessed us all along?''

\v{15}So Joshua replied to them, ``Since you're so numerous, go up to the forest and clear ground there for yourselves in the territory where the Perizzites and Rephaim\fnote{\fbackref{17:15} I.e. a race of giants that formerly populated Canaan; cf. Num 13:22, 33; Deut 9:2} are, because the hill country of Ephraim is too narrow for you.''

\v{16}The descendants of Joseph replied, ``The hill country isn't sufficient for us, but all the Canaanites who live on the plain have iron chariots, both those in Beth-shean and its villages as well as the inhabitants of the Jezreel Valley.''

\v{17}So Joshua told the tribes of Joseph, which were Ephraim and Manasseh, ``You're truly a numerous group, and you have great power. You are not to have only one allotment, \v{18}but the hill country will also belong to you. Even though it's a forest, you will clear it and possess it to its farthest borders. You'll drive out the Canaanites, even though they have iron chariots and even though they're strong.''
\labelchapt{18}
\passage{Other Tribal Allotments}

\chapt{18}
\v{1}After this, the entire assembly of the Israelis gathered together at Shiloh and set up the Tent of Meeting there, where the land lay conquered before them. \v{2}Seven tribes remained among the Israelis for whom their inheritances had not yet been allocated.

\v{3}So Joshua told the Israelis, ``How long will you delay invading and taking possession of the land that the \divine{Lord} God of your ancestors has given you? \v{4}Appoint three men from each tribe and I'll send them. They'll begin to go through the land and record a description of it, categorized according to their inheritance, and then they'll report\fnote{\fbackref{18:4} Lit. \fbib{come}} back to me. \v{5}They'll divide it seven ways. Judah will stay in its territory on the south and the house of Joseph will remain in its territory on the north. \v{6}Lay out the land in seven divisions, then bring your report\fnote{\fbackref{18:6} The Heb. lacks \fbib{your report}} here to me. I will then cast lots in the presence of the \divine{Lord} our God. \v{7}The descendants of Levi have no allotment among you, since the priesthood of the \divine{Lord} is their inheritance. Gad, Reuben, and the half-tribe of Manasseh received their inheritance to the east, beyond the Jordan River given to them by Moses the servant of the \divine{Lord}.''

\v{8}So the men started out, following Joshua's command to those who went to scout the land, ``Go through the land and record a description of it, and then return to me. I will then cast lots in the presence of the \divine{Lord} your God in Shiloh.'' \v{9}Then the men left camp and went throughout the land, describing its cities in a book with seven divisions. Then they returned to Joshua at the camp at Shiloh. \v{10}Joshua threw lots in Shiloh in the \divine{Lord}'s presence and divided the land accordingly among the Israelis according to their divisions.
\passage{Benjamin's Allocation}

\v{11}The allotment of the tribe of the descendants of Benjamin came up according to their families, and their territorial allotment fell between the descendants of Judah and the descendants of Joseph. \v{12}Their border started on the north side at the Jordan River, proceeded to the slope of Jericho on the north, then westward through the hill country, and terminated at the wilderness of Beth-aven. \v{13}From there the boundary proceeded south in the direction of Luz to the slope of Luz (also known as Bethel), then proceeded to Ataroth-addar, on the mountain that lies south of Lower Beth-horon. \v{14}From there the boundary changed direction, turning southward on the western side opposite Beth-horon, terminating at Kiriath-baal (also known as Kiriath-jearim), which belongs to Judah. This formed the western boundary.

\v{15}The southern boundary began at the edge of Kiriath-jearim, proceeding from there to Ephron and then to the spring at the Nephtoah Waters. \v{16}From there the boundary proceeded to the border of the mountain that overlooks the Ben-hinnom Valley at the northern end of the Rephaim Valley, where it proceeded down the Hinnom Valley south of the slope of the Jebusites toward En-rogel. \v{17}Then it turned north toward En-shemesh and proceeded from there to Geliloth opposite the ascent of Adummim, where it turned toward the Stone of Bohan, Reuben's son, \v{18}and proceeded north of the slope of Beth-arabah down to the Arabah. \v{19}From there the boundary proceeded to north of the slope of Beth-hoglah and terminated at the northern bay of the Salt\fnote{\fbackref{18:19} Lit. \fbib{Dead}} Sea where the Jordan River ends in the south. This was the southern border. \v{20}The Jordan River formed its boundary on the east. This is the inheritance for the tribe of Benjamin according to its families, boundary by boundary around the entire territory.\fnote{\fbackref{18:20} Lit. \fbib{boundary all around}}

\v{21}The towns belonging to the tribe of Benjamin according to their families were Jericho, Beth-hoglah, Emek-keziz, \v{22}Beth-arabah, Zemaraim, Bethel, \v{23}Avvim, Parah, Ophrah, \v{24}Chephar-ammoni, Ophni, and Geba, for a total of twelve towns and villages. \v{25}Also included were\fnote{\fbackref{18:25} The Heb. lacks \fbib{Also included were}} Gibeon, Ramah, Beeroth, \v{26}Mizpeh, Chephirah, Mozah, \v{27}Rekem, Irpeel, Taralah, \v{28}Zela, Haeleph, Jebus (also known as Jerusalem), Gibeah, and Kiriath-jearim, for a total of fourteen towns and villages. This is the inheritance of the tribe of Benjamin according to their families.
\labelchapt{19}
\passage{Simeon's Allocation}

\chapt{19}
\v{1}The second lottery went to the tribe of Simeon according to its families. Its inheritance was enclosed within the inheritance of the tribe of Judah. \v{2}Its inheritance included Beer-sheba (also known as\fnote{\fbackref{19:2} The Heb. lacks \fbib{also known as}} Shebah), Moladah, \v{3}Hazar-shual, Balah, Ezem, \v{4}Eltolad, Bethul, Hormah, \v{5}Ziklag, Beth-marcaboth, Hazar-susah, \v{6}Beth-lebaoth, and Sharuhen, for a total of thirteen towns and villages. \v{7}Also included were\fnote{\fbackref{19:7} The Heb. lacks \fbib{Also included were}} Ain, Rimmon, Ether, and Ashan, for a total of four towns and villages. \v{8}Also included were\fnote{\fbackref{19:8} The Heb. lacks \fbib{Also included were}} all the surrounding villages as far as Baalath-beer, in Ramah of the Negev.\fnote{\fbackref{19:8} I.e. the southern regions of the Sinai peninsula; cf. Josh 10:40} This was the inheritance of the tribe of Simeon, according to its families. \v{9}The inheritance of the tribe of Simeon was contained in part of the territory of Judah; that is, because the portion allotted to the tribe of Judah was large enough for both tribes, the tribe of Simeon obtained an inheritance within that of Judah.\fnote{\fbackref{19:9} Lit. \fbib{within their inheritance}}
\passage{Zebulun's Allocation}

\v{10}The third lottery went to the tribe of Zebulun according to its families. The boundary of its inheritance extended to Sarid, \v{11}then turned westward toward Maralah, proceeding to Dabbesheth, and then to the valley that is east of Jokneam. \v{12}From Sarid it proceeded back eastward, creating a sunrise boundary at Chisloth-tabor, and proceeded from there to Daberath, then to Japhia, \v{13}from which it proceeded toward the east to Gath-hepher, then to Eth-kazin, then going to Rimmon, where it turned toward Neah. \v{14}On the north of Neah, the boundary turned toward Hannathon, terminating at Iphtah-el Valley \v{15}and Kattath, Nahalal, Shimron, Idalah, and Bethlehem, for a total of twelve towns and villages. \v{16}These towns and villages are the inheritance of the tribe of Zebulun according to its families.
\passage{Issachar's Allocation}

\v{17}The fourth lottery went to the tribe of Issachar according to its families. \v{18}The territorial allotment included Jezreel, Chesulloth, Shunem, \v{19}Hapharaim, Shion, Anaharath, \v{20}Rabbith, Kishion, Ebez, \v{21}Remeth, En-gannim, En-haddah, Beth-pazzez, \v{22}with the boundary including Tabor, Shahazumah, and Beth-shemesh. The boundary terminated at the Jordan River, for a total of sixteen towns and villages. \v{23}These towns and villages comprise the inheritance of the tribe of Issachar, according to its families.
\passage{Asher's Allocation}

\v{24}The fifth lottery went to the tribe of Asher according to its families. \v{25}The territorial boundary included Helkath, Hali, Beten, Achshaph, \v{26}Allammelech, Amad, and Mishal, and on the west Carmel and Shihor-libnath, \v{27}then proceeded east to Beth-dagon. It proceeded to Zebulun and the Iphtah-el Valley, turned north to Beth-emek and Neiel, then proceeded north to Cabul, \v{28}Ebron, Rehob, Hammon, and Kanah as far as Great Sidon. \v{29}There the boundary turned toward Ramah, reaching to the fortress city of Tyre and turned to Hosah, where it terminated at the Mediterranean\fnote{\fbackref{19:29} The Heb. lacks \fbib{Mediterranean}} Sea. Also included were\fnote{\fbackref{19:29} The Heb. lacks \fbib{Also included were}} Mahalab, Achzib, \v{30}Ummah, Aphek, and Rehob, for a total of 22 towns and villages. \v{31}These towns and villages are the inheritance of the tribe of Asher according to its families.
\passage{Naphtali's Allocation}

\v{32}The sixth lottery went to the tribe of Naphtali according to its families. \v{33}The territorial boundary proceeded from Heleph, the oak in Zaanannim, and Adami-nekeb, and Jabneel as far as Lakkum, terminating at the Jordan River. \v{34}There the boundary proceeded west to Aznoth-tabor and then to Hukkok, reaching Zebulun at the south, Asher on the west, and Judah on the east at the Jordan River. \v{35}Also included were\fnote{\fbackref{19:35} The Heb. lacks \fbib{Also included were}} the fortress towns of Ziddim, Zer, Hammath, Rakkath, Chinnereth, \v{36}Adamah, Ramah, Hazor, \v{37}Kedesh, Edrei, En-hazor, \v{38}Iron, Migdal-el, Horem, Beth-anath, and Beth-shemesh, for a total of nineteen towns and their villages. \v{39}These towns and villages comprised the inheritance of the tribe of Naphtali according to its families.
\passage{Dan's Allocation}

\v{40}The seventh lottery went to the tribe of Dan according to its families. \v{41}The territorial allotment included Zorah, Eshtaol, Ir-shemesh, \v{42}Shaalabbin, Aijalon, Ithlah, \v{43}Elon, Timnah, Ekron, \v{44}Eltekeh, Gibbethon, Baalath, \v{45}Jehud, Bene-berak, Gath-rimmon, \v{46}Me-jarkon, and Rakkon at the border near Joppa. \v{47}Later, when the descendants of Dan lost their territory, they went up and attacked Leshem. After they captured it and executed its inhabitants, they took possession of it and settled there, renaming the city of Leshem to Dan in memory of their ancestor Dan. \v{48}These towns and villages comprised the inheritance of the tribe of Dan according to their families.
\passage{Joshua's Allocation}

\v{49}When the Israelis had completed distribution of the various territories of the land as inheritances, they provided an inheritance to Nun's son Joshua. \v{50}By a command from the \divine{Lord}, they allocated the town that he requested, Timnath-serah in the hill country of Ephraim. He rebuilt the town and settled there. \v{51}These are the inheritances that Eleazar the priest, Nun's son Joshua, and the heads of the families of the Israeli tribes distributed by lot in the \divine{Lord}'s presence at the entrance to the Tent of Meeting. So they finished dividing the land.
\labelchapt{20}
\passage{The Cities of Refuge}
\passageinfo{(Numbers 35:9-28; Deuteronomy 19:1-13)}

\chapt{20}
\v{1}Then the \divine{Lord} told Joshua, \v{2}``Tell the people of Israel to set apart cities of refuge about which I spoke to you through Moses, \v{3}so that anyone who kills a person unintentionally and without premeditation may run there and be protected from closely related\fnote{\fbackref{20:3} Lit. \fbib{from blood}} avengers. \v{4}He may run to one of those cities, stand at the entrance to the city gate, and tell his side of the story to the elders of the city. They are to bring him inside the city with them and provide him a place to live among them. \v{5}Now if the closely related\fnote{\fbackref{20:5} Lit. \fbib{the blood}} avenger pursues him, then they are not to hand the killer over to him, because he killed his neighbor without premeditation and without hating him beforehand. \v{6}He is to live in that city until he stands trial before the community, until the death of the one who is high priest at that time. Then the killer may return to his own city and to his own home, that is, to the city from which he fled.''

\v{7}So they set apart Kedesh in Galilee in the hill country of Naphtali, Shechem in the hill country of Ephraim, and Kiriath-arba (also known as Hebron) in the hill country of Judah. \v{8}East of Jericho beyond the Jordan River, they reserved Bezer in the wilderness on the plain from the tribe of Reuben, Ramoth in Gilead from the tribe of Gad, and Golan in Bashan from the tribe of Manasseh. \v{9}These were appointed to be cities for all the Israelis and for the foreigner who lives among them, so that whoever kills anyone unintentionally may run there and not die at the hands of a closely related\fnote{\fbackref{20:9} Lit. \fbib{a blood}} avenger until he stands for trial before the community.
\labelchapt{21}
\passage{Reservation of the Levitical Cities}

\chapt{21}
\v{1}Then the family leaders of the descendants of Levi approached Eleazar the priest and Nun's son Joshua, along with the family leaders of the people of Israel. \v{2}It was at Shiloh in the land of Canaan that they told them, ``The \divine{Lord} ordered through Moses that we be given cities in which to live, along with their pasture lands for our livestock.''
\passage{Allocation for the Descendants of Kohath and Descendants of Gershon}

\v{3}So, just as the Lord had said, the Israelis set aside cities for the descendants of Levi from a portion of their own inheritances, along with their grazing lands. \v{4}The lottery went to the families of the descendants of Kohath. So the descendants of Aaron the priest, who were descendants of Levi, received thirteen cities by random lot from the tribes of Judah, Simeon, and Benjamin. \v{5}The rest of the descendants of Kohath received ten cities by random lot from the families of the tribes of Ephraim, Dan, and the half-tribe of Manasseh.

\v{6}The descendants of Gershon received thirteen cities by random lot from the families of the tribes of Issachar, Asher, Naphtali, and from the half-tribe of Manasseh located in Bashan. \v{7}The descendants of Merari, allocated according to their families, received twelve cities from the tribes of Reuben, Gad, and Zebulun.

\v{8}The Israelis apportioned these cities, along with their pasture lands, to the descendants of Levi by random lot, just as the \divine{Lord} had commanded through Moses.

\v{9}From the tribes of the descendants of Judah and Simeon, they gave these cities, delineated by name: \v{10}for the descendants of Aaron, one of the families of the descendants of Kohath, of the descendants of Levi, since the lot fell in their favor first. \v{11}They gave them Kiriath-arba, also known as Hebron, (Arba was named after\fnote{\fbackref{21:11} The Heb. lacks \fbib{was named after}} the ancestor of Anak), in the hill country of Judah, along with its surrounding pasture lands. \v{12}But the fields adjoining the city and its surrounding villages were given to Jephunneh's son Caleb.

\v{13}So they gave Hebron to the descendants of Aaron the priest to serve as a city of refuge for unintentional killers, along with its pasture lands, Libnah with its pasture lands, \v{14}Jattir with its pasture lands, Eshtemoa with its pasture lands, \v{15}Holon with its pasture lands, Debir with its pasture lands, \v{16}Ain with its pasture lands, Juttah with its pasture lands, and Beth-shemesh with its pasture lands, for a total of nine cities from these two tribes.

\v{17}From the tribe of Benjamin, Gibeon with its pasture lands, Geba with its pasture lands, \v{18}Anathoth with its pasture lands, and Almon with its pasture lands, for a total of four cities. \v{19}All of the cities allocated\fnote{\fbackref{21:19} The Heb. lacks \fbib{allocated}} to the priests, who were descendants of Aaron, numbered thirteen, along with their pasture lands.

\v{20}Cities from the tribe of Ephraim were also allotted to the families of the descendants of Kohath, that is, to the rest of the descendants of Kohath, who were descendants of Levi. \v{21}Shechem was allocated to them as a city of refuge for unintentional killers, along with its pasture lands, in the mountainous region\fnote{\fbackref{21:21} Or \fbib{the hill country}} of Ephraim, Gezer with its pasture lands, \v{22}Kibzaim with its pasture lands, and Beth-horon with its pasture lands, for a total of four cities.

\v{23}From the tribe of Dan were allocated\fnote{\fbackref{21:23} The Heb. lacks \fbib{were allocated}} Elteke with its pasture lands, Gibbethon with its pasture lands, \v{24}Aijalon with its pasture lands, and Gath-rimmon with its pasture lands, for a total of four cities.

\v{25}From the half-tribe of Manasseh were allocated Taanach with its pasture lands and Gath-rimmon with its pasture lands, for a total of two cities. \v{26}All the cities with their pasture lands for the families of the rest of the descendants of Kohath numbered ten.

\v{27}To the descendants of Gershon (one of the Levitical families) from the half-tribe of Manasseh were allocated\fnote{\fbackref{21:27} The Heb. lacks \fbib{were allocated}} Golan in Bashan as a city of refuge for unintentional killers, along with its pasture lands, and Beeshterah with its pasture lands, for a total of two cities.

\v{28}From the tribe of Issachar were allocated\fnote{\fbackref{21:28} The Heb. lacks \fbib{were allocated}} Kishion with its pasture lands, Daberath with its pasture lands, \v{29}Jarmuth with its pasture lands, and En-gannim with its pasture lands, for a total of four cities.

\v{30}From the tribe of Asher were allocated\fnote{\fbackref{21:30} The Heb. lacks \fbib{were allocated}} Mishal with its pasture lands, Abdon with its pasture lands, \v{31}Helkath with its pasture lands, and Rehob with its pasture lands, for a total of four cities.

\v{32}From the tribe of Naphtali, Kedesh in Galilee with its pasture lands were allocated\fnote{\fbackref{21:32} The Heb. lacks \fbib{were allocated}} as a city of refuge for the unintentional killer, Hammoth-dor with its pasture lands, and Kartan with its pasture lands, for a total of three cities.

\v{33}All the cities of the descendants of Gershon according to their families totaled thirteen, including their pasture lands.
\passage{Allocation for the Descendants of Merari}

\v{34}From the tribe of Zebulun were allocated\fnote{\fbackref{21:34} The Heb. lacks \fbib{were allocated}} to the descendants of Merari (that is, the rest of the descendants of Levi) Jokneam with its pasture lands, Kartah with its pasture lands, \v{35}Dimnah with its pasture lands, and Nahalal with its pasture lands, for a total of four cities.

\v{36}\fnote{\fbackref{21:36} vv. 36-37 are usu. included in MT as a quotation from 1Chr 6:63-64}From the tribe of Reuben were allocated\fnote{\fbackref{21:36} The Heb. lacks \fbib{were allocated}} Bezer with its pasture lands, Jahaz with its pasture lands, \v{37}Kedemoth with its pasture lands, and Mephaath with its pasture lands, for a total of four cities.

\v{38}From the tribe of Gad were allocated\fnote{\fbackref{21:38} The Heb. lacks \fbib{were allocated}} Ramoth in Gilead with its pasture lands, to serve as a city of refuge for the unintentional killer, Mahanaim with its pasture lands, \v{39}Heshbon with its pasture lands, and Jazer with its pasture lands, for a total of four cities in all.

\v{40}So the entire allocation to the descendants of Merari (that is, the rest of the families of the descendants of Levi) according to their families totaled twelve cities.
\passage{Summary of Allocations to the Descendants of Levi}

\v{41}All of the cities of the descendants of Levi that had been set apart\fnote{\fbackref{21:41} The Heb. lacks \fbib{that had been set apart}} among the territorial\fnote{\fbackref{21:41} The Heb. lacks \fbib{territorial}} possession of the Israelis totaled 48, along with their pasture lands. \v{42}These cities were each surrounded by pasture lands, as was the case with all of these cities. \v{43}So the \divine{Lord} gave all of the land to Israel that he had promised to give their ancestors, and they took possession and settled there in it. \v{44}The \divine{Lord} gave them peace\fnote{\fbackref{21:44} Lit. \fbib{rest}} all around them, just as he had promised their ancestors. Not one of their enemies was able to oppose them---the \divine{Lord} placed all of their enemies under their control. \v{45}Not one of the good promises that the \divine{Lord} had made to the house of Israel failed---all of them came about.\fnote{\fbackref{21:45} The Heb. lacks \fbib{came about}}
\labelchapt{22}
\passage{The Tribes East of the Jordan}

\chapt{22}
\v{1}Later, Joshua called together the descendants of Reuben, the descendants of Gad, and the half-tribe of Manasseh \v{2}and told them, ``You have done everything that Moses the servant of the \divine{Lord} commanded you, and you have listened to me in everything that I commanded you. \v{3}You haven't abandoned your relatives these past days to the present, and you have met the obligation contained in\fnote{\fbackref{22:3} The Heb. lacks \fbib{contained in}} the commands of the \divine{Lord} your God. \v{4}Now the \divine{Lord} has given peace\fnote{\fbackref{22:4} Lit. \fbib{rest}} to your relatives, just as he told them. Therefore, proceed to your tents---to the land that is yours to possess---that Moses the servant of the \divine{Lord} gave you east of\fnote{\fbackref{22:4} Lit. \fbib{you beyond}} the Jordan River. \v{5}Only be very careful to keep the commands and the Law that Moses the servant of the \divine{Lord} commanded you---that is,\fnote{\fbackref{22:5} The Heb. lacks \fbib{that is}} to love the \divine{Lord} your God, to follow in all of his ways, to keep his commands, to stay close to him, and to serve him with all your heart and soul.'' \v{6}That's how Joshua blessed them. Then he sent them on their way and they returned to their tents.

\v{7}Now Moses had made an allotment in Bashan to one half-tribe of Manasseh, but Joshua made an allotment west of the Jordan River to the other half-tribe of their relatives. So when Joshua sent them on their way back to their tents, he also blessed them by saying \v{8}``Return to your tents with great wealth, plenty of livestock, silver, gold, bronze, iron, and lots of clothing. Divide the spoil from your enemies among your relatives.''

\v{9}The descendants of Reuben, the descendants of Gad, and the half-tribe of Manasseh went back to the land of Gilead, leaving the Israelis at Shiloh in the land of Canaan, for their territorial possession that they had inherited in accordance with the command of the \divine{Lord} given through Moses.
\passage{An Unauthorized Altar is Constructed}

\v{10}After they arrived at an area of the Jordan River that is in the land of Canaan, the descendants of Reuben, the descendants of Gad, and the half-tribe of Manasseh constructed an altar there by the Jordan River, and it was very large. \v{11}When the Israelis heard about it, they announced, ``Look here, the descendants of Reuben, the descendants of Gad, and the half-tribe of Manasseh have constructed an altar in Canaan's frontier district of the Jordan River, on the side apportioned to the Israelis.'' \v{12}When the Israelis heard that announcement,\fnote{\fbackref{22:12} The Heb. lacks \fbib{that announcement}} the entire community of the Israelis gathered together at Shiloh in preparation for war.

\v{13}Then the Israelis sent a delegation\fnote{\fbackref{22:13} The Heb. lacks \fbib{a delegation}} to the descendants of Reuben, the descendants of Gad, and the half-tribe of Manasseh in the land of Gilead. They sent\fnote{\fbackref{22:13} The Heb. lacks \fbib{They sent}} Eleazar's son Phinehas the priest, \v{14}and ten officials with him (one for each of the tribal families of Israel, each one of them a family leader among the tribes\fnote{\fbackref{22:14} Lit. \fbib{thousands}} of Israel). \v{15}They approached the descendants of Reuben, the descendants of Gad, and the half-tribe of Manasseh in the land of Gilead and told them: \v{16}``This is what the entire community of the \divine{Lord} has to say: `What is this treacherous act by which you have acted deceitfully against the God of Israel by turning away from following the \divine{Lord} today, and by building yourselves an altar today, so you can rebel against the \divine{Lord}? \v{17}Isn't the evil that happened at Peor enough for us, from which we have yet to be completely cleansed even to this point,\fnote{\fbackref{22:17} Lit. \fbib{day}} and because of which a plague came upon the community of the \divine{Lord}? \v{18}Now then, are you turning away from following the \divine{Lord} today? If you rebel against the \divine{Lord} today, by tomorrow he will be angry with the entire community of Israel. \v{19}If the land of your inheritance remains unclean, then cross back over into the land that the \divine{Lord} possesses, and receive an inheritance among us. Don't rebel against the \divine{Lord} and against us by constructing an altar for yourselves besides the altar of the \divine{Lord} our God. \v{20}Didn't Zerah's son Achan act treacherously with respect to the things banned by God,\fnote{\fbackref{22:20} The Heb. lacks \fbib{by God}} and as a result God became angry at\fnote{\fbackref{22:20} Lit. \fbib{result anger fell on}} the entire community of Israel? And that man was not the only one to die because of his iniquity.'\,''

\v{21}The descendants of Reuben, descendants of Gad, and the half-tribe of Manasseh answered the officials of the tribes\fnote{\fbackref{22:21} Lit. \fbib{thousands}} of Israel, \v{22}``The God of gods, the \divine{Lord}, the God of gods, the \divine{Lord} is the one who knows! And may Israel itself be aware that if this\fnote{\fbackref{22:22} The Heb. lacks \fbib{this}} was an act of rebellion or an act of treachery against the \divine{Lord}, may he not deliver us today! \v{23}If we have built an altar for ourselves intended to turn us away from following the \divine{Lord}, or to offer burnt offerings, grain offerings, or peace offerings on it, may the \divine{Lord} himself demand an accounting from us!\fnote{\fbackref{22:23} The Heb. lacks \fbib{an accounting from us}} \v{24}But we did this because we were concerned for a reason, since we thought, `Sometime in the future your descendants may say to our descendants, ``What do you have in common\fnote{\fbackref{22:24} The Heb. lacks \fbib{have in common}} with the \divine{Lord}, the God of Israel? \v{25}The \divine{Lord} has established the Jordan River to be a territorial border between us and you. descendants of Reuben and descendants of Gad have no allotment from the \divine{Lord}.'' So your descendants may cause our descendants to stop fearing the \divine{Lord}.'

\v{26}``That's why we said, `Let's build an altar for ourselves, not for burnt offerings or sacrifice, \v{27}but instead it will serve as a reminder\fnote{\fbackref{22:27} Or \fbib{witness}} between us and you and between our generations after us, that we are to serve the \divine{Lord} with our burnt offerings, sacrifices, and peace offerings. That way your descendants will not say to our descendants in the future, ``You have no allotment from the \divine{Lord}.''\,'

\v{28}``That's also why we said, `It may be if they say these things\fnote{\fbackref{22:28} The Heb. lacks \fbib{these things}} to us and to our descendants in the future, so we will respond, ``Look at this replica of the altar of the \divine{Lord} that our ancestors made, not for burnt offerings or sacrifice, but rather as a reminder\fnote{\fbackref{22:28} Or \fbib{witness}} between us and you. \v{29}May we never rebel against the \divine{Lord} today by building an altar for burnt offerings, for grain offerings, or for sacrifice to replace\fnote{\fbackref{22:29} Or \fbib{sacrifice besides}} the altar of the \divine{Lord} our God which stands before his Tent.''\,'\,''

\v{30}When Phinehas the priest, the leaders of the community, and the heads of the families of Israel who were with him heard what the descendants of Reuben, the descendants of Gad, and the descendants of Manasseh said, they were pleased. \v{31}So Eleazar's son Phinehas the priest replied to the descendants of Reuben, the descendants of Gad, and the descendants of Manasseh, ``Today we've demonstrated\fnote{\fbackref{22:31} Lit. \fbib{known}} that the \divine{Lord} is among us, because you have not acted treacherously against the \divine{Lord}. Now you have delivered the Israelis from the anger\fnote{\fbackref{22:31} Lit. \fbib{hand}} of the \divine{Lord}.''

\v{32}So Eleazar's son Phinehas the priest and the leaders returned from the descendants of Reuben, the descendants of Gad, and from the land of Gilead to the land of Canaan and to the people of Israel, bringing back word to them. \v{33}What they said pleased the people of Israel, so they\fnote{\fbackref{22:33} Lit. \fbib{the Israelis}} blessed God and said no more about going up to attack them in war and to destroy the land where the descendants of Reuben and the descendants of Gad were living. \v{34}The descendants of Reuben and the descendants of Gad named the altar ``Witness,'' because they claimed,\fnote{\fbackref{22:34} The Heb. lacks \fbib{they claimed}} ``It stands as a witness between us that the \divine{Lord} is God.''
\labelchapt{23}
\passage{Joshua's Retirement Address to Israel}

\chapt{23}
\v{1}Many years later, after the \divine{Lord} had given peace\fnote{\fbackref{23:1} Lit. \fbib{rest}} between Israel and all its surrounding enemies, and after Joshua had become very old, \v{2}Joshua called together all Israel, including their leaders, officials, judges, and tribal officers. He told them, ``I am old now after having lived many years. \v{3}You have seen everything that the \divine{Lord} your God has done to all of these nations on your behalf, because it has been the \divine{Lord} your God who has been fighting on your behalf. \v{4}Now look, I have allocated these nations that remain as an inheritance for your tribes, including all of the nations that I have eliminated, from the Jordan River to the Mediterranean\fnote{\fbackref{23:4} Lit. \fbib{Great}} Sea to the west.\fnote{\fbackref{23:4} Lit. \fbib{Sea that faces the setting sun}} \v{5}The \divine{Lord} your God will expel them in front of you, driving them out of your sight. You will take possession of this land, just as the \divine{Lord} your God promised you. \v{6}Stand very strong, then, so you can obey and carry out everything written in the Book of the Law of Moses, turning neither to the right nor to the left of it. \v{7}That way, you will not mingle with those nations that remain among you, nor mention the name of their gods, nor make oaths by them,\fnote{\fbackref{23:7} The Heb. lacks \fbib{by them}} nor serve them, nor worship them. \v{8}Instead, you are to hold fast to the \divine{Lord} your God, as you have done today, \v{9}because the \divine{Lord} has expelled great and strong nations ahead of you. Now as for you, not a single man has been able to oppose you right to this day. \v{10}A single man makes a thousand flee, because the \divine{Lord} your God is the one who is fighting for you, just as he promised you.

\v{11}``So be very diligent to love the \divine{Lord} your God, \v{12}because if you ever turn back and cling to those who remain of these nations by intermarrying with them and associating one with another, \v{13}know for certain that the \divine{Lord} your God will not continue to drive out these nations ahead of you. Instead, they will be a snare and a trap for you, a whip to your backs, and thorns in your eyes, until you perish from this good land that the \divine{Lord} your God has given you.

\v{14}``Look here: today I'm going down the path that everyone on earth takes, and you know with all your hearts and souls that not a single word of all of the good things that the \divine{Lord} your God spoke about you has failed to happen. Everything has been fulfilled about you---not one of them has failed. \v{15}However, just as all of the good things have come about that the \divine{Lord} your God promised, so also the \divine{Lord} will bring upon you all of the threats until he has destroyed you from possessing this good land that he\fnote{\fbackref{23:15} Lit. \fbib{the \divine{Lord} your God}} has given you. \v{16}When you break the covenant of the \divine{Lord} your God that he commanded you to obey by going to serve other gods and worship them, then the anger of the \divine{Lord} will blaze against you, and you will perish quickly from this good land that he gave you.''
\labelchapt{24}
\passage{Joshua's Final Exhortation}

\chapt{24}
\v{1}Then Joshua assembled together all of the tribes of Israel at Shechem. He called for the leaders, officials, judges, and tribal officers of Israel. They assembled in formation before God, \v{2}and Joshua told all of the people, ``This is what the \divine{Lord} God of Israel has to say:

\begin{poetry}
\poeml `Long ago your ancestors lived beyond the Euphrates\fnote{\fbackref{24:2} The Heb. lacks \fbib{Euphrates}} River, including Terah, father of both Abraham and Nahor, where they served other gods. \v{3}Then I took your ancestor Abraham from the other side of the Euphrates\fnote{\fbackref{24:3} The Heb. lacks \fbib{Euphrates}} River and led him through the entire land of Canaan. I multiplied his descendants, and gave him his son\fnote{\fbackref{24:3} The Heb. lacks \fbib{his son}} Isaac. \v{4}I gave Jacob and Esau to Isaac. And I gave Mount Seir\fnote{\fbackref{24:4} This mountain, the modern \fbib{Jebel esh-sher\'{a}}, is located in the mountain range that extends south of the Dead Sea toward the Gulf of Aqaba, and is bordered by the Arabah Valley to the west.} to Esau as his possession, but Jacob and his children went down to Egypt. \\
\poeml \v{5}`Later I commissioned Moses and Aaron, and I inflicted plagues on Egypt by what I did among them. Afterwards, I brought all of you\fnote{\fbackref{24:5} Lit. \fbib{brought you} (pl.)} out. \\
\poeml \v{6}`Then I brought your ancestors out of Egypt, and you came to the Sea, and the Egyptians followed your ancestors with chariots and horsemen to the Reed\fnote{\fbackref{24:6} So MT; LXX reads \fbib{Red}} Sea. \v{7}But when they cried out to the \divine{Lord}, he placed darkness between you and the Egyptians, brought the sea upon the Egyptians,\fnote{\fbackref{24:7} Lit. \fbib{upon them}} and swallowed them up. Your own eyes saw what I did in Egypt. Then you lived in the desert for a long time. \\
\poeml \v{8}`I brought you into the territory of the Amorites, who lived on the other side of the Jordan River. They fought you, but I gave them into your control, and you took possession of their land. I destroyed them from your presence. \\
\poeml \v{9}`Then Zippor's son, King Balak of Moab, showed up and fought against Israel. He sent word\fnote{\fbackref{24:9} The Heb. lacks \fbib{word}} to Balaam, summoning Beor's son to put a curse on you. \v{10}But I wasn't willing to listen to Balaam. So he had to bless you, and I delivered you from his control. \\
\poeml \v{11}`Next, you crossed the Jordan River and arrived at Jericho. But the citizens of Jericho fought you, as did the Amorites, Perizzites, Canaanites, Hittites, Girgashites, Hivites, and the Jebusites, so I gave them into your control. \\
\poeml \v{12}`Then I sent hornets ahead of you to drive out two kings of the Amorites before you without your using either sword or bow. \v{13}I gave you a land for which you never worked and cities that you didn't build, but that you have lived in. You're eating from vineyards and olive groves that you didn't plant.'
\end{poetry}

\v{14}``Now you must fear the \divine{Lord} and serve him in faithfulness and truth. Throw away the gods that your ancestors served beyond the Euphrates\fnote{\fbackref{24:14} The Heb. lacks \fbib{Euphrates}} River and in Egypt. Instead, serve the \divine{Lord}. \v{15}If you think it's the wrong thing for you to serve the \divine{Lord}, then choose for yourselves today whom you will serve---the gods whom your ancestors served on the other side of the Euphrates\fnote{\fbackref{24:15} The Heb. lacks \fbib{Euphrates}} River, or the gods of the Amorites in whose territories you are living. But as for me and my household, we will serve the \divine{Lord}.''
\passage{The Response of the People}

\v{16}In response, the people said, ``Far be it from us that we should abandon the \divine{Lord} to serve other gods, \v{17}since the \divine{Lord} our God is the one who brought us and our ancestors up from the land of Egypt, from a life of slavery. He did those great things right in front of us, preserving us along the way that we traveled and among all the peoples through whose territory we passed. \v{18}The \divine{Lord} expelled all the people before us, including the Amorites who lived in the land. Therefore, we also will serve the \divine{Lord}, since he is our God.''

\v{19}So Joshua told the people, ``You will not be able to serve the \divine{Lord}, because he is a God of Holiness. He is a jealous God, and he will forgive neither your transgressions nor your sins. \v{20}If you abandon the \divine{Lord} and serve foreign deities, then he will turn and do you harm, consuming you after all\fnote{\fbackref{24:20} The Heb. lacks \fbib{all}} the good he has done for you.''

\v{21}``No,'' the people replied to Joshua. ``We will serve the \divine{Lord}.''

\v{22}Joshua responded, ``You are giving testimony against yourselves, that you have chosen to serve the \divine{Lord}.''

They replied, ``We are witnesses!''

\v{23}Joshua said,\fnote{\fbackref{24:23} The Heb. lacks \fbib{Josh said}} ``Therefore abandon the foreign gods that are among you, and turn your hearts to the \divine{Lord}, the God of Israel.''

\v{24}The people replied,\fnote{\fbackref{24:24} Lit. \fbib{replied to Josh}} ``We will serve the \divine{Lord} our God and obey his voice.''

\v{25}So Joshua made a covenant with the people that day, making statutes and ordinances in Shechem. \v{26}He\fnote{\fbackref{24:26} Lit. \fbib{Josh}} wrote these words in the Book of the Law of God, took a large stone, moved it under the shade of\fnote{\fbackref{24:26} The Heb. lacks \fbib{the shade of}} the oak tree that was near the sanctuary of the \divine{Lord}, \v{27}and then\fnote{\fbackref{24:27} Lit. \fbib{Josh}} told all of the people, ``Look! This stone will testify for us, because it has heard everything that the \divine{Lord} has spoken to us. So it will stand as a witness against you in the event that you deny your God.'' \v{28}Then Joshua dismissed the people, and each man returned\fnote{\fbackref{24:28} The Heb. lacks \fbib{returned}} to his territorial inheritance.
\passage{The Death of Joshua}
\passageinfo{(Judges 2:6-9)}

\v{29}Some time later, Joshua servant of the \divine{Lord} died at the age of 110 years, and \v{30}they buried him in his territorial inheritance at Timnath-serah in the mountainous region\fnote{\fbackref{24:30} Or \fbib{the hill country}} of Ephraim, north of Mount Gaash. \v{31}Israel served the \divine{Lord} for the entire lifetimes of Joshua and of the officials who outlived Joshua, that is, the ones who had personally known everything that the \divine{Lord} had done for Israel. \v{32}They also buried the bones of Joseph, which the Israelis brought up from Egypt, in the parcel of ground at Shechem that Jacob had purchased from the descendants of Shechem's father Hamor, for 100 pieces of silver. It became part of the inheritance of the descendants of Joseph.
\passage{The Death of Eleazar the Priest}

\v{33}Later, Aaron's son Eleazar also died, and they buried him at Gibeah, which belonged to his son Phinehas, and which had been given to him in the mountainous region\fnote{\fbackref{24:33} Or \fbib{the hill country}} of Ephraim.

\bookheader{Judges}
\labelbook{Judg}

\bookpretitle{The Book of}
\booktitle{Judges}

\labelchapt{1}
\passage{The Capture of Jerusalem}

\chapt{1}
\v{1}Sometime after Joshua had died, the Israelis asked the \divine{Lord}, ``Who is to lead\fnote{Lit. \fbib{to go up for}} us against the Canaanites in our opening attack against them?''

\v{2}The \divine{Lord} replied, ``The tribe of\fnote{The Heb. lacks \fbib{the tribe of}; and so throughout the chapter} Judah is to lead you.\fnote{Lit. \fbib{to go up}} Look! I've given the land into their control.''

\v{3}But the tribe of Judah told the tribe of Simeon, the descendants of Judah's\fnote{Lit. \fbib{Simeon, his}} brother, ``Come with us\fnote{Lit. \fbib{him}} into our territory, and we'll both fight the Canaanites. In return, we'll\fnote{Lit. \fbib{I'll}} go with you when you fight in your territory.'' So the army of\fnote{The Heb. lacks \fbib{the army of}; and so throughout the chapter} the tribe of Simeon accompanied the army of the tribe of Judah.

\v{4}When the army of the tribe of Judah went into battle, the \divine{Lord} gave the Canaanites and the Perizzites into their control, and they defeated 10,000 men at Bezek. \v{5}They located Adoni-bezek in Bezek, fought him, and defeated the Canaanites and the Perizzites. \v{6}Adoni-bezek ran off, but they pursued him, caught him, and amputated his thumbs and big toes. \v{7}Adoni-bezek used to brag, ``Seventy kings without thumbs and big toes used to eat what was left under my table. God has repaid me for what I've done.'' They brought him to Jerusalem, and he later died there.

\v{8}Then the army of Judah attacked Jerusalem, captured it, executed its inhabitants, and set fire to the city. \v{9}Later, the army of Judah left Jerusalem\fnote{Lit. \fbib{Judah went down}} to attack the Canaanites who lived in the hill country, the Negev,\fnote{I.e. the southern regions of the Sinai peninsula; cf. Josh 10:40} and the Shephelah.\fnote{I.e. the verdant central lowlands of Israel, and so throughout the book; cf. Josh 10:40} \v{10}They\fnote{Lit. \fbib{Judah}} attacked the Canaanites who inhabited Hebron (formerly known as Kiriath-arba) and fought Sheshai, Ahiman, and Talmai.
\passage{The Capture of Debir}
\passageinfo{(Joshua 15:13-19)}

\v{11}The army of Judah then proceeded to attack the inhabitants of Debir, which used to be known as Kiriath-sepher. \v{12}Caleb announced, ``I'll give my daughter Achsah in marriage to whomever leads the attack against Kiriath-sepher and captures it.'' \v{13}Othniel, Caleb's nephew through his younger brother Kenaz, captured the city, so Caleb\fnote{Lit. \fbib{he}} awarded him his daughter Achsah in marriage.

\v{14}Later on, after she had arrived, she urged Othniel\fnote{Lit. \fbib{him}} to ask her father for a field. As she got off her donkey, Caleb asked her, ``What do you want\fnote{The Heb. lacks \fbib{do you want}} for yourself?''

\v{15}``I want this blessing from you,'' she replied. ``Since you've given me land in the Negev,\fnote{I.e. the southern regions of the Sinai peninsula; cf. Josh 10:40} give me water springs, too.'' So Caleb gave her both the upper and lower springs.
\passage{The Capture of Certain Southern Territories}

\v{16}The descendants of the Kenites, the tribe from which\fnote{The Heb. lacks \fbib{the tribe from which}} Moses' father-in-law came, accompanied the descendants of Judah from the city of the palms to the Judean wilderness, which is in the desert area south of Arad, and lived with the people there. \v{17}The army of Judah accompanied the army of Simeon, Judah's\fnote{Lit. \fbib{his}} brother, as they attacked the Canaanites who were living in Zephath, and they completely destroyed it. Then they renamed the city Hormah. \v{18}The army of Judah captured Gaza and its territory, Ashkelon and its territory, and Ekron and its territory. \v{19}The \divine{Lord} was with the army of Judah, and they captured the hill country, but did not expel the inhabitants of the valley because they were equipped with iron chariots.
\passage{Hebron Awarded to Caleb}
\passageinfo{(Joshua 14:6-15; 15:63)}

\v{20}They gave Hebron to Caleb, just as Moses had promised,\fnote{Cf. Josh 14:9} and he drove out the three sons of Anak from there. \v{21}However, the descendants of Benjamin did not expel the Jebusites who lived in Jerusalem, so the Jebusites have lived with the descendants of Benjamin in Jerusalem to this day.
\passage{The Capture of Bethel}

\v{22}Then the army of the tribe\fnote{Lit. \fbib{house}} of Joseph attacked Bethel, and the \divine{Lord} was with them. \v{23}The army of the tribe of Joseph scouted out Bethel, which had been formerly named Luz. \v{24}The scouts observed a man coming out of the city and they promised him, ``Please show us the entrance to the city and we'll deal kindly with you.'' \v{25}So he showed them the entrance to the city, and they attacked the city with swords, but they let the man and his entire family escape. \v{26}So the man traveled to the land of the Hittites and built a city that he named ``Luz,'' and it is called by that name to this day.
\passage{Unconquered Territories}

\v{27}The army of the tribe of Manasseh did not conquer Beth-shean and its villages, Taanach and its villages, the inhabitants of Dor and its villages, the inhabitants of Ibleam and its villages, and the inhabitants of Megiddo and its villages. Instead, the Canaanites continued to live in that land. \v{28}When Israel had grown strong, they subjected the Canaanites to conscripted labor and never did expel them completely.

\v{29}The army of the tribe of Ephraim did not expel the Canaanites who were living in Gezer, so the Canaanites lived in Gezer among them.

\v{30}The army of the tribe of Zebulun did not expel the inhabitants of Kitron or the inhabitants of Nahalol, so the Canaanites lived among them, but were subjected to conscripted labor.

\v{31}The army of the tribe of Asher did not expel the inhabitants of Acco nor the inhabitants of Sidon, Ahlab, Achzib, Helbah, Aphik, or Rehob. \v{32}So the descendants of Asher lived among the Canaanites who continued to inhabit the land, because they did not expel them.

\v{33}The army of the tribe of Naphtali did not expel the inhabitants of Beth-shemesh and the inhabitants of Beth-anath. Instead, they lived among the Canaanites who inhabited the land. However, the inhabitants of Beth-shemesh and Beth-anath were subjected to conscripted labor.

\v{34}Later on, the Amorites forced the descendants of Dan into the hill country and did not permit them to come into the valleys of the hills. \v{35}Furthermore, the Amorites continued to inhabit Mount Heres in Aijalon and Shaalbim. Eventually, however, after the tribe\fnote{Lit. \fbib{house}} of Joseph had become strong, the Amorites\fnote{Lit. \fbib{they}} were subjected to conscripted labor. \v{36}The Amorite border extended upward from the Akrabbim Ascent, that is, from Sela.
\labelchapt{2}
\passage{Israel is Rebuked}

\chapt{2}
\v{1}Some time later, the angel of the \divine{Lord} came up from Gilgal to Bochim and announced to Israel,\fnote{The Heb. lacks \fbib{to Israel}} ``I brought you up from Egypt and led you into the land that I promised to your ancestors. I had told them,\fnote{The Heb. lacks \fbib{to them}} `I'll never breach my covenant with you. \v{2}As for you, you must not make any treaties\fnote{Or \fbib{covenants}} with the inhabitants of this land. Instead, tear down their altars.' But you haven't obeyed me. What have you done? \v{3}Therefore I'm now saying,\fnote{Lit. \fbib{I also said}} `I won't expel them before you. Instead, they'll remain at your side, and their gods will ensnare you.'\,''

\v{4}Because the angel of the \divine{Lord} said these things to all of the Israelis, the people wept out loud, \v{5}which is why they named the place Bochim.\fnote{MT \fbib{Bochim} means \fbib{weeping}} And there they sacrificed to the \divine{Lord}. \v{6}After Joshua had dismissed the people, the Israelis returned to their respective inheritances to take possession of the land.
\passage{The Death of Joshua}
\passageinfo{(Joshua 24:29-31)}

\v{7}The people served the \divine{Lord} during the entire lifetime of Joshua as well as the lifetimes of all the elders who outlived Joshua and who had observed all the great deeds that the \divine{Lord} had done for Israel. \v{8}But then Nun's son Joshua, the servant of the \divine{Lord}, died at the age of 110 years. \v{9}They buried him in Timnath-heres, within the boundaries of his inheritance in the mountainous region\fnote{Or \fbib{the hill country}} of Ephraim, north of Mount Gaash. \v{10}After that whole generation had died,\fnote{Lit. \fbib{had been gathered to their fathers}} another generation grew up after them that was not acquainted with the \divine{Lord} or with what he had done for Israel.
\passage{The Rise of the Judges}

\v{11}So the Israelis practiced what the \divine{Lord} considered to be evil by worshiping Canaanite deities.\fnote{Lit. \fbib{worshiping the Baals}} \v{12}They abandoned the \divine{Lord} God of their ancestors, who had brought them out of the land of Egypt. They followed other gods from among the gods of the peoples who surrounded them. They bowed down in worship of them, and by doing so angered the \divine{Lord}. \v{13}As a result, they abandoned the \divine{Lord} by serving both Baal\fnote{I.e. the supreme male deity of the Canaanites} and Ashtaroth.\fnote{I.e. various female deities of the Canaanites} \v{14}So in his burning anger against Israel, the \divine{Lord} gave them into the domination of marauders who plundered them. The enemies who surrounded the Israelis\fnote{Lit. \fbib{them}} controlled them, and they were no longer able to withstand their adversaries. \v{15}Wherever they went, the \divine{Lord} worked\fnote{Lit. \fbib{the hand of the \divine{Lord} was}} against them to bring misfortune, just as the \divine{Lord} had warned, and just as the \divine{Lord} had promised them. As a result, they suffered greatly.

\v{16}Then the \divine{Lord} raised up leaders,\fnote{Or \fbib{judges}; and so throughout the chapter} who delivered Israel\fnote{Lit. \fbib{them}} from domination by their marauders. \v{17}But they didn't listen to their leaders, because they were committing spiritual immorality by following other gods and worshiping them. They quickly turned away from the road on which their ancestors had walked in obedience to the commands of the \divine{Lord}. They didn't follow their example. \v{18}As a result, whenever the \divine{Lord} raised up leaders for them, the \divine{Lord} remained present with their leader, delivering Israel\fnote{Lit. \fbib{them}} from the control of their enemies during the lifetime of that leader. The \divine{Lord}\fnote{Lit. \fbib{For he}} was moved with compassion by their groaning that had been caused by those who were oppressing and persecuting them. \v{19}However, after the leader had died, they would relapse to a condition more corrupt than their ancestors, following other gods, serving them, and worshiping them. They would not abandon their activities or their obstinate lifestyles.

\v{20}In his burning anger against Israel, the \divine{Lord} said, ``Because the people have transgressed my covenant that I commanded their ancestors to keep, and because they haven't obeyed me, \v{21}I'm also going to stop expelling any of the nations that remained after Joshua died. \v{22}That way, I'll use them to demonstrate whether or not Israel will keep the \divine{Lord}'s lifestyle by walking on that road like their ancestors did.'' \v{23}So the \divine{Lord} caused those nations to remain and did not expel them quickly. He did not give them into Joshua's control.
\labelchapt{3}
\passage{Unconquered Canaanite Nations}

\chapt{3}
\v{1}Here's a list of nations that the \divine{Lord} caused to remain in order to test Israel (that is,\fnote{The Heb. lacks \fbib{that is}} everyone who had not gained any battle experience in Canaan) \v{2}only so that successive Israeli generations, who had not known war previously, might come to know it by experience. \v{3}These nations included\fnote{The Heb. lacks \fbib{These nations included}} the five lords of the Philistines, all of the Canaanites, the Sidonians, and the Hivites who lived in Mount Baal-hermon as far as Lebo-hamath. \v{4}They remained there to test Israel, to reveal if they would obey the commands of the \divine{Lord} that he issued to their ancestors through Moses.
\passage{Othniel, Israel's First Judge}

\v{5}The Israelis continued to live among the Canaanites, the Hittites, the Amorites, the Perizzites, the Hivites, and the Jebusites, \v{6}taking their daughters as wives for themselves, giving their own daughters to their sons, and serving their gods. \v{7}The Israelis kept on practicing evil in full view of the \divine{Lord}. They forgot the \divine{Lord} their God and served Canaanite male and female deities.\fnote{Lit. \fbib{served the Baals and the Ashtaroth}} \v{8}Then in his burning anger against Israel, the \divine{Lord} delivered them to domination by King Cushan-rishathaim of Aram-naharaim.\fnote{Or \fbib{Aram of the Two Rivers}; i.e. Mesopotamia} So the Israelis served Cushan-rishathaim for eight years. \v{9}When the Israelis cried out to the \divine{Lord}, the \divine{Lord} raised up Othniel son of Caleb's younger brother Kenaz, to deliver\fnote{Lit. \fbib{to be a deliverer for}; or \fbib{to be a messiah}} them,\fnote{Lit. \fbib{deliver the Israelis}} and he did. \v{10}The Spirit of the \divine{Lord} was on him, and he governed Israel. When Othniel\fnote{Lit. \fbib{he}} went out to battle, the \divine{Lord} handed king Cushan-rishathaim of Aram-naharaim\fnote{Or \fbib{Aram of the Two Rivers}; i.e. Mesopotamia} into his control, and Othniel's\fnote{Lit. \fbib{his}} domination of Cushan-rishathaim was strong. \v{11}As a result, the land was quiet for 40 years. Then Kenaz' son Othniel died.
\passage{Ehud, Israel's Second Judge}

\v{12}The Israelis again practiced evil in full view of the \divine{Lord}. So the \divine{Lord} strengthened Eglon king of Moab in his control over Israel, because they had practiced evil in full view of the \divine{Lord}. \v{13}Eglon\fnote{Lit. \fbib{He}} assembled together the Ammonites and the Amalekites, proceeded to attack Israel, and captured the cities of palms. \v{14}So the Israelis served king Eglon of Moab for eighteen years.

\v{15}But when the Israelis cried out to the \divine{Lord}, the \divine{Lord} raised up Gera's son Ehud, a left-handed descendant of Benjamin, as a deliverer for them. The Israelis paid tribute through him to king Eglon of Moab. \v{16}Ehud forged a double-edged sword that was one cubit\fnote{I.e. about a foot and a half} long, tied it to his right thigh under his cloak, \v{17}and went to present the tribute to King Eglon of Moab. Now Eglon happened to be a very obese man.

\v{18}As he finished presenting the tribute, Ehud\fnote{Lit. \fbib{he}} sent away the people who had been carrying it. \v{19}He had turned away from the idols that were at Gilgal. So he told Eglon, ``I have a secret message for you, king.''

King Eglon\fnote{Lit. \fbib{So he}} responded ``Silence!'' and all of his attendants left him.

\v{20}Ehud approached him while he was sitting by himself in the cool roof chamber of his palace.\fnote{The Heb. lacks \fbib{of his palace}} He said, ``I have a message from God for you!'' So when Eglon\fnote{Lit. \fbib{he}} got up from his seat, \v{21}Ehud used his left hand to take the sword from his right thigh and then plunged it into Eglon's\fnote{Lit. \fbib{his}} abdomen. \v{22}The hilt also penetrated along with the sword blade, and Eglon's fat closed in over the blade. Because he did not withdraw the sword from Eglon's abdomen, the sword point\fnote{So LXX. MT reads \fbib{abdomen, it}} exited from Eglon's entrails.\fnote{Or \fbib{from behind}}

\v{23}Then Ehud left the cool chamber in the direction of the vestibule, shutting and locking the doors behind him. \v{24}After he left, Eglon's\fnote{Lit. \fbib{his}} attendants came to look, but the doors to the cool chamber were locked! So they said, ``He must be relieving himself\fnote{Lit. \fbib{be covering his feet}} in the inner part of the cool chamber.''\fnote{Or \fbib{cool area}; i.e. a private room (usually on a roof) for residence in warm weather} \v{25}They waited until they were embarrassed, since he never opened the doors to the chamber. Eventually they took a key, opened the doors, and found their master dead on the ground.

\v{26}Meanwhile, Ehud escaped while they were delayed, passed by the idols, and escaped in the direction of Seirah. \v{27}When he arrived there, he sounded a trumpet in the mountainous region\fnote{Or \fbib{the hill country}} of Ephraim. While the Israeli army accompanied Ehud from the mountainous regions,\fnote{Or \fbib{the hill country}} \v{28}he told them, ``Attack them, because the \divine{Lord} has given your enemies---the Moabites---into your control.'' So the Israeli army\fnote{Lit. \fbib{he}} followed after him, seized the fords of the Jordan River opposite Moab, and did not allow anyone to cross. \v{29}At that time they attacked about 10,000 Moabites, all of whom were strong and valiant men. Not one man escaped. \v{30}As a result, Moab was subdued under the control of Israel, and the land remained quiet for 80 years.
\passage{Shamgar, Israel's Third Judge}

\v{31}After Ehud,\fnote{Lit. \fbib{him}} Anath's son Shamgar attacked 600 Philistines with a cattle prod. He also delivered Israel.
\labelchapt{4}
\passage{Deborah, Israel's Fourth Judge}

\chapt{4}
\v{1}After Ehud died, while the \divine{Lord} was watching, the Israelis made the evil they had been practicing even worse, \v{2}so the \divine{Lord} turned them over to domination by Jabin king of Canaan, who reigned in Hazor. Sisera, the commanding officer of his army, lived in Harosheth-haggoyim.\fnote{Or \fbib{in the gentile district of Harosheth}} \v{3}The Israelis cried out to the \divine{Lord}, because of his 900 iron chariots. Jabin\fnote{Lit. \fbib{he}} oppressed the Israelis forcefully for twenty years.

\v{4}Deborah, a woman, prophet, and wife of Lappidoth, was herself judging Israel during that time. \v{5}She regularly took her seat\fnote{I.e. in her capacity as governor} under the Palm Tree of Deborah between Ramah and Bethel in the mountainous region\fnote{Or \fbib{the hill country}} of Ephraim, where the Israelis would approach her for decisions. \v{6}She sent word to Abinoam's son Barak from Kedesh-naphtali, summoning him. She asked him, ``The \divine{Lord} God of Israel has commanded you, hasn't he? He told you,\fnote{The Heb. lacks \fbib{He told you}} `Go out, march to Mount Tabor, and take 10,000 men with you from the tribes\fnote{Lit. \fbib{children}} of Naphtali and Zebulun. \v{7}I will draw out Sisera, the commanding officer of Jabin's army, along with his chariots and troops, to the Kishon River, where I will drop him right into your hands.'\,''

\v{8}``If you'll go with me, I'll go,'' Barak replied. ``But if you won't go with me, then I'm not going.''

\v{9}She responded, ``I will surely go with you, but the road that you're about to take will not lead to honor for you. The \divine{Lord} will sell Sisera into the hands of a woman.'' Then Deborah got up and went with Barak toward Kedesh. \v{10}Barak called out the army of the tribes of Zebulun and Naphtali to march on Kedesh, and 10,000 men went out to war with him, along with Deborah.

\v{11}Meanwhile, Heber the Kenite had been separated from the Kenites, the descendants of Moses' father-in-law Hobab. He had pitched his tents far away, near the Elon-bezaanannim.\fnote{Or \fbib{the Plain of Zaanannim}} \v{12}Furthermore, Sisera had been informed that Abinoam's son Barak had marched on Mount Tabor. \v{13}So Sisera gathered his iron chariots together from Harosheth-haggoyim\fnote{Or \fbib{from the gentile district of Harosheth}}---all 900 of them, along with all the people who were assigned to them---and they assembled at the Kishon River.

\v{14}``Get going!'' Deborah told Barak. ``Because today's the day when the \divine{Lord} has dropped Sisera into your hands! Look! The \divine{Lord} has already gone out ahead of you!'' So Barak left Mount Tabor, followed by 10,000 men, \v{15}and the \divine{Lord} threw Sisera, all the chariots, and his entire army into a panic right in front of Barak. Then Sisera abandoned his chariot and escaped on foot \v{16}while Barak chased the chariots and army as far as Harosheth-haggoyim.\fnote{Or \fbib{as the gentile district of Harosheth}} Sisera's entire army died in the battle---not even one soldier\fnote{The Heb. lacks \fbib{soldier}} remained.
\passage{Heber's Wife Jael Kills Sisera}

\v{17}Meanwhile, Sisera had escaped on foot to a tent belonging to Jael, wife of Heber the Kenite, since there was peace between Jabin king of Hazor and the household of Heber the Kenite. \v{18}Jael went out to greet Sisera. ``Turn aside, sir!'' she told him. ``Turn aside to me! Don't be afraid.'' So he turned aside to her and entered her tent, where she concealed him behind a curtain.\fnote{Or \fbib{she covered him with a blanket}}

\v{19}He asked her, ``Please give me some water to drink, because I'm thirsty.'' Instead, she opened a leather container of milk, gave him a drink, and then covered him up. \v{20}He told her, ``Stand in the doorway of the tent, and if anyone comes and asks `Is anybody here?' say `No'.''

\v{21}But Heber's wife Jael grabbed a tent peg in one hand and a hammer in the other,\fnote{The Heb. lacks \fbib{in the other}} crept up to him quietly, and drove the tent peg right through his temple into the ground below after he had fallen sound asleep from exhaustion. That's how\fnote{The Heb. lacks \fbib{That's how}} he died.

\v{22}Meanwhile, as Barak continued chasing Sisera, Jael went out to meet him. ``Come with me,'' she told him, ``and I'll show you the man you're looking for!'' So he went with her, and there was Sisera, lying dead with the tent peg still embedded in his temple! \v{23}That's how God subdued Jabin, king of Canaan right in front of the Israelis that day. \v{24}And the Israelis gained greater control over King Jabin of Canaan until they had eliminated him.
\labelchapt{5}
\passage{Deborah and Barak Celebrate in Song}

\chapt{5}
\v{1}Later that day, Deborah and Abinoam's son Barak celebrated by singing this song:

\begin{poetry}
\poeml \v{2}``When hair grows long\fnote{I.e. in keeping with having made a Nazirite vow} in Israel,\fnote{Or \fbib{When leaders carry out vengeance in Israel}} \\
\poemll    when the people give themselves willingly, \\
\poemlll       bless the \divine{Lord}! \\
\poeml \v{3}Listen, you kings! \\
\poemll    Turn your ears to me, you rulers! \\
\poeml As for me, to the \divine{Lord} I will sing! \\
\poemll    I will sing praise to the \divine{Lord} God of Israel. \\
\poeml \v{4}\divine{Lord}, when you left Seir, \\
\poemll    when you marched out \\
\poemlll       from the grain field of Edom, \\
\poeml the earth quaked \\
\poemll    and the heavens poured out rain;\fnote{The Heb. lacks \fbib{rain}} \\
\poemlll       indeed, the clouds poured out water. \\
\poeml \v{5}Mountains tremble at the presence of the \divine{Lord} --- \\
\poemll    even\fnote{Lit. \fbib{this}} Sinai!---at the presence of the \divine{Lord} God of Israel. \\
\poeml \v{6}During the lifetime of Anath's son Shamgar \\
\poemll    and during the lifetime of Jael \\
\poeml highways remained deserted, \\
\poemll    while travelers kept to back roads. \\
\poeml \v{7}Rural populations plummeted\fnote{Lit. \fbib{ceased}} in Israel; \\
\poemll    until I, Deborah, arose; \\
\poemlll       until I---an Israeli mother---arose. \\
\poeml \v{8}New gods were chosen, \\
\poemll    then war came to the city\fnote{The Heb. lacks city} gates, \\
\poeml but there wasn't a shield or spear to be seen \\
\poemll    among 40,000 soldiers\fnote{The Heb. lacks \fbib{soldiers}} of Israel. \\
\poeml \v{9}My heart is for the commanders of Israel, \\
\poemll    to those who work willingly among the people. \\
\poemlll       Bless the \divine{Lord}! \\
\poeml \v{10}``Speak up, you who ride white donkeys, \\
\poemll    sitting on cloth saddles\fnote{Or \fbib{wearing rich clothing}} \\
\poemlll       while you travel on your way! \\
\poeml \v{11}From the sound of those who divide their work loads \\
\poemll    at the watering troughs, \\
\poeml there they will retell the righteous deeds of the \divine{Lord}, \\
\poemll    the righteous victories for his rural people in Israel.''
\end{poetry}

Then the people of the \divine{Lord} went down to the gates.

\begin{poetry}
\poeml \v{12}``Wake up! Wake up, Deborah! \\
\poemll    Wake up! Wake up, Deborah! \\
\poeml Get up, Barak, and dispose of your captives, \\
\poemll    you son of Abinoam! \\
\poeml \v{13}Then the survivors approached the nobles; \\
\poemll    the people of the \divine{Lord} approached me in battle array. \\
\poeml \v{14}Some came\fnote{The Heb. lacks \fbib{came}} from Ephraim \\
\poemll    who had been harassed by\fnote{Or \fbib{who routed}; So LXX.} Amalek, \\
\poemlll       followed by Benjamin with your people. \\
\poeml Some commanders came\fnote{The Heb. lacks \fbib{came}} from Machir, \\
\poemll    along with some from Zebulun \\
\poemlll       who carry a badge\fnote{Lit. \fbib{scepter}} of office.\fnote{Or \fbib{who wield official authority}} \\
\poeml \v{15}The officials of Issachar were with Deborah, \\
\poemll    as was the tribe of Issachar and Barak. \\
\poeml They rushed out into the valley at his heels \\
\poemll    along with divisions from Reuben's army. \\
\poemlll       Great was their resolve of heart! \\
\poeml \v{16}Why did you sit down among the sheepfolds? \\
\poemll    To hear the bleating of the flocks? \\
\poeml Among the divisions of the army of Reuben \\
\poemll    there was great searching of heart. \\
\poeml \v{17}The tribe of Gilead remained \\
\poemll    on the other side of the Jordan River. \\
\poeml As for the tribe of Dan, \\
\poemll    why did they stay on board their ships? \\
\poeml The tribe of Asher sat by the seashore \\
\poemll    and remained near its harbors. \\
\poeml \v{18}The tribe of Zebulun did not worry about their lives \\
\poemll    at the price of death; \\
\poeml neither did the tribe of Naphtali also \\
\poemll    on high places of the field.\fnote{I.e. as they fought within idolatrous worship centers} \\
\poeml \v{19}``Kings came to fight, \\
\poemll    then battled the kings of Canaan \\
\poemlll       at Taanach near the waters of Megiddo. \\
\poeml They took no silver \\
\poemll    as the spoils of war. \\
\poeml \v{20}The stars fought from heaven; \\
\poemll    they fought against Sisera from their orbits. \\
\poeml \v{21}The current\fnote{Or \fbib{wadi}; i.e. a seasonal river, and so throughout the verse} of the Kishon River swept them downstream, \\
\poemll    that ancient current, the Kishon's current! \\
\poemlll       March on strongly, my soul! \\
\poeml \v{22}Then loud was the beat of the horses' hooves--- \\
\poemll    from the galloping, galloping war steeds! \\
\poeml \v{23}```Meroz is cursed!' declared the angel of the \divine{Lord}. \\
\poemll    `Utterly and totally cursed are its inhabitants, \\
\poeml because they never came to the aid of the \divine{Lord}, \\
\poemll    to the aid of the \divine{Lord} against the valiant warriors!'\,'' \\
\poeml \v{24}``Blessed above all women is Jael, \\
\poemll    wife of Heber the Kenite; \\
\poemlll       most blessed is she among women who live in tents! \\
\poeml \v{25}Sisera\fnote{Lit. \fbib{He}} asked for water--- \\
\poemll    she gave him milk. \\
\poemlll       In a magnificent bowl she brought him yogurt!\fnote{I.e. a processed milk product} \\
\poeml \v{26}She reached out one hand for the tent peg, \\
\poemll    and her other\fnote{Lit. \fbib{right}} for the workman's mallet. \\
\poeml Then she struck Sisera, \\
\poemll    smashing his head, \\
\poemlll       shattering and piercing his temple. \\
\poeml \v{27}He crumpled to the ground between her feet, \\
\poemll    where he fell down and collapsed. \\
\poeml Between her feet he crumpled, \\
\poemll    Fallen dead! \\
\poeml \v{28}``Back at home,\fnote{The Heb. lacks \fbib{Back at home}} out the window Sisera's mother peered, \\
\poemll    lamenting through the lattice. \\
\poeml `Why is his chariot delayed in returning? \\
\poemll    `Why do the hoof beats of his chariots wait?' \\
\poeml \v{29}Her wise attendants\fnote{Or \fbib{officials}} find an answer for her; \\
\poemll    in fact, she tells the same words to herself: \\
\poeml \v{30}`They're busy finding and dividing the war booty, aren't they? \\
\poemll    A girl or two for each valiant warrior, \\
\poeml and some dyed materials for Sisera--- \\
\poemll    perhaps dyed, embroidered war booty--- \\
\poeml or some detailed embroidery for my neck \\
\poemll    as the booty of war! \\
\poeml \v{31}``May all of your enemies perish like this, \divine{Lord}! \\
\poemll    But may those who love him be \\
\poemlll       like the ascending sun in its strength!''
\end{poetry}

Then the land enjoyed quiet for 40 years.
\labelchapt{6}
\passage{Gideon, Israel's Fifth Judge}

\chapt{6}
\v{1}Later on, the Israelis practiced what the \divine{Lord} considered to be evil, so the \divine{Lord} handed them over to the domination of Midian for seven years. \v{2}Midian's control predominated throughout Israel, and because of Midian the Israelis went out to find temporary hiding places for themselves in the mountains, caves, and fortified places.

\v{3}Whenever the Israelis sowed their crops,\fnote{The Heb. lacks \fbib{their crops}} the Midianites, the Amalekites, and certain groups\fnote{Lit. \fbib{and sons}} from the east would come up and invade them. \v{4}They set up their military encampments to fight them, destroyed the harvest of the land as far as Gaza, and left nothing in Israel, whether harvested grain, sheep, oxen, or donkeys. \v{5}They would invade with their livestock and tents, swooping in as numerous as locusts. It was impossible to count them or their camels---and they came into the land to destroy it. \v{6}Because Israel was deeply impoverished due to the Midianites, they\fnote{Lit. \fbib{Midianites, the Israelis}} cried out to the \divine{Lord}.

\v{7}When the Israelis cried out to him about Midian, \v{8}the \divine{Lord} sent a man who was a prophet to the Israelis and told them, ``This is what the \divine{Lord} God of Israel says: `I was the one who brought you up from the land of Egypt, delivering you from the house of servitude. \v{9}I delivered you from the domination of Egypt and from the domination of all of your oppressors, expelling them right in front of you and giving their land to you. \v{10}I told you, ``I am the \divine{Lord} your God. You are not to fear the gods of the Amorites in whose land you'll be living.''\,' But you haven't obeyed what I said.''
\passage{Gideon is Visited by the Angel of the \divine{Lord}}

\v{11}After this, the angel of the \divine{Lord} arrived and sat down in the shade of\fnote{The Heb. lacks \fbib{the shade of}} the oak tree in Ophrah that belonged to Joash, a descendant of Abiezer, while his son Gideon was threshing wheat in a wine press in order to safeguard it\fnote{The Heb. lacks \fbib{it}} from the Midianites. \v{12}The angel of the \divine{Lord} appeared to him and told him, ``The \divine{Lord} is with you, you valiant warrior!''

\v{13}But Gideon replied, ``Right{\ldots} Sir, if the \divine{Lord} is with us, then why has all of this happened to us? And where are all of his miraculous works that our ancestors recounted to us when they said, `The \divine{Lord} brought us up from Egypt, didn't he?' But now the \divine{Lord} has abandoned us and given us over to Midian!''

\v{14}The \divine{Lord} looked straight at him and replied, ``Go with this determination\fnote{Or \fbib{strength}} of yours and deliver Israel from Midian's domination. I've directed\fnote{Or \fbib{sent}} you, haven't I?''

\v{15}``Right{\ldots},'' Gideon\fnote{Lit. \fbib{he}} responded. ``Sir, how will I deliver Israel? Look---my family is the weakest in Manasseh, and I'm the youngest in my father's household.''

\v{16}The \divine{Lord} told him, ``Because I'll be with you, and you'll defeat Midian---every single one of them!''

\v{17}So Gideon asked him, ``Please, if I have received favor from you, then do a miracle for me that shows that you're making this\fnote{The Heb. lacks \fbib{this}} promise to me. \v{18}And please don't leave here until I've come back to you, brought my offering, and set it down in front of you.''

The \divine{Lord}\fnote{Lit. \fbib{So he}} replied, ``I'll stay until you return.''

\v{19}Then Gideon went and prepared a young goat and unleavened bread from an ephah of flour. He put the meat in a basket and poured the broth into a pot, and brought them to the angel\fnote{Lit. \fbib{to him}} right under the oak tree. Then he made his offering. \v{20}The angel, who was God,\fnote{Or \fbib{angel of God}} replied, ``Take the meat and the unleavened bread and lay them on this boulder. Then pour out the broth.'' So he did that. \v{21}The angel of the \divine{Lord} extended the tip of the staff that was in his hand and touched the meat and unleavened bread. Fire broke out from inside the boulder, consuming the meat and unleavened bread. Then the angel of the \divine{Lord} vanished in front of him.\fnote{Lit. \fbib{\divine{Lord} left his eyes}}
\passage{God Reassures Gideon}

\v{22}When Gideon realized that he had seen the angel of the \divine{Lord} himself, he cried out, ``Oh no! Lord \divine{God}! I've been looking right at the angel of the \divine{Lord}---and face-to-face at that!''

\v{23}``Calm down!\fnote{Lit. \fbib{Peace to you!}} Don't be afraid.'' the \divine{Lord} replied. ``You're not going to die!'' \v{24}So Gideon built an altar right there to the \divine{Lord} and called it ``The \divine{Lord} is peace.'' (To this very day it still stands in Ophrah, which belongs to the descendants of Abiezer.)

\v{25}Later that very night, the \divine{Lord} told Gideon,\fnote{Lit. \fbib{him}} ``Take the bull that belongs to your father, along with a second bull that's seven years old. Then tear down the altar to Baal\fnote{I.e. the supreme male deity of the Canaanites} that your father owns, cut down the Asherah\fnote{I.e. a carved wooden pillar dedicated to various female deities of the Canaanites, and so throughout the book} that's beside it, \v{26}and build an altar to the \divine{Lord} your God on top of this stronghold in an orderly manner. Then take the second bull and offer it as a burnt offering using the wood from the Asherah that you'll be cutting down.''
\passage{Gideon Destroys His Father's Altar}

\v{27}So Gideon went with ten men who were his servants and did just what the \divine{Lord} had told him to do, though he did it at night because he was too afraid of his father's family and the leading\fnote{The Heb. lacks \fbib{leading}} men of the city to do it during the day. \v{28}When the leading\fnote{The Heb. lacks \fbib{leading}} men of the city got up early the next morning, the altar to Baal had been torn down, along with the Asherah that had stood beside it, and the second bull had been offered on the altar that had been erected.

\v{29}They asked each other, ``Who did this thing?'' When they looked into it and asked around, they concluded, ``Joash's son Gideon did it.''\fnote{Lit. \fbib{did this thing}} \v{30}So the leading\fnote{The Heb. lacks \fbib{leading}} men of the city ordered Joash, ``Bring us that son of yours. He's going to die, because he tore down the altar to Baal and cut down the Asherah that stood beside it!''

\v{31}But Joash responded to everyone who was opposing him, ``Do you really intend to fight on Baal's behalf? Do you really intend to rescue him by ordering\fnote{The Heb. lacks \fbib{by ordering}} that whoever fights him will be executed by morning? If Baal\fnote{Lit. \fbib{he}} is a god, let him fight for himself. After all, it was his altar that was torn down.'' \v{32}So that very day he named Gideon\fnote{Lit. \fbib{him}} Jerubbaal, that is, ``Let Baal fight,'' since he had torn down his altar.

\v{33}Then all the Midianites, Amalekites, and certain groups\fnote{Lit. \fbib{and sons}} from the east gathered together, crossed the Jordan River, and set up camp in the Jezreel Valley. \v{34}So the Spirit of the \divine{Lord} took control of\fnote{Lit. \fbib{\divine{Lord} clothed himself with}} Gideon, who blew a trumpet, mustering the descendants of Abiezer to follow him into battle.\fnote{The Heb. lacks \fbib{into battle}} \v{35}He sent messengers to the entire tribe of Manasseh, calling them to follow him, and he also sent word to the tribes of Asher, Zebulun, and Naphtali, calling them to meet him.
\passage{Gideon Asks for a Sign from God}

\v{36}Then Gideon told God, ``If you intend to deliver Israel by my efforts\fnote{Lit. \fbib{hand}} as you've said, \v{37}then take a look at this wool fleece that I'm placing on the threshing floor. If dew appears only on the fleece---and it's dry on the ground all around it---then I'll know that you'll deliver Israel by my efforts\fnote{Lit. \fbib{hand}} like you've said.'' \v{38}And that is what happened:\fnote{Lit. \fbib{And so it was}} When he got up early the next morning, he wrung out the fleece to drain the dew from it and extracted\fnote{The Heb. lacks \fbib{and extracted}} a bowl full of water.

\v{39}Then Gideon told God, ``Don't let yourself be angry with me! I want to ask you once again: please let me make a test with the fleece just once more. Cause it to be dry only on the fleece, but let there be dew all around on the ground.'' \v{40}And God did it just like that later that night. It was dry only on the fleece, but dew was all around on the ground.
\labelchapt{7}
\passage{God Chooses Gideon's 300 Soldiers}

\chapt{7}
\v{1}Then Jerubbaal, also known as Gideon, got up early along with all of his soldiers. They encamped near the Harod Spring. The Midian encampment lay in the valley to their north, near the hill of Moreh. \v{2}The \divine{Lord} told Gideon, ``You have too many soldiers with you for me to drop Midian into their hands, because Israel would become arrogant and say, `It was my own abilities that delivered me.' \v{3}That's why you're to ask in full view of the soldiers, ``Whoever is afraid or is trembling may go back from Mount Gilead and return home.''\fnote{The Heb. lacks \fbib{home}} So 22,000 soldiers left and 10,000 remained.

\v{4}``There are still too many soldiers,'' the \divine{Lord} told Gideon. ``Bring them down to the water and I'll refine them for you there. Therefore when I say to you, `This one will be going with you,' he'll go with you, but no one may go about whom I tell you, `This one won't be going with you.'\,''

\v{5}So he brought his soldiers down to the water, and the \divine{Lord} told Gideon, ``You are to cull out everyone who laps up water with his tongue like a dog from everyone who kneels to drink.'' \v{6}The contingent of soldiers who lapped water\fnote{The Heb. lacks \fbib{water}} with their hands to their mouths numbered 300 men, but everyone else kneeled to drink water.

\v{7}Then the \divine{Lord} told Gideon, ``I'm going to deliver you with the 300 soldiers who lapped by giving the Midianites into your control. Send everyone else back to their own homes.''\fnote{Lit. \fbib{place}}

\v{8}So the soldiers took provisions with them, along with their trumpets, and Gideon\fnote{Lit. \fbib{he}} sent all the rest of the soldiers of Israel back to their own tents, but he retained the 300 men. And the Midian encampment was below him in the valley.
\passage{Gideon Sneaks Down to the Midianite Encampment}

\v{9}Later that same night, the \divine{Lord} directed Gideon,\fnote{Lit. \fbib{him}} ``Get up and go down to the Midianite\fnote{The Heb. lacks \fbib{Midianite}} encampment, because I've given it into your control. \v{10}But if you're afraid to go down there, you may take your servant Purah with you to their encampment, \v{11}where you will hear what they're talking about. That way, you'll be encouraged to attack the encampment.'' So he and his servant Purah went down to the perimeter outposts of the encamped army.

\v{12}The Midianites, the Amalekites, and certain groups\fnote{Lit. \fbib{and sons}} from the east lay encamped in the valley, as thick as locusts. The number of their camels couldn't be calculated---they seemed as numerous as the sand on the seashore. \v{13}Gideon arrived just as a soldier was talking to a friend about a dream. ``Look!'' he was saying. ``I had a dream that went like this: A loaf of barley bread rolled into the Midianite encampment, came to a tent, and collided with it. The loaf of bread fell down, turned upside down, and the tent collapsed!''

\v{14}Then his friend replied, ``Can this be anything else than the sword of Joash's son Gideon, that man from Israel? God must have given Midian and the entire encampment into his control!''

\v{15}When Gideon\fnote{Lit. \fbib{he}} heard the tale of the dream and its interpretation, he bowed down in worship and then returned to the Israeli encampment.
\passage{Gideon's 300 Attack}

There he announced, ``Get up! The \divine{Lord} has given the Midianite army into your control!'' \v{16}Then he separated the 300 men into three companies, gave them each trumpets to carry, along with jars into which he placed lit torches.

\v{17}He instructed them, ``Watch me, and do what I do. When we come to the outer perimeter of the encampment, do what I do. \v{18}When I sound my trumpet, accompanied by everyone who is with me, you must blow your trumpets all around the entire encampment. Then shout out, `For the \divine{Lord} and for Gideon!'\,''

\v{19}So Gideon and the 100 men with him arrived at the outer perimeter of the encampment at the beginning of the middle watch, just after they had posted sentries. They blew their trumpets and smashed the jars that they were carrying in their hands. \v{20}When the three companies sounded their trumpets and broke the jars, they held the torches in their left hands and sounded their trumpets with their right hands. Then they cried out, ``A sword for the \divine{Lord} and for Gideon!'' \v{21}They stood up, each soldier in his assigned\fnote{The Heb. lacks \fbib{assigned}} place surrounding the encampment, and the entire army ran away, sounding the alarm to retreat.

\v{22}As the 300 trumpets were being sounded, the \divine{Lord} turned the swords of the Midianite\fnote{The Heb. lacks \fbib{Midianite}} soldiers against one another throughout the entire army, and the army ran away as far as Beth-shittah in the direction of Zererah. They got as far as the outskirts of Abel-meholah, near Tabbath. \v{23}Israeli soldiers were called out from the territories of\fnote{The Heb. lacks \fbib{the territories of}} Naphtali, Asher, and throughout Manasseh, and they chased after the Midianites.

\v{24}Gideon dispatched messengers throughout the mountainous region\fnote{Or \fbib{the hill country}} of Ephraim, notifying them, ``Come down to fight Midian. Capture the water crossings\fnote{The Heb. lacks \fbib{crossings}} as far as Beth-barah and the Jordan River before they can get to them.'' \v{25}They captured two Midianite leaders, Oreb and Zeeb. While they were pursuing the Midianites, they executed Oreb at Oreb's Rock and Zeeb at Zeeb's Winepress, and then they carried the heads of Oreb and Zeeb to Gideon from the east bank\fnote{Lit. \fbib{the other side}} of the Jordan River.
\labelchapt{8}
\passage{Gideon Assuages the Anger of Ephraim}

\chapt{8}
\v{1}Later on, the descendants of Ephraim spoke to Gideon.\fnote{Lit. \fbib{him}} They argued vehemently, ``What are you doing to us? You never called us! But you went out to fight Midian!''

\v{2}``What have I accomplished compared to you?'' he responded. ``Isn't what's left from Ephraim's harvest better than the best vintage of Abiezer? \v{3}God gave Oreb and Zeeb, the leaders of Midian, into your control. What was I able to do compared to you?'' When he said this, their anger calmed down.

\v{4}Meanwhile, Gideon and the 300 soldiers with him came to the Jordan, exhausted but continuing their pursuit. \v{5}He told the men of Succoth, ``Please give loaves of bread to the soldiers who are following behind me. They're tired, and I'm pursuing Zebah and Zalmunna, the kings of Midian.''

\v{6}But the officials of Succoth replied, ``Do you have Zebah and Zalmunna in custody\fnote{Lit. \fbib{have the hands of Zebah and Zalmunna}} already, so that we should give food to your army?''

\v{7}So Gideon responded, ``Very well then, but when the \divine{Lord} has turned over Zebah and Zalmunna into my control, I'm going to whip you with thorns and briers from the desert!''

\v{8}Then he left there to go to Penuel and asked the same thing from them, but the men of Penuel responded the same way the men of Succoth did. \v{9}So he responded the same way to the men of Penuel, ``When I come back safely,\fnote{Lit. \fbib{return in peace}} I'm going to tear down this tower.''

\v{10}Now Zebah and Zalmunna were in Karkor, along with their armies, about 15,000 men who survived from the entire army of the group from\fnote{Lit. \fbib{the sons of}} the east, since 120,000 swordsmen had already fallen. \v{11}Gideon went up by a caravan route east of Nobah and Jogbehah and attacked their encampment when they were off guard. \v{12}When Zebah and Zalmunna escaped, he pursued them, captured those two kings of Midian,\fnote{Lit. \fbib{Midian, Zebah and Zalmunna,}} and threw the entire army into a panic.

\v{13}Then Joash's son Gideon returned from the battle along the Heres Ascent. \v{14}He caught a young man from Succoth and interrogated him. He wrote out for Gideon\fnote{Lit. \fbib{him}} a list of the 77 officials of Succoth, including its elders. \v{15}Then Gideon\fnote{Lit. \fbib{he}} approached the men of Succoth and announced, ``Here are Zebah and Zalmunna. You criticized me about them when you said, `Do you have Zebah and Zalmunna in custody\fnote{Lit. \fbib{have the hands of Zebah and Zalmunna}} already, so that we should give food to your weary army?'\,'' \v{16}So he took the elders of the city and disciplined the men of Succoth with thorns and briers from the desert. \v{17}He also demolished the tower in Penuel and killed the men of the city.

\v{18}Afterwards, he asked Zebah and Zalmunna, ``What were the men like whom you killed at Tabor?''

They answered, ``Like you, each one like the son of a king{\ldots}''

\v{19}Gideon replied, ``They were my brothers---sons from my own mother. As the \divine{Lord} lives, if you had let them live, I wouldn't be killing you.'' \v{20}Then he told his firstborn son Jether, ``Get up and kill them!'' But he was afraid, since he was still only a youngster.

\v{21}Then Zebah and Zalmunna responded, ``Get up and attack us yourself, since a man's valor is only as good as the man himself.'' So Gideon got up, killed Zebah and Zalmunna, and took away the crescent-shaped necklaces that adorned the necks of their camels.

\v{22}Then the men of Israel asked Gideon, ``Rule over us---you, your son, and your grandsons---because you have delivered us from Midian's domination.''

\v{23}But Gideon told them, ``I won't rule over you and my son won't rule over you. The \divine{Lord} will rule you.''
\passage{Gideon Falls into Idolatry}

\v{24}But Gideon also added, ``I would like to ask that each of you give me a ring from his war booty'' because, as Ishmaelites, the Midianites\fnote{Lit. \fbib{they}} had been wearing gold rings.

\v{25}They responded, ``We'll be happy to give them.'' So they laid out a garment, and each of them contributed a ring from his war booty. \v{26}The weight of the rings that he had asked for was 1,700 gold coins,\fnote{The Heb. lacks \fbib{coins}} not counting the crescent-shaped necklaces, pendants, and purple garments worn by the Midian kings, and also not counting the bands adorning the necks of their camels.

\v{27}Gideon crafted the booty into an ephod\fnote{cf. Lev 8:7; a golden garment rightly worn only by the Levitical high priest} and enshrined it in his home town of Ophrah. Then all of Israel committed spiritual adultery with it there, and it became a snare for Gideon and his household.
\passage{Gideon Dies}

\v{28}Midian remained subjugated to the Israelis, and they didn't so much as raise their heads anymore, so the land was peaceful for 40 years during the lifetime of Gideon. \v{29}Afterwards, Joash's son Jerubbaal went home and retired.\fnote{Lit. \fbib{lived}} \v{30}Gideon raised 70 sons as his direct descendants, since he had many wives. \v{31}His mistress\fnote{Or \fbib{concubine}; i.e. a secondary wife} in Shechem bore him a son whom he named Abimelech.\fnote{The Heb. name \fbib{Abimelech} means \fbib{My father is king}} \v{32}Later, Joash's son Gideon died at a ripe\fnote{Lit. \fbib{good}} old age and was buried in the tomb of his father Joash at Ophrah, which belonged to the descendants of Abiezer.

\v{33}Later on, as soon as Gideon was dead, the Israelis again committed spiritual adultery with various Canaanite deities\fnote{Lit. \fbib{baals}} and appointed Baal-berith\fnote{The Heb. name \fbib{Baal-berith} means \fbib{Lord of the Covenant}} to be their god. \v{34}The Israelis did not remember the \divine{Lord} their God, who continually delivered them from the domination of their enemies who surrounded them on every side. \v{35}And they showed no gracious love to the household of Jerubbaal---also known as Gideon---despite all the good that he had done for Israel.
\labelchapt{9}
\passage{Abimelech Attempts to Become King}

\chapt{9}
\v{1}Then Jerubbaal's son Abimelech went to his mother's relatives in Shechem. He spoke to the entire family of his mother's father, telling them, \v{2}``Ask all the ``lords''\fnote{Lit. \fbib{baals}; i.e. the leaders---a pun contrasting the Heb. word \fbib{lords} with Baal, the chief male Canaanite deity; and so through v. 47} of Shechem, `What's better for you? That 70 men, each of them Jerubbaal's sons, rule over you? Or that one man rule over you?' Keep in mind that I'm like your own close relative.''\fnote{Lit. \fbib{your skin and flesh}}

\v{3}So his mother's relatives spoke all of this on his behalf in the presence\fnote{Lit. \fbib{hearing}} of all the ``lords'' of Shechem. Since they were inclined to follow Abimelech, they said, ``He's our relative!'' \v{4}and they gave him 70 silver coins from the temple that they had built to\fnote{Lit. \fbib{temple of}} Baal-berith. Abimelech hired some worthless and useless men, who followed him \v{5}to his father's house in Ophrah. There he murdered his own brothers, Jerubbaal's sons---all 70 of them---in one place.\fnote{Lit. \fbib{them---on one stone}} But Jerubbaal's youngest son Jotham survived by hiding himself.

\v{6}All the men from Shechem and Beth-millo\fnote{Or \fbib{and from the household of Rampart}; and so throughout the chapter} gathered together and set up Abimelech as king near the pillar erected\fnote{I.e. a cultic object of worship} in Shechem. \v{7}When Jotham was informed about this, he went out, took his stand on top of Mount Gerizim, and cried out loudly, ``Listen to me, you ``lords'' of Shechem, and God will listen to you.

\begin{poetry}
\poeml \v{8}``Once upon a time\fnote{The Heb. lacks \fbib{Once upon a time}} the trees went out \\
\poemll    to consecrate\fnote{Or \fbib{anoint}} a king for themselves. \\
\poeml ``So they told the olive tree, \\
\poemll    `Reign over us!' \\
\poeml \v{9}But the olive tree asked them, \\
\poemll    `Should I stop producing my rich oils \\
\poemlll       by which both God and men are honored \\
\poemll    and go take dominion over trees?' \\
\poeml \v{10}``So the trees told the fig tree, \\
\poemll    `Hey you! Come and reign over us!' \\
\poeml \v{11}But the fig tree asked them, \\
\poemll    `Should I leave my sweet, good fruit \\
\poemlll       and go take dominion over trees?' \\
\poeml \v{12}``So the trees told the grape vine, \\
\poemll    `Hey you! Come and reign over us!' \\
\poeml \v{13}But the grape vine asked them, \\
\poemll    `Should I leave my new wine, \\
\poemlll       which cheers God and man, \\
\poemll    and go take dominion over trees?' \\
\poeml \v{14}``So all the trees told the bramble bush, \\
\poemll    `Hey you! Come and reign over us!' \\
\poeml \v{15}Then the bramble bush replied to the trees, \\
\poemll    `If you really are consecrating\fnote{Or \fbib{anointing}} me to rule you, \\
\poemlll       come and put your confidence in my shade; \\
\poemll    but if not, may fire spring out from the bramble bush \\
\poemlll       and burn up the cedars\fnote{I.e. a genus of coniferous evergreen in the family \fbib{Pinaceae}; and so throughout the book} of Lebanon{\ldots}'
\end{poetry}

\v{16}``Now then, if you have been acting in good faith and integrity by making a king out of Abimelech, if you have treated Jerubbaal and his household appropriately by acting toward him as he deserved\fnote{Lit. \fbib{as his hands acted}}--- \v{17}because my father fought on your behalf, throwing away all concern for his own life, and delivered you from Midian's domination.

\v{18}``But now as for you, you've rebelled against my father's house today. You've murdered his sons---70 men---in one place,\fnote{Lit. \fbib{men---on one stone}} and you've installed Abimelech, the son of his mistress, as king to rule over the ``lords'' of Shechem, since he's related to you. \v{19}So if you've acted in good faith and integrity toward Jerubbaal and his household today, then you're welcome to\fnote{Lit. \fbib{then rejoice in}} Abimelech, and he's welcome to\fnote{Lit. \fbib{and let him rejoice in}} you{\ldots} \v{20}But if not, may fire spring out from Abimelech and consume the ``lords'' of Shechem and Beth-millo, and may fire spring out from the ``lords'' of Shechem and Beth-millo to consume Abimelech.'' \v{21}Then Jotham escaped by running away. He went to Beer and remained there because of his brother Abimelech.
\passage{The Destruction of Shechem}

\v{22}Abimelech dominated Israel for three years. \v{23}Then God sent an evil spirit to divide Abimelech and the ``lords'' of Shechem \v{24}so that the violence committed against the 70 sons of Jerubbaal might come back on their brother Abimelech, who murdered them, and so it might come back on the ``lords'' of Shechem, who provoked him to murder his brothers. \v{25}The ``lords'' of Shechem sent out men to ambush him on the mountain tops, and they robbed everyone who came by them along the roads, and this was reported to Abimelech.

\v{26}Meanwhile, Ebed's son Gaal arrived with his relatives and crossed over into Shechem. The ``lords'' of Shechem put their faith in him. \v{27}They went out into the fields, harvested their vineyards, made some wine, and threw a party. Then they went into the temple of their god, ate, drank, and cursed Abimelech.

\v{28}Then Ebed's son Gaal remarked, ``Who is this Abimelech? And who is Shechem? Should we serve him? Isn't he Jerubbaal's son? Isn't Zebul his lieutenant? Serve the men of Hamor, Shechem's ancestor---but why are we serving him? \v{29}If only authority over this people were given to me. Then I would remove Abimelech!'' Then he challenged Abimelech: ``Build up your army and then come out and fight!''

\v{30}When Zebul, the ruler of the city, heard what Ebed's son Gaal had said, he flew into a rage. \v{31}He sent messengers to Abimelech in secret\fnote{Or \fbib{in Tormah}} and told him, ``Look out! Ebed's son Gaal and his family have arrived here in Shechem. Watch out! They're stirring up the city against you. \v{32}So get up at night, take your soldiers with you, and wait in ambush out in the field. \v{33}Tomorrow morning when the sun is up, get up early and attack the city. When Gaal\fnote{Lit. \fbib{he}} and his army come out to fight you, do whatever you can to them.''

\v{34}So Abimelech and his entire army got up that night and waited in ambush against Shechem in four separate companies.

\v{35}Ebed's son Gaal went out and stood in the entrance to the city gate while Abimelech and his army were creeping out of their ambush. \v{36}When Gaal saw the army, he observed to Zebul, ``Look there! People are coming down from the top of the mountains.''

But Zebul replied to him, ``You're looking at morning shadows cast by the mountains. They just look\fnote{Lit. \fbib{mountains. You are seeing}} like men to you.''

\v{37}Gaal spoke up again to say, ``Look! People are coming down from the highest part of the land, and there's a company approaching from the diviner's oak tree.''\fnote{Or \fbib{from Elon-meonenim}}

\v{38}So Zebul replied, ``Right... So where's your boasting now? You said, `Who is Abimelech? Should we serve him?' Isn't this the army that you insulted? So go out right now and fight them!''

\v{39}So Gaal went out in full view of the ``lords'' of Shechem and fought Abimelech. \v{40}Abimelech chased him, and Gaal ran away from him. Many fell wounded right up to the entrance to the city gate. \v{41}Afterwards, Abimelech remained at Arumah, but Zebul expelled Gaal and his family so they couldn't remain in Shechem.

\v{42}The next day, the people went out to the field, and Abimelech learned about it. \v{43}So he took his army, divided it into three separate companies, and laid in ambush out in the field. When Abimelech\fnote{Lit. \fbib{he}} noticed the people coming out from the city, his\fnote{Lit. \fbib{the}} army attacked them and killed them. \v{44}Then Abimelech and the soldiers who were with him rushed forward and commandeered the entrance to the city gate while the other two companies ran out to kill everyone who was in the field. \v{45}Abimelech fought against the city all that day, captured the city, killed the people in it, then tore the city to the ground and sowed it with salt.

\v{46}When all the ``lords'' at the tower of Shechem heard what had happened, they retreated into the inner chamber of the temple of El-berith. \v{47}Abimilech was told that all of the ``lords'' of the Shechem Tower had assembled there. \v{48}So he\fnote{Lit. \fbib{Abimelech}} went up to Mount Zalmon, accompanied by his entire army. Abimelech had an axe in his hand, so he cut down a branch from a tree, lifted it up, and laid it on his shoulder. Then he told the army that had accompanied\fnote{The Heb. lacks \fbib{had accompanied}} him, ``You've seen what I just did. Hurry up! Do the same thing!''

\v{49}Then his entire army also cut down a branch for each soldier, followed Abimelech to the inner chamber, and set fire to it\fnote{Lit. \fbib{set the inner chamber}} while they were inside. As a result, all the men of the tower of Shechem died, including about a thousand men and women.
\passage{The Death of Abimelech}

\v{50}Later on, Abimelech went to Thebez, set up a siege encampment there, and captured it. \v{51}But there was a fortified tower in the center of the city, and all the men, women, and leaders of the city escaped to it, shut themselves in, and went up to the roof of the tower. \v{52}So Abimelech approached the tower, attacked it, and approached the tower's gate, intending\fnote{The Heb. lacks \fbib{intending}} to burn it down. \v{53}But a certain woman threw an upper millstone down on Abimelech's head, fracturing his skull.

\v{54}So he cried out to his young armor bearer and ordered him, ``Draw your sword and kill me, so no one will say about me that `A woman killed him.'\,'' So the young man pierced him through, and he died. \v{55}When the men of Israel noticed that Abimelech was dead, they each left for home.\fnote{Lit. \fbib{each man left to his place}} \v{56}That's how God repaid Abimelech for the evil thing he did to his father by killing his 70 brothers. \v{57}God also repaid\fnote{Lit. \fbib{repaid on the heads of}} the men of Shechem for their wickedness, and the curse of Jerubbaal's son Jotham came true for them.
\labelchapt{10}
\passage{Tola, Israel's Sixth Judge}

\chapt{10}
\v{1}A man from the tribe of Issachar, Puah's son Tola, grandson of Dodo, arose to save Israel. He lived in Shamir, in the mountainous region\fnote{Or \fbib{the hill country}} of Ephraim. \v{2}He governed Israel for 23 years and then died. He was buried in Shamir.
\passage{Jair, Israel's Seventh Judge}

\v{3}After him, Jair the Gileadite arose and governed Israel for 22 years. \v{4}His 30 sons rode on 30 donkeys, controlling 30 cities in the territory of Gilead named Havvoth-jair\fnote{The Heb. name \fbib{Havvoth-jair} means \fbib{Jair's Villages}} to this day. \v{5}Jair died and was buried in Kamon.
\passage{Israel Descends into Apostasy}

\v{6}Later on, the Israelis again practiced what the \divine{Lord} considered to be evil by serving the Baals, the stars, the gods of Aram, the gods of Sidon, the gods of Moab, the gods of the descendants of Ammon, and the gods of the Philistines. In doing so, they ignored\fnote{Or \fbib{forgot}} the \divine{Lord} and wouldn't serve him. \v{7}In his burning anger against Israel, he sold them into domination by the Philistines and the Ammonites, \v{8}who trampled and troubled the Israelis during that year---eighteen years for the Israelis who lived east of the Jordan River in Gilead, the land occupied by\fnote{Lit. \fbib{land of}} the Amorites. \v{9}The Ammonites crossed the Jordan River to fight against the tribes of Judah, Benjamin, and the house of Ephraim. As a result, Israel was deeply distressed. \v{10}Then the Israelis cried out to the \divine{Lord} and told him,\fnote{The Heb. lacks \fbib{him}} ``We have sinned against you because we have abandoned our God to serve the Baals.''

\v{11}The \divine{Lord} replied to the Israelis, ``Aren't you away from the Egyptians, the Amorites, the Ammonites, and the Philistines? \v{12}And when the Sidonians, the Amalekites, and the Maonites harassed you, you cried out to me, and I delivered you from under their domination. \v{13}But you have abandoned me and served other gods. Therefore I will no longer be delivering you. \v{14}Go and cry out to the gods that you have chosen for yourselves. Let them deliver you in your time of trouble.''

\v{15}The Israelis replied to the \divine{Lord}, ``We have sinned, so do to us anything that's right to do in your opinion, just please deliver us right now.'' \v{16}When they put away their foreign gods and served the \divine{Lord}, he brought Israel's misery to an end. \v{17}The Ammonites were summoned and they encamped in Gilead. The Israelis assembled together and encamped in Mizpah. \v{18}The people and Gilead's officials inquired among themselves, ``Who will begin our attack against the Ammonites? He'll become head over everyone who lives in Gilead.''
\labelchapt{11}
\passage{Jephthah, Israel's Eighth Judge}

\chapt{11}
\v{1}Now Jephthah the Gileadite was a valiant soldier, but he was also the son of a prostitute and Jephthah's father Gilead. \v{2}Gilead's wife bore two sons through him, but when his wife's sons grew up, they expelled Jephthah and declared to him, ``You won't have an inheritance in this\fnote{Lit. \fbib{in our father's}} house, since you're the son of a different woman.'' \v{3}So Jephthah escaped from his brothers and lived in the territory of Tob, where worthless men gathered themselves around him and went out on raiding parties with him.

\v{4}Later on, the Ammonites attacked Israel. \v{5}When this happened,\fnote{Lit. \fbib{When the Ammonites attacked Israel}} the elders of Gilead went to the territory of Tob to find Jephthah. \v{6}They told him, ``Come and be our commander so we can fight the Ammonites!''

\v{7}But Jephthah replied to the elders of Gilead, ``Weren't you the ones who hated me and drove me out of my father's house? And you come to me now that you're in trouble?''

\v{8}So the elders of Gilead told Jephthah, ``Well, we're coming back to you now so you can accompany us, fight the Ammonites, and become the head of all the inhabitants of Gilead.''

\v{9}Then Jephthah asked the elders of Gilead, ``If you all send me to fight against the Ammonites and the \divine{Lord} hands them over right in front of me, will I really become your head?''

\v{10}The elders of Gilead responded to Jephthah, ``May the Lord serve\fnote{Lit. \fbib{hear}} as a witness that we're making this agreement between ourselves to do as we've said.'' \v{11}So Jephthah went with the elders of Gilead, and the people appointed him head and military commander over them. Jephthah uttered everything he had to say with the solemnity of an oath\fnote{Lit. \fbib{uttered all his words}} in the \divine{Lord}'s presence at Mizpah.
\passage{Jephthah's Dialogue with the Ammonites}

\v{12}Afterwards, Jephthah sent messengers to the king of the Ammonites to ask him, ``What's your dispute between us that prompted you to come and attack my land?''

\v{13}The king of the Ammonites answered the messengers of Jephthah, ``We're here\fnote{The Heb. lacks \fbib{We're here}} because Israel took away my land from the Arnon River as far as the Jabbok River and as far as the Jordan River when they came up from Egypt! So restore it as a gesture of good will.''\fnote{Lit. \fbib{restore them in peace}}

\v{14}But Jephthah sent additional messengers again to the king of the Ammonites \v{15}and they informed him, ``This is Jephthah's response:

\begin{poetry}
\poeml `Israel didn't seize the land of Moab nor the land of the Ammonites. \v{16}Here's what happened:\fnote{Lit. \fbib{Because}} When Israel came up from Egypt, passed through the desert to the Red\fnote{Lit. \fbib{Reed}} Sea, and arrived at Kadesh, \v{17}Israel sent a delegation to the king of Edom and asked him, ``Please let us pass through your territory.'' \\
\poeml `But the king of Edom wouldn't listen. So they also sent word to the king of Moab, but he wouldn't consent, either. So Israel stayed at Kadesh. \v{18}Then they went through the desert, circumventing the territory belonging to Edom and Moab. They encamped on the other side of the Arnon River, but never entered the territory of Moab because the Arnon River is the border of Moab. \\
\poeml \v{19}`Then Israel sent a delegation to Sihon, king of the Amorites and king of Heshbon. Israel requested of him, ``Please let us pass through your territory to our place.'' \v{20}But Sihon didn't trust Israel to pass through his territory, so he assembled his entire army, encamped in Jahaz, and fought against Israel. \v{21}The \divine{Lord} God of Israel handed Sihon and his entire army into the control of Israel, and defeated them. As a result, Israel took control over the entire land of the Amorites, who were living in that country. \v{22}They took possession of the entire territory of the Amorites from the Arnon River as far as the Jabbok River and from the desert as far as the Jordan River. \\
\poeml \v{23}`Now then, since the \divine{Lord} God of Israel expelled the Amorites right in front of his people Israel, are you going to control their territory? \v{24}Don't you control what your god Chemosh gives you? In the same way, we'll take control of whomever the \divine{Lord} our God has driven out in front of us. \v{25}Also ask yourselves:\fnote{Lit. \fbib{And now}} do you have a better case\fnote{Lit. \fbib{are you better}} than Zippor's son Balak, king of Moab? Did he ever have a quarrel with Israel or ever win a\fnote{The Heb. lacks \fbib{ever win a}} fight against them? \v{26}When Israel was living in Heshbon and its surrounding villages, in Aroer and its surrounding villages, and in all the cities that line the banks of the Arnon River these past three hundred years, why didn't you retake them during that time? \v{27}I haven't sinned against you, but you are acting wrongly against me by declaring war on me. May the \divine{Lord}, the Judge, sit in judgment today between the Israelis and the Ammonites.'\,''
\end{poetry}

\v{28}But the king of the Ammonites wouldn't heed the message that Jephthah had sent to him.
\passage{Jephthah's Vow}

\v{29}The Spirit of the \divine{Lord} came\fnote{Lit. \fbib{was}} on Jephthah, so he swept through Gilead and the territory of\fnote{The Heb. lacks \fbib{the territory of}} Manasseh, then swept through Mizpah in Gilead, and from Mizpah in Gilead he proceeded toward where the Ammonites were encamped. \v{30}Jephthah made this solemn vow to the \divine{Lord}: ``If you truly give the Ammonites into my control, \v{31}then if I return from the Ammonites without incident,\fnote{Lit. \fbib{Ammonites in peace}} whatever comes\fnote{MT participle is masculine} out the doors of my house to meet me will become the \divine{Lord}'s, and I will offer it\fnote{MT suffix is masculine} up as a burnt offering.''

\v{32}Then Jephthah crossed over to the Ammonites and attacked them. The \divine{Lord} gave them into his control. \v{33}He attacked them from Aroer to the entrance of Minnith---twenty cities in all\fnote{The Heb. lacks \fbib{in all}}---even as far as Abel-keramim. As a result, the Ammonites were subdued right in front of the Israelis. \v{34}When Jephthah arrived at his home in Mizpah---surprise!---it was his daughter who came out to meet him, playing tambourines and dancing. She was his one and only child. Except for her, he had no other son or daughter. \v{35}When he saw her, he ripped his clothes and cried out, ``Oh no! My daughter! You have terribly burdened me! You've joined those who are causing me trouble, because I've given my word\fnote{Lit. \fbib{I've opened my mouth}} to the \divine{Lord}, and I cannot go back on it.\fnote{The Heb. lacks \fbib{on it}}

\v{36}She told him, ``My father, you have given your word\fnote{Lit. \fbib{You've opened your mouth}} to the \divine{Lord}. Do to me according to what has come out of your own mouth, considering that the \divine{Lord} has paid back your enemies, the Ammonites.'' \v{37}Then she continued talking with her father, ``Do this for me: leave me alone by myself for two months. I'll go up to the mountains and cry there because I'll never marry.\fnote{Lit. \fbib{there on behalf of my virginity}; i.e. terminating the genealogy of Jephthah} My friends and I will go.''\fnote{The Heb. lacks \fbib{will go}}

\v{38}So he said, ``Go!'' He sent her away for two months. She left with her friends and cried there on the mountains because she would never marry.\fnote{Lit. \fbib{there for her virginity}} \v{39}Later, after the two months were concluded, she returned to her father, and he fulfilled what he had solemnly vowed---and she never married.\fnote{Lit. \fbib{she did not know a man}} That's how the custom arose in Israel \v{40}that for four days out of every year the Israeli women would go to mourn the daughter of Jephthah the Gileadite in commemoration.
\labelchapt{12}
\passage{Jephthah's Dispute with the Tribe of Ephraim}

\chapt{12}
\v{1}A little while later, the army of Ephraim was mustered, and they crossed to Zaphon. They confronted Jephthah and asked, ``Why did you cross over to fight the Ammonites without calling us to accompany you? We're going to burn your house down around you!''

\v{2}But Jephthah replied to them, ``My army and I were engaged in a serious fight with the Ammonites. I called for you, but you didn't deliver me from their control. \v{3}When I saw that you wouldn't be delivering me, I took my own life in my hands, crossed over to fight the Ammonites, and the \divine{Lord} gave them into my control. So why have you come here today to fight me?'' \v{4}Then Jephthah mustered all the men of Gilead, fought the tribe of Ephraim, and defeated them, because they had been claiming, ``You descendants of Gilead are fugitives in the midst of the tribes of Ephraim and Manasseh.''
\passage{Shibboleth vs. Sibboleth}

\v{5}The descendants of Gilead seized control of the Jordan River's fords along the border of Ephraim's territory.\fnote{Lit. \fbib{fords opposite Ephraim}} Later on, when any fugitive from Ephraim asked them, ``Let me cross over,'' the men from Gilead would ask him, ``Are you an Ephraimite?'' If he said ``No,'' \v{6}they would order him, ``Pronounce the word `Shibboleth' right now.'' If he said ``Sibboleth,'' not being able to pronounce it correctly, they would seize him and slaughter him there at the fords of the Jordan River. During those days 42,000 descendants of Ephraim died that way. \v{7}Jephthah governed Israel for six years. Then Jephthah died and was buried somewhere in the cities of Gilead.
\passage{Ibzan, Israel's Ninth Judge}

\v{8}After he died,\fnote{Lit. \fbib{After him}} Ibzan from Bethlehem governed Israel for ten years. \v{9}He had 30 sons and 30 daughters, but he gave his daughters\fnote{Lit. \fbib{gave them}} in marriage to outsiders and brought in 30 outsiders\fnote{Lit. \fbib{30 daughters from outside}} for his sons. He governed Israel for seven years, \v{10}then he died and was buried in Bethlehem.
\passage{Elon, Israel's Tenth Judge}

\v{11}Elon the Zebulunite governed Israel after him for ten years. \v{12}Then Elon the Zebulunite died and was buried in Aijalon within the territory of Zebulun.
\passage{Abdon, Israel's Eleventh Judge}

\v{13}Hillel the Pirathonite's son Abdon governed Israel after him. \v{14}He had 40 sons and 30 grandsons who rode on 70 donkeys. He governed Israel for eight years. \v{15}Then he died and was buried at Pirathon in the territory of Ephraim, in the mountainous region\fnote{Or \fbib{the hill country}} of the Amalekites.
\labelchapt{13}
\passage{The Birth of Samson, Israel's Twelfth Judge}

\chapt{13}
\v{1}Some time later, the Israelis again practiced what the \divine{Lord} considered to be evil, so the \divine{Lord} handed them over into the domination of the Philistines for 40 years. \v{2}There was one man from Zorah, from the family of the descendants of Dan, whose name was Manoah. Since his wife was infertile, she hadn't borne children.\fnote{The Heb. lacks \fbib{children}} \v{3}One day the angel of the \divine{Lord} presented himself to the woman. ``Hello!'' he greeted\fnote{Lit. \fbib{and told}} her. ``Though you are infertile at this time and haven't borne a child, you're about to conceive and give birth to a son. \v{4}So be sure that you don't drink wine or anything intoxicating, and don't eat anything unclean \v{5}because---surprise!---you're going to conceive and give birth to a son! Don't put a razor to his head, because the young man will be a Nazirite, dedicated\fnote{The Heb. lacks \fbib{dedicated}} to God from inside the womb. He will begin to deliver Israel from domination by the Philistines.''

\v{6}Then the woman went to tell her husband. She said, ``A man of God appeared\fnote{Or \fbib{came}} to me. He looked like what an angel of God would look like---very frightening.\fnote{Or \fbib{very awe-inspiring}} I didn't ask him where he had come from and he didn't tell me his name. \v{7}He told me, `Surprise!---you're going to conceive and give birth to a son!' and as for you, `Be sure that you don't drink wine or anything intoxicating, and don't eat anything unclean,' `because the young man will be a Nazirite dedicated to God from inside the womb' until the day he dies.''

\v{8}So Manoah prayed to the \divine{Lord}, ``Please, Lord, have the man of God whom you sent before\fnote{The Heb. lacks \fbib{before}} come again so he can instruct us what to do on behalf of the child who is to be born.''

\v{9}God listened to Manoah's request,\fnote{Lit. \fbib{voice}} and the angel of God came again to the woman as she was sitting out in the pasture. But her husband Manoah wasn't with her, \v{10}so the woman ran quickly to tell her husband, ``Look! The man who came the other\fnote{The Heb. lacks \fbib{other}} day appeared to me!''

\v{11}So Manoah got up quickly and followed his wife, and when he came to the man he told him, ``Are you the man who spoke to my\fnote{The Heb. lacks \fbib{my}} wife?''

He replied, ``I am.''

\v{12}Manoah asked, ``Now, when what you've said occurs, what is to be the young man's way of life and work?''

\v{13}The angel of the \divine{Lord} replied to Manoah, ``Just have your wife\fnote{Lit. \fbib{have the woman}} be careful to carry out everything that I told her. \v{14}She must not consume anything extracted from grape vines, including wine or anything intoxicating, and she must not eat anything unclean, doing everything that I commissioned her to do.''

\v{15}Manoah responded to the angel of the \divine{Lord}, ``Please, let us detain you while we prepare a young goat for you.''

\v{16}The angel of the \divine{Lord} answered Manoah, ``If you detain me, I won't be eating your food, but if you prepare a burnt offering, you'll be making a sacrifice to the \divine{Lord}.'' The angel of the \divine{Lord}\fnote{Lit. \fbib{He}} said this\fnote{The Heb. lacks \fbib{He said this}} because Manoah didn't know that he was the angel of the \divine{Lord}.

\v{17}Manoah asked the angel of the \divine{Lord}, ``What's your name, because when what you've said happens, we'll glorify\fnote{Or \fbib{honor}} you?''

\v{18}The angel of the \divine{Lord} answered him, ``Why are you asking this about my name? It's `Wonderful.'\,''\fnote{cf. Isa 9:5}

\v{19}So Manoah prepared a young goat and a grain offering and offered it on a boulder to the \divine{Lord}, who kept on performing miracles while Manoah and his wife watched continually. \v{20}When the burnt offering was engulfed in flames that sprang up from the altar toward heaven, the angel of the \divine{Lord} ascended in the flame that came from the altar. When Manoah and his wife observed this, they collapsed on their faces to the ground. \v{21}The angel of the \divine{Lord} did not appear again to Manoah or to his wife, and then Manoah knew confidently that the visitor\fnote{Lit. \fbib{that he}} had been the angel of the \divine{Lord}.

\v{22}Then Manoah told his wife, ``We're going to die for sure, because we've seen God!''

\v{23}But his wife replied to him, ``If the \divine{Lord} had intended to kill us, he wouldn't have accepted a burnt offering and a grain offering from us,\fnote{Lit. \fbib{from our hands}} he wouldn't have shown us all these things, and he wouldn't have permitted us to hear things\fnote{The Heb. lacks \fbib{things}} like this, now would he?''\fnote{The Heb. lacks \fbib{would he}}

\v{24}Later on, the woman gave birth to a son and named him Samson.\fnote{The Heb. name \fbib{Samson} means \fbib{Like the sun}} The child grew strong and the \divine{Lord} blessed him. \v{25}Then the Spirit of the \divine{Lord} began to rouse him where the tribe of Dan was encamped,\fnote{Or \fbib{him in Mahaneh-dan}} between Zorah and Eshtaol.
\labelchapt{14}
\passage{Samson's Marriage}

\chapt{14}
\v{1}A while later, Samson went down to Timnah and observed a woman in Timnah who was of Philistine origin.\fnote{Lit. \fbib{Timnah from the daughters of Philistines}} \v{2}Then he returned and told his father and mother, ``In Timnah I saw a woman of Philistine origin.''\fnote{Lit. \fbib{woman from the daughters of Philistines}} He ordered them, ``Get her for me as a wife. Now!''\fnote{Or \fbib{So get her}}

\v{3}His father and mother asked him, ``Isn't there a woman suitable\fnote{The Heb. lacks \fbib{suitable}} among the daughters of your relatives or among all of our people, since you're going to get your\fnote{Lit. \fbib{a}} wife from the uncircumcised Philistines?''

But Samson retorted to his father, ``Get her for me, since she looks fine to me.'' \v{4}Meanwhile, his father and mother did not know that she was from the \divine{Lord}, because he had been seeking a favorable opportunity concerning the Philistines, since\fnote{Lit. \fbib{and}} the Philistines were dominating Israel at that time.

\v{5}Then Samson went down in the direction of Timnah with his father and mother and arrived as far as the vineyards of Timnah. And---surprise!---a young lion came roaring at him! \v{6}The Spirit of the \divine{Lord} rushed upon him, and he ripped the lion\fnote{Lit. \fbib{ripped it}} apart as one might dissect a young goat, even though he carried nothing in his hand. But he didn't tell his father and mother what he had done. \v{7}Then he went down and talked to the woman, and she looked fine to Samson. \v{8}When he came back later to marry\fnote{Lit. \fbib{take}} her, he turned aside to observe the lion's carcass. Amazingly, there was a swarm of bees in the body of the lion, complete with honey. \v{9}So he scraped some out into his hands and went on his way, eating all the while. When he met his father and mother, he gave some\fnote{The Heb. lacks \fbib{some}} to them, and they ate it, too. But he didn't inform them that he had scraped the honey from the carcass of the lion.
\passage{Samson's Riddle}

\v{10}Later on, when his father went down to visit\fnote{The Heb. lacks \fbib{visit}} the woman, Samson threw a party there, since young men customarily did this. \v{11}When they saw him, they brought 30 companions to accompany him. \v{12}``Let me tell you a riddle,'' Samson told them. ``If you can solve it during this week-long festival, I'll give you 30 linen garments and 30 formal garments.\fnote{Or \fbib{30 changes of clothes}} \v{13}But if you don't solve it,\fnote{Lit. \fbib{don't tell me}} then you'll give me 30 linen garments and 30 formal garments.''\fnote{Or \fbib{30 changes of clothes}}

``Tell us your riddle and we'll solve it,'' they responded.

\v{14}So he told them:

\begin{poetry}
\poeml From the eater came something edible; \\
\poemll    from the strong something sweet.
\end{poetry}

For three days they couldn't solve the riddle.

\v{15}The next\fnote{Lit. \fbib{On the fourth}} day, they told Samson's wife, ``Coax your husband to explain the riddle or we'll set fire to your father's house---with you in it! You've invited us here to make us paupers, haven't you?''

\v{16}So Samson's wife cried in front of him and accused him, ``You only hate me. You don't love me. You've told a riddle to my relatives, but you haven't told the solution\fnote{Lit. \fbib{told it}} to me.''

Samson responded, ``Look, I haven't told my parents,\fnote{Lit. \fbib{my father and my mother}} either. Why\fnote{The Heb. lacks \fbib{either. Why}} should I tell you?''

\v{17}So she kept on crying in front of him for the entire seven days of the wedding party. On the seventh day he told the solution\fnote{Lit. \fbib{it}} to her because she nagged him, and then she told the solution to\fnote{The Heb. lacks \fbib{the solution to}} the riddle to her relatives.

\v{18}Then the men of the city answered him just before sunset on the seventh day:

\begin{poetry}
\poeml ``What's sweeter than honey? \\
\poemll    What's stronger than lions?''
\end{poetry}

Samson\fnote{Lit. \fbib{He}} responded,

\begin{poetry}
\poeml ``If you hadn't plowed with my heifer \\
\poemll    you wouldn't have solved my riddle.''
\end{poetry}

\v{19}Then the Spirit of the \divine{Lord} rushed upon him, and he went down to Ashkelon, killed 30 men, took their belongings, and gave the garments to those who had told him the solution to\fnote{The Heb. lacks \fbib{the solution to}} the riddle. He remained furious, left for his father's house, \v{20}and Samson's wife went to the best man at his wedding.\fnote{Lit. \fbib{wife was to an acquaintance who was his friend}; cf. Judg 15:2, 7}
\labelchapt{15}
\passage{Samson Burns the Philistine Harvest}

\chapt{15}
\v{1}A while later during the wheat harvest, Samson visited his wife, bringing along a young goat, and told his father-in-law,\fnote{The Heb. lacks \fbib{to his father-in-law}} ``I'm going into my wife's room.'' But her father wouldn't give permission for him\fnote{The Heb. lacks \fbib{permission for him}} to go.

\v{2}Her father said, ``Because I honestly thought that you hated her deeply, I gave her in marriage to your best man.\fnote{Lit. \fbib{your acquaintance}; cf. Judg 14:20; 15:7} Isn't her younger sister better than she? Please then, let her be yours instead.''

\v{3}Samson replied to them, ``This time I'll be blameless when I do something evil to the Philistines.'' \v{4}So Samson went out, caught 300 foxes, grabbed some torches,\fnote{Or \fbib{firebrands}} tied\fnote{Lit. \fbib{turned}} the foxes together in pairs at their tails,\fnote{Lit. \fbib{foxes tail to tail}} and fastened a torch\fnote{Or \fbib{firebrand}} between each pair of tails. \v{5}Then he ignited the torches, set the foxes loose into the Philistines' unharvested grain, and burned up both the harvested shocks and the standing grain, along with their vineyards and olive groves.

\v{6}Then the Philistines demanded, ``Who did this?''

Someone said, ``Samson, son-in-law of the Timnite, because his father-in-law\fnote{Lit. \fbib{because he}} took Samson's\fnote{Lit. \fbib{his}} wife and gave her to the best man at Samson's wedding.''\fnote{Lit. \fbib{to his acquaintance}; cf. Judg 14:20, 15:2} In retaliation, the Philistines came up and burned her and her father to death.

\v{7}Samson replied to them, ``Because you did this, I'm not going to stop until I get my revenge against you!'' \v{8}So he attacked them ruthlessly\fnote{Lit. \fbib{them hip and thigh}} in a massive slaughter, then left to live in the caves of Etam. \v{9}In response, the Philistines went up, encamped in the territory of\fnote{The Heb. lacks \fbib{the territory of}} Judah, and raided\fnote{Or \fbib{and spread out in}} Lehi.

\v{10}The leading\fnote{The Heb. lacks \fbib{leading}} men of Judah asked, ``Why have you invaded us?''

They replied, ``We're here to arrest Samson. Then we're going to do to him what he did to us.''

\v{11}In response, 3,000 soldiers from the tribe of Judah went down to the caves of the rock of Etam and asked Samson, ``Don't you know that the Philistines have us in their control? What have you done to us?''

``I did to them what they did to me,'' he answered.

\v{12}They responded, ``We've come here to arrest you and transfer you to the custody of the Philistines.''

Samson told them, ``Promise me that you won't kill me.''

\v{13}So they said, ``No, we won't. But we're going to tie you up securely and transfer you to their custody. But we won't kill you.'' Then they bound him with two ropes and brought him up from the caves.\fnote{Lit. \fbib{rock}}
\passage{Samson Kills 1,000 Philistines}

\v{14}When Samson\fnote{Lit. \fbib{he}} arrived at Lehi, the Philistines came shouting to meet him. Then the Spirit of the \divine{Lord} rushed upon him, so that the ropes that bound him were like flax that's been burned by fire, and his bonds dissolved. \v{15}He happened upon a jawbone from a putrefying donkey, reached out to grab it, and killed 1,000 men with it. \v{16}Then Samson declared,

\begin{poetry}
\poeml ``With a jawbone from the donkey--- \\
\poemll    here a heap, there a pair of heaps---\fnote{I.e. multiple encounters with the Philistines; MT word \fbib{heap} is a word play on the identically spelled Heb. word \fbib{donkey}} \\
\poeml with the jawbone of the donkey \\
\poemll    I've killed 1,000 men.''
\end{poetry}

\v{17}When he finally finished bragging, he discarded the jawbone and named that place ``Jawbone Heights.''\fnote{Lit. \fbib{Ramath-lehi}}

\v{18}Aferward, he became thirsty, called out to the \divine{Lord}, and told him, ``So, you provided this great deliverance at the hands\fnote{Lit. \fbib{hand}} of your servant, but now I'm to die of thirst and fall into the hands of the uncircumcised?'' \v{19}So God split a hollow place that's in Lehi, and water sprang out of it. After he had taken a drink, his strength returned, and he revived. That's why it was named ``En-hakkore,''\fnote{MT word \fbib{En-hakkore} means \fbib{The Spring of the One Who Calls Out}} which is in Lehi to this day. \v{20}Samson\fnote{Lit. \fbib{He}} governed Israel for twenty years during the Philistine domination.
\labelchapt{16}
\passage{Samson's Troubles in Gaza}

\chapt{16}
\v{1}Sometime later, Samson went to Gaza, saw a prostitute there, and went in to have sex with her. \v{2}When the Gazites were informed,\fnote{Lit. \fbib{were told}} ``Samson has come here!'' they surrounded him, intending to lay in wait for him at the city gate throughout the entire night. They kept quiet all night, telling each other,\fnote{The Heb. lacks \fbib{each other}} ``At first light, let's kill him!''

\v{3}Meanwhile, Samson had sex until midnight, then at midnight he got up, grabbed the doors, the two door posts, and the bars of the city gate, and uprooted them. He put them on his shoulders and carried them to the top of the mountain opposite Hebron.
\passage{Samson Meets Delilah}

\v{4}After this incident, he loved a woman in Sorek Valley whose name was Delilah. \v{5}The Philistine officials approached her and told her, ``Entice him to discover where his great strength is, and how we can overpower him. We intend to tie him up and torture him. We'll each pay you 1,100 silver coins.''

\v{6}So Delilah asked Samson, ``Please tell me the secret to\fnote{Lit. \fbib{the location of}} your great strength and how you may be tied up and tortured.''

\v{7}Samson replied, ``If I'm tied up with seven green cords\fnote{Or \fbib{bowstrings}} that have never been dried out, then I'll become weak and just like any other\fnote{Lit. \fbib{like one}} human being.''

\v{8}Then the Philistine leaders brought her seven green cords\fnote{Or \fbib{bowstrings}} that had never been dried, and she tied him up with them. \v{9}Meanwhile, some kidnappers were hiding inside an inner room, waiting for her signal.\fnote{Lit. \fbib{sitting for her}} So she told him, ``The Philistines are attacking you!'' But he snapped the cords\fnote{Or \fbib{bowstrings}} as one might break a burned candle wick.\fnote{Lit. \fbib{burned strand of fiber}} So his secret\fnote{Lit. \fbib{strength}} remained undiscovered.

\v{10}Some time later, Delilah told Samson, ``Look here! You've been mocking me and lying to me. Now please tell me how you can be tied up.''

\v{11}He told her, ``If I'm tied up securely with new ropes that have never been used, then I'll become weak and just like any other\fnote{Lit. \fbib{like one}} human being.''

\v{12}So Delilah grabbed some new ropes and tied him up. Then she told him, ``The Philistines are attacking you, Samson!'' because some kidnappers were hiding inside an inner room. But he snapped the ropes\fnote{Lit. \fbib{snapped them}} from his arms like thread.

\v{13}Later on, Delilah told Samson, ``You're still mocking me and telling me lies! Tell me how to tie you up!''

He answered her, ``If you weave the seven locks on my head into a loom and fasten it with a peg, then I will become weak and just like any other human being.''

\v{14}So Delilah took the seven locks on his head and wove them into the loom while he slept.\fnote{So LXX. MT omits \fbib{and fasten it with a peg, then I will become weak and just like any other human being.} \fbib{\v{14}So Delilah took the seven locks of his hair and wove them into the loom while he slept.}} She fastened his hair with a peg and then told him, ``The Philistines are attacking you, Samson!'' But he woke up from his nap and pulled the pin from the loom and the weaving.
\passage{Samson Tells Delilah His Secret}

\v{15}Some time later, she asked him, ``How can you keep saying `I love you!' when your heart isn't with me? These three times you've lied to me and haven't told me where your great strength lies.'' \v{16}She nagged him every day with this speech, pestering him until he\fnote{Lit. \fbib{until his soul}} was annoyed nearly\fnote{The Heb. lacks \fbib{nearly}} to death.

\v{17}So he finally disclosed everything. He told her,\fnote{Lit. \fbib{disclosed his entire heart to her}} ``A razor has never touched my head, because I've been a Nazirite for God before I was born.\fnote{Lit. \fbib{God from my mother's womb}} If I am shaved, then my strength will abandon me and I will become weak like every human being.''

\v{18}When Delilah realized that he had disclosed everything\fnote{Lit. \fbib{disclosed his entire heart}} to her, she sent for the Philistine officials and told them, ``Hurry up and come here at once, because he has told me everything.''\fnote{Lit. \fbib{me his entire heart}} So the Philistine officials went to her and brought their money with them. \v{19}So she enticed him to fall asleep on her lap, called for a man to shave off his seven locks of hair\fnote{The Heb. lacks \fbib{of hair}} from his head, and so began to humiliate him. Then his strength abandoned him.

\v{20}When she cried out, ``The Philistines are attacking you, Samson!'' he woke from his sleep and told himself,\fnote{The Heb. lacks \fbib{to himself}} ``I'll go out like I did at other times like this and shake myself free.'' But he didn't know that the \divine{Lord} had abandoned him.
\passage{Samson is Imprisoned by the Philistines}

\v{21}Then the Philistines grabbed him, gouged out his eyes, brought him down to Gaza, tied him up in bronze chains,\fnote{The Heb. lacks \fbib{chains}} and made him grind grain in their prison.\fnote{Lit. \fbib{in the house of captives}} \v{22}But the hair on his head began to grow again after it had been shaved off.

\v{23}Some time later, the Philistine officials got together to present a magnificent sacrifice to their god Dagon, and to throw a party, because they were claiming, ``Our god has given Samson into our control!''

\v{24}When the people saw Samson,\fnote{Lit. \fbib{him}} they praised their god, claiming:

\begin{poetry}
\poeml Our god has given our enemy into our control; \\
\poemll    the one who was destroying our land, \\
\poemlll       and who has killed many of us.
\end{poetry}

\v{25}Because they all got good and drunk,\fnote{Lit. \fbib{Because their hearts were merry}} they ordered, ``Go get Samson, so he can entertain us.'' So they called for Samson from the prison, and he entertained them while they made him stand between the pillars.
\passage{Samson Kills Himself and 3,000 Philistines}

\v{26}Then Samson told the young man who had been leading him around by the hand, ``Let me touch and feel the pillars on which this building rests, and I'll support myself against them.'' \v{27}Now the building was full of men, women, and all the Philistine officials, with about 3,000 men and women on the roof watching Samson while he was entertaining them.

\v{28}Then Samson cried out to the \divine{Lord}, ``Lord \divine{God}, please remember me. And please strengthen me this one time, God, so that I can repay the Philistines right now for my two eyes.'' \v{29}Then Samson grabbed the two middle pillars upon which the house rested and braced himself against them with one pillar in his right hand and the other in his left.

\v{30}Then Samson said, ``Let me die with the Philistines!'' He strained with all his strength until the building collapsed on the officials and every person in it. As a result, the dead whom he killed at his death were more than those whom he killed during his lifetime. \v{31}Afterwards, his brothers and his father's household servants\fnote{The Heb. lacks \fbib{servants}} came down, took him, brought him back, and buried him in his father Manoah's tomb between Zorah and Eshtaol. He had governed Israel for 20 years.
\labelchapt{17}
\passage{Micah's Descent into Idolatry}

\chapt{17}
\v{1}A man named Micah lived in the mountainous region\fnote{Or \fbib{the hill country}} of the territory of\fnote{The Heb. lacks \fbib{the territory of}} Ephraim. \v{2}He told his mother, ``Do you remember\fnote{The Heb. lacks \fbib{Do you remember}} those 1,100 silver coins that were stolen from you and about which you uttered a curse when I could hear it? Well, I have the silver. I took it.''

So she replied, ``May my son be blessed by the \divine{Lord}.''

\v{3}Her son gave back the 1,100 silver coins to his mother, and she said, ``I'm totally giving this silver---from my hand to the \divine{Lord}---so my son can make a carved image and a cast image. So I'm returning it to you.''

\v{4}When he had returned the silver to his mother, his mother took 200 of the silver coins and handed them over to a silversmith. He crafted them into a carved image and into a cast image, and they were set up\fnote{The Heb. lacks \fbib{set up}} in Micah's house. \v{5}This man Micah had his own shrine,\fnote{Lit. \fbib{own house of God}} had crafted his own ephod and some household idols,\fnote{Lit. \fbib{and teraphim}; i.e. images of pagan gods used in divination; and so throughout the book} and had installed one of his sons as a priest.

\v{6}Back in those days, Israel didn't yet have a king, so each person did whatever seemed right in his own opinion.

\v{7}A young male descendant of Levi happened to be visiting there from Bethlehem in the territory of\fnote{The Heb. lacks \fbib{the territory of}} Judah. \v{8}The man had left his city Bethlehem in Judah to live wherever he could. As he traveled along, he eventually arrived at Micah's house in the mountainous region\fnote{Or \fbib{the hill country}} of Ephraim, looking for work.

\v{9}Micah asked him, ``Where did you come from?''

He replied, ``I'm a descendant of Levi from Bethlehem in Judah, and I'm going to stay temporarily wherever I can find a place.''\fnote{The Heb. lacks \fbib{a place}}

\v{10}So Micah replied, ``Come live with me! You can be a spiritual father\fnote{The Heb. lacks \fbib{spiritual}; cf. Judg 18:19} to me, as well as a priest. I'll pay you ten silver coins a year, plus a priestly uniform\fnote{Or \fbib{a suit of clothes}} and an income.'' So the descendant of Levi moved in. \v{11}The descendant of Levi agreed to live with the man, and the young man became like one of the family.\fnote{Lit. \fbib{of his sons}} \v{12}Micah set up the descendant of Levi in ministry, and the young man became his priest while he lived in Micah's house. \v{13}As for Micah, he kept saying, ``Now I know the \divine{Lord} will make me rich, because I have a descendant of Levi for a priest!''
\labelchapt{18}
\passage{The Descendants of Dan Learn about Micah}

\chapt{18}
\v{1}Back in those days, Israel didn't have a king yet, and during that time the tribe of Dan had been seeking a territorial inheritance to live in, because up until that time no territory had been allotted to them as a possession among the tribes of Israel. \v{2}So the tribe\fnote{Lit. \fbib{sons}} of Dan sent from their families five valiant men of their number from Zorah and Eshtaol to scout the land and search through it. Following their orders, which were ``Go and scout the land,'' they came to the mountainous region\fnote{Or \fbib{the hill country}} of Ephraim, arrived at Micah's home, and stayed there.

\v{3}As they approached Micah's home, they recognized the voice of the young male descendant of Levi. They turned aside from there and spoke to him, asking him, ``Who brought you here? What work are you doing here? And what's your business here?''

\v{4}He answered, ``Micah did such and such for me, and has hired me, so I've become his priest.''

\v{5}They replied, ``Go ask God, please, about whether or not we'll be successful in this journey.''

\v{6}The priest responded to them, ``Travel in peace. The mission that you're to accomplish is from the \divine{Lord}.''

\v{7}So the five men left and went to Laish, and observed the people who were living there carefree, as Sidonians tend to do, in peace and quiet. There was no ruler in the land oppressing them for any reason. They were living far away from the Sidonians, and had no dealings with anyone.\fnote{So MT; LXX reads \fbib{with Syria}; Symmachus reads \fbib{with Aram}; cf. Judg 18:28} \v{8}When they returned to their relatives at Zorah and Eshtaol, their relatives asked them, ``What's your report?''\fnote{The Heb. lacks \fbib{report}}

\v{9}They replied, ``Let's get going and attack them. We've scouted out the land---and look!---it's a very good one. Why should we sit still? We can't wait to go back, invade, and take over the land. \v{10}When you invade, you'll meet a carefree people living in a spacious territory. God has given it into your control---it's a place that lacks nothing on this earth!'' \v{11}So 600 descendants of Dan from Zorah and Eshtaol set out for battle, armed with military weapons. \v{12}They went out and encamped at Kiriath-jearim in the territory of Judah. (That's why they call the place Mahaneh-dan to this day. It lies west of Kiriath-jearim.) \v{13}They proceeded from there to the mountainous region\fnote{Or \fbib{the hill country}} of Ephraim and arrived at Micah's house.
\passage{The Descendants of Dan Commandeer Micah's Idols}

\v{14}Then the five men who had gone to scout out the territory of Laish told their relatives, ``Are you aware that in these houses there's an ephod, some household idols,\fnote{Lit. \fbib{teraphim}; i.e. images of pagan gods used in divination} a carved image, and a cast image? You know what you need to do.'' \v{15}So they turned aside from there, went to Micah's house, and greeted him.

\v{16}While the 600 Danite soldiers, armed with military weapons, stood guard at the entrance to the gate, \v{17}the five men who had gone to scout out the land arrived, entered Micah's home\fnote{The Heb. lacks \fbib{Micah's home}} and confiscated the carved image, the ephod, the household idols,\fnote{Lit. \fbib{teraphim}; i.e. images of pagan gods used in divination} and the cast image. Meanwhile, the priest stood outside by the entrance to the gate with the 600 men armed with military weapons. \v{18}After they went into Micah's home and took possession of the carved image, the ephod, the household idols,\fnote{Lit. \fbib{and teraphim}; i.e. images of pagan gods used in divination} and the cast image, the priest challenged them. ``What are you doing?'' he asked them.

\v{19}They told him, ``Shut up and keep quiet.\fnote{Lit. \fbib{and put your hand over your mouth}} Come with us and be our spiritual\fnote{The Heb. lacks \fbib{spiritual}; cf. Judg 17:10} father and priest. It's better for you, isn't it, to be a priest to an entire\fnote{The Heb. lacks \fbib{entire}} tribe and family in Israel than to be priest to the home of one man?''

\v{20}The priest was happy to oblige,\fnote{Lit. \fbib{happy in heart}} so he took the ephod, the household idols,\fnote{Lit. \fbib{teraphim}; i.e. images of pagan gods used in divination} and the carved image and went along with the army. \v{21}Then they turned around and left, sending their little ones, their livestock, and their valuables on ahead. \v{22}When they had been gone a short distance from Micah's home, some of Micah's neighbors assembled a search party and overtook the descendants of Dan. \v{23}They yelled at the descendants of Dan, who turned around to face Micah and asked, ``What's wrong\fnote{The Heb. lacks \fbib{wrong}} with you? You've assembled together{\ldots}?''

\v{24}Micah\fnote{Lit. \fbib{He}} replied, ``You took my gods that I crafted, along with the priest, and left! What do I have left? So what's with this `What's wrong with you?'\,''

\v{25}The descendants of Dan answered him, ``You had better not talk to us about this,\fnote{The Heb. lacks \fbib{about this}} or else these bad guys here will attack you. You will lose your life, along with the lives of your whole\fnote{The Heb. lacks \fbib{whole}} household.''

\v{26}Then the descendants of Dan went on their way. Because Micah saw that they were too strong for him, he turned and went back home. \v{27}But the descendants of Dan\fnote{Lit. \fbib{But they}} took what Micah had made, along with the priest who had worked for him, and went to Laish, to a quiet and carefree people, and killed them with swords. Then they set fire to the city. \v{28}They had no one else to deliver them,\fnote{The Heb. lacks \fbib{them}} because they lived far from Sidon and had no dealings with anyone.\fnote{Cf. Judg 18:7} It lay in the valley near Beth-rehob. They rebuilt the city and lived in it. \v{29}They renamed the city Dan, after the name of their ancestor Dan, who had been born in Israel. The former name of the city was Laish. \v{30}The descendants of Dan set up the carved image, and Gershom's son Jonathan, a descendant of Manasseh, served along with his descendants as priests to the tribe of Dan until the land was taken captive. \v{31}Micah's carved image, that he himself had crafted, was in place during the entire time that God's tent was set up at Shiloh.
\labelchapt{19}
\passage{The Levite's Mistress}

\chapt{19}
\v{1}Now it happened in those days, before there was a king in Israel, that a certain male descendant of Levi, who lived in a remote part of the mountainous region\fnote{Or \fbib{the hill country}} of Ephraim, took a mistress for himself from Bethlehem in the territory of Judah. \v{2}But his mistress was sexually unfaithful to him, and then she left him to live in her father's home in Bethlehem in the territory of Judah. She had been living there for a period\fnote{Lit. \fbib{days}} of about four months \v{3}when her husband got up and went after her, intending to speak lovingly to her\fnote{Lit. \fbib{speak to her heart}} in order to win her back. He took with him his young man servant and a pair of donkeys. When she brought him into her father's house to see him, her father was happy to have met him.

\v{4}The young woman's father (that is, his father-in-law) made him stay there for three days while they ate and drank during his visit there. \v{5}On the fourth day, they got up early that morning, and the descendant of Levi\fnote{Lit. \fbib{and he}} got ready to leave. Then the young woman's father-in-law told him, ``Fortify yourself\fnote{Lit. \fbib{Fortify your heart}} by eating some food before you go.'' \v{6}So both of them sat down for a bit, ate and drank together, and the young woman's father invited the man, ``Please, enjoy yourself and spend another night.'' \v{7}The man got up, intending\fnote{The Heb. lacks \fbib{intending}} to leave, but his father-in-law urged him to spend the night there again.

\v{8}On the fifth day, he got up early in the morning, but the young woman's father-in-law told him, ``Please, fortify yourself,''\fnote{Lit. \fbib{Fortify your heart}} so they delayed until later that afternoon while both of them ate together. \v{9}When the man got up to leave with his mistress and servant, his father-in-law, the young woman's father, told him, ``Look now, evening is coming, so please spend another night. See how the daylight is fading, so spend the night here and enjoy yourself. Then tomorrow get up early and leave on your journey home.''

\v{10}Because the man was unwilling to spend the night, he got up, left, and arrived opposite Jebus (now known as Jerusalem). He had with him a pair of saddled donkeys, along with his mistress. \v{11}As they approached Jebus, the daylight was almost gone, so the servant suggested to his master, ``Come on, let's spend the night in this Jebusite city.''

\v{12}But his master replied, ``We're not going to turn aside into a city of foreigners who are not part of the Israelis. Instead, we'll go on to Gibeah.'' \v{13}He also told his servant, ``Come on,\fnote{So Codex Leningradensis.} let's go to one of these places and spend the night in Gibeah or Ramah.'' \v{14}So they continued on their way, and the sun set on them near Gibeah, which is part of Benjamin's territorial allotment.\fnote{The Heb. lacks \fbib{territorial allotment}} \v{15}They turned aside there, intending to enter Gibeah and spend the night.
\passage{The Homosexual Descendants of Benjamin in Gibeah}

After they entered the city, they had to sit down in the public square because no one would take them into their\fnote{The Heb. lacks \fbib{their}} home for the night. \v{16}Just then, an old man was coming out of the fields that evening from work. The man was from the mountainous region\fnote{Or \fbib{the hill country}} of Ephraim and had been staying in Gibeah, even though the men of that place were descendants of Benjamin. \v{17}As the old man looked up and saw the traveling man in the public square of the city, he asked, ``Now then, where are you headed? And where are you from?''

\v{18}He replied, ``We're traveling from Bethlehem in Judah to the remote part of the mountainous region\fnote{Or \fbib{the hill country}} of Ephraim, because I'm from there, and I've been visiting Bethlehem in Judah. I'm going home now, but no one will take me into his home. \v{19}Meanwhile, we also have straw and fodder for our donkeys, and bread and wine for me, for this\fnote{Lit. \fbib{your};} young woman servant, and for the young man who is with your servants. We don't need anything else.''

\v{20}The old man replied, ``Don't be alarmed. I'll take care of all your needs. Just don't spend the night in the public square.'' \v{21}So he took him into his home and fed the donkeys while they refreshed themselves and had dinner.''\fnote{Lit. \fbib{they washed their feet and ate and drank}}

\v{22}While they were enjoying themselves, all of a sudden certain ungodly men\fnote{Lit. \fbib{men of Belial}; i.e. men so wicked as to be worthy of death} who lived in the city surrounded the house, pounded on the door, and ordered the old man who owned the home, ``Bring out the man who came to visit your home so we can have sex with him.''

\v{23}The man who owned the house went out to talk to them and pleaded with them, ``No, my brothers, please don't act so wickedly. This man is my guest! Don't try to do this stupid thing. \v{24}Instead, here's my virgin daughter and my visitor's\fnote{Lit. \fbib{and his}} mistress. Please let me bring them out to you. Occupy yourselves with them, and do to them whatever you would like. But don't commit such a stupid thing against this man.''
\passage{The Men of Gibeah Rape and Murder the Mistress}

\v{25}But the men were unwilling to listen to him. So the descendant of Levi\fnote{Lit. \fbib{man}} grabbed his mistress, took her out to them, and they raped and tortured her all night until morning. Then they released her as the first daylight was beginning to appear. \v{26}As dawn was breaking, the woman approached the door of the man's home where her master was and collapsed. Eventually, full daylight came. \v{27}When her master got up that morning and opened the doors of the house to leave on his way, there was his mistress, fallen dead at the door of the house with her hands grasping the threshold.

\v{28}He spoke to her, ``Get up, and let's go.''

But there was no response. So he placed her on the donkey, mounted his own animal,\fnote{Lit. \fbib{donkey, got up}} and went home. \v{29}When he arrived home, he grabbed a knife, took hold of his mistress, cut her apart limb by limb into twelve pieces, and sent her remains\fnote{The Heb. lacks \fbib{remains}} throughout the land of Israel. \v{30}All the witnesses said, ``Nothing has happened or has been seen like this from the day the Israelis came here from the land of Egypt to this day! Think about it, get some advice about it, and then speak up about it!''
\labelchapt{20}
\passage{The Israelis Attack the Tribe of Benjamin}

\chapt{20}
\v{1}Then the entire Israeli nation---from Dan to Beer-sheba, including the territory of Gilead---came out for war. The army assembled as one united force to God at Mizpah. \v{2}The officials of the entire nation, including every tribe of Israel, took their stand in the assembly of the people of God: 400,000 foot soldiers, all of them\fnote{The Heb. lacks \fbib{all of them}} expert swordsmen. \v{3}While the descendants of Benjamin were learning that the Israelis had gone up to Mizpah, the Israelis asked, ``Somebody tell us how this evil could happen?''

\v{4}So the descendant of Levi, the husband of the murdered woman, spoke up and replied, ``I came to spend the night at Gibeah, which is part of Benjamin, along with my mistress. \v{5}But the officials of Gibeah attacked me and surrounded the house because of me. They intended to kill me, but instead they tortured my mistress to death. \v{6}So I grabbed my mistress, cut her in pieces, and sent her remains\fnote{The Heb. lacks \fbib{remains}} throughout the territory of Israel's inheritance, because they've committed a vile, stupid outrage in Israel. \v{7}So look, all you Israelis! Speak up and give us your advice!''

\v{8}Then the entire army stood up as a single unit and declared, ``Nobody's going back to his tent, and nobody's going home! \v{9}This is what we'll do to Gibeah: we're going to assemble an army by lottery. \v{10}We'll take ten men out of 100 from all of the tribes of Israel. We'll appoint 100 out of 1,000 and 1,000 out of 10,000 to supply provisions for the army. And when we reach Gibeah in the territory of Benjamin, we'll punish them for all of the stupid things that they've done in Israel.'' \v{11}That's how the army of Israel came to be gathered together to attack the city, united as a single unit.

\v{12}The tribes of Israel sent men throughout the entire tribe of Benjamin to ask them, ``What is this evil thing that has occurred among you? \v{13}Now then, hand over the men---those ungodly men,\fnote{Lit. \fbib{men of Belial}; i.e. men so wicked as to be worthy of death} and we'll execute them in order to remove this evil from Israel.''

But the descendants of Benjamin wouldn't obey the request of their own relatives, the Israelis, \v{14}so the descendants of Benjamin assembled from the cities of Gibeah to fight the Israelis in battle. \v{15}The day of the battle,\fnote{The Heb. lacks \fbib{of the battle}} the army from the descendants of Benjamin numbered 26,000 expert swordsmen from their cities, not including the inhabitants of Gibeah, who numbered 700 special forces soldiers. \v{16}Out of all these soldiers, 700 of them were left-handed---and each one could sling a stone at a hair and never miss. \v{17}But the Israeli army---not counting the tribe of Benjamin---numbered 400,000 expert swordsmen, all of them battle-hardened soldiers.\fnote{Lit. \fbib{them men of war}}
\passage{Civil War Lays Waste to the Tribe of Benjamin}

\v{18}The Israelis mounted up, traveled to Bethel, and asked God what to do.\fnote{The Heb. lacks \fbib{what to do}} They said, ``Who is to lead us in our opening attack against the descendants of Benjamin?''

The \divine{Lord} replied, ``Judah is to open the attack.''

\v{19}So the Israelis got up in the morning, encamped near Gibeah, \v{20}and the army of Israel went out to fight the tribe of Benjamin, assembling in battle array against them at Gibeah. \v{21}The descendants of Benjamin came out of Gibeah, and 22,000 soldiers of Israel fell in battle that day.

\v{22}But the army---the men of Israel---encouraged themselves and arrayed for battle again the next day in the same place where they had gathered the day before. \v{23}From there\fnote{Lit. \fbib{Then}} the Israelis went up and wept in the \divine{Lord}'s presence until evening. Then they asked the \divine{Lord}, ``Should we attack the descendants of\fnote{Lit. \fbib{of my brother}; the descendants of Benjamin personified as an individual} Benjamin again?''

The \divine{Lord} replied, ``Attack them.''\fnote{Lit. \fbib{him}; i.e. the descendants of Benjamin personified as an individual}

\v{24}So the Israelis attacked the descendants of Benjamin for a second day, \v{25}and the tribe of Benjamin went to war against them from Gibeah during that second day, and 18,000 soldiers from the Israelis---all of them expert swordsmen---fell to the ground. \v{26}All the Israelis, including its army, went up from there to Bethel and wept, remaining there in the \divine{Lord}'s presence, fasting throughout the day until dusk, when they offered burnt offerings and peace offerings in the \divine{Lord}'s presence. \v{27}The Israelis inquired of the \divine{Lord}, since the Ark of the Covenant was there\fnote{The Heb. lacks \fbib{there}} at that time \v{28}while Eleazar's son Phinehas, a descendant of Aaron, served before it in those days. They asked, ``Should we go out to war again against the descendants of our relative Benjamin, or shall we cease?''

And the \divine{Lord} answered, ``Go out, and tomorrow I will deliver them into your control.''

\v{29}So Israel set soldiers in ambush around Gibeah. \v{30}The Israelis went out against the descendants of Benjamin on the third day, arraying themselves against Gibeah as they had done previously. \v{31}They attacked the army and were drawn away from the city as they began to inflict casualties on the soldiers along the roads to Bethel and Gibeah, just as they had done the other times. About 30 soldiers from Israel fell in battle there\fnote{The Heb. lacks \fbib{fell in battle there}} and in the fields.

\v{32}Then the descendants of Benjamin told themselves,\fnote{The Heb. lacks \fbib{told themselves}} ``They're falling right in front of us, just like before!''

But the army of Israel told themselves, ``Let's draw them away by escaping to the highways from the city.'' \v{33}So the entire army of Israel moved from their location and arrayed themselves at Baal-tamer while that part of their army moved from their ambush positions from Maareh-geba. \v{34}As 10,000 of Israel's best soldiers came to fight Gibeah, the battle became fierce, but the army of Benjamin didn't know that disaster was close at hand. \v{35}The \divine{Lord} struck Benjamin in the full view of Israel. As a result, the Israelis destroyed 25,100 soldiers of Benjamin that day, all expert swordsmen.

\v{36}Then the descendants of Benjamin realized that they had been defeated. The army of Israel pretended to retreat from the army of Benjamin, knowing that they had set some soldiers in ambush near Gibeah. \v{37}The soldiers in ambush rushed out to attack Gibeah, deploying in force\fnote{The Heb. lacks \fbib{in force}} and executing the entire city with swords. \v{38}Meanwhile, the army of Israel had arranged to signal their soldiers who had been hiding in ambush by sending up a cloud of smoke from the city. \v{39}The army of Israel turned around in the battle, and the army of Benjamin began to attack and kill about 30 soldiers, thinking, ``Now we're really defeating them,\fnote{Lit. \fbib{Now they are defeated in front of us}} just like before.''

\v{40}But then the smoke began to rise from the city in a column. The army of Benjamin observed behind them that the whole city was going up in flames\fnote{The Heb. lacks \fbib{in flames}} straight into the sky! \v{41}At that point, as the army of Israel turned back to face the army of Benjamin,\fnote{The Heb. lacks \fbib{back to face the army of Benjamin}} the army of Benjamin was filled with terror, because they realized that disaster was about to overtake them. \v{42}So they turned tail and ran away from the army of Israel toward the wilderness, but they were overtaken in battle when soldiers came out from the cities to destroy them.\fnote{Lit. \fbib{them among them}} \v{43}They surrounded the army of Benjamin, pursuing them ceaselessly until they defeated them near the east-facing\fnote{Lit. \fbib{near the rising of the sun}} border of Gibeah. \v{44}That's how 18,000 men from the tribe of Benjamin fell in battle, all of whom were valiant soldiers. \v{45}The rest of them turned and ran into the wilderness in the direction of the rock of Rimmon, but 5,000 of them were killed on the highways while 2,000 of them were overtaken and killed near Gidom.

\v{46}To sum up, the soldiers from the tribe of Benjamin who died that day totaled 25,000 men, all of them expert swordsmen and valiant soldiers. \v{47}However, 600 soldiers ran into the wilderness in the direction of the rock of Rimmon, where they remained as fugitives for four months. \v{48}Meanwhile, the army of Israel went back to fight the surviving\fnote{The Heb. lacks \fbib{surviving}} descendants of Benjamin. They attacked the entire city with swords, including its cattle and everyone they could find. Then they set fire to all of the cities that they could find.
\labelchapt{21}
\passage{The Israelis Mourn the Tribe of Benjamin}

\chapt{21}
\v{1}Now the people of Israel had taken a vow in Mizpah that went like this: ``Not even one of us will give his daughter in marriage to a descendant of Benjamin!'' \v{2}So the people went to Bethel, sat before God until dusk, where they cried out loud and wept bitterly. \v{3}``Why, \divine{Lord} God of Israel,'' they asked him, ``is one tribe missing\fnote{The Heb. lacks \fbib{missing}} from Israel?''

\v{4}The next day, the people got up early, built an altar, and offered burnt offerings and peace offerings. \v{5}The Israelis asked themselves, ``Who didn't come up in our assembly in the \divine{Lord}'s presence from among all of the tribes of Israel?'' They had taken a solemn oath concerning those who didn't come up to meet with the \divine{Lord} at Mizpah that ``They will certainly be executed.''

\v{6}But the Israelis were mourning for their relatives in the tribe of Benjamin. They announced, ``One tribe has been eliminated from Israel today! \v{7}What can we do to find wives for the survivors who remain, since we've already taken an oath in the \divine{Lord}'s presence not to give them any of our daughters in marriage?''
\passage{The Israelis Attempt to Mitigate Their Disaster}

\v{8}They asked, ``What one group of the tribes of Israel didn't come up to meet the \divine{Lord} at Mizpah?'' It turned out that no one had come to the encampment from Jabesh-gilead, \v{9}since when they took a census of the assembly, not even one of the inhabitants of Jabesh-gilead was in attendance. \v{10}So the congregation sent out 12,000 of their valiant soldiers, issuing these orders to them: ``Go and attack the inhabitants of Jabesh-gilead with swords, including the women and little ones. \v{11}You're to completely destroy every man and every married woman.''\fnote{Lit. \fbib{woman who has had sexual relations with a man}}

\v{12}They discovered among the inhabitants of Jabesh-gilead 400 young virgins who hadn't had sex with a man, and they brought them to the encampment at Shiloh in the territory of Canaan. \v{13}Then the entire congregation sent for the surviving\fnote{The Heb. lacks \fbib{surviving}} descendants of Benjamin who were living at the rock of Rimmon and assured them that their intentions toward them were peaceful.\fnote{Lit. \fbib{and proclaimed peace}} \v{14}So the survivors of the tribe of Benjamin\fnote{The Heb. lacks \fbib{the survivors of the tribe of}} returned at that time, and the Israelis\fnote{Lit. \fbib{and they}} gave them the women whom they had kept alive from the raid on\fnote{Lit. \fbib{the women of}} Jabesh-gilead. Even so, there weren't enough for them.

\v{15}The people felt sorry for the tribe of Benjamin because the \divine{Lord} had broken one of the tribes of Israel. \v{16}So the elders of the congregation asked, ``What will we do to obtain wives for the survivors, since the women of Benjamin have been devastated?'' \v{17}They continued, ``Let's make sure that there's an inheritance for the survivors of the tribe of Benjamin, so that a tribe won't be blotted out from Israel. \v{18}But we can't give them wives from our own daughters, since we've\fnote{Lit. \fbib{since the Israelis had}} taken this vow: `May the \divine{Lord} curse\fnote{Lit. \fbib{vow: `Cursed be}} anyone who gives his daughter as\fnote{The Heb. lacks \fbib{his daughter as}} a wife to the tribe of Benjamin!'\,''

\v{19}So they concluded, ``Look, there's a festival to the \divine{Lord} every year in Shiloh on the north side of Bethel, south of Lebonah and on the east side of the highway that runs from Bethel to Shechem{\ldots}'' \v{20}So they told the descendants of Benjamin, ``Go and hide in the vineyards. \v{21}Watch when the unmarried women\fnote{Lit. \fbib{the daughters}} from Shiloh come out to participate in the dances. Then come out of the vineyards and each of you grab a wife from the unmarried women\fnote{Lit. \fbib{the daughters}} from Shiloh. Then go back home to the territory of Benjamin. \v{22}If their fathers or brothers come complaining to us, we'll tell them `Be generous! Give them to us voluntarily, because we didn't take anyone to be a wife for the men of the tribe of Benjamin\fnote{The Heb. lacks \fbib{of the tribe of Benjamin}} as a result of the battle. And you haven't incurred guilt by giving your daughters to them.'\,''

\v{23}So the descendants of Benjamin did all of this: they chose and carried away just enough wives from those who danced to meet the number needed, then they left to return to their inheritance, to rebuild their cities, and to live there. \v{24}The Israelis left there at that time, each man to his tribe and family, and each of them went down from there to his territorial allotment.

\v{25}Back in those days, Israel didn't yet have a king, so each person did whatever seemed right in his own opinion.

\bookheader{Ruth}
\labelbook{Ruth}

\bookpretitle{The Book of}
\booktitle{Ruth}

\labelchapt{1}
\passage{Naomi's Family}

\chapt{1}
\v{1}Now there came a time of famine while judges were ruling in the land of Israel.\fnote{The Heb. lacks \fbib{of Israel}} A man from Bethlehem of Judah, his wife, and his two sons left to live in the country of Moab. \v{2}The man's name was Elimelech, his wife's name was Naomi, and their two sons were named Mahlon and Chilion---Ephrathites from Bethlehem of Judah. They journeyed to the country of Moab and lived there for some time.\fnote{The Heb. lacks \fbib{for some time}} \v{3}Then Naomi's husband Elimelech died, and she was left with her two sons. \v{4}Each of her sons\fnote{Lit. \fbib{They}} married Moabite women: one named Orpah and the other named Ruth. After they lived there about ten years, \v{5}both Mahlon and Chilion died, leaving Naomi\fnote{Lit. \fbib{the woman}} alone with neither her husband nor her two sons.
\passage{Naomi Returns to Judah}

\v{6}She and her daughters-in-law prepared to return from the country of Moab, because she had heard while living there\fnote{Lit. \fbib{living in the country of Moab}} how the \divine{Lord} had come to the aid of his people, giving them relief.\fnote{Lit. \fbib{bread} or \fbib{food}} \v{7}So she left the place where she had been, along with her two daughters-in-law, and they traveled along the return road to the land of Judah. \v{8}But along the way,\fnote{The Heb. lacks \fbib{along the way}} Naomi told her two daughters-in-law, ``Each of you go back home. Return to your mother's house. May the \divine{Lord} show his gracious love to you, as you have shown me and our loved ones who have died.\fnote{Lit. \fbib{and the dead}} \v{9}May the \divine{Lord} grant each of you security in your new\fnote{The Heb. lacks \fbib{new}} husbands' households.'' Then she kissed them good-bye,\fnote{The Heb. lacks \fbib{good-bye}} and they cried loudly.

\v{10}They both replied to her, ``No! We'll go back with you to your people.''

\v{11}But Naomi responded, ``Go back, my daughters. Why go with me? Are there still sons to be born to me\fnote{Lit. \fbib{sons in my womb}} as future husbands for you? \v{12}So go on back, my daughters! Be on your way! I'm too old to remarry.\fnote{Lit. \fbib{to have a husband}} If I were to say that I'm hoping for a husband tonight and then also bore sons this very night,\fnote{The Heb. lacks \fbib{this very night}} \v{13}would you wait for them until they were grown? Would you refrain from marriage for them? No, my daughters! I'm more deeply grieved than you, because\fnote{Lit. \fbib{because the hand of}} the \divine{Lord} is working against me!''
\passage{Ruth Remains with Naomi}

\v{14}They began to cry loudly again. So Orpah kissed her mother-in-law good-bye,\fnote{The Heb. lacks \fbib{good-bye}} but Ruth remained with her. \v{15}Naomi told Ruth,\fnote{The Heb. lacks \fbib{to Ruth}} ``Look, your sister-in-law has returned to her people and to her gods. Follow your sister-in-law!''

\v{16}But Ruth answered, ``Stop urging me to abandon you and to turn back from following you. Because wherever you go, I'll go. Wherever you live, I'll live. Your people will be my people, and your God, my God. \v{17}Where you die, I'll die and be buried. May the \divine{Lord} do this to me---and more---if anything\fnote{The Heb. lacks \fbib{anything}} except death comes between you and me.''

\v{18}When Naomi\fnote{The Heb. lacks \fbib{Naomi}} observed Ruth's\fnote{Lit. \fbib{her}} determination to travel with her, she ended the conversation. \v{19}So they continued on until they reached Bethlehem.
\passage{Naomi and Ruth Arrive in Bethlehem}

Now when the two of them arrived in Bethlehem, the entire town got excited at the news of their arrival\fnote{Lit. \fbib{at them}} and they asked one another, ``Can this be Naomi?''

\v{20}But Naomi replied, ``Don't call me `Naomi'!\fnote{I.e. \fbib{pleasant}} Call me `Mara'!\fnote{I.e. \fbib{bitter}} That's because the Almighty has dealt bitterly with me. \v{21}I left here full, but the \divine{Lord} brought me back empty. So why call me `Naomi'? After all, the \divine{Lord} is against me, and the Almighty has broken\fnote{Or \fbib{has done evil toward}} me.''

\v{22}So Naomi returned to Bethlehem\fnote{The Heb. lacks \fbib{to Bethlehem}} from the country of Moab, along with her daughter-in-law Ruth the Moabite woman. And they arrived in Bethlehem at the beginning of the barley harvest.
\labelchapt{2}
\passage{Boaz Meets Ruth}

\chapt{2}
\v{1}Naomi had a close relative of her late\fnote{The Heb. lacks \fbib{late}} husband, a man of considerable wealth from the family of Elimelech. His name was Boaz.

\v{2}Ruth the Moabite told Naomi, ``Please allow me to go out to the fields and glean grain behind anyone who shows me kindness.''

So Naomi replied, ``Go ahead, my daughter.''

\v{3}So she went out, proceeded to the field, and gleaned behind the harvesters. And it happened that she came to the portion of land belonging to Boaz, of the family of Elimelech.

\v{4}Now when Boaz arrived from Bethlehem, he told the harvesters, ``The \divine{Lord} be with you.''

``May the \divine{Lord} bless you!'' they replied.

\v{5}At this point, Boaz asked the foreman of\fnote{Or \fbib{the young man over}} his harvesters, ``To whom does this young woman belong?''

\v{6}The foreman of\fnote{Or \fbib{The young man over}} the harvesters answered, ``She is the Moabite who came back with Naomi from the country of Moab. \v{7}She asked us, `Please allow me to glean what's left of the grain behind the harvesters.' So she came out and has continued working\fnote{The Heb. lacks \fbib{working}} from dawn until now, except for a short time in a shelter.''
\passage{Boaz Shows Kindness to Ruth}

\v{8}Boaz then addressed Ruth: ``Listen, my daughter!\fnote{Lit. \fbib{Will you not listen, my daughter?}} Don't glean in any other field. Don't even leave this one, and be sure to stay close to my women servants. \v{9}Keep your eyes on the field where they are harvesting, and follow them. I've ordered my young men not to bother\fnote{Or \fbib{touch}} you, haven't I? And when you are thirsty, drink from the water vessels that the young men have filled.''

\v{10}At this she fell prostrate, bowing low to the ground, and asked him, ``Why is it that you're showing me kindness by noticing me, since I'm a foreigner?''

\v{11}Boaz answered her, ``It has been clearly disclosed to me all that you have done for your mother-in-law following the death of your husband---how you abandoned your father, your mother, and your own land, and came to a people you did not previously know. \v{12}May the \divine{Lord} repay you for your work, and may a full reward be given you from the \divine{Lord} God of Israel, under whose wings\fnote{Or \fbib{garment}; cf. 3:9} you have come for refuge.''

\v{13}She responded, ``May I continue to find favor in your sight, sir, since you've been comforting me and you have spoken graciously to\fnote{Lit. \fbib{spoken to the heart of}} your servant, even though I am not one of your servants.''

\v{14}At lunchtime, Boaz invited her, ``Come on over, have some food, and dip your bread in our oil and\fnote{The Heb. lacks \fbib{oil and}} vinegar.'' So she sat down beside the harvesters, and he handed her some roasted grain, which she ate until she was satisfied. She kept what was left over.
\passage{Boaz the Benefactor}

\v{15}After she had left to glean, Boaz commanded his servants,\fnote{Or \fbib{young men}} ``Allow her to glean also among the cut sheaves, and don't taunt her. \v{16}One other thing\fnote{Lit. \fbib{her}. \fbib{\v{16}Also}}---drop some handfuls\fnote{Or \fbib{portions}} deliberately, leaving them for her so she can gather it. And don't bother her.'' \v{17}So Ruth\fnote{Lit. \fbib{she}} gathered grain out in the field until dusk, and then threshed what she had gathered---about a week's supply\fnote{Lit. \fbib{an ephah}; i.e. enough to support Naomi and Ruth for at least five or six days} of barley. \v{18}She picked up her grain\fnote{The Heb. lacks \fbib{her grain}} and went back to town.

Her mother-in-law noticed how much Ruth\fnote{Lit. \fbib{she}} had gleaned and had brought back from what was left over from her lunch. \v{19}So her mother-in-law quizzed her, ``Where did you glean today? Where, precisely, did you work? May the one who took notice of you be blessed.''

So Ruth told her mother-in-law with whom she had worked. She said, ``The man's name with whom I worked today is Boaz.''

\v{20}Naomi replied, ``May the one who hasn't abandoned his gracious love to the living or to the dead be blessed by the \divine{Lord}.'' Naomi added, ``This man is closely related to us, our related redeemer,\fnote{I.e. a close male relative responsible to redeem inheritances (Lev. 25:25), to free relatives from indentured servitude (Lev. 25:47-55), to avenge deaths (Deut. 19:1-13), and to financially support, care for, and (in certain limited cases) to marry a widow related to him (Deut. 25:5-10); and so throughout the book} as a matter of fact!''

\v{21}Then Ruth the Moabite woman added, ``He also told me `Stay close to my young men until they have completed my entire harvest.'\,''

\v{22}Naomi responded to her daughter-in-law Ruth, ``It is prudent, my daughter, for you to go out with his women servants, so someone won't attack you in another field.'' \v{23}So Ruth\fnote{Lit. \fbib{she}} continued to stay close to the young women who worked for Boaz, gathering grain until both the barley and wheat harvests were complete, all the while living with her mother-in-law.
\labelchapt{3}
\passage{Naomi Offers to Find a Husband for Ruth}

\chapt{3}
\v{1}Ruth's\fnote{Lit. \fbib{her}} mother-in-law Naomi, told her, ``My daughter, should I not make inquiries about your financial security,\fnote{Lit. \fbib{about a resting place}} so you'll be better off in life?\fnote{Lit. \fbib{so your life may go well}} \v{2}Isn't Boaz one of our close relatives? You've been associating with his women servants lately. Look, he'll be winnowing barley at the threshing floor tonight. \v{3}So get cleaned up, put on some perfume, dress up, and go to the threshing floor, but don't let him see you\fnote{Lit. \fbib{but do not make yourself known to the man}} until after he's finished eating and drinking. \v{4}When he lies down, be sure to notice where he is resting. Then go over, uncover his feet, and lie down. He'll tell you what to do.''

\v{5}Ruth replied, ``I'll do everything you've said.'' \v{6}So she went out to the threshing floor and did precisely what her mother-in-law had advised.
\passage{Ruth's Meeting with Boaz}

\v{7}After Boaz had finished eating and drinking to his heart's content, he went over and lay down next to the pile of threshed grain. Ruth\fnote{Lit. \fbib{She}} came in quietly, uncovered his feet, and lay down. \v{8}In the middle of the night, Boaz\fnote{Lit. \fbib{the man}} was startled awake and turned over in surprise to see a woman lying at his feet.

\v{9}He asked her, ``Who are you?''

She answered, ``I'm only Ruth, your servant. Spread the edge\fnote{Or \fbib{wing}; cf. 2:12} of your garment over your servant, because you are my related redeemer.''

\v{10}He replied, ``May you be blessed by the \divine{Lord}, my daughter. You've added to the gracious love you've already demonstrated\fnote{Lit. \fbib{You've showed more gracious love in the latter end than in the beginning}} by not pursuing younger men, whether rich or poor. \v{11}Don't be afraid, my daughter. I'll do for you everything that you have asked, since all of my people in town are aware that you're a virtuous\fnote{The Heb. word for \fbib{virtuous} is identical to the word for \fbib{of considerable wealth} in 2:1} woman. \v{12}It's true that I'm your related redeemer, but there is another related redeemer even closer than I. \v{13}Stay the night, and if he fulfills his duty as your related redeemer in the morning, that will be acceptable. But if he isn't inclined to do so,\fnote{I.e. act as related redeemer} then, as the \divine{Lord} lives, I will act as your related redeemer myself. So lie down until morning.''

\v{14}So she lay down at his feet until dawn approached, then got up while it was still difficult for anyone to be recognized. Then he told her, ``It shouldn't be known that a woman has come to the threshing floor.'' \v{15}So he said, ``Take your cloak and hold it out.'' She did so, and he measured out six units\fnote{The Heb. lacks \fbib{units}} of barley and placed them in a sack\fnote{The Heb. lacks \fbib{in a sack}} on her. Then she left for town.
\passage{Naomi's Response to Ruth}

\v{16}When Ruth\fnote{Lit. \fbib{she}} returned to her mother-in-law, Naomi\fnote{Lit. \fbib{she}} asked her, ``How did it go, my daughter?''

Then she related everything that the man had done for her. \v{17}Ruth\fnote{Lit. \fbib{She}} also said, ``He gave me these six units\fnote{The Heb. lacks \fbib{units}} of barley and told me, `Don't go back to your mother-in-law empty-handed.'\,''\fnote{Lit. \fbib{in vain}}

\v{18}Naomi\fnote{Lit. \fbib{She}} replied, ``Be patient, my daughter, until you learn how this works out, because the man won't rest until he finishes everything today.''
\labelchapt{4}
\passage{Boaz Acts to Fulfill His Responsibilities}

\chapt{4}
\v{1}Meanwhile, Boaz approached the city gate\fnote{I.e. the place of public business} and sat down there. Just then, the very same related redeemer whom Boaz had mentioned came by, so Boaz\fnote{Lit. \fbib{he}} called out to him, ``Come over and sit down here, my friend!'' So the man came over and sat down.

\v{2}Boaz\fnote{Lit. \fbib{He}} selected ten of the town elders and spoke to them, ``Sit down here.'' So they sat down \v{3}and Boaz\fnote{Lit. \fbib{he}} addressed the related redeemer directly: ``A portion of a field belonging to our relative Elimelech is up for sale by Naomi, who recently returned from the country of Moab. \v{4}So I thought to myself I ought to tell you that you must make a public purchase of this before the town residents and the elders of my people. So if you intend to act as the related redeemer, then do so.\fnote{Lit. \fbib{then act as related redeemer}} But if not, let me know, because except for you---and I after you---there is no one to fulfill the duties of a related redeemer.''

The man responded, ``I will act as related redeemer.''
\passage{A Complication Arises and is Resolved}

\v{5}Boaz continued, ``On the very day you buy the field from Naomi,\fnote{Lit. \fbib{from the hand of Naomi}} you're also ``buying'' Ruth the Moabite woman, the wife of her dead husband,\fnote{The Heb. lacks \fbib{husband}} so the family name may be continued\fnote{Or \fbib{raised up}} as an inheritance.''

\v{6}At this, the nearer related redeemer replied, ``Then I am unable to act as related redeemer, because that would complicate my own inheritance. You act instead as the related redeemer, because I cannot do so.''\fnote{Lit. \fbib{cannot act as related redeemer}}

\v{7}During Israel's earlier history,\fnote{Or \fbib{years}} all things concerning redeeming or changing inheritances were confirmed by a man taking off his sandal and giving it to the other party,\fnote{Or \fbib{neighbor}; cf. Deut 25:9} thereby creating a public\fnote{The Heb. lacks \fbib{public}} record in Israel. \v{8}So when the nearer related redeemer told Boaz, ``Make the purchase yourself,'' he then took off his sandal.
\passage{Boaz's Public Commitment}

\v{9}At this, Boaz addressed the elders and all of the people: ``You all are witnesses today that I hereby redeem everything from Naomi that belonged to Elimelech, including what belonged to Chilion and Mahlon, \v{10}along with Mahlon's wife Ruth the Moabite woman. I will marry her to continue the family name as an inheritance, so that the name of the deceased does not disappear from among his relatives, nor from the public record.\fnote{Lit. \fbib{the gate of his place}} You are all witnesses today!''

\v{11}Then all of the assembled people,\fnote{Lit. \fbib{the people in the gate}} including the elders who were there, said, ``We are witnesses! May the \divine{Lord} make this woman who enters your house like Rachel and Leah, who together established the house of Israel. May you prosper in Ephrathah, and may you excel in Bethlehem! \v{12}Moreover, may your house be like the house of Perez, whom Tamar bore for Judah, from the offspring that the \divine{Lord} gives you from this young woman.''
\passage{The Marriage of Boaz and Ruth}

\v{13}So Boaz took Ruth to be his wife, and when he had marital relations with her, the \divine{Lord} made her conceive, and she bore a son. \v{14}Then the women of Bethlehem\fnote{The Heb. lacks \fbib{of Bethlehem}} told Naomi, ``May the \divine{Lord} be blessed,\fnote{Or \fbib{Blessed be the \divine{Lord}}} who has not left you today without a related redeemer. May his name become famous throughout Israel! \v{15}And he will restore your life for you and will support you in your old age, because your daughter-in-law, who loves you and who has borne him, is better for you than seven sons!''

\v{16}Naomi took care of the child, taking him to her breast and becoming his nurse. \v{17}So her women neighbors gave the child a nickname, which is ``Naomi has a son!'' They named him Obed. He became the father of Jesse, who was the father of David.
\passage{The Ancestry of David}

\v{18}This is the genealogy of Perez: Perez fathered Hezron, \v{19}Hezron fathered Ram, and Ram fathered Amminadab. \v{20}Amminadab fathered Nahshon, and Nahshon fathered Salmon. \v{21}Salmon fathered Boaz, and Boaz fathered Obed. \v{22}Then Obed fathered Jesse, who fathered David.

\bookheader{1 Samuel}
\labelbook{1Sam}

\bookpretitle{The Book of}
\booktitle{First Samuel}

\labelchapt{1}
\passage{The Birth of Samuel}

\chapt{1}
\v{1}A certain man lived in Ramathaim-zophim, which is in the hill country of Ephraim. He was Jeroham's son Elkanah, the grandson of Elihu and grandson of Tohu, who was the son of Zuph, an Ephraimite. \v{2}He had two wives; the name of one was Hannah and the name of the other was Peninnah. Peninnah had children, but Hannah had no children. \v{3}That man would go up from his town each year to worship and sacrifice to the \divine{Lord} of the Heavenly Armies at Shiloh, where Eli's two sons Hophni and Phineas served as priests of the \divine{Lord}. \v{4}On the day when Elkanah offered sacrifices, he would give portions to his wife Peninnah and to all her sons and daughters, \v{5}but he would give twice as much to Hannah because he loved her.

Now the \divine{Lord} had closed her womb. \v{6}Her rival would provoke her severely so that she complained loudly\fnote{Or \fbib{severely to irritate her}} because the \divine{Lord} had closed her womb. \v{7}Elkanah\fnote{Lit. \fbib{He}} would do this year after year, as often as Hannah\fnote{Lit. \fbib{she}} went up to the house of the \divine{Lord}. Likewise, Peninnah\fnote{Lit. \fbib{she}} would provoke her, and Hannah\fnote{Lit. \fbib{she}} would cry and would not eat. \v{8}Elkanah her husband told her, ``Hannah, why are you crying and why don't you eat? Why are you upset?\fnote{Lit. \fbib{is your heart troubled}} Am I not better to you than ten sons?''

\v{9}Hannah got up after she had finished eating and drinking in Shiloh. Now Eli the priest was sitting on the chair by the doorpost of the tent\fnote{Or \fbib{temple}} of the \divine{Lord}. \v{10}Deeply distressed, she prayed to the \divine{Lord} and wept bitterly. \v{11}Hannah\fnote{Lit. \fbib{She}} made a vow: ``\divine{Lord} of the Heavenly Armies, if you just look at the misery of your maid servant, remember me, and don't forget your maid servant. If you give your maid servant a son,\fnote{Lit. \fbib{seed of men}} then I'll give him to the \divine{Lord}\fnote{So MT; LXX reads \fbib{him to your presence as a gift until the day of his death}} for all the days of his life,\fnote{So MT; LXX and DSS read \fbib{life, and wine or strong drink he won't drink}; Cf. 4QSam\textsuperscript{a}} and a razor is never to touch\fnote{Or \fbib{cross}; MT reads \fbib{to go over;} 4QSam\textsuperscript{a} reads \fbib{to come upon};} his head.''

\v{12}As she continued to pray in the \divine{Lord}'s presence, Eli was watching her mouth. \v{13}Hannah\fnote{So MT; 4QSam\textsuperscript{a} LXX read \fbib{She}} was praying inwardly.\fnote{Lit. \fbib{in her heart}} Her lips were quivering, and her voice could not be heard. So Eli thought she was drunk. \v{14}Eli told her, ``How long will you stay drunk? Put away your wine!''

\v{15}``No, sir!''\fnote{Lit. \fbib{No, my lord}} Hannah replied. ``I'm a deeply troubled\fnote{Lit. \fbib{harsh of spirit}; i.e. one whose life is hard} woman. I've drunk neither wine nor beer. I've been pouring out my soul in the \divine{Lord}'s presence. \v{16}Don't consider your maid servant a worthless woman. Rather, all this time I've been speaking because I'm very anxious and distressed.''

\v{17}``Go in peace,'' Eli answered. ``May the God of Israel grant the request you have asked of him.''

\v{18}She said, ``Let your servant\fnote{Lit. \fbib{maid servant}} find favor in your eyes.'' Then she\fnote{Lit. \fbib{Then the woman}} went on her way and ate, and her face was no longer sad.\fnote{So LXX; The meaning of MT is uncertain.}

\v{19}They got up early the next morning and worshipped in the \divine{Lord}'s presence, and then they returned and came to their house at Ramah. Elkanah had marital relations with\fnote{Lit. \fbib{Elkanah knew}} his wife Hannah, and the \divine{Lord} remembered her. \v{20}By the time of the next year's sacrifice,\fnote{Or \fbib{in due time}} Hannah had become pregnant and had borne a son. She named him Samuel\fnote{The Heb. name \fbib{Samuel} means \fbib{God has heard}} because she said,\fnote{The Heb. lacks \fbib{she said}} ``I asked the \divine{Lord} for him.''
\passage{Hannah Dedicates Samuel to the \divine{Lord}}

\v{21}Then Elkanah went up with all his family to offer the yearly sacrifice to the \divine{Lord} and pay his vow. \v{22}Hannah did not go up because she had told her husband, ``As soon as the child is weaned, I'll take him to appear in the \divine{Lord}'s presence and remain there\fnote{So MT; 4QSam\textsuperscript{a} reads \fbib{before there}} forever.\fnote{So MT LXX; 4QSam\textsuperscript{a} reads \fbib{there, and I'll dedicate him as a Nazirite forever, all his days}}

\v{23}``Do what you want,''\fnote{Lit. \fbib{is good in your eye}} Elkanah told her. ``Stay until you have weaned him, only may the \divine{Lord} bring about what you've said.''\fnote{So LXX and DSS; lit. \fbib{about what comes out of your mouth}; MT reads \fbib{about his word}} So Hannah\fnote{Lit. \fbib{So the woman}} stayed and nursed her son until she had weaned him. \v{24}Then, when she had weaned him, she brought him\fnote{So 4QSam\textsuperscript{a}; MT employs a Heb. suffix} up with her to Shiloh,\fnote{The Heb. lacks \fbib{to Shiloh;} LXX and DSS add \fbib{to Shiloh}} along with a three-year-old bull,\fnote{4QSam\textsuperscript{a} LXX read \fbib{three bulls}; meaning of MT is uncertain.} an ephah\fnote{I.e. about a half-bushel; an ephah was a measure of dry capacity equal to about one half of a bushel} of flour, and a skin of wine. She brought him to the house of the \divine{Lord} at Shiloh, and the boy\fnote{So MT LXX; 4QSam\textsuperscript{a} reads \fbib{child}} was young.\fnote{The meaning of MT is uncertain; lit. \fbib{the lad was a lad}; LXX and DSS read \fbib{and the boy was with them.}} \v{25}They slaughtered the bull and brought the boy\fnote{So MT; Cf. 4QSam\textsuperscript{a}; LXX reads \fbib{was with them. When they brought him into the presence of the Lord, his father slaughtered the sacrifice, as he did annually for the Lord. Then he brought the child,} \fbib{\v{25}and he slaughtered the calf. And Hannah his mother brought him}} to Eli.

\v{26}Hannah\fnote{Lit. \fbib{she}} said, ``Sir,\fnote{Lit. \fbib{My lord}} as surely as you are alive, I'm the woman who stood before you here praying to the \divine{Lord}. \v{27}I prayed for this boy, and the \divine{Lord} granted me the request I asked of him. \v{28}Now\fnote{Lit. \fbib{Also}} I'm dedicating\fnote{Or \fbib{lending}} him to the \divine{Lord}, and as long as he lives,\fnote{So LXX and other Heb. MSS; MT, \fbib{is}} he will be dedicated\fnote{Or \fbib{lent}} to the \divine{Lord}.'' Then they worshipped\fnote{So 4QSam\textsuperscript{a}; MT LXX\textsuperscript{mss} read \fbib{So he worshipped}; LXX lacks \fbib{Then they worshipped the Lord there}} the \divine{Lord} there.
\labelchapt{2}
\passage{Hannah's Thanksgiving Psalm}

\chapt{2}
\v{1}Then Hannah prayed:

\begin{poetry}
\poeml ``My heart exults in the \divine{Lord}; \\
\poemll    my strength\fnote{Lit. \fbib{horn}; i.e. a symbol of strength and well-being} is increased by the \divine{Lord}. \\
\poeml I will open my mouth to speak\fnote{The Heb. lacks \fbib{to speak}} against my enemies, \\
\poemll    because I rejoice in your deliverance. \\
\poeml \v{2}Indeed,\fnote{So 4QSam\textsuperscript{a} LXX; MT lacks \fbib{Indeed}} there is no one holy like the \divine{Lord}, \\
\poemll    indeed, there is no one besides you, \\
\poeml there is no rock like our God. \\
\poeml \v{3}Don't continue to talk proudly, \\
\poemll    and don't speak arrogantly, \\
\poeml for the \divine{Lord} is a God of knowledge, \\
\poemll    and by him actions are weighed. \\
\poeml \v{4}The bows of warriors are shattered,\fnote{Lit. \fbib{are shattering}; 4QSam\textsuperscript{a} reads \fbib{warriors shatter}} \\
\poemll    but those who stumble are equipped with\fnote{Lit. \fbib{stumble gird on}} strength. \\
\poeml \v{5}Those who had an abundance of bread \\
\poemll    now hire themselves out, \\
\poeml and those who were hungry \\
\poemll    hunger no more.\fnote{Lit. \fbib{cease}} \\
\poeml While the barren woman gives birth to seven children,\fnote{The Heb. lacks \fbib{children}} \\
\poemll    she who had many children languishes. \\
\poeml \v{6}The \divine{Lord} kills and gives life, \\
\poemll    he brings people down to where the dead are\fnote{Lit. \fbib{Sheol}; i.e. the realm of the dead} \\
\poemlll       and he raises them up. \\
\poeml \v{7}The \divine{Lord} makes people poor \\
\poemll    and he makes people rich, \\
\poeml he brings them low, \\
\poemll    and he also exalts them. \\
\poeml \v{8}He raises the poor up from the dust, \\
\poemll    he lifts up the needy from the trash heap \\
\poeml to make them sit with princes \\
\poemll    and inherit a seat of honor. \\
\poeml Indeed the pillars of the earth belong to the \divine{Lord}, \\
\poemll    and he has set the world on them. \\
\poeml \v{9}He guards the steps\fnote{Lit. \fbib{feet}; cf. 4QSam\textsuperscript{a}} of his faithful ones, \\
\poemll    while the wicked are made silent\fnote{Or \fbib{are destroyed}} in darkness. \\
\poeml He grants the request of the one who prays.\fnote{So 4QSam\textsuperscript{a} LXX; MT lacks this line} \\
\poemll    He blesses the year of the righteous.\fnote{So 4QSam\textsuperscript{a} LXX; MT lacks this line} \\
\poeml Indeed it's not by strength that a person prevails. \\
\poeml \v{10}The \divine{Lord} will shatter his enemies\fnote{So 4QSam\textsuperscript{a} LXX; MT lacks \fbib{his enemies}} \\
\poemll    ---those who contend against him. \\
\poeml Who is holy?\fnote{So 4QSam\textsuperscript{a}; MT lacks this line; LXX reads \fbib{The Lord is holy}} \\
\poemll    The one who will thunder\fnote{So MT; or \fbib{He thundered}; 4QSam\textsuperscript{a} reads \fbib{And he thundered}} against them in the heavens. \\
\poeml The \divine{Lord} will judge the ends of the earth, \\
\poemll    he will give strength to his king, \\
\poemlll       and he will increase the strength of His anointed one.''
\end{poetry}

\v{11}Then Elkanah went to his house at Ramah, while the boy was ministering to the \divine{Lord} in the presence of Eli the priest.
\passage{Eli's Wicked Sons}

\v{12}Now the sons of Eli were worthless men who did not know the \divine{Lord}. \v{13}The custom of the priests with the people was that whenever a person offered a sacrifice, a servant\fnote{Lit. \fbib{lad}} of the priest would come with a three pronged fork in his hand while the meat was boiling, and\fnote{So 4QSam\textsuperscript{a}; 4QSam\textsuperscript{a} reads \fbib{and take}; MT LXX read \fbib{when anyone offered a sacrifice, the priest's servant came while the meat was boiling with a three-pronged fork in his hand.}} \v{14}he would stick it into the boiler or pot,\fnote{So 4QSam\textsuperscript{a}; MT reads \fbib{the pan, kettle, caldron, or pot}; LXX reads \fbib{the caldron, kettle, or pot}} and take everything\fnote{So 4QSam\textsuperscript{a}; MT LXX read \fbib{the priest took for himself}} the fork brought up---that is, the priest would take it for himself. This is what they were supposed to do with all the Israelis who came there to Shiloh. \v{15}But even before they burned the fat, the servant of the priest would come and say to the person offering the sacrifice, ``Give me meat to roast for the priest. He won't accept boiled meat from you, but only raw.''

\v{16}If the man told him\fnote{So MT; 4QSam\textsuperscript{a} reads \fbib{man answered and told the priest's servant}; LXX reads \fbib{man who was making the sacrifice told him}}, ``They must surely\fnote{So MT; 4QSam\textsuperscript{a} reads \fbib{Let the priest} LXX reads \fbib{Let him}} burn up the fat first, and then take for yourself whatever\fnote{So MT; 4QSam\textsuperscript{a} LXX read \fbib{everything}} you desire,'' the servant would say, ``No,\fnote{Lit. \fbib{Rather}} give it now, and if you don't,\fnote{So MT LXX; 4QSam\textsuperscript{a} \fbib{now, or}} I'll take it by force!''\fnote{So MT LXX; 4QSam\textsuperscript{a} reads \fbib{now, or I'll take the ram by force to give him the meat}} \v{17}By doing this, the sin of the young men was very serious in the \divine{Lord}'s sight because the men\fnote{So MT; 4QSam\textsuperscript{a} LXX read \fbib{because they}} despised the \divine{Lord}'s offering.
\passage{Samuel at Shiloh}

\v{18}Now Samuel was ministering in the \divine{Lord}'s presence, as a boy wearing a linen ephod.\fnote{The ephod was a type of vest normally worn by the Levite priests.} \v{19}His mother would make a small robe for him, and she would bring it each year when she went up with her husband to offer the yearly sacrifice. \v{20}Then Eli would bless Elkanah and his wife and say,\fnote{So MT; 4QSam\textsuperscript{a} reads \fbib{wife, saying}} ``May the \divine{Lord} give\fnote{So MT; 4QSam\textsuperscript{a} read \fbib{\divine{Lord} repay you by giving}} you descendants\fnote{Lit. \fbib{seed}} from this woman in place of the one she dedicated\fnote{So LXX and DSS; MT, \fbib{he dedicated} or \fbib{asked}} to the \divine{Lord}.'' Then they would return to their\fnote{So MT; 4QSam\textsuperscript{a} LXX read \fbib{his}} home.\fnote{Lit. \fbib{place}}

\v{21}The \divine{Lord} took note of Hannah,\fnote{So MT; 4QSam\textsuperscript{a} LXX read \fbib{\divine{Lord} visited}} and she became pregnant and gave birth to\fnote{So MT; 4QSam\textsuperscript{a} LXX read \fbib{she bore more children---}} three sons and two daughters. Meanwhile, the boy Samuel continued to grow,\fnote{So MT LXX; 4QSam\textsuperscript{a} reads \fbib{grow there}} and the \divine{Lord} was constantly\fnote{The Heb. lacks \fbib{constantly}} with him.
\passage{Judgment against Eli's Sons}

\v{22}Now Eli was very\fnote{So MT LXX; 4QSam\textsuperscript{a} reads \fbib{was 98 years}} old, and he had heard everything that\fnote{So MT; 4QSam\textsuperscript{a} LXX read \fbib{heard what}} his sons were doing\fnote{So 4QSam\textsuperscript{a}; MT reads \fbib{sons would do}} to the Israelis,\fnote{So 4QSam\textsuperscript{a} LXX; MT reads \fbib{to all of Israel}} and how they lay with the women who were serving regularly\fnote{Lit. \fbib{who had assembled in order}} at the entrance to the Tent of Meeting. \v{23}``Why are you doing these things that I'm hearing about?'' he asked his sons,\fnote{Lit. \fbib{them}} ``These reports about your evil deeds are coming from all these\fnote{So MT; 4QSam\textsuperscript{a} LXX read \fbib{from the \divine{Lord}'s}} people! \v{24}No, my sons, I'm not hearing good news being circulated by the \divine{Lord}'s people. \v{25}If a person sins\fnote{So MT; 4QSam\textsuperscript{a} LXX read \fbib{sins gravely}} against another, God\fnote{Or \fbib{a judge}} will mediate for him,\fnote{So MT; 4QSam\textsuperscript{a} reads \fbib{another, he will appeal to the \divine{Lord}}} but if a person sins against the \divine{Lord}, who can intercede for him?''

But they would not follow the advice of\fnote{So MT LXX; 4QSam\textsuperscript{a} lacks \fbib{the advice of}} their father; for the \divine{Lord} wanted to put them to death. \v{26}But the boy Samuel continued to grow both physically and in favor with the \divine{Lord} and the people.

\v{27}A man of God came to Eli, saying to him,\fnote{So MT; 4QSam\textsuperscript{a} LXX read \fbib{Eli and said}} ``This is what the \divine{Lord} says: `When they were in Egypt and slaves\fnote{So LXX and DSS; the Heb. lacks \fbib{slaves}} to the house of Pharaoh, did I not reveal\fnote{So MT; 4QSam\textsuperscript{a} LXX read \fbib{Pharaoh, I did indeed}} to the family of your ancestor Aaron\fnote{The Heb. lacks \fbib{Aaron}} \v{28}that I had chosen him\fnote{So MT; 4QSam\textsuperscript{a} LXX read \fbib{chosen your father's house}} out of all the tribes of Israel to be my priest, to offer up burnt offerings on my altar, burn incense, and carry\fnote{So MT; 4QSam\textsuperscript{a} LXX read \fbib{wear}} the ephod in my presence? And did I not give to your ancestors' family all the Israeli fire-offerings? \v{29}Why, then, do all of you show contempt for\fnote{Lit. \fbib{do you} (pl.) \fbib{tread down}; 4QSam\textsuperscript{a} LXX read \fbib{do you} (sing) \fbib{look down on}} my sacrifice and offering that I've commanded for my\fnote{The Heb. lacks \fbib{my}} dwelling? And you honor your sons more than me in order to fatten yourselves\fnote{So MT; 4QSam\textsuperscript{a} reads \fbib{yourself}} from the best of all the offerings of my people Israel.'\fnote{So MT; 4QSam\textsuperscript{a} LXX read \fbib{Israel in my presence}}

\v{30}``Therefore, the \divine{Lord} God of Israel has declared, `I did, in fact, say\fnote{So MT LXX; 4QSam\textsuperscript{a} reads \fbib{seek}} that your family and your ancestor's family would walk before me forever,' but now the \divine{Lord} declares, `Far be it from me! The one who honors me I'll honor, and the one who despises me is to be treated with contempt. \v{31}The time is coming when I'll cut away\fnote{So MT; 4QSam\textsuperscript{a} LXX read \fbib{away your strength and the strength of your father's house}} at your family\fnote{Lit. \fbib{arm}} and your ancestor's family\fnote{Lit. \fbib{arm}} until there are no old men left in your family. \v{32}Distress will settle down to live in your household, and despite all the good that I do for Israel,\fnote{Or \fbib{you are to look with envy on all that happens good with Israel, and}} there will never be an old man in your family forever, and you will never again have an old man in my house.\fnote{So 4QSam\textsuperscript{a} LXX; MT reads \fbib{left in my dwelling. You will look in distress at all the prosperity given to Israel, while there never again will be an old man in your house}} \v{33}Any of you whom I don't eliminate from serving at my altar will grow tired from weeping,\fnote{The Heb. lacks \fbib{from weeping}} and their\fnote{Lit. \fbib{your}} souls will grieve. All the increase of your family will die by violence.\fnote{Lit. \fbib{by the sword}; so with LXX and DSS; MT \fbib{die like men}} \v{34}Here's a sign for you---your two sons Hophni and Phineas will both die on the same day! \v{35}And I'll raise up for myself a faithful\fnote{Or \fbib{an enduring}; MT word translated \fbib{faithful} here is translated \fbib{enduring} later in this verse.} priest who will do according to what is in my heart and according to my desire. I'll build for him an enduring\fnote{Or \fbib{faithful}; MT word translated \fbib{enduring} here is translated \fbib{faithful} earlier in this verse.} house and he will walk before my anointed one forever. \v{36}Anyone who remains in your family will come and prostrate themselves before him for a small wage\fnote{Lit. \fbib{a payment of silver}} or a loaf of bread and will say,\fnote{So MT; 4QSam\textsuperscript{a} LXX read \fbib{bread, saying}} ``Please put me in one of the priest's offices so I can eat a piece of bread.''\,'\,''
\labelchapt{3}
\passage{The \divine{Lord} Calls Samuel}

\chapt{3}
\v{1}Meanwhile the boy Samuel was serving the \divine{Lord} before Eli. A word from the \divine{Lord} was rare in those days, and visions were infrequent. \v{2}At that time Eli, whose vision was growing dim,\fnote{Lit. \fbib{dim so that he was unable to see}} was lying down in his bedroom.\fnote{Lit. \fbib{place}} \v{3}The lamp of God had not yet been extinguished, and Samuel was lying down in the tent\fnote{Or \fbib{temple}} of the \divine{Lord} where the Ark of God was. \v{4}The \divine{Lord} called out to Samuel, who answered, ``Here I am.''

\v{5}He ran to Eli and said, ``Here I am! You called me.''

``I didn't call you,'' Eli\fnote{Lit. \fbib{He}} said. ``Go back and lie down.'' So he went and lay down.

\v{6}Then the \divine{Lord} again called out, ``Samuel!''

So Samuel got up, went to Eli, and said, ``Here I am! You called me.''

He said, ``I didn't call you, my son. Go back and lie down.'' \v{7}Now Samuel did not yet know the \divine{Lord} and had not yet had the word of the \divine{Lord} revealed to him.

\v{8}Then the \divine{Lord} called out to Samuel again a third time, and he got up, went to Eli, and said, ``Here I am! You called me.''

Then Eli understood that the \divine{Lord} was calling the boy, \v{9}so Eli told Samuel, ``Go lie down, and then if he calls you, answer, `Speak, \divine{Lord}, because your servant is listening.'\,'' Then Samuel\fnote{Lit. \fbib{he}} went and lay down.

\v{10}Later, the \divine{Lord} came and stood there, calling out, ``Samuel! Samuel!'' as he had before.

Samuel said, ``Speak, because your servant is listening.''

\v{11}``Look,'' the \divine{Lord} told Samuel. ``I'm about to do something\fnote{Lit. \fbib{do a work}} in Israel that will make both ears of anyone who hears it tingle. \v{12}I'll fulfill every promise that I've spoken concerning Eli's family, from beginning to end. \v{13}I've told him that I'm about to judge his family forever because of the iniquity that he knew about. His sons committed blasphemy\fnote{LXX \fbib{cursed God}; MT \fbib{cursed themselves}} and he did not rebuke them. \v{14}Therefore I've sworn concerning Eli's family that the iniquity of his family is not to be atoned for by sacrifice or offering forever.''
\passage{Samuel Delivers God's Message}

\v{15}Samuel lay down until morning and then opened the doors of the house of the \divine{Lord}, but he\fnote{Lit. \fbib{Samuel}} was afraid to report the vision to Eli. \v{16}Then Eli called Samuel: ``Samuel, my son.''

He said, ``Here I am.''

\v{17}Eli\fnote{Lit. \fbib{He}} said, ``What did the \divine{Lord}\fnote{Lit. \fbib{he}} say to you? Please don't conceal anything\fnote{The Heb. lacks \fbib{anything}} from me. May God do this to you and even more\fnote{I.e. \fbib{May God punish you}} if you conceal from me one word of all that he spoke to you.'' \v{18}So Samuel told him everything---he did not conceal anything\fnote{The Heb. lacks \fbib{anything}} from him. Eli\fnote{Lit. \fbib{He}} said, ``He is the \divine{Lord}. May he do what seems good to him.''

\v{19}As Samuel grew, the \divine{Lord} was with him and did not let any of Samuel's\fnote{Lit. \fbib{his}} predictions fail.\fnote{Lit. \fbib{words fall to the ground}} \v{20}All Israel from Dan to Beer-sheba knew that Samuel was confirmed as the \divine{Lord}'s prophet. \v{21}The \divine{Lord} continued to appear at Shiloh, because he\fnote{Lit. \fbib{the \divine{Lord}}} revealed himself to Samuel at Shiloh by means of messages from\fnote{Lit. \fbib{by the word of}} the \divine{Lord}.
\labelchapt{4}
\passage{The Philistines Capture the Ark}

\chapt{4}
\v{1}What Samuel had to say was directed to all Israel, and Israel went out to engage the Philistines in battle. The Israelis\fnote{Lit. \fbib{They}} were camped at Ebenezer, while the Philistines were camped at Aphek. \v{2}The Philistines deployed their forces to meet Israel, and as the battle spread Israel was defeated by the Philistines, who killed about four thousand men on the battlefield.

\v{3}When the people came to the camp, the elders of Israel said, ``Why did the \divine{Lord} defeat us today when we fought the Philistines? Let's take the Ark of the Covenant of the \divine{Lord} from Shiloh, so it\fnote{Or \fbib{he}} may go with us and deliver us from the power of our enemies.'' \v{4}So the people sent word\fnote{The Heb. lacks \fbib{word}} to Shiloh and took away from there the Ark of the Covenant of the \divine{Lord} of the Heavenly Armies, who sits above\fnote{Lit. \fbib{sits}} the cherubim.

Now the two sons of Eli, Hophni and Phineas, were there with the Ark of the Covenant of God. \v{5}When the Ark of the Covenant of the \divine{Lord} came into the camp, all Israel gave a great shout and the earth reverberated! \v{6}When the Philistines heard the noise of the shout, they asked, ``What is this noise coming from shouting in the camp of the Hebrews?'' Then they realized that the Ark of the \divine{Lord} had come into the camp, \v{7}and the Philistines were terrified. ``God has come\fnote{So MT Some LXX texts read \fbib{The gods have come} and other LXX texts read \fbib{Their God has come}} into the camp,'' they said. ``How terrible for us, because nothing like this has ever happened before! \v{8}How terrible for us! Who will deliver us from the hand of these mighty gods? These are the gods who struck the Egyptians with all kinds of plagues in the desert. \v{9}Philistines, be strong and be men, or you will become slaves to the Hebrews just as they have been slaves to you! Be men and fight!''

\v{10}The Philistines fought and Israel was defeated; each of them fled to his own tent. It was a very great slaughter, and 30,000 soldiers of Israel died. \v{11}The Ark of God was captured, and the two sons of Eli, Hophni and Phineas, died.
\passage{The Death of Eli}

\v{12}That very same day, a man who was a descendant of Benjamin ran from the battle line and came to Shiloh, with his garments torn and dirt on his head. \v{13}When he arrived, Eli was sitting there on a seat beside the road, watching because his heart trembled for the Ark of God. The man went into the town to give the report, and the whole town cried out. \v{14}Eli heard the sound of the cry and asked, ``What is the meaning\fnote{Lit. \fbib{sound}} of this commotion?'' Then the man quickly came and told Eli. \v{15}Now Eli was 98 years old, and his vision had failed.\fnote{Lit. \fbib{were set}}

\v{16}The man told Eli, ``I've just come from the battle line, and I escaped from the battle today.''

He asked, ``What happened, my son?''

\v{17}The messenger answered, ``Israel fled from the Philistines and the people suffered a great defeat as well. Moreover, your two sons, Hophni and Phineas, are dead, and the Ark of God was captured.''

\v{18}When he mentioned the Ark of God, Eli\fnote{Lit. \fbib{he}} fell off the seat backwards by the side of the gate. His neck was broken and he died, since he was old and heavy. Eli had judged Israel for 40 years.
\passage{Ichabod is Born}

\v{19}Eli's\fnote{Lit. \fbib{His}} daughter-in-law, the wife of Phineas, was pregnant and ready to give birth. When she heard the report about the capture of the Ark of God and that her father-in-law and husband were dead, she crouched down and gave birth, because her labor pains suddenly began. \v{20}As she was about to die, the women standing around her said, ``Don't be afraid! You've given birth to a son.'' But she did not respond or pay attention. \v{21}She had named the boy Ichabod,\fnote{Ichabod means \fbib{Where is the glory?}} saying, ``Glory has departed from Israel,'' because the Ark of God had been captured and because her father-in-law and husband were dead.\fnote{Lit. \fbib{because of her father-in-law and husband}} \v{22}She said, ``Glory has departed from Israel, because the Ark of God has been captured.''
\labelchapt{5}
\passage{The Philistines' Troubles because of the Ark}

\chapt{5}
\v{1}The Philistines took the Ark of God and brought it from Ebenezer to Ashdod. \v{2}Then the Philistines took the Ark of God, brought it to the temple of Dagon,\fnote{Dagon was the principal deity of the Philistines.} and placed it beside Dagon. \v{3}When the people of Ashdod got up the next morning, there was Dagon, lying on the ground in front of the Ark of the \divine{Lord}. They took Dagon and put him back in his place. \v{4}But when they got up the next morning, there was Dagon, lying on the ground again in front of the Ark of the \divine{Lord}. Dagon's head and both of his arms\fnote{Lit. \fbib{both of the palms of his hands}} were broken off and lying on the threshold.\fnote{Lit. \fbib{broken off on the threshold}} Only the trunk of\fnote{The Heb. lacks \fbib{the trunk of}} Dagon was left intact.\fnote{Lit. \fbib{on it}} \v{5}This is why neither the priests of Dagon nor anyone who enters the temple of Dagon step on the threshold of Dagon in Ashdod to this day.

\v{6}The \divine{Lord} heavily oppressed the people of Ashdod, devastating and afflicting Ashdod and its territories with tumors of the groin. \v{7}When the men of Ashdod saw how things were, they said, ``Don't let the Ark of the God of Israel stay with us, because he is severely attacking us and our god Dagon.'' \v{8}They sent messengers\fnote{The Heb. lacks \fbib{messengers}} and gathered together all the lords of the Philistines and asked, ``What are we to do with the Ark of the God of Israel?''

They said, ``Let the Ark of the God of Israel move to Gath.'' So they moved the Ark of the God of Israel.

\v{9}After they moved it, the \divine{Lord} moved against the town, causing\fnote{Lit. \fbib{with}} a very great panic. He struck the men of the town, from young to old with tumors of the groin. \v{10}Then they sent the Ark of God to Ekron. When the Ark of God arrived in Ekron, the people of Ekron cried out, ``They have brought the Ark of the God of Israel to us to kill us and our people!''

\v{11}They sent messengers\fnote{The Heb. lacks \fbib{messengers}} and gathered together all the Philistine lords: ``Send away the Ark of the God of Israel, and let it return to where it belongs so that it does not kill us and our people.'' Meanwhile, a deadly panic had spread all over the town, and God kept on pressuring\fnote{Lit. \fbib{and the hand of God was on}} them there. \v{12}The people who did not die were afflicted with tumors of the groin, and the cry of the town went up to heaven.
\labelchapt{6}
\passage{The Philistines Return the Ark to Israel}

\chapt{6}
\v{1}The Ark of the \divine{Lord} remained in Philistine territory\fnote{Lit. \fbib{field}} for seven months. \v{2}The Philistines summoned the priests and diviners and asked, ``What should we do about the Ark of the \divine{Lord}? Tell us how we should send it back to its place.''

\v{3}They said, ``If you send the Ark of the God of Israel back, don't send it empty, but rather be sure to send back to him a guilt offering. Then you will be healed and will know why his oppression\fnote{Lit. \fbib{hand}} has not been removed from you.''

\v{4}They asked, ``What is the guilt offering that we should send back to him?''

``Five gold tumors and five gold mice,'' they answered, ``according to the number of the lords of the Philistines, since the same plague was on all of you and on your lords. \v{5}Make images of your tumors and images of the mice that are destroying your land, and you are to give glory to the God of Israel. Perhaps he will remove his pressure from you, your gods, and your land. \v{6}Why should you harden your hearts just as the Egyptians and Pharaoh hardened their hearts? Isn't it true that after God\fnote{Lit. \fbib{he}} toyed with them, they let Israel\fnote{Lit. \fbib{them}} go, and off they went?

\v{7}``So make a new cart, and take two milk cows that have never had a yoke on them. Hitch the cows to the cart and take their calves away from them and back to the house. \v{8}Take the Ark of the \divine{Lord}, put it on the cart, and put the gold objects that you are returning to him as a guilt offering in a box beside it. Then send it away and let it go. \v{9}Keep watching it. If it goes up along the road to its own territory to Beth-shemesh, it's the \divine{Lord}\fnote{Lit. \fbib{he}} who has done this great evil to us. But if it does not, then we will know that he wasn't pressuring us. It happened to us as a natural event.''

\v{10}The men did this. They took two milk cows, hitched them to the cart, and penned up their calves in the house. \v{11}They put the Ark of the \divine{Lord}, the box, the gold mice, and the images of their tumors on the cart. \v{12}The cows took a straight path along the road to Beth-shemesh. They stayed on the highway, lowing as they went, and did not turn to the right or the left. The Philistine lords followed them as far as the border of Beth-shemesh.

\v{13}Now the people of Beth-shemesh were gathering their wheat harvest in the valley. They looked up, saw the Ark, and rejoiced to see it. \v{14}The cart came to the field of Joshua of Beth-shemesh, and stopped there. In that place there was a large stone. They broke up the wood from the cart, and offered up the cows as a burnt offering to the \divine{Lord}. \v{15}The descendants of Levi took down the Ark of the \divine{Lord}, along with the box that was with it, containing the objects of gold, and they put them on the large stone. The men of Beth-shemesh offered burnt offerings and made sacrifices to the \divine{Lord} that day. \v{16}When the five Philistine lords saw this, they returned to Ekron that very day.

\v{17}These are the gold tumors that the Philistines returned as a guilt offering to the \divine{Lord}: one for Ashdod, one for Gaza, one for Ashkelon, one for Gath, and one for Ekron. \v{18}The gold mice represented\fnote{Lit. \fbib{were according to}} the number of all the Philistine towns belonging to the five lords, both fortified towns and unwalled villages. The large stone, beside which they put the Ark of the \divine{Lord}, is a witness to this day in the field of Joshua of Beth-shemesh.

\v{19}God struck down the men of Beth-shemesh because they had looked into the Ark of the \divine{Lord}. He struck down 50,070\fnote{So MT; LXX reads \fbib{70}} men among the people, and the people mourned because the \divine{Lord} struck down the people with a great slaughter. \v{20}The men of Beth-shemesh asked themselves, ``Who can stand in the presence of the \divine{Lord}, this holy God? And to whom will the Ark\fnote{Lit. \fbib{it}} go from here?''\fnote{Lit. \fbib{us}}

\v{21}They sent messengers to the residents of Kiriath-jearim, who told them, ``The Philistines have returned the Ark of the \divine{Lord}. Come down and take it up with you.''
\labelchapt{7}
\passage{The Ark is Stored in Kiriath-Jearim}

\chapt{7}
\v{1}The men of Kiriath-jearim came and took the Ark of the \divine{Lord}. They brought it to the house of Abinadab on the hill, and they consecrated his son Eleazar to care for the Ark of the \divine{Lord}.

\v{2}A long time passed---it was twenty years---from the time the Ark came to reside in Kiriath-jearim, and all the house of Israel mourned because of the \divine{Lord}.
\passage{The Philistines are Defeated at Ebenezer}

\v{3}Then Samuel told the whole house of Israel, ``If you're returning to the \divine{Lord} with all your heart, then remove the foreign gods and the Ashtaroth\fnote{I.e. trees or images representing the Canaanite deity Asherah} from among you, direct your hearts back to the \divine{Lord}, and serve him only. Then he will deliver you from the control of the Philistines.'' \v{4}So the Israelis removed the Baals\fnote{Images representing the Canaanite storm god} and Ashtaroth, and served the \divine{Lord} only.

\v{5}Samuel said, ``Bring all Israel together at Mizpah, and I'll pray to the \divine{Lord} on your behalf.'' \v{6}So they came together at Mizpah, drew water, and poured it out in the \divine{Lord}'s presence.

On that day they fasted there and said, ``We have sinned against the \divine{Lord}.'' Then Samuel judged the Israelis at Mizpah. \v{7}When the Philistines heard that the Israelis had gathered at Mizpah, the Philistine lords came up against Israel. When the Israelis heard this, they were afraid of the Philistines.

\v{8}The Israelis told Samuel, ``Don't stop crying out to the \divine{Lord} our God for us that he may deliver us from the hand of the Philistines.'' \v{9}Then Samuel took a nursing lamb and offered it as a whole burnt offering to the \divine{Lord}. Samuel cried out to the \divine{Lord} on behalf of Israel, and the \divine{Lord} answered him. \v{10}While Samuel was sacrificing the burnt offering, the Philistines approached to attack Israel. But that day the \divine{Lord} thundered against the Philistines and threw them into panic, and they were defeated before Israel. \v{11}The men of Israel went out from Mizpah, pursued the Philistines, and struck them down as far as a point below Beth-car. \v{12}Then Samuel took a stone, placed it between Mizpah and Shen\fnote{Lit. \fbib{the tooth}; perhaps referring to a prominent rock formation. Syr reads \fbib{Jeshanah}} and named it Ebenezer.\fnote{MT means \fbib{Stone of Help}} He said, ``The \divine{Lord} has helped us this far.'' \v{13}The Philistines were subdued, and they did not continue to enter the territory of Israel.

The \divine{Lord} continued to oppose the Philistines all during Samuel's life time. \v{14}The towns that the Philistines had taken from Israel were returned to Israel, from Ekron to Gath, and Israel delivered their territory from Philistine control. There was also peace between Israel and the Amorites.

\v{15}Samuel judged Israel all the days of his life. \v{16}He went on a circuit each year to Bethel, Gilgal, and Mizpah, and he judged Israel in all those places. \v{17}He would return to Ramah because his house was there, and judged Israel from there. He also built an altar to the \divine{Lord} there.
\labelchapt{8}
\passage{Israel Demands a King}

\chapt{8}
\v{1}When Samuel became old, he appointed his sons judges over Israel. \v{2}The name of his firstborn son was Joel and the name of his second was Abijah. They were judges in Beer-sheba. \v{3}His sons did not follow Samuel's example.\fnote{Lit. \fbib{not walk in his ways}} Instead, they pursued\fnote{Lit. \fbib{turned after}} dishonest gain, took bribes, and perverted justice.\fnote{Lit. \fbib{caused justice to be turned aside}}

\v{4}All the elders of Israel gathered together, and came to Samuel at Ramah. \v{5}They told him, ``Look, you're old, and your sons don't follow your example.\fnote{Lit. \fbib{not walk in his ways}} So appoint a king to govern us like all the other\fnote{The Heb. lacks \fbib{other}} nations.'' \v{6}Samuel was displeased\fnote{Lit. \fbib{the thing was bad in the eyes of}} when they said, ``Give us a king to govern us.'' So Samuel prayed to the \divine{Lord}.

\v{7}The \divine{Lord} told Samuel, ``Listen to the people\fnote{Lit. \fbib{the voice of the people}} in all that they say to you. In fact, it's not you they have rejected, but rather they have rejected me from being their king. \v{8}Like all the things they have done from the day I brought them up out of Egypt until this very day, they have forsaken me and followed other gods. They're also doing the same thing to you. \v{9}Now, listen to them, but you are to clearly warn them and inform them about how the king who rules over them will operate.''\fnote{Lit. \fbib{the practice of the king}}

\v{10}Samuel reported everything the \divine{Lord} told him to the people who were asking him for a king. \v{11}He said, ``This is how the king who rules over you will operate: He will conscript your sons and assign them\fnote{Lit. \fbib{them for himself}} to his chariots. He will conscript them\fnote{The Heb. lacks \fbib{conscripting them}} as his horsemen, and they'll run in front of his chariots. \v{12}He will appoint his officers over thousands and officers over fifties---some will plow his fields,\fnote{Lit. \fbib{and to plow his plowing}} reap his harvest, and craft his war implements and equipment for his chariots. \v{13}He will take your daughters for perfumers, cooks, and bakers. \v{14}He will take the best products of your fields, your vineyards, and your olive groves and give them to his servants.\fnote{Or \fbib{officials}} \v{15}He will take a tenth of your seed and your vineyards and give it to his officers and servants.\fnote{Or \fbib{officials}} \v{16}He will take your male and female servants, your best young men, and your donkeys to do his work. \v{17}He will take a tenth of your flock, and you will become his servants. \v{18}When all of this comes about, you will cry out because of your king whom you chose for yourselves, but the \divine{Lord} won't answer you at that time.''

\v{19}The people refused to listen to Samuel.\fnote{Lit. \fbib{to the voice of Samuel}} Instead, they insisted, ``No! Let a king rule over us instead! \v{20}We, too, will be like all the nations! Our king will govern us and go out before us to fight our battles.''

\v{21}So Samuel listened to all the words of the people, and he repeated them directly to\fnote{Lit. \fbib{them in the ears of}} the \divine{Lord}. \v{22}The \divine{Lord} told Samuel, ``Listen to them, and appoint a king for them.''

Then Samuel told the men of Israel, ``Each of you go to his own town.''
\labelchapt{9}
\passage{Saul Selected as Israel's First King}

\chapt{9}
\v{1}There was a man from Benjamin named Kish, Abiel's son, the grandson of Zeror and great-grandson of Aphiah's son Becorath. A prominent man\fnote{I.e. a man of wealth, military skill, or high reputation} from Benjamin, \v{2}he had a son named Saul, who was a choice and handsome\fnote{Or \fbib{good}} young man. There was no one among the Israelis as handsome as he, and he was a head taller\fnote{Lit. \fbib{from his shoulder up he was taller}} than any of the other people.

\v{3}The donkeys belonging to Kish, Saul's father, were lost, and Kish told his son Saul, ``Take one of the young men with you, get up, and go look for the donkeys.'' \v{4}He went through the hill country of Ephraim and through the region of Shalishah, but they did not find them. Then they went through the region of Shaalim, but they were not there. They also went through the territory of the descendants of Benjamin, but they did not find them.

\v{5}When they entered the region of Zuph, Saul told the\fnote{Lit. \fbib{his}} young man with him, ``Come on, let's go back so my father does not stop worrying\fnote{The Heb. lacks \fbib{worrying}} about the donkeys and become anxious about us.''

\v{6}The young man\fnote{Lit. \fbib{He}} said, ``Look, there's a man of God in this town. The man is respected, and everything he predicts happens. Now, let's go there. Perhaps he can tell us about the\fnote{Lit. \fbib{our}} journey on which we have set out.''

\v{7}Saul told the\fnote{Lit. \fbib{his}} young man, ``Look, we could go, but what could we bring the man? The bread is gone from our bags, and there is no present to bring to the man of God. What do we have with us?''

\v{8}The young man answered Saul again, ``Look here! I have in my hand a quarter shekel\fnote{I.e. about 0.1 ounces at 0.4 shekels per ounce} of silver. I'll give it to the man of God, and he will tell us about our journey.''

\v{9}(Previously in Israel, a person would say when he went to inquire of God, ``Come on! Let's go to the seer!'' because the person known as a prophet\fnote{Lit. \fbib{the prophet}} today was formerly called a seer.)

\v{10}Saul told his young man, ``That's a good suggestion! Come on, let's go!'' Then they entered the town where the man of God was.

\v{11}As they were going up the hill to the town, they met some young women going out to draw water, and they told them, ``Is the seer here?''

\v{12}They answered them: ``Yes, he's right there ahead of you. Hurry, for he came to town just today because there is a sacrifice for the people on the high place today. \v{13}When you come into town you can find him before he goes up to the high place to eat. For the people don't eat until he arrives, because he must bless the sacrifice and then after that those who are invited will eat. So go up right now because you can find him now.'' \v{14}They went up to the town, and as they were coming to the center of the town, Samuel was coming out to meet them, on his way\fnote{Lit. \fbib{going}} up to the high place.
\passage{The \divine{Lord}'s Revelation to Samuel}

\v{15}Now one day before Saul's arrival, the \divine{Lord} had revealed to\fnote{Lit. \fbib{uncovered the ear of}} Samuel: \v{16}``About this time tomorrow I'll send you a man from the land of Benjamin, and you are to anoint him as Commander-in-Chief\fnote{Lit. \fbib{Nagid}; i.e. a senior officer entrusted with dual roles of operational oversight and administrative authority} over my people Israel. He'll deliver my people from the control\fnote{Lit. \fbib{hand}} of the Philistines, because I've seen the suffering of\fnote{So LXX; the Heb. lacks \fbib{the suffering of}} my people and because their cry has come up to me.'' \v{17}When Samuel saw Saul, the \divine{Lord} told him, ``Here is the man I told you about. This man will rule over my people.''

\v{18}As Saul approached Samuel in the middle of the gate, he said, ``Please tell me where the seer's house is.''

\v{19}Samuel answered Saul: ``I'm the seer. Go up ahead of me to the high place, and eat with me today. In the morning I'll send you away and tell you everything that is on your mind. \v{20}Now as for your donkeys that were lost three days ago, don't give any thought to them, because they've been found. Meanwhile, to whom is all Israel looking, if not to you and all of your father's household?''

\v{21}Saul answered: ``Am I not a descendant of Benjamin from the least of the tribes of Israel? Isn't my family the least important of all the families of the tribe of Benjamin? Why have you spoken to me like this?''

\v{22}Then Samuel took Saul and his young man and brought them to a room where he gave them a place at the head of those who were invited, of whom there were about 30 men. \v{23}Then Samuel told the cook, ``Bring the portion that I gave you, the one I told you to set aside.'' \v{24}The cook picked up the thigh and what was on it and set it in front of Saul. Then Samuel\fnote{Lit. \fbib{he}} said, ``Here is what is left! Set it before you and eat, for it has been kept for you until the appointed time, about which I said,\fnote{Lit. \fbib{appointed time, saying}} `I've invited the people.'\,'' So Saul ate with Samuel that day.

\v{25}When they had come down from the high place into town,\fnote{LXX adds, \fbib{they made a bed for Saul on the roof and he slept}} Samuel\fnote{Lit. \fbib{he}} spoke to Saul on the roof. \v{26}They got up early in the morning, and about daybreak Samuel called to Saul on the roof, ``Get up and I'll send you off.'' Saul got up and the two of them, he and Samuel, went outside. \v{27}As they were going down to the edge of the town, Samuel told Saul, ``Tell your young man to go ahead of us and when he has gone ahead, stop for a while so I may declare God's word to you.''
\labelchapt{10}
\passage{Saul is Anointed King}

\chapt{10}
\v{1}Samuel took a flask of oil, poured it on Saul's\fnote{Lit. \fbib{his}} head, kissed him, and said, ``The \divine{Lord} has anointed you Commander-in-Chief\fnote{Lit. \fbib{Nagid}; i.e. a senior officer entrusted with dual roles of operational oversight and administrative authority} over his inheritance, has he not? \v{2}When you leave me today, you will find two men by Rachel's tomb in the territory of Benjamin at Zelzah. They'll tell you, `The donkeys you went to look for have been found. Now your father has stopped worrying about the donkeys\fnote{Lit. \fbib{about the matter of the donkeys}} and he's anxious about you. He's asking, `What will I do about my son?' \v{3}Then you'll go on further from there and come to the oak at Tabor. There three men going up to the \divine{Lord} at Bethel will meet you. One will be herding\fnote{Lit. \fbib{carrying}} three young goats, one will be carrying three loaves of bread, and one will be carrying a bottle\fnote{Lit. \fbib{skin}} of wine. \v{4}They'll greet you and give you two loaves of bread, which you're to accept from them.

\v{5}``After that you will come to Gibeath-elohim\fnote{Or \fbib{the hill of God}} where the Philistine garrison is. As you arrive there at the town, you'll meet a band of prophets coming down from the high place with a harp, tambourine, flute, and lyre being played in front of them, and they'll be prophesying. \v{6}The Spirit of the \divine{Lord} will come upon you, and you'll prophesy with them and be changed into a different person. \v{7}When these signs occur,\fnote{Lit. \fbib{signs come to you}} do whatever you want\fnote{Lit. \fbib{whatever your hand finds}} to do, because the \divine{Lord} is with you. \v{8}You are to go down ahead of me to Gilgal, and then I'll come down to offer burnt offerings and to sacrifice peace offerings. You are to wait seven days until I come to you to let you know what you are to do.''
\passage{The Spirit of God Comes on Saul}

\v{9}Now it happened as Saul\fnote{Lit. \fbib{he}} turned his back to leave Samuel, that God gave him another heart,\fnote{Lit. \fbib{changed for him another heart}} and all these signs occurred on that day. \v{10}When they arrived there at Gibeah,\fnote{Or \fbib{the hill}} a band of prophets was right there to meet them. The Spirit of God came upon Saul,\fnote{Lit. \fbib{him}} and he prophesied\fnote{Or \fbib{he was caught up in prophetic ecstasy}} along with them. \v{11}When all those who had known Saul previously saw that he was there among the prophets prophesying, the people told one another, ``What has happened to Kish's son? Is Saul also among the prophets?''

\v{12}A man from there answered: ``Now who is their father?'' Therefore it became a proverb, ``Is Saul also among the prophets?'' \v{13}When he had finished prophesying, he went to the high place.

\v{14}Saul's uncle told him and to his young man, ``Where did you go?''

He said, ``To look for the donkeys, and when we saw that they couldn't be found, we went to Samuel.''

\v{15}Then Saul's uncle said, ``Please tell me what Samuel told you.''

\v{16}Saul told his uncle, ``He actually told us that the donkeys had been found,'' but he did not tell him about the matter of kingship about which Samuel had spoken.
\passage{Saul is Proclaimed King}

\v{17}Samuel summoned the people to the \divine{Lord} at Mizpah. \v{18}He told the Israelis, ``This is what the \divine{Lord} God of Israel says: `I brought Israel up out of Egypt, and I rescued you from the power\fnote{Lit. \fbib{hand}} of Egypt and from the power\fnote{Lit. \fbib{hand}} of all the kingdoms that were oppressing you.' \v{19}But today you have rejected your God who delivers you from all your troubles and difficulties. You have said, `No!\fnote{So with numerous mss and versions; MT reads \fbib{told him, `Instead,}} Instead, appoint a king over us.' Now present yourselves in the \divine{Lord}'s presence by your tribes and families.''

\v{20}Samuel brought forward all the tribes of Israel, and the tribe of Benjamin was chosen. \v{21}Then he brought forward the tribe of Benjamin according to its families, and the family of Matri was chosen. Finally, Kish's son Saul was chosen, but when they looked for him, they couldn't find him. \v{22}So they inquired further of the \divine{Lord}, ``Has the man come here yet?''

The \divine{Lord} said, ``He is here, hiding among the baggage.''

\v{23}They ran and brought him from there. When he stood among the people, he was taller than any of the others by a head.\fnote{Lit. \fbib{than all the people from his shoulder up}} \v{24}Then Samuel told all the people, ``Do you see the man whom the \divine{Lord} has chosen? For there is no one like him among all the people.''

Then all the people shouted, ``Long live the king!''

\v{25}Samuel explained to the people the regulations\fnote{Or \fbib{practices}} concerning kingship. He wrote them in a scroll and placed it in the \divine{Lord}'s presence. Then Samuel sent all the people to their own houses. \v{26}Saul also went to his house in Gibeah, and the soldiers\fnote{Or \fbib{valiant men}} whose hearts God had touched went with him. \v{27}But some troublemakers\fnote{Lit. \fbib{sons of Belial}; i.e. worthless men} said, ``How can this man deliver us?'' They despised him and did not bring him a gift. But Saul\fnote{Lit. \fbib{he}} remained silent.
\passage{The Ammonites Threaten Jabesh-gilead}

\v{28}Meanwhile, Nahash, king of the Ammonites, had been severely oppressing the descendants of Gad and descendants of Reuben, gouging out their right eyes and not allowing Israel to have a deliverer. No one was left among the Israelis across the Jordan whose right eye Nahash, king of the Ammonites, had not gouged out. However, 7,000 men had escaped from the Ammonites and entered Jabesh-gilead.\fnote{So DSS 4QSam\textsuperscript{a} and Josephus; MT and LXX lack 10:28.}
\labelchapt{11}
\passage{Saul Defeats the Ammonites}

\chapt{11}
\v{1}So after a month,\fnote{So LXX and DSS 4QSam\textsuperscript{a}; the Heb. lacks \fbib{after a month}} Nahash the Ammonite came up and laid siege to\fnote{Lit. \fbib{camped against}} Jabesh-gilead. All the men of Jabesh told Nahash, ``Make a covenant with us, and we will serve you.''

\v{2}Nahash the Ammonite told them, ``I'll make a covenant with you on the condition that I gouge out the right eye of every one of you and so bring disgrace on all Israel.''

\v{3}The elders of Jabesh told him, ``Leave us alone for seven days so that we may send messengers through all the territory of Israel. Then if no one delivers us, we will come out to you and surrender.''\fnote{The Heb. lacks \fbib{and surrender}} \v{4}When the messengers came to Gibeah of Saul and reported the terms to the people,\fnote{Lit. \fbib{in the ears of the people}} all the people cried loudly.\fnote{Lit. \fbib{lifted their voices and wept}}

\v{5}Just then Saul was coming in from the field behind the oxen and he said, ``What's with the people? Why are they crying?'' They reported to him what the men of Jabesh had said.\fnote{Lit. \fbib{the words of the men of Jabesh}}

\v{6}When Saul heard these words, the Spirit of God came on him, and he was very angry. \v{7}He took a yoke of oxen, cut them in pieces, and sent the pieces\fnote{Lit. \fbib{sent}} by messengers through all the territory of Israel: ``This is what will be done to the oxen of anyone who does not come out and join\fnote{Lit. \fbib{out after}} Saul and Samuel!'' The fear of the \divine{Lord} fell on the people and they came out as one man.

\v{8}Saul\fnote{Lit. \fbib{He}} mustered them at Bezek, and there were 300,000 Israelis and 30,000 men of Judah. \v{9}They told the messengers who had come, ``You are to say this to the men of Jabesh-gilead, `Tomorrow, by the time the sun is hot, you will be delivered.'\,'' The messengers went and reported to the men of Jabesh, and they rejoiced.

\v{10}The men of Jabesh said, ``Tomorrow we will come out to you and surrender.\fnote{The Heb. lacks \fbib{and surrender}} Then you can do whatever you want to us.''

\v{11}The next day Saul separated the people into three companies. They came into the camp during the morning watch, and struck down the Ammonites until the heat of the day. Those who survived were scattered so that no two of them remained together.

\v{12}The people told Samuel, ``Who said, `Will Saul reign over us?' Bring them to us\fnote{Lit. \fbib{Give the men}} and we will put them to death!''

\v{13}But Saul said, ``Let no one be put to death this day, because today the \divine{Lord} has delivered Israel.''

\v{14}Then Samuel told the people, ``Come, let's go to Gilgal and reaffirm the kingship there.'' \v{15}So all the people went to Gilgal and there they made Saul king in the \divine{Lord}'s presence in Gilgal. There they sacrificed peace offerings in the \divine{Lord}'s presence, and there Saul and all the men of Israel rejoiced greatly.
\labelchapt{12}
\passage{Samuel's Farewell}

\chapt{12}
\v{1}Then Samuel told all Israel, ``Take note! I've listened to you, to everything you have told me, and I've appointed a king over you. \v{2}Now here is the king walking before you,\fnote{I.e. leading you} while I'm old and gray, and my sons are with you. I've walked before you\fnote{I.e. led you} from my youth until this day. \v{3}Here I am. Testify against me in the \divine{Lord}'s presence and before his anointed. Whose ox have I taken, or whose donkey have I taken? Who have I cheated? Who have I oppressed? Who bribed me to look the other way?\fnote{Lit. \fbib{From whose hand did I accept a bribe to blind my eyes}} I'll restore it to you.''

\v{4}They said, ``You haven't cheated us or oppressed us, and you haven't taken anything from anyone's hand.''

\v{5}He told them, ``Today the \divine{Lord} is testifying, along with his anointed, that you haven't found any bribes in my possession.''

They said, ``He's a witness.''

\v{6}Then Samuel told the people, ``It is the \divine{Lord} who appointed Moses and Aaron and who brought your ancestors up out of the land of Egypt. \v{7}Now stand up and I'll pass judgment on you in light of the \divine{Lord}'s righteous acts that he did for you and your ancestors. \v{8}After Jacob went to Egypt, and your ancestors cried out to the \divine{Lord}, he sent Moses and Aaron, who brought your ancestors out of Egypt and settled them in this place. \v{9}But they forgot the \divine{Lord} their God, so he handed them over to the domination of Sisera, the commander of the army of Hazor, and into domination by the Philistines and by the king of Moab, and Israel fought against them.

\v{10}``Then they cried out to the \divine{Lord}: `We have sinned because we have forsaken the \divine{Lord} and have served\fnote{Or \fbib{worshipped}} the Baals\fnote{I.e. images representing the Canaanite storm god} and the Ashtaroth.\fnote{I.e. trees or other symbols representing the Canaanite deity Asherah} Now deliver us from the hand of our enemies, and we will serve\fnote{Or \fbib{worship}} you.' \v{11}Then the \divine{Lord} sent Jerubbaal,\fnote{I.e. Gideon} Barak,\fnote{So LXX and Syr; MT reads \fbib{Bedan}} Jephthah, and Samuel and he delivered you from the hand of your enemies on every side, so that you lived securely. \v{12}But when you saw that Nahash, king of the Ammonites, was coming to fight you, you told me, `No, let a king rule over us instead,' even though the \divine{Lord} your God was your king.

\v{13}``Now, here is the king you have chosen, the one whom you asked for. See, the Lord has appointed a king over you. \v{14}If you fear the \divine{Lord}, serve him, obey him, and don't rebel against the commandment of the \divine{Lord}, then both you and the king who rules over you will truly follow the \divine{Lord} your God. \v{15}But if you don't obey the \divine{Lord} and rebel against the commandment of the \divine{Lord}, then the \divine{Lord} will turn against you as he did against your ancestors.\fnote{Lit. \fbib{and against your ancestors}}

\v{16}``Now then, stand up and see this great thing that the \divine{Lord} is about to do before your eyes. \v{17}Is it not the wheat harvest today? I'll call upon the \divine{Lord}, and he will send thunder and rain. Then you will know and understand that you have done a great evil in the sight of the \divine{Lord} by asking for a king for yourselves.'' \v{18}Samuel called upon the \divine{Lord} that same day, and the \divine{Lord} sent thunder and rain. So all the people greatly feared the \divine{Lord} and Samuel.

\v{19}Then all the people told Samuel, ``Pray to the \divine{Lord} your God for your servants, so that we don't die, because we made all our sins worse by asking for a king for ourselves.''

\v{20}Samuel told all the people, ``Don't be afraid. You have done all this evil. Yet don't turn aside from following the \divine{Lord}, but serve\fnote{Or \fbib{worship}} the \divine{Lord} with all your heart. \v{21}Don't turn aside after useless things\fnote{I.e. idols or false gods} that cannot profit or deliver because they're useless. \v{22}Indeed, the \divine{Lord} won't abandon His people for the sake of His great name, for the \divine{Lord} desires to make you a people for himself. \v{23}Now as for me, far be it from me that I should sin against the \divine{Lord} by ceasing to pray for you. I'll also instruct you in the way that is good and right. \v{24}Only, fear the \divine{Lord} and serve him faithfully with all your heart. Indeed, consider what great things he has done for you. \v{25}But if you persist in doing evil, both you and your king will be swept away.''
\labelchapt{13}
\passage{Saul's Battles against the Philistines}

\chapt{13}
\v{1}Saul was 30\fnote{So a few late LXX mss.; the Heb. lacks \fbib{30}} years old when he began to reign, and he ruled for 42\fnote{Lit. \fbib{two}; cf. Acts 13:21; Josephus's \fbib{Antiquities} VI.14.9 cites Saul as reigning 18 years before Samuel's death and 22 years after. But \fbib{Antiquities} X.8.4 cites only 20 years for Saul's reign.} years over Israel. \v{2}Saul chose for himself 3,000 men from Israel. There were 2,000 with Saul in Michmash and the hill country of Bethel, while 1,000 were with Jonathan in Gibeah of Benjamin. He had sent the rest of the people home.\fnote{Lit. \fbib{each to his own tent}}

\v{3}Jonathan attacked the Philistine garrison\fnote{Or \fbib{struck down the Philistine leader}} in Geba, and the Philistines heard about it. Saul blew the trumpet throughout the land: ``Listen, Hebrews!'' \v{4}All Israel heard the report,\fnote{Lit. \fbib{heard, saying}} ``Saul has attacked the Philistine garrison\fnote{Or \fbib{struck down the Philistine leader}} and Israel has also become repulsive to the Philistines.'' Then the people were summoned to Saul at Gilgal.

\v{5}The Philistines assembled to fight against Israel with 30,000 chariots, 6,000 horsemen, and people as numerous as the sand on the seashore. And they advanced and camped in Michmash, east of Beth-aven. \v{6}When the men of Israel saw that they were in distress (for the people were in difficult circumstances), the people hid themselves in caves, in thickets, in crags, in tombs, and in pits. \v{7}Hebrews went across the Jordan to the land of Gad and Gilead, but Saul remained in Gilgal, and all the people followed him, trembling.

\v{8}Saul\fnote{Lit. \fbib{He}} waited seven days for the appointment set by Samuel. When Samuel did not arrive at Gilgal, as the people began to scatter from Saul,\fnote{Lit. \fbib{him}} \v{9}Saul said, ``Bring the burnt offering and the peace offering to me,'' and he offered the burnt offering. \v{10}Just as he finished offering the burnt offering, Samuel arrived, and Saul went out to meet and greet him.

\v{11}Samuel said, ``What have you done?''

Saul replied, ``When? I saw that the people were scattering from me, that you didn't come at the appointed time, and that the Philistines were assembling at Michmash. \v{12}I\fnote{Or \fbib{When I {\ldots} Michmash,} \fbib{\v{12}I}} thought, `The Philistines will come down against me at Gilgal but I've not sought the favor of the \divine{Lord},' so I forced myself to offer the burnt offering.''

\v{13}Then Samuel told Saul, ``You have acted foolishly. You haven't obeyed the commandment of the \divine{Lord} your God, which he commanded you. For then the \divine{Lord} would have established your kingdom over Israel forever, \v{14}but now your kingdom won't be established. The \divine{Lord} has sought for himself a man after his own heart, and the \divine{Lord} has appointed him as Commander-in-Chief\fnote{Lit. \fbib{Nagid}; i.e. a senior officer entrusted with dual roles of operational oversight and administrative authority} over his people because you didn't obey that which the \divine{Lord} commanded you.''

\v{15}Then Samuel got up and went from Gilgal to Gibeah of Benjamin. Saul mustered the people present with him, about 600 men. \v{16}Saul, his son Jonathan, and the people present with them remained in Geba of Benjamin, while the Philistines camped in Michmash. \v{17}Raiders went out of the Philistine camp in three companies. One company turned in the direction of\fnote{Or \fbib{along the road to}} Ophrah, to the land of Shual, \v{18}one company turned in the direction of\fnote{Or \fbib{along the road to}} Beth-horon, while the one company turned toward the border\fnote{Or \fbib{along the border road}} that overlooks the valley of Zeboiim toward the desert.
\passage{The Philistine Monopoly on Metal Working}

\v{19}No blacksmith could be found in all the land of Israel because the Philistines thought, ``This will keep the Hebrews from making swords or spears.'' \v{20}Everyone in Israel would have to go to the Philistines so each person could sharpen his plow, his mattock, his axe, and his sickle.\fnote{So LXX; MT, \fbib{plow}} \v{21}The charge was one pin\fnote{I.e. a unit of measurement equal to about 2/3 of a shekel, weighing about 0.3 ounces; one shekel weighed about 0.4 ounces} for plows, mattocks, three pronged forks,\fnote{The meaning of MT is uncertain} and axes, or for setting the goads. \v{22}On the day of battle, none of the people who were with Saul and Jonathan were armed with swords or spears, but Saul and his son Jonathan did have\fnote{Lit. \fbib{were found with}} them. \v{23}Now a garrison of the Philistines had gone out to the pass of Michmash.
\labelchapt{14}
\passage{Jonathan's Heroic Exploits}

\chapt{14}
\v{1}One day Jonathan told his armor bearer,\fnote{Lit. \fbib{the young man who carries his weapons}} ``Come, let's go over to the Philistine garrison which is on the other side,'' but he did not tell his father. \v{2}Saul was sitting on the outskirts of Geba under the pomegranate tree which was at Migron, and with him\fnote{Lit. \fbib{the people with him}} were about 600 men. \v{3}Along with him were Ahitub's son Ahijah, Ichabod's brother, who was Phineas' son and a grandson of Eli the priest of the \divine{Lord} at Shiloh, who was carrying the ephod. The people did not know that Jonathan had gone.

\v{4}Now in the pass\fnote{Lit. \fbib{between the passes}} through which Jonathan planned to get across to the Philistine garrison, there was a sharp crag\fnote{Lit. \fbib{tooth of a crag}} on one side and a sharp crag on the other side. The name of the one was Bozez, and the name of the other was Seneh. \v{5}One crag rose on the north opposite Michmash, and the other on the south opposite Geba.

\v{6}Jonathan told his armor bearer,\fnote{Lit. \fbib{the young man carrying his armor}} ``Come, let's go over to the garrison of these uncircumcised ones. Perhaps the \divine{Lord} will work for us, since nothing prevents the \divine{Lord} from delivering, whether by many or by a few.''

\v{7}His armor bearer told him, ``Do whatever you want.\fnote{Lit. \fbib{is in your heart}} Let's move out!\fnote{Lit. \fbib{Turn}} I'm right here with you, as you wish.''\fnote{Lit. \fbib{according to your heart}}

\v{8}Jonathan said, ``Look, we're going over to the men, and we will show ourselves to them. \v{9}If they say to us, `Stay there until we come to you,' then we will stay where we are\fnote{Lit. \fbib{in our place}} and not go up to them. \v{10}But if they say, `Come up and fight us,' then we will go up, for the \divine{Lord} has given them into our hands, and this will be the sign for us.''

\v{11}When the two of them showed themselves to the Philistine garrison, the Philistines said, ``Look, the Hebrews are coming out of the holes where they have been hiding.''

\v{12}The men of the garrison responded to Jonathan and his armor bearer: ``Come up and fight us, and we will show you something.''

Jonathan then told his armor bearer, ``Follow me, for the \divine{Lord} has given them into Israel's control.''

\v{13}Jonathan crawled up on his hands and feet, with his armor bearer following him. The Philistines\fnote{Lit. \fbib{They}} fell before Jonathan, and his armor bearer who was behind him also killed some. \v{14}In the initial attack, Jonathan and his armor bearer struck down about twenty men in an area of about half an acre\fnote{An acre represents the amount of land a yoke of oxen could plow in a day.} of land. \v{15}There was terror in the camp, in the field, and among all the people. Even the garrison and the raiders were terrified. The earth shook, and there was even greater terror.\fnote{Lit. \fbib{it became a terror of God}}

\v{16}Saul's sentries in Gibeah of Benjamin watched as the camp\fnote{Lit. \fbib{the multitude}} was in disarray,\fnote{Lit. \fbib{melted away}} going this way and that.\fnote{Lit. \fbib{here}} \v{17}Saul told the people who were with him, ``Do a roll call\fnote{Lit. \fbib{Number}} and see who has left us.'' They did a roll call,\fnote{Lit. \fbib{numbered}} and Jonathan and his armor bearer were not there.

\v{18}Saul told Ahijah, ``Bring the Ark of God here.'' For at that time the Ark of God was with\fnote{So some mss and ancient versions; MT \fbib{and the Israelis}} the Israelis.

\v{19}While Saul was still speaking to the priest, the commotion in the Philistine camp increased more and more, and Saul told the priest, ``Remove your hand.''\fnote{I.e. from the ephod that the priest was wearing in order to determine God's will as to what the army should do}

\v{20}Then Saul and all the people who were with him assembled and went into battle. Now the swords of all the Philistines were against each other,\fnote{Lit. \fbib{the sword of each man was against his companion}} and there was very great confusion. \v{21}The Hebrews who had previously been with the Philistines, who had gone up with them from the surrounding areas to the camp, even they joined Israel and those who were with Saul and Jonathan. \v{22}All the Israelis who had been hiding in the hill country of Ephraim heard that the Philistines were fleeing, and even they pursued the Philistines\fnote{Lit. \fbib{them}} in the battle. \v{23}On that day the \divine{Lord} delivered Israel, and the battle moved past Beth-aven.
\passage{Saul Issues a Rash Edict}

\v{24}The men of Israel were hard pressed on that day, and Saul required the army to take an oath: ``Cursed is the person who eats food before evening and before I've been avenged of my enemies.'' So no one tasted food.

\v{25}Later on, all the soldiers\fnote{Lit. \fbib{land}} entered the woods, and there was honey on the ground. \v{26}The people came into the woods and there was flowing honey, but no one put his hand to his mouth to eat it because the people were afraid due to the oath. \v{27}But Jonathan had not heard that his father had required the army to swear an oath, so he stretched out the end of the staff that was in his hand and dipped it in the honeycomb. He brought it back to his mouth and his eyes brightened. \v{28}Then one of the people responded: ``Your father strictly ordered the army to take an oath. That's why he said, `Cursed is the person who eats food today,' and so the army is exhausted.''

\v{29}Jonathan said, ``My father has troubled the land. See how my eyes have brightened because I tasted a little of this honey. \v{30}How much better if the army had eaten freely today of their enemy's spoil that they found, because the slaughter among the Philistines has not been great.''

\v{31}That day they struck down the Philistines from Michmash to Aijalon, and the army was very weary. \v{32}The army grabbed the spoil, took sheep, oxen, and calves, and slaughtered them on the ground, and then the army ate them with the blood. \v{33}Someone\fnote{Lit. \fbib{They}} reported this to Saul: ``Right now the army is sinning against the \divine{Lord} by eating meat\fnote{The Heb. lacks \fbib{meat}} with the blood.'' He said, ``You have acted treacherously. Roll a large stone to me today.''

\v{34}Then Saul said, ``Disperse yourselves among the soldiers and say to them, `Let each man bring his ox and his sheep to me, and you are to slaughter them here and eat. But don't sin against the \divine{Lord} by eating meat\fnote{The Heb. lacks \fbib{meat}} with the blood.'\,'' So every soldier brought his ox with him that night, and they slaughtered them there. \v{35}Saul built an altar to the \divine{Lord}; it was the first altar that he built to the \divine{Lord}.

\v{36}Saul said, ``Let's go down after the Philistines tonight and plunder them until dawn, and let's not leave a single one\fnote{Lit. \fbib{a man}} of them alive.''

They said, ``Do whatever seems good to you!''

But the priest said, ``Let's draw near to God here.''

\v{37}Saul inquired of God, ``Shall I go down after the Philistines? Will you give them into the hand of Israel?'' But God\fnote{Lit. \fbib{he}} did not answer him that day.

\v{38}Saul said, ``All you army officers are to come here to find out\fnote{Lit. \fbib{know and see}} what constitutes\fnote{Lit. \fbib{in what is}} this sin today. \v{39}Indeed, as the \divine{Lord} who delivers Israel lives, even if the sin\fnote{Lit. \fbib{it}} is with my son Jonathan, he will surely die!'' Not a single one of the soldiers answered him. \v{40}Then he told all Israel, ``You will be on one side, and I and my son Jonathan will be on the other side.''

The people told Saul, ``Do what seems good to you.''

\v{41}Then Saul told the \divine{Lord} God of Israel, ``Judge us properly.''\fnote{Lit. \fbib{Give perfect}} Jonathan and Saul were selected, but the army was cleared.\fnote{Lit. \fbib{went out}} \v{42}Saul said, ``Cast lots between me and my son Jonathan,'' and Jonathan was selected. \v{43}Saul told Jonathan, ``Tell me what you've done.''

So Jonathan spoke to him: ``I did taste a little honey from the end of the staff that was in my hand. Here I am; I'm ready to die!''

\v{44}Saul said, ``May God do this to me\fnote{So LXX; i.e. may God strike me dead} and even more, if you don't surely die, Jonathan!''

\v{45}Then the army told Saul, ``Shall Jonathan die, who brought about this great deliverance in Israel? As the \divine{Lord} lives, not one hair of his head will fall to the ground, because today he did this with God's help.''\fnote{Lit. \fbib{with God}}

\v{46}Then Saul stopped pursuing\fnote{Lit. \fbib{went up from after}} the Philistines, and the Philistines went back to their territory.
\passage{Saul's Military Victories}

\v{47}When Saul became king over Israel, he fought against all his enemies on every side---against Moab, the Ammonites, Edom, the kings of Zobah, and the Philistines. Everywhere he turned he was victorious.\fnote{Cf. LXX} \v{48}He acted valiantly, defeated Amalek, and delivered Israel from those who had been plundering them.
\passage{Saul's Family}

\v{49}Saul's sons included Jonathan, Ishvi, and Malchi-shua. Of his two daughters, the firstborn was named Merab, and the younger one was named Michal. \v{50}Saul's wife was Ahinoam, daughter of Ahimaaz, while the commander of his army was Saul's uncle Ner's son Abner. \v{51}Saul's father Kish and Abner's father Ner were sons of Abiel. \v{52}There was intense fighting against the Philistines during Saul's entire reign, and whenever Saul discovered a strong or valiant warrior, he would enlist him for service.\fnote{Lit. \fbib{gather him to himself}}
\labelchapt{15}
\passage{Saul Disobeys the \divine{Lord}}

\chapt{15}
\v{1}Samuel told Saul, ``The \divine{Lord} sent me to anoint you king over his people, Israel. Now listen to the words\fnote{Lit. \fbib{the sound of the words}} of the \divine{Lord}. \v{2}This is what the \divine{Lord} of the Heavenly Armies says: `I'll punish Amalek for what he did to Israel, when he set himself against Israel\fnote{Lit. \fbib{him}} in the way, as they were going up from Egypt. \v{3}Now, go and attack Amalek. Completely destroy\fnote{The Heb. term \fbib{destroy} involved consecration of things or people to the \fbib{\divine{Lord}} either by destruction or by an offering; and so throughout the chapter} all that they have. Don't spare them, but put to death both man and woman, child and infant, both ox and sheep, camel and donkey.'\,''

\v{4}Saul summoned the people and mustered them in Telaim, 200,000 foot soldiers and 10,000 men from Judah. \v{5}Saul came to the city of Amalek and set an ambush in the valley. \v{6}Saul told the Kenites, ``Withdraw from the Amalekites so that I don't destroy you with them, for you showed kindness to all the Israelis when they departed from Egypt.'' So the Kenites withdrew from the Amalekites. \v{7}Saul attacked the Amalekites from Havilah to Shur, which is east of Egypt. \v{8}He captured alive Agag king of Amalek, but he completely destroyed all the people, executing them with swords. \v{9}Saul and the people spared Agag and the best of the sheep and cattle---the fattened animals and lambs---along with all that was good. They were not willing to completely destroy them, but they did completely destroy everything that was worthless and inferior.
\passage{The \divine{Lord} Rejects Saul}

\v{10}This message from the \divine{Lord} came to Samuel: \v{11}``I regret that I made Saul king, because he has turned away from following me and has not carried out my commands.'' Samuel was angry, and he cried out to the \divine{Lord} all night.

\v{12}Samuel got up early in the morning to meet Saul, but Samuel was told, ``Saul went up to Carmel to set up a monument for himself. Then he turned around and traveled on to Gilgal.''

\v{13}Samuel approached Saul. ``May the \divine{Lord} bless you,'' Saul said. ``I've carried out the \divine{Lord}'s command.''

\v{14}Samuel said, ``Then what is this bleating of sheep in my ears and the lowing of cattle that I hear?''

\v{15}Saul replied, ``They brought them from the Amalekites. The people spared the best of the sheep and cattle to offer sacrifices to the \divine{Lord} your God, and the rest they completely destroyed.''

\v{16}``Be quiet!'' Samuel said. ``I'll tell you what the \divine{Lord} told me last night.''

Saul told him, ``Speak.''

\v{17}So Samuel replied, ``Is it not true that though you were small in your own eyes you became head of the tribes of Israel, and the \divine{Lord} anointed you king over Israel? \v{18}The \divine{Lord} sent you on a mission: `Go and completely destroy the sinners, the Amalekites, and fight against them until they're destroyed.' \v{19}Why didn't you obey the \divine{Lord}, but grabbed the spoil and did evil in the \divine{Lord}'s sight?''

\v{20}Saul told Samuel, ``I did obey the \divine{Lord}. I went on the mission on which the \divine{Lord} sent me, I brought Agag king of Amalek, and I completely destroyed the Amalekites. \v{21}The people took some of the spoil---sheep, cattle, and the best of what was to be completely destroyed---to sacrifice to the \divine{Lord} your God at Gilgal.''

\v{22}Samuel said,

\begin{poetry}
\poeml ``Does the \divine{Lord} delight as much in burnt offerings and sacrifices \\
\poemll    as in obeying the \divine{Lord}? \\
\poeml Surely, to obey is better than sacrifice, \\
\poemll    to pay attention is better\fnote{The Heb. lacks \fbib{is better}} than the fat of rams. \\
\poeml \v{23}Indeed, rebellion is the sin of divination, \\
\poemll    and arrogance is iniquity and idolatry. \\
\poeml Because you have rejected this message from the \divine{Lord}, \\
\poemll    he has rejected you from being king.''
\end{poetry}

\v{24}``I've sinned,'' Saul replied to Samuel. ``I've broken the \divine{Lord}'s command and your word, because I was afraid of the people and listened to them. \v{25}Now, please forgive my sin and return with me so I may worship the \divine{Lord}.''

\v{26}Samuel told Saul, ``I won't return with you because you have rejected the message from the \divine{Lord}, and the \divine{Lord} has rejected you from being king over Israel.''

\v{27}As Samuel turned to go Saul\fnote{Lit. \fbib{he}} seized him by the corner of his robe, and it tore. \v{28}Samuel told him, ``The \divine{Lord} has torn the kingdom of Israel away from you today, and he has given it to your neighbor who is better than you. \v{29}Moreover, the Glory of Israel does not lie or change his mind, for he's not a man that he should change his mind.''

\v{30}``I've sinned,'' Saul\fnote{Lit. \fbib{He}} said. ``But please honor me now before the elders of my people and before Israel, and return with me so I may worship the \divine{Lord} your God.'' \v{31}Samuel returned, following Saul, and Saul worshipped the \divine{Lord}.
\passage{Samuel Executes King Agag}

\v{32}Then Samuel said, ``Bring Agag king of Amalek to me.''

Agag came to him in fetters, saying to himself,\fnote{The Heb. lacks \fbib{to himself}} ``Surely the bitterness of death is past.''

\v{33}Samuel said, ``Just as your sword has made women childless, so your mother will be childless among women.'' Then Samuel cut Agag into pieces in the \divine{Lord}'s presence in Gilgal.

\v{34}Then Samuel went to Ramah, and Saul went to his house in Gibeah of Saul. \v{35}Samuel did not see Saul again until the day of his death, but Samuel grieved over Saul, and the \divine{Lord} regretted that he had made Saul king over Israel.
\labelchapt{16}
\passage{David Anointed to Succeed Saul}

\chapt{16}
\v{1}The \divine{Lord} told Samuel, ``How long will you grieve over Saul, since I've rejected him from being king over Israel? Fill your horn with oil and go. I'm sending you to Jesse from Bethlehem because I've chosen for myself one of his sons as king.''

\v{2}Samuel said, ``How can I go? Saul will hear about this\fnote{The Heb. lacks \fbib{about this}} and kill me!''

The \divine{Lord} said, ``Take a heifer\fnote{I.e. a young cow that has not yet had a calf} with you and say, `I've come to offer a sacrifice to the \divine{Lord}.' \v{3}You are to invite Jesse to the sacrifice, and I'll show you what you are to do. You are to anoint for me the one I tell you.''

\v{4}Samuel did what the \divine{Lord} said and went to Bethlehem. The elders of the town came out to meet him trembling, and said, ``May your coming be in peace.''

\v{5}He said, ``Peace, I've come to sacrifice to the \divine{Lord}. Consecrate yourselves and come with me to the sacrifice.'' Samuel\fnote{Lit. \fbib{He}} consecrated Jesse and his sons and invited them to the sacrifice.

\v{6}When they arrived, Samuel\fnote{Lit. \fbib{he}} saw Eliab, and said, ``Surely he's the \divine{Lord}'s\fnote{Lit. \fbib{his}} anointed.''\fnote{Lit. \fbib{surely the \divine{Lord}'s anointed is before him}}

\v{7}The \divine{Lord} told Samuel, ``Don't look at his appearance or his height,\fnote{Lit. \fbib{the height of his stature}} for I've rejected him. Truly, God does not see\fnote{The Heb. lacks \fbib{see}} what man sees, for man looks at the outward appearance, but the \divine{Lord} sees the heart.''

\v{8}Then Jesse summoned Abinadab and brought him before Samuel, and he said, ``Neither has the \divine{Lord} chosen this one.'' \v{9}Then Jesse brought Shammah, and he said, ``Neither has the \divine{Lord} chosen this one.'' \v{10}Jesse brought seven of his sons before Samuel, and Samuel told Jesse, ``The \divine{Lord} has not chosen these.''

\v{11}Then Samuel told Jesse, ``Are these all the young men?'' He said, ``There yet remains the youngest one, and right now he's tending the sheep.'' Samuel told Jesse, ``Send someone to get him,\fnote{Lit. \fbib{send and get him}} for we won't do anything else\fnote{Lit. \fbib{we won't turn aside}} until he arrives here.'' \v{12}So he sent and brought him. He had a dark, healthy complexion, with beautiful eyes, and he was handsome. The \divine{Lord} said, ``Get up and anoint him, for this is the one.''
\passage{God's Spirit Comes on David and Departs from Saul}

\v{13}Then Samuel took the horn of oil and anointed David\fnote{Lit. \fbib{him}} in the presence of his brothers, and the Spirit of the \divine{Lord} came on David from that day forward. Then Samuel got up and went to Ramah.

\v{14}The Spirit of the \divine{Lord} departed from Saul, and an evil spirit from the \divine{Lord} troubled him. \v{15}Saul's servants told him, ``Look, an evil spirit from God is troubling you. \v{16}Let our lord order his servants who attend you\fnote{Lit. \fbib{who are before you}} to look for a man who is skilled in playing the lyre. And then when an evil spirit from God comes on you, he will play\fnote{Lit. \fbib{play with his hand}} and you will be better.''

\v{17}Saul told his servants, ``Find\fnote{Lit. \fbib{Provide}} a man for me who can play well and bring him to me.''

\v{18}One of the young men answered: ``Look, I've seen a son of Jesse the Bethlehemite who is skilled in playing. The man is a valiant soldier, gifted in speech, and handsome. And the \divine{Lord} is with him.''

\v{19}So Saul sent messengers to Jesse and said, ``Send me your son David, who is with the sheep.''

\v{20}Jesse took a donkey loaded with bread, a container of wine, and one kid, and sent them to Saul along with his son David. \v{21}David went to Saul and began to serve him.\fnote{Lit. \fbib{stood before him}} Saul loved him very much, and he became his armor bearer. \v{22}Saul sent a messenger\fnote{The Heb. lacks \fbib{a messenger}} to Jesse to tell him, ``Allow David to serve me, because I'm pleased with him.''\fnote{Lit. \fbib{because he has found favor in my sight}} \v{23}Whenever an evil\fnote{The Heb. lacks \fbib{evil}} spirit from God came to Saul, David would take the lyre and play it.\fnote{Lit. \fbib{play with his hand}} Relief would come to Saul and he would be better, because the evil spirit would leave him.
\labelchapt{17}
\passage{Goliath Challenges the Israelis}

\chapt{17}
\v{1}The Philistines assembled their army for battle. They were assembled at Socoh, which belongs to Judah, and they camped between Socoh and Azekah, in Ephes-dammim. \v{2}Saul and the Israelis assembled and camped in the valley of Elah, where they set up their forces to meet the Philistines. \v{3}The Philistines were standing on the hill on one side while the Israelis were standing on the hill on the other side, with the valley between them.

\v{4}A champion named Goliath from Gath came out from the Philistine camp. He was four cubits and a span\fnote{I.e. about six and a half feet; so DSS 4QSam\textsuperscript{a} and LXX; MT reads \fbib{six cubits and a span} (i.e. nine and a half feet)} tall, \v{5}wore a bronze helmet on his head, and wore bronze scale armor that weighed about 5,000 shekels.\fnote{I.e. about 125 pounds at 0.4 shekels per ounce} \v{6}He had bronze armor on his legs\fnote{Or \fbib{bronze greaves}; i.e. leg armor worn below the knees} and carried a bronze javelin slung\fnote{The Heb. lacks \fbib{slung}} between his shoulders. \v{7}The shaft of his spear was like a weaver's beam and the iron point of his spear weighed 600 shekels.\fnote{I.e. about 15 pounds at 0.4 shekels per ounce} A man carrying his shield walked in front of him.

\v{8}He stood still and called out to the ranks of Israel, ``Why should you move into position for battle? Am I not a Philistine and you Saul's servants? Choose a man for yourselves to come down against me. \v{9}If he's able to fight me and strike me down, then we will become your servants; but if I prevail against him and strike him down, then you will become our servants and serve us.'' \v{10}The Philistine said, ``I defy\fnote{Or \fbib{challenge}} the ranks of Israel today. Send me one man and let's fight together.'' \v{11}When Saul and all the Israelis heard these words of the Philistine, they were dismayed and very frightened.
\passage{David Comes to the Camp}

\v{12}David was the son of that Ephrathite man named Jesse from Bethlehem in Judah. He had eight sons; at the time when Saul was king he was old, having lived to an advanced age. \v{13}The three oldest sons of Jesse followed Saul into battle. The names of his three sons who went to the battle were his firstborn Eliab, Abinadab, his second son, and Shammah, the third. \v{14}David was the youngest, while the three oldest had followed Saul. \v{15}And David would go back and forth from Saul to tend his father's sheep in Bethlehem. \v{16}For 40 days the Philistine would come forward, morning and evening, to take his position.

\v{17}Jesse told his son David, ``Take this ephah\fnote{I.e. about a half-bushel; an ephah was a measure of dry capacity equal to about one half of a bushel} of roasted grain to your brothers, along with these ten loaves of bread, and quickly take them to your brothers in the camp. \v{18}Take these ten pieces of cheese to the commander of the unit,\fnote{Lit. \fbib{thousand}} check on the well-being of your brothers, and bring something back from them. \v{19}Saul, your brothers,\fnote{Lit. \fbib{they}} and all the men of Israel are in the valley of Elah fighting with the Philistines.'' \v{20}David got up early in the morning, left the sheep with a keeper, took the supplies,\fnote{The Heb. lacks \fbib{the supplies}} and went as Jesse had directed him. He arrived at the encampment\fnote{Or \fbib{entrenchment}} as the army was going out to the battle line, shouting the battle cry.
\passage{David Hears Goliath's Challenge}

\v{21}Israel and the Philistines moved into position for battle, battle line facing battle line. \v{22}David left the supplies he had with him in the care of the supply keeper and ran to the battle line. When he arrived there, he asked his brothers about their well-being. \v{23}As he was speaking with them, the Philistine champion named Goliath from Gath came up from the Philistine battle lines and spoke his usual words,\fnote{Lit. \fbib{according to these words}} as David listened. \v{24}When all the Israelis saw the man, they fled from him and were very frightened.

\v{25}``Did all of you see this man coming up?'' one Israeli asked. ``He comes up to defy\fnote{Or \fbib{challenge}} Israel, and the king will richly reward the man who kills him. He will give his daughter to him and will make his father's house tax\fnote{The Heb. lacks \fbib{tax}} free in Israel.''

\v{26}David asked the men who were standing by him, ``What will be done for the man who kills this Philistine and takes away the reproach from Israel? Indeed, who is this uncircumcised Philistine that he should defy\fnote{Or \fbib{challenge}} the armies of the living God?''

\v{27}The people also told him the same thing,\fnote{Lit. \fbib{spoke to him according to this word}} saying, ``This is what will be done for the man who kills him.''

\v{28}Eliab his oldest brother heard him talking to the men. Eliab was angry with David and said, ``Why did you come down here? And who did you leave those few sheep with in the wilderness? I know your insolence and wicked intentions.\fnote{Lit. \fbib{wickedness of your heart}} You came down just to see the battle!''

\v{29}``What have I done now?'' David asked. ``It was just a question,\fnote{Lit. \fbib{a word}} wasn't it?'' \v{30}Then he turned from him toward another person and asked the same thing. The people replied to him the same way as the first one had.
\passage{David Accepts the Challenge}

\v{31}When the words that David had spoken were heard, they were reported to Saul, and he sent for him. \v{32}David told Saul, ``Let no one's courage\fnote{Lit. \fbib{heart}} fail because of him; your servant will go fight this Philistine.''

\v{33}Saul told David, ``You can't go against this Philistine and fight him. You are only a young man, but he has been a warrior since his youth.''

\v{34}David told Saul, ``Your servant has been a shepherd for his father. When a lion or bear came and took a lamb from the flock, \v{35}I would go out after it, strike it down, and rescue the lamb\fnote{The Heb. lacks \fbib{the lamb}} from its mouth. Then when it rose up against me, I would grab it by its fur,\fnote{Lit. \fbib{beard}} strike it down, and kill it. \v{36}Your servant has struck down both lions and bears, and this uncircumcised Philistine will be like one of them, since he defied\fnote{Or \fbib{challenged}} the armies of the living God.'' \v{37}David continued, ``The \divine{Lord} who delivered me from the power of\fnote{Or \fbib{hand of}} the lion and the power of\fnote{Or \fbib{hand of}} the bear will also deliver me from the power of\fnote{Or \fbib{hand of}} this Philistine.''

Saul told David, ``Go! And may the \divine{Lord} be with you.''

\v{38}Saul put his garments on David, set a bronze helmet on his head, and put armor on him. \v{39}David strapped Saul's\fnote{Lit. \fbib{his}} sword over his garments and tried to walk, but\fnote{Lit. \fbib{for}} he was not used to the armor.\fnote{Lit. \fbib{he had not tested}} David told Saul, ``I can't walk in these because I'm not used to them,''\fnote{Lit. \fbib{I have not tested}} and then took them off. \v{40}He took his staff in his hand and chose for himself five smooth stones from the brook and put them in the pouch in his shepherd's bag. He approached the Philistine with his sling in his hand.
\passage{David Defeats Goliath}

\v{41}With a man carrying his shield in front of him, the Philistine kept coming closer to David. \v{42}When the Philistine looked and saw David, he had contempt for him, because he was only a young man. David had a dark, healthy complexion and was handsome. \v{43}The Philistine asked David, ``Am I a dog that you come at me with sticks?'' Then the Philistine cursed David by his own gods and \v{44}told David, ``Come to me! I'll give your flesh to the birds of the sky and to the beasts of the field.''

\v{45}Then David told the Philistine, ``You come at me with a sword, a spear, and a javelin, but I come to you in the name of the \divine{Lord} of the Heavenly Armies, the God of the armies of Israel whom you have defied.\fnote{Or \fbib{challenged}} \v{46}This very day the \divine{Lord} will deliver you into my hand, and I'll strike you down and remove your head from you. And this very day I'll give the dead bodies of the Philistine army to the birds of the sky and to the animals of the earth, so that all the earth will know that there is a God in Israel, \v{47}and this whole congregation will know that the \divine{Lord} does not deliver by sword or spear. Indeed, the battle is the \divine{Lord}'s and he will give you into our hands.''

\v{48}When the Philistine got up and came closer to meet David, David quickly ran to the battle line to meet the Philistine. \v{49}David reached his hand into the bag, took out a stone, slung it, and struck the Philistine in his forehead. The stone sunk into his forehead, and he fell on his face to the ground. \v{50}David defeated the Philistine with a sling and a stone; he struck down the Philistine and killed him, and there was no sword in David's hand. \v{51}David ran and stood over the Philistine. He took the Philistine's\fnote{Lit. \fbib{his}} sword, pulled it from its sheath, killed him, and then he cut off his head with it. When the Philistines saw that their champion was dead, they fled. \v{52}The men of Israel and Judah got up with a shout and pursued the Philistines as far as the entrance to\fnote{Lit. \fbib{until you enter}} the valley and to the gates of Ekron. Wounded Philistines fell along the way to Shaaraim as far as Gath and Ekron. \v{53}The Israelis returned from pursuing the Philistines and plundered their camp. \v{54}David took the Philistine's head and brought it to Jerusalem, but he put Goliath's\fnote{Lit. \fbib{his}} weapons in his tent.

\v{55}When Saul saw David going out to meet the Philistine, he asked Abner, the commander of the army, ``Whose son is this young man, Abner?''

Abner said, ``As surely as you live, your majesty, I don't know.''

\v{56}The king replied, ``Go find out whose son the young man is.''

\v{57}When David returned from striking down the Philistine, Abner took him and brought him to Saul with the Philistine's head in his hand. \v{58}Saul told him, ``Whose son are you, young man?''

David said, ``The son of your servant Jesse of Bethlehem.''
\labelchapt{18}
\passage{Jonathan and David's Friendship}

\chapt{18}
\v{1}When David finished speaking with Saul, Jonathan became a close friend to David,\fnote{Lit. \fbib{Jonathan's soul was knit with David's soul}} and Jonathan\fnote{Lit. \fbib{he}} loved him as himself. \v{2}Saul took David\fnote{Lit. \fbib{him}} that day and did not let him return to his father's house. \v{3}Jonathan made a covenant with David because he loved him as he loved himself. \v{4}Jonathan took off the robe that he had on and gave it to David, along with his coat, his sword, his bow, and his belt. \v{5}David went out and was successful everywhere Saul sent him, and Saul put him in charge of the troops. This pleased the entire army,\fnote{Or \fbib{pleased all the people}} as well as Saul's officials.\fnote{Or \fbib{servants}}
\passage{Saul's Jealousy of David}

\v{6}When David returned from defeating the Philistine, as they were entering the city, women from all the towns of Israel came out to meet King Saul, singing and dancing as they joyously played tambourines and lyres. \v{7}As the women sang and played, they said,

\begin{poetry}
\poeml ``Saul has struck down his thousands \\
\poemll    but David his ten thousands.''
\end{poetry}

\v{8}Saul was very angry and he did not like what the women sang. He told himself,\fnote{The Heb. lacks \fbib{to himself}} ``They have attributed tens of thousands to David, but to me they have attributed thousands. What else can he have but the kingdom?'' \v{9}From then on Saul kept his eye on David.\fnote{Or \fbib{eyed David with suspicion}}

\v{10}The next day, while David was playing the lyre\fnote{Lit. \fbib{playing with his hand}} as he had before, the evil spirit from the \divine{Lord} attacked Saul, and he began to rave\fnote{Or \fbib{prophesy}} inside the house with a spear in his hand. \v{11}Saul hurled it, thinking,\fnote{Lit. \fbib{saying}} ``I'll pin David to the wall.'' But David escaped from him twice.

\v{12}Now Saul was afraid of David because the \divine{Lord} was with him and had departed from Saul. \v{13}Saul removed David\fnote{Lit. \fbib{him}} from his presence and made him an officer over a division of soldiers.\fnote{Lit. \fbib{over a thousand}} So David led the troops in battle.\fnote{Lit. \fbib{went out and came in before the people} (i.e. the soldiers)} \v{14}David was successful in all that he did, for the \divine{Lord} was with him. \v{15}When Saul saw that David\fnote{Lit. \fbib{he}} was highly successful, he feared him. \v{16}But all Israel and Judah loved David because he led them in battle.\fnote{Lit. \fbib{went out and came in before them}}
\passage{David Marries Saul's Daughter}

\v{17}Saul told David, ``Here is my older daughter Merab. I'll give her to you as a wife. Just be an excellent soldier for me and fight the \divine{Lord}'s battles.'' Now Saul told himself,\fnote{The Heb. lacks \fbib{to himself}} ``I won't harm him myself.\fnote{Lit. \fbib{Let not my hand be against him}} Instead, I'll let the Philistines harm him.''\fnote{Lit. \fbib{let the hand of the Philistines be against him}}

\v{18}David told Saul, ``Who am I and what is my life or my father's family in Israel that I should be the king's son-in-law?'' \v{19}And when the time came to give Saul's daughter Merab to David, she was given as a wife to Adriel of Meholah.

\v{20}Saul's daughter Michal loved David. Saul was informed of this and he liked the idea.\fnote{Lit. \fbib{the matter was straight in his eyes}} \v{21}Saul told himself,\fnote{The Heb. lacks \fbib{to himself}} ``I'll give her to him and she can be a snare to him and the Philistines will harm him.''\fnote{Lit. \fbib{so the hand of the Philistines will be against him}} So Saul told David, ``For a second time you can be my son-in-law today.''

\v{22}Saul commanded his officials,\fnote{Or \fbib{servants}} ``Speak with David privately and say, `Look, the king delights in you, and all his officials\fnote{Or \fbib{servants}} love you. Now become the king's son-in-law.'\,''

\v{23}Saul's officials\fnote{Or \fbib{servants}} delivered this message to David,\fnote{Lit. \fbib{spoke these words in the ears of David}} and he\fnote{Lit. \fbib{David}} asked, ``Is becoming the king's son-in-law an unimportant thing to you? I'm a poor and unimportant man.''

\v{24}Saul's officials\fnote{Or \fbib{servants}} reported to him: ``This is what David said.''

\v{25}Saul said, ``This is what you are to tell David, `The king desires no bride price except 100 Philistine foreskins to take vengeance on the king's enemies.'\,'' Now Saul thought he would cause David to die at the hand of the Philistines. \v{26}When his officials\fnote{Or \fbib{servants}} delivered this message to David, David decided it would be a good thing to become the king's son-in-law. Before the time was up, \v{27}David got up, went out with his men, and struck down 200 Philistine men. David brought their foreskins and gave them all to the king so he could become the king's son-in-law. So Saul gave him his daughter Michal as a wife. \v{28}As Saul continued to observe, he realized that the \divine{Lord} was with David and that Saul's daughter Michal loved him. \v{29}Then Saul was even more afraid of David, and Saul was David's enemy from that time on.\fnote{Lit. \fbib{all the days}}

\v{30}The Philistine commanders would go out to fight\fnote{The Heb. lacks \fbib{to fight}} and whenever they did, David was more successful than any of Saul's other leaders.\fnote{Or \fbib{servants}} His name was held in high esteem.
\labelchapt{19}
\passage{Jonathan Intercedes for David}

\chapt{19}
\v{1}Saul told his son Jonathan and all his officials\fnote{Or \fbib{servants}} to kill David, but Saul's son Jonathan was very fond of\fnote{Lit. \fbib{took great delight in}} David. \v{2}So Jonathan told David, ``My father Saul is trying to kill you. In the morning be careful and stay hidden in a secret place. \v{3}I'll go out and stand by my father in the field where you are. I'll speak to my father about you. If I find out what he intends to do,\fnote{The Heb. lacks \fbib{he intends to do}} I'll tell you.''

\v{4}Jonathan spoke to his father Saul favorably about David. ``The king shouldn't wrong his servant David because he has not wronged you and because what he has done has been very beneficial for you. \v{5}He risked his life\fnote{Lit. \fbib{put his life in his hand}} and struck down the Philistine, and the \divine{Lord} brought about a spectacular deliverance for all Israel. You saw that and rejoiced, so why would you do wrong and shed innocent blood\fnote{Lit. \fbib{do wrong with innocent blood}} by killing David without cause?'' \v{6}Saul listened to Jonathan, and swore by the life of the \divine{Lord} that David\fnote{Lit. \fbib{he}} would not be killed. \v{7}Jonathan summoned David and told him all this.\fnote{Lit. \fbib{all these words}} Then Jonathan brought David to Saul, and David\fnote{Lit. \fbib{he}} served him\fnote{Lit. \fbib{was in his presence}} as before.
\passage{Saul Again Tries to Kill David}

\v{8}The war continued and David went out to fight against the Philistines. He thoroughly defeated them,\fnote{Lit. \fbib{he struck them down with a great slaughter}} and they fled before David.\fnote{Lit. \fbib{him}} \v{9}The evil spirit from the \divine{Lord} attacked Saul while he was sitting in his house with his spear in his hand and David was playing the lyre. \v{10}Saul tried to pin David to the wall with the spear, but he jumped away from Saul and the spear stuck in the wall. That night David escaped and fled.
\passage{Michal Helps David Escape}

\v{11}Saul sent messengers to David's house to watch him so they could kill him in the morning. David's wife, Michal, told him, ``If you don't escape with your life tonight, tomorrow you'll be put to death.'' \v{12}So Michal let David down through the window, and he escaped and fled. \v{13}Then Michal took the household idol\fnote{Heb. \fbib{teraphim}} and laid it on the bed with a cover of goat hair placed at its head. Then she covered it with clothes.

\v{14}When Saul sent the messengers to take David, Michal said, ``He's sick.''

\v{15}Then Saul sent messengers to check on\fnote{Or \fbib{to see}} David. He told them, ``Bring him to me on the bed so I may kill him.''\fnote{Lit. \fbib{in order to kill him}} \v{16}The messengers went in, and there was the household idol in the bed with the cover of goat hair at its head!

\v{17}Then Saul told Michal, ``Why did you deceive me like this and let my enemy go so he could escape?''

Michal told Saul, ``He told me, `Let me go or I'll kill you!'\,''\fnote{Lit. \fbib{why should I kill you?}}
\passage{Saul Prophesies at Ramah and David Escapes}

\v{18}David escaped and fled. He came to Samuel at Ramah and told him all that Saul had done to him. Then he and Samuel went and stayed at Naioth. \v{19}It was reported to Saul saying, ``David is at Naioth in Ramah right now.'' \v{20}Saul sent messengers to take David, and they saw a group of prophets caught up in prophetic ecstasy,\fnote{Or \fbib{prophesying}} with Samuel standing beside them leading them. Then the Spirit of God came on Saul's messengers, and they also were caught up in prophetic ecstasy.\fnote{Or \fbib{prophesied}}

\v{21}They reported this to Saul, he sent other messengers, and they also were caught up in prophetic ecstasy.\fnote{Or \fbib{prophesied}} \v{22}Then Saul himself went to Ramah, and he arrived at the large well that is in Secu. He asked, ``Where are Samuel and David?''

Someone\fnote{Lit. \fbib{He}} replied, ``They're at Naioth in Ramah.'' \v{23}Saul went to Naioth in Ramah, and the Spirit of God came on him also. He continued in prophetic ecstasy\fnote{Or \fbib{he continued to prophesy}} until he came to Naioth in Ramah. \v{24}He also removed his clothes and was caught up in prophetic ecstasy\fnote{Or \fbib{prophesied}} right in front of Samuel! He fell down naked and remained there all that day and all night. That is why people say,\fnote{Lit. \fbib{Therefore, they say}} ``Is Saul also among the prophets?''
\labelchapt{20}
\passage{David and Jonathan's Discussion}

\chapt{20}
\v{1}David fled from Naioth in Ramah. He came to Jonathan and said, ``What have I done? What is my crime, and how have I wronged your father so that he's determined to kill me?\fnote{Lit. \fbib{seeks my life}}

\v{2}Jonathan\fnote{Lit. \fbib{He}} told him, ``Far from it! You won't die. Look, my father never does anything, great or small, without telling me;\fnote{Lit. \fbib{revealing it in my ear}} so why should my father hide this thing from me? It's not like that!''

\v{3}David again took an oath: ``Your father certainly knows that I've found favor with you, and so he told himself,\fnote{The Heb. lacks \fbib{to himself}} `Jonathan must not know this so he won't be upset.' But as certainly as the \divine{Lord} is alive and living, and as certainly as I'm alive and living, too, there is only a step between me and death.''

\v{4}Jonathan told David, ``Whatever you say, I'll do.''

\v{5}David told Jonathan, ``Look, the New Moon is tomorrow, and I'm expected to sit down with the king to eat. Let me go so I can hide in the field until the evening of the day after tomorrow.\fnote{Lit. \fbib{until the third evening}} \v{6}If your father actually notices that I'm not there,\fnote{The Heb. lacks \fbib{that I'm not there}} then you are to say, `David urgently requested that I allow him to run to his hometown of Bethlehem because the yearly sacrifice for the entire family was taking place there.' \v{7}If he says, `Good,' then your servant will be safe.\fnote{Lit. \fbib{there will be peace for your servant}} But if he actually gets angry, you will know that his intentions are evil.\fnote{Lit. \fbib{that evil has been determined by him}} \v{8}Now, show gracious kindness to your servant because you have entered into a sacred covenant\fnote{Lit. \fbib{a covenant of the \divine{Lord}}} with your servant. If there is iniquity in me, then kill me yourself---why should you bring me to your father?''

\v{9}``Nonsense!'' Jonathan replied. ``If I actually knew that my father intended evil against you, wouldn't I tell you about it?''

\v{10}Then David told Jonathan, ``Who will tell me if your father answers you harshly?''
\passage{David and Jonathan Make a Covenant}

\v{11}Then Jonathan told David, ``Come, let's go into the field.'' So the two of them went into the field. \v{12}Jonathan told David, ``The \divine{Lord} God of Israel is my witness\fnote{The Heb. lacks \fbib{is my witness}} that I'll carefully question my father by tomorrow or the next day. And if the response\fnote{Lit. \fbib{it}} is favorable for David, will I not then send word\fnote{The Heb. lacks \fbib{word}} to you and let you know?\fnote{Lit. \fbib{reveal in your ear}} \v{13}But if my father intends to harm you, may the \divine{Lord} strike me dead\fnote{Lit. \fbib{may the \divine{Lord} do to Jonathan and more also}; This oath would have been accompanied by some symbolic action such as simulating the plunge of a knife into one's heart.} if I don't let you know and send you away so you may go safely. May the \divine{Lord} be with you as he has been with my father. \v{14}If I remain alive, don't fail to show me the \divine{Lord}'s gracious love so that I don't die. \v{15}And don't stop showing your gracious love to my family forever, not even when the \divine{Lord} eliminates each of David's enemies from the surface of the earth.'' \v{16}Jonathan made a covenant with the house of David: ``May the \divine{Lord} punish any violation of this covenant by the hand of David's enemies.''\fnote{Lit. \fbib{may the \divine{Lord} seek from the hand of David's enemies}} \v{17}Jonathan made David vow again out of his love for him, because he loved him as himself.
\passage{Jonathan's Signal to David}

\v{18}Jonathan told him, ``Tomorrow is the New Moon, and you will be missed because your seat is empty. \v{19}On the third day go down quickly and come to the place where you hid earlier.\fnote{Lit. \fbib{on the day of the event}} Remain beside the rock at Ezel. \v{20}I'll shoot three arrows to the side of the rock\fnote{The Heb. lacks \fbib{of the rock}} as though I were shooting at a target. \v{21}Then I'll send a servant,\fnote{Or \fbib{boy}} saying,\fnote{The Heb. lacks \fbib{saying}} `Go, find the arrows.' If I specifically say to the servant,\fnote{Or \fbib{boy}} `Look, the arrows are on this side of you, get them,' then come out because it's safe for you, and, as surely as the \divine{Lord} lives, there is no danger.\fnote{Lit. \fbib{thing}} \v{22}But if I say this to the young man: `Look, the arrows are beyond you,' then go, for the \divine{Lord} has sent you away. \v{23}As for the matter about which you and I spoke, remember that\fnote{Or \fbib{look,}} the \divine{Lord} is a witness\fnote{The Heb. lacks \fbib{a witness}} between us forever.''
\passage{Jonathan Intercedes for David}

\v{24}David hid in the field. When the New Moon arrived, the king sat down to eat. \v{25}The king sat down at his place as before, in the seat by the wall. Jonathan stood while Abner sat next to Saul, but David's place was empty. \v{26}Saul didn't say anything that day because he told himself,\fnote{The Heb. lacks \fbib{to himself}} ``Something has happened; he's unclean; surely he's not clean.''

\v{27}But the next day, on the second day of the New Moon, David's place was empty, and so Saul told his son Jonathan, ``Why didn't Jesse's son come to the festival, either yesterday or today?''

\v{28}Jonathan answered Saul, ``David urgently requested that I let him go to Bethlehem. \v{29}He said, `Please let me go because our family has a sacrifice in the town, and my brother has ordered me to come. Now, if it's acceptable to you,\fnote{Lit. \fbib{if I have found favor in your eyes}} please let me get away so I can see my brothers.' That's the reason he didn't come to the king's table.''
\passage{Saul's Anger toward Jonathan}

\v{30}Saul flew into a rage and told Jonathan, ``You son of a perverse and rebellious woman! Don't I know that you have chosen Jesse's son to your shame and to the shame of your mother who bore you?\fnote{Lit. \fbib{to the shame of your mother's nakedness}} \v{31}As long as\fnote{Lit. \fbib{all the days that}} Jesse's son lives on the earth, neither you nor your kingdom will be established! Now send someone and bring David\fnote{Lit. \fbib{him}} to me. He's a dead man!''

\v{32}Jonathan asked his father Saul, ``Why should he be killed? What did he do?'' \v{33}Then Saul threw the spear that was beside him to strike Jonathan\fnote{Lit. \fbib{him}} down. So Jonathan realized that his father was determined to kill David. \v{34}So on the second day of the New Moon Jonathan angrily got up from the table without eating because he was upset about David, and because his father had humiliated him.
\passage{Jonathan Warns David}

\v{35}In the morning Jonathan, accompanied by a servant,\fnote{Lit. \fbib{young man}} went out to the field for the appointment with David. \v{36}Jonathan\fnote{Lit. \fbib{He}} told his servant,\fnote{Lit. \fbib{young man}} ``Run, find the arrows that I'm shooting.'' As the servant\fnote{Lit. \fbib{young man}} ran, Jonathan\fnote{Lit. \fbib{he}} shot the arrow beyond him. \v{37}The servant\fnote{Lit. \fbib{young man}} came to the place where Jonathan had shot it, and Jonathan called out to him,\fnote{Lit. \fbib{young man}} ``The arrow is beyond you, isn't it?'' \v{38}Jonathan called out to the servant,\fnote{Lit. \fbib{young man}} ``Hurry, be quick, don't stand around.'' Jonathan's servant\fnote{Lit. \fbib{young man}} picked up the arrow and brought it to his master. \v{39}The servant was not aware of anything. Only Jonathan and David understood what had happened.\fnote{Lit. \fbib{the matter}}

\v{40}Then Jonathan gave his equipment to the servant\fnote{Lit. \fbib{young man}} who was with him and told him, ``Go, take these things to the city.'' \v{41}The servant\fnote{Lit. \fbib{young man}} went. Then David came out from the south side of the rock,\fnote{The Heb. lacks \fbib{of the rock}} fell on his face, and bowed down three times. The men kissed each other, and both of them cried, but David even more. \v{42}Jonathan told David, ``Go in peace since both of us swore in the name of the \divine{Lord}: `May the \divine{Lord} be between me and you, and between my descendants and your descendants forever.'\,''

\fnote{This sentence is 21:1 in MT}Then David\fnote{Lit. \fbib{he}} got up and left, while Jonathan went to the city.
\labelchapt{21}
\passage{David Flees to Nob}

\chapt{21}
\v{1}\fnote{This verse is 21:2 in MT}David came to Nob to Ahimelech the priest, and Ahimelech was trembling as he came\fnote{The Heb. lacks \fbib{as he came}} to meet David. Ahimelech\fnote{Lit. \fbib{He}} told him, ``Why are you alone, and no one with you?''

\v{2}David told Ahimelech the priest, ``The king commanded me about a matter, saying to me, `Don't let anyone know anything about the matter I'm sending you to do\fnote{The Heb. lacks \fbib{to do}} and about which I've commanded you. I've directed the young men to a certain place.' \v{3}Now, what do you have available?\fnote{Lit. \fbib{under your control}} Give me five loaves of bread or whatever you have.''\fnote{Lit. \fbib{what is found}}

\v{4}The priest answered David: ``There is no ordinary bread available;\fnote{Lit. \fbib{under my control}} only consecrated bread, provided that the young men have kept themselves from women.''

\v{5}David answered the priest, saying to him, ``Indeed, women were kept from us as is usual\fnote{Lit. \fbib{as previously}} whenever I go out on a mission,\fnote{The Heb. lacks \fbib{on a mission}} and the equipment\fnote{Or \fbib{vessels}} of the young men is consecrated even when it's an ordinary journey, so how much more is their equipment\fnote{Or \fbib{are their vessels}} consecrated today?'' \v{6}So the priest gave him consecrated bread because no bread was there except the Bread of the Presence that had been removed from the \divine{Lord}'s presence and replaced with hot bread on the day it was taken away.

\v{7}Now, Doeg the Edomite, one of Saul's officials,\fnote{Or \fbib{servants}} was there that day, detained in the \divine{Lord}'s presence. He was the chief of Saul's shepherds.
\passage{David Takes Goliath's Sword}

\v{8}David told Ahimelech, ``Is there no spear or sword available\fnote{Lit. \fbib{under your control}} here? I took neither my sword nor my weapons with me, because the king's mission is urgent.''

\v{9}The priest said, ``The sword of Goliath the Philistine, whom you struck down in the Valley of Elah is wrapped up in a cloth behind the ephod.\fnote{The ephod was a type of vest normally worn by the priests} If you want it, take it because there is no other except it here.''

So David said, ``There is none like it. Give it to me.''
\passage{David Flees to Gath}

\v{10}David got up that day and fled from Saul, and he went to King Achish of Gath. \v{11}The officials\fnote{Or \fbib{servants}} of Achish told him, ``Isn't this David, king of the land? Isn't this the one about whom they sang as they danced,

\begin{poetry}
\poeml `Saul has struck down his thousands, \\
\poemll    but David his ten thousands'?''
\end{poetry}

\v{12}David took these words seriously,\fnote{Lit. \fbib{paid attention to these words}} and he was very frightened of King Achish of Gath. \v{13}So David changed his behavior before them and acted like he was crazy in their presence. He scribbled on the doors of the gate, and let his saliva run down his beard. \v{14}Achish told his officials,\fnote{Or \fbib{servants}} ``Look, you see a person acting like a madman. Why'd you bring him to me? \v{15}Am I lacking madmen that you bring me this one to act like a madman around me? Shall this one come into my house?''
\labelchapt{22}
\passage{David at the Cave of Adullam}

\chapt{22}
\v{1}David left from there and escaped to the Cave of Adullam. His brothers and all his father's family heard about this and went down to him there. \v{2}Everyone who was in distress, everyone who was in debt, and everyone who was malcontent\fnote{Lit. \fbib{bitter of spirit}} gathered around him, and he became their leader. There were about 400 men with him.
\passage{David Seeks Protection for His Family}

\v{3}David went from there to Mizpah of Moab, and he told the king of Moab, ``Please let my father and mother come and stay with you\fnote{Lit. \fbib{come with you}} until I know what God is going to do for me.'' \v{4}David left them with the king of Moab, and they stayed with him all the time David was in the stronghold.

\v{5}The prophet Gad told David, ``Don't remain in the stronghold. Go and enter the territory of Judah.'' So David left and went into the forest of Hereth.
\passage{Doeg the Edomite Reports to Saul}

\v{6}When Saul heard that David and the men who were with him had been found,\fnote{Lit. \fbib{were known}} he\fnote{Lit. \fbib{Saul}} was sitting in Gibeah, under the tamarisk tree on the hill, with his spear in his hand. All his officials\fnote{Or \fbib{servants}} were standing around him. \v{7}Saul told his officials who were standing around him, ``Listen, men of Benjamin! Will Jesse's son also give fields and vineyards to all of you? Will he make all of you officers over thousands and officers over hundreds? \v{8}But all of you have conspired against me, and no one tells me\fnote{Lit. \fbib{reveals in my ear}} about my son's covenant\fnote{Or \fbib{agreement}} with Jesse's son. None of you feels sorry for me and tells me that my son has stirred up my servant against me to lie in wait, as he's doing\fnote{The Heb. lacks \fbib{he is doing}} this day.''

\v{9}Then Doeg the Edomite, who was in charge of Saul's servants answered: ``I saw Jesse's son coming to Nob to Ahitub's son Ahimelech. \v{10}Ahimelech\fnote{Lit. \fbib{He}} inquired of the \divine{Lord} for him, gave him provisions, and gave him the sword of Goliath the Philistine.''
\passage{Saul Orders the Execution of the Priests}

\v{11}The king sent for Ahitub's son Ahimelech the priest and for all his father's family who were priests at Nob. All of them came to the king. \v{12}Saul said, ``Listen, son of Ahitub!''

And he said, ``Here I am, your majesty.''

\v{13}Then Saul\fnote{Lit. \fbib{he}} asked him, ``Why have you conspired against me---you and Jesse's son---by giving him food and a sword, and by inquiring of God for him, so he can rise up against me to lie in wait, as he's doing\fnote{The Heb. lacks \fbib{he is doing}} today?''

\v{14}Ahimelech answered the king, ``Who among all your officials\fnote{Or \fbib{servants}} is as faithful as David? He is the king's son-in-law, the captain of your bodyguard, and he's honored in your household. \v{15}Is today the first time I inquired of God for him? Absolutely not! The king shouldn't accuse his servant, or any of my father's family of anything, because your servant didn't know anything at all\fnote{Lit. \fbib{anything, great or small}} about this.''

\v{16}The king said, ``Ahimelech, you will surely die, you and all your father's family!'' \v{17}The king told the guards, who were standing beside him, ``Turn and kill the priests of the \divine{Lord} because they supported David,\fnote{Lit. \fbib{their hand was with David}} and because they knew he was fleeing, but didn't inform me.''\fnote{Lit. \fbib{reveal in my ear}} But the officials of the king did not want to lift their hands\fnote{Lit. \fbib{to send their hand}} to attack the priests of the \divine{Lord}.

\v{18}Then the king told Doeg, ``You turn and attack the priests.'' Doeg the Edomite turned and attacked the priests. That day he killed eighty-five men who carry the linen ephod.\fnote{I.e. the priests} \v{19}He attacked the priestly town of Nob with the sword. Men and women, children and infants, oxen, donkeys and sheep were put to the sword.
\passage{Abiathar Takes the Ephod to David}

\v{20}One man, Ahimelech's son Abiathar, a grandson of Ahitub, escaped and fled to David. \v{21}Abiathar told David that Saul had killed the priests of the \divine{Lord}. \v{22}David told Abiathar, ``I knew on that day when Doeg the Edomite was there that he would certainly tell Saul! I'm responsible for the deaths of your father's whole family. \v{23}Stay with me, and don't be afraid because the one who seeks my life, seeks your life. Indeed, you will be safe with me.''
\labelchapt{23}
\passage{David Delivers Keilah}

\chapt{23}
\v{1}Someone\fnote{Lit. \fbib{They}} told David, ``Look, the Philistines are fighting at Keilah and are plundering the threshing floors.''

\v{2}David inquired of the \divine{Lord}: ``Shall I go and strike down these Philistines?''

The \divine{Lord} told David, ``Go strike down the Philistines and deliver Keilah.''

\v{3}David's men told him, ``Look, we're afraid here in Judah. How much then, if we go to Keilah against the Philistine army?''

\v{4}David inquired of the \divine{Lord} again, and the \divine{Lord} answered him: ``Get up, go down to Keilah. I'll give the Philistines into your control.''\fnote{Lit. \fbib{hand}} \v{5}David and his men went to Keilah and fought the Philistines. He carried off their livestock and defeated them decisively,\fnote{Lit. \fbib{struck them down with a great slaughter}} and so David delivered the inhabitants of Keilah. \v{6}Now when Ahimelech's son Abiathar had fled to David in Keilah, the ephod\fnote{The ephod was a type of vest normally worn by the priests.} had come down with him.

\v{7}It was reported to Saul that David had come to Keilah, and Saul said, ``The \divine{Lord} has delivered\fnote{So with LXX} him into my hand because he has shut himself in by going into a town with double gates and bars.'' \v{8}Saul summoned for battle all his forces\fnote{Lit. \fbib{all the people}} to go down to Keilah, to besiege David and his men.

\v{9}David knew that Saul was devising evil plans against him, and so he told Abiathar the priest, ``Bring the ephod.''

\v{10}David said, ``\divine{Lord} God of Israel. Your servant has definitely heard that Saul intends to come to Keilah to destroy the town because of me. \v{11}Will the people of Keilah hand me over to him?\fnote{Lit. \fbib{into his hand}} Will Saul come down just as your servant has heard? \divine{Lord} God of Israel, please inform your servant.''

The \divine{Lord} said, ``He will come down.''

\v{12}Then David said, ``Will the people of Keilah hand me over to Saul?''\fnote{Lit. \fbib{into Saul's hand}}

The \divine{Lord} said, ``They'll hand you over.'' \v{13}David and his men, about 600 strong, got up and left Keilah. They moved around wherever they could go. Saul was advised that David had escaped from Keilah, so he stopped the campaign.\fnote{Lit. \fbib{stopped going out}}
\passage{Jonathan Visits David}

\v{14}David stayed in the wilderness in the strongholds, and he lived in the hill country in the wilderness of Ziph. Saul sought him every day, but God did not let David\fnote{Lit. \fbib{him}} slip into Saul's\fnote{Lit. \fbib{his}} control. \v{15}David was afraid because\fnote{Or \fbib{David saw that}} Saul had come out to seek his life while David was in the wilderness of Ziph at Horesh. \v{16}Saul's son Jonathan got up and went to David at Horesh, and he encouraged him to trust\fnote{Lit. \fbib{he strengthened his hand}} in God. \v{17}Jonathan told him, ``Don't be afraid. My father Saul won't find you, and you will be king over Israel. I'll be your second-in-command. My father Saul also knows this.'' \v{18}The two of them made a covenant\fnote{Or \fbib{agreement}} in the \divine{Lord}'s presence. David remained at Horesh while Jonathan went home.
\passage{The People of Ziph Betray David}

\v{19}People from Ziph came up to Saul at Gibeah and informed him, ``David is hiding with us in the strongholds in Horesh and on the hill of Hachilah south of Jeshimon, isn't he? \v{20}Now, your majesty, whenever you want to come down,\fnote{Lit. \fbib{according to your desire to come down}} come down, and our part will be to hand him over to the king.''

\v{21}Saul said, ``May you be blessed by the \divine{Lord}, because you have been gracious to me. \v{22}Go and again make sure, find out and investigate where he is\fnote{Lit. \fbib{his place where his foot is}} and who has seen him there, for people tell me that he's very clever. \v{23}Investigate and find out all the hiding places there where he hides, and return to me with reliable information. Then I'll go down with you, and if he's in the land, I'll search him out among all the thousands of Judah.'' \v{24}The people from Ziph got up and left Saul, while David and his men were in the wilderness of Maon in the Arabah south of Jeshimon.

\v{25}When Saul and his men went to search for David,\fnote{The Heb. lacks \fbib{for David}} some people\fnote{Lit. \fbib{David, they}} told David, and he went down to the Rock of Escape\fnote{The Heb. lacks \fbib{of Escape}; cf. v.28} and remained in the wilderness of Maon. Saul heard this and he pursued David into the wilderness of Maon. \v{26}Saul went on one side of the mountain while David and his men went on the other side of the mountain. David was hurrying to get away from Saul while Saul and his men were closing in on David and his men to capture them.

\v{27}Then a messenger came to Saul with this news: ``Come quickly, because the Philistines have made a raid on the land!'' \v{28}So Saul turned around from pursuing David and went to meet the Philistines. Therefore, they call that place the Rock of Escape. \v{29}\fnote{This v. is 24:1 in MT}David went up from there and stayed in the strongholds of En-gedi.
\labelchapt{24}
\passage{David Spares Saul's Life}

\chapt{24}
\v{1}\fnote{This v. is 24:2 in MT}When Saul returned from pursuing the Philistines, he was told,\fnote{Lit. \fbib{they told him}} ``Look, David is in the wilderness of En-gedi.'' \v{2}Saul took 3,000 of his best troops\fnote{Lit. \fbib{choice men}} from all over Israel, and he went to look for David and his men in the direction of the Rocks of the Wild Goats. \v{3}He came to the sheepfolds beside the road. There was a cave there, and Saul went in to relieve himself.\fnote{Lit. \fbib{to cover his feet}} Now David and his men were sitting in the inner recesses\fnote{Or \fbib{in the interior}} of the cave.

\v{4}David's men told him, ``Look, today is the day about which the \divine{Lord} spoke to you when he said,\fnote{The Heb. lacks \fbib{when he said}} `I'll give your enemy into your hand.' Do to him whatever you want!''

David rose and stealthily cut off the corner of Saul's robe. \v{5}Afterwards, David's conscience bothered him because he had cut off the corner of Saul's robe. \v{6}He told his men, ``God forbid that I should do this thing to your majesty, the \divine{Lord}'s anointed, by stretching out my hand against him, since he's the \divine{Lord}'s anointed.'' \v{7}David restrained his men with his\fnote{The Heb. lacks \fbib{his}} words and did not allow them to rebel against Saul. Saul got up from the cave and started off.\fnote{Lit. \fbib{went on the way}}
\passage{David Rebukes Saul}

\v{8}Then David got up, went out of the cave, and called out to Saul: ``Your majesty!''\fnote{Lit. \fbib{My lord, O king!}} Saul looked behind him, and David bowed down with his face to the ground and prostrated himself. \v{9}Then David told Saul, ``Why do you listen to the words of those who say, `Look, David is trying to harm you?' \v{10}Look, this very day you saw with your own eyes\fnote{Lit. \fbib{your eyes saw}} that the \divine{Lord} gave you into my control in the cave, and one of my men\fnote{Lit. \fbib{and he}} told me to kill you, but I had pity\fnote{So LXX} on you and responded, `I won't lift my hand against his majesty because he's the \divine{Lord}'s anointed.' \v{11}Looke, my father, look! The corner of your robe is in my hand. Indeed, by my cutting off the corner of your robe and not killing you, you may know and understand that I have no evil intent or transgression---I haven't wronged you, even though you are hunting me to take my life. \v{12}May the \divine{Lord} judge between me and you, and may he take vengeance on you for me, but I won't be attacking you. \v{13}Just like the ancient proverb says, `From wicked people comes wickedness,' but I'm not against you. \v{14}After whom is the king of Israel going out? Whom are you pursuing? A dead dog or a single flea? \v{15}May the \divine{Lord} act as judge, and may he decide between me and you. May he see, may he plead my case, and may he vindicate me in this dispute against you.''\fnote{Or \fbib{he deliver me from your hand}}
\passage{Saul's Apparent Repentance}

\v{16}When David had finished saying these things to Saul, Saul asked, ``Is this your voice, my son David?'' Then Saul cried loudly \v{17}to David, ``You are more righteous than I am, because you have treated me well even though I've treated you poorly. \v{18}You have explained how you treated me well, in that the \divine{Lord} delivered me into your hand but you didn't kill me. \v{19}For who would find his enemy and then send him away safely?\fnote{Lit. \fbib{on a good road}} May the \divine{Lord} repay you for what you have done for me today. \v{20}Now I know for certain that you will be king, and that the kingdom will be established under your authority.\fnote{Lit. \fbib{hand}} \v{21}Now swear to me by the \divine{Lord} that you will never eliminate my descendants after me, and that you won't erase my name from my father's family.'' \v{22}David made this vow to Saul, and then Saul went home, while David and his men went up to the stronghold.
\labelchapt{25}
\passage{The Death of Samuel}

\chapt{25}
\v{1}Samuel died and all Israel assembled to mourn for him. They buried him at his home in Ramah.
\passage{David, Nabal, and Abigail}

David got up and went down to the Wilderness of Paran.\fnote{LXX reads \fbib{Maoch}} \v{2}Now there was a man in Maon whose business was in Carmel of Judah,\fnote{The Heb. lacks \fbib{of Judah}} and the man was very rich. He had 3,000 sheep and 1,000 goats, and he was shearing his sheep in Carmel. \v{3}The man's name was Nabal and his wife's name was Abigail. The woman was intelligent and beautiful, while the man was harsh and wicked in his dealings. He was a descendant of Caleb.

\v{4}While David was in the wilderness, he heard that Nabal was shearing his sheep. \v{5}David sent ten young men, saying to the young men, ``Go up to Carmel, find Nabal, and greet him in my name. \v{6}Then say, `May you live long. Peace to you, peace to your family, and peace to all that you have. \v{7}Now, I've heard that the sheep shearers are with you. Now, your shepherds have been with us. We didn't harm them, and they didn't miss anything all the time they were in Carmel. \v{8}Ask your young men and they'll tell you. Therefore let my\fnote{Lit. \fbib{the}} young men find favor with you since we came on a special\fnote{Lit. \fbib{good}} day. Please give whatever you have available to your servants and to your son David.'\,''

\v{9}David's young men came to Nabal and told him all this\fnote{Lit. \fbib{according to all these words}} in David's name, and then they waited. \v{10}Nabal answered David's servants: ``Who is David? Who is this son of Jesse? There are many servants today who are breaking away from their masters. \v{11}Should I take my food, my water, and my meat that I've slaughtered for my shearers and give it to men who came from who knows where?''\fnote{Lit. \fbib{men whom I don't know where they're from}}

\v{12}David's men turned and went on\fnote{Lit. \fbib{turned to}} their way. They came back and told David\fnote{Lit. \fbib{him}} everything. \v{13}David told his men, ``Put on your swords.'' They put on their swords, and David put on his sword. Then about 400 men followed David, while 200 stayed with the supplies.
\passage{Abigail Intercedes with David}

\v{14}Now, one of the young men told Nabal's wife Abigail: ``Look, David sent messengers from the wilderness to greet\fnote{Lit. \fbib{bless}} our lord, but he screamed insults at them. \v{15}The men were very good to us. They didn't harm us, and we didn't miss anything all the time we moved around with them when we were in the field. \v{16}They were a wall around us both day and night, all the time we were with them taking care of the sheep. \v{17}Now, be aware of this\fnote{Lit. \fbib{Know}} and consider what you should do. Calamity is being planned against our master and against his entire household. He's such a worthless person\fnote{Lit. \fbib{a son of Belial}; i.e. a worthless person} that no one can talk to him.''

\v{18}Abigail quickly took 200 loaves of bread, two skins of wine, five butchered sheep, five measures of roasted grain, 100 bunches of raisins, and 200 fig cakes and loaded them on donkeys. \v{19}She told her young men, ``Go ahead of me, I'll be coming right behind you.'' But she said nothing to her husband Nabal. \v{20}She was riding on the donkey and as she went down a protected part\fnote{Or \fbib{a hidden part}} of the mountain, David was there with his men, coming down to meet her, and she went toward them.

\v{21}Now David had said, ``Surely it was for nothing that I protected everything that belonged to this man in the wilderness, and nothing was missing of all that belonged to him. But he has repaid me\fnote{Lit. \fbib{returned to me}} with evil for good! \v{22}May the \divine{Lord} do this to the enemies of David\fnote{LXX reads \fbib{to David}}---and more also---if by the morning I've left alive a single male\fnote{Lit. \fbib{single one who urinates on a wall}} of all those who belong to him.''

\v{23}When Abigail saw David, she quickly got down from the donkey and fell on her face before David, prostrating herself on the ground. \v{24}She fell at his feet and pleaded, ``Your majesty, let the guilt be on me alone, and please let your servant\fnote{Lit. \fbib{maidservant}} speak to you.\fnote{Lit. \fbib{speak in your ear}} Listen to the words of your servant.\fnote{Lit. \fbib{maidservant}} \v{25}Please, your majesty, don't pay attention to this worthless man Nabal, for he's just like his name. Nabal\fnote{\fbib{Nabal} means \fbib{fool} in Heb.} is his name and folly is his constant companion. But I, your servant,\fnote{Lit. \fbib{maidservant}} didn't see your majesty's young men whom you sent. \v{26}Now, your majesty, as the \divine{Lord} lives and as you live, the \divine{Lord} has kept you from shedding blood\fnote{Lit. \fbib{coming with blood}} and from delivering yourself by your own actions. Now, may your enemies and those seeking to do evil to your majesty be like Nabal. \v{27}Now let this present that your servant\fnote{Lit. \fbib{maidservant}} has brought to your majesty be given to the young men who follow\fnote{Lit. \fbib{who are walking at the feet of}} your majesty. \v{28}Please forgive the offense of your servant.\fnote{Lit. \fbib{maidservant}} For the \divine{Lord} will certainly make a strong dynasty for your majesty, for your majesty is fighting the \divine{Lord}'s battles. May evil not be found in you for all of your life.\fnote{Lit. \fbib{all the days}} \v{29}If anyone should arise to pursue you and seek your life, may the life of your majesty be bound up with the \divine{Lord} your God in a bundle of the living, and may he sling out the lives of your enemies from the pocket of a sling. \v{30}When the \divine{Lord} does for your majesty all the good that he promised concerning you and appoints you Commander-in-Chief\fnote{Lit. \fbib{Nagid}; i.e. a senior officer entrusted with dual roles of operational oversight and administrative authority} over Israel, \v{31}this shouldn't be an obstacle or stumbling block for your majesty's conscience, that he poured out blood without cause or that your majesty delivered himself. When the \divine{Lord} does good things for your majesty, remember your servant.''\fnote{Lit. \fbib{maidservant}}

\v{32}David told Abigail, ``Blessed be the \divine{Lord} God of Israel, who sent you to meet me today. \v{33}Blessed be your good judgment, and blessed be you, who today stopped me from shedding blood\fnote{Lit. \fbib{from coming with blood}} and delivering myself by my own actions. \v{34}For as surely as the \divine{Lord} God of Israel lives, the one who restrained me from harming you---indeed, had you not quickly come to meet me, by dawn\fnote{Lit. \fbib{the light of the morning}} there wouldn't be a single male\fnote{Lit. \fbib{one who urinates on a wall}} left to Nabal.''

\v{35}David took from her what she had brought him and told her, ``Go up to your house in peace. Look, I've heard your request and will grant it.''
\passage{Nabal's Death}

\v{36}Abigail returned to Nabal, and he was there in his house holding a festival like the festival of a king. Nabal's heart was glad, and he was very drunk, so she didn't tell him anything at all\fnote{Lit. \fbib{anything great or small}} until morning. \v{37}After Nabal became sober the next morning,\fnote{Lit. \fbib{When the wine had gone out of Nabal}} his wife told him all that had happened.\fnote{Lit. \fbib{all these things}} Nabal's\fnote{Lit. \fbib{His}} heart failed and he became paralyzed.\fnote{Lit. \fbib{became like a stone}} \v{38}About ten days later the \divine{Lord} struck Nabal, and he died.

\v{39}When David heard that Nabal had died, he said, ``Blessed be the \divine{Lord} who has judged the dispute over my insult at the hand of Nabal, and has held back his servant from evil. The \divine{Lord} has repaid Nabal's wickedness.''

Then David sent word to Abigail that he would take her as his wife. \v{40}David's servants went to Abigail at Carmel and told her, ``David sent us to you to take you to him as his wife.''

\v{41}She got up, prostrated herself face down on the ground, and replied, ``Your servant would be a slave to wash the feet of your majesty's servants.'' \v{42}Then Abigail quickly got up and got on a donkey, with five young women walking behind her.\fnote{Lit. \fbib{walking at her feet}; i.e. as her attendants} She followed David's messengers, and she became his wife. \v{43}David also married Ahinoam of Jezreel, and both of them became his wives. \v{44}Meanwhile, Saul had given his daughter Michal, David's wife, to Laish's son Palti from Gallim.
\labelchapt{26}
\passage{David Again Spares Saul's Life}

\chapt{26}
\v{1}People from Ziph came to Saul in Gibeah and informed him, ``David is hiding on the hill of Hachilah which is across from Jeshimon, isn't he?'' \v{2}So Saul rose and went down with 3,000 select men of Israel to the Wilderness of Ziph, to look for David in the Wilderness of Ziph. \v{3}Saul camped by the road on the hill of Hachilah, across from Jeshimon, while David was staying in the wilderness. When he realized\fnote{Lit. \fbib{saw}} that Saul had come after him in the wilderness, \v{4}David sent out spies and found out for certain that Saul had arrived. \v{5}David rose and went to the place where Saul was camped. David saw the place where Saul and Abner, his Commander-in-Chief, lay down. Saul was lying down within the encampment, and the army was\fnote{Or \fbib{the people were}} camped all around him.

\v{6}David said\fnote{Lit. \fbib{answered, saying}} to Ahimelech the Hittite, and to Joab's brother Abishai, Zeruiah's son, ``Who will go down with me to Saul in the camp?''

Abishai said, ``I'll go down with you.''

\v{7}David and Abishai went to the army\fnote{Or \fbib{the people}} at night, and Saul was lying there asleep in the encampment. His spear was stuck in the ground at his head, and Abner and the army\fnote{Or \fbib{the people}} were lying all around him. \v{8}Abishai told David, ``Today God has delivered your enemy into your hand. Let me run the spear through him into the ground with a single blow. I won't need to strike him twice!''

\v{9}David told Abishai, ``Don't destroy him. Who can raise his hand to strike the \divine{Lord}'s anointed and remain innocent? \v{10}As the \divine{Lord} lives, the \divine{Lord} will strike him down, or his time will come to die, or he will go into battle and perish. \v{11}The \divine{Lord} forbid that I should raise my hand against the \divine{Lord}'s anointed. Now take the spear that is at his head and the jug of water, and let's go.'' \v{12}So David took the spear and the jug of water at Saul's head, and they left. No one saw, and no one knew, because no one was awake. They were all asleep, because a deep sleep from the \divine{Lord} had fallen over them.

\v{13}Then David crossed over to the other side and stood on top of the hill some distance away with a large distance between them. \v{14}David called out to the army\fnote{Or \fbib{the people}} and to Ner's son Abner, ``Abner, won't you answer me?''

Abner answered: ``Who are you who calls out to the king?''

\v{15}David told Abner, ``Are you not a man, and who is like you in Israel? Why didn't you guard your lord, the king? Indeed, a soldier came to destroy the king, your lord. \v{16}This thing that you did is not good. As the \divine{Lord} lives, you deserve to die,\fnote{Lit. \fbib{you are sons of death;} i.e. dead men} you who didn't guard your lord, the \divine{Lord}'s anointed. Where is the king's spear and where is the jug of water that was at his head?''

\v{17}Saul recognized David's voice and said, ``Is this your voice, my son David?''

David replied, ``It is my voice, your majesty.''\fnote{Lit. \fbib{My lord the king}} \v{18}David\fnote{Lit. \fbib{He}} said, ``Why is your majesty pursuing his servant? For what have I done, and what evil do I bear toward you? \v{19}Now let your majesty\fnote{Lit. \fbib{My lord the king}} listen to the words of his servant. If the \divine{Lord} incited you against me, then may he accept an offering. But if it is people, may they be cursed in the \divine{Lord}'s presence, because they have driven me out today from sharing in the inheritance of the \divine{Lord} by saying, `Go serve other gods.' \v{20}Now, don't let my blood fall to the ground away from the \divine{Lord}'s presence. Indeed, the king of Israel has come out to seek a single flea, like someone hunts a partridge in the mountains.''
\passage{Saul Apologizes Again}

\v{21}Then Saul said, ``I've wronged you. Return, my son David, for I won't harm you again because my life was precious to you\fnote{Lit. \fbib{in your sight}} today. Look, I've acted foolishly and have made a very great mistake.''

\v{22}David replied, ``Here's the king's spear. Have one of the young men come over and get it. \v{23}The \divine{Lord} repays a person for his righteousness and his faithfulness. The \divine{Lord} gave you into my control today, but I refused to raise my hand against the \divine{Lord}'s anointed. \v{24}Look, just as your life was valuable in my eyes today, so may my life be valuable in the \divine{Lord}'s eyes, and may he deliver me from all trouble.''

\v{25}Saul told David, ``Blessed are you, my son David. In whatever you do you will surely succeed.'' So David went on his way, and Saul returned to his place.
\labelchapt{27}
\passage{David Escapes to Philistine Territory}

\chapt{27}
\v{1}David told himself, ``One of these days I'll perish by Saul's hand. There is nothing better for me to do than to escape to Philistine territory. Saul will give up searching for me anymore within the borders of Israel, so I'll escape from him.'' \v{2}So David got up, and he and the 600 men who were with him went to Maoch's son Achish, the king of Gath. \v{3}David stayed with Achish in Gath along with his men, each of whom was with his household. David had his two wives, Ahinoam from Jezreel and Abigail, who had been the wife of Nabal of Carmel. \v{4}Saul was told that David had fled to Gath, and he did not continue to search for him.
\passage{Achish Gives Ziklag to David}

\v{5}David told Achish, ``If it pleases you, give me a place in one of the outlying towns,\fnote{Lit. \fbib{one of the towns of the field}} so I may live there. Why should your servant live with you in the royal city?'' \v{6}So that day Achish gave him Ziklag, and therefore, Ziklag has belonged to the kings of Judah until the present time. \v{7}David lived in Philistine territory for a year and four months.
\passage{David's Raids on the Land}

\v{8}David and his men went up and raided the descendants of Geshur, the descendants of Girzi, and the Amalekites, for they had been living in the land since ancient times, from the entrance of\fnote{Lit. \fbib{times, where you enter}} Shur all the way to the land of Egypt. \v{9}David struck the land and did not leave a man or woman alive. He took sheep, cattle, donkeys, camels, and clothing, and then came back and went to Achish.

\v{10}Achish said, ``Where did you raid today?''

David answered, ``Against the Negev\fnote{I.e. southern regions of the Sinai peninsula; cf. Josh 10:40} of Judah, against the Negev\fnote{Or \fbib{south}} of the Jerahmeelites, and against the Negev\fnote{Or \fbib{south}} of the Kenites.'' \v{11}David did not leave a man or woman alive to bring to Gath. He told himself,\fnote{The Heb. lacks \fbib{himself}} ``Otherwise, they'll say, `This is what David is doing, and this has been his practice all the time he has lived in Philistine territory.'\,''

\v{12}Achish believed David, telling himself,\fnote{The Heb. lacks \fbib{himself}} ``He has certainly made himself repulsive to his people in Israel. He will be my servant forever.''
\labelchapt{28}
\passage{The Philistines Prepare to Fight against Israel}

\chapt{28}
\v{1}At that time the Philistines assembled their army for war to fight against Israel. Achish told David, ``You know, of course, that you and your men will go out with me into the battle.''

\v{2}David told Achish, ``Very well, you will now see\fnote{Lit. \fbib{you will know}} what your servant will do.''

Achish told David, ``Very well, I'll appoint you as my permanent bodyguard.''
\passage{Saul and the Medium at Endor}

\v{3}Now Samuel had died, and all Israel had mourned for him and buried him in his own town of Ramah. Saul had expelled the mediums and spiritists from the land.

\v{4}The Philistines assembled, moved out, and camped at Shunem, while Saul assembled all Israel and camped at Gilboa. \v{5}When Saul saw the Philistine camp, he was afraid, and his heart trembled greatly. \v{6}Saul inquired of the \divine{Lord}, but the \divine{Lord} did not answer him, either through dreams or Urim\fnote{I.e. a device used by the priest to determine God's will} or through prophets. \v{7}Saul told his servants, ``Find me a woman who is a medium so I can go to her and make my inquiry through her.''

His servants told him, ``Look, there's a woman at Endor who is a medium.''

\v{8}Saul disguised himself, putting on different clothes. He went along with two men to the woman at night. He said, ``Consult a familiar spirit for me and bring up for me the one whom I tell you.''

\v{9}The woman told him, ``Look, you know what Saul has done. He has removed mediums and spiritists from the land, so why are you trying to entrap me, so as to cause my death?''

\v{10}Saul swore to her by the \divine{Lord}: ``As surely as the \divine{Lord} lives, no punishment will come on you for this thing.''

\v{11}The woman said, ``Whom shall I bring up for you?''

Saul\fnote{Lit. \fbib{He}} said, ``Bring up Samuel for me.''

\v{12}When the woman saw Samuel, she cried out loudly.\fnote{Lit. \fbib{with a loud voice}} The woman told Saul, ``Why have you deceived me? You are Saul!''

\v{13}The king told her, ``Don't be afraid; but what do you see?''

The woman told Saul, ``I see a divine being\fnote{Or a \fbib{spirit}; or a \fbib{god}} coming up out of the ground.''

\v{14}Saul\fnote{Lit. \fbib{He}} told her, ``What does he look like?''

She said, ``An old man is coming up, and he's wrapped in a robe.'' Saul knew that it was Samuel, and he bowed low to the ground and prostrated himself.
\passage{Samuel's Message to Saul}

\v{15}Samuel told Saul, ``Why did you disturb me by bringing me up?''

Saul said, ``I'm in great distress. The Philistines are waging war against me. God has departed from me and won't answer me anymore, either by messages written by\fnote{The Heb. lacks \fbib{messages written by}} the hand of the prophets or by dreams. So I've summoned you to tell me what I should do.''

\v{16}Samuel said, ``Why do you ask me, since the \divine{Lord} has departed from you and become your enemy? \v{17}The \divine{Lord} has done to you exactly as he spoke through me.\fnote{Lit. \fbib{by my hand}} The \divine{Lord} has torn the kingdom away from you\fnote{Lit. \fbib{from your hand}} and has given it to your colleague David. \v{18}Because you didn't obey the \divine{Lord} and didn't display his fierce anger against Amalek, therefore, the \divine{Lord} will do this thing to you today. \v{19}The \divine{Lord} is giving both you, and Israel with you, into Philistine control. Tomorrow, the \divine{Lord} will give you, your sons with you, and also the army of Israel into the control\fnote{Lit. \fbib{hand}} of the Philistines.''
\passage{The Medium Attends to Saul}

\v{20}Saul immediately fell down full-length on the ground. He was terrified because of Samuel's words, and he had no strength because he had not eaten food all day and all night. \v{21}Then the woman came to Saul and saw that he was very disturbed. She told him, ``Look, your servant\fnote{Lit. \fbib{maidservant}} obeyed you. I put my life into your hands, and I listened to your words that you spoke to me. \v{22}Now, please listen to your servant.\fnote{Lit. \fbib{maidservant}} I'll put a piece of bread before you so you can eat and have strength to go on your way.''\fnote{Lit. \fbib{the way}}

\v{23}Saul\fnote{Lit. \fbib{He}} refused, saying, ``I won't eat!''

Both his servants and the woman urged him, and so he listened to them. He got up off the ground and sat on the bed. \v{24}The woman had a fattened calf in the house, and she quickly slaughtered it. She took flour, kneaded it, and baked unleavened bread. \v{25}She brought it to Saul and to his servants, and they ate. Then they got up and went out that night.
\labelchapt{29}
\passage{The Philistine Leaders Reject David}

\chapt{29}
\v{1}The Philistines gathered all their troops at Aphek, while Israel was camped at the spring in Jezreel. \v{2}The Philistine leaders were passing in review among\fnote{The Heb. lacks \fbib{among}} the military units,\fnote{Lit. \fbib{the hundreds and the thousands}} and David and his men were among\fnote{Lit. \fbib{were passing}} them in the rear with Achish.

\v{3}The Philistine leaders said, ``What are these Hebrews doing here?''

Achish asked the Philistine leaders, ``Isn't this David, the servant of King Saul of Israel, who has been with me these days, or rather\fnote{The Heb. lacks \fbib{rather}} these years? I've found no fault in him from the day he deserted\fnote{Lit. \fbib{fell}} until now.''

\v{4}But the Philistine leaders were angry with him, so they\fnote{Lit. \fbib{the Philistine leaders}} pleaded with him, ``Send the man back! Let him return to the\fnote{Lit. \fbib{his}} place you assigned him. He mustn't go into battle with us. Otherwise, he may become our adversary in the battle! How could there be a better way for\fnote{The Heb. lacks \fbib{there be a better way for}} this fellow to reconcile himself with his lord? Wouldn't it be with the heads of these men? \v{5}Isn't this the same\fnote{The Heb. lacks \fbib{same}} David about whom the maidens\fnote{Lit. \fbib{they}} sang when they were dancing,

\begin{poetry}
\poeml `Saul has struck down his thousands, \\
\poemll    but David his ten thousands'?''
\end{poetry}
\passage{Achish Sends David Home}

\v{6}Then Achish summoned David and told him, ``As surely as the \divine{Lord} lives, you are trustworthy,\fnote{Or \fbib{upright}} and it seems good to me for you to campaign\fnote{Lit. \fbib{for you to go out and come in}} with me as part of the army. Indeed, I've not found any evil in you from the time you came to me until now.\fnote{Lit. \fbib{until this day}} But the leaders don't approve of you. \v{7}Now return and go in peace, so you do nothing to displease the Philistine leaders.''

\v{8}David told Achish, ``What have I done, and what have you found in your servant from the time I came before you until this very moment,\fnote{Lit. \fbib{until this day}} that I shouldn't go out and fight the enemies of your majesty?''\fnote{Lit. \fbib{my lord the king}}

\v{9}Achish answered David, ``I know that I'm pleased with you. You're\fnote{The Heb. lacks \fbib{You're}} like an angel of God. But the Philistine leaders have said, `He mustn't go into battle with us.' \v{10}Now, get up early in the morning along with your lord's servants who came with you.\fnote{LXX reads \fbib{with you and go to the place that I've assigned you. Harbor no bitter thought in your heart, for you are acceptable to me.}} Get up early in the morning, and go as soon as you have light.'' \v{11}So\fnote{The Heb. lacks \fbib{So}} David and his men got up early in the morning to return to Philistine territory, while the Philistines went up to Jezreel.
\labelchapt{30}
\passage{Trouble on David's Return to Ziklag}

\chapt{30}
\v{1}When David and his men came to Ziklag on the third day, the Amalekites had raided the Negev\fnote{I.e. the southern regions of the Sinai peninsula; cf. Josh 10:40} and Ziklag. They had attacked Ziklag and set it on fire. \v{2}They took the women in it captive, from young to old.\fnote{Lit. \fbib{from small to great}} They did not kill anyone. Instead, they carried them off and went on their way. \v{3}David and his men came to the town, and it had been burned down. Their wives, their sons, and their daughters had been taken captive. \v{4}Then David and the people who were with him lifted their voices and cried until they had no more strength left to cry. \v{5}David's two wives, Ahinoam from Jezreel and Abigail, Nabal's former\fnote{The Heb. lacks \fbib{former}} wife, had been captured. \v{6}David was in great danger\fnote{Or \fbib{greatly distressed}} because all the people were bitter because of their sons and daughters, and they were talking about stoning him. But David found strength\fnote{Or \fbib{strengthened himself}} in the \divine{Lord} his God.
\passage{David Pursues the Amalekites}

\v{7}David told Ahimelech's son Abiathar the priest, ``Bring me the ephod.''\fnote{The ephod was a type of vest worn by the priest and was used to determine God's will.} So Abiathar brought the ephod to David. \v{8}David inquired of the \divine{Lord}: ``Shall I pursue this raiding party?\fnote{Or \fbib{band}} Will I overtake them?''

The \divine{Lord}\fnote{Lit. \fbib{He}} told David,\fnote{Lit. \fbib{him}} ``Pursue them! You will definitely overtake them and rescue the captives.''\fnote{Lit. \fbib{and you will definitely rescue}} \v{9}So David and 600 men who were with him set out. They came to the Wadi\fnote{I.e. a seasonal stream or river that channels water during rain seasons but is dry at other times} Besor where those who were left behind stayed. \v{10}David and 400 men continued the pursuit,\fnote{Lit. \fbib{pursued}} while the 200 men who were too exhausted to cross over the Wadi\fnote{I.e. a seasonal stream or river that channels water during rain seasons but is dry at other times} Besor remained there.\fnote{The Heb. lacks \fbib{there}}
\passage{An Egyptian Leads David to the Amalekites}

\v{11}They found an Egyptian man in the field, and they took him to David. They gave him food to eat and provided water for him. \v{12}They gave him part of a fig cake and two bunches of raisins. After he had eaten, he revived,\fnote{Lit. \fbib{his spirit revived}} since he had neither eaten food nor had he drunk water for three days and three nights. \v{13}David told him, ``To whom do you belong and where are you from?''

The Egyptian\fnote{Lit. \fbib{He}} replied, ``I'm a young Egyptian man, the slave of an Amalekite man. My master abandoned me, because I got sick three days ago. \v{14}We raided the Negev\fnote{Or \fbib{the southern region}} of the Cherethites, the territory that belongs to Judah,\fnote{Lit. \fbib{what belongs to Judah}} and the Negev\fnote{Or \fbib{the southern region}} of Caleb, and we set Ziklag on fire.''

\v{15}David asked him, ``Will you take me to this raiding party?''\fnote{Or \fbib{band}}

He said, ``Swear to me by God that you won't kill me or turn me over to my master, and I'll take you to the raiding party.''\fnote{Or \fbib{band}}
\passage{David Defeats the Amalekites}

\v{16}The Egyptian\fnote{Lit. \fbib{He}} led him to the camp,\fnote{Lit. \fbib{him down}} and there the Amalekites\fnote{Lit. \fbib{they}} were spread out over the whole area, eating, drinking, and celebrating with the great amount of spoil they had taken from the territory belonging to the Philistines and to Judah. \v{17}David struck them down from twilight until the evening of the next day, and not one of them escaped except for 400 young men who mounted camels and fled. \v{18}David rescued everyone whom the Amalekites\fnote{Lit. \fbib{whom Amalek}; i.e., those who lived in the town of Amalek} had captured, including\fnote{Lit. \fbib{captured, and David rescued}} his two wives. \v{19}Nothing of theirs was missing, whether small or large, sons or daughters, spoil, or anything that they had taken for themselves---David brought back everything. \v{20}David took all the rest of\fnote{The Heb. lacks \fbib{the rest of}} the sheep and cattle, driving them ahead of their rescued livestock.\fnote{Lit. \fbib{ahead of those livestock}} People said about all this,\fnote{Lit. \fbib{about them}} ``This is David's spoil.''
\passage{David Divides the Spoil}

\v{21}David came to the 200 men who were too exhausted to follow him\fnote{Lit. \fbib{David}} and who had been left at the Wadi\fnote{I.e. a seasonal stream or river that channels water during rain seasons but is dry at other times} Besor. They came out to meet David and the people who were with him. As David approached the people, he asked them how they were doing.\fnote{Or \fbib{he greeted them}} \v{22}At this point, all the wicked and worthless men of the group who had gone with David answered, ``Because they didn't go with us, we won't give them any of the spoil that we recovered, except that each person may take his wife and his children and go.''

\v{23}David said, ``No, you won't do this, my brothers, with what the \divine{Lord} has given us. He guarded us and gave the raiding party\fnote{Or \fbib{band}} that came against us into our hand. \v{24}Who will listen to you in this matter? Indeed, the share of those who went down into battle and the share of those who stayed with the supplies will be the same. They'll share alike.'' \v{25}From that day forward he made it a statute and an ordinance for Israel, and it remains\fnote{The Heb. lacks \fbib{and it remains}} to this present\fnote{The Heb. lacks \fbib{present}} day.
\passage{David Shares the Spoil with the People of Judah}

\v{26}David came to Ziklag, and he sent some of the spoil to the elders of Judah, and to his friends, telling them, ``Look, this is a gift for you from the spoil of the enemies of the \divine{Lord} \v{27}in Bethel, Ramoth-negev, Jattir, \v{28}Aroer, Siphmoth, Eshtemoa, \v{29}Rachal, in the Jerahmeelite towns, in the Kenite towns, \v{30}in Hormah, Bor-ashan, Athach, \v{31}Hebron, and for all those places where David and his men had frequented.''
\labelchapt{31}
\passage{Saul Killed by the Philistines}
\passageinfo{(1 Chronicles 10:1-7)}

\chapt{31}
\v{1}The Philistines fought against Israel, and the army\fnote{Lit. \fbib{and the men}} of Israel fled before the Philistines. They fell slain on Mount Gilboa. \v{2}The Philistines pursued Saul and his sons. The Philistines struck down Jonathan, Abinadab, and Malchi-shua, Saul's sons. \v{3}The heaviest fighting was directed toward Saul,\fnote{Lit. \fbib{was heavy toward}} and when the bowmen who were shooting located Saul, he was severely wounded by them.\fnote{Lit. \fbib{the archers}}

\v{4}Saul told his armor bearer, ``Draw your sword and run me through with it, or these uncircumcised people will come and run me through and make sport of me.'' But his armor bearer did not want to do it\fnote{The Heb. lacks \fbib{to do it}} because he was very frightened, so Saul took the sword and fell on it. \v{5}When his armor bearer saw that Saul was dead, he also fell on his sword and died with him. \v{6}As a result, Saul, his three sons, his armor bearer, and all his men died together that day. \v{7}When the men of Israel who were across the valley and who were across the Jordan saw that the army\fnote{Lit. \fbib{men}} of Israel had fled, and that Saul and his sons were dead, they abandoned the cities and fled, and the Philistines came and occupied them.
\passage{The Philistines Desecrate Saul's Body}
\passageinfo{(1 Chronicles 10:8-10)}

\v{8}The next day, the Philistines came to strip the dead, and they found Saul and his three sons fallen on Mount Gilboa. \v{9}They cut off his head and stripped him of his weapons. They sent people throughout the territory of the Philistines to report the good news in the temples of their idols and to the people. \v{10}They put Saul's\fnote{Lit. \fbib{his}} weapons in the temple of Asherah\fnote{Asherah was a female deity worshipped by the Canaanites and the Philistines} and fastened his corpse to the wall of Beth-shan.
\passage{The People of Jabesh-gilead Give Saul a Proper Burial}
\passageinfo{(1 Chronicles 10:11-13)}

\v{11}When the residents of Jabesh-gilead heard what\fnote{Lit. \fbib{heard about it, what}} the Philistines had done to Saul, \v{12}every valiant soldier\fnote{Lit. \fbib{man}} got up, traveled all night, and removed Saul's body and the bodies of his sons from the wall of Beth-shan. Then they went to Jabesh and cremated the bodies\fnote{Lit. \fbib{them}} there. \v{13}They took their bones, buried them under the tamarisk tree in Jabesh, and fasted for seven days.

\bookheader{2 Samuel}
\labelbook{2Sam}

\bookpretitle{The Book of}
\booktitle{Second Samuel}

\labelchapt{1}
\passage{David Mourns for Saul and Jonathan}

\chapt{1}
\v{1}Shortly after Saul had died, David returned from defeating the Amalekites and remained in Ziklag for two days. \v{2}The next\fnote{\fbackref{1:2} Lit. \fbib{third}} day, a man escaped from Saul's camp! With torn clothes and dirty hair, he approached David, fell to the ground, and bowed down to him.

\v{3}David asked him, ``Where did you come from?

He answered him, ``I just escaped from Israel's encampment.''

\v{4}David continued questioning him, ``How did things go? Please tell me!''

He replied, ``The army has fled the battlefield, many of the army are wounded\fnote{\fbackref{1:4} Lit. \fbib{fallen}} or have died, and Saul and his son Jonathan are also dead.''

\v{5}David asked the young man who related the story,\fnote{\fbackref{1:5} The Heb. lacks \fbib{the story}} ``How do you know that Saul and his son Jonathan are dead?''

\v{6}The young man who had been relating the story\fnote{\fbackref{1:6} The Heb. lacks \fbib{the story}} answered, ``I happened to be on Mount Gilboa and there was Saul, leaning on his spear! Meanwhile, the chariots and horsemen were rapidly drawing near. \v{7}Saul\fnote{\fbackref{1:7} Lit. \fbib{He}} glanced behind him, saw me, and called out to me, so I replied, `Here I am!' \v{8}He asked me, `Who are you?' So I answered him, `I'm an Amalekite!' \v{9}He begged me, `Please---come stand here next to me and kill me, because I'm still alive.' \v{10}So I stood next to him and killed him, because I knew that he wouldn't live after he had fallen. I took the crown that had been on his head, along with the bracelet that had been on his arm, and I have brought them to your majesty.''\fnote{\fbackref{1:10} Lit. \fbib{my lord}; and so throughout the book}

\v{11}On hearing this,\fnote{\fbackref{1:11} Lit. \fbib{Then}} David grabbed his clothes and tore them, as did all the men who were attending to him. \v{12}They mourned and wept, and then decided to fast\fnote{\fbackref{1:12} Lit. \fbib{wept, fasting}} until dusk for Saul, for his son Jonathan, for the army of the \divine{Lord}, and for the house of Israel, because they had fallen in battle.\fnote{\fbackref{1:12} Lit. \fbib{fallen by the sword}}

\v{13}Meanwhile, David asked the young man who had told him the story,\fnote{\fbackref{1:13} The Heb. lacks \fbib{story}} ``Where are you from?''

He answered, ``I'm an Amalekite, the son of a foreign man.''

\v{14}At this David asked him, ``How is it that you weren't afraid to raise your hand to strike the \divine{Lord}'s anointed?''

\v{15}Then David called out to one of his young men and ordered him, ``Go up to him and cut him down!'' So he attacked him and killed him.

\v{16}David told him, ``Your blood is on your own head, because your own words\fnote{\fbackref{1:16} Lit. \fbib{mouth}} testified against you! After all, you said, `I myself have killed the \divine{Lord}'s anointed!'\,''
\passage{David's Song for Saul and Jonathan}

\v{17}So David intoned this song of lament about Saul and his son Jonathan, \v{18}and he gave orders\fnote{\fbackref{1:18} Lit. \fbib{he said}} to teach the descendants of Judah the art of warfare,\fnote{\fbackref{1:18} Lit. \fbib{Judah the bow}; or \fbib{Judah the Song of the Bow}; i.e., David's lament in vs. 19-27} as is recorded in the Book of Jashar:\fnote{\fbackref{1:18} Lit. \fbib{the Book of the Upright}; i.e. an ancient chronicle of Israel, apparently now lost.}

\begin{poetry}
\poeml \v{19}``Your beauty, Israel, lies slain on your high places! \\
\poemll    O, how the valiant have fallen! \\
\poeml \v{20}Don't make it known in Gath! \\
\poemll    Don't declare it in the avenues of Ashkelon! \\
\poeml Otherwise, the daughters of Philistia will rejoice; \\
\poemll    and the daughters of the uncircumcised will triumph. \\
\poeml \v{21}Mountains of Gilboa, \\
\poemll    let no dew or rain fall on you, \\
\poemlll       and may none of your fields be filled with plenty, \\
\poeml because in that place the shield of the valiant ones was defiled, \\
\poemll    the shield of Saul without an anointing with oil. \\
\poeml \v{22}From the blood of the slain, \\
\poemll    from the blood of the valiant, \\
\poeml Jonathan's bow would not retreat \\
\poemll    nor would Saul's sword return empty. \\
\poeml \v{23}Saul and Jonathan, loved and handsome in life, \\
\poemll    in death were not separated. \\
\poeml Swifter than eagles they were, \\
\poemll    and more valiant than lions. \\
\poeml \v{24}Daughters of Israel, weep over Saul! \\
\poemll    He clothed you in scarlet luxury \\
\poemlll       and decorated your garments with gold. \\
\poeml \v{25}How have the valiant fallen in the tumult of battle! \\
\poemll    Jonathan lies slain on your high places. \\
\poeml \v{26}I am in distress for you, my brother Jonathan. \\
\poemll    You have been most kind\fnote{\fbackref{1:26} Or \fbib{pleasant}} to me. \\
\poeml Your love for me was extraordinary\fnote{\fbackref{1:26} Or \fbib{wonderful}}--- \\
\poemll    beyond love from women. \\
\poeml \v{27}How the valiant have fallen! \\
\poemll    How the weapons of war are destroyed!''
\end{poetry}
\labelchapt{2}
\passage{David Becomes King over Judah}

\chapt{2}
\v{1}Some time later, David inquired of the \divine{Lord} to ask, ``Am I to move\fnote{\fbackref{2:1} Lit. \fbib{to go up}} to any one of the cities of Judah?''

The \divine{Lord} told him, ``Go.''

So David asked, ``To which one?''

He replied, ``To Hebron.''

\v{2}So David went there, along with his two wives Ahinoam from Jezreel and Abigail, widow of Nabal from Carmel. \v{3}David brought his army\fnote{\fbackref{2:3} Lit. \fbib{men}} with him, each soldier accompanied by his household, and they settled in the cities of Hebron. \v{4}After this, the army of Judah arrived, and they anointed David king over the house of Judah.

There they informed David, ``The men of Jabesh-gilead buried Saul.''

\v{5}So David sent messengers to the people\fnote{\fbackref{2:5} Lit. \fbib{men}} of Jabesh-gilead and told them, ``May the \divine{Lord} bless you, because you showed gracious love like\fnote{\fbackref{2:5} The Heb. lacks \fbib{like}} this to your lord Saul by burying him. \v{6}Now may the Lord reward you with gracious love, as well as faithfulness, to you, too! And I will also reward you because you did this good thing. \v{7}So strengthen yourselves, and be valiant in heart, because your lord Saul has died, and the household of Judah has anointed me to be king over them.''
\passage{Abner's Rebellion and the Battle at Gibeon}

\v{8}Meanwhile, Ner's son Abner, the commander of Saul's army, had taken Saul's son Ish-bosheth\fnote{\fbackref{2:8} MT means \fbib{Shameful Man}; cf. 1Chr 8:33, where he is named \fbib{Esh-baal}} and brought him to Mahanaim. \v{9}He installed him as king over Gilead, the Ashurites, Jezreel, Ephraim, Benjamin, and all of the rest of\fnote{\fbackref{2:9} The Heb. lacks \fbib{the rest of}} Israel. \v{10}Ish-bosheth began to reign over Israel at the age of 40 years, and he reigned for two years, even though Judah's lineage followed David. \v{11}The period of David's kingship in Hebron lasted seven years and six months.

\v{12}Ner's son Abner and the servants of Saul's son Ish-bosheth set out from Mahanaim for Gibeon. \v{13}Zeruiah's son Joab and some of David's staff went out to meet them at the pool of Gibeon. One side encamped on one side of the pool while the other encamped on the other side of the pool.

\v{14}Abner told Joab, ``Let's have the young men get up and fight in our presence.''

Joab replied, ``Let them come.''

\v{15}So they got up and twelve were counted to represent Benjamin and Saul's son Ish-bosheth and twelve to represent members of David's staff. \v{16}Each man grabbed his opponent by the head, plunged\fnote{\fbackref{2:16} The Heb. lacks \fbib{plunged}} his sword into his opponent's side, and then they both fell together. That's why the place at Gibeon was named The Field of Swords.\fnote{\fbackref{2:16} Lit. \fbib{Helkath-hazzurim}} \v{17}The battle was very violent that day, with Abner and the men of Israel being defeated in the presence of David's servants.
\passage{Abner Kills Joab's Brother Asahel}

\v{18}Zeruiah's three sons Joab, Abishai, and Asahel were there. As a runner, Asahel was fast, like one of the wild gazelles. \v{19}So Asahel ran straight\fnote{\fbackref{2:19} Lit. \fbib{ran turning neither to the right nor to the left}} after Abner, following him. \v{20}When Abner looked behind him, he said, ``Is that you, Asahel?''

He answered, ``I am.''

\v{21}Abner told him, ``Go off to your right or left after one of the young men and grab some war spoils.'' But Asahel would not stop following him, \v{22}so Abner told Asahel again, ``Stop following me. Why should I strike you down? How could I show my face to your brother Joab?''

\v{23}But Asahel\fnote{\fbackref{2:23} Lit. \fbib{he}} refused to turn away, so Abner struck Asahel in the abdomen with the butt end of his spear, and the spear protruded through his back. He collapsed to the ground and died where he fell. Everyone gathered round the place where Asahel had collapsed and died, and stood still there.

\v{24}Meanwhile, Joab and Abishai continued to chase Abner. At dusk, as they approached the hill of Ammah that is located near Giah on the way to the Gibeon desert, \v{25}the descendants of Benjamin rallied around Abner, forming a single military force. They took their stand on top of the hill.

\v{26}Then Abner called out to Joab, ``Must the battle sword keep on devouring forever? Don't you realize that the end result is bitterness? How long will it take for you to order your army\fnote{\fbackref{2:26} Lit. \fbib{people}; and so throughout the chapter} to stop pursuing their own relatives?''

\v{27}Joab answered, ``As God lives, if you hadn't spoken up, by morning my army would have broken off their pursuit of their own relatives.'' \v{28}So Joab sounded his battle trumpet, his entire army stopped pursuing Israel any longer, and they quit fighting.

\v{29}Abner and his army traveled through the Arabah by night, crossed the Jordan, and arrived at Mahanaim after marching all morning. \v{30}Joab returned from his pursuit of Abner, and when he had mustered his entire army, nineteen of David's soldiers\fnote{\fbackref{2:30} Lit. \fbib{servants}} were missing besides Asahel. \v{31}Meanwhile, other\fnote{\fbackref{2:31} The Heb. lacks \fbib{other}} soldiers of David had killed 360 of Abner's men from the tribe of\fnote{\fbackref{2:31} The Heb. lacks \fbib{the tribe of}} Benjamin. \v{32}They retrieved Asahel's body and buried him in his father's tomb at Bethlehem. Then Joab and his men marched all night until daybreak and arrived back in Hebron.
\labelchapt{3}
\passage{Abner Changes Loyalties}

\chapt{3}
\v{1}After this, a state of protracted war existed between Saul's dynasty and David's dynasty, and the dynasty of David continued to grow and become strong while the dynasty of Saul continued to grow weaker. \v{2}During this time, sons were born to David while he was living in Hebron. His firstborn was Amnon by Ahinoam from Jezreel, \v{3}his second was Chileab by Abigail, widow of Nabal from Carmel, his third was Absalom by Maacah, daughter of King Talmai from Geshur, \v{4}his fourth was Adonijah by Haggith, his fifth was Shephatiah by Abital, \v{5}and his sixth was Ithream by David's wife Eglah. They were all\fnote{\fbackref{3:5} The Heb. lacks \fbib{all}} born to David in Hebron.

\v{6}While war continued between the dynasties of Saul and David, Abner was growing in influence within the dynasty of Saul. \v{7}Meanwhile, Saul had a mistress\fnote{\fbackref{3:7} Or \fbib{concubine}; i.e. a secondary wife; and so throughout the chapter} named Rizpah, who was the\fnote{\fbackref{3:7} The Heb. lacks \fbib{who was the}} daughter of Aiah. Ish-bosheth\fnote{\fbackref{3:7} Lit. \fbib{And he}; cf. vs. 8} asked Abner, ``Why did you have sex with my father's mistress?''

\v{8}What Ish-bosheth\fnote{\fbackref{3:8} Cf. 1Chr 8:33, where he is named \fbib{Esh-baal}; i.e., a man devoted to Baal} said made Abner furious, so he replied, ``A dog's head for Judah---is that what I am? Up until today I've kept on showing loyalty to your father Saul's dynasty, to his relatives and friends, and I haven't turned you over to David, but you're charging me today with moral guilt regarding this woman! \v{9}Therefore may God do to me\fnote{\fbackref{3:9} Lit. \fbib{to Abner}}---and more also!---just as the \divine{Lord} has promised to David, since I'm doing this for him: \v{10}I will take away the kingdom from the dynasty of Saul by making the throne of David firm over Israel and Judah---from Dan to Beer-sheba!''

\v{11}Ish-bosheth\fnote{\fbackref{3:11} Lit. \fbib{he}} couldn't say another word in response to Abner, because he was terrified of him. \v{12}So Abner sent messengers to David at Hebron to ask him, ``Who owns this land? Cut a deal\fnote{\fbackref{3:12} Lit. \fbib{covenant}} with me, and look!---I'll lend my hand in bringing all of Israel over to you!''

\v{13}David replied, ``Sounds good to me! I'll cut a deal\fnote{\fbackref{3:13} Lit. \fbib{covenant}} with you under one condition: you're not to show yourself in my presence unless you bring Saul's daughter with you when you come to see me.'' \v{14}Then David sent a delegation to Saul's son Ish-bosheth to say, ``Give me my wife Michal, to whom I was engaged with a dowry of 100 Philistine foreskins.''\fnote{\fbackref{3:14} Cf. 1Sam 18:25ff}

\v{15}So Ish-bosheth ordered that she be taken away from her husband, Laish's son Paltiel. \v{16}Her husband accompanied her, crying as he followed after her all the way to Bahurim, where Abner told him, ``Leave! Go back!'' So he went back.
\passage{David's Dynasty is Strengthened}

\v{17}Later, Abner had a talk with the elders of Israel. He said, ``In the past you were looking to see David made king over you. \v{18}So do it, then! Because the \divine{Lord} has said this about David:

\begin{poetry}
\poeml `Through my servant David I will save my people Israel \\
\poemll    from the control of the Philistines \\
\poemlll       and from all of their enemies.'\,''
\end{poetry}

\v{19}Abner also addressed the tribe of Benjamin. Furthermore, with David's permission,\fnote{\fbackref{3:19} Lit. \fbib{in the hearing of David}; i.e., with David's tacit knowledge} Abner said anything that seemed like it would be good for Israel and for the entire tribe of Benjamin.

\v{20}Afterwards, Abner brought 20 soldiers to David at Hebron, and David threw a party for Abner and the men who were with him. \v{21}So Abner told David, ``Give me permission to go out and rally all of Israel to your majesty the king so they can enter into a formal agreement with you to reign over everything that your heart desires.'' So David sent Abner off, and he went away in peace.
\passage{Joab Murders Abner}

\v{22}Right about then, David's servants returned from a raid, bringing plenty of war booty with them, but Abner wasn't in Hebron with David, since David\fnote{\fbackref{3:22} Lit. \fbib{he}} had sent him away and Abner\fnote{\fbackref{3:22} Lit. \fbib{he}} had left in peace. \v{23}When Joab returned with his entire army, Joab was informed, ``Ner's son Abner visited the king, and he has dismissed him. He has left in peace.''

\v{24}So Joab approached the king and asked him, ``What have you done? Look, Abner came to you! What's this? You sent him away? He's long gone now! \v{25}You know Ner's son Abner came to mislead you, to learn your troop movements,\fnote{\fbackref{3:25} Lit. \fbib{to know your comings and goings}} and to learn everything you're doing!''

\v{26}As soon as Joab left David, Joab\fnote{\fbackref{3:26} Lit. \fbib{he}} sent messengers after Abner, and they brought him back from the cistern at Sirah, but David was not aware of this. \v{27}When Abner returned to Hebron, Joab brought him aside within the gateway to talk to him alone and then stabbed him in the abdomen.\fnote{\fbackref{3:27} Lit. \fbib{him there the fifth}; i.e., below the fifth rib} So he died for shedding\fnote{\fbackref{3:27} The Heb. lacks \fbib{shedding}} the blood of Joab's\fnote{\fbackref{3:27} Lit. \fbib{his}} brother Asahel.

\v{28}Later on, David found out about it and proclaimed, ``Let me and my kingdom remain guiltless forever in the \divine{Lord}'s presence for the death\fnote{\fbackref{3:28} Lit. \fbib{blood}} of Ner's son Abner. \v{29}May judgment\fnote{\fbackref{3:29} Lit. \fbib{guilt}} rest on Joab's head and on his father's entire household. May Joab's dynasty never be without one who has a discharge,\fnote{\fbackref{3:29} I.e. one who is ceremonially unfit to serve God; cf. Lev 13:46} who is a leper, who walks with a cane,\fnote{\fbackref{3:29} Lit. \fbib{who needs a staff}} who commits suicide,\fnote{\fbackref{3:29} Lit. \fbib{who falls on a sword}} or who lacks food!'' \v{30}He said this\fnote{\fbackref{3:30} The Heb. lacks \fbib{He did this}} because Joab and his brother Abishai murdered Abner after he had killed their brother Asahel in the battle at Gibeon.

\v{31}David ordered Joab and all the people who were with him, ``Tear your clothes, put on sackcloth, and mourn for Abner.'' King David walked behind the funeral procession, \v{32}and they buried Abner at Hebron. The king wept loudly at Abner's grave, and all the people wept, too. \v{33}The king composed this mourning song for Abner:

\begin{poetry}
\poeml ``Should Abner's death be like a fool's? \\
\poeml \v{34}Your hands were not bound, \\
\poemlll       nor were your feet in irons. \\
\poeml As one falls before the wicked, \\
\poemll    you have fallen.''
\end{poetry}

Then all the people cried again because of him. \v{35}Everyone tried to persuade David to have a meal while there was still daylight, but David took an oath by saying, ``May God to do like this to me and more, if I taste bread or anything else before the sun sets!''

\v{36}Everybody took note of this and was very pleased, just as everything else the king did pleased everyone. \v{37}As a result, the entire army and all of Israel understood that day that the king had nothing to do with the murder of Ner's son Abner.

\v{38}The king reminded his staff,\fnote{\fbackref{3:38} Lit. \fbib{servants}} ``Don't you know that a prince and a great man has fallen today in Israel? \v{39}Today, even though I'm anointed as king, I'm weak. These men, sons of Zeruiah, are too difficult\fnote{\fbackref{3:39} Or \fbib{violent}} for me. May the \divine{Lord} repay the one who acts wickedly in accordance with his wickedness!''
\labelchapt{4}
\passage{The Murder of Ish-bosheth}

\chapt{4}
\v{1}When Saul's son heard that Abner had died in Hebron, his courage\fnote{\fbackref{4:1} Lit. \fbib{hands}} failed and all of Israel was disturbed. \v{2}Now Saul's son had two officers in charge of some raiding parties. One was named Baanah and the other was named Rechab. They were sons of Rimmon, a descendant of Benjamin from Beeroth, which was considered to belong to the tribe of\fnote{\fbackref{4:2} The Heb. lacks \fbib{the tribe of}} Benjamin. \v{3}(The residents of Beeroth had evacuated to Gittaim and live there as resident aliens to this day.)

\v{4}Meanwhile, Saul's son Jonathan had a son whose feet were crippled. When he was five years old, news had arrived about Saul and Jonathan from Jezreel, and his nurse picked him up to flee, but in her hurry to leave, he happened to fall and became lame. His name was Mephibosheth.\fnote{\fbackref{4:4} Cf. 1Chr 8:34; 9:40, where his name is recorded as \fbib{Merib-baal}}

\v{5}Rechab and Baanah, the sons of Rimmon the Beerothite, left and arrived during the hottest part of the day at the home of Ish-bosheth while he was taking a noon day nap. \v{6}They entered the house as though they intended to obtain some grain and stabbed him in the abdomen. Then Rechab and his brother Baanah escaped. \v{7}While they were in the house, they struck him, killed him, and cut off his head while he was lying on his bed in his bedroom. They took his head, and traveled all night along the Arabah road.
\passage{David Punishes the Killers of Ish-bosheth}

\v{8}They brought Ish-bosheth's head to David at Hebron and told the king, ``Look! Here's the head of your enemy Ish-bosheth, Saul's son, who sought your life. Today the \divine{Lord} has given your majesty the king vengeance on Saul and his descendants.''\fnote{\fbackref{4:8} Lit. \fbib{seed}}

\v{9}David responded to Rechab and his brother Baanah, the sons of Rimmon the Beerothite: ``As the \divine{Lord} lives, who has saved my life in every adversity, \v{10}when the man who told me `Look! Saul is dead!' thought he was bringing me good news, I arrested him and had him killed at Ziklag as the reward I gave him for his news. \v{11}How much worse will it be, then, when evil men kill an innocent man on his own bed in his own house! Shouldn't I avenge his blood---which you are responsible for shedding\fnote{\fbackref{4:11} Lit. \fbib{blood from your hand}}---by removing you from the earth?'' \v{12}So David commanded his personal guards,\fnote{\fbackref{4:12} Lit. \fbib{his young men}} and they killed Rechab and Baanah,\fnote{\fbackref{4:12} Lit. \fbib{killed them}} cut off their hands and feet, and hung up their bodies beside the pool at Hebron. They took Ish-bosheth's head and buried it in Abner's tomb at Hebron.
\labelchapt{5}
\passage{David Becomes King over Israel}
\passageinfo{(1 Chronicles 11:1-3)}

\chapt{5}
\v{1}After this, all of the tribes of Israel assembled with David at Hebron and declared, ``Look, we're your own flesh and blood!\fnote{\fbackref{5:1} Lit. \fbib{bone}} \v{2}Even back when Saul was our king, it was you who kept on leading Israel out to battle\fnote{\fbackref{5:2} The Heb. lacks \fbib{to battle}} and bringing them back again.\fnote{\fbackref{5:2} The Heb. lacks \fbib{back again}} The \divine{Lord} told you, `You yourself will shepherd my people Israel and serve as Commander-in-Chief\fnote{\fbackref{5:2} Lit. \fbib{Nagid}; i.e. a senior officer entrusted with dual roles of operational oversight and administrative authority} over Israel.'\,'' \v{3}So all the elders of Israel approached the king at Hebron, where King David entered into a covenant with them in the presence of the \divine{Lord}. Then they anointed David to be king over Israel.
\passage{David Establishes Jerusalem as His Capital}
\passageinfo{(1 Chronicles 11:4-9; 14:1-7)}

\v{4}David began to reign when he was 30 years old, and he reigned 40 years. \v{5}He reigned over Judah for seven years and six months in Hebron, and he reigned over all of Israel including Judah for 33 years in Jerusalem. \v{6}Later, the king and his army marched on Jerusalem against the Jebusites, who were inhabiting the territory at that time\fnote{\fbackref{5:6} The Heb. lacks \fbib{at that time}} and who had told David, ``You're not coming in here! Even the blind and the lame could turn you away!'' because they were thinking\fnote{\fbackref{5:6} Lit. \fbib{saying}} ``David can't come here.'' \v{7}Even so, David captured the stronghold of Zion, which is now known as\fnote{\fbackref{5:7} The Heb. lacks \fbib{now known as}} the City of David.

\v{8}At that time,\fnote{\fbackref{5:8} Lit. \fbib{day}} David had said, ``Whoever intends to attack the Jebusites will have to climb up the water shaft to attack the lame and blind, who hate David.''\fnote{\fbackref{5:8} Or \fbib{whom David hates}; LXX reads \fbib{blind, and those who hate David}}

Therefore they say, ``The blind and lame are never to come into the house.'' \v{9}David occupied\fnote{\fbackref{5:9} Or \fbib{lived in} } the fortress, naming it the City of David. He\fnote{\fbackref{5:9} Lit. \fbib{David}} built up the surroundings from the terrace ramparts\fnote{\fbackref{5:9} Lit. \fbib{the Millo}, fortified areas of ancient Jerusalem with terraces and retaining walls} inward. \v{10}David became more and more esteemed because the \divine{Lord} God of the Heavenly Armies was with him.

\v{11}Later, King Hiram of Tyre sent a delegation to David, accompanied by cedar\fnote{\fbackref{5:11} I.e. a genus of coniferous evergreen in the family \fbib{Pinaceae}; and so throughout the book} logs, carpenters, and stone masons. They built a palace for David. \v{12}So David concluded\fnote{\fbackref{5:12} Lit. \fbib{knew}} that the \divine{Lord} had established him as king over Israel and that he had exalted his kingdom in order to benefit his people Israel. \v{13}But after arriving in Jerusalem after leaving Hebron, David took more wives and mistresses,\fnote{\fbackref{5:13} Or \fbib{concubines}; i.e. secondary wives} and more sons and daughters were born to David. \v{14}These are the names of those who were born to him in Jerusalem: Shammua, Shobab, Nathan, Solomon, \v{15}Ibhar, Elishua, Nepheg, Japhia, \v{16}Elishama, Eliada, and Eliphelet.
\passage{David Battles the Philistines}
\passageinfo{(1 Chronicles 14:8-17)}

\v{17}When the Philistines eventually learned that Israel\fnote{\fbackref{5:17} The Heb. lacks \fbib{Israel}} had anointed David to be king over Israel, they marched out in search of him.\fnote{\fbackref{5:17} Lit. \fbib{David}} But David heard about it and retreated to his stronghold. \v{18}Meanwhile, the Philistines arrived and encamped in the Rephaim Valley, \v{19}so David asked the \divine{Lord}, ``Am I to go attack the Philistines? Will you give me victory over them?''\fnote{\fbackref{5:19} Lit. \fbib{give them into my hand}}

``Go get them,'' the \divine{Lord} replied to David, ``because I'm going to put the Philistines right into your hand!''

\v{20}So David went to Baal-perazim and defeated them there. He called the place Baal-perazim,\fnote{\fbackref{5:20} The Heb. name means \fbib{Lord of breaking forth}; cf. 2Sam 6:8} because he said, ``Like a bursting flood, the \divine{Lord} has jumped out in front of me to fight my enemies.'' \v{21}The Philistines abandoned their idols there, and David and his army carried them off.

\v{22}Later, the Philistines once again marched out and encamped in the Rephaim Valley. \v{23}When David asked the \divine{Lord} about it, he said, ``Don't attack them directly. Instead, go around to the rear and attack them opposite those balsam trees. \v{24}When you hear the sound of marching coming from the tops of the balsam trees, then be sure to act quickly, since the \divine{Lord} will have gone out ahead of you to cut down the Philistine army.'' \v{25}So David did exactly what the \divine{Lord} ordered him to do, and he struck down the Philistines from Geba to Gezer.
\labelchapt{6}
\passage{Troubles in Mishandling the Ark}
\passageinfo{(1 Chronicles 13:1-14; 15:25-16:3)}

\chapt{6}
\v{1}After this, David gathered together again 30,000 men from all of the choicest men of Israel. \v{2}Then David and all the people with him set out from Baal-judah to bring up from there the Ark of God, who is called the Name, the name of the \divine{Lord} of the Heavenly Armies, and who is enthroned on the cherubim. \v{3}They mounted the Ark of God on a new cart and brought it from Abinadab's home in Gibeah,\fnote{\fbackref{6:3} Or \fbib{was on the hill}} with Abinadab's sons Uzzah and Ahio\fnote{\fbackref{6:3} Or \fbib{and his brother}} driving the new cart. \v{4}As they left Abinadab's house in Gibeah accompanied by the Ark of God, Ahio was walking ahead of the ark. \v{5}David and the entire assembly\fnote{\fbackref{6:5} Lit. \fbib{house}} of Israel were dancing in the presence of the \divine{Lord} with all of their strength, accompanied by all sorts of wood instruments,\fnote{\fbackref{6:5} Cf. 1Chr 13:8, where MT letters of the word \fbib{cypress} may be transposed as MT word \fbib{song}} harps, tambourines, castanets, and cymbals.

\v{6}When they arrived at Nacon's threshing floor, Uzzah reached out and grabbed the Ark of God because the oxen had stumbled. \v{7}Just then, the anger of the \divine{Lord} blazed against Uzzah, and God struck him down right there because of his failure, and he died there beside the Ark of God.

\v{8}David flew into a rage because the \divine{Lord} had killed\fnote{\fbackref{6:8} Lit. \fbib{had burst out against}} Uzzah. That's why that place is called Perez-uzzah\fnote{\fbackref{6:8} The Heb. name \fbib{Perez-uzzah} means \fbib{Overwhelming Uzzah}; cf. 2Sam 5:20} to this day. \v{9}But David feared the \divine{Lord} that day, and asked, ``How can the Ark of God come to me?'' \v{10}As a result, David was unwilling to take the ark of the \divine{Lord} into his care in the City of David. Instead, David left it at the home of Obed-edom the Gittite. \v{11}So the ark of the \divine{Lord} remained for three months in the household of Obed-edom the Gittite while the \divine{Lord} blessed Obed-edom and his entire household.

\v{12}Later on, David was informed, ``The \divine{Lord} has blessed the home of Obed-edom and everything he has since he's in possession\fnote{\fbackref{6:12} Or \fbib{has on account}} of the Ark of God.'' So David went out joyfully and brought up the Ark of God to the City of David from Obed-edom's home. \v{13}After those who were carrying the ark of the \divine{Lord} had taken six steps, he sacrificed oxen and fattened animals, \v{14}dancing in front of the \divine{Lord} with all of his strength and wearing a linen ephod. \v{15}So David and the entire assembly\fnote{\fbackref{6:15} Lit. \fbib{house}} of Israel brought up the ark of the \divine{Lord} with shouting and trumpet blasts.
\passage{David's Wife Michal Disrespects David's Worship}

\v{16}As the ark of the \divine{Lord} was coming into the City of David, Saul's daughter Michal was peering out a window, watching King David jumping and dancing in the \divine{Lord}'s presence, and she despised him in her heart. \v{17}They brought in the ark of the \divine{Lord}, set it in place inside the tent that David had erected for it, and David sacrificed burnt offerings and peace offerings in the presence of the \divine{Lord}.

\v{18}After David had finished sacrificing the burnt offerings and peace offerings, he blessed the people in the name of the \divine{Lord} of the Heavenly Armies \v{19}and distributed to all the people---the entire multitude of Israel, including both men and women---a cake made of bread, one made of dates, and one made of raisins to each one. Then all the people left, each headed for home.

\v{20}When David returned to bless his household, Saul's daughter Michal came out to meet him and called out, ``How the king of Israel honored himself today by undressing himself right in front of his women staff members, just like any pervert\fnote{\fbackref{6:20} Lit. \fbib{like one of the worthless ones}} would dare to expose himself!''

\v{21}But David replied to Michal, ``It was in front of the \divine{Lord}, who appointed me to replace your father and his entire household by selecting me as Commander-in-Chief\fnote{\fbackref{6:21} Lit. \fbib{Nagid}; i.e. a senior officer entrusted with dual roles of operational oversight and administrative authority} over Israel, the people of the \divine{Lord}, that I danced in front of the \divine{Lord}. \v{22}I'm going to act more shamelessly than this, even to humbling myself in my own eyes. Now as to the women staff members about whom you have spoken, they are to hold me in honor!'' \v{23}And Saul's daughter Michal bore no children from that day on until the day she died.
\labelchapt{7}
\passage{David Plans to Build the Temple}
\passageinfo{(1 Chronicles 17:1-15)}

\chapt{7}
\v{1}After the king had settled down in his palace and the \divine{Lord} had given him respite from all of his surrounding enemies, \v{2}he\fnote{\fbackref{7:2} Lit. \fbib{the king}} told the prophet Nathan, ``Look now, I'm living in a cedar palace, but the Ark of God resides behind\fnote{\fbackref{7:2} Lit. \fbib{between}} a tent\fnote{\fbackref{7:2} The Heb. lacks \fbib{tent}} curtain.''

\v{3}Nathan replied to the king, ``Go do everything you have in mind,\fnote{\fbackref{7:3} Lit. \fbib{heart}} because the \divine{Lord} is with you.''

\v{4}But later that same night, this message came to Nathan from the \divine{Lord}:

\begin{poetry}
\poeml \v{5}``Go tell my servant David, `This is what the \divine{Lord} says: \\
\poeml `````Are you going to build a house\fnote{\fbackref{7:5} I.e. a temple, and so throughout the chapter} for me to inhabit? \v{6}After all, I haven't lived in a house since the day I brought up the Israelis from Egypt until now. Instead, I've moved around in a tent that served as my\fnote{\fbackref{7:6} Lit. \fbib{tent and}} dwelling place. \v{7}Wherever I moved among the Israelis, did I ever ask even one tribal leader\fnote{\fbackref{7:7} Lit. \fbib{ask the tribes}} of Israel whom I commanded to shepherd my people Israel, `Why haven't you built me a cedar house?' \\
\poeml \v{8}`````Now therefore this is what you are to tell my servant David: `This is what the \divine{Lord} of the Heavenly Armies says: ``I took you from the pasture myself---from tending sheep---to become Commander-in-Chief\fnote{\fbackref{7:8} Lit. \fbib{Nagid}; i.e. a senior officer entrusted with dual roles of operational oversight and administrative authority} over my people, that is, over Israel. \\
\poeml \v{9}`````Furthermore, I have remained with you everywhere you have gone, annihilating all your enemies right in front of you. I will make a great reputation\fnote{\fbackref{7:9} Lit. \fbib{name}} for you, like the reputation\fnote{\fbackref{7:9} Lit. \fbib{name}} of great ones who have lived on\fnote{\fbackref{7:9} The Heb. lacks \fbib{have lived}} earth. \v{10}I will establish a homeland\fnote{\fbackref{7:10} Lit. \fbib{place}} for my people---for Israel---planting them so they may live in a secure location where they will never be disturbed anymore. Wicked people\fnote{\fbackref{7:10} Lit. \fbib{Children of wickedness}} will no longer afflict them, as happened in the past \v{11}when I had commanded judges to administer\fnote{\fbackref{7:11} Lit. \fbib{judges over}} my people Israel. I'll also grant you relief from all your enemies.''\,' \\
\poeml ```The \divine{Lord} also announces to you: ``The \divine{Lord} will himself build a house\fnote{\fbackref{7:11} I.e. a dynasty} for you. \v{12}When your life\fnote{\fbackref{7:12} Lit. \fbib{days}} is complete and you go to join\fnote{\fbackref{7:12} Lit. \fbib{you rest with}} your ancestors, I will raise up your offspring\fnote{\fbackref{7:12} Lit. \fbib{seed}; MT is sing.} after you, who will come forth from your body,\fnote{\fbackref{7:12} Lit. \fbib{your inward parts}} and I will fortify his kingdom. \v{13}He will build a Temple dedicated to my Name, and I will make the throne of his kingdom last forever. \v{14}I will be a father to him, and he will be to me a son who, when he commits iniquity, I will discipline with the rod wielded by armies\fnote{\fbackref{7:14} Lit. \fbib{men}} and with wounds inflicted by human beings.\fnote{\fbackref{7:14} Lit. \fbib{by children of Adam}} \v{15}But I'll never remove my gracious love from him as I did from Saul, whom I removed from your presence. \v{16}Your dynasty and your kingdom will remain forever in my presence---your throne will be secure forever.''\,'\,''
\end{poetry}

\v{17}Nathan communicated this complete oracle to David with precisely these words.
\passage{David's Prayer}
\passageinfo{(1 Chronicles 17:16-27)}

\v{18}Then King David went in to the presence of the \divine{Lord}, sat down, and said:

\begin{poetry}
\poeml ``Who am I, Lord \divine{God}, and what is my family,\fnote{\fbackref{7:18} Lit. \fbib{house} or \fbib{household}, and so throughout the chapter} that you have brought me to this? \v{19}And this is still a small thing to you, Lord \divine{God}---you also have spoken about the future of your servant's house, and this is the charter\fnote{\fbackref{7:19} Or \fbib{law} or \fbib{instruction}} for mankind, O Lord \divine{God}! \\
\poeml \v{20}``What more can David say to you, and you surely know your servant, Lord \divine{God}. \v{21}For the sake of your word and consistent with your desire,\fnote{\fbackref{7:21} Lit. \fbib{heart}; cf. Eph 1:5} you have done all of these great things, informing your servant. \v{22}And therefore you are great, Lord \divine{God}, there is no one like you, there is no God except for you, just as we've heard with our own ears. \\
\poeml \v{23}``And who is like your people, like Israel, the one nation on earth that God went out to redeem as a people for himself, to make a name for himself, and to carry out for them great and awe-inspiring accomplishments, driving out nations and their gods in front of your people, whom you redeemed to yourself from Egypt? \v{24}You have prepared your people Israel to be your very own people for ever, and you, \divine{Lord}, have become their God! \\
\poeml \v{25}``And now, \divine{Lord} God, let what you have spoken concerning your servant and his household be done---and let it be done just as you've promised. \v{26}May your name be made great forever with the result that it is said that the \divine{Lord} of the Heavenly Armies is God over Israel, and that the household of your servant David may be established before you. \v{27}For you, \divine{Lord} of the Heavenly Armies, the God of Israel, have revealed this to your servant, telling him, `I will build a dynasty for you,' so that your servant has found fortitude\fnote{\fbackref{7:27} Lit. \fbib{heart}} to pray this prayer to you. \\
\poeml \v{28}``Now therefore, Lord \divine{God}, you are God, and your words are true, and you have spoken to your servant these good things. \v{29}So may it please you to bless the household of your servant, so that it might remain forever in your presence, because you, Lord \divine{God}, have spoken, and from your blessing may the household of your servant be blessed forever.''
\end{poetry}
\labelchapt{8}
\passage{David's Military Victories}
\passageinfo{(1 Chronicles 18:1-13)}

\chapt{8}
\v{1}Sometime later, David defeated and subdued the Philistines, taking Metheg-ammah away from the Philistines. \v{2}David also conquered Moab, then measured them with a cord, making them lie down on the ground. He executed everyone measured out in each two lengths' measurement of the cord, but spared the ones measured out by every third length. Then the Moabites were placed under servitude to David, and made to pay tribute.

\v{3}David also attacked King Hadadezer, Rehob's son from Zobah, when he was attempting to restore his hegemony\fnote{\fbackref{8:3} Lit. \fbib{hand}} over the Euphrates\fnote{\fbackref{8:3} The Heb. lacks \fbib{Euphrates}} River. \v{4}David captured 1,000 of his chariots, 1,700\fnote{\fbackref{8:4} So MT; LXX reads 7,000; cf. 1Chr 18:4} horsemen, and 20,000 foot soldiers. David hamstrung all the chariot horses except for enough to supply\fnote{\fbackref{8:4} The Heb. lacks \fbib{enough to supply}} 100 chariots. \v{5}When Arameans came from Damascus to help King Hadadezer of Zobah, David killed 22,000 of them. \v{6}David erected garrisons in the Aramean kingdom of Damascus, placing the Arameans under servitude to him,\fnote{\fbackref{8:6} Lit. \fbib{David}} and they paid tribute to him. \v{7}David also confiscated the gold shields that belonged to Hadadezer's officers and took them to Jerusalem. \v{8}He\fnote{\fbackref{8:8} Lit. \fbib{David}} also confiscated a vast quantity of bronze from Betah and Berothai, cities under Hadadezer's control.

\v{9}When King Tou of Hamath learned that David had conquered the entire army of King Hadadezer of Zobah, \v{10}Tou sent his son Joram to King David to greet him and congratulate him on his victory over Hadadezer, because he had been at war with Tou. Joram brought articles of silver, gold, and bronze with him, \v{11}and King David dedicated them to the \divine{Lord}, along with the silver and gold that had been dedicated from all the nations that he had conquered, \v{12}including from Edom, Moab, the Ammonites, the Philistines, Amalek, and spoil from King Hadadezer, Rehob's son from Zobah.

\v{13}David made a name for himself when he returned from killing 18,000 Edomites in the Salt Valley. \v{14}He erected garrisons throughout Edom, and all the Edomites became subservient to David, while the \divine{Lord} gave victory to David wherever he went.
\passage{David's Leaders}
\passageinfo{(1 Chronicles 18:14-17)}

\v{15}David reigned over all of Israel, administering\fnote{\fbackref{8:15} Lit. \fbib{with David administering}} justice and equity to every one of his people. \v{16}Zeruiah's son Joab served in charge of the army, Ahilud's son Jehoshaphat was his personal archivist,\fnote{\fbackref{8:16} Or \fbib{recorder}; an officer who kept official records of David's administration} \v{17}Ahitub's son Zadok and Abiathar's son Ahimelech were priests, Seraiah\fnote{\fbackref{8:17} Cf. 1Chr 18:16, which reads \fbib{Shavsha}} was his personal secretary,\fnote{\fbackref{8:17} Or \fbib{scribe}} \v{18}Jehoida's son Benaiah supervised the special forces\fnote{\fbackref{8:18} Lit. \fbib{Cherethites}; i.e. elite body guards} and mercenaries,\fnote{\fbackref{8:18} Lit. \fbib{Pelethites}; i.e. special couriers} and David's sons were priests.\fnote{\fbackref{8:18} Cf. 1Chr 18:17, which describes them as special officials}
\labelchapt{9}
\passage{David Shows Kindness to Mephibosheth}

\chapt{9}
\v{1}Later on, David asked, ``Is there anyone left alive from Saul's household to whom I can show gracious love in memory\fnote{\fbackref{9:1} Lit. \fbib{love for the sake}} of Jonathan?''

\v{2}A household servant of Saul named Ziba was called to appear before David, and the king asked him, ``Are you Ziba?''

``I am your servant,'' Ziba replied.

\v{3}At this the king asked, ``Isn't there still someone left from Saul's household to whom I may show God's gracious love?''

``There's Jonathan's son. He has maimed feet, '' Ziba answered.

\v{4}So David asked, ``Where is he?''

Ziba responded, ``He's in Lo-debar at the home of Ammiel's son Makir.''

\v{5}At this, King David sent for him and brought him from the home of Ammiel's son Makir in Lo-debar. \v{6}When Mephibosheth, Jonathan's son and a grandson of Saul, approached David, he threw himself on his face out of respect.

``Mephibosheth!'' David said as he greeted him.

``Hello! I am your servant,'' he replied.

\v{7}``Don't be afraid,'' David reassured him, ``because I'm going to show gracious love to you in memory\fnote{\fbackref{9:7} Lit. \fbib{love for the sake}} of your father Jonathan. I'm going to restore to you all the land that belonged to your grandfather Saul, and you'll always have a place\fnote{\fbackref{9:7} Lit. always eat} at my table!''

\v{8}Mephibosheth\fnote{\fbackref{9:8} Lit. \fbib{He}} bowed low again and asked, ``Who am I, your servant, that you would pay attention to a dead dog like me?''

\v{9}At this, the king called for Saul's servant Ziba and told him, ``I'm restoring to your master's grandson everything that belonged to Saul and his family. \v{10}You and your servants are to farm the land on his behalf and bring in the crops in order to provide for your master's grandson. Meanwhile, Mephibosheth, your master's grandson, will always have a place\fnote{\fbackref{9:10} Lit. \fbib{always eat}} at my table.'' (Now Ziba had fifteen sons and 20 servants.)

\v{11}Later, Ziba told the king, ``Your servant will do everything that your majesty the king commands him.'' So Mephibosheth ate at David's table like one of the king's sons. \v{12}Mephibosheth fathered a son named Mica, and everyone who lived in Ziba's house became Mephibosheth's servants. \v{13}Mephibosheth continued to live in Jerusalem, always eating at the king's table, since he was maimed in both feet.
\labelchapt{10}
\passage{Subjugation of Ammon and Aram}
\passageinfo{(1 Chronicles 19:1-19)}

\chapt{10}
\v{1}Sometime later, the Ammonite king died and his son Hanun succeeded him as king, \v{2}so David told himself, ``I will be loyal to Nahash's son Hanun, since in his loyalty his father showed gracious love to me.'' So David sent a delegation\fnote{\fbackref{10:2} Lit. \fbib{sent by the hand of his servants}} to Hanun to console him about his loss of\fnote{\fbackref{10:2} The Heb. lacks \fbib{his loss of}} his father.

But when David's delegation arrived in Ammonite territory, \v{3}the Ammonite officials asked their lord Hanun, ``Do you think that because David has sent a delegation of consolers to you that he is honoring your father? His delegation has arrived intending to search, scout the land, and then overthrow it, hasn't it?'' \v{4}So Hanun arrested David's delegation, shaved off half of their beards, cut off their clothes at the waist line, and sent them away in disgrace.\fnote{\fbackref{10:4} The Heb. lacks \fbib{in disgrace}}

\v{5}When David had been informed about the incident,\fnote{\fbackref{10:5} The Heb. lacks \fbib{about the incident}} he sent word\fnote{\fbackref{10:5} The Heb. lacks \fbib{word}} to them, since the men had been deeply humiliated. The king told them, ``Stay at Jericho until your beards have grown back, and then return.''

\v{6}When the Ammonites realized that they had created quite a stink with\fnote{\fbackref{10:6} Lit. \fbib{had become odious to}} David, they hired 20,000 Aramean mercenaries from Beth-rehob and Zobah, along with the king of Maacah and 1,000 men, and 12,000 men from Tob. \v{7}In response, David sent out Joab and his entire army of elite soldiers. \v{8}The Ammonites went out in battle formation at the entrance to the city\fnote{\fbackref{10:8} The Heb. lacks \fbib{city}} gate, while the Arameans from Zobah and Rehob, along with the army\fnote{\fbackref{10:8} Lit. \fbib{men}} from Tob and Maacah, were out by themselves in the open fields.

\v{9}When Joab observed that the battle lines were set up to oppose him both in front and behind, he appointed the best troops in Israel and arrayed them to oppose the Arameans, \v{10}putting the rest of his forces under the command of his brother Abishai, who arrayed them to oppose the Ammonites. \v{11}He said, ``If the Arameans prove too strong for me, then you are to help me. If the Ammonites prove too strong for you, then I will come help you. \v{12}Be strong, be courageous on behalf of our people and for the cities of our God, and may the \divine{Lord} do what he thinks is best.''

\v{13}So Joab and the soldiers who were with him attacked the Arameans in battle formation, and the Arameans retreated in front of him. \v{14}When the Ammonites saw the Arameans retreating, they also retreated from Abishai back to the city. Then Joab broke off his attack against the Ammonites and went back to Jerusalem. \v{15}After the Arameans realized that they had been defeated by Israel, they regrouped. \v{16}Hadadezer sent for the Arameans who lived beyond the Euphrates River,\fnote{\fbackref{10:16} The Heb. lacks \fbib{Euphrates}} and they set out for Helam, with Shobach\fnote{\fbackref{10:16} Cf. 1Chr 19:16, which reads \fbib{Shophach}} leading them as commander of Hadadezer's army.

\v{17}When David learned this, he mustered all of Israel, crossed the Jordan River, and approached Helam. The Arameans assembled in battle array to attack David, and started their assault. \v{18}But the Arameans retreated from Israel, and David's forces\fnote{\fbackref{10:18} Lit. \fbib{David}} killed 700 of their charioteers, 40,000 soldiers, and mortally wounded Shobach, the commander of their army. As a result, Shobach\fnote{\fbackref{10:18} Lit. \fbib{he}} died there. \v{19}When all the kings who were allied with\fnote{\fbackref{10:19} Lit. \fbib{were servants of}} Hadadezer saw that they had been defeated by Israel, they sought terms of peace with the Israelis and became subservient to them. Furthermore, the Arameans were afraid to help the Ammonites anymore.
\labelchapt{11}
\passage{David's Adultery}

\chapt{11}
\v{1}One spring day, during the time of year when kings go off to war, David sent out Joab, along with his personal staff\fnote{\fbackref{11:1} Lit. \fbib{his servants}} and all of Israel's army. They utterly destroyed the Ammonites and then attacked Rabbah while David remained in Jerusalem. \v{2}Late one afternoon about dusk,\fnote{\fbackref{11:2} Lit. \fbib{It happened at the time of the evening}} David got up from his couch and was walking around on the roof of the royal palace. From there\fnote{\fbackref{11:2} Lit. \fbib{From the roof}} he watched a woman taking a bath, and she\fnote{\fbackref{11:2} Lit. \fbib{and the woman}} was very beautiful to look at.

\v{3}David sent word\fnote{\fbackref{11:3} The Heb. lacks \fbib{word}} to inquire about her,\fnote{\fbackref{11:3} Lit. \fbib{the woman}} and someone told him, ``This is Eliam's daughter Bathsheba,\fnote{\fbackref{11:3} Eliam's father was Ahithophel, Bathsheba's grandfather; cf. 2Sam 15:12; 23:34} the wife of Uriah the Hittite, isn't it?'' \v{4}So David sent some messengers, took her from her home,\fnote{\fbackref{11:4} The Heb. lacks \fbib{from her home}} and she went to him, and he had sex with her. (She had been consecrating herself following her menstrual separation.)\fnote{\fbackref{11:4} I.e. a week-long period of ritual exemption from participation in Israel's social and worship community; cf. Lev 15:19, 28; 18:19} Then she returned to her home.

\v{5}The woman conceived, and she sent this message\fnote{\fbackref{11:5} The Heb. lacks \fbib{this message}} to David: ``I'm pregnant.''

\v{6}So David summoned Joab, and told him,\fnote{\fbackref{11:6} The Heb. lacks \fbib{and told him}} ``Send me Uriah the Hittite.'' So Joab sent Uriah to David. \v{7}When Uriah arrived, David inquired about how Joab was doing, how the army was\fnote{\fbackref{11:7} Lit. \fbib{the people were}} doing, and how the war was progressing.

\v{8}Then David told Uriah, ``Go on down to your house and relax a while.''\fnote{\fbackref{11:8} Lit. \fbib{and wash your feet}} So Uriah left the king's palace, and the king sent a gift along after him. \v{9}But Uriah spent the night sleeping in the alcove of the king's palace in the company of all his master's staff members. He refused to go down to his own home.

\v{10}When David was told that Uriah hadn't gone home the previous night,\fnote{\fbackref{11:10} The Heb. lacks \fbib{the previous night} } he quizzed him,\fnote{\fbackref{11:10} Lit. \fbib{Uriah}} ``You just arrived from a long journey, so why didn't you go down to your own house?''

\v{11}Uriah replied, ``The ark, along with Israel and Judah, are encamped in tents, while my commanding officer Joab and my master's staff members are camping out in the open fields. Should I go home, eat, drink, and have sex with my wife? Not on your life!\fnote{\fbackref{11:11} Lit. \fbib{As you live and as your soul lives}} I won't do something like this, will I?''

\v{12}Then David invited Uriah, ``Stay here today, and tomorrow I'll send you back.'' So Uriah remained in Jerusalem all that day and the next. \v{13}Then at David's invitation, he and Uriah dined and drank wine together, and David got him drunk. Later that evening, Uriah went out to lie on a couch in the company of his lord's servants, and he did not go down to his house.
\passage{David Orders Uriah Killed}

\v{14}The next morning, David sent a message to Joab that Uriah took with him in his hand. \v{15}In the message, he wrote: ``Assign Uriah to the most difficult fighting at the battle front, and then withdraw from him so that he will be struck down and killed.'' \v{16}So as Joab began to attack the city, he assigned Uriah to a place where he knew valiant men would be stationed.\fnote{\fbackref{11:16} The Heb. lacks \fbib{stationed}} \v{17}When the men of the city came out to fight Joab, some of David's army staff members fell, and Uriah the Hittite died, too.

\v{18}Then Joab sent word to David about everything that had happened at the battle. \v{19}He instructed the courier, ``When you have finished conveying all the news about the battle to the king, \v{20}if the king starts to get angry and asks you, `Why did you get so near the city to fight? Didn't you know they would shoot from the wall? \v{21}Who killed Jerubbesheth's\fnote{\fbackref{11:21} I.e. Gideon (cf. Judg 8:30-31), also called \fbib{Jerubbaal} (cf. Judg 8:35)} son Abimelech? Didn't a woman kill him by throwing an upper millstone on him from the wall at Thebez? Why did you go so close to the wall?' then tell him, `Your servant Uriah the Hittite also died.'\,''

\v{22}So the messenger left Joab, set out for Jerusalem,\fnote{\fbackref{11:22} The Heb. lacks \fbib{for Jerusalem}} and disclosed to David everything that Joab had sent him to say. \v{23}The messenger told David, ``The men surprised us and attacked us in the field, but we drove them back to the entrance of the city gate. \v{24}Then the archers shot at your servants from the wall. Some of the king's staff members are dead, and your servant Uriah the Hittite has died as well.''

\v{25}David responded to the messenger, ``Here's what you're to tell Joab: `Don't be troubled by this incident, because the battle sword consumes one or another from time to time. Consolidate your attack against the city and conquer it.' Be sure to encourage him.''

\v{26}When Uriah's wife heard about the death of her husband\fnote{\fbackref{11:26} The Heb. word for \fbib{husband} (\fbib{isha}) describes a husband with respect to his relationship with his wife.} Uriah, she went into mourning for the head of her household.\fnote{\fbackref{11:26} Lit. \fbib{for her husband}; the Heb. word for \fbib{husband} (\fbib{baal}) describes a husband with respect to his role as a household leader.} \v{27}When her mourning period was completed, David sent for her, brought her to his palace, and she became his wife. Later on, she bore him a son.

Meanwhile, what David had done grieved the \divine{Lord},\fnote{\fbackref{11:27} Lit. \fbib{done was grieving in the \divine{Lord}'s sight}; i.e., the act itself is personified here as being distressed in the \fbib{}\divine{Lord}'s sight}\chapt{12}
\v{1}so the \divine{Lord} sent Nathan to David.
\labelchapt{12}
\passage{Nathan's Rebuke}

Nathan\fnote{\fbackref{12:1} Lit. \fbib{He}} approached David\fnote{\fbackref{12:1} Lit. him} and said, ``There are two men in the city. One is rich and one is poor. \v{2}The rich man has many flocks and herds, \v{3}but the poor man had nothing except for one little ewe lamb that he had bought. He raised it, and it grew up with him and his children. It used to share his food and drink from his own cup. It even slept in his arms. It was like a daughter to him. \v{4}A traveler arrived to visit the rich man. Because he was unwilling to take an animal from one of his own flocks or herds to prepare for the guest who had come to visit him, he took the poor man's lamb and prepared it for the man who had come to visit him.''

\v{5}David flew into a rage at the man and told Nathan, ``As the \divine{Lord} lives, the man who did this deserves to die! \v{6}He will restore the lamb four times its value, because he did this thing, and because he did it without compassion.''

\v{7}But Nathan replied to David, ``You are the man! This is what the \divine{Lord} God of Israel says:

```I anointed you king---and you became king over Israel.

```I delivered you from Saul's control.

\v{8}```I gave you your former\fnote{\fbackref{12:8} The Heb. lacks \fbib{former}} master's household.

```I placed your former\fnote{\fbackref{12:8} The Heb. lacks \fbib{former}} master's wives right in your arms.

```I gave you\fnote{\fbackref{12:8} Lit. \fbib{you the house of}} Israel and Judah.

```And if this had been too little, I would have added much more than that to you!

\v{9}```Why did you despise what the \divine{Lord} has promised by doing what is detestable in his sight?

```You struck down Uriah the Hittite with a battle sword.

```You took his wife to be your own.\fnote{\fbackref{12:9} Lit. \fbib{wife}}

```You killed him with the sword of the Ammonite army.

\v{10}```Therefore the sword will never leave your household, because you have despised me by taking the wife of Uriah the Hittite to be your own.'\fnote{\fbackref{12:10} Lit. \fbib{wife}}

\v{11}``This is what the \divine{Lord} says:

```Listen very carefully!

```I'm raising up evil against you right out of your own household.

```I'm going to take your wives away from you right before your eyes.

```Then I'll give them to your neighbor.

```And then he's going to have sex with your wives in broad daylight!

\v{12}```What you did in secret I'm going to do right in front of all Israel and in broad daylight as well!'\,''

\v{13}At this point, David told Nathan, ``I have sinned against the \divine{Lord}.''

Nathan responded to David, ``There's one other thing: the \divine{Lord} has forgiven your sin.\fnote{\fbackref{12:13} Or \fbib{has caused your sin to go away}; lit. \fbib{has caused your sin to cross over eastward}} You won't die. \v{14}Nevertheless, because you have despised the \divine{Lord}'s enemies with utter contempt,\fnote{\fbackref{12:14} Or \fbib{because you have given occasion for the \divine{Lord}'s enemies to show contempt}} the son born to you will most certainly die.'' \v{15}Then Nathan went home.
\passage{David's Infant Son Dies}

After this, the \divine{Lord} afflicted the child that Uriah's wife had born to David, and the child\fnote{\fbackref{12:15} Lit. \fbib{and he}} became very ill. \v{16}David begged God on behalf of the youngster. He\fnote{\fbackref{12:16} Lit. \fbib{David}} fasted, went inside, and spent the night lying on the ground. \v{17}His closest advisors at the palace\fnote{\fbackref{12:17} Lit. \fbib{The elders of the house}} got up, remained with him, and tried to help him get up from the ground, but he would not do so. He also wouldn't eat with them.

\v{18}A week later, the child died, and David's staff was afraid to tell him that the child had died. They were telling themselves, ``Look, when the child was still alive, we talked to him but he wouldn't listen to what we said. Now what kind of trouble will he bring on himself if we tell him that the child has died?''

\v{19}But as David observed his staff whispering together, he perceived that the child had died, so he asked his staff, ``Is the child dead?''

They replied, ``He has died.''

\v{20}At this, David got up from the ground, washed, anointed himself, changed his clothes, and went into the \divine{Lord}'s tent\fnote{\fbackref{12:20} Lit. \fbib{house}} to worship. Then he went back to his palace where, at his request, they served him food and he ate.

\v{21}His staff asked him, ``What's this about? When the child was alive, you fasted and cried. Now that the child has died, you get up and eat!''

\v{22}He answered, ``When the child was alive, I fasted and cried. I asked myself, `Who knows? Maybe the \divine{Lord} will show grace to me and the child will live.' \v{23}But now that he has died, what's the point of fasting? Can I bring him back again? I'll be going to be with him, but he won't be returning to me.''
\passage{The Birth of Solomon}

\v{24}Then David consoled his wife Bathsheba. He went in and had sex with her, and she bore a son whom he named Solomon. The \divine{Lord} loved him, \v{25}and sent a message written by Nathan the prophet to call his name Jedidiah,\fnote{\fbackref{12:25} The Heb. name \fbib{Jedidiah} means \fbib{loved by the \divine{Lord}}} for the Lord's sake.
\passage{The Ammonites are Defeated}

\v{26}Meanwhile, Joab attacked the Ammonite city of\fnote{\fbackref{12:26} The Heb. lacks \fbib{city of}} Rabbah and captured its stronghold. \v{27}Then Joab sent messengers to David to tell him, ``I just attacked Rabbah and captured its municipal water supply, \v{28}so call out the rest of the army, attack the city, and capture it. Otherwise, I'll take the city myself and name it after me.'' \v{29}So David mustered his entire army and marched on Rabbah, attacked it, and captured it. \v{30}He confiscated the crown of their king\fnote{\fbackref{12:30} Lit. \fbib{of Malcam}; LXX reads \fbib{king Molchol}; cf. 1King 11:5, 33; Zeph 1:5} from his head---it weighed one talent\fnote{\fbackref{12:30} I.e. about 75 pounds} in gold and was set with precious stones---and it was placed on David's head. He confiscated a great amount of war booty that had been plundered from the city, \v{31}brought back the people who had lived in it, placing them under conscripted labor with saws, iron picks, and axes. He did this to every Ammonite city, and then David and his entire army\fnote{\fbackref{12:31} Lit. \fbib{people}} returned to Jerusalem.
\labelchapt{13}
\passage{Amnon's Rape of Tamar}

\chapt{13}
\v{1}Sometime after this, David's son Amnon fell in love with David's other\fnote{\fbackref{13:1} The Heb. lacks \fbib{other}} son Absalom's beautiful sister Tamar. \v{2}Amnon became so emotionally distressed that he fell sick over his half-sister Tamar. She was a virgin, and Amnon found it difficult to do anything to her.

\v{3}Meanwhile, Amnon had a friend named Jonadab, who was the son of David's brother Shimeah. Now Jonadab was a very shrewd man. \v{4}``Why are you so depressed these past few mornings,''\fnote{\fbackref{13:4} Lit. \fbib{depressed morning by morning}} Jonadab\fnote{\fbackref{13:4} Lit. \fbib{he}} asked Amnon, ``since you're a son of the king? Why not tell me?''

Amnon replied, ``I'm in love with my brother Absalom's sister Tamar.''

\v{5}Jonadab advised him, ``Lie down and fake being sick. When your father visits you, ask him, `Please let my sister Tamar come and give me something to eat that she prepares especially for me,\fnote{\fbackref{13:5} Lit. \fbib{prepares in my sight}} and after she makes dinner for me, let her feed it to me personally.'\,''\fnote{\fbackref{13:5} Lit. \fbib{it from her hand}}

\v{6}So Amnon lay down and faked being sick. When the king came to visit him, Amnon asked the king, ``Please let my sister Tamar come and make some of her bread especially for me,\fnote{\fbackref{13:6} Lit. \fbib{bread in my sight}} so she can feed it to me personally.''\fnote{\fbackref{13:6} Lit. \fbib{it from her hand}}

\v{7}So David sent for Tamar back at the palace, telling her, ``Please go to your brother Amnon's home and prepare some food for him.'' \v{8}Tamar went to her brother Amnon's home, where he was lying down. She brought along some dough, kneaded it, prepared some cakes especially for him,\fnote{\fbackref{13:8} Lit. \fbib{some bread in his sight}} baked them, \v{9}and emptied the baking skillet just for him, but he refused to eat.

``Send everybody out of here,'' Amnon said. So everyone left the room. \v{10}Amnon told Tamar, ``Bring the food into my private bedroom, so I can eat it with you personally.''\fnote{\fbackref{13:10} Lit. \fbib{it from your hand}} So Tamar took the cakes she had prepared and brought them into the private bedroom for her brother Amnon.

\v{11}But as soon as she brought them near him to eat, he overpowered her and told her, ``Come here and have sex with me, my sister!''

\v{12}``No, my brother!'' she kept telling him. ``Don't humiliate me like this! This just isn't done in Israel! Don't do this utterly foolish thing! \v{13}And what about me? Where will I go to escape\fnote{\fbackref{13:13} Or \fbib{carry}} this disgrace? And as for you, you'll be known as one of Israel's greatest fools! So please talk to the king, because he won't withhold me from you!''

\v{14}But he was unwilling to listen to what she was saying. Since he was stronger than she was, he forced her into having sex with him. \v{15}Afterwards, though, Amnon hated her very intensely. As a result, his hatred for her exceeded the love that he had previously for her. So Amnon told her, ``Get up! Leave!''

\v{16}Even so, she tried to tell him, ``No! After all, it's more wrong to send me away than what you just did to me!''

But he was unwilling to listen to her. \v{17}So he called out to a young man who was serving him, and told him: ``Send this woman away from me and lock the door after her.''

\v{18}Now she was clothed in a long sleeved, multi-colored ornamental tunic, commonly worn by the king's virgin daughters. When Amnon's\fnote{\fbackref{13:18} Lit. \fbib{his}} servant threw her out and locked the door after her, \v{19}Tamar rubbed her head with ashes, tore her tunic that she was wearing, put her hand to her head, and ran off, crying aloud as she went away.
\passage{Absalom's Plans Revenge}

\v{20}Later, her brother Absalom asked her, ``Has Amnon, that brother of yours, raped\fnote{\fbackref{13:20} Or \fbib{yours, been with}} you? Then keep quiet about your half-brother for now, my sister. Stop taking this so personally.''\fnote{\fbackref{13:20} Lit. \fbib{this matter to your heart}} From that time on, Tamar lived in continuous desolation within her brother Absalom's house. \v{21}When King David heard all about these developments, he flew into a rage over it. \v{22}But Absalom never said a word, either good or bad, to Amnon. Nevertheless, he hated Amnon because he had humiliated his sister Tamar.
\passage{Absalom's Men Kill Amnon}

\v{23}Two full years later, Absalom took some men to Baal-hazor near Ephraim to shear his sheep. He\fnote{\fbackref{13:23} Lit. \fbib{Absalom}} also invited all of the king's sons to come. \v{24}Absalom had gone to the king to ask him, ``I've brought some men to shear the sheep. Won't you please come and join me, along with your senior staff?''

\v{25}But King David declined,\fnote{\fbackref{13:25} Lit. \fbib{David replied}} saying to Absalom, ``No, my son, we won't all go, since that would be too much trouble for you.'' Although Absalom begged David, he would not go, even though he did give his blessing.

\v{26}So Absalom responded, ``If you aren't coming, please allow my brother Amnon to accompany us.''

The king asked, ``Why should he go with you?''

\v{27}But Absalom kept begging David\fnote{\fbackref{13:27} Lit. \fbib{him}} until he sent Amnon and all of David's\fnote{\fbackref{13:27} Lit. \fbib{all the king's}} sons to accompany Absalom.

\v{28}Then Absalom instructed his young men, ``Please keep watching Amnon until he's drunk. Then I'll tell you, `Attack Amnon!' As soon as I do, kill him and don't be afraid! You have your orders, so be strong and brave!'' \v{29}So Absalom's young men did to Amnon just as they had been\fnote{\fbackref{13:29} Lit. \fbib{as Absalom had}} ordered, but the rest of David's sons jumped up, mounted their mules, and escaped.

\v{30}While they were still on the road, this rumor came to David: ``Absalom has struck down all of the king's sons and none of them has survived.'' \v{31}David arose, ripped his clothes in anguish,\fnote{\fbackref{13:31} The Heb. lacks \fbib{in anguish}} and collapsed to the ground while all of his staff stood by with their own clothes torn.

\v{32}But David's brother Shimeah's son Jonadab reported, ``Your majesty, don't assume they've killed all of the young men---the king's sons---only Amnon has died, since that was Absalom's intention from the day Amnon raped\fnote{\fbackref{13:32} Lit. \fbib{humiliated}} his sister Tamar. \v{33}Now your majesty, don't be concerned about this rumor that all the king's sons have died, because only Amnon is dead.''

\v{34}Meanwhile, Absalom had run away. While the young man standing watch was looking around, all of a sudden he observed many people coming down the road behind and to the west of the mountain! So the watchman left his post and reported, ``I have seen men coming from the direction of Horonaim.''\fnote{\fbackref{13:34} So LXX; the Heb. lacks \fbib{So the {\ldots} of Horonaim}}

\v{35}Jonadab told the king, ``Look! Here come the king's sons. This thing has turned out just like your servant reported.'' \v{36}Just as he finished his comments, the king's sons arrived, crying loudly. At this, with tears overflowing, the king and his entire staff wept bitterly.

\v{37}Absalom continued to flee, eventually going to Ammihud's son King Talmai of Geshur, while King David continued to mourn for his son every day. \v{38}After fleeing to Geshur, Absalom remained there for three years. \v{39}Meanwhile, King David longed to visit Absalom, since he was moved to compassion over Amnon's death.
\labelchapt{14}
\passage{Joab's Plan Regarding Absalom}

\chapt{14}
\v{1}Meanwhile, Zeruiah's son Joab knew that the king's attention was focused on Absalom,\fnote{\fbackref{14:1} Lit. \fbib{king's heart was toward}} \v{2}so he\fnote{\fbackref{14:2} Lit. \fbib{Joab}} sent messengers\fnote{\fbackref{14:2} The Heb. lacks \fbib{messengers}} to Tekoa to bring a wise woman from there. He told her, ``Please play the role of a mourner, wear the clothes of a mourner, and refrain from using makeup.\fnote{\fbackref{14:2} Lit. \fbib{using anointing oil}} Act like a woman who's been in mourning for the dead for many days. \v{3}Then go to the king and speak to him like this{\ldots}'' Then Joab told her what to say.

\v{4}When the woman from Tekoa spoke to the king, she fell on her face to the ground, prostrating herself to address him, ``Help, your majesty!''

\v{5}The king asked her, ``What's your problem?''\fnote{\fbackref{14:5} The Heb. lacks \fbib{problem}}

``I've been a widowed woman\fnote{\fbackref{14:5} I.e. a widow of meager resources, low social status, and limited circumstances, therefore eligible to receive special assistance from Israel's society.} ever since my husband died,'' she answered. \v{6}``Your humble servant used to have two sons, but they got into a fight out in the field. Because there was no one to keep them apart, one of them attacked the other and killed him. \v{7}Now please pay attention closely! My\fnote{\fbackref{14:7} Lit. \fbib{The}} whole family is attacking your humble servant! They're saying, `Turn over the one who attacked his brother and we'll put him to death in retribution for his brother, whose life he took. That way, we'll kill the heir also!' They're going to extinguish the only light\fnote{\fbackref{14:7} Lit. \fbib{the coal that is}; i.e. the only remaining heir} left in my family, leaving my late husband neither an ongoing name nor a survivor on the face of the earth!''

\v{8}Then the king replied to the woman, ``Go home and I'll issue a special order just for you.''

\v{9}But the woman from Tekoa told the king, ``Your majesty, let any guilt for this be on me and on my ancestors' household, and not on my king or his throne!''

\v{10}The king replied, ``Bring anyone who talks to you about this to me, and he certainly won't be bothering\fnote{\fbackref{14:10} Lit. \fbib{touching}} you anymore!''

\v{11}Then she said, ``Your majesty, please remember the \divine{Lord} your God, so that blood avengers don't do any more damage! Otherwise, they'll destroy my son!''

So he promised, ``As the \divine{Lord} lives, not even a single hair from your son's head\fnote{\fbackref{14:11} The Heb. lacks \fbib{head}} will fall to the ground!''

\v{12}At this, the woman responded, ``Would your majesty the king please allow your humble servant to say one more thing?''

``Say it{\ldots}''\fnote{\fbackref{14:12} The Heb. lacks \fbib{it}} he replied.

\v{13}``Why, then,'' the woman asked, ``are you planning to act just like this against God's people? Based on what your majesty has said, you're acting like one who is guilty himself, because you're not bringing back the one whom you've banished! \v{14}After all, even though we all die,\fnote{\fbackref{14:14} Lit. \fbib{though to death we all die}} and we're\fnote{\fbackref{14:14} The Heb. lacks \fbib{we're}} all like water being spilled on the ground that cannot be recovered, nevertheless God doesn't take away life, but carries out his plans so as not to cast away permanently from him those who are presently estranged.\fnote{\fbackref{14:14} MT verb for \fbib{cast away permanently} is an intensive form of the verb \fbib{estranged}}

\v{15}``Now as to why I've come to speak with your majesty the king, it's because the people have made me afraid, so your humble servant told herself,\fnote{\fbackref{14:15} The Heb. lacks \fbib{to herself}} `I'll go speak to the king, so perhaps the king will do what his humble servant has requested. \v{16}Perhaps the king will listen and deliver his humble servant from the oppression\fnote{\fbackref{14:16} Lit. \fbib{palm}} of the man who intends to eliminate both me and my son from what God has apportioned to us!'\fnote{\fbackref{14:16} The Heb. lacks \fbib{to us}}

\v{17}``So your humble servant is saying, `Please, your majesty, let what the king has to say be of comfort, because just as the angel of God is, so also is your majesty the king to discern both good and evil. And may the \divine{Lord} your God remain present with you.'\,''

\v{18}In reply, the king asked the woman, ``Please don't conceal anything about which I'm going to be asking you now.''

So the woman replied, ``Please, your majesty, let the king speak.''

\v{19}Then the king asked, ``Is Joab behind all of this with you?''\fnote{\fbackref{14:19} Lit. \fbib{Is the hand of Joab with you in}}

``As your soul lives, your majesty, the king,'' the woman answered, ``no one can divert anything left or right from what your majesty the king has spoken! As a matter of fact, it was your servant Joab! He was there, giving me precise orders about everything that your humble servant was to say. Your servant Joab did this, \v{20}intending to change the outcome of this matter. Nevertheless, your majesty is wise, like the wisdom of the angel of God, to be aware of everything that's going on throughout the earth.''\fnote{\fbackref{14:20} Or \fbib{land}; or \fbib{going on in the land}}
\passage{David Authorizes Absalom's Return}

\v{21}Then the king addressed Joab, ``Look! I'll do this thing that you've requested.\fnote{\fbackref{14:21} The Heb. lacks \fbib{that you've requested}} Go bring back the young man Absalom.''

\v{22}At this, Joab fell on his face to the ground, prostrating himself to bless the king, and then\fnote{\fbackref{14:22} Lit. \fbib{Joab}} said, ``Today your servant realizes that he's found favor with you, your majesty, in that the king has acted on the request of his servant.'' \v{23}Then Joab got up, went to Geshur, and brought Absalom back to Jerusalem.

\v{24}Nevertheless, the king said, ``Let him return to his own home and not show his face to me.'' So Absalom returned to his own home and did not show his face to the king.
\passage{David's Son Absalom}

\v{25}Now throughout all of Israel no one was as handsome as Absalom or so highly praised, from the sole of his foot to the crown of his head there wasn't a single thing wrong about him. \v{26}Whenever he cut his hair ---he cut it at the end of every year, because it grew thick on his head,\fnote{\fbackref{14:26} Lit. \fbib{grew heavy on him}} which is why he cut it---his hair weighed in at 200 shekels\fnote{\fbackref{14:26} I.e. about five pounds at 0.4 shekels per ounce} measured by the royal standard.\fnote{\fbackref{14:26} Lit. \fbib{the king's weight}} \v{27}Absalom fathered three sons and one daughter, whom he named Tamar. She was a beautiful woman, both in form and appearance.

\v{28}Meanwhile, Absalom lived in Jerusalem for two years, but never saw the king's face. \v{29}After this, Absalom sent for Joab, intending to send him to the king, but Joab\fnote{\fbackref{14:29} Lit. \fbib{he}} would not come. Absalom\fnote{\fbackref{14:29} Lit. \fbib{he}} sent for him a second time, but he still\fnote{\fbackref{14:29} The Heb. lacks \fbib{still}} would not come. \v{30}So Absalom\fnote{\fbackref{14:30} Lit. \fbib{he}} told his servants, ``Observe that Joab's grain field lies next to mine. He has barley planted there. Go set it on fire.'' So Absalom's servants set the field on fire.

\v{31}At this, Joab got up, went to Absalom's home, and demanded of him, ``Why did your servants set fire to my grain field?''

\v{32}In answer to Joab, Absalom replied, ``Look, I sent for you, telling you `Come here so I can send you to the king to ask him ``What's the point in moving here from Geshur? I would have been better off to have remained there!''\,' So let me see the king's face, and if I'm guilty of anything, let him execute me!''

\v{33}So when Joab approached the king and told him what Absalom had said,\fnote{\fbackref{14:33} The Heb. lacks \fbib{what Absalom had said}} he summoned Absalom, who then came to the king and fell to the ground on his face in front of him.\fnote{\fbackref{14:33} Lit. \fbib{of the king}} Then the king kissed Absalom.
\labelchapt{15}
\passage{Absalom Instigates Civil War}

\chapt{15}
\v{1}Sometime later, Absalom acquired a chariot equipped with horses and recruited\fnote{\fbackref{15:1} The Heb. lacks \fbib{recruited}} 50 men to accompany\fnote{\fbackref{15:1} Lit. \fbib{to run before}} him.\fnote{\fbackref{15:1} Cf. 1Sam 8:11} \v{2}Then he\fnote{\fbackref{15:2} Lit. \fbib{Absalom}} would get up early, stand near the passageway to the palace\fnote{\fbackref{15:2} The Heb. lacks \fbib{palace}} gate, and when anyone arrived to file a legal complaint for a hearing before the king, Absalom would call to him and ask, ``You're from what city?'' If\fnote{\fbackref{15:2} The Heb. lacks \fbib{If}} he replied, ``Your servant is from one of Israel's tribes,'' \v{3}Absalom would respond, ``Look, your claims are valid and defensible, but nobody will listen to you on behalf of the king. \v{4}Who will appoint me to be a judge in the land? When anyone arrived to file a legal complaint or other cause, he could approach me for justice and I would settle it!'' \v{5}Furthermore, if a man approached him to bow down in front of him, he would put out his hand, grab him, and embrace him. \v{6}By doing all of this to anyone who came to the king for a hearing, Absalom stole the loyalty\fnote{\fbackref{15:6} Lit. \fbib{hearts}} of the men of Israel.

\v{7}And so it was that forty\fnote{\fbackref{15:7} So MT and LXX; Syr Peshitta and Lucian recension of LXX read \fbib{four}} years after Israel had demanded a king,\fnote{\fbackref{15:7} The Heb. lacks \fbib{after Israel had demanded a king}; i.e. about ten years before David began his reign. Or \fbib{forty years after David's anointing at Bethlehem}; cf. 1Sam 16:13} Absalom asked the king, ``Please let me go to Hebron so I can pay my vow that I made to the \divine{Lord}, \v{8}because when I was living at Geshur in Aram, your servant made this solemn promise:\fnote{\fbackref{15:8} Lit. \fbib{servant vowed a vow}} `If the \divine{Lord} ever brings me back to Jerusalem, then I will serve the \divine{Lord}.'\,''

\v{9}The king replied to him, ``Go in peace!'' So Absalom\fnote{\fbackref{15:9} Lit. \fbib{he}} got up and left for Hebron.

\v{10}But Absalom sent agents throughout all of the tribes of Israel, telling them, ``When you hear the sound of the battle trumpet, you're to announce that Absalom is king in Hebron.'' \v{11}Meanwhile, 200 men left Jerusalem with Absalom. They had been invited to go along, but were innocent, not knowing anything about what was happening.\fnote{\fbackref{15:11} Lit. \fbib{about the matter}} \v{12}Absalom also sent for Ahithophel\fnote{\fbackref{15:12} Ahithophel was Bathsheba's grandfather; cf. 2Sam 11:3; 23:34} the Gilonite, David's counselor, to come\fnote{\fbackref{15:12} The Heb. lacks \fbib{to come}} from his home town of Giloh while Absalom\fnote{\fbackref{15:12} Lit. \fbib{he}} was presenting the sacrificial offerings. And so the conspiracy widened, because the common people increasingly sided with Absalom.
\passage{David Flees from Jerusalem}

\v{13}Then a messenger arrived to inform David, ``The loyalties of the men\fnote{\fbackref{15:13} Lit. \fbib{heart of the man}} of Israel have shifted to\fnote{\fbackref{15:13} Lit. \fbib{have followed after}} Absalom.''

\v{14}So David told all of his staff who were with him in Jerusalem, ``Let's get up and get out of here! Otherwise, none of us will escape from Absalom. Hurry, or he'll overtake us quickly, bring disaster on all of us, and execute the inhabitants of the city!''

\v{15}``Look!'' the king's staff replied. ``Your servants will do whatever the king chooses.'' \v{16}So the king left, taking his entire household with him except for ten mistresses,\fnote{\fbackref{15:16} Or \fbib{concubines}; i.e. secondary wives} who were to keep the palace in order. \v{17}The king left, along with all of his people with him, and they paused at the last house. \v{18}All of his staff were going on ahead of\fnote{\fbackref{15:18} Lit. \fbib{on beside}} him---that is, all of the special forces\fnote{\fbackref{15:18} Lit. \fbib{Cherethites}; i.e. elite body guards} and mercenaries,\fnote{\fbackref{15:18} Lit. \fbib{Pelethites}; i.e. special couriers} all of the Gittites, and 600 men who had come to serve\fnote{\fbackref{15:18} Lit. \fbib{come at his feet}} him from Gath, went on ahead of the king.

\v{19}Then the king suggested to Ittai the Gittite, ``Why should you have to go with us? Return and stay with the new\fnote{\fbackref{15:19} The Heb. lacks \fbib{new}} king, since you're a foreigner and exile. Stay where you want to stay.\fnote{\fbackref{15:19} Lit. \fbib{Stay in your own place}} \v{20}It seems only yesterday that you arrived, so should I make you wander around with us while I go wherever I can? Go back, and take your brothers with you. May gracious love and truth accompany you!''

\v{21}``As the \divine{Lord} lives,'' Ittai answered in reply, ``and as your majesty the king lives, wherever your majesty my king may be---whether living or dying---that's where your servant will be!''

\v{22}So David replied, ``Come along, then!'' So Ittai the Gittite went along also, accompanied by all of his men and all of his little ones. \v{23}With all of the people in\fnote{\fbackref{15:23} The Heb. lacks \fbib{of the people in}} the territory crying loudly, everybody passed over the Kidron brook, along with the king. Then everyone headed out toward the road that leads to the wilderness.

\v{24}Meanwhile, Zadok showed up also, along with all of the descendants of Levi with him, carrying the Ark of the Covenant of God. They set down the Ark of God and Abiathar approached while all the people finished abandoning the city. \v{25}The king told Zadok, ``Take the Ark of God back to the city. If I'm shown favor in the \divine{Lord}'s sight, then he'll bring me back again and show me both it and the place where it rests.\fnote{\fbackref{15:25} Lit. \fbib{and his habitation}} \v{26}But if he should say something like `I'm not pleased with you,' well then, here I am---let him do to me whatever seems right to him.''

\v{27}The king also asked Zadok the priest, ``Aren't you a seer, too? Go back to the city in comfort, along with your son Ahimaaz and Abiathar's son Jonathan. \v{28}Look! I'll camp at the wilderness fords until you send word to inform me.''

\v{29}So Zadok and Abiathar returned the Ark of God to Jerusalem and remained there. \v{30}David then left, going up the Mount of Olives,\fnote{\fbackref{15:30} Lit. \fbib{the Olivet}} crying as he went, with his head covered and his feet bare. All of the people who were with him covered their own heads and climbed up the Mount of Olives,\fnote{\fbackref{15:30} The Heb. lacks \fbib{the Mount of Olives}} crying as they went along.

\v{31}Just then, someone told David, ``Ahithophel is one of Absalom's conspirators!''

So David prayed, ``\divine{Lord}, please turn Ahithophel's counsel into foolishness.''
\passage{Hushai Serves as a Spy}

\v{32}Just as David was coming to the top of the Mount of Olives where God was being worshipped, there was Hushai the Archite to meet him, with his coat ripped and dust all over his head! \v{33}David greeted him, ``If you come along with me, you'll be a burden to me. \v{34}So go back to the city and tell Absalom, `I'll be your servant, your majesty! Just as I served your father in the past, I can be your servant now.' That way you can manipulate Ahithophel's advice to my benefit. \v{35}Won't Zadok and Abiathar the priests be there with you? So whatever you hear from the king's palace, you're to report to Zadok and Abiathar the priests. \v{36}Their two sons---Zadok's son Ahimaaz and Abiathar's son Jonathan---are with them there. You'll be sending me everything that you hear through them.'' \v{37}So David's friend Hushai went back to the city just as Absalom was arriving in Jerusalem.
\labelchapt{16}
\passage{David's Experience with His Adversaries}

\chapt{16}
\v{1}Now just as David happened to have passed the summit of the Mount of Olives,\fnote{\fbackref{16:1} The Heb. lacks \fbib{of the Mount of Olives}} suddenly Mephibosheth's servant Ziba met him, accompanied by a couple of saddled donkeys loaded with 200 loaves of bread, 100 clusters of raisins, 100 pieces of summer fruit, and a skin of wine! \v{2}The king asked Ziba, ``What are those for?''

Ziba replied, ``The donkeys are for the king's household to ride, the bread and summer fruit are for your young men to eat, and the wine is for whoever wants to drink if they get weary in the wilderness.''

\v{3}Then the king asked, ``Where is your master's son?''

``He's staying in Jerusalem!'' Ziba answered the king. ``He's saying `The nation\fnote{\fbackref{16:3} Lit. \fbib{house}} of Israel will restore my father's kingdom to me today!'\,''

\v{4}So the king told Ziba, ``Pay attention! Everything that belongs to Mephibosheth is now yours!''

In response Ziba said, ``I'm submitting to you. Let me find favor in your sight, your majesty the king!''
\passage{Shimei Curses David}

\v{5}Later on, as King David approached Bahurim, Gera's son Shimei, who was related to the family of Saul's household, went out to meet David,\fnote{\fbackref{16:5} The Heb. lacks \fbib{to meet David}} cursing continually as he approached. \v{6}He threw rocks at David and all of David's staff who were accompanying him, while all the rest of the entourage, including all of David's security detail, were close by him.\fnote{\fbackref{16:6} Lit. \fbib{were at his right and left hands}} \v{7}``Get out of here!\fnote{\fbackref{16:7} The Heb. lacks \fbib{of here}} Get out!'' Shimei yelled as he cursed. ``You murderer! You who think you're above the law!\fnote{\fbackref{16:7} So LXX. MT reads \fbib{You man of Belial!}} \v{8}The \divine{Lord} has repaid you personally for murdering the entire dynasty of Saul, whose place you've taken to reign! And the \divine{Lord} has given the kingdom into your son Absalom's control. Now look! Your own evil has caught up with you, because you're guilty of murder!''

\v{9}At this point, Zeruiah's son Abishai asked the king, ``Why should this dead dog be cursing your majesty the king? May I have permission to go over and cut off his head?''

\v{10}But the king responded, ``What do I have in common with you sons of Zeruiah? If he continues to curse---and if the \divine{Lord} has told him, `Curse David!'---then who are you to be demanding to know\fnote{\fbackref{16:10} Lit. \fbib{be saying}} `Why have you done this?'\,''

\v{11}So David ordered Abishai and all of his staff: ``Look! My own son wants to kill me! How much more now is this descendant of Benjamin? Leave him alone and let him go on cursing, because the \divine{Lord} has ordered him to do this.\fnote{\fbackref{16:11} The Heb. lacks \fbib{to do this}} \v{12}Perhaps the \divine{Lord} will take note of my troubles and return good to me instead of curses today!''

\v{13}So David and his entourage went on their way, and Shimei walked along the hillside with him, cursing, throwing rocks, and tossing dirt at David\fnote{\fbackref{16:13} Lit. \fbib{him}} as they went along. \v{14}Eventually, the king and his entourage arrived exhausted at their destination, and David\fnote{\fbackref{16:14} Lit. \fbib{he}} refreshed himself there.
\passage{Absalom Captures Jerusalem}

\v{15}Right about then, Absalom and his entourage from the people of Israel entered Jerusalem, accompanied by Ahithophel. \v{16}When David's friend Hushai the Archite approached Absalom, Hushai greeted Absalom, ``Long live the king! Long live the king!''

\v{17}But Absalom asked Hushai, ``So this is how you demonstrate your loyalty\fnote{\fbackref{16:17} Lit. \fbib{gracious love}} to your closest friends? Why didn't you leave with your friend?''

\v{18}Hushai replied, ``No! On the contrary, whomever the \divine{Lord}, this group, and all the men of Israel choose is where I'll be, and I'll remain with him! \v{19}Besides, who else should I be serving? Why not the son? The same way I served your father, I'll serve you.''\fnote{\fbackref{16:19} Lit \fbib{served in your father's presence, I'll serve in your presence}}
\passage{Ahithophel Counsels Absalom}

\v{20}So Absalom asked Ahithophel, ``What's your advice? What should we do?''

\v{21}Ahithophel responded, ``Go inside and have sex with your father's mistresses\fnote{\fbackref{16:21} Or \fbib{concubines}; i.e. secondary wives}, whom he left to keep the palace in order. Then everyone in Israel will hear how your father has come to hate you and everyone who has joined you will be emboldened to act.'' \v{22}So they erected a tent for Absalom on the palace roof and Absalom went in and had sex with his father's mistresses right in front of all Israel.
\labelchapt{17}
\passage{Ahithophel Tries to Crush David's Supporters}

\v{23}Now Ahithophel's advice that he provided at that time was being compared to one who inquired of God, so highly regarded was Ahithophel's counsel by both David and Absalom.\chapt{17}
\v{1}``Give me 12,000 men! I'll leave\fnote{\fbackref{17:1} Lit. \fbib{get up}} tonight and pursue David,'' Ahithophel advised Absalom. \v{2}``I'll catch him while he is still tired and weak.\fnote{\fbackref{17:2} Lit. \fbib{and weak-handed}} I'll frighten him so all his people with him desert him. But I'll only kill the king. \v{3}Then I'll bring everybody else back to you. When the man you're looking for is dead, all the rest of the people will return quietly.''

\v{4}Even though this plan seemed like a good idea to Absalom and to all of the elders of Israel, \v{5}Absalom replied, ``Call in Hushai the Archite so I can hear what he has to say, too!'' \v{6}When Hushai approached Absalom, Absalom asked him, ``Here's what Ahithophel had to advise. Should we do what he says? Or if not, say so!''
\passage{Hushai Counters Ahithophel's Advice}

\v{7}``Ahithophel's advice is not best at this time,'' Hushai suggested to Absalom. \v{8}``You know how strong your father and his men are. They're as mad as a bear robbed of her cubs! Furthermore, your father is a skilled warrior. He won't stay with his army at night. \v{9}Look! He's probably already hiding in a cave or someplace like that. If the first attack fails, people will hear about it and think, `Absalom's army is losing!' \v{10}Then even men who would otherwise be as brave as lions will be scared, because every Israeli knows your father is a mighty man, and they know his men are valiant! \v{11}So here's my advice: Muster everybody from one end of the country to the other!\fnote{\fbackref{17:11} Lit. \fbib{from Dan to Beer-sheba}; i.e. Hushai was stalling for time, since Dan was the northernmost Israeli city and Beer-sheba its southernmost.} You'll have an army in number like the sand on the seashore! Then you'll go into battle! \v{12}We'll go find David wherever he's hiding. We'll fall on him like dew on the ground! We'll kill him and all of his men, and we won't leave even one man alive! \v{13}If he escapes into a city, we'll bring ropes to that city and tear it down! We won't leave a single stone left in the valley!''

\v{14}Absalom and all of the Israelis replied, ``The advice of Hushai the Archite is better than Ahithophel's!''
\passage{Hushai Warns David}

But the \divine{Lord} had planned to circumvent the sound advice of Ahithophel so the \divine{Lord} could bring Absalom to destruction. \v{15}So Hushai told Zadok and Abiathar, the priests, what Ahithophel had suggested to Absalom and the elders of Israel. He also reported what he himself had proposed. Hushai said, \v{16}``Quick! Get word to David! Tell him not to spend the night at the crossings that lead to the desert. Instead, he must cross the Jordan River immediately. That way, if he crosses the river, the king and his entourage\fnote{\fbackref{17:16} Lit. \fbib{people}; and so throughout the chapter} will survive.''

\v{17}Meanwhile, since they could not risk being seen entering the city, Jonathan and Ahimaaz had been waiting at En-rogel, where a young servant woman was to go to inform them and they would then go brief King David. \v{18}But a young man observed Jonathan and Ahimaaz and informed Absalom, so they left in a hurry, arrived at the home of a man who lived at Bahurim, and hid inside a well that was in his courtyard. \v{19}The man's wife grabbed a sheet, covered the mouth of the well with it, and spread some dried grain over it. As a result, nobody could tell it was a hiding place.\fnote{\fbackref{17:19} Lit. \fbib{And nothing was known}}

\v{20}When Absalom's servants approached the woman of the house, they asked her, ``Where are Ahimaaz and Jonathan?''

``They've already crossed the brook,'' the woman answered. So Absalom's servants went away in search of Jonathan and Ahimaaz, but they couldn't find them, so they returned to Jerusalem.

\v{21}A little while later, the men crawled up out of the well and went off to talk to King David. They told David, ``Get up! Cross the water quickly, because this is what Ahithophel advised about you{\ldots}'' \v{22}So David got up and all of his entourage crossed the Jordan River.\fnote{\fbackref{17:22} The Heb. lacks \fbib{River}; and so throughout the chapter} Everyone had crossed the Jordan River by dawn's first light.
\passage{Ahithophel's Suicide}

\v{23}Meanwhile, when Ahithophel observed that his counsel was not being acted upon, he saddled his donkey, got up, and left for his hometown. Leaving behind a set of orders for his household, he hanged\fnote{\fbackref{17:23} Lit. \fbib{strangled}} himself. After his death he was buried in his father's tomb.
\passage{David Receives Supplies in the Wilderness}

\v{24}Later, David arrived at Mahanaim. Absalom and all of the Israelis who supported him crossed the Jordan River. \v{25}Absalom had installed Amasa in place of Joab over the army. (Amasa was the son of a man named Jether the Ishmaelite. His mother was Abigail, a daughter of Nahash and a sister of Zeruiah, Joab's mother.) \v{26}Absalom and the Israelis with him\fnote{\fbackref{17:26} The Heb. lacks \fbib{with him}} camped in the territory of Gilead. \v{27}When David arrived at Mahanaim, Shobi (Nahash's son from the Ammonite town of Rabbah), Makir (Ammiel's son from Lo-debar), and Barzillai (from Rogelim in Gilead) were already there. \v{28}They brought along bedding, bowls, clay basins, wheat, barley, flour, roasted grains, beans, peas, \v{29}honey, cheeses,\fnote{\fbackref{17:29} Or \fbib{milk curds}} sheep, and cheese made from cow's milk for David and his entourage because they had been reasoning, ``The people are hungry, tired, and thirsty there in the wilderness.''
\labelchapt{18}
\passage{The Battle Begins}

\chapt{18}
\v{1}David mustered his forces and appointed officers in charge of regiments and companies.\fnote{\fbackref{18:1} Lit. \fbib{of thousands and hundreds}} \v{2}Dividing his forces into three groups, he set Joab as commander of one third of his army, Zeruiah's son Abishai, Joab's brother, as commander of another third, and Ittai from Gath as commander of another third. The king informed the army, ``I'm going out to battle\fnote{\fbackref{18:2} The Heb. lacks \fbib{to battle}} with you, too.''

\v{3}``No way!'' his army responded. ``If we have to retreat from the battle, Absalom's men won't care about us. Even if half of us die, they won't care about us. But you are worth 10,000 of us. The best thing you can do for us is to remain in the city.''

\v{4}So David responded, ``I'll do what you think best.'' Then he stood alongside the city gate as the army went out in battle array by hundreds and thousands. \v{5}As they were going out, the king ordered Joab, Abishai, and Ittai, ``Treat young Absalom gently for my sake.'' Everyone heard what the king had ordered his commanders about Absalom.

\v{6}David's army left for the battlefield to fight Absalom and his Israeli followers, and they also fought in the Ephraim forest, \v{7}where David's army of servants defeated the Israelis. Many died that day---20,000 men. \v{8}The battle spread throughout the entire countryside, and the forest claimed more casualties that day than did the sword fighting.
\passage{Joab Kills Absalom}

\v{9}Absalom happened to run into David's soldiers. While Absalom was trying to get away on his mule, it ran under the thick branches of a giant oak tree, and Absalom's head got caught in the tree! As his mule ran out from under him, Absalom was left hanging above the ground. \v{10}When one of the soldiers saw what had happened, he told Joab, ``I saw Absalom stuck in an oak tree!''

\v{11}Joab asked the man who was reporting to him, ``What! You saw him? Why didn't you kill him right then and there? I would've given you ten pieces\fnote{\fbackref{18:11} The Heb. lacks \fbib{pieces}; the unit of payment is unspecified} of silver and a warrior's sash!''\fnote{\fbackref{18:11} Lit. \fbib{belt}; i.e., a commemorative battle decoration}

\v{12}But the soldier replied to Joab, ``I wouldn't have touched the king's son even if you dropped 1,000 pieces\fnote{\fbackref{18:12} The Heb. lacks \fbib{pieces}; the unit of payment is unspecified} of silver right into my hands, because we heard the king command you, Abishai, and Ittai, `Watch how you treat the young man Absalom!' \v{13}If I had taken his life,\fnote{\fbackref{18:13} Or \fbib{If I had put my life in jeopardy}; i.e. by disobeying David's order} the king would have uncovered everything about it, and you would never have protected me!''

\v{14}``There's no reason to wait for you!'' Joab retorted. Then he took three spears\fnote{\fbackref{18:14} Or \fbib{sticks}} in his hand and stabbed Absalom in the heart while he was still alive, dangling from the branches of\fnote{\fbackref{18:14} The Heb. lacks \fbib{the branches of}} the oak tree. \v{15}Ten young men who served as Joab's personal assistants then surrounded Absalom, striking him repeatedly and killing him. \v{16}At this, Joab sounded his battle trumpet and his troops stopped pursuing the other\fnote{\fbackref{18:16} The Heb. lacks \fbib{other}} Israelis. \v{17}Meanwhile, Joab's army grabbed Absalom's body, tossed it into a large pit in the forest, and filled it up with a huge pile of rocks. Then the Israelis ran away back to their homes.

\v{18}While Absalom had been living, he had erected a pillar as a monument\fnote{\fbackref{18:18} The Heb. lacks \fbib{as a monument}} to himself in King's Valley because he had been telling himself, ``I don't have a son to carry on my family name.''\fnote{\fbackref{18:18} Lit. \fbib{on memory of my name}} So he named the pillar after himself---it's called Absalom's Monument even today.
\passage{David Learns of Absalom's Death}

\v{19}Zadok's son Ahimaaz told Joab, ``Let me run over to King David and take him the news. I'll mention that the \divine{Lord} has delivered him from his enemies.''

\v{20}But Joab answered Ahimaaz, ``You're not the man to deliver news today. Do it any other time, but not today, because the king's son is dead.'' \v{21}So Joab ordered a man from Ethiopia,\fnote{\fbackref{18:21} Lit. \fbib{Cush}} ``Go tell the king what you've seen.'' So the Ethiopian\fnote{\fbackref{18:21} Lit. \fbib{Cushite}; and so throughout the chapter} saluted\fnote{\fbackref{18:21} Lit. \fbib{bowed to}} Joab and then ran to tell David.

\v{22}``Please,'' Zadok's son Ahimaaz continued, ``No matter what happens, let me follow the Ethiopian!''

Joab asked him, ``Why this request\fnote{\fbackref{18:22} The Heb. lacks \fbib{request}} to run, my son? There's no reward in it for you.''

\v{23}``No matter what, I'm running,'' Ahimaaz replied.\fnote{\fbackref{18:23} The Heb. lacks \fbib{Ahimaaz replied}}

So Joab told Ahimaaz, ``Run!'' And Ahimaaz ran, taking the Jordan Valley road, passing the Ethiopian.

\v{24}Meanwhile, David was sitting between the inner and outer gates of the city. The watchman was up on the roof of the gateway near the walls, looking around, and there was a man running by himself! \v{25}So the watchman\fnote{\fbackref{18:25} Lit. \fbib{he}} called out his news to the king.

The king responded, ``If he's alone, he's bringing some news to report.''\fnote{\fbackref{18:25} Lit. \fbib{news in his mouth}} As the man continued to draw near and approach the palace,\fnote{\fbackref{18:25} The Heb. lacks \fbib{the palace}} \v{26}the watchman observed another man running. So he called out to the gatekeeper, ``There's another\fnote{\fbackref{18:26} The Heb. lacks \fbib{another}} man running by himself!''

The king replied, ``He's also bringing some news to report!''

\v{27}Then the watchman observed, ``It looks to me that the runner out in front is running like Zadok's son Ahimaaz!''

The king replied, ``This is a good man bearing good news!''

\v{28}``Everything's fine!''\fnote{\fbackref{18:28} Lit. \fbib{Peace!}} Ahimaaz announced to the king. He bowed low with his face to the ground\fnote{\fbackref{18:28} The Heb. lacks \fbib{to the ground}} before the king and said, ``Praise be to the \divine{Lord} your God! He has handed over the men who rebelled against your majesty the king.''

\v{29}``Are things fine\fnote{\fbackref{18:29} Lit. \fbib{Peace!}} with respect to the young man Absalom?'' the king asked.

Ahimaaz answered, ``I saw a lot of confusion about the time Joab was getting ready to send the king's courier and me, your servant, but I'm not sure what was going on.''\fnote{\fbackref{18:29} The Heb. lacks \fbib{was going on}}

\v{30}The king replied, ``Stand here at attention and wait.'' So he stepped to the side and stood there waiting.

\v{31}Just then the Ethiopian arrived. He\fnote{\fbackref{18:31} Lit. \fbib{The Cushite}} reported, ``Good news, your majesty the king! The \divine{Lord} has delivered you from the control of everyone who rebelled against you!''

\v{32}The king asked the Ethiopian, ``Is the young man safe?''

The Ethiopian answered, ``May the enemies of your majesty the king---including everyone who rebels and tries to harm you---become like that young man{\ldots}.''
\passage{David Mourns for Absalom}

\v{33}\fnote{\fbackref{18:33} This v. is 19:1 in MT}Deeply shaken, the king went up to the chamber overlooking the city gate, weeping bitterly and crying out as he went along, ``My son Absalom! My son! My son Absalom! I wish I had died instead of you, Absalom my son, my son!''
\labelchapt{19}
\passage{Joab Rebukes David}

\chapt{19}
\v{1}\fnote{\fbackref{19:1} This v. is 19:2 in MT, 19:2 is 19:3 in MT, and so through 19:43}Someone informed Joab, ``The king is weeping bitterly, mourning for Absalom.'' \v{2}The victory had become an occasion for the army to mourn, because on that very day the troops heard the announcement, ``The king is grieving for his son!'' \v{3}So men snuck into the city that day like men do who are ashamed after they've run away from a battle.

\v{4}Meanwhile, the king veiled his face and kept on crying loudly, ``My son Absalom! Absalom my son, my son!''

\v{5}Joab went up to the palace and rebuked the king: ``Today you've humiliated your entire army who just saved your life, the lives of your sons and daughters, and the lives of your wives and mistresses! \v{6}You love those who hate you and hate those who love you! You've made it abundantly clear today that your officers and the men under them\fnote{\fbackref{19:6} Lit. \fbib{and the servants}} mean nothing to you! I've learned today that you would rather have Absalom alive today and all the rest of us dead! \v{7}Now get up and restore the morale of\fnote{\fbackref{19:7} Lit. \fbib{and encourage}} your army. I swear by the \divine{Lord} that if you don't get out there, you won't have a single man left in your army\fnote{\fbackref{19:7} Lit. \fbib{left with you}} by nightfall! You'll be in more trouble today than all the disasters you've been through from your boyhood until now!'' \v{8}So the king got up and took his seat in the gateway. When the army was informed, ``The king is sitting in the gateway!'' they all gathered together in his presence.
\passage{David is Reinstated as King}

Meanwhile, the Israelis had run away back to their own homes. \v{9}Throughout the tribes of Israel, everyone was quarreling with one another:

``The king delivered us from the domination of our enemies{\ldots}.''

``He's the one who rescued us from Philistine control{\ldots}.''

``Now he's fleeing the country because of Absalom{\ldots}!''

\v{10}``The very same Absalom we anointed to rule just died in battle{\ldots}!''

``Now then, why remain silent about bringing the king back{\ldots}?''

\v{11}So King David sent this message\fnote{\fbackref{19:11} The Heb. lacks \fbib{this message}} to Zadok and Abiathar, the priests: ``Ask the elders of Judah, `Why are you the last to bring the king back to his palace, considering that what's being reported throughout all of Israel has come to the king at his palace? \v{12}You're my relatives! You're my own flesh and blood! So why are you the last to bring back the king?' \v{13}Then ask Amasa, `Aren't you my own flesh and blood? So may God deal with me, no matter how severely, if from this day forward you don't take Joab's place as commander of my army.'

\v{14}By doing things like this,\fnote{\fbackref{19:14} The Heb. lacks \fbib{By doing things like this}} he persuaded all the men of Judah to unite in support of him.\fnote{\fbackref{19:14} The Heb. lacks \fbib{in support of him}} They sent the king this message: ``Come on back, you and all of your army!'' \v{15}So the king returned to Israel as far as the Jordan River.\fnote{\fbackref{19:15} The Heb. lacks \fbib{River}; and so throughout the chapter}
\passage{Shimei is Shown Mercy}

The men of Judah went out as far as Gilgal to greet the king and escort him across the Jordan River \v{16}while Gera's son Shimei,\fnote{\fbackref{19:16} Cf. 2Sam 16:5-12} a descendant of Benjamin from Bahurim, accompanied them to meet King David. \v{17}Ziba, the steward in charge of Saul's household, and 1,000 descendants of Benjamin accompanied him, along with Ziba's fifteen sons and 20 servants. They rushed toward the Jordan River ahead of the king \v{18}and forded it to assist the king at the crossing so he could do whatever he wished.

Just as the king was about to ford the Jordan River, Gera's son Shimei fell down in front of the king \v{19}and addressed him,\fnote{\fbackref{19:19} Lit. \fbib{addressed the king}} ``May your majesty not hold me guilty. Don't remember how your servant did wrong the day your majesty the king left Jerusalem. May the king not let it burden his heart, \v{20}because your servant knows that I have sinned, but today I have come here as the first one from the entire house of Joseph to meet your majesty the king.''

\v{21}But Zeruiah's son Abishai asked, ``Why shouldn't Shimei be put to death for this? After all, he cursed the \divine{Lord}'s anointed!''

\v{22}David replied, ``What do you sons of Zeruiah have in common with me?\fnote{\fbackref{19:22} Cf. 2Sam 16:10} You've become my enemies today! Should anyone be executed in Israel today? Don't you know that I've been reinstated as king over Israel today?'' \v{23}Then the king addressed Shimei, ``You won't die!'' affirming his promise with an oath.
\passage{David Meets Mephibosheth}

\v{24}Meanwhile, Saul's grandson Mephibosheth also went out to greet the king. He had not taken care of his feet, trimmed his mustache, or washed his clothes from the day the king left until the day he returned safely. \v{25}When he arrived from Jerusalem to greet the king, the king asked him, ``So why didn't you come with me, Mephibosheth?''

\v{26}He replied, ``Well, your majesty, since your servant is lame, I told myself, `I'll have my donkey saddled and I'll ride on it so I can leave with the king.' But my servant Ziba deceived me \v{27}by slandering your servant to your majesty.\fnote{\fbackref{19:27} Cf. 2Sam 16:1-4} But your majesty the king is like an angel from God: so do what you think is best. \v{28}Everyone from my grandfather's household deserved nothing but death from your majesty the king, but you provided a place for your servant among those who have been eating from your table. So what right do I have to ask for anything more from the king?''

\v{29}In response, the king told him, ``What's the point of us talking anymore? My decision is that you and Ziba divide the fields.''

\v{30}But Mephibosheth told the king, ``Let him take all of it, now that your majesty the king has returned safely to his palace.''
\passage{David's Mercy for Barzillai}

\v{31}Barzillai the Gileadite also had come down from Rogelim to cross the Jordan River with the king and to see him on his way from there. \v{32}Now Barzillai was a very old man at the age of 80 years. A very wealthy man, Barzillai\fnote{\fbackref{19:32} Lit. \fbib{he}} had provided for king David during his sojourn in Mahanaim.\fnote{\fbackref{19:32} Cf. 2Sam 17:27} \v{33}So the king invited Barzillai, ``Cross the Jordan River\fnote{\fbackref{19:33} The Heb. lacks \fbib{the Jordan River}} with me, live with me in Jerusalem, and I'll provide for you there.''\fnote{\fbackref{19:33} The Heb. lacks \fbib{there}}

\v{34}``How many more years do I have to live,'' Barzillai replied to the king, ``that I should move to Jerusalem with the king? \v{35}I'm now 80 years old! I can hardly tell the difference between what tastes\fnote{\fbackref{19:35} The Heb. lacks \fbib{what tastes}} good or bad! I can't tell what I eat or drink! I can't hear the voice of men and women when they sing! So why should your servant be an added burden to your majesty the king? \v{36}Your servant will cross the Jordan River\fnote{\fbackref{19:36} The Heb. lacks \fbib{River}} with the king for a short distance, but why should the king offer me this reward? \v{37}Please let your servant return so I can die in my own home town near the grave of my father and mother. Meanwhile, here is your servant Chimham!\fnote{\fbackref{19:37} I.e., a son of Barzillai to whom David later gave a land grant near Bethlehem and on which Chimham built an inn that remained at least until the exile; cf. Jer 41:17} Let him accompany your majesty the king. Please do for him whatever seems best to you.''

\v{38}So the king answered, ``Chimham will accompany me, and I'll do for him whatever seems best to you! I'll do anything for you that you want!'' \v{39}Then all the people crossed the Jordan River,\fnote{\fbackref{19:39} The Heb. lacks \fbib{River}} followed by the king. The king embraced\fnote{\fbackref{19:39} Or \fbib{kissed}} Barzillai, blessed him, and then Barzillai\fnote{\fbackref{19:39} Lit. \fbib{he}} returned to his home.\fnote{\fbackref{19:39} Lit. \fbib{place}} \v{40}As the king crossed over the Jordan River\fnote{\fbackref{19:40} The Heb. lacks \fbib{the Jordan River}} to Gilgal, Chimham accompanied him, as did all the troops of Judah and half the troops of Israel.
\passage{Petty Quarrels Arise between Israel and Judah}

\v{41}Not long afterward, all the men of Israel started coming to the king, complaining to him,\fnote{\fbackref{19:41} Lit. \fbib{to the king}} ``Why did our relatives in Judah's army sneak you away, taking the king and his household over the Jordan River,\fnote{\fbackref{19:41} The Heb. lacks \fbib{River}} along with David's army?''

\v{42}Everybody from Judah shouted to the men from Israel, ``We did this because the king is closely related to us. So why are you angry about this? Have we lived off\fnote{\fbackref{19:42} Lit. \fbib{we eaten from}} the king's expense? Have we appropriated anything for ourselves?''

\v{43}But the men from Israel answered the men from Judah: ``We\fnote{\fbackref{19:43} Lit. \fbib{I}} represent ten of the tribes\fnote{\fbackref{19:43} Lit. \fbib{ten hands}; i.e. ten fractional portions} of Israel! So we\fnote{\fbackref{19:43} Lit. \fbib{I}} have more right to David than you\fnote{\fbackref{19:43} MT \fbib{you} is sing.} do! Why haven't you\fnote{\fbackref{19:43} MT \fbib{you} is sing.} taken us\fnote{\fbackref{19:43} Lit. \fbib{me}} seriously? Weren't we\fnote{\fbackref{19:43} Lit. \fbib{Wasn't I}} the first to talk about bringing back our\fnote{\fbackref{19:43} Lit. \fbib{my}} king?'' But what the people of Judah had to say was harsher than what the people of Israel were saying.
\labelchapt{20}
\passage{Sheba's Rebellion}

\chapt{20}
\v{1}Right about then, Bichri's son Sheba, an ungodly man\fnote{\fbackref{20:1} Lit. \fbib{a son of Belial}} from the tribe of Benjamin, sounded a battle trumpet and announced:

\begin{poetry}
\poeml We've never been a part of David! \\
\poemll    We'll never gain anything from Jesse's son! \\
\poemlll       It's every man to his tent, Israel!
\end{poetry}

\v{2}So all of the other Israeli soldiers\fnote{\fbackref{20:2} I.e. the ten tribes apparently mentioned in 2Sam 19:43; the Heb. lacks \fbib{other}} abandoned David to follow Bichri's son Sheba, while the army of Judah remained with the king all the way from the Jordan River\fnote{\fbackref{20:2} The Heb. lacks \fbib{River}} to Jerusalem.

\v{3}When David arrived at his palace in Jerusalem, the king took the ten mistresses\fnote{\fbackref{20:3} Lit. \fbib{concubines}; i.e. secondary wives} whom he had left behind to keep the palace in order and placed them in a separate house, providing for them under the care of a protective guard. He never visited them again, so they were under care until they died, living as if their husbands had died.

\v{4}Meanwhile, David ordered Amasa, ``Muster the army of Judah here within three days, and be here yourself!''

\v{5}But when Amasa went out to muster the army of\fnote{\fbackref{20:5} The Heb. lacks \fbib{the army of}} Judah, he delayed to act within the time allotted to him. \v{6}So David told Abishai, ``Now Bichri's son Sheba is about to do more damage than did Absalom. So take my personal guards and go after them. Otherwise, he'll run to one of the fortified cities and escape from us.'' \v{7}So Joab's men, the special forces\fnote{\fbackref{20:7} Lit. \fbib{Cherethites}; i.e. elite body guards} and mercenaries,\fnote{\fbackref{20:7} Lit. \fbib{Pelethites}; i.e. special couriers} and all of David's elite forces left Jerusalem in pursuit of Bichri's son Sheba.
\passage{Joab Murders Amasa}

\v{8}When they arrived at the great stone that is in Gibeon, Amasa came out to meet them. Joab was dressed in a soldier's uniform, over which was a belt that fastened a sword sheath to his thigh. As he walked forward, the sword was exposed. \v{9}Joab asked Amasa, ``Is everything going well with you, my brother?'' As Joab took Amasa by his beard to greet him, \v{10}Amasa did not notice the sword that Joab was holding in his hand. Joab stabbed him in the abdomen, spilling his intestines to the ground in a single stroke and killing him. After this, Joab and his brother pursued Bichri's son Sheba.

\v{11}One of Joab's soldiers stood by Amasa while he lay dying\fnote{\fbackref{20:11} The Heb. lacks \fbib{while he lay dying}} and announced, ``Whoever is in favor of Joab and David, let him follow Joab.'' \v{12}While Amasa lay wallowing in his blood in the middle of the highway, everybody who passed by was stopping to stare at him, so when the soldier saw that all of the army was stopping, he carried Amasa off the highway into a nearby field and covered him with a garment. \v{13}After Amasa\fnote{\fbackref{20:13} it. \fbib{he}} had been removed from the highway, the rest of the army followed Joab in pursuit of Bichri's son Sheba.
\passage{Sheba Dies at Abel of Beth-maacah}

\v{14}Meanwhile, Sheba traveled throughout the tribes of Israel in the direction of Abel and Beth-maacah, and all of the descendants of Beri\fnote{\fbackref{20:14} So MT; some ancient versions read \fbib{descendants of Bichri}} gathered together and followed him inside. \v{15}All of the men who had accompanied Joab arrived and besieged Sheba in Abel of Beth-maacah. They threw up a siege ramp against the city rampart and began to batter the wall to demolish it. \v{16}Just then a wise woman called out from the city. ``Attention!'' she said, ``Go tell Joab `Come here! I want to talk to you!'\,'' \v{17}Joab came over and the woman asked him, ``Are you Joab?''

``I am,'' he answered.

So she told him, ``Listen to what your servant has to say!''

``I'm listening,'' he replied.

\v{18}So she said, ``In days past, people used to settle a dispute by saying `Let's ask for advice at Abel!' \v{19}I'm one of the peaceful and faithful citizens of Israel. You're trying to destroy a city that's a mother in Israel. Why are you devouring the heritage of the \divine{Lord}?''

\v{20}But Joab replied, ``No way! No way! I'm not here to devour or destroy! \v{21}That's a lie! But there is a man from the Ephraim hill country---he's known as Bichri's son Sheba---who has rebelled against King David. Turn him over and I'll withdraw from the city!''

So the woman replied, ``Watch this! His head will be thrown to you over the city wall.'' \v{22}Then the woman wisely went back to her people. They cut off the head of Bichri's son Sheba and threw it out to Joab, so Joab sounded his battle trumpet and they withdrew from the city. Everybody went back home and Joab returned to the king at Jerusalem.

\v{23}Joab commanded the entire army of Israel, Jehoiada's son Benaiah commanded the special forces\fnote{\fbackref{20:23} Lit. \fbib{Cherethites}; i.e. elite body guards} and mercenaries,\fnote{\fbackref{20:23} Lit. \fbib{Pelethites}; i.e. special couriers} \v{24}Adoram supervised conscripted labor, Ahilud's son Jehoshaphat was the recorder, \v{25}Sheva was secretary, Zadok and Abiathar were priests, \v{26}and Ira the Jairite\fnote{\fbackref{20:26} Cf. 2Sam 23:38, where he is also known as \fbib{Ira the Ithrite}} was David's priest.
\labelchapt{21}
\passage{Retribution for the Gibeonites}

\chapt{21}
\v{1}One time there was a famine during David's reign that went on for three straight years. David sought the \divine{Lord}, who\fnote{\fbackref{21:1} Lit. \fbib{sought the face of the \divine{Lord}, and the \divine{Lord}}} said, ``Saul and his household are guilty because he executed the Gibeonites.''

\v{2}So the king called together the Gibeonites and conferred with them. Now the Gibeonites weren't part of the nation of Israel, but were the survivors from the Amorites. Although the Israelis had promised to spare them, Saul had started to execute them in his zeal for the people of Israel and Judah.

\v{3}So David asked the Gibeonites, ``What am I to do for you? How am I to make atonement so that you will bless the \divine{Lord}'s heritage?''

\v{4}``We're not looking for mere silver or gold to be paid by Saul or his household to us,'' the Gibeonites responded to him. ``And it's not for us to execute anyone in Israel.''

In reply, David\fnote{\fbackref{21:4} Lit. \fbib{he}} asked, ``So what are you asking me to do for you?''

\v{5}They told the king, ``The man who consumed us, who planned our destruction---intending to leave us with nothing in the territory of Israel--- \v{6}is to have\fnote{\fbackref{21:6} Lit. \fbib{Israel}{\ldots} \fbib{\v{6}Let seven}} seven of his sons turned over to us. We will hang\fnote{\fbackref{21:6} Or \fbib{impale}; i.e. they would execute them and then expose the bodies} them in the presence of the \divine{Lord} at Gibeah, which belonged to Saul, whom the \divine{Lord} chose.''

So the king answered, ``I will give them.''\fnote{\fbackref{21:6} The Heb. lacks \fbib{them}} \v{7}The king exempted Mephibosheth, the son of Saul's son Jonathan, because of the promise to the \divine{Lord} that existed between David and Saul's son Jonathan.

\v{8}Instead, the king arrested Aiah's daughter Rizpah's two sons Armoni and Mephibosheth, whom she had borne to Saul, and the five sons of Saul's daughter Merab, whom she had borne to Barzillai the Meholathite's son Adriel. \v{9}Then he turned them over to the custody of the Gibeonites, who hanged them on the mountain in the presence of the \divine{Lord}. All seven of them died at the same time. They were executed during the first days of harvest, just as the barley began to be gathered in.

\v{10}Then Aiah's daughter Rizpah grabbed some sackcloth and spread it out for herself on the rock where her children had been hanged\fnote{\fbackref{21:10} The Heb. lacks \fbib{where her children had been hanged}} from the beginning of harvest until the first rain fell from the sky. She would not allow any scavenger birds\fnote{\fbackref{21:10} Lit. \fbib{any birds of the sky}} to land on them during the day nor the beasts of the field to approach them\fnote{\fbackref{21:10} The Heb. lacks \fbib{to approach them}} at night.

\v{11}When David was informed what Rizpah, the daughter of Saul's mistress\fnote{\fbackref{21:11} Lit. \fbib{concubine}; a secondary wife} had done, \v{12}David had Saul's bones and the bones of his son Jonathan removed from the custody of certain men from Jabesh-gilead, who had stolen them from the public square in Beth-shan, where the Philistines had hanged them---that is, back on the day when the Philistines had killed Saul on Mount\fnote{\fbackref{21:12} The Heb. lacks \fbib{Mount}} Gilboa. \v{13}He brought the bones of Saul and his son Jonathan from there along with the bones of those who had been hanged, \v{14}and they buried Saul's bones and his son Jonathan's bones in the territory of Benjamin in Zela, in the tomb of Saul's\fnote{\fbackref{21:14} Lit. \fbib{his}} father Kish. After they had done everything that the king commanded, God responded to prayers for the land.\fnote{\fbackref{21:14} Cf. 2Sam 24:25}
\passage{Israel Battles Four Giants from Gath}
\passageinfo{(1 Chronicles 20:4-8)}

\v{15}Afterwards, war broke out between the Philistines and Israel, so David went down to fight the Philistines. David became weary, \v{16}and Ishbi-benob, who had been fathered by giants,\fnote{\fbackref{21:16} Lit. \fbib{by the Rapha}; and so throughout the chapter} said he intended to kill David. (His bronze spearhead weighed 300 shekels,\fnote{\fbackref{21:16} I.e., about seven and a half pounds at 0.4 shekels per ounce} and he carried state-of-the-art\fnote{\fbackref{21:16} Or \fbib{newly-issued}; lit. \fbib{newly girded}} weaponry.) \v{17}But Zeruiah's son Abishai came to David's aid, attacked the Philistine, and killed him. After this, David's army told him, ``You're not going out anymore with us to battle, so Israel's beacon won't be extinguished!'' \v{18}Sometime later after this incident, there was another battle with the Philistines at Gob. Sibbecai the Hushathite killed Saph, who had been fathered by giants. \v{19}In yet another battle at Gob, Jaare-oregim the Bethlehemite's son Elhanan killed Goliath the Gittite, the shaft of whose spear resembled that of a weaver's beam. \v{20}Later on, there was another battle at Gath, where there was a very tall man with six fingers on each hand and six toes on each foot---24 in number---who had also been fathered by giants. \v{21}When he defied Israel, David's brother Shimeah's son Jonathan killed him. \v{22}These four giants, who had been fathered by a giant in Gath, were killed at the hands of David and his servants.
\labelchapt{22}
\passage{David's Psalm of Deliverance}

\chapt{22}
\v{1}David composed the words of this song to the \divine{Lord} the very day the \divine{Lord} delivered him from the domination\fnote{\fbackref{22:1} Lit. \fbib{hand}} of all of his enemies, including from Saul's hands. \v{2}This is what\fnote{\fbackref{22:2} The Heb. lacks \fbib{This is what}} he said:

\begin{poetry}
\poeml \divine{Lord}, you are\fnote{\fbackref{22:2} So LXX. MT reads \fbib{The \divine{Lord} is}} my stone stronghold \\
\poemll    and my fortified place; \\
\poemlll       you are continuously delivering\fnote{\fbackref{22:2} So MT LXX reads \fbib{rescuing}} me. \\
\poeml \v{3}He is my God, \\
\poemll    my strong stone--- \\
\poemlll       in him I will find my refuge--- \\
\poeml my shield, \\
\poemll    the strength\fnote{\fbackref{22:3} Lit. \fbib{horn}} of my salvation, \\
\poemlll       my high tower, \\
\poeml my way of escape, \\
\poemll    and the one who is saving me. \\
\poemlll       You will save me from violence. \\
\poeml \v{4}As I am praising him,\fnote{\fbackref{22:4} The Heb. lacks \fbib{him}} \\
\poemll    I will call out to the \divine{Lord}, \\
\poemlll       and I will be saved from my enemies. \\
\poeml \v{5}Because deadly breakers\fnote{\fbackref{22:5} Or \fbib{currents}} engulfed me, \\
\poemll    while torrents of abuse\fnote{\fbackref{22:5} The Heb. lacks \fbib{of abuse}} from the ungodly overwhelmed\fnote{\fbackref{22:5} Or \fbib{terrified}} me. \\
\poeml \v{6}Binding ropes from Sheol entangled me \\
\poemll    while lethal snares hindered me. \\
\poeml \v{7}I cried out to the \divine{Lord} in the middle of my troubles; \\
\poemll    I cried out to my God. \\
\poeml He listened to my voice from his sanctuary, \\
\poemll    and my call for help was heard. \\
\poeml \v{8}Just then the earth shook and trembled! \\
\poemll    The foundations of heaven reeled and quaked \\
\poemlll       because the \divine{Lord}\fnote{\fbackref{22:8} Lit. \fbib{because he}} was angry. \\
\poeml \v{9}Smoke poured out of his nostrils, \\
\poemll    and fire from his mouth \\
\poemlll       kindling coals to flame by it. \\
\poeml \v{10}He deformed heaven itself as he descended. \\
\poemll    Thick darkness enveloped his feet. \\
\poeml \v{11}He rode on a cherub and flew, \\
\poemll    soaring on the wings of the wind! \\
\poeml \v{12}The darkness around him was his canopies--- \\
\poemll    amassed water was his overhanging clouds! \\
\poeml \v{13}From the shining light\fnote{\fbackref{22:13} Or \fbib{the brightness}} that was his presence\fnote{\fbackref{22:13} Or \fbib{counterpart}; MT word is perhaps a word play on the noun \fbib{shining light}} \\
\poemll    coals of fire blazed into flame! \\
\poeml \v{14}The \divine{Lord} roared from heaven! \\
\poemll    The Most High let his voice be heard! \\
\poeml \v{15}He launched his arrows and scattered them--- \\
\poemll    his lightning routed them. \\
\poeml \v{16}The currents of the sea were revealed \\
\poemll    and the foundations of the world were exposed \\
\poeml at the rebuke of the \divine{Lord} \\
\poemll    and at the blazing breath from his nostrils! \\
\poeml \v{17}He sent for me from on high! \\
\poemll    He grabbed hold of me, \\
\poemlll       drawing me out of deep water. \\
\poeml \v{18}He rescued me from my strong enemy--- \\
\poemll    from those who hate me continually, \\
\poemlll       since they were stronger than I. \\
\poeml \v{19}They confronted me when I was in trouble,\fnote{\fbackref{22:19} Lit. \fbib{in the day of my calamity}} \\
\poemll    but the \divine{Lord} remained my support! \\
\poeml \v{20}He brought me to a wide open area, \\
\poemll    rescuing me because he was pleased with me! \\
\poeml \v{21}The \divine{Lord} dealt with me according to my righteousness, \\
\poemll    rewarding me according to the degree of my innocence,\fnote{\fbackref{22:21} Lit. \fbib{the cleanness of my hands}} \\
\poeml \v{22}because I have kept the \divine{Lord}'s way--- \\
\poemll    I haven't willfully abandoned my God--- \\
\poeml \v{23}and because all of his decrees remain in my thoughts,\fnote{\fbackref{22:23} Lit. \fbib{presence}} \\
\poemll    I have not turned aside from his statutes, \\
\poeml \v{24}I have been innocent before him, \\
\poemll    and I've kept myself from incurring\fnote{\fbackref{22:24} The Heb. lacks \fbib{incurring}} guilt. \\
\poeml \v{25}The Lord has repaid me according to my righteousness, \\
\poemll    that is, according to my clean standing as he\fnote{\fbackref{22:25} Lit. \fbib{standing to the counterpart}} looks at me.\fnote{\fbackref{22:25} Or \fbib{standing in his eyes}} \\
\poeml \v{26}In the company of the gracious \\
\poemll    you demonstrate your gracious love. \\
\poeml In the company of the blamelessly valiant \\
\poemll    you demonstrate your blamelessness. \\
\poeml \v{27}In the company of the pure \\
\poemll    you demonstrate your purity. \\
\poeml In the company of the perverted \\
\poemll    you will appear to be perverse. \\
\poeml \v{28}You save the nation who is humble \\
\poemll    but your eyes watch the proud, \\
\poemlll       to bring them down. \\
\poeml \v{29}For you are my lamp, \divine{Lord}, \\
\poemll    the \divine{Lord} who illuminates my darkness. \\
\poeml \v{30}By you I devastate armies, \\
\poemll    by my God I scale walls. \\
\poeml \v{31}This God! His way is perfect! \\
\poemll    What the \divine{Lord} declares proves true. \\
\poemlll       He shields\fnote{\fbackref{22:31} Lit. \fbib{He is a shield for}} everyone who flees for protection to him! \\
\poeml \v{32}For who is God apart from the \divine{Lord}? \\
\poemll    And who is a Rock, apart from our God? \\
\poeml \v{33}This God is my strong place of valor! \\
\poemll    He has made my life\fnote{\fbackref{22:33} Lit. \fbib{way}} blameless. \\
\poeml \v{34}He has made my feet like those of a deer, \\
\poemll    setting me secure on his high places! \\
\poeml \v{35}He has trained my hands for battle readiness--- \\
\poemll    I can bend a bow made out of bronze. \\
\poeml \v{36}He has equipped me with the shield that is your salvation, \\
\poemll    Your gentleness\fnote{\fbackref{22:36} So MT; LXX reads \fbib{Obedience to you}} has made me great. \\
\poeml \v{37}You've made room beneath me for my footsteps, \\
\poemll    and my feet didn't slip. \\
\poeml \v{38}I pursued my enemies and conquered them; \\
\poemll    I didn't return until they were consumed. \\
\poeml \v{39}I devoured them, \\
\poemll    striking them down \\
\poeml until they could not get up again. \\
\poemll    They fell beneath my feet. \\
\poeml \v{40}You strengthened me with valor sufficient for the battle; \\
\poemll    you made those who rebelled against me fall beneath me. \\
\poeml \v{41}You made my enemies turn and run---\fnote{\fbackref{22:41} Lit. \fbib{away their backs to me}} \\
\poemll    that is, those who hate me--- \\
\poemlll       and I destroyed them! \\
\poeml \v{42}They looked around, but there was no one to save\fnote{\fbackref{22:42} MT verb \fbib{looked around} sounds like MT verb \fbib{to save}} them\fnote{\fbackref{22:42} The Heb. lacks \fbib{them}}--- \\
\poemll    they looked\fnote{\fbackref{22:42} The Heb. lacks \fbib{they looked}} to the \divine{Lord}, but he paid no attention! \\
\poeml \v{43}I pulverized them to powder, \\
\poemll    like the dust of the earth; \\
\poeml I crushed them, \\
\poemll    stomping on them like mud on a street. \\
\poeml \v{44}You delivered me from civil war among my own people. \\
\poemll    You preserved me as head of the nations. \\
\poemlll       People whom I had never known served me! \\
\poeml \v{45}Foreigners\fnote{\fbackref{22:45} Lit. \fbib{Children of foreigners}} came cringing to me; \\
\poemll    they obeyed as soon as they heard\fnote{\fbackref{22:45} MT verbs \fbib{obeyed} and \fbib{heard} are identical in spelling} me. \\
\poeml \v{46}Foreigners\fnote{\fbackref{22:46} Lit. \fbib{Children of foreigners}} lost their courage, \\
\poemll    coming trembling from their strongholds. \\
\poeml \v{47}The \divine{Lord} lives! \\
\poemll    Blessed be my Rock, \\
\poeml and may my God be exalted, \\
\poemll    the Rock who is my salvation! \\
\poeml \v{48}The God who keeps on avenging me, \\
\poemll    subjugating people beneath me, \\
\poeml \v{49}delivering me from my enemies. \\
\poeml You exalted me above those who rebelled against me, \\
\poemll    delivering me from violent men. \\
\poeml \v{50}Because of all of this I will praise you among the nations, \divine{Lord}, \\
\poemll    and I will sing praises to your name! \\
\poeml \v{51}Great is the salvation he brings to his king, \\
\poemll    showing gracious love to his anointed, \\
\poemlll       to David and to his offspring\fnote{\fbackref{22:51} Lit. \fbib{seed}; MT is sing.} forever.
\end{poetry}
\labelchapt{23}
\passage{David's Oracle}

\chapt{23}
\v{1}This was David's last composition:

\begin{poetry}
\poeml The oracle of David, son of Jesse, \\
\poemll    an oracle by the valiant one who was exalted--- \\
\poeml anointed by the God of Jacob, \\
\poemll    the contented psalm writer of Israel. \\
\poeml \v{2}The Spirit of the \divine{Lord} speaks within\fnote{\fbackref{23:2} Or \fbib{through}} me; \\
\poemll    his word is on my tongue! \\
\poeml \v{3}The God of Israel has spoken; \\
\poemll    the Rock of Israel has talked to me. \\
\poeml ``When one is governing men justly, \\
\poemll    he fears God while governing. \\
\poeml \v{4}He is like dawn's first\fnote{\fbackref{23:4} The Heb. lacks \fbib{first} } light, \\
\poemll    like bright sun blazing on a cloudless morning, \\
\poemlll       glistening on grassland that flourishes after a rain shower. \\
\poeml \v{5}Is not my dynasty\fnote{\fbackref{23:5} Lit. \fbib{house}} like this with God? \\
\poemll    Has he not made an eternal covenant with me, \\
\poemlll       preparing every detail of it? \\
\poeml And he has made it secure, \\
\poemll    including my complete\fnote{\fbackref{23:5} Lit. \fbib{including all of my}} salvation, has he not? \\
\poeml He has been of continual\fnote{\fbackref{23:5} Lit. \fbib{He has been all}} help, has he not, \\
\poemll    even with respect to all of my desires? \\
\poeml \v{6}But ungodly men\fnote{\fbackref{23:6} Lit. \fbib{But Belial}} are like thorns that are discarded \\
\poemll    because they cannot be safely\fnote{\fbackref{23:6} The Heb. lacks \fbib{safely}} handled. \\
\poeml \v{7}Whoever handles them \\
\poemll    wears heavy duty clothing,\fnote{\fbackref{23:7} Lit. \fbib{arms himself with iron}} \\
\poeml carries strong tools,\fnote{\fbackref{23:7} Lit. \fbib{and a spear shaft}} \\
\poemll    and burns them to ashes on the spot!\fnote{\fbackref{23:7} Lit. \fbib{ashes where they sit}}
\end{poetry}
\passage{David's Elite Soldiers}
\passageinfo{(1 Chronicles 11:10-19)}

\v{8}Here's a list of the names of David's special forces: Josheb-basshebeth the Tahkemonite\fnote{\fbackref{23:8} Cf. 1Chr 11:11, where this individual is named \fbib{Hachmoni's son Jashobeam}} was head of the Three;\fnote{\fbackref{23:8} I.e. a group of three distinguished officers who served David, and so throughout the chapter; cf. 1Chr 11:12} he was nicknamed Adino the Eznite\fnote{\fbackref{23:8} The two Heb. names comprise a word play that roughly translates as \fbib{Thin as a Spear}} because he killed 800 men in a single battle engagement.

\v{9}Next was Dodai\fnote{\fbackref{23:9} Cf. 1Chr 11:12, where this individual is named \fbib{Dodo}} the Ahohite's son Eleazar. Eleazar, who also was one of the Three, was with David when they challenged the Philistines. When the Philistines had assembled in battle array, the Israeli army retreated, \v{10}but Eleazar remained standing right where he was and fought so hard against the Philistines that he became exhausted---he couldn't even let go of his sword! The \divine{Lord} magnificently delivered them that day. After Eleazar had won the battle, the other soldiers returned, but only to strip the weapons and armor from the dead.\fnote{\fbackref{23:10} The Heb. lacks \fbib{the weapons and armor from the dead}}

\v{11}Next was Shammah, Agee the Hararite's son. One time the Philistines assembled to fight\fnote{\fbackref{23:11} Or \fbib{assembled at Lehi}} in a field where lentils had been growing. Israel's army retreated from the Philistines, \v{12}but Shammah stood his ground in the middle of the field, defended it, and killed the Philistines. And the \divine{Lord} brought about a great victory.

\v{13}One day while the Philistine army was camping in the valley of giants,\fnote{\fbackref{23:13} Or \fbib{the Rephaim Valley}} three of the 30 leaders joined David at the cave of Adullam. \v{14}David was living in that stronghold at the time, while a Philistine garrison was then at Bethlehem.

\v{15}David expressed his longing, ``Oh, how I wish someone would get me a drink of water from the Bethlehem well that's by the city gate!'' \v{16}So the Three elite warriors broke through the Philistine ranks, drew some water from the Bethlehem well that was next to the city gate, and brought it back to David. But he refused to drink it. Instead, he poured it out in the \divine{Lord}'s presence, \v{17}and said, ``The \divine{Lord} forbid that I drink this---this is the blood of men who endangered their own lives!'' The Three elite warriors did these things.
\passage{David's Other Valiant Soldiers}
\passageinfo{(1 Chronicles 11:20-47)}

\v{18}Zeruiah's son Abishai, Joab's brother, was the lieutenant\fnote{\fbackref{23:18} Lit. \fbib{chief}} in charge of the platoons.\fnote{\fbackref{23:18} So Syr; MT reads \fbib{Three}} He used his spear to fight and kill 300 men, gaining a reputation distinct from the Three. \v{19}He was more well-known than the Three, and became their commander, but he never measured up to\fnote{\fbackref{23:19} Or \fbib{never attained the stature of}} the Three.

\v{20}Jehoiada's son Benaiah, who was a valiant man, accomplished great things. He was from Kabzeel. He killed two men named\fnote{\fbackref{23:20} The Heb. lacks \fbib{men named}} Ariel from Moab\fnote{\fbackref{23:20} The Heb. name \fbib{Ariel} means \fbib{lion}} and then he also went down into a pit and struck down a lion during a snow storm one day. \v{21}He also killed a soldier\fnote{\fbackref{23:21} Lit. \fbib{man}} from Egypt. Of handsome appearance, the Egyptian carried a spear, but Benaiah attacked him with a staff, snatched the spear out of the Egyptian's hand and killed him with his own spear. \v{22}Benaiah did things like this and gained a reputation comparable to the Three warriors. \v{23}He was well known among the platoons, but he didn't measure up to\fnote{\fbackref{23:23} Or \fbib{he never attained the stature of}} the Three. David placed him in charge of his security detail.

\v{24}Among the Thirty were Joab's brother Asahel, Dodo's son Elhanan of Bethlehem, \v{25}Shammah from Harod; Elika from Harod, \v{26}Helez the Paltite,\fnote{\fbackref{23:26} Cf. 1Chr 11:27, where he is named \fbib{Helez the Pelonite}} Ikkesh's son Ira from Tekoa, \v{27}Abiezer from Anathoth, Mebunnai the Hushathite, \v{28}Zalmon the Ahohite, Maharai of Netophah, \v{29}Baanah's son Heleb from Netophah, Ribai's son Ittai from Gibeah of the descendants of Benjamin, \v{30}Benaiah from Pirathon, Hiddai from the Gaash creeks area,\fnote{\fbackref{23:30} The Heb. lacks \fbib{area}; i.e. a region in Gaash containing numerous seasonal streams} \v{31}Abi-albon the Arbathite, Azmaveth from Bahurim, \v{32}Eliahba from Shaalbon, Jashen's sons, \v{33}Shammah's son from Harar, Sharar the Hararite's son Ahiam, \v{34}Ahasbai the Maacathite's son Eliphelet, Ahithophel the Gilonite's son Eliam,\fnote{\fbackref{23:34} Bathsheba's father was Eliam; her grandfather was Ahithophel; cf. 2Sam 11:3; 15:12} \v{35}Hezro from Carmel, Paarai the Arbite, \v{36}Nathan's son Igal from Zobah, Bani the Gadite, \v{37}Zelek the Ammonite, Naharai from Beeroth (the armor-bearer for Zeruiah's son Joab), \v{38}Ira the Ithrite,\fnote{\fbackref{23:38} Cf. 2Sam 20:26, where he is also known as \fbib{Ira the Jairite}} Gareb the Ithrite, \v{39}and Uriah the Hittite---for a total of 37.
\labelchapt{24}
\passage{David Takes a Census of Israel}
\passageinfo{(1 Chronicles 21:1-6)}

\chapt{24}
\v{1}Later, God's anger blazed forth against Israel, so he incited David to move against them by telling him, ``Go take a census of Israel and Judah.''

\v{2}So the king ordered Joab, commander of the special forces, who was with him, ``Go throughout the tribes of Israel from Dan to Beer-sheba and take a census of the people so I can be made aware of the total number.''

\v{3}But Joab replied, ``May the \divine{Lord} your God increase the population of the people a hundredfold while your majesty the king is still alive to see it happen! But why does your majesty the king want to do this?''

\v{4}But the king's order overruled Joab and the commanders of the special forces, so Joab and the commanders of the special forces left David's presence to take a census of the people of Israel. \v{5}They crossed the Jordan River,\fnote{\fbackref{24:5} The Heb. lacks \fbib{River}} encamped at Aroer south of the town that is located in the river valley, proceeding through Gad and then on toward Jazer. \v{6}They went on to Gilead and the territory of Tahtim-hodshi, then on toward Dan. From Dan they went around to Sidon \v{7}and arrived at the fortified city of Tyre and all of the towns of the Hivites and Canaanites.

Eventually they proceeded to Beer-sheba in the Judean Negev.\fnote{\fbackref{24:7} I.e. southern regions of the Sinai peninsula; cf. Josh 10:40} \v{8}After they had traveled throughout the entire land, they returned to Jerusalem at the end of nine months and 20 days. \v{9}Joab reported the total number of men to the king. In Israel there were 800,000 men trained for war.\fnote{\fbackref{24:9} Lit. \fbib{men in wielding a sword}} In Judah there were 500,000.
\passage{Discipline for David's Sin}
\passageinfo{(1 Chronicles 21:7-17)}

\v{10}Later, David's conscience bothered\fnote{\fbackref{24:10} Lit. \fbib{David's heart struck}} him after he had numbered the army,\fnote{\fbackref{24:10} Lit. \fbib{people}} so David told the \divine{Lord}, ``I have sinned greatly by what I did. But now I am asking you, please remove the guilt of your servant, since I have acted very foolishly.''

\v{11}Before David arose the next morning, this message from the \divine{Lord} came to Gad, David's seer: \v{12}``Go tell David, `This is what the \divine{Lord} says: ``I'm holding three choices out for you: pick one of them for yourself, and I will do it to you.''\,'\,''

\v{13}So Gad went to David and asked him, ``Shall seven years of famine come to your land, or three months of reversals\fnote{\fbackref{24:13} Or \fbib{destruction}} while you flee from your enemies as they pursue you, or three days of pestilence in your land? Decide right now what I am to answer to the one who sent me.''

\v{14}So David replied to Gad, ``This is a very difficult choice for me to make! Let me now please fall into the hand of the \divine{Lord}, since his mercy is very great, but may I never fall into human hands!''

\v{15}That very morning, the \divine{Lord} sent a pestilence to Israel until the conclusion of the time designated, and 70,000 men\fnote{\fbackref{24:15} Or \fbib{soldiers}} died from Dan to Beer-sheba. \v{16}As the angel was stretching out his hand to destroy Jerusalem, the \divine{Lord} was grieved because of the calamity, so he told the angel who was afflicting the people, ``Enough! Stay your hand!'' So the angel of the \divine{Lord} remained near the threshing floor that belonged to Araunah\fnote{\fbackref{24:16} Araunah was also known as Ornan; cf. 1Chr 21:15} the Jebusite.\fnote{\fbackref{24:16} I.e. a descendant of Canaan's third son (cf. Gen 10:15-16), Jebusites were native to Jebus, the ancient name of the city of Jerusalem}

\v{17}When David saw the angel who had been attacking the people, he told the \divine{Lord}, ``Look, I'm the one who has sinned! I did the evil. These are only sheep! What did they do? Please, let your hand fall on me and on my household!''
\passage{David Buys Araunah's Threshing Floor}
\passageinfo{(1 Chronicles 21:18-27)}

\v{18}That very day, Gad approached David and told him, ``Go up and build an altar to the \divine{Lord} on the threshing floor that belongs to Araunah the Jebusite.'' \v{19}So David went up, just as Gad had ordered, consistent with the \divine{Lord}'s command.

\v{20}When Araunah looked down, he saw the king and his staff approaching him. Araunah went out, bowed down before the king with his face on the ground, \v{21}and asked\fnote{\fbackref{24:21} Lit. \fbib{and Araunah said}} him, ``Why has your majesty the king come to his servant?''

David replied, ``To purchase your threshing floor and to build an altar to the \divine{Lord}, so the pestilence can be averted from the people.''

\v{22}Araunah responded to David, ``May your majesty the king take it and offer whatever pleases him. Here are oxen for a burnt offering, along with the threshing sledges and yokes from the oxen for wood! \v{23}Your majesty, Araunah gives all of this\fnote{\fbackref{24:23} The Heb. lacks \fbib{of this}} to the king.'' Araunah also told the king, ``May the \divine{Lord} your God be pleased with you!''

\v{24}``No!'' the king replied to Araunah. ``I will buy them from you at full\fnote{\fbackref{24:24} The Heb. lacks \fbib{full}} price. I won't offer to the \divine{Lord} my God burnt offerings that cost me nothing.'' So David bought the threshing floor and the oxen for 50 silver shekels,\fnote{\fbackref{24:24} I.e. about one and one quarter pounds at 0.4 shekels per ounce} \v{25}built\fnote{\fbackref{24:25} Lit. \fbib{David built}} an altar to the \divine{Lord} there, and presented burnt offerings and peace offerings. So the \divine{Lord} answered David's prayers for the land\fnote{\fbackref{24:25} Cf. 2Sam 21:14} and the pestilence on Israel was averted.

\bookheader{1 Kings}
\labelbook{1King}

\bookpretitle{The Book of}
\booktitle{First Kings}

\labelchapt{1}
\passage{Adonijah's Attempted Coup}

\chapt{1}
\v{1}When David had grown very old, they covered him with blankets, but he could not keep warm, \v{2}so his servants suggested to him, ``Let's look for a young virgin woman to take care of you, your majesty. She will be of use to you if you have her lie down near you\fnote{Lit. \fbib{lie in your lap}} so that your majesty may keep warm.'' \v{3}So they conducted a search throughout the territory of Israel for a beautiful young woman, and Abishag the Shunammite was located and brought to the king. \v{4}The young woman was absolutely beautiful. She served the king and was very useful to him. The king was not sexually involved with her.

\v{5}Meanwhile, about this time Haggith's son Adonijah began to seek a reputation for himself and decided,\fnote{Lit. \fbib{said}} ``I'm going to be king!'' So he prepared chariots, cavalry, and 50 soldiers to serve as a security detail to guard him.\fnote{Lit. \fbib{soldiers to run ahead of him}} \v{6}His father had never challenged him at any time during his life by asking him, ``Why are you acting like this?'' Adonijah\fnote{Lit. \fbib{He}} was very handsome and had been born after Absalom. \v{7}He had the support of Zeruiah's son Joab and of Abiathar the priest, who followed Adonijah\fnote{Lit. \fbib{him}} and assisted him, \v{8}but Zadok the priest, Jehoiada's son Benaiah, Nathan the prophet, Shimei, Rei, and David's personal elite forces would have nothing to do with Adonijah.

\v{9}Adonijah sacrificed sheep, oxen, and fatted cattle by the Serpent Stone\fnote{Or \fbib{the stone of Omelet}} near En-rogel,\fnote{Cf. Josh 15:7; 18:16; 2Sam 17:17} inviting all of his relatives, the king's sons, and all of the men of Judah who worked for the king, \v{10}but he did not invite Nathan the prophet, Benaiah, David's\fnote{Lit. \fbib{the}} personal elite forces, or his brother Solomon.
\passage{Nathan and Bathsheba Confer about Adonijah}

\v{11}``Haven't you heard?'' Nathan asked Solomon's mother Bathsheba. ``Haggith's son Adonijah has become king and David, our true king,\fnote{Lit. \fbib{our lord}} isn't aware of it. \v{12}If you listen to me, you'll save your life and the life of your son Solomon. \v{13}Go right now to King David and ask him, `Your majesty, you promised your servant that ``Your son Solomon will certainly become king after me and will sit on my throne,'' didn't you? So why has Adonijah become king?' \v{14}Then, while you are still talking to the king, I'll come in after you and verify your statement.''

\v{15}So Bathsheba went to the king in his private room. Now the king was very old, and Abishag the Shunammite was attending to him.\fnote{Lit. \fbib{to the king}} \v{16}Bathsheba knelt and bowed down to the king, and the king asked her, ``What do you wish?''

\v{17}``Your majesty,'' she replied, ``you promised your servant in the name of\fnote{The Heb. lacks \fbib{the name of}} the \divine{Lord} your God, `Your son Solomon will certainly become king after me and will sit on my throne.' \v{18}Now look, Adonijah has become king, and your majesty is not aware of it. \v{19}Adonijah\fnote{Lit. \fbib{He}} has sacrificed myriads of oxen, fattened cattle, and sheep, and he has invited all of the king's sons, Abiathar the priest, and Joab the commander of the army, but he has not invited your servant Solomon. \v{20}And as for you, your majesty, everyone in Israel is looking to you to tell them who will sit on your majesty's throne after you.\fnote{Lit. \fbib{him}} \v{21}Otherwise, as soon as your majesty is laid to rest with his ancestors, my son Solomon and I will be branded as traitors.''\fnote{Lit. \fbib{sinners}}

\v{22}While she was still talking to the king, Nathan the prophet arrived. \v{23}They informed the king, ``Nathan the prophet is here.''

When he had been ushered into the presence of the king, Nathan bowed low in front of the king with his face to the ground \v{24}and asked, ``Your majesty, did you say `Adonijah will be king after me and will sit on my throne'? \v{25}Well now, he went down today and sacrificed lots of oxen, fattened cattle, and sheep, and has invited all the king's sons, the army commanders, and Abiathar the priest. They're having a party together and saying, `Long live King Adonijah!' \v{26}Of course, he never invited me, Zadok the priest, Jehoiada's son Benaiah, nor your servant Solomon. \v{27}Were you behind this, your majesty, without letting your servants know who would sit on your majesty's throne after him?''
\passage{David Affirms Solomon as King}

\v{28}``Call Bathsheba for me,'' King David replied. So she came in and stood in front of the king. \v{29}``As the \divine{Lord} lives,'' the king said with an oath, ``who has redeemed me from all sorts of troubles, \v{30}I certainly did tell you in the name of\fnote{The Heb. lacks \fbib{the name of}} the \divine{Lord} God of Israel, `Your son Solomon will be king after me and will sit on my throne in my place.' I'm certainly going to make this happen today!''

\v{31}``King David,'' Bathsheba said as she bowed low in front of the king with her face to the ground, ``your majesty, may you live forever.''

\v{32}``Get me Zadok the priest,'' King David said, ``along with Nathan the prophet, and Jehoiada's son Benaiah.'' So they were ushered into the king's presence \v{33}and David addressed them. ``Take your lord's servants, have my son Solomon ride on my own mule, and take him down to Gihon. \v{34}Have Zadok the priest and Nathan the prophet anoint him there as king over Israel. Then sound a trumpet and declare `Long live King Solomon!' \v{35}After this, you are to follow him back here, and he is to come and sit on my throne and take my place as king, because I've appointed him to be Commander-in-Chief\fnote{Lit. \fbib{Nagid}; i.e. a senior officer entrusted with dual roles of operational oversight and administrative authority} over Israel and Judah.''

\v{36}``Amen!'' replied Jehoiada's son Benaiah to the king. ``May the \divine{Lord} God of your majesty make this happen! \v{37}As the \divine{Lord} has been with your majesty the king, so may he be with Solomon. May he make his throne greater than the throne of your majesty, King David.''
\passage{Solomon is Anointed King}

\v{38}So Zadok the priest, Nathan the prophet, Jehoiada's son Benaiah, the special forces\fnote{Lit. \fbib{Cherethites}; i.e. elite body guards} and mercenaries\fnote{Lit. \fbib{Pelethites}; i.e. special couriers} went out and had Solomon ride the king's mule all the way to Gihon. \v{39}Zadok the priest brought from his tent a horn filled with oil and anointed Solomon, a trumpet was sounded, and everybody yelled out, ``Long live King Solomon!'' \v{40}All the people followed after him, playing on wind pipes and so full of joy that the earth shook because of all the noise!

\v{41}Right about then, Adonijah and all of his guests were just finishing their meal when they heard all the noise. ``Why is the city in such an uproar?'' Joab asked as he heard the trumpet sounds.

\v{42}While he was still asking that question, Jonathan, the son of Abiathar the priest arrived, so Adonijah told him, ``Come on in, since you're a worthy man and are bringing us good news!''

\v{43}``No,'' Jonathan answered. ``Our lord King David has installed Solomon as king. \v{44}The king has sent Zadok the priest, Nathan the prophet, Jehoiada's son Benaiah, the special forces\fnote{Lit. \fbib{Cherethites}; i.e. elite body guards} and mercenaries,\fnote{Lit. \fbib{Pelethites}; i.e. special couriers} along with Solomon, who is riding the king's personal mule. \v{45}Zadok the priest and Nathan the prophet have anointed him in Gihon, and they just left from there rejoicing, and that's why the city is all in an uproar. That's the noise that you've been hearing! \v{46}Solomon now sits on the royal throne. \v{47}In addition to all of this, the king's servants have come along to congratulate our lord King David. They've been telling David `May your God make Solomon's reputation even more famous than yours, and may he make his throne greater than yours!' The king has himself bowed in worship on his own bed\fnote{I.e. a possible allusion to sacred oaths such as Joseph's promise to Jacob in Gen 47:31} \v{48}and said `Blessed be the \divine{Lord} God of Israel, who has provided someone to sit on my throne today. I've seen it with my own eyes!'\,''

\v{49}Terrified, all of Adonijah's guests jumped up and ran away. \v{50}Afraid of Solomon, Adonijah also jumped up and headed straight for the horns of the altar.\fnote{I.e. the altar associated with sacrifices in the tent}

\v{51}``Hey look!'' somebody informed Solomon. ``Adonijah is terrified of King Solomon! He's gone out, grabbed hold of the horns of the altar, and now he's begging King Solomon, `Swear to me that you won't put your servant to death with a sword!'\,''

\v{52}``If he's done nothing wrong, not a hair of his head will be harmed,'' Solomon replied. ``But if we find evil in him, he's a dead man.''

\v{53}So King Solomon sent for him, and he was brought down from the altar. When he had arrived, he fell on his face in front of King Solomon, so Solomon told him, ``Go home!''
\labelchapt{2}
\passage{David Instructs Solomon}

\chapt{2}
\v{1}As David's time to die approached, he addressed his son Solomon with these words:

\begin{poetry}
\poeml \v{2}``I'm headed down the road that everyone who lives on earth travels, so be strong and demonstrate that you're a grown man \v{3}by keeping the charge that the \divine{Lord} your God entrusted to you. Live life his way, keep his statutes, his commands, his ordinances, and his testimonies, just as they're written down in the Law of Moses, so that you may succeed in everything you do and wherever you go,\fnote{Lit. \fbib{turn}} \v{4}and so that the \divine{Lord} may fulfill his promise that he spoke about me when he said, `If your sons pay attention to how they live by walking truthfully in my presence with all their heart and with all their soul, you will never lack a man on the throne of Israel.' \\
\poeml \v{5}``Furthermore, you're aware of what Zeruiah's son Joab did to me and to those two commanders of the armies of Israel, Ner's son Abner and Jether's son Amasa, whom he killed, and how he shed the blood of wartime during times of peace, staining the very belt he wears around his waist and the sandals he wears on his feet. \v{6}So act consistently with your wisdom, and don't let him die as a peaceful old man.\fnote{Lit. \fbib{let his gray hair descend to Sheol in peace}} \v{7}Be gracious to the descendants of Barzillai the Gileadite, and provide for them in your household,\fnote{Lit. \fbib{them at your table}} because they helped me when I had to run from your brother Absalom. \\
\poeml \v{8}``Pay attention now! You have with you Gera's son Shimei the descendant of Benjamin from Bahurim. He cursed me violently that day when I had to leave for Mahanaim. When he visited me at the Jordan River,\fnote{The Heb. lacks \fbib{River}} I made an oath to the \divine{Lord} and told him, `I won't execute you with a sword.' \v{9}But don't let him off unpunished, since you're a wise man and you'll know what you need to do to him. Find a way that he dies in his old age\fnote{Lit. \fbib{Bring his gray hair down to Sheol}} by shedding his blood.''
\end{poetry}
\passage{David Dies and Solomon Consolidates His Reign}
\passageinfo{(1 Chronicles 3:4; 29:26-28)}

\v{10}After this, David died, as had\fnote{Lit. \fbib{David slept with}} his ancestors, and he was buried in the City of David. \v{11}David had reigned over Israel for 40 years. He reigned in Hebron for seven years and in Jerusalem for 33 years. \v{12}Solomon then assumed his father David's throne, and his kingdom was firmly established.
\passage{Adonijah asks for Abishag}

\v{13}Later, Haggith's son Adonijah approached Solomon's mother. ``Are you here on a peaceful mission?'' she asked.

``Yes,'' he replied. \v{14}``I have something to ask you about.''

``Talk,'' she told him.

\v{15}So he replied, ``You know that the kingdom should have come to me, and that everyone in Israel intended to place me as the next\fnote{The Heb. lacks \fbib{the next}} king. However, the kingdom has turned around and now belongs to my brother, because it went to him from the \divine{Lord}. \v{16}So now I'm asking one thing from you. Don't refuse me.''

``Talk,'' she told him.

\v{17}Then he asked her, ``Please talk to King Solomon for me, since he won't refuse you. Ask him to give me Abishag the Shunammite as a wife.''

\v{18}``Very well,'' Bathsheba replied. ``I'll talk to the king for you.'' \v{19}So Bathsheba went to talk to King Solomon for Adonijah. The king rose to meet her, bowed to her, and sat down on his throne. He ordered a throne be set in place for his mother. She sat on a throne to his right \v{20}and told him,\fnote{The Heb. lacks \fbib{to him}} ``I would like to make a minor request of you. Please don't refuse me.''

``What is your request, mother?'' the king asked her. ``I won't turn you down.''

\v{21}So she asked him, ``Give Abishag the Shunammite to your brother Adonijah as a wife.''

\v{22}But King Solomon replied to his mother, ``Why are you asking Abishag the Shunammite for Adonijah? Why not ask me to give up the kingdom for him, since he's my older brother, and why not ask\fnote{The Heb. lacks \fbib{why not ask}} for Abiathar the priest, and for Zeruiah's son Joab?''

\v{23}Then King Solomon took this oath in the name of the \divine{Lord}: ``May God do so to me, and more besides, if Adonijah hasn't endangered his life by bringing up this subject. \v{24}Now therefore, as the \divine{Lord} lives, who has established me and set me on the throne of my father David, and who has established a dynasty, just like he promised, Adonijah will surely be executed today.'' \v{25}So King Solomon sent for Jehoiada's son Benaiah, who attacked and killed Adonijah.\fnote{Lit. \fbib{him}}

\v{26}The king also told Abiathar the priest, ``Go home to Anathoth. You deserve to die, but I won't kill you today, because you carried the ark of the Lord \divine{God} before my father David and because you shared all the troubles that my father went through.'' \v{27}So Solomon fired Abiathar as the \divine{Lord}'s priest, thus fulfilling the promise that the \divine{Lord} had spoken in Shiloh concerning Eli's household.\fnote{Cf. 1Sam 2:27-36}
\passage{Joab is Executed}

\v{28}When Joab learned what had happened, he ran to the \divine{Lord}'s tent and grabbed hold of the horns of the altar, since Joab had supported Adonijah (though he had not supported Absalom). \v{29}Somebody informed King Solomon, ``Joab just ran to the \divine{Lord}'s tent and now he's standing beside the altar!''

But Solomon ordered Jehoiada's son Benaiah, ``Go kill him!''

\v{30}So Benaiah went into the \divine{Lord}'s tent and told Joab,\fnote{Lit. \fbib{him}} ``The king orders you to come out!''

``No,'' Joab said, ``I'd rather die here!''

So Benaiah went and informed the king, ``This is how Joab answered me.''

\v{31}The king replied to him, ``Do just what he asked. Kill him and bury him so that you may remove from me and from my father's household the guilt that Joab shed needlessly. \v{32}The \divine{Lord} will repay him for his bloodshed because, without my father David's consent he attacked and murdered two men more righteous and better than he, Ner's son Abner, the commander of Israel's army and Jether's son Amasa, commander of Judah's army. \v{33}May their blood be repaid to Joab and to his descendants forever, and may there be peace shown from the \divine{Lord} forever to David, to his descendants, to his household, and to his throne.''

\v{34}Jehoiada's son Benaiah then approached Joab, attacked him, killed him, and had him buried at Joab's\fnote{Lit. \fbib{his}} home in the wilderness. \v{35}The king appointed Jehoiada's son Benaiah in charge of the army to replace Joab and also appointed Zadok the priest to replace Abiathar.
\passage{Shimei is Executed}

\v{36}The king sent for Shimei and told him, ``Build yourself a house in Jerusalem and live there, but don't go anywhere from there. \v{37}If you ever leave and cross the Kidron Brook, you can be sure that you'll die. You'll be responsible for your own death.''

\v{38}Shimei replied to the king, ``What your majesty has decreed is acceptable to me. I'll do what you've said.'' So Shimei lived in Jerusalem for quite some time. \v{39}But three years later, two of Shimei's servants escaped to Maacah's son Achish, the king of Gath.

Somebody told Shimei, ``Look! Your servants went to Gath!'' \v{40}So Shimei got up, saddled a donkey, and traveled to Gath to find his servants. He found them and brought them back from Gath.

\v{41}Later, Solomon found out that Shimei had left Jerusalem, gone to Gath, and had returned, \v{42}so the king sent for Shimei and asked him, ``Didn't I make a promise to the \divine{Lord} and warn you, `The day you leave and go anywhere else, you can be sure you'll die'? And you told me, `What your majesty has decreed is acceptable to me.' \v{43}So why haven't you kept the oath you made to the \divine{Lord}, and why didn't you obey my personal order to you?''

\v{44}The king also reminded Shimei, ``You know all the evil things that you admit you did to my father David. Therefore the \divine{Lord} is going to repay you for\fnote{Lit. \fbib{repay on your head}} all of your evil. \v{45}But King Solomon will be blessed, and David's throne will be established in the presence of the \divine{Lord} forever.'' \v{46}So the king gave orders to Jehoiada's son Benaiah to go out, attack Shimei, and kill him. That is how the kingdom was established under Solomon's control.
\labelchapt{3}
\passage{Solomon Prays for Wisdom}
\passageinfo{(2 Chronicles 1:2-13)}

\chapt{3}
\v{1}Later, Solomon intermarried with the family of\fnote{Lit. The Heb. lacks \fbib{the family of}} Pharaoh, the king of Egypt by taking his daughter and bringing her to the City of David to live until he had completed building his own palace, the \divine{Lord}'s Temple, and the wall around Jerusalem. \v{2}The people were sacrificing at various high places because the Temple had not yet been built and dedicated to\fnote{Lit. \fbib{built for the name of}} the \divine{Lord}.

\v{3}Solomon loved the \divine{Lord}, and lived according to the statutes that his father David obeyed, except that he sacrificed and burned offerings at the high places. \v{4}The king used to go to Gibeon to sacrifice, since there was a famous high place there, where Solomon once offered 1,000 burnt offerings on that altar. \v{5}The \divine{Lord} appeared to Solomon one night in a dream and told him, ``Ask me for whatever you want and I'll give it to you.''

\v{6}So Solomon said:

\begin{poetry}
\poeml ``You have demonstrated abundant gracious love to your servant David, my father, as he lived in your presence truthfully, righteously, and uprightly in his heart. In addition, you have kept on showing this abundant gracious love by giving him a son to sit on his throne today. \v{7}Now, \divine{Lord} my God, you have set me as king to replace my father David, but I'm still young. I don't have any leadership skills.\fnote{Lit. \fbib{I'm} only \fbib{a youth and don't know how to come and go}} \v{8}Your servant lives in the midst of your people that you have chosen, a great people that is too numerous to be counted. \v{9}So give your servant an understanding mind to govern your people, so I can discern between good and evil. Otherwise, how will I be able to govern this great people of yours?''
\end{poetry}

\v{10}The \divine{Lord} was pleased that Solomon had asked for this, \v{11}so God told him:

\begin{poetry}
\poeml ``Because you asked for this, and you didn't ask for a long life for yourself, and you didn't ask for the lives of your enemies, but instead you've asked for discernment so you can understand how to govern, \v{12}look how I'm going to do precisely what you asked. I'm giving you a wise and discerning mind, so that there will have been no one like you before you and no one will arise after you like you. \v{13}I'm also giving you what you haven't requested: both riches and honor, so that no other king will be comparable to you during your lifetime. \v{14}If you will live life my way, keeping my statutes and my commands, just like your father David did, I'll also increase the length of your life.''
\end{poetry}

\v{15}Then Solomon woke up and realized that he had dreamed a dream. Then he went back to Jerusalem, stood before the ark of the \divine{Lord}'s covenant, offered burnt offerings and peace offerings, and threw a party for all of his servants.
\passage{Solomon's Wisdom is Tested}

\v{16}Right about then, two prostitutes approached the king and requested an audience with him. \v{17}One woman said, ``Your majesty, this woman and I live in the same house. I gave birth to a child while she was in the house. \v{18}Three days later, this woman also gave birth. We lived alone there. There was nobody else with us in the house. It was just the two of us. \v{19}This woman's son died overnight because she laid on top of him. \v{20}She got up in the middle of the night, took my son from me while your servant was asleep, and laid him to her breast after laying her dead son next to me. \v{21}The next morning, I got up to nurse my son, and he was dead. But when I examined him carefully in the light of day, he turned out not to be my son whom I had borne!''

\v{22}``Not so,'' claimed the other woman. ``The living child is my son, and the dead one is yours.''

But the first woman said, ``Not so! The dead child is your son and the living one is my son.'' This is what they testified before the king.

\v{23}The king said, ``One of them claims, `This living son is mine, and your son is the dead one' and the other claims `No. Your son is the dead one and my son is the living one.' \v{24}``Somebody get me a sword.'' So they brought a sword to the king. \v{25}``Divide the living child in two!'' he ordered. ``Give half to the one and half to the other.''

\v{26}The woman whose child was still alive cried out to the king, because her heart yearned for her son. ``Oh no, your majesty!'' she said. ``Give her the living child. Please don't kill him.''

But the other woman said, ``Cut him in half! That way, he'll belong to neither one of us.''

\v{27}The king announced his decision: ``Give the living child to the first woman. Don't kill him. She is his mother.'' \v{28}When this decision that the king had handed down was announced, everybody in Israel was amazed at\fnote{Lit. \fbib{Israel feared}} the king, because they all saw that God's wisdom was in him, enabling him to administer justice.
\labelchapt{4}
\passage{Solomon's Administration}

\chapt{4}
\v{1}And so King Solomon ruled over all of Israel. \v{2}Here's a list of his officials: Zadok's son Azariah was priest, \v{3}Shisha's sons Elihoreph and Ahijah were his secretaries, Ahilud's son Jehoshaphat was recorder, \v{4}Jehoiada's son Benaiah commanded the army, Zadok and Abiathar served as priests, \v{5}Nathan's son Azariah supervised the governors, Nathan's son Zabud the priest was the king's counselor, \v{6}Ahishar supervised palace matters, and Abda's son Adoniram supervised conscripted labor. \v{7}Solomon also appointed twelve governors over all of Israel, each of whom were responsible for providing one month's food provisions to the king and to his administration during each year.

\v{8}Here's a list of their names: Ben-hur from the hill country of Ephraim; \v{9}Ben-deker in Makaz, Shaalbim and Beth-shemesh and Elonbeth-hanan; \v{10}Ben-hesed served in Arubboth (where he supervised Socoh and all of the territory of Hepher); \v{11}Ben-abinadab supervised the Dor heights (Solomon's daughter Taphath was his wife); \v{12}Ahilud's son Baana served Taanach, Megiddo, and all of Beth-shean near Zarethan below Jezreel, including from Beth-shean to Abel-meholah as far as the other side of Jokmeam; \v{13}Ben-geber in Ramoth-gilead, including the towns that belonged to Manasseh's descendant Jair that are in Gilead; \v{14}Iddo's son Ahinadab served in Mahanaim; \v{15}Ahimaaz served in Naphtali (he was married to Solomon's daughter Basemath); \v{16}Hushai's son Baana served in Asher and Bealoth; \v{17}Paruah's son Jehoshaphat served in Issachar; \v{18}Ela's son Shimei served in Benjamin; \v{19}and Uri's son Geber served in the territory of Gilead, the territory formerly ruled by King Sihon of the Amorites and King Og of Bashan (he was the only governor over that territory).
\passage{Solomon's Magnificence}

\v{20}Judah and Israel became as numerous as the sand on the seashore. They enjoyed abundance, and ate, drank, and rejoiced regularly. \v{21}\fnote{This v. is 5:1 in MT, 4:22 is 5:2, and so on through 4:34}Solomon ruled over all the kingdoms from the Euphrates River\fnote{The Heb. lacks \fbib{River}} to the territory of the Philistines and south\fnote{The Heb. lacks \fbib{south}} to the border of Egypt. They brought tribute and served Solomon throughout his lifetime. \v{22}Solomon's daily provisions were 30 kors of fine flour, 60 kors of meal, \v{23}ten fattened oxen, 20 pasture-fed cattle, 100 sheep, as well as deer, gazelles, roebucks, and domestic poultry. \v{24}He ruled over everything west of the Euphrates\fnote{The Heb. lacks \fbib{Euphrates}} River from Tiphsah to Gaza, over all of the kings west of the Euphrates\fnote{The Heb. lacks \fbib{Euphrates}} River, and he enjoyed peace on all sides around him.

\v{25}Judah and Israel lived safely, and everyone enjoyed their own vine and fig tree from Dan to Beer-sheba through all of Solomon's life. \v{26}Solomon owned 40,000 stalls for the horses that drove his chariots, and he employed 12,000 men to drive them.\fnote{The Heb. lacks \fbib{to drive them}} \v{27}His officers supplied provisions for King Solomon and for everyone who visited King Solomon's palace,\fnote{Lit. \fbib{table}} each in their respective month of service responsibility.\fnote{The Heb. lacks \fbib{of service responsibility}} Nothing ever ran out. \v{28}They also provided barley and straw for the horses and camels to their respective locations, each consistent with their responsibilities.
\passage{Solomon's Fame}

\v{29}God gave Solomon wisdom and great discernment. His insights were as numerous as sand on the seashore. \v{30}Solomon was wiser than any of the eastern leaders and wiser than anyone in Egypt. \v{31}He was wiser than anyone of his day---wiser than Ethan the Ezrahite, Heman, and wiser than Mahol's sons Calcol and Darda.

His reputation was known throughout the surrounding nations. \v{32}Solomon wrote 3,000 proverbs and 1,005 songs. \v{33}He described trees---everything from cedars\fnote{I.e. a genus of coniferous evergreen in the family \fbib{Pinaceae}; and so throughout the book} that grow in Lebanon to hyssop that grows on a garden wall. He described animals, birds, reptiles, and fish. \v{34}People came from everywhere to hear Solomon's advice. Every king on the earth heard of his wisdom.
\labelchapt{5}
\passage{Preparations to Build the Temple}
\passageinfo{(2 Chronicles 2:1-18)}

\chapt{5}
\v{1}\fnote{This v. is 4:15 in MT, 5:2 is 4:16, and so on through 5:18, which is 4:32 in MT}King Hiram of Tyre sent his servants to Solomon when he learned that Solomon\fnote{Lit. \fbib{he}} had been anointed king to replace his father, because Hiram had been David's lifelong friend.\fnote{Lit. \fbib{David's friend all his days}} \v{2}Solomon sent this message to Hiram:

\begin{poetry}
\poeml \v{3}``You know that my father David was unable to build a temple dedicated to\fnote{Lit. \fbib{temple to the name of}} the \divine{Lord} his God because he was busy fighting wars all around him until the \divine{Lord} defeated his enemies. \v{4}But now the \divine{Lord} has given me rest all around, since I have neither foreign adversaries nor domestic crises. \v{5}So now I'm planning to build a temple dedicated to\fnote{Lit. \fbib{temple to the name of}} the \divine{Lord} my God, just as the \divine{Lord} told my father when he said, `Your son, whom I will set on your throne to replace you, will build the Temple dedicated to me.'\fnote{Lit. \fbib{to my name}} \v{6}Now therefore please order that cedars of Lebanon be cut for me. My servants will work with your servants, and I will pay your servants whatever wages you set, because you know there is no one among us who knows how to cut timber like the Sidonians do.''
\end{poetry}

\v{7}As soon as Hiram received the message from Solomon, he became so ecstatic that he exclaimed, ``Blessed be the \divine{Lord} today, who has given David a wise son to rule this great people!'' Then he sent this message to Solomon:

\begin{poetry}
\poeml \v{8}``I have read the letter that you sent me. I'll do what you've asked about the cedar and cypress timber. \v{9}My servants will transport them from Lebanon to the sea, where we'll make them into rafts and float them by sea to the port that you tell me to send them. We'll have them prepared for transport there and then you can carry them from there. You can meet my needs by providing provisions for my household.''
\end{poetry}

\v{10}That's how Hiram came to provide Solomon as much cedar and cypress timber as he needed. \v{11}In return, Solomon paid Hiram 20,000 kors of wheat as food for his household, and 20 kors of beaten oil. Solomon provided this amount every year during the construction.\fnote{The Heb. lacks \fbib{during the construction}}

\v{12}The \divine{Lord} continued giving Solomon wisdom, just as he had promised, and Hiram and Solomon entered into a peace treaty between themselves.
\passage{Conscripted Labor for the Building Program}

\v{13}King Solomon conscripted laborers from throughout Israel. The work force numbered 30,000 men. \v{14}He sent 10,000 men to Lebanon in shifts lasting one month. They worked one month in Lebanon for every two months they worked at home. Adoniram was placed in charge of the conscripted labor. \v{15}Solomon also employed 70,000 heavy-lift workers and 80,000 stonecutters in the hill country. \v{16}Solomon also employed 3,300 officials to supervise the work and to manage the people employed in the construction. \v{17}The king specified that large, expensive stones be quarried so the foundation of the Temple could be laid with cut stones. \v{18}As a result, Solomon's builders worked with Hiram's builders, accompanied by the Gebalites, to quarry the stone and to prepare the timber and other\fnote{The Heb. lacks \fbib{other}} stone for the Temple's construction.
\labelchapt{6}
\passage{Temple Construction Begins}
\passageinfo{(2 Chronicles 3:1-14)}

\chapt{6}
\v{1}During the month of Ziv, which was the second month of the fourth year of Solomon's reign over Israel, 480 years after the Israelis left the land of Egypt, Solomon began to build the \divine{Lord}'s Temple. \v{2}The Temple for the \divine{Lord} that Solomon was building was 60 cubits\fnote{I.e. about 90 feet; a cubit was about eighteen inches} long and 20 cubits\fnote{I.e. about 30 feet; a cubit was about eighteen inches} wide. \v{3}A portico extended in front of the Temple for 20 cubits\fnote{I.e. about 30 feet; a cubit was about eighteen inches} outward, corresponding to the width of the Temple. Along the front of the Temple its depth was ten cubits.\fnote{I.e. about fifteen feet; a cubit was about eighteen inches} \v{4}Solomon\fnote{Lit. \fbib{He}} also constructed windows in the Temple with specially designed\fnote{Or \fbib{latticed}} frames.

\v{5}Against the wall of the Temple he built a series of rooms that encompassed the exterior of the Temple walls around the inner sanctuary. He built these side chambers all around the building.\fnote{The Heb. lacks \fbib{the building}} \v{6}The lower structures were five cubits\fnote{I.e. about seven and a half feet; a cubit was about eighteen inches} wide, the middle structures were six cubits\fnote{I.e. about nine feet; a cubit was about eighteen inches} wide and the third structures were seven cubits\fnote{I.e. about ten and a half feet; a cubit was about eighteen inches} wide. Offsets were placed all around the Temple so that beams would not protrude through the walls of the Temple. \v{7}The Temple was constructed of stone precut at the quarry so that no hammer, axe, or any other iron implement would be heard in the Temple while it was being built. \v{8}A passageway to the side chamber was constructed on the south side of the Temple by which people\fnote{Lit. \fbib{they}} could ascend winding stairs to the middle story, then from there to the third story.
\passage{Interior Finishing with Gold and Cedar}

\v{9}After Solomon\fnote{Lit. \fbib{he}} built the Temple and finished it, he covered the Temple with beams and planks made of cedar. \v{10}He constructed this structure to adjoin the entire Temple, five cubits\fnote{I.e. about seven and a half feet; a cubit was about eighteen inches} high, and fastened it to the Temple with cedar timbers.

\v{11}Then this message from the \divine{Lord} came to Solomon: \v{12}``Concerning\fnote{The Heb. lacks \fbib{Concerning}} this Temple that you're building, if you live your life\fnote{Lit. \fbib{you walk}} according to my statutes, carry out my ordinances, and keep all of my commands, and live according to them, then I will do what I promised to your father David. \v{13}I will reside among the Israelis and will never abandon my people Israel.''

\v{14}So Solomon kept on building the Temple and finished it. \v{15}Then he built the inside walls of the Temple, lining them from floor to ceiling with cedar boards, and overlaying the Temple floor with boards made of cypress wood. \v{16}He lined 20 cubits\fnote{I.e. about 30 feet; a cubit was about eighteen inches} of the rear part of the Temple from floor to ceiling with cedar boards specially constructed for the inside to serve as the Most Holy Place. \v{17}The rest of the main nave in the front was 40 cubits\fnote{I.e. about 60 feet; a cubit was about eighteen inches} long. \v{18}Cedar\fnote{I.e. a genus of coniferous evergreen in the family \fbib{Pinaceae}; and so throughout the book} carvings in the form of gourds and blooming flowers covered the entire interior of the Temple so that no stone could be seen.

\v{19}Solomon\fnote{Lit. \fbib{He}} also prepared an inner sanctuary within the Temple where the \divine{Lord}'s Ark of the Covenant was placed. \v{20}The inner sanctuary was 20 cubits\fnote{I.e. about 30 feet; a cubit was about eighteen inches} long, 20 cubits\fnote{I.e. about 30 feet; a cubit was about eighteen inches} wide, and 20 cubits\fnote{I.e. about 30 feet; a cubit was about eighteen inches} high, and overlaid with pure gold. The altar was also overlaid with cedar. \v{21}Solomon overlaid the inside of the Temple with pure gold, fastened gold chains across the front of the inner sanctuary, and overlaid it with gold. \v{22}He finished the Temple by overlaying it entirely with gold, including overlaying with gold the whole altar that was by the inner sanctuary.
\passage{Temple Furnishings}
\passageinfo{(2 Chronicles 4:1-10, 19-22; 5:1)}

\v{23}Inside the inner sanctuary Solomon\fnote{Lit. \fbib{he}} placed two cherubim crafted from olive wood, each ten cubits\fnote{I.e. about fifteen feet; a cubit was about eighteen inches} high. \v{24}Each wing of one cherub was five cubits\fnote{I.e. about seven and a half feet; a cubit was about eighteen inches} long, and each wing of the other cherub was five cubits\fnote{I.e. about seven and a half feet; a cubit was about eighteen inches} long, so that the distance from the end of one wing to the end of the other wing was ten cubits.\fnote{I.e. about fifteen feet; a cubit was about eighteen inches} \v{25}Each cherub was ten cubits\fnote{I.e. about fifteen feet; a cubit was about eighteen inches} high, and both were of the same size and shape, \v{26}the height of one cherub being ten cubits,\fnote{I.e. about fifteen feet; a cubit was about eighteen inches} as was the height of the other.

\v{27}Solomon\fnote{Lit. \fbib{He}} placed the cherubim in the middle of the inner sanctuary, with their wings spread in such a way that the wing of one was touching the one wall and the opposite wing of the other cherub was touching the opposite wall. Furthermore, their wings in the center of the wall were touching each other wing-to-wing. \v{28}Each cherub was overlaid with gold.

\v{29}Solomon\fnote{Lit. \fbib{He}} also inlaid all the inner walls of the Temple---both the inner and outer sanctuaries---with carved engravings of cherubim, palm trees, and blooming flowers. \v{30}He also overlaid the floor of the Temple with gold in both the inner and outer sanctuaries.

\v{31}Solomon\fnote{Lit. \fbib{He}} also provided doors, lintels, and five-sided doorposts for the entrance to the inner sanctuary. \v{32}He installed two doors made of olive wood, inlaying them with carvings of cherubim, palm trees, and blooming flowers, and overlaying them with gold. Then he added more gold to cover the cherubim and palm trees.

\v{33}Solomon\fnote{Lit. \fbib{He}} also provided four-sided doorposts made of cypress wood for the entrance to the outer sanctuary, \v{34}along with two doors of cypress wood, one door of which had two leaves that turned on hinges, as did the other door, which also had two leaves that turned on hinges.

\v{35}Solomon\fnote{Lit. \fbib{He}} also inlaid the doors with\fnote{The Heb. lacks \fbib{the doors with}} cherubim, palm trees, and blooming flowers. He overlaid them with gold that was carefully\fnote{Or \fbib{evenly}} applied on the engraved work. \v{36}He constructed the inner court with three rows of precut stone and a row of cedar beams.
\passage{Temple Construction is Completed}

\v{37}The foundation for the \divine{Lord}'s Temple was laid in the month of Ziv during the fourth year of Solomon's reign, \v{38}and the Temple was completely finished according to its plans and specifications in the eighth month of the eleventh year of Solomon's\fnote{Lit. \fbib{his}} reign, that is, during the month of Bul. It took about seven years to build.
\labelchapt{7}
\passage{Solomon's Palace}

\chapt{7}
\v{1}But Solomon took thirteen years to build his own palace, and finally finished it. \v{2}He built his own palace out of timber supplied from the forest of Lebanon. It was 100 cubits\fnote{I.e. about 150 feet; a cubit was about eighteen inches} long, 50 cubits\fnote{I.e. about 75 feet; a cubit was about eighteen inches} wide, 20 cubits\fnote{I.e. about 30 feet; a cubit was about eighteen inches} tall, and was constructed on four rows of cedar pillars, with cedar beams interlocking the pillars. \v{3}There were 45 pillars paneled with cedar above the side chambers, with rows of fifteen pillars, \v{4}with three rows of framed windows facing each other in three ranks. \v{5}All the doorways and doorposts had rectangular frames, with the doorways facing each other in three tiers. \v{6}There was also a hall of pillars 50 cubits\fnote{I.e. about 75 feet; a cubit was about eighteen inches} long and 30 cubits\fnote{I.e. about 45 feet; a cubit was about eighteen inches} wide, and a porch in front with pillars, and a canopy in front of the pillars.\fnote{Lit. \fbib{of them}} \v{7}He constructed the Judgment Hall for the throne room where he would be ruling, paneling it with cedar from floor to ceiling.\fnote{Lit. \fbib{floor to floor}} \v{8}Solomon's\fnote{Lit. \fbib{His}} personal dwelling quarters, a separate court behind the hall, was of similar workmanship. Solomon\fnote{Lit. \fbib{He}} also built a house similar to this for Pharaoh's daughter, whom Solomon had married.

\v{9}All of these were made with expensive stones, pre-cut according to specifications, hand-sawed inside and out from the foundation to the coping, including from inside to the great court. \v{10}The foundation was made of expensive stone, including large stones ten cubits\fnote{I.e. about 15 feet; a cubit was about eighteen inches} long and stones eight cubits\fnote{I.e. about 12 feet; a cubit was about eighteen inches} long. \v{11}Above these were expensive stones cut according to specifications, and cedar. \v{12}So the great court was surrounded by three rows of cut stone, along with a row of cedar beams, just like the inner court of the \divine{Lord}'s Temple and the porch surrounding the Temple.
\passage{Contributions by Hiram the Bronzeworker}
\passageinfo{(2 Chronicles 3:15-17; 4:11-18)}

\v{13}King Solomon sent for Hiram\fnote{2Chr 2:13 identifies the man as \fbib{Hiram-abi}} from Tyre, \v{14}the son of a widow from the tribe of Naphtali, whose father was from Tyre. A bronze worker, he was wise, knowledgeable, and was skilled in all sorts of bronze working. He went to King Solomon and did all of his work.

\v{15}He fashioned two bronze pillars, each one eighteen cubits\fnote{I.e. about 27 feet; a cubit was about eighteen inches} high, with a circumference of twelve cubits.\fnote{I.e. about 18 feet; a cubit was about eighteen inches} \v{16}He also crafted two capitals of cast bronze and set them on top of the pillars. The height of one capital was five cubits,\fnote{I.e. about seven and a half feet; a cubit was about eighteen inches} and the height of the other capital was five cubits.\fnote{I.e. about seven and a half feet; a cubit was about eighteen inches} \v{17}A network of latticework on top of the pillars was inlaid with ornamental wreaths and chains, the top of each pillar containing seven groups of ornamental structures. \v{18}The pillars contained two rows of ornaments shaped like pomegranates around the latticework covering the top of each pillar. \v{19}The capitals on top of each pillar above the rounded latticework contained four cubits\fnote{I.e. about six feet; a cubit was about eighteen inches} of lily designs, \v{20}with the capitals on the two pillars covered by 200 pomegranates in rows around both the capitals above and adjoining the rounded latticework. \v{21}That's how he designed the pillars at the portico of the sanctuary. When he set up the right pillar, he named it Jachin.\fnote{The name means \fbib{He Established}} When he set up the left pillar, he named it Boaz.\fnote{The name means \fbib{In Strength}} \v{22}The work on the pillars was finished with a lily design on top of the pillars.
\passage{The Bronze Sea}

\v{23}Hiram\fnote{Lit. \fbib{He}} also made a sea of cast metal ten cubits\fnote{I.e. about fifteen feet; a cubit was about eighteen inches} from brim to brim, circular in shape and five cubits\fnote{I.e. about seven and a half feet; a cubit was about eighteen inches} and 30 cubits\fnote{I.e. 45 feet; a cubit was about eighteen inches} in its inner circumference. \v{24}Under the brim, completely encircling it, were two rows of gourds inlaid as part of the original casting, ten to a cubit.\fnote{I.e. ten in each one and a half feet; a cubit was about eighteen inches} \v{25}The sea stood on top of twelve oxen. Three faced north, three faced west, three faced south, and three faced east. The sea was set on top of them, and their hind parts faced the center.\fnote{Lit. \fbib{were inward}} \v{26}The reservoir, which held about 2,000 baths,\fnote{I.e. about 12,000 gallons; Cf. 2Chron 4:52, where the volume is given at 3,000 baths} stood about a handbreadth\fnote{I.e. about three inches; a handbreadth was about one sixth of a cubit} thick, and its rim looked like the brim of a cup or of a lily blossom.
\passage{The Ten Water Carts}

\v{27}Hiram\fnote{Lit. \fbib{He}} also made ten bronze water carts.\fnote{Or \fbib{stands}, and so throughout this paragraph} Each one was four cubits\fnote{I.e. about six feet; a cubit was about eighteen inches} wide, four cubits long,\fnote{I.e. about six feet; a cubit was about eighteen inches} and three cubits\fnote{I.e. about four and a half feet; a cubit was about eighteen inches} high. \v{28}The carts were designed with borders between cross-pieces, \v{29}and on the borders between the cross-pieces were lions, oxen, and cherubim. A pedestal was placed above the cross-pieces, and beneath the lions and oxen there were wreaths hanging down. \v{30}Each cart had four bronze wheels equipped with bronze axles with four support feet. Beneath the basin were cast support structures made like wreaths on each side. \v{31}The opening to each water cart inside the crown on top was one cubit\fnote{I.e. about one and a half feet; a cubit was about eighteen inches} wide, with engravings on the opening. The borders to the frames surrounding the opening were square, not round. \v{32}The four wheels were placed underneath the borders, and the axles for the wheels were on the stand. Each wheel stood one and a half cubits\fnote{I.e. about 27 inches; a cubit was about eighteen inches} high. \v{33}The wheels resembled those of a chariot, with their axles, rims, spokes, and hubs made of cast bronze. \v{34}Four supports stood at the four corners of each cart, built into the carts themselves. \v{35}On top of each stand was a circular structure one half of one cubit\fnote{I.e. about 9 inches; a cubit was about eighteen inches} high, with its braces and support frames integral with it, forming a single piece. \v{36}Hiram\fnote{Lit. \fbib{He}} engraved ornamental cherubim, lions, and palm trees on the surfaces of the supports and frames wherever there was space to do so, and encircled the artwork with wreaths. \v{37}He made ten identical water carts by using the same plans, castings, and shapes for all of them.
\passage{The Other Bronze Implements}

\v{38}Hiram\fnote{Lit. \fbib{He}} also fashioned ten bronze basins, each holding about 40 baths,\fnote{I.e. about 240 gallons; a bath held about six gallons} each basin measuring four cubits\fnote{I.e. about six feet; a cubit was about eighteen inches} in diameter,\fnote{The Heb. lacks \fbib{in diameter}} with one basin for each stand. \v{39}He set five of the stands on the right side of the Temple and five on the left side of the Temple. He set the bronze sea on the right side of the Temple eastward facing the south. \v{40}Hiram also made the basins, shovels, and bowls to complete the work that he performed for King Solomon in the \divine{Lord}'s Temple, \v{41}including the two pillars and the bowls for the capitals that stood on top of the two pillars, along with the two lattices that covered the two bowls of the capitals that stood on top of the pillars, \v{42}plus the 400 pomegranates for the two lattices (that is, the two rows of pomegranates for each lattice to cover the two bowls of the capitals that stood on top of the pillars), \v{43}the ten stands with the ten basins on the stands, \v{44}the single bronze\fnote{The Heb. lacks \fbib{bronze}} sea and the twelve oxen that stood under the sea, \v{45}and the pots, shovels, and bowls---all of these utensils that Hiram made for King Solomon for the \divine{Lord}'s Temple were made from polished bronze.

\v{46}The king had them cast in the clay ground between Succoth and Zarethan in the Jordan plain. \v{47}Solomon never inventoried the weight of the bronze used, because there were too many utensils, so the weight of the bronze used was never ascertained. \v{48}Solomon made all the furnishings that were placed in the \divine{Lord}'s Temple, including the golden altar and the golden table on which the bread of the Presence was placed, \v{49}along with the lamp stands (five on the right side and five on the left in front of the inner sanctuary), all made of pure gold, as well as the flower blossoms, lamps, and tongs of gold, \v{50}and the cups, snuffers, bowls, spoons, and the fire pans, all made of pure gold, and hinges for the doors of the inner sanctuary, the Most Holy Place, and for the gates of the Temple that led to the nave, also of gold.

\v{51}Thus all the work that King Solomon performed in the \divine{Lord}'s Temple was finished. Then Solomon brought in the articles that had been dedicated by his father David, including silver, gold, and other utensils, and he placed them into storage in the treasuries of the \divine{Lord}'s Temple.
\labelchapt{8}
\passage{The Temple is Dedicated}
\passageinfo{(2 Chronicles 5:2-6:2)}

\chapt{8}
\v{1}Then Solomon gathered together the elders of Israel, including all the heads of the tribes and the leaders of the ancestral households of the Israelis, to meet with him in Jerusalem so they could bring up the Ark of the Covenant of the \divine{Lord} from Zion, the City of David. \v{2}So all the men gathered together to meet with King Solomon at the Festival of Tents\fnote{The Heb. lacks \fbib{of Tents}; cf. Lev 23:34} in the month Ethanim, the seventh month. \v{3}All the Elders of Israel showed up, and the priests picked up the ark \v{4}and brought it, the Tent of Meeting, and all the holy implements that were in the tent. The priests and descendants of Levi carried them up to Jerusalem.\fnote{The Heb. lacks \fbib{to Jerusalem}}

\v{5}King Solomon and the entire congregation of Israel that had assembled to be with him stood in front of the ark, sacrificing so many sheep and oxen that they were neither counted nor inventoried. \v{6}After this, the priests brought the Ark of the Covenant of the \divine{Lord} to the place prepared for it, into the inner sanctuary of the Temple, under the wings of the cherubim in the Most Holy Place. \v{7}The wings of the cherubim spread over the resting place for the ark, so that the cherubim made a covering over the ark and its poles when viewed\fnote{The Heb. lacks \fbib{when viewed}} from above. \v{8}The poles extended so far that their ends could be seen from the Holy Place in front of the inner sanctuary, but they could not be seen from outside. They remain there to this day. \v{9}The ark was empty except for the two stone tablets that Moses had placed there at Horeb when the \divine{Lord} had made a covenant with the Israelis after they had come out of the land of Egypt. \v{10}When the priests left the Holy Place after setting the ark in place,\fnote{The Heb. lacks \fbib{after setting the ark in place}} the cloud filled the \divine{Lord}'s Temple \v{11}so that the priests could not stand to minister because of the cloud, since the glory of the \divine{Lord} filled the \divine{Lord}'s Temple.
\passage{Solomon's Speech of Dedication}
\passageinfo{(2 Chronicles 6:3-11)}

\v{12}Then Solomon said, ``The \divine{Lord} has said that he lives shrouded in darkness. \v{13}Now I have been constructing a magnificent Temple dedicated to you that will serve as a place for you to inhabit forever.''

\v{14}Then the king turned to face the entire congregation of Israel while the congregation of Israel remained standing. \v{15}Then Solomon\fnote{Lit. \fbib{He}} prayed:

\begin{poetry}
\poeml ``Blessed is the \divine{Lord} God of Israel, who made a commitment\fnote{Lit. \fbib{who spoke by his mouth}} to my father David and then personally\fnote{Lit. \fbib{and by his hand}} fulfilled what he had promised when he said:\fnote{Cf. 1Chr 17:5ff}
\end{poetry}

\begin{poetry}
\poeml \v{16}`From the day I brought out my people Israel from Egypt I never chose a city from all the tribes of Israel to build a temple where my name might reside. I have chosen David to be over my people Israel.'
\end{poetry}

\begin{poetry}
\poeml \v{17}``My father David wanted to build a temple for the name of the \divine{Lord} God of Israel. \v{18}The \divine{Lord} told my father David: \\
\poeml `Therefore, since you determined\fnote{Lit. \fbib{since it was in your heart}} to build a temple for my name, you acted well, because it was your choice\fnote{Lit. \fbib{because it was in your heart}} to do so. \v{19}Nevertheless, you are not to build the Temple, but your son who will be born\fnote{Lit. \fbib{will come from your loins}} to you is to build a temple for my name.' \\
\poeml \v{20}``The \divine{Lord} has brought to fulfillment\fnote{Lit. \fbib{has caused to stand up}} what he promised, and now here I stand,\fnote{MT verb is a pun on the verb \fbib{brought to fulfillment}} having succeeded my father David to sit on the throne of Israel, as the \divine{Lord} promised. I have built the Temple for the name of the \divine{Lord} God of Israel. \v{21}I have placed there the ark in which the covenant is stored that the \divine{Lord} made with our ancestors when he brought them out of the land of Egypt.''
\end{poetry}
\passage{Solomon's Prayer of Dedication}
\passageinfo{(2 Chronicles 6:12-43)}

\v{22}Then Solomon took his place in front of the \divine{Lord}'s altar in the presence of the entire congregation of Israel, spread out his hands toward heaven, \v{23}and said:

\begin{poetry}
\poeml ``\divine{Lord} God of Israel, there is no one like you, God in heaven above or on the earth below, who watches over\fnote{Or \fbib{who keeps}} his covenant, showing gracious love to your servants who live their lives in your presence\fnote{Lit. \fbib{who walk before you}} with all their hearts. \v{24}It is you, \divine{Lord} God,\fnote{The Heb. lacks \fbib{It is you, \divine{Lord} God}} who have kept your promise to my father, your servant David, that you made to him. Indeed, you made a commitment\fnote{Lit. \fbib{you spoke by your mouth}} to my father David and then personally fulfilled\fnote{Lit. \fbib{and by your hand full}} what you had promised today. \\
\poeml \v{25}``Now therefore, \divine{Lord} God of Israel, keep your promise that you made\fnote{Lit. \fbib{spoke}} to my father, your servant David, when you said, `You will not lack a man to sit on the throne of Israel,\fnote{Cf. 1King 2:4; 2Chr 7:18} if only your descendants will watch their lives,\fnote{Lit. \fbib{ways}} to live\fnote{Lit. \fbib{walk}} in my presence just as you have lived\fnote{Lit. \fbib{walked}} in my presence.'\fnote{Or \fbib{have walked before me}} \\
\poeml \v{26}``Now therefore, God of Israel, may your promise that you made\fnote{Lit. \fbib{spoke}} to your servant David my father be fulfilled{\ldots} \v{27}and yet, will God truly reside on earth? Look! Neither the sky nor the highest heaven can contain you! How much less this Temple that I have built! \v{28}Pay attention to the prayer of your servant and to his request, \divine{Lord} my God, and listen to the cry and prayer that your servant is praying in your presence today. \v{29}Let your eyes always look toward this Temple night and day, toward the location where you have said `My name will reside there.' Listen to the prayer that your servant prays in this direction.\fnote{Lit. \fbib{prays toward this place}} \v{30}Listen to the requests from your servant and from your people Israel as they pray in this direction,\fnote{Lit. \fbib{pray toward this place}} listen from the place where you reside in heaven, then hear and forgive. \\
\poeml \v{31}``If a man should sin against his neighbor and he is required to take an oath, and he then comes to take an oath in front of your altar in this Temple, \v{32}then listen in heaven, act, and judge your servants, condemning the wicked by bringing back to him the consequences of his choices\fnote{Lit. \fbib{by bringing his way upon his head}} and by justifying the righteous by recompensing him according to his righteousness. \\
\poeml \v{33}``If your people Israel are defeated in a battle with\fnote{Lit. \fbib{defeated before}} their enemy because they have sinned against you, when they return to you and confess to you,\fnote{Lit. \fbib{confess your name}} pray, and in this Temple they ask you to show grace to them, \v{34}then hear in heaven, forgive the sin of your people Israel, and return them to the soil\fnote{Or \fbib{land}} that you gave to their ancestors. \\
\poeml \v{35}``When heaven remains closed, and there is no rain because they have sinned against you, and they pray in the direction of this place, confessing your name and turning from their sin when you afflict them,\fnote{So MT; LXX reads \fbib{you bring them low}} \v{36}then hear in heaven and forgive the sin of your servants and of your people Israel. Indeed, teach them the best way to live and send rain on your land that you have given to your people as an inheritance. \\
\poeml \v{37}``If a famine comes to the land, or if plant diseases, mildew, locust, or grasshoppers\fnote{Or \fbib{caterpillars}} appear, or if their enemies attack them in their settlements of the land, no matter what the epidemic or illness is, \v{38}whatever prayer or request is made, no matter whether it's made by a single man or by all of your people Israel, each praying out of his own hurting heart and anguish and stretching out his hands toward this Temple, \v{39}then hear from heaven, the place where you reside, and forgive, repaying each person according to all of his ways, since you know their hearts---for you alone know the hearts of all human beings--- \v{40}so they will fear you every day and live on the surface of the land that you have given to our ancestors. \\
\poeml \v{41}``Now concerning the foreigner who is not from your people Israel, when he comes from a land far away for the sake of your name \v{42}(for people will hear of your great name, your mighty acts,\fnote{Lit. \fbib{hand}} and your obvious power\fnote{Lit. \fbib{your outstretched arm}}), when he comes and prays facing this Temple, \v{43}then hear in heaven where you reside, and do whatever the foreigner asks of you, so that all the people of the earth may know your name, fear you as do your people Israel, and so they may know that this Temple that I have built is called by your name. \\
\poeml \v{44}``When your people go out to war against their enemies, no matter what way you send them, and they pray to the \divine{Lord} in the direction of the city that you have chosen and in the direction of the Temple that I have built for your name, \v{45}then hear their prayer and their request in heaven, and fight for their cause. \\
\poeml \v{46}``When they sin against you---because there isn't a single human being who doesn't sin---and you become angry with them and deliver them over to their enemy, who takes them away captive to the land that belongs to their enemy, whether near or far away, \v{47}if they turn their hearts back to you\fnote{The Heb. lacks \fbib{back to you}} in the land where they have been taken captive, repent, and pray to you---even if they do so in the land of their captivity---confessing, `We have sinned, we have committed abominations, and practiced wickedness,' \v{48}if they return to you with all of their heart and with all of their soul in the land of their enemies who have taken them captive, as they pray to you in the direction of their land that you have given to their ancestors and to the city that you have chosen, and to the Temple that I have built for your name, \v{49}then hear their prayer and requests in heaven, where you reside, and fight for their cause, \v{50}forgiving your people who have sinned against you, along with their transgressions by which they have transgressed against you. \\
\poeml ``Show your compassion in the presence of those who have taken them captive, so they may show compassion on them, \v{51}since they are your people and your heritage, which you brought out of Egypt, from an iron fire furnace. \v{52}Do this\fnote{The Heb. lacks \fbib{Do this}} so your eyes may remain open to the requests of your servant and to the requests of your people's prayers, to listen to them whenever they call out to you, \v{53}because you have separated them to yourself as your heritage from all the people of the earth, as you spoke through your servant Moses when you brought our ancestors out of Egypt, Lord \divine{God}.
\end{poetry}
\passage{Solomon's Blessing to the Assembly}
\passageinfo{(2 Chronicles 6:40-42)}

\v{54}When Solomon had completed saying this entire prayer to the \divine{Lord}, he got up from kneeling with his hands spread out toward heaven in the presence of the \divine{Lord}'s altar, \v{55}stood up, and blessed all of the assembly of Israel in a loud voice. He said:

\begin{poetry}
\poeml \v{56}``Blessed is the \divine{Lord}, who has given security to his people Israel, just as he promised. Not one of his promises has failed to come about that he gave through his servant Moses. \v{57}May the \divine{Lord} our God be with us, just as he was with our ancestors. May he never leave us or abandon us, \v{58}so that he may turn our hearts toward him, so that we may live life\fnote{Lit. \fbib{may walk in}} his way, keeping his commands, statutes, and ordinances that he gave to our ancestors. \v{59}And may what I've had to say to the \divine{Lord} remain with the \divine{Lord} our God both day and night, so that he may defend the cause of his servant and the cause of his people Israel, as the need of the day may require it, \v{60}so that, in turn,\fnote{Lit. \fbib{therefore}} all the people of the earth may know that the \divine{Lord} is God---there is no one else. \v{61}Now let your heart be completely devoted to the \divine{Lord} our God, to live according to his statutes and to keep his commands, as we are doing today.''
\end{poetry}
\passage{Solomon's Initial Offerings}
\passageinfo{(2 Chronicles 7:4-11)}

\v{62}Then the king and all of Israel with him offered sacrifices to the \divine{Lord}. \v{63}Solomon offered peace offerings to the \divine{Lord} consisting of 22,000 oxen and 120,000 sheep. So the king and all the Israelis dedicated the \divine{Lord}'s Temple. \v{64}That same day, the king consecrated the middle court that stood in front of the \divine{Lord}'s Temple, because that was where he offered burnt offerings, grain offerings, and fat from the peace offerings and because the bronze altar that was in the \divine{Lord}'s presence was too small to hold the burnt offerings, grain offerings, and fat from the peace offerings. \v{65}So Solomon observed the Festival of Tents\fnote{The Heb. lacks \fbib{of Tents}; cf. Lev 23:34} at that time, as did all of Israel with him. A large assembly came up from as far away as Lebo-hamath and the Wadi\fnote{I.e. a seasonal stream or river that channels water during rain seasons but is dry at other times} of Egypt to appear in the presence of the \divine{Lord} our God, not just for seven days, but for seven days after that, a total of fourteen days. \v{66}The following\fnote{Lit. \fbib{eighth}} day, Solomon\fnote{Lit. \fbib{he}} sent the people away as they blessed the king. Then they went back to their tents, rejoicing and glad for all the good things that the \divine{Lord} had done for his servant David and to his people Israel.
\labelchapt{9}
\passage{God Appears to Solomon}
\passageinfo{(2 Chronicles 7:11-22)}

\chapt{9}
\v{1}Later, after Solomon had finished building the \divine{Lord}'s Temple, the royal palace, and everything else that Solomon wanted to do, \v{2}the \divine{Lord} appeared to Solomon for a second time, just as he had appeared to him at Gibeon. \v{3}The \divine{Lord} told him:

\begin{poetry}
\poeml ``I've heard your prayer and your request that you made to me. I have consecrated this Temple that you have built by placing my name there forever. My eyes and my heart will be there continuously. \\
\poeml \v{4}``Now as for you, if you commune with me like your father did, with an upright heart of integrity and doing everything that I've commanded you and keeping my statutes and ordinances, \v{5}then I'll make your royal throne secure forever, just as I agreed to do so for your father David when I said, `You are to not lack a man on the throne of Israel.' \v{6}But if you or your descendants abandon me, and do not keep my commandments and statutes that I have given to you, and if you go away, serve other gods, and worship them, \v{7}then I will eliminate Israel from the land that I gave them and from the Temple that I've consecrated for my name. I will throw them out of my sight, and Israel will become the butt of jokes\fnote{Lit. \fbib{become an object of mockery}} and a means of ridicule among people worldwide! \\
\poeml \v{8}``This Temple will become a pile of ruins. Everyone who passes by it will be so astounded that they will ask, `Why did the \divine{Lord} do this to this land and to this Temple?' \v{9}They will answer, `Because they abandoned the \divine{Lord} their God, who brought their ancestors out of the land of Egypt, and they adopted other gods and served them. That's why the Lord has brought all of this disaster on them.'\,''
\end{poetry}
\passage{Solomon Cedes Cities to Hiram}

\v{10}It took 20 years for Solomon to finish working on the two houses---the \divine{Lord}'s Temple and the royal palace--- \v{11}after which King Solomon gave Hiram 20 cities in the land of Galilee, because King Hiram of Tyre had provided Solomon with as much cedar, cypress timber, and gold that he wanted. \v{12}Hiram came out from Tyre to see the cities that Solomon had given him, but he wasn't happy with them, \v{13}so he asked him, ``What are these cities that you have given to me, my brother?'' That's why these cities were named ``the land of Cabal''\fnote{The Heb. name \fbib{Cabul} means \fbib{as good as nothing}} to this day. \v{14}Then Hiram paid the king 120 talents\fnote{I.e. about 9,000 pounds; a talent weighed about 75 pounds} of gold.
\passage{Solomon's Other Accomplishments}
\passageinfo{(2 Chronicles 8:3-16)}

\v{15}Here is a summary of the conscripted labor that King Solomon required to build the \divine{Lord}'s Temple, his royal palace, the terrace ramparts in the City of David,\fnote{Lit. \fbib{the Millo}, fortified areas of ancient Jerusalem with terraces and retaining walls} the wall of Jerusalem, Hazor, Megiddo, and Gezer. \v{16}Pharaoh, the king of Egypt, had attacked and captured Gezer, burned it down, killed the Canaanites who lived in the city, and then gave it as a dowry for his daughter, Solomon's wife. \v{17}So Solomon rebuilt Gezer, lower Beth-horon, \v{18}Baalath, and Tamar in the wilderness, \v{19}along with the storage cities that Solomon used for his chariots and for his cavalry, everything that Solomon felt like building in Jerusalem, in Lebanon, and in every territory under his control.

\v{20}The people who survived from the Amorites, the Hittites, the Perizzites, the Hivites, and the Jebusites, who were not related to the Israelis, \v{21}and whose descendants had survived them and continued to live in the land because the Israelis were unable to completely eliminate them, Solomon placed under conscripted labor, a situation that remains in effect to this day. \v{22}However, Solomon did not force Israelis into conscripted labor, but they did serve as his soldiers, servants, princes, captains, chariot commanders, and cavalry. \v{23}There were 550 chief officers who supervised Solomon's activities and managed the staff that was doing the work.

\v{24}As soon as Pharaoh's daughter arrived from the City of David to live in her house that Solomon\fnote{Lit. \fbib{he}} had built for her, then he fortified the terrace ramparts in the City of David.\fnote{Lit. \fbib{the Millo}, fortified areas of ancient Jerusalem with terraces and retaining walls} \v{25}Three times every year Solomon offered burnt offerings and peace offerings on the altar that he had built to the \divine{Lord}, burning incense with the offerings in the presence of the Lord.

This concludes the record of the Temple construction.
\passage{Solomon's Business Ventures}
\passageinfo{(2 Chronicles 8:17-18)}

\v{26}King Solomon also built a fleet of ships at Ezion-geber, which is near Eloth on the shore of the Reed\fnote{So MT; LXX reads \fbib{Red}} Sea in the land of Edom. \v{27}Hiram sent his servants to sail with the fleet, since they were expert seamen, and so they accompanied Solomon's servants. \v{28}They sailed as far as Ophir\fnote{Or \fbib{as a source of fine gold}; cf. 1Chr 29:4} and brought back 420 talents\fnote{I.e. about 31,500 pounds; a talent weighed about 75 pounds} of gold for Solomon.
\labelchapt{10}
\passage{The Queen of Sheba Visits Solomon}
\passageinfo{(2 Chronicles 9:1-28)}

\chapt{10}
\v{1}When the queen of Sheba heard about Solomon's reputation with the \divine{Lord}, she came to test him\fnote{Lit. \fbib{Solomon}} with difficult questions. \v{2}She brought along a large retinue, camels laden with spices, and lots of gold and precious stones. Upon her arrival, she spoke with Solomon about everything that was on her mind.\fnote{Lit. \fbib{was with her heart}} \v{3}Solomon answered all of her questions. Nothing was hidden from Solomon that he did not explain to her. \v{4}When the queen of Sheba had seen all of Solomon's wisdom for herself, the palace that he had built, \v{5}the food set at his table, his servants who sat with him, his ministers in attendance and how they were dressed, his personal staff\fnote{Lit. \fbib{his cupbearers}} and how they were dressed, and even his personal stairway by which he went up to the \divine{Lord}'s Temple, she was breathless!

\v{6}``Everything I heard about your wisdom and what you have to say is true!'' she gasped, \v{7}``but I didn't believe it at first! But then I came here and I've seen it for myself! It's amazing! I wasn't told half of what's really great about your wisdom. You're far better in person than what the reports have said about you! \v{8}How blessed are your staff! And how blessed are your employees,\fnote{Lit. \fbib{servants}} who serve you continuously and get to listen to your wisdom! \v{9}And blessed be the \divine{Lord} your God, who is delighted with you! He set you in place on the throne of Israel because the \divine{Lord} loved Israel forever. That's why he made you to be king, so you could carry out justice and implement righteousness.''

\v{10}Then she gave the king 120 talents\fnote{I.e. about 9,000 pounds; a talent weighed about 75 pounds} of gold, a vast quantity of spices, and precious stones. No spices ever came again that were comparable to those that the queen of Sheba gave to King Solomon. \v{11}Hiram's ships that brought gold from Ophir,\fnote{Or \fbib{from a source of fine gold}; cf. 1Chr 29:4} also brought from Ophir\fnote{Or \fbib{from a source of fine gold}; cf. 1Chr 29:4} lots of algum wood\fnote{Or \fbib{presented Juniper trees}} and precious stones. \v{12}The king used the algum wood\fnote{Or \fbib{the Juniper trees}} to have supports made for the \divine{Lord}'s Temple and for the royal palace, as well as lyres and harps for the choir,\fnote{Lit. \fbib{singers}} and nothing like that wood\fnote{The Heb. lacks \fbib{wood}} has ever come again or even been seen since right to this day. \v{13}In return, King Solomon gave the queen of Sheba everything she wanted and had requested in addition to what he had given her consistent with his generosity. Afterward, she returned to her own land with her servants.
\passage{Solomon's Wealth}
\passageinfo{(2 Chronicles 1:14-17; 10:13-28)}

\v{14}Solomon's annual revenue was 666 talents\fnote{I.e. about 49,950 pounds; a talent weighed about 75 pounds} of gold, \v{15}not including revenue from traders, merchants, and from all the kings of Arabia and the governors of the land. \v{16}King Solomon made 200 large shields of beaten gold, overlaying each large shield with the gold from 600 gold pieces,\fnote{MT does not identify the individual unit of measure} \v{17}and 300 shields from beaten gold, overlaying each shield with the gold from 300 gold pieces.\fnote{MT does not identify the individual unit of measure} The king put them in his palace in the Lebanon forest. \v{18}The king also made a great ivory throne and overlaid it with pure gold. \v{19}Six steps led up to the throne, which had a round canopy fastened to the rear of the throne and armrests on each side of the seat and two lions standing on either side of each armrest. \v{20}Twelve lions were placed on both sides of the six steps leading to the throne,\fnote{The Heb. lacks \fbib{leading to the throne}} and nothing comparable was made for any other\fnote{The Heb. lacks \fbib{other}} kingdoms. \v{21}All of King Solomon's drinking vessels were made of\fnote{The Heb. lacks \fbib{made of}} gold, and all the vessels in his palace in the Lebanon forest were made of\fnote{The Heb. lacks \fbib{made of}} pure gold. None were of silver, because silver was never considered to be valuable during Solomon's lifetime, \v{22}because the king had ships that sailed to Tarshish accompanied by Hiram's ships. Once every three years ships from Tarshish returned, bringing gold, silver, ivory, apes, and peacocks. \v{23}As a result, King Solomon became greater than all the kings of the earth in regards to wealth and wisdom. \v{24}All the earth continued to seek audiences with Solomon so they could hear the wise things that God had put in his heart. \v{25}Everyone kept on bringing gifts on an annual basis, including items made of silver and gold, garments, myrrh, spices, horses, and mules. \v{26}Solomon accumulated chariots and cavalry. He had 1,400 chariots and 12,000 cavalry soldiers. He stationed them in various chariot cities and with the king in Jerusalem. \v{27}The king made silver as common as\fnote{The Heb. lacks \fbib{as common as}} stones in Jerusalem, and made cedar trees as abundant as sycamore\fnote{The sycamore fruit tree native to Israel bears figs} trees in the Shephelah.\fnote{I.e. the verdant central lowlands of Israel; cf. Josh 10:40} \v{28}Solomon imported horses from Egypt and Kue, and the king's buyers procured them at market price from Kue. \v{29}A chariot from Egypt cost 600 pieces\fnote{The denomination of silver coin is not specified.} of silver, and a horse 150 pieces of silver,\fnote{The Heb. lacks \fbib{pieces of silver}} but then they were exported to all the Hittite kings and to the Aramean kings.
\labelchapt{11}
\passage{Solomon's Forbidden Marriages and Idolatry}
\passageinfo{(2 Chronicles 9:1-28)}

\chapt{11}
\v{1}But King Solomon married\fnote{Lit. \fbib{loved}} many foreign women besides the daughter of Pharaoh: women from Moab, Ammon, Edom, and Sidonia, along with Hittite women, too, \v{2}all of them from nations that the \divine{Lord} had ordered the Israelis, ``You are not to associate with\fnote{Lit. \fbib{to go in to}} them and they are not to associate with you, because they will most certainly turn your affections\fnote{Lit. \fbib{hearts}} away to follow their gods.'' Solomon became deeply attached to them by falling in love. \v{3}He had 700 princess wives and 300 mistresses\fnote{Or \fbib{concubines}; i.e. secondary wives} who\fnote{Lit. \fbib{mistresses, and his wives}} turned his heart away from the \divine{Lord},\fnote{The Heb. lacks \fbib{from the \divine{Lord}}} \v{4}because as Solomon grew older, his wives turned his affections away after other gods, and his heart was not fully as devoted to the \divine{Lord} his God as his father David's heart had been. \v{5}Solomon pursued Astarte, the Sidonian goddess, and Milcom, that detestable Ammonite idol. \v{6}Solomon practiced what the \divine{Lord} considered to be evil by not fully following the \divine{Lord}, as had his father David. \v{7}Later, Solomon even constructed a high place on the mountain east of Jerusalem that was dedicated to Chemosh, that detestable Moabite idol, and to Molech, the detestable Ammonite idol. \v{8}Solomon\fnote{Lit. \fbib{He}} did this for all of his foreign wives, who burned incense and sacrificed to their own gods.

\v{9}The \divine{Lord} became angry at Solomon because his heart wandered away from the \divine{Lord} God of Israel, who had appeared to him twice\fnote{Cf. 1King 3:5, 9:2} \v{10}and warned him about this so he would not pursue other gods. But he did not obey what the \divine{Lord} had commanded, \v{11}so the \divine{Lord} told Solomon, ``Because you have done this and haven't kept my covenant and statutes that I commanded you, I'm going to tear the kingdom from you and give it to your servant. \v{12}I'm not going to do this during your lifetime, for the sake of your father David, but I will tear it out of your son's control.\fnote{Lit. \fbib{hand}} \v{13}For the sake of my servant David and for the sake of Jerusalem, I won't tear away the entire kingdom. I'll leave one tribe for your son to govern.''\fnote{The Heb. lacks \fbib{to govern}}
\passage{Solomon's Enemies}

\v{14}After this, the \divine{Lord} allowed\fnote{Lit. \fbib{raised up}} Hadad the Edomite to oppose Solomon. He was part of the royal line of Edom. \v{15}During David's military campaign against Edom, when his army commander Joab had gone out to bury the dead, he killed every male in Edom. \v{16}Joab had his entire army of Israel stay there for six months until he had eliminated every male in Edom.

\v{17}But Hadad escaped to Egypt in the company of some of his father's Edomite servants, while Hadad was still a little child. \v{18}They left Midian, arrived in Paran, and left from Paran with some men and traveled on to Egypt, where Pharaoh, king of Egypt, gave him a house to live in, assigned a food allotment to him, and gave him some land. \v{19}Hadad won the affection of the Pharaoh, who gave permission for Hadad to marry the sister of his own wife, Queen Tahpenes. \v{20}Queen Tahpenes' sister bore him his son Genubath, whom Tahpenes weaned in Pharaoh's palace while Genubath lived in Pharaoh's palace with the Pharaoh's own sons.

\v{21}Later on, Hadad learned in Egypt that David had been buried\fnote{Lit. \fbib{had slept}} with his ancestors and that Joab the army commander was dead. So Hadad asked Pharaoh, ``Please send me out so I can go back to my own land.''

\v{22}Pharaoh asked him, ``But have you lacked anything from me that would make you want to go back to your own country?''

``No,'' he answered, ``but I still really must leave.''

\v{23}God also raised up Eliada's son Rezon, who had escaped from his master King Hadadezer of Zobah. \v{24}He raised an army and commanded a gang of raiders after David had eliminated those who lived in Zobah. Rezon and his army\fnote{Lit. \fbib{They}} moved to Damascus, remained there, and Rezon ruled from Damascus. \v{25}He opposed Israel during Solomon's entire reign, in addition to all of the evil things that Hadad did. Rezon\fnote{Lit. \fbib{He}} also hated Israel while he reigned over Aram.
\passage{Jeroboam Rebels against Solomon}

\v{26}Solomon had a servant, Nebat's son Jeroboam, who was an Ephraimite from Zeredah. His widowed mother was named Zeruah. Jeroboam rebelled against Solomon, \v{27}and this is why he rose in rebellion against the king: Solomon had built up the terrace ramparts\fnote{Lit. \fbib{the Millo}, fortified areas of ancient Jerusalem with terraces and retaining walls} in the city of his father David in order to repair a weakness. \v{28}Jeroboam was a valiant soldier, and because Solomon observed that the young man was able to get things done, he set him in charge over all of the conscripted labor from the household of Joseph. \v{29}During that time, Jeroboam left Jerusalem and the prophet Ahijah the Shilonite met him on the road. Ahijah had wrapped himself up in a new cloak, and both of them were alone on the open road. \v{30}Ahijah grabbed the new cloak that he was wearing and tore it into twelve pieces! \v{31}Then he told Jeroboam, ``Take ten pieces for yourself, because this is what the \divine{Lord} God of Israel says:

\begin{poetry}
\poeml `Pay attention! I'm going to tear the kingdom out of Solomon's control\fnote{Lit. \fbib{hand}} and give you ten tribes. \v{32}I'll leave him one tribe for the sake of my servant David and one tribe\fnote{The Heb. lacks \fbib{one tribe}} for the sake of Jerusalem, the city that I chose from all of the tribes of Israel. \v{33}I'm doing this\fnote{The Heb. lacks \fbib{I'm doing this}} because they have abandoned me and worshipped that Sidonian goddess Astarte, the Moabite god Chemosh, and the Ammonite god Milcom. They haven't lived my way by doing what I consider to be right and observing my statutes and my ordinances, like his father David did. \\
\poeml \v{34}`Nevertheless, I won't take the entire kingdom away from him, but I'll let him reign for the rest of his life, because of my servant David, whom I chose, who obeyed my commandments and statutes, \v{35}but I will take the kingdom away from his son's control\fnote{Lit. \fbib{hand}} and give ten tribes to you. \v{36}I'll give one tribe to his son, so my servant David will always have a light shining in my presence in Jerusalem, the city that I chose for myself and where I have placed my name. \v{37}I'm going to take you and have you reign over whatever you desire. You will be king over Israel. \v{38}If you listen to everything that I command you to do, and if you live your life my way,\fnote{Lit. \fbib{you walk in my ways}} and if you do what I consider to be right by observing my statutes and my commandments, just like my servant David did, then I will be with you, I will build an enduring dynasty for you,\fnote{Lit. \fbib{enduring house}} just like I did for David, and I'll give Israel to you. \v{39}This is how I'm going to afflict David's descendants because of what they have done, though I won't do it continuously.'\,''
\end{poetry}

\v{40}That's why Solomon tried to execute Jeroboam, but Jeroboam got up and fled to Egypt, where he lived as a guest of King Shishak and remained until Solomon had died.
\passage{The Death of Solomon}
\passageinfo{(2 Chronicles 9:29-31)}

\v{41}Now the rest of Solomon's accomplishments, including everything else he did, as well as records of\fnote{The Heb. lacks \fbib{records of}} his wisdom, are recorded in the Book of the Acts of Solomon, are they not? \v{42}Solomon reigned over all of Israel from Jerusalem for a total of 40 years. \v{43}Then Solomon died, as had\fnote{Lit. \fbib{Solomon slept with}} his ancestors, and he was buried in the city of his father David. His son Rehoboam reigned in his place.
\labelchapt{12}
\passage{Secession of the Northern Tribes}
\passageinfo{(2 Chronicles 10:1-19)}

\chapt{12}
\v{1}Rehoboam traveled to Shechem because all of Israel went there to install him as king. \v{2}Nebat's son Jeroboam heard about it while he was still in Egypt, where he had fled to get away from King Solomon. Jeroboam returned from Egypt \v{3}after being summoned. When Jeroboam and the entire assembly of Israel arrived, they spoke to Rehoboam, \v{4}``Your father made our burdens unbearable.\fnote{Lit. \fbib{our yoke heavy}} Therefore lighten your father's requirements and his heavy burdens that he placed on us, and we'll serve you.''

\v{5}``Come again in three days,'' Rehoboam\fnote{Lit. \fbib{He}} told them. So the people left \v{6}while King Rehoboam conferred with his advisors who had worked for his father Solomon during his administration. He asked them, ``What is your advice as to how I should respond to these people?''

\v{7}They advised him, ``If today you are a servant, you will serve this people by answering them and speaking kindly to them. Then they will serve you forever.''

\v{8}But Rehoboam\fnote{Lit. \fbib{he}} ignored the counsel that his elder advisors had given him. Instead, he consulted the younger men who had grown up with him and who worked for\fnote{Lit. \fbib{who stood before}} him. \v{9}As a result, he asked them, ``What's your advice so that we can give an answer to these people who have asked me, `Please lighten the burden that your father put on us.'?''

\v{10}``This is what you should tell these people who asked you `Your father made our burden heavy, but you must make it lighter for us!'\,'' the young men who grew up with Rehoboam\fnote{Lit. \fbib{him}} replied. ``Tell them, `My little finger will be thicker than my father's whole body!\fnote{Lit. \fbib{father's loin}} \v{11}Not only that, but since my father loaded you down heavily, I'm going to add to that burden. My father disciplined you with whips, but I'm going to discipline you with scorpions!'\,''

\v{12}So Jeroboam and all the people went back to Rehoboam on the third day, just as they had been directed when the king said, ``Come back again in three days.'' \v{13}But the king gave the people a harsh response, because he was ignoring the counsel that his elders had given him. \v{14}Instead, Rehoboam\fnote{Lit. \fbib{he}} spoke to them along the lines of what the younger men suggested. He told them ``My father burdened you heavily, but I will add to that burden. If my father disciplined you with whips, I'm going to discipline you with scorpions!''

\v{15}The king would not listen to the people, because the turn of events was from the \divine{Lord}, to fulfill his prediction that the \divine{Lord} spoke by means of Ahijah the Shilonite to Nebat's son Jeroboam \v{16}When all of Israel saw that the king wasn't listening to them, the people responded to the king's message, ``What's the point in following David? We have no inheritance in the descendants of Jesse. Let's go home,\fnote{Lit. \fbib{Each man to his tent}} Israel! David, take care of your own household!' So Israel left for home.\fnote{Lit. \fbib{left for their tents}} \v{17}And so Rehoboam ruled over the Israelis who lived in the cities of Judah.

\v{18}King Rehoboam sent Hadoram, who was in charge of conscripted labor, but all of Israel stoned him to death, and King Rehoboam had to jump in his chariot and flee back in a hurry to Jerusalem. \v{19}That's how Israel came to be in rebellion against David's dynasty to this day.
\passage{Jeroboam Reigns over Israel}
\passageinfo{(2 Chronicles 11:1-4)}

\v{20}Now when all of Israel heard that Jeroboam had returned, they sent for him and invited him to visit their assembly, where they installed him as king over all of Israel. Nobody (with the sole exception of the tribe of Judah) would align with David's dynasty. \v{21}As soon as Rehoboam returned to Jerusalem, he assembled 180,000 elite soldiers from the tribes of Judah and Benjamin, intending to attack the dynasty of Israel and restore the kingdom to Solomon's son Rehoboam. \v{22}But a message from God came to Shemaiah, a man of God: \v{23}``Tell Solomon's son Rehoboam, king of Judah, all the dynasty of Judah, Benjamin, and the rest of the people, \v{24}`This is what the \divine{Lord} says: ``You are not to fight or even approach your fellow Israelis in battle. Every soldier is to return to his own home, because this development comes from me.''\,'\,'' So they listened to what the \divine{Lord} had to say and returned home,\fnote{The Heb. lacks \fbib{home}} just as the \divine{Lord} had directed.
\passage{Jeroboam's Idolatry}

\v{25}Later on, Jeroboam fortified Shechem in the hill country of Ephraim and lived there. He also expanded from there and built Penuel. \v{26}Jeroboam was thinking to himself, ``The kingdom is about to return to David's control.\fnote{Lit. \fbib{house}} \v{27}If these people keep going up to Jerusalem to offer sacrifices to the \divine{Lord} there, the hearts of these people will return to their lord, King Rehoboam of Judah. Then they'll kill me and return to Rehoboam, king of Judah!'' \v{28}So the king sought some advice and then built two golden calves and announced, ``It's too difficult for you to travel to Jerusalem. So here are your gods, Israel, who brought you up from the land of Egypt!'' \v{29}He set one of them in Bethel and placed the other one in Dan. \v{30}Doing this was sinful, because the people traveled as far as Dan to appear before one of their idols.\fnote{The Heb. lacks \fbib{of their idols}} \v{31}Jeroboam\fnote{Lit. \fbib{He}} built temples on the high places, and appointed his own priests from the fringe elements of the people who were not descendants of Levi.

\v{32}Jeroboam invented a festival for the fifteenth day of the eighth month similar to the festival that takes place in Judah. He approached the altar that he had set up in Bethel and sacrificed to the calves that he had made, having stationed in Bethel the priests that he had appointed. \v{33}Then, on the fifteenth day of the eighth month, he went up to burn incense on the altar that he had set up in Bethel, thus beginning the festival that he had made up out of his own heart for the Israelis.
\labelchapt{13}
\passage{Josiah's Desecration Predicted by a Man of God}

\chapt{13}
\v{1}Right when Jeroboam was standing by the altar to burn some incense, a man of God arrived in Bethel from Judah in obedience to a command from the \divine{Lord}. \v{2}He cursed\fnote{Or \fbib{rebuked}} the altar in this\fnote{Lit. \fbib{a}} message from the \divine{Lord}: ``Hey altar! Hey altar! This is what the \divine{Lord} says: `Pay attention to this! A son is going to be born in David's dynasty. His name will be Josiah. He will sacrifice the priests who burn incense on you in these high places. Human bones will be burned on you!'\,''\fnote{Cf. 2King 23:15-16}

\v{3}Later that same day, he gave them a special display of power\fnote{Or \fbib{a sign}} of what was to come when he said, ``Here's proof\fnote{Or \fbib{Here's a sign}} that the \divine{Lord} has decreed this:\fnote{Lit. \fbib{spoken}} Look! This altar will be split apart and the ashes that are on it will spill out.''

\v{4}When he heard the man of God curse\fnote{Or \fbib{rebuke}} the altar in Bethel, the king pointed at the man of God from where the king was standing at the altar. ``Seize him!'' he ordered. But all of a sudden his hand that he had stretched out dried up, and he could not bring it back to his side! \v{5}Also, the altar broke apart and the ashes that were on it spilled out from the altar, providing just the proof that the man of God had predicted in his message from the \divine{Lord}!

\v{6}``Please!'' the king begged the man of God, ``Ask the \divine{Lord} your God and pray for me that my hand may be restored for me!'' So the man of God asked the \divine{Lord}, and the king's hand was immediately and fully restored, just like it had been before. \v{7}So the king told the man of God, ``Come back to my palace and rest a while. I'd like to give you a reward.''

\v{8}But the man of God replied to the king, ``Even if you were to offer me half of your house, I wouldn't go with you, and I'm sure not going to eat even a piece of bread or drink water in this place, \v{9}because the \divine{Lord} commanded me specifically, `You are not to eat bread, drink water, or return by the way that you came to arrive here!'\,'' \v{10}Then he left, returning a different way than the one by which he had traveled to Bethel.
\passage{An Old Prophet Rebukes the Man of God}

\v{11}Now there was an old prophet who lived in Bethel, and his sons went to him and told him everything that the man of God had accomplished that day in Bethel, including the message that he had delivered to the king. \v{12}``Which way did he go?'' their father asked him, since his sons had observed the way that the man of God had taken to return to Judah from Bethel. \v{13}``Saddle my donkey for me!'' he ordered.\fnote{The Heb. lacks \fbib{he ordered}} So they saddled the donkey for him \v{14}and he rode off after the man of God and found him sitting under an oak tree.\fnote{or \fbib{under a terebinth tree}; i.e. an oak tree used in idol worship} ``You're the man of God who came from Judah, aren't you?'' the old prophet\fnote{Lit. \fbib{He}} asked him.

``I am,'' he replied.

\v{15}``Come home with me and have a meal,'' he told him.

\v{16}But he replied, ``I can't go back with you to your home, be in your company, or even eat food or drink water with you in this place, \v{17}because I've been given a command in the form of this message from the \divine{Lord}: `You are to eat no food, drink no water, and do not return to Judah\fnote{The Heb. lacks \fbib{to Judah}} by traveling the way by which you go there.'\,''

\v{18}``I'm a prophet like you,'' the old man replied, ``and an angel spoke to me and delivered this message from the \divine{Lord}: `Bring him back with you to your house and give him food and water.'\,'' But he was lying, \v{19}and the man of God\fnote{Lit. \fbib{So he}} accompanied the old prophet\fnote{Lit. \fbib{accompanied him}} back to his house, ate some food, and drank some water.

\v{20}Later, while they were sitting down at the table, a message from the \divine{Lord} was delivered to the prophet who had brought him back, \v{21}so he cried out to the man of God from Judah: ``This is what the \divine{Lord} says: `Because you disobeyed a command from the \divine{Lord} and haven't done what the \divine{Lord} your God commanded you to do, \v{22}but instead you returned to eat and drink in the very place that he told you ``Eat no food and drink no water,'' your body will not be buried in the same grave as your ancestors.'\,''
\passage{A Lion Kills the Man of God}

\v{23}After the meal was over, and the man had eaten food and had drunk water, the old prophet saddled the donkey for him---that is, for the man of God whom he had brought back. \v{24}Not long after the man of God\fnote{Lit. \fbib{after he}} had left, a lion met him along the road and killed him. His body was left lying in the middle of the road with the donkey standing beside it and with the lion also standing next to the body. \v{25}When some men passed by and noticed the body lying in the middle of the road and the lion standing beside the body, they went straight to the city and told what had happened in the city where the old prophet lived.

\v{26}The prophet who had brought the man of God\fnote{Lit. \fbib{brought him}} back from the road learned about it. ``It's the man of God who disobeyed the message from the \divine{Lord},'' he said. ``That's why the \divine{Lord} gave him to that lion, which mauled him and killed him, just as the message from the \divine{Lord} told rebuke him.'' \v{27}Then he ordered his sons, ``Saddle the donkey for me.'' So they did. \v{28}The old prophet\fnote{Lit. \fbib{He}} went out, located the body on the road where the donkey and the lion were standing beside the body. The lion had not eaten the body nor mauled the donkey. \v{29}The prophet picked up the body of the man of God, laid it on the donkey, and brought it back to the city where the old man lived so he could mourn and bury him.

\v{30}He buried the corpse in his own grave and his family mourned for him, crying out, ``Oh, no! My brother!''

\v{31}After he had buried the man of God,\fnote{Lit. \fbib{buried him}} he gave these instructions to his children: ``When I die, bury me in the same grave in which the man of God is buried. Place my bones beside his, \v{32}because what he predicted by a message from the \divine{Lord} against the altar in Bethel and the temples built in the high places of the cities of Samaria will certainly come about.''

\v{33}Despite everything that happened, Jeroboam never did repent of his evil practices. Instead, he appointed even more people to act as priests for the high places. Anyone who wanted to be a priest was ordained to be a priest in the high places. \v{34}This practice became so sinful that the \divine{Lord} decided\fnote{The Heb. lacks \fbib{that the \divine{Lord} decided}} to erase Jeroboam's dynasty, thus eliminating it from the face of the earth.
\labelchapt{14}
\passage{God Disciplines Jeroboam's Family}

\chapt{14}
\v{1}Right at that time, Jeroboam's son Abijah became ill, \v{2}so Jeroboam suggested to his wife, ``Get up, disguise yourself so that no one will know that you're Jeroboam's wife, and go to Shiloh where the prophet Ahijah lives. He's the one who told me that I would be king over this people. \v{3}Take ten loaves with you, some\fnote{Lit. \fbib{loaves in your hand}} cakes, and a jar of honey and go visit him. He will tell you what will happen to the boy.''

\v{4}So that's what Jeroboam's wife did. She got up, went to Shiloh, and found Ahijah's home. Ahijah was blind, because his eyes could not focus\fnote{Lit. \fbib{eyes were set}} due to his age. \v{5}Meanwhile, the \divine{Lord} had spoken to Ahijah, ``Be on your guard! Jeroboam's wife is coming to ask you about her son, because he is ill. You're to say such and such to her. When she arrives, she will pretend to be someone else!''

\v{6}When she arrived, Ahijah heard the sound of her feet as she came through the doorway. He said this to her:

\begin{poetry}
\poeml ``Come in, wife of Jeroboam. What is this pretension at being someone else? I have some harsh news.\fnote{The Heb. lacks \fbib{news}} \v{7}Go tell Jeroboam: \\
\poeml `I raised you up from among the people. \\
\poeml `I made you Commander-in-Chief\fnote{Lit. \fbib{Nagid}; i.e. a senior officer entrusted with dual roles of operational oversight and administrative authority} over my people Israel. \\
\poeml \v{8}`I tore the kingdom away from David's dynasty. \\
\poeml `Then I gave it to you. \\
\poeml But you have not lived like my servant David, who kept my commands with all his heart, and did only what I considered to be right. \\
\poeml \v{9}`Instead, you have done more evil than everyone who lived before you. \\
\poeml `You have gone out and crafted other gods for yourself. \\
\poeml `You made cast images. \\
\poeml `You have provoked me to anger. \\
\poeml `You have thrown me behind your back. \\
\poeml \v{10}`Therefore, watch while I bring calamity on Jeroboam's dynasty! \\
\poeml `I will eliminate every male,\fnote{Lit. \fbib{everyone who urinates against a wall}} both slave and free in Israel, from Jeroboam. \\
\poeml `I will burn up Jeroboam's dynasty, as a man burns up manure until it is gone. \v{11}Dogs will eat anyone who dies in the city that belongs to Jeroboam's household. The birds of the sky will eat anyone who dies in the open field, because the \divine{Lord} has determined it.' \\
\poeml \v{12}``Now get up and go home. When your feet cross the city line, your child will die. \v{13}Everyone in Israel will mourn for him and will bury him, because he alone from Jeroboam's family will receive a decent burial, because something good was observed in him with respect to the \divine{Lord} God of Israel out of all the household of Jeroboam! \\
\poeml \v{14}``In addition to this, the \divine{Lord} will raise up for himself a king over Israel who will eliminate Jeroboam's dynasty, starting today and from now on. \v{15}The \divine{Lord} will attack Israel, and Israel will shake like a reed shakes in a river current! He will uproot Israel from this good land that he gave to their ancestors and he will scatter them beyond the Euphrates\fnote{The Heb. lacks \fbib{Euphrates}} River, because they erected their Asherim\fnote{I.e. cultic pillars erected in worship to Canaanite deities} and provoked the \divine{Lord} to become angry! \v{16}He will give up Israel because of Jeroboam's sins that he committed and by which Jeroboam\fnote{Lit. \fbib{he}} caused Israel to sin.''
\end{poetry}

\v{17}Then Jeroboam's wife got up and left for Tirzah. As soon as she set foot over the threshold of the house, the child died. \v{18}All of Israel mourned him at his burial, just as the \divine{Lord} had said when he spoke through Ahijah the prophet.
\passage{The Death of Jeroboam}

\v{19}Now as for the rest of Jeroboam's accomplishments, including how he waged war and how he reigned, you may read about them in the Book of the Chronicles of the Kings of Israel. \v{20}Jeroboam reigned for 22 years and then died, as had his ancestors, and his son Nadab reigned in his place.
\passage{Rehoboam Reigns over Judah}
\passageinfo{(2 Chronicles 11:5-12:6)}

\v{21}Meanwhile, Solomon's son Rehoboam reigned in Judah. Rehoboam was 41 years old when he became king, and he reigned for seventeen years in Jerusalem, the city where the Lord had chosen from all the tribes of Israel to place his Name. His mother was an Ammonite named Naamah. \v{22}Judah practiced what the \divine{Lord} considered to be evil. They did more to provoke him to jealousy than their ancestors had ever done by committing the sins that they committed. \v{23}They erected high places, sacred pillars, and Asherim\fnote{I.e. cultic pillars erected in worship to Canaanite deities} for themselves on every high hill and under every green tree. \v{24}They even maintained male shrine prostitutes throughout the land, and imitated every detestable practice that the nations practiced whom the \divine{Lord} had expelled in front of the Israelis.

\v{25}As a result, during the fifth year of the reign of\fnote{The Heb. lacks \fbib{the reign of}} King Rehoboam, King Shishak of Egypt invaded and attacked Jerusalem. \v{26}He stripped the \divine{Lord}'s Temple and the royal palace of their treasures. He took everything, even the gold shields that Solomon had made. \v{27}King Rehoboam made shields out of bronze to take their place, and then committed them to the care and custody of the commanders of those who guarded the entrance to the royal palace. \v{28}Whenever the king entered the \divine{Lord}'s Temple, the guards would carry them to and from the guard's quarters.

\v{29}As to the rest of Rehoboam's accomplishments, and everything else that he undertook, they are recorded in the Book of the Chronicles of the Kings of Judah, aren't they? \v{30}There was continual warfare between Rehoboam and Jeroboam, \v{31}but eventually Rehoboam died, as had his ancestors, and he was buried with his ancestors in the City of David. His mother's name had been Naamah the Ammonite, and his son Abijah became king to replace him.
\labelchapt{15}
\passage{Abijah's Reign over Judah}
\passageinfo{(2 Chronicles 13:1-14:1)}

\chapt{15}
\v{1}Abijah reigned over Judah starting in the eighteenth year of Nebat's son Jeroboam's reign. \v{2}He reigned for three years in Jerusalem. His mother's name was Maacah, the daughter of Abishalom. \v{3}He practiced the same sins that his father committed before he was born. Unlike his ancestor David, his heart never became devoted to the \divine{Lord} his God. \v{4}Nevertheless, for the sake of David, the \divine{Lord} his God maintained a lamp for David\fnote{Lit. \fbib{him}} in Jerusalem by raising up his son after him so that Jerusalem would be established, \v{5}because David had practiced what the \divine{Lord} considered to be right. He never avoided anything that the \divine{Lord} had commanded him during his entire lifetime, except for the case of Uriah the Hittite.

\v{6}There was continual military conflict between Rehoboam and Jeroboam throughout his entire lifetime. \v{7}The rest of Abijah's accomplishments, including everything he undertook, are written in the Chronicles of the Kings of Judah, are they not? And a state of war continued to exist between Abijah and Jeroboam. \v{8}Eventually, Abijah died, as did his ancestors, and he was buried in the City of David. His son Asa succeeded him as king.
\passage{Asa Reigns over Judah}
\passageinfo{(2 Chronicles 14:1-15:19)}

\v{9}Asa began to reign as Judah's king during the twentieth year of the reign of\fnote{The Heb. lacks \fbib{the reign of}} Jeroboam as king over Israel. \v{10}He reigned 41 years in Jerusalem. His mother's name was Maacah, the daughter of Abishalom. \v{11}Asa practiced what the \divine{Lord} considered to be right, just like his ancestor David. \v{12}He also removed the male cult prostitutes\fnote{Or \fbib{sodomites}} from the land and destroyed all the idols that his ancestors had made. \v{13}He removed his mother Maacah from her position as Queen Mother because she had made a detestable image dedicated to Asherah.\fnote{I.e. cultic pillars erected in worship to Canaanite deities} Asa cut down his mother's idol, crushed it, and burned it at the Kidron Brook. \v{14}Nevertheless, the high places were not removed, even though Asa's heart was blameless toward the \divine{Lord} all of his life. \v{15}Asa brought into the \divine{Lord}'s Temple the things that his father had dedicated, as well as his own dedicated gifts such as silver, gold, and temple service\fnote{The Heb. lacks \fbib{temple service}} implements.
\passage{Alliances with Aram against Israel}
\passageinfo{(2 Chronicles 16:1-17:1)}

\v{16}A state of continual military unrest existed between Asa and King Baasha of Israel throughout their lifetimes. \v{17}King Baasha of Israel invaded Judah and interdicted Ramah by building fortifications around it so no one could enter or leave to join King Asa of Judah. \v{18}But Asa removed all the silver and gold from the treasuries of the Lord's Temple and from his royal palace, placed them into the care of some servants, and then sent them to Tabrimmon's son King Ben-hadad of Aram, the grandson of Hezion, who lived in Damascus.

\v{19}``Let's make a treaty between you and me,'' he said, ``just like the one between my father and your father. Notice that I've sent you silver and gold to break your treaty with King Baasha of Israel, so he'll retreat from his attack\fnote{The Heb. lacks \fbib{his attack}} on me.''

\v{20}So King Ben-hadad did just what King Asa had asked: he sent his commanding officers to attack the cities of Israel, conquering Ijon, Dan, Abel-beth-maacah, all of Chinneroth,\fnote{I.e. the region encompassing the Sea of Galilee} and the territory of Naphtali. \v{21}When Baasha learned of this, he stopped fortifying Ramah and remained in Tirzah, \v{22}so King Asa published a proclamation throughout Judah (no one was left out) and they carried away the stones and timber with which Baasha had been fortifying Judah. King Asa used them to fortify Geba in Benjamin and Mizpah.

\v{23}The rest of Asa's accomplishments, his strength, everything that he undertook, and the cities that he fortified are written in the Book of the Chronicles of the Kings of Judah, are they not? However, as he approached old age, he became diseased in his feet. \v{24}Then Asa died, as had his ancestors, and he was buried with his ancestors in the City of David, his ancestor. His son Jehoshaphat reigned in his place.
\passage{Nadab Reigns over Israel}

\v{25}Jeroboam's son Nadab became king over Israel during the second year of the reign of\fnote{The Heb. lacks \fbib{the reign of}} King Asa over Judah. He reigned over Israel for two years, \v{26}practicing what the \divine{Lord} considered to be evil, living the way his father did, committing sins, and leading Israel to sin. \v{27}So Ahijah's son Baasha from the household of Issachar conspired against him and killed Nadab at Gibbethon in Philistia while Nadab and all of Israel were attacking Gibbethon. \v{28}Baasha killed him during the third year of the reign of\fnote{The Heb. lacks \fbib{the reign of}} King Asa of Judah and took Nadab's\fnote{Lit. \fbib{his}} place as king.

\v{29}As soon as he was established as king, he killed everyone in the household of Jeroboam. He left not even one single person alive. He destroyed them completely, just as the \divine{Lord} had spoken through his servant Ahijah the Shilonite,\fnote{Cf. 1King 14:7-16} \v{30}because of the sins that Jeroboam had committed, and because he led Israel into sin, provoking the \divine{Lord} God of Israel to become angry.

\v{31}Now the rest of Nadab's accomplishments, including everything he undertook, are written in the Book of the Chronicles of the Kings of Israel, are they not? \v{32}Meanwhile, a state of war continued to exist between Asa and Baasha king of Israel, throughout their reigns.
\passage{Baasha Reigns over Israel}

\v{33}During the third year of the reign of\fnote{The Heb. lacks \fbib{the reign of}} King Asa of Judah, Ahijah's son Baasha became king over all of Israel. He reigned for 24 years at Tirzah. \v{34}He practiced what the \divine{Lord} considered to be evil, living like Jeroboam did and leading Israel into sin.
\labelchapt{16}
\passage{Jehu Rebukes Baasha}

\chapt{16}
\v{1}Later, a message came from the \divine{Lord} to Hanani's son Jehu. It was directed to rebuke Baasha, and this is what it said:

\begin{poetry}
\poeml \v{2}I raised you from the dirt to become Commander-in-Chief\fnote{Lit. \fbib{Nagid}; i.e. a senior officer entrusted with dual roles of operational oversight and administrative authority} over my people Israel, but you've been living like Jeroboam, you've been leading my people Israel into sin, and you've been provoking me to anger with their sins. \v{3}So watch out! I'm going to devour Baasha and his household. I'm going to make your household just like the household of Jeroboam, Nebat's son. \v{4}Anyone from Baasha's household\fnote{The Heb. lacks \fbib{household}} who dies in the city will be eaten by dogs, and anyone of his who dies in the field the birds of the sky will eat.''
\end{poetry}

\v{5}Now the rest of Baasha's accomplishments, including everything that he undertook, as well as his strengths, are recorded in the Book of the Chronicles of the Kings of Israel, are they not? \v{6}Eventually, Baasha died, as had his ancestors, and he was buried in Tirzah. His son Elah was installed as king in his place.

\v{7}In addition, a message from the \divine{Lord} came through Hanani's son Jehu the prophet against Baasha and his household, not only because of all of the things that Baasha\fnote{Lit. \fbib{he}} did that the \divine{Lord} considered to be evil, including provoking the \divine{Lord}\fnote{Lit. \fbib{him}} to anger by what he did and by being like the household of Jeroboam, but also because Baasha\fnote{Lit. \fbib{he}} had destroyed Jeroboam's household.\fnote{Lit. \fbib{destroyed it}}
\passage{Elah Reigns over Israel}

\v{8}During the twenty-sixth year of the reign of\fnote{The Heb. lacks \fbib{the reign of}} King Asa of Judah, Baasha's son Elah became king over Israel and reigned at Tirzah for two years. \v{9}But his servant Zimri, who commanded half of his chariot forces, conspired against Elah while he was drinking himself drunk in the home of Arza, who managed the household at Tirzah. \v{10}Zimri went inside, attacked him, and killed him in the twenty-seventh year of the reign of King Asa of Judah, and then became king in Elah's place. \v{11}As soon as he had consolidated his reign, he executed the entire household of Baasha. He did not leave a single male alive, including any of Baasha's relatives or friends. \v{12}In doing so, Zimri destroyed the entire household of Baasha, in keeping with the message from the \divine{Lord} that he had spoken against Baasha through Jehu the prophet \v{13}because of all the sins that Baasha and his son Elah had committed and because of what they did to lead Israel into sin, thus provoking the \divine{Lord} God of Israel to anger with their idolatry. \v{14}Now the rests of Elah's accomplishments, including everything he undertook, are written in the Book of the Chronicles of the Kings of Israel, are they not?
\passage{Zimri Reigns over Israel}

\v{15}Zimri reigned for seven days at Tirzah during the twenty-seventh year of the reign of\fnote{The Heb. lacks \fbib{the reign of}} King Asa of Judah. At that time, the army was encamped in a siege against Gibbethon of Philistia. \v{16}The army at the encampment heard this report: ``Zimri has conspired against the king and killed him.'' So the entire army of\fnote{The Heb. lacks \fbib{army of}} Israel made Omri, their commander, king over Israel. \v{17}Then Omri and the entire army of\fnote{The Heb. lacks \fbib{army of}} Israel left from Gibbethon and attacked Tirzah. \v{18}When Zimri observed that the city had been captured, he retreated into the king's palace, set fire to the citadel, and died when the palace burned down around him \v{19}because of the sins that he committed by doing what the \divine{Lord} considered to be evil, living like Jeroboam did, and sinning so as to lead Israel into sin. \v{20}The rest of Zimri's accomplishments, including his conspiracy that he carried out, are written in the Book of the Chronicles of the Kings of Israel, are they not?
\passage{Omri Reigns over Israel and Builds Samaria}

\v{21}The army\fnote{Or \fbib{people}} of Israel was divided into two parties: half of the army\fnote{Or \fbib{people}} were loyal to Ginath's son Tibni and wanted to make him king, and half were loyal to Omri. \v{22}But the army\fnote{Or \fbib{people}} that was loyal to Omri was victorious over Ginath's son Tibni. Tibni later died and Omri became king. \v{23}During the thirty-first year of the reign of\fnote{The Heb. lacks \fbib{the reign of}} King Asa of Judah, Omri became king over Israel. He reigned for twelve years, six of them at Tirzah. \v{24}He bought the hill of Samaria from Shemer for two talents\fnote{I.e. about 150 pounds; a talent weighed about 75 pounds} of silver, fortified the hill, and named the city Samaria after Shemer, the former owner of the hill. \v{25}Omri practiced what the \divine{Lord} considered to be evil, doing far more evil than anyone who had reigned before him. \v{26}He lived just like Nebat's son Jeroboam, and by his sin he led Israel into sin, provoking the \divine{Lord} God of Israel with their idolatry. \v{27}Now the rest of Omri's accomplishments, including the power that he demonstrated, are recorded in the Book of the Chronicles of the Kings of Israel, are they not? \v{28}So Omri died, as had his ancestors, and he was buried in Samaria. His son Ahab became king in his place.
\passage{Ahab Reigns over Israel and Marries Jezebel}

\v{29}Omri's son Ahab became king over Israel in the thirty-eighth year of King Asa of Judah. He\fnote{Lit. \fbib{Omri's son Ahab}} reigned over Israel in Samaria for 22 years. \v{30}Omri's son Ahab practiced more of what the \divine{Lord} considered to be evil than anyone who had lived before him. \v{31}In fact, as if it were nothing for him to live like Nebat's son Jeroboam, Ahab married Jezebel, the daughter of King Ethbaal of Sidon. Then he went out to serve Baal and worship him. \v{32}He built an altar for Baal in a temple for Baal that he constructed in Samaria. \v{33}Ahab also erected an Asherah, doing more to provoke the \divine{Lord} God of Israel than all of the kings of Israel who had reigned before him. \v{34}It was during Ahab's reign that Hiel the Bethelite rebuilt Jericho. He laid its foundations just as his firstborn son Abiram was dying, and he erected its gates while his youngest son Segub was dying, thus fulfilling the message that the \divine{Lord} delivered through Nun's son Joshua.\fnote{Cf. Josh 6:26}
\labelchapt{17}
\passage{Elijah Calls for a Drought}

\chapt{17}
\v{1}Elijah the foreigner,\fnote{Lit. \fbib{Tishbite}; or \fbib{sojourner}} who was an alien resident from Gilead, told Ahab, ``As the \divine{Lord} God of Israel lives, in whose presence I'm standing, there will be neither dew nor rain these next several years, except when I say so.''

\v{2}Later, this message came to him from the \divine{Lord}: \v{3}``Leave here and go into hiding at the Wadi\fnote{I.e. a seasonal stream or river that channels water during rain seasons but is dry at other times} Cherith, where it enters the Jordan River.\fnote{The Heb. lacks \fbib{River}; and so throughout the chapter} \v{4}You will be able to drink from that brook, and I've commanded some crows to sustain you there.''

\v{5}So Elijah\fnote{Lit. \fbib{he}} left and did exactly what the \divine{Lord} had told him to do---he went to live near the Wadi\fnote{I.e. a seasonal stream or river that channels water during rain seasons but is dry at other times} Cherith, where it enters the Jordan River. \v{6}Crows would bring him bread and meat both in the morning and in the evening, and he would drink from the brook. \v{7}But after a while,\fnote{Lit. \fbib{But at the end of days}} the brook dried up because there had been no rain in the land.
\passage{Elijah Visits the Widowed Mother of Zarephath}

\v{8}Then this message came to him from the \divine{Lord}: \v{9}``Get up, move to Zarephath in Sidon, and stay there. Look! I've commanded a widow to sustain you there.''

\v{10}So he got up and went to Zarephath. As he arrived at the entrance to the city, a widow was there gathering sticks. So he asked her, ``Please, may I have some water in a cup so I can have a drink.'' \v{11}While she was on her way to get the water, he called out to her, ``Would you please also bring me a piece of bread while you're at it?''\fnote{Lit. \fbib{bread in your hand}}

\v{12}``As the \divine{Lord} your God lives,'' she replied, ``I don't have so much as a muffin, just a handful of flour in a bowl and some oil left in a bottle. Now I'm going to find some sticks so I can cook a last meal for my son and for me. Then we're going to eat it and die.''

\v{13}But Elijah told her, ``You can stop being afraid. Go and do what you said, but first make me a muffin and bring it to me. Then make a meal for yourself and for your son, \v{14}because this is what the \divine{Lord} God of Israel says: `That jar of flour will not run out, nor will that bottle of oil become empty until the very day that the \divine{Lord} sends rain on the surface of the ground.'\,''

\v{15}So she went out and did precisely what Elijah told her to do. As a result, Elijah,\fnote{Lit. \fbib{he}} the widow,\fnote{Lit. \fbib{she}} and her son\fnote{Lit. \fbib{household}} were fed for days. \v{16}The jar of flour never ran out and the bottle of oil never became empty, just as the \divine{Lord} had promised\fnote{Lit. \fbib{spoken}} through\fnote{Lit. \fbib{through the hand of}} Elijah.
\passage{Elijah Restores the Widow's Son}

\v{17}Sometime later, the son of the woman who owned the house became ill. In fact, his illness became so severe that he died.\fnote{Lit. \fbib{that no breath remained in him}} \v{18}``What do we have in common, you man of God?'' she accused Elijah. ``You came to me so you could uncover my guilt! And you're responsible for the death of my son!''

\v{19}``Give me your son,'' he replied. Then he took him from her lap, carried him upstairs to the room where he lived, and laid him on his bed. \v{20}Then he called out to the \divine{Lord} and asked him, ``\divine{Lord} my God, have you also brought evil to this dear widow with whom I am living as her guest? Have you caused the death of her son?'' \v{21}Then he stretched himself three times and cried out to the \divine{Lord}, ``\divine{Lord} my God, please cause the soul of this little boy to return to him.''

\v{22}The \divine{Lord} listened to Elijah, and the soul of the little boy returned to him, and he revived. \v{23}Then Elijah took the little boy downstairs from the upper chamber back into the main house and delivered him to his mother. ``Look,'' Elijah told her, ``your son is alive.''

\v{24}The woman responded to Elijah, ``Now at last I've really learned that you are a man of God and that what you have to say about the \divine{Lord}\fnote{Lit. \fbib{that the word of the \divine{Lord} in your mouth}} is the truth.''
\labelchapt{18}
\passage{Elijah Rebukes Ahab}

\chapt{18}
\v{1}Quite some time later---three years later!---this message from the \divine{Lord} came to Elijah: ``Go visit Ahab, and I'll send some rain to the surface of the ground.'' \v{2}So Elijah went to show himself to Ahab, right when the famine in Samaria was most severe.

\v{3}Ahab called for Obadiah, his household supervisor. This man, who feared the \divine{Lord} very much, \v{4}had taken 100 prophets and had hidden them by fifties in a cave, providing them with food and water when Jezebel was trying to destroy the \divine{Lord}'s prophets.

\v{5}Ahab had instructed Obadiah, ``Go throughout the land to all of the water springs and to all of the valleys. Maybe we'll find some grass to keep the horses and mules alive. Also, maybe we won't have to kill some of our cattle.'' \v{6}So they divided the land between them so they could conduct their survey. Ahab went off by himself in one direction and Obadiah went off by himself in the other.

\v{7}While Obadiah was on the road, Elijah met him. Obadiah recognized him and bowed down with his face to the ground. ``It's you, isn't it, my master Elijah?''

\v{8}``I am,'' he replied. ``Go tell your master, `Look! Elijah!'\,''

\v{9}But Obadiah replied, ``What did I do wrong, that you would put me in a position where Ahab would execute me? \v{10}As surely as the \divine{Lord} your God lives, there isn't a nation or kingdom where my master hasn't tried to find you. Whenever they would say `He isn't here,' he forced that kingdom or nation to swear that they hadn't seen you. \v{11}But now you're saying `Go tell your master, ``Elijah is here!''\,' \v{12}As soon as I've left you, the Spirit of the \divine{Lord} will carry you off to I don't know where! Then when I go tell Ahab and he can't find you, he'll kill me, even though I have been your servant and have feared the \divine{Lord} since I was young! \v{13}Hasn't anyone told you, my master, what I did when Jezebel was killing the \divine{Lord}'s prophets? I hid 100 of the \divine{Lord}'s prophets by fifties in a cave and provided food and water for them. \v{14}Now you're saying, `Go tell your master, ``Elijah's here!''\,' He's sure to kill me!''

\v{15}But Elijah promised him, ``As the \divine{Lord} of the Heavenly Armies lives, in whose presence I stand, I will appear to Ahab today.''

\v{16}So Obadiah went out to meet Ahab and reported to him. Then Ahab went to meet Elijah. \v{17}When Ahab saw Elijah, Ahab asked him, ``Is it really you, you destroyer of Israel?''

\v{18}But Elijah\fnote{Lit. \fbib{he}} replied, ``I'm no destroyer of Israel. But you and your ancestor's household have been doing that, because you have abandoned the \divine{Lord}'s commandments and have followed the Baals. \v{19}So go gather all of Israel to meet me on Mount Carmel. Bring along 450 prophets of Baal and 400 prophets of the Asherah who are funded at Jezebel's expense.''\fnote{Lit. \fbib{who eat at Jezebel's table}}
\passage{Elijah Defeats the Prophets of Baal}

\v{20}Ahab sent for the Israelis and brought the prophets together at Mount Carmel, \v{21}where Elijah approached all the people and asked them, ``How long will you keep hesitating\fnote{Lit. \fbib{dancing}; or \fbib{limping}} between both sides? If the \divine{Lord} is God, go after him. If Baal, go after him.''

But the people didn't say a word.

\v{22}So Elijah told the people, ``I'm the only one left over as a prophet of the \divine{Lord}, am I? But Baal's prophets number 450 men? \v{23}So let them provide two oxen. They can choose one ox for themselves. Cut it up, lay it on top of some wood, but don't set fire to it. I will prepare the other ox and lay it on top of some wood, and I won't set fire to it. \v{24}Then you can call on the name of your god, and I'll call on the name of the \divine{Lord}. Let the God who answers by fire be our God!''

``That's a good idea!'' all the people shouted.

\v{25}So Elijah told the prophets of Baal, ``Choose an ox for yourselves and you prepare it first, since there are so many of you. Call on the name of your god, but don't set fire to the offering.''

\v{26}So they took the ox that was given to them, prepared it, and called on the name of Baal from early morning until noon. ``Baal! Answer us!'' they cried. But there was no response. Nobody answered. So they kept on dancing\fnote{Or \fbib{limping}} around the altar that they had made.

\v{27}Starting about noon, Elijah began to tease them:

``Shout louder!

``He's a god, so maybe he's busy.

``Maybe he's relieving himself.

``Maybe he's busy someplace.

``Maybe he's taking a nap and somebody needs to wake him up.''

\v{28}So the prophets of Baal\fnote{Lit. \fbib{So they}} cried even louder and slashed themselves with swords and lances until their blood gushed out all over them, as was their custom. \v{29}They kept on raving right through midday and until it was time to offer the evening sacrifice, but there was still no response. Nobody answered, and nobody paid attention.

\v{30}Eventually, Elijah told everybody, ``Come here!'' So everybody approached him, and he repaired the \divine{Lord}'s altar that had been torn down. \v{31}Elijah took twelve stones, one for each of the tribes of Jacob's descendants, to whom the message from the \divine{Lord} had come that ``Israel is to be your name.'' \v{32}So Elijah used the stones to build an altar to the name of the \divine{Lord}. But then he dug a trench around the altar large enough to hold two measures\fnote{Lit. \fbib{seahs}; or \fbib{hold four gallons}; i.e. a trench encircling the altar and wide enough that a container holding about four gallons could be laid inside it} of seed. \v{33}Then he laid the wood in order, cut the bull into pieces, and laid them on top of the wood.

``Fill four pitchers with water,'' he ordered. ``Then pour them out on the burnt offering and the wood.''

\v{34}``Do it a second time,'' he ordered. So they did it a second time.

``Do it a third time,'' he said. So they did it a third time. \v{35}The water ran down around the altar and completely filled the trench.\fnote{Lit. \fbib{trench with water}}
\passage{Elijah's Prayer and God's Answer by Fire}

\v{36}As the time for the evening offering arrived, Elijah the prophet approached and said, ``\divine{Lord} God of Abraham, Isaac, and Israel, let it be known today that you are God in Israel and that I, your servant, have done all of this in obedience to your word. \v{37}Answer me, \divine{Lord}! Answer me so that this people may know that you, \divine{Lord}, are God, and that you are turning back their hearts again.''

\v{38}Right then the \divine{Lord}'s fire fell and consumed the burnt offering, the wood, the stones, the dust, and even the water that was in the trench! \v{39}When all the people saw what had happened, they fell flat on their faces and cried out ``The \divine{Lord} is God! The \divine{Lord} is God!''

\v{40}But Elijah said, ``Arrest the prophets of Baal. Don't let even one of them get away.'' So the people\fnote{Lit. \fbib{So they}} seized them, and Elijah brought them down to the Wadi\fnote{I.e. a seasonal stream or river that channels water during rain seasons but is dry at other times} Kishon and executed them there.
\passage{The Rain Storm Ends the Drought}

\v{41}After this, Elijah told Ahab, ``Get up and have something to eat and drink, because there's the sound of a coming rainstorm.'' \v{42}So Ahab got up to get something to eat and drink while Elijah went back up to the top of Mount\fnote{The Heb. lacks \fbib{Mount}} Carmel, where he bowed low to the ground and placed his face between his knees.

\v{43}Then he told his young servant, ``Go and look toward the sea.''

So he went and looked out to sea. ``Nothing there,'' he said.

But Elijah told him to go back seven times. \v{44}On the seventh look, he said, ``Look! There's a cloud, a small one, about the size of a man's hand. It's coming up out of the sea!''

``Get up and find Ahab!'' Elijah\fnote{Lit. \fbib{he}} said. ``Tell him, `Mount your chariot and ride down the mountain\fnote{The Heb. lacks \fbib{the mountain}} so the storm doesn't stop you.'\,''

\v{45}A little while later, the sky turned black with storm clouds and winds, and there was a heavy shower. So Ahab rode off to Jezreel. \v{46}After Ahab had left,\fnote{Lit. \fbib{Then}} the hand of the \divine{Lord} came upon Elijah, and he tucked his mantle into his belt and outran Ahab in a race to the city gate of Jezreel.
\labelchapt{19}
\passage{Elijah Runs from Jezebel}

\chapt{19}
\v{1}Ahab complained to Jezebel about everything that Elijah had done, especially the part about him killing all the prophets of Baal with a sword. \v{2}Jezebel sent a messenger to tell Elijah, ``May the gods do the same to me and even more if tomorrow about this time I haven't made you like one of those prophets you had killed.''\fnote{The Heb. lacks \fbib{prophets you had killed}}

\v{3}Elijah was terrified, so he got up and ran for his life to Beer-sheba, which is part of Judah, and left his servant there \v{4}and ran for a day's journey deep into the wilderness. He found a juniper tree, sat down under it, and prayed that he could die. He asked God, ``Enough! \divine{Lord}! Take my life, because I'm not better than my ancestors!'' \v{5}Then he lay down and went to sleep under the juniper tree. All of a sudden, there was an angel, who kept grabbing him and telling him, ``Get up! Eat!''

\v{6}So he looked around, and there near his head was a muffin sitting on top of some heated stones, along with a jar of water. Elijah ate and drank and then lay down again. \v{7}Later, the angel of the \divine{Lord} came a second time, grabbed him, and said ``Get up! Eat! The journey ahead\fnote{The Heb. lacks \fbib{ahead}} is too difficult for you!'' \v{8}So Elijah\fnote{Lit. \fbib{he}} got up, ate and drank, and survived on that one meal for 40 days and nights as he set out on his journey to Horeb, God's mountain.
\passage{Elijah Talks to God at Horeb}

\v{9}Elijah\fnote{Lit. \fbib{He}} arrived at a cave and stayed there. All of a sudden this message came from the \divine{Lord}: ``What are you doing here, Elijah?''

\v{10}``I've been very zealous for the \divine{Lord} God of the Heavenly Armies,'' he replied. ``The Israelis have abandoned your covenant, demolished your altars, executed your prophets with swords, and I---that's right, just me!---am the only one left. Now they're seeking my life, to get rid of me!''

\v{11}``Go out,'' he responded, ``and stand on the mountain in the presence of the \divine{Lord}.'' And there was the \divine{Lord}, passing by! A tremendous, mighty windstorm was tearing at the mountains and breaking the rocks in pieces in the presence of the \divine{Lord}, but the \divine{Lord} was not in the windstorm. After the wind there came an earthquake, but the \divine{Lord} was not in the earthquake. \v{12}After the earthquake there came fire, but the \divine{Lord} was not in the fire. And after the fire, there was the sound of a gentle whisper. \v{13}As soon as Elijah heard it, he covered his face in his mantle, went outside, and stood at the entrance to the cave. And there a voice spoke to him and said, ``What are you doing here, Elijah?''

\v{14}``I've been very zealous for the \divine{Lord} God of the Heavenly Armies,'' he replied. ``The Israelis have abandoned your covenant, demolished your altars, executed your prophets with swords, and I---that's right, just me!---am the only one left. Now they're seeking my life, to get rid of me!''

\v{15}The \divine{Lord} replied to him, ``Go! Return to Damascus, and when you get there, anoint Hazael as king over Aram, \v{16}anoint Nimshi's son Jehu as king over Israel, and anoint Shaphat's son Elisha from Abel-meholah as a prophet to replace you. \v{17}Whoever escapes from Hazael's sword Jehu will execute, and whoever escapes from Jehu's sword Elisha will put to death. \v{18}Nevertheless, I've reserved 7,000 in Israel who have neither bowed their knees to Baal nor kissed him.''
\passage{Elisha Chosen to Replace Elijah}

\v{19}Elijah left there and located Shaphat's son Elisha, who was plowing, along with a total of\fnote{The Heb. lacks \fbib{a total of}} twelve pairs of oxen.\fnote{The Heb. lacks \fbib{of oxen}} (He was plowing with the twelfth pair.) As Elijah passed by, he tossed his cloak at Elisha.\fnote{Lit. \fbib{him}} \v{20}He abandoned the oxen, ran off to follow Elijah, and asked him, ``Please, let me kiss my mother and father good-bye, and then I'll come after you.''

``Go back again,'' Elijah replied. ``What have I done to you?''

\v{21}So Elisha\fnote{Lit. \fbib{he}} turned back, took the pair of oxen, sacrificed them, boiled their flesh using the farm implements for fuel, and gave the food to the people with him.\fnote{The Heb. lacks \fbib{with him}} Then he got up, followed Elijah, and became his servant.
\labelchapt{20}
\passage{Ahab Attacks the Arameans}

\chapt{20}
\v{1}A little while later, King Ben-hadad of Aram mustered an army of cavalry and chariots in a military confederacy with 32 kings, invaded Samaria, and set up siege encampments there. \v{2}Then he sent envoys to visit King Ahab of Israel and told him, ``This is what Ben-hadad says: \v{3}`Your silver and gold belong to me. So do the most beautiful of your wives and children.'\,''

\v{4}``Whatever you want, your majesty,'' the king of Israel answered. ``I belong to you, as does everything I own.''

\v{5}After delivering Ahab's answer,\fnote{The Heb. lacks \fbib{After delivering Ahab's answer}} the envoys returned with this message: ``This is what Ben-hadad says: `I've sent my envoys to you to tell you that your silver, gold, wives, and children are to be given to me. \v{6}About this time tomorrow, I'll send my servants to you, and they'll search through your palace and your servants' houses. Whatever is important to you will be seized\fnote{Lit. \fbib{seized in their hand}} and taken away.'\,''

\v{7}Then the king of Israel called together all of the elders of the land and told them, ``Please note that this man is here looking for trouble. He sent a message to me, demanding my wives, my children, and my silver and gold, and I haven't refused him.''

\v{8}``Don't listen to him,'' all the elders and the people replied. ``And don't agree to his terms.''\fnote{The Heb. lacks \fbib{to his terms}}

\v{9}So he told Ben-hadad's envoys, ``Tell his majesty the king, `Everything that you asked for the first time I will do, but this thing I cannot do.'\,'' So the envoys left to deliver Ahab's response. They\fnote{Lit. \fbib{deliver and}} returned a little while later.

\v{10}Beh-hadad sent this message back: ``May the gods do so to me, and more than that also, if the dust that remains of Samaria is enough to fill up a few handfuls for all of the armies at my disposal.''

\v{11}But the king of Israel replied, ``Tell him, `The one who is starting to strap on his battle armor should never brag like the one who is taking it off.'\,''

\v{12}Ben-hadad received Ahab's response\fnote{Lit. \fbib{message}} while he was celebrating with his kings in the battle pavilions. ``Sound `Battle Stations!'\,'' he ordered, and the army began to prepare their attack.
\passage{God's Prophets Rebuke Ahab}

\v{13}Right about then, a prophet approached King Ahab of Israel and told him, ``This is what the \divine{Lord} says: `You see all of this great big army, do you? Well now, I'm going to deliver them all right into your hand, and you will learn that I am the \divine{Lord}!'\,''

\v{14}``By whom?'' Ahab asked.

``This is what the \divine{Lord} says,'' the prophet replied. ```By the young men who serve as officials within the provinces.'\,''

``Who is to begin the battle?'' Ahab asked.

``You,'' the prophet answered.

\v{15}So Ahab\fnote{Lit. \fbib{he}} gathered together 232 young men who served as officials within the provinces and then mustered 7,000 soldiers from among the Israelis. \v{16}They attacked at noon, just as Ben-hadad was drinking himself drunk in the battle pavilions, along with the 32 kings who had joined him. \v{17}The young men who served as officials within the provinces led the charge, and somebody informed Ben-hadad, ``Some men have come out from Samaria.''

\v{18}``Take them alive, whether they've come in peace or not,'' he ordered.

\v{19}Meanwhile, as the young men who served as officials within the provinces left the city, their army followed after them. \v{20}Each man struck down his opponent, and the Arameans ran away with Israel in pursuit. King Ben-hadad of Aram escaped on horseback with the help of\fnote{The Heb. lacks \fbib{the help of}} his cavalry. \v{21}The king of Israel went out and attacked the cavalry and chariots and killed the Arameans in a massive victory.\fnote{;21 Or \fbib{slaughter}}

\v{22}The prophet approached the king of Israel and told him, ``Go replenish your forces and prepare for the future, because early this next year the king of Aram will attack you again.''
\passage{The Arameans are Defeated}

\v{23}Sure enough, the advisors to the king of Aram told him, ``Their gods are mountain gods. That's why they were stronger than we were. But when we fight them on the plains, we're certain to be the stronger army! \v{24}So do this: remove the kings from command\fnote{The Heb. lacks \fbib{from command}} and replace them with captains. \v{25}Then replace the army that you lost, horse-for-horse and chariot-for-chariot. We'll fight them on the plains, and we're certain to be the stronger army.'' Ben-hadad\fnote{Lit. \fbib{He}} listened to what they had to say and carried out their advice.

\v{26}Early the next year, Ben-hadad mustered the Arameans and invaded Aphek in a battle against Israel. \v{27}The Israelis were mustered, equipped with provisions, and sent out to fight. The Israeli encampment looked like two little flocks of goats compared to how the Aramean encampments\fnote{The Heb. lacks \fbib{encampments}} filled the countryside!

\v{28}Right about then, a man of God approached and told the king of Israel, ``This is what the \divine{Lord} says: `Because the Arameans keep saying ``The \divine{Lord} is a mountain god, but isn't a valley god,'' I'm going to deliver this entire vast army right into your control, so you'll learn that I really am the \divine{Lord}.'\,'' \v{29}So they remained in opposing camps for seven days. Then on the seventh day the battle commenced, and the Israelis killed 100,000 Aramean infantry troops in a single day. \v{30}The rest of the Aramean army retreated into Aphek, but the city wall collapsed on 27,000 soldiers who had taken shelter there. Ben-hadad himself ran away and hid inside a closet\fnote{Lit. \fbib{inside an inner room}} somewhere in the city.

\v{31}``Look, now,'' his advisors suggested, ``we've heard that the Israeli kings are merciful. So let's clothe ourselves with sackcloth, tie our hair back with ropes, and go out to the king of Israel. Maybe he'll spare your life.'' \v{32}So they put on some sackcloth, tied their hair back with ropes, and approached the king of Israel. ``Your servant Ben-hadad says this,'' they said. ``Please let me live.''

``Is he still alive?'' Ahab asked. ``He's my brother.''

\v{33}Ben-hadad's advisors,\fnote{Lit. \fbib{The men}} quickly analyzing the signs in what Ahab was saying, responded, ``Yes, your brother Ben-hadad.''

``Go get him,'' Ahab responded. So Ben-hadad came out to him, and Ahab took him up into his personal chariot.

\v{34}Ben-hadad made this promise to Ahab: ``I will restore the cities that my ancestors took from your ancestors. You'll be able to build streets named after yourself in Damascus, as my father did in Samaria.''

``With this promise I will release you,'' Ahab\fnote{Lit. \fbib{he}} replied. So Ahab\fnote{Lit. \fbib{he}} made a treaty with Ben-hadad\fnote{Lit. \fbib{him}} and let him go.
\passage{Ahab is Condemned}

\v{35}Right about then, one of the members of the guild\fnote{Lit. \fbib{sons}} of prophets told another through a message from the \divine{Lord}: ``Please strike me!'' But the man refused to do so, \v{36}so he told him, ``Because you haven't obeyed the \divine{Lord}'s voice, as soon as you leave here, a lion will kill you.'' As soon as the man left, a lion found him and killed him.

\v{37}Later, he found another man and told him, ``Please strike me!'' So the man struck him and wounded him. \v{38}Then the prophet left and waited for the king to pass by, disguising himself with a bandage over his eyes.

\v{39}As the king was passing by, he cried out to the king and told him, ``Your servant went out into the middle of the battle, and a soldier turned aside, brought a prisoner to me, and told me, `Guard this man. If he turns up missing for any reason at all, you'll pay for it with your life or be fined one talent\fnote{I.e. about 75 pounds} of silver.' \v{40}While your servant was busy here and there, the prisoner escaped.''

The king told him, ``By your actions you've earned the proper judgment!''

\v{41}Then the prophet quickly tore off his bandage, and the king of Israel recognized him as being one of the prophets. \v{42}He told the king,\fnote{Lit. \fbib{told him}} ``This is what the \divine{Lord} says: `Because you let the man whom I had dedicated to destruction go free, therefore your life is to be forfeited for his life, and your people for his people.'\,''

\v{43}After hearing this, the king of Israel rode back to his palace in Samaria, frustrated and in a foul mood.
\labelchapt{21}
\passage{The Naboth Vineyard Incident}

\chapt{21}
\v{1}Meanwhile, there was a man named Naboth from Jezreel who owned a vineyard that was located contiguous to King Ahab's palace in Samaria. \v{2}Ahab addressed Naboth and asked him, ``I would like to plant a vegetable garden near my house. Please exchange your vineyard with a better one from me, or if you'd rather have cash, I'll buy it for its full value.''

\v{3}But Naboth replied to Ahab, ``No way! The \divine{Lord} prohibits the sale to you of the inheritance of my ancestors!''

\v{4}Ahab went back to his palace, sullen and in a foul mood, because Naboth the Jezreelite had turned down Ahab's offer by saying ``I will not transfer my ancestors' inheritance to you!'' He laid down on his bed, curled up with his face to the wall, and refused to eat.

\v{5}But his wife Jezebel went to him and asked him, ``How is it that you're so sullen and refusing to eat?''

\v{6}``I asked Naboth the Jezreelite, `Sell me your vineyard for cash, or if you want, I'll give you a better one in its place.' But he refused. He told me, `I won't give you my vineyard!'\,''

\v{7}``Aren't you the reigning king of Israel,'' his wife Jezebel replied. ``Get up, have a meal, and get ready to be happy. I'll go get you the vineyard that Naboth the Jezreelite owns.'' \v{8}So she wrote some memos in Ahab's name, set his personal seal to them, and sent them to the elders and nobles who lived with Naboth in his city. \v{9}In the memos, she wrote the following directives: ``Proclaim a public fast and seat Naboth in the front row. \v{10}Seat two wicked men in front of him, and make them testify against him. Tell them to claim `You cursed God and the king.' Then take him out and stone him to death.''

\v{11}So the leading men of the city, along with the elders and nobles who lived there, did precisely what Jezebel had directed them to do. They followed the instructions that she had set forth in the memos: \v{12}They proclaimed a public fast and seated Naboth in the front row. \v{13}Two wicked men came in, sat down in front of them, and testified against Naboth in public, ``Naboth cursed God and the king!'' So they took him outside the city and stoned him to death.\fnote{Lit. \fbib{death with stones}}

\v{14}Afterwards, they sent a message\fnote{The Heb. lacks \fbib{a message}} to Jezebel that said, ``Naboth has been stoned. He's dead.''

\v{15}When Jezebel heard that Naboth had been stoned to death, she told Ahab, ``Get up and confiscate Naboth's vineyard that he refused to sell you for cash. Naboth the Jezreelite isn't alive anymore. He's dead!'' \v{16}So once he heard that Naboth was dead, Ahab got up, went down to the vineyard of Naboth the Jezreelite, and confiscated it.
\passage{Elijah Rebukes Ahab}

\v{17}That's when this message from the \divine{Lord} came to Elijah the foreigner:\fnote{Lit. \fbib{Tishbite}; or \fbib{sojourner}} \v{18}``Get up and go down to meet King Ahab of Israel. He's in Samaria. Look! He's in Naboth's vineyard, where he's gone to confiscate it. \v{19}Ask the king, `Did you commit murder? And now you're going to steal as well?' Also tell him, `This is what the \divine{Lord} says: ``Where the dogs were licking up Naboth's blood, dogs will also lick up your blood---that's right---yours!''\,'\,''

\v{20}Later on, Ahab asked Elijah, ``Have you found me, my enemy?''

But Elijah answered, ``I've found you because you sold yourself to do what the \divine{Lord} considers to be evil! \v{21}Now pay attention! I'm going to send evil in your direction! I will completely sweep you away and eliminate from Ahab every male, whether indentured servant or free, throughout Israel. \v{22}I will make your household resemble that of Nebat's son Jeroboam, or like the household of Ahijah's son Baasha, because of how you've provoked me to anger and made Israel to sin. \v{23}The \divine{Lord} also has this to say about Jezebel: `Dogs will eat Jezebel within the outer ramparts of Jezreel. \v{24}Dogs will eat whoever belongs to Ahab and who dies in the city. The birds of the sky will eat whoever dies in the fields.'\,''

\v{25}It can be truly said that no one else sold himself to practice what the \divine{Lord} considered to be evil quite like the way Ahab did, because his wife Jezebel incited him. \v{26}His behavior in pursuing idolatry was detestable, just like the Amorites had done whom the \divine{Lord} had expelled in front of the army of Israel. \v{27}Nevertheless, as soon as Ahab heard this message, he tore his clothes, put on sackcloth, and fasted. He even slept in sackcloth and wandered around meekly.

\v{28}Later, this message from the \divine{Lord} came to Elijah the foreigner:\fnote{Lit. \fbib{Tishbite}; or \fbib{sojourner}} \v{29}``Have you noticed that Ahab has humbled himself in my presence? Because he has humbled himself in my presence, I will not bring his evil to harvest\fnote{The Heb. lacks \fbib{to harvest}} during his lifetime, but I will bring evil to his household during his son's lifetime.''
\labelchapt{22}
\passage{King Ahab Invites Jehoshaphat to Invade Aram}
\passageinfo{(2 Chronicles 18:1-11)}

\chapt{22}
\v{1}Three years passed without war between Aram and Israel. \v{2}During that third year, King Jehoshaphat of Judah went to visit the king of Israel. \v{3}The king of Israel asked his servants, ``Were you aware that Ramoth-gilead belongs to us, but we aren't doing anything to remove it from the control of the king of Aram?''

\v{4}Then he asked Jehoshaphat, ``Will you join me in battle against Ramoth-gilead?''

``I'm with you,'' Jehoshaphat answered the king of Israel. ``My army will join yours, and my cavalry will be your cavalry.'' \v{5}But Jehoshaphat also asked the king of Israel, ``Please ask for a message from the \divine{Lord}, first.''

\v{6}So the king of Israel called in about 400 prophets and asked them, ``Should we go attack Ramoth-gilead, or should I call off the attack?''\fnote{The Heb. lacks \fbib{the attack}}

``Go attack them,'' they all said, ``because the Lord will drop them right into the king's hand!''

\v{7}But Jehoshaphat asked, ``Isn't there a prophet of the \divine{Lord} left here that we could talk to?''

\v{8}``There is still one man left by whom we could ask the \divine{Lord} what to do,''\fnote{The Heb. lacks \fbib{what to do}} the king of Israel replied to Jehoshaphat, ``but I hate him because he never prophesies anything good about me. Instead, he prophesies evil. He is Imla's son Micaiah.''

But Jehoshaphat rebuked Ahab, ``Kings\fnote{Lit. \fbib{The king}} should never talk like that.''

\v{9}Nevertheless, the king of Israel called one of his officers and ordered him, ``Bring me Imla's son Micaiah quickly.''

\v{10}Now the king of Israel and King Jehoshaphat of Judah were each sitting on their respective thrones, arrayed in their robes, on the threshing floor at the entrance to the city gate of Samaria, and all of the prophets were prophesying in front of them. \v{11}Chenaanah's son Zedekiah made iron horns for himself and told them, ``This is what the \divine{Lord} says, `With these horns you are to gore the Arameans until they are eliminated!'\,''

\v{12}All the other prophets were saying similar things, like ``Go up to Ramoth-gilead and you will be successful, because the \divine{Lord} will hand it over to the king!''
\passage{Micaiah Predicts Failure}
\passageinfo{(2 Chronicles 18:12-27)}

\v{13}Meanwhile, the messenger who had gone off to summon Micaiah advised him, ``Look, everything that the other prophets were saying was unanimously favorable to the king. So please, cooperate with them and speak favorably.''

\v{14}``As the \divine{Lord} lives,'' Micaiah replied, ``I'll say what my God tells me to say.''

\v{15}When Micaiah\fnote{Lit. \fbib{he}} approached the king, the king asked him, ``Micaiah, should we go to war against Ramoth-gilead, or should I not?''

``Go to war,'' Micaiah\fnote{Lit. \fbib{he}} replied, ``and you will be successful, because the \divine{Lord} will hand it over to the king!''

\v{16}When he heard this, the king asked him, ``How many times do I have to make you swear to tell me nothing but the truth? Now do it in the name of the \divine{Lord}!''

\v{17}So Micaiah replied:

\begin{poetry}
\poeml ``I saw all of Israel \\
\poemll    scattered on the mountains \\
\poemlll       like sheep without a shepherd. \\
\poeml And the \divine{Lord} told me, \\
\poemll    `These have no master, \\
\poemlll       so let them each return to his own home in peace.'\,''
\end{poetry}

\v{18}Then the king of Israel told Jehoshaphat, ``Didn't I tell you that he wouldn't prophesy anything good about me, but only evil?''

\v{19}But Micaiah responded, ``Therefore, listen to what the \divine{Lord} has to say. I saw the \divine{Lord}, sitting on his throne, and the entire Heavenly Army was standing around him on his right hand and on his left hand.

\v{20}``The \divine{Lord} asked, `Who will tempt King Ahab of Israel to attack Ramoth-gilead, so that he will die there?' And one was saying one thing and one was saying another.

\v{21}``But then a spirit approached, stood in front of the \divine{Lord}, and said, `I will entice him.'

\v{22}``And the \divine{Lord} asked him, `How?'

```I will go,' he announced, `and I will be a deceiving spirit in the mouth of all of his prophets!'

``So the \divine{Lord} said, `You're just the one to deceive him. You will be successful. Go and do it.'

\v{23}``Now therefore, listen! The \divine{Lord} has placed a lying spirit in the mouth of all of these prophets of yours, because the \divine{Lord} has determined to bring disaster upon you.''

\v{24}Right then, Chenaanah's son Zedekiah approached Micaiah and struck him on the cheek. Then he asked him, ``How did the Spirit of the \divine{Lord} move from me to speak to you?''

\v{25}Micaiah replied, ``You'll see how when the day comes that you run away to hide yourself in a closet!''

\v{26}Then the king of Israel ordered, ``Take Micaiah and place him in the custody of Amon, the city governor. Hand him over to Joash, the king's son. \v{27}Give him this order: `Place him in prison on survival rations of bread and water only until I come back safely.'\,''

\v{28}``If you return alive,'' Micaiah responded, ``then the \divine{Lord} has not spoken by me.'' Then he added, ``Listen, all you people!''
\passage{Ahab Dies at Ramoth-gilead}
\passageinfo{(2 Chronicles 18:28-34)}

\v{29}So the king of Israel and King Jehoshaphat of Judah both attacked Ramoth-gilead. \v{30}The king of Israel suggested to Jehoshaphat, ``I'll go into battle in disguise, but you keep your royal uniform on.'' So the king of Israel disguised himself and they both went into the battle.

\v{31}Meanwhile, the king of Aram had issued these orders to 32 of his chariot commanders: ``Don't attack unimportant soldiers or ranking officers. Go after only the king of Israel.''

\v{32}So when the chariot commanders observed Jehoshaphat, they said by mistake,\fnote{The Heb. lacks \fbib{by mistake}} ``It's the king of Israel!'' and they turned aside to attack him. But Jehoshaphat cried out. \v{33}When the chariot commanders saw that their target\fnote{Lit. \fbib{that he}} was not the king of Israel, they stopped pursuing him.

\v{34}Meanwhile, somebody drew his bow aimlessly and struck the king of Israel between the scales where his armor breastplates joined, so he instructed his chariot driver, ``Turn around and take me out of the battle, because I've been severely wounded.'' \v{35}The battle continued on for the rest of the day while the king of Israel was propped up in front of the Arameans until the sun set, at which time he died. The blood from Ahab's wound ran down into the bottom of the chariot.

\v{36}As the day drew to a close, this order was circulated throughout the army telling the soldiers, ``Everybody go back to his city and to his own land.'' \v{37}So the king died and was brought back to Samaria, and they buried the king in Samaria. \v{38}They washed the chariot by the reservoir of Samaria, and the dogs licked up his blood near where the prostitutes went to bathe, in keeping with the message that the \divine{Lord} had spoken.

\v{39}Now as to the rest of Ahab's accomplishments, everything that he undertook, the ivory palace he built, and the cities that he built, they are written in the Book of the Chronicles of the Kings of Israel, are they not? \v{40}That's how Ahab died, just as his ancestors had, and his son Ahaziah became king in his place.
\passage{Jehoshaphat Reigns over Judah}

\v{41}Asa's son Jehoshaphat became king over Judah during the fourth year of the reign of\fnote{The Heb. lacks \fbib{the reign of}} King Ahab of Israel. \v{42}Jehoshaphat was 35 years old when he became king. He reigned 25 years in Jerusalem. His mother's name was Azubah. She was the daughter of Shilhi. \v{43}He lived like his father Asa and never abandoned that life. He did what the \divine{Lord} considered to be right. Nevertheless, the high places were not demolished, and the people continued to sacrifice and burn incense on the high places.\fnote{This last sentence of v 43 is v44 in MT, v44 is v45 in MT, and so through the rest of the chapter.} \v{44}Jehoshaphat also made a peace treaty with the king of Israel.

\v{45}Now the rest of Jehoshaphat's accomplishments, the power that he demonstrated, and how he waged war are written in the book of the Chronicles of the Kings of Judah, are they not? \v{46}He also eliminated the male cult prostitutes who still remained from the time of his father Asa.

\v{47}There was no king reigning in Edom; there was only a stand-in\fnote{Or \fbib{deputy}} king. \v{48}Jehoshaphat had ocean-going vessels from Tarshish sail to Ophir\fnote{Or \fbib{a source of fine gold}; cf. 1Chr 29:4} for gold, but they never made it because they were shipwrecked at Ezion-geber. \v{49}Ahab's son Ahaziah had offered to go. ``Let my servants go with your servants in the ships!'' he said. But Jehoshaphat was not willing. \v{50}Later, Jehoshaphat died, as did his ancestors, and he was buried alongside his ancestors in the City of David. Jehoram his son became king in his place.
\passage{Ahaziah Reigns over Israel}

\v{51}Ahab's son Ahaziah became king over Israel in Samaria in the seventeenth year of King Jehoshaphat of Judah. He reigned for two years over Israel. \v{52}He practiced what the Lord considered to be evil by living life like his father and mother did. He lived like Nebat's son Jeroboam, who led Israel into sin. \v{53}He served Baal, worshipped him, and provoked the \divine{Lord} God of Israel to anger, in accordance with everything his father had done.

\bookheader{2 Kings}
\labelbook{2King}

\bookpretitle{The Book of}
\booktitle{Second Kings}

\labelchapt{1}
\passage{Elijah Rebukes King Ahaziah}

\chapt{1}
\v{1}Moab rebelled against Israel\fnote{\fbackref{1:1} Cf. 2Sam 8:2} after Ahab died. \v{2}Meanwhile, Ahaziah had fallen through the lattice in his upper room in Samaria and lay injured. He sent messengers to Ekron with these orders: ``Go and consult with Ekron's god Baal-zebub to find out\fnote{\fbackref{1:2} The Heb. lacks \fbib{and find out}} if I'm going to recover from this injury.''\fnote{\fbackref{1:2} Lit. \fbib{sickness}}

\v{3}But the angel of the \divine{Lord} spoke to Elijah the foreigner,\fnote{\fbackref{1:3} Lit. \fbib{Tishbite}; or \fbib{sojourner}} ``Get up and go meet the messengers from the king of Samaria. Ask them `Is it because there is no God in Israel that you're going to consult with Ekron's god Baal-zebub? \v{4}Now therefore this is what the \divine{Lord} says: ``You won't be getting up from that bed of yours on which you're lying. You will most certainly die!''\,'\,'' So Elijah got up and\fnote{\fbackref{1:4} The Heb. lacks \fbib{got up and}} went.

\v{5}The messengers returned to the king and he asked them, ``What's this? You've come back?''

\v{6}They replied, ``We met a man who told us, `Go back to the king who sent you and ask him, ``Is it because there is no God in Israel that you're going to consult with Baal-zebub, the god of Ekron? Therefore you won't be getting up from that bed on which you're lying. You will most certainly die!''\,'\,''

\v{7}He told them, ``Describe the man who met you and told you these things.''

\v{8}They answered, ``The man was a hairy fellow. He wore a leather sash around his waist.''

The king\fnote{\fbackref{1:8} Lit. \fbib{He}} responded, ``It's Elijah, that foreigner!''\fnote{\fbackref{1:8} Lit. \fbib{Elijah the Tishbite}; or \fbib{Elijah, the sojourner}}
\passage{Fire from Heaven Destroys the King's Henchmen}

\v{9}So the king sent out 50 men, along with their leader.\fnote{\fbackref{1:9} Lit. \fbib{a captain of fifty}; and so through v. 14; modern equivalent to a Second Lieutenant commanding a platoon of four twelve-member squads} The leader\fnote{\fbackref{1:9} Lit. \fbib{He}} approached Elijah, who was sitting at the top of a hill. He ordered Elijah,\fnote{\fbackref{1:9} Lit. \fbib{him}} ``Hey, man of God! The king orders you to come down!''

\v{10}Elijah responded to the leader who was in charge of the 50 soldiers, ``So I'm a man of God, am I? If so, may fire\fnote{\fbackref{1:10} MT word \fbib{fire} sounds like MT word \fbib{man}} fall from heaven and devour you and your 50 soldiers{\ldots}''\fnote{\fbackref{1:10} The Heb. lacks \fbib{soldiers}} Just then, fire fell from heaven and devoured that leader and his 50 soldiers.\fnote{\fbackref{1:10} The Heb. lacks \fbib{soldiers}}

\v{11}Later the king tried again---he sent another company of 50 soldiers, along with their leader, who ordered Elijah, ``Hey, man of God! This is what the king orders: `Come down!'\,''

\v{12}Elijah responded to the leader and to his entire company,\fnote{\fbackref{1:12} Lit. \fbib{to them}} ``So I'm a man of God, am I? If so, may fire\fnote{\fbackref{1:12} MT word \fbib{fire} sounds like MT word \fbib{man}} fall from heaven and devour you and your 50 soldiers{\ldots}''\fnote{\fbackref{1:12} The Heb. lacks \fbib{soldiers}} Just then, fire fell from heaven and devoured him and his 50 soldiers.\fnote{\fbackref{1:12} The Heb. lacks \fbib{soldiers}}

\v{13}Then the king tried yet again! The king sent a third company of 50 soldiers along with their leader. The third leader went up the hill,\fnote{\fbackref{1:13} The Heb. lacks \fbib{the hill}} approached Elijah,\fnote{\fbackref{1:13} The Heb. lacks \fbib{Elijah}} fell on his knees in front of him, and begged him,\fnote{\fbackref{1:13} Lit. \fbib{Elijah}} ``Hey, man of God, please treat\fnote{\fbackref{1:13} Lit. \fbib{see}} my life and the lives of these servants of yours as precious! \v{14}Look how fire fell from heaven and devoured the two other companies of 50 soldiers, along with their captains, but now please treat me as if my life were precious!''

\v{15}The angel of the \divine{Lord} told Elijah, ``Go down the hill with that man. Don't be afraid of him!'' So Elijah\fnote{\fbackref{1:15} Lit. \fbib{he}} got up and went down with him to meet the king.

\v{16}Then Elijah spoke to the king, ``This is what the \divine{Lord} says: `Since you sent messengers to consult with Baal-zebub, the god of Ekron---is it because there is no God in Israel with whom to consult regarding his word?---therefore you're not getting up from the bed on which you're lying. You certainly will die!'\,'' \v{17}And die he did, just as the \divine{Lord} had said and just as Elijah had spoken!

After this, Jehoram ascended to the throne during the second year of the reign of Jehoshaphat's son Jehoram from Judah. He took the place of Ahaziah, who had no son. \v{18}The rest of Ahaziah's activities are recorded in the Book of the Chronicles of the Kings of Israel,\fnote{\fbackref{1:18} An ancient chronicle of Israel, apparently now lost; and so throughout the book} are they not?
\labelchapt{2}
\passage{Elijah is Taken to Heaven}

\chapt{2}
\v{1}As the time drew near when the \divine{Lord} was about to take Elijah to heaven in a wind storm, Elijah and Elisha were on their way from Gilgal. \v{2}Elijah instructed Elisha, ``Remain here on this side, please, because the \divine{Lord} is sending me as far as Bethel.''

But Elisha replied, ``As the \divine{Lord} lives, I'm not going to leave you while you're still alive!'' So they both went on to Bethel.

\v{3}When the Guild of Prophets\fnote{\fbackref{2:3} Lit. \fbib{The children of the prophets}; and so throughout the book; i.e. a group of disciples within Israel's prophetic community whose precise association with Elijah and Elisha is never specified} who lived in Bethel came out to greet Elisha, they asked him, ``You are aware, aren't you, that later today the \divine{Lord} is going to remove your master from being your mentor?''\fnote{\fbackref{2:3} Lit. \fbib{your head over you}}

``Of course I'm aware of it,'' he said. ``Calm down.''

\v{4}Elijah also spoke to him, ``Elisha, remain here on this side, please, because the \divine{Lord} is sending me to Jericho.''

But Elisha responded, ``As the \divine{Lord} lives, and while you're still alive, I'm not going to leave you!'' So they went to Jericho.

\v{5}The Guild of Prophets who lived in Jericho approached Elisha and asked him, ``You are aware, aren't you, that later today the \divine{Lord} is going to remove your master from being your mentor?''\fnote{\fbackref{2:5} Lit. \fbib{your head over you}}

``Of course I'm aware of it,'' he said. ``Calm down.''

\v{6}Elijah also spoke to him, ``Elisha, remain here on this side, please, because the \divine{Lord} is sending me to the Jordan River.''\fnote{\fbackref{2:6} The Heb. lacks \fbib{River}}

But Elisha responded, ``As the \divine{Lord} lives, and while you're still alive, I'm not going to leave you!'' So they went on their way,\fnote{\fbackref{2:6} The Heb. lacks \fbib{their way}} \v{7}accompanied by 50 men from the Guild of Prophets, who stood at a short distance from them while they were both standing by the Jordan. \v{8}Elijah took off his ornamented cloak, wrapped it up in a roll, struck the water, and all of a sudden the water divided into two parts! One side of the river stood still opposite the other until the two of them crossed over on dry ground.

\v{9}When they had crossed the Jordan River,\fnote{\fbackref{2:9} The Heb. lacks \fbib{the Jordan River}} Elijah invited Elisha, ``Ask me what you want me to do for you before I'm taken away from you.''

So Elisha asked, ``Please, may there be a double portion of\fnote{\fbackref{2:9} Or \fbib{double desire for}; Lit. \fbib{double mouth for}} your spirit upon me!''

\v{10}``That's a hard thing to ask for,'' Elijah answered, ``but if you see me while I'm being taken from you, it will happen for you. But if you don't see me, it won't happen.''

\v{11}As they continued on, talking as they went, suddenly chariots blazing with fire and pulled by fiery horses appeared, separated the two of them, and Elijah ascended in a wind storm to heaven! \v{12}As Elisha continued to watch, he cried out, ``My father! My father! The chariots of Israel and its cavalry!''\fnote{\fbackref{2:12} The Heb. word \fbib{cavalry} can refer to \fbib{horses}, to \fbib{horsemen}, or to both} Then he did not see Elijah anymore.

After this, Elisha\fnote{\fbackref{2:12} Lit. \fbib{he}} gripped his clothes that he was wearing, tore them apart into two pieces, \v{13}picked up Elijah's ornamented cloak that had fallen from him, and went back to stand on the bank of the Jordan River.\fnote{\fbackref{2:13} The Heb. lacks \fbib{River}} \v{14}Elisha\fnote{\fbackref{2:14} Lit. \fbib{He}} took hold of Elijah's ornamental cloak that had been left behind,\fnote{\fbackref{2:14} Lit. \fbib{had fallen from him}} struck the water, and cried out: ``Where is the \divine{Lord} God of Elijah?'' All of a sudden, after he had struck the water, the water divided into two parts! One side of the river stood opposite the other, and Elisha crossed over.
\passage{Elisha is Recognized as Elijah's Successor}

\v{15}As soon as the Guild of Prophets who lived adjacent to Jericho saw Elisha,\fnote{\fbackref{2:15} Lit. \fbib{him}} they began to announce, ``The spirit\fnote{\fbackref{2:15} Or \fbib{Spirit}} of Elijah is at rest on Elisha!'' So they came out to meet him and they greeted him by bowing low to the ground in front of him.

\v{16}Then they asked Elisha,\fnote{\fbackref{2:16} Lit. \fbib{him}} ``Look! We have 50 valiant men here with your servant! Please let them go out and search for your master Elijah.\fnote{\fbackref{2:16} The Heb. lacks \fbib{Elijah}} Perhaps the Spirit of the \divine{Lord} has taken him up on a mountain or into a valley.''

Elisha responded, ``Don't bother searching.''

\v{17}But they persisted until he was frustrated, so he said, ``Send them out!'' So they sent out the 50 men, and they looked around for three days but did not find Elijah.\fnote{\fbackref{2:17} Lit. \fbib{him}} \v{18}By the time they returned, Elisha\fnote{\fbackref{2:18} Lit. \fbib{he}} was living in Jericho. Then Elisha asked them, ``Didn't I tell you not to go?''
\passage{Elisha Cures the Waters of Jericho}

\v{19}The men who lived in the city addressed Elisha. ``Look now,'' they said, ``our\fnote{\fbackref{2:19} Lit. \fbib{The}} city's location is good, as you\fnote{\fbackref{2:19} Lit. \fbib{as my Lord}} have been observing, but the water springs\fnote{\fbackref{2:19} Lit. \fbib{the waters}} here are bad and the land isn't sustaining crops.''

\v{20}Elisha ordered them, ``Bring me a new bowl and put some salt in it.'' So they brought him what he had requested.\fnote{\fbackref{2:20} The Heb. lacks \fbib{what he had requested}}

\v{21}Elisha went out to the springs, threw the salt into them, and declared, ``This is what the \divine{Lord} says: `I have purified these waters. Neither death nor barrenness is to flow from them anymore.'\,'' \v{22}As a result, the water springs\fnote{\fbackref{2:22} Lit. \fbib{the waters}} remain pure to this day, just as\fnote{\fbackref{2:22} Lit. \fbib{as the word of}} Elisha had declared.
\passage{Elisha Rebukes Some Mockers}

\v{23}Later, Elisha\fnote{\fbackref{2:23} Lit. \fbib{he}} left there to go up to Bethel, and as he was traveling along the road, some insignificant\fnote{\fbackref{2:23} Or \fbib{small}; i.e. in significance, not stature or age; the individuals were adults} young men came from the city and started mocking him. They told him, ``Get on up,\fnote{\fbackref{2:23} The taunt may be an allusion to Elijah's experience described in v. 11.} baldy! Get on up, baldy!'' \v{24}He looked behind him, took note of the young men, and cursed them in the name of the \divine{Lord}. Suddenly two female bears emerged from the woods and mauled 42 of the young men. \v{25}After this, he left from there to go to Mt. Carmel, and from there he went back to Samaria.
\labelchapt{3}
\passage{Jehoram Becomes King}

\chapt{3}
\v{1}Ahab's son Jehoram ascended to the throne of Israel at Samaria during the eighteenth year of the reign of Judah's king Jehoshaphat. He reigned for twelve years, \v{2}practicing evil in the \divine{Lord}'s presence,\fnote{\fbackref{3:2} Lit. \fbib{sight}} only not to the extent that his mother and father had done\fnote{\fbackref{3:2} The Heb. lacks \fbib{had done}}---he forced abolition of the sacred pillar to Baal\fnote{\fbackref{3:2} I.e. the main Canaanite male deity, and so throughout the book} that his father had crafted. \v{3}Even so,\fnote{\fbackref{3:3} Lit. \fbib{Only}} he kept on committing the sins that Nebat's son Jeroboam had done, which ensnared Israel in sin---he never abandoned them.
\passage{Moab Rebels against Israel}

\v{4}Meanwhile, Moab's king Mesha was a sheep breeder. He used to pay 100,000 lambs and the wool from 100,000 rams to the king of Israel as tribute. \v{5}After Ahab died, the king of Moab rebelled against the king of Israel. \v{6}So king Jehoram left Samaria at that time\fnote{\fbackref{3:6} Lit. \fbib{in those days}} and mustered the entire army of\fnote{\fbackref{3:6} The Heb. lacks \fbib{army of}} Israel. \v{7}As he was going out, he sent this message\fnote{\fbackref{3:7} The Heb. lacks \fbib{this message}} to King Jehoshaphat of Judah: ``The king of Moab has rebelled against me. Will you go with me to fight Moab?''

``I'm coming,'' Jehoshaphat\fnote{\fbackref{3:7} Lit. \fbib{He}} replied. ``I'm like you! My army will act like your army and my cavalry like your cavalry,'' Then Jehoshaphat\fnote{\fbackref{3:7} Lit. \fbib{He}} added: \v{8}``What road do we take?''

Jehoram\fnote{\fbackref{3:8} Lit. \fbib{He}} answered, ``We'll go along the Edom desert road.''

\v{9}So the king of Israel, the king of Judah, and the king of Edom made a complete circuit on the road for seven days, but there was no water for the army or for the livestock that accompanied\fnote{\fbackref{3:9} Or \fbib{followed}} them.

\v{10}Then the king of Israel remarked, ``Oh no! The \divine{Lord} has summoned us three kings so he can hand us over to Moab, hasn't he?''
\passage{The Kings Seek Elisha's Counsel}

\v{11}Jehoshaphat asked, ``Isn't there a prophet who belongs to the \divine{Lord} and through whom we can ask the \divine{Lord} a question?''

One of the king of Israel's attendants replied, ``Shaphat's son Elisha lives here. He used to be Elijah's personal attendant.''\fnote{\fbackref{3:11} \fbib{to pour water on Elijah's hands}}

\v{12}Jehoshaphat answered, ``He receives messages from\fnote{\fbackref{3:12} Lit. \fbib{He has the word of}} the \divine{Lord}.'' So the king of Israel, Jehoshaphat, and the king of Edom went to visit Elisha.\fnote{\fbackref{3:12} Lit. \fbib{him}}

\v{13}Elisha asked the king of Israel, ``What do I have in common\fnote{\fbackref{3:13} The Heb. lacks \fbib{in common}} with you? Go visit your parents' prophets.''\fnote{\fbackref{3:13} Lit. \fbib{the prophets of your father and the prophets of your mother}}

The king of Israel replied, ``No! The \divine{Lord} has summoned these three kings so he can hand them over to Moab!''

\v{14}But Elisha responded, ``As the \divine{Lord} of the Heavenly Armies lives, in whose presence I stand, I would never pay attention to you or even look in your direction were it not for my continuous respect for the presence of King Jehoshaphat of Judah. \v{15}Now bring me a musician.''

As the musician played, the hand of the \divine{Lord} rested on Elisha, \v{16}so he said, ``This is what the \divine{Lord} says: `Fill this valley with trench after trench!' \v{17}This is what the \divine{Lord} says: `Though you won't see wind or storm, nevertheless that river\fnote{\fbackref{3:17} Or \fbib{seasonal stream}} will overflow with water so that you, your cattle, and your livestock may drink.' \v{18}And this is the easy part for the \divine{Lord}\fnote{\fbackref{3:18} Lit. \fbib{part in the \divine{Lord}'s eyes}}---he's also going to hand the Moabites over to you! \v{19}Then you are to attack every fortified city and every significant city. Cut down every significant tree, fill in all of the water springs, and ruin every prime piece of land with stones.''
\passage{War with Moab}

\v{20}The very next day, about the time of the morning offering, water suddenly appeared, coming from the direction of Edom, and the land overflowed with water! \v{21}Meanwhile, all the Moabites heard that the kings had come up to attack them, so everyone old enough to wear battle armor was mustered to stand guard at the border. \v{22}As the Moabites arose early that morning, the sun cast its rays on the water, and to the Moabites, the water across from them appeared to be red like blood. \v{23}So they concluded,\fnote{\fbackref{3:23} Lit. \fbib{said}} ``This must be blood! The kings must have had one mighty big fight and each man killed the other! So let's go get the battle spoil, Moab!''

\v{24}But when the Moabites arrived at the Israeli encampment, the Israelis got up and attacked them. The Moabites ran away from the Israelis,\fnote{\fbackref{3:24} Lit. \fbib{of them}} who followed them into the land as they continued their pursuit against Moab. \v{25}They destroyed their cities, and all of them threw stones onto every piece of farm\fnote{\fbackref{3:25} Or \fbib{good}} land, ruining the fields.\fnote{\fbackref{3:25} I.e. for future cultivation} Then they filled in all the water wells\fnote{\fbackref{3:25} Or \fbib{springs}} and chopped down all of the useful\fnote{\fbackref{3:25} Or \fbib{good}} trees. Stone walls remained surrounding Kir-hareseth only, until the archers surrounded and attacked that city. \v{26}When the king of Moab realized that the battle was going strongly against him, he took 700 expert swordsmen to attempt to break through to the king of Edom, but was unable to do so. \v{27}So he took his firstborn son, whom he intended to reign after him, and offered him up as a burnt offering on the wall. There subsequently came great anger against Israel, so they abandoned the attack and returned to their homeland.
\labelchapt{4}
\passage{The Miracle of the Oil Vessels}
\passageinfo{(1 Kings 17:14-16)}

\chapt{4}
\v{1}Now there happened to be a certain woman who had been the wife of a member of the Guild of Prophets. She cried out to Elisha, ``My husband who served you has died, and you know that your servant feared the \divine{Lord}. But a creditor has come to take away my children into indentured servitude!''

\v{2}Elisha responded, ``What shall I do for you? Tell me what you have in your house.''

She replied, ``Your servant has nothing in the entire house except for a flask of oil.''

\v{3}He told her, ``Go out to all of your neighbors in the surrounding streets and borrow lots of pots from them. Don't get just a few empty vessels, either. \v{4}Then go in and shut the door behind you, taking only your children, and pour oil\fnote{\fbackref{4:4} The Heb. lacks \fbib{oil}} into all of the pots. As each one is filled, set it aside.''

\v{5}So she left Elisha,\fnote{\fbackref{4:5} Lit. \fbib{him}} shut the door behind her and her children, and while they kept on bringing vessels to her, she kept on pouring oil.\fnote{\fbackref{4:5} The Heb. lacks \fbib{oil}} \v{6}When the last of\fnote{\fbackref{4:6} The Heb. lacks \fbib{last of}} the vessels had been filled, she told her son, ``Bring me another pot!''

But he replied, ``There isn't even one pot left.'' Then the oil stopped flowing. \v{7}After this, she went and told the man of God what had happened.\fnote{\fbackref{4:7} The Heb. lacks \fbib{what had happened}} So he said, ``Go sell the oil, pay your debt, and you and your children will be able to live on the proceeds.''
\passage{The Hospitality of a Woman from Shunem}

\v{8}Some time later, Elisha went to Shunem,\fnote{\fbackref{4:8} I.e. a town in the territory belonging to Issachar} where he met a prominent and wealthy\fnote{\fbackref{4:8} Lit. \fbib{strong}} woman who persuaded him to have a meal with her. As a result, whenever he was in the area, he stopped by to eat with her. \v{9}So she had a talk with her husband. ``Look here! I've learned that this is a holy and godly man\fnote{\fbackref{4:9} Or \fbib{a holy man of God}} who comes by here on a regular basis. \v{10}Now then, let's build a small upper room and put a bed in it for him there, along with a table, a chair, and a lamp stand. That way, when he comes to visit, he can rest\fnote{\fbackref{4:10} Or \fbib{can turn in}} there.''

\v{11}One day, Elisha\fnote{\fbackref{4:11} Lit. \fbib{he}} came by to visit and stopped in to rest in the upper chamber. \v{12}He told his attendant\fnote{\fbackref{4:12} Lit. \fbib{his young man}; and so throughout the chapter} Gehazi, ``Call this Shunammite.'' So when he had summoned her, she stood in front of him.

\v{13}Elisha\fnote{\fbackref{4:13} Lit. \fbib{He}} told him, ``Ask her, `Look how you've gone to all this trouble to care for us! What can I do for you? Do you wish to be mentioned to the king or to the head of the army?'\,''

She replied, ``I'm at home\fnote{\fbackref{4:13} So LXX; the Heb. lacks \fbib{at home}} living among my own people.''

\v{14}He responded, ``What, then, is to be done on her behalf?''

Gehazi answered, ``Well, she has no son and her husband is growing old.''

\v{15}``Call her,'' Elisha\fnote{\fbackref{4:15} Lit. \fbib{he}} ordered. After he called her, she came and stood in the doorway, \v{16}and he told her, ``About this time next year you will be embracing a son.''

``No, sir! Please, as a godly man,\fnote{\fbackref{4:16} Or \fbib{a man of God}} don't mislead your servant!'' \v{17}But the woman did conceive and did bear a son at that very same time the next year, just as Elisha had told her.
\passage{Elisha Raises the Shunammite's Son}
\passageinfo{(1 Kings 17:17-24)}

\v{18}After the child had grown up a bit, one day he went out to visit his father, who was with the harvesters. \v{19}He told his father, ``My head! My head!''

So his father ordered his servant, ``Carry him over to his mother!'' \v{20}So the servant carried him over to his mother, where he rested on her lap until mid-day,\fnote{\fbackref{4:20} Or \fbib{noon}} and then he died. \v{21}The woman went upstairs, laid him on the bed belonging to the man of God, and shut the door, leaving him behind as she left.

\v{22}Then she called to her husband and asked him, ``Please send me one of the servants, along with one of the donkeys, so I can ride quickly to see that godly man.\fnote{\fbackref{4:22} Or \fbib{that man of God}} I'll be right back.''

\v{23}He asked her, ``What's the point of visiting him today? It's not a New Moon, and it isn't the Sabbath!''

But she kept saying, ``Things will go well.''\fnote{\fbackref{4:23} Lit. \fbib{Peace}; i.e. a general statement of good will; and so through v. 26}

\v{24}So she saddled a donkey and told her servant, ``Forward, driver! Don't slow down on my account, unless I tell you!'' \v{25}So out she went and eventually she arrived at Mount Carmel to visit the man of God.

When the man of God noticed her from a distance, he told his attendant Gehazi, ``Look! There's the woman from Shunem! \v{26}Please run out quickly and greet her. Ask her, `Are things going well with you? Are things going well with your husband? Are things going well with your child?'\,''

She answered Gehazi,\fnote{\fbackref{4:26} The Heb. lacks \fbib{Gehazi}} ``Things are going well.''

\v{27}As she came near the man of God on the mountain, she grabbed his feet. When Gehazi intervened to push her away, the man of God said, ``Leave her alone! She is deeply troubled! The \divine{Lord} has concealed the thing from me, and hasn't informed me.''

\v{28}Then she asked, ``Did I ask my lord for a son? Didn't I beg you, `Don't mislead me?'\,''

\v{29}At this he told Gehazi, ``Get ready to run!\fnote{\fbackref{4:29} Lit. \fbib{Tie up your garments}; i.e. to secure one's robes with a belt in preparation for running} Take my staff in your hand, and get on the road. Don't greet anyone you meet. If anyone greets you, don't respond. Just go lay my staff on the youngster's face.''

\v{30}At this, the youngster's mother replied, ``As long as you and the \divine{Lord} live, I'm not leaving you!'' So he got up and followed her.

\v{31}Meanwhile, Gehazi went on ahead of them and placed the staff on the youngster's face, but when there was no sound or reaction, he returned, met Elisha,\fnote{\fbackref{4:31} Lit. \fbib{him}} and told him, ``The youngster has shown no sign of awakening.''

\v{32}When Elisha entered the house, there was the youngster, dead and laid out on Elisha's\fnote{\fbackref{4:32} Lit. \fbib{his}} bed! \v{33}So he entered, shut the door behind them both, and prayed to the \divine{Lord}. \v{34}Then he approached the child and lay down with his mouth near the child's, with his eyes near those of the child, and taking the child's hands in his. As Elisha\fnote{\fbackref{4:34} Lit. \fbib{he}} stretched himself on the child, the child's flesh began to grow warm. \v{35}Then he went downstairs, walked around back and forth inside the house once, went back up to his upper chamber,\fnote{\fbackref{4:35} The Heb. lacks \fbib{up to his upper chamber}} and stretched himself over the child again. The young man sneezed seven times and then opened his eyes.

\v{36}He called out to Gehazi, ``Go get the Shunammite woman!'' So he called her. When she came in to see Elisha,\fnote{\fbackref{4:36} Lit. \fbib{him}} he told her, ``Take back your son!'' \v{37}Then she approached him, fell at his feet, bowing low to the ground, took back her son, and went out.
\passage{Poisoned Stew is Purified}

\v{38}Elisha returned to Gilgal during a time of famine in the land. While the Guild of Prophets were having a meal\fnote{\fbackref{4:38} Lit. \fbib{were sitting}} with him, he instructed his attendant, ``Put a large pot on the fire and boil some stew for the Guild of Prophets.'' \v{39}Somebody went out into the fields to grab some herbs, found a wild vine, and gathered a lap full of wild gourds, which he came and sliced up into the stew pot, but nobody else knew.

\v{40}When they served the men, they began to eat the stew. But they cried out, ``That pot of stew is deadly, you man of God!'' So they couldn't eat the stew.

\v{41}But he replied, ``Bring me some flour.'' He tossed it into the pot and said, ``Serve the people so they can eat.'' Then there was nothing harmful in the pot.
\passage{Feeding of the Crowd}
\passageinfo{(Matthew 14:13-31; 15:32-39)}

\v{42}Later on, a man arrived from Baal-shalishah, bringing the man of God some bread as a first fruit offering. He had 20 loaves of barley and ripe ears of corn in his sack. So Elisha\fnote{\fbackref{4:42} Lit \fbib{he}} said, ``Give them to the people so they can eat.''

\v{43}Elisha's attendant asked, ``What? Will this serve 100 men?''

But he replied, ``Distribute it to the people so they can eat, because this is what the \divine{Lord} says: `They will eat and have a surplus!'\,'' \v{44}So he served them, and they ate and had some left over, just as the \divine{Lord} had indicated.
\labelchapt{5}
\passage{The Healing of Naaman}

\chapt{5}
\v{1}Naaman, the commander of the army of the king of Aram,\fnote{\fbackref{5:1} I.e., ancient Assyria, and so throughout the book} was a great man in the opinion\fnote{\fbackref{5:1} Lit. \fbib{eyes}} of his master. He was highly favored, because by him the \divine{Lord} had given victory to Aram. Though he was a mighty and valiant man, he was suffering from leprosy. \v{2}On one of their raids to the territory of Israel, Aram had taken captive a young girl when she was an infant,\fnote{\fbackref{5:2} Or \fbib{young little girl}; cf. v. 14; i.e., a young girl of small size} who had eventually become an attendant to\fnote{\fbackref{5:2} Lit. \fbib{girl, and she was in the presence of}} Naaman's wife. \v{3}She mentioned to her mistress, ``If only my master were to visit the prophet who is in Samaria! He would cure him of his leprosy.''

\v{4}Later, Naaman\fnote{\fbackref{5:4} Lit. \fbib{he}} went to inform his master and told him something like this: ``Thus and so spoke the young woman from the territory of Israel.''

\v{5}The king of Aram replied, ``Go now, and I'll send a letter to the king of Israel.'' So he left and took with him ten talents\fnote{\fbackref{5:5} I.e. about 750 pounds; a talent weighed about 75 pounds} of silver and 6,000 units\fnote{\fbackref{5:5} The unit of weight is unspecified.} of gold, along with ten sets\fnote{\fbackref{5:5} So MT; LXX reads \fbib{changes}} of clothing. \v{6}He also brought the letter to the king of Israel, which read as follows: ``{\ldots}and now as this letter finds its way to you, look! I've sent my servant Naaman to you so you may heal him of his leprosy.''

\v{7}When the king of Israel read the letter, he ripped his clothes and cried out, ``Am I God? Can I kill and give life? Is this man sending me a request\fnote{\fbackref{5:7} The Heb. lacks \fbib{a request}} to heal a man's leprosy? Let's think about this---he's looking for a reason to start a fight\fnote{\fbackref{5:7} The Heb. lacks \fbib{to start a fight}} with me!''

\v{8}When Elisha the man of God heard that the king of Israel had torn his clothes, he sent a message\fnote{\fbackref{5:8} The Heb. lacks \fbib{a message}} to the king and asked, ``Why did you tear your clothes? Please, let the man come visit me and he will learn that there is a prophet in Israel!''

\v{9}So Naaman arrived with his horses and chariots and stood in front of the door to Elisha's house. \v{10}Elisha sent a messenger out to him, who told him, ``Go bathe in the Jordan River\fnote{\fbackref{5:10} The Heb. lacks \fbib{River}} seven times. Your flesh will be restored for you. Now stay clean!''

\v{11}But Naaman flew into a rage and left, telling himself, ``Look! I thought `He's surely going to come out to me, stand still, call out in the name of the \divine{Lord} his God, wave his hand over the infection,\fnote{\fbackref{5:11} Lit. \fbib{place}} and cure the leprosy!' \v{12}Aren't the Abana and Pharpar rivers in Damascus better than all of the water in Israel? Couldn't I just bathe in them and become clean?'' So he turned away and left, filled with anger.

\v{13}But then his servants approached him and spoke with him. They said, ``My father, had the prophet only asked of you something great, you would have done it, wouldn't you? Yet he told you, `Bathe, and be clean{\ldots}!'\,'' \v{14}So he went down and plunged himself into the Jordan River\fnote{\fbackref{5:14} The Heb. lacks \fbib{River}} seven times, just as the man of God had said, and his flesh rejuvenated like the flesh of a newborn child. And he was clean.
\passage{Gehazi's Greed is Punished}

\v{15}Naaman\fnote{\fbackref{5:15} Lit. \fbib{He}} went back to the man of God, along with his entire entourage, and stood before him. ``Please look!'' he said. ``I know that there is no God in all the earth, except in Israel! So please, take a present from your servant.''

\v{16}But Elisha\fnote{\fbackref{5:16} Lit. \fbib{he}} replied, ``As the \divine{Lord} lives, before whom I stand, I will not receive anything from you.'' Though Naaman\fnote{\fbackref{5:16} Lit. \fbib{he}} urged him to take it, Elisha\fnote{\fbackref{5:16} Lit. \fbib{he}} declined.

\v{17}So Naaman asked, ``No? Then please let your servant load two mules with dirt from Israel,\fnote{\fbackref{5:17} The Heb. lacks \fbib{from Israel}} because your servant will no longer offer any burnt offering or sacrifice to any other god but the \divine{Lord}. \v{18}In this one area may the \divine{Lord} pardon your servant: Whenever my master enters the temple of Rimmon to worship there, he will lean on my hand while I bow down in the temple of Rimmon. So may the \divine{Lord} pardon your servant in this one area.''

\v{19}``Go in peace,'' he said. So Naaman\fnote{\fbackref{5:19} Lit. \fbib{he}} left.

After Naaman had gone only a short distance, \v{20}Gehazi, the attendant to Elisha, the man of God, told himself, ``Look how my master has spared this Aramean, Naaman! He declined to take from him what he brought. As the \divine{Lord} lives, I'm going to run after him and get something from him.'' \v{21}So Gehazi ran after Naaman.

When Naaman noticed someone running after him, he came down from his chariot, greeted him and asked, ``Is everything all right?''\fnote{\fbackref{5:21} Lit. \fbib{Peace}; i.e. a general statement of good will; and so through v. 26}

\v{22}Gehazi said, ``Everything's all right. My master sent me to tell you, `Just now two men from the Guild of Prophets have arrived from the hill country of Ephraim. Please give them each a talent\fnote{\fbackref{5:22} I.e. about 75 pounds; a talent weighed about 75 pounds} of silver bullion and two sets\fnote{\fbackref{5:22} So MT; LXX reads \fbib{changes}} of clothes.'\,''

\v{23}But Naaman said, ``Please accept my invitation to take two talents\fnote{\fbackref{5:23} The Heb. is dual; i.e. about 150 pounds; a talent weighed about 75 pounds} of silver.'' He urged him, binding two talents\fnote{\fbackref{5:23} The Heb. is dual; i.e. about 150 pounds; a talent weighed about 75 pounds} of silver in two bags, along with two sets of clothes. He placed them in the care of two of his young men, and they went on ahead of Gehazi.\fnote{\fbackref{5:23} Lit. \fbib{him}} \v{24}When he arrived at the stronghold, Gehazi\fnote{\fbackref{5:24} Lit. \fbib{he}} took the bags from their custody and hid them away in the house. Then he sent the men away and they left.

\v{25}Later he went to address\fnote{\fbackref{5:25} Or \fbib{to stand before}} his master. Elisha asked him, ``Where did you go, Gehazi?''

``Your servant went nowhere in particular,'' he said.

\v{26}But Elisha\fnote{\fbackref{5:26} Lit. \fbib{he}} responded, ``Didn't my heart break\fnote{\fbackref{5:26} Lit. \fbib{go}} as the man was turning from his chariot to greet you? Is now the time to receive money? To receive clothes? And olive groves, vineyards, sheep, oxen, servants, or female attendants? \v{27}Naaman's leprosy will plague you and your descendants forever!'' As he left Elisha's presence, he was infected with leprosy that looked like white snow.
\labelchapt{6}
\passage{The Miracle of the Ax Head}

\chapt{6}
\v{1}One day the Guild of Prophets told Elisha, ``Notice how the place where we are living is too small for us. \v{2}Let's go to the Jordan River,\fnote{\fbackref{6:2} The Heb. lacks \fbib{River}} fashion some rafters,\fnote{\fbackref{6:2} Lit. \fbib{take a beam}} and build a place for us so we can live there.''

So he said, ``Go!''

\v{3}Someone asked, ``Would you be willing to come with your servants?''

``I'm willing,'' he replied. \v{4}So he accompanied them, and when they came to the Jordan River,\fnote{\fbackref{6:4} The Heb. lacks \fbib{River}} they cut down some trees.

\v{5}It happened that as one of them was felling a beam, his axe head fell into the water. He cried out, ``Oh no! Master! The axe was on loan to me!''

\v{6}The man of God asked, ``Where did it fall?'' When he was shown the place, he cut off a branch, tossed it there, and made the iron axe head float. \v{7}Then Elisha said, ``Pick it up!'' So the young man reached out and picked it\fnote{\fbackref{6:7} The Heb. lacks \fbib{it}} up.
\passage{The Arameans Attack}

\v{8}Eventually the king of Aram went to war against Israel, taking counsel with his advisors and concluding, ``In such and such a place I'll build my encampment.''

\v{9}So the man of God sent a message\fnote{\fbackref{6:9} The Heb. lacks \fbib{a message}} to the king of Israel, warning him, ``Keep an eye on that area, because the Arameans are going to be there!'' \v{10}The king of Israel confirmed the matter\fnote{\fbackref{6:10} Lit. \fbib{Israel sent}} about which the man of God had warned him. Having been forewarned, he was able to protect himself there on more than one or two occasions.

\v{11}The king of Aram flew into a rage over this, so he called in his advisors and asked them, ``Will you please tell me which of us has joined the king of Israel?''

\v{12}``No, your majesty,'' one of his servants said. ``Elisha the prophet, who lives in Israel, tells the king of Israel what you talk about in your bedroom!''

\v{13}So the king\fnote{\fbackref{6:13} Lit. \fbib{So he}} ordered, ``Go and discover where he is, so I may send men\fnote{\fbackref{6:13} The Heb. lacks \fbib{men}} to take him into custody.''

Later somebody told him, ``Look! He's in Dothan!''

\v{14}So the king of Aram\fnote{\fbackref{6:14} Lit. \fbib{So he}} sent out horses, chariots, and an elite force, and they arrived during the night and surrounded the city. \v{15}Meanwhile, the attendant to the man of God got up early in the morning and went outside, and there were the elite forces, surrounding the city, accompanied by horses and chariots! So Elisha's attendant cried out to him, ``Oh no! Master! What will we do!?''

\v{16}Elisha\fnote{\fbackref{6:16} Lit. \fbib{He}} replied, ``Stop being afraid, because there are more with us than with them!'' \v{17}Then Elisha prayed, asking the \divine{Lord}, ``Please make him able to really see!'' And so when the \divine{Lord} enabled the young man to see, he looked, and there was the mountain, filled with horses and fiery chariots surrounding Elisha!

\v{18}When the army approached him, Elisha spoke to the \divine{Lord}, asking him, ``\divine{Lord}, I'm asking you please to afflict this group of people with blindness!'' So he afflicted them with blindness, just as Elisha had asked.

\v{19}Then Elisha told the army, ``This isn't the way, and this isn't the city! Follow me, and I'll bring you to the man you're seeking. Then he led them to Samaria. \v{20}When they arrived in Samaria, Elisha asked the \divine{Lord}, ``Enable them to see again.'' So the \divine{Lord} did so, and there they were---right in the middle of Samaria!

\v{21}When the king of Israel saw Elisha, he asked him, ``Shall I execute them, my father?''

\v{22}But he replied, ``No! You're not to kill them! Would you execute those whom you've taken captive at the point of a sword or with your bow? Give them food and water so they can eat and drink. Then send them back to their master!'' \v{23}So he prepared a large festival for them, and when they had finished eating and drinking, he sent them back to their master, and marauding gangs of Arameans never came into the territory of Israel again.
\passage{Ben-hadad Attacks Samaria}

\v{24}Some time later, King Ben-hadad from Aram mustered his army, invaded the land,\fnote{\fbackref{6:24} The Heb. lacks \fbib{the land}} and attacked Samaria \v{25}until there was a great famine throughout Samaria. The siege lasted until a donkey's head cost\fnote{\fbackref{6:25} The Heb. lacks \fbib{sold}} 80 silver coins\fnote{\fbackref{6:25} The exact weight or denomination of silver coin is unspecified.} and one quarter of a unit\fnote{\fbackref{6:25} Lit. \fbib{a kab}; a unit of dry weight in volume about 1.5 quarts} of dove's dung cost\fnote{\fbackref{6:25} The Heb. lacks \fbib{cost}} five silver coins.\fnote{\fbackref{6:25} The exact weight or denomination of silver coin is unspecified.}

\v{26}While the king of Israel was walking along the city\fnote{\fbackref{6:26} The Heb. lacks \fbib{city}} wall, a woman cried out to him. ``Help me, your majesty!''\fnote{\fbackref{6:26} Lit \fbib{me, my lord the king}} she said.

\v{27}He replied, ``No! Since the \divine{Lord} won't give you victory, how will I be able to deliver you? From the threshing floor? From the wine press?'' \v{28}Then the king asked her, ``What's bothering\fnote{\fbackref{6:28} The Heb. lacks \fbib{bothering}} you?''

She said, ``This woman told me, `Give up your son, and we'll eat him today, and we'll eat your son tomorrow.'\,'' \v{29}So we boiled my son and ate him. The next day, I told her, `Give me your son so we can eat him!' But she has hidden her son!''

\v{30}When the king heard what the woman said, he ripped his garments as he continued walking along the city\fnote{\fbackref{6:30} The Heb. lacks \fbib{city}} wall. As the people watched, all of a sudden they noticed he was wearing sackcloth underneath his clothes, inside next to his flesh! \v{31}He said, ``May God do to me---and more also!---if the head of Shaphat's son Elisha remains on his shoulders\fnote{\fbackref{6:31} Lit. \fbib{on him}} today!''

\v{32}Meanwhile, Elisha was sitting in his house, along with the elders, when the king\fnote{\fbackref{6:32} Lit. \fbib{when he}} sent a man to kill him,\fnote{\fbackref{6:32} The Heb. lacks \fbib{to kill him}} but before the messenger arrived, Elisha\fnote{\fbackref{6:32} Lit. \fbib{he}} told the elders, ``Are you watching how this descendant of murderers has ordered my head be cut off? Look, when the messenger arrives, shut the door and hold it to shut them out! Don't you hear the sound of his master's feet right behind him?''

\v{33}While he was still talking with them, the messenger arrived to see him and delivered the king's message to Elisha,\fnote{\fbackref{6:33} Lit. \fbib{and told him}} ``Look! This evil has come from the \divine{Lord}! Why should I wait for the \divine{Lord} anymore?''
\labelchapt{7}
\passage{Elisha Predicts Deliverance the Next Day}

\chapt{7}
\v{1}So Elisha responded, ``Listen to this message from the \divine{Lord}! `This is what the \divine{Lord} says: ``At about this time tomorrow, in Samaria's city gate, a seah\fnote{\fbackref{7:1} I.e. a dry measure of grain equal to about 2 gallons in volume.} of finely ground flour will sell for a shekel, and two seahs of barley for a shekel.''\,'\,''

\v{2}But the royal attendant on whom the king depended responded to the man of God: ``Look here! Even if the \divine{Lord} were to open a window in the sky, how could this happen?''

He replied, ``No, you look! You'll see it with your eyes, but you won't eat any of it!''
\passage{The Arameans Flee}

\v{3}Now there happened to be four lepers who were at that very moment at the entrance to the city gate. As they were talking with one another, they said, ``Why are we sitting here waiting to die? \v{4}If we tell ourselves, `Let's remain in the city,' we'll die there since there's famine in the city. But if we sit here, we'll die, too. So let's go over\fnote{\fbackref{7:4} Lit. \fbib{let's fall}} to the Arameans! If they spare our lives, we'll live, and if they kill us{\ldots}we're dying anyway!''\fnote{\fbackref{7:4} The Heb. lacks \fbib{anyway}}

\v{5}So they got up at dusk and went out to the Aramean encampment. But when they arrived at the outskirts of the Aramean encampment, there was no one there! \v{6}The \divine{Lord} had made the Aramean army hear the sounds of chariots, horses, and a large army, so they told one another, ``Look! The king of Israel has hired the kings of the Hittites and the Egyptians to come attack us!'' \v{7}So the Arameans\fnote{\fbackref{7:7} Lit. \fbib{So they}} got up and ran away in the gathering darkness. They left behind their tents, horses, and donkeys just as they were---and fled for their lives!

\v{8}When the lepers arrived at the outskirts of the encampment, they entered one tent and ate and drank. Then they carried off from there some silver, gold, and clothes, and went out and hid them. After this, they returned, entered another tent, raided it, and went and hid all of that,\fnote{\fbackref{7:8} The Heb. lacks \fbib{all of that}} too! \v{9}But then they told each other, ``We're not doing the right thing. This is a day of good news, but if we keep quiet until morning, we're sure to be punished! So let's leave and go tell the king's household!'' \v{10}So they left, called out to the city gatekeepers, and reported to them: ``We went out to the Aramean encampment, and there was nobody there! Not even the sound of men---only horses and donkeys tied up, and tents left just as they were!''

\v{11}The gatekeepers announced the report to the king's attendants, \v{12}so the king got up in the middle of the night and ordered his servants: ``Let me explain what the Arameans have done to us. They know that we're hungry, so they've left their encampment to conceal themselves in the surrounding fields. They're telling themselves, `When they come out of the city, we'll capture them alive and enter the city!'\,''

\v{13}One of his attendants suggested, ``Please, let's take five of the remaining horses, since those who remain here will end up like the rest of Israel, which has already died, and we'll send them out to look.'' \v{14}So they took two chariots and horses, and the king sent them out after the Aramean army with the orders, ``Go and look!''
\passage{The Prophecy is Fulfilled}

\v{15}They went out in the direction of the Jordan River,\fnote{\fbackref{7:15} The Heb. lacks \fbib{River}} and the entire roadway was strewn with clothes and equipment that the Arameans had abandoned in their haste to leave!\fnote{\fbackref{7:15} The Heb. lacks \fbib{to leave}} So the messengers returned and reported to the king. \v{16}At this, the people went out and plundered the camp of the Arameans. At that time, a seah\fnote{\fbackref{7:16} I.e. a dry measure of grain equal to about 2 gallons in volume.} of finely ground flour was sold for a shekel, and two seahs of barley for a shekel, in accordance with the \divine{Lord}'s message.

\v{17}Meanwhile, the king appointed the same royal attendant on whom he depended\fnote{\fbackref{7:17} Cf. v. 2} to take control of the city gate, but the people trampled him to death in the gate, just as the man of God had told the king when the king came down to him. \v{18}It happened just as the man of God had spoken to the king:

\begin{poetry}
\poeml ``At about this time tomorrow, in Samaria's city gate, a seah\fnote{\fbackref{7:18} I.e. a dry measure of grain equal to about 2 gallons in volume.} of finely ground flour will sell for a shekel, and two seahs of barley for a shekel.''
\end{poetry}

\v{19}But the royal attendant on whom the king depended responded to the man of God: ``Look here! Even if the \divine{Lord} were to make a window in the sky, how could this happen?''

He replied, ``No, you look! You'll see it with your eyes, but you won't eat any of it!''\fnote{\fbackref{7:19} Cf. v. 1-2}

\v{20}And so it happened to him, because the people trampled him in the city gate and he died.
\labelchapt{8}
\passage{The Shunammite's Land is Restored}

\chapt{8}
\v{1}Meanwhile, Elisha urged the woman whose son he had restored to life, ``You must get up and leave with your household to go live wherever you can, because the \divine{Lord} has called for a famine, and it's going to come over the land for seven years.'' \v{2}So the woman followed the instructions given to her by the man of God, and she went to the territory of the Philistines to live for seven years with her household. \v{3}At the end of the seven years, the woman returned from the territory of the Philistines and went to the king in order to file an appeal regarding her house and her grain field.

\v{4}The king was talking with Gehazi, the attendant of the man of God. He had asked Gehazi, ``Please tell me about all of the great things that Elisha has done.'' \v{5}Just as he was telling the king about Elisha's having restored the dead to life, the woman whose son had been restored arrived and appealed to the king for her house and her land!

Gehazi told the king, ``Your majesty, this is the woman! And here's her son, whom Elisha restored to life!''

\v{6}The king consulted with the woman, who related the story. So the king appointed a court official to represent her and ordered him: ``Restore to her everything that belonged to her, including all of the produce that her fields yielded from the day she left the land until now.''
\passage{The Murder of King Ben-hadad of Aram}

\v{7}Later on, Elisha traveled to Damascus. King Ben-hadad of Aram was ill, but someone informed him, ``The man of God has come here!''

\v{8}So the king told Hazael, ``Take a gift with you and go meet the man of God. Inquire of the \divine{Lord} through him and ask, `Will I recover from this sickness?'\,''

\v{9}So Hazael went out to meet with him and took a gift with him---40 camel loads filled with samples of everything good in Damascus. He approached the man of God\fnote{\fbackref{8:9} Lit. \fbib{approached him}} and said, ``Your son King Ben-hadad from Aram has sent me to you to ask you, `Will I recover from this sickness?'\,''

\v{10}But Elisha told him, ``Go tell him, `You will certainly recover,' but the \divine{Lord} has shown me that he will certainly die.'' \v{11}Then Elisha\fnote{\fbackref{8:11} Lit. \fbib{he}} looked steadily at Hazael\fnote{\fbackref{8:11} The Heb. lacks \fbib{at Hazael}} until Hazael grew ashamed, and then the man of God began to cry.

\v{12}``Why are you crying, sir?'' Hazael asked.

``Because I know the evil that you're about to bring on the Israelis,'' he replied. ``You'll burn down their fortified cities, execute their young men with swords, dash to pieces their little ones, and you'll tear open their pregnant women!''

\v{13}But Hazael responded, ``What? Who am I, your servant, that I should do such a horrible thing?''

But Elisha answered, ``The \divine{Lord} has shown me that you will be king over Aram.''

\v{14}So he left Elisha and returned to his master, who asked him, ``What did Elisha tell you?''

He replied, ``He told me that you would certainly get better.''

\v{15}But the very next day, Hazael\fnote{\fbackref{8:15} Lit. \fbib{he}} grabbed a thick covering, soaked it in water, and spread it over the king's\fnote{\fbackref{8:15} Lit. \fbib{over his}} face, and he suffocated.\fnote{\fbackref{8:15} Lit. \fbib{died}} Then Hazael succeeded Ben-hadad\fnote{\fbackref{8:15} Lit. \fbib{succeeded him}} as king.
\passage{Jehoram Comes to the Throne of Judah}

\v{16}Sometime during the fifth year of the reign of Ahab's son Joram, king of Israel (while Jehoshaphat was still ruling as king of Judah), Jehoshaphat's son Jehoram ascended to the throne of Judah. \v{17}He was 32 years old when he became king, and he reigned in Jerusalem for eight years. \v{18}He lived his life like the kings of Israel did, following the example of Ahab's household when he married Ahab's daughter and practiced what was evil in the \divine{Lord}'s presence.\fnote{\fbackref{8:18} Lit. \fbib{sight}} \v{19}But the \divine{Lord} remained unwilling to destroy Judah for the sake of his servant David, since he had promised to keep\fnote{\fbackref{8:19} Lit. \fbib{give}} David's lamp burning brightly through his descendants every day.

\v{20}During Jehoram's lifetime, Edom rebelled from Judah's hegemony and appointed a king to rule over themselves. \v{21}Then Joram crossed over to Zair, along with all of his chariots. At night he attacked the Edomites who had surrounded him and the commanders of his chariots, but the army\fnote{\fbackref{8:21} Lit. \fbib{people}} ran away to their tents. \v{22}Edom remains in rebellion against Judah to this day, and Libnah revolted at the same time. \v{23}The rest of the official\fnote{\fbackref{8:23} The Heb. lacks \fbib{official}} acts of Joram, along with everything else that he did, are recorded in the Book of the Chronicles of the Kings of Judah,\fnote{\fbackref{8:23} An ancient chronicle of Israel, apparently now lost; and so throughout the book} are they not?
\passage{Ahaziah Succeeds Jehoram}

\v{24}After Joram was laid to rest with his ancestors in the City of David, his son Ahaziah replaced him as king. \v{25}Joram's son Ahaziah began to reign as king of Judah during the twelfth year of the reign of Ahab's son Joram, king of Israel. \v{26}Ahaziah was 22 years old when he became king, and he reigned in Jerusalem for one year.

His mother was named Athaliah. She was the granddaughter of Omri, king of Israel. \v{27}Ahaziah lived his life following the example of Ahab's household, practicing what the \divine{Lord} considered to be evil, just like the household of Ahab, because he was a son-in-law to Ahab's household. \v{28}He joined Ahab's son Joram in an attack on King Hazael of Aram at Ramoth-gilead, and that's where the Arameans wounded Joram. \v{29}Then King Joram retreated to Jezreel to recover from the wounds that the Arameans had inflicted on him at Ramah during the battle against King Hazael of Aram. Jehoram's son Ahaziah, king of Judah, went to visit Ahab's son Joram in Jezreel because Joram was sick.\fnote{\fbackref{8:29} I.e. during Joram's recovery from his battle wounds}
\labelchapt{9}
\passage{Jehu Anointed King of Israel}

\chapt{9}
\v{1}Elisha called one of the members of the\fnote{\fbackref{9:1} The Heb. lacks \fbib{members of the}} Guild of Prophets and told him, ``Get ready to run,\fnote{\fbackref{9:1} Lit. \fbib{Tie up your garments}} take this flask of oil in your hand, and go to Ramoth-gilead. \v{2}As soon as you get there, go find Jehoshaphat's son Jehu, the grandson of Nimshi. When you do,\fnote{\fbackref{9:2} The Heb. lacks \fbib{When you do}} go in, tell him to get up and go apart with you away from his brothers. Lead him into a private chamber, \v{3}take the flask of oil, and pour it out on his head. Then tell him, `This is what the \divine{Lord} says: I'm anointing you king over Israel.' Then open the door and leave. Don't linger there!''

\v{4}So the young man, who was an attendant to the prophet, went to Ramoth-gilead. \v{5}When he arrived, the army commanders were seated, so he said, ``I have a message for you, captain!''

Jehu asked, ``For which one of us?''

``For you, captain!'' he answered.

\v{6}So Jehu\fnote{\fbackref{9:6} Lit. \fbib{he}} got up and went inside the house, and the young man\fnote{\fbackref{9:6} Lit. \fbib{and he}} told him, ``This is what the \divine{Lord}, the God of Israel says: `I have anointed you king over the people of the \divine{Lord}---that is, over Israel. \v{7}You are to attack the household of your master Ahab, so I may avenge the blood of my servants the prophets, as well as the blood of all of the servants of the \divine{Lord} that has been spilled\fnote{\fbackref{9:7} The Heb. lacks \fbib{that has been spilled}} at Jezebel's orders.\fnote{\fbackref{9:7} Lit. \fbib{hand}} \v{8}The entire household of Ahab will die, and I will cut off from Ahab every male person in Israel, whether imprisoned or surviving.\fnote{\fbackref{9:8} Or \fbib{whether in servitude or left behind}} \v{9}I will make the household of Ahab like the household of Nebat's son Jeroboam and the household of Ahijah's son Baasha. \v{10}Furthermore, the dogs will eat Jezebel in the territory of Jezreel. There will be no burial for her.'\,'' Then he opened the door and left.

\v{11}As Jehu was coming out to his master's attendants, one of them asked him, ``Is everything all right? Why did this maniac visit you?''

``You know the man and how he speculates,'' Jehu replied.

\v{12}``That's a lie!'' they said. ``Tell us what's going on!''

``He said `This and that' to me,'' he responded. ```This is what the \divine{Lord} says: ``I have anointed you king over Israel.''\,'\,''

\v{13}At this, each man quickly grabbed his own garment, placed it under him at the top of the stairs,\fnote{\fbackref{9:13} Or \fbib{him on the uncovered ascent}; i.e. on the roof of the building} sounded a trumpet, and announced, ``Jehu is king!''
\passage{Joram (Also Known as Jehoram) is Assassinated}
\passageinfo{(2 Chronicles 22:7-9)}

\v{14}Meanwhile, Jehoshaphat's son Jehu, the grandson of Nimshi, had been conspiring against Joram while Joram and all the army of\fnote{\fbackref{9:14} The Heb. lacks \fbib{the army of}} Israel had been defending Ramoth-gilead against King Hazael from Aram. \v{15}King Jehoram had returned to Jezreel to recover from wounds he had sustained from the Arameans when he had fought against King Hazael from Aram. So Jehu concluded, ``Since this is what you've decided,\fnote{\fbackref{9:15} Lit. \fbib{is your soul}} then let no one get away, leave the city, and go report to Jezreel!'' \v{16}Then Jehu rode by chariot to Jezreel, since Joram was recovering\fnote{\fbackref{9:16} Lit. \fbib{lying}} there. King Ahaziah from Judah had come to visit Joram.

\v{17}While the watchman was standing guard in the tower at Jezreel, he watched Jehu's entourage arrive. So he called out, ``I see a group arriving.''

Joram ordered, ``Take a horseman, send him out to meet them, and have him ask, `Have you come in peace?'\,''\fnote{\fbackref{9:17} Lit. \fbib{The peace}; i.e. a general inquiry of welfare}

\v{18}So a horseman went out, greeted Jehu and said, ``This is what the king said: `Have you come in peace?'\,''

But Jehu responded, ``What do you have to do with peace? Fall in behind me.''

The watchman reported, ``The messenger arrived there, but he hasn't returned.''

\v{19}Then Joram sent out a second horseman, who went out to them and said, ``This is what the king said: `Have you come in peace?'\,''

Jehu responded, ``What do you have to do with peace? Fall in behind me.''

\v{20}The watchman reported to Joram, ``He arrived there, but he hasn't returned. Also, he drives like Nimshi's son Jehu drives---irrationally!''

\v{21}Joram replied, ``Let's begin our attack!'' As soon as his chariot was prepared, both King Joram of Israel and King Ahaziah of Judah went out, each in his own chariot, to fight against Jehu. They met together in the property that had belonged to Naboth the Jezreelite.\fnote{\fbackref{9:21} Cf. 1 King 21:1-19}

\v{22}As soon as Joram noticed Jehu, he cried out, ``Peace, Jehu?''

Jehu\fnote{\fbackref{9:22} Lit. \fbib{He}} replied, ``What peace, given\fnote{\fbackref{9:22} The Heb. lacks \fbib{given}} your mother Jezebel's prostitution and all of\fnote{\fbackref{9:22} Lit. \fbib{and many}} her witchcraft?''\fnote{\fbackref{9:22} Or \fbib{sorcery}; i.e. wielding power through demonic spirits}

\v{23}Joram reined his horse\fnote{\fbackref{9:23} Lit. \fbib{hands}} around to flee and cried out to Ahaziah, ``Ahaziah! Treachery!'' \v{24}But Jehu drew his bow with all of his strength, shooting Joram between his shoulder blades.\fnote{\fbackref{9:24} Lit. \fbib{his arms}} The arrow pierced his heart, and he collapsed in his chariot.

\v{25}After this, Jehu called out to Bidkar, his third in command, ``Pick up Joram's body and throw it in the field, the property that belonged to Naboth the Jezreelite, because you and I remember how when we were riding together in pursuit of his father Ahab, that the \divine{Lord} pronounced this oracle\fnote{\fbackref{9:25} Lit. \fbib{burden}; a prophetic message of solemn import} against him:

\begin{poetry}
\poeml \v{26}`This is what the \divine{Lord} says, ``I have certainly observed the blood of Naboth and his sons, and I will repay you on this property,'' declares the \divine{Lord}.' \\
\poeml ``Therefore take the body and throw it in the field, just as the \divine{Lord} said.''
\end{poetry}
\passage{King Ahaziah is Also Killed}
\passageinfo{(2 Chronicles 22:7-9)}

\v{27}As soon as King Ahaziah of Judah observed this, he attempted to flee by the garden house road, but Jehu pursued him. At the ascent toward Gur which is near Ibleam, he ordered, ``Shoot him in the chariot, too!''

Ahaziah fled to Megiddo, where he died. \v{28}Ahaziah's servants transported the king's body\fnote{\fbackref{9:28} Lit. \fbib{transported him}} by chariot to Jerusalem and buried it in his own sepulcher near his ancestors in the City of David. \v{29}Ahaziah had begun to reign over Judah in the eleventh year of the reign of\fnote{\fbackref{9:29} The Heb. lacks \fbib{the reign of}} Ahab's son Joram.
\passage{Jezebel is Executed}

\v{30}As soon as Jehu arrived at Jezreel, Jezebel adorned her eyes, arranged her hair, and peered out a window. \v{31}When Jehu had entered through the gate, she asked, ``Was Zimri, who murdered his master,\fnote{\fbackref{9:31} Cf. 1King 16:9-10} received well?''

\v{32}Jehu\fnote{\fbackref{9:32} Lit. \fbib{He}} looked up toward the window and called out, ``Who is on my side? Who?'' When two or three eunuchs looked out at him, \v{33}he ordered, ``Throw her down!''

So they did, and her blood splashed against the wall and on the horses, while Jehu trampled her underfoot. \v{34}Later on, after he had come in to eat and drink, he ordered, ``Go and see to this cursed woman, and bury her, because she was a king's daughter.'' \v{35}But when they went out to bury her, they found nothing left of her except her skull, her feet, and the palms of her hands. \v{36}So they returned and reported to Jehu,\fnote{\fbackref{9:36} Lit. \fbib{him}} and he responded, ``This fulfills\fnote{\fbackref{9:36} The Heb. lacks \fbib{fulfills}} this message from the \divine{Lord} that he spoke through his servant Elijah the foreigner,\fnote{\fbackref{9:36} Lit. \fbib{Tishbite}; or \fbib{sojourner}} who said:

\begin{poetry}
\poeml `Dogs will eat Jezebel's flesh on the property of Jezreel, \v{37}and her corpse will lie like dung on the surface of the field on the property in Jezreel, but no one will say, ``This is Jezebel.''\,'\,''
\end{poetry}
\labelchapt{10}
\passage{Ahab's Dynasty is Ended}

\chapt{10}
\v{1}Meanwhile, Ahab had 70 sons who lived in Samaria. So Jehu wrote letters and sent them to Samaria---to the rulers of Jezreel, the elders, and the guardians of Ahab's children.\fnote{\fbackref{10:1} The Heb. lacks \fbib{children}} He told them, \v{2}``As soon as you receive this letter (since your master's children are with you, you have chariots and horses there with you, and you are protected by a walled city and weaponry), \v{3}select the best and most qualified of your master's sons, set him in place on his father's throne, and fight for your master's dynasty!''

\v{4}But they were too terrified, and so they told one another,\fnote{\fbackref{10:4} The Heb. lacks \fbib{to one another}} ``Look! Two previous kings couldn't stand up to Jehu, so how can we?'' \v{5}So the household overseer and the city supervisor, along with the elders and the children's guardians, sent word\fnote{\fbackref{10:5} The Heb. lacks \fbib{word}} to Jehu, telling him, ``We will serve you and do everything you ask. We won't set up a king, so do what you want to do.''

\v{6}But Jehu wrote them another letter: ``If you're loyal to me, and if you intend to obey my commands,\fnote{\fbackref{10:6} Lit. \fbib{voice}} then bring the heads of your master's sons and meet me in Jezreel about this time tomorrow.''

Now the king's sons, totaling 70 men, were living with the leading men of the city, who were their guardians. \v{7}When the letter from Jehu\fnote{\fbackref{10:7} The Heb. lacks \fbib{from Jehu}} arrived, the city leaders arrested the king's sons, slaughtered all 70 of them, put their heads in baskets, and sent them to Jehu\fnote{\fbackref{10:7} Lit. \fbib{him}} at Jezreel.

\v{8}When the messenger arrived to report to the king, he said, ``They have brought the heads of the king's sons.''

Jehu\fnote{\fbackref{10:8} Lit. \fbib{He}} replied, ``Put them in two piles at the entrance of the city gate until morning.'' \v{9}The next morning, Jehu went out, stood still, and announced to all the people: ``Are you righteous? I conspired against my master and killed him, but who slaughtered all of these? \v{10}Keep this in mind---not a single statement by the \divine{Lord} will fail to come about that he spoke concerning Ahab's dynasty, because the \divine{Lord} has accomplished what he predicted by his servant Elijah.''

\v{11}So Jehu executed all those who remained from Ahab's dynasty in Jezreel, including all of Ahab's men, his friends, and his priests, until there remained not even one survivor. \v{12}Then Jehu got up, left the city,\fnote{\fbackref{10:12} The Heb. lacks \fbib{the city}} and went to Samaria. When he arrived at the shearing house\fnote{\fbackref{10:12} Or \fbib{Beth-eked}; Lit. \fbib{the house of binding}; i.e. binding sheep to shear them} that was located on the way, \v{13}Jehu met up with the relatives of king Ahaziah of Judah. He asked them, ``Who are you?''

They answered, ``We're Ahaziah's relatives, and we've come down to greet the king's sons and the sons of the queen mother.''

\v{14}Jehu ordered, ``Take them alive!'' So Jehu's soldiers captured them and executed all 42 of them near the pit at the shearing house.\fnote{\fbackref{10:14} Or \fbib{Beth-eked}; Lit. \fbib{the house of binding}; i.e. binding sheep to shear them} He left none of them alive.

\v{15}After he left there, he encountered Rechab's son Jehonadab. After he greeted him, Jehu\fnote{\fbackref{10:15} Lit. \fbib{he}} asked him, ``Is your heart right, as my heart is with yours?''

``It is,'' Jehonadab answered.

``If it is,'' Jehu replied,\fnote{\fbackref{10:15} The Heb. lacks \fbib{Jehu replied}} ``Put out your hand.'' So Jehonadab stuck out his hand, and Jehu took him up to stand in his chariot. \v{16}He told him, ``Come with me and see my enthusiasm for the \divine{Lord}!'' So Jehu\fnote{\fbackref{10:16} Lit. \fbib{he}} had Jehonadab\fnote{\fbackref{10:16} Lit. \fbib{him}} ride in his chariot.

\v{17}When Jehu\fnote{\fbackref{10:17} Lit. \fbib{he}} arrived in Samaria, he executed everyone who remained of Ahab's household in Samaria, until he had utterly destroyed Ahab in accordance with the message from the \divine{Lord} that he spoke to Elijah.
\passage{Jehu Executes the Prophets of Baal}

\v{18}Then Jehu assembled all the people and announced to them, ``Ahab served Baal a little, but Jehu will serve him a lot! \v{19}Therefore summon all of Baal's prophets to me, including all his worshipers and all his priests. Don't leave even one out, because I've prepared a great sacrifice for Baal. Whoever doesn't show up doesn't live!'' But Jehu did this deceptively, intending to destroy Baal's worshippers. \v{20}Jehu ordered, ``Set aside a solemn assembly for Baal!''

And so they proclaimed it. \v{21}Jehu sent the proclamation\fnote{\fbackref{10:21} The Heb. lacks \fbib{the proclamation}} throughout Israel, and all the Baal worshipers came. There wasn't a single man left who failed to come. When they entered Baal's temple, it was filled from one end to the other. \v{22}Then Jehu\fnote{\fbackref{10:22} Lit. \fbib{he}} ordered the one in charge of the wardrobe, ``Bring out garments for all of the worshipers of Baal.'' So he brought out garments for them.

\v{23}Jehu and Rechab's son Jehonadab entered Baal's temple, and Jehu told the Baal worshipers, ``Look around and be sure that no servant of the \divine{Lord} is here among you, but only worshipers of Baal.'' \v{24}Then they went in to offer sacrifices and burnt offerings. Meanwhile, Jehu had stationed 80 men outside, ordering them, ``If any of these men whom I've brought into your control escape, the one who allows it will forfeit his life.''

\v{25}As soon as he had completed the burnt offering, Jehu ordered the guards and the officers, ``Go in and execute them. Don't let even one man escape.'' So they executed them with swords, and the guards and the officers threw the bodies out and proceeded into the inner room of Baal's temple, \v{26}from which they brought out the sacred pillars and burned them. \v{27}They also cut down the pillar to Baal, tore apart Baal's temple, and turned it into a latrine---and it remains that way today. \v{28}That's how Jehu eradicated Baal from Israel. \v{29}Even so, Jehu never abandoned the sins of Nebat's son Jeroboam, who caused Israel to sin, regarding the golden calves that were at Bethel and Dan.
\passage{Israel Begins to Reduce in Size}

\v{30}Nevertheless, the \divine{Lord} told Jehu, ``Because you have done well in carrying out what I saw as the right thing to do by completing everything I had in mind regarding Ahab's dynasty, your sons will sit on the throne of Israel to the fourth generation.'' \v{31}But Jehu did not remain careful to walk in the instruction\fnote{\fbackref{10:31} Or \fbib{Law}} of the \divine{Lord} God of Israel with all his heart. He never abandoned the sins of Jeroboam that had caused Israel to sin. \v{32}In those days, the \divine{Lord} began to reduce Israel in size: Hazael defeated them throughout the territory of Israel, \v{33}from the Jordan River\fnote{\fbackref{10:33} The Heb. lacks \fbib{River}} eastward, all the territory of Gilead, the descendants of Gad, the descendants of Reuben, and the descendants of Manasseh, from Aroer by the Valley of the Arnon, including Gilead and Bashan.
\passage{Jehoahaz Succeeds Jehu}

\v{34}Now as to the rest of Jehu's activities, including his valiant deeds, they are recorded in the Book of the Chronicles of the Kings of Israel, are they not? \v{35}Then Jehu died, as did\fnote{\fbackref{10:35} Lit. \fbib{Jehu slept with}} his ancestors, and they buried him in Samaria. His son Jehoahaz reigned in his place. \v{36}Jehu reigned over Israel in Samaria for 28 years.
\labelchapt{11}
\passage{Athaliah Reigns as Queen of Judah}
\passageinfo{(2 Chronicles 22:10-23:11)}

\chapt{11}
\v{1}As soon as Ahaziah's mother Athaliah learned that her son had died, she seized the throne\fnote{\fbackref{11:1} Lit. \fbib{she arose}} and executed the entire royal bloodline.\fnote{\fbackref{11:1} Lit. \fbib{seed}} \v{2}But King Joram's daughter Jehosheba, who was Ahaziah's sister, rescued\fnote{\fbackref{11:2} Lit. \fbib{took}} Ahaziah's son Joash from the group of the king's sons who were being executed and hid him and his nurse in her bedroom, concealing him from Athaliah so he was not put to death. \v{3}So Joash remained hidden with her in the \divine{Lord}'s Temple for six years while Athaliah reigned over the land.

\v{4}But during the seventh year of her reign,\fnote{\fbackref{11:4} The Heb. lacks \fbib{of her reign}} Jehoiada went out and called together the rulers of hundreds, the captains, and the guards, and assembled them together inside the \divine{Lord}'s Temple. He made a covenant with them, making them take an oath in the \divine{Lord}'s Temple, and then he revealed the king's son to them. \v{5}He ordered them:

\begin{poetry}
\poeml ``Here's what we'll do: A third of you will enter here on this coming\fnote{\fbackref{11:5} The Heb. lacks \fbib{this coming}} Sabbath dressed\fnote{\fbackref{11:5} The Heb. lacks \fbib{dressed}} as guardians of the watch for the king's palace, \v{6}with a third of you at the Sur gate, and a third at the gate behind the guards. Keep watch over the palace\fnote{\fbackref{11:6} Or \fbib{Temple}; Lit. \fbib{house}} and defend it. \v{7}Two\fnote{\fbackref{11:7} Lit. \fbib{Two hands}} of you who enter here on this coming\fnote{\fbackref{11:7} The Heb. lacks \fbib{this coming}} Sabbath are to stand watch at the \divine{Lord}'s Temple, \v{8}guarding the king and surrounding him with weapons in hand. Whoever comes within range is to be killed. Stay with the king wherever he goes, coming or going.''
\end{poetry}

\v{9}So the captains of hundreds did just as Jehoiada the priest ordered. Each one of them assembled his men who were to enter on the Sabbath, along with those who were to leave on the Sabbath, and approached Jehoiada the priest.

\v{10}The priest issued king David's personal spears and shields that had been stored in the \divine{Lord}'s Temple to the captains of hundreds. \v{11}So the guards stood assembled, every soldier with weapons in hand, surrounding the king from the right side corner of the Temple to the left side corner, including around the altar and the Temple.

\v{12}Then he brought out the king's son, put the royal crown on him, presented him with the Testimony,\fnote{\fbackref{11:12} I.e. the tablets that were stored in the ark; cf. Ex 25:16, 31:18} and installed him as king. They anointed him, applauded, and said, ``May the king live!''

\v{13}When Athaliah heard all of the commotion coming from those who were guarding the people, she approached the people who were in the \divine{Lord}'s Temple. \v{14}She looked around---and there was the king, standing near a column, as was the royal custom! He was accompanied by the commanding officers, along with trumpeters who stood beside the king. All the people of the land sounded trumpets in their excitement.

But Athaliah tore her clothes and bellowed, ``It's a plot! A conspiracy!''
\passage{Athaliah is Executed}
\passageinfo{(2 Chronicles 23:12-15)}

\v{15}Jehoiada the priest commanded the captains in charge of\fnote{\fbackref{11:15} Lit. \fbib{captains of hundred over}} the army, ``Take her out the back way\fnote{\fbackref{11:15} Lit. \fbib{out by the ranks}; i.e. using a utility entrance (cf. vs. 16)} and execute anybody who follows her,'' since the priest had also issued this order: ``Let's not put her to death in the \divine{Lord}'s Temple.'' \v{16}So they arrested Athaliah, took her out through the same entrance used by the horses for entering the king's palace, and executed her.
\passage{A Covenant is Made}
\passageinfo{(2 Chronicles 23:16-21)}

\v{17}Then Jehoiada entered into a covenant with the \divine{Lord}, the king, and the people, that they would live as the \divine{Lord}'s people, and also entered into a covenant with the king and the people. \v{18}Then all of the people of the land entered Baal's temple, tore it down, and broke his altars and his images to pieces, killing Mattan the priest of Baal right in front of the altars. Furthermore, Jehoiada\fnote{\fbackref{11:18} The Heb. lacks \fbib{Jehoiada}} the priest appointed officers to guard the \divine{Lord}'s Temple, \v{19}and brought the commanders of hundreds, the Carites, the guards, and all the people of the land, taking the king out of the \divine{Lord}'s Temple, marching through the guard's gate to the king's palace, where Joash\fnote{\fbackref{11:19} Lit. \fbib{he}} took his seat on the throne of the kings. \v{20}After this, everyone throughout the land rejoiced and the city was at peace, because they had executed Athaliah at the king's palace.
\labelchapt{12}
\passage{Jehoash (Joash) Reigns over Judah}

\v{21}\fnote{\fbackref{11:21} This vs. is 12:1 in MT} Jehoash began to reign as king when he was seven years old,\chapt{12}
\v{1}\fnote{\fbackref{12:1} This vs. is 12:2 in MT, and so throughout the chapter} ascending to the throne in the seventh year of the reign of\fnote{\fbackref{12:1} The Heb. lacks \fbib{the reign of}} Jehu and then reigning for 40 years in Jerusalem. His mother's name was Zibiah from Beer-sheba. \v{2}Jehoash did what the \divine{Lord} considered to be right during the entire time when Jehoiada the priest was instructing him, \v{3}except that the high places were not demolished, so the people continued to sacrifice and burn incense on the high places.
\passage{Jehoash Institutes Temple Repairs}

\v{4}Jehoash spoke to the priests about all of the proceeds\fnote{\fbackref{12:4} Lit. \fbib{silver}; i.e., money from conversion of gifts into cash} of the consecrated gifts that were being brought into the \divine{Lord}'s Temple, cash from every man who was traveling through the area,\fnote{\fbackref{12:4} The Heb. lacks \fbib{the area}} cash obtained by personal assessment,\fnote{\fbackref{12:4} Lit. \fbib{cash from souls to their appointment}} and all the cash that came through voluntary gifts\fnote{\fbackref{12:4} Lit. \fbib{through the heart of a man}} into the \divine{Lord}'s Temple:

\begin{poetry}
\poeml \v{5}``Let the priests get support for themselves from their own donors, and let them repair the Temple wherever a leak in need of repair is discovered.''
\end{poetry}

\v{6}But 23 years into the reign of king Jehoash, the priests still had not repaired the leaks in the Temple. \v{7}So king Jehoash called for Jehoiada the priest, along with other\fnote{\fbackref{12:7} Lit. \fbib{the}} priests, and asked them, ``Why haven't you fixed the leaks in the Temple? Stop receiving donations from your acquaintances for repairing the leaks in the Temple.''

\v{8}So the priests agreed to receive no more cash from the people, but they didn't repair the leaks in the Temple, either. \v{9}So Jehoiada the priest grabbed a chest, bored an opening in its lid, and placed it next to the altar, on the right side as one enters the \divine{Lord}'s Temple. The priests who tended the entryway put all the money that was brought into the \divine{Lord}'s Temple into the chest.\fnote{\fbackref{12:9} Lit. \fbib{into it}} \v{10}As a result, whenever they noticed that there was a lot of money in the chest, the king's secretary and the high priest went forward, put the money in bags, counted the money that had been given over to the \divine{Lord}'s Temple, \v{11}and disbursed the cash directly into the hands of those who did the work and who were in charge of the oversight of the \divine{Lord}'s Temple. They paid it to the carpenters and builders who worked on the \divine{Lord}'s Temple, \v{12}to masons and stonecutters, and for procurement of timber and quarried stone for making repairs to the \divine{Lord}'s Temple, and for all outlays needed for repairs of the Temple.

\v{13}But no provision was included for the \divine{Lord}'s Temple from the money that was brought into the \divine{Lord}'s Temple for silver basins, snuffers, bowls, trumpets, or any vessels made of gold or silver, \v{14}because that money had been allocated to the workmen who were repairing the \divine{Lord}'s Temple. \v{15}Furthermore, they required no accounting from the men into whose hand they had paid the money to do the work, because the workers acted in good faith. \v{16}The money from the guilt offerings and\fnote{\fbackref{12:16} Lit. \fbib{and the money}} from the sin offerings was not brought into the \divine{Lord}'s Temple, because it was allocated to the priests.
\passage{Hazael Attacks Israel}

\v{17}Later, King Hazael of Aram invaded and attacked Gath, captured it, and then set out to approach Jerusalem. \v{18}So King Jehoash of Judah took all of the sacred things that his ancestors Jehoshaphat, Jehoram, and Ahaziah, kings of Judah, had dedicated, along with his own dedicated things, and all the gold that could be located within the treasure vaults of the \divine{Lord}'s Temple and in the king's palace, and paid off King Hazael of Aram. Then Hazael\fnote{\fbackref{12:18} Lit. \fbib{he}} left Jerusalem.
\passage{Amaziah Succeeds Jehoash (Joash)}
\passageinfo{(2 Chronicles 24:23-27)}

\v{19}Now the rest of the Joash's activities---everything he did---are written in the Book of the Chronicles of the Kings of Judah, are they not? \v{20}His servants rose up in rebellion, formed a conspiracy, and assassinated Joash in the palace at the terrace ramparts\fnote{\fbackref{12:20} Lit. \fbib{the Millo}, fortified areas of ancient Jerusalem with terraces and retaining walls} while he was on his way down to Silla. \v{21}Shimeath's son Jozacar and Shomer's son Jehozabad, his servants, attacked him and he died. They buried him alongside his ancestors in the City of David, and his son Amaziah became king to replace him.
\labelchapt{13}
\passage{Jehoahaz Becomes King of Israel}

\chapt{13}
\v{1}During the twenty-third year of the reign of\fnote{\fbackref{13:1} The Heb. lacks \fbib{the reign of}} Ahaziah's son Joash, king of Judah, Jehu's son Jehoahaz began his seventeen year reign in Samaria over Israel.\fnote{\fbackref{13:1} I.e. over the northern kingdom} \v{2}He did what the \divine{Lord} considered to be evil, after the pattern of Nebat's son Jeroboam. By doing so, he caused Israel to sin, and he never changed course from it. \v{3}As a result, the \divine{Lord}'s wrath flared up against Israel, so he handed them over to domination by king Hazael of Aram and later into constant domination\fnote{\fbackref{13:3} Lit. \fbib{into domination all their days}} by Hazael's son Ben-hadad. \v{4}But Jehoahaz sought the \divine{Lord},\fnote{\fbackref{13:4} Lit. \fbib{the \divine{Lord}'s face}} and the \divine{Lord} paid attention to him, because the \divine{Lord}\fnote{\fbackref{13:4} Lit. \fbib{because he}} had been watching the oppression that Israel was enduring from the king of Aram.\fnote{\fbackref{13:4} The Heb. lacks \fbib{that Israel was enduring from the king of Aram.}}
\passage{God Delivers Israel}

\v{5}The \divine{Lord} provided Israel with a deliverer, so they escaped the Aramean oppression while the descendants of Israel lived in tents as they had formerly. \v{6}Nevertheless, they did not change course away from the sins of Jeroboam's household, by which he caused Israel to sin, but continued on that same course, with Asherah poles\fnote{\fbackref{13:6} I.e. cultic pillars used in pagan worship, and so throughout the book} remaining in place in Samaria. \v{7}For the Aramean king\fnote{\fbackref{13:7} Lit. \fbib{For he}} had left only 50 cavalry, ten chariots, and 10,000 soldiers out of the army belonging to Jehoahaz, because the king of Aram had destroyed the others,\fnote{\fbackref{13:7} Lit. \fbib{destroyed them}} making them like chaff left over after threshing.

\v{8}Now the rest of the activities of Jehoahaz, including everything he did and his grandeur, are recorded in the Book of the Chronicles of the Kings of Israel, are they not? \v{9}So Jehoahaz died, as did\fnote{\fbackref{13:9} Lit. \fbib{Jehoahaz slept with}} his ancestors, and he was buried in Samaria while his son Joash replaced him as king.
\passage{Jehoash Reigns in Samaria}

\v{10}During the thirty-seventh year of the reign of\fnote{\fbackref{13:10} The Heb. lacks \fbib{the reign of}} king Joash\fnote{\fbackref{3:10} \fbib{Joash} and \fbib{Jehoash} are alternate spellings of the same name} of Judah, Jehoahaz's son Jehoash began a sixteen year reign as king over Israel in Samaria. \v{11}He practiced what the \divine{Lord} considered to be evil, not changing course from all of the sins practiced by Nebat's son Jeroboam by which he caused Israel to sin. Instead, he continued on that same course. \v{12}The rest of Joash's activities, including everything he did and the vehemence with which he fought against king Amaziah of Judah are recorded in the Book of the Chronicles of the Kings of Israel, are they not? \v{13}So Joash died, as did\fnote{\fbackref{13:13} Lit. \fbib{Joash slept with}} his ancestors, and Jeroboam assumed his throne after Joash was buried in Samaria with the kings of Israel.
\passage{Elisha Predicts Partial Victory for Joash}

\v{14}When Elisha fell ill with the sickness from which he was about to die, king Joash of Israel came down to see\fnote{\fbackref{13:14} The Heb. lacks \fbib{see}} him, wept in his presence, and told him, ``My father, Israel's chariots and horsemen!''

\v{15}Elisha told him, ``Pick up a bow and some arrows.'' So he picked up a bow and some arrows.

\v{16}Then Elisha told Israel's king, ``Draw the bow!'' As he did so, Elisha laid his hands on top of the king's hands \v{17}and ordered him, ``Open a window that faces east.'' So he did so.

Elisha ordered him, ``Shoot!'' So he shot.

Then Elisha said, ``This is the \divine{Lord}'s arrow of victory---the victory arrow against Aram, because you will defeat the Arameans at Aphek until you will have utterly finished them off.''

\v{18}After this Elisha said, ``Pick up the arrows.'' So the king picked them up.

Then Elisha told the king of Israel, ``Strike the ground!'' So he struck it three times and then stood still.

\v{19}At this, the man of God became angry at him and told him, ``You should have struck five or six times! Then you would have attacked Aram until you would have destroyed it! But as it is now, you'll defeat Aram only three times!''
\passage{The Death of Elisha}

\v{20}Later, Elisha died and was buried. Now at that time, various Moabite marauders had been invading the land each spring. \v{21}One day while some Israelis\fnote{\fbackref{13:21} Lit. \fbib{As they}} were burying a man, they saw some marauders, so they threw the man into Elisha's grave. But when the man fell against Elisha's remains,\fnote{\fbackref{13:21} Lit. \fbib{bones}} he revived and rose to his feet.
\passage{Elisha's Prophecy of Partial Victory is Fulfilled}

\v{22}Meanwhile, king Hazael of Aram had been oppressing Israel throughout the reign of Jehoahaz, \v{23}but the \divine{Lord} showed grace to them, displayed his compassion toward them, and turned to them due to his covenant with Abraham, Isaac, and Jacob. He would not destroy them or evict them from his presence up until that time. \v{24}After king Hazael of Aram died, his son Ben-hadad replaced him as king. \v{25}At that time, Jehoahaz's son Jehoash recaptured from Hazael's son Ben-hadad the cities that Hazael\fnote{\fbackref{13:25} Lit. \fbib{he}} had captured through warfare from the control of Jehoahaz, Jehoash's father. Joash\fnote{\fbackref{13:25} \fbib{Joash} and \fbib{Jehoash} are alternate spellings of the same name} defeated and recovered cities of Israel from Ben-hadad\fnote{\fbackref{13:25} Lit. \fbib{him}} three times.
\labelchapt{14}
\passage{Amaziah Becomes King of Judah}

\chapt{14}
\v{1}Amaziah, son of Judah's king Joash, became king during the second year of the reign of\fnote{\fbackref{14:1} The Heb. lacks \fbib{the reign of}} Joash, son of king Joahaz of Israel, \v{2}at the age of 25. He reigned 29 years in Jerusalem. His mother's name was Jehoaddin; she was\fnote{\fbackref{14:2} The Heb. lacks \fbib{she was}} from Jerusalem.

\v{3}He practiced what the \divine{Lord} considered to be right, but not like his ancestor David did. He acted as his father Joash had done, \v{4}except that the high places were not abolished. The people continued to offer sacrifices and to burn incense on the high places. \v{5}Later on, as soon as he was in firm control of his kingdom, he executed the servants who had murdered his father the king, \v{6}but he did not execute the children of the murderers, in keeping with what is written in the Book of the Law of Moses, as the \divine{Lord} had commanded: ``Fathers must not be put to death because of their children's sin; nor are children to die because of their fathers' sin, for each person is to be put to death for his own sin.''\fnote{\fbackref{14:6} Cf. Deut 24:16}
\passage{The Edomites are Defeated}
\passageinfo{(2 Chronicles 25:5-16)}

\v{7}Joash executed 10,000 Edomites in the Salt Valley and captured Sela in battle, renaming it Joktheel, which remains its name to this day. \v{8}Later, Amaziah sent couriers to Jehoahaz's son Jehoash, grandson of king Jehu of Israel, challenging him, ``Come on! Let's fight face to face!''

\v{9}But king Jehoash of Israel sent this message to king Amaziah of Judah: ``The thorn bush in Lebanon sent this message to the cedar\fnote{\fbackref{14:9} I.e. a genus of coniferous evergreen in the family \fbib{Pinaceae}; and so throughout the book} of Lebanon: `Give your daughter to my son in marriage.' But just then a wild beast from Lebanon wandered by and trampled down the thorn bush. \v{10}You just defeated Edom and you're\fnote{\fbackref{14:10} Lit. \fbib{and your heart is}} arrogant. Bask in your victory and stay home. Why incite trouble so that you---yes, you!---fall, along with Judah with you?''

\v{11}But Amaziah refused to listen. So Israel's king Jehoash and Judah's king Amaziah faced each other at Beth-shemesh, which is part of Judah. \v{12}Judah was defeated by Israel, and everybody fled to their own tents. \v{13}Then king Jehoash of Israel captured Judah's king Amaziah, the son of Jehoash and grandson of Ahaziah, at Beth-shemesh. He went to Jerusalem and demolished 400 cubits\fnote{\fbackref{14:13} I.e. about 600 feet a cubit was about eighteen inches} of the wall of Jerusalem from the Ephraim Gate to the Corner Gate. \v{14}He confiscated all the gold and silver, all the instruments he could find in the \divine{Lord}'s Temple and in the palace treasuries. He also captured some hostages and then returned to Samaria.
\passage{Jeroboam Succeeds Israel's King Jehoash}

\v{15}The rest of Jehoash's activities that he undertook, including his valor in fighting king Amaziah of Judah, are recorded in the Book of the Chronicles of the Kings of Israel, are they not? \v{16}Jehoash died, as had\fnote{\fbackref{14:16} Lit. \fbib{Jehoash slept with}} his ancestors, and he was buried in Samaria alongside the kings of Israel. His son Jeroboam reigned in his place.
\passage{The Death of Judah's King Amaziah}
\passageinfo{(2 Chronicles 25:25-28)}

\v{17}Joash's son, king Amaziah of Judah, lived for fifteen years after Jehoahaz' son, king Jehoash of Israel, died. \v{18}The rest of Amaziah's activities are recorded in the Book of the Chronicles of the Kings of Judah, are they not? \v{19}A conspiracy arose against him in Jerusalem, and he ran off to Lachish, but he was pursued to Lachish and killed there. \v{20}His body was brought back on horses and he was buried at Jerusalem alongside his ancestors in the City of David.
\passage{Azariah's Reign over Judah}

\v{21}All the people of Judah took Azariah, who was sixteen years old, and installed him as king to take the place of his father Amaziah. \v{22}He rebuilt Elath and restored it to Judah. Later on the king died, as did\fnote{\fbackref{14:22} Lit. \fbib{king slept with}} his ancestors.
\passage{Jeroboam's Reign over Israel}

\v{23}In the fifteenth year of the reign of\fnote{\fbackref{14:23} The Heb. lacks \fbib{the reign of}} Amaziah son of Joash, king of Judah, Jeroboam son of Joash, king of Israel, began a 41 year reign in Samaria. \v{24}He did what the \divine{Lord} considered to be evil by not abandoning all the sins of Nebat's son Jeroboam, who made Israel sin. \v{25}He rebuilt Israel's coastline from the entrance of Hamath as far as the Sea of the Arabah,\fnote{\fbackref{14:25} I.e. the Dead Sea; cf. Deut 3:17} in accordance with the message from the \divine{Lord} God of Israel that he spoke through his servant Jonah the prophet, Amittai's son, who was from Gath-hepher. \v{26}For the \divine{Lord} observed Israel's bitter misery, and there was no one left, neither slave nor free, and there was no deliverer for Israel. \v{27}The \divine{Lord} had never said that he would erase the name of Israel from under heaven. Instead, he delivered them by Joash's son Jeroboam. \v{28}The rest of Jeroboam's actions---everything he did, including his powerful fighting and how on behalf of Israel he restored Damascus and Hamath to Judah---are recorded in the Book of the Chronicles of the Kings of Israel, are they not?
\passage{Zechariah's Reign over Israel}

\v{29}Jeroboam died, as had\fnote{\fbackref{14:29} Lit. \fbib{Jeroboam slept with}} his ancestors the kings of Israel, and his son Zechariah became king in his place.
\labelchapt{15}
\passage{Azariah Becomes King of Judah}

\chapt{15}
\v{1}Amaziah's son Azariah began reigning during the twenty-seventh year of the reign of\fnote{\fbackref{15:1} The Heb. lacks \fbib{the reign of}} Jeroboam, king of Israel. \v{2}He was sixteen years old when he began to reign, and he reigned 52 years in Jerusalem. His mother's name was Jecoliah; she was\fnote{\fbackref{15:2} The Heb. lacks \fbib{she was}} from Jerusalem. \v{3}He did what the \divine{Lord} considered to be right, just as his father Amaziah had done in everything, \v{4}except that the high places were never removed, and the people kept on sacrificing and burning incense on the high places.

\v{5}The \divine{Lord} struck the king so that he was afflicted with leprosy until the day he died. He lived in a separate house while his son Jotham managed the household and ruled\fnote{\fbackref{15:5} Lit. \fbib{judged}} the people who lived in the land. \v{6}Now the rest of Azariah's activities, including everything he did, are recorded in the Book of the Chronicles of the Kings of Judah, are they not? \v{7}Later, Azariah died, as had\fnote{\fbackref{15:7} Lit. \fbib{Azariah slept with}} his ancestors, and they buried him with his ancestors in the City of David. His son Jotham then reigned in his place.
\passage{Zachariah's Reign over Israel}

\v{8}During the thirty-eighth year of the reign of\fnote{\fbackref{15:8} The Heb. lacks \fbib{the reign of}} Azariah, king of Judah, Jeroboam's son Zachariah began a six-month reign in Samaria. \v{9}He did what the \divine{Lord} considered to be evil, just as his ancestors had done. He never abandoned the sins of Nebat's son Jeroboam, who caused Israel to sin. \v{10}So Jabesh's son Shallum conspired against him and attacked him in full view of the people, killed him, and reigned in his place. \v{11}The rest of Zachariah's activities are recorded in the Book of the Chronicles of the Kings of Israel.
\passage{Shallum's Reign over Israel}

\v{12}This is what the \divine{Lord} told Jehu: ``Your children will sit on Israel's throne for the next four generations.''\fnote{\fbackref{15:12} The Heb. lacks \fbib{generations}} And that is what happened:\fnote{\fbackref{15:12} Lit. \fbib{And so it was}} \v{13}Jabesh's son Shallum began his reign in the thirty-ninth year of the reign of Uzziah,\fnote{\fbackref{15:13} \fbib{Uzziah} is an alternate spelling of \fbib{Azariah} (cf. 2King 15:1)} king of Judah. He reigned a full month\fnote{\fbackref{15:13} Lit. \fbib{a lunar of days}; i.e. one complete lunar month (through all four phases of the moon)} in Samaria, \v{14}then Gadi's son Menahem approached Samaria from Tirzah and attacked Jabesh's son Shallum, executed him, and reigned in his place. \v{15}The rest of Shallum's activities, including the conspiracy that he carried out, are recorded in the Book of the Chronicles of the Kings of Israel, are they not?
\passage{Menahem's Reign over Israel}

\v{16}At another time, Menahem attacked Tiphsah and all of its inhabitants, including its coastlands from Tirzah, because they would not open the city gate for him. After defeating them, he ripped open all of their pregnant women. \v{17}In the thirty-ninth year of the reign of\fnote{\fbackref{15:17} The Heb. lacks \fbib{the reign of}} Azariah, king of Judah, Gadi's son Menahem began a ten-year reign over Israel from Samaria. \v{18}He did what the \divine{Lord} considered to be evil by never abandoning the sins of Nebat's son Jeroboam, who caused Israel to sin, as long as he lived.

\v{19}Later on, King Pul of Aram attacked the land, and Menahem paid Pul 1,000 silver talents\fnote{\fbackref{15:19} I.e. about 75,000 pounds; a talent weighed about 75 pounds} so Pul\fnote{\fbackref{15:19} Lit. \fbib{he}} would join forces with Menahem\fnote{\fbackref{15:19} Lit. \fbib{him}} to secure his hold on the kingdom. \v{20}Menahem exacted the money from all of Israel's powerful and wealthy men, 50 shekels\fnote{\fbackref{15:20} I.e. about 20 ounces; a shekel weighed about 0.4 ounce} from each, to pay the king of Aram. As a result, the king of Aram retreated and did not remain there in the land. \v{21}The rest of Menahem's activities, including everything that he did, are recorded in the Book of the Chronicles of the Kings of Israel, are they not? \v{22}Then Menahem died, as did\fnote{\fbackref{15:22} Lit. \fbib{Menahem slept with}} his ancestors, and his son Pekahiah reigned in his place.
\passage{Pekahiah's Reign over Israel}

\v{23}Menahem's son Pekahiah became king over Israel for two years during the fiftieth year of the reign of\fnote{\fbackref{15:23} The Heb. lacks \fbib{the reign of}} King Azariah of Judah. \v{24}He did what the \divine{Lord} considered to be evil. Just as Nebat's son Jeroboam had led Israel into sin, so also Pekahiah did not stop doing the same thing. \v{25}Then Remaliah's son Pekah, Pekahiah's\fnote{\fbackref{15:25} Lit. \fbib{his}} officer, conspired against him with Argob and Arieh. Accompanied by 50 Gileadite men, Pekah attacked Pekahiah inside the palace of the king's compound\fnote{\fbackref{15:25} Lit. \fbib{house}} in Samaria, executed him, and reigned as king in his place. \v{26}The rest of Pekahiah's activities, including everything he did, are written in the Book of the Chronicles of the Kings of Israel.
\passage{Pekah's Reign over Israel}

\v{27}Remaliah's son Pekah began a 20-year reign as Israel's king during the fifty-second year of King Azariah of Judah. \v{28}He did what the \divine{Lord} considered to be evil by never abandoning the sins of Nebat's son Jeroboam, by which he caused Israel to sin. \v{29}During the lifetime of King Pekah of Israel, King Tiglath-pileser of Assyria attacked. He captured the cities of Ijon, Abel Beth Maacah, Janoah, Kedesh, and Hazor. He also captured Gilead, Galilee, and the entire territory of Naphtali, and carried its people off to Assyria. \v{30}So during the twentieth year of the reign of\fnote{\fbackref{15:30} The Heb. lacks \fbib{the reign of}} Uzziah's son Jotham, Elah's son Hoshea conspired against Remaliah's son Pekah, attacked him, executed him, and became king in his place. \v{31}The rest of Pekah's activities, including everything that he accomplished, are written in the Book of the Chronicles of the Kings of Israel.
\passage{Jotham's Reign over Judah}

\v{32}Uzziah's son Jotham became king over Judah during the second year of the reign of\fnote{\fbackref{15:32} The Heb. lacks \fbib{the reign of}} Remaliah's son Pekah, king of Israel. \v{33}He was 25 years old when he became king. He reigned sixteen years in Jerusalem. Zadok's daughter Jerusha was his mother. \v{34}He did what the \divine{Lord} considered to be right, following everything his father Uzziah had done, \v{35}except the high places were not torn down, and the people still sacrificed and burned incense on the high places. But he rebuilt the upper gate of the \divine{Lord}'s Temple. \v{36}The rest of Jotham's activities, including everything that he accomplished, are recorded in the Book of the Chronicles of the Kings of Judah, are they not?

\v{37}Right about that time, the \divine{Lord} began to send King Rezin of Aram and Remaliah's son Pekah against Judah. \v{38}Meanwhile, Jotham died, as did\fnote{\fbackref{15:38} Lit. \fbib{Jotham slept with}} his ancestors, and was buried with them\fnote{\fbackref{15:38} Lit. \fbib{with his ancestors}} in the City of David, his ancestor. Then Jotham's son Ahaz reigned in his place.
\labelchapt{16}
\passage{Ahaz Becomes King of Judah}

\chapt{16}
\v{1}During the seventeenth year of the reign of\fnote{\fbackref{16:1} The Heb. lacks \fbib{the reign of}} Remaliah's son Pekah, Jotham's son Ahaz became king of Judah. \v{2}Ahaz was 20 years old when he became king, and he ruled in Jerusalem for sixteen years. He did not practice what the \divine{Lord} considered to be right, as had his ancestor David. \v{3}Instead, he behaved like the kings of Israel did by making his son pass through fire, the very same abomination that the heathen practiced, whom the \divine{Lord} evicted from the land right in front of the Israelis. \v{4}Furthermore, Ahaz\fnote{\fbackref{16:4} Lit. \fbib{he}} sacrificed and burned incense on the high places, on top of hills, and under every green tree.
\passage{Ahaz Seeks Help from Assyria}
\passageinfo{(2 Chronicles 28:16-21; Isaiah 7:1-16)}

\v{5}Later, King Rezin of Aram and Remaliah's son Pekah, king of Israel, approached Jerusalem to attack it. They besieged Ahaz but could not conquer him. \v{6}But at that time, King Rezin of Aram recovered Elath for Aram, completely removing the Judeans from Elath. Then the Arameans returned to Elath and have remained there to this day. \v{7}So Ahaz sent envoys to Tiglath-pileser, king of Assyria, to tell him, ``I am your servant and son. Save me from the king of Aram and the king of Israel, who are attacking me.'' \v{8}Then Ahaz took the silver and gold that was in the \divine{Lord}'s Temple and in the palace treasuries and sent them as a gift to the king of Assyria, \v{9}so the king of Assyria listened to Ahaz. He attacked Damascus, captured it, sent its people away into exile to Kir, and executed Rezin.
\passage{King Ahaz Constructs a Pagan Altar}
\passageinfo{(2 Chronicles 28:22-25)}

\v{10}King Ahaz traveled to Damascus and met with King Tiglath-pileser of Assyria, where he observed the altar at Damascus. So King Ahaz sent a set of construction patterns of this altar to Uriah the priest. \v{11}Uriah the priest built an altar, following the plans that King Ahaz had sent him from Damascus and finishing the altar before King Ahaz returned from Damascus. \v{12}When the king returned from Damascus, as soon as he saw the altar, he\fnote{\fbackref{16:12} Lit. \fbib{altar, the king}} approached it and offered sacrifices on it. \v{13}He presented a burnt offering, a meat offering, poured out a drink offering, and sprinkled the blood of a peace offering on his altar. \v{14}Then he took the bronze altar that stood in the \divine{Lord}'s presence from in front of the Temple, moved it to the north side of his altar, \v{15}and issued these orders to Uriah the priest:

\begin{poetry}
\poeml ``Burn the morning burnt offering, the evening grain offering, the king's burnt offering and grain offering, the whole burnt offering, the grain offering, and the drink offering on behalf of all the people of the land on the large altar. And sprinkle all the blood from the burnt offering and from the sacrifice. But I will use the bronze altar to ask God questions.''
\end{poetry}

\v{16}So Uriah the priest did precisely what King Ahaz ordered. \v{17}Later, King Ahaz ordered the side panels removed from the bases, along with the washing bowls that had stood on top of the bases. He also removed the large bowl that was called the Sea from on top of the bronze bulls that supported it, and put it on a stone base. \v{18}Then Ahaz removed the covered walkway for use on the Sabbath that they had built in the Temple. Because of the king of Assyria, he also removed the outside entrance from the \divine{Lord}'s Temple that had been built exclusively\fnote{\fbackref{16:18} The Heb. lacks \fbib{that had been built exclusively}} for the king.

\v{19}Now the rest of Ahaz's activities are recorded in the Book of the Chronicles of the Kings of Judah, are they not? \v{20}Later, Ahaz died, as did\fnote{\fbackref{16:20} Lit. \fbib{Ahaz slept with}} his ancestors, and was buried alongside his ancestors in the City of David. His son Hezekiah reigned in his place.
\labelchapt{17}
\passage{Israel Falls to Assyria during Hoshea's Reign}

\chapt{17}
\v{1}During the twelfth year of the reign of\fnote{\fbackref{17:1} The Heb. lacks \fbib{the reign of}} King Ahaz of Judah, Elah's son Hoshea became king over Israel for nine years in Samaria. \v{2}He practiced what the \divine{Lord} considered to be evil,\fnote{\fbackref{17:2} Lit. \fbib{sight}} though not like the kings of Israel who had preceded him. \v{3}King Shalmaneser of Assyria attacked him, and Hoshea became his servant and paid tribute to him. \v{4}But the king of Assyria uncovered a conspiracy involving Hoshea, who had sent envoys to King So of Egypt and stopped offering tribute to the king of Assyria, as he had done annually. As a result, the king of Assyria placed him under arrest and sent him to prison. \v{5}After this, the king of Assyria invaded the entire land, approached Samaria, and began a three year siege. \v{6}As a result, during the ninth year of the reign of\fnote{\fbackref{17:6} The Heb. lacks \fbib{the reign of}} Hoshea, the king of Assyria captured Samaria and took the Israelis off to Assyria, placing them in Halah, along the Habor River in Gozan, and in cities ruled by the Medes.
\passage{The Idolatry of the Northern Kingdom}

\v{7}This happened because the Israelis had sinned against the \divine{Lord} their God, who had brought them up from the land of Egypt and from the domination\fnote{\fbackref{17:7} Lit. \fbib{hand}} of Pharaoh, king of Egypt, because\fnote{\fbackref{17:7} The Heb. lacks \fbib{because}} they were fearing other gods, \v{8}and because they were following\fnote{\fbackref{17:8} Lit. \fbib{were walking in}} the rules of the nations whom the \divine{Lord} had expelled before the Israelis and that the kings of Israel had practiced.

\v{9}The Israelis practiced secret things that were not right, offending the \divine{Lord} their God. In addition, they built high places for use by all their towns, watchtowers, and fortified cities. \v{10}They set up pillars and Asherim on every high hill and in the shade of every green tree, \v{11}where they made offerings on all the high places, as did the nations whom the \divine{Lord} had expelled before them. They also practiced other\fnote{\fbackref{17:11} The Heb. lacks \fbib{other}} wickedness, provoking the \divine{Lord} to become angry, \v{12}and they served idols, a practice that the \divine{Lord} had warned them, ``You are not to do this.''

\v{13}Nevertheless, the \divine{Lord} had warned both Israel and Judah by means\fnote{\fbackref{17:13} Lit. \fbib{by the hand}} of every prophet and seer: ``Turn away from your evil practices\fnote{\fbackref{17:13} Lit. \fbib{ways}} and keep my commandments and statutes according to the entire Law that I gave your ancestors and that I sent to you through my servants, the prophets.'' \v{14}But they would not listen. Instead, they were stubborn,\fnote{\fbackref{17:14} Lit. \fbib{they hardened their necks}} just like their ancestors had been, who did not believe in the \divine{Lord} their God. \v{15}They rejected the \divine{Lord}'s\fnote{\fbackref{17:15} Lit. \fbib{rejected his}} statutes, the covenant that he had made with their ancestors, and his warnings that he gave them. They pursued meaninglessness---and became meaningless themselves---as they followed the lifestyles of the nations that surrounded them, a practice that the \divine{Lord} had warned them not to do.

\v{16}They abandoned all of the commands given by\fnote{\fbackref{17:16} The Heb. lacks \fbib{given by}} the \divine{Lord} their God, crafted for themselves cast images of two calves, constructed an Asherah, worshipped all of the stars in heaven, and served Baal. \v{17}They passed their sons and daughters through fire, practiced divination, cast spells, and sold themselves to practice what the \divine{Lord} considered to be evil, thereby\fnote{\fbackref{17:17} The Heb. lacks \fbib{thereby}} provoking him. \v{18}As a result, the \divine{Lord} was angry with Israel and removed them from his presence. No one was left except for the tribe of Judah.

\v{19}But Judah, too, did not keep the commands of the \divine{Lord} their God. Instead, they lived the lifestyle\fnote{\fbackref{17:19} Lit. \fbib{customs}} that Israel had chosen, \v{20}so the \divine{Lord} rejected all of the descendants\fnote{\fbackref{17:20} Lit. \fbib{seed}} of Israel, afflicted them, and handed them over to the control of plunderers until he had thrown them away from his presence.\fnote{\fbackref{17:20} Lit. \fbib{face}} \v{21}He ripped them away from the heritage of David, even as the people\fnote{\fbackref{17:21} Lit. \fbib{David, and they}} appointed Nebat's son Jeroboam to be king. Jeroboam drove Israel away from following the \divine{Lord} and made them commit great sin.

\v{22}The Israelis practiced\fnote{\fbackref{17:22} Lit. \fbib{Israelis walked in}} all the sins that Jeroboam had practiced, and never wavered from them \v{23}until the \divine{Lord} removed Israel from his presence,\fnote{\fbackref{17:23} Lit. \fbib{sight}} just as he had warned through\fnote{\fbackref{17:23} Lit. \fbib{spoken by the hand of}} all of his prophets who served him. So Israel was carried off into exile from their own land into Assyria, where they remain to this day.\fnote{\fbackref{17:23} I.e. c. during the late Babylonian captivity or slightly after that time}
\passage{Assyria Supplants the Northern Kingdom}

\v{24}Because the king of Assyria brought captives\fnote{\fbackref{17:24} The Heb. lacks \fbib{captives}} from Babylon, Cuthah, Avva, Hamath, and Sephar-vaim and settled them in the cities of Samaria to replace the Israelis, the settlers\fnote{\fbackref{17:24} The Heb. lacks \fbib{the settlers}} possessed Samaria and lived in its cities. \v{25}When they first began to live there, the settlers\fnote{\fbackref{17:25} The Heb. lacks \fbib{the settlers}} did not fear the \divine{Lord}, so he sent lions among them, and they killed a few of them. \v{26}As a result, they reported to the king of Assyria, ``Because the nations whom you exiled to live in the cities of Samaria don't know the law\fnote{\fbackref{17:26} Or \fbib{justice}} of the god of the land, he has sent lions among them. Look how the lions\fnote{\fbackref{17:26} Lit. \fbib{how they}} are killing them, because they don't know the law of the god of the land!''

\v{27}So the king of Assyria issued this order: ``Take one of the priests whom you carried away and let him go back and live there. Let him teach them the law of the god of the land.'' \v{28}So one of the priests whom they had carried away from Samaria went to live in Bethel to teach them how they ought to fear the \divine{Lord}.
\passage{Assyrian Settlers Create Lasting Corruption}

\v{29}Nevertheless, each nation continued to craft their own gods and install them in the temples on the high places that the people of Samaria had constructed---every nation in their own cities where they continued to live. \v{30}Settlers\fnote{\fbackref{17:30} Lit. \fbib{Men}} from Babylon built Succoth-benoth, settlers\fnote{\fbackref{17:30} Lit. \fbib{men}} from Cuth built Nergal, settlers\fnote{\fbackref{17:30} Lit. \fbib{men}} from Hamath built Ashima, \v{31}and settlers\fnote{\fbackref{17:31} Lit. \fbib{men}} from Avva built Nibhaz and Tartak. The residents of Sephar-vaim burned their children in fire to Adrammelech and Anammelech, the gods of Sephar-vaim.

\v{32}Because they feared the \divine{Lord}, they also appointed from among themselves priests for the high places who acted on their behalf in the temples on the high places. \v{33}While they continued to fear the \divine{Lord}, they served their own gods, following the custom of the nations whom they had carried away from there. \v{34}To this very day, they still follow the former customs: they don't fear the \divine{Lord} and they don't live in accordance with the statutes, ordinances, laws, or commandments that the \divine{Lord} had given to the descendants of Jacob---whom he renamed Israel--- \v{35}and with whom the \divine{Lord} had made a covenant when he gave these\fnote{\fbackref{17:35} The Heb. lacks \fbib{these}} orders to them:

\begin{poetry}
\poeml ``You are not to fear other gods, bow down to them, serve them, or sacrifice to them. \v{36}Instead, it is to be the \divine{Lord}, who brought you up from the land of Egypt, showing great power and public demonstrations of might,\fnote{\fbackref{17:36} Lit. \fbib{and with an outstretched arm}} whom you are to fear, worship, and to whom you are to offer sacrifice. \v{37}Furthermore, you are to be careful to observe forever the statutes, ordinances, law, and the commandment that he wrote for you. And you are not to fear other gods. \v{38}You are not to forget the covenant that I've made with you, and you are not to fear other gods. \v{39}But you are to fear the \divine{Lord}, and he will deliver you from the control\fnote{\fbackref{17:39} Lit. \fbib{hand}} of all your enemies.''
\end{poetry}

\v{40}But they wouldn't listen. Instead, they did what they had been doing before. \v{41}These nations feared the \divine{Lord} and also served their carved images. Their descendants did the same thing, as did their grandchildren. Just as their ancestors had done, they also do the same thing to this day.
\labelchapt{18}
\passage{Hezekiah Becomes King of Judah}
\passageinfo{(2 Chronicles 29:1-2)}

\chapt{18}
\v{1}Now it happened that during the third year of the reign of\fnote{\fbackref{18:1} The Heb. lacks \fbib{the reign of}} Elah's son Hoshea, king of Israel, that Ahaz' son Hezekiah became king. \v{2}He was 25 years old when he became king, and he reigned in Jerusalem for 29 years. His mother was Zechariah's daughter Abi. \v{3}He did what the \divine{Lord} considered to be right, according to everything that his ancestor David had done.
\passage{Hezekiah's Reforms}
\passageinfo{(2 Chronicles 29:3; 31:1)}

\v{4}He removed the high places, demolished the sacred pillars, and tore down the Asherah poles. He also demolished the bronze serpent that Moses had crafted, because the Israelis had been burning incense to it right up until that time. Hezekiah\fnote{\fbackref{18:4} Lit. \fbib{He}} called it a piece of brass.\fnote{\fbackref{18:4} Lit. \fbib{Nehushtan}; so MT; LXX reads \fbib{Neeshthan}} \v{5}He trusted the \divine{Lord} God of Israel, and after him there were none like him among all the kings of Judah, \v{6}because he depended on the \divine{Lord}, not abandoning pursuit of him, and keeping the \divine{Lord}'s commands that he had commanded Moses. \v{7}So the \divine{Lord} was with him, and Hezekiah prospered wherever he went, even when he rebelled against the king of Assyria, refusing to serve him. \v{8}He attacked the Philistines, invading Gaza and its borders from watchtower to fortified garrison.
\passage{Shalmaneser Attacks Samaria}

\v{9}In the fourth year of King Hezekiah's reign (that is, during the seventh year of Elah's son Hoshea's reign as king of Israel), King Shalmaneser from Assyria invaded Samaria and besieged it. \v{10}Three years later, they captured Samaria during the sixth year of Hezekiah's reign,\fnote{\fbackref{18:10} The Heb. lacks \fbib{reign}} which was the ninth year of Hoshea's reign as king of Israel. \v{11}After this, the king of Assyria carried Israel off into exile in Assyria, settling them in Halah, on the Habor River in Gozan, and in cities controlled by the Medes, \v{12}because they would not obey the voice of the \divine{Lord} their God. Instead, they transgressed his covenant, including everything that Moses, the servant of the \divine{Lord}, had commanded, by neither listening nor putting what he had commanded\fnote{\fbackref{18:12} The Heb. lacks \fbib{what he had commanded}} into practice.

\v{13}During the fourteenth year of the reign of\fnote{\fbackref{18:13} The Heb. lacks \fbib{the reign of}} King Hezekiah, King Sennacherib of Assyria approached all of the walled cities of Judah and seized them. \v{14}So Hezekiah sent this message to the king of Assyria at Lachish: ``I have offended you. Withdraw from me, and I'll accept whatever tribute you impose.'' So the king of Assyria required Hezekiah to pay him 300 talents\fnote{\fbackref{18:14} I.e. about 11,500 pounds; a talent weighed about 75 pounds} of silver and 30 talents\fnote{\fbackref{18:14} I.e. about 1,150 pounds; a talent weighed about 75 pounds} of gold. \v{15}Hezekiah gave him all the silver that could be removed from the \divine{Lord}'s Temple and from the treasuries in the king's palace. \v{16}At that time, Hezekiah removed the doors to the \divine{Lord}'s Temple and the doorposts that he had overlaid with gold,\fnote{\fbackref{18:16} The Heb. lacks \fbib{with gold}} and gave the gold\fnote{\fbackref{18:16} Lit. \fbib{gave it}} to the king of Assyria.
\passage{Assyria's King Taunts Hezekiah}
\passageinfo{(2 Chronicles 29:9-19)}

\v{17}Sometime later, the king of Assyria sent Tartan, Rab-saris, and Rab-shakeh from Lachish to King Hezekiah in Jerusalem, accompanied with a large army. \v{18}When they called for the king, Hilkiah's son Eliakim, who managed the household, Shebnah the scribe, and Asaph's son Joah the recorder went out to them. \v{19}Rab-shakeh told them, ``Tell Hezekiah right now, `This is what the great king, the king of Assyria says:

\begin{poetry}
\poeml ```Why are you so confident? \v{20}You're saying---but they're only empty words---`I have enough\fnote{\fbackref{18:20} The Heb. lacks \fbib{I have enough}} advice and resources to conduct warfare!' \\
\poeml ```Now who are you relying on, that you have rebelled against me? \v{21}Look, you're trusting on Egypt to lean on like a staff, but it's a crushed reed, and if you lean on it, it will collapse and pierce your hand. Pharaoh, king of Egypt, is just like that to everyone who relies on him! \\
\poeml \v{22}```Of course, you might tell me, ``We rely on the \divine{Lord} our God!'' But isn't it he whose high places and whose altars Hezekiah has demolished, all the while telling Jerusalem, ``You're to worship in front of this altar in Jerusalem?''\,' \\
\poeml \v{23}```Come now, and make a deal with my master, the king of Assyria, and I'll give you 2,000 horses, if you can furnish them with riders. \v{24}How can you refuse even one official from the least of my master's servants and rely on Egypt for chariots and horsemen? \v{25}``Now then, haven't I come up---apart from the \divine{Lord}---to attack and destroy this place? The \divine{Lord} told me, `Go up against this land and destroy it!'\,''\,'\,''
\end{poetry}

\v{26}At this, Hilkiah's son Eliakim, Shebnah, and Joah asked Rab-shakeh, ``Please speak to your servants in Aramaic, because we understand it, but don't speak the language of Judah to us within the hearing of the people who are on the wall.''

\v{27}But Rab-shakeh spoke to them, ``Has my master sent me to talk about this just to your master and to you, and not also to the men who are sitting on the wall, who will soon be eating their own feces and drinking their own urine\fnote{\fbackref{18:27} An alternate MT reading is \fbib{own water at their feet}}---along with you?'' \v{28}Then Rab-shakeh stood up and cried out loud, ``Listen to what the great king, the king of Assyria has to say. \v{29}This is what the king says:

\begin{poetry}
\poeml `Don't let Hezekiah deceive you, because he will prove to be unable to deliver you from my control.\fnote{\fbackref{18:29} Lit. \fbib{hand}} \v{30}And don't let Hezekiah make you trust in the \divine{Lord} by telling you, ``The \divine{Lord} will certainly deliver us and this city will not be handed over to the king of Assyria.'' \v{31}Don't listen to Hezekiah, because this is what the king of Assyria says: ``Make peace with me and come out to me! Each of you will eat from his own vine. Each will eat from his own fig tree. And each of you will drink water from his own cistern \v{32}until I come and take you away to a land like your own land, one overflowing with grain and new wine, a land filled with bread and vineyards, with olive trees and honey, so you may live and not die.'' \\
\poeml `But don't listen to Hezekiah when he misleads you by saying, ``The \divine{Lord} will deliver us!'' \v{33}Has any of the gods of the nations delivered his land from control by\fnote{\fbackref{18:33} Lit. \fbib{from the hand of}} the king of Assyria? \v{34}Where are the gods of Hamath and Arpad? Where are the gods of Sephar-vaim, of Hena, and Ivvah? Have they delivered Samaria from my control?\fnote{\fbackref{18:34} Lit. \fbib{hand}} \v{35}Who among all the gods of these lands has delivered their land from my control\fnote{\fbackref{18:35} Lit. \fbib{hand}}, so that the \divine{Lord} should deliver Jerusalem from me?'\,''\fnote{\fbackref{18:35} Lit. \fbib{from my hand}}
\end{poetry}

\v{36}But the people remained silent and did not answer with even so much as a word, because the king's order was, ``Don't answer him.''

\v{37}But Hilkiah's son Eliakim, who managed the household, Shebna the scribe, and Asaph's son Joah the recorder came back to Hezekiah with their clothes torn\fnote{\fbackref{18:37} I.e. as a visible response to the pending calamity} and told him what Rab-shakeh had said.
\labelchapt{19}
\passage{Isaiah Encourages Hezekiah}

\chapt{19}
\v{1}When King Hezekiah heard Eliakim's report,\fnote{\fbackref{19:1} The Heb. lacks \fbib{Eliakim's report}} he tore his clothes, put on a sackcloth covering, entered the \divine{Lord}'s Temple, \v{2}and sent Eliakim the household supervisor, Shebna the scribe, and the elders of the priests---all of them covered in sackcloth---to Amoz's son, the prophet Isaiah. \v{3}They announced to him:

\begin{poetry}
\poeml ``This is what Hezekiah says: `Today is a day of trouble, rebuke, and blasphemy,\fnote{\fbackref{19:3} Or \fbib{contempt}} because children are about to be born, but there is no strength to bring them to birth. \v{4}Perhaps the \divine{Lord} your God will take note of everything that Rab-shakeh has said, whom his master the king of Assyria sent to taunt the living God, and then he will rebuke the words that the \divine{Lord} your God has heard. Therefore offer a prayer for the survivors who remain.'\,''
\end{poetry}

\v{5}That is how the King Hezekiah's servants approached Isaiah.

\v{6}In reply, Isaiah responded to them, ``Here's how you're to report to your master:

\begin{poetry}
\poeml `This is what the \divine{Lord} says: ``Never be afraid of the words that you have heard by which the servants of the king of Assyria have blasphemed me. \v{7}Look! I'm going to cause an attitude\fnote{\fbackref{19:7} Or \fbib{to bring a spirit}} to grow within him so that he'll hear a rumor and return to his own territory, where I'll make him die by the sword in his own land!''\,'\,''
\end{poetry}
\passage{Sennacherib Defies God}
\passageinfo{(2 Chronicles 29:17-19)}

\v{8}So Rab-shakeh returned and found the king of Assyria at war with Libnah, because Rab-shakeh had heard that the king had left Lachish. \v{9}When he heard that it was being said about King Tirhakah of Ethiopia,\fnote{\fbackref{19:9} Lit. \fbib{Cush}} ``Look! He has come out to attack you!'' he again sent messengers to Hezekiah.

The messengers were told, \v{10}``This is what you are to say to King Hezekiah of Judah: `Don't let your God in whom you trust deceive you by telling you\fnote{\fbackref{19:10} The Heb. lacks \fbib{you}} ``Jerusalem won't be turned over to the control\fnote{\fbackref{19:10} Lit. \fbib{hand}} of Assyria's king.'' \v{11}`Look! you've heard what the kings of Assyria have done to all the lands---they completely destroyed them! Will you be spared? \v{12}Did the gods of those nations whom my ancestors destroyed deliver them, including Gozan, Haran, Rezeph, and Eden's descendants in Telassar? \v{13}Where is the king of Hamath, the king of Arpad, the king of the city of Sephar-vaim, the king of Hena, or the king of Ivvah?'\,''
\passage{Hezekiah's Prayer for Help}

\v{14}Hezekiah took the messages from the couriers, read them, went up to the \divine{Lord}'s Temple, and laid them out in the presence of the \divine{Lord}. \v{15}Then Hezekiah prayed in the presence of the \divine{Lord}, ``\divine{Lord} God of Israel! You live between the cherubim! You alone are the God of all the kingdoms of the earth. You have fashioned the heavens and the earth. \v{16}Turn\fnote{\fbackref{19:16} Or \fbib{Bow down}} your ear, \divine{Lord}, and listen! Open your eyes, \divine{Lord}, and observe! Listen to the message sent by Sennacherib to insult the living God! \v{17}Truly, \divine{Lord}, the kings of Assyria have devastated nations and their territories, \v{18}throwing their gods into the fire, since they weren't gods but rather were the product of men's handiwork---wood and stone. And so they destroyed them. \v{19}Now, \divine{Lord} our God, I'm praying that you will deliver us from his control, so that all the kingdoms of the earth may know that you alone, \divine{Lord}, are God!''
\passage{God's Answer through Isaiah the Prophet}

\v{20}Then Amoz's son Isaiah sent word to Hezekiah, ``This is what the \divine{Lord}, the God of Israel says: `Because you have prayed to me about King Sennacherib of Assyria, I have listened.'\,''

\v{21}``This is what the \divine{Lord} has spoken against him:

\begin{poetry}
\poeml `She despises and mocks you, \\
\poemll    this virgin daughter of Zion! \\
\poeml Behind your back she shakes her head, \\
\poemll    this daughter of Jerusalem! \\
\poeml \v{22}Who are you reproaching and blaspheming? \\
\poemll    Against whom have you raised your voice? \\
\poeml And against whom\fnote{\fbackref{19:22} The Heb. lacks \fbib{against whom}} have you lifted up your eyes in arrogance? \\
\poemll    Against the Holy One of Israel! \\
\poeml \v{23}By your messengers you have insulted the \divine{Lord}. \\
\poemll    You have claimed, \\
\poeml ``With my many chariots \\
\poemll    I ascended the heights of the mountains, \\
\poemlll       including the remotest regions of Lebanon; \\
\poeml I cut down its tall cedars \\
\poemll    and the best of its cypress trees. \\
\poeml I entered its most remote lodging place \\
\poemll    and its most fruitful\fnote{\fbackref{19:23} Or \fbib{its densest}} forest. \\
\poeml \v{24}I myself dug for and drank foreign water. \\
\poemll    With the sole of my foot I dried up all the streams of Egypt!'' \\
\poeml \v{25}`Didn't you hear? \\
\poemll    I determined it years ago! \\
\poeml I planned this from ancient times, \\
\poemll    and now I've brought it to pass, \\
\poeml to turn fortified cities \\
\poemll    into piles of ruins \\
\poeml \v{26}while their inhabitants, lacking strength, \\
\poemll    stand dismayed and confused. \\
\poeml They were like vegetation out in the fields, \\
\poemll    and like green herbs--- \\
\poeml just as grass that grows on a housetop \\
\poemll    dries out before it can grow. \\
\poeml \v{27}`But when you sit down, \\
\poemll    when you go out, \\
\poeml and when you come in, \\
\poemll    I'm aware of it! \\
\poeml \v{28}Because of your rage against me, \\
\poemll    your complacency has reached my ears. \\
\poeml I'll put my hook into your nostrils \\
\poemll    and my bit into your mouth. \\
\poeml Then I'll turn you back on the road \\
\poemll    by which you came.'
\end{poetry}

\v{29}``This will serve as a sign for you: you'll eat this year from what grows by itself, in the second year what grows from that, and in the third year you'll sow, reap, plant vineyards, and enjoy\fnote{\fbackref{19:29} Lit. \fbib{eat}} their fruit. \v{30}Those who survive from Judah's household will again put down deep roots and bear fruit extensively,\fnote{\fbackref{19:30} Or \fbib{upwards}} \v{31}because a remnant will go out from Jerusalem, and survivors from Mount Zion. The zeal of the \divine{Lord}\fnote{\fbackref{19:31} So MT; LXX and a MT variant read \fbib{\divine{Lord} of the Heavenly Armies}} will bring this about.''

\v{32}``Therefore this is what the \divine{Lord} says concerning the king of Assyria: `Not only will he not approach this city or shoot an arrow in its direction, he won't approach it with so much as a shield, nor will he throw up a siege ramp against it. \v{33}He'll return on the same route by which he came---he won't come to this city,' declares the \divine{Lord}. \v{34}`I will defend this city and preserve it for my own reasons, and because of my servant David.'\,''
\passage{God Destroys the Assyrian Army}
\passageinfo{(2 Chronicles 32:20-21)}

\v{35}That very night, the angel of the \divine{Lord} went out to the camp of the Assyrian army and killed 185,000 men. Early the next morning, when the army of Israel\fnote{\fbackref{19:35} Lit. \fbib{when they}} arose, all 185,000 soldiers\fnote{\fbackref{19:35} The Heb. lacks \fbib{185,000 soldiers}} were dead. \v{36}As a result, King Sennacherib of Assyria left and returned to Nineveh where he lived. \v{37}Later on, as he was worshiping in the temple of his god Nisroch, Adrammelech\fnote{\fbackref{19:37} So MT; LXX and a MT variant read \fbib{his sons Adrammelech}} and Sharezer killed him with a sword and fled into the territory of Ararat. Then Sennacherib's\fnote{\fbackref{19:37} Lit. \fbib{his}} son Esarhaddon became king in his place.
\labelchapt{20}
\passage{Hezekiah's Sickness and Recovery}
\passageinfo{(2 Chronicles 32:24-26)}

\chapt{20}
\v{1}During this time, Hezekiah became sick with a fatal illness, so Isaiah the prophet, the son of Amoz, approached him and told him, ``This is what the \divine{Lord} says: `Put your household in order, because you are dying. You will not survive.'\,''

\v{2}So Hezekiah turned his face to the wall and prayed to the \divine{Lord}. \v{3}``Remember me, \divine{Lord},'' he said, ``how I have walked in your presence with integrity, with an undivided heart, and I have accomplished what is good in your sight.'' And Hezekiah wept deeply.

\v{4}Before Isaiah had left the middle court, this message from the \divine{Lord} came to him. \v{5}``Return to Hezekiah,'' he said, ``and tell the Commander-in-Chief\fnote{\fbackref{20:5} Lit. \fbib{Nagid}; i.e. a senior officer entrusted with dual roles of operational oversight and administrative authority} of my people: `This is what the \divine{Lord}, the God of your ancestor David, says: ``I've heard your prayer and I've observed your tears. Look! I'm healing you. Three days from now, you'll go visit the \divine{Lord}'s Temple. \v{6}Furthermore, I'll add fifteen years to your life. I'll deliver you and this city from domination by\fnote{\fbackref{20:6} Lit. \fbib{from the hand of}} the king of Assyria, and I'll defend this city for my own sake and for the sake of my servant David.''\,'\,''

\v{7}Isaiah said, ``Take a fig cake.'' So some attendants\fnote{\fbackref{20:7} Lit. \fbib{So they}} took it, laid it on Hezekiah's\fnote{\fbackref{20:7} Lit. \fbib{the}} boil, and he recovered.

\v{8}Now Hezekiah had asked Isaiah, ``What is to be the sign that the \divine{Lord} is healing me and that I'll be going up to the \divine{Lord}'s Temple three days from now?''

\v{9}So Isaiah replied, ``This will be your sign from the \divine{Lord} that the \divine{Lord} will do what he has promised. Shall the shadow go forward ten steps or go back ten steps?''

\v{10}Hezekiah answered, ``It's an easy thing for a shadow to lengthen ten steps. So let the shadow go backward ten steps.''

\v{11}So Isaiah cried out to the \divine{Lord}, who brought the shadow back ten steps after it had gone down the stairway of Ahaz.
\passage{Hezekiah Shows His Treasure to the Babylonian Envoys}

\v{12}Some time later, Berodach-baladan,\fnote{\fbackref{20:12} So MT; LXX and a MT variant read \fbib{Marodach-baladan}} the son of King Baladan of Babylon, sent letters and a gift to Hezekiah, because he had heard that Hezekiah had been ill. \v{13}Hezekiah listened to the entourage\fnote{\fbackref{20:13} Lit. \fbib{to them}} and showed them his entire treasury, including the silver, gold, and spices, the precious oil, his armory, and everything that was inventoried in his treasuries. There was nothing in his household or in his holdings that Hezekiah did not show them.

\v{14}Then Isaiah the prophet came to King Hezekiah and asked him, ``What did these men have to say, and where did they come from?''

Hezekiah replied, ``They came from a country far away---from Babylon.''

\v{15}He asked, ``What did they see in your household?''

Hezekiah answered, ``They have seen everything. In my household there is nothing in my treasuries that I haven't shown them.''

\v{16}Then Isaiah replied to Hezekiah, ``Listen to this message from the \divine{Lord}: \v{17}`Watch out! The days are coming when everything that's in your house---everything that your ancestors have saved up right to this day---will be carried off to Babylon. Nothing will be left,' declares the \divine{Lord}. \v{18}`Some of your descendants---your very own seed, whom you will father---will be carried away to become officials\fnote{\fbackref{20:18} Or \fbib{court officials}; the position may have mandated castration as a condition of service} in the palace of the king of Babylon.'\,''

\v{19}At this, Hezekiah replied to Isaiah, ``What you've spoken from the \divine{Lord} is good,'' because he had been thinking, ``Why not, as long as there's peace and security\fnote{\fbackref{20:19} Lit. \fbib{truth}} in my lifetime{\ldots}?''

\v{20}Now the rest of Hezekiah's actions, as well as his glorious deeds, including how he constructed the pool and the conduit to bring water into the city, are recorded in the Book of the Chronicles of the Kings of Judah, are they not? \v{21}Hezekiah died, as did\fnote{\fbackref{20:21} Lit. \fbib{Hezekiah slept with}} his ancestors, and his son Manasseh became king in his place.
\labelchapt{21}
\passage{Manasseh Succeeds Hezekiah}

\chapt{21}
\v{1}Manasseh began to reign at the age of twelve, and he reigned for 55 years in Jerusalem. His mother was named Hephzibah. \v{2}He did what the \divine{Lord} considered to be evil, following the despicable practices of the nations whom the \divine{Lord} had expelled in full view of the people of Israel. \v{3}He rebuilt the high places that his father Hezekiah had destroyed. He erected altars for Baal, crafted an Asherah, just as King Ahab of Israel had done, and worshipped and served the stars of heaven. \v{4}He also built altars in the \divine{Lord}'s Temple, about which the \divine{Lord} had said, ``In Jerusalem I will place my Name.'' \v{5}He built two altars to every star in the heavens in the two courts of the \divine{Lord}'s Temple. \v{6}He made his son into a burnt offering, practiced witchcraft, used divination, and consorted with mediums and spirit-channelers.\fnote{\fbackref{21:6} Or \fbib{wizards}} He practiced many things that the \divine{Lord} considered to be evil and provoked him.

\v{7}He also erected the carved image of Asherah that he had made inside the Temple about which the \divine{Lord} had spoken to David and to his son Solomon, ``I will put my Name forever in this Temple and in Jerusalem, which I have chosen from all of the tribes of Israel. \v{8}And I will not make Israel's feet to wander anymore from the land that I have given to their ancestors, if they will only be careful to do everything that I have commanded them according to the entire Law that my servant Moses commanded them.'' \v{9}But they would not listen. Manasseh led them astray to practice more evil than the nations whom the \divine{Lord} had destroyed in the presence of the Israelis.
\passage{The \divine{Lord} Rebukes Manasseh's Idolatry}

\v{10}So the \divine{Lord} announced through his prophets, \v{11}``Because King Manasseh of Judah has committed these despicable things, acting more sinfully than did all of the Amorites who preceded him, including making Judah sin with its idols, \v{12}therefore this is what the \divine{Lord} God of Israel says: `Look! I'm going to bring such a\fnote{\fbackref{21:12} The Heb. lacks \fbib{such a}} disaster to Jerusalem and Judah that both ears of those who hear about it will ring. \v{13}I'll stretch out over Jerusalem the measuring line that is Samaria and the plumb line that is Ahab's dynasty. Then I'll wipe Jerusalem like one wipes a dish, wiping it and turning it upside down! \v{14}I will abandon the survivors of my heritage and hand them over to their enemies. They will become war booty and spoil to all of their enemies, \v{15}because they have done what I consider to be evil and they have provoked me from the day their ancestors left Egypt right up to this day!'\,''

\v{16}In addition to this, Manasseh shed lots of innocent blood---until he had filled Jerusalem from one end to another---besides his sin by which he caused Judah to sin by practicing what the \divine{Lord} considered to be evil. \v{17}The rest of Manasseh's deeds, including everything that he accomplished and the sin that he practiced, are recorded in the Book of the Chronicles of the Kings of Judah, are they not? \v{18}Manasseh died, as did\fnote{\fbackref{21:18} Lit. \fbib{Manasseh slept with}} his ancestors, and he was buried in the garden at his home in the Garden of Uzza. His son Amon became king in his place.
\passage{Amon Reigns in Judah}

\v{19}Amon began to reign at the age of 22, and ruled for two years in Jerusalem. His mother was named Meshullemeth, the daughter of Haruz of Jotbah. \v{20}He practiced what the \divine{Lord} considered to be evil, just as his father Manasseh had done, \v{21}because he completely adopted his father's lifestyle, serving the same idols his father had served and worshipped. \v{22}As a result, he abandoned the \divine{Lord} God of his ancestors and did not walk in the \divine{Lord}'s way. \v{23}Later on, Amon's staff conspired against him and killed the king inside his own home. \v{24}But afterward, the people of the land executed everyone who had conspired against King Amon, and the people of the land installed his son Josiah to be king in his place.

\v{25}Now the rest of Amon's activities that he undertook are recorded in the Book of the Chronicles of the Kings of Judah, are they not? \v{26}He was buried in his own grave in the Garden of Uzza, and his son Josiah became king in his place.
\labelchapt{22}
\passage{Josiah Succeeds Amon}

\chapt{22}
\v{1}Josiah was an eight year old child when he began to reign, and he reigned for 31 years in Jerusalem. His mother was named Jedidah, the daughter of Adaiah of Bozkath. \v{2}He practiced what the \divine{Lord} considered to be right, living the way his ancestor David had lived, turning neither to the right nor to the left.

\v{3}Eighteen years after King Josiah had begun to reign, the king sent Azaliah's son Shaphan, grandson of Meshullam the scribe, to the \divine{Lord}'s Temple. He told him, \v{4}``Go to the high priest Hilkiah, so he can count the money that has been brought into the \divine{Lord}'s Temple by the doorkeepers who have been gathering it from the people. \v{5}Have them deliver it to the workmen who are supervising the \divine{Lord}'s Temple, so that they may pay it over to the workmen who serve in the \divine{Lord}'s Temple to repair its damages, \v{6}including paying\fnote{\fbackref{22:6} The Heb. lacks \fbib{paying}} the carpenters, builders, and masons, as well as buying timber and pre-carved stone to repair the Temple. \v{7}But you won't need to force them to be accountable for money already paid to them, since they're faithful.''
\passage{Hilkiah Discovers an Ancient Archive}

\v{8}Later on, Hilkiah the high priest informed Shaphan the scribe, ``I've discovered the Book of the Law in the \divine{Lord}'s Temple.'' Hilkiah gave the book to Shaphan, and he began to read it.

\v{9}Shaphan the scribe reported to King Josiah, brought up the matter to him, and told him, ``Your servants have distributed the money that was found in the Temple by giving it to the workmen who supervise the \divine{Lord}'s Temple.'' \v{10}Then Shaphan the scribe informed the king, ``Hilkiah the priest has given me a book.'' Then Shaphan read from it in the king's presence.

\v{11}When the king heard what was written in the Book of the Law, he tore his clothes \v{12}and issued these orders to Hilkiah the priest, Shaphan's son Ahikam, Micaiah's son Achbor, Shaphan the scribe, and the king's servant Asaiah: \v{13}``Go ask the \divine{Lord} for me, for the people, and for all of Judah about what's written in this book that has been discovered, because the \divine{Lord}'s anger is burning against us, since our ancestors have not listened to the words written in this book and have not lived according to everything that is written concerning us.''
\passage{Huldah Predicts Disaster}

\v{14}So Hilkiah the priest, Ahikam, Achbor, Shaphan, and Asaiah went to the prophet Huldah, the wife of Tikvah's son Shallum, the grandson of Harhas and supervisor of the royal wardrobe, who lived in the Second Quarter in Jerusalem. They spoke with her, \v{15}and she told them, ``This is what the \divine{Lord} God of Israel says: `Tell the man who sent you to me: \v{16}``This is what the Lord says: `Look! I'm bringing disaster on this place and on its inhabitants---everything written in the book that the king of Judah has read---\v{17}because they have abandoned me, burned incense to other gods, and they have provoked me to anger with everything that they've done. Therefore my anger is kindled against this place and it won't be quenched!'\,'' \v{18}Nevertheless, tell the king of Judah who sent you to ask the \divine{Lord} about this,\fnote{\fbackref{22:18} The Heb. lacks \fbib{about this}} ``This is what the \divine{Lord} God of Israel says: `Now about what you've heard, \v{19}because your heart was sensitive, and you humbled yourself in the \divine{Lord}'s presence when you heard what I had to say against this place and against its inhabitants---that they would become a desolation and a curse---and you have torn your clothes and cried out before me, be assured that I have truly heard you,' declares the \divine{Lord}. \v{20}`Therefore, look! I will gather you to your ancestors, and you will be placed in your grave in peace. Your eyes will never see all the evil that I will bring on this place.'\,''\,'\,''
\labelchapt{23}
\passage{Josiah's Covenant}

\chapt{23}
\v{1}At this, the king sent for and gathered together all the elders of Judah and Jerusalem. \v{2}The king went up to the \divine{Lord}'s Temple, accompanied by all the men of Judah, everyone who lived in Jerusalem, the priests, the prophets, and everyone---including those who were unimportant and those who were important---and he read to them everything written in the Book of the Covenant that had been discovered in the \divine{Lord}'s Temple. \v{3}The king stood beside a pillar and made a covenant in the presence of the \divine{Lord}: to follow after the \divine{Lord}, to keep his commandments, his testimonies, and his statutes with all of his heart and soul, and to carry out what was written in the covenant contained in the book. All the people consented to enter into the covenant.
\passage{Josiah Abolishes Idolatry}

\v{4}The king ordered Hilkiah the high priest, the priests of the secondary order, and the doorkeepers to take out of the \divine{Lord}'s Temple all of the implements that had been crafted for Baal, for Asherah, and for every star in the heavens. Then he burned them outside Jerusalem in the fields of the Kidron and carried the ashes to Bethel. \v{5}The king unseated the idolatrous priests whom the kings of Judah had appointed to burn incense in the high places throughout the cities of Judah and in the environs surrounding Jerusalem, including those who had been burning incense to Baal, to the sun, to the moon, to the constellations, and to every star in the heavens. \v{6}He brought the Asherah from the \divine{Lord}'s Temple to the Kidron Brook outside Jerusalem, burned it at the Kidron brook, pulverized the ashes\fnote{\fbackref{23:6} The Heb. lacks \fbib{the ashes}} to dust, and scattered it\fnote{\fbackref{23:6} The Heb. lacks \fbib{it}} over the graves of the common people.

\v{7}He also demolished the temples of the cultic male prostitutes that had been operating\fnote{\fbackref{23:7} The Heb. lacks \fbib{operating}} in the \divine{Lord}'s Temple, where the women had been doing weaving for the Asherah. \v{8}Then he gathered together all the priests from the cities of Judah and defiled the high places from Geba to Beer-sheba, where the priests had burned incense. He also demolished the high places of the gates that had been erected to the left as one enters the city gate---that is, near the entrance operated by Joshua, the governor of the city. \v{9}Nevertheless, the priests of the high places did not approach the \divine{Lord}'s altar in Jerusalem, but instead they ate unleavened bread given to them by their\fnote{\fbackref{23:9} Or \fbib{bread among}} relatives.

\v{10}He also defiled Topheth, which is located in the Ben-hinnom Valley,\fnote{\fbackref{23:10} So MT; LXX and MT variant read \fbib{the valley of the descendants of Hinnom}} so that no one would force his son or daughter to pass through the fire in dedication to Molech. \v{11}He abolished the horses that the kings of Judah had dedicated to the sun at the entrance to the \divine{Lord}'s Temple, near the offices of Nathan-melech, the official, that were in the precincts. He also set fire to the chariots of the sun.

\v{12}The king demolished the rooftop altars on top of Ahaz's upper chamber that the kings of Judah had erected, as well as the altars that Manasseh had made in the two courts of the \divine{Lord}'s Temple. He pulverized them where they stood and cast their dust into the Kidron Brook. \v{13}The king defiled the high places which faced\fnote{\fbackref{23:13} So LXX.} Jerusalem on the south\fnote{\fbackref{23:13} Lit. \fbib{right}; i.e. the side on the right when facing east} side of Corruption Mountain, which King Solomon of Israel had constructed for Ashtoreth, the Sidonian abomination, for Chemosh, the Moabite abomination, and for Milcom, the Ammonite abomination. \v{14}He broke the pillars to pieces, cut down the Asherim, and filled their locations with human bones.

\v{15}Furthermore, he even broke down the altar that had been at Bethel as well as the high place constructed by Nebat's son Jeroboam, who had caused Israel to sin. He demolished its stones, pulverized them to dust, and burned the Asherah. \v{16}As Josiah turned around, he observed the graves located there on the mountain, so he sent for and recovered the bones from the graves and burned them on the altar to defile it, in keeping with the message from the \divine{Lord} that the godly man had proclaimed when he was declaring these things. \v{17}He asked, ``What is this monument that I'm looking at?''

The men who lived in that city answered him, ``It's the grave of that godly man who came from Judah and predicted these things that you've done against the altar at Bethel!''

\v{18}Josiah\fnote{\fbackref{23:18} Lit. \fbib{He}} replied, ``Leave him alone. No one is to disturb his bones.'' So they preserved his bones undisturbed, along with the bones of the prophet who had come from Samaria. \v{19}Josiah also removed all of the temples on the high places that had been in the cities of Samaria and that the kings of Israel had erected, thereby provoking the \divine{Lord}.\fnote{\fbackref{23:19} So LXX. The Heb. lacks \fbib{the \divine{Lord}}} He treated Samaria\fnote{\fbackref{23:19} Lit. \fbib{them}} just as he had Bethel. \v{20}After he had slaughtered all the priests who served at the high places and burned their bones on those high places, he returned to Jerusalem.
\passage{Josiah Reinstates the Passover}

\v{21}After this, the king commanded all of the people, ``Celebrate the Passover to the \divine{Lord} your God, just as it's prescribed in this Book of the Covenant.'' \v{22}From the days of the judges who ruled in Israel, no Passover had been celebrated like this, not even in all the reigns of the kings of Israel and the kings of Judah. \v{23}In the eighteenth year of the reign of\fnote{\fbackref{23:23} The Heb. lacks \fbib{the reign of}} King Josiah, this Passover was observed in Jerusalem to honor the \divine{Lord}. \v{24}Furthermore, Josiah removed the mediums, the necromancers, the household gods,\fnote{\fbackref{23:24} Lit. \fbib{the teraphim}} the idols, and every despicable thing that could be seen in the territory of Judah and in Jerusalem, so that he might confirm the words of the Law that had been written in the book that Hilkiah the priest had discovered in the \divine{Lord}'s Temple. \v{25}There had been no king like him before him, who turned to the \divine{Lord} with all his heart, with all his soul, and with all his strength, in obeying everything in the Law of Moses. No king arose like Josiah after him.

\v{26}Even so, the \divine{Lord} did not turn away from his fierce and great anger that burned against Judah because of everything with which Manasseh had provoked him. \v{27}The \divine{Lord} said, ``I'm going to remove Judah from my sight as well, just as I've removed Israel. I will abandon Jerusalem, this city that I've chosen, as well as the Temple, about which I've spoken, `My Name shall remain there.'\,''
\passage{Pharaoh Neco Kills Josiah}

\v{28}Now the rest of Josiah's actions, including everything that he did, are recorded in the Book of the Chronicles of the Kings of Judah, are they not? \v{29}During his reign, Pharaoh Neco, king of Egypt, marched out toward the Euphrates River to meet the king of Assyria. King Josiah went out to engage him in battle, but Pharaoh Neco\fnote{\fbackref{23:29} Lit. \fbib{but he}} killed him at Megiddo as soon as he saw him. \v{30}Josiah's servants drove his corpse in a chariot from Megiddo to Jerusalem and buried him in a tomb made for him.
\passage{Jehoahaz is Anointed King}

The people of the land took Josiah's son Jehoahaz, anointed him, and installed him as king in his father's place. \v{31}Jehoahaz was 23 years old when he became king. He reigned three months in Jerusalem. His mother's name was Hamutal. She was the daughter of Jeremiah of Libnah. \v{32}He practiced what the \divine{Lord} considered to be evil, just as all of his ancestors had done. \v{33}Pharaoh Neco placed him in custody at Riblah, in the land of Hamath, so that he would not reign in Jerusalem, and imposed a tribute of 100 talents\fnote{\fbackref{23:33} I.e. about 7,500 pounds; a talent weighed about 75 pounds} of silver and a talent\fnote{\fbackref{23:33} I.e. about 75 pounds; a talent weighed about 75 pounds} of gold.
\passage{Jehoiakim is Made King by Pharaoh Neco}

\v{34}Pharaoh Neco installed Josiah's son Eliakim as king to replace his father Josiah and changed his name to Jehoiakim. He transported Jehoahaz off to Egypt, where he died. \v{35}As a result, Jehoiakim paid the silver and gold tribute\fnote{\fbackref{23:35} The Heb. lacks \fbib{tribute}} to Pharaoh, but he passed on the costs to the inhabitants of the land in taxes, in keeping with Pharaoh's orders. He exacted the silver and gold from the people who lived in the land, from each according to his assessment, in order to pay it to Pharaoh Neco. \v{36}Jehoiakim was 25 years old when he became king, and he reigned for eleven years in Jerusalem. His mother was named Zebidah. She was the daughter of Pedaiah of Rumah. \v{37}Eliakim practiced what the \divine{Lord} considered to be evil, just as his ancestors had done.
\labelchapt{24}
\passage{Jehoiakim Serves Nebuchadnezzar}

\chapt{24}
\v{1}During his lifetime, King Nebuchadnezzar of Babylon attacked Jehoiakim, who became his vassal for three years, after which he turned against Nebuchadnezzar\fnote{\fbackref{24:1} Lit. \fbib{him}} and rebelled. \v{2}The \divine{Lord} sent raiding parties from the Chaldeans, Arameans, Moabites, and Ammonites against Jehoiakim. He sent them against Judah to destroy it, in keeping with the message from the \divine{Lord} that he had spoken through his servants, the prophets. \v{3}It was truly by the command of the \divine{Lord} against Judah that it came, in order to remove them from his sight, because of every sin that Manasseh had committed, \v{4}as well as for the innocent blood that he had shed. He had filled Jerusalem with innocent blood, and the \divine{Lord} would not forgive them.\fnote{\fbackref{24:4} The Heb. lacks \fbib{them}} \v{5}Now the rest of Jehoiakim's actions, and everything that he undertook, are recorded in the Book of the Chronicles of the Kings of Judah, are they not? \v{6}Jehoiakim died, as did\fnote{\fbackref{24:6} Lit. \fbib{Jehoiakim slept with}} his ancestors, and his son Jehoiachin became king in his place. \v{7}The king of Egypt did not leave his territory again, because the king of Babylon had taken everything that belonged to the king of Egypt from the Wadi\fnote{\fbackref{24:7} I.e. a seasonal stream or river that channels water during rain seasons but is dry at other times} of Egypt to the Euphrates River.
\passage{Jehoiachin Becomes King}

\v{8}Jehoiachin became king at the age of eighteen years, and he reigned for three months in Jerusalem. His mother was named Hausa. She was the daughter of Elzaphan of Jerusalem. \v{9}He practiced what the \divine{Lord} considered to be evil, just as his ancestors had done. \v{10}At that time, the servants of King Nebuchadnezzar of Babylon attacked Jerusalem and the city was placed under siege. \v{11}King Nebuchadnezzar of Babylon came up against the city, along with his servants, who besieged it. \v{12}King Jehoiachin of Judah surrendered to the king of Babylon (as did his mother, his servants, his princes, and his officers) during the eighth year of his reign.
\passage{Jerusalem's Citizens are Sent into Exile}

\v{13}Nebuchadnezzar\fnote{\fbackref{24:13} Lit. \fbib{He}} carried off from there all of the treasures of the \divine{Lord}'s Temple, along with the treasures in the king's palace. He cut into pieces all the gold vessels in the \divine{Lord}'s Temple that King Solomon of Israel had made, just as the \divine{Lord} had said would happen.\fnote{\fbackref{24:13} The Heb. lacks \fbib{would happen}} \v{14}Then Nebuchadnezzar sent away into exile all of Jerusalem---all the captains, all the valiant soldiers, 10,000 captives, and all of the craftsmen and ironworkers. Nobody remained except the poorest people of the land. \v{15}He sent Jehoiachin into exile to Babylon, along with the king's mother, the king's wives, his officials, and the leading men of the land. He took them into exile from Jerusalem to Babylon. \v{16}All 7,000 of the most valiant soldiers and 1,000 of the craftsmen and ironworkers---all physically fit and trained for battle---were brought by the king of Babylon into exile in Babylon.
\passage{Zedekiah is Installed as King}

\v{17}The king of Babylon installed Jehoiachin's\fnote{\fbackref{24:17} Lit. \fbib{installed his}} uncle Mattaniah as king in his place and then changed his name to Zedekiah. \v{18}Zedekiah was 21 years old when he became king. He reigned for eleven years in Jerusalem. His mother was named Hamutal. She was the daughter of Jeremiah of Libnah. \v{19}Zedekiah practiced what the \divine{Lord} considered to be evil, just as Jehoiakim had done, \v{20}because through the \divine{Lord}'s anger these things happened\fnote{\fbackref{24:20} The Heb. lacks \fbib{these things}} to Jerusalem and Judah until he threw them from his presence.
\labelchapt{25}
\passage{Nebuchadnezzar Captures Jerusalem}

Zedekiah then rebelled against the king of Babylon,\chapt{25}
\v{1}so on the tenth day of the tenth month of the ninth year of Zedekiah's\fnote{\fbackref{25:1} Lit. \fbib{his}; but cf. 25:3, which suggests it refers to Zedekiah} reign, King Nebuchadnezzar of Babylon and his entire army approached Jerusalem, attacked it, encamped against it, and built a siege wall that surrounded the city. \v{2}The city remained under siege until the eleventh year of the reign of\fnote{\fbackref{25:2} The Heb. lacks \fbib{the reign of}} King Zedekiah. \v{3}By the ninth day of the fourth\fnote{\fbackref{25:3} The Heb. lacks \fbib{fourth}; but cf. Jer. 52:6} month, the resulting\fnote{\fbackref{25:3} The Heb. lacks \fbib{resulting}} famine had become so severe in the city that no food remained for the people who lived in the land. \v{4}The city was breached, and the entire army left during the night through the gate that stood between the two walls beside the royal garden, even though the Chaldeans had surrounded the city. They escaped through the Arabah, \v{5}but the Chaldean army pursued the king and overtook him in the Jericho plains, where his entire army was scattered. \v{6}The Chaldeans captured the king and brought him to Riblah, where the king of Babylon determined his sentence. \v{7}They executed Zedekiah's sons in his presence, blinded Zedekiah, bound him with bronze chains, and transported him to Babylon.
\passage{Jerusalem is Burned and the Temple Demolished}

\v{8}On the seventh\fnote{\fbackref{25:8} Cf. Jer 52:12, which reads \fbib{tenth}} day of the fifth month, which was during the nineteenth year of King Nebuchadnezzar's reign as king of Babylon, captain of the guard Nebuzaradan, a servant of the king of Babylon, arrived in Jerusalem \v{9}and set fire to the \divine{Lord}'s Temple, the royal palace, and all the houses of Jerusalem. He even incinerated the lavish\fnote{\fbackref{25:9} Lit. \fbib{great}} homes. \v{10}The Chaldean army that accompanied the captain of the guard demolished the walls that surrounded Jerusalem. \v{11}Nebuzaradan, the captain of the guard, carried the survivors of the people who remained in the city, those who had deserted to the king of Babylon, and the rest of the multitude into exile. \v{12}However, the captain of the guard left some of the poor people of the land to work as vinedressers and farmers.

\v{13}The Chaldeans also broke into pieces and carried back to Babylon the bronze pillars that stood in the \divine{Lord}'s Temple, along with the stands and the bronze sea\fnote{\fbackref{25:13} Cf. 1King 7:23-26; 2Chr 4:2-4} that used to be in the \divine{Lord}'s Temple. \v{14}They also confiscated\fnote{\fbackref{25:14} Or \fbib{took away}} the pots, shovels, snuffers, spoons, and the rest of the bronze vessels that were used in ministry. \v{15}The captain of the guard also confiscated\fnote{\fbackref{25:15} Or \fbib{took away}} the fire pans, basins, and whatever had been crafted of pure gold and pure silver. \v{16}The bronze contained in the two pillars, the one sea, and the stands that Solomon had crafted for the \divine{Lord}'s Temple could not be inventoried for weight. \v{17}The height of one of the pillars was eighteen cubits,\fnote{\fbackref{25:17} I.e. about 24 feet; a cubit was about eighteen inches long} and the capital on top of it was three cubits\fnote{\fbackref{25:17} I.e. about 4 and a half feet; a cubit was about eighteen inches long} high.\fnote{\fbackref{25:17} The Heb. lacks \fbib{high}} A latticework carved in the form of pomegranates encircled the capital, crafted completely out of brass. The second pillar was identical to the first.\fnote{\fbackref{25:17} Lit. \fbib{to these with latticework}}
\passage{Judah's Leaders are Executed}

\v{18}The captain of the guard arrested Seraiah the chief priest, Zephaniah the second priest, three temple officials,\fnote{\fbackref{25:18} Lit. \fbib{three threshold keepers}} \v{19}one overseer from the city who supervised the soldiers, five of the king's advisors who had been discovered in the city, the scribe who served the army captain who mustered the army of the land, and 60 men of the land who were discovered in the city. \v{20}Nebuzaradan, the captain of the guard, took them to the king of Babylon at Riblah, \v{21}where the king of Babylon executed them in the land of Hamath. And so Judah was transported into exile from the land.
\passage{Gedaliah is Appointed Governor}

\v{22}Now as for the people who remained in the land of Judah whom King Nebuchadnezzar of Babylon had left behind, he appointed Ahikam's son Gedaliah, the grandson of Shaphan, to rule. \v{23}When all the captains of the armies, along with their men, heard that the king of Babylon had appointed Gedaliah, these men visited Gedaliah at Mizpah: Nethaniah's son Ishmael, Kareah's son Johanan, Tanhumeth the Netophathite's son Seraiah, and Jaazaniah, who was descended from the Maacathites. \v{24}Gedaliah made this promise to them and to their men: ``Don't be afraid of the servants of the Chaldeans. Live in the land and serve the king of Babylon, and things will go well with you.'' \v{25}Nevertheless, seven months later, Nethaniah's son Ishmael, the grandson of Elishama from the royal family, came with ten men and attacked Gedaliah. As a result, he died along with the Jews and Chaldeans who were with him at Mizpah. \v{26}Then all the people, including those who were insignificant and those who were important, fled with the captains of the armed forces to Egypt, because they were afraid of the Chaldeans.
\passage{Jehoiachin Leaves Prison}

\v{27}Later on, after King Jehoiachin of Judah had been in exile for 37 years, on the twenty-seventh day of the twelfth month, during the first year of his reign, King Evil-merodach of Babylon released King Jehoiachin of Judah from prison. \v{28}He spoke kindly to him and elevated his position\fnote{\fbackref{25:28} Lit. \fbib{throne}} above the thrones of the kings with him in Babylon. \v{29}Jehoiachin changed out of his prison clothes and had regular meals in the king's presence every day for the rest of his life, \v{30}and a regular stipend was provided to him by the king in accordance with his needs for as long as he lived.

\bookheader{1 Chronicles}
\labelbook{1Chr}

\bookpretitle{The Book of}
\booktitle{First Chronicles}

\labelchapt{1}
\passage{Genealogies from Adam}
\passageinfo{(Gen 5:1-32; 10:1-32; 11:10-26; Lk 3:34-38)}

\chapt{1}
\v{1}Adam fathered\fnote{The Heb. lacks \fbib{fathered}} Seth, who fathered\fnote{The Heb. lacks \fbib{who fathered}; and so through v 4} Enosh, \v{2}who fathered Kenan, who fathered Mahalalel, who fathered Jared, \v{3}who fathered Enoch, who fathered Methuselah, who fathered Lamech, \v{4}who fathered Noah, who fathered Shem, Ham, and Japheth.

\v{5}Japheth's descendants were\fnote{The Heb. lacks \fbib{were}; and so throughout the genealogies} Gomer, Magog, Madai, Javan, Tubal, Meshech, and Tiras.

\v{6}Gomer's descendants were Ashkenaz, Diphath,\fnote{So MT LXX and Vg read \fbib{Riphath}; cf. Gen 10:3 \fbib{Riphath}} and Togarmah.

\v{7}Javan's descendants were Elishah, Tarshish, Kittim, and Rodanim.\fnote{So MT and LXX. Syr and Vg read \fbib{Dodanim}; cf. Gen 10:4 \fbib{Dodanim}}

\v{8}Ham's descendants were Cush, Mizraim, Put, and Canaan.

\v{9}Cush's descendants were Seba, Havilah, Sabta, Raama, and Sabteca.

Raamah's descendants were Sheba and Dedan.

\v{10}Cush fathered Nimrod. He became the first powerful ruler on the earth.

\v{11}Mitzraim fathered the Ludim, the Anamim, the Lehabim, the Naphtuhim, \v{12}the Pathrusim, the Casluhim (from whom the Philistines descended)\fnote{Lit. \fbib{Philistines are named}}, and the Caphtorim.

\v{13}Canaan fathered Sidon his firstborn, as well as Heth, \v{14}and the Jebusites, the Amorites, the Girgashites, \v{15}the Hivites, the Archites, the Sinites, \v{16}the Arvadites, the Zemarites, and the Hamathites.

\v{17}Shem's descendants were Elam, Asshur, Arpachshad, Lud, Aram, Uz, Hul, Gether, and Meshech.\fnote{So MT; cf. Gen 10:23 \fbib{Mash}} \v{18}Arpachshad fathered Shelah and Shelah fathered Eber. \v{19}Eber fathered two sons. The name of the one was Peleg (because the earth was divided during his lifetime) and his brother was named Joktan. \v{20}Joktan fathered Almodad, Sheleph, Hazarmaveth, Jerah, \v{21}Hadoram, Uzal, Diklah, \v{22}Ebal, Abimael, Sheba, \v{23}Ophir, Havilah, and Jobab---all of these were Joktan's descendants.

\v{24}In summary,\fnote{The Heb. lacks \fbib{In summary}} Shem fathered\fnote{The Heb. lacks \fbib{fathered}} Arpachshad, who fathered\fnote{The Heb. lacks \fbib{who fathered}; and so through v 27} Shelah, \v{25}who fathered Eber, who fathered Peleg, who fathered Reu, \v{26}who fathered Serug, who fathered Nahor, who fathered Terah, \v{27}who fathered Abram---that is, Abraham.
\passage{Genealogy of Abraham's Family}
\passageinfo{(Gen. 25:1-4, 12-16; 36:1-30)}

\v{28}Abraham's descendants were Isaac and Ishmael. \v{29}These are their genealogies: the firstborn Ishmael fathered Nebaioth, and then Kedar, Adbeel, Mibsam, \v{30}Mishma, Dumah, Massa, Hadad, Tema, \v{31}Jetur, Naphish, and Kedemah---these are the Ishmaelites.

\v{32}The descendants born to Keturah, Abraham's mistress,\fnote{Or \fbib{concubine}; i.e. a secondary wife, and so throughout the book} were Zimran, Jokshan, Medan, Midian, Ishbak, and Shuah. The descendants of Jokshan were Sheba and Dedan. \v{33}The descendants of Midian were Ephah, Epher, Hanoch, Abida, and Eldaah. All these were the descendants of Keturah.

\v{34}Abraham fathered Isaac. Isaac's descendants were Esau and Israel.
\passage{Esau's Genealogy}

\v{35}Esau's descendants were Eliphaz, Reuel, Jeush, Jalam, and Korah.

\v{36}Eliphaz's descendants were Teman, Omar, Zephi, Gatam, Kenaz, Timna, and Amalek.

\v{37}Reuel's descendants were Nahath, Zerah, Shammah, and Mizzah.

\v{38}Seir's descendants were Lotan, Shobal, Zibeon, Anah, Dishon, Ezer, and Dishan.

\v{39}Lotan's descendants were Hori and Homam. Lotan's sister was Timna.

\v{40}Shobal's descendants were Alian, Manahath, Ebal, Shephi, and Onam.

Zibeon's descendants were Aiah and Anah.

\v{41}Anah's descendant was Dishon.

Dishon's descendants were Hamran, Eshban, Ithran, and Cheran.

\v{42}Ezer's descendants were Bilhan, Zaavan, and Jaakan.\fnote{Or \fbib{Akan}; cf. Gen 36:27}

Dishan's\fnote{MT reads \fbib{Dishon}; cf. Gen 36:28} descendants were Uz and Aran.
\passage{The Kings of Edom}

\v{43}Here's a list of kings who reigned in the land of Edom before any king reigned over the Israelis, beginning with\fnote{The Heb. lacks \fbib{beginning with}} Beor's son Bela (his city was named Dinhabah). \v{44}After Bela died, Zerah's son Jobab from Bozrah succeeded him.

\v{45}After Jobab died, Husham from the land of the Temanites succeeded him.

\v{46}After Husham died, Bedad's son Hadad, who defeated Midian in the country of Moab, succeeded him. His city was named Avith.

\v{47}After Hadad died, Samlah from Masrekah succeeded him.

\v{48}After Samlah died, Shaul\fnote{Or \fbib{Saul}} from Rehoboth on the Euphrates River\fnote{The Heb. lacks \fbib{River}} succeeded him.

\v{49}After Shaul\fnote{Or \fbib{Saul}} died, Achbor's son Baal-hanan succeeded him.

\v{50}After Baal-hanan died, Hadad succeeded him. His city was named Pai, and his wife's name was Mehetabel. She was the daughter of Matred, who was the daughter of Me-zahab. \v{51}Then Hadad died.

The chiefs of Edom included the chiefs of Timna, Aliah, Jetheth, \v{52}Oholibamah, Elah, Pinon, \v{53}Kenaz, Teman, Mibzar, \v{54}Magdiel, and Iram---these are the clans of Edom.
\labelchapt{2}
\passage{Genealogies of Israel and Judah}
\passageinfo{(Gen. 29:31-30:24; 46:8-25; Ruth 4:18-22; Matt 1:2-6; Lk 3:1-33)}

\chapt{2}
\v{1}Here's a list of Israel's sons: Reuben, Simeon, Levi, Judah, Issachar, Zebulun, \v{2}Dan, Joseph, Benjamin, Naphtali, Gad, and Asher.

\v{3}Judah's three sons Er, Onan, and Shelah were born to him through Bath-shua, a Canaanite. Er, Judah's firstborn, became wicked in the \divine{Lord}'s sight, so he put him to death. \v{4}Judah's\fnote{Lit. \fbib{His}} daughter-in-law Tamar also bore him Perez and Zerah, so Judah had five sons in all.

\v{5}Perez's sons were Hezron and Hamul.

\v{6}Zerah had five sons in all: Zimri, Ethan, Heman, Calcol, and Dara.\fnote{So MT and LXX; cf. Syr, Targ, and some Gk. mss; cf. 1King 4.31}

\v{7}Carmi's son was Achar,\fnote{Cf. Josh 7:1 \fbib{Achan}} who became Israel's troublemaker by transgressing the \divine{Lord}'s commandment\fnote{The Heb. lacks \fbib{the \divine{Lord}'s commandment}} regarding things that were to be destroyed.

\v{8}Ethan's son was Azariah.

\v{9}Hezron's sons born to him were Jerahmeel, Ram, and Chelubai. \v{10}Ram fathered Amminadab, and Amminadab fathered Nahshon, who was leader of the descendants of Judah.

\v{11}Nahshon fathered Salma, Salma fathered Boaz, \v{12}Boaz fathered Obed, and Obed fathered Jesse. \v{13}Jesse fathered Eliab his firstborn, Abinadab his second born, Shimea his third born, \v{14}Nethanel his fourth born, Raddai his fifth born, \v{15}Ozem his sixth born, David his seventh born; \v{16}along with their sisters Zeruiah and Abigail.

Zeruiah's three sons were Abishai, Joab, and Asahel. \v{17}Abigail bore Amasa, whose father was Jether the Ishmaelite.

\v{18}Hezron's son Caleb had children by his wife Azubah and by Jerioth. These were her sons: Jesher, Shobab, and Ardon. \v{19}When Azubah died, Caleb married Ephrath, who bore him Hur. \v{20}Hur fathered Uri, and Uri fathered Bezalel.

\v{21}Later, Hezron married\fnote{Lit. \fbib{Hezron went in to}} the daughter of Machir, who had fathered Gilead. He married her when he was 60 years old, and she bore him Segub. \v{22}Segub fathered Jair, who had 23 towns in the land of Gilead. \v{23}But Geshur and Aram took 60 towns from Gilead,\fnote{Lit. \fbib{them}} including Havvoth-jair and Kenath, along with their villages. All these were descendants of Machir, who fathered Gilead.

\v{24}After Hezron died in Caleb-ephrathah, Abijah wife of Hezron bore him Ashhur, who fathered Tekoa.

\v{25}The descendants of Jerahmeel, the firstborn of Hezron, were Ram his firstborn, Bunah, Oren, Ozem, and Ahijah.

\v{26}Jerahmeel also had another wife, whose name was Atarah; she was the mother of Onam.

\v{27}The descendants of Ram, firstborn of Jerahmeel were Maaz, Jamin, and Eker.

\v{28}Onam's descendants were Shammai and Jada.

Shammai's descendants were Nadab and Abishur.

\v{29}Abishur's wife was named Abihail. She bore him Ahban and Molid.

\v{30}Nadab's descendants were Seled and Appaim. Seled died childless.

\v{31}Appaim's son\fnote{Lit. \fbib{sons}} was Ishi. Ishi's son\fnote{Lit. \fbib{sons}} was Sheshan. Sheshan's son\fnote{Lit. \fbib{sons}} was Ahlai.

\v{32}Shammai's brother Jada's descendants were Jether and Jonathan, but Jether died childless.

\v{33}Jonathan's descendants were Peleth and Zaza. These were the descendants of Jerahmeel.

\v{34}Now Sheshan had no sons, only daughters. However, Sheshan had an Egyptian slave named Jarha. \v{35}So Sheshan gave his daughter in marriage to his slave Jarha, and she bore him Attai.

\v{36}Attai fathered Nathan, and Nathan fathered Zabad. \v{37}Zabad fathered Ephlal, Ephlal fathered Obed, \v{38}Obed fathered Jehu, Jehu fathered Azariah, \v{39}Azariah fathered Helez, Helez fathered Eleasah, \v{40}Eleasah fathered Sismai, Sismai fathered Shallum. \v{41}Shallum fathered Jekamiah, and Jekamiah fathered Elishama.

\v{42}Jerahmeel's brother Caleb's descendants were his firstborn Mesha,\fnote{So MT; LXX reads \fbib{Maresa}} who fathered Ziph.

The descendants of Mareshah, who fathered Hebron, were as follows:\fnote{The Heb. lacks \fbib{as follows}}

\v{43}Hebron's descendants were Korah, Tappuah, Rekem, and Shema.

\v{44}Shema fathered Raham, who fathered Jorkeam.

Rekem fathered Shammai. \v{45}Shammai's descendants included\fnote{The Heb. lacks \fbib{included}; and so throughout the genealogies} Maon, who fathered Beth-zur. \v{46}Caleb's mistress Ephah also bore Haran, Moza, and Gazez.

Haran fathered Gazez. \v{47}Jahdai's descendants were Regem, Jotham, Geshan, Pelet, Ephah, and Shaaph. \v{48}Caleb's mistress Maacah bore Sheber, Tirhanah, \v{49}and Shaaph, who fathered Madmannah. Sheva fathered Machbenah and Gibe. Caleb's daughter was Achsah. \v{50}These were Caleb's descendants.

The son of Hur, the firstborn of Ephrathah, was Shobal, who fathered Kiriath-jearim, \v{51}Salma, who fathered Bethlehem, and Hareph, who fathered Beth-gader.

\v{52}Shobal, who fathered Kiriath-jearim, had other sons, including Haroeh, half of the Menuhoth. \v{53}The families of Kiriath-jearim included the Ithrites, the Puthites, the Shumathites, and the Mishraites. The Zorathites and the Eshtaolites came from them.

\v{54}Salma's descendants were Bethlehem, the Netophathites, Atroth-beth-joab, and half of the Manahathites, the Zorites.

\v{55}The families of the scribes who lived at Jabez included the Tirathites, the Shimeathites, and the Sucathites. These are the Kenites who came from Hammath, who fathered the house of Rechab.
\labelchapt{3}
\passage{Genealogy of David and Solomon}
\passageinfo{(Matt 1:6-12)}

\chapt{3}
\v{1}These are David's descendants who were born to him in Hebron: Amnon his firstborn by Ahinoam the Jezreelite, Daniel his second born by Abigail the Carmelite, \v{2}Absalom his third born by Maacah daughter of King Talmai of Geshur, Adonijah his fourth born by\fnote{Lit. \fbib{son of}} Haggith, \v{3}Shephatiah his fifth born by Abital, and Ithream his sixth born by his wife Eglah. \v{4}These six were born to him in Hebron, where he reigned for seven years and six months.

He reigned 33 years in Jerusalem. \v{5}These four children\fnote{The Heb. lacks \fbib{children}} were born to David\fnote{Lit. \fbib{him}} by Bath-shua\fnote{An alternate spelling for \fbib{Bathsheba}, wife of Uriah} daughter of Ammiel while he was living\fnote{The Heb. lacks \fbib{while he was living}} in Jerusalem: Shimea, Shobab, Nathan, and Solomon, \v{6}followed by nine more: Ibhar, Elishama, Eliphelet, \v{7}Nogah, Nepheg, Japhia, \v{8}Elishama, Eliada, and Eliphelet. \v{9}All these were David's sons, besides children born to his mistresses. Tamar was their sister.

\v{10}Solomon's descendants included Rehoboam, his son Abijah, his son Asa, his son Jehoshaphat, \v{11}his son Joram, his son Ahaziah, his son Joash, \v{12}his son Amaziah, his son Azariah, his son Jotham, \v{13}his son Ahaz, his son Hezekiah, his son Manasseh, \v{14}his son Amon, and his son Josiah.

\v{15}Josiah's descendants included Johanan his firstborn, his second born Jehoiakim, his third born Zedekiah, and his fourth born Shallum.

\v{16}Jehoiakim's descendants included his son Jeconiah, and his son Zedekiah.

\v{17}The descendants of Jeconiah, who was taken\fnote{The Heb. lacks \fbib{who was taken}} captive to Babylon\fnote{The Heb. lacks \fbib{to Babylon}}, included his son Shealtiel, \v{18}Malchiram, Pedaiah, Shenazzar, Jekamiah, Hoshama, and Nedabiah.

\v{19}Pedaiah's descendants included Zerubbabel and Shimei.

Zerubbabel's descendants included Meshullam and Hananiah, along with Shelomith their sister \v{20}and five others:\fnote{The Heb. lacks \fbib{others}} Hashubah, Ohel, Berechiah, Hasadiah, and Jushab-hesed.

\v{21}Hananiah's descendants included Pelatiah and Jeshaiah, his son\fnote{Lit. \fbib{sons}; LXX \fbib{son}; and so through v. 22} Rephaiah, his son Arnan, his son Obadiah, and his son Shecaniah.

\v{22}Shecaniah's son was Shemaiah, and the six\fnote{So MT, LXX; the name of one descendant is omitted.} sons of Shemaiah were Hattush, Igal, Bariah, Neariah, and Shaphat.

\v{23}The three sons of Neariah were Elioenai, Hizkiah, and Azrikam.

\v{24}The seven sons of Elioenai were Hodaviah, Eliashib, Pelaiah, Akkub, Johanan, Delaiah, and Anani.
\labelchapt{4}
\passage{Genealogy of Judah}

\chapt{4}
\v{1}Judah's descendants were Perez, Hezron, Carmi, Hur, and Shobal.

\v{2}Shobal's son Reaiah fathered Jahath, and Jahath fathered Ahumai and Lahad. These were the families of the Zorathites.

\v{3}These were the descendants of\fnote{The Heb. lacks \fbib{the descendants of}; MT reads \fbib{fathers}} the ancestor of Etam: Jezreel, Ishma, and Idbash; and their sister's name was Hazzelelponi.

\v{4}Penuel fathered Gedor and Ezer fathered Hushah.

These were the descendants of Hur, Ephrathah's firstborn, who fathered Bethlehem: \v{5}Tekoa's father Ashhur had two wives, Helah and Naarah. \v{6}Naarah bore him these sons: Ahuzzam, Hepher, Temeni, and Haahashtari.\fnote{Or \fbib{the Ahastarite}}

\v{7}The sons of Helah were Zereth, Izhar,\fnote{Or \fbib{Zohar}} and Ethnan.

\v{8}Koz fathered Anub, Zobebah, and the families of Harum's son Aharhel.

\v{9}Jabez enjoyed more honor than his relatives---his mother named him Jabez, she said, ``because I bore him in pain.''\fnote{The name \fbib{Jabez} is related to MT word \fbib{pain}}

\v{10}Later on, Jabez called on the God of Israel, asking him,\fnote{The Heb. lacks \fbib{him}} ``{\ldots}whether you would bless me again and again, enlarge my territory, keep your power\fnote{Lit. \fbib{hand}} with me, keep me from evil, and keep me from harm!'' And God granted what he had requested.

\v{11}Chelub, Shuhah's brother, fathered Mehir, who fathered Eshton. \v{12}Eshton fathered Beth-rapha, Paseah, and Tehinnah, who fathered Ir-nahash. These are the men of Recah.

\v{13}Kenaz's descendants were Othniel and Seraiah.

Othniel's descendants were Hathath \v{14}and Meonothai, who fathered Ophrah.

Seraiah fathered Joab, who fathered the Ge-harashim,\fnote{Lit. \fbib{Valley of the Artists}} because they became artisans.

\v{15}The descendants of Jephunneh's son Caleb were Iru, Elah, and Naam.

Elah's son\fnote{Lit. \fbib{sons}} was Kenaz.

\v{16}Jehallelel's descendants were Ziph, Ziphah, Tiria, and Asarel.

\v{17}Ezrah's descendants were Jether, Mered, Epher, and Jalon.

Mered's wife\fnote{The Heb. lacks \fbib{Mered's wife}} conceived Miriam, Shammai, and Ishbah, who fathered Eshtemoa. \v{18}Then his Judean wife bore Jered, who fathered Gedor and then Heber, who fathered Soco and Jekuthiel, who fathered Zanoah. These are the descendants of Bithiah, daughter of Pharaoh, whom Mered married.

\v{19}The descendants of Hodiah's wife, Naham's sister, fathered Keilah the Garmite and Eshtemoa the Maacathite.

\v{20}Shimon's descendants were Amnon, Rinnah, Ben-hanan, and Tilon.

Ishi's descendants were Zoheth and Ben-zoheth.

\v{21}The descendants of Judah's son Shelah were Er, who fathered Lecah, Laadah (who fathered Mareshah and the families who belonged to the guild\fnote{Lit. \fbib{house}} of linen workers at Beth-ashbea), \v{22}Jokim, the men who lived in Cozeba, Joash, and Saraph (who married Moabite families),\fnote{Lit. \fbib{married into Moab}} and Jashubi-lehem.\fnote{Or \fbib{and returned to Lehem}} (The records are ancient.)\fnote{Or \fbib{missing}} \v{23}These people\fnote{The Heb. lacks \fbib{people}} were potters who lived in Netaim and Gederah in service to their king, who lived there.
\passage{Genealogy of Simeon}
\passageinfo{(Genesis 46:10)}

\v{24}Simeon's descendants were Nemuel, Jamin, Jarib, Zerah, Shaul, \v{25}his son Shallum, his son Mibsam, and his son Mishma.

\v{26}Mishma's descendants were his son Hammuel, his son Zaccur, and his son Shimei.

\v{27}Shimei had 16 sons and six daughters, but his relatives did not have many children, nor did their entire family multiply like the Judeans did. \v{28}They lived in Beer-sheba, Moladah, Hazar-shual, \v{29}Bilhah, Ezem, Tolad, \v{30}Bethuel, Hormah, Ziklag, \v{31}Beth-marcaboth, Hazar-susim, Beth-biri, and Shaaraim. These were their cities until David began to reign.

\v{32}Their cities were Etam, Ain, Rimmon, Tochen, and Ashan, for a total of\fnote{The Heb. lacks \fbib{for a total of}} five cities, \v{33}along with all their settlements that surrounded these cities as far as Baal---this is their settlement history.\fnote{The Heb. lacks \fbib{history}}

They kept this genealogical record for themselves: \v{34}Meshobab, Jamlech, Amaziah's son Joshah, \v{35}Joel, Joshibiah's son Jehu (who was the grandson of Seraiah and great-grandson of Asiel), \v{36}Elioenai, Jaakobah, Jeshohaiah, Asaiah, Adiel, Jesimiel, Benaiah, \v{37}Shiphi's son Ziza (who was the grandson of Shiphi, who was fathered by Allon, who was fathered by Jedaiah, who was fathered by Shimri, who was fathered by Shemaiah)---\v{38}these people,\fnote{The Heb. lacks \fbib{people}} enumerated by name, were leaders in their respective families, and their clans grew to be very abundant.

\v{39}They journeyed as far as the entrance of Gedor on the east side of the valley in order to find pasture for their flocks. \v{40}They discovered abundant and excellent grazing lands there, where the land was very broad, secure, and tranquil, because the former inhabitants there were descendants of Ham. \v{41}Later on, during the reign\fnote{Lit. \fbib{days}} of King Hezekiah of Judah, these people,\fnote{The Heb. lacks \fbib{people}} enumerated by name, came and attacked both their homes\fnote{Lit. \fbib{tents}} and the Meunim who had settled there and who remain exterminated to this day. They settled down there, taking their place, because there was pasture there for their flocks. \v{42}Some of them---that is, 500 Simeonite men---went to Mount Seir.\fnote{This mountain, the modern \fbib{Jebel esh-sher\'{a}}, is located in the mountain range that extends south of the Dead Sea toward the Gulf of Aqaba, and is bordered by the Arabah Valley to the west.} Under the leadership of Ishi's sons Pelatiah, Neariah, Rephaiah, and Uzziel, \v{43}they destroyed the survivors of the Amalekites who had escaped, and they have lived there to this day.
\labelchapt{5}
\passage{Genealogy of Reuben}
\passageinfo{(Genesis 46:8-9)}

\chapt{5}
\v{1}Here is a record of\fnote{The Heb. lacks \fbib{Here is a record of}} the descendants of Reuben, Israel's firstborn. (He was the firstborn, but because he defiled his father's marriage bed, his birthright was transferred to the descendants of Israel's son Joseph. As a result, Reuben is not enrolled in the genealogy according to the birthright. \v{2}Even though Judah became prominent among his relatives---that is, the Commander-in-chief\fnote{Or \fbib{Prince}; i.e. a title of Messiah; lit. \fbib{Nagid}; i.e. a senior officer entrusted with dual roles of operational oversight and administrative authority} will be his descendant---nevertheless the right of the firstborn went to Joseph.)

\v{3}The descendants of Reuben, Israel's firstborn, included Hanoch, Pallu, Hezron, and Carmi.

\v{4}Joel's descendants were his son Shemaiah, his son Gog, his son Shimei, \v{5}his son Micah, his son Reaiah, his son Baal, \v{6}and his son Beerah, whom King Tiglath-pileser of Assyria carried away into exile, and who was a governor of the descendants of Reuben.

\v{7}His relatives, listed by families when the genealogy was enrolled according to their generations, included\fnote{The Heb. lacks \fbib{included}} the chief, Jeiel, Zechariah, \v{8}and Azaz's son Bela, grandson of Shema, and great-grandson of Joel, who lived in Aroer, near Nebo and Baal-meon. \v{9}He also lived eastward as far as the entrance to the wilderness this side of the Euphrates River,\fnote{The Heb. lacks \fbib{River}} because their cattle had increased in the territory of Gilead. \v{10}During the reign\fnote{Lit. \fbib{days}} of Saul they declared war on the Hagrites, who fell in battle by their hand. They lived in their tents throughout all of east Gilead.
\passage{Genealogy of Gad}

\v{11}Gad's descendants lived beside them in the land of Bashan as far as Salecah: \v{12}They included\fnote{The Heb. lacks \fbib{They included}} Joel their chief, Shapham their second in command,\fnote{The Heb. lacks \fbib{in command}} Janai, and Shaphat, who lived\fnote{The Heb. lacks \fbib{who lived}} in Bashan. \v{13}Their seven relatives, according to the households of their clans, included Michael, Meshullam, Sheba, Jorai, Jacan, Zia, and Eber. \v{14}These were the descendants of Huri's son Abihail, who was fathered by Jaroah, who was fathered by Gilead, who was fathered by Michael, who was fathered by Jeshishai, who was fathered by Jahdo, and who was fathered by Buz: \v{15}Abdiel's son Ahi, who was the grandson of Guni, was chief in their clan. \v{16}They lived in Gilead, in Bashan and its villages, and in all the surrounding suburbs\fnote{Or \fbib{all its pasture lands}} of Sharon as far as their borders. \v{17}All of them were enrolled by genealogies during the reign\fnote{Lit. \fbib{days}} of King Jotham of Judah and during the reign\fnote{Lit. \fbib{days}} of King Jeroboam of Israel.

\v{18}The descendants of Reuben, the descendants of Gad, and the half-tribe of Manasseh produced 44,700 valiant soldiers expert in shield, sword, and bow. Trained in warfare, they were equipped to serve at a moment's notice. \v{19}They fought in battle against the Hagrites, Jetur, Naphish, and Nodab. \v{20}When they received assistance against them, the Hagrites and all of their allies were handed over to their control, because they cried out to God during the battle. He honored their entreaty, because they had placed their trust in him. \v{21}They captured 50,000 camels, 250,000 sheep, 2,000 donkeys, and 100,000 war captives from their possessions. \v{22}Many fell slain, because the battle's outcome was directed by God. They lived in their territory\fnote{Lit. \fbib{lived in place of them}} until the exile.
\passage{Genealogy of Manasseh}

\v{23}The half-tribe of Manasseh lived in the land, spread out from Bashan to Baal-hermon, including\fnote{The Heb. lacks \fbib{including}} Senir and Mount Hermon. \v{24}These were the leaders of their clans: Epher, Ishi, Eliel, Azriel, Jeremiah, Hodaviah, and Jahdiel---they were mighty warriors, well known men, and leaders of their clans. \v{25}But they were unfaithful to the God of their ancestors by prostituting themselves to the gods of the peoples of the land, whom God had exterminated right in front of them. \v{26}So the God of Israel incited\fnote{Lit. \fbib{incited the spirit of}} King Pul of Assyria (also known as\fnote{Lit. \fbib{Assyria and the spirit of}} King Tiglath-pileser of Assyria), who took them prisoner and brought the descendants of Reuben, the descendants of Gad, and the half-tribe of Manasseh to Halah, Habor, Hara, and to the Gozan River, where they remain\fnote{The Heb. lacks \fbib{where they remain}} to this day.
\labelchapt{6}
\passage{Genealogy of Levi's Sons Kohath}
\passageinfo{(Genesis 46:11)}

\chapt{6}
\v{1}\fnote{This v. is 5:27 in MT, 16:2 is MT 5:28, and so through 6:15 (5:41 in MT)}Levi's descendants included\fnote{The Heb. lacks \fbib{included}; and so throughout the chapter} Gershom, Kohath, and Merari. \v{2}Kohath's sons included Amram, Izhar, Hebron, and Uzziel. \v{3}Amram's descendants included Aaron, Moses, and Miriam. Aaron's sons included Nadab, Abihu, Eleazar, and Ithamar. \v{4}Eleazar fathered Phinehas, Phinehas fathered Abishua, \v{5}Abishua fathered Bukki, Bukki fathered Uzzi, \v{6}Uzzi fathered Zerahiah, Zerahiah fathered Meraioth, \v{7}Meraioth fathered Amariah, Amariah fathered Ahitub, \v{8}Ahitub fathered Zadok, Zadok fathered Ahimaaz, \v{9}Ahimaaz fathered Azariah, Azariah fathered Johanan, \v{10}and Johanan fathered Azariah, who served as priest in the Temple that Solomon built in Jerusalem. \v{11}Azariah fathered Amariah, Amariah fathered Ahitub, \v{12}Ahitub fathered Zadok, Zadok fathered Shallum, \v{13}Shallum fathered Hilkiah, Hilkiah fathered Azariah, \v{14}Azariah fathered Seraiah, and Seraiah fathered Jehozadak. \v{15}The \divine{Lord} sent Jehozadak, Judah, and Jerusalem into exile, using Nebuchadnezzar to do it.\fnote{Lit. \fbib{exile by the hand of Nebuchadnezzar}}
\passage{Genealogy of Levi's Son Gershom}

\v{16}\fnote{This v. is 6:1 in MT, 6:17 is MT 6:2, and so through 6:81 (6:66 in MT)}Levi's descendants included Gershom, Kohath, and Merari. \v{17}These are the names of Gershom's descendants: Libni and Shimei. \v{18}Kohath's sons included Amram, Izhar, Hebron, and Uzziel. \v{19}Merari's sons included Mahli and Mushi.

These are the clans of the descendants of Levi according to their ancestry: \v{20}Gershom's clan included\fnote{The Heb. lacks \fbib{clan included}} his son Libni, his son Jahath, his son Zimmah, \v{21}his son Joah, his son Iddo, his son Zerah, and his son Jeatherai.

\v{22}Kohath's descendants included Amminadab, his son Korah, his son Assir, \v{23}his son Elkanah, his son Ebiasaph, his son Assir, \v{24}his son Tahath, his son Uriel, his son Uzziah, and his son Shaul.

\v{25}Elkanah's descendants included Amasai and Ahimoth, \v{26}his son Elkanah, his son Zophai, his son Nahath, \v{27}his son Eliab, his son Jeroham, and his son Elkanah.

\v{28}Samuel's descendants included Joel his firstborn and his second son\fnote{The Heb. lacks \fbib{son}} Abijah.

\v{29}Merari's descendants included Mahli, his son Libni, his son Shimei, his son Uzzah, \v{30}his son Shimea, his son Haggiah, and his son Asaiah.
\passage{David's Musicians}

\v{31}These are the men\fnote{The Heb. lacks \fbib{are the men}} to whom David handed responsibility for music in the Temple of the \divine{Lord}, after the ark came to rest there. \v{32}They ministered in song\fnote{Or \fbib{music}} in front of the Tent of Meeting, until Solomon had built the Temple of the \divine{Lord} in Jerusalem. They served in accordance with orders of service designated for them.

\v{33}These are the men who served, including their descendants: From the descendants of Kohath, there was\fnote{The Heb. lacks \fbib{there was}} Heman the singer, who had been fathered by Joel, who had been fathered by Samuel, \v{34}who had been fathered by Elkanah, who had been fathered by Jeroham, who had been fathered by Eliel, who had been fathered by Toah, \v{35}who had been fathered by Zuph, who had been fathered by Elkanah, who had been fathered by Mahath, who had been fathered by Amasai, \v{36}who had been fathered by Elkanah, who had been fathered by Joel, who had been fathered by Azariah, who had been fathered by Zephaniah, \v{37}who had been fathered by Tahath, who had been fathered by Assir, who had been fathered by Ebiasaph, who had been fathered by Korah, \v{38}who had been fathered by Izhar, who had been fathered by Kohath, who had been fathered by Levi, who had been fathered by Israel.

\v{39}There was also\fnote{The Heb. lacks \fbib{There was also}} his brother Asaph, who stood to Heman's\fnote{Lit. \fbib{his}} right. Asaph had been fathered by Berechiah, who had been fathered by Shimea, \v{40}who had been fathered by Michael, who had been fathered by Baaseiah, who had been fathered by Malchijah, \v{41}who had been fathered by Ethni, who had been fathered by Zerah, who had been fathered by Adaiah, \v{42}who had been fathered by Ethan, who had been fathered by Zimmah, who had been fathered by Shimei, \v{43}who had been fathered by Jahath, who had been fathered by Gershom, and who had been fathered by Levi.

\v{44}To Heman's\fnote{Lit. \fbib{the}} left were their relatives who were Merari's sons: Ethan, who had been fathered by Kishi, who had been fathered by Abdi, who had been fathered by Malluch, \v{45}who had been fathered by Hashabiah, who had been fathered by Amaziah, who had been fathered by Hilkiah, \v{46}who had been fathered by Amzi, who had been fathered by Bani, who had been fathered by Shemer, \v{47}who had been fathered by Mahli, who had been fathered by Mushi, who had been fathered by Merari, who had been fathered by Levi, \v{48}along with their relatives, descendants of Levi who had been appointed for all the service of the tent of the Temple of God.

\v{49}Meanwhile, Aaron and his sons presented offerings on the altar of burnt offering and on the altar of incense, carrying out the work of the Most Holy Place, making atonement for Israel in accordance with everything that Moses the servant of God had commanded.

\v{50}These are Aaron's sons: his son Eleazar, his son Phinehas, his son Abishua, \v{51}his son Bukki, his son Uzzi, his son Zerahiah, \v{52}his son Meraioth, his son Amariah, his son Ahitub, \v{53}his son Zadok, and his son Ahimaaz.
\passage{Levitical Settlements}
\passageinfo{(Joshua 21:1-42)}

\v{54}These are the settlement locations allotted within their borders to Aaron's descendants in the Kohathite clan since the lot was cast in their favor first. \v{55}Hebron in the territory of Judah was allotted to them, along with its surrounding suburbs.\fnote{Or \fbib{its pasture lands}; and so throughout the chapter} \v{56}The fields adjacent to\fnote{Lit. \fbib{fields of}} the city and its villages were allotted to Jephunneh's son Caleb. \v{57}They allotted these cities of refuge to the descendants of Aaron: Hebron, Libnah with its surrounding suburbs, Jattir, Eshtemoa with its surrounding suburbs, \v{58}Hilen with its surrounding suburbs, Debir with its surrounding suburbs, \v{59}Ashan with its surrounding suburbs, and Beth-shemesh with its surrounding suburbs. \v{60}From the tribe of Benjamin were allotted\fnote{The Heb. lacks \fbib{were allotted}; and so throughout the chapter} Geba with its surrounding suburbs, Alemeth with its surrounding suburbs, and Anathoth with its surrounding suburbs. All their towns allotted to their families totaled thirteen.

\v{61}Ten towns were allocated to the rest of the descendants of Kohath by lot out of the family of the tribe, that is, the half-tribe of Manasseh. \v{62}To the descendants of Gershom according to their families were allotted 13 towns in Bashan from the tribes of Issachar, Asher, Naphtali, and Manasseh. \v{63}The descendants of Merari were allotted 12 towns according to their families from the tribes of Reuben, Gad, and Zebulun. \v{64}So the people of Israel gave the descendants of Levi the towns with their surrounding suburbs, \v{65}allocating these towns from the tribes of Judah, Simeon, and Benjamin.

\v{66}A few of the families of Kohath's descendants had towns of their territory allotted from\fnote{Lit. \fbib{territory out of}} the tribe of Ephraim. \v{67}They were given these cities of refuge: Shechem with its surrounding suburbs in the hill country of Ephraim, Gezer with its surrounding suburbs, \v{68}Jokmeam with its surrounding suburbs, Beth-horon with its surrounding suburbs, \v{69}Aijalon with its surrounding suburbs, Gath-rimmon with its surrounding suburbs, \v{70}and (from of the half-tribe of Manasseh), Aner with its surrounding suburbs, and Bileam with its surrounding suburbs for the rest of the Kohathite families.

\v{71}From the half-tribe of Manasseh the descendants of Gershom were allotted Golan in Bashan with its surrounding suburbs and Ashtaroth with its surrounding suburbs. \v{72}From the tribe of Issachar were allotted Kedesh with its surrounding suburbs, Daberath with its surrounding suburbs, \v{73}Ramoth with its surrounding suburbs, and Anem with its surrounding suburbs. \v{74}From of the tribe of Asher were allotted Mashal with its surrounding suburbs, Abdon with its surrounding suburbs, \v{75}Hukok with its surrounding suburbs, and Rehob with its surrounding suburbs. \v{76}From the tribe of Naphtali were allotted Kedesh in Galilee with its surrounding suburbs, Hammon with its surrounding suburbs, and Kiriathaim with its surrounding suburbs. \v{77}From the tribe of Zebulun the rest of the descendants of Merari were allotted Rimmono with its surrounding suburbs, and Tabor with its surrounding suburbs, \v{78}across the Jordan from Jericho, that is, on the east side of the Jordan, from the tribe of Reuben were allotted Bezer in the steppe with its surrounding suburbs, Jahzah with its surrounding suburbs, \v{79}Kedemoth with its surrounding suburbs, and Mephaath with its surrounding suburbs. \v{80}From the tribe of Gad were allotted Ramoth in Gilead with its surrounding suburbs, Mahanaim with its surrounding suburbs, \v{81}Heshbon with its surrounding suburbs, and Jazer with its surrounding suburbs.
\labelchapt{7}
\passage{Genealogy of Issachar}
\passageinfo{(Genesis 46:13)}

\chapt{7}
\v{1}The four descendants of Issachar included\fnote{The Heb. lacks \fbib{included}; and so throughout the chapter} Tola, Puah, Jashub, and Shimron. \v{2}Tola's descendants included Uzzi, Rephaiah, Jeriel, Jahmai, Ibsam, and Shemuel, leaders of their ancestral house of Tola, who were valiant warriors during their lifetimes. During the life of David, they numbered 22,600. \v{3}Uzzi fathered Izrahiah, and Izrahiah fathered Michael, Obadiah, Joel, and Isshiah, all five of them leaders. \v{4}In addition to them, according to their ancestral records were 36,000 members of their trained army by their generations, because they had many wives and children. \v{5}As recorded in their genealogy, a total of 87,000 trained warriors belonged to all of the clans of Issachar.
\passage{Genealogy of Benjamin}
\passageinfo{(Genesis 46:21)}

\v{6}Benjamin's three descendants included Bela, Becher, and Jediael.

\v{7}Bela's five descendants included Ezbon, Uzzi, Uzziel, Jerimoth, and Iri, who were leaders of their ancestral households. Valiant warriors, their enrollment totaled 22,034 according to their genealogies.

\v{8}Becher's descendants included Zemirah, Joash, Eliezer, Elioenai, Omri, Jeremoth, Abijah, Anathoth, and Alemeth. All these were descendants of Becher, \v{9}and their genealogical enrollment totaled 20,200 valiant warriors, delineated according to their generations as leaders of their ancestral households.

\v{10}Jediael fathered Bilhan, and Bilhan's descendants included Jeush, Benjamin, Ehud, Chenaanah, Zethan, Tarshish, and Ahishahar. \v{11}All these were descendants through Jediael according to the heads of their ancestral households. Their valiant warriors totaled 17,200 equipped and ready for battle.

\v{12}In addition, Shuppim and Huppim were the sons of Ir, and the Hushites were\fnote{Or \fbib{and Hushim was}} descended from Aher.
\passage{Genealogy of Naphtali}
\passageinfo{(Genesis 46:24)}

\v{13}Naphtali's descendants included Jahziel, Guni, Jezer, and Shallum, descended through Bilhah.
\passage{Genealogy of Manasseh}

\v{14}Manasseh's descendants included Asriel, whom his Aramean mistress bore, along with Machir, who fathered Gilead. \v{15}Machir chose wives for his sons\fnote{The Heb. lacks \fbib{his sons}} Huppim and for Shuppim. He had a sister named Maacah. His second son\fnote{The Heb. lacks \fbib{son}} was named Zelophehad, and Zelophehad fathered only\fnote{The Heb. lacks \fbib{only}} daughters.\fnote{Cf. Num 26:33-27:7; 36:6-11; Josh 17:3} \v{16}Machir's wife Maacah bore a son whom she named Peresh. His brother was named Sheresh, and his sons were Ulam and Rekem. \v{17}Ulam's son was Bedan. These were the children of Machir's son Gilead, who was also a descendant of Manasseh. \v{18}His sister Hammolecheth bore Ishhod, Abiezer, and Mahlah. \v{19}Shemida's sons included Ahian, Shechem, Likhi, and Aniam.
\passage{Genealogy of Ephraim}

\v{20}Ephraim's descendants included Shuthelah, his son Bered, his son Tahath, his son Eleadah, his son Tahath, \v{21}his son Zabad, his son Shuthelah, his son Ezer, and Elead. The people of Gath, who were native to the land, killed them when\fnote{Lit. \fbib{because}} they came down to raid their cattle. \v{22}So their father Ephraim mourned many days, and his relatives came to comfort him. \v{23}Later, Ephraim had marital relations with his wife, and she conceived and gave birth to a son, whom he named Beriah,\fnote{The Heb. name \fbib{Beriah} means \fbib{in disaster}} because his household had been visited with disaster.

\v{24}His daughter Sheerah built both Lower and Upper Beth-horon, along with Uzzen-sheerah. \v{25}Rephah was also his descendant,\fnote{Lit. \fbib{son}} as were Resheph, Telah, Tahan, \v{26}Ladan, Ammihud, Elishama, \v{27}Nun, and Joshua. \v{28}Their possessions and settlements included Bethel and its towns,\fnote{Lit. \fbib{daughters}; i.e. surrounding villages, and so through v.29} Naaran to the east, Gezer and its towns to the west, Shechem and its towns as far as Ayyah and its towns \v{29}along the borders of the descendants of Manasseh, Beth-shean and its towns, Taanach and its towns, Megiddo and its towns, and Dor and its towns. In these lived the descendants of Israel's son Joseph.
\passage{Genealogy of Asher}
\passageinfo{(Genesis 46:17)}

\v{30}Asher's descendants included Imnah, Ishvah, Ishvi, Beriah, and their sister Serah.

\v{31}Beriah's descendants included Heber and Malchiel, who fathered Birzaith. \v{32}Heber fathered Japhlet, Shomer, Hotham, and their sister Shua.

\v{33}Japhlet's descendants included Pasach, Bimhal, and Ashvath. These were the descendants of Japhlet.

\v{34}Shemer's descendants included Ahi, Rohgah, Hubbah, and Aram.

\v{35}His brother Helem's descendants included Zophah, Imna, Shelesh, and Amal.

\v{36}Zophah's descendants included Suah, Harnepher, Shual, Beri, Imrah, \v{37}Bezer, Hod, Shamma, Shilshah, Ithran, and Beera.

\v{38}Jether's descendants included Jephunneh, Pispa, and Ara.

\v{39}Ulla's descendants included Arah, Hanniel, and Rizia.

\v{40}All of these were men of Asher, leaders of ancestral households, choice valiant mighty warriors, and chiefs among princes. Their enrolled genealogies for battle conscription\fnote{Or \fbib{service}} totaled 26,000 men.
\labelchapt{8}
\passage{Genealogy of Benjamin}
\passageinfo{(Genesis 46:21)}

\chapt{8}
\v{1}Benjamin fathered Bela his firstborn, Ashbel his second born, Aharah his third born, \v{2}Nohah his fourth born, and Rapha his fifth born.

\v{3}Bela's descendants included\fnote{The Heb. lacks \fbib{included}; and so throughout the chapter} Addar, Gera, Abihud, \v{4}Abishua, Naaman, Ahoah, \v{5}Gera, Shephuphan, and Huram.

\v{6}Ehud's descendants, who were leaders of their ancestral households in Geba and who were taken into exile to Manahath, included: \v{7}Naaman, Ahijah, and Gera (also known as Heglam), who fathered Uzza and Ahihud.

\v{8}Shaharaim fathered sons in the land of Moab after he had divorced\fnote{Lit. \fbib{had sent away}} his wives Hushim and Baara. \v{9}By his wife Hodesh he fathered Jobab, Zibia, Mesha, Malcam, \v{10}Jeuz, Sachia, and Mirmah. These were his sons and leaders of ancestral households.

\v{11}He also fathered his sons Abitub and Elpaal by Hushim.

\v{12}Elpaal's descendants included Eber, Misham, Shemed (who built Ono and Lod, along with its towns),\fnote{Lit. \fbib{daughters}; i.e. surrounding villages} \v{13}Beriah and Shema, leaders of ancestral households in Aijalon who put to flight the inhabitants of Gath, \v{14}Ahio, Shashak, Jeremoth, \v{15}Zebadiah, Arad, and Eder.

\v{16}Beriah's descendants included Michael, Ishpah, and Joha.

\v{17}Elpaal's descendants included Zebadiah, Meshullam, Hizki, Heber, \v{18}Ishmerai, Izliah, and Jobab.

\v{19}Shimei's descendants included Jakim, Zichri, Zabdi, \v{20}Elienai, Zillethai, Eliel, \v{21}Adaiah, Beraiah, and Shimrath.

\v{22}Shashak's descendants included Ishpan, Eber, Eliel, \v{23}Abdon, Zichri, Hanan, \v{24}Hananiah, Elam, Anthothijah, \v{25}Iphdeiah, and Penuel.

\v{26}Jeroham's descendants included Shamsherai, Shehariah, Athaliah, \v{27}Jaareshiah, Elijah, and Zichri.

\v{28}All of these were the leaders of ancestral households, chiefs according to their generations. They lived in Jerusalem.

\v{29}Jeiel the father of Gibeon lived in Gibeon, and his wife was named Maacah. \v{30}His firstborn son was Abdon, then Zur, Kish, Baal, Nadab, \v{31}Gedor, Ahio, Zecher, \v{32}and Mikloth, who fathered Shimeah. Now these also lived with their relatives across town in Jerusalem from\fnote{Or \fbib{lived opposite}; LXX reads \fbib{lived in sight of}} their other\fnote{The Heb. lacks \fbib{other}} relatives.

\v{33}Ner fathered Kish, Kish fathered Saul, Saul fathered Jonathan, Malchi-shua, Abinadab, and Esh-baal.\fnote{The Heb. name means \fbib{Man of Baal}; cf. 2Sam 2:8, where he is named \fbib{Ish-bosheth}}

\v{34}Jonathan fathered Merib-baal and Merib-baal fathered Micah.

\v{35}Micah's descendants included Pithon, Melech, Tarea, and Ahaz.

\v{36}Ahaz fathered Jehoaddah and Jehoaddah fathered Alemeth, Azmaveth, and Zimri. Zimri fathered Moza. \v{37}Moza fathered Binea, and Raphah was his son, Eleasah his son, and Azel his son.

\v{38}Azel had six sons. Their names were Azrikam, Bocheru, Ishmael, Sheariah, Obadiah, and Hanan---all of these were the sons of Azel. \v{39}The sons of his brother Eshek included Ulam his firstborn, Jeush his second, and Eliphelet his third. \v{40}Ulam's descendants were valiant warriors and archers. They had 150 children and grandchildren, all descendants of Benjamin.
\labelchapt{9}
\passage{Summary of the Genealogies}

\chapt{9}
\v{1}All of Israel was enumerated by genealogy and recorded in the Book of the Kings of Israel\fnote{An ancient chronicle of Israel, apparently now lost} as\fnote{Or \fbib{and}} Judah was being taken captive into exile to Babylon due to their disobedience.\fnote{Or \fbib{unfaithfulness}} \v{2}The first to settle on their own property in their own towns of Israel were priests, descendants of Levi, and the Temple Servants.\fnote{Heb. \fbib{Nethinim}; i.e. a division of special assistants to the descendants of Levi, originally appointed by King David; and so throughout the book; cf. Ezra 2:58; 2:70; 7:7,24; 8:17,20.}
\passage{Jerusalem after the Exile}

\v{3}In Jerusalem there lived some of the people of Judah, Benjamin, Ephraim, and Manasseh including\fnote{The Heb. lacks \fbib{including}} \v{4}Ammihud's son Uthai, who was the grandson of Omri, who was the great-grandson of Imri, who was fathered by Bani from the descendants of Judah's son Perez. \v{5}From the descendants of Shilon there was\fnote{The Heb. lacks \fbib{there was}; and so throughout the chapter} Asaiah the firstborn, along with his descendants. \v{6}From the descendants of Zerah there was Jeuel, along with 690 of their relatives. \v{7}From the descendants of Benjamin there was Meshullam's son Sallu, who was also the grandson of Hodaviah and great-grandson of Hassenuah, \v{8}Jeroham's son Ibneiah, Uzzi's son Elah, who was also Michri's grandson, and Shephatiah's son Meshullam, who was the grandson of Reuel and great-grandson of Ibnijah, \v{9}along with 956 of their relatives according to their generations. All of these were leaders of families according to their ancestral households.
\passage{Priests in Service}

\v{10}From the priests there were Jedaiah, Jehoiarib, Jachin, \v{11}and Hilkiah's son Azariah, who was fathered by Meshullam, who was fathered by Zadok, who was fathered by Meraioth, who was fathered by Ahitub, the Chief Operating Officer\fnote{Lit. \fbib{Nagid}; i.e. a senior officer entrusted with dual roles of operational oversight and administrative authority; and so throughout the chapter} of the Temple of God. \v{12}There was\fnote{The Heb. lacks \fbib{There was}} Jeroham's son Adaiah, who was fathered by Pashhur, who was fathered by Malchijah, and Adiel's son Maasai, who was fathered by Jahzerah, who was fathered by Meshullam, who was fathered by Meshillemith, who was fathered by Immer, \v{13}along with 1,760 of their relatives, who were leaders of their ancestral households, valiant and qualified to serve in the Temple of God.
\passage{Levitical Families}

\v{14}From the descendants of Levi there was Hasshub's son Shemaiah, who was the grandson of Azrikam, who was fathered by Hashabiah, from the descendants of Merari; \v{15}along with Bakbakkar, Heresh, Galal, and Mica's son Mattaniah, who was the grandson of Zichri and great-grandson of Asaph, \v{16}and Shemaiah's son Obadiah, who was the grandson of Galal, who was fathered by Jeduthun, and Asa's son Berechiah, who was the grandson of Elkanah, who lived in the villages of the Netophathites.

\v{17}The gatekeepers included\fnote{The Heb. lacks \fbib{included}; and so throughout the chapter} Shallum, Akkub, Talmon, Ahiman, and other\fnote{The Heb. lacks \fbib{other}} relatives. Shallum was the leader. \v{18}He used to be stationed in the King's Gate on the east side as one of\fnote{Lit. \fbib{side. These were}} the gatekeepers of the camp belonging to the descendants of Levi. \v{19}Kore's son Shallum, who was the grandson of Ebiasaph and the great-grandson of Korah, and the descendants of Korah (who were relatives of his ancestral house) were over the service responsibilities and served as guardians of the entrances of the Tent, just as their ancestors had been in charge of the camp of the \divine{Lord} and guardians of the entrance. \v{20}Eleazar's son Phinehas used to be Commander-in-Chief\fnote{Lit. \fbib{Nagid}; i.e. a senior officer entrusted with dual roles of operational oversight and administrative authority} over them---the \divine{Lord} was with him. \v{21}Meshelemiah's son Zechariah was gatekeeper at the entrance to the Tent of Meeting. \v{22}All these, who had been set apart as gatekeepers at the entrances, numbered 212 and had been enrolled by genealogies in their villages.

David and Samuel the seer installed them in their positions of trust, \v{23}so they and their descendants were in charge of the gates of the house of the \divine{Lord}, that is, the House of the Tent, as guardians. \v{24}The guardians were stationed on four sides---east, west, north, and south. \v{25}Their relatives who lived in their villages were required to visit every seven days to be with them in turn, \v{26}because the four senior gatekeepers (who were descendants of Levi) had been placed in charge of the chambers and the treasury of the Temple of God. \v{27}They spent the night near the Temple of God, since they had been entrusted to guard it. They were in charge of opening it every morning.

\v{28}Some were responsible for the service utensils, and they were required to take an inventory of them when they were brought in and out. \v{29}Others were responsible for the furniture and for all of the holy utensils, including the flour, wine, oil, incense, and spices. \v{30}Other descendants of the priests prepared the mixed spices. \v{31}Mattithiah, a descendant of Levi and firstborn of Shallum the Korahite, was in charge of making the offering\fnote{Or \fbib{flat}} cakes. \v{32}Some of their Kohathite relatives were responsible to prepare the rows of bread for each Sabbath. \v{33}These singers, leaders of ancestral households of the descendants of Levi, were living in the chambers of the Temple. Freed from other service responsibilities, they were on duty day and night. \v{34}These leaders of the descendants of Levi, enrolled according to their genealogies, lived in Jerusalem.
\passage{Genealogy of King Saul}

\v{35}Jeiel, who fathered Gibeon, lived in the city of\fnote{The Heb. lacks \fbib{the city of}} Gibeon. His wife was named Maacah. \v{36}His firstborn son was Abdon, followed by\fnote{Lit. \fbib{Abdon and}} Zur, Kish, Baal, Ner, Nadab, \v{37}Gedor, Ahio, Zechariah, and Mikloth. \v{38}Mikloth fathered Shimeam. They lived across town from\fnote{Or \fbib{lived opposite}} their relatives in Jerusalem. \v{39}Ner fathered Kish, Kish fathered Saul, and Saul fathered Jonathan, Malchi-shua, Abinadab, and Esh-baal. \v{40}Jonathan fathered Merib-baal, and Merib-baal fathered Micah.

\v{41}Micah's descendants included Pithon, Melech, Tahrea, and Ahaz. \v{42}Ahaz fathered Jarah, and Jarah fathered Alemeth, Azmaveth, and Zimri. Zimri fathered Moza, and \v{43}Moza fathered Binea, and Rephaiah was his son, Eleasah his son, and Azel his son. \v{44}Azel had six descendants with these names: Azrikam, Bocheru, Ishmael, Sheariah, Obadiah, and Hanan---these were the descendants of Azel.
\labelchapt{10}
\passage{The Death of Saul and His Sons}
\passageinfo{(1 Samuel 31:1-7)}

\chapt{10}
\v{1}The Philistines were fighting against Israel, and each\fnote{The Heb. lacks \fbib{each}} soldier\fnote{Lit. \fbib{a man}} of Israel fled before the Philistines. They fell slain on the mountain of Gilboa. \v{2}The Philistines followed after Saul and after his sons, and the Philistines struck down Jonathan, Abinadab, and Malchi-shua, Saul's sons. \v{3}The heaviest fighting was against Saul,\fnote{Lit. \fbib{was heavy toward}} and when the archers who were shooting located Saul, he was gravely wounded by them.\fnote{Lit. \fbib{the archers}}

\v{4}Saul ordered his armor bearer, ``Draw your sword and run me through with it, or these uncircumcised people will come and abuse me.''

But his armor bearer did not want to do it\fnote{The Heb. lacks \fbib{to do it}} because he was very frightened, so Saul took the sword and fell on it. \v{5}When his armor bearer saw that Saul was dead, he also fell on his\fnote{The Heb. lacks \fbib{his}} sword and died. \v{6}Therefore Saul, his three sons, and all his entire household died together. \v{7}When that part of the army\fnote{Lit. \fbib{man}} of Israel that was in the valley saw that the rest of the\fnote{The Heb. lacks \fbib{rest of the}} army of Israel had fled and that Saul and his sons were dead, they abandoned their cities and fled, and the Philistines came and occupied them.
\passage{The Philistines Desecrate Saul's Body}
\passageinfo{(1 Samuel 31:8-10)}

\v{8}The Philistines came to strip the dead the next day, and they found Saul dead on Gilboa mountain, along with his sons. \v{9}They stripped him, took his head and armor, and sent messengers throughout the territory of the Philistines to report the news to their idols and to the people. \v{10}Then they put Saul's\fnote{Lit. \fbib{his}} armor in the temple of their gods and fastened his skull to the wall of\fnote{The Heb. lacks \fbib{to the wall of}} the temple of Dagon.
\passage{The People of Jabesh-gilead Give Saul a Proper Burial}
\passageinfo{(1 Samuel 31:11-13)}

\v{11}When all the residents of\fnote{The Heb. lacks \fbib{the residents of}} Jabesh-gilead heard everything that the Philistines had done to Saul, \v{12}every valiant soldier\fnote{Lit. \fbib{man}} got up, removed the bodies of Saul and his sons, took them to Jabesh, and buried their bones under the tamarisk\fnote{Or \fbib{great}} tree in Jabesh. Then they fasted for seven days. \v{13}So Saul died for his transgressions; that is, he acted unfaithfully to the \divine{Lord} by transgressing the message from the \divine{Lord} (which he did not keep), by consulting a medium for advice, \v{14}and by not seeking counsel\fnote{The Heb. verb \fbib{to seek counsel} sounds like the name \fbib{Saul}} from the \divine{Lord}, who therefore put him to death and turned the kingdom over to Jesse's son David.
\labelchapt{11}
\passage{David is Anointed King}
\passageinfo{(2 Samuel 5:1-10)}

\chapt{11}
\v{1}Later on, all of Israel gathered together at Hebron in order to tell David, ``Look, we're your own flesh and blood!\fnote{Lit. \fbib{bone}} \v{2}Even back when Saul was ruling as king, you kept on leading the army of Israel out to battle\fnote{The Heb. lacks \fbib{out to battle}} and bringing them in again.\fnote{The Heb. lacks \fbib{in again}} The \divine{Lord} your God told you, `You yourself will shepherd my people Israel and will be Commander-in-Chief\fnote{Lit. \fbib{Nagid}; i.e. a senior officer entrusted with dual roles of operational oversight and management authority} over my people Israel.'\,'' \v{3}So all the elders of Israel approached the king at Hebron, where David entered into a covenant in\fnote{Lit. \fbib{covenant---that is, at Hebron---in}} the presence of the \divine{Lord}. Then they anointed David to be king over Israel, just as the \divine{Lord} had sent word through\fnote{Lit. \fbib{word by the hand of}} Samuel.
\passage{David Captures Jerusalem}

\v{4}Later, David and all of Israel marched to Jerusalem (then known as Jebus, where the Jebusites lived when they inhabited the land). \v{5}The inhabitants of Jebus told David, ``You're not coming in here!'' Nevertheless, David captured the fortress of Zion, now known as the City of David.

\v{6}David had announced, ``Whoever first attacks the Jebusites will be appointed chief and commander.'' When Zeruiah's son Joab went up first, he became chief. \v{7}David occupied\fnote{Or \fbib{lived in}} the fortress, so it was named the City of David after him. \v{8}He built up the walls surrounding the city in a complete circle from the terrace ramparts,\fnote{Lit. \fbib{the Millo}, fortified areas of ancient Jerusalem with terraces and retaining walls} and Joab repaired the rest of the city. \v{9}David became more and more prestigious because the \divine{Lord} of the Heavenly Armies was with him.
\passage{David's Elite Soldiers}
\passageinfo{(2 Samuel 23:8-17)}

\v{10}These are the leaders of the elite warriors who were strong supporters of David in his kingdom, along with all of Israel, in keeping with the message from the \divine{Lord} concerning Israel. \v{11}This record of the warriors who were for David included\fnote{The Heb. lacks \fbib{included}} Hachmoni's son Jashobeam,\fnote{Or \fbib{Jashobeam son of a Hachmonite}; cf. 2Sam 23:8, where this individual is named \fbib{Josheb-basshebeth the Tahkemonite}} leader of the platoons,\fnote{Lit. \fbib{thirties}; i.e. a military unit roughly analogous to two or more squads; and so throughout the chapter; or a group of distinguished officers who served David; cf. 2Sam 23:8} who killed 300 with his spear in a single encounter.

\v{12}Next to him among the Three Warriors\fnote{Lit. \fbib{the three valiant ones}; i.e. a group of three distinguished officers who served David, and so throughout the chapter; cf. 2Sam 23:8} was Dodo\fnote{Cf. 2Sam 23:9, where this individual is named \fbib{Dodai}} the Ahohite's son Eleazar. \v{13}He was with David at Pas-dammim when the Philistines were there to engage them in battle. There was a field planted with barley, and the army had run away from the Philistines, \v{14}but they took a defensive stand in the middle of the field and killed the Philistines while the \divine{Lord} saved them by means of a great victory.\fnote{Or \fbib{deliverance}}

\v{15}Later, the Three Warriors went down to David's hideout\fnote{Lit. \fbib{rock}} at the cave of Adullam when the Philistine army was camping in the valley of giants.\fnote{Or \fbib{the Rephaim Valley}} \v{16}David was living in that stronghold at the time, while a Philistine garrison was then at Bethlehem. \v{17}David expressed a longing, ``Oh, how I wish someone would get me a drink of water from the Bethlehem well that's by the city gate!'' \v{18}So the Three Warriors broke through the Philistine ranks, drew some water from the Bethlehem well that was next to the city gate, and brought it back to David. But David refused to drink it, poured it out in the \divine{Lord}'s presence, and \v{19}said in response, ``May God forbid me to do this! I won't drink the blood of these men, will I? After all, they risked their lives to bring it to me.''\fnote{The Heb. lacks \fbib{to me}} That's why he wouldn't drink it. The Three Warriors did these things.
\passage{David's Other Valiant Soldiers}
\passageinfo{(2 Samuel 23:18-39)}

\v{20}Joab's brother Abishai was the lieutenant\fnote{Lit. \fbib{chief}} in charge of the platoons. He used his spear to fight and kill 300 men, gaining a reputation distinct from the Three. \v{21}He was more well-known than the Three,\fnote{So MT; the Syr reads \fbib{thirty}} but he never attained the stature of the Three.

\v{22}Jehoiada's son Benaiah, who was a valiant man, accomplished great things. He was from Kabzeel. He killed two men named\fnote{The Heb. lacks \fbib{men named}} Ariel from Moab\fnote{The Heb. name \fbib{Ariel} means \fbib{lion}} and then he also went down into a pit and struck down a lion during a snow storm one day. \v{23}He also killed a soldier\fnote{Lit. \fbib{man}} from Egypt of enormous height---five cubits\fnote{I.e. about seven and a half feet; a cubit was about eighteen inches} tall. The Egyptian carried a spear comparable in size to a weaver's beam, but Benaiah attacked him with a staff, snatched the spear out of the Egyptian's hand and killed him with his own spear. \v{24}Benaiah did things like this and gained a reputation comparable to the Three Warriors. \v{25}He was well known among the platoons, but he didn't measure up to\fnote{Or \fbib{never attained the stature of}} the Three Warriors. David placed him in charge of his security detail.

\v{26}The elite forces included Asahel (Joab's brother), Dodo's son Elhanan from Bethlehem, \v{27}Shammoth from Haror,\fnote{Or \fbib{Shammoth from Haror}; also cf. 2Sam 23:25, where he is named \fbib{Shammah from Harod}} Helez the Pelonite,\fnote{Cf. 2Sam 23:26, where he is named \fbib{Helez the Paltite}} \v{28}Ikkesh's son Ira from Tekoa, Abiezer from Anathoth, \v{29}Sibbecai the Hushathite, Ilai the Ahohite, \v{30}Maharai from Netophah, Baanah's son Heled from Netophah, \v{31}Ribai's son Ithai from Gibeah, controlled by\fnote{The Heb. lacks \fbib{controlled by}} the descendants of Benjamin, Benaiah of Pirathon, \v{32}Hurai from the wadis\fnote{I.e. seasonal streams or rivers that channel water during rain seasons but are dry at other times} of Gaash, Abiel the Arbathite, \v{33}Azmaveth from Baharum, Eliahba from Shaalbon, \v{34}Hashem the Gizonite, Shagee the Hararite's son Jonathan, \v{35}Sachar the Hararite's son Ahiam, Ur's son Eliphal, \v{36}Hepher the Mecherathite, Ahijah the Pelonite, \v{37}Hezro from Carmel, Ezbai's son Naarai, \v{38}Joel (Nathan's brother), Hagri's son Mibhar, \v{39}Zelek the Ammonite, Naharai from Beeroth, who was the armor-bearer for Zeruiah's son Joab, \v{40}Ira the Ithrite, Gareb the Ithrite, \v{41}Uriah the Hittite, Ahlai's son Zabad, \v{42}Shiza the Reubenite's son Adina, a leader of the descendants of Reuben, along with thirty others with him, \v{43}Maacah's son Hanan, Joshaphat the Mithnite, \v{44}Uzzia the Ashterathite, Hotham the Aroerite's sons Shama and Jeiel, \v{45}Shimri's son Jediael and his brother Joha the Tizite, \v{46}Eliel the Mahavite, Elnaam's sons Jeribai and Joshaviah, Ithmah the Moabite, \v{47}Eliel, Obed, and Jaasiel the Mezobaite.
\labelchapt{12}
\passage{David's Time in the Wilderness}
\passageinfo{(1 Samuel 22:1-2)}

\chapt{12}
\v{1}Here's a list of those who came to David at Ziklag when he was unable to travel freely due to Saul son of Kish. They were among the elite soldiers who assisted him in battle. \v{2}Equipped as archers, they could use both their right and left hands to shoot arrows and hurl stones. As descendants of Benjamin, they were Saul's relatives. \v{3}Their leaders were Shemaah's sons Ahiezer and Joash from Gibeah, Azmaveth's sons Jeziel and Pelet, Beracah, Jehu from Anathoth, \v{4}Ishmaiah from Gibeon (who was one of the elite among the Thirty and in charge over them),\fnote{Lit. \fbib{over the Thirty}} Jeremiah,\fnote{The remainder of this v. is 12:5 in MT} Jahaziel, Johanan, Jozabad from Gederah, \v{5}\fnote{This v. is 12:6 in MT, and so throughout the chapter}Eluzai, Jerimoth, Bealiah, Shemariah, Shephatiah the Haruphite, \v{6}Elkanah, Isshiah, Azarel, Joezer, Jashobeam, the descendants of Korah, \v{7}and Jeroham's sons Joelah and Zebadiah from Gedor.

\v{8}Mighty and experienced warriors from the descendants of Gad joined David at his wilderness stronghold. They were expert handlers of both shield and spear, with hardened looks\fnote{Lit. \fbib{with faces like those of lions}} and as agile\fnote{Or \fbib{swift}} as a gazelle on a mountain slope. \v{9}Their leader was Ezer, Obadiah was second, Eliab third, \v{10}Mishmannah fourth, Jeremiah fifth, \v{11}Attai sixth, Eliel seventh, \v{12}Johanan eighth, Elzabad ninth, \v{13}Jeremiah tenth, and Machbannai eleventh. \v{14}These descendants of Gad were army leaders. The least of them\fnote{Lit. \fbib{One of their number}} was equal to a hundred other soldiers\fnote{The Heb. lacks \fbib{other soldiers}} and the greatest to a thousand. \v{15}These men\fnote{Lit. \fbib{These are they who}} crossed the Jordan in the first month of the year\fnote{The Heb. lacks \fbib{of the year}} during flood season and chased out everyone in the valleys, to the east and to the west.

\v{16}Later, some descendants of Benjamin and Judah approached David at his stronghold, \v{17}and David went out to meet them. He told them, ``If you've come in peace to be of help to me, then you'll have my commitment.\fnote{Lit. \fbib{then my heart will be knit to you}} But if you've come to betray me to my enemies, even though I'm innocent of wrongdoing, then may the God of our ancestors watch and judge.''

\v{18}Then the Spirit came upon Amasai, leader of the Thirty, and he said,

\begin{poetry}
\poeml ``David, we belong to you; \\
\poemll    we're with you, son of Jesse! \\
\poeml Peace, peace to you, \\
\poemll    and peace to the one who helps you! \\
\poemlll       For your deliverer is your God.''
\end{poetry}

So David received them and assigned them to be officers over troops. \v{19}Some of the descendants of Manasseh joined\fnote{Lit. \fbib{fell}} David when he was going to fight against Saul, accompanied by the Philistines. Even so, David was of no help to them, because the Philistine rulers were counseled to send him away. They told themselves, ``He's going to go over to his master Saul at the cost of our heads.''

\v{20}As he traveled toward Ziklag, these descendants of Manasseh joined\fnote{Lit. \fbib{fell}} him: Adnah, Jozabad, Jediael, Michael, Jozabad, Elihu, and Zillethai, leaders in charge thousands in Manasseh. \v{21}They helped David against raiders, since they were all warriors and commanders in the army. \v{22}Indeed people kept coming to David every day to help him, until his army became a great, vast army.\fnote{Lit. \fbib{great, like an army of God}}
\passage{David's Army at Hebron}

\v{23}What follows is a listing of the divisions of battle-ready troops who joined David in Hebron to turn the kingdom of Saul over to him, in accordance with what the \divine{Lord} had spoken. \v{24}The army of Judah, equipped with both shields and spears, numbered 6,800 warriors, \v{25}the elite warriors of Simeon numbered 7,100, \v{26}and the descendants of Levi numbered 4,600.

\v{27}Jehoiada, a senior officer\fnote{Lit. \fbib{Nagid}; i.e. a senior officer entrusted with dual roles of operational oversight and administrative authority} in the house of Aaron, brought\fnote{The Heb. lacks \fbib{brought}; and so throughout the chapter} with him 3,700. \v{28}Zadok, a young and valiant soldier, brought 22 commanders from his own ancestral house.

\v{29}The tribe of\fnote{The Heb. lacks \fbib{The tribe of}; and so throughout the chapter} Benjamin, relatives of Saul numbered 3,000, of whom most had remained allied to what remained of\fnote{The Heb. lacks \fbib{what remained of}} Saul's dynasty.

\v{30}The tribe of Ephraim supplied\fnote{The Heb. lacks \fbib{supplied}; and so throughout the chapter} 20,800 valiant soldiers who were well known in their ancestral households.

\v{31}The half-tribe of Manasseh supplied 18,000, who had been appointed specifically to come and establish David as king.

\v{32}The tribe of Issachar supplied 200 leaders, along with all of their relatives under their command. They kept up-to-date in their understanding of the times and knew what Israel should do.

\v{33}The tribe of Zebulun supplied 50,000 experienced troops, trained in the use of every kind of war weapon, in order to help David\fnote{So LXX. The Heb. lacks \fbib{David}} with undivided loyalty.

\v{34}The tribe of Naphtali supplied 1,000 commanders, accompanied by 37,000 troops armed with shields and spears.

\v{35}The tribe of Dan supplied 28,600 battle-ready troops.

\v{36}The tribe of Asher supplied 40,000 experienced, battle-ready troops.

\v{37}The tribes of Reuben and Gad, along with the half-tribe of Manasseh east of\fnote{Lit. \fbib{Manasseh beyond}} the Jordan supplied 120,000 men armed with every kind of war weapon.

\v{38}All these warriors arrived in battle order at Hebron, fully intending to establish David as king over all Israel. Furthermore, all of the rest of Israel were united in their intent to make David king. \v{39}They spent three days eating and drinking with David, since their relatives had supplied provisions for them.

\v{40}Their neighbors came from as far away as the territories of Issachar, Zebulun, and Naphtali, bringing provisions loaded on donkeys, camels, mules, and oxen. They brought\fnote{The Heb. lacks \fbib{They brought}} abundant provisions of meal, fig bars, raisins, wine, oil, oxen, and sheep, because there was joy in Israel.
\labelchapt{13}
\passage{The Ark is Moved from Kiriath-jearim}
\passageinfo{(2 Samuel 6:1-11)}

\chapt{13}
\v{1}Later, David conferred with every officer\fnote{Lit. \fbib{Nagid}; i.e. a senior officer entrusted with dual roles of operational oversight and administrative authority} in charge of groups of thousands and groups of\fnote{The Heb. lacks \fbib{groups of}} hundreds. \v{2}Then he\fnote{Lit. \fbib{David}} addressed the entire community of Israel, ``If it seems good to you and something from the Lord our God, let's spread word to all of our relatives who remain throughout the entire land of Israel, including the priests and descendants of Levi in the cities and pasture lands, so they can gather together with us. \v{3}Then let's bring the Ark of God back to us, because we didn't consult it during Saul's reign.''\fnote{Lit. \fbib{days}} \v{4}The entire community consented, because doing so pleased all the people. \v{5}So David assembled all of Israel---from the Shihor River of Egypt to Lebo-hamath---in order to bring the Ark of God from Kiriath-jearim.

\v{6}David, accompanied by all of Israel, went up to Baalah (the former name of Kiriath-jearim), which belonged to Judah, to bring from there the Ark of God, the \divine{Lord}, who sits enthroned on the cherubim, and who is called the Name.\fnote{The Heb. lacks \fbib{the}} \v{7}They mounted the Ark of God on a new cart, bringing it from Abinadab's home, with Uzzah and Ahio driving the cart. \v{8}David and all of Israel were dancing in the presence of God with all of their\fnote{The Heb. lacks \fbib{their}} might with songs,\fnote{Cf. 2Sam 6:5, where MT letters of the word \fbib{song} may be transposed as MT word \fbib{cypress}} harps, tambourines, cymbals, and trumpets. \v{9}As they approached Chidon's threshing floor, Uzzah put out his hand to steady the ark, because the oxen had stumbled. \v{10}Just then, the anger of the \divine{Lord} blazed against Uzzah, and he struck him down because he had put his hand on the ark, and he died right there in the presence of God.

\v{11}David flew into a rage because the \divine{Lord} had killed\fnote{Or \fbib{had burst out against}} Uzzah. As a result, that place was called Perez-uzzah\fnote{The Heb. name \fbib{Perez-uzzah} means \fbib{Overwhelming Uzzah}; cf. 2Sam 5:20, 6:8} to this day. \v{12}But David feared God that day, and asked ``How am I to bring the Ark of God to me?'' \v{13}As a result, David would not take the ark into the City of David for it to be in his care. Instead, he took it to the home of Obed-edom the Gittite. \v{14}So the Ark of God remained in the care of Obed-edom's household for three months, and God blessed Obed-edom's household, along with everyone associated with it.
\labelchapt{14}
\passage{David Settles in Jerusalem}
\passageinfo{(2 Samuel 5:11-16)}

\chapt{14}
\v{1}After this, King Hiram of Tyre sent a delegation to David, accompanied by cedar\fnote{I.e. a genus of coniferous evergreen in the family \fbib{Pinaceae}; and so throughout the book} logs, stone masons, and wood workers, to construct a palace for him. \v{2}David realized that the \divine{Lord} was affirming him as king over Israel, and that his government was being exalted in order to benefit his people Israel. \v{3}But while he was living in Jerusalem, David married more wives and fathered more sons and daughters. \v{4}Here's a list of the children whom he fathered while in Jerusalem: Shammua, Shobab, Nathan, Solomon, \v{5}Ibhar, Elishua, Elpelet, \v{6}Nogah, Nepheg, Japhia, \v{7}Elishama, Beeliada, and Eliphelet.
\passage{David Defeats the Philistines}
\passageinfo{(2 Samuel 5:17-25)}

\v{8}When the Philistines learned that David had been anointed king over all of Israel, all of the Philistines invaded to look for David. David heard about it and went out to fight them. \v{9}Meanwhile, the Philistines had invaded and raided the Rephaim Valley. \v{10}So David asked God, ``Am I to go out against the Philistines? Will you give me victory over them?''\fnote{Lit. \fbib{give them into my hand}}

``Go out,'' the \divine{Lord} replied to him, ``and I'll put them right into your hand.''

\v{11}So David\fnote{Lit. \fbib{he}} went out to Baal-perazim and defeated the Philistines\fnote{Lit. \fbib{defeated them}} there. David observed, ``Like an overwhelming flood, God has overwhelmed\fnote{Or \fbib{has burst out against}} my enemies, using me to do it.''\fnote{Lit. \fbib{using my own hand}} That's why that place is called Baal-perazim.\fnote{The Heb. name \fbib{Baal-perazim} means \fbib{Lord of Overwhelming}} \v{12}The Philistines\fnote{Lit. \fbib{They}} abandoned their gods there, so David ordered that their idols be incinerated.

\v{13}Later the Philistines invaded the Rephaim\fnote{The Heb. lacks \fbib{Rephaim}} Valley again. \v{14}When David asked God about it, God told him, ``Don't directly attack them. Instead, go around them and come up against them opposite those balsam trees. \v{15}When you hear the sound of marching coming from the tops of the balsam trees, then go out to battle, because God will have gone out ahead of you to destroy the Philistine army.'' \v{16}So David did just as God had ordered, and they struck down the Philistine army from Gibeon to Gezer. \v{17}Then David's reputation spread through all of the neighboring countries,\fnote{Lit. \fbib{the lands}} and the \divine{Lord} caused all nations\fnote{Or \fbib{gentiles}} to be afraid of David.
\labelchapt{15}
\passage{A Place for the Ark is Prepared}
\passageinfo{(2 Samuel 6:12-16)}

\chapt{15}
\v{1}David built palaces for himself in the City of David, and he prepared a place for the Ark of God and erected a tent for it. \v{2}Then David ordered that the Ark of God was to be carried by no one except the descendants of Levi, because the \divine{Lord} had chosen them to carry the ark of the \divine{Lord} and to serve him forever. \v{3}David assembled all of Israel in Jerusalem to bring up the ark of the \divine{Lord} to its proper place that he had prepared for it.
\passage{Ministry Appointments}

\v{4}David also assembled the descendants of Aaron, who were descendants of Levi, \v{5}including\fnote{The Heb. lacks \fbib{including}} Uriel their leader from the descendants of Kohath, along with 120 of his relatives, \v{6}from the descendants of Merari, Asaiah their leader, along with 220 of his relatives, \v{7}from the descendants of Gershom, Joel their chief, along with 130 of his relatives, \v{8}from the descendants of Elizaphan, Shemaiah their leader, along with 200 of his relatives, \v{9}from Hebron's descendants, Eliel their leader, along with 80 of his relatives, \v{10}and from Uzziel's descendants, Amminadab their leader, along with 112 of his relatives.

\v{11}Then David summoned the priests Zadok and Abiathar, along with the descendants of Levi Uriel, Asaiah, Joel, Shemaiah, Eliel, and Amminadab \v{12}and addressed them: ``As leaders of your Levitical families, set yourselves apart, both you and your relatives, so you can be qualified to\fnote{The Heb. lacks \fbib{be qualified to}} bring up the ark of the \divine{Lord} God of Israel to the place I've prepared for it. \v{13}Because you didn't carry it from the very first, the \divine{Lord} our God attacked\fnote{Lit. \fbib{overwhelmed}} us, since we didn't care for it appropriately.'' \v{14}So the priests and descendants of Levi set themselves apart to carry the ark of the \divine{Lord} God of Israel. \v{15}The descendants of Levi carried the Ark of God the way Moses had commanded and in accordance with the command from\fnote{Lit. \fbib{the word of}} the \divine{Lord}---that is, with poles\fnote{Lit. \fbib{yolk bars}} on their shoulders.
\passage{Music Ministry Appointments}

\v{16}David also told the leaders of the descendants of Levi to appoint their relatives as singers, to play musical instruments such as harps, lyres, and cymbals, and to keep sounding aloud with joyful voices. \v{17}So the descendants of Levi appointed Joel's son Heman, his relative Berechiah's son Asaph, as well as certain\fnote{The Heb. lacks \fbib{certain}} relatives of Merari's sons, including\fnote{The Heb. lacks \fbib{including}} Kushaiah's son Ethan, \v{18}their second order relatives\fnote{Lit. \fbib{their second relatives}; i.e. a supplementary ministry team} Zechariah, Jaaziel, Shemiramoth, Jehiel, Unni, Eliab, Benaiah, Maaseiah, Mattithiah, Eliphelehu, and Mikneiah, as well as the trustees\fnote{Or \fbib{gatekeepers}} Obed-edom and Jeiel. \v{19}The singers included Heman, Asaph, and Ethan (who played bronze cymbals). \v{20}Zechariah, Aziel, Shemiramoth, Jehiel, Unni, Eliab, Maaseiah, and Benaiah played harps to accompany the women singers,\fnote{Lit. \fbib{harps according to Alamoth}; i.e. \fbib{harps according to young women}} \v{21}and Mattithiah, Eliphelehu, Mikneiah, Obed-edom, Jeiel, and Azaziah led on lyres, sounding the octaves.\fnote{Lit. \fbib{lyres according to Sheminith to lead}} \v{22}Chenaniah, music leader for the descendants of Levi, served as music director, because he was expert at it. \v{23}Berechiah and Elkanah served as gatekeepers for the ark. \v{24}Shebaniah, Joshaphat, Nethanel, Amasai, Zechariah, Benaiah, and Eliezer the priests were appointed to sound the trumpets before the Ark of God, and Obed-edom and Jehiah were trustees\fnote{Or \fbib{gatekeepers}} for the ark.
\passage{The Ark is Moved to Jerusalem}

\v{25}Then David, the elders of Israel, and the leaders of groups of thousands\fnote{Lit. \fbib{the Elefim}, a community leader representing 1,000 Israelis} proceeded to bring the Ark of the Covenant of the \divine{Lord} from Obed-edom's house, rejoicing as they went.\fnote{The Heb. lacks \fbib{as they went}} \v{26}As God helped the descendants of Levi who were carrying the Ark of the Covenant of the \divine{Lord}, they sacrificed seven bulls and seven rams. \v{27}David wore a robe made from fine linen, as did all of the descendants of Levi who were carrying the ark, the singers, and Chenaniah the music and choir director. David also wore a linen ephod. \v{28}All of Israel were bringing up the Ark of the Covenant of the \divine{Lord}, accompanied by shouting, sounding of horns, trumpets, and cymbals, along with loud music on harps and lyres. \v{29}But as the Ark of the Covenant of the \divine{Lord} approached the City of David, Saul's daughter Michal was peering out a window, watching King David dancing and cavorting around, and she despised him in her heart.
\labelchapt{16}
\passage{The Ark is Placed in the Tent}
\passageinfo{(2 Samuel 6:17-19)}

\chapt{16}
\v{1}They brought the Ark of God, placed it within the tent that David had erected, and offered burnt offerings and peace offerings in the presence of God. \v{2}After David had finished sacrificing the burnt offerings and peace offerings, he blessed the people in the name of the \divine{Lord} \v{3}and distributed a loaf of bread, a date bar, and a raisin bar to every person in Israel---that is, to each man and to each woman. \v{4}In the presence of the ark of the \divine{Lord}, he appointed some of the descendants of Levi to minister continually by remembering,\fnote{Lit. \fbib{invoking}; i.e. to speak to God in light of his past works} giving thanks, and praising the \divine{Lord} God of Israel. \v{5}Their director Asaph played cymbals, and next to him was Zechariah, followed by Jeiel, Shemiramoth, Jehiel, Mattithiah, Eliab, Benaiah, Obed-edom, and Jeiel, who played harps and lyres. \v{6}The priests Benaiah and Jahaziel played the trumpets continually in the presence of the Ark of the Covenant of God.
\passage{David's Psalm of Thanksgiving}
\passageinfo{(Psalm 96:1-13; 105:1-15; 106:1,47-48)}

\v{7}On that very day, David composed this psalm of thanksgiving to the \divine{Lord} just for\fnote{Lit. \fbib{\divine{Lord} in the hand of}} Asaph and his companions:\fnote{Lit. \fbib{brothers}; i.e. his fellow descendants of Levi}

\begin{poetry}
\poeml \v{8}Give thanks to the \divine{Lord}, \\
\poemll    calling on his name. \\
\poemlll       Make what he has done known among the people. \\
\poeml \v{9}Sing to him, \\
\poemll    sing psalms to him, \\
\poemlll       and think\fnote{Or \fbib{and talk}} about all of his miraculous deeds. \\
\poeml \v{10}Find joy in his holy name; \\
\poemll    let the hearts of those who keep on seeking the \divine{Lord} rejoice. \\
\poeml \v{11}Seek the \divine{Lord} and his strength. \\
\poemll    Always look to him.\fnote{Lit. \fbib{to his face}} \\
\poeml \v{12}Keep remembering the awesome deeds that he has done, \\
\poemll    along with his miracles \\
\poemlll       and the rulings that he has handed down, \\
\poeml \v{13}you descendants of his servant Israel, \\
\poemll    you descendants of Jacob, \\
\poemlll       the ones he has chosen. \\
\poeml \v{14}He is the \divine{Lord} our God. \\
\poemll    His justice is in all of the land. \\
\poeml \v{15}Remember his covenant forever, \\
\poemll    his promise that he made to the thousandth generation, \\
\poeml \v{16}the covenant\fnote{The Heb. lacks \fbib{the covenant}} that he made with Abraham, \\
\poemll    and the oath he swore to Isaac. \\
\poeml \v{17}He confirmed it to Jacob in the form of an ordinance, \\
\poemll    an eternal covenant to Israel, \\
\poeml \v{18}when he told Israel, \\
\poemll    ``To you I will give the land of Canaan \\
\poemlll       as your joyful inheritance.''\fnote{Or \fbib{your special portion}} \\
\poeml \v{19}When you were few in number--- \\
\poemll    very few, and strangers at that--- \\
\poeml \v{20}wandering from nation to nation, \\
\poemll    from one kingdom to another, \\
\poeml \v{21}he did not let anyone wrong them. \\
\poemll    He warned kings on their behalf, \\
\poeml \v{22}``Don't touch my chosen ones, \\
\poemll    and don't hurt my prophets!'' \\
\poeml \v{23}Let all the earth sing to the \divine{Lord}! \\
\poemll    Day after day proclaim his deliverance!\fnote{Or \fbib{day preach his salvation}} \\
\poeml \v{24}Declare his glory among the nations, \\
\poemll    and his miraculous deeds to all people, \\
\poeml \v{25}because the \divine{Lord} is great, \\
\poemll    and he is praised greatly! \\
\poemlll       He is feared above every god. \\
\poeml \v{26}For all of the gods of the other\fnote{The Heb. lacks \fbib{other}} nations are mere\fnote{The Heb. lacks \fbib{mere}} idols, \\
\poemll    but the \divine{Lord} fashioned the heavens! \\
\poeml \v{27}Splendor and majesty surround him, \\
\poemll    and strength and joy fill his palace.\fnote{Lit. \fbib{place}} \\
\poeml \v{28}Let the families of earth recognize the \divine{Lord}--- \\
\poemll    that he is glorious and powerful. \\
\poeml \v{29}Recognize the glory that is due the \divine{Lord}! \\
\poemll    Bring your offering, \\
\poeml and come into his presence, \\
\poemll    worshiping the \divine{Lord} in all of his holy splendor. \\
\poeml \v{30}Tremble in his presence, all the earth! \\
\poemll    Surely the inhabited world\fnote{Or \fbib{the inhabitants of the world}} stands firm--- \\
\poemlll       it cannot be moved. \\
\poeml \v{31}Let the heavens rejoice, \\
\poemll    and the earth be glad! \\
\poeml Say to the nations, \\
\poemll    ``The \divine{Lord} reigns!'' \\
\poeml \v{32}Let the sea roar \\
\poemll    along with everything that fills it! \\
\poeml Let the fields exult, \\
\poemll    along with everything in them! \\
\poeml \v{33}Then let the trees in the forest sing out in praise, \\
\poemll    for the \divine{Lord} is coming to judge the world. \\
\poeml \v{34}Give thanks to the \divine{Lord}, \\
\poemll    because he is good \\
\poemlll       and because his gracious love is eternal! \\
\poeml \v{35}Call out,\fnote{Lit. \fbib{Say}} ``Save us, God, you who delivers us! \\
\poemll    Gather us and rescue us from the nations! \\
\poeml We will thank your holy name \\
\poemll    and rejoice as we praise you! \\
\poeml \v{36}Praise the \divine{Lord} God of Israel, \\
\poemll    who lives from eternity to eternity!
\end{poetry}

Then all of the people shouted ``Amen!'' and praised the \divine{Lord}.
\passage{David's Establishes Regular Worship}

\v{37}Later David\fnote{Lit. \fbib{he}} left the presence of the Ark of the Covenant of the \divine{Lord} so Asaph and his fellow descendants of Levi could serve the ark there continually each day, doing whatever was required. \v{38}Obed-edom and 68 of his relatives remained also, with Jeduthun's son Obed-edom and Hosah serving as trustees.\fnote{Or \fbib{gatekeepers}} \v{39}He left Zadok the priest and his relatives at the Tent of the \divine{Lord} at the high place in Gibeon, where they ministered in the \divine{Lord}'s presence, \v{40}sacrificing the regular burnt offerings regularly each morning and evening to the \divine{Lord} on the altar dedicated to that purpose, doing everything written in the Law of the \divine{Lord}, just as he had commanded Israel.

\v{41}David\fnote{Lit. \fbib{He}} also appointed Heman, Jeduthun, and others chosen by name to give thanks to the \divine{Lord}, because ``his gracious love is eternal.''\fnote{Cf. v.34} \v{42}They accompanied their songs of praise to God with trumpets, cymbals, and other musical instruments while Jeduthun's children served as trustees.\fnote{Or \fbib{gatekeepers}} \v{43}After this, everyone left for their own homes and David went home to bless his own household.
\labelchapt{17}
\passage{God Establishes His Covenant with David}
\passageinfo{(2 Samuel 7:1-17)}

\chapt{17}
\v{1}After David had settled down to live in his palace, he\fnote{Lit. \fbib{David}} spoke with the prophet Nathan. ``Look, here I am living in this\fnote{Lit. \fbib{the}} cedar palace, but the ark of the \divine{Lord}'s covenant remains surrounded by curtains!''

\v{2}``Do everything you have in mind,''\fnote{Lit. \fbib{heart}} Nathan replied to David, ``because God is with you.''

\v{3}But later that same night, this message came to Nathan from God:

\begin{poetry}
\poeml \v{4}``Go tell David, my servant, `This is what the \divine{Lord} says: \\
\poeml `````You won't be building a house\fnote{Lit. \fbib{house}; and so throughout the chapter} for me to inhabit, will you? \v{5}After all, I haven't lived in a house from the day I brought out Israel until today. Instead, I've lived from tent to tent and from one place to another.\fnote{The Heb. lacks \fbib{to another}} \v{6}Wherever I've moved within all of Israel, did I ever ask even one judge of Israel whom I commanded to shepherd my people, `Why haven't you built me a cedar house?'\,''\,'
\end{poetry}

\begin{poetry}
\poeml \v{7}``Now therefore this is what you are to tell my servant David: \\
\poeml `This is what the \divine{Lord} of the Heavenly Armies says: ``I took you from the pasture myself---from tending sheep---to become Commander-in-Chief\fnote{Lit. \fbib{Nagid}; i.e. a senior officer entrusted with dual roles of operational oversight and management authority} over my people Israel. \\
\poeml \v{8}`````Furthermore, I have remained with you everywhere you have gone, annihilating all your enemies right in front of you. I will make your reputation\fnote{Lit. \fbib{name}} great, like the reputation\fnote{Lit. \fbib{name}} of the great ones who have lived on\fnote{The Heb. lacks \fbib{have lived}} earth. \v{9}I will establish a homeland\fnote{Lit. \fbib{place}} for my people Israel, planting them in a secure location where they will never be disturbed anymore. Wicked people\fnote{Lit. \fbib{Children of wickedness}} will not oppress them as happened in the past, \v{10}during the time I had commanded judges to administer\fnote{Lit. \fbib{judges over}} my people Israel. I'll also grant you deliverance from all your enemies. \\
\poeml `````I'm also announcing to you that the \divine{Lord} also will himself build a house\fnote{I.e. a dynasty} for you. \v{11}It will come about that when your life\fnote{Lit. \fbib{days}} is complete and you go to join your ancestors, I will raise up your offspring\fnote{Lit. \fbib{seed}; MT is sing.} after you, who is related to one of\fnote{Or \fbib{is from}} your sons, and I will fortify his kingdom. \v{12}He will build a temple dedicated to me, and I will make his throne last forever. \v{13}I will be a father to him and he will be a son to me. I will never remove my gracious love from him as I did from the one who preceded you. \v{14}I will confirm him in my Temple and in my kingdom forever, and his throne will remain secure forever.''\,'\,''
\end{poetry}

\v{15}Using precisely these words, Nathan communicated this complete oracle to David.
\passage{David's Prayer}
\passageinfo{(2 Samuel 7:18-29)}

\v{16}Then King David went in, sat down in the presence of the \divine{Lord}, and said:

\begin{poetry}
\poeml ``Who am I, \divine{Lord} God, and what is my household,\fnote{Lit. \fbib{house}, and so throughout the chapter} since you have brought me to this? \v{17}Furthermore, this is a small thing to you, God, and yet you have spoken concerning your servant's household for a great while to come, and you have seen in me the fulfillment\fnote{Lit. \fbib{turning}} of man's purpose, \divine{Lord} God. \\
\poeml \v{18}``What more can David say to you about how you are honoring your servant, and you surely know your servant. \v{19}\divine{Lord}, for the sake of your servant, and consistent with your heart, you have done all of these great things and are now making these\fnote{The Heb. lacks \fbib{these}} great things known. \\
\poeml \v{20}``\divine{Lord}, there is no one like you, and we have heard from no god other than you. \v{21}What other one nation on the earth is like your people Israel, God, which you have redeemed from slavery to become your own people, making a great name for yourself when you redeemed your people from Egypt. You did awesome miraculous deeds, driving out nations that stood in their way. \v{22}You took\fnote{Lit. \fbib{gave}} your people Israel to be your very own people forever, and you, \divine{Lord}, have become their God. \\
\poeml \v{23}``And now, \divine{Lord}, let what you have spoken concerning your servant and his household be done forever---and let it be done just as you've promised. \v{24}May your name be made great and honored forever: The \divine{Lord} of the Heavenly Armies, the God of Israel, is God for Israel, and may the family of David your servant stand before you forever. \\
\poeml \v{25}``Because of you, my God, I have been bold to pray to you, as you have told your servant that you will build him a dynasty. \v{26}And now, \divine{Lord}, you are God, and you have promised all of these good things to your servant. \v{27}Furthermore, it has pleased you to bless the dynasty of your servant, so that it will continue in place forever in your presence, because when you, \divine{Lord}, grant a blessing, it is an eternal blessing.''
\end{poetry}
\labelchapt{18}
\passage{David's Military Victories}
\passageinfo{(2 Samuel 8:1-14)}

\chapt{18}
\v{1}After this, David defeated and subdued the Philistines, and then took possession of Gath and its towns from Philistine control. \v{2}He also conquered Moab, placing them in servitude and making them pay tribute.

\v{3}David also defeated King Hadadezer of Zobah, which is near Hamath,\fnote{A city in Syria on the Orontes River} while he was going about establishing his hegemony\fnote{Lit. \fbib{hand}} as far as the Euphrates\fnote{Or \fbib{Perath}; a river valley near Parah (cf. Jer 13:4-7)} River. \v{4}David confiscated 1,000 chariots, 7,000 horsemen, and 20,000 foot soldiers from him, and hamstrung all of the chariot horses except for a reserve force of 100 chariots. \v{5}When Arameans came from Damascus to help King Hadadezer of Zobah, David killed 22,000 of them. \v{6}David later erected garrisons\fnote{So LXX. The Heb. lacks \fbib{garrisons}} in Aram of Damascus, and the Arameans were placed under servitude to David, to whom they paid tribute. \v{7}David also confiscated the gold shields that belonged to Hadadezer's officials and took them to Jerusalem. \v{8}David also confiscated a vast quantity of bronze from Tibhath\fnote{So MT; cf. 2Sam 8:8} and Cun, cities under Hadadezer's control. Later on, Solomon crafted the bronze sea, the pillars, and the bronze vessels for the Temple.\fnote{The Heb. lacks \fbib{for the Temple}}

\v{9}When King Tou of Hamath learned that David had conquered King Hadadezer of Zobah's entire army, \v{10}he sent his son Hadoram to King David to meet and congratulate him, because he had fought against and defeated Hadadezer. Since Hadadezer had often been to war against Tou, he sent all sorts of gold, silver, and bronze goods \v{11}to King David, which David\fnote{Lit. \fbib{he}} also dedicated to the \divine{Lord}, along with silver and gold that he confiscated from all the surrounding\fnote{The Heb. lacks \fbib{surrounding}} nations, including Edom, Moab, the Ammonites, the Philistines, and Amalek.

\v{12}Zeruiah's son Abishai killed 18,000 Edomites in the Salt Valley. \v{13}He erected garrisons in Edom, and all the Edomites became subservient to David, while the \divine{Lord} gave victory to David wherever he went.
\passage{David's Reign}
\passageinfo{(2 Samuel 8:15-18)}

\v{14}So David reigned over all of Israel, administering justice and equity to all of his people. \v{15}Zeruiah's son Joab served in charge of the army, Ahilud's son Jehoshaphat was his personal archivist,\fnote{Or \fbib{recorder}; an officer who kept official records of David's administration} \v{16}Ahitub's son Zadok and Abiathar's son Ahimelech were priests, Shavsha\fnote{Cf. 2Sam 8:16, which reads \fbib{Seraiah}} was his personal secretary,\fnote{Or \fbib{scribe}} \v{17}Jehoiada's son Benaiah supervised the special forces\fnote{Lit. \fbib{Cherethites}; i.e. elite body guards} and mercenaries,\fnote{Lit. \fbib{Pelethites}; i.e. special couriers} while David's sons worked as chief officials in service to the king.\fnote{Cf. 2Sam 8:19, which describes them as priests}
\labelchapt{19}
\passage{Subjugation of Ammon and Aram}
\passageinfo{(2 Samuel 10:1-19)}

\chapt{19}
\v{1}Some time later, King Nahash of Ammon died and his son succeeded him, \v{2}so David told himself, ``I will be loyal to Nahash's son Hanun, since his father showed loyal, gracious love to me.'' So David sent a delegation\fnote{Lit. \fbib{servants}; and so throughout the section} to console him about his loss of his\fnote{The Heb. lacks \fbib{his loss of}} father.

But when David's delegation arrived to visit\fnote{The Heb. lacks \fbib{visit}} Hanun in Ammonite territory to console him, \v{3}the Ammonite officials asked Hanun, ``Do you think that because David has sent a delegation of consolers to you that he is honoring your father? His delegation has arrived to search, overthrow, and scout the land, hasn't it?'' \v{4}So Hanun arrested David's delegation, shaved off their beards, cut off their clothes at the waist line, and sent them away in disgrace.\fnote{The Heb. lacks \fbib{in disgrace}}

\v{5}After they had departed, David was informed about the men, so he sent word\fnote{The Heb. lacks \fbib{word}} to them, since they had been deeply humiliated. He told them, ``Stay at Jericho until your beards have grown back, and then return.''

\v{6}When the Ammonites realized that they had created quite a stink with David, Hanun and the Ammonites spent 1,000 silver talents\fnote{I.e., about 75,000 pounds; a talent weighed about 75 pounds} to hire chariots and mercenaries from Mesopotamia, from Aram-maacah, and from Zobah. \v{7}They hired 32,000 chariots, along with the king of Maacah and his army, who arrived and encamped at Medeba. The Ammonites also were mustered and came out to battle from their home cities. \v{8}In response, David sent out Joab and his entire army of elite soldiers. \v{9}The Ammonites went out in battle formation in front of the entrance to the city while the kings who had come stayed by themselves in the open fields.

\v{10}When Joab observed that the battle lines were set up to oppose him both in front and behind, he appointed some special forces from Israel and arrayed them to oppose the Arameans, \v{11}putting the rest of his forces under command of his brother Abishai, who arrayed them to oppose the Ammonites. \v{12}He told Abishai,\fnote{The Heb. lacks \fbib{to Abishai}} ``If the Arameans prove too strong for me, then you are to help me. If the Ammonites prove too strong for you, then I will help you. \v{13}Be strong, be courageous on behalf of our people and for the cities of our God, and may the \divine{Lord} do what he thinks is best.'' \v{14}So Joab and the soldiers who were with him attacked the Arameans in battle formation, and the Arameans retreated in front of him. \v{15}When the Ammonites saw the Arameans retreating, they also retreated from Joab's brother Abishai back to the city and Joab left for Jerusalem. \v{16}After the Arameans realized that they had been defeated by Israel, they sent for the Arameans who lived beyond the Euphrates River.\fnote{The Heb. lacks \fbib{Euphrates}} Shophach\fnote{Cf. 2Sam 10:16, which reads \fbib{Shobach}} was leading them as commander of Hadadezer's army.

\v{17}When David learned this, he mustered all of Israel, crossed the Jordan, approached the Arameans, and drew up his forces against them. After David had assembled in battle array against the Arameans, the Arameans\fnote{Lit. \fbib{Arameans, they}} attacked him. \v{18}The Arameans retreated from Israel, and David's forces\fnote{Lit. \fbib{David}} killed 7,000 Aramean charioteers, 40,000 soldiers, and Shophach, the commander of their army. \v{19}When Hadadezer's officials saw that they had been defeated by Israel, they sought terms of peace with David and became subservient to him. After this, the Arameans were unwilling to help the Ammonites anymore.
\labelchapt{20}
\passage{The Capture of Rabbah}
\passageinfo{(2 Samuel 11:1; 12:26-31)}

\chapt{20}
\v{1}Later the next spring, at the time that kings go out to fight, Joab led out the army, ravaged the territory of the Ammonites, and then went out and attacked Rabbah, while David remained behind in Jerusalem. Joab besieged Rabbah and conquered it. \v{2}David confiscated the crown of their king\fnote{Lit. \fbib{of Malcam}; LXX reads \fbib{king Molchol}. Cf. 1King 11:5, 33; Zeph 1:5} from his head, and found that its weight was a talent\fnote{I.e. about 75 pounds; a talent weighed about 75 pounds} in gold. A precious stone had been set in it, and it was placed on David's head. He also confiscated a great amount of war booty that had been plundered from the city, \v{3}brought back the people who had lived in it, and put them to conscripted labor with saws, iron picks, and axes. David did this to every Ammonite city, and then David and his entire army\fnote{Lit. \fbib{people}} returned to Jerusalem.
\passage{Fighting Philistine Giants}
\passageinfo{(2 Samuel 21:15-22)}

\v{4}Afterwards, war broke out against the Philistines at Gezer, where Sibbecai the Hushathite killed Sippai, one of the descendants of the Rephaim,\fnote{Or \fbib{the giants}} defeating the Philistines. \v{5}There was also another battle against the Philistines, when Jair's son Elhanan killed Lahmi the Gittite, Goliath's brother, whose spear was as big as\fnote{Lit. \fbib{was like}} a weaver's beam. \v{6}There was also a battle at Gath, where there was a very tall man with six fingers on each hand and six toes on each foot---for a total of 24 digits---who was a descendant of the Rephaim.\fnote{Or \fbib{the giants}} \v{7}When he challenged Israel, Shimei's son Jonathan, David's nephew,\fnote{Lit. \fbib{brother}} killed him. \v{8}These descendants from the giants in Gath died at the hands of David and his servants.
\labelchapt{21}
\passage{David's Unauthorized Census}
\passageinfo{(2 Samuel 24:1-17)}

\chapt{21}
\v{1}Then Satan attacked Israel by inciting David to enumerate a census of Israel. \v{2}David ordered Joab and the commanders of the army,\fnote{Lit. \fbib{people}} ``Go take a census of Israel from Beer-sheba to Dan, and bring me a report so I can be aware of the total number.''

\v{3}But Joab replied, ``May the \divine{Lord} increase the population of his people a hundredfold! Your majesty,\fnote{Lit. \fbib{my lord the king}} all of them are your majesty's servants, aren't they? So why should your majesty demand this? Why should he bring guilt to Israel?''

\v{4}But the king's order overruled Joab, so Joab left, traveled throughout all of Israel, and then returned to Jerusalem \v{5}to report the total population count to David. Throughout all of Israel there were 1,100,000 men trained for war.\fnote{Lit. \fbib{men in wielding a sword}} In Judah there were 470,000 men trained for war. \v{6}Levi and Benjamin were not included in the census, because what the king had commanded was unethical to Joab.
\passage{David Chooses His Punishment}
\passageinfo{(2 Samuel 24:10-18)}

\v{7}God considered this behavior\fnote{Lit. \fbib{this matter}} to be evil, so he attacked Israel. \v{8}David responded to God, ``I sinned greatly by behaving this way. But now I am asking you, please remove the guilt of your servant, since I have acted very foolishly.''

\v{9}So the \divine{Lord} responded through Gad, David's seer. \v{10}``Go and tell David, `This is what the \divine{Lord} says: ``I'm holding three choices out for you: pick one of them for yourself, and I will do it to you.''\,'\,''\fnote{MT pronouns are sing. in this vs.}

\v{11}Gad went to David and told him, ``This is what the \divine{Lord} says: `Make a choice for yourself: \v{12}Either three years of famine, or three months of reversals\fnote{Or \fbib{destruction}} as you are swept away by your enemies while the sword of your enemies overtakes you, or three days with the sword of the \divine{Lord}, consisting of pestilence infecting the land, with the angel of the \divine{Lord} wreaking destruction from border to border throughout all\fnote{Lit. \fbib{destruction in all the border}} of Israel.' Decide right now what I am to answer to the one who sent me.''

\v{13}So David replied to Gad, ``This is a very bad choice for me to make! Let me now please fall into the hand of the \divine{Lord}, because his mercy is very great, but may I never fall into human hands!''

\v{14}Then the \divine{Lord} sent a pestilence to Israel, and 70,000 men died in Israel. \v{15}God also sent an angel to destroy Jerusalem, but as he was about to do so, the \divine{Lord} looked and withdrew\fnote{Or \fbib{and relented concerning}} the calamity by saying to the destroying angel, ``Enough! Stop what you're doing!''\fnote{Lit. \fbib{Stay your hand.}}

So the angel of the \divine{Lord} remained standing near the threshing floor that belonged to Ornan\fnote{Ornan was also known as Araunah; cf. 2Sam 24:16} the Jebusite.\fnote{I.e. a descendant of Canaan's third son (cf. Gen 10:15-16), Jebusites were native to Jebus, the ancient name of the city of Jerusalem} \v{16}David looked up and saw the angel of the \divine{Lord} standing between earth and heaven, with a drawn sword in his hand stretched out over Jerusalem. Then David and the elders, clothed in sackcloth, fell on their faces.

\v{17}David told God, ``Wasn't I the one who ordered the census of the population? Wasn't it I who sinned and acted wickedly? Now as for these sheep, what have they done? \divine{Lord} God, please let your hand be against me and my ancestral household, but don't let your people be ravaged by plague!''
\passage{David's Altar}
\passageinfo{(2 Samuel 24:18-25)}

\v{18}The angel of the \divine{Lord} told Gad to tell David that David was to go up and build an altar to the \divine{Lord} on the threshing floor that belonged to Ornan the Jebusite. \v{19}So David went up, obeying Gad's directive that he had spoken in the name of the \divine{Lord}. \v{20}Ornan turned around and saw the angel. While his four sons with him ran away to hide, Ornan continued to thresh wheat. \v{21}As David approached Ornan, Ornan looked around and observed David, left the threshing floor, and fell to the ground before David with his face on the ground.

\v{22}David told Ornan, ``Give me the threshing floor as a site to build an altar to the \divine{Lord} on it. Give it to me at its full price, so the plague may be averted from the people.''

\v{23}But Ornan replied to David, ``Take it! Let your majesty the king do whatever seems like a good idea to him. Look here! I'm giving the oxen for burnt offerings, the threshing machinery for the wood, and the wheat for a grain offering. I'm giving all of it.''

\v{24}But King David told Ornan, ``No. I will buy them for the full price\fnote{Lit. \fbib{silver}} because I will not offer to the \divine{Lord} what is yours or offer burnt offerings that cost me nothing.''

\v{25}So David paid Ornan 600 shekels weight worth in gold for the site, \v{26}built an altar to the \divine{Lord} there, and presented burnt offerings and peace offerings. He called out to the \divine{Lord}, and he answered him from heaven with fire on the altar of burnt offerings. \v{27}After this, the \divine{Lord} spoke to the angel, who then sheathed his sword.

\v{28}From that time on, after David had observed that the \divine{Lord} had answered him at the threshing floor of Ornan the Jebusite, he made his sacrifices there. \v{29}Meanwhile, the tent of the \divine{Lord} that Moses had crafted in the desert, along with the altar of burnt offerings, were being stored at the high place in Gibeon at that time, \v{30}but David was not going before it to inquire of God, because he was afraid of the sword carried by the angel of the \divine{Lord}.\chapt{22}
\v{1}David said, ``This is where the \divine{Lord} God's Temple will be, along with the altar of burnt offerings for Israel.''
\labelchapt{22}
\passage{David's Plan to Build the Temple}

\v{2}David subsequently issued orders to conscript the resident aliens who lived in the land of Israel and appointed stonecutters to prepare stones for building a temple for God. \v{3}David also provisioned abundant supplies of iron for nails to build the doors for gates and to build clamps. Furthermore, he provided so much bronze it wasn't inventoried, \v{4}as well as an innumerable amount of cedar logs, since the Sidonians and Tyrians brought vast amounts of cedar to David.

\v{5}David thought, ``My son Solomon is young and inexperienced. The temple that will be built for the \divine{Lord} is to be magnificent, well known, and internationally honored, so I will complete preparations for it.'' So before his death, David finished providing a great quantity of materials for it.
\passage{David Commissions Solomon to Build the Temple}

\v{6}Later, David called for his son Solomon and directed him to build a temple to the \divine{Lord} God of Israel. \v{7}David addressed Solomon: ``I have attempted to build a temple to the name of the \divine{Lord} my God. \v{8}But this message from the \divine{Lord} came to me, telling me

\begin{poetry}
\poeml `You have shed a lot of blood and fought great battles. You won't be building a house for my name, since you have shed so much blood on the earth in my sight. \v{9}But look! A son born to you will live comfortably,\fnote{Lit. \fbib{will be a man of comfort}} because I will give him rest from all his enemies that surround him on every side, since his name will be ``Solomon''---I will give peace and quiet for Israel during his lifetime. \v{10}He will build a temple to my name. He will be a son to me, I myself will be a father to him, and I will secure his royal throne in Israel forever.'
\end{poetry}

\v{11}So now, my son, may the \divine{Lord} be with you, so that you are successful in constructing the Temple of the \divine{Lord} your God, just as he has spoken about you.

\v{12}``Only may the \divine{Lord} give you discretion and understanding as he places you in charge over Israel, so you can keep the Law of the \divine{Lord} your God. \v{13}Then you will be successful, if you keep on observing the statutes and ordinances that the \divine{Lord} commanded Moses concerning Israel. Be strong, be courageous, and never give in to fear or dismay. \v{14}At great effort I have provided for the Temple of the \divine{Lord} 100,000 gold talents,\fnote{I.e. about 7,500,000 pounds; a talent weighed about 75 pounds} 1,000,000 silver talents,\fnote{I.e. about 75,000,000 pounds; a talent weighed about 75 pounds} as well as bronze and iron beyond calculation, since there is so much of it. I've also provided timber and stone, but you'll need to obtain more. \v{15}You already have plenty of workers, including stonecutters, masons, carpenters, and an innumerable group of artisans who are skilled at working in \v{16}gold, silver, bronze, and iron. So begin the work, and may the \divine{Lord} be with you.''

\v{17}David also issued these orders to all of the leaders of Israel to assist his son Solomon: \v{18}``Isn't the \divine{Lord} your God with you? Hasn't he surrounded you with comfort? He has delivered the inhabitants of the land into my control, and the land lies subdued both in the \divine{Lord}'s presence and before his people. \v{19}So set your minds and hearts to seek the \divine{Lord} your God, to get up, and to build the sanctuary of the \divine{Lord} God, so the Ark of the Covenant of the \divine{Lord} and the holy vessels of God may be stored in a temple built for the name of the \divine{Lord}.''
\labelchapt{23}
\passage{The Levitical Divisions}

\chapt{23}
\v{1}After David had reached old age, and had completed his reign,\fnote{Lit. \fbib{days}} he set his son Solomon as king over Israel. \v{2}David then gathered together all of the leaders of Israel, including the priests and descendants of Levi. \v{3}descendants of Levi 30 years old and above were counted for a total of 38,000. \v{4}``24,000 of these,'' David said, ``are to be set in charge of the work of the Temple of the \divine{Lord}, with 6,000 serving as officers and judges, \v{5}with 4,000 gatekeepers, and with 4,000 offering praises to the \divine{Lord} with the musical instruments that I have had crafted.''

\v{6}David divided them into divisions based on Gershon, Kohath, and Merari, Levi's sons.
\passage{An Abbreviated Genealogy of Levi's Sons}

\v{7}The descendants of Gershon were Ladan and Shimei. \v{8}The three descendants of Ladan included\fnote{The Heb. lacks \fbib{included}; and so throughout the chapter.} Jehiel (their chief), Zetham, and Joel. \v{9}The three descendants of Shimei included Shelomoth, Haziel, and Haran. These were the heads of families of Ladan.

\v{10}The descendants of Shimei included Jahath, Zina, Jeush, and Beriah. These four were sons of Shimei. \v{11}Jahath served as chief and Zizah was second in rank, but since Jeush and Beriah did not have many sons, they were enrolled as a single family unit.

\v{12}The four descendants of Kohath included Amram, Izhar, Hebron, and Uzziel. \v{13}The descendants of Amram included Aaron and Moses. Aaron had been set apart to consecrate the most holy things, with the intent that he and his sons should present offerings in the \divine{Lord}'s presence forever, ministering to him and pronouncing blessings in his name forever.

\v{14}Meanwhile, as for Moses the man of God, his sons were considered among the tribe of Levi. \v{15}The descendants of Moses included Gershom and Eliezer. \v{16}The descendants of Gershom included Shebuel as their chief.

\v{17}The descendants of Eliezer included Rehabiah as their chief. Eliezer had no other sons, but Rehabiah had many descendants.

\v{18}The descendants of Izhar included Shelomith their chief.

\v{19}The descendants of Hebron included Jeriah their chief, Amariah their second in rank, Jahaziel their third, and Jekameam their fourth.

\v{20}The descendants of Uzziel included Micah their chief and Isshiah their second in rank.

\v{21}The descendants of Merari included Mahli and Mushi. The descendants of Mahli included Eleazar and Kish, \v{22}but Eleazar died having no sons, but only daughters. Their relatives (the descendants of Kish) married them. \v{23}The three descendants of Mushi included Mahli, Eder, and Jeremoth.

\v{24}These were the descendants of Levi according to their ancestral households, with family heads documented according to the names of persons 20 years and older who were appointed to perform work in service to the Temple of the \divine{Lord}.

\v{25}For David had said ``The \divine{Lord} God of Israel has granted rest to his people, and he has taken Israel as his eternal residence. \v{26}Therefore\fnote{Lit. \fbib{Also}} the descendants of Levi are no longer to carry the Tent or its service implements.''\fnote{The quotation possibly concludes at the end of vs. 25.} \v{27}Since, according to David's final instructions, the list above\fnote{The Heb. lacks \fbib{above}} contains the total number of descendants of Levi from the age of 20 years and upward, \v{28}David issued these orders:\fnote{The Heb. lacks \fbib{David issued these orders}}

\begin{poetry}
\poeml ``Instead, they are to assist by lending a hand to the descendants of Aaron regarding service to the Temple of the \divine{Lord} relating to the courts, the chambers, purification of everything pertaining to holiness, and to anything else pertaining to service on behalf of the Temple of God, \v{29}including assisting with the rows of showbread, selecting flour for the grain offerings, the unleavened bread, baked offerings, and oil-based offerings, no matter what the quantity or sizes. \v{30}They are to take their stand morning by morning, thanking and praising the \divine{Lord} right through until the evening, \v{31}whenever burnt offerings are presented to the \divine{Lord}, whether on Sabbaths, New Moons, or scheduled festivals, regularly in the \divine{Lord}'s presence in accordance with the number required to conduct their service. \v{32}By doing this, they will fulfill their obligation as trustees over the Tent of Assembly and the Sanctuary, attending to the needs of\fnote{The Heb. lacks \fbib{to the needs of}} their relatives, who are descendants of Aaron, in keeping with their service on behalf of the Temple of the \divine{Lord}.''
\end{poetry}
\labelchapt{24}
\passage{The Priestly Divisions}

\chapt{24}
\v{1}With respect to the descendants of Aaron, classes of service were organized for Nadab, Abihu, Eleazar, and Ithamar, the descendants of Aaron. \v{2}But Nadab and Abihu died before their father did, leaving no sons, so Eleazar and Ithamar became priests. \v{3}Along with Zadok, one of Eleazar's descendants, and Ahimelech, one of Ithamar's descendants, David organized their service according to their assigned responsibilities.

\v{4}More leaders were located among Eleazar's descendants than among those of Ithamar, so sixteen leaders were appointed from the leaders of the ancestral households of Eleazar's descendants and eight from those of Ithamar. \v{5}They were chosen by impartial lottery, since there were trustees\fnote{Lit. \fbib{officers}} of the sanctuary and officers of God among both Eleazar's descendants and among Ithamar's descendants. \v{6}Nethanel's son Shemaiah, a Levitical scribe, made an official record of them for the king, the officers, Zadok the priest, Abiathar's son Ahimelech, and the heads of ancestral households of both the priests and the descendants of Levi. One ancestral house was chosen for Eleazar and one for Ithamar.

\v{7}The first lottery was chosen in favor of Jehoiarib, the second for Jedaiah, \v{8}third for Harim, the fourth for Seorim, \v{9}the fifth for Malchijah, the sixth for Mijamin, \v{10}the seventh for Hakkoz, the eighth for Abijah, \v{11}the ninth for Jeshua, the tenth for Shecaniah, \v{12}the eleventh for Eliashib, the twelfth for Jakim, \v{13}the thirteenth for Huppah, the fourteenth for Jeshebeab, \v{14}the fifteenth for Bilgah, the sixteenth for Immer, \v{15}the seventeenth for Hezir, the eighteenth for Happizzez, \v{16}the nineteenth for Pethahiah, the twentieth for Jehezkel, \v{17}the twenty-first for Jachin, the twenty-second for Gamul, \v{18}the twenty-third for Delaiah, and the twenty-fourth for Maaziah. \v{19}These were appointed to enter the Temple of the \divine{Lord} according to their protocols established by their ancestor Aaron, as commanded by the \divine{Lord} God of Israel.
\passage{Other Levitical Divisions}

\v{20}Now with respect to the descendants of Levi there remained Shubael from the descendants of Amram and Jehdeiah from the descendants of Shubael; \v{21}with respect to Rehabiah, Isshiah their chief from the descendants Rehabiah; \v{22}with respect to the Izharites, Shelomoth, Jahath from the descendants of Shelomoth; \v{23}with respect to the descendants of Hebron, Jeriah their chief, Amariah their second in rank, Jahaziel their third, and Jekameam their fourth; \v{24}with respect to the descendants of Uzziel, Micah; with respect to the descendants of Micah, Shamir; \v{25}with respect to Micah's brother Isshiah; with respect to the descendants of Isshiah, Zechariah; \v{26}with respect to Merari's sons, Mahli and Mushi; with respect to the sons of Jaaziah, Beno; \v{27}with respect to the sons of Merari, Jaaziah, Beno, Shoham, Zaccur, and Ibri; \v{28}with respect to Mahli, Eleazar, who had no sons; \v{29}with respect to Kish, Jerahmeel, one of the descendants of Kish; \v{30}and with respect to the descendants of Mushi, Mahli, Eder, and Jerimoth. These were the descendants of Levi according to their ancestral households. \v{31}These individuals also cast lots corresponding to their relatives, Aaron's descendants, in the presence of King David, Zadok, and Ahimelech, and in the presence of the heads of the ancestral households of the priests and of the descendants of Levi, and the eldest was treated as impartially as was the younger brother.
\labelchapt{25}
\passage{The Musicians}

\chapt{25}
\v{1}Along with officers in his army, David consecrated to assist in service to the descendants of Asaph, Heman, and Jeduthun those who prophesy with lyres, harps, and cymbals.

The list of those who participated in this service included: \v{2}from the descendants of Asaph: Zaccur, Joseph, Nethaniah, and Asarelah, sons of Asaph mentored by\fnote{Lit. \fbib{under the hand of}; and so throughout the chapter} Asaph himself, who prophesied under the supervision\fnote{Lit. \fbib{hand}} of the king; \v{3}from Jeduthun, these six of his descendants: Gedaliah, Zeri, Jeshaiah, Shimei, Hashabiah, and Mattithiah, mentored by their father Jeduthun, who played a lyre and prophesied, giving thanks and praise to the \divine{Lord}; \v{4}from Heman, these descendants: Bukkiah, Mattaniah, Uzziel, Shebuel, Jerimoth, Hananiah, Hanani, Eliathah, Giddalti, Romamti-ezer, Joshbekashah, Mallothi, Hothir, and Mahazioth. \v{5}All of these were descendants of Heman the king's seer, according to God's promise to exalt him, since God had given Heman fourteen sons and three daughters. \v{6}They were all under their father's supervision regarding music in the Temple of the \divine{Lord} with cymbals, harps, and lyres for the service of the Temple of God.

Asaph, Jeduthun, and Heman were under command of the king. \v{7}They and their relatives who had been skillfully trained in singing to the \divine{Lord}, numbered 288. \v{8}Their duties, whether significant or insignificant, whether performed by teacher or pupil alike, were assigned by lottery.

\v{9}Asaph's first lottery was cast in favor of Joseph; the second went to Gedaliah, that is, to him, to his relatives, and his sons, for a total of twelve;\fnote{The Heb. lacks \fbib{for a total of}; and so throughout the chapter} \v{10}the third to Zaccur, his sons and his relatives, for a total of twelve; \v{11}the fourth to Izri, his sons and his relatives, for a total of twelve; \v{12}the fifth to Nethaniah, his sons and his relatives, for a total of twelve; \v{13}the sixth to Bukkiah, his sons and his relatives, for a total of twelve; \v{14}the seventh to Jesharelah, his sons and his relatives, for a total of twelve; \v{15}the eighth to Jeshaiah, his sons and his relatives, for a total of twelve; \v{16}the ninth to Mattaniah, his sons and his relatives, for a total of twelve; \v{17}the tenth to Shimei, his sons and his relatives, for a total of twelve; \v{18}the eleventh to Azarel, his sons and his relatives, for a total of twelve; \v{19}the twelfth to Hashabiah, his sons and his relatives, for a total of twelve; \v{20}the thirteenth to Shubael, his sons and his relatives, for a total of twelve; \v{21}the fourteenth to Mattithiah, his sons and his relatives, for a total of twelve; \v{22}the fifteenth to Jeremoth, his sons and his relatives, for a total of twelve; \v{23}the sixteenth to Hananiah, his sons and his relatives, for a total of twelve; \v{24}the seventeenth to Joshbekashah, his sons and his relatives, for a total of twelve; \v{25}the eighteenth to Hanani, his sons and his relatives, for a total of twelve; \v{26}the nineteenth to Mallothi, his sons and his relatives, for a total of twelve; \v{27}the twentieth to Eliathah, his sons and his relatives, for a total of twelve; \v{28}the twenty-first to Hothir, his sons and his relatives, for a total of twelve; \v{29}the twenty-second to Giddalti, his sons and his relatives, for a total of twelve; \v{30}the twenty-third to Mahazioth, his sons and his relatives, for a total of twelve; \v{31}the twenty-fourth to Romamti-ezer, his sons and his relatives, for a total of twelve.
\labelchapt{26}
\passage{The Korahite Trustees}

\chapt{26}
\v{1}The guild\fnote{Lit. \fbib{divisions} or \fbib{courses}} of trustees\fnote{Lit. \fbib{gatekeepers} or \fbib{porters}; i.e. attendants who administered access to the Temple} included, from the descendants of Korah, Kore's son Meshelemiah from Asaph's descendants; \v{2}Meshelemiah's sons Zechariah, his firstborn, Jediael his second, Zebadiah his third, Jathniel his fourth, \v{3}Elam his fifth, Jehohanan his sixth, and Eliehoenai his seventh; \v{4}Obed-edom's sons Shemaiah, his firstborn, Jehozabad his second, Joah his third, Sachar his fourth, Nethanel his fifth, \v{5}Ammiel his sixth, Issachar his seventh, and Peullethai his eighth, since God had blessed him.

\v{6}Furthermore, his son Shemaiah had sons born to him who wielded authority in their ancestral households, since they were mighty men of valor. \v{7}These sons of Shemaiah included\fnote{The Heb. lacks \fbib{included}} Othni, Rephael, Obed, and Elzabad, whose brothers were valiant, able men, Elihu and Semachiah. \v{8}All of these sons of Obed-edom, along with their sons and brothers, were valiant men, fully qualified for duty---62 descendants\fnote{The Heb. lacks \fbib{descendants}} of Obed-edom. \v{9}Meshelemiah had 18 sons and brothers who were valiant men. \v{10}Hosah, one of Merari's sons, had these\fnote{The Heb. lacks \fbib{these}} sons: Shimri their chief (though not the firstborn, his father had appointed him chief), \v{11}Hilkiah his second, Tebaliah his third, and Zechariah his fourth, with a total of 13 sons and brothers of Hosah

\v{12}With respect to their leaders, these courses of trustees had responsibilities, along with their relatives, regarding ministry within the Temple of the \divine{Lord} \v{13}assigned by lottery according to their ancestral households, whether large or small alike, for their gate assignments. \v{14}The lot for the eastern gate\fnote{The Heb. lacks \fbib{gate}} fell to Shelemiah. They also cast lots for his son Zechariah, who was a wise counselor, and his lot indicated the northern gate.\fnote{The Heb. lacks \fbib{gate}} \v{15}Obed-edom's lot indicated the south gate,\fnote{The Heb. lacks \fbib{gate}} and his sons were also allotted responsibility for the storehouse. \v{16}For Shuppim and Hosah the lot indicated the west at the gate of Shallecheth on the ascending road.

Each guard corresponding to each guard, \v{17}on the east six descendants of Levi were assigned\fnote{The Heb. lacks \fbib{assigned}; and so throughout the chapter} for each day, on the north four for each day, on the south four for each day (as well as two pairs of guards assigned\fnote{Lit. \fbib{two and two}} to the storehouse), \v{18}and for the colonnade on the west four were assigned at the road and two at the colonnade. \v{19}These were the ranks of trustees assigned among the descendants of Korah and the sons of Merari.
\passage{Oversight of the Treasuries}

\v{20}Now with respect to the descendants of Levi, Ahijah was responsible for the treasuries of the Temple of God, including the treasuries containing dedicated gifts. \v{21}With respect to the descendants of Ladan, the Gershonite descendants pertaining to Ladan, the heads of families pertaining to Ladan the Gershonite, there was Jehieli. \v{22}The descendants of Jehieli, Zetham and his brother Joel, were responsible for the treasuries of the Temple of the \divine{Lord}.

\v{23}From the descendants of Amram, Izhar, Hebron, and Uzziel were assigned \v{24}Shebuel, a descendant of Gershom and a descendant of Moses (as chief officer\fnote{Lit. \fbib{Nagid}; i.e. a senior officer entrusted with dual roles of operational oversight and administrative authority} in charge of the treasuries) \v{25}and his brothers from Eliezer, including his son Rehabiah, his son Jeshaiah, his son Joram, his son Zichri, and his son Shelomoth.

\v{26}Shelomoth and his brothers were responsible for all of the treasuries of dedicated gifts given by King David, by the heads of families, by the officers of groups of thousands and groups of hundreds, and by the leading army officers. \v{27}They dedicated gifts for the maintenance of the Temple of the \divine{Lord} from spoils of war. \v{28}Furthermore, everything that Samuel the seer, Kish's son Saul, Ner's son Abner, and Zeruiah's son Joab had dedicated---all of their dedicated gifts---were under the care of Shelomoth and his brothers.

\v{29}From the descendants of Izhar, Chenaniah and his sons were assigned as officers and judges with responsibilities relating to external duties. \v{30}From the descendants of Hebron, Hashabiah and his relatives---1,700 outstanding men---were assigned oversight of Israel west of the Jordan regarding all of the \divine{Lord}'s work and services on behalf of the king.

\v{31}From the descendants of Hebron, Jerijah was assigned chief of the descendants of Hebron. During the fortieth year of David's administration, a search was made by genealogical record, family by family, to find men of great ability, including those found at Jazer in Gilead. \v{32}King David appointed Jerijah,\fnote{Lit. \fbib{him}} his relatives, and 2,700 competent men who were each family heads, to oversee the tribes of Reuben and Gad, and the half-tribe of Manasseh regarding everything pertaining to God as well as matters relating to the king.
\labelchapt{27}
\passage{Military Divisions}

\chapt{27}
\v{1}The Israelis, according to the number of the leaders of their families, the officers of groups of thousands and groups of hundreds, and their leaders who served the king on behalf of the army divisions of 24,000 soldiers on duty month by month throughout the year, consisted of the following.

\v{2}Zabdiel's son Jashobeam was responsible\fnote{Lit. \fbib{over}; and so throughout the chapter} for the first division of 24,000 soldiers\fnote{The Heb. lacks \fbib{soldiers}; and so throughout the chapter} for the first month. \v{3}A descendant of Perez, he was chief of all the commanders of the army for the first month.

\v{4}Dodai the Ahohite was responsible for the division of the second month. Mikloth served as chief officer\fnote{Lit. \fbib{Nagid}; i.e. a senior officer entrusted with dual roles of operational oversight and administrative authority} of his division, consisting of 24,000 soldiers.

\v{5}Jehoiada's son Benaiah the priest was commander of the third division for the third month, consisting of 24,000 soldiers. \v{6}This was the same Benaiah who was one of the elite men of the Thirty and in command of the Thirty. His son Ammizabad was responsible for his division.

\v{7}Joab's brother Asahel was fourth for the fourth month, assisted\fnote{Or \fbib{followed}} by his son Zebadiah, with 24,000 soldiers in his division.

\v{8}The fifth commander for the fifth month was Shamhuth the Izrahite. His division consisted of 24,000 soldiers.

\v{9}Ikkesh's son Ira from Tekoa was sixth for the sixth month; there were 24,000 soldiers in his division.

\v{10}Helez the Pelonite, an Ephraimite, was seventh for the seventh month; 24,000 soldiers served in his division.

\v{11}Sibbecai the Hushathite, a Zerahite, was eighth for the eighth month; 24,000 soldiers served in his division.

\v{12}Abiezer from Anathoth, a descendant of Benjamin, was ninth for the ninth month; 24,000 soldiers served in his division.

\v{13}Mahari from Netophah, a Zerahite, was tenth for the tenth month; 24,000 soldiers served in his division.

\v{14}Benaiah from Pirathon, an Ephraimite, was eleventh for the eleventh month; 24,000 soldiers served in his division.

\v{15}Heldai the Netophathite, from Othniel, was twelfth for the twelfth month; 24,000 soldiers served in his division.
\passage{Tribal Leaders}

\v{16}Wielding the scepters of Israel for the descendants of Reuben, there was\fnote{The Heb. lacks \fbib{there was}; and so throughout the chapter} Zichri's son Eliezer as chief officer;\fnote{Lit. \fbib{Nagid}; i.e. a senior officer entrusted with dual roles of operational oversight and administrative authority} for the descendants of Simeon there was Maacah's son Shephatiah; \v{17}for Levi there was Kemuel's son Hashabiah; for Aaron there was Zadok; \v{18}for Judah there was Elihu, one of David's brothers; for Issachar there was Michael's son Omri; \v{19}for Zebulun there was Obadiah's son Ishmaiah; for Naphtali, there was Azriel's son Jerimoth; \v{20}for the descendants of Ephraim, there was Azaziah's son Hoshea; for the half-tribe of Manasseh, there was Pedaiah's son Joel; \v{21}for the half-tribe of Manasseh in Gilead, there was Zechariah's son Iddo; for Benjamin, there was Abner's son Jaasiel; \v{22}for Dan, there was Jeroham's son Azarel. These were the leaders of the tribes of Israel.

\v{23}David did not complete a census of those younger than 20 years of age, since the \divine{Lord} had said he would make Israel as numerous as the stars of heaven. \v{24}Zeruiah's son Joab began the census, but never completed it. Nevertheless, God became angry with Israel because of this, so the number was never entered into the official records of the Annals of King David.\fnote{An ancient chronicle of Israel, apparently now lost}
\passage{Civic Leaders}

\v{25}Adiel's son Azmaveth was responsible for the king's treasuries. Uzziah's son Jonathan was in charge of treasuries located in the country, in cities, in villages, and in towers. \v{26}Chelub's son Ezri supervised the field workers who tilled the soil. \v{27}Shimei the Ramathite supervised the vineyards. In charge over the produce of the vineyards for the wine cellars was Zabdi the Shiphmite. \v{28}Baal-hanan the Gederite supervised the olive and sycamore\fnote{The sycamore fruit tree native to Israel bears figs} trees in the Shephelah.\fnote{I.e. the verdant central lowlands of Israel; cf. Josh 10:40} Joash supervised the oil reserves. \v{29}Shitrai the Sharonite supervised the herds that were pastured in Sharon. Adlai's son Shaphat supervised the herds in the valleys. \v{30}Obil the Ishmaelite supervised the camels. Jehdeiah the Meronothite supervised the donkeys. Jaziz the Hagrite supervised the flocks. \v{31}All of these served as stewards over King David's property.

\v{32}David's uncle Jonathan was a counselor, since he was a man of understanding and a scribe, and Hachmoni's son Jehiel was an attendant to the king's sons. \v{33}Ahithophel served as an advisor to the king, Hushai the Archite was the king's trusted associate, \v{34}and under Ahithophel there was Benaiah's son Jehoiada and Abiathar. Joab served as commander of the king's army.
\labelchapt{28}
\passage{David Addresses Israel}

\chapt{28}
\v{1}David gathered together all of the leaders of Israel, the leaders of the tribes, division officers who reported to the king, the commanders of thousands, commanders of hundreds, the supervisors of the property and livestock that belonged to the king and to his sons, along with all of the officers of the palace, the elite forces, and all of the soldiers.

\v{2}King David rose to his feet and said, ``My fellow citizens,\fnote{Lit. \fbib{My brothers and my people}} may I have your attention. I intended to build a house of rest for the Ark of the Covenant of the \divine{Lord}, for a footstool of our God, so I began preparations for its construction. \v{3}But then God told me, `You will not build a temple to my name, because you are a man of war, and you have committed bloodshed.'\fnote{I.e. perhaps an allusion to Uriah the Hittite} \v{4}Nevertheless, the \divine{Lord} God of Israel chose me from my entire ancestral household to be king over Israel forever, since he had chosen Judah as Commander-in-Chief.\fnote{Lit. \fbib{Nagid}; i.e. a senior officer entrusted with dual roles of operational oversight and administrative authority} In my ancestor Judah's household, from my father's household, and from among my father's sons it pleased him to make me king over all of Israel.

\v{5}``Now out of all of my sons (since the \divine{Lord} has given me many of them), he has selected my son Solomon to sit on the throne of the kingdom of the \divine{Lord}, ruling\fnote{The Heb. lacks \fbib{ruling}} over Israel. \v{6}He told me,

\begin{poetry}
\poeml `I chose your son Solomon to be the one who will construct my Temple and my courts, because I have chosen him to be a son to me, and I will be a father to him. \v{7}I will establish his kingdom forever, assuming he remains strongly committed to carry out my commandments and ordinances, as he is doing today.'
\end{poetry}

\v{8}Therefore, in the presence\fnote{Lit. \fbib{eyes}} of all of Israel, the assembly of the \divine{Lord}, and while our God is listening, observe and search through all of the commandments of the \divine{Lord} your God, so that you may continue to possess this good land, leaving it for an inheritance forever to benefit your descendants who come after you.''
\passage{David Addresses Solomon Directly}

\v{9}``Now as for you, my son Solomon, get to know the God of your father. Serve him with a sound heart and a devoted soul, because the \divine{Lord} is searching every heart, every plan and thought. He will be found by you, assuming you are seeking him, but if you abandon him, he will abandon you forever. \v{10}So keep watching, because the \divine{Lord} has chosen you to build the Temple of his sanctuary. So be strong, and get to work!''
\passage{David Transfers Plans and Materials to Solomon}

\v{11}At this point in his address,\fnote{The Heb. lacks \fbib{At this point in his address}} David transferred to his son Solomon the construction plans for the Hall of Justice,\fnote{Or \fbib{Temple vestibule}} its buildings, its treasure vaults, its upper rooms, its inner chambers, the housing for the Mercy Seat, \v{12}and the plans for everything else that he had in mind for the courtyards of the Temple of the \divine{Lord}. Included were plans for\fnote{The Heb. lacks \fbib{were plans for}} all of the surrounding vaults and treasuries of the Temple of God intended for storage of\fnote{The Heb. lacks \fbib{intended for storage of}} dedicated gifts, \v{13}for use by the ranks of priests and descendants of Levi, for all the work of service responsibilities in the Temple of the \divine{Lord}, and for all of the utensils used in the work of the Temple of the \divine{Lord}. \v{14}David also transferred to him\fnote{The Heb. lacks \fbib{David also transferred to him}} by weight the gold that was to be used to craft the\fnote{The Heb. lacks \fbib{that was to be used to craft the}} service utensils, the silver that was to be used to craft the\fnote{The Heb. lacks \fbib{that was to be used to craft the}} service utensils, \v{15}the gold for the golden lamp stands and their lamps, the silver for a lamp stand and its lamps (each according to its intended use in the service), \v{16}the gold by weight for each table of the rows of bread, the silver for the silver tables, \v{17}pure gold for the forks, the basins, the cups, the golden bowls (along with enough gold by weight for each one), enough weight for each of the silver bowls, \v{18}refined gold for the altar of incense, by weight, along with his plans for crafting\fnote{The Heb. lacks \fbib{for crafting}} the golden chariot for the cherubim that spread out their wings to cover the Ark of the Covenant of the \divine{Lord}.
\passage{David Continues His Address}

\v{19}``All of these things the \divine{Lord} made clear to me in writing at his direction---the construction plans for all of the building.''

\v{20}David continued with these words for his son Solomon: ``Be strong and courageous, and get to work. Never be afraid or discouraged, for the \divine{Lord} God, my God, is with you. He will not fail you nor will he abandon you right up to your completion of the work for the service of the Temple of the \divine{Lord}. \v{21}Now look! Here are the ranks of the priests and the descendants of Levi for the entire service of the Temple of God, and in all of the work there will be all types of volunteers who have skills for anything needed for the services. Furthermore, the officers and all of the people will be at your complete command.''
\labelchapt{29}
\passage{Offerings for the Temple}

\chapt{29}
\v{1}Then King David addressed the entire assembly: ``My son Solomon, the one whom God alone has chosen, is still young and inexperienced, and the task is great, since this structure will be a citadel to the \divine{Lord} God and not for human beings. \v{2}To the extent that I have been able to do so, I have provided supplies for the Temple of my God, including gold for what is to be made of gold, silver for what is to be made of silver, bronze for what is to be made of bronze, iron for what is to be made of iron, wood for what is to be made of wood, and great quantities of onyx, precious stones, antimony, colored stones, all types of other semi-precious stones, and plenty of marble.

\v{3}``In addition to everything that I have supplied for the Temple, it pleases me to provide my own treasure of gold and silver, so because of my love for the Temple of my God I hereby give to the Temple of my God the following: \v{4}3,000 gold talents\fnote{I.e. about 225,000 pounds; a talent weighed about 75 pounds} imported from Ophir,\fnote{Or \fbib{from a source of fine gold}; cf. 2Chr 8:18} 7,000 talents\fnote{I.e. about 525,000 pounds; a talent weighed about 75 pounds} of refined silver for gilding the walls of the Temple \v{5}and for all the work to be undertaken by skilled artists, gold for what is to be made of gold, and silver for what is to be made of silver. Who then, will be dedicating the productivity\fnote{Lit. \fbib{filling}} of his own work\fnote{Lit. \fbib{hand}} to the \divine{Lord} today?''

\v{6}So the leaders of the ancestral households presented their voluntary offerings, as did the leaders of the tribes, the commanders of thousands and hundreds, and the officials in charge of the king's business. \v{7}They presented 5,000 gold talents\fnote{I.e. about 375,000 pounds; a talent weighed about 75 pounds} and 10,000 gold darics\fnote{I.e. about 156 pounds; a daric weighed about one quarter of an ounce} for the work of the Temple of God, 10,000 silver talents\fnote{I.e. about 750,000 pounds; a talent weighed about 75 pounds}, 18,000 bronze talents,\fnote{I.e. about 1,350,000 pounds; a talent weighed about 75 pounds} and 100,000 iron talents.\fnote{I.e. about 7,500,000 pounds; a talent weighed about 75 pounds} \v{8}Whoever owned precious stones gave them to the treasury of the Temple of the \divine{Lord}, in care of Jehiel the Gershonite. \v{9}Then the people rejoiced because they had given voluntarily, since with a devoted heart they had freely given to the \divine{Lord}.
\passage{David's Praise to God}

King David also rejoiced greatly. \v{10}Then David blessed the \divine{Lord} in the presence of the entire assembly. David said,

\begin{poetry}
\poeml How blessed you are, \divine{Lord}, \\
\poemll    the God of our ancestor Israel, \\
\poemlll       from eternity to eternity! \\
\poeml \v{11}To you, \divine{Lord}, belongs the greatness, and the valor, \\
\poemll    and the splendor, and the endurance, and the majesty \\
\poeml because all that is in heaven \\
\poemll    and on earth is yours. \\
\poeml To you belongs the kingdom, \divine{Lord}, \\
\poemll    and you are exalted as head over all. \\
\poeml \v{12}Both wealth and honor proceed from you, \\
\poemll    and you are ruling over them all. \\
\poeml You control\fnote{Lit. \fbib{You have in your hand}} power--- \\
\poemll    you control who is made great, \\
\poemlll       and how everyone becomes strong. \\
\poeml \v{13}And so, our God, we are giving you thanks, \\
\poemll    and we are praising your wonderful name! \\
\poeml \v{14}But who am I, \\
\poemll    and who are my people, \\
\poemlll       that we make such voluntary offerings as these? \\
\poeml For all things come from you, \\
\poemll    and from your own hand we are giving to you. \\
\poeml \v{15}For we are aliens and vagrants in your presence, \\
\poemll    as were all of our ancestors. \\
\poeml Our days on the earth pass away like shadows, \\
\poemll    and we have no hope. \\
\poeml \v{16}\divine{Lord} our God, all of this abundance that we have given \\
\poemll    for building a temple for your great name \\
\poeml was provided by you\fnote{Lit. \fbib{by your hand}} \\
\poemll    and all of it belongs to you. \\
\poeml \v{17}And I know, God, \\
\poemll    that it is you who searches the heart \\
\poemlll       and you who finds pleasure in righteousness. \\
\poeml With a righteous heart I have freely given all these things, \\
\poemll    and now I have seen all of these people of yours \\
\poemlll       giving freely and joyfully to you! \\
\poeml \v{18}\divine{Lord} God of Abraham, Isaac, and Israel, our ancestors, \\
\poemll    keep your purposes and thoughts \\
\poemlll       constantly in the hearts of your people \\
\poeml and direct their hearts toward you, \\
\poeml \v{19}granting to my son Solomon to keep with a devoted heart \\
\poeml your commands, your decrees, and your statutes, \\
\poemll    carrying out all of them, \\
\poeml and that he may build the Temple \\
\poemll    for which I have made the preparations.
\end{poetry}

\v{20}Then David told the entire assembly, ``Bless the \divine{Lord} your God, please.'' So the entire assembly blessed the \divine{Lord} God of their ancestors, bowing their heads and falling in the \divine{Lord}'s presence and before the king. \v{21}The next day, they offered sacrifices and burnt offerings to the \divine{Lord} amounting to\fnote{The Heb. lacks \fbib{amounting to}} 1,000 bulls, 1,000 rams, and 1,000 lambs, along with their libations. Sacrifices were abundant throughout all Israel, \v{22}and they ate and drank in the \divine{Lord}'s presence with great joy.
\passage{Solomon is Anointed King}
\passageinfo{(1 Kings 1:38-40; 2:12)}

They crowned David's son Solomon king a second time and anointed him to serve\fnote{The Heb. lacks \fbib{serve}} as Commander-in-Chief\fnote{Or \fbib{prince}; lit. \fbib{Nagid}; i.e. an officer entrusted with dual roles of operational oversight and management authority} to the \divine{Lord} and Zadok to serve\fnote{The Heb. lacks \fbib{serve}} as priest. \v{23}So Solomon sat on the throne of the \divine{Lord} as king in the place of\fnote{Or \fbib{king under}} his father David. He prospered, and all of Israel obeyed\fnote{Or \fbib{listened to}} him. \v{24}All of the officials, all of the valiant soldiers, and all of King David's sons submitted to King Solomon's control, \v{25}and the \divine{Lord} exalted Solomon magnificently in the sight of all Israel, bestowing upon him royal majesty such as had not been given to any king in Israel before him.
\passage{Summary of the Reign of King David}

\v{26}Jesse's son David reigned as king over all of Israel, \v{27}serving as king over Israel for 40 years. He reigned for seven years in Hebron and for 33 in Jerusalem. \v{28}He died at a good old age, having lived a full life, replete with riches and honor, and with his son Solomon reigning in his place. \v{29}The activities of David the king are recorded in the History of Samuel the Seer,\fnote{An ancient chronicle of Israel, apparently now lost} in the History of Nathan the Prophet,\fnote{An ancient chronicle of Israel, apparently now lost} and in the History of Gad the Seer,\fnote{An ancient chronicle of Israel, apparently now lost} \v{30}including details regarding\fnote{The Heb. lacks \fbib{details regarding}} his reign, his power, the circumstances that attended his life, Israel, and all of the kingdoms of the countries that surrounded him.\fnote{The Heb. lacks \fbib{that surrounded him}}

\bookheader{2 Chronicles}
\labelbook{2Chr}

\bookpretitle{The Book of}
\booktitle{Second Chronicles}

\labelchapt{1}
\passage{The Beginnings of Solomon's Administration}
\passageinfo{(1 Kings 3:1-15)}

\chapt{1}
\v{1}As David's son Solomon consolidated\fnote{Or \fbib{strengthened}} his administration,\fnote{Lit. \fbib{kingdom}} the \divine{Lord} his God was with him to make him very successful.\fnote{Lit. \fbib{great}} \v{2}Solomon addressed the entire nation of Israel, including the commanders of thousands and hundreds, the judges, all the other leaders of Israel, and all of the heads of the ancestral houses of Israel.

\v{3}Solomon, along with the whole assembly with him, met at the high place in Gibeon because that's where God's Tent of Meeting that the \divine{Lord}'s servant Moses had constructed in the wilderness was located. \v{4}Nevertheless, David had brought the Ark of God from Kiriath-jearim to the place that David had prepared for it, after having erected a tent for it in Jerusalem. \v{5}Also, the bronze altar that Uri's son Bezalel, Hur's grandson had erected, was in place in front of the \divine{Lord}'s tent. Solomon and the assembly sought the \divine{Lord}\fnote{The Heb. lacks \fbib{the \divine{Lord}}} there. \v{6}Solomon approached the presence of the \divine{Lord} at the bronze altar that had been placed at the Tent of Meeting and offered 1,000 burnt offerings on it.
\passage{Solomon Asks God for Wisdom}

\v{7}That very night God appeared to Solomon and told him, ``Ask what I am to give you.''

\v{8}Solomon replied to God, ``You showed great gracious love to my father David, and have established me as king in his place. \v{9}Now, \divine{Lord} God, your promise to my father David is fulfilled, because you have made me king over a people as numerous as the dust of the earth. \v{10}Give me wisdom now, so I may go in and out among\fnote{Or \fbib{out in front of}} this people, because who can rule this great people that belongs to you?

\v{11}God told Solomon, ``Since you had this in mind,\fnote{Lit. \fbib{heart}} to ask neither to focus on riches, wealth, honor, or the lives of those who hate you, nor have you requested a long life, but instead you have asked for wisdom and knowledge for yourself, so that you may rule my people over whom I have established you as king, \v{12}wisdom and knowledge have been granted to you. Furthermore, I will give you riches, wealth, and honor---such as none of the kings owned who lived before you and none after you are to ever attain their equal.''
\passage{Solomon's Wealth}
\passageinfo{(1 Kings 10:26-29; 2 Chronicles 9:25-28)}

\v{13}So Solomon returned from the Tent of Meeting at the high place in Gibeon to Jerusalem, where he reigned over Israel. \v{14}Solomon amassed both chariots and horsemen: he owned 1,400 chariots and 12,000 horsemen, stationing them in armories\fnote{Lit. \fbib{in chariot cities}} and with the king in Jerusalem. \v{15}The king made silver and gold as common in Jerusalem as stones, and made cedar\fnote{I.e. a genus of coniferous evergreen in the family \fbib{Pinaceae}; and so throughout the book} trees as plentiful as sycamore\fnote{The sycamore fruit tree native to Israel bears figs} trees that grow in the Shephelah.\fnote{I.e. the verdant central lowlands of Israel; cf. Josh 10:40} \v{16}Solomon's horses were imported from Egypt and from Kue; the king's procurement officials obtained them from Kue at great\fnote{The Heb. lacks \fbib{great}} price. \v{17}Chariots were imported from Egypt for 600 shekels\fnote{The Heb. lacks \fbib{shekels}} each, and horses cost 150 shekels\fnote{The Heb. lacks \fbib{shekels}} each, and then they exported them to all of the kings of the Hittites and to the kings of Aram.
\labelchapt{2}
\passage{Solomon Enlists King Hiram's Help to Build the Temple}
\passageinfo{(1 Kings 5:1-18)}

\chapt{2}
\v{1}\fnote{This v. is 1:18 in MT and LXX}Now Solomon was determined\fnote{Lit. \fbib{saying}} to build a temple dedicated to the Name of the \divine{Lord} as well as his own royal palace. \v{2}\fnote{This v. is 2:1 in MT and LXX, and so throughout the chapter}So Solomon conscripted 70,000 men to do heavy work, 80,000 men to quarry in the hill country, and 3,600 to supervise them. \v{3}Solomon also sent this message to King Hiram\fnote{Or \fbib{Huram}, and so throughout the chapter} of Tyre:

\begin{poetry}
\poeml ``Just as you did with my father David, sending him cedars to build him a palace to live in, do the same for me. \v{4}Look, I'm building a temple dedicated to the name of the \divine{Lord} my God, to his glory, so we can burn fragrant incense in his presence, display rows of the bread of his presence continuously, and make burnt offerings in the morning, evening, on Sabbath days,\fnote{The Heb. lacks \fbib{days}} during New Moon festivals,\fnote{The Heb. lacks \fbib{festivals}} and during appointed festivals scheduled\fnote{The Heb. lacks \fbib{scheduled}} by the \divine{Lord} our God. This is mandated forever in Israel. \\
\poeml \v{5}``The Temple that I'm building will be great, because the greatness of our God surpasses that of\fnote{The Heb. lacks \fbib{surpasses that of}} all gods. \v{6}But who can build a temple for him, since neither heaven nor the highest of the heavens can contain him? So who am I, that I should build a temple to him, except to burn incense in his presence? \\
\poeml \v{7}``At any rate, send me an individual who is a skilled craftsman in gold, silver, bronze, and iron, as well as in purple, crimson, and blue\fnote{Or \fbib{violet}} materials,\fnote{The Heb. lacks \fbib{materials}} who knows how to craft engravings, so he may work with the craftsmen whom I have assembled in Judah and Jerusalem, as provided for by my father David. \v{8}Also send me cedar, cypress, and algum timber from Lebanon, since I'm aware that your servants know how to cut down timber from Lebanon. My servants will accompany your servants \v{9}to prepare an abundant amount of timber for me, because the Temple that I'm building is to be great and awesome. \\
\poeml \v{10}``Now look! I will pay your servants, the lumberjacks who prepare the timber, 20,000 measures\fnote{Lit. \fbib{homers}, about 720,000 bushels or 4,400,000 liters, at six bushels or 220 liters per homer} of barley, 20,000 baths\fnote{I.e. about 720,000 gallons or about 440,000 liters, at six gallons or 22 liters per bath} of wine, and 20,000 baths\fnote{I.e. about 720,000 gallons or about 440,000 liters, at six gallons or 22 liters per bath} of oil.''
\end{poetry}
\passage{Solomon's Wealth}

\v{11}In a letter that he sent to Solomon, King Hiram of Tyre wrote,\fnote{Lit. \fbib{responded to Solomon}} ``Because he loves his people, the \divine{Lord} has placed you as king over them.'' \v{12}Hiram also wrote:

\begin{poetry}
\poeml ``Blessed be the \divine{Lord} God of Israel, who made the heavens and the earth. He gave King David a wise son, who is acquainted with discretion and understanding, and who is building a temple to the \divine{Lord}, as well as a royal palace for himself. \\
\poeml \v{13}Now I'm sending along Hiram-abi,\fnote{Or \fbib{Huram-abi}; the Heb. name means \fbib{Hiram is my father}} a skilled craftsman, who is very creative.\fnote{Or \fbib{discerning}} \v{14}He is the son of a mother from the tribe of Dan, and his father is from Tyre. He's skilled in working with gold, silver, bronze, iron, stone, and timber, as well as in purple, blue,\fnote{Or \fbib{violet}} linen, and crimson materials.\fnote{The Heb. lacks \fbib{materials}} He is skilled in engravings, and can craft any design to which he may be assigned. He will work with your skilled artisans and with all of your craftsmen who have been assigned by my lord David, your father. \\
\poeml \v{15}``So then, may my lord send to his servants the wheat, barley, oil, and wine about which he has spoken. \v{16}We'll cut down the timber you need from Lebanon and transport it to you on rafts by sea to Joppa, so you can move it to Jerusalem.''
\end{poetry}

\v{17}Solomon took a census of all the non-Israeli men\fnote{Lit. \fbib{aliens}} who lived in the land of Israel, after the census that his father David had taken, and 153,600 were counted. \v{18}He conscripted 70,000 of them to do heavy work, 80,000 to quarry in the hill country, and 3,600 men to supervise the people.
\labelchapt{3}
\passage{Temple Construction}
\passageinfo{(1 Kings 6:1-22)}

\chapt{3}
\v{1}So Solomon began construction of the \divine{Lord}'s Temple in Jerusalem on Mount Moriah where the \divine{Lord}\fnote{The Heb. lacks \fbib{the \divine{Lord}}} had appeared to his father David, that is, where David had prepared Ornan the Jebusite's threshing floor. \v{2}He began construction on the second day\fnote{The Heb. lacks \fbib{day}} of the second month of the fourth year of his reign.
\passage{Dimensions of the Temple}

\v{3}These are the foundations that Solomon set in place for God's Temple. The length in terms of the former standard measurements: 60 cubits;\fnote{I.e. about 90 feet; a cubit was about eighteen inches} its width: 20 cubits.\fnote{I.e. about 30 feet; a cubit was about eighteen inches} \v{4}A portico extended in front of the Temple for its entire width of 20 cubits,\fnote{I.e. about 30 feet; a cubit was about eighteen inches} and was\fnote{The Heb. lacks \fbib{was}} 120 cubits\fnote{I.e. about 180 feet; a cubit was about eighteen inches} high. Inside he had it overlaid with pure gold. \v{5}The main room of the Temple was trimmed with a wainscoting composed of cypress wood, overlaid with fine gold ornamented with palm trees and chains. \v{6}The Temple was adorned with precious stones, including gold from the Orient.\fnote{Lit. \fbib{from Parvaim}} \v{7}The Temple was overlaid with gold, including the beams, thresholds, walls, and doors. Cherubim were engraved on the walls. \v{8}With respect to the Most Holy Place in the Temple, its length across the width of the Temple was 20 cubits,\fnote{I.e. about 30 feet; a cubit was about eighteen inches} and its width extended 20 cubits.\fnote{I.e. about 30 feet; a cubit was about eighteen inches}
\passage{Materials of the Temple}

Solomon\fnote{Lit. \fbib{He}} overlaid it with 600 talents\fnote{I.e. about 45,000 pounds; a talent weighed about 75 pounds} of pure gold. \v{9}The gold nails weighed 50 shekels.\fnote{I.e. about 20 ounces; a shekel weighed about 0.4 ounces} He also overlaid the upper rooms with gold. \v{10}He crafted two cherubim from wood, overlaid them with gold, and placed them in the Most Holy Place in the Temple. \v{11}The wingspan of the cherubim was 20 cubits;\fnote{I.e. about 30 feet; a cubit was about eighteen inches} the wing of one, five cubits\fnote{I.e. about seven and a half feet; a cubit was about eighteen inches} long, touched the wall of the Temple, and its other wing, five cubits\fnote{I.e. about seven and a half feet; a cubit was about eighteen inches} long, touched the wing of the other cherub. \v{12}The wing of the other cherub, five cubits\fnote{I.e. about seven and a half feet; a cubit was about eighteen inches} long, touched the opposite\fnote{The Heb. lacks \fbib{opposite}} wall of the Temple and its other wing, five cubits\fnote{I.e. about seven and a half feet; a cubit was about eighteen inches} long, touched the wing of the first\fnote{Lit. \fbib{other}} cherub. \v{13}The wings of these cherubim extended for 20 cubits\fnote{I.e. about 30 feet; a cubit was about eighteen inches} as they stood on their feet and faced the front of\fnote{The Heb. lacks \fbib{the front of}} the Temple. \v{14}He constructed the veil from blue,\fnote{Or \fbib{violet}} purple, crimson, and fine linen, embroidering cherubim on it. \v{15}He also made two pillars 35 cubits\fnote{I.e. about 52 and a half feet; a cubit was about eighteen inches} high for the front of the Temple, topped by a capital that was five cubits\fnote{I.e. about seven and a half feet; a cubit was about eighteen inches} high. \v{16}He crafted chains for the inner sanctuary and placed them on top of the pillars, attaching 100 pomegranates to each of the chains. \v{17}He set up the pillars at the front of the Temple, one on the south side of the entrance\fnote{The Heb. lacks \fbib{of the entrance}} and the other on the north side of the entrance.\fnote{The Heb. lacks \fbib{of the entrance}} He named the south pillar Jachin\fnote{The Heb. name means \fbib{He will establish}} and the north pillar Boaz.\fnote{The Heb. name means \fbib{In him is strength}}
\labelchapt{4}
\passage{Furnishing the Temple}
\passageinfo{(1 Kings 6:23-38; 7:13-51)}

\chapt{4}
\v{1}Solomon\fnote{Lit. \fbib{Then he}} also constructed a bronze\fnote{Or \fbib{brass}} altar 20 cubits\fnote{I.e. about 30 feet; a cubit was about eighteen inches} long, 20 cubits\fnote{I.e. about 30 feet; a cubit was about eighteen inches} wide, and ten cubits\fnote{I.e. about 15 feet; a cubit was about eighteen inches} high. \v{2}He crafted a circular sea of cast metal 10 cubits\fnote{I.e. about 15 feet; a cubit was about eighteen inches} from rim to rim and five cubits\fnote{I.e. about seven and a half feet; a cubit was about eighteen inches} tall. A line 30 cubits\fnote{I.e. about 45 feet, perhaps its external circumference; a cubit was about eighteen inches} long surrounded it. \v{3}Underneath, figurines resembling oxen\fnote{Or \fbib{cattle}; and so throughout the chapter} encircled the circular sea\fnote{Lit. \fbib{encircled it}} beneath it, ten oxen\fnote{The Heb. lacks \fbib{oxen}} every cubit,\fnote{Lit. \fbib{each cubit}} and encircling the sea completely. The oxen were in two rows, cast all at the same time. \v{4}The sea stood on top of twelve oxen, three of which faced to the north, three of which faced to the west, three of which faced to the south, and three of which faced toward the east. The sea was placed on top of the oxen, with all of their hindquarters turned inwards. \v{5}It was a handbreadth\fnote{I.e. about three inches; a handbreadth was about one sixth of a cubit} thick, with its brim fashioned like the brim of a cup. Similar in shape to a lily blossom, it could hold 3,000 baths.\fnote{I.e. about 18,000 gallons; Cf. 1King 7:26, where the volume is given at 2,000 baths} \v{6}Solomon\fnote{Lit. \fbib{He}} also made ten wash basins, placing five on the right side and five on the left. The basins were intended for use to rinse burnt offerings, and the sea was intended for use by the priests to wash in.

\v{7}Solomon\fnote{Lit. \fbib{He}} made ten gold lamp stands as he had been directed and set them in the Temple, five on the south side and five on the north side. \v{8}He also made ten tables and placed them in the Temple, five on the right side and five on the left side. He also constructed 100 gold basins. \v{9}He made the court of the priests, the great court, and doors for the court, overlaying their doors with bronze. \v{10}He set the sea at the southeast corner of the Temple.

\v{11}Hiram-abi\fnote{Lit. Huram; cf. v. 16 and 2Chr 2:13} crafted the pots, shovels, and basins, thus completing the work that he did for King Solomon on the Temple of God; \v{12}that is, the two pillars, the bowls, the two capitals on top of the pillars, the two lattice works that covered the two bowls for the capitals that were on top of the pillars; \v{13}the 400 pomegranate-shaped ornaments for the latticework of the two pillars (each latticework having two rows of ornaments at the bowl-shaped top of each pillar); \v{14}the ten\fnote{Or \fbib{he made the}} stands with their ten basins; \v{15}the large bronze basin called the Sea with the twelve oxen underneath, \v{16}along with its pots, shovels, forks, and all of its other implements that Hiram-abi made from polished bronze for King Solomon and the \divine{Lord}'s Temple. \v{17}The king had them forged in the clay ground between Succoth and Zeredah in the Jordan plain. \v{18}Solomon made so many utensils in such great quantities that the weight of the bronze was never fully recorded.

\v{19}Solomon also made these items for God's Temple: the golden altar, the tables for the Bread of the Presence, \v{20}the lamp stands and their lamps made of pure gold to burn in front\fnote{Or \fbib{burn at the entrance}} of the inner sanctuary, as required, \v{21}the pure gold ornaments in the shape of flowers, the lamps, and the tongs (all made of the purest gold), \v{22}the gold trimming instruments, basins, pans, censers, and the gold door sockets for the inner sanctuary (that is, the Most Holy Place), and for the doors to the main hall of the Temple.
\labelchapt{5}
\passage{The Ark is Placed in the Temple}
\passageinfo{(1 Kings 8:1-11)}

\chapt{5}
\v{1}As soon as Solomon had completed the \divine{Lord}'s Temple, he installed the holy items that had belonged to his father David, including the silver, gold, and all the other items in the treasure rooms of God's Temple. \v{2}Then Solomon called Israel's elders together, including all the leaders of the tribes and families of Israel. They met in Jerusalem to transfer the Ark of the Covenant of the \divine{Lord} from Zion, the City of David. \v{3}All the men of Israel assembled in front of the king during the Festival of Tents\fnote{The Heb. lacks \fbib{of Tents}} that takes place in the seventh month\fnote{I.e. sometime during mid-September to mid-October} of the year.\fnote{The Heb. lacks \fbib{of the year}}

\v{4}As soon as all of Israel's elders had arrived, the descendants of Levi lifted the ark \v{5}and carried it, the tent where God met with his people,\fnote{Lit. \fbib{the Tent of Meeting}} and all of the sacred implements that belonged in the tent. The Levitical priests carried these up to the City of David.\fnote{The Heb. lacks \fbib{to the City of David}} \v{6}King Solomon and all the Israelis who had assembled together proceeded ahead of the ark and sacrificed more sheep and oxen than could be counted or recorded due to the number of sacrifices.\fnote{The Heb. lacks \fbib{due to the number of sacrifices}}

\v{7}The priests transported the Ark of the Covenant of the \divine{Lord} to the place created for it within the inner sanctuary of the Temple, into the Most Holy Place under the wings of the cherubim. \v{8}The wings of the cherubim extended over where the ark and its carrying poles\fnote{Cf. Ex 25:13-15} had been placed, \v{9}but the poles were long enough for their ends to extend to the front of the inner sanctuary, even though they could not be seen from outside. They remain there to this day. \v{10}There was nothing in the ark except for the two tablets that Moses had placed there while Israel was encamped\fnote{The Heb. lacks \fbib{while Israel was encamped}} at Horeb, where the \divine{Lord} made a covenant with the Israelis after he had brought them out of the land of Egypt.

\v{11}After this, the priests vacated the Holy Place. (Meanwhile, all the priests who were participating consecrated themselves, irrespective of their Levitical divisions. \v{12}All the musicians who were descendants of Levi, including Asaph, Heman, Jeduthun, and their sons and relatives wore linen and played cymbals and stringed instruments as they stood east of the altar. Accompanied by 120 priests who played trumpets, \v{13}the trumpeters and musicians played in union, praising and giving thanks to the \divine{Lord}. They praised the \divine{Lord} loudly and sang, ``He is good, and his gracious love is eternal,'' accompanied by the trumpets, cymbals, and other musical instruments.) As they did this,\fnote{The Heb. lacks \fbib{As they did this}} a cloud filled the Temple, that is, the \divine{Lord}'s Temple, \v{14}and the priests were unable to complete their duties because of the cloud, since the glory of the \divine{Lord} had filled God's Temple.
\labelchapt{6}
\passage{Solomon Dedicates the Temple}
\passageinfo{(1 Kings 8:12-21)}

\chapt{6}
\v{1}Then Solomon said, ``The \divine{Lord} has said that he lives shrouded in darkness. \v{2}Now I have constructed a magnificent temple dedicated to you that will serve as a place for you to inhabit forever.''

\v{3}Then the king turned to face the entire congregation of Israel while the congregation of Israel remained standing. \v{4}Then Solomon\fnote{Lit. \fbib{He}} prayed:

\begin{poetry}
\poeml ``Blessed is the \divine{Lord} God of Israel, who made a commitment\fnote{Lit. \fbib{who spoke by his mouth}} to my father David and then personally\fnote{Lit. \fbib{and by his hand}} fulfilled what he had promised when he said:\fnote{Cf. 1Chr 17:5ff} \\
\poeml \v{5}`From the day I brought out my people from the land of Egypt I never chose a city from all the tribes of Israel to build a temple where my name might reside. And I never chose any man to become Commander-in-Chief\fnote{Lit. \fbib{Nagid}; i.e. a senior officer entrusted with dual roles of operational oversight and management authority} over my people Israel. \v{6}But I have chosen Jerusalem, where my name will reside. And I have chosen David to be over my people Israel.' \\
\poeml \v{7}``My father David wanted to build a temple for the name of the \divine{Lord} God of Israel. \v{8}The \divine{Lord} told my father David: \\
\poeml `Therefore, since you determined\fnote{Lit. \fbib{since it was in your heart}} to build a temple for my name, you acted well, because it was your choice\fnote{Lit. \fbib{because it was in your heart}} to do so. \v{9}Nevertheless, you are not to build the Temple, but your son who will be born\fnote{Lit. \fbib{will come from your loins}} to you is to build a temple for my name.' \\
\poeml \v{10}``The \divine{Lord} has brought to fulfillment\fnote{Lit. \fbib{has caused to stand up}} what he promised, and now here I stand,\fnote{MT verb is a pun on the verb \fbib{brought to fulfillment}} having succeeded my father David to sit on the throne of Israel, as the \divine{Lord} promised. I have built the Temple for the name of the Lord \divine{God} of Israel. \v{11}I have placed in it the ark in which the covenant that the \divine{Lord} made with the Israelis is stored.''
\end{poetry}
\passage{Solomon's Prayer of Dedication}
\passageinfo{(1 Kings 8:22-53)}

\v{12}Then Solomon\fnote{Lit. \fbib{he}} took his place in front of the \divine{Lord}'s altar in the presence of the entire congregation of Israel and spread out his hands. \v{13}Solomon had a bronze platform constructed five cubits\fnote{I.e. about seven and a half feet; a cubit was about eighteen inches} square and three cubits\fnote{I.e. about four and a half feet; a cubit was about eighteen inches} high. He had it erected in the middle of the courtyard, and stood on it. Then he knelt down on his knees in front of the entire congregation of Israel, spread out his hands toward heaven, \v{14}and said:

\begin{poetry}
\poeml ``\divine{Lord} God of Israel, there is no one like you, God of heaven and earth, who watches over\fnote{Or \fbib{who keeps}} his covenant, showing gracious love to your servants who live their lives in your presence\fnote{Lit. \fbib{who walk before you}} with all their hearts. \v{15}It is you, \divine{Lord} God,\fnote{The Heb. lacks \fbib{It is you, \divine{Lord} God}} who has kept your promise to my father, your servant David, that you made to him. Indeed, you made a commitment\fnote{Lit. \fbib{you spoke by your mouth}} to my father David and then personally fulfilled\fnote{Lit. \fbib{and by your hand full}} what you had promised today. \\
\poeml \v{16}``Now therefore, \divine{Lord} God of Israel, keep your promise that you made\fnote{Lit. \fbib{spoke}} to my father, your servant David, when you said, `You are to not lack a man to sit on the throne of Israel,\fnote{Cf. 1King 2:4; 2Chr 7:18} if only your descendants will watch their lives,\fnote{Lit. \fbib{ways}} to live according to my Law, just as you have lived\fnote{Lit. \fbib{walked}} in my presence.'\fnote{Or \fbib{have walked before me}} \\
\poeml \v{17}``Now therefore, \divine{Lord} God of Israel, may your promise that you made\fnote{Lit. \fbib{spoke}} to your servant David be fulfilled{\ldots} \v{18}and yet, will God truly reside on earth with human beings? Look! Neither the sky nor the highest heaven can contain you! How much less this Temple that I have built! \v{19}Pay attention to the prayer of your servant and to his request, \divine{Lord} my God, and listen to the cry and prayer that your servant is praying in your presence. \v{20}Let your eyes always look toward this Temple day and night, toward the location where you have said you would place your name. Listen to the prayer that your servant prays in this direction.\fnote{Lit. \fbib{prays toward this place}} \v{21}Listen to the requests from your servant and from your people Israel as they pray in this direction,\fnote{Lit. \fbib{pray toward this place}} and listen from the place where you reside---from heaven!---then hear and forgive. \\
\poeml \v{22}``If a man sins against his neighbor and he is required to take an oath, and he then comes to take an oath in front of your altar in this Temple, \v{23}then listen from heaven, act, and judge your servants, recompensing the wicked by bringing back to him the consequences of his choices\fnote{Lit. \fbib{by bringing his way upon his head}} and by justifying the righteous by recompensing him according to his righteousness. \\
\poeml \v{24}``If your people Israel are defeated in a battle with\fnote{Lit. \fbib{defeated before}} their enemy because they have sinned against you, when they return to you\fnote{The Heb. lacks \fbib{to you}} and confess to you,\fnote{Lit. \fbib{confess your name}} pray, and in this Temple they ask you to show grace to them, \v{25}then hear from heaven, forgive the sin of your people Israel, and return them to the soil\fnote{Or \fbib{land}} that you gave to them and to their ancestors. \\
\poeml \v{26}``When the skies remain closed, and there is no rain because they have sinned against you, and they pray in the direction of this place, confessing your name and turning from their sin when you afflict them,\fnote{So MT; LXX reads \fbib{you bring them low}} \v{27}then hear in heaven and forgive the sin of your servants and of your people Israel. Indeed, teach them the best way to live and send rain on your land that you have given to your people as an inheritance. \\
\poeml \v{28}``If a famine comes to the land, or if there comes plant diseases, mildew, locusts, or grasshoppers,\fnote{Or \fbib{caterpillars}} or if their enemies attack them in their settlements of the land, no matter what the epidemic or illness is, \v{29}whatever prayer or request is made, no matter whether it's made by a single man or by all of your people Israel, each praying out of his own illness and anguish and stretching out their hands toward this Temple, \v{30}then hear from heaven, the place where you reside, and forgive, repaying each person according to all of his ways, since you know their hearts---for you alone know the hearts of human beings--- \v{31}so they will fear you and live life\fnote{Lit. \fbib{and walk in}} your way as long as they live in the land that you have given to our ancestors. \\
\poeml \v{32}``Now concerning the foreigner who is not from your people Israel, when he comes from a land far away for the sake of your great name, your mighty acts,\fnote{Lit. \fbib{hand}} and your obvious power,\fnote{Lit. \fbib{your outstretched arm}} when they come and pray in the direction of this Temple, \v{33}then hear from heaven where you reside, and do whatever the foreigner asks of you, so that all the people of the earth may know your name, fear you as do your people Israel, and so they may know that this Temple that I have built is called by your name. \\
\poeml \v{34}``When your people go out to war against their enemies, no matter what way you send them, and they pray to you in the direction of this city that you have chosen and in the direction of the Temple that I have built for your name, \v{35}then hear their prayer and their request from heaven, and fight for their cause. \\
\poeml \v{36}``When they sin against you---because there isn't a single human being who doesn't sin---and you become angry with them and deliver them over to their enemy, who takes them away captive to a land that's near or far away, \v{37}if they turn their hearts back to you\fnote{The Heb. lacks \fbib{back to you}} in the land where they have been taken captive, repent, and pray to you---even if they do so in the land where they have been taken captive---confessing, `We have sinned, we have committed abominations, and practiced wickedness,' \v{38}if they return to you with all of their heart and with all of their soul in the land where they have been taken captive, as they pray in the direction of their land that you have given to their ancestors and to the city that you have chosen, and to the Temple that I have built for your name, \v{39}then hear their prayer and requests from heaven, where you reside, and fight for their cause, forgiving your people who have sinned against you. \\
\poeml \v{40}``And now, my God, please let your eyes be open and your ears attentive to the prayers that are uttered in\fnote{The Heb. lacks \fbib{that are uttered in}} this place. \\
\poeml \v{41}``And now may the \divine{Lord} God arise, to your place of rest, you, and the ark of your power! Let your priests, \divine{Lord} God, be clothed with salvation, and cause your godly ones to find their joy in what is good. \\
\poeml \v{42}``\divine{Lord} God, do not turn your face away from your anointed one.\fnote{Or \fbib{your Messiah}} Remember your gracious love to your servant David.''
\end{poetry}
\labelchapt{7}
\passage{The Glory of God Fills the Temple}
\passageinfo{(1 Kings 8:62-66)}

\chapt{7}
\v{1}As soon as Solomon finished his prayer, fire descended from heaven and burned up the burnt offerings and sacrifices, and the glory of the \divine{Lord} filled the Temple. \v{2}The priests could not enter into the Temple because the glory of the \divine{Lord} had filled the \divine{Lord}'s Temple. \v{3}When all of the Israelis saw the fire coming down and the glory of the \divine{Lord} resting\fnote{The Heb. lacks \fbib{resting}} on the Temple, they bowed down with their faces\fnote{Lit. \fbib{nostrils}} to the ground on the pavement, worshipped, and gave thanks to the \divine{Lord},

\begin{poetry}
\poeml ``Because he is good; \\
\poemll    because his gracious love is eternal.''
\end{poetry}

\v{4}Then the king and all the people kept on offering sacrifices in the presence of the \divine{Lord}. \v{5}King Solomon offered a sacrifice of 22,000 oxen and 120,000 sheep, which is how\fnote{The Heb. lacks \fbib{which is how}} the king and all of the people dedicated God's Temple. \v{6}The priests stood in waiting at their assigned places, along with the descendants of Levi who carried musical instruments used in service to the \divine{Lord} that King David had made for giving thanks to the \divine{Lord}---because his gracious love is eternal---whenever David, accompanied by priests\fnote{Lit. David by their hand, that is, the priests,} sounding trumpets, offered praises while all of Israel stood in the assembly.\fnote{The Heb. lacks \fbib{in the assembly}}

\v{7}Solomon also dedicated the middle of the court in front of the \divine{Lord}'s Temple by offering there burnt offerings and fat from peace offerings because the bronze altar that Solomon had made could not contain the burnt offerings, grain offerings, and fat portion offerings. \v{8}At that time Solomon also held a week-long festival attended by all of Israel. The assembly was very large, and included people from as far away as Lebo-hamath\fnote{I.e. the principal city of Syria to the north of Israel in the Orontes Valley.} to the Wadi\fnote{I.e. a seasonal stream or river that channels water during rain seasons but is dry at other times} of Egypt.\fnote{Or \fbib{Brook of Egypt}; the southwestern-most border of Israel} \v{9}On the day after the festival ended,\fnote{Lit. \fbib{On the eighth day}} they convened a solemn assembly, because they had been dedicating the altar for seven days and observing the festival for seven days. \v{10}On the twenty-third day of the seventh month, King Solomon\fnote{Lit. \fbib{he}} sent the people back home,\fnote{Lit. \fbib{back to their tents}} and they returned\fnote{The Heb. lacks \fbib{and they returned}} rejoicing and in good spirits because of the goodness that the \divine{Lord} had shown to David, to Solomon, and to his people Israel. \v{11}And so Solomon completed the \divine{Lord}'s Temple, bringing to completion everything that he had planned on doing for the \divine{Lord}'s Temple and for his own palace.
\passage{God Appears to Solomon}
\passageinfo{(1 Kings 9:1-9)}

\v{12}Later, the \divine{Lord} appeared to Solomon during the night and told him:

\begin{poetry}
\poeml ``I have heard your prayer and have chosen this place for a sacrificial temple to me. \v{13}Whenever I close the skies so there is no rain, or whenever I command locusts to lay waste to the land, or whenever I send epidemics among my people, \v{14}when my people humble themselves---the ones who are called by my name---and pray, seek me,\fnote{Lit. \fbib{seek my face}} and turn away from their evil practices, I myself will listen from heaven, I will pardon their sins, and I will restore their land. \\
\poeml \v{15}``Now therefore my eyes will remain open and my ears will remain listening to the prayers that are offered in this place. \v{16}Furthermore, I have chosen and have set apart for myself\fnote{The Heb. lacks \fbib{for myself}} this Temple, intending my name to reside there forever. My eyes and my heart will reside there every day. \v{17}Now as for you, if you commune with me like your father did, doing everything that I have commanded you, including obeying my statutes and my legal decisions, \v{18}then I will make your royal throne secure, just as I agreed to do for your father David when I said, `You are to not lack a man to rule over Israel.'\fnote{Cf. 1King 2:4; 2Chr 6:16} \\
\poeml \v{19}``But if you\fnote{MT pronoun is pl.} turn away and abandon my statutes and my commands that I have given you, and if you\fnote{MT pronoun is pl.} walk away to serve other gods and worship them, \v{20}then I will tear them up by the roots from the ground that I had given them! And as for this Temple that I have set apart for my name, I will throw it out of my sight and make it the butt of jokes\fnote{Lit. \fbib{it an object of mockery}} and a means of ridicule among people worldwide! \\
\poeml \v{21}``Furthermore, even though this Temple seems so exalted, everyone who passes by it will be so astounded that they will ask, `Why did the \divine{Lord} do this to this land and to this Temple?' \v{22}They will answer, `Because they abandoned the \divine{Lord} God of their ancestors, who brought them from the land of Egypt, adopted other gods, worshipped them, and served them, therefore the \divine{Lord}\fnote{Lit. \fbib{he}} has brought all of this disaster on them.'\,''
\end{poetry}
\labelchapt{8}
\passage{Solomon's Accomplishments}
\passageinfo{(1 Kings 9:10-28)}

\chapt{8}
\v{1}It took Solomon 20 years to build the \divine{Lord}'s Temple and his own palace. \v{2}During this time, he also rebuilt the towns that Hiram had restored to him, and he settled Israelis in them. \v{3}After this, Solomon traveled to Hamath-zobah and captured it. \v{4}Then he rebuilt Tadmor in the desert, along with supply centers\fnote{Lit. \fbib{cities}} that he had built in Hamath. \v{5}He also built upper and lower Beth-horon as fortified cities, installing\fnote{The Heb. lacks \fbib{installing}} walls, gates, and bars, \v{6}and he rebuilt Baalath and its supply centers\fnote{Lit. \fbib{cities}} that belonged to Solomon, along with all the cities that he utilized to garrison his chariots and cavalry forces. Solomon was pleased also to build in Jerusalem, in Lebanon, and in every territory\fnote{Or \fbib{land}} that he controlled.
\passage{Conscripted Laborers}

\v{7}All of the survivors who remained living in the land but who were not Israelis (including Hittites, Amorites, Perizzites, Hivites, and Jebusites) \v{8}were descendants of the nations whom the people of Israel had not eliminated. Solomon put them to work as conscripted laborers, which they continue to do\fnote{The Heb. lacks \fbib{which they continue to do}} to this day. \v{9}However, Solomon never made conscripted laborers from among the Israelis, but they did serve as his army, as his chief captains, and as commanders in charge of his chariots and cavalry. \v{10}King Solomon appointed 250 chief officers to command his army.\fnote{Or \fbib{people}} \v{11}Later, Solomon moved Pharaoh's daughter from the City of David to the palace that he had constructed to house her, because he reasoned, ``My wife isn't going to live in the palace where King David of Israel lived, because wherever the ark of the \divine{Lord} entered is holy.''

\v{12}Solomon offered burnt offerings to the \divine{Lord} on the \divine{Lord}'s altar that he had built in front of the porch of the Temple,\fnote{The Heb. lacks \fbib{of the temple}} \v{13}acting\fnote{The Heb. lacks \fbib{acting}} in compliance with the daily rule by offering them in conformity to commands issued by Moses for the Sabbaths, the New Moons, the three annual festivals (the Festival of Unleavened Bread, the Festival of Weeks, and the Festival of Tents). \v{14}Following proscriptions laid down by his father David, Solomon\fnote{Lit. \fbib{he}} appointed divisions of priests for their service as well as descendants of Levi for duties of praise and ministry before the priests consistent with the daily rules. Furthermore, because David, the man of God, had commanded it, Solomon\fnote{Lit. \fbib{he}} also appointed gatekeepers to serve by divisions at every gate of the Temple.\fnote{The Heb. lacks \fbib{of the temple}} \v{15}They scrupulously adhered to\fnote{Lit. \fbib{They did not depart from}} the orders issued by the king to the priests and descendants of Levi in everything, including matters pertaining to operation of\fnote{The Heb. lacks \fbib{to operation of}} the treasuries.
\passage{Work on the Temple is Completed}

\v{16}And so Solomon completed all of the work, from the day that the foundation stone of the \divine{Lord}'s Temple was laid\fnote{The Heb. lacks \fbib{was laid}} until the \divine{Lord}'s Temple was completely finished. \v{17}After this, Solomon visited Ezion-geber and Elath at the seashore in the land of Edom. \v{18}Hiram sent Solomon\fnote{Lit. \fbib{him}} ships and servants who were expert mariners, and they sailed with Solomon's servants to Ophir,\fnote{Or \fbib{to a source of fine gold}; cf. 1Chr 29:4} where they brought back 450 talents\fnote{I.e. about 33,750 pounds; a talent weighed about 75 pounds} of gold for Solomon.
\labelchapt{9}
\passage{The Queen of Sheba Visits Jerusalem}
\passageinfo{(1 Kings 10:1-13)}

\chapt{9}
\v{1}When the queen of Sheba heard about Solomon's reputation, she traveled to Jerusalem and tested him\fnote{Lit. \fbib{Solomon}} with difficult questions. She brought along a large retinue, camels laden with spices, and lots of gold and precious stones. Upon her arrival, she spoke with Solomon about everything that was on her mind.\fnote{Lit. \fbib{heart}} \v{2}Solomon answered all of her questions. Because nothing was hidden from Solomon, he hid nothing from her. \v{3}When the queen of Sheba had seen Solomon's wisdom for herself, the palace that he had built, \v{4}the food set at his table, his servants who waited on him, his ministers in attendance and how they were dressed, his personal staff\fnote{Lit. \fbib{his cupbearers}} and how they were dressed, and even his personal stairway by which he went up to the \divine{Lord}'s Temple, she was breathless!

\v{5}``Everything I heard about your wisdom and what you have to say is true!'' she gasped, \v{6}``but I didn't believe it at first! But then I came here and I've seen it for myself! It's amazing! I wasn't told half of what's really great about your wisdom. You're far better in person than what the reports have said about you! \v{7}How blessed are your staff! And how blessed are your employees,\fnote{Lit. \fbib{servants}} who serve you continually and get to listen to your wisdom! \v{8}Blessed be the \divine{Lord} your God, who is delighted with you! He set you in place on his throne to be king for the \divine{Lord} your God. He made you king over them so you could carry out justice and implement righteousness, because your God loves Israel and intends to establish them\fnote{Lit. \fbib{him}; i.e. the nation personified as an individual} forever.''

\v{9}Then she gave the king 120 talents\fnote{I.e. about 9,000 pounds; a talent weighed about 75 pounds} of gold, a vast quantity of spices, and precious stones. There were no spices comparable to those that the queen of Sheba gave to King Solomon. \v{10}Hiram's servants and Solomon's servants, who brought gold from Ophir,\fnote{Or \fbib{from a source of fine gold}; cf. 1Chr 29:4} also presented algum wood\fnote{Or \fbib{presented Juniper trees}} and other precious stones. \v{11}The king used the algum wood\fnote{Or \fbib{the Juniper trees}} to have steps made for the \divine{Lord}'s Temple and for the royal palace, as well as lyres and harps for the choir,\fnote{Lit. \fbib{singers}} and nothing like that wood\fnote{The Heb. lacks \fbib{wood}} had been seen before in the territory of Judah. \v{12}In return, King Solomon gave the queen of Sheba everything she wanted and requested in addition to what she had brought for the king. Afterward, she returned to her own land, accompanied by her servants.
\passage{Solomon's Wealth}
\passageinfo{(1 Kings 10:14-29; 2 Chronicles 1:14-17)}

\v{13}Solomon received in any given year about 666 talents\fnote{I.e. about 49,950 pounds; a talent weighed about 75 pounds} of gold, \v{14}not including revenue from traders and merchants. In addition, all the kings of Arabia and the governors of the nation brought gold and silver to Solomon. \v{15}King Solomon made 200 large shields of beaten gold, overlaying each shield with the gold from 600 gold pieces,\fnote{MT does not identify the individual unit of measure} \v{16}and 300 shields from beaten gold, overlaying each shield with the gold from 300 gold pieces.\fnote{MT does not identify the individual unit of measure} The king put them in his palace in the Lebanon forest. \v{17}The king also made a great ivory throne and overlaid it with pure gold. \v{18}Six steps led up to the throne. A golden footstool was attached to the throne, which had armrests on each side of the seat and two lions standing on either side of each armrest. \v{19}Twelve lions were placed on both sides of the six steps leading to the throne,\fnote{The Heb. lacks \fbib{leading to the throne}} and nothing comparable was made for any other\fnote{The Heb. lacks \fbib{other}} kingdom. \v{20}All of King Solomon's drinking vessels were made of\fnote{The Heb. lacks \fbib{made of}} gold, and all the vessels in his palace in the Lebanon forest were made of\fnote{The Heb. lacks \fbib{made of}} pure gold. Silver was never considered to be valuable during the lifetime of Solomon, \v{21}because the king had ships that sailed to Tarshish accompanied by Hiram's servants. Once every three years ships from Tarshish returned, bringing gold, silver, ivory, apes, and peacocks.

\v{22}As a result, King Solomon became greater than all the kings of the earth in regards to wealth and wisdom. \v{23}All the kings of the earth continued to seek audiences with Solomon so they could hear the wise things that God had put in his heart. \v{24}Everyone kept on bringing gifts on an annual basis, including items made of silver and gold, garments, myrrh, spices, horses, and mules. \v{25}Solomon had 4,000 stalls for horses and chariots, along with 12,000 cavalry soldiers. He stationed them in various chariot cities and with the king in Jerusalem. \v{26}King Solomon\fnote{Lit. \fbib{He}} ruled over all the kings from the Euphrates\fnote{The Heb. lacks \fbib{Euphrates}} River west\fnote{The Heb. lacks \fbib{west}} to the land of the Philistines and as far south as the boundary with Egypt.

\v{27}The king made silver as common as\fnote{The Heb. lacks \fbib{as common as}} stones in Jerusalem, and made cedar trees as abundant as sycamore trees in the Shephelah.\fnote{I.e. the verdant central lowlands of Israel; cf. Josh 10:40} \v{28}They also kept bringing horses to Solomon from Egypt and from all of the surrounding\fnote{The Heb. lacks \fbib{surrounding}} countries.
\passage{The Death of Solomon}
\passageinfo{(1 Kings 11:41-43)}

\v{29}Now the rest of Solomon's accomplishments, from first to last, are written in the records of Nathan the prophet, in the prophecy of Ahijah the Shilonite, and in the visions of Iddo the seer pertaining to Nebat's son Jeroboam, are they not? \v{30}Solomon reigned for 40 years in Jerusalem over all of Israel. \v{31}Then Solomon died, as had\fnote{Lit. \fbib{Solomon slept with}; and so throughout the book} his ancestors, and his son Rehoboam reigned in his place.
\labelchapt{10}
\passage{Rehoboam's Foolish Choices}
\passageinfo{(1 Kings 12:1-19)}

\chapt{10}
\v{1}Rehoboam traveled to Shechem, because all of Israel went there to install him as king. \v{2}Nebat's son Jeroboam heard about it in Egypt, where he had fled to get away from Solomon the king. Jeroboam returned from Egypt \v{3}after being summoned. When Jeroboam and all of Israel arrived, they spoke to Rehoboam, \v{4}``Your father made our burdens unbearable.\fnote{Lit. \fbib{our yoke heavy}} Therefore you must lighten your father's requirements and his heavy burden that he placed on us, and we'll serve you.''

\v{5}``Come back again in three days,'' Rehoboam\fnote{Lit. \fbib{He}} told them. So the people left \v{6}while King Rehoboam conferred with his advisors who had worked with his father Solomon during his administration. He asked them, ``What is your advice as to what response I should return to these people?''

\v{7}In reply, they told him, ``If you will be kind to this people, please them, and speak appropriately to them with kind words, they'll serve you forever.''

\v{8}But Rehoboam\fnote{Lit. \fbib{he}} ignored the counsel that his elder advisors had given him. Instead, he consulted the younger men who had grown up with him and worked for\fnote{Lit. \fbib{and stood before}} him. \v{9}As a result, he asked them, ``What's your advice, so we can give an answer to these people who have asked me, `Please lighten the burden that your father put on us'?''

\v{10}``This is what you should tell the people who asked you `Your father made our burden heavy, but you must make it lighter for us!'\,'' the young men who had grown up with Rehoboam\fnote{Lit. \fbib{him}} replied. ``Tell them `My little finger will be thicker than my father's whole body!\fnote{Lit. \fbib{father's loin}} \v{11}Not only that, but since my father loaded you down heavily, I'm going to add to that burden. If my father disciplined you with whips, I'm going to do so\fnote{The Heb. lacks \fbib{to do so}} with scorpions!'\,''

\v{12}So Jeroboam and all the people went back to Rehoboam on the third day, just as they had been directed when the king said, ``Come back again in three days.'' \v{13}But the king answered them strictly and ignored the counsel of his elders. \v{14}Instead, Rehoboam\fnote{Lit. \fbib{he}} spoke to them along the lines of what the younger men suggested. He told them ``My father burdened you heavily, but I will add to that burden. If my father disciplined you with whips, I will, too---with scorpions!''

\v{15}The king would not listen to the people because the turn of events was from God, so that the \divine{Lord} might fulfill his prediction that he spoke through Nebat's son Ahijah the Shilonite. \v{16}All of Israel---since the king wasn't going to listen to them---the people responded to the king, ``What's the point in following David? We have no inheritance in the descendants of Jesse. Let's go home,\fnote{Lit. \fbib{Each man to his tent}} Israel! David, take care of your own household!' So all of Israel left for home.\fnote{Lit. \fbib{left for their tents}} \v{17}And so Rehoboam ruled over the Israelis who lived in the cities of Judah.

\v{18}King Rehoboam sent Hadoram, who was in charge of conscripted labor, but the Israelis stoned him to death, and King Rehoboam had to jump in his chariot and flee back in a hurry to Jerusalem. \v{19}That's how Israel came to be in rebellion against David's dynasty to this day.
\labelchapt{11}
\passage{Rehoboam Reigns over Judah Only}
\passageinfo{(1 Kings 12:20-24)}

\chapt{11}
\v{1}When Rehoboam returned to Jerusalem, he gathered together 180,000 specially chosen soldiers from the households of Judah and Benjamin to fight against Israel and restore the kingdom to Rehoboam. \v{2}But a message from the \divine{Lord} came to Shemaiah, a man of God: \v{3}``Tell Solomon's son Rehoboam, king of Judah and all of Israel in Judah and Benjamin: \v{4}`This is what the \divine{Lord} says: ``You are not to fight or even to approach your relatives in battle. Every soldier is to return to his own home, for this development comes from me.''\,'\,'' So they listened to what the \divine{Lord} had to say and called off their attack on Jeroboam.

\v{5}Rehoboam continued to live in Jerusalem and built defensive fortification cities throughout Judah, \v{6}including Bethlehem, Etam, Tekoa, \v{7}Beth-zur, Soco, Adullam, \v{8}Gath, Mareshah, Ziph, \v{9}Adoraim, Lachish, Azekah, \v{10}Zorah, Aijalon, and Hebron. These were all fortified cities throughout Judah and Benjamin. \v{11}He also strengthened the fortified cities, assigned officers to them, and stockpiled food, oil, and wine. \v{12}He also stockpiled shields and spears in every city and fortified them greatly to secure his rule over Judah and Benjamin.
\passage{The Priests and Levites Support Rehoboam}
\passageinfo{(1 Kings 14:21-24)}

\v{13}The priests and descendants of Levi throughout Israel also supported him in their districts, \v{14}because the descendants of Levi left their pasture lands and their property to live in Judah and Jerusalem, since Jeroboam and his sons had excluded them from participating in priestly services to the \divine{Lord}. \v{15}Jeroboam had appointed his own priests to serve at the high places and to serve the satyrs\fnote{Lit. \fbib{goat idols}} and calves that he had made. \v{16}As a result, anyone from all of the tribes of Israel who was determined to seek the \divine{Lord} God of Israel followed the descendants of Levi\fnote{Lit. \fbib{followed them}} to Jerusalem so they could sacrifice to the \divine{Lord} God of their ancestors, \v{17}and they continued to strengthen the kingdom of Judah, supporting Solomon's son Rehoboam for three years, by living\fnote{Lit. \fbib{by walking in}} the way David and Solomon did for three years.
\passage{Rehoboam's Wives and Children}

\v{18}Rehoboam married Mahalath, the daughter of David's son Jerimoth, along with Abihail, the daughter of Jesse's son Eliab, \v{19}who bore him these sons: Jeush, Shemariah, and Zaham. \v{20}After this he married Absalom's daughter Maacah, who bore him Abijah, Attai, Ziza, and Shelomith. \v{21}Rehoboam loved Absalom's daughter Maacah more than he did all of his wives and mistresses. (He married eighteen wives and 60 concubines, fathering 28 sons and 60 daughters.) \v{22}Later, Rehoboam appointed Abijah, his son from Maacah, as senior family leader among his brothers, since he intended to establish Abijah\fnote{Lit. \fbib{him}} as king. \v{23}Rehoboam\fnote{Lit. \fbib{He}} was wise to distribute some his children throughout all of the territories of Judah and Benjamin, placing them in all of the fortified cities. He allotted them abundant supplies of food and sought many wives for them.\fnote{The Heb. lacks \fbib{for them}}
\labelchapt{12}
\passage{Shishak Invades Judah}
\passageinfo{(1 Kings 14:25-28)}

\chapt{12}
\v{1}At the height of his power, after he had consolidated his rule, Rehoboam abandoned the \divine{Lord}'s Law, along with all of Israel with him. \v{2}Because he had been unfaithful to the \divine{Lord}, during the fifth year of King Rehoboam's reign, King Shishak of Egypt attacked Jerusalem \v{3}with 1,200 chariots and 60,000 cavalry. The Lubim, Sukkiim, and the Ethiopians who invaded from Egypt with Shishak\fnote{Lit. \fbib{him}} were innumerable. \v{4}Shishak\fnote{Lit. \fbib{He}} captured the fortified cities of Judah and invaded as far as Jerusalem.

\v{5}Right then, Shemaiah the prophet approached Rehoboam and the princes of Judah who had gathered together in Jerusalem because of Shishak, and he told them, ``This is what the \divine{Lord} says: `You abandoned me, so I've abandoned you to Shishak.'\,''

\v{6}In response, the princes of Israel and the king humbled themselves and declared, ``The \divine{Lord} is righteous.''

\v{7}When the \divine{Lord} observed that they had humbled themselves, the \divine{Lord} spoke to Shemaiah, ``They have humbled themselves, so I won't destroy them. Instead, I'll grant them some deliverance by not pouring out my indignation on Jerusalem, using Shishak to do it. \v{8}Nevertheless, they will become his slaves so they may learn to differentiate between what it means to serve me and to serve the kingdoms of these nations.'' \v{9}So King Shishak of Egypt invaded Jerusalem and looted the treasure stores in the \divine{Lord}'s Temple and in the royal palace. He took everything, including the golden shields that Solomon had made. \v{10}After this, King Rehoboam made shields out of bronze to take their place, committing them to the care and custody of the commanders of those who guarded the entrance to the royal palace. \v{11}As often as the king entered the \divine{Lord}'s Temple, the guards came and transported the shields\fnote{Lit. \fbib{transported them}} to the Temple\fnote{The Heb. lacks \fbib{to the temple}} and then brought them back to the guard's quarters. \v{12}After he had humbled himself, the \divine{Lord} stopped being angry with him, and did not destroy Rehoboam\fnote{Lit. \fbib{him}} completely. Furthermore, conditions became good in Judah.
\passage{The Death of Rehoboam}
\passageinfo{(1 Kings 14:21-22; 29-31)}

\v{13}King Rehoboam consolidated his reign in Jerusalem. Rehoboam was 41 years old when he began to reign, and he reigned for seventeen years in Jerusalem, the city that that \divine{Lord} had chosen from all the tribes of Israel in which to establish his name. Rehoboam's mother was Naamah from Ammon. \v{14}He practiced evil by not setting his heart to seek the \divine{Lord}. \v{15}Now Rehoboam's accomplishments, from first to last, are written in the records of Shemaiah the prophet and of Iddo the seer, enrolled by genealogy, are they not? \v{16}Later, Rehoboam died, as had his ancestors, and his son Abijah became king to replace him.
\labelchapt{13}
\passage{Abijah Succeeds Rehoboam}
\passageinfo{(1 Kings 15:1-9)}

\chapt{13}
\v{1}During the eighteenth year of the reign of\fnote{The Heb. lacks \fbib{the reign of}} King Jeroboam, Abijah became king over Judah. \v{2}He reigned for three years in Jerusalem. His mother was Uriel's daughter Micaiah from Gibeah.

A war started between Abijah and Jeroboam. \v{3}Abijah started the battle with an army of 400,000 specially chosen valiant soldiers, but Jeroboam opposed him with 800,000 specially chosen valiant soldiers. \v{4}Abijah stood on Mount Zemaraim in the hill country of Ephraim and announced:

\begin{poetry}
\poeml ``Listen to me, Jeroboam and Israel! \v{5}Don't you know that the \divine{Lord} God of Israel assigned the kingship over Israel to David and his descendants forever by a salt covenant?\fnote{Cf. Lev 2:13; Num 18:19} \v{6}Even so, Nebat's son Jeroboam, who used to serve David's son Solomon, rose in rebellion against his own master! \v{7}Useless troublemakers\fnote{Lit. \fbib{sons of Belial}} soon gathered around him, who turned out to be too strong for Rehoboam, because he was young, timid, and unable to withstand them. \\
\poeml \v{8}``So now you think you'll be able to withstand the \divine{Lord}'s kingdom as controlled by David's descendants, just because you have a large crown and have brought with you the golden calves that Jeroboam made for you as gods. \v{9}Haven't you already driven away the \divine{Lord}'s priests, the descendants of Aaron and the descendants of Levi? Haven't you established your own priests like the people of other\fnote{The Heb. lacks \fbib{other}} lands? \\
\poeml \v{10}``Now as far as we're concerned, the \divine{Lord} is our God, and we haven't abandoned him. The descendants of Aaron are ministering to the \divine{Lord} as priests, and the descendants of Levi continue their work. \v{11}Every morning and evening, they're offering burnt offerings and fragrant incense to the \divine{Lord}, the showbread is set out on the pure table, and they take care of the golden lamp stand so its lamps can continue to burn every evening. We continue to be faithful over what the \divine{Lord} our God entrusted to us, but you have abandoned him. \v{12}Now listen! God is with us to lead us, and his priests are about to sound their battle trumpets against you. Descendants of Israel, don't fight against the \divine{Lord} God of your ancestors, because you won't succeed!''
\end{poetry}

\v{13}But Jeroboam had sent an ambush to attack from the rear, so Israel was in front of Judah, with the ambush set in place behind them. \v{14}When the army of\fnote{The Heb. lacks \fbib{the army of}} Judah turned around to look, they were being attacked from both front and rear, so they cried out to the \divine{Lord} while the priests sounded their trumpets. \v{15}Then the army of Judah sounded a war cry, and God routed Jeroboam and the entire army of Israel in front of Abijah and Judah. \v{16}When the descendants of Israel ran away from the army of Judah, God handed them over to the army of Judah. \v{17}Abijah and his army defeated them in a tremendous slaughter that resulted in 500,000 special forces from Israel being slain. \v{18}And so the descendants of Israel were defeated at that time. The descendants of Judah were victorious because they trusted in the \divine{Lord} God of their ancestors. \v{19}After this Abijah pursued Jeroboam and captured Bethel and its villages, Jeshanah and its villages, and Ephron and its villages.
\passage{Jeroboam's Death and Asa's Reign in Judah}

\v{20}Jeroboam never recovered his strength for the rest of Abijah's life. The \divine{Lord} struck Jeroboam,\fnote{Lit. \fbib{him}} and he died, \v{21}but Abijah continued to grow more powerful. He took fourteen wives for himself and fathered 22 sons and sixteen daughters. \v{22}The rest of Abijah's accomplishments, his lifestyle and his memoirs are recorded in the Midrash\fnote{Or \fbib{Commentary}} of the Prophet Iddo.\chapt{14}
\v{1}\fnote{This v. is 13:23 in MT}Then Abijah died, as had his ancestors, and he was buried in the City of David. Abijah's\fnote{Lit. \fbib{His}} son Asa reigned in his place, and during his lifetime the land enjoyed rest for ten years.
\labelchapt{14}
\passage{Asa Chooses to do What is Right}
\passageinfo{(1 Kings 15:9-15)}

\v{2}\fnote{This v. is 14:1 in MT, and so throughout the chapter}Asa practiced what the \divine{Lord} his God considered to be right \v{3}by removing the foreign altars and high places, tearing down the sacred pillars, cutting down the Asherim,\fnote{I.e. cultic pillars erected in worship to Canaanite deities; or \fbib{groves}} and \v{4}commanding Judah to seek the \divine{Lord} God of their ancestors and to keep the Law and the commandments. \v{5}He also removed the high places and incense altars from all of the cities of Judah. As a result, the kingdom enjoyed rest under Asa's leadership.\fnote{Lit. \fbib{under him}}

\v{6}Asa\fnote{Lit. \fbib{He}} built fortified cities throughout Judah while the land lay undisturbed, because the \divine{Lord} had given him peace so that no one went to war against him during those years. \v{7}He had told Judah, ``Let's build up these cities, surrounding them with walls, towers, gates, and bars. The land still belongs to us, because we have kept on seeking the \divine{Lord} our God. We have sought him out, and he has given us rest all around us.'' So the people built and prospered. \v{8}Asa kept a standing army of 300,000 soldiers from Judah equipped with large shields and spears, as well as 280,000 soldiers from Benjamin, also bearing shields and wielding bows. All of them were valiant soldiers.
\passage{Ethiopia Invades and is Repulsed}

\v{9}Sometime later, Zerah the Ethiopian went to war against him at Mareshah with an army of one million troops and 300 chariots. \v{10}Asa went out to engage him in battle, and they drew up their battle lines at Mareshah in the Zephathah Valley. \v{11}Asa cried out to the \divine{Lord} his God, telling him, ``\divine{Lord}, there is no one except for you to help between the powerful and the weak. So help us, \divine{Lord} God, because we're depending on you and have come against this vast group in your name. \divine{Lord}, you are our God. Let no mere mortal man defeat you!''

\v{12}So the Lord defeated the Ethiopians right in front of Asa and Judah, and the Ethiopians ran away. \v{13}Asa and his army pursued the Ethiopians\fnote{Lit. \fbib{them}} as far as Gerar. So many Ethiopians died that their army could not recover, because it had been shattered in the \divine{Lord}'s presence and in the presence of his army. The Israelis\fnote{Lit. \fbib{They}} carried off a lot of plunder, too, \v{14}They attacked all the cities that surrounded Gerar, because fear of the \divine{Lord} had overwhelmed them. The Israelis spoiled all the cities, because there was a lot to plunder in them. \v{15}They also attacked the tents of those who owned livestock and carried off lots of sheep and camels. Then they returned to Jerusalem.
\labelchapt{15}
\passage{Azariah the Prophet Encourages Asa}

\chapt{15}
\v{1}After this, the Spirit of God came to rest on Oded's son Azariah, \v{2}so he went out to meet Asa and rebuked him:

\begin{poetry}
\poeml ``Listen to me, Asa, Judah, and Benjamin! The \divine{Lord} is with you when you are with him. If you seek him, he will allow you to find him, but if you abandon him, he will abandon you. \v{3}Israel lived for years without the true God, priests to teach them, and the Law, \v{4}but they turned to the \divine{Lord} God of Israel in their distress. When they sought him, he let them become reacquainted with him. \\
\poeml \v{5}``During those days, it wasn't safe for anyone to come and go, because many civil\fnote{The Heb. lacks \fbib{civil}} disturbances afflicted everyone who lived in the territories. \v{6}Nation battled nation, and city fought city, because God was afflicting them all with every kind of distress. \v{7}Now as for you,\fnote{MT pronoun is pl.} be strong\fnote{MT verb is pl.} and never be discouraged,\fnote{MT verb is pl.} because there will be reward for your\fnote{MT pronoun is pl.} work.''
\end{poetry}
\passage{Asa Institutes Reforms}

\v{8}Encouraged by what Oded's son Azariah the prophet had said in his prophecy, Asa\fnote{Lit. \fbib{he}} removed the detestable idols from throughout the entire territories of Judah and Benjamin, and from the cities that he had captured in the hill country of Ephraim. He repaired the \divine{Lord}'s altar that stood in front of the vestibule of the \divine{Lord}'s Temple. \v{9}Then he gathered together all of Judah, Benjamin, and people from Ephraim, Manasseh, and Simeon who were living among them, since many people had defected to him from Israel when they learned that the \divine{Lord} his God was with him. \v{10}They all assembled in Jerusalem during the third month of the fifteenth year of Asa's reign. \v{11}They sacrificed to the \divine{Lord} that day 700 oxen and 7,000 sheep from the spoil that they had brought with them. \v{12}They also entered into a covenant to seek the \divine{Lord} God of their ancestors with all their heart and soul, \v{13}and they further agreed that\fnote{The Heb. lacks \fbib{they further agreed that}} whoever would refuse to seek the \divine{Lord} God of Israel was to be executed, whether important or unimportant, man or woman. \v{14}They also made a vow to the \divine{Lord} with loud voices, shouting, trumpets, and horns. \v{15}Everybody in Judah was very glad to make their oath, because they had made their vow with all their heart and had sought him with all of their might,\fnote{Or \fbib{desire}} and they found him! The \divine{Lord} also gave them rest in their surrounding lands.

\v{16}King Asa removed his mother Maacah from her position as Queen Mother because she had made a detestable image dedicated to Asherah.\fnote{I.e. cultic pillars erected in worship to Canaanite deities} He cut down his mother's idol, crushed it, and burned it at the Kidron Brook. \v{17}Nevertheless, the high places were not removed from Israel, even though Asa's heart was blameless all of his life. \v{18}Asa brought into God's Temple the things that his father had dedicated, as well as his own dedicated gifts such as silver, gold, and temple service\fnote{The Heb. lacks \fbib{temple service}} implements. \v{19}Asa experienced no more war until the end of the\fnote{The Heb. lacks \fbib{end of the}} thirty-fifth year of his reign.
\labelchapt{16}
\passage{Asa Attacks Baasha}
\passageinfo{(1 Kings 15:16-22)}

\chapt{16}
\v{1}During the thirty-sixth year of Asa's reign, King Baasha of Israel invaded Judah and interdicted Ramah by building fortifications around it so no one could enter or leave to join King Asa of Judah. \v{2}But Asa removed some silver and gold from the treasuries of the \divine{Lord}'s Temple and from his royal palace and sent them to King Ben-hadad of Aram, who lived in Damascus. \v{3}``Let's make a treaty between you and me,'' he said, ``just like the one between my father and your father. Notice that I've sent you silver and gold to break your treaty with King Baasha of Israel, so he'll retreat from his attack\fnote{The Heb. lacks \fbib{his attack}} on me.''

\v{4}So King Ben-hadad did just what King Asa had asked: he sent his commanding officers to attack the cities of Israel. They conquered Ijon, Dan, Bel-maim, and all of the storage centers in Naphtali. \v{5}When Baasha learned of the attack, he withdrew from Ramah and stopped his interdiction. \v{6}Then King Asa brought his entire army of Judah to carry away the building stones and the timber that Baasha had been using to surround Ramah, and he used those materials to fortify Geba and Mizpah.
\passage{Asa is Rebuked by Hanani the Seer}
\passageinfo{(1 Kings 15:23-24)}

\v{7}Right about then, Hanani the seer came to King Asa of Judah and rebuked him. ``Because you have put your trust in the king of Aram and have not relied on the \divine{Lord} your God, the army of the king of Aram has escaped from your control. \v{8}Weren't the Ethiopians and the Libyans a vast army with many chariots and cavalry? Yet because you relied on the \divine{Lord}, he gave them into your control! \v{9}The \divine{Lord}'s eyes keep on roaming throughout the earth, looking for those whose hearts completely belong to him, so that he may strongly support them. But because you have acted foolishly in this, from now on you will have wars.'' \v{10}In response, Asa flew into a rage and locked up the seer in stocks in the palace prison\fnote{The Heb. lacks \fbib{prison}} because of what Hanani\fnote{Lit. \fbib{he}} had told him. Asa also tortured some of the people of Israel\fnote{The Heb. lacks \fbib{of Israel}} at that time.
\passage{Asa's Illness and Death}
\passageinfo{(1 Kings 15:23-24)}

\v{11}Now the accomplishments of Asa from first to last are written in the Book of the Kings of Judah. \v{12}In the thirty-ninth year of his reign, Asa suffered from a foot disease. Even though he suffered greatly, he never sought the \divine{Lord}, but instead looked to doctors. \v{13}As a result, in the forty-first year of his reign, Asa died, as had his ancestors, \v{14}and he was buried in his own tomb that he had prepared\fnote{Lit. \fbib{had carved out}} for himself in the City of David. He was laid out on a bier that had been filled with various spices prepared by morticians,\fnote{Lit. \fbib{by the perfumers' art}} and the mourners\fnote{Lit. \fbib{and they}} built a massive bonfire to honor his memory.
\labelchapt{17}
\passage{Jehoshaphat Succeeds Asa}

\chapt{17}
\v{1}Asa's son Jehoshaphat succeeded him as king, and he consolidated his authority over Israel \v{2}by placing troops in all of the fortified citadels through Judah and by establishing garrisons throughout the land of Judah and in the cities that his father Asa had captured.

\v{3}The \divine{Lord} was with Jehoshaphat because he followed the example set during his ancestor David's preliminary years by not pursuing the Baals.\fnote{I.e. the supreme male divinity of the Philistines and Canaanites} \v{4}Instead, Jehoshaphat\fnote{Lit. \fbib{he}} sought the God of his ancestors and obeyed his commands, unlike Israel. \v{5}Therefore the \divine{Lord} secured Jehoshaphat's\fnote{Lit. \fbib{his}} kingdom under his control, with all of Judah paying him tribute, and Jehoshaphat became very wealthy and greatly respected. \v{6}He remained committed to following the \divine{Lord}, and he removed the high places and Asherah poles from Judah.
\passage{Jehoshaphat Institutes Teaching Programs}

\v{7}During the third year of his reign, Jehoshaphat sent his officials Ben-hail, Obadiah, Zechariah, Nethanel, and Micaiah to teach throughout the cities of Judah. \v{8}They were accompanied by the descendants of Levi, including\fnote{The Heb. lacks \fbib{including}} Shemaiah, Nethaniah, Zebadiah, Asahel, Shemiramoth, Jehonathan, Adonijah, Tobijah, and Tobadonijah. These descendants of Levi were accompanied by the priests Elishama and Jehoram. \v{9}They taught throughout Judah from a copy of the Book of the Law of the \divine{Lord} that they took with them as they passed through all the cities of Judah, teaching among all the people.
\passage{Jehoshaphat's Military and Economic Stability}

\v{10}Because they were afraid of the \divine{Lord}, none of the kingdoms of the lands that surrounded Judah dared go to war against Jehoshaphat. \v{11}Some of the Philistines brought gifts and silver as tribute to Jehoshaphat, and Arabians brought him flocks of 7,700 rams and 7,700 male goats. \v{12}As a result, Jehoshaphat grew more and more powerful, and built up fortresses and storage centers throughout Judah. \v{13}He placed a large amount of supplies into storage throughout the cities of Judah and stationed soldiers---all of them valiant men---in Jerusalem. \v{14}Here's how they were mustered, listed according to their ancestral houses and listed by commanders of thousands: Adnah commanded 300,000 elite forces. \v{15}Near him was Johanan, commander of 280,000 \v{16}and next to him was Zichri's son Amasiah, who had volunteered to serve the \divine{Lord}. He commanded 200,000 elite forces. \v{17}There was also Eliada from Benjamin, himself a valiant soldier. He was accompanied by 200,000 expert archers bearing shields. \v{18}Near him was Jehozabad, who was accompanied by 180,000 soldiers equipped for warfare. \v{19}These men served the king, and there were others whom the king garrisoned inside fortified cities throughout all of Judah.
\labelchapt{18}
\passage{Jehoshaphat Allies with Ahab}
\passageinfo{(1 Kings 22:1-12)}

\chapt{18}
\v{1}After Jehoshaphat had become wealthy and was enjoying abundant honor, he allied himself to Ahab. \v{2}After a few years, he visited Ahab in Samaria. Ahab slaughtered lots of sheep and oxen for him, and the people who were with him persuaded Jehoshaphat to attack Ramoth-gilead. \v{3}King Ahab of Israel asked King Jehoshaphat of Judah, ``Will you join me in attacking Ramoth-gilead?''

``I'm with you,'' Jehoshaphat\fnote{Lit. \fbib{he}} replied. ``and my army is with you. We'll join you in the battle.'' \v{4}But then Jehoshaphat asked the king of Israel, ``Please ask for a message from the \divine{Lord}, first.''

\v{5}So the king of Israel gathered together 400 prophets and asked them, ``Should we go attack Ramoth-gilead, or should I call off the attack?''\fnote{The Heb. lacks \fbib{the attack}}

``Go attack them,'' they all said, ``because God will drop them right in the king's hand.''

\v{6}But Jehoshaphat asked, ``Isn't there a prophet of the \divine{Lord} left here that we could talk to?''

\v{7}``There is still one man left by whom we could ask the \divine{Lord} what to do,'' the king of Israel replied to Jehoshaphat, ``but I hate him because he won't prophesy anything good about me. Instead, he always prophesies evil. He is Imla's son Micaiah.''

But Jehoshaphat rebuked Ahab, ``Kings\fnote{Lit. \fbib{The king}} should never talk like that.''

\v{8}Nevertheless, the king of Israel called an officer and ordered him, ``Bring me Imla's son Micaiah quickly.''

\v{9}Now the king of Israel and King Jehoshaphat of Judah were each sitting on their own thrones, arrayed in their robes, and sitting on the threshing floor at the entrance to the city gate of Samaria, and all of the prophets were prophesying in front of them. \v{10}Chenaanah's son Zedekiah made iron horns for himself and told them, ``This is what the \divine{Lord} says, `With these horns you are to gore the Arameans until they are eliminated!'\,''

\v{11}All the other prophets were saying similar things, like ``Go up to Ramoth-gilead and you will be successful, because the \divine{Lord} will hand it over to the king!''
\passage{Micaiah the True Prophet Warns Ahab and Jehoshaphat}
\passageinfo{(1 Kings 22:13-28)}

\v{12}Meanwhile, the messenger who had gone off to summon Micaiah advised him, ``Look, everything that the other prophets were saying has been unanimously favorable to the king. So please, cooperate with them and speak favorably.''

\v{13}``As the \divine{Lord} lives,'' Micaiah replied, ``I'll say what my God tells me to say.''

\v{14}When Micaiah\fnote{Lit. \fbib{he}} approached the king, the king asked him, ``Micaiah, should we go to war against Ramoth-gilead, or should I not?''

``Go to war,'' Micaiah\fnote{Lit. \fbib{he}} replied, ``and you will be successful, because the \divine{Lord} will hand it over to the king!''

\v{15}When he heard this, the king asked him, ``How many times do I have to ask you? Tell me nothing but the truth, and do it in the name of the \divine{Lord}!''

\v{16}And so Micaiah replied:

\begin{poetry}
\poeml I saw all of Israel \\
\poemll    scattered on the mountains \\
\poemlll       like sheep without a shepherd. \\
\poeml And the \divine{Lord} told me, \\
\poemll    `These have no master, \\
\poemlll       so let them each return to his own home in peace.'\,''
\end{poetry}

\v{17}Then the king of Israel told Jehoshaphat, ``Didn't I tell you that he wouldn't prophesy anything good about me, but only evil?''

\v{18}But Micaiah responded, ``Therefore, listen to what the \divine{Lord} has to say. I saw the \divine{Lord}, sitting on his throne, and the entire Heavenly Army was surrounding him on his right hand and on his left hand.

\v{19}``The \divine{Lord} asked, `Who will tempt King Ahab of Israel to attack Ramoth-gilead, so that he will die there?' And one was saying one thing and one was saying another.

\v{20}``But then a spirit approached, stood in front of the \divine{Lord}, and said, `I will entice him.'

``And the \divine{Lord} asked him, `How?'

\v{21}```I will go,' he announced, `and I will be a deceiving spirit in the mouth of all of his prophets!'

``So the \divine{Lord} said, `You're just the one to deceive him. You will be successful. Go and do it.'

\v{22}Now therefore, listen! The \divine{Lord} has placed a lying spirit in the mouth of all of these prophets of yours, because the \divine{Lord} has determined to bring disaster upon you.''

\v{23}As if on cue, Chenaanah's son Zedekiah approached Micaiah and struck him on the cheek. Then he asked him, ``How did the Spirit of the \divine{Lord} move from me to speak to you?''

\v{24}Micaiah replied, ``You'll learn the answer to that question when the day comes that you run away to hide yourself in a closet!''

\v{25}Then the king of Israel ordered, ``Take Micaiah and place him in the custody of Amon, the city governor. Hand him over to Joash, the king's son. \v{26}Give him this order: `Place him in prison on survival rations only until I come back safely.'\,''

\v{27}``If you return alive,'' Micaiah responded, ``then the \divine{Lord} has not spoken by me.'' Then he added, ``Listen, everybody!''
\passage{Ahab's Dies at Ramoth-gilead}
\passageinfo{(1 Kings 22:29-40)}

\v{28}So the king of Israel and King Jehoshaphat of Judah both attacked Ramoth-gilead. \v{29}The king of Israel suggested to Jehoshaphat, ``I'll go into battle in disguise, but you keep your royal uniform on.'' So the king of Israel disguised himself and they both went into the battle.

\v{30}Meanwhile, the king of Aram had issued these orders to his chariot commanders: ``Don't attack unimportant soldiers or ranking officers. Go after only the king of Israel.'' \v{31}So when the chariot commanders observed Jehoshaphat, they said by mistake, ``It's the king of Israel!'' and they turned aside to attack him. But Jehoshaphat cried out to the \divine{Lord}, who helped him, and God diverted them from him. \v{32}When the chariot commanders saw that their target\fnote{Lit. \fbib{that he}} was not the king of Israel, they stopped pursuing him.

\v{33}Meanwhile, somebody drew his bow and struck the king of Israel at a weak spot where his armor plates joined, so he instructed his chariot driver, ``Turn around and take me out of the battle, because I've been severely wounded.'' \v{34}The battle continued on for the rest of the day while the king of Israel propped himself up in front of the Arameans until the sun set, at which time he died.''
\labelchapt{19}
\passage{Jehu the Seer Warns Jehoshaphat}

\chapt{19}
\v{1}After this, King Jehoshaphat of Judah returned safely to his palace in Jerusalem, \v{2}where Hanani's son Jehu, the seer, went out to meet him. He asked king Jehoshaphat, ``Should you be helping those who are wicked, yes or no? Should you love those who hate the \divine{Lord}? Wrath is headed your way directly from the \divine{Lord} because of this. \v{3}Nevertheless, a few good things have been found in you, in that you have removed the Asheroth\fnote{I.e. cultic pillars erected in worship to Canaanite deities} from the land and you have disciplined yourself to seek God.''
\passage{Judges are Appointed}

\v{4}Jehoshaphat continued to live in Jerusalem, but he travelled again throughout the people from Beer-sheba to Mount Ephraim, bringing them back to the \divine{Lord} God of their ancestors \v{5}and appointing judges throughout the land in all of the walled cities of Judah, city by city. He issued this reminder to the judges:

\begin{poetry}
\poeml \v{6}``Pay careful attention to your duties, because you are judging not only for the sake of human beings but also for the \divine{Lord}---and he is present with you as you make your rulings. \v{7}So let the fear of the \divine{Lord} rest upon you, be on your guard, and act carefully, because with the \divine{Lord} our God there is neither injustice, nor partiality, nor bribery.''
\end{poetry}

\v{8}In Jerusalem, Jehoshaphat also appointed certain descendants of Levi, priests, and family leaders of Israel to render verdicts for the \divine{Lord} and to decide difficult cases. Their offices were in Jerusalem. \v{9}He issued this reminder to them:

\begin{poetry}
\poeml ``You are to carry out your duties in the fear of the \divine{Lord}, serving him\fnote{The Heb. lacks \fbib{serving him}} faithfully\fnote{Or \fbib{truthfully}} with your whole heart. \v{10}No matter what case comes before you from your fellow citizens who live in their own cities, whether it's a dispute between blood relatives\fnote{Lit. \fbib{blood and blood}} or a dispute regarding the Law and the commands, statutes, or verdicts, you are to warn the parties\fnote{Lit. \fbib{warn them}} so that they do not become guilty in the \divine{Lord}'s presence and so that anger does not come upon you and your fellow citizens. \v{11}Take notice, please, that Amariah the Chief Priest is presiding over all cases\fnote{Lit. \fbib{is over you in all things}} that pertain to the \divine{Lord}, Ishmael's son Zebadiah is presiding as ruler of the household of Judah with respect to all cases that pertain to the national government,\fnote{Lit. \fbib{the king's matters}} and the descendants of Levi will preside over your other civil cases.\fnote{Lit. \fbib{over you}} Serve courageously, and the \divine{Lord} will be with the upright.''
\end{poetry}
\labelchapt{20}
\passage{Judah is Invaded Unexpectedly}

\chapt{20}
\v{1}Sometime after these events, the Moabites and the Ammonites, accompanied by some other descendants of Ammon,\fnote{Or \fbib{some Meunites}; cf. 2Chr 26:7} attacked Jehoshaphat and started a war. \v{2}Jehoshaphat's military advisors\fnote{Lit. \fbib{They}} came and informed him, ``We've been attacked by a vast invasion force from Aram,\fnote{I.e., from Edom} beyond the Dead\fnote{The Heb. lacks \fbib{Dead}} Sea. Be advised---they've already reached Hazazon-tamar, also known as En-gedi.''

\v{3}In mounting fear, Jehoshaphat devoted himself\fnote{Lit. \fbib{devoted his face}} to seek the \divine{Lord}. He proclaimed a period of\fnote{The Heb. lacks \fbib{a period of}} fasting throughout all of the territory of\fnote{The Heb. lacks \fbib{of the territory of}} Judah, \v{4}and the tribe of\fnote{The Heb. lacks \fbib{the tribe of}} Judah assembled together to seek the \divine{Lord}. People\fnote{Lit. \fbib{They}} came from all of the cities of Judah to seek the \divine{Lord}.
\passage{Jehoshaphat Prays and the People Wait}

\v{5}Jehoshaphat stood among the assembly of Judah and Jerusalem in the \divine{Lord}'s Temple in the vicinity of the new court \v{6}and said:

\begin{poetry}
\poeml ``\divine{Lord} God of our ancestors, you are the God who lives in heaven, are you not? You rule over all the kingdoms of the nations, don't you? In your own hands you grasp both strength and power, don't you? As a result, no one can oppose you, can they? \v{7}You are our God, who expelled the former inhabitants of this land right in front of our people Israel, aren't you? Then you gave it to your friend Abraham's descendant\fnote{Lit \fbib{seed} (sing.)} forever, didn't you? \v{8}They lived in it and have built there a sanctuary for your name, where they said, \v{9}`If evil comes upon us, such as war\fnote{Lit. \fbib{us the sword}} as punishment, disease, or famine and we stand in your presence in this Temple (because your Name is in this Temple) and cry out to you in our distress, then you will hear and deliver.' \v{10}Now therefore look! The Ammonites, the Moabites, and the inhabitants of\fnote{The Heb. lacks \fbib{inhabitants of}} Mount Seir,\fnote{This mountain, the modern \fbib{Jebel esh-sher\'{a}}, is located in the mountain range that extends south of the Dead Sea toward the Gulf of Aqaba, and is bordered by the Arabah Valley to the west.} whom you would not permit Israel to attack when they arrived from the land of Egypt---since they turned away from them and did not eliminate them--- \v{11}Look how they're rewarding us! They're coming to drive us from your property that you gave us to be our inheritance. \v{12}Our God, you are going to punish them, aren't you? We have no strength to face this vast multitude that has come against us, nor do we know what to do, except that our eyes are on you.''
\end{poetry}

\v{13}All of Judah was standing in the \divine{Lord}'s presence, along with their little babies, their wives, and their children.
\passage{The Prophetic Response of Jahaziel}

\v{14}Then the Spirit of the \divine{Lord} came upon Zechariah's son Jahaziel, the son of Benaiah, the son of Jeiel, the son of Mattaniah, a descendant of Levi from the descendants of Asaph in the middle of the assembly, and he said:

\begin{poetry}
\poeml \v{15}``Pay attention, everyone in Judah, in Jerusalem, and you, too, King Jehoshaphat! This is what the \divine{Lord} says to you: `Stop being afraid, and stop being discouraged because of this vast invasion force,\fnote{Lit. \fbib{vast multitude}} because the battle doesn't belong to you, but to God. \v{16}Tomorrow you are to go down to attack them. Pay attention, now---they'll be coming up near the ascent of Ziz.\fnote{I.e. a mountain pass extending from the Dead Sea to the Judean wilderness near Tekoa} You'll find them at the end of the valley that looks out over the Jeruel wilderness. \v{17}You won't be fighting in this battle. Take your stand, but stand still, and watch the \divine{Lord}'s salvation on your behalf, Judah and Jerusalem! Never fear and never be discouraged. Go out to face them tomorrow, since the \divine{Lord} is with you.'\,''
\end{poetry}

\v{18}Jehoshaphat bowed down with his face\fnote{Lit. \fbib{nostrils}} to the ground, and all the assembled inhabitants of Judah and Jerusalem fell face down in the \divine{Lord}'s presence and worshipped the \divine{Lord}. \v{19}Descendants of Levi from the descendants of Kohath and from the descendants of Korah stood up to praise the \divine{Lord} God of Israel in a very loud voice that ascended to heaven.\fnote{Lit. \fbib{ascended on high}}
\passage{Jehoshaphat's Instructions the Next Morning}

\v{20}The army\fnote{Lit. \fbib{They}} got up early the next morning and headed out into the wilderness of Tekoa. Jehoshaphat stood up and addressed them. ``Listen to me, you inhabitants of Judah and Jerusalem,'' he said. ``Have faith in the \divine{Lord} your God and you'll be established! Have faith in his prophets and you'll succeed!'' \v{21}After he had consulted with the people, Jehoshaphat\fnote{Lit. \fbib{he}} appointed some choir members\fnote{The Heb. lacks \fbib{choir members}} to sing to the \divine{Lord} and to praise him in sacred splendor as they marched out in front of the armed forces. They kept saying

\begin{poetry}
\poeml ``Give thanks to the \divine{Lord}, \\
\poemll    because his gracious love is eternal!''
\end{poetry}

\v{22}Right on time, as they began to sing and praise, the \divine{Lord} ambushed\fnote{Or \fbib{surprised}; i.e. attacked the invaders from concealment} the Ammonites, Moabites, and the inhabitants of\fnote{The Heb. lacks \fbib{inhabitants of}} Mount Seir\fnote{This mountain, the modern \fbib{Jebel esh-sher\'{a}}, is located in the mountain range that extends south of the Dead Sea toward the Gulf of Aqaba, and is bordered by the Arabah Valley to the west.} who had attacked Judah, and they were defeated. \v{23}The Ammonites and Moabites attacked the inhabitants of Mount Seir, destroying them, and after they had finished with the inhabitants of Mount Seir, they worked on destroying one another!\fnote{Lit. \fbib{destroying each man his neighbor}}

\v{24}When the army of\fnote{The Heb. lacks \fbib{the army of}} Judah arrived at the remotest watchtower in the wilderness, they looked around at the invasion force, and to their surprise, there were dead bodies lying all around on the ground---not one had escaped! \v{25}Later on, when Jehoshaphat and his army arrived to collect the spoils of war, they discovered there were far more goods, garments, and other valuable items to collect than they could carry off in a single day.\fnote{The Heb. lacks \fbib{in a single day}} There was so much material that it took three days to finish their collection efforts.
\passage{A Victory Celebration in Beracah Valley}

\v{26}Three days later, they assembled together in the Beracah Valley, where they blessed the \divine{Lord}, which is why the name of that place is called Beracah\fnote{The Heb. name \fbib{Beracah} means \fbib{blessing}} Valley to this day. \v{27}Then they all returned with joy to Jerusalem, every soldier from Judah and Jerusalem, with Jehoshaphat at the head of the procession, because the \divine{Lord} had made them rejoice over their enemies. \v{28}They proceeded directly to the \divine{Lord}'s Temple, carrying lyres, harps, and trumpets. \v{29}Fear of God seized all of the kingdoms in the surrounding territories when they heard that the \divine{Lord} had battled Israel's enemies. \v{30}As a result, Jehoshaphat's kingdom enjoyed peace, because his God had provided rest for him all around.
\passage{A Summary of Jehoshaphat's Reign}
\passageinfo{(1 Kings 22:41-50)}

\v{31}Jehoshaphat reigned over Judah, having become king at the age of 35. He reigned in Jerusalem for 25 years. His mother's name was Azubah, the daughter of Shilhi. \v{32}He followed the example of his father Asa and never departed from it, practicing what the \divine{Lord} considered to be right. \v{33}However, the high places were not removed, since the people had not yet directed their hearts to the God of their ancestors. \v{34}The rest of Jehoshaphat's accomplishments, from first to last, are recorded in the annals of Hanani's son Jehu, which appears in the Book of the Kings of Israel.
\passage{Jehoshaphat's Evil Alliance with Ahaziah}

\v{35}Sometime later, King Jehoshaphat of Judah entered into a military alliance with King Ahaziah of Israel, acting wickedly by doing so. \v{36}He also agreed with King Ahaziah\fnote{Lit. \fbib{with him}} to build ships to sail toward Tarshish, which they built in Ezion-geber. \v{37}But Dodavahu's son Eliezer from Mareshah prophesied in opposition to Jehoshaphat, ``Because you have entered into an alliance with Ahaziah, the \divine{Lord} has destroyed your efforts.'' So the ships were destroyed and were never able to sail for Tarshish.
\labelchapt{21}
\passage{Jehoram Succeeds Jehoshaphat}
\passageinfo{(1 Kings 22:50; 2 Kings 8:16-19)}

\chapt{21}
\v{1}Jehoshaphat died, as had his ancestors, and was buried in the City of David alongside his ancestors. His son Jehoram became king in his place. \v{2}Jehoshaphat's sons, Jehoram's\fnote{Lit. \fbib{his}} brothers, included Azariah, Jehiel, Zechariah, Azariah,\fnote{Lit. \fbib{Azaryahu}} Michael, and Shephatiah. All of these were sons of Jehoshaphat, king of Israel.

\v{3}Their father gave them many gifts made of silver, and gold, as well as valuable things, along with fortified cities in Judah, but he passed the kingdom to Jehoram because Jehoram was his firstborn. \v{4}But after Jehoram had assumed the throne and consolidated his rule over his father's kingdom, he executed all of his brothers, along with some of the rulers of Israel. \v{5}Jehoram was 32 years old when he became king, and he reigned for eight years in Jerusalem. \v{6}He lived like\fnote{Lit. \fbib{He walked in the ways of}} the kings of Israel, following the example of Ahab's dynasty, since he had married Ahab's daughter, and he practiced what the \divine{Lord} considered to be evil. \v{7}Nevertheless, the Lord was unwilling to destroy David's dynasty because of the covenant that he had made with David, especially since he had promised to give him and to his sons the reigning presence of an heir\fnote{Lit. \fbib{sons a lamp}} forever.
\passage{Edom Revolts}
\passageinfo{(2 Kings 8:20-22)}

\v{8}Nevertheless, Edom revolted against Judah's rule and set up their own king to rule them during Jehoram's reign.\fnote{Lit. \fbib{days}} \v{9}So Jehoram invaded Edom\fnote{Lit. \fbib{So he crossed over}} with his commanders and his chariots by night and killed the Edomites who had surrounded him and his chariot commanders. \v{10}Edom remains in revolt against Judah to this day. Libnah revolted against Jehoram's rule, too, because he had abandoned the \divine{Lord} God of his ancestors. \v{11}In addition to all of this, he built high places in the mountains of Judah, led the inhabitants of Jerusalem into cultic sexual immorality, and made Judah go astray.
\passage{Elijah Writes a Letter}

\v{12}After this, a letter arrived from Elijah the prophet. It said:

\begin{poetry}
\poeml ``This is what the \divine{Lord} God of your ancestor David says: `You haven't lived like your father Jehoshaphat and like King Asa of Judah. \v{13}Instead, you have lived like the kings of Israel by causing Judah and the inhabitants of Jerusalem to commit cultic sexual immorality---just like Ahab's dynasty did! And you've killed your brothers who were better than you---your own father's dynasty! \v{14}Look what's going to happen! The \divine{Lord} is going to strike your people, your children, your wives, and everything you own with a massive tragedy. \v{15}And as for you, you will suffer from a serious disease of your bowels. Eventually, day-by-day you will excrete your own bowels because of this disease.''
\end{poetry}

\v{16}The \divine{Lord} also provoked the attitude of the Philistines and the Arabs who bordered the Ethiopians against Jehoram, \v{17}and they attacked Judah, invading it and carried off everything he owned in his royal palace, along with all of his sons and wives except for his youngest son Jehoahaz.\fnote{This individual is also identified as Ahaziah in 2Chr 22:1}
\passage{Jehoram's Illness and Death}
\passageinfo{(2 Kings 8:23-24)}

\v{18}After all of this happened, the \divine{Lord} struck him in his bowels with an incurable illness. \v{19}In due course, as time passed, two years later\fnote{Lit. \fbib{And it came about with respect to the days from the days, as time went out, at the end of two days}} his bowels came out because of his sickness and he died in agony. His people lit no memorial bonfire for him as they had done for his ancestors. \v{20}Jehoram\fnote{Lit. \fbib{He}} was 32 years old when he became king, and he reigned in Jerusalem for eight years. He left this earth\fnote{The Heb. lacks \fbib{this earth}}---to nobody's regret---and they buried him in the City of David, but not in the tombs of the kings.
\labelchapt{22}
\passage{Ahaziah Succeeds Jehoram}
\passageinfo{(2 Kings 8:25-29; 9:14-16; 27-29)}

\chapt{22}
\v{1}The residents of Jerusalem made Jehoram's\fnote{Lit. \fbib{his}} son Ahaziah\fnote{This individual is also identified as Jehoahaz in 2Chr 21:17} king in his place after the raiding party that had invaded the city with the Arabs had killed all of the older sons. That's how Jehoram's son Ahaziah became king of Judah. \v{2}Ahaziah was 22\fnote{Cf. 2 King 8:26, Syr, and LXX. MT reads 42.} years old when he became king, and he reigned for one year in Jerusalem. His mother was Athaliah, Omri's granddaughter.

\v{3}He followed the example\fnote{Lit. \fbib{footsteps}} of Ahab's dynasty because his mother gave him evil counsel. \v{4}So he practiced what the \divine{Lord} considered to be evil, just like Ahab's dynasty had done, because after his father died, he was given advice that resulted in his destruction. \v{5}He followed their counsel and accompanied Ahab's son Joram, king of Israel, to wage war against King Hazael of Aram at Ramoth-gilead. But the Arameans wounded Joram, \v{6}so he returned to Jezreel to recover from the wounds that he had received at Ramah in the battle against King Hazael of Aram. King Ahaziah of Judah, Jehoram's son, went to visit Ahab's son Joram, because he was wounded.
\passage{Ahaziah is Executed}
\passageinfo{(2 Kings 9:27-28)}

\v{7}God used Ahaziah's visit to Joram to destroy Ahaziah. As soon as he arrived, Ahaziah\fnote{Lit. \fbib{he}} went out with Joram to attack Nimshi's son Jehu, whom the \divine{Lord} had appointed to eliminate Ahab's dynasty. \v{8}And that's exactly what happened. While Jehu was punishing\fnote{Lit. \fbib{was executing judgment}} Ahab's dynasty, he located the princes of Judah and the sons of Ahaziah's brothers who were ministering to Ahaziah, and he put them to death. \v{9}Jehu\fnote{Lit. \fbib{He}} also searched for Ahaziah, had him apprehended while Ahaziah\fnote{Lit. \fbib{he}} was hiding out in Samaria, and had Ahaziah\fnote{Lit. \fbib{him}} brought to him. Jehu\fnote{Lit. \fbib{He}} had Ahaziah\fnote{Lit. \fbib{him}} executed and buried. It was said of Jehu,\fnote{Lit. \fbib{him}} ``He is the son of Jehoshaphat, who sought the \divine{Lord} with all of his heart.'' As a result, there was no one left in the household of Ahaziah strong enough to reign in the kingdom.
\passage{Athaliah's Revolt}
\passageinfo{(2 Kings 11:1-8)}

\v{10}As soon as Ahaziah's mother Athaliah learned that her son had died, she set out to destroy the entire royal family of Judah. \v{11}However, the king's daughter Jehoshabeath took Ahaziah's son Joash away from the king's children who were about to be assassinated and hid him and his nurse in a bedroom. That's how King Jehoram's daughter Jehoshabeath, who was also the priest Jehoiada's wife and Ahaziah's sister, hid him from Athaliah. As a result, she was not able to kill him. \v{12}Joash\fnote{Lit. \fbib{He}} remained with them for six years, hidden in God's Temple while Athaliah reigned over the land.
\labelchapt{23}
\passage{Jehoiada Establishes Joash as King}
\passageinfo{(2 Kings 11:9-12)}

\chapt{23}
\v{1}Seven years later, Jehoiada mustered up some courage and made a deal with the officers who commanded units of hundreds of soldiers, including Jehoram's son Azariah, Jehochanan's son Ishmael, Obed's son Azariah, Adaiah's son Maaseiah, and Zichri's son Elishaphat. \v{2}They traveled throughout Judah and gathered together the descendants of Levi from all the cities of Judah, along with the Israeli family leaders. \v{3}Everybody went to Jerusalem, and the whole group made a covenant with the king in God's Temple, where Jehoiada\fnote{Lit. \fbib{he}} addressed them:

\begin{poetry}
\poeml ``Look! The king's son is going to rule, just as the \divine{Lord} promised David's descendants. \v{4}So here's what you'll need to do: One third of you priests and descendants of Levi who are on duty during the Sabbath will serve as guards at the temple gates. \v{5}Another third of you priests and descendants of Levi\fnote{The Heb. lacks \fbib{priests and descendants of Levi}} will take your places in the royal palace, while another third of you priests and descendants of Levi\fnote{The Heb. lacks \fbib{priests and descendants of Levi}} will stand near the Foundation Gate. The rest of you will remain in the courtyard of the \divine{Lord}'s Temple. \v{6}Nobody is to enter the \divine{Lord}'s Temple except for the priests and descendants of Levi who are on duty. They may enter because they are ceremonially holy, but all the rest of the people must observe the \divine{Lord}'s instructions. \v{7}The descendants of Levi will surround the king, brandishing weapons in their hands, and anybody who enters the Temple will be killed. Stay near the king wherever he enters and leaves.''
\end{poetry}

\v{8}What Jehoiada the priest ordered is precisely what the descendants of Levi and all of Judah did. Each of them took the men who were on duty on the Sabbath as well as those who were off duty. Jehoiada the priest did not release the divisions from service, \v{9}and Jehoiada the priest issued the spears and shields that King David had placed in storage in God's Temple to the officers in charge of the units of hundreds. \v{10}He set the rest of the people to serve as guards for the king, and each one brandished weapons in his hand, from the south side of the Temple to the north side of the Temple, around the altar, and surrounding the palace. \v{11}Then he brought out the king's son, put a crown on him, and presented him with the Testimony,\fnote{I.e. the tablets that were stored in the ark; cf. Ex 25:16, 31:18}
\passage{Joash is Crowned and Athaliah Executed}
\passageinfo{(2 Kings 11:9-12)}

\v{12}When Athaliah heard all the commotion of the people running around and praising the king, she went straight to the \divine{Lord}'s Temple to confront\fnote{The Heb. lacks \fbib{confront}} the people. \v{13}She looked around, and there was the king, standing by his pillar at the gate, accompanied by officers and trumpeters who stood beside the king, along with all the people of the land rejoicing and sounding trumpets while singers lead the celebration with their musical instruments. Athaliah tore her robes and yelled ``Treason! Treason!''

\v{14}But Jehoiada the priest summoned the captains of hundreds who had been appointed in charge over the army and ordered them, ``Bring her out between the ranks, and execute anyone who follows her.'' The priest also told them, ``Don't execute her in the \divine{Lord}'s Temple.'' \v{15}So they arrested her when she arrived at the entrance to the Horse Gate near the royal palace, and then they executed her there.
\passage{Jehoiada's Reforms}
\passageinfo{(2 Kings 12:17-20)}

\v{16}After this, Jehoiada drew up a covenant between himself as an individual with all the people, and between himself as king, that they would be the \divine{Lord}'s people. \v{17}Then all the people went to the temple of Baal, broke its altars and idols to pieces, and executed Mattan, the priest of Baal, in front of the altars. \v{18}Jehoiada also placed the offices of the \divine{Lord}'s Temple under the authority of the Levitical priests whom David had assigned over the \divine{Lord}'s Temple, just as is required by the Law of Moses, to offer the \divine{Lord}'s burnt offerings with joy and singing, just as David had ordered. \v{19}Jehoiada\fnote{Lit. \fbib{He}} also stationed inspectors\fnote{Lit. \fbib{gatekeepers}} at the \divine{Lord}'s Temple so that no one would enter who was ritually unclean in any manner. \v{20}He also took the captains of hundreds, the nobles, the people's governors, and all the people of the land, and they all marched with the king from the \divine{Lord}'s Temple through the upper gate to the royal palace, where they installed the king on his royal throne. \v{21}There all of the people of the land rejoiced and the city stayed quiet, because they had executed Athaliah with a sword.
\labelchapt{24}
\passage{Joash Follows Jehoiada's Example}
\passageinfo{(2 Kings 11:21-12:16)}

\chapt{24}
\v{1}Joash was seven years old when he began his reign, and he reigned forty years in Jerusalem. His mother's name was Zibiah. She was from Beer-sheba. \v{2}Joash practiced what the \divine{Lord} considered to be right during the lifetime\fnote{Lit. \fbib{days}} of Jehoiada the priest, \v{3}who found two wives for him, so he fathered sons and daughters.

\v{4}Later on, Joash decided to rebuild the \divine{Lord}'s Temple, \v{5}so he assembled the priests and descendants of Levi and ordered them, ``Go throughout the cities of Judah and take up a collection\fnote{Lit. \fbib{and collect silver}} from all of Israel for the annual upkeep\fnote{Lit. \fbib{strengthening}} of the Temple of your God. And make sure that you act quickly.'' But the descendants of Levi did not act quickly, \v{6}so the king summoned Jehoiada the chief priest and asked him, ``Why haven't you required the descendants of Levi to bring from Judah and Jerusalem the tax levied by Moses, the \divine{Lord}'s servant, and the assembly of Israel for the Tent of Testimony?''

\v{7}Because that wicked woman Athaliah's family members had broken into the Temple of God and used the consecrated implements of the \divine{Lord}'s Temple for service to the Baals, \v{8}the king issued an order and a chest was made and set outside the entrance gate to the \divine{Lord}'s Temple. \v{9}A public notice was sent throughout Judah and Jerusalem to bring in the tax that Moses the servant of the \divine{Lord} had levied on Israel when they were in the wilderness. \v{10}So all the princes and all the people gladly brought their tax and placed it into the chest until they had completed paying the tax.\fnote{The Heb. lacks \fbib{paying the tax.}} \v{11}Whenever the chest was brought to the king's officials by the descendants of Levi, the royal secretary and the chief priest's designated officer would come, empty the chest, and take it back to its place. They did this day after day until they had collected a large amount of cash.\fnote{Lit. \fbib{silver}}

\v{12}Both the king and Jehoiada paid the money to those who were working to maintain the service of the \divine{Lord}'s Temple, and they, in turn, hired masons and carpenters to restore the \divine{Lord}'s Temple. Iron and bronze workers also were brought in to repair the Lord's Temple. \v{13}As a result, the workmen did their labor, and the repair work progressed steadily under their supervision,\fnote{Lit. \fbib{progressed in their hands}} and they restored God's Temple back to what it should be, and strengthened it, too. \v{14}When they had completed the work, they brought what was left of the money to the king and to Jehoiada, and it was used to cast utensils for the \divine{Lord}'s Temple that were to be utilized for daily service and for burnt offerings, for incense vessels, and for both gold and silver vessels. Burnt offerings were offered on a regular basis in the \divine{Lord}'s Temple throughout Jehoiada's lifetime.
\passage{Joash Apostatizes and Kills Jehoiada's Son}

\v{15}Eventually, Jehoiada grew old and died at the age of 130 years, after having lived a full life. \v{16}He was buried in the City of David among the graves of\fnote{The Heb. lacks \fbib{the graves of}} the kings, because he had accomplished many good things in Israel on behalf of God and his Temple. \v{17}But after Jehoiada had died, officials from Judah came, bowed down to the king, and the king listened to what they had to say. \v{18}They abandoned the \divine{Lord}'s Temple and the God of their fathers, and they served Asherim\fnote{I.e. cultic pillars erected in worship to Canaanite deities} and idols. As a result this guilt of theirs resulted in wrath coming upon Judah and Jerusalem. \v{19}Nevertheless, God\fnote{Lit. \fbib{he}} sent prophets among them to bring them back to the \divine{Lord}.

\v{20}Then Jehoiada the priest's son Zechariah was clothed by the Spirit of God, and he stood above the people and told them, ``This is what God has to say: `Why are you breaking the \divine{Lord}'s commandments. You'll never be successful! Because you have abandoned the \divine{Lord}, he has abandoned you.'\,''

\v{21}But the people\fnote{Lit. \fbib{But they}} conspired against him, and at the direct orders of the king they stoned him to death in the courtyard of the \divine{Lord}'s Temple. \v{22}This is how King Joash failed to remember the kindness that Zechariah's father Jehoiada had shown him: he killed his son. As he lay dying, Zechariah cried out, ``May the \divine{Lord} watch this and avenge.''
\passage{The Death of Joash}
\passageinfo{(2 Kings 12:19-21)}

\v{23}At the end of that year, the Aramean army attacked Joash. They invaded Judah and Jerusalem, destroyed every senior official among the people, and sent all of their possessions to the king of Damascus. \v{24}The Aramean army attacked with only a small force, but the \divine{Lord} delivered a much larger army into their control because Judah\fnote{Lit. \fbib{they}} had abandoned the \divine{Lord} God of their ancestors. And so the Aramean army carried out God's\fnote{The Heb. lacks \fbib{of God's}} judgment on Joash. \v{25}After the Arameans left him very sick, Joash's\fnote{Lit. \fbib{his}} own servants conspired against him because Joash\fnote{Lit. \fbib{he}} had murdered Jehoiada the priest's son, and they killed him on his sick bed. \v{26}The conspirators included Shimeath the Ammonite's son Zabad and Shimrith the Moabite's son Jehozabad. \v{27}Records concerning his sons, the various prophetic statements rebuking him, and records of the reconstruction work on God's Temple are written in the Midrash\fnote{Or \fbib{Commentary}} of the Book of the Kings. Joash's\fnote{Lit. \fbib{His}} son Amaziah reigned in his place.
\labelchapt{25}
\passage{Amaziah Succeeds Joash}
\passageinfo{(2 Kings 14:7)}

\chapt{25}
\v{1}Amaziah began his reign at the age of 25 years, and he reigned 29 years in Jerusalem. His mother's name was Jehoaddan. She was from Jerusalem. \v{2}He practiced what the \divine{Lord} considered to be right, but not with a perfect heart. \v{3}As soon as he had consolidated his royal authority, he executed the servants who had killed his father, the king, \v{4}but he did not execute their children in obedience to what is written in the Law, the writings of Moses, where the \divine{Lord} commanded, ``Fathers are not to die because of what their children do, nor are children to die because of what their fathers do, but each person is to die for his own sins.''\fnote{Cf. Deut 24:16; Jer 31:30; Eze 18:20}
\passage{The Edomites are Defeated}
\passageinfo{(2 Kings 14:7)}

\v{5}Amaziah gathered Judah together and organized them according to their ancestral households under commanders of thousands and hundreds throughout Judah and Benjamin. He then mustered an army from those who were 20 years old and older. He discovered that there were 300,000 elite soldiers qualified for war duty and capable of handling spears and shields. \v{6}He also hired 100,000 elite forces from Israel, paying 100 talents\fnote{I.e. about 7,500 pounds; a talent weighed about 75 pounds} of silver for their services.

\v{7}A man came from God and warned him, ``Your majesty, don't let the army of Israel accompany you into battle, because the \divine{Lord} isn't with any of the descendants of Ephraim. \v{8}But if you do go, strengthen yourself for war. Do you think God will throw you down before the enemy, since God has the power both to help or to overthrow?''

\v{9}Amaziah asked the man of God, ``What are we to do about the 100 talents\fnote{I.e. about 7,500 pounds; a talent weighed about 75 pounds} that I have paid to the army of Israel?''

The man of God answered, ``The \divine{Lord} has a lot more than that to give you!'' \v{10}So Amaziah sent the troops home who had arrived from Ephraim. They flew into a rage against Judah but left for home very angry.

\v{11}But Amaziah encouraged himself and led his army out to the Salt Valley to kill 10,000 soldiers from Seir. \v{12}The army of Judah captured another 10,000 prisoners and took them to the top of a cliff and threw them down from there where they all were dashed to pieces. \v{13}Meanwhile, the troops that Amaziah had sent home from the battle raided the cities of Judah from Samaria to Beth-horon, killing 3,000 people and taking a large amount of war booty.

\v{14}Later, Amaziah returned from slaughtering the Edomites, but he brought back the gods that had belonged to the men of Seir, set them up as his own gods, worshipped them, and sacrificed offerings to them. \v{15}As a result, the Lord became angry with Amaziah and sent a prophet to him, who asked him, ``Why did you seek the gods of a people who were unable to deliver their own nation from you?''

\v{16}But even while the prophet\fnote{Lit. \fbib{while he}} was speaking, the king asked him, ``Did we appoint you to be a royal counselor? Stop! Why should you be struck down?''

So the prophet stopped speaking, but he also said, ``I know God has determined to destroy you, because you've done all this and ignored my counsel.''
\passage{Israel Defeats Judah}
\passageinfo{(2 Kings 14:8-14)}

\v{17}After this, King Amaziah of Judah sought some advice and then challenged Jehoahaz' son King Joash of Israel, the grandson of Jehu, telling him, ``Come out and let's fight each other!''

\v{18}But King Joash of Israel replied to King Amaziah of Judah, ``There once was a thorn bush in Lebanon that sent an invitation to the cedar of Lebanon that read `Give your daughter to my son in marriage.' Right about then, a wild animal in Lebanon passed by and trampled the thorn bush. \v{19}You claim you've defeated Edom, but you're really only puffed up with arrogant boasting. So stay home. Why stir up trouble so you die, and the rest of Judah with you?''

\v{20}But Amaziah refused to listen, because the situation was being orchestrated by God in order to turn them over to the control of their enemies because they had pursued those Edomite gods. \v{21}So King Joash of Israel went out to battle against King Amaziah of Judah, and they fought at Beth-shemesh, which is part of Judah's territory. \v{22}Judah was defeated by Israel, and every soldier ran home. \v{23}King Joash of Israel captured Joash's son King Amaziah of Judah, the grandson of Ahaziah, at Beth-shemesh and brought him back to Jerusalem, where he broke down 400 cubits\fnote{I.e. about 600 feet; a cubit was about eighteen inches} of the wall of Jerusalem from the Ephraim Gate to the Corner Gate. \v{24}He confiscated all the gold, silver, and utensils that he could find in the care of Obed-edom inside of God's Temple and inside the royal palace. Then he took some hostages and returned to Samaria.
\passage{The Death of Amaziah}
\passageinfo{(2 Kings 14:17-20)}

\v{25}Joash's son Amaziah, king of Judah, lived for fifteen years after the death of Jehoahaz' son Joash, king of Israel. \v{26}The rest of Amaziah's accomplishments, from first to last, are recorded in the Book of the Kings of Judah and Israel, are they not? \v{27}From the time that Amaziah abandoned his seeking the \divine{Lord}, some people conspired against him in Jerusalem, so he ran away to Lachish, but they pursued him to Lachish and killed him there. \v{28}They brought him back on horses and buried him with his ancestors in the city of Judah.
\labelchapt{26}
\passage{Uzziah Succeeds Amaziah}
\passageinfo{(2 Kings 14:21-22; 15:1-3)}

\chapt{26}
\v{1}All the people of Judah made Uzziah king in place of his father Amaziah. Uzziah was sixteen years old at the time. \v{2}He rebuilt Eloth and restored it to Judah after King Amaziah\fnote{Lit. \fbib{after the king}} had been laid to rest\fnote{Lit. \fbib{after the king slept}} with his ancestors. \v{3}Uzziah was sixteen years old when he became king, and he reigned for 52 years in Jerusalem. His mother's name was Jecholiah. She was from Jerusalem. \v{4}He practiced what the \divine{Lord} considered to be right, following the example set by his father Amaziah's accomplishments. \v{5}Uzziah\fnote{Lit. \fbib{He}} kept on seeking God during the lifetime of Zechariah, who taught him how to fear God, and as long as he sought the \divine{Lord}, God made him prosperous.
\passage{Uzziah's Initial Successes}

\v{6}One time Uzziah\fnote{Lit. \fbib{he}} went out and battled the Philistines. He tore down the walls of Gath, Jabneh, and Ashdod, and built cities in the Ashdod area among the Philistines. \v{7}God helped Uzziah\fnote{Lit. \fbib{him}} defeat the Philistines, the Arabians who lived in Gur-baal, and the Meunites. \v{8}The Ammonites paid tribute to Uzziah, and his reputation extended as far as the border with Egypt as he became stronger and stronger. \v{9}Uzziah also built towers in Jerusalem, at the Corner Gate, at the Valley Gate, and at the Angle\fnote{Or \fbib{the Corner Portion; i}.e., a portion of Jerusalem's wall near an armory; cf. Neh 3:19} and fortified them. \v{10}He also built watchtowers in the wilderness and had many cisterns hewed out, since he also possessed large herds, both in the Shephelah\fnote{I.e. the verdant central lowlands of Israel; cf. Josh 10:40} and in the midland plains. He had many farmers and vinedressers throughout the hills and fertile lands because he loved farming.\fnote{Lit. \fbib{loved the ground}}

\v{11}Uzziah kept a standing army, equipped for battle, garrisoned in divisions according to an organizational structure devised by his royal secretary Jeiel and his officer Maaseiah, who reported to Hananiah, one of the king's commanders. \v{12}The number of senior leaders of the ancestral houses of his elite forces numbered 2,600. \v{13}Uzziah\fnote{Lit. \fbib{He}} commanded an army of 307,500 who could fight formidably on behalf of the king against any enemy. \v{14}In addition, Uzziah equipped the entire army with shields, spears, helmets, body armor, bows, and stones for use in slings. \v{15}He also had various siege engines built by skilled designers and placed them on the towers and on the corner ramparts that could fire arrows and very large stones. His reputation spread far and wide, and he was marvelously assisted until he grew very strong.
\passage{Uzziah's Arrogance and Apostasy}
\passageinfo{(2 Kings 15:4-7)}

\v{16}But after he had become strong, in his arrogance he acted corruptly and became unfaithful to the \divine{Lord} his God, and he dared to enter the \divine{Lord}'s Temple to burn incense on the incense altar. \v{17}Azariah the priest ran after him, along with 80 of the \divine{Lord}'s valiant priests, \v{18}and they opposed King Uzziah. ``Uzziah, it's not for you to burn incense to the \divine{Lord},'' they told him, ``but for the priests to do, Aaron's descendants who are consecrated to burn incense. Leave the sanctuary now, because you have been unfaithful and won't receive any honor from the \divine{Lord} God.''

\v{19}Uzziah flew into a rage while he held in his hand a censer to burn incense. As he got angry at the priests, leprosy broke out all over his forehead right in front of the priests beside the incense altar in the \divine{Lord}'s Temple. \v{20}So Azariah the chief priest and all the priests stared at Uzziah, who was infected with leprosy in his forehead! They all rushed at him and hurried him out of the Temple. Uzziah\fnote{Lit. \fbib{He}} was in a hurry to get out anyway, because the \divine{Lord} had struck him.

\v{21}King Uzziah remained a leper until the day he died. Because he was a leper, he lived in a separate residence and remained disqualified to enter the \divine{Lord}'s Temple. His son Jotham served in the royal palace, judging the people of the land. \v{22}Now the rest of Uzziah's accomplishments, from first to last, have been recorded by Amoz's son Isaiah the prophet. \v{23}Uzziah died, as had his ancestors, and they buried him alongside his ancestors in a grave in a field that belonged to the kings, because they said, ``He was a leper.'' Uzziah's\fnote{Lit. \fbib{His}} son Jotham became king to replace him.
\labelchapt{27}
\passage{Jotham Succeeds Uzziah}
\passageinfo{(2 Kings 15:32-38)}

\chapt{27}
\v{1}Jotham was 25 years old when he began his reign, and he reigned for sixteen years in Jerusalem. His mother was Zadok's daughter Jerusha. \v{2}He practiced what the \divine{Lord} considered to be right, just as his father Uzziah had done, even though he did not enter the Temple. Nevertheless, the people continued acting corruptly.

\v{3}Jotham\fnote{Lit. \fbib{He}} constructed the Upper Gate of the \divine{Lord}'s Temple and did extensive work on the wall of Ophel.\fnote{I.e. a ridge of hills in Jerusalem fortified for defense of the city} \v{4}He also built cities in the hill country of Judah, along with fortresses and guard towers in the forests. \v{5}He launched a military excursion against the king of the Ammonites and defeated him. As a result, that year the Ammonites paid 100 talents\fnote{I.e. about 7,500 pounds, if this talent weighed about 75 pounds; but Babylonian era talents are known to have weighed as much as 130 pounds} of silver in tribute, as well as 10,000 kors\fnote{I.e. about 60,000 bushels; the \fbib{kor} was a dry measure equal to about six bushels} of wheat and 10,000 kors\fnote{The Heb. lacks \fbib{kors}} of barley. The Ammonites continued to pay this same amount in tribute over the following two years. \v{6}Jotham grew in power because he had determined to live his life in the presence of the \divine{Lord} his God. \v{7}The rest of the accomplishments of Jotham's reign, including all of his military exploits and campaigns, are recorded in the book of the Kings of Israel and Judah. \v{8}He started his reign at the age of 25 years and he reigned for sixteen years in Jerusalem. \v{9}Then Jotham died, as had his fathers, and he was buried in the City of David. His son Ahaz became king in his place.
\labelchapt{28}
\passage{Ahaz Succeeds Jotham}
\passageinfo{(2 Kings 16:1-4)}

\chapt{28}
\v{1}Ahaz was 20 years old when he began to reign, and he reigned 16 years in Jerusalem, but he did not practice what the \divine{Lord} considered to be right, as his ancestor David had done. \v{2}Instead, he lived like\fnote{Lit. \fbib{he walked in the ways}} the kings of Israel did. He cast metal images of Baal,\fnote{I.e. the supreme male deity of the Canaanites} \v{3}burned incense in the Ben-hinnom Valley, and burned his sons\fnote{Lit. \fbib{and passed his sons through fire}} as an offering, following the detestable activities of the nations whom the \divine{Lord} had expelled in front of the people of Israel. \v{4}He sacrificed and burned incense on high places, on the top of hills, and under every green tree.
\passage{Aram and Israel Defeat Judah}
\passageinfo{(2 Kings 16:5-6; Isaiah 7:1)}

\v{5}As a result, the \divine{Lord} his God handed Ahaz\fnote{Lit. \fbib{him}} over to the king of Aram, who defeated him and took a large number of captives away to Damascus. Ahaz\fnote{Lit. \fbib{He}} was also delivered over to the control of the King of Israel, who defeated him with many heavy casualties. \v{6}Remaliah's son Pekah killed 120,000 soldiers in a single day, all of them elite forces, because they had forsaken the \divine{Lord} God of their ancestors. \v{7}Zichri, a valiant soldier from Ephraim, killed the king's son Maaseiah, Azrikam, the palace manager, and Elkanah, who was second in rank to the king. \v{8}The Israelis carried away 200,000 women, sons, and daughters from among their own relatives. They also took a great deal of plunder, and brought it all to Samaria.
\passage{Oded the Prophet Rebukes Israel}

\v{9}But a prophet of the \divine{Lord} was there named Oded. He went out to greet the army as it arrived in Samaria. He warned them, ``Look! Because the \divine{Lord} God of your ancestors was angry at Judah, he delivered them into your control, but you have killed them with a vehemence that has reached all the way to heaven! \v{10}Now you're intending to make the men and women of Judah and Jerusalem to be your slaves. Surely you have your own sins against the \divine{Lord} your God for which you're accountable,\fnote{The Heb. lacks \fbib{for which you're accountable}} don't you? \v{11}So listen to me! Return the captives whom you've captured from your brothers, because the anger of the \divine{Lord} is burning hot against you!''

\v{12}Some of the leaders of the descendants of Ephraim, including Johanan's son Azariah, Meshillemoth's son Berechiah, Shallum's son Jehizkiah, and Hadlai's son Amasa, stood up to the army as they were coming back from the battle \v{13}and told them, ``Don't bring those captives here! You'll bring even more guilt on us from the \divine{Lord}, in addition to our own existing sin and guilt! He's already mad enough against Israel because of our guilt!''

\v{14}So the army abandoned the captives and the war booty in front of the officers and the entire assembled retinue. \v{15}After this, some men who were chosen by name took charge of the captives, clothed those who were naked with clothes appropriated from the war booty, gave them clothes and sandals, fed them, gave them something to drink, anointed them with oil, provided those who weren't able to walk\fnote{Lit. \fbib{who were feeble}} with donkeys to ride on, and took them back to their relatives at Jericho, the city of palm trees. Then they returned to Samaria.
\passage{Assyria Plunders the Temple}
\passageinfo{(2 Kings 16:7-9)}

\v{16}Right about then, King Ahaz sent for help from the kings of Assyria \v{17}because the Edomites had invaded, attacked Judah, and carried off some captives. \v{18}The Philistines also invaded some of the cities in the Shephelah\fnote{I.e. the verdant central lowlands of Israel; cf. Josh 10:40} and in the Negev\fnote{I.e. southern regions of the Sinai peninsula; cf. Josh 10:40} of Judah. They captured Beth-shemesh, Aijalon, Gederoth, Soco, and their surrounding villages, Timnah and its villages, and Gimzo and its villages. Then the Philistines\fnote{Lit. \fbib{Then they}} settled there, \v{19}because the \divine{Lord} was humiliating Judah because of King Ahaz of Israel, since Ahaz had brought about a lack of restraint within Judah and had remained unfaithful to the \divine{Lord}. \v{20}King Tiglath-pileser of Assyria attacked Ahaz\fnote{Lit. \fbib{him}} and, instead of helping him, attacked him. \v{21}Even though Ahaz took some of the assets belonging to the \divine{Lord}'s Temple from the royal palace, and from the palaces belonging to the princes, and gave them to the king of Assyria, none of his gifts did any good.
\passage{The Apostasy and Death of Ahaz}
\passageinfo{(2 Kings 16:12-20)}

\v{22}In the midst of his troubles, King Ahaz became more and more unfaithful to the \divine{Lord}. \v{23}He sacrificed to the gods of Damascus that had defeated him, reasoning, ``The gods of the kings of Aram helped them, so I'll sacrifice to them so they will help me!'' But those gods\fnote{Lit. \fbib{But they}} brought about his downfall, and the downfall of all of Israel, too. \v{24}Ahaz also collected the utensils of God's Temple, cut them all into pieces, and closed the doors of the \divine{Lord}'s Temple. Then he made altars to\fnote{Or \fbib{for}} himself on every corner in Jerusalem \v{25}and established high places in every city of Judah where incense was burned to other gods, thus provoking the \divine{Lord} God of his ancestors to anger. \v{26}The rest of his accomplishments, and records of everything he did from first to last are written in the Book of the Kings of Judah and Israel. \v{27}So Ahaz died, as had his ancestors, and he was buried in the city of Jerusalem, but they didn't bury him among the tombs of the kings of Israel. Ahaz's son Hezekiah reigned in his place.
\labelchapt{29}
\passage{Hezekiah Succeeds Ahaz}
\passageinfo{(2 Kings 18:1-3}

\chapt{29}
\v{1}Hezekiah began his reign at the age of 25. He reigned for 29 years in Jerusalem. His mother's name was Abijah, Zechariah's daughter. \v{2}He practiced what the \divine{Lord} considered to be right, following all of the examples set by his ancestor David.
\passage{Hezekiah's Temple Restoration Project}
\passageinfo{(2 Kings 18:4)}

\v{3}In the first month of the first year of his reign he repaired and reopened the doors of the \divine{Lord}'s Temple. \v{4}Then he brought in the priests and descendants of Levi, gathered them into the square in the eastern part of the Temple,\fnote{The Heb. lacks \fbib{part of the temple}} \v{5}and told them,

\begin{poetry}
\poeml ``Pay attention to me, you descendants of Levi! Consecrate yourselves and the Temple of the \divine{Lord} God of your ancestors by taking out from the Holy Place whatever is unclean. \v{6}Our ancestors have been unfaithful. They practiced what the \divine{Lord} considers to be evil, abandoned him, turned their faces away from the place where the \divine{Lord} resides, and turned their backs to him. \v{7}They shut the doors to the vestibule\fnote{Or \fbib{the outer courtyard}} of the Temple,\fnote{The Heb. lacks \fbib{of the temple}} extinguished its lamps, and have not burned incense or offered burnt offerings to the God of Israel in the Holy Place. \v{8}That's why the \divine{Lord} was angry with Judah and Jerusalem and made them an object of terror, horror, and derision, as you've seen with your own eyes. \v{9}Now look! Our ancestors have been killed with swords and our sons, daughters, and wives are being held captive because of all of this. \v{10}I'm intending to make a covenant with the \divine{Lord} God of Israel so his burning anger may turn away from us. \v{11}Please don't be careless, you descendants of Aaron,\fnote{The Heb. lacks \fbib{of Aaron}} because the \divine{Lord} has chosen you to minister in his presence, to serve him, to be his ministers, and to burn incense.''
\end{poetry}

\v{12}Here are the names of the descendants of Levi who made themselves available to God: Amasai's son Mahath and Azariah's son Joel from the descendants of Kohath; Abdi's son Kish and Jehallelel's son Azariah from the descendants of Merari; Zimmah's son Joah and Joah's son Eden from the descendants of Gershon; \v{13}Elizaphan's sons Shimri and Jeiel; Asaph's sons Zechariah and Mattaniah; \v{14}Heman's sons Jehiel and Shimei; and Jeduthun's sons Shemaiah and Uzziel. \v{15}They also brought together their brothers, consecrated themselves, and proceeded to cleanse the \divine{Lord}'s Temple, just as the king had ordered in accordance with what the \divine{Lord} had told him. \v{16}The priests entered the inner courts of the \divine{Lord}'s Temple to cleanse it, and they brought out everything unclean that they found there to the outer court of the \divine{Lord}'s Temple. Then the descendants of Levi carried everything from there out to the Kidron Valley. \v{17}They began their consecration duties on the first day of the first month and finished at the \divine{Lord}'s outer vestibule\fnote{Or \fbib{courtyard}} on the eighth day of the month. Another eight days was used to consecrate the \divine{Lord}'s Temple, so they completed the work on the sixteenth day of the first month.

\v{18}After this, they went to King Hezekiah and told him, ``We have cleansed all of the \divine{Lord}'s Temple, including the altar for burnt offerings, all of its utensils, the table of showbread, and all of its utensils. \v{19}In addition, we have prepared and rededicated all of the utensils that King Ahaz threw away during his unfaithful reign, and now they're back in service at the \divine{Lord}'s altar.''
\passage{Temple Worship is Restored}

\v{20}Early the next morning, King Hezekiah got up and assembled the city officials and went up to the \divine{Lord}'s Temple, \v{21}where they brought seven rams, seven lambs, and seven male goats for a sin offering on behalf of the kingdom, the Holy Place, and Judah. He ordered that the priests, as descendants of Aaron, place the offerings\fnote{The Heb. lacks \fbib{the offerings}} on the \divine{Lord}'s altar. \v{22}So they slaughtered the bulls and the priests sprinkled the blood on the altar. They also slaughtered the rams and sprinkled the blood on the altar, and they also slaughtered the lambs and sprinkled the blood on the altar. \v{23}They brought the male goats for the sin offering to the king within the assembled gathering, laid their hands on them, \v{24}and then the priests slaughtered them and purged the altar with their blood as a sin offering to atone for all Israel, because the king ordered that the burnt offering and the sin offering be made for all Israel.

\v{25}Hezekiah\fnote{Lit. \fbib{He}} stationed descendants of Levi in the \divine{Lord}'s Temple to play cymbals and stringed instruments, just as David, Gad the seer,\fnote{Cf. 2Sam 24:11} and Nathan the prophet\fnote{Cf. 2Sam 7:2} had directed, because the command to do so was from the \divine{Lord} through those prophets. \v{26}The descendants of Levi played instruments that had been crafted by David and the priests sounded trumpets.

\v{27}Hezekiah gave a command to offer burnt offerings on the altar, and when the burnt offerings began, a song to the \divine{Lord} also began with trumpets sounding and with the instruments that King David of Israel had crafted. \v{28}Everybody in the assembly worshipped, the singers sang, and the trumpets sounded. They continued doing this until the burnt offering sacrifice was completed. \v{29}When the sacrifices had been offered, the king and everyone else who was present with him bowed down and worshipped. \v{30}King Hezekiah and his officials ordered the descendants of Levi to sing praises to the \divine{Lord} based on psalms that had been written by David and Asaph the seer.\fnote{I.e. portions of the book of Psalms; cf. Prov25:1} So they all joyfully sang praises, bowed low, and worshipped.

\v{31}After this, Hezekiah announced, ``Now that you've consecrated yourselves to the \divine{Lord}, come near and bring your sacrifices and thanksgiving offerings to the \divine{Lord}'s Temple. So the assembly brought sacrifices and thanksgiving offerings, and everyone who was willing to do so brought burnt offerings. \v{32}The number of burnt offerings brought by the assembly was 70 bulls, 100 rams, and 200 lambs. All of these were burnt offerings to the \divine{Lord}. \v{33}The consecrated offerings numbered 600 bulls and 3,000 sheep. \v{34}Because there weren't enough priests, they were unable to prepare all the burnt offerings until other priests came forward after having consecrated themselves, so their descendant of Levi relatives assisted them until the services were complete. (The descendants of Levi had been more conscientious in consecrating themselves than had been the priests.) \v{35}Furthermore, there were also many burnt offerings, fat from peace offerings, and drink offerings. And that's how the service of the Lord's Temple was restored. \v{36}Hezekiah and all of the people were ecstatic with joy because of what God had done for the people, since everything had come about so suddenly.
\labelchapt{30}
\passage{Israel Celebrates the Passover}

\chapt{30}
\v{1}Hezekiah also sent word to all of Israel and Judah, and wrote letters to Ephraim and Manasseh that they should come to the \divine{Lord}'s Temple in Jerusalem to observe the Passover to the \divine{Lord} God of Israel. \v{2}The king, his princes, and the entire assembly in Jerusalem had mutually decided to observe the Passover in the second month, \v{3}but they had been unable to celebrate it then because not enough priests had consecrated themselves and the people had not yet been gathered together in Jerusalem. \v{4}This decision seemed to be a good one in the opinion of the king and of the entire assembly, \v{5}so they published a decree that was circulated throughout Israel from Beer-sheba to Dan that they are to come celebrate the Passover to the \divine{Lord} God of Israel in Jerusalem. The Passover\fnote{Lit. \fbib{Jerusalem, since they}} had not been celebrated in great numbers as was being prescribed by the decree.\fnote{The Heb. lacks \fbib{by the decree}}

\v{6}Couriers were sent throughout all of Israel and Judah with letters written by the king and his princes, just as the king had commanded:

\begin{poetry}
\poeml ``Listen, you descendants of Israel! Come back to the \divine{Lord} God of Abraham, Isaac, and Israel, so he may come back to those of you who have escaped and survived from domination by\fnote{Lit. \fbib{from the palm of}} the kings of Assyria. \v{7}Don't be like your ancestors and your relatives, who weren't faithful to the \divine{Lord} God of their ancestors, who, as a result, made them a desolate horror, as you well know. \v{8}So don't be stiff-necked like your ancestors were. Instead, submit to the \divine{Lord}, enter his sanctuary that he has sanctified forever, and serve the \divine{Lord} your God so that he'll stop being angry with you. \v{9}If you return to the \divine{Lord}, your relatives and children will receive compassion from those who took them away captive, and they'll return to this land, because the \divine{Lord} is both gracious and compassionate---he will not turn away from you if you return to him.''
\end{poetry}

\v{10}Couriers crossed from city to city throughout the territories of Ephraim and Manasseh as far as Zebulun, but those people\fnote{Lit. \fbib{but they}} just mocked them and laughed at them. \v{11}Nevertheless, a few men from Asher, Manasseh, and Zebulun humbled themselves and traveled to Jerusalem. \v{12}God also poured out his grace throughout\fnote{Lit. \fbib{The hand of God also rested on}} Judah, giving them a dedicated\fnote{Lit. \fbib{them one}} heart to do what the king and princes had decreed according to the message from the \divine{Lord}. \v{13}Many of the people gathered together in Jerusalem to observe the Festival of Unleavened Bread during the second month. It was a very large assembly. \v{14}They all got to work and removed the idolatrous\fnote{The Heb. lacks \fbib{idolatrous}} altars that were throughout Jerusalem. They also removed all the incense altars and threw them into the Kidron Brook. \v{15}Then they slaughtered the Passover lamb on the fourteenth day of the second month.

The priests and descendants of Levi felt ashamed of themselves, so they consecrated themselves and brought burnt offerings to the \divine{Lord}'s Temple. \v{16}Then they took their customary places, as the Law of Moses the man of God prescribes, and the priests sprinkled the blood that they were given by the descendants of Levi. \v{17}Because there were so many in the assembly that had not consecrated themselves, therefore the descendants of Levi supervised the slaughter of the Passover sacrifices on behalf of everyone who remained unclean, so they could be consecrated to the \divine{Lord}. \v{18}Even though a large crowd of people from as far away as Ephraim, Manasseh, Issachar, and Zebulun had not completed consecrating themselves, they still ate the Passover in a manner not proscribed by the Law,\fnote{The Heb. lacks \fbib{by the Law}} because Hezekiah had prayed like this for them: ``May the good \divine{Lord} extend a pardon on behalf of \v{19}everyone who prepares his own heart to seek God, the \divine{Lord} God of his ancestors, even though he does so inconsistent with the laws of consecration.'' \v{20}The \divine{Lord} listened to Hezekiah and healed the people.
\passage{The Festival of Unleavened Bread is Observed}

\v{21}The Israelis who were present in Jerusalem observed the Festival of Unleavened Bread for seven days with immense gladness, and the descendants of Levi and priests praised the \divine{Lord} throughout each day, singing mightily to the \divine{Lord}. \v{22}Hezekiah encouraged all the descendants of Levi who demonstrated significant insight in their service to the \divine{Lord}, so they all participated in the festival meals for seven days, all the while sacrificing peace offerings and giving thanks to the \divine{Lord} God of their ancestors. \v{23}After this, the whole assembly agreed to celebrate for another seven days, and so they did---and they were very happy to do so! \v{24}King Hezekiah of Judah gave the assembly 1,000 bulls and 7,000 sheep for offerings, and the princes contributed 1,000 bulls and 10,000 sheep, and a large number of priests consecrated themselves.

\v{25}Everyone in the assembly of Judah rejoiced, as did the priests, the descendants of Levi, and the people who gathered together from throughout Israel, including those who came from the land of Israel and those who lived in Judah. \v{26}There was great joy throughout Jerusalem, because nothing had happened like this in Jerusalem since the days of David's son Solomon, king of Israel. \v{27}After this, the priests arose, blessed the people, and their voices were heard in prayer all the way to heaven, where God resides in holiness.
\labelchapt{31}
\passage{Idols are Eliminated from Judah}
\passageinfo{(2 Kings 18:4)}

\chapt{31}
\v{1}At the conclusion of all of these activities, everybody in Israel who was in attendance traveled throughout the cities of Judah, broke down the sacred pillars, cut down the Asherim, and broke down the high places and altars throughout the territories of\fnote{The Heb. lacks \fbib{the territories of}} Judah, Benjamin, Ephraim, and Manasseh until they had eliminated all of them. Then the people of Israel went back to their cities and back to their work.\fnote{Lit. \fbib{possessions}}
\passage{Hezekiah Continues His Reforms}

\v{2}Hezekiah appointed the priestly divisions and the divisions of the descendants of Levi, each according to their service duties, including both priests and descendants of Levi who offered morning and evening burnt offerings, peace offerings, general\fnote{The Heb. lacks \fbib{general}} ministry, thanksgiving, and praise in the gateways to the \divine{Lord}'s campgrounds.\fnote{I.e. a portion of land set aside for temporary tents used by visitors to Jerusalem} \v{3}He also gave a portion of his own income for both morning and evening burnt offerings, for burnt offerings on the Sabbath, New Moons, and for the scheduled festivals, as is recorded in the \divine{Lord}'s Law.\fnote{Cf. Num 28:1-29:40} \v{4}Hezekiah\fnote{Lit. \fbib{He}} also directed the people who lived in Jerusalem to give what was due to the priests and descendants of Levi, so they could be strengthened in the \divine{Lord}'s Law. \v{5}As the word spread around, the people of Israel gave generously for the first fruits of grain, wine, oil, honey, and all of the produce of the fields. They generously gave a tithe of everything. \v{6}The descendants of Israel and Judah who lived throughout the cities of Judah also brought tithes of cattle and sheep, as well as tithes of gifts that had been dedicated to the \divine{Lord} their God.

As these gifts were given, they were laid in piles. \v{7}They began to make these piles of gifts\fnote{The Heb. lacks \fbib{of gifts}} during the third month, and it took them until the seventh month to finish. \v{8}When Hezekiah and the officials arrived and saw the piles of gifts,\fnote{The Heb. lacks \fbib{of gifts}} they blessed the \divine{Lord} and his people Israel, \v{9}and Hezekiah quizzed the priests and the descendants of Levi about the piles of gifts.\fnote{The Heb. lacks \fbib{of gifts}} \v{10}Azariah replied, ``Since they began to bring their gifts into the \divine{Lord}'s Temple, we have eaten and have been satisfied. Now we still have plenty left, because the \divine{Lord} has blessed his people so that we have all of this left over.''
\passage{The Priests and Descendants of Levi Reorganized}

\v{11}Hezekiah gave an order to prepare storerooms in the \divine{Lord}'s Temple, and so they did. \v{12}They faithfully brought in the gifts, tithes, and consecrated materials, and Conaniah the descendant of Levi was placed in charge of them. His brother Shimei was second in command, \v{13}Jehiel, Azaziah, Nahath, Asahel, Jerimoth, Jozabad, Eliel, Ismachiah, Mahath, and Benaiah served as supervisors under Conaniah and his brother Shimei, who had been appointed by King Hezekiah. Azariah served as senior officer of God's Temple. \v{14}Imnah the descendant of Levi's son Kore, keeper of the eastern gate, was in charge of voluntary offerings to God, apportioning contributions for the \divine{Lord} and the most holy things. \v{15}Under his authority, Eden, Miniamin, Jeshua, Shemaiah, Amariah, and Shecaniah served in the priestly cities, making sure contributions were distributed faithfully to their relatives division by division, no matter how large or how small, \v{16}without regard to genealogical enrollment, to every male 30\fnote{Lit. \fbib{three}; cf. 1Chr 23:3, which records 30 years of age as the year of enrollment eligibility} years old and older---that is, to everyone who entered the \divine{Lord}'s Temple as their duty obligations required---for their work and duties according to their divisions \v{17}as well as the priests who were enrolled in the genealogies according to their ancestral households. \v{18}These genealogical enrollments also included all of their little children, their wives, and their sons and daughters for the entire assembly, because they were being faithful to consecrating themselves in holiness. \v{19}Furthermore, with respect to the descendants of Aaron, that is, the priests who lived out in the country away from the cities, or who lived in each and every city, men were designated by name to distribute portions to every male among the priests and to everyone who had been enrolled by genealogy among the descendants of Levi.

\v{20}Hezekiah did this throughout all of Judah, and he acted well, doing what the \divine{Lord} his God considered to be right and true. \v{21}Everything that Hezekiah\fnote{Lit. \fbib{he}} began in the service of God's Temple was done according to the Law and to the commandments as he sought his God, worked with all of his heart, and became successful.
\labelchapt{32}
\passage{Sennacherib Invades Judah}
\passageinfo{(2 Kings 18:13-19:34; Isaiah 36:2-22)}

\chapt{32}
\v{1}After all of these acts of faithfulness occurred, King Sennacherib of Assyria came, invaded Judah, and laid siege to the fortified cities, thinking to conquer them for himself. \v{2}As soon as Hezekiah learned that Sennacherib had arrived and had determined to attack Jerusalem, \v{3}he developed a plan with his commanders and his elite forces to cut off the water supply from the springs that were outside the city, and they helped him to carry it out. \v{4}Many people gathered together and plugged up all the springs, along with the stream that flowed through the region. They were thinking to themselves, ``Why should the Assyrian kings invade and discover an abundant water supply?''

\v{5}Hezekiah took courage and rebuilt all of the walls that had been broken down. Then he erected watch towers on them, and added another external wall. He fortified the terrace ramparts\fnote{Lit. \fbib{the Millo}, fortified areas of ancient Jerusalem with terraces and retaining walls} in the City of David and prepared a large number of weapons and shields. \v{6}He appointed military officers to take charge of the people, who gathered them together in the square near the city gate and spoke to them encouragingly, \v{7}``Be strong and courageous.\fnote{Cf. Josh 1:7} Don't be afraid or disheartened because of the king of Assyria or because of the army that accompanies him, because the one who is with us is greater than the one with him. \v{8}He only has the strength of his own flesh, but the \divine{Lord} our God is with us to help us and to fight our battles.'' So the people were encouraged from what King Hezekiah of Judah told them.
\passage{Sennacherib Blasphemes God}
\passageinfo{(2 Kings 18:17-37)}

\v{9}After this, King Sennacherib of Assyria sent his messengers to Jerusalem while he was in the middle of a vigorous attack on Lachish. They delivered this message to King Hezekiah of Judah and to all the people of Judah who had gathered in Jerusalem:

\begin{poetry}
\poeml \v{10}``This is what King Sennacherib of Assyria says: `What are you leaning on that makes you stay behind while Jerusalem comes under siege? \v{11}Isn't Hezekiah lying to you so he can hand you over to die by famine and thirst? After all, he's telling you ``The \divine{Lord} our God will deliver us from the king of Assyria's control.''\fnote{Lit. \fbib{hand}} \v{12}Isn't this the very same Hezekiah who removed this god's high places and altars? Isn't this the same Hezekiah who\fnote{Lit. \fbib{altars and}} issued this order to Judah and Jerusalem: ``You are to worship in front of only one altar and burn your sacrifices only on it.''? \v{13}Don't you know what my predecessors\fnote{Lit. \fbib{fathers}} have done to all the other people in other lands? Were the gods of the people who lived in those lands able to deliver their countries out of my control?\fnote{Lit. \fbib{hand}} \v{14}What god, out of all the gods of those nations that my predecessors\fnote{Lit. \fbib{fathers}} utterly destroyed, has been able to deliver his people from my control\fnote{Lit. \fbib{hand}} or from the control\fnote{Lit. \fbib{hand}} of my predecessors?\fnote{Lit. \fbib{fathers}} \v{15}Now therefore, don't let Hezekiah lie to you or mislead you like this. Don't believe him, because no god of any nation has been able to deliver his people from my control\fnote{Lit. \fbib{hand}} or from the control\fnote{Lit. \fbib{hand}} of my predecessors. So how much less will your God deliver you from me?'\,''\fnote{Lit. \fbib{from my hand}}
\end{poetry}

\v{16}King Sennacherib's\fnote{Lit. \fbib{His}} spokesmen said even worse things against the \divine{Lord} God and against his servant Hezekiah.

\v{17}Sennacherib\fnote{Lit. \fbib{He}} also wrote letters like this that insulted and slandered the \divine{Lord} God of Israel: ``Just as the gods of the nations in other\fnote{Lit. \fbib{the}} lands haven't delivered their people from my control,\fnote{Lit. \fbib{hand}} so also the god of Hezekiah won't deliver his people from me!''\fnote{Lit. \fbib{from my hand}} \v{18}His spokesmen\fnote{Lit. \fbib{They}} shouted these things out with loud voices in the language of Judah to frighten and terrify the people of Jerusalem who were stationed on the city walls, to make it easier to conquer the city. \v{19}In doing so,\fnote{The Heb. lacks \fbib{In doing so}} they spoke about the God of Jerusalem as if he were like the gods of the nations of the earth that are made by the hands of human beings.
\passage{Sennacherib is Defeated and Killed}
\passageinfo{(2 Kings 19:35-37)}

\v{20}Meanwhile, King Hezekiah and Amoz's son Isaiah the prophet were praying about this and crying out to heaven. \v{21}So the \divine{Lord} sent an angel, who eliminated all of the elite forces, commanders, and officers within the encampment of the king of Assyria. As a result, he retreated to his own country, deeply ashamed and humiliated. When he visited the temple of his god, some of his sons killed him right there with swords. \v{22}That's how the \divine{Lord} delivered Hezekiah, as well as those who lived in Jerusalem, from Assyria's King Sennacherib and all his forces, and provided for all of their needs.\fnote{Or \fbib{and guided them on every side}} \v{23}Many brought gifts to the \divine{Lord} in Jerusalem and brought presents to King Hezekiah of Judah. As a result, he was exalted in the opinion of all nations thereafter.
\passage{Hezekiah's Illness and Recovery}
\passageinfo{(2 Kings 20:1-11; Isaiah 38:1-8)}

\v{24}During this time Hezekiah became critically ill, and he prayed to the \divine{Lord}. The \divine{Lord} spoke to him and gave him a sign.\fnote{Cf. Isa 38:7-8} \v{25}But Hezekiah's response wasn't commensurate with what had been done for him because he was arrogant in heart, so wrath came upon him, upon Judah, and upon Jerusalem. \v{26}But Hezekiah humbled himself while he was arrogant in heart, and the inhabitants of Jerusalem joined him in this. As a result, the \divine{Lord}'s wrath did not come upon them during Hezekiah's lifetime.
\passage{Hezekiah's Wealth and Accomplishments}
\passageinfo{(2 Kings 20:12-21; Isaiah 39:1-8)}

\v{27}Hezekiah received immense wealth and honor. He built treasuries for himself to store silver, gold, precious stones, spices, shields, and all sorts of valuable items, \v{28}along with storage facilities for grain, wine, oil, stalls for all sorts of cattle, and sheepfolds for his flocks. \v{29}He also built cities for himself and stored up flocks and herds in abundance, because God had given him great riches. \v{30}Hezekiah stopped up the upper outlet of the Gihon springs and diverted them down to the western side of the City of David. He prospered in everything he did.
\passage{Hezekiah's Heart is Tested by God}

\v{31}Later on, envoys came from the princes of Babylon to inquire about the miracle that had happened in the land.\fnote{I.e. the miracle recorded in Isa 38:7-8 and alluded to in v. 24} God left Hezekiah\fnote{Lit. \fbib{him}} to himself, so that he might make known\fnote{Or \fbib{know}} what was really in Hezekiah's\fnote{Lit. \fbib{his}} heart. \v{32}Now the rest of Hezekiah's accomplishments and his faithful deeds are recorded in the vision of Amoz's son Isaiah the prophet, and in the Book of the Kings of Judah and Israel. \v{33}Hezekiah died, as had his fathers, and they buried him in the upper part of the tombs of the descendants of David. All of Judah and the inhabitants of Jerusalem honored him at his death. But his son Manasseh reigned in his place.
\labelchapt{33}
\passage{Manasseh Succeeds Hezekiah}
\passageinfo{(2 Kings 21:1-9)}

\chapt{33}
\v{1}Manasseh began to reign at the age of twelve years, and continued to reign for 55 years in Jerusalem. \v{2}But he practiced what the \divine{Lord} considered to be evil by behaving detestably, as did the nations whom the \divine{Lord} expelled in front of the Israelis.
\passage{The Sins of Manasseh}

\v{3}He re-established the high places that his father Hezekiah had demolished, he built altars to the Baals, erected Asherim, and worshipped and served the armies\fnote{Or \fbib{stars}} of heaven. \v{4}He also built altars in the \divine{Lord}'s Temple, about which the \divine{Lord} had spoken ``My name will reside in Jerusalem forever.''\fnote{Cf. 2Sam 7:13; 2Chr 7:16} \v{5}He built altars for all the armies\fnote{Or \fbib{stars}} of heaven in the two courtyards of the \divine{Lord}'s Temple.\fnote{I.e. the court of the priests and the great court; cf. 2Chr 4:9} \v{6}He burned his sons\fnote{Lit. \fbib{He passed his sons through fire}} as an offering in the Ben-hinnom Valley, practiced fortune-telling, witchcraft, sorcery, and communicated with mediums and separatists. He did a lot of things that the \divine{Lord} considered to be evil, thus provoking him. \v{7}He also placed an image that he had carved in God's Temple, the place about which God had told to David and to his son Solomon, ``I will place my name in this Temple and in Jerusalem, which I have chosen out of all the tribes of Israel,''\fnote{Cf. 1King 9:3-5; 2Chr 7:16; 33:4} \v{8}and ``I won't let Israel's foothold slip on the land that I've given to your ancestors, if only they will be careful to keep everything that I commanded them in the Law, in the statutes, and in the ordinance through Moses.''\fnote{Cf. 2Sam 7:10} \v{9}This is how Manasseh deceived Judah and the inhabitants of Jerusalem to practice more evil than the nations whom the \divine{Lord} had eliminated in front of the Israelis.
\passage{Manasseh Repents and is Restored}

\v{10}The \divine{Lord} kept on speaking to Manasseh and to his people, but they paid no attention to him, \v{11}so the \divine{Lord} brought in the army commanders who worked for the king of Assyria, who captured Manasseh with hooks, bound him in bronze chains, and took him off to Babylon. \v{12}But when he was in trouble, he sought the face of the \divine{Lord} his God, humbled himself magnificently before the God of his ancestors, \v{13}and prayed to him. Moved by Manasseh's\fnote{Lit. \fbib{his}} entreaties, the \divine{Lord} heard his supplications and brought him back to his kingdom in Jerusalem. That's how Manasseh learned that the \divine{Lord} is God.

\v{14}Later on, Manasseh\fnote{Lit. \fbib{he}} reinforced the outer wall to the City of David on the west side overlooking the Gihon Valley as far as the Fish Gate. He encircled the Ophel,\fnote{I.e. a ridge of hills in Jerusalem fortified for defense of the city; cf. 2 Chr 27:3} raising it to a great height. \v{15}He also eliminated the foreign gods and idols from the \divine{Lord}'s Temple, along with all of the altars that he had built in Jerusalem and on the mountain where the \divine{Lord}'s Temple was located, and he discarded them outside the city. \v{16}He set up an altar to the \divine{Lord}, sacrificed peace offerings on it, and ordered Judah to serve the \divine{Lord} God of Israel. \v{17}Even so, the people continued to sacrifice in the high places, but only to the \divine{Lord} their God.
\passage{The Death of Manasseh}
\passageinfo{(2 Kings 21:17-18)}

\v{18}Now as to the rest of Manasseh's accomplishments, including his prayer to God and what the seers had to say to him in the name of the \divine{Lord} God of Israel, they are included among the Acts of the Kings of Israel. \v{19}His prayer, how God was moved by him, all of his sin and unfaithfulness, and a record of the sites where he constructed high places, erected Asherim and carved images before he humbled himself are written in the Acts of the Seers.\fnote{Or \fbib{the Record Keepers}} \v{20}So Manasseh died, as had his ancestors, and they buried him in his own palace while his son Amon became king in his place.
\passage{Amon's Reign and Death}
\passageinfo{(2 Kings 21:19-26)}

\v{21}Amon was 22 years old when he became king, and he reigned two years in Jerusalem. \v{22}He practiced what the \divine{Lord} considered to be evil, just as his father Manasseh had done, sacrificing to and serving all the carved images that his father Manasseh had made, \v{23}except that he never humbled himself to the \divine{Lord} like his father Manasseh had done. In fact, Amon multiplied his own guilt \v{24}until his servants finally conspired against him and executed him in his own palace. \v{25}But the people of the land executed all of the conspirators against King Amon and installed his son Josiah as king to succeed him.
\labelchapt{34}
\passage{Josiah Succeeds Amon}
\passageinfo{(2 Kins 22:1-2)}

\chapt{34}
\v{1}Josiah was eight years old when he began to reign, and he reigned for 31 years in Jerusalem. \v{2}He practiced what the \divine{Lord} considered to be right, following the example\fnote{Lit. \fbib{right, walking in the ways}} of his ancestor David, turning neither to the right nor to the left. \v{3}In the eighth year of his reign, while he was still young, he began to seek the God of his ancestor David. In the twelfth year of his reign,\fnote{The Heb. lacks \fbib{of his reign}} he began to remove the high places, Asherim, carved images, and cast images from Judah and Jerusalem.

\v{4}They tore down the altars of Baals in his presence. He chopped down the incense altars that stood high above them. He broke into pieces the Asherim, the carved images, and the cast images, ground them to dust, and scattered the residue on the graves of those who had sacrificed to them. \v{5}He burned the bones of the priests on their altars, thus purging Judah and Jerusalem. \v{6}In the cities of Manasseh, Ephraim, Simeon, and as far as Naphtali and their surrounding ruins, \v{7}he also tore down altars, destroyed the Asherim and the carved images, grinding them\fnote{The Heb. lacks \fbib{grinding them}} into dust, and chopped down all the incense altars throughout the land of Israel. Then he went back to Jerusalem.
\passage{Josiah's Restoration Work}
\passageinfo{(2 Kings 22:3-20)}

\v{8}In the eighteenth year of his reign, after he had purged the land and the Temple, he sent Azaliah's son Shaphan, Maaseiah, mayor\fnote{Lit. \fbib{governor}} of Jerusalem,\fnote{Lit. \fbib{of the city}} and Joahaz's son Joah, the recorder, to repair the Temple of the \divine{Lord} his God. \v{9}They approached Hilkiah the high priest and delivered to him the money that had been brought into God's Temple that the descendants of Levi and gatekeepers had collected from Manasseh, Ephraim, the surviving Israelis, Judah, Benjamin, and the inhabitants of Jerusalem. \v{10}They paid it to the workmen who supervised the \divine{Lord}'s Temple, and the workmen who were employed in the \divine{Lord}'s Temple to supervise restoration and repair of the Temple. \v{11}They, in turn, paid the carpenters and builders to purchase quarried stone and timber for binders and beams for the buildings that previous\fnote{The Heb. lacks \fbib{previous}} kings of Judah had let deteriorate. \v{12}The workmen did their duties faithfully with these foremen supervising them: Jahath and Obadiah, descendants of Levi who were Merari's sons, Zechariah and Meshullam, descendants of Kohath, and various descendants of Levi, who were skilled musicians. \v{13}These men also supervised the heavy lift workers and supervised all the workmen from job to job, while some of the descendants of Levi served as scribes, officials, and gatekeepers.
\passage{The Book of the Law is Discovered}
\passageinfo{(2 Kings 22:3-20)}

\v{14}While they were bringing out the money that had come in as gifts to the \divine{Lord}'s Temple, Hilkiah the priest discovered the Book of the Law of the \divine{Lord} that had been handed down by Moses. \v{15}Hilkiah reported his finding to Shaphan the scribe, telling him, ``I found the Book of the Law in the \divine{Lord}'s Temple. Then he gave the book to Shaphan. \v{16}Shaphan took the book to the king and gave an additional report to the king, telling him ``Everything that you've entrusted to your servants is being carried out. \v{17}They've removed the money that was found in the \divine{Lord}'s Temple and have passed it on to the supervisors and the workmen.'' \v{18}Shaphan the scribe also informed the king, ``Hilkiah the priest gave me a book.'' Shaphan read from its contents to the king.

\v{19}As soon as he heard what the Law said, he tore his clothes. \v{20}He issued these orders to Hilkiah, Shaphan's son Ahikam, Micah's son Abdon, Shaphan the scribe, and the king's personal assistant Asaiah: \v{21}``Go ask the \divine{Lord} for me and for those who survive in Israel and in Judah about the words that we've read in this book that we found, because the wrath of the \divine{Lord} that we deserve to have poured out on us is very great, since our ancestors haven't obeyed the command from\fnote{Lit. \fbib{the word of}} the \divine{Lord} that required us to do everything that is written in this book.''
\passage{Hilkiah Consults with Huldah, the Woman Prophet}

\v{22}So Hilkiah and the others who had received orders from the king went to visit Huldah the prophetess, the wife of Tokhath's son Shallum, grandson of Hasrah. She was the king's wardrobe supervisor, and she lived in Jerusalem's Second Quarter. They asked her about what had happened. \v{23}In response, she replied:

\begin{poetry}
\poeml ``This is what the \divine{Lord} God of Israel says: `Tell the man who sent you to me, \v{24}``This is what the \divine{Lord} says: `Pay attention! I'm bringing evil to visit this place and its inhabitants---every single curse written in the book that they've been reading to the King of Judah. \v{25}Because they abandoned me and have burned incense to other gods, provoking me to become angry at everything they're doing,\fnote{Lit. \fbib{doing with their hands}} therefore my wrath is about to be poured out on this place, and it won't be quenched.'\,''\,' \\
\poeml \v{26}``Now tell the king of Judah who sent you to ask the \divine{Lord} about this: `This is what the \divine{Lord} God of Israel says about what you've heard: \v{27}``Because your heart was sensitive, and you humbled yourself before God when you heard what he had to say about this place and its inhabitants---indeed, because you humbled yourself before me, tore your clothes, and cried out to me, I have heard you,'' declares the \divine{Lord}. \v{28}``Look! I'm going to take you to your ancestors, and you will be buried in your grave in peace so that you won't have to see all the evil that I'm going to bring to this place and to its inhabitants.''\,'\,''
\end{poetry}

So they all brought back this message to the king.
\passage{The Covenant is Renewed}
\passageinfo{(2 Kings 23:1-20)}

\v{29}The king sent word to gather all the elders of Judah and Jerusalem. \v{30}Then the king went up to the \divine{Lord}'s Temple, accompanied by the men of Judah, the inhabitants of Jerusalem, the priests and descendants of Levi, and everyone else from the most important to the least important, and he read out loud\fnote{Lit. \fbib{read in their hearing}} all the words of the book of the covenant that had been found in the \divine{Lord}'s Temple. \v{31}While standing in his appointed place, the king made a public covenant with the \divine{Lord} to follow the \divine{Lord}, to keep his commandments, his testimonies, and his statutes, and to do so with all of his heart and soul, and to carry out what was written in the covenant contained in the book. \v{32}He also made everyone who was present in Jerusalem and Benjamin to stand in agreement with him. As a result, the inhabitants of Jerusalem reconfirmed the covenant of God, the God of their ancestors. \v{33}Josiah also removed all the detestable things from the territories that belonged to the people of Israel, and made everyone who lived in Israel to serve the \divine{Lord} their God. For the rest of his life, they didn't abandon their quest to follow the \divine{Lord} God of their ancestors.
\labelchapt{35}
\passage{Passover is Observed Again}
\passageinfo{(2 Kings 23:21-23)}

\chapt{35}
\v{1}Josiah observed the Passover to the \divine{Lord} in Jerusalem. They slaughtered the Passover on the fourteenth day of the first month. \v{2}He appointed priests to their offices, encouraging them in their service at the \divine{Lord}'s Temple. \v{3}He addressed the descendants of Levi who were teaching all Israel and who had consecrated themselves to the \divine{Lord}, telling them:

\begin{poetry}
\poeml ``Put the holy ark in the Temple that Solomon, the son of Israel's King David, built. It will no longer be a burden on their shoulders. Now go serve the \divine{Lord} your God and his people Israel. \v{4}Prepare yourselves by divisions according to your ancestral households, keeping to what King David of Israel and his son Solomon wrote about this.\fnote{The Heb. lacks \fbib{about this}} \v{5}In addition to this, take your place in the Holy Place according to the groupings of the ancestral households of your relatives consistent with the division of the descendants of Levi by their ancestral households. \v{6}Now slaughter the Passover, consecrate yourselves, and prepare your relatives to obey the command from\fnote{Lit. \fbib{the word of}} the \divine{Lord} given by Moses.''
\end{poetry}

\v{7}Josiah contributed 30,000 animals from the flocks of lambs and young goats, giving Passover offerings to all of the people who were present, plus an additional 3,000 bulls from the king's private possessions. \v{8}His officers contributed a voluntary offering to the people, the priests, and the descendants of Levi. Hilkiah, Zechariah, and Jehiel, the officials who supervised God's Temple, gave 2,600 animals from their flocks to the priests for Passover offerings, along with 300 bulls. \v{9}Also, Conaniah, and his relatives Shemaiah, and Nethanel, along with Hashabiah, Jeiel, and Jozabad, the officers in charge of the descendants of Levi, contributed 5,000 animals from the flocks to the descendants of Levi for the Passover offerings, along with 500 bulls. \v{10}As a result, the Passover service was prepared, the priests took their assigned places, and the descendants of Levi stood in their divisions as the king had commanded.

\v{11}They slaughtered the Passover lamb, and the priests poured out the blood that they had received from the lambs\fnote{Lit. \fbib{from them}} while the descendants of Levi flayed the sacrifices. \v{12}They set aside in reserve the burnt offerings, so they could distribute them in proportion to the divisions of their ancestral households for presentation by the people to the \divine{Lord}, as is required by the book of Moses. They did this with respect to the bulls, also. \v{13}They roasted the Passover in fire, as required by the ordinances, and boiled the holy things in pots, kettles, and pans, and delivered them quickly to all the people. \v{14}After this, because the priests, who were descendants of Aaron, were busy offering the burnt offerings and fat portions until evening, the descendants of Levi prepared the Passover for themselves and their fellow-descendants of Aaron, the priests. \v{15}The singers, as descendants of Asaph, remained at their stations as David, Asaph, Heman, and the king's seer Jeduthun required, and the gatekeepers did not have to leave their posts because their descendant of Levi relatives prepared the Passover for them.

\v{16}That's how the \divine{Lord}'s service was prepared that day to celebrate the Passover and to offer burnt offerings on the \divine{Lord}'s altar according to what King Josiah had commanded. \v{17}The Israelis who were present celebrated the Passover that day, as well as the Festival of Unleavened Bread for seven days. \v{18}There had not been a Passover celebration like it in Israel since Samuel the prophet was alive, nor had any of the kings of Israel celebrated a Passover like Josiah did at that time\fnote{The Heb. lacks \fbib{at that time}} with the priests, the descendants of Levi, everyone from Judah and Israel who were present, and the inhabitants of Jerusalem. \v{19}This Passover celebration was observed during the eighteenth year of the reign of Josiah.
\passage{Pharaoh Neco and Josiah's Death}
\passageinfo{(2 Kings 23:29-30)}

\v{20}Some time after all of this, after Josiah had finished preparing the Temple, King Neco of Egypt invaded Carchemish on the Euphrates River,\fnote{The Heb. lacks \fbib{River}} and Josiah went out to fight him. \v{21}But he sent messengers to him, who asked him, ``What do we have in common, King of Judah? I am not here today opposing you. I am fighting the dynasty that is fighting me, and God has ordered me to hurry. For your own good, stop interfering with God, who is with me, and he won't destroy you!''

\v{22}But Josiah wouldn't turn around. In fact, he put on a disguise so he could fight Neco.\fnote{Lit. \fbib{him}} He wouldn't listen to what God told him through what Neco had to say, and as a result, Josiah came to attack Neco\fnote{The Heb. lacks \fbib{Neco}} on the Megiddo plain. \v{23}Some archers shot King Josiah, and the king told his servants, ``Take me away, because I'm badly wounded.'' \v{24}So his servants removed him from the chariot he was in and carried him away in a backup chariot that he had and took him back to Jerusalem, where he died and was buried in the tombs of his ancestors. All of Judah and Jerusalem went into mourning for Josiah.

\v{25}Jeremiah sang a lament for Josiah, and all the male and female singers recite that lamentation about Josiah to this day. In fact, they made singing it an ordinance in Israel, and they are recorded in the Lamentations.\fnote{This is \fbib{not} a reference to the Book of Lamentations in the Bible.} \v{26}Now the rest of the accomplishments of Josiah, including his faithful acts of devotion as required in the Law of the \divine{Lord}, \v{27}and his other\fnote{The Heb. lacks \fbib{other}} activities from first to last, are recorded in the Book of the Kings of Israel and Judah.
\labelchapt{36}
\passage{Jehoahaz Becomes King}
\passageinfo{(2 Kings 23:31-33)}

\chapt{36}
\v{1}After this, the people of the land installed Josiah's son Jehoahaz in Jerusalem as king to take his father's place. \v{2}Jehoahaz was 23 years old when he became king, and he reigned for three months in Jerusalem, \v{3}after which the king of Egypt dethroned him and imposed a fine on the land of 100 talents\fnote{I.e. about 7,500 pounds; a talent weighed about 75 pounds} of silver and one talent\fnote{I.e. about 75 pounds; a talent weighed about 75 pounds} of gold. \v{4}King Neco of Egypt installed Jehoahaz's\fnote{Lit. \fbib{his}} brother Eliakim as king over Judah and Jerusalem, changed Eliakim's name to Jehoiakim, and took his brother Joahaz back to Egypt.
\passage{Jehoiakim's Reign; Nebuchadnezzar's First Capture of Jerusalem}

\v{5}Jehoiakim was 25 years old when he became king, and he reigned eleven years in Jerusalem, but he practiced what the \divine{Lord} his God considered to be evil. \v{6}As a result, King Nebuchadnezzar of Babylon attacked him, bound him in bronze shackles, and took him to Babylon. \v{7}Nebuchadnezzar also took articles from the \divine{Lord}'s Temple to Babylon and placed them in his temple in Babylon. \v{8}The rest of Jehoiakim's accomplishments---along with the detestable things that he did that were recorded in his disfavor---are written in the Book of the Kings of Israel and Judah. His son Jehoiachin became king to replace him.
\passage{Jechoiachin's Reign; Nebuchadnezzar's Second Capture of Jerusalem}

\v{9}Jehoiachin was eight years old when he became king, and he reigned for three months and ten days in Jerusalem, all the while doing what the \divine{Lord} considered to be evil. \v{10}At the beginning of the next year, King Nebuchadnezzar sent for him and brought him to Babylon, along with valuable articles from the \divine{Lord}'s Temple, and he installed Jehoiachin's relative Zedekiah as king over Judah and Jerusalem.
\passage{Zedekiah Rules in Judah}
\passageinfo{(2 Kings 24:18-20; Jeremiah 52:1-3a)}

\v{11}Zedekiah was 21 years old when he became king, and he reigned for eleven years in Jerusalem. \v{12}He practiced what the \divine{Lord} his God considered to be evil and never humbled himself before Jeremiah the prophet who spoke for the \divine{Lord}. \v{13}Zedekiah rebelled against King Nebuchadnezzar, who had made him swear allegiance in the name of\fnote{The Heb. lacks \fbib{allegiance in the name of}} God. Instead, he stiffened his resolve,\fnote{Lit. \fbib{neck}} and hardened his heart, and would not return to the \divine{Lord} God of Israel.
\passage{Nebuchadnezzar's Third Capture of Jerusalem}
\passageinfo{(2 Kings 25:1-21; Jeremiah 52:3b-30)}

\v{14}Meanwhile, all the officials who supervised the priests and the people remained unfaithful, following the detestable example of the surrounding nations. They polluted the \divine{Lord}'s Temple that he had consecrated in Jerusalem. \v{15}The \divine{Lord} God of their ancestors pleaded with them time and again through his messengers, because he had compassion on his people and on the place of his residence, \v{16}but they mocked God's messengers, despised his words, and scoffed at his prophets, until there was no remedy for the wrath of the \divine{Lord} that arose to punish\fnote{Lit. \fbib{arose against}} his people. \v{17}Therefore he brought up the king of the Chaldeans against them, who executed their young men in the holy Temple, showing no compassion on young man or young virgin, adult men or the aged. God gave them all into the king's control, \v{18}who took back to Babylon every article in God's Temple, whether large or small, including the treasuries of the \divine{Lord}'s Temple, the king's assets, and those of his officers. \v{19}After this, they set fire to God's Temple, demolished the wall around Jerusalem, burned all of its fortified buildings, and destroyed everything of value. \v{20}Nebuchadnezzar\fnote{Lit. \fbib{He}} carried off to Babylon those who survived the executions, and they served him and his descendants until the kingdom of Persia came to power. \v{21}All of this fulfilled what the \divine{Lord} had predicted through Jeremiah. And so the land enjoyed its Sabbaths, and the length of the land's desolation lasted until a 70-year long Sabbath had been completed.
\passage{An Edict to Rebuild the Temple}
\passageinfo{(Ezra 1:1-4)}

\v{22}During the first year of Cyrus, king of Persia, in fulfillment of the message from the \divine{Lord} spoken by Jeremiah, the \divine{Lord} prompted\fnote{Lit. \fbib{\divine{Lord} stirred up the spirit of}} Cyrus, king of Persia, to make this proclamation throughout his entire kingdom, which was also released in written form:

\v{23}\divine{An Official Statement} \divine{from}\fnote{Lit. \fbib{Thus says}} \divine{Cyrus, King of Persia}

\begin{poetry}
\poeml All of the kingdoms of the earth have been given to me by the \divine{Lord} God of Heaven, and he specifically charged me to build a temple\fnote{Or \fbib{house}} for him in Jerusalem, which is in Judah. Therefore, who among the \divine{Lord}'s\fnote{Lit. \fbib{among all of his}} people trusts in his God? Whoever among this group wishes to do so may travel to Jerusalem.\fnote{The Heb. lacks \fbib{to Jerusalem}}
\end{poetry}

\addcontentsline{toc}{chapter}{History After the Exile}
\bookheader{Ezra}
\labelbook{Ezra}

\bookpretitle{The Book of}
\booktitle{Ezra}

\labelchapt{1}
\passage{An Edict to Rebuild the Temple}
\passageinfo{(2 Chronicles 36:22-23)}

\chapt{1}
\v{1}During the first year of Cyrus, king of Persia, in fulfillment of the message from the \divine{Lord} spoken through Jeremiah, the \divine{Lord} prompted\fnote{Lit. \fbib{stirred up the spirit of}} Cyrus, king of Persia, to make this proclamation throughout his entire kingdom, which was also released in written form:

\v{2}\divine{An Official Statement}

\divine{from}\fnote{Lit. \fbib{Thus says}} \divine{Cyrus, King of Persia}

\begin{poetry}
\poeml All of the kingdoms of the earth have been given to me by the \divine{Lord} God of Heaven, and he specifically charged me to build a temple\fnote{Or \fbib{house}, and so throughout the book} for him in Jerusalem, which is in Judah. \v{3}Therefore, who among the \divine{Lord}'s\fnote{Lit. \fbib{among all of his}} people trusts in his God? Whoever among this group wishes to do so may travel to Jerusalem of Judah to rebuild the Temple of the \divine{Lord}\fnote{Lit. \fbib{of his \divine{Lord}}} God of Israel, the God of Jerusalem. \v{4}Furthermore, everyone who wishes to repatriate\fnote{Lit. \fbib{who remains}} from any territory where he now resides is to receive assistance from his fellow residents in the form of silver, gold, equipment, and pack animals, in addition to voluntary offerings for the Temple of the God of Jerusalem.
\end{poetry}

\v{5}In response, the heads of the families\fnote{Lit. \fbib{fathers}} of Judah and Benjamin, the priests and descendants of Levi, and all those who had been prompted\fnote{Lit. \fbib{all whose spirit had been stirred up}} by God, prepared to travel to rebuild the Temple of the \divine{Lord}, which was in Jerusalem. \v{6}So all of their neighbors equipped the travelers\fnote{Lit. \fbib{strengthened their hands}} with silver, gold, equipment, pack animals, and valuable goods, in addition to voluntary offerings.
\passage{Temple Instruments Returned}

\v{7}King Cyrus also brought out from storage\fnote{The Heb. lacks \fbib{from storage}} the service instruments from the Temple of the \divine{Lord}, which Nebuchadnezzar had taken from Jerusalem and had placed in the temple of his gods.\fnote{LXX \fbib{his god}} \v{8}Cyrus, king of Persia, had them brought out to Mithredath the Treasurer, had them inventoried, and had them placed in care of\fnote{Lit. \fbib{Treasurer, and numbered them to}} Sheshbazzar,\fnote{I.e. Zerubbabel; \fbib{Sheshbazzar} is the Persian equivalent (cf. 2:2)} governor of Judah. \v{9}Here is a partial inventory:\fnote{Lit. \fbib{This was their number}}

Gold dishes: 30

Silver dishes: 1,000

Sacrificial knives: 29

\v{10}Gold bowls: 30

Silver bowls of another kind:\fnote{Lit. \fbib{of a second}} 410

Miscellaneous instruments: 1,000

\v{11}The complete inventory of gold and silver vessels totaled 5,400. Sheshbazzar\fnote{I.e. Zerubbabel; \fbib{Sheshbazzar} is the Persian equivalent (cf. 2:2)} brought them all to Jerusalem, along with the exiles from Babylon.
\labelchapt{2}
\passage{A List of Those who Returned}
\passageinfo{(Nehemiah 7:6-73)}

\chapt{2}
\v{1}Here is a list\fnote{Cf. Neh 7:6} of descendants of the province of Judah\fnote{The Heb. lacks \fbib{of Judah}} who returned from the captivity, from those who had been exiled. Nebuchadnezzar, king of Babylon, had taken them to Babylon. They came back to Jerusalem and Judah, each one to his town, \v{2}along with Zerubbabel, Jeshua, Nehemiah, Seraiah, Reelaiah,\fnote{MT of Neh 7:7 lacks \fbib{Seraiah, Reelaiah}} Mordecai, Bilshan, Mispar,\fnote{Cf. Neh 7:7 \fbib{Mispereth}} Bigvai, Rehum,\fnote{Cf. Neh 7:7 \fbib{Nehum}} and Baanah. Here is the enumeration of:

The Men of Israel:

\v{3}Descendants of\fnote{Lit. \fbib{Sons of}; and so throughout the chapter} Parosh: 2,172

\v{4}Descendants of Shephatiah: 372

\v{5}Descendants of Arah: 775\fnote{Cf. Neh 7:10 \fbib{652}}

\v{6}Descendants of Pahath-moab; that is, through Jeshua and Joab: 2,812\fnote{Cf. Neh 7:11 \fbib{2,818}}

\v{7}Descendants of Elam: 1,254

\v{8}Descendants of Zattu: 945\fnote{Cf. Neh 7:13 \fbib{845}}

\v{9}Descendants of Zaccai: 760

\v{10}Descendants of Bani:\fnote{Cf. Neh 7:13 \fbib{Binnui}} 642\fnote{Cf. Neh 7:13 \fbib{648}}

\v{11}Descendants of Bebai: 623\fnote{Cf. Neh 7:16 \fbib{628}}

\v{12}Descendants of Azgad: 1,222\fnote{Cf. Neh 7:17 \fbib{2,322}}

\v{13}Descendants of Adonikam: 666\fnote{Cf. Neh 7:18 \fbib{667}}

\v{14}Descendants of Bigvai: 2,056\fnote{Cf. Neh 7:19 \fbib{2,067}}

\v{15}Descendants of Adin: 454\fnote{Cf. Neh 7:20 \fbib{655}}

\v{16}Descendants of Ater through Hezekiah: 98

\v{17}Descendants of Bezai: 323\fnote{Cf. Neh 7:22 \fbib{328}}

\v{18}Descendants of Jorah:\fnote{Cf. Neh 7:24 \fbib{Hariph}} 112

\v{19}Descendants of Hashum: 223\fnote{Cf. Neh 7:22 \fbib{328}}

\v{20}Descendants of Gibbar:\fnote{Cf. Neh 7:25 \fbib{Gibeon}} 95

\v{21}Descendants of exiles from\fnote{The Heb. lacks \fbib{exiles from}; and so through v. 35} Bethlehem: 123

\v{22}People from\fnote{Lit. \fbib{Men of}; and so in vv. 23, 27, and 28} Netophah: 56\fnote{Cf. Neh 7:26, where the combined total is \fbib{188}}

\v{23}People from Anathoth: 128

\v{24}Descendants of exiles from Azmaveth:\fnote{Cf. Neh 7:28 \fbib{Beth-azmaveth}} 42

\v{25}Descendants of exiles from Kiriath-arim;\fnote{Cf. Neh 7:29 \fbib{Kiriath-jearim}} that is, Chephirah and Beeroth: 743

\v{26}Descendants of exiles from Ramah and Geba: 621

\v{27}People from Michmas: 122

\v{28}People from Bethel and Ai: 223\fnote{Cf. Neh 7:32 \fbib{123}}

\v{29}Descendants of exiles from Nebo: 52

\v{30}Descendants of exiles from Magbish: 156

\v{31}Descendants of exiles from the other Elam: 1,254

\v{32}Descendants of exiles from Harim: 320

\v{33}Descendants of exiles from Lod, Hadid, and Ono: 725\fnote{Cf. Neh 7:37 \fbib{721}}

\v{34}Descendants of exiles from Jericho: 345

\v{35}Descendants of exiles from Senaah: 3,630\fnote{Cf. Neh 7:38 3,930}

\v{36}The Priests:

Descendants of Jedaiah from the household of Jeshua: 973

\v{37}Descendants of Immer: 1,052

\v{38}Descendants of Pashhur: 1,247

\v{39}Descendants of Harim: 1,017

\v{40}The Descendants of Levi:

Descendants of Jeshua and Kadmiel; that is, descendants of Hodaviah:\fnote{Cf. Neh 7:43 \fbib{Hodevah}} 74

\v{41}The Singers:

Descendants of Asaph: 128\fnote{Cf. Neh 7:44 \fbib{148}}

\v{42}The Descendants of the Gatekeepers:

Descendants of Shallum, Ater, Talmon, Akkub, Hatita, and Shobai, totaling: 139\fnote{Cf. Neh 7:45 \fbib{138}}

\v{43}The Temple Servants:\fnote{Heb. \fbib{Nethinim}; i.e. a division of special assistants to the descendants of Levi, originally appointed by King David; and so throughout the book; cf. Ezra 2:58; 2:70; 7:7,24; 8:17,20.}

Descendants of Ziha, Hasupha, and Tabbaoth.

\v{44}Descendants of Keros, Siaha,\fnote{Cf. Neh 7:47 \fbib{Sia}} and Padon.

\v{45}Descendants of Lebanah, Hagabah, and Akkub.\fnote{Cf. Neh 7:48 \fbib{Shalmai}}

\v{46}Descendants of Hagab, Shalmai, and Hanan.

\v{47}Descendants of Giddel, Gahar, and Reaiah.

\v{48}Descendants of Rezin, Nekoda, and Gazzam.

\v{49}Descendants of Uzza, Paseah, and Besai.

\v{50}Descendants of Asnah,\fnote{Cf. Neh 7:52 \fbib{Besai}} Meunim, and Nephusim.

\v{51}Descendants of Bakbuk, Hakupha, and Harhur.

\v{52}Descendants of Bazluth, Mehida, and Harsha.

\v{53}Descendants of Barkos, Sisera, and Temah.

\v{54}Descendants of Neziah and Hatipha.

\v{55}The Descendants of Solomon's Servants:

Descendants of Sotai, Hassophereth,\fnote{Cf. Neh 7:57 \fbib{Sophereth}} and Peruda.\fnote{Cf. Neh 7:57 \fbib{Perida}}

\v{56}Descendants of Jaalah,\fnote{Cf. Neh 7:58 \fbib{Jaala}} Darkon, and Giddel.

\v{57}Descendants of Shephatiah, Hattil, Pochereth-hazzebaim, and Ami.\fnote{Cf. Neh 7:59 \fbib{Ammon}}

\v{58}All of the Temple Servants and descendants of Solomon's servants numbered 392.
\passage{Non-Documented Persons}
\passageinfo{(Nehemiah 7:61-69)}

\v{59}Here is a list of returnees from Tel-melah, Tel-harsha, Cherub, Addan, and Immer who could not prove their ancestry and lineage from Israel:

\v{60}Descendants of Delaiah, Tobiah, and Nekoda: 652\fnote{Cf. Neh 7:62 \fbib{642}}

\v{61}Descendants of the Priests:

Descendants of Habaiah, Hakkoz,\fnote{Cf. Neh 7:63 \fbib{Koz}} and Barzillai, who married one of the daughters of Barzillai from Gilead and took that name.

\v{62}These people searched for their ancestral registrations but they couldn't be located. Accordingly, they were assigned an ``unclean'' status and couldn't be priests. \v{63}Governor Zerubbabel\fnote{The Heb. lacks \fbib{Zerubbabel}} also ruled that they shouldn't eat anything holy until a priest arose with Urim and Thummim.\fnote{I.e. a high priest to whom God would reveal his will through the jewel-encrusted breastplate that he wore; cf. Exod 28:30, Neh 7:65}

\v{64}The entire assembly numbered 42,360, \v{65}not including 7,337 male and female servants, along with 200\fnote{Cf. Neh 7:66 \fbib{245}} singing men and women. \v{66}In addition, they had 736 horses, 245 mules, \v{67}435 camels, and 6,720 donkeys.
\passage{Gifts for the Temple}
\passageinfo{(Nehemiah 7:70-73)}

\v{68}When they arrived at the Temple of the \divine{Lord} in Jerusalem, some of the heads of the families contributed toward building the Temple of God on its former site. \v{69}They contributed to the treasury for this work in accordance with their ability: 61,000 golden drachma, 5,000 units\fnote{Lit. \fbib{mina}} of silver, and 100 priestly robes. \v{70}As a result, the priests, descendants of Levi, certain people, the singers, door-keepers, and the Temple Servants were able to settle in their original cities, with the rest of the Israelis in their cities.
\labelchapt{3}
\passage{Initial Offering Ceremonies}
\passageinfo{(Nehemiah 7:72)}

\chapt{3}
\v{1}Seven months after the Israelis had settled in their cities, they all gathered together in Jerusalem as a united body.\fnote{Lit. \fbib{together as one man in Jerusalem}} \v{2}Then Jozadak's son Jeshua and his brothers got up, along with Shealtiel's son Zerubbabel and his brothers. They built an altar of the God of Israel in order to offer burnt offerings, as prescribed by the Law of Moses, the man of God.

\v{3}Even though they feared the people in neighboring regions, they rebuilt the altar where it had stood before.\fnote{Lit. \fbib{altar on its bases}} They offered burnt offerings on it to the \divine{Lord}---burnt offerings both in the morning and in the evening. \v{4}They also observed the Festival of Tents\fnote{Or \fbib{Shelters}} as has been prescribed, offering a specific number of daily burnt offerings in accordance with the ordinance of each day. \v{5}After that, they offered\fnote{The Heb. lacks \fbib{they offered}} all of the continual burnt offerings and the New Moon sacrifices\fnote{Lit. \fbib{the moons}} for all of the designated festivals of the \divine{Lord} that were being consecrated, along with all the voluntary offerings that were dedicated to the \divine{Lord}. \v{6}They began to offer burnt offerings to the \divine{Lord} from the first day of the seventh month, even though the foundation of the Temple of the \divine{Lord} had not yet been laid.
\passage{Construction Begins on the Temple}

\v{7}They paid masons and carpenters in cash.\fnote{Lit. \fbib{silver}} They paid\fnote{The Heb. lacks \fbib{They paid}} the residents of Sidon and Tyre with food, drink, and oil, for them to bring cedar trees by sea from Lebanon to Joppa in accordance with the order they had obtained from Cyrus, king of Persia.

\v{8}Two years and two months after arriving at the site of the Temple of God in Jerusalem, Shealtiel's son Zerubbabel, Jozadak's son Jeshua, the relatives of the priests and descendants of Levi, and everyone else who had left the Babylonian\fnote{The Heb. lacks \fbib{Babylonian}} captivity for Jerusalem appointed descendants of Levi who were 20 years old and older to oversee the work of the \divine{Lord}'s Temple.

\v{9}At this time Jeshua, along with his children and relatives, and Kadmiel, with his children and the descendants of Judah, joined the family of Henadad with his children and relatives, and the descendants of Levi in overseeing the work on the Temple of God.
\passage{The Temple Foundation is Laid}

\v{10}After the builders laid the foundation for the \divine{Lord}'s Temple, the priests stood in their ministerial robes with trumpets and the descendants of Levi (who were also descendants of Asaph) with cymbals to praise the \divine{Lord}, according to instructions prepared by\fnote{Lit. \fbib{\divine{Lord} according to the hand of}} David, king of Israel. \v{11}And they sang in unison\fnote{Or \fbib{sang by antiphonal courses}} to one another, giving thanks to the \divine{Lord}:

\begin{poetry}
\poeml ``He is good, \\
\poemll    and his gracious love to Israel endures forever.''
\end{poetry}

And all the people shouted out loudly in praise to the \divine{Lord} when the foundation of the \divine{Lord}'s Temple was laid.
\passage{Remembering the Former Temple}

\v{12}Now a number of the priests, the Levities, and the leading officials of the elders---who were very\fnote{The Heb. lacks \fbib{very}} elderly---had seen the former Temple with their own eyes. When they observed the foundation of the Temple being laid, they wept with a loud voice, while the rest of them shouted for joy. \v{13}As a result, the people couldn't distinguish between the noise coming from the shouts of joy and the noise coming from the weeping people, because everyone\fnote{Lit. \fbib{the people}} was shouting loudly and could be heard a long way off.
\labelchapt{4}
\passage{A Plot to Hinder the Work}

\chapt{4}
\v{1}When the enemies of Judah and Benjamin learned that the descendants of the Babylonian\fnote{The Heb. lacks \fbib{Babylonian}} captivity had built their Temple to the \divine{Lord}, the God of Israel, \v{2}they approached Zerubbabel and the heads of the families\fnote{Lit. \fbib{fathers}} with this message: ``Let's build along with you, because, like you, we seek your God, as do you, and we've been making sacrifices to him since the reign of Esarhaddon, king of Assyria, who brought us here.''

\v{3}But Zerubbabel, Jeshua, and the rest of the heads of the families\fnote{Lit. \fbib{fathers}} of Israel replied, ``You have no part in our plans for\fnote{The Heb. lacks \fbib{plans for}} building a temple to our God, because we alone will build to the \divine{Lord}, the God of Israel, in accordance with the decree issued by King Cyrus, king of Persia.''
\passage{The Plot Succeeds---for a While}

\v{4}After this, the non-Israeli inhabitants\fnote{Lit. \fbib{the people}} of the land undermined\fnote{Lit. \fbib{weakened the hands of}} the people of Judah, harassing them in their construction work \v{5}by bribing their consultants in order to frustrate their plans throughout the reign of Cyrus, king of Persia until Darius became king.\fnote{Lit. \fbib{until the reign of Darius, king of Persia}}

\v{6}At the beginning of the reign of Ahasuerus, they lodged a formal accusation against the inhabitants of Judah and Jerusalem. \v{7}While Artaxerxes was king of Persia, Bishlam, Mithredath, Tabeel, and the rest of their co-conspirators wrote in the Aramaic language and script to King Artaxerxes of Persia.

Aramaic:\fnote{From this point through 6:18, the text of MT is in Aramaic.}

\v{8}Governor Rehum and Shimshai the scribe wrote a letter concerning Jerusalem to King Artaxerxes as follows:

\v{9}From Governor Rehum

Shimshai the scribe

The rest of their colleagues---

Judges, envoys, officials, Persians, the people of Erech, the Babylonians, the people of Susa (that is, the Elamites) \v{10}and many other nations whom the great and honorable Osnappar deported and resettled in Samaria and in the rest of the province beyond the Euphrates\fnote{The Aram. lacks \fbib{Euphrates}} River.

\v{11}This is the text of the letter they sent.

\begin{poetry}
\poeml To: King Artaxerxes \\
\poeml From: Your servants, the men of the province beyond the Euphrates\fnote{The Aram. lacks \fbib{Euphrates}} River. \\
\poeml \v{12}May the king be advised that the Jews who came from you to us have reached Jerusalem and are rebuilding a rebellious and wicked city, having completed its walls and repaired its foundations. \\
\poeml \v{13}May the king be further advised that if this city is rebuilt and its walls erected, its citizens\fnote{Lit. \fbib{erected, they}} will refuse to pay tributes, taxes, and tariffs, thereby restricting royal revenues. \\
\poeml \v{14}Now, because we are royal employees\fnote{Lit. \fbib{we received salt from the palace}} and are committed to preserving the reputation of the king, we have written to the king and have declared its contents to be true,\fnote{Lit. \fbib{and certified to the king}} \v{15}urging\fnote{The Aram. lacks \fbib{urging}} that a search may be made in the official registers of your predecessors.\fnote{Lit. \fbib{fathers}} You will discover in the registers that\fnote{Lit. \fbib{books and will know}} this city is a rebellious city, that it is damaging to both kings and provinces, that it has been moved to sedition from time immemorial, and that because of this it was destroyed. \\
\poeml \v{16}We certify to the king that if this city is rebuilt and its walls completed, you will lose your land holdings in the province beyond the Euphrates\fnote{The Aram. lacks \fbib{Euphrates}} River.
\end{poetry}
\passage{The Response of Ahasuerus}

\v{17}The king replied:

\begin{poetry}
\poeml To: Governor Rehum, Shimshai the scribe, and their colleagues living in Samaria, and the remainder living beyond the Euphrates\fnote{The Aram. lacks \fbib{Euphrates}} River. \\
\poeml Greetings:\fnote{Lit. \fbib{Peace, and now.}} \\
\poeml \v{18}The memorandum you sent to us has been read and carefully considered.\fnote{Lit. \fbib{been read plainly before me}} \v{19}Pursuant to my edict, an investigation has been undertaken. It is noted that this city has fomented rebellion against kings from time immemorial, and that rebellion and sedition has occurred in it. \\
\poeml \v{20}Powerful kings have reigned over Jerusalem, including ruling over all lands beyond the Euphrates\fnote{The Aram. lacks \fbib{Euphrates}} River. Furthermore, taxes, tribute, and tolls have been paid to them. \\
\poeml \v{21}Accordingly, issue an order to force these men to cease their work\fnote{The Aram. lacks \fbib{their work}} so that this city is not rebuilt until you receive further notice from me. \\
\poeml \v{22}Be diligent and take precautions so that you do not neglect your responsibility in this matter. Why should the kingdom sustain any more damage?
\end{poetry}
\passage{Reconstruction Ceases}

\v{23}As soon as a copy of the letter from King Artaxerxes was read to Rehum, to Shimshai the scribe, and to their colleagues, they traveled quickly to Jerusalem and compelled the Jews to cease by force of arms. \v{24}As a result, work on the Temple of God in Jerusalem ceased and did not begin again until the second year of the reign of King Darius of Persia.
\labelchapt{5}
\passage{Rebuilding Efforts Begin Again}
\passageinfo{(Haggai 1:1; Zechariah 1:1)}

\chapt{5}
\v{1}At that time, the prophets Haggai and Iddo's son Zechariah prophesied specifically to the Jews in Judah and Jerusalem in the name of the God of Israel. \v{2}So Shealtiel's son Zerubbabel and Jozadak's son Jeshua restarted construction of the Temple of God in Jerusalem. And the prophets of God were there supporting them.
\passage{Government Interference}

\v{3}Right about then, Trans-Euphrates\fnote{Lit. \fbib{Beyond the River}} Governor Tattenai, Shethar-bozenai, and their colleagues approached and challenged them. They asked, ``Who authorized you to build this Temple and to reconstruct this wall?'' \v{4}In answer, we responded with a list of\fnote{Lit. \fbib{responded thus: ``What are}} the names of the men who were building the structure. \v{5}But God watched over the Jewish leaders, who could not be forced to stop working until Darius received a report and responded in reply.
\passage{A Memorandum}

\v{6}Here is a copy of the letter that Trans-Euphrates\fnote{Lit. \fbib{Beyond the River}} Governor Tattenai, Shethar-bozenai, and his colleagues the Trans-Euphrates Persians sent to King Darius. \v{7}The letter sent to him was written like this:

\begin{poetry}
\poeml To: King Darius: \\
\poeml Greetings!\fnote{Lit. \fbib{All peace!}} \\
\poeml \v{8}This is to inform\fnote{Lit. \fbib{Let it be known to}} the king that we traveled to the Temple of the great God in the Judean province, which is being built with large stones and reinforced with wooden beams in its walls. The work proceeds diligently and is in capable hands.\fnote{Lit. \fbib{and prospers in their hands}} \\
\poeml \v{9}We asked the elders, ``Who authorized you to build this Temple and to reinforce these walls?'' \v{10}We also asked them their names so that we could certify the identities\fnote{Lit. \fbib{could write the names}} of their leaders to you. \\
\poeml \v{11}In answer they responded, ``We are servants of the God of heaven and earth, and are rebuilding the Temple that was built many years ago by a great king of Israel. \v{12}But because our predecessors provoked the God of Heaven to become angry, he handed them over to the control\fnote{Lit. \fbib{hand}} of King Nebuchadnezzar of Babylon, the Chaldean who destroyed this Temple and transported the people to Babylon. \\
\poeml \v{13}However, during King Cyrus' first year---that same King Cyrus of Babylon---issued a decree to reconstruct this Temple of God. \v{14}He delivered into the care of Sheshbazzar (whom he appointed governor) the gold and silver utensils that Nebuchadnezzar had taken from the Jerusalem Temple and brought into the Babylonian temple. \\
\poeml \v{15}``And Cyrus\fnote{Lit. \fbib{he}} told him, `Take these utensils, go to Jerusalem, and carry them to the Temple, after the Temple of God has been built\fnote{Lit. \fbib{temple, and let the temple of God be built}} in its appropriate place.' \\
\poeml \v{16}``Then this very same Sheshbazzar arrived and laid the foundations for the Temple of God in Jerusalem. Since that time until now the Temple has been under construction and is not yet completed.'' \\
\poeml \v{17}Accordingly, with your approval we suggest that\fnote{Lit. \fbib{Accordingly, if it seems good to the king, let}} a search be conducted within the king's treasury at Babylon to verify\fnote{The Aram. lacks \fbib{to verify}} whether or not King Cyrus ever issued such a decree to reconstruct this Temple of God in Jerusalem. Then please notify us concerning the king's pleasure in this matter.
\end{poetry}
\labelchapt{6}
\passage{King Darius Verifies the Decree}

\chapt{6}
\v{1}Then King Darius issued an order to search the Hall of Records where the Babylonian archives were stored. \v{2}The following was found written on a scroll in Ecbatana at the summer\fnote{The Aram. lacks \fbib{summer}} palace of the province of Media:

\begin{poetry}
\poeml \v{3}\divine{Date}: First year of Cyrus the King \\
\poeml \divine{From}: King Cyrus \\
\poeml \divine{Subject}: The Temple of God in Jerusalem \\
\poeml Let the Temple be rebuilt where they offered sacrifices. Let the foundations thereof be laid with a height of 60 cubits\fnote{I.e. about 90 feet; a cubit was about eighteen inches} and a width of 60 cubits,\fnote{I.e. about 90 feet; a cubit was about eighteen inches} \v{4}constructed\fnote{The Aram. lacks \fbib{constructed}} with three layers of foundation\fnote{Lit. \fbib{heavy}} stone interlaced with a row of new timber, the expenses for which are to be paid from the king's treasury. \\
\poeml \v{5}Furthermore, let the gold and silver utensils from the Temple of God (that Nebuchadnezzar took from the Temple in Jerusalem and carried off to Babylon) be brought back to the Temple at Jerusalem and restored to their respective places in the Temple of God.
\passage{King Darius Confirms the Decree}
\poeml \v{6}To: Tattenai, Trans-Euphrates Governor, Shethar-bozenai, and your colleagues living beyond the Euphrates\fnote{The Aram. lacks \fbib{Euphrates}} River. \\
\poeml Stay away from there! \\
\poeml \v{7}Leave the work on this Temple of God alone! \\
\poeml Let the Jewish governor and the Jewish leaders build this Temple of God on its site. \\
\poeml \v{8}Furthermore, I hereby decree what you are to do for the Jewish leaders who are building this Temple of God: you are to pay the expenses of these men out of the king's assets from taxes collected\fnote{The Aram. lacks \fbib{collected}} beyond the River so that they are not hindered. \\
\poeml \v{9}And be sure that you don't fail to provide their daily needs---including young bulls, rams, and lambs for the burnt offerings of the God of Heaven, along with wheat, salt, wine, and oil, as the priests in Jerusalem tell you--- \v{10}so they may approach the God of Heaven with fragrant sacrifices and pray for the life of this king and his sons. \\
\poeml \v{11}I hereby also decree that whoever shall alter the wording of this edict, let his residence be torn down for timber to build a gallows,\fnote{The Aram. lacks \fbib{a gallows}} hang\fnote{Or \fbib{impale}} him on it, and turn his home into an outhouse. \v{12}And may the God who causes his Name to rest there destroy any king or people who might attempt\fnote{Lit. \fbib{shall put their hand out}} to destroy this Temple of God in Jerusalem. \\
\poeml I, Darius, have issued this decree. Let it be carried out quickly.
\end{poetry}

\v{13}Because of what King Darius had mandated, Tattenai, the Trans-Euphrates Governor, Shethar-bozenai, and their colleagues carried out his orders quickly.
\passage{Progress and Completion}

\v{14}And so the Jewish leaders continued their building, and prospered because of the prophecies of Haggai the prophet and Iddo's son Zechariah. They completed the rebuilding in accordance with the commandment from the God of Israel and the edicts of Cyrus, Darius, and Artaxerxes, king of Persia. \v{15}The Temple was completed on the third day of the month Adar during the sixth year of the reign of King Darius.

\v{16}The Israelis---the priests, the descendants of Levi, and the other related descendants who had returned from captivity---celebrated with joy at the dedication of the Temple of God. \v{17}At the dedication offering of the Temple of God, they presented 100 bulls, 200 rams, and 400 lambs, along with a sin offering of twelve male goats for the entire nation of Israel according to the number of the tribes of Israel.

\v{18}Furthermore, they established the priests in their divisions and the descendants of Levi in their positions for the service of God conducted at Jerusalem, as is proscribed in the Book of Moses.
\passage{The First Post-Captivity Passover}
\passageinfo{(Deuteronomy 16:1-8)}

\v{19}\fnote{At this point, the text of MT reverts to Heb.}The former exiles\fnote{Lit. \fbib{The sons of the captivity}} observed the Passover on the fourteenth day of the first month \v{20}because the priests and descendants of Levi had purified themselves together---all of them were pure---and they killed the Passover lamb\fnote{The Heb. lacks \fbib{lamb}} for every former exile,\fnote{Lit. \fbib{for all of the sons of the captivity}} for their relatives the priests, and for themselves.

\v{21}So the Israelis who had returned from captivity ate the Passover with all who had consecrated themselves from the uncleanness of the nations of the land in order to seek the \divine{Lord} God of Israel. \v{22}Then they observed the Festival of Unleavened Bread for seven days with joy, because the \divine{Lord} had made them glad, turning the heart of the king of Assyria toward them and strengthening them for their work on the Temple of God, the God of Israel.
\labelchapt{7}
\passage{Ezra's Return to Jerusalem}
\passageinfo{(Ezra 2:1-70)}

\chapt{7}
\v{1}After all of this, during the reign of King Artaxerxes of Persia, Seraiah's son Ezra (who was the grandson of Azariah, son of Hilkiah, \v{2}son of Shallum, son of Zadok, son of Ahitub, \v{3}son of Amariah, son of Azariah, son of Meraioth, \v{4}son of Zerahiah, son of Uzzi, son of Bukki, \v{5}son of Abishua, son of Phinehas, son of Eleazar, son of Aaron the chief priest) \v{6}left\fnote{Lit. \fbib{Ezra himself left}} Babylon. He was a skillful scribe of the Law of Moses that the \divine{Lord} God of Israel had given. And the king granted him everything he had requested because the hand of the \divine{Lord} his God was upon him. \v{7}Some of the descendants of Israel also left for Jerusalem, including the priests, the descendants of Levi, the singers, the gatekeepers, and the Temple Servants, during the seventh year of king Artaxerxes.

\v{8}He arrived in Jerusalem during the fifth month of the seventh year of the king's reign.\fnote{Lit. \fbib{seventh of the king}} \v{9}On the first day\fnote{The Heb. lacks \fbib{day}} of the first month he left Babylon and he arrived in Jerusalem on the first day\fnote{The Heb. lacks \fbib{day}} of the fifth month, since the beneficent hand of his God was upon him. \v{10}For Ezra had set his heart to seek the Law of the \divine{Lord}, to obey it, and to teach God's\fnote{The Heb. lacks \fbib{God's}} statutes and judgments in Israel.
\passage{The Letter from King Artaxerxes}

\v{11}Here is a copy of the letter that King Artaxerxes gave to Ezra, the priest-scribe, a scholar\fnote{Or \fbib{scribe}} in matters concerning the commandments of the \divine{Lord} and concerning his statutes pertaining to Israel:

\begin{poetry}
\poeml \v{12}From:\fnote{At this point, the text of MT changes to Aramaic through verse 26.} Artaxerxes, King of Kings \\
\poeml To: Ezra, the Priest, a scholar\fnote{Or \fbib{scribe}} in matters concerning the laws of the God of Heaven \\
\poeml Greetings!\fnote{Lit. \fbib{Perfect and so forth}} \\
\poeml \v{13}I hereby decree that all of the people of Israel--- along with their priests and descendants of Levi in my kingdom---who are determined to return to Jerusalem with you may do so. \v{14}You have authority to act for the king and for his Council of Seven to conduct an inquiry concerning Judah and Jerusalem in accordance with the Law of your God, which is in your possession. \v{15}You are carrying silver and gold that the King and his advisors have freely given to the God of Israel, whose Temple is in Jerusalem, \v{16}together with all of the silver and gold that you can raise in the province of Babylon, plus the freewill offerings given by the people and the priests, contributed for the Temple of their God, which is in Jerusalem. \\
\poeml \v{17}Accordingly, you are to exercise due diligence to utilize this money to purchase bulls, rams, lambs, grain offerings, and drink offerings, and to offer them upon the altar of the Temple of your God, which is in Jerusalem. \\
\poeml \v{18}Furthermore, the balance remaining of the silver and gold may be used for whatever other purpose you and your people desire, as long as such use is consistent with the will of your God. \\
\poeml \v{19}Furthermore, you are to deliver to the God of Jerusalem the vessels for the service of the Temple of your God that have been given to you. \\
\poeml \v{20}Furthermore, provide from the royal treasury whatever else may be needed for the Temple of your God. \\
\poeml \v{21}I, Artaxerxes, in my capacity as king,\fnote{Lit. \fbib{And I, even I, Artaxerxes the King}} hereby decree to all royal treasuries beyond the Euphrates\fnote{The Aram. lacks \fbib{Euphrates}} River that whatever Ezra the priest-scribe of the Law of the God of Heaven, may require of you are to be performed with all due diligence, \v{22}up to 100 silver talents,\fnote{I.e. about 7,500 pounds; a talent weighed about 75 pounds} 100 measures of wheat, 100 measures of wine, 100 measures of oil, and salt without limitation. \v{23}Whatever is commanded by the God of Heaven is to be done with all due diligence for the Temple of the God of Heaven, or wrath will come against the king's realm and his sons. \\
\poeml \v{24}Furthermore, we decree that with respect to any of the priests, descendants of Levi, singers, gatekeepers, Temple Servants, or other servants of this Temple of God, it is not to be lawful to impose any tribute, tax, or toll on them. \\
\poeml \v{25}And you, Ezra, in accordance with the wisdom given to you by your God, are to appoint magistrates and judges to administer justice to all the people beyond the Euphrates\fnote{The Aram. lacks \fbib{Euphrates}} River. All of them are to know the laws of your God, and you are to instruct those who do not know them. \v{26}Whoever refuses to practice the law of your God and the law of the king is to see judgment executed quickly, whether to death, banishment, confiscation of goods, or imprisonment.
\passage{Ezra's Response to the Letter}
\poeml \v{27}Blessed be the \divine{Lord} God of our ancestors, \\
\poemll    who placed this decree\fnote{The Heb. lacks \fbib{decree}} into the king's heart \\
\poemlll       to beautify the Temple of the \divine{Lord} in Jerusalem \\
\poeml \v{28}and who showed gracious love to me before the king, \\
\poemlll       before his advisors, \\
\poemlll       and before all of the king's mighty officials.
\end{poetry}

And I was strengthened because the hand of the \divine{Lord} my God was upon me. So I gathered together the leaders of Israel to go with me.
\labelchapt{8}
\passage{Ezra's List of Family Leaders}

\chapt{8}
\v{1}These are the leaders of the families listed among those who left Babylon with me during the reign of King Artaxerxes: \v{2}From Phinehas's descendants: Gershom. From Ithamar's descendants: Daniel. From David's descendants: Hattush. \v{3}From Shecaniah's descendants and\fnote{The Heb. lacks \fbib{and}} from Parosh's descendants: Zechariah, along with 150 men whose genealogies had been certified. \v{4}From Pahath-moab's descendants: Zerahiah's son Eliehoenai and 200 men with him. \v{5}From Zattu's descendants: Jahaziel's son Shecaniah and 300 men with him. \v{6}From Adin's descendants: Jonathan's son Ebed and 50 men with him. \v{7}From Elam's descendants: Athaliah's son Jeshaiah and 70 men with him. \v{8}From Shephatiah's descendants: Michael's son Zebadiah and 80 men with him. \v{9}From Joab's descendants: Jehiel's son Obadiah and 218 men with him. \v{10}From Bani's descendants:\fnote{So LXX. The Heb. lacks \fbib{Bani}} Josiphiah's son Shelomith and 160 men with him. \v{11}From Bebai's descendants: Bebai's son Zechariah and 28 men with him. \v{12}From Azgad's descendants: Hakkatan's son Johanan and 110 men with him. \v{13}From Adonikam's later descendants: Eliphelet, Jeuel, Shemaiah, and 60 men with him. \v{14}From Bigvai's descendants: Uthai, Zabbud,\fnote{So MT, but \fbib{qere} directs that the name be read \fbib{Zaccur}, perhaps due to confusion with the nearly identical Heb. word for \fbib{men} (\fbib{Zecarim})} and 70 men with him.
\passage{Ezra Calls the Leaders to Fast}

\v{15}I gathered them together at the river that flows toward Ahava,\fnote{I.e. about 80 miles northwest of Babylon (cf. 2Kings 17:24)} where we camped three days. Afterwards, I examined the people and the priests, but found no descendants of Levi there. \v{16}So I sent for Eliezer, Ariel, Shemaiah, Elnathan, Jarib, Elnathan, Nathan, Zechariah, and Meshullam, who were senior leaders, as well as for Joiarib and Elnathan, who were men of discernment. \v{17}I told them to go see Iddo, a leader of Casiphia, and tell him and his relatives (administrators of Casiphia) to bring us men who could serve in the Temple of our God. \v{18}By the grace\fnote{Lit. \fbib{the good hand}} of our God they brought back a discerning man from the descendants of Mahli, a descendant of Israel's son Levi, along with Sherebiah and eighteen of his sons and brothers; \v{19}Hashabiah and Jeshaiah from the descendants of Merari and 20 of his brothers and their sons; \v{20}220 descendants of the Temple Servants whom David and the leaders had appointed to serve the descendants of Levi, listed by name.

\v{21}Then I called for a fast there at the Ahava River so we could humble ourselves before our God and seek from him an appropriate way for us and our little ones to live, and how we should guard our personal wealth,\fnote{Lit. \fbib{and for our wealth}} \v{22}because I was ashamed to ask the king for a contingent of soldiers and cavalry to protect us from enemies we might encounter\fnote{The Heb. lacks \fbib{we might encounter}} on the way. After all, we had told the king, ``The hand of our God seeks the good of all who seek him,\fnote{Lit. \fbib{God upon all who seek him to the good}} but his power and anger are against everyone who forsakes him.'' \v{23}So we fasted and asked our God about this, and he listened to us.
\passage{Ezra Delegates Responsibilities}

\v{24}Next I selected twelve of the chief priests---Sherebiah, Hashabiah, and ten of their brothers with them--- \v{25}and divided between them the silver, the gold, the vessels, and the offering for the Temple of our God which the king had offered, along with his advisors, his senior officials, and all of Israel assembled there. \v{26}I divided among them\fnote{Lit. \fbib{divided upon their hand}} 650 silver talents,\fnote{I.e. about 48,750 pounds; a talent weighed about 75 pounds} silver utensils weighing 100 talents,\fnote{I.e. about 7,500 pounds; a talent weighed about 75 pounds} 100 talents\fnote{I.e. about 7,500 pounds; a talent weighed about 75 pounds} of gold, \v{27}20 gold basins weighing 1,000 darics\fnote{I.e. about 15.6 pounds; a daric weighed about one quarter of an ounce} each, and two vessels made of polished brass, as valuable as gold.

\v{28}I told them, ``You are consecrated\fnote{Or \fbib{holy}} to the \divine{Lord}, and the vessels are also consecrated.\fnote{Or \fbib{holy}} The silver and the gold are a freely given offering to the \divine{Lord} God of your ancestors. \v{29}Guard and protect them until you disperse them to the chief priests, the descendants of Levi, and to the family leaders of Israel at Jerusalem in the chambers of the Temple of the \divine{Lord}.''

\v{30}So the priests and descendants of Levi took possession of the silver, the gold, and the vessels in order to bring them to Jerusalem, to the Temple of our God.

\v{31}Then we left the Ahava River for Jerusalem on the twelfth day\fnote{The Heb. lacks \fbib{day}} of the first month. Our God's protection\fnote{Lit. \fbib{hand}} was with us, and he delivered us from the enemy's power\fnote{Lit. \fbib{hand}} and from ambush along the way.
\passage{Ezra's Delegation Arrives in Jerusalem}

\v{32}We arrived in Jerusalem and remained there three days. \v{33}On the fourth day the silver, the gold, and the vessels were distributed at the Temple of our God into the care\fnote{Lit. \fbib{hand}} of Uriah's son Meremoth the priest, Phinehas' son Eleazar, Jeshua's son Jozabad, and Binnui's son Noadiah, the descendants of Levi. \v{34}Distribution was according to inventory\fnote{Lit. \fbib{By number}} and weight, with all weights being recorded at that time.

\v{35}The descendants of those who had been taken into captivity and who had returned from captivity offered burnt offerings to the God of Israel: twelve bulls for all of Israel, 96 rams, 77 lambs, and twelve male goats as a sin offering---all of them burnt offerings to the \divine{Lord}. \v{36}Then they delivered copies of\fnote{The Heb. lacks \fbib{copies of}} the king's orders to the king's officers, and governors on this side of the Euphrates\fnote{The Heb. lacks \fbib{Euphrates}} River. The orders were in support of the people and God's Temple.
\labelchapt{9}
\passage{Ezra's Reaction to Foreign Marriages}
\passageinfo{(Nehemiah 13:23)}

\chapt{9}
\v{1}After these things occurred, certain\fnote{The Heb. lacks \fbib{certain}} officials approached me and said ``The people of Israel, the priests, and the descendants of Levi have not separated themselves from the people of the lands or from the detestable behavior of the Canaanites, the Hittites, the Perizzites, the Jebusites, the Ammonites, the Moabites, the Egyptians, and the Amorites, \v{2}because they and their sons have married foreign women.\fnote{Lit. \fbib{married some of their daughters}} As a result, the holy people\fnote{Lit. \fbib{seed}} have mingled themselves among the people who live in these lands. As a matter of fact, the senior officials and the rulers have been foremost in this sin.''

\v{3}When I heard this, I tore both my garment and robe, plucked hair from both my head and my beard, and collapsed in shock! \v{4}Then everyone who trembled at the words of the God of Israel gathered together as a group because of the sin committed by those who had been led astray. As for me, I remained seated, in shock, until the evening sacrifice.
\passage{Ezra's Prayer of Repentance}

\v{5}At the time of the evening sacrifice, I arose from my discouragement. Still in my torn garment and robe, I fell to my knees with my hands outstretched to the \divine{Lord} my God, \v{6}and said,

\begin{poetry}
\poeml ``My God, I am too ashamed and hurt to turn to you, because we're in our iniquities over our heads. Furthermore, my God, our sins have grown as high as the heavens. \v{7}We have lived in great sin from the days of our ancestors even until today, and because of those iniquities we, our kings, and our priests have been delivered over to foreign kings, for execution, for captivity, for plunder, and for humiliation, as is the case\fnote{The Heb. lacks \fbib{is the case}} today. \v{8}Though now, for a moment, grace has been shown\fnote{The Heb. lacks \fbib{shown}} from the \divine{Lord} our God, leaving a few survivors to escape, and providing us a secure hold in his Holy Place, so that our God might enlighten us and give us relief from our servitude. \v{9}Even though we are slaves, our God has not abandoned us in our slavery. Instead, he has extended gracious love to us in the presence of the kings of Persia, to grant us revival, to set up the Temple of our God, to repair its damage, and to give us a protective wall for Judah and Jerusalem. \\
\poeml \v{10}Now, our God, what can we say besides this? Because we have abandoned your commandments \v{11}that you gave in the writings\fnote{Lit. \fbib{you ordered by the hand}} of your servants, the prophets: \\
\poeml `The land you are entering to possess is a morally unclean land due to the moral uncleanness of the people who live in there---along with their abominations---that has filled it from one end to the other with their impurities. \v{12}So, therefore, do not give your daughters in marriage to their sons, nor marry their daughters to your sons, and under no circumstances are you to seek their well-being or their wealth, so that you may remain strong, enjoying the best things the land has to give, and so that you may establish an inheritance for your children forever.'\fnote{This quotation from the prophets is not contained in MT} \\
\poeml \v{13}``After all that has happened to us because of our evil behavior, and because of our great sin---considering that you our God have punished us less than our iniquities deserve\fnote{The Heb. lacks \fbib{deserve}} and have given us this deliverance--- \v{14}should we violate your commandments by intermarrying with the nations\fnote{Lit. \fbib{peoples}} who practice these abominations? Would you not be angry with us until you had consumed us, with not even a remnant surviving\fnote{The Heb. lacks \fbib{surviving}} to escape? \\
\poeml \v{15}\divine{Lord} God of Israel, you are just: As a result, we remain here today delivered. Look at us! Because of our sin, we cannot stand in your presence as a result of everything that has happened.''
\end{poetry}
\labelchapt{10}
\passage{The People Gather with Ezra}

\chapt{10}
\v{1}Now while Ezra was praying and confessing in tears, having prostrated himself to the ground before the Temple of God, a very large crowd of Israelis---men, women, and children---gathered around him. Indeed, the people were crying bitterly.

\v{2}Jehiel's son Shecaniah, one of Elam's descendants, responded to Ezra: ``We have sinned against our God by marrying foreign wives from the people of the land, but even now there is hope in Israel, despite this. \v{3}So let's make a promise to our God by which we divorce\fnote{Or \fbib{expel}} all of these foreign\fnote{The Heb. lacks \fbib{of these foreign}} wives---as well as those born to them---in accordance with the counsel of our Lord and of those who tremble at our God's command. Furthermore, let it be done according to the Law. \v{4}So get up---it's your responsibility! We're with you. Be strong, and get to work.''\fnote{Or \fbib{and do it}}
\passage{The People Agree to Dissolve Their Marriages}

\v{5}So Ezra got up and made the chief priests, the descendants of Levi, and all of Israel vow to carry out everything they promised. And so they agreed.\fnote{Lit. \fbib{swore}} \v{6}Ezra arose in front of the Temple of God to visit the apartment of Eliashib's son Jehohanan. While there, he neither ate nor drank because he was in mourning over the sins of those who had returned from exile. \v{7}Then they sent word throughout Judah and Jerusalem to everyone who had returned from the exile, to gather together in Jerusalem. \v{8}Whoever would not come within three days would forfeit his assets and be separated from the community of the returning exiles, just as the high officials and elders had advised.

\v{9}Less than three days later, all of the men of Judah and Benjamin gathered together on the twentieth day\fnote{The Heb. lacks \fbib{day}} of the ninth month. Everyone sat in the plaza of the Temple of God, trembling because of everything that was happening, and also because it was raining heavily. \v{10}Ezra the priest stood up and spoke to them, ``You have sinned by marrying foreign wives, thereby increasing the transgressions of Israel. \v{11}Now confess this to the \divine{Lord} God of your ancestors, and separate yourselves from the people who live in the land and from foreign wives.''

\v{12}At this, the entire community responded with a loud cry, ``We will do just as you've spoken! \v{13}However, many people are involved, and it's raining heavily. Furthermore, this is not just a matter of a day or two of work, because many of us have sinned in this. \v{14}So let's have our officials remain on behalf of the whole community. Then all who have married foreign wives are to come appear at specific times before\fnote{Lit. \fbib{with}} the elders and judges of each city until the fierce anger of our God has been turned away from us in this matter.''

\v{15}Only Asahel's son Jonathan and Tikvah's son Jahzeiah opposed this, and they were supported by Meshullam and Shabbethai the descendant of Levi.
\passage{The People Carry Out Their Promise}

\v{16}So those who had returned from exile did this. Ezra the priest and leaders of certain ancestral groups listed by name devoted themselves to examine the situation on the first day of the tenth month. \v{17}By the first day of the first month they concluded their investigation of all of the men who had married foreign wives.
\passage{Those who Married Foreign Women}

\v{18}Here is a list of priestly descendants who were found to have married foreign women. From Jeshua's descendants:\fnote{Lit. \fbib{Children of}; and so through v. 43} Jozadak's son and his brothers Maaseiah, Eliezer, Jarib, and Gedaliah. \v{19}Pleading guilty, they promised to divorce their wives. Then they offered a ram from their flocks for their offense.

\v{20}From Immer's descendants: Hanani and Zebadiah. \v{21}From Harim's descendants: Maaseiah, Elijah, Shemaiah, Jehiel, and Uzziah. \v{22}From Pashhur's descendants: Elioenai, Maaseiah, Ishmael, Nethanel, Jozabad, and Elasah. \v{23}From the descendants of Levi: Jozabad, Shimei, Kelaiah (that is, Kelita), Pethahiah, Judah, and Eliezer. \v{24}From the singers: Eliashib. From the gatekeepers: Shallum, Telem, and Uri.

\v{25}From the Israelis: Parosh's descendants: Ramiah, Izziah, Malchijah, Mijamin, Eleazar, Malchijah, and Benaiah. \v{26}From Elam's descendants: Mattaniah, Zechariah, Jehiel, Abdi, Jeremoth, and Elijah. \v{27}From Zattu's descendants: Elioenai, Eliashib, Mattaniah, Jeremoth, Zabad, and Aziza. \v{28}From Bebai's descendants: Jehohanan, Hananiah, Zabbai, and Athlai. \v{29}From Bani's descendants: Meshullam, Malluch, Adaiah, Jashub, Sheal, and Jeremoth. \v{30}From Pahath-moab's descendants: Adna, Chelal, Benaiah, Maaseiah, Mattaniah, Bezalel, Binnui, and Manasseh. \v{31}From Harim's descendants: Eliezer, Isshijah, Malchijah, Shemaiah, Shimeon, \v{32}Benjamin, Malluch, and Shemariah.

\v{33}From Hashum's descendants: Mattenai, Mattattah, Zabad, Eliphelet, Jeremai, Manasseh, and Shimei. \v{34}From Bani's descendants: Maadai, Amram, Uel, \v{35}Benaiah, Bedeiah, Cheluhi, \v{36}Vaniah, Meremoth, Eliashib, \v{37}Matanza, Maternai, Jas, \v{38}Ban\'{i}, Binai, Shihezi, \v{39}Shelemiah, Nathan, Adaiah, \v{40}Machnadebai, Shashai, Sharai, \v{41}Azarel, Shelemiah, Shemariah, \v{42}Shallum, Amariah, and Joseph. \v{43}From Nebo's descendants: Jeiel, Mattithiah, Zabad, Zebina, Jaddai, Joel, and Benaiah.

\v{44}All of these had married foreign wives, and some of them had children by them.

\bookheader{Nehemiah}
\labelbook{Neh}

\bookpretitle{The Book of}
\booktitle{Nehemiah}

\labelchapt{1}
\passage{Introduction}

\chapt{1}
\v{1}In this document, I,\fnote{Lit. \fbib{The words of}} Hacaliah's son Nehemiah, recount\fnote{The Heb. lacks \fbib{recount}} what occurred during the twentieth year of Artaxerxes.\fnote{The Heb. lacks \fbib{of Artaxerxes}; cf. 2:1}
\passage{Background}

In the month of Chislev,\fnote{I.e. about 445-444 BC} while I was in Shushan at the palace, \v{2}Hanani, one of my brothers, arrived with some men from Judah. I asked them about the Jews who had escaped, about those who had survived the Babylonian\fnote{The Heb. lacks \fbib{Babylonian}} captivity, and about Jerusalem.

\v{3}They told me, ``The survivors of the captivity there in the province are living in great distress and shame. Furthermore, the Jerusalem wall remains broken down and its gates have been burned by fire.''
\passage{Nehemiah's Prayer}

\v{4}When I heard this, I sat down and cried, mourning for a number of days while I fasted and prayed in the presence of the God of Heaven. \v{5}I said, ``Please, \divine{Lord}, God of Heaven, the great and fearsome God who keeps the covenant, showing\fnote{The Heb. lacks \fbib{showing}} gracious love to those who love you and keep your commands, \v{6}please turn your attention to observe carefully and listen to the prayer of your servant today that I am presenting to you day and night on behalf of your servants, the Israelis.

``I confess the sins that we Israelis have committed against you. Both I and my father's house have sinned. \v{7}We have abandoned you by not keeping your commands, your ceremonies, and your judgments that you proscribed to your servant Moses. \v{8}Please remember what you spoke in commanding your servant Moses. You said,

\begin{poetry}
\poeml `If you rebel, I will scatter you among the nations\fnote{Lit. \fbib{people}} \v{9}but if you return to me, keeping my commands and doing them, even if your exiled people are in the farthest horizon, I will gather them from there and bring them to the place where I have chosen to establish my Name.'\fnote{Cf. Deut 30:1-5}
\end{poetry}

\v{10}These are your servants as well as your people, whom you have redeemed by your great power and by your strong hand.

\v{11}``And now, Lord, I ask you to listen to the prayer of your servant---and to the prayers of your servants who delight in revering your Name. I ask you, please prosper your servant today by granting him to receive favor from this man.''\fnote{I.e. King Artaxerxes}

Now I was the king's senior security advisor.\fnote{Lit. \fbib{king's cupbearer}; a servant who tested the king's food and beverages for poison; cf. Gen 41:9}
\labelchapt{2}
\passage{Nehemiah's Conversation with the King}

\chapt{2}
\v{1}It came about in the twentieth year of Artaxerxes, during the month of Nissan, the king was about to drink some wine that I was preparing for him.\fnote{Lit. \fbib{I took up the wine and gave it to the king}} Now I had never looked troubled in his presence.

\v{2}The king asked me, ``Why do you look so troubled, since you're not ill? This cannot be anything else but troubles of the heart.''

Then I was filled with fear. \v{3}I replied to the king, ``May the king live forever. Why shouldn't I be troubled, since the city where my ancestral sepulchers are located lies waste, with its gates burned by fire?''

\v{4}The king answered, ``What do you want?''

So I prayed to the God of heaven \v{5}and I replied to the king, ``If it seems good to you, and if your servant has found favor with you, would you send me to Judah, to the city where my ancestral sepulchers are located, so I can rebuild it?''

\v{6}With his queen seated beside him, the king asked me, ``How long will your journey take, and when will you return?'' The king thought it was a good idea\fnote{Lit. \fbib{It was good to the king}} to send me, so I presented him with a prepared plan.\fnote{Lit. \fbib{a season}}

\v{7}I also asked the king, ``If it seems good to you, order that letters of authorization be given me for the Trans-Euphrates\fnote{Lit. \fbib{Beyond the River}} governors, so they will allow me to pass through to Judah, \v{8}along with a letter to Asaph, the royal Commissioner of Forests, so that he will supply me with timber to craft beams for the gatehouses of the Temple, for the city walls, and for the house in which I will be living.''

The king granted this for me, according to the good hand of my God. \v{9}So I went to the Trans-Euphrates\fnote{Lit. \fbib{Beyond the River}} governors and gave them the king's letters of authorization. The king also sent army officers and cavalry to accompany me.
\passage{Opposition and Inspection}

\v{10}But when Sanballat the Horonite and his servant Tobiah the Ammonite heard of this, they were greatly distressed because someone had come to do good for the Israelis. \v{11}I arrived in Jerusalem and remained there for three days. \v{12}Then I got up at night, along with a few men with me. I had not confided to any person what my God had put in my heart to do for Jerusalem. Furthermore, there was no other animal with me except for the one I was riding.

\v{13}So I went out during the night through the Valley Gate toward Dragon's\fnote{Or \fbib{Jackal}} Well, and from there to the Dung Gate, inspecting the walls of Jerusalem that were broken down and burned by fire. \v{14}I proceeded to the Fountain Gate, and then to the King's Pool, but there wasn't sufficient clearance for the animal I was riding\fnote{Lit. \fbib{animal under me}} to pass. \v{15}I traveled the valley by night to inspect the wall, returning through the Valley Gate. \v{16}The local officials did not know where I had gone or what I had done---I informed neither the Judeans, nor the priests, nor the nobles, nor the officials, nor any of the rest who would be doing the work.

\v{17}Later I told them, ``You all are watching the predicament we're in, how Jerusalem lies desolate, with its gates burned by fire. Let's rebuild the Jerusalem wall so we're no longer a disgrace.'' \v{18}Then I told them how good my God had been to\fnote{Lit. \fbib{them the good hand of my God upon}} me, and about what the king had told me.

They replied, ``Let's get out there and build!'' So they encouraged themselves to do good.
\passage{Nehemiah Replies to Sanballat}

\v{19}But when Sanballat the Horonite, his servant Tobiah the Ammonite, and Geshem the Arab heard about it,\fnote{The Heb. lacks \fbib{about it}} they jeered at us and despised us when they said, ``What is this thing that you're doing? You're rebelling against the king, aren't you?''

\v{20}In reply to them, I said, ``The God of Heaven will prosper us. That's why we're preparing to build. But as far as you're concerned, there exists no ancestral heritage, no legal right, nor any historic claim in Jerusalem.
\labelchapt{3}
\passage{Those who Worked on the Wall}

\chapt{3}
\v{1}So Eliashib the high priest came forward, along with his fellow priests, and reconstructed the Sheep Gate. They consecrated it and installed its doors. They also consecrated the wall as far as the Tower of the Hundred and the Tower of Hananel. \v{2}Men from Jericho did repairs next to him, and Imri's son Zaccur did repairs next to them.

\v{3}The Fish Gate was repaired by Hassenaah's sons. They built its framework and installed its doors, including locks and security\fnote{The Heb. lacks \fbib{security}} bars, \v{4}with Uriah's son Meremoth (who was also a grandson of Hakkoz) next to them, Berechiah's son Meshullam (who was also a grandson of Meshezabel) next to them, and next to him Baana's son Zadok. \v{5}Next to them the Tekoites worked valiantly, even though their leading officials weren't fully dedicated\fnote{Lit. \fbib{officials did not bind their necks}} to the work of their lord.\fnote{Or \fbib{governor}}

\v{6}Paseah's son Joiada and Besodeiah's son Meshullam repaired the Old Gate. They built its framework and installed its doors, including locks and security\fnote{The Heb. lacks \fbib{security}} bars. \v{7}Next to them, Melatiah the Gibeonite and Jadon the Meronothite were working with men from Gibeon and men from Mizpah under the Trans-Euphrates\fnote{Lit. \fbib{from across the river}} regional governor. \v{8}Nearby, Harhaiah's son Uzziel the goldsmith was carrying on repairs, and next to him Hananiah, a perfume-maker, rebuilt Jerusalem as far as the Broad Wall.

\v{9}Next to him, Hur's son Rephaiah, ruling official for half of the Jerusalem district, did repairs. \v{10}Then next to them, Harumaph's son Jedaiah did repairs opposite his house, and next to him Hashabneiah's son Hattush carried on repairs. \v{11}Harim's son Malchijah and Pahath-moab's son Hasshub repaired another section, along with the Tower of the Ovens, \v{12}and next to him Hallohesh's son Shallum, ruling official for the other\fnote{The Heb. lacks \fbib{the other}} half of the Jerusalem district, did repairs, as did his daughters.

\v{13}Hanun and the residents of Zanoah repaired the Valley Gate, reconstructing it and installing its doors, including locks and security\fnote{The Heb. lacks \fbib{security}} bars. They also rebuilt 1,000 cubits\fnote{I.e. about 500 yards; a cubit was about eighteen inches} of the wall\fnote{The Heb. lacks \fbib{of the wall}} as far as the Dung Gate. \v{14}Rechab's descendant\fnote{Lit. \fbib{son}; cf. Jer 35:19} Malchijah, ruling official of the Beth-haccherem district, repaired the Dung Gate, reconstructing it, installing its doors, its locks, and its security\fnote{The Heb. lacks \fbib{security}} bars.

\v{15}Colhozeh's son Shallum, ruling official of the Mizpah district, repaired the Fountain Gate, reconstructing it, installing its doors, its locks, and its security\fnote{The Heb. lacks \fbib{security}} bars, as well as the Pool of Shelach near the royal garden as far as the stairway that descends from the City of David.

\v{16}Next to him Azbuk's son Nehemiah, ruling official of half of the Beth-zur district, carried on repairs as far as the tombs of David, then to the artificial pool that had been installed there, and then as far as the military barracks.\fnote{Lit. \fbib{the house of the mighty}} \v{17}Next to him the descendants of Levi, led by\fnote{The Heb. lacks \fbib{led by}} Bani's son Rehum, carried on repairs. Next to him Hashabiah, ruling official for half of the Keilah district, did repairs for his district. \v{18}Next to him their brothers, led by\fnote{The Heb. lacks \fbib{led by}} Henadad's son Bavvai, ruling official for the other\fnote{The Heb. lacks \fbib{the other}} half of the Keilah district, carried on repairs. \v{19}Next to him Jeshua's son Ezer, ruling official of Mizpah, repaired another section near the ascent to the armory at the Angle.\fnote{Cf. 2Chr 26:9} \v{20}Next to him Zabbai's son Baruch worked valiantly on another section from the angle of the wall\fnote{The Heb. lacks \fbib{of the wall}} as far as the door to the house belonging to Eliashib the high priest.

\v{21}Then next to him Uriah's son Meremoth, grandson of Hakkoz, repaired another section from the door of Eliashib's house as far as the rear of the property,\fnote{Lit. \fbib{the house of Eliashib}} \v{22}Next to him the priests, men from the plain, carried on repairs. \v{23}Next to them Benjamin and Hasshub carried on repairs near their house, followed by Maaseiah's son Azariah, grandson of Ananiah, who worked beside his own house. \v{24}Following him, Henadad's son Binnui repaired another section from Azariah's house to the angle of the wall,\fnote{The Heb. lacks \fbib{of the wall}} and then to the corner. \v{25}Uzai's son Palal carried on repairs over against the angle of the wall\fnote{The Heb. lacks \fbib{of the wall}} at the tower that stands out from the king's upper palace, which is located by the royal guard's court. Next to him, Parosh's son Pedaiah carried on repairs. \v{26}(Now the Temple Servants\fnote{Heb. \fbib{Nethinim}; i.e. a division of special assistants to the descendants of Levi, originally appointed by King David; and so throughout the book; cf. Ezra 2:58; 2:70; 7:7,24; 8:17,20.} were living on the Ophel as far as the Water Gate that faces eastward with its prominent tower.) \v{27}Next to Pedaiah,\fnote{Lit. \fbib{him}} the Tekoites repaired another section from the prominent tower as far as the wall of the Ophel.

\v{28}The priests carried on repairs from above the Horse Gate as far as their own houses. \v{29}Then next to them, Immer's son Zadok did repairs as far as his own house. Next to him, Shecaniah's son Shemaiah, custodian of the East Gate, carried on repairs. \v{30}Next to him, Shelemiah's son Hananiah and Zalaph's sixth son Hanun repaired another section. Next to him, Berechiah's son Meshullam carried on repairs up to his chamber. \v{31}Next to him, Malchijah, one of the goldsmiths, carried on repairs up to the house of the Temple Servants and the merchants, up to the Muster Gate as far as the ascent to the corner. \v{32}Between the ascent of the corner and the Sheep Gate, the goldsmiths and merchants carried on repairs.
\labelchapt{4}
\passage{Sanballat Opposes the Reconstruction}

\chapt{4}
\v{1}\fnote{This v. is 3:33 in MT, and so through v. 6.}When Sanballat heard that we were reconstructing the wall, he flew into a rage, became indignant, and mocked the Jews. \v{2}He addressed his allies and the Samaritan officials,\fnote{Or \fbib{army}} saying ``What are these pathetic Jews doing? Are they intending to rebuild it by themselves? Do they intend to offer sacrifices? Will they finish in a single day? Can they make stones from this burned out rubble?''

\v{3}Tobiah the Ammonite stood to the side, commenting, ``If a fox were to jump onto what they're building, it would collapse their stone wall!''
\passage{Nehemiah's Prayer}

\v{4}``Listen, our God, because we are being mocked. Let their insults fall back on them,\fnote{Lit. \fbib{on their heads}} and let them be dragged away as captives into exile. \v{5}Don't atone their iniquity, and don't let their sin be blotted out from before you, because they have demoralized the builders.''

\v{6}So we rebuilt the wall, completing it halfway up, because the people were committed to working.
\passage{Sanballat Reacts to the Progress}

\v{7}\fnote{This v. is 4:1 in MT, and so through v. 23.}But when Sanballat, Tobiah, the Arabs, the Ammonites, and the Ashdodites heard that the repair work on the Jerusalem wall was progressing and that its breaches were being repaired, they flew into a rage. \v{8}So they all conspired together to invade and fight against Jerusalem, creating confusion there.
\passage{Nehemiah Reacts to Sanballat}

\v{9}But we prayed to our God. We also set up guards day and night because of them.

\v{10}Meanwhile, the people of\fnote{The Heb. lacks \fbib{the people of}} Judah said, ``The builders are tired and there's so much rubble that we can't reconstruct the wall!''

\v{11}Our enemies said, ``Before they notice or see us, we'll penetrate their midst, kill them, and stop the work!''

\v{12}The Jews who lived near them kept coming to us, reporting at least\fnote{The Heb. lacks \fbib{at least}} ten times, ``They'll attack us from every direction.'' \v{13}So I stationed the people by families behind the wall in the lower exposed areas, equipping them with their swords, spears, and bows.

\v{14}Looking things over, I stood up and spoke to the officials, the military leaders, and the rest of the people: ``Don't fear them. Remember the great and awe-inspiring Lord. Fight for your brothers, your sons, your daughters, your wives, and your homes.''

\v{15}Our opponents heard that we had learned about them, that God had brought their plans to failure, and that each and every one of us had come to work on the wall. \v{16}From that day on, half of my helpers engaged in the work while the other half kept spears, shields, bows, and armor ready. The senior officials backed all of the Judeans \v{17}who worked on the wall. Those who carried building materials worked with one hand, carrying a spear in the other. \v{18}Each builder worked with a sword strapped to his side, while a trumpeter remained beside me to sound an alarm.\fnote{The Heb. lacks \fbib{to sound an alarm}}

\v{19}I told the officials, rulers, and the rest of the people, ``The project is large and extensive, and we are separated from each other on the wall, \v{20}so wherever you hear the sound of the trumpet, come over to us, and our God will fight for us!'' \v{21}So we worked hard, half of us holding spears from dawn to dusk.

\v{22}At the same time I told the people, ``Let's have everyone sleep at night inside Jerusalem with their servants, so they can guard us at night and work during the day. \v{23}No one---neither I, my allies, my servants, nor the bodyguards who accompanied me---changed clothes. Everyone carried a weapon even while going for water.
\labelchapt{5}
\passage{Settling Some Civil Disputes}

\chapt{5}
\v{1}Now the people along with their spouses complained loudly against their fellow\fnote{I.e. wealthy} Jews, \v{2}because certain of them kept claiming, ``Since we have so many sons and daughters, we must get some grain so we can eat and survive.''

\v{3}Others were saying, ``We're having to mortgage our fields, our vineyards, and our homes so we can buy grain during this famine.''

\v{4}Still others were saying ``We've borrowed money against our fields and vineyards to pay the king's taxes. \v{5}Now our bodies are no different than the bodies of our relatives, and our children are like their children. Nevertheless, we're about to force our sons and daughters into slavery, and some of our daughters are already in bondage. It's beyond our power to do anything about it, because our fields and vineyards belong to others.''

\v{6}I became very livid when I heard their complaining and these charges. \v{7}So after thinking it over carefully, I accused the officials and nobles openly, ``Every one of you is charging your fellow countrymen interest!'' So I opened a public investigation against them.

\v{8}I accused them, ``To the best of our ability, we've been buying back our fellow Jews who had been sold to foreigners. Even now you're selling your fellow countrymen, only for them to be sold back to us!'' They kept quiet and never spoke a word.

\v{9}So I said, ``What you're doing isn't right! Shouldn't you live in the fear of our God to avoid shame from our foreign enemies? \v{10}I'm also lending money and grain, as are my fellow-Jews and my servants, but let's not charge interest. \v{11}So today please restore to them their fields, vineyards, olive orchards, and homes, along with the one percent interest charge\fnote{Lit. \fbib{the one hundredth part}} that you've assessed them on the grain, wine, and oil.''

\v{12}They responded, ``We will restore these things,\fnote{The Heb. lacks \fbib{these things}} and will assess no interest charges\fnote{Lit. \fbib{will require nothing}} against them. We will do what you are requesting!''

So I called the priests and made them take an oath to fulfill this promise. \v{13}I also shook my robes,\fnote{Lit. \fbib{lap}} and said, ``May God shake out every man from his house and his possessions who does not keep this promise. May he be emptied out and shaken just like this.''

All the assembly said, ``Amen!'' and praised the \divine{Lord}. And the people kept their promise.
\passage{Nehemiah Refuses the Governor's Allotment}

\v{14}In addition, from the time that I was appointed to be their governor in the land of Judah (that is, during the twelve years from the twentieth to the thirty-second year of King Artaxerxes), neither I nor my relatives relied on the provisions\fnote{Lit. \fbib{have eaten the bread}} allotted to the governor. \v{15}Nevertheless, the former governors before me placed a heavy burden on the people. They received food and wine, plus a tax of\fnote{The Heb. lacks \fbib{a tax of}} 40 shekels\fnote{I.e. about a pound; a shekel weighed about 0.4 ounces} of silver. Even their young men took advantage of the people, but I never did so because I feared God.

\v{16}Also, as I continued to work on the wall, we purchased no land, and all of my young men were employed in the work. \v{17}I fed 150 Jews and officials every day, not counting those who came from the nations around us. \v{18}Our daily requirements were one ox and six choice sheep, along with various kinds of poultry prepared for me. Every ten days there was a delivery of an abundant supply of wine. Despite all this, I refused the governor's allotment,\fnote{Lit. \fbib{bread}} because demands on the people were heavy.

\v{19}``Remember me with favor, my God, for everything I've done for this people.''
\labelchapt{6}
\passage{A Diversion is Attempted}

\chapt{6}
\v{1}When Sanballat, Tobiah, Geshem the Arab, and the rest of our enemies heard that I had completed the wall and that no break remained in it (even though by that time I hadn't yet installed the doors in the gates), \v{2}Sanballat and Geshem sent word\fnote{The Heb. lacks \fbib{word}} to me, saying ``Come, let's meet together at Kephirim on the Ono Plain.'' But they were just trying to do me harm.

\v{3}So I sent messengers to them, replying ``I am involved in a great endeavor, so I can't leave. Why should the work stop while I leave it to come down to you?'' \v{4}They sent me this message four times, and I answered them the same way.

\v{5}Then Sanballat sent his assistant to me the fifth time. But this time the letter was sent\fnote{The Heb. lacks \fbib{sent}} unsealed, and \v{6}in it was written:

\begin{poetry}
\poeml It is reported among the nations---and Gashmu confirms this---that you and the Jews are planning a revolt, and that you're rebuilding the wall in order to declare yourself king. According to these reports, \v{7}you also have appointed prophets to proclaim about you in Jerusalem, ``There is a king in Judah!'' Since these words are being reported to the king, come and let's meet together.
\end{poetry}

\v{8}I sent word back\fnote{The Heb. lacks \fbib{word back}} to him, ``Nothing has happened as you've claimed. You're making up these charges\fnote{The Heb. lacks \fbib{charges}} in your imagination.''\fnote{Lit. \fbib{heart}} \v{9}For they all were trying to make us afraid by saying, ``Their hands will become tired from laboring, so the work won't be completed.''

``Therefore, \divine{Lord},\fnote{The Heb. lacks \fbib{\divine{Lord}}} strengthen my hands!''
\passage{A Conspiracy Charge Emerges}

\v{10}Later I visited Delaiah's son Shemaiah, a grandson of Mehetabel, who was confined at home. He kept urging me, ``Let's meet together at the house of God, within the Temple, and take refuge there,\fnote{Lit. \fbib{and shut the doors of the temple}} because they're coming to kill you. In fact, they're coming at night to kill you!''

\v{11}But I asked him, ``Should a man like me run? Should someone like me run into the Temple to save his life? I won't go there!'' \v{12}I perceived that God had not sent him. Instead, Tobiah and Sanballat had hired him to pronounce this prophecy against me. \v{13}He had been hired to make me afraid so I would sin by doing what he suggested.\fnote{Lit. \fbib{doing this}} Then they could create a slanderous report to use against me.

\v{14}``Remember me, my God, and take note of what Tobiah and Sanballat are doing. Also take note of the prophetess Noadiah and the rest of the prophets who intend to make me afraid.''

\v{15}So the wall was completed on the twenty-fifth day of Elul in 52 days.
\passage{Tobiah's Continued Harassment}

\v{16}When all of our enemies---including the surrounding nations---heard this, they became very discouraged, since they saw that the work had been done by our God. \v{17}Meanwhile, at that time the nobles of Judah continued to send many letters to Tobiah, and Tobiah kept sending letters\fnote{The Heb. lacks \fbib{letters}} to them. \v{18}For many Judeans had sworn allegiance to him, since he was son-in-law to Arah's son Shecaniah, and his son Jehohanan had married the daughter of Berechiah's son Meshullam. \v{19}Furthermore, they kept reporting Tobiah's\fnote{Lit. \fbib{his}} good deeds to me, and kept repeating what I told him. Tobiah kept sending letters to intimidate me.
\labelchapt{7}
\passage{Nehemiah Appoints Administrators}

\chapt{7}
\v{1}After the wall had been completed and its doors installed, then the gatekeepers, singers, and descendants of Levi were appointed. \v{2}I appointed my brother Hanani and fortress commander Hananiah to be over Jerusalem, since he was a faithful person who revered God more than many others did. \v{3}I charged them, ``Do not open the gates of Jerusalem until mid-day.\fnote{Lit. \fbib{until the sun is hot}} Until then, let everyone stand watch, keeping the gates shut and locked. Appoint security watches from those who live in Jerusalem. Everyone should maintain his own watch near his house.'' \v{4}Even though the city was large and spread out, not many people were living there and not many houses had been built. \v{5}So my God gave me the idea to gather together the nobles, the officials, and the people so they could be registered according to their genealogies.
\passage{A List of Those who Returned}
\passageinfo{(Ezra 2:1-58)}

I found a register of the original inhabitants in which there was recorded \v{6}a list of descendants\fnote{Lit. \fbib{These are the descendants}} of the province of Judah\fnote{The Heb. lacks \fbib{of Judah}} who returned from captivity, from those who had been exiled by Nebuchadnezzar king of Babylon. They had come back to Jerusalem and to Judah, each one to his town. \v{7}They were coming with Zerubbabel, Jeshua, Nehemiah, Azariah, Raamiah, Nahamani,\fnote{MT of Ezra 2:2 lacks \fbib{Azariah, Raamiah, Nahamani}} Mordecai, Bilshan, Mispereth,\fnote{Cf. MT of Ezra 2:2 \fbib{Mispar}} Bigvai, Nehum,\fnote{Cf. MT of Ezra 2:2 \fbib{Rehum}} and Baanah. Here is the enumeration of:

The Men of Israel:

\v{8}Parosh's descendants:\fnote{Lit. \fbib{Sons of}; and so throughout the chapter} 2,172

\v{9}Shephatiah's descendants: 372

\v{10}Arah's descendants: 652\fnote{Cf. Ezra 2:3 \fbib{775}}

\v{11}Pahath-moab's descendants; that is, through Jeshua and Joab: 2,818\fnote{Cf. Ezra 2:6 \fbib{2,812}}

\v{12}Elam's descendants: 1,254

\v{13}Zattu's descendants: 845\fnote{Cf. Ezra 2:8 \fbib{945}}

\v{14}Zaccai's descendants: 760

\v{15}Binnui's descendants:\fnote{Cf. Ezra 2:10 \fbib{Bani}} 648\fnote{Cf. Ezra 2:10 \fbib{642}}

\v{16}Bebai's descendants: 628\fnote{Cf. Ezra 2:11 \fbib{623}}

\v{17}Azgad's descendants: 2,322\fnote{Cf. Ezra 2:12 \fbib{1,222}}

\v{18}Adonikam's descendants: 667\fnote{Cf. Ezra 2:13 \fbib{666}}

\v{19}Bigvai's descendants: 2,067\fnote{Cf. Ezra 2:14 \fbib{2,056}}

\v{20}Adin's descendants: 655\fnote{Cf. Ezra 2:15 \fbib{454}}

\v{21}Ater's descendants through Hezekiah: 98

\v{22}Hashum's descendants: 328\fnote{Cf. Ezra 2:19 \fbib{223}}

\v{23}Bezai's descendants: 324\fnote{Cf. Ezra 2:17 \fbib{323}}

\v{24}Hariph's descendants:\fnote{Cf. Ezra 2:18 \fbib{Jorah}} 112

\v{25}Gibeon's descendants:\fnote{Cf. Ezra 2:19 \fbib{Gibbar}} 95

\v{26}People from Bethlehem and Netophah: 188\fnote{Cf. Ezra 2:21-22 where the total is \fbib{179}}

\v{27}People from Anathoth: 128

\v{28}People from Beth-azmaveth:\fnote{Cf. Ezra 2:24 \fbib{Azmaveth}} 42

\v{29}People from Kiriath-jearim,\fnote{Cf. Ezra 2:25 \fbib{Kiriath-arim}} Chephirah, and Beeroth: 743

\v{30}People from Ramah and Geba: 621

\v{31}People from Michmas: 122

\v{32}People from Bethel and Ai: 123\fnote{Cf. Ezra 2:28 \fbib{223}}

\v{33}People from the other Nebo: 52

\v{34}The other Elam's descendants: 1,254

\v{35}Harim's descendants: 320

\v{36}Jericho's descendants: 345

\v{37}Descendants of Lod, Hadid, and Ono: 721\fnote{Cf. Ezra 2:33 \fbib{725}}

\v{38}Senaah's descendants: 3,930\fnote{Cf. Ezra 2:35 \fbib{3,630}}

\v{39}The Priests:

Jedaiah's descendants from the household of Jeshua: 973

\v{40}Immer's descendants: 1,052

\v{41}Pashhur's descendants: 1,247

\v{42}Harim's descendants: 1,017

\v{43}The Descendants of Levi:

Jeshua of Kadmiel's descendants: that is, Hodevah's descendants:\fnote{Cf. Ezra 2:40 \fbib{Hodaviah}} 74

\v{44}The Singers:

Asaph's descendants: 148\fnote{Cf. Ezra 2:41 \fbib{128}}

\v{45}The Gatekeepers:

Shallum's descendants, Ater's descendants, Talmon's descendants, Akkub's descendants, Hatita's descendants, Shobai's descendants: 138\fnote{Cf. Ezra 2:42 \fbib{139}}

\v{46}The Temple Servants:

Descendants of Ziha, Hasupha, and Tabbaoth.

\v{47}Descendants of Keros, Sia,\fnote{Cf. Ezra 2:44 \fbib{Siaha}} and Padon.

\v{48}Descendants of Lebanah, Hagabah, and Shalmai.\fnote{Cf. Ezra 2:45-46 \fbib{and Akkub} \fbib{\v{46}Descendants of Hagab, Shalmai}}

\v{49}Descendants of Hanan, Giddel, and Gahar.

\v{50}Descendants of Reaiah, Rezin, and Nekoda.

\v{51}Descendants of Gazzam, Uzza, and Paseah.

\v{52}Descendants of Besai,\fnote{Cf. Ezra 2:49-50 \fbib{Besai}. \fbib{\v{50}Descendants of Asnah,}} Meunim, and Nephushesim,\fnote{Cf. Ezra 2:50 \fbib{Nephusim}}

\v{53}Descendants of Bakbuk, Hakupha, and Harhur.

\v{54}Descendants of Bazlith,\fnote{Cf. Ezra 2:52 \fbib{Bazluth}} Mehida, and Harsha.

\v{55}Descendants of Barkos, Sisera, and Temah.

\v{56}Descendants of Neziah and Hatipha.

\v{57}The Descendants of Solomon's Servants:

Descendants of Sotai, Sophereth,\fnote{Cf. Ezra 2:55 \fbib{Hassophereth}} and Perida,\fnote{Cf. Ezra 2:55 \fbib{Peruda}}

\v{58}Descendants of Jaala, Darkon, and Giddel,

\v{59}Descendants of Shephatiah, Hattil, Pochereth-hazzebaim, and Ammon;\fnote{Cf. Ezra 2:47 \fbib{Ami}}

\v{60}All of the Temple Servants and descendants of Solomon's servants numbered 392.
\passage{Non-Documented Persons}
\passageinfo{(Ezra 2:59-67)}

\v{61}Here is a list of returnees from Tel-melah, Tel-harsha, Cherub, Addan, and Immer, who could not prove their ancestry and lineage from Israel:

\v{62}Descendants of Delaiah, Tobiah, and Nekoda: 642\fnote{Cf. Ezra 2:60 \fbib{652}}

\v{63}Of the Priests:

Descendants of Habaiah, Koz,\fnote{Cf. Ezra 7:61 \fbib{Hakkoz}} and Barzillai, who married one of the daughters of Barzillai from Gilead and took that name.

\v{64}These people searched for their ancestral records, but they couldn't be located. Accordingly, they were considered disqualified\fnote{Lit. \fbib{unclean}} from the priesthood. \v{65}The governor\fnote{Lit. \fbib{Tirshatha}; i.e. a Persian title} ordered them not to eat anything holy until a priest would be installed with Urim and Thummim.\fnote{I.e. a high priest to whom God would reveal his will through the jewel-encrusted breastplate that he wore; cf. Exod 28:30, Ezra 2:63}

\v{66}The entire assembly numbered 42,360, \v{67}not including their 7,337 male and female servants. They had 245\fnote{Cf. Ezra 2:65 \fbib{200}} men and women singers. \v{68}\fnote{Some MT mss. lack this v.}They had 736 horses, 245 mules, \v{69}435 camels, and 6,720 donkeys.
\passage{Gifts for the Temple}
\passageinfo{(Ezra 2:68-70)}

\v{70}Some of the heads of the families contributed to the work. The governor\fnote{Lit. \fbib{Tirshatha}; i.e. a Persian title} contributed 1,000 gold drachmas to the treasury, along with 50 basins, and 530 priestly garments. \v{71}Some of the heads of the families gave to the treasury 20,000 gold drachmas and 2,200 silver units\fnote{Lit. \fbib{mina}} for the work. \v{72}The rest of the people gave 20,000 gold drachmas, 2,000 silver units\fnote{Lit. \fbib{mina}}, and 67 priestly garments.

\v{73}The priests, descendants of Levi, gatekeepers, singers, some of the people, the Temple Servants, and all the Israelis settled in their cities.
\labelchapt{8}
\passage{Ezra Reads the Law}
\passageinfo{(Ezra 3:1)}

Seven months later,\fnote{Lit. \fbib{When the seventh month came}; cf. Ezra 3:1} the Israelis had settled in their own cities.\chapt{8}
\v{1}All the people gathered as a united body\fnote{Lit. \fbib{as one man}} into the plaza in front of the Water Gate. They asked Ezra the scribe to bring out the Book of the Law of Moses, which the \divine{Lord} had commanded for Israel. \v{2}So on the first day of the seventh month, Ezra the priest brought out the Law before the assembled people. Both men and women were in attendance, as well as\fnote{The Heb. lacks \fbib{were in attendance, as well as}} all\fnote{Lit. \fbib{women and everyone}} who could understand what they were hearing.

\v{3}Ezra\fnote{Lit. \fbib{He}} read from it, facing the plaza in front of the Water Gate, from early in the morning until mid-day in the presence of the men and women, as well as all who could understand. All the people were attentive to the Book of the Law. \v{4}Ezra the scribe stood on a wooden rostrum erected for that purpose. Beside him to his right stood Mattithiah, Shema, Anaiah, Uriah, Hilkiah, and Maasseiah. Beside him to his left stood Pedaiah, Mishael, Malchijah, Hashum, Hashbaddanah, Zechariah, and Meshullam.

\v{5}Ezra opened the book in the sight of all the people. Because he was visible\fnote{The Heb. lacks \fbib{visible}} above all the people there, as he opened it, all the people stood up. \v{6}Ezra blessed the \divine{Lord}, the great God, and with uplifted hands, all the people responded, ``Amen! Amen!'' They bowed down and worshipped the \divine{Lord} prostrate on the ground.

\v{7}Furthermore, Jeshua, Bani, Sherebiah, Jamin, Akkub, Shabbethai, Hodiah, Maaseiah, Kelita, Azariah, Jozabad, Hanan, Pelaiah, and the descendants of Levi taught the Law to the people while the people remained standing. \v{8}They read from the Book of the Law of God, distinctly communicating its meaning, so they could understand the reading.
\passage{A Declaration to Rejoice}

\v{9}Because all the people were weeping as they listened to the words of the Law, Nehemiah the governor,\fnote{Lit. \fbib{Tirshatha}; i.e. a Persian title} Ezra the priest and scribe, and the descendants of Levi who taught the people told everyone, ``This day is holy to the \divine{Lord} your God. Do not mourn or weep.'' \v{10}He also told them, ``Go eat the best food, drink the best wine,\fnote{Or \fbib{drink sweet drinks}} and give something to those who have nothing, since this day is holy to our Lord. Don't be sorrowful, because the joy of the \divine{Lord} is your strength.''

\v{11}The descendants of Levi also calmed all the people by saying, ``Be still, for the day is holy. Don't be sorrowful!''

\v{12}So all the people went to eat, to drink, to send something to those who had nothing,\fnote{The Heb. lacks \fbib{to those who had nothing}} and to celebrate with great joy, because they understood the words that were being declared to them.
\passage{The Festival of Tents is Reinstituted}
\passageinfo{(Leviticus 23:33-43)}

\v{13}The next day, the heads of the families of all the people were gathered together, along with the priests and the descendants of Levi, to meet with\fnote{The Heb. lacks \fbib{meet with}} Ezra the scribe in order to understand the words of the Law. \v{14}They found written in the Law that the \divine{Lord} had commanded through Moses that the Israelis were to live in tents\fnote{I.e. booth-like structures covered with branches; cf. Lev 23:34,40,42} during the festival scheduled for the seventh month. \v{15}So they circulated a proclamation throughout their towns and in Jerusalem. It said, ``Go out to the hill country and bring back olive branches, wild olive branches, myrtle branches, palm branches, and branches of mature trees, in order to set up tents, as has been written.''

\v{16}Then the people went out and found branches to make tents for themselves on the roofs of their houses, in their courtyards, and in the courts of God's Temple, in the plaza near the Water Gate, and in the plaza near the Gate of Ephraim. \v{17}The entire assembly of those who had returned from exile erected tents and lived in them. Indeed, from the days of Nun's son Joshua until that day the Israelis had not done so. Joy was everywhere,\fnote{Lit. \fbib{was very abundant}} \v{18}and Ezra\fnote{Lit. \fbib{he}} continued to read from the Book of the Law of God day by day, from the first day through the last. They celebrated for seven days, and on the eighth day they held a solemn assembly according to regulation.
\labelchapt{9}
\passage{The People Confess Their Sins}

\chapt{9}
\v{1}On the twenty-fourth day of this same month, the Israelis gathered together while fasting, wearing sackcloth, and covering themselves with dust. \v{2}The remnant\fnote{Lit. \fbib{seed}} of Israel separated themselves from all foreigners. Then they stood and confessed their sins and the iniquities of their ancestors. \v{3}While they stood there, they read from the Book of the Law of the \divine{Lord} their God for one fourth of the day, and they confessed and worshipped the \divine{Lord} their God for another\fnote{Lit. \fbib{one}} fourth of the day.
\passage{The Descendants of Levi's Prayer of Blessing}

\v{4}Jeshua, Bani, Kadmiel, Shebaniah, Bunni, Sherebiah, Bani, and Chenani stood on the rostrum assigned for use by the descendants of Levi and cried out loudly to the \divine{Lord} their God. \v{5}Then the descendants of Levi---Jeshua, Kadmiel, Bani, Hashabneiah, Sherebiah, Hodiah, Shebaniah, and Pethahiah---said,

\begin{poetry}
\poeml ``Stand up and bless the \divine{Lord} your God \\
\poemll    from eternity to eternity! \\
\poeml Blessed be your glorious name! \\
\poemll    May it be exalted above all blessing and praise! \\
\poeml \v{6}``You are the \divine{Lord}; \\
\poemll    you alone crafted the heavens, \\
\poeml the highest heavens \\
\poemll    with all of their armies; \\
\poeml the earth, and everything in it; \\
\poemll    the seas, and everything in them; \\
\poeml you keep giving all of them life, \\
\poemll    and the army of heaven continuously worships you. \\
\poeml \v{7}You are the \divine{Lord}, \\
\poemll    the God who chose Abram, \\
\poeml whom you brought from Ur of the Chaldeans \\
\poemll    and to whom you gave the name Abraham. \\
\poeml \v{8}You found him\fnote{Lit. \fbib{found his heart}} faithful in your sight; \\
\poemll    you made a covenant with him \\
\poeml and you gave the land of the Canaanites, the Hittites, \\
\poemll    the Amorites, the Perizzites, the Jebusites, \\
\poemlll       and the Girgashites to his descendants. \\
\poeml And you have kept your word, \\
\poemll    because you are righteous. \\
\poeml \v{9}``You took note of the affliction of our ancestors in Egypt, \\
\poemll    and listened to their cry at the Red Sea. \\
\poeml \v{10}You sent signs and wonders against Pharaoh, \\
\poemll    against all of his officials, \\
\poeml and against all the people of his land, \\
\poemll    because you knew they acted arrogantly against your people.\fnote{Lit. \fbib{against them}} \\
\poeml So you established your name with them, \\
\poemll    as it remains to this day. \\
\poeml \v{11}You divided the sea in front of them, \\
\poemll    and they traveled through the midst of the sea on dry ground. \\
\poeml You hurled their pursuers into the depths, \\
\poemll    as one throws\fnote{The Heb. lacks \fbib{one throws}} a stone into turbulent waters. \\
\poeml \v{12}You led them during the day by a pillar of cloud, \\
\poemll    and by a pillar of fire at night \\
\poeml to provide light for them \\
\poemll    on the path they took. \\
\poeml \v{13}``You also came down to Mount Sinai, \\
\poemll    spoke with them from heaven, \\
\poeml and gave them impartial regulations, true laws, \\
\poemll    statutes, and good commands. \\
\poeml \v{14}You revealed to them your holy Sabbath, \\
\poemll    and you mandated precepts, statutes, and laws \\
\poemlll       through Moses your servant. \\
\poeml \v{15}You gave them food from heaven for their hunger \\
\poemll    and water from the rock for their thirst. \\
\poeml You directed them to enter and possess the land \\
\poemll    that you had promised to give them. \\
\poeml \v{16}``But they---our ancestors---became arrogant and stubborn, \\
\poemll    refusing to listen\fnote{Or \fbib{obey}} to your commands. \\
\poeml \v{17}They would not listen,\fnote{Or \fbib{obey}} \\
\poemll    and did not remember the miracles you did among them. \\
\poeml Instead, they became stubborn \\
\poemll    and appointed a leader \\
\poemlll       to return them to their slavery. \\
\poeml ``But you are a God of forgiveness, \\
\poemll    gracious and compassionate, \\
\poeml slow to anger, \\
\poemll    and rich in gracious love; \\
\poemlll       therefore you did not abandon them. \\
\poeml \v{18}Moreover, after they had cast a golden calf for themselves, they said, \\
\poemll    ``This is your god who brought you out of Egypt!'' \\
\poemlll       and committed terrible\fnote{Or \fbib{great}} blasphemies. \\
\poeml \v{19}You, in your great compassion, \\
\poemll    did not abandon them in the wilderness. \\
\poeml The pillar of cloud did not leave them in daylight, \\
\poemll    in order to provide light for them on the path they took. \\
\poeml Nor did the pillar of fire abandon them\fnote{The Heb. lacks \fbib{abandon them}} at night, \\
\poemll    in order to provide light for them \\
\poemlll       and lead them on the path they took. \\
\poeml \v{20}``You gave your good Spirit to instruct them, \\
\poemll    not withholding manna from them,\fnote{Lit. \fbib{from their mouths}} \\
\poemlll       and providing water to quench\fnote{The Heb. lacks \fbib{quench}} their thirst. \\
\poeml \v{21}You sustained them in the wilderness for 40 years. \\
\poemll    They lacked nothing. \\
\poeml Their clothes did not wear out, \\
\poemll    and their feet did not swell. \\
\poeml \v{22}You gave them kingdoms and nations, \\
\poemll    apportioning them as frontier boundaries. \\
\poeml They took possession of the land of Sihon, \\
\poemll    the land of the king of Heshbon, \\
\poemlll       and the land of Og, king of Bashan. \\
\poeml \v{23}``You multiplied their descendants like the stars in heaven \\
\poemll    and brought them to the land \\
\poeml about which you told their ancestors \\
\poemll    to enter and possess. \\
\poeml \v{24}So their descendants entered \\
\poemll    and took possession of the land. \\
\poeml Before their eyes you subdued those living in the land---the Canaanites--- \\
\poemll    putting them under their control, \\
\poeml along with their kings and the peoples of the land, \\
\poemll    so they could do with them as they pleased. \\
\poeml \v{25}They conquered fortified cities and fertile ground, \\
\poemll    possessing houses filled with all kinds of good things, \\
\poeml wells already dug, with vineyards, \\
\poemll    olive orchards, and fruit trees in abundance. \\
\poeml So they ate, were satiated, and were well nourished, \\
\poemll    delighting themselves in your great goodness. \\
\poeml \v{26}``Then they disobeyed, rebelled against you, \\
\poemll    and threw your Law behind their backs. \\
\poeml They murdered your prophets \\
\poemll    who had admonished the people\fnote{Lit. \fbib{admonished them}} to return to you, \\
\poemlll       committing terrible blasphemies. \\
\poeml \v{27}So you delivered them into the control of their enemies, \\
\poemll    who oppressed them. \\
\poeml But when they were oppressed, \\
\poemll    they cried out to you, \\
\poemlll       and you heard from heaven. \\
\poeml In your great compassion \\
\poemll    you gave them deliverers who rescued them \\
\poemlll       from the control of their enemies. \\
\poeml \v{28}``But after they had gained relief, \\
\poemll    they returned to doing evil before you. \\
\poeml Therefore you abandoned them to the control of their enemies, \\
\poemll    who continued to oppress them. \\
\poeml But when they came back and cried out to you, \\
\poemll    you listened from heaven \\
\poemlll       and delivered them in your compassion on many occasions. \\
\poeml \v{29}You admonished them to return to your Law, \\
\poemll    but they acted arrogantly, \\
\poemlll       and would not listen\fnote{Or \fbib{obey}} to your commands. \\
\poeml They sinned against your regulations, \\
\poemll    which if anyone obeys, \\
\poemlll       he will live by them. \\
\poeml They turned away, \\
\poemll    being stubborn and stiff-necked, \\
\poemlll       and they did not listen.\fnote{Or \fbib{obey}} \\
\poeml \v{30}You were patient with them for many years, \\
\poemll    warning them by your Spirit \\
\poemlll       through\fnote{Lit. \fbib{through the hand of}} your prophets. \\
\poeml But they would not listen, \\
\poemll    so you turned them over \\
\poemlll       to the control of people in other\fnote{Lit. \fbib{the}} lands. \\
\poeml \v{31}Nevertheless, in your great compassion \\
\poemll    you did not completely destroy them \\
\poemlll       or abandon them, \\
\poeml because you are a God of grace \\
\poemll    and you are merciful. \\
\poeml \v{32}``Now therefore, our God, \\
\poemll    the great, mighty, and awesome God, \\
\poemlll       who keeps the covenant and gracious love, \\
\poeml don't let all of the difficulties seem trifling to you, \\
\poemll    all of hardships that have come upon us, upon our kings, \\
\poeml upon our leaders, upon our priests, \\
\poemll    upon our prophets, upon our ancestors, \\
\poeml and upon all of your people \\
\poemll    from the time of the kings of Assyria until this day. \\
\poeml \v{33}You are righteous in all that is happening to us, \\
\poemll    because you have acted faithfully \\
\poemlll       while we have practiced evil. \\
\poeml \v{34}Furthermore, neither our kings, \\
\poemll    nor our leaders, nor our priests \\
\poemlll       nor our ancestors have practiced your Law \\
\poeml or paid attention to your commands and warnings \\
\poemll    by which you admonished them. \\
\poeml \v{35}But they in their kingdom--- \\
\poemll    in the midst of your great goodness that you gave them \\
\poeml and in the large and fertile land \\
\poemll    that you provided them--- \\
\poeml did not serve you \\
\poemll    or turn away from their evil deeds. \\
\poeml \v{36}``Look! Today we are your servants, \\
\poemll    along with the land that you gave to our ancestors, \\
\poeml so they could enjoy its fruit and its value--- \\
\poemll    behold, in it we are your servants! \\
\poeml \v{37}But now its abundant produce belongs to the kings \\
\poemll    whom you placed over us \\
\poemlll       because of our sin. \\
\poeml They also have power over our bodies and our herds \\
\poemll    at their pleasure, \\
\poemlll       and we are in great distress. \\
\poeml \v{38}``Because of all this, we are making a binding agreement, \\
\poemll    putting it in writing, \\
\poeml and our leaders, our descendants of Levi, and our priests \\
\poemll    hereby set their seals upon it.''\fnote{The Heb. lacks \fbib{it}}
\end{poetry}
\labelchapt{10}
\passage{Signatories to the Agreement}

\chapt{10}
\v{1}\fnote{This v. is 10:2 in MT, and so throughout the chapter}Here is a list of those who signed: Hacaliah's son Nehemiah the governor, Zedekiah, \v{2}Seraiah, Azariah, Jeremiah, \v{3}Pashur, Amariah, Malchijah, \v{4}Hattush, Shebaniah, Malluch, \v{5}Harim, Meremoth, Obadiah, \v{6}Daniel, Ginnethon, Baruch, \v{7}Meshullam, Abijah, Mijamin, \v{8}Maaziah, Bilgai, and Shemaiah---these are the priests.

\v{9}These were the descendants of Levi: Azaniah's son Jeshua, Binnui from the descendants of Henadad, Kadmiel, \v{10}also their relatives Shebaniah, Hodiah, Kelita, Pelaiah, Hanan, \v{11}Mica, Rehob, Hashabiah, \v{12}Zaccur, Sherebiah, Shebaniah, \v{13}Hodiah, Bani, and Beninu.

\v{14}The leaders of the people included\fnote{The Heb. lacks \fbib{included}} Parosh, Pahath-moab, Elam, Zattu, Bani, \v{15}Bunni, Azgad, Bebai, \v{16}Adonijah, Bigvai, Adin, \v{17}Ater, Hezekiah, Azzur, \v{18}Hodiah, Hashum, Bezai, \v{19}Hariph, Anathoth, Nebai, \v{20}Magpiash, Meshullam, Hezir, \v{21}Meshezabel, Zadok, Jaddua, \v{22}Pelatiah, Hanan, Anaiah, \v{23}Hoshea, Hananiah, Hasshub, \v{24}Hallohesh, Pilha, Shobek, \v{25}Rehum, Hashabnah, Maaseiah, \v{26}Ahiah, Hanan, Anan, \v{27}Malluch, Harim, and Baanah.
\passage{Commitments of the Covenant}

\v{28}The rest of the people, the priests, the descendants of Levi, the gatekeepers, the singers, the Temple Servants, and everyone who had separated themselves from the peoples of the surrounding\fnote{The Heb. lacks \fbib{surrounding}} lands for the Law of God---their wives, their sons, their daughters, and all who had knowledge and understanding--- \v{29}joined with their relatives and their leaders. They entered into an oath---enforced by a curse\fnote{Lit. \fbib{into a curse and an oath}}---to walk in God's Law that was given through God's servant Moses, and to be careful to obey all of the commands of the \divine{Lord}, our Lord, as well as his regulations and statutes: \v{30}``We will not give our daughters in marriage\fnote{The Heb. lacks \fbib{in marriage}} to the people of the land, nor take their daughters for our sons. \v{31}As for the people of the land who bring merchandise or grain to sell on the Sabbath day, we will not buy from them on the Sabbath or on any holy day. We will forego planting crops, and we will cancel debts during every seventh year.''
\passage{Commitments for Temple Service}

\v{32}We also obligated ourselves to contribute annually a third of a shekel\fnote{I.e. 0.13 ounces; a shekel weighed about 0.4 ounces} for services relating to the Temple of our God--- \v{33}for the bread set out on the table,\fnote{The Heb. lacks \fbib{on the table}} for the daily grain offering, for the continual burnt offering, for the Sabbath offerings, for the New Moon festivals, for the appointed festivals, for the holy offerings, for the sin offerings to make atonement for Israel, and for all the service of the Temple of our God.

\v{34}We---the priests, the descendants of Levi, and the people---cast lots to determine when to bring the wood offering into the Temple of our God, just as our ancestors' families were appointed annually to maintain the altar fire of the \divine{Lord} our God, as recorded in the Law. \v{35}We also cast lots to determine when\fnote{The Heb. lacks \fbib{We also cast lots to determine when}} to bring the first fruits of our land and the annual first fruits of all fruit of every tree to the Temple of the \divine{Lord}, \v{36}as well as the firstborn of our sons and our cattle, as recorded in the Law, along with the firstlings of our herds and our flocks, to present to the Temple of our God for the priests that minister in the Temple of our God. \v{37}We also determined\fnote{The Heb. lacks \fbib{determined}} to present the first fruits of our ground grain, our offerings, the fruit of all kinds of trees, wines, and oil to the priests, to the chambers of the Temple of our God, and the tithes of our land to the descendants of Levi, so those descendants of Levi could collect the tithes in all the towns where we worked: \v{38}``And the priest, the descendant of Aaron, will be with the descendants of Levi when the descendants of Levi receive tithes, and the descendants of Levi will bring the tithe of the tithes into the store rooms of the Temple of our God. \v{39}For the Israelis and the descendants of Levi will bring the grain offering, the wine, and the oil into the chambers where the vessels of the sanctuary are, along with the ministering priests, the porters, and the singers. We will not neglect the Temple of our God.''
\labelchapt{11}
\passage{Inhabitants of Jerusalem}

\chapt{11}
\v{1}The leaders of the people who lived in Jerusalem, along with the rest of the people, decided to choose one out of ten of them by lot to live in Jerusalem, the holy city, leaving the other nine of them in their towns. \v{2}And the people blessed all of the men who volunteered to live in Jerusalem.

\v{3}These are the leaders of the provinces who lived in Jerusalem. Some lived in the towns of Judah---each on their property in their respective towns---that is, the Israelis, the priests, the descendants of Levi, the Temple Servants, and the descendants of Solomon's servants.

\v{4}Some of the descendants of Judah and Benjamin lived in Jerusalem.

From Judah's Descendants:

Uzziah's son Athaiah, who was the son of Zechariah, the son of Amariah, the son of Shephatiah, the son of Mahalalel;

From Perez's Descendants

\v{5}Baruch's son Maaseiah, who was the son of Col-hozeh, the son of Hazaiah, the son of Adaiah, the son of Joiarib, the son of Zechariah, the son of the Shilonite. \v{6}All of the descendants of Perez who lived in Jerusalem numbered\fnote{The Heb. lacks \fbib{numbered}} 468 men of valor.

\v{7}These Benjamin's Descendants:

Meshullam's son Sallu, who was the son of Joed, the son of Pedaiah, the son of Kolaiah, the son of Maaseiah, the son of Ithiel, the son of Jeshaiah; \v{8}and after him Gabbai and Sallai, numbering\fnote{The Heb. lacks \fbib{numbering}} 928.

\v{9}Zichri's son Joel was their overseer, and Hassenuah's son Judah was in command of the second district of the city.

\v{10}From the Priests:

Joiarib's son Jedaiah, Jachin, \v{11}Hilkiah's son Seraiah, the son of Meshullam, the son of Zadok, the son of Meraioth, the son of Ahitub, the administrator of the Temple of God. \v{12}Their associates who performed the work of the Temple numbered\fnote{The Heb. lacks \fbib{numbered}} 822. Jeroham's son Adaiah, the son of Pelaliah, the son of Amzi, the son of Zechariah, the son of Pashhur, the son of Malchijah, \v{13}along with his associates, the leaders of the families,\fnote{Lit. \fbib{fathers}} numbered\fnote{The Heb. lacks \fbib{numbered}} 242, along with Amashsai, the son of Azarel, the son of Ahzai, the son of Meshillemoth, the son of Immer, \v{14}along with their relatives, 128 mighty, valiant men, and their overseer Zabdiel son of Haggedolim.

\v{15}From the descendants of Levi:

Shemaiah son of Hasshub, the son of Azrikam, the son of Hashabiah, the son of Bunni, \v{16}and Shabbethai and Jozabad, from the leaders of the descendants of Levi who oversaw the exterior work of the Temple of God, \v{17}and Mattaniah son of Mica, the son of Zabdi, the son of Asaph, who led the thanksgiving prayer, and Bakbukiah, second among his relatives, and Abda son of Shammua, the son of Galal, the son of Jeduthun. \v{18}All of the descendants of Levi in the holy city numbered\fnote{The Heb. lacks \fbib{numbered}} 284.

\v{19}The Gatekeepers:

Akkub, Talmon, and their relatives, who kept watch at the gates, numbered\fnote{The Heb. lacks \fbib{numbered}} 172.
\passage{Those who Lived Outside Jerusalem}

\v{20}The rest of Israel---the priests and the descendants of Levi---lived in all the cities of Judah, each on his own property, \v{21}but the Temple Servants lived on Ophel. Ziha and Gishpa oversaw the Temple Servants.

\v{22}The overseer of the descendants of Levi at Jerusalem was Uzzi son of Bani, the son of Hashabiah, the son of Mattaniah, the son of Mica. Singers from the descendants of Asaph oversaw the work of the Temple of God. \v{23}They were subject to the commands of the king, who provided for the singers daily. \v{24}Pethahiah son of Meshezabel, a descendant of Zerah son of Judah, represented the king\fnote{Lit. \fbib{Judah, was at the king's hand}} in all matters concerning the people.
\passage{Outlying Towns}

\v{25}Now concerning the villages and their fields, some of the people of Judah lived in Kiriath-arba and its villages, in Dibon and its villages, in Jekabzeel and its villages, \v{26}in Jeshua, in Moladah, and Beth-pelet, \v{27}in Hazar-shual, in Beer-sheba and its villages, \v{28}in Ziklag, in Meconah and its villages, \v{29}in En-rimmon, in Zorah, in Jarmuth, \v{30}in Zanoah, Adullam, and their villages, Lachish and its fields, and Azekah and its villages. They encamped from Beer-sheba to the Hinnom Valley.

\v{31}The descendants of Benjamin lived from Geba to Michmash, Aija, Bethel and its villages, \v{32}Anathoth, Nob, Ananiah, \v{33}Hazor, Ramah, Gittaim, \v{34}Hadid, Zeboim, Neballat, \v{35}Lod, and Ono's Craftsmen Valley, \v{36}with some Levitical divisions of Judah pertaining to Benjamin.
\labelchapt{12}
\passage{Priests and Descendants of Levi}
\passageinfo{(Ezra 2:36-40)}

\chapt{12}
\v{1}These are the priests and descendants of Levi who had returned with Shealtiel's son Zerubbabel and with Jeshua: Seraiah, Jeremiah, Ezra, \v{2}Amariah, Malluch, Hattush, \v{3}Shecaniah, Rehum, Meremoth, \v{4}Iddo, Ginnethoi, Abijah, \v{5}Mijamin, Maadiah, Bilgah, \v{6}Shemaih, Joiarib, Jedaiah, \v{7}Sallu, Amok, Hilkiah, and Jedaiah. These were the leaders of the priests and their associates in the time of Jeshua.

\v{8}The descendants of Levi included Jeshua, Binnui, Kadmiel, Sherebiah, Judah, and Mattaniah, who with his associates was in charge of the songs of thanksgiving. \v{9}Bakbukiah and Unni and their associates stood opposite them in the service. \v{10}Jeshua fathered Joiakim, Joiakim fathered Eliashib, and Eliashib fathered Joiada. \v{11}Joiada fathered Jonathan and Jonathan fathered Jaddua.

\v{12}These were the priests and heads of their ancestors' houses in the time of Joiakim: of Seraiah, Meraiah; of Jeremiah, Hananiah; \v{13}of Ezra, Meshullam; of Amariah, Jehohanan; \v{14}of Malluchi, Jonathan; of Shebaniah, Joseph; \v{15}of Harim, Adna; of Meraioth, Helkai; \v{16}of Iddo, Zechariah; of Ginnethon, Meshullam; \v{17}of Abijah, Zichri; of Miniamin, of Moadiah, Piltai; \v{18}of Bilgah, Shammua; of Shemaiah, Jehonathan; \v{19}of Joiarib, Mattenai; of Jedaiah, Uzzi; \v{20}of Sallai, Kallai; of Amok, Eber; \v{21}of Hilkiah, Hashabiah; of Jedaiah, Nethanel.

\v{22}When Eliashib, Joiada, Johanan, and Jaddua were serving, the descendants of Levi were recorded as heads of their ancestors' houses, as were the priests during the reign of Darius the Persian. \v{23}The leaders of the ancestors of Levi were written in the Book of Annals until the time of Eliashib's son Johanan.

\v{24}The leaders of the descendants of Levi were: Hashabiah, Sherebiah, and Kadmiel's son Jeshua, along with their associates who served opposite them to give praise and thanks, division by division, according to the commands given by David the man of God. \v{25}Mattaniah, Bakbukiah, Obadiah, Meshullam, Talmon, and Akkub were gatekeepers who guarded the store houses of the gates. \v{26}These were at the time of Jeshua's son Joiakim, the grandson of Jozadak, and in the time of Nehemiah the governor and Ezra the priest and scribe.
\passage{The Wall is Dedicated}

\v{27}At the dedication of the wall of Jerusalem, they invited the descendants of Levi to come from wherever they lived to Jerusalem so they could celebrate the dedication with joy, thanksgiving, and songs, accompanied by\fnote{The Heb. lacks \fbib{accompanied by}} cymbals, lyres, and harps. \v{28}So the descendants of the singers gathered themselves together from the region surrounding Jerusalem, from the villages of Netophathi, \v{29}from Beth-gilgal, and from the area of Geba and Azmaveth, because the singers had built villages for themselves in the vicinity of Jerusalem. \v{30}The priests and the descendants of Levi purified themselves, and also purified the people, the gates, and the wall.
\passage{The Procession on the Wall}

\v{31}Then I brought up the leaders of Judah to the crest of the wall, and appointed two large thanksgiving choirs, the first of which\fnote{The Heb. lacks \fbib{the first of which}} proceeded on the wall to the right toward the Dung Gate. \v{32}Following them were Hoshaiah and half of the leaders of Judah, \v{33}including Azariah, Ezra, Meshullam, \v{34}Judah, Benjamin, Shemaiah, and Jeremiah. \v{35}Some of the priests' sons were trumpeters, including Zechariah son of Jonathan, the son of Shemaiah, the son of Mattaniah, the son of Micaiah, the son of Zaccur, the son of Asaph, \v{36}with his associates Shemaiah, Azarel, Milalai, Gilalai, Maai, Nethanel, Judah, and Hanani, accompanied by the musical instruments of David, the man of God. Ezra the scribe led the procession. \v{37}At the Fountain Gate, which stood opposite them, they ascended the stairs of the City of David where the wall rose above the house of David east of the Water Gate.

\v{38}The second thanksgiving choir approached opposite them, and I followed them. Half of the people stood on the crest of the wall from beyond the Tower of the Ovens to the Broad Wall, \v{39}and from above the Ephraim Gate, above the Fish Gate, the Tower of Hananel and the Tower of the Hundred, as far as the Sheep Gate. They stopped at the Guard Gate. \v{40}Then the two choirs assembled in the Temple of God, as did I, along with half of the officials who accompanied me, \v{41}and the priests Eliakim, Maaseiah, Miniamin, Micaiah, Elioenai, Zechariah, Hananiah with trumpeters \v{42}Maaseiah, Shemaiah, Eleazar, Uzzi, Jehohanan, Malchijah, Elam, and Ezer. And the singers made their presence known, with Jezrahiah to lead them.

\v{43}That day they offered a large number of sacrifices, and they rejoiced, because God had caused them to rejoice enthusiastically. Their wives and children rejoiced, so that Jerusalem's joy was heard from a long distance. \v{44}Also at that time men were appointed over the storerooms for the contributions, for the first fruits, and for the tithes, so those portions required by the Law could be gathered from the fields adjacent to the towns to benefit the priests and descendants of Levi, for the people of\fnote{The Heb. lacks \fbib{the people of}} Judah rejoiced over the priests and the descendants of Levi who were serving. \v{45}They carried out their service obligations to their God and their service obligations of purification according to what David and his son Solomon had commanded. \v{46}For in David's lifetime---and in the lifetime of Asaph, choir master of old---there were songs of praise and thanksgiving to God. \v{47}All Israel in the time of Zerubbabel and in the time of Nehemiah gave allotments to each of the singers and gate keepers on a daily basis, setting them apart to benefit the descendants of Levi. And the descendants of Levi set them apart to benefit the descendants of Aaron.
\labelchapt{13}
\passage{Enemies of Israel Excluded}
\passageinfo{(Numbers 22:1-24:25)}

\chapt{13}
\v{1}Later that day the book of Moses was read aloud so the people could hear it, and a written command was discovered therein\fnote{Cf. Deut 23:3-5} permanently prohibiting the Ammonites and Moabites from coming into the congregation of God \v{2}because they did not greet the Israelis with food and water, but instead hired Balaam to oppose them by cursing them, even though our God turned the curse into a blessing. \v{3}When they heard the Law, they separated all those of foreign descent from Israel.
\passage{Tobiah Evicted from the Temple}

\v{4}Now prior to this, Eliashib the priest, who supervised the store rooms of the Temple of our God and who was related to Tobiah, \v{5}had prepared a great chamber for him, in the place where they used to place the grain offerings, incense, and vessels, along with the tithes of the grain, the new wine, and the oil that was mandated for the descendants of Levi, the singers, the gate keepers, and the priests' offerings. \v{6}During all of this time, I was not in Jerusalem, because I had returned to the king in the thirty-second year of Artaxerxes, king of Babylon. After a while I obtained permission from the king \v{7}to return to Jerusalem. I learned of the evil thing that Eliashib had done for Tobiah in furnishing him with a room in the courts of the Temple of God. \v{8}I was greatly upset, so I threw out all of Tobiah's property from the room. \v{9}I ordered them to purify the chambers, and then they brought back the vessels from the Temple of God, along with the grain offerings and incense.
\passage{Neglecting Levitical Allotments}

\v{10}I also learned that the allotments for the descendants of Levi had not been distributed. As a result, the descendants of Levi and singers who were responsible for the service had each left to go back to their fields. \v{11}So I confronted the officials and asked, ``Why is the Temple of God neglected?'' Then I gathered them together and put them back in their places. \v{12}Then all of Judah brought the tithe of the grain, the new wine, and the oil into the storerooms. \v{13}I appointed over the storerooms: Shelemiah the priest, Zadok the scribe, and Pedaiah from the descendants of Levi; and next to them Zaccur's son Hanan, the grandson of Mattaniah, because they had been considered faithful. Their duties were to distribute to their associates.

\v{14}Remember me, my God, concerning this, and do not erase my faithful deeds that I have undertaken for the Temple of my God, and for its services.
\passage{Prohibiting Work on the Sabbath}

\v{15}At that time I saw in Judah some who were treading wine presses on the Sabbath, bringing in sacks of grain, loading them onto donkeys, along with wine, grapes, figs, and all kinds of loads. They brought them into Jerusalem on the Sabbath day. So I rebuked them on the day on which they were selling food. \v{16}Furthermore, Tyrians were living there who were importing fish and all kinds of merchandise, selling them to the people of Judah on the Sabbath, even in Jerusalem.

\v{17}I rebuked the officials of Judah, saying to them, ``What's this evil thing that you're doing by profaning the Sabbath day? \v{18}Didn't your ancestors do the same? And didn't our God bring on us and on this city all of this trouble? Now you're adding to the wrath against Israel by profaning the Sabbath!''

\v{19}As the Sabbath approached and it began to get dark at the gates of Jerusalem, I gave word to shut the gates, charging that they should not be opened until after the Sabbath. I stationed some of my men at the gates to ensure that no loads would be brought in on the Sabbath day. \v{20}As a result, the merchants and sellers of all sorts of goods remained outside Jerusalem a couple of times. \v{21}I argued with them, ``Why are you staying outside the wall? If you do this again, I'll arrest you.'' From that time on, they didn't come anymore on the Sabbath. \v{22}Then I commanded the descendants of Levi to purify themselves and to come as gate keepers to sanctify the Sabbath day.

Remember me, my God, and show mercy to me according to the greatness of your gracious love.
\passage{Removing Foreign Spouses}
\passageinfo{(Ezra 9:1-4)}

\v{23}At that time I also noticed that Jews had married women from Ashdod, Ammon, and Moab. \v{24}Furthermore, their children spoke half of the time in the language of Ashdod, and could not speak in the language of Judah. Instead, they spoke in the languages of various peoples. \v{25}So I rebuked them, cursed them, struck some of their men, tore out their hair, and made them take this oath in the name of God: ``You are not to give your daughters to their sons nor take their daughters for your sons or for yourselves. \v{26}Didn't Solomon, king of Israel, sin by doing these things, even though among many nations there was no king like him who was loved by his God, and God made him king over all Israel? Even so, foreign women caused him to sin. \v{27}Should we listen to you and do all of this terrible evil by transgressing against our God to marry foreign wives?'' \v{28}One of the sons of Eliashib the high priest's son Joiada was a son-in-law to Sanballat the Horonite, so I drove him away from me.

\v{29}Remember them, my God, because they have defiled the priesthood and the covenant of the priesthood and the descendants of Levi.

\v{30}I purified them from everything foreign, arranged duties for the priests and the descendants of Levi, each to his task, \v{31}and I arranged at the appointed time for the supply of wood, and for the first fruits.

Remember me, my God, with favor.

\bookheader{Esther}
\labelbook{Esth}

\bookpretitle{The Book of}
\booktitle{Esther}

\labelchapt{1}
\passage{The Wealth of King Ahasuerus}

\chapt{1}
\v{1}This is a record of\fnote{\fbackref{1:1} The Heb. lacks \fbib{a record of}} what happened during the reign\fnote{\fbackref{1:1} Lit. \fbib{days}} of Ahasuerus,\fnote{\fbackref{1:1} Or possibly \fbib{Xerxes}, and so throughout the book} the Ahasuerus who ruled over 127 provinces from India to Cush.\fnote{\fbackref{1:1} I.e. the upper Nile region} \v{2}At that time King Ahasuerus was ruling from\fnote{\fbackref{1:2} Lit. \fbib{was sitting on his royal throne in}} Susa the capital.\fnote{\fbackref{1:2} Or \fbib{Susa the fortress,} and so throughout the book} \v{3}In the third year of his reign, he gave a banquet for all his officials and ministers,\fnote{\fbackref{1:3} Or \fbib{his servants}} and the military leaders\fnote{\fbackref{1:3} Lit. \fbib{army}} of Persia and Media, the nobles, and the provincial officials were present.\fnote{\fbackref{1:3} Lit. \fbib{before him}} \v{4}He displayed the enormous wealth of his kingdom, along with its splendid beauty and greatness for many days---for 180 days in all.\fnote{\fbackref{1:4} The Heb. lacks \fbib{in all}}
\passage{The Banquet of King Ahasuerus}

\v{5}When those days were over, the king held a seven-day banquet in the courtyard of the garden of his\fnote{\fbackref{1:5} Lit. \fbib{the king's}} palace for all the people who were present in Susa the capital, from the greatest to the least important. \v{6}There were curtains of white and blue linen tied with cords of fine linen and purple material\fnote{\fbackref{1:6} The Heb. lacks \fbib{material}} to silver rings on\fnote{\fbackref{1:6} Lit. \fbib{and}} marble columns. There were couches of gold and silver on a mosaic pavement of porphyry, marble, mother-of-pearl and other precious stones. \v{7}Drinks were served in gold vessels of various kinds, and there was plenty of royal wine because the king was generous.\fnote{\fbackref{1:7} Lit. \fbib{wine according to the hand of the king}} \v{8}According to the king's\fnote{\fbackref{1:8} The Heb. lacks \fbib{king's}} decree the drinking was not compulsory because the king instructed\fnote{\fbackref{1:8} Lit. \fbib{established}} every steward in his house to serve each individual what he desired. \v{9}Queen Vashti also held a banquet in the royal palace of King Ahasuerus for the women.
\passage{Vashti Refuses to Obey the King}

\v{10}A week later, when the king was under the influence of all that wine,\fnote{\fbackref{1:10} Lit. \fbib{the heart of the king was happy with wine}} he ordered Mehuman, Biztha, Harbona, Bigtha, Abagtha, Zethar, and Carcas, the seven eunuchs who served King Ahasuerus, \v{11}to bring Queen Vashti to the king, wearing\fnote{\fbackref{1:11} Lit. \fbib{with}} the royal crown to display her beauty to the people and the officials, since she was lovely to look at. \v{12}Queen Vashti refused to come at the king's order that was brought by the eunuchs.
\passage{The King Removes Vashti as Queen}

Then the king flew into a rage. \v{13}The king spoke to the wise men who understood the times, for it was the king's custom to consult\fnote{\fbackref{1:13} Lit. \fbib{to speak before}} all those who understood law and justice. \v{14}(His closest advisors\fnote{\fbackref{1:14} Lit. \fbib{Those closest to him}} were: Carshena, Shethar, Admatha, Tarshish, Meres, Marsena, and Memucan, the seven officials of Persia and Media who had direct access\fnote{\fbackref{1:14} Lit. \fbib{saw the face of}} to the king and who held the highest rank\fnote{\fbackref{1:14} Lit. \fbib{sat first in the kingdom}} in the kingdom.) \v{15}The king inquired,\fnote{\fbackref{1:15} The Heb. lacks \fbib{The king inquired}} ``According to law, what should be done to Queen Vashti because she did not obey the order of King Ahasuerus that was delivered by the eunuchs?''

\v{16}Then Memucan replied in the presence of the king and his officials, ``It is not the king alone whom Vashti has wronged, but rather all of the officials and all of the people who are in the provinces of King Ahasuerus. \v{17}When the report about the queen goes out to all the women, it will cause them to despise their husbands.\fnote{\fbackref{1:17} Lit. \fbib{husbands in their eyes}} They'll say, `King Ahasuerus ordered Queen Vashti to be brought before him, but she wouldn't come.' \v{18}This very day the wives of the officials\fnote{\fbackref{1:18} Or \fbib{women of nobility}} of Persia and Media who hear the report about the queen will speak in the same way to all the officials of the king, and there will be more than enough contempt and anger. \v{19}If it seems good to the king, let a royal decree go out from him and let it be written in the laws of Persia and Media, which cannot be repealed, that Vashti is never again to enter the presence of King Ahasuerus. Let the king give her royal position to another woman who is better than she. \v{20}When the edict of the king that he issues is heard throughout his kingdom---for it's vast---then all the women will give honor to their husbands, from the greatest to the least important.''

\v{21}This seemed like a good idea\fnote{\fbackref{1:21} The Heb. lacks \fbib{idea}} to the king and his officials, so the king did what Memucan suggested.\fnote{\fbackref{1:21} Lit. \fbib{according to the word of Memucan}} \v{22}He sent letters to all the provinces of the king, written in the script of that province,\fnote{\fbackref{1:22} Lit. \fbib{to each province according to its writing}} and to each people in their own language, ordering that every man should be the master in his house and speak the language of his own people.
\labelchapt{2}
\passage{The King Searches for a New Queen}

\chapt{2}
\v{1}After this, when the anger of King Ahasuerus had subsided, he remembered Vashti, what she had done, and what had been decreed about her. \v{2}Then the young men who attended the king suggested, ``Let beautiful young virgins be sought for the king. \v{3}Let the king appoint officials in all the provinces of his kingdom to bring all the beautiful young virgins into the harem in Susa the capital. Let them be placed under the care of Hegai, the king's eunuch, who is in charge of the women to give them their beauty treatments.\fnote{\fbackref{2:3} Lit. \fbib{their massages}} \v{4}Then let the young woman who pleases the king rule in place of Vashti.'' This advice\fnote{\fbackref{2:4} The Heb. lacks \fbib{advice}} pleased the king, and he did this.
\passage{Esther's Background}

\v{5}In Susa the capital there was a Jewish man from the tribe of Benjamin, Jair's son Mordecai, who was a descendant of Kish's son Shimei the descendant of Benjamin. \v{6}He had been taken into captivity from Jerusalem along with the exiles who had been deported with Jeconiah, king of Judah, whom Nebuchadnezzar, king of Babylon had taken into exile.\fnote{\fbackref{2:6} This deportation took place in 597 B.C.} \v{7}Mordecai\fnote{\fbackref{2:7} Lit. \fbib{He}} had raised his cousin\fnote{\fbackref{2:7} Lit. \fbib{his uncle's daughter}} Hadassah, or Esther,\fnote{\fbackref{2:7} The Heb. name \fbib{Hadassah} means \fbib{Myrtle}; The Persian name \fbib{Esther} means \fbib{Star}} because she had no father or mother. The young woman had a beautiful figure and was very attractive. When her mother and father died Mordecai had taken her as his daughter.

\v{8}The king's order and edict was proclaimed, and many young women were brought to Susa the capital under the care of Hegai. Esther was taken to the palace into the care of Hegai, who was in charge of the women. \v{9}The young woman pleased him and gained his favor. He quickly provided her beauty treatments and gave her portions of food to her. He also assigned her seven suitable young women from the palace and transferred her and her young women to the best place in the harem. \v{10}Esther did not make known her people or heritage\fnote{\fbackref{2:10} Or \fbib{her ancestry}} because Mordecai had instructed her not to make it known. \v{11}Every day Mordecai would walk back and forth in front of the court of the harem to find out about Esther's well-being and what was happening to her.
\passage{Esther Becomes Queen}

\v{12}Each young woman's turn came to go in to King Ahasuerus at the end of the twelve month period, at which time she was treated according to the regulations for women. This process\fnote{\fbackref{2:12} The Heb. lacks \fbib{process}} completed the period of her beauty treatments---six months with oil of myrrh and six months with spices and cosmetics for women. \v{13}After that the young woman would go in to the king, and whatever she asked for would be given to her to take with her from the harem to the palace. \v{14}In the evening she would go into the palace\fnote{\fbackref{2:14} The Heb. lacks \fbib{the palace}} and in the morning she would return to the second harem, into the care of Shaashgaz, the king's eunuch who was in charge of the mistresses.\fnote{\fbackref{2:14} Or \fbib{concubines}; i.e. secondary wives} She would not go again to the king unless the king wanted her and she was called for by name. \v{15}Now Esther was the daughter of Abihail, who had been Mordecai's uncle. Mordecai had taken Esther in as his own\fnote{\fbackref{2:15} The Heb. lacks \fbib{own}} daughter. When her turn came to go in to the king, she did not want anything except what Hegai, the king's eunuch in charge of the harem, advised. Esther found favor with everyone who saw her. \v{16}Esther was taken to King Ahasuerus to his royal palace in the tenth month, which is the month Tebeth, in the seventh year of his reign.

\v{17}The king loved Esther more than any of the other women, so he favored her and was kinder to her than he was to any of the other virgins. He put the royal crown on her head and made her queen in place of Vashti. \v{18}The king put on a great banquet for all his officials and ministers\fnote{\fbackref{2:18} Or \fbib{servants}} to honor Esther. He declared a holiday for the provinces and gave generous gifts.\fnote{\fbackref{2:18} Lit. \fbib{gave gifts according to the hand of the king}}
\passage{Mordecai Thwarts a Plot to Kill Ahasuerus}

\v{19}When the virgins were gathered a second time, Mordecai was sitting in the king's gate. \v{20}Now Esther had not declared her heritage\fnote{\fbackref{2:20} Or \fbib{her ancestry}} or her people, just as Mordecai had instructed her, for Esther did what Mordecai told her just as she had done when she was raised by him. \v{21}At that time when Mordecai was sitting in the king's gate, Bigthan and Teresh, two of the king's eunuchs among those who guarded the threshold,\fnote{\fbackref{2:21} I.e. the entrance to the restricted areas of the palace} became angry and conspired to assassinate\fnote{\fbackref{2:21} Lit. \fbib{to send a hand against}} King Ahasuerus. \v{22}When Mordecai learned about the plot, he told Queen Esther, and she told the king in Mordecai's name. \v{23}After the matter had been fully investigated, Bigthan and Teresh\fnote{\fbackref{2:23} Lit. \fbib{investigated, the two of them}} were hanged\fnote{\fbackref{2:23} Or \fbib{impaled}} on a pole, and this was recorded in the Book of the Chronicles in the presence of the king.
\labelchapt{3}
\passage{Haman is Promoted by Ahasuerus}

\chapt{3}
\v{1}Sometime later King Ahasuerus promoted Hammedatha the Agagite's son Haman, elevating him to a position above\fnote{\fbackref{3:1} Lit. \fbib{setting his seat above}} all the other officials who were with him. \v{2}All the king's ministers\fnote{\fbackref{3:2} Or \fbib{servants}} who were in the king's gate would kneel and bow down to Haman, because the king had commanded that Haman\fnote{\fbackref{3:2} Lit. \fbib{commanded concerning him that he}} be honored in this way. Mordecai, however, would not kneel and would not bow down.

\v{3}The king's ministers\fnote{\fbackref{3:3} Or \fbib{servants}} who were in the king's gate asked Mordecai, ``Why are you disobeying the king's command?'' \v{4}They asked him this day after day, and he would not listen to them, so they told Haman to see whether or not Mordecai would get away with his disobedience,\fnote{\fbackref{3:4} Lit. \fbib{would stand}} since he also had told them that he was Jewish. \v{5}When Haman saw that Mordecai would not kneel and bow down to him, he\fnote{\fbackref{3:5} Lit. \fbib{Haman}} flew into a rage. \v{6}Because they had told him who the people of Mordecai were, Haman\fnote{\fbackref{3:6} Lit. \fbib{he}} found it unacceptable\fnote{\fbackref{3:6} Lit. \fbib{contemptible}} to kill\fnote{\fbackref{3:6} Lit. \fbib{sending a hand against}} only Mordecai. So Haman sought to destroy all of Mordecai's people, the Jewish people, who were in all the kingdom of Ahasuerus.
\passage{Haman's Plot against the Jewish People}

\v{7}In the twelfth year of the reign of\fnote{\fbackref{3:7} The Heb. lacks \fbib{the reign of}} King Ahasuerus, in the first month (the month Nisan), the pur (that is, the lot) was cast in Haman's presence to determine the best day and month to carry out his plot.\fnote{\fbackref{3:7} Lit. \fbib{before Haman from day to day and month to month}} The lot indicated the twelfth month, the month Adar.\fnote{\fbackref{3:7} Lit. \fbib{day and month, the twelfth, the month Adar}} \v{8}Then Haman told King Ahasuerus, ``There is a certain people scattered and divided among the people throughout the provinces of your kingdom. Their laws are different than all the other people, they don't obey the king's laws, and it's not in the king's best interest\fnote{\fbackref{3:8} Lit. \fbib{there is no advantage for the king}} to leave them alone. \v{9}If the king approves, let it be decreed\fnote{\fbackref{3:9} Lit. \fbib{written}} that they're to be destroyed, and I'll measure out 10,000 silver talents\fnote{\fbackref{3:9} I.e. about 750,000 pounds; a talent weighed about 75 pounds} and bring it to the king's treasury for those who will do the work.''

\v{10}The king removed his signet ring from his hand and gave it to Hammedatha the Agagite's son Haman, the enemy of the Jewish people. \v{11}The king told Haman, ``The silver is given to you, along with the people, to do with them as you see fit.''

\v{12}The king's scribes were summoned on the thirteenth day of the first month, and all that Haman commanded was written to the regional authorities\fnote{\fbackref{3:12} Lit. \fbib{satraps}; Persian government officials similar in authority to a governor} of the king, to the governors who were over each province, and to the officials of each people. This order\fnote{\fbackref{3:12} The Heb. lacks \fbib{order}} was translated in the name of King Ahasuerus into the language of each province\fnote{\fbackref{3:12} Lit. \fbib{Ahasuerus to the people, the script}} and bore the seal of the king's signet ring. \v{13}Letters were sent by couriers to all of the king's provinces to annihilate, to kill, and to destroy all the Jewish people, both young and old, women and children, and to confiscate their goods\fnote{\fbackref{3:13} Lit. \fbib{spoil}} on a single day---the thirteenth day of the twelfth month of Adar. \v{14}A copy of the letter was to be issued as an edict in every province and published to all the people, telling them\fnote{\fbackref{3:14} The Heb. lacks \fbib{telling them}} to be ready for that day. \v{15}The couriers went out, urged on by the king's command, and the edict was issued in Susa the capital. The king and Haman sat down to drink, while the city of Susa was thrown into confusion.
\labelchapt{4}
\passage{Mordecai Seeks Esther's Help}

\chapt{4}
\v{1}When Mordecai learned all that had been done, he tore his garments and clothed himself in sackcloth and ashes. He went into the middle of the city and cried out with a loud and bitter cry. \v{2}He came as far as\fnote{\fbackref{4:2} Lit. \fbib{came to}} the front of the king's gate, because no one was allowed to enter the king's gate clothed in sackcloth. \v{3}In every province where the order of the king and his edict reached, among the Jewish people there was great mourning, fasting, weeping, and lamenting, and many lay down on sackcloth and ashes.

\v{4}When Esther's young women and her eunuchs came and told her, the queen was greatly distressed. She sent clothes for Mordecai to put on so he could take off the sackcloth that he had on, but he would not take them. \v{5}Then Esther summoned Hathach, one of the king's eunuchs, whom he had assigned to her, and she ordered him to go to Mordecai to find out what was happening and why it was happening. \v{6}Hathach went to Mordecai in the city square that was in front of the king's gate. \v{7}Mordecai told him everything that had happened and the exact amount of money that Haman had said he would pay into the king's treasury in order to destroy the Jewish people. \v{8}Mordecai\fnote{\fbackref{4:8} Lit. \fbib{He}} gave Hathach\fnote{\fbackref{4:8} Lit. \fbib{him}} a copy of the written decree ordering the Jews' destruction that had been issued in Susa. Mordecai\fnote{\fbackref{4:8} Lit. \fbib{He}} wanted him to show it to Esther, to explain it to her, and then to instruct her to go in to the king to seek his favor and plead with him for her people.

\v{9}Hathach went and told Esther what Mordecai had said.\fnote{\fbackref{4:9} Lit. \fbib{the words of Mordecai}} \v{10}Then Esther spoke to Hathach, instructing him\fnote{\fbackref{4:10} Lit. \fbib{and she instructed him}} to go back\fnote{\fbackref{4:10} The Heb. lacks \fbib{to go back}} to Mordecai with this message:\fnote{\fbackref{4:10} The Heb. lacks \fbib{with this message:}} \v{11}``Every servant of the king and every person in the king's provinces knows that for any man or woman who goes to the king in the inner court without being summoned there is only\fnote{\fbackref{4:11} The Heb. lacks \fbib{only}} one law---that he be put to death---unless the king holds out the golden scepter to him. Only\fnote{\fbackref{4:11} The Heb. lacks \fbib{Only}} then he will live. For these last\fnote{\fbackref{4:11} The Heb. lacks \fbib{last}} 30 days I've not been summoned to come to the king.''

\v{12}They reported Esther's message to Mordecai.

\v{13}Mordecai told them to reply to Esther, ``Don't suppose that because you are in the palace, you will escape any more than the other Jewish people.\fnote{\fbackref{4:13} Lit. \fbib{than all the Jews}} \v{14}Indeed, if you are silent at this time, relief and deliverance will come to the Jewish people from another place, but you and your father's family will perish. Who knows but that you were brought to the kingdom for a time like this?''

\v{15}Then Esther replied to Mordecai, \v{16}``Go and gather all the Jewish people who are in Susa and fast for me. Don't eat or drink for three days, night or day. Both I and my young women will also fast in the same way, and then I'll go in to the king, even though it's against the law. And if I perish, I perish.''

\v{17}Then Mordecai left and did everything that Esther had ordered him.
\labelchapt{5}
\passage{Esther Goes before the King}

\chapt{5}
\v{1}On the third day, Esther put on her royal attire and stood in the inner courtyard of the palace in front of the king's quarters.\fnote{\fbackref{5:1} Or \fbib{house}} The king was sitting on his royal throne in the throne room, opposite the entrance to the building. \v{2}When the king saw Queen Esther standing in the courtyard, she won his favor, and the king extended to Esther the gold scepter that he was holding. Esther approached and touched the top of the scepter. \v{3}The king asked her, ``What do you want, Queen Esther? What is your request? Even if it's half of the kingdom, it will be given to you.''

\v{4}Esther replied, ``If it pleases the king, let the king and Haman come today to the banquet I've prepared for him.''

\v{5}The king responded, ``Bring Haman quickly so we may do what Esther has requested.'' So the king and Haman went to the banquet that Esther had prepared.

\v{6}While they were drinking wine,\fnote{\fbackref{5:6} Lit. \fbib{at the banquet of wine}} the king asked Esther, ``What's your petition? It will be given to you. What's your request? Up to half of the kingdom, and it will be done.''

\v{7}Esther answered, ``This is my petition and my request: \v{8}If I've found favor with the king and if it pleases the king to grant my petition and to honor my request, let the king and Haman come to the banquet that I'll prepare for them tomorrow, and then I'll do what the king has said.''
\passage{Haman's Plan to Kill Mordecai}

\v{9}Haman went out that day pleased and happy, but when Haman saw Mordecai in the king's gate, and that he did not stand up and tremble in his presence, Haman was furious with Mordecai. \v{10}Haman restrained himself, went to his house, and sent for\fnote{\fbackref{5:10} Lit. \fbib{sent and brought}} his friends and his wife Zeresh. \v{11}Then Haman told them about his splendid wealth, the number\fnote{\fbackref{5:11} Lit. \fbib{multitude}} of his sons, all the ways the king had honored\fnote{\fbackref{5:11} Or \fbib{promoted}} him, and that he had promoted him above all the other officials and ministers\fnote{\fbackref{5:11} Or \fbib{servants}} of the king.

\v{12}Then Haman said, ``Even Queen Esther brought no one except me with the king to the banquet that she held. Furthermore, I (along with the king) have also been invited by her tomorrow. \v{13}But all this does not satisfy me every time I see Mordecai the Jew sitting at the king's gate.''

\v{14}Then Zeresh his wife and all his friends said, ``Have a pole made 50 cubits\fnote{\fbackref{5:14}I.e. about 75 feet; a cubit was about eighteen inches} high, and then in the morning speak to the king and have Mordecai hanged\fnote{\fbackref{5:14} Or \fbib{impaled}} on it. Then go with the king to the banquet happy.'' This advice pleased Haman, and he had the pole made.
\labelchapt{6}
\passage{Haman's Plan Begins to Unravel}

\chapt{6}
\v{1}That night the king could not sleep, so he gave instructions to bring the book of records, the chronicles, and they were read to the king. \v{2}It was found recorded there that Mordecai had reported about Bigthana and Teresh, two of the king's eunuchs who guarded the entrance to the restricted areas of the palace,\fnote{\fbackref{6:2} Lit. \fbib{guarded the threshold}} and that they had conspired to assassinate\fnote{\fbackref{6:2} Lit. \fbib{to send a hand against}} King Ahasuerus. \v{3}So the king asked, ``What honor and distinction was bestowed on Mordecai for this?''

The young men who served the king answered, ``Nothing was done for him.''

\v{4}The king said, ``Who is in the courtyard?'' Now Haman had just entered the outer courtyard of the palace to speak to the king about having Mordecai hanged on the pole he had set up.

\v{5}The king's young men told him, ``Look, Haman is standing in the courtyard.''

The king said, ``Let him come in.''

\v{6}After Haman came in, the king asked him, ``What should be done for the man whom the king desires to honor?''

Haman told himself, ``Whom would the king desire to honor more than me?'' \v{7}Haman answered the king, ``For a man whom the king desires to honor, \v{8}let them bring royal robes that the king has worn and a horse on which the king has ridden, with a royal crown placed on its head. \v{9}Then give the robes and the horse to one of the king's most noble officials. Let them put the robes on the man whom the king desires to honor, and let them put him on the horse in the main\fnote{\fbackref{6:9} The Heb. lacks \fbib{main}} square of the city. Then let them announce in front of him, `This is what is done for the man whom the king desires to honor.'\,''

\v{10}Then the king told Haman, ``Quick! Take the clothes and the horse just as you have suggested and do this for Mordecai the Jew who sits in the king's gate. And don't let anything you've suggested fall through the cracks.''\fnote{\fbackref{6:10} The Heb. lacks \fbib{through the cracks}}

\v{11}So Haman took the clothes and the horse, dressed Mordecai, and put him on the horse in the main\fnote{\fbackref{6:11} The Heb. lacks \fbib{main}} square of the city. He cried out in front of him, ``This is what is done for the man whom the king desires to honor.''

\v{12}Then Mordecai returned to the king's gate, while Haman hurried to his house, mourning and hiding his face.\fnote{\fbackref{6:12} Lit. \fbib{and covering his head}} \v{13}Haman told his wife Zeresh and all his friends everything that had happened to him. His wise friends and his wife Zeresh told him, ``If Mordecai, before whom you have begun to fall, is one of the Jewish people,\fnote{\fbackref{6:13} Lit. \fbib{of the seed of the Jews}} you won't prevail against him. Instead, you will surely fall before him.''

\v{14}While they were still talking to him, the king's eunuchs arrived, and they quickly took him to the banquet that Esther had prepared.
\labelchapt{7}
\passage{Haman is Executed}

\chapt{7}
\v{1}The king and Haman went in to have a drink with Queen Esther. \v{2}On the second day the king again told Esther as they drank wine, ``What's your petition, Queen Esther? It will be given to you. What's your request? Up to half of the kingdom, and it will be done.''

\v{3}Queen Esther answered: ``If I've found favor with you, your majesty, and if it seems good to the king, let my life be given to me as my petition and my people as my request. \v{4}Indeed, I and my people have been sold to be annihilated, killed, and destroyed. If we had just been sold as male and female slaves, I would have kept quiet, because the trouble wouldn't have been sufficient to bother the king.''\fnote{\fbackref{7:4} Or \fbib{no enemy could compensate for this damage to the king}}

\v{5}Then King Ahasuerus asked Queen Esther, ``Who is this, and where is the person who would dare\fnote{\fbackref{7:5} Lit. \fbib{whose heart has filled him}} do this?''

\v{6}Esther replied, ``An adversary and an enemy---it's this wicked Haman!'' So Haman was terrified before the king and the queen. \v{7}The king got up from the banquet in anger and went out to the palace garden, while Haman stood there begging Queen Esther to spare his life,\fnote{\fbackref{7:7} Lit. \fbib{the queen for his life}} because he realized that the king intended to harm him.\fnote{\fbackref{7:7} Lit. \fbib{to bring evil on him}}

\v{8}When the king returned to the banquet hall from the palace garden, Haman was prostrate on the couch where Esther was. The king asked, ``Will this man\fnote{\fbackref{7:8} Lit. \fbib{he}} even assault the queen with me in the house?'' The king had no sooner spoken than they covered Haman's face. \v{9}Then Harbonah, one of the eunuchs attending the king, observed, ``Look there! A pole is standing 50 cubits\fnote{\fbackref{7:9} I.e. about 75 feet; a cubit was about eighteen inches} high at Haman's house that he prepared for Mordecai, whose report benefitted\fnote{\fbackref{7:9} Lit. \fbib{who spoke good for}} the king!''

The king said, ``Hang\fnote{\fbackref{7:9} Or \fbib{Impale}} him on it.'' \v{10}So they hanged\fnote{\fbackref{7:10} Or \fbib{impaled}} Haman on the pole he had set up for Mordecai, and then the king's anger subsided.
\labelchapt{8}
\passage{The Promotion of Mordecai}

\chapt{8}
\v{1}That day King Ahasuerus gave Queen Esther the property\fnote{\fbackref{8:1} Lit. \fbib{house}} of Haman, the enemy of the Jewish people, and Mordecai came into the king's presence because Esther had told him how Mordecai\fnote{\fbackref{8:1} Lit. \fbib{he}} was related to her. \v{2}The king took off his signet ring that he had taken from Haman and gave it to Mordecai. Esther then put Mordecai in charge of Haman's property.\fnote{\fbackref{8:2} Lit. \fbib{house}}
\passage{Esther Asks that the Jewish People be Spared}

\v{3}Then Esther spoke to the king again and fell at his feet. She wept and pleaded with him for mercy to overturn the evil plan devised\fnote{\fbackref{8:3} The Heb. lacks \fbib{devised}} by Haman the Agagite and his plot against the Jewish people. \v{4}The king extended the golden scepter to Esther, and she got up and stood before the king. \v{5}She said, ``If it pleases the king, and if I've found favor with him, and if the matter is proper in the king's opinion, and if I'm pleasing to the king, let an order be issued\fnote{\fbackref{8:5} Lit. \fbib{let it be written}} revoking the letters devised by Hammedatha the Agagite's son Haman, which ordered\fnote{\fbackref{8:5} Lit. \fbib{were written}} the destruction of the Jewish people throughout the king's provinces. \v{6}Indeed, how can I bear to see this disaster happen to my people? How can I bear to see the destruction of my kinsmen?''

\v{7}King Ahasuerus told Queen Esther and Mordecai the Jew, ``Look, I've given Haman's property\fnote{\fbackref{8:7} Lit. \fbib{house}} to Esther, and they have hanged\fnote{\fbackref{8:7} Or \fbib{impaled}} him on the pole because he tried to harm\fnote{\fbackref{8:7} Lit. \fbib{sent his hand against}} the Jewish people. \v{8}Now, in the name of the king, you write what seems good to you concerning the Jewish people, and seal it with the king's signet ring, for a document written in the king's name and sealed with the king's signet ring cannot be revoked.''

\v{9}The king's scribes were summoned at that time, on the twenty-third day of the third month, which is the month Sivan, and everything that Mordecai commanded the Jewish people, the regional authorities,\fnote{\fbackref{8:9} Lit. \fbib{satraps}; Persian government officials similar in authority to a governor} the governors, and the provincial officials of the 127 provinces from India to Cush\fnote{\fbackref{8:9} I.e. the upper Nile region} was written down for each province according to its script, for each people according to their language, and for the Jewish people according to their script and language. \v{10}He wrote in the name of King Ahasuerus and sealed it with the king's signet ring. He sent the letters by couriers on horseback, riding steeds especially bred for the king.\fnote{\fbackref{8:10} Or \fbib{sired by the royal stud}; i.e. the horses were strong and fast and specially bred for the task}

\v{11}What the king granted the Jewish people in every town was the right\fnote{\fbackref{8:11} The Heb. lacks \fbib{the right}} to assemble and defend themselves,\fnote{\fbackref{8:11} Lit. \fbib{and stand; or their lives}} to annihilate, kill, and destroy every armed force of a people or a province that was hostile to them, including children and women, and to plunder their property.\fnote{\fbackref{8:11} Lit. \fbib{spoil}} \v{12}Throughout all the provinces of King Ahasuerus, the one day for the Jewish people to do this was the thirteenth day of the twelfth month, which is the month Adar. \v{13}A copy of the document was to be issued as law in each and every province and published for all people, indicating that the Jewish people were to be ready to take vengeance on their enemies on that day. \v{14}The couriers, mounted on the royal steeds, left quickly, urged on by the king's command. The edict was also issued in Susa the capital.
\passage{The Jewish People Celebrate the King's Edict}

\v{15}Mordecai left the king's presence in royal robes of blue and white, wearing a large golden crown and a purple robe made of fine linen; and the city of Susa shouted with joy. \v{16}For the Jewish people, there was light and joy, gladness and honor. \v{17}In each and every province, and in each and every city, in the places where the king's order and edict reached, there was gladness and joy among the Jewish people, along with a festival and a holiday. Many of the people of the land became\fnote{\fbackref{8:17} Or \fbib{professed to be}} Jews, because they had come to fear the Jewish people.
\labelchapt{9}
\passage{The Jewish People Defeat Their Enemies}

\chapt{9}
\v{1}On the thirteenth day of the twelfth month, which is the month Adar, when the king's order and edict was about to be carried out, on the day when the enemies of the Jewish people expected to prevail over them, things were turned around so that the Jewish people themselves prevailed over those who hated them.

\v{2}The Jewish people assembled in their towns throughout the provinces of King Ahasuerus to strike out against those who intended to harm them, and no one could oppose them because all the people had come to fear the Jews.\fnote{\fbackref{9:2} Lit. \fbib{them}} \v{3}All the provincial officials, the regional authorities,\fnote{\fbackref{9:3} Lit. \fbib{satraps}; Persian government officials similar in authority to a governor} the governors, and those doing the king's work supported the Jewish people because the fear of Mordecai had come over\fnote{\fbackref{9:3} Lit. \fbib{fallen on}} them. \v{4}Indeed, Mordecai was a powerful\fnote{\fbackref{9:4} Or \fbib{great}} official in the palace and his fame spread throughout the provinces. Indeed, the man Mordecai grew more and more powerful.\fnote{\fbackref{9:4} Or \fbib{greater and greater}}

\v{5}The Jewish people struck down all their enemies with the sword, killing and destroying them, and they did with their enemies as they pleased. \v{6}In Susa the capital the Jewish people killed and destroyed 500 people. \v{7}They killed Parshandatha, Dalphon, Aspatha, \v{8}Poratha, Adalia, Aridatha, \v{9}Parmashta, Arisai, Aridai, and Vaizatha, \v{10}the ten sons of Hammedatha's son Haman, the enemy of the Jewish people, but they did not lay their hands on the spoils.

\v{11}On that day the number of those slain in Susa the capital was reported to the king. \v{12}The king told Queen Esther, ``In Susa the capital the Jewish people have killed and destroyed 500 people, including Haman's ten sons. What have they done in the rest of the king's provinces? Now what's your petition? It will be given to you. What's your further request? It will be done.''

\v{13}Then Esther said, ``If it pleases the king, let it also be granted to the Jewish people in Susa to do tomorrow what the edict allowed them to do today,\fnote{\fbackref{9:13} Lit. \fbib{according to today's edict}} and let Haman's ten sons be hanged\fnote{\fbackref{9:13} Or \fbib{impaled}} on poles.''

\v{14}The king said, ``Let this be done.'' So an edict was issued in Susa, and Haman's ten sons were hanged\fnote{\fbackref{9:14} Or \fbib{impaled}} on poles. \v{15}The Jewish people in Susa assembled again on that day, the fourteenth of Adar, and they killed 300 people in Susa, but they did not lay their hands on the spoils.
\passage{The Festival of Purim is Celebrated}

\v{16}The rest of the Jewish people in the king's provinces assembled to defend\fnote{\fbackref{9:16} Lit. \fbib{stand; or their lives}} themselves, and they gained relief from their enemies, killing 75,000 of those who hated them. But they did not lay their hands on the spoils. \v{17}They did this on the thirteenth day of Adar and rested on the fourteenth day, making it a day of feasting and joy. \v{18}The Jewish people in Susa assembled on the thirteenth day and again on the fourteenth, and then rested on the fifteenth day and made it a day of feasting and joy. \v{19}Therefore the Jewish people in the rural areas who live in unwalled towns make the fourteenth day of the month Adar a holiday for joy and feasting, and people send presents\fnote{\fbackref{9:19} Or \fbib{portions of food}} to one another.
\passage{Official Instructions for Celebrating Purim}

\v{20}Mordecai wrote these instructions and sent letters to all the Jewish people in all the provinces of King Ahasuerus, both near and far, \v{21}establishing that they should celebrate the fourteenth and fifteenth days of the month Adar every year, \v{22}as the days on which the Jewish people enjoyed relief\fnote{\fbackref{9:22} Or \fbib{the Jews rested}} from their enemies. It was a month when things turned around for them, from sorrow to joy and from mourning to a holiday. They were to celebrate these days as days of feasting and joy, and they were to send presents\fnote{\fbackref{9:22} Or \fbib{portions of food}} to one another and gifts to the poor. \v{23}So the Jewish people made a tradition\fnote{\fbackref{9:23} Lit. \fbib{the Jews accepted}} out of what they had begun to do and of what Mordecai had written to them, \v{24}since Hammedatha's son Haman, the enemy of the Jewish people, had plotted against the Jewish people to destroy them, and he had cast the pur (that is, the lot) to determine when\fnote{\fbackref{9:24} The Heb. lacks \fbib{to determine when}} to confuse and destroy them.

\v{25}But when Esther came before the king, he ordered through a letter that the evil plot that Haman\fnote{\fbackref{9:25} Lit. \fbib{he}} had devised against the Jewish people be rescinded,\fnote{\fbackref{9:25} Lit. \fbib{be turned back on his own head}} and that he and his sons be hanged on poles. \v{26}Therefore these days were called Purim, from the word pur. Because of all that was written in this letter, because of what they experienced in this matter, and because of what happened to them, \v{27}the Jewish people established this celebration, making it a tradition\fnote{\fbackref{9:27} Lit. \fbib{people accepted it}} for themselves, for their descendants, and for all who joined with them\fnote{\fbackref{9:27} I.e. those who became Jews} that they should not fail to observe these two days each year, based on the written instructions, and at the prescribed time. \v{28}These days should be remembered and observed in every generation by each family in every province and town. These days of Purim should not be neglected by\fnote{\fbackref{9:28} Lit. \fbib{should not pass by}} the Jewish people, and that they should not be forgotten by their descendants.
\passage{Queen Esther Confirms the Instructions for Purim}

\v{29}Queen Esther, the daughter of Abihail, and Mordecai the Jew wrote with full authority confirming this second letter about Purim. \v{30}Letters containing wishes for peace and stability were sent to all the Jewish people, to the 127 provinces of Ahasuerus' kingdom, \v{31}establishing these days of Purim at the prescribed time, just as Mordecai the Jew and Queen Esther had established, and just as the Jewish people\fnote{\fbackref{9:31} Lit. \fbib{they}} had established for themselves and for their descendants. The letter included instructions for their fasting\fnote{\fbackref{9:31} Lit. \fbib{descendants, instructions for their fasting}} and lamentations. \v{32}The order of Esther established these instructions for Purim, and it was officially recorded.\fnote{\fbackref{9:32} Lit. \fbib{recorded in a record}}
\labelchapt{10}
\passage{Other Things King Ahasuerus Did}

\chapt{10}
\v{1}King Ahasuerus imposed tribute\fnote{\fbackref{10:1} I.e. a kind of tax} on the land and on the islands of the sea. \v{2}Now as to all the powerful and great deeds of Ahasuerus, along with an exact statement about the high position\fnote{\fbackref{10:2} Lit. \fbib{greatness}} of Mordecai to which the king promoted him, these things are written in the Book of the Chronicles of the Kings of Media and Persia, are they not? \v{3}Indeed, Mordecai the Jew was second in authority only\fnote{\fbackref{10:3} Lit. \fbib{second to King}} to King Ahasuerus and was a powerful official\fnote{\fbackref{10:3} Or \fbib{great}} among the Jewish people. Mordecai\fnote{\fbackref{10:3} Lit. \fbib{He}} was accepted favorably by his many kinsmen, and he sought the good of his countrymen and spoke out for the welfare of all his people.\fnote{\fbackref{10:3} Lit. \fbib{his see}}

\bookheader{Job}
\labelbook{Job}

\bookpretitle{The Book of}
\booktitle{Job}

\labelchapt{1}
\passage{Job's Faithfulness}

\chapt{1}
\v{1}There once was a man in the land of Uz\fnote{\fbackref{1:1} I.e. a city east of Israel in Arabia; the name means \fbib{Wooded}} named Job. The man was blameless as well as upright. He feared God and kept away from evil. \v{2}Seven sons and three daughters had been born to him. \v{3}His livestock included 7,000 sheep, 3,000 camels, 500 teams\fnote{\fbackref{1:3} Or \fbib{pairs}} of oxen, 500 female donkeys, and many servants. Indeed, the man's stature greatly exceeded that of many people who lived in the East. \v{4}His sons used to travel to each other's houses in turn on a regular schedule and hold festivals, inviting their three sisters to celebrate\fnote{\fbackref{1:4} Lit. \fbib{to eat and drink}} with them.

\v{5}When their time of feasting had concluded, Job would rise early in the morning to send for them\fnote{\fbackref{1:5} The Heb. lacks \fbib{for them}} and consecrate them to God.\fnote{\fbackref{1:5} The Heb. lacks \fbib{to God}} He would offer a burnt offering for each one,\fnote{\fbackref{1:5} Lit. \fbib{offering according to their number}} because Job thought, ``Perhaps my children sinned by cursing God in their hearts.'' Job did this time and again.\fnote{\fbackref{1:5} Lit. \fbib{all the days}}
\passage{Satan's First Attack on Job}

\v{6}One day, divine beings\fnote{\fbackref{1:6} Lit. \fbib{day, sons of God}} presented themselves to the \divine{Lord}, and Satan\fnote{\fbackref{1:6} The Heb. name \fbib{Satan} means \fbib{The Opponent} or \fbib{The Accuser}; and so throughout the book} accompanied them. \v{7}The \divine{Lord} asked Satan, ``Where have you come from?''

In response, Satan answered the \divine{Lord}, ``From wandering all over the earth and walking back and forth throughout it.''

\v{8}Then the \divine{Lord} asked Satan, ``Have you considered\fnote{\fbackref{1:8} Lit. \fbib{you set your heart over}} my servant Job? There is no one like him on earth. The man is blameless as well as upright. He fears God and keeps away from evil.''

\v{9}But in response, Satan asked the \divine{Lord}, ``Does Job fear God for nothing? \v{10}Haven't you surrounded him with a fence on all sides, around his house, and around all that he owns? You have blessed everything he puts his hands on and you have increased his livestock in the land. \v{11}However, stretch out your hand and strike everything he owns, and he will curse you to your face.''

\v{12}Then the \divine{Lord} told Satan, ``Very well then, everything he owns is under your control,\fnote{\fbackref{1:12} Lit. \fbib{hand}} only you may not extend your hand against him.'' So Satan left the \divine{Lord}'s presence.

\v{13}Some time later, when his children\fnote{\fbackref{1:13} Lit. \fbib{his sons and daughters}} were celebrating\fnote{\fbackref{1:13} Lit. \fbib{were eating and drinking wine}} in their oldest\fnote{\fbackref{1:13} Lit. \fbib{their firstborn}} brother's house, \v{14}a messenger approached Job and said, ``The oxen were plowing and the female donkeys were grazing nearby \v{15}when the Sabeans attacked, captured the servants, and killed them with swords. I alone escaped to tell you!''

\v{16}While this messenger\fnote{\fbackref{1:16} The Heb. lacks \fbib{messenger}} was still speaking, another\fnote{\fbackref{1:16} Lit. \fbib{this} } came and announced, ``A lightning storm struck\fnote{\fbackref{1:16} Lit. \fbib{Fire of God fell from heaven}} and incinerated the flock and the servants while they were eating. I alone escaped to tell you!''

\v{17}While this messenger\fnote{\fbackref{1:17} The Heb. lacks \fbib{messenger}} was still speaking, another\fnote{\fbackref{1:17} Lit. \fbib{this} } came and announced, ``The Chaldeans formed three companies, raided the camels, captured the servants, and killed them with swords. Only I alone escaped to tell you.''

\v{18}While this messenger\fnote{\fbackref{1:18} The Heb. lacks \fbib{messenger}} was still speaking, another\fnote{\fbackref{1:18} Lit. \fbib{this}} came and announced, ``Your children were celebrating\fnote{\fbackref{1:18} Lit. \fbib{eating and drinking wine}} in their oldest\fnote{\fbackref{1:18} Lit. \fbib{their firstborn}} brother's house \v{19}when a strong wind came straight out of the wilderness and struck the four corners of the house. It collapsed on the young people, and they died. I alone escaped to tell you!''
\passage{Job Blesses God Despite the Catastrophe}

\v{20}Then Job stood up, tore his robe, shaved his head, fell to the ground, bowed very low, \v{21}and exclaimed:

\begin{poetry}
\poeml ``I left my mother's womb naked, \\
\poemll    and I will return to God naked. \\
\poeml The \divine{Lord} has given, \\
\poemll    and the \divine{Lord} has taken. \\
\poemlll       May the name of the \divine{Lord} be blessed.''
\end{poetry}

\v{22}Job neither sinned nor charged God with wrongdoing in all of this.
\labelchapt{2}
\passage{Satan's Second Attack on Job}

\chapt{2}
\v{1}Some time later, divine beings again\fnote{\fbackref{2:1} Lit. \fbib{later, the sons of God}} presented themselves to the \divine{Lord}, and Satan accompanied them to present himself to the \divine{Lord}. \v{2}The \divine{Lord} asked Satan, ``Where have you come from?''

In response, Satan told the \divine{Lord}, ``From wandering all over the earth and walking back and forth throughout it.''

\v{3}The \divine{Lord} asked Satan, ``Have you considered\fnote{\fbackref{2:3} Lit. \fbib{you set your heart over}} my servant Job? There is no one like him on earth. The man is blameless as well as upright. He fears God and keeps away from evil. He remains firm in his integrity, even though you have been urging me to overwhelm him without cause.''

\v{4}Satan answered the \divine{Lord}, ``Skin for skin! The man will give up everything that he owns in exchange for his health.\fnote{\fbackref{2:4} Lit. \fbib{his soul}} \v{5}However, stretch out your hand\fnote{\fbackref{2:5} Or \fbib{send judgment}} and strike his bones and flesh, and he'll curse you to your face, won't he?''

\v{6}Then the \divine{Lord} told Satan, ``Very well then, he is under your control.\fnote{\fbackref{2:6} Lit. \fbib{hand}} Just preserve his life.''\fnote{\fbackref{2:6} Lit. \fbib{his soul}}

\v{7}So Satan left the \divine{Lord}'s presence and struck Job with terrible boils from the sole of his feet to the top of his head. \v{8}Job\fnote{\fbackref{2:8} Lit. \fbib{He}} took a broken piece of pottery to scrape himself while sitting among the ashes.
\passage{Job Refuses to Curse God}

\v{9}Then his wife told him, ``Do you remain firm in your integrity? Curse God and die!''

\v{10}But he replied to her, ``You're talking like foolish women do. Are we to accept\fnote{\fbackref{2:10} Or \fbib{receive}} what is good from God but not tragedy?''

Throughout all of this, Job did not sin by what he said.\fnote{\fbackref{2:10} Lit. \fbib{by his lips}}
\passage{Job's Friends Visit}

\v{11}When Job's three friends heard all these tragedies that happened to him, they each traveled from their home towns\fnote{\fbackref{2:11} Lit. \fbib{from his place}} to visit him. Eliphaz came from Teman,\fnote{\fbackref{2:11} Lit. \fbib{Eliphaz the Temanite}; i.e. from Teman in Edom, and so throughout the book} Bildad came from Shuah,\fnote{\fbackref{2:11} Lit. \fbib{Bildad the Shuhite}; i.e. from Shuah, and so throughout the book} and Zophar came from Naamath.\fnote{\fbackref{2:11} Lit. \fbib{Zophar the Naamathite}; i.e. from Naamath in Arabia, and so throughout the book} They met together and went to console and comfort him. \v{12}Observing him from a distance, at first they didn't even recognize him, so they raised their voices and burst into tears. They each ripped their robes, threw ashes into the air on their heads, \v{13}and sat with Job\fnote{\fbackref{2:13} Lit. \fbib{him}} on the ground for a full week\fnote{\fbackref{2:13} Lit. \fbib{for seven days and seven nights}} without saying a word, since they could see the great extent of his anguish.
\labelchapt{3}
\passage{Job Laments the Day He was Born}

\chapt{3}
\v{1}After this, Job spoke up solemnly, cursing\fnote{\fbackref{3:1} Or \fbib{Job opened his mouth and cursed}} the day he was born.\fnote{\fbackref{3:1} The Heb. lacks \fbib{he was born}} \v{2}This is what Job said:

\begin{poetry}
\poeml \v{3}``Let the day when I was born be annihilated, \\
\poemll    along with the night when it was announced, \\
\poemlll       `It's a boy!'\fnote{\fbackref{3:3} Lit. \fbib{A man has been conceived.}} \\
\poeml \v{4}Let that day be dark; \\
\poemll    let God above not care about it; \\
\poemlll       let no light shine over it. \\
\poeml \v{5}Let darkness and deep gloom reclaim it; \\
\poemll    let clouds settle down on it; \\
\poemlll       let blackness in mid-day terrify it. \\
\poeml \v{6}Let darkness carry that night away; \\
\poemll    let it not take its place joyfully among the days of the year; \\
\poemlll       let it not be entered into the calendar.\fnote{\fbackref{3:6} Lit. \fbib{entered among the numbering of months}} \\
\poeml \v{7}``Yes, let that night be barren; \\
\poemll    let it not appear with its joyful shout. \\
\poeml \v{8}Let whoever curses days curse it--- \\
\poemll    those who are ready to awaken monsters.\fnote{\fbackref{3:8} Lit. \fbib{Leviathan}; i.e. an ancient sea creature} \\
\poeml \v{9}Let the stars of its evening twilight be dark; \\
\poemll    let it hope for light but let there be none; \\
\poemlll       let it not see the breaking rays\fnote{\fbackref{3:9} Lit. \fbib{the eyelashes}} of the dawn. \\
\poeml \v{10}``Because that night\fnote{\fbackref{3:10} Lit. \fbib{It}} refused to shut the doors of my mother's\fnote{\fbackref{3:10} The Heb. lacks \fbib{mother's}} womb; \\
\poemll    it failed to keep me from seeing this trouble. \\
\poeml \v{11}Why didn't I die while I was still in the womb, \\
\poemll    or die while I was being born? \\
\poeml \v{12}Why was there a lap\fnote{\fbackref{3:12} Lit. \fbib{Why were there knees}} to hold me, \\
\poemll    and why were there breasts to nurse me? \\
\poeml \v{13}``If I had died,\fnote{\fbackref{3:13} Lit. \fbib{For}} I would be lying down by now, \\
\poemll    undisturbed, asleep, and at rest, \\
\poeml \v{14}along with kings and counselors of the earth, \\
\poemll    who used to build for themselves what are now only\fnote{\fbackref{3:14} The Heb. lacks \fbib{only}} ruins, \\
\poeml \v{15}or princes who amassed\fnote{\fbackref{3:15} The Heb. lacks \fbib{who amassed}} gold for themselves, \\
\poemll    and who kept filling their houses with silver. \\
\poeml \v{16}``Or why was I not buried\fnote{\fbackref{3:16} Lit. \fbib{hidden}} like a stillborn child,\fnote{\fbackref{3:16} Or \fbib{miscarriage}} \\
\poemll    like babies\fnote{\fbackref{3:16} Lit. \fbib{children}} who never saw the light? \\
\poeml \v{17}In that place, the wicked stop causing trouble, \\
\poemll    and there, those whose strength is exhausted are at rest. \\
\poeml \v{18}In that place, those who once were prisoners will be at ease together; \\
\poemll    they won't hear the voice of oppressors. \\
\poeml \v{19}The unimportant and the important are both there, \\
\poemll    and the servant is free from his master. \\
\poeml \v{20}``Why does God\fnote{\fbackref{3:20} Lit. \fbib{he}} give light to the sufferer \\
\poemll    or life to the bitter person: \\
\poeml \v{21}To those who are longing for death--- \\
\poemll    even though it does not come? \\
\poeml To those who search for it \\
\poemll    more than for hidden treasure? \\
\poeml \v{22}To those who are happy beyond measure\fnote{\fbackref{3:22} Lit. \fbib{happy with great rejoicing}} \\
\poemll    when they reach their graves? \\
\poeml \v{23}To the formerly successful\fnote{\fbackref{3:23} Lit. \fbib{the valiant}} man who lost his way in life, \\
\poemll    and God fenced him in? \\
\poeml \v{24}``As far as I'm concerned, my food comes to me in the form of sighs, \\
\poemll    and my cries of anguish pour out like water. \\
\poeml \v{25}For the dreaded thing that I feared has happened to me, \\
\poemll    what caused me to worry has engulfed\fnote{\fbackref{3:25} Lit. \fbib{come}} me. \\
\poeml \v{26}I will not be at ease; \\
\poemll    I will not be quiet; \\
\poeml I will not rest; \\
\poemll    because trouble has arrived.''
\end{poetry}
\labelchapt{4}
\passage{Eliphaz: the Innocent Don't Suffer}

\chapt{4}
\v{1}In reply, Eliphaz from Teman answered:

\begin{poetry}
\poeml \v{2}``Will you get offended if somebody tries to talk to you? \\
\poemll    Who can keep from speaking at a time like this?\fnote{\fbackref{4:3} The Heb. lacks \fbib{at a time like this}} \\
\poeml \v{3}Look! You've admonished many people,\fnote{\fbackref{4:3} The Heb. lacks \fbib{people}} \\
\poemll    and you've strengthened feeble hands. \\
\poeml \v{4}A word from you has supported those who have stumbled, \\
\poemll    and has strengthened faltering knees. \\
\poeml \v{5}``But now it's your turn, \\
\poemll    and you're the one who is worn out!\fnote{\fbackref{4:5} Or \fbib{impatient}} \\
\poeml Now it's striking you, \\
\poemll    and you're dismayed! \\
\poeml \v{6}``Your fear of God has been your confidence, hasn't it? \\
\poemll    The integrity of your life has been your hope, hasn't it? \\
\poeml \v{7}Now please think: \\
\poemll    Who has ever perished when they're innocent? \\
\poemlll       Where have the upright been destroyed? \\
\poeml \v{8}It's been my experience that those who plow the soil of\fnote{\fbackref{4:8} The Heb. lacks \fbib{the soil of}} iniquity \\
\poemll    and those who sow the seed of\fnote{\fbackref{4:8} The Heb. lacks \fbib{the seed of}} trouble will reap their harvest!\fnote{\fbackref{4:8} The Heb. lacks \fbib{their harvest}} \\
\poeml \v{9}They perish by the breath of God; \\
\poemll    they are consumed by the storm that is\fnote{\fbackref{4:9} Or \fbib{the breath of}} his anger.\fnote{\fbackref{4:9} Or \fbib{anger}} \\
\poeml \v{10}``The lioness may roar, \\
\poemll    and the lion cub may growl; \\
\poemlll       but even the ivory teeth of the full grown lion are broken. \\
\poeml \v{11}Full grown lions die when they cannot find prey; \\
\poemll    that's when the lion cubs are scattered. \\
\poeml \v{12}``A message was confided\fnote{\fbackref{4:12} Or \fbib{was delivered in secret}} to me; \\
\poemll    my ear caught a whisper of it. \\
\poeml \v{13}Disquieting thoughts from dreams at night; \\
\poemll    when deep sleep falls on everyone.\fnote{\fbackref{4:13} Lit. \fbib{men}} \\
\poeml \v{14}A fear fell upon me, along with trembling \\
\poemll    that caused all my bones to shake in terror.\fnote{\fbackref{4:14} Or \fbib{dread}} \\
\poeml \v{15}A spirit glided past me \\
\poemll    and made the hair on my skin\fnote{\fbackref{4:15} Lit. \fbib{flesh}} to bristle. \\
\poeml \v{16}It remained standing, \\
\poemll    but I couldn't recognize its appearance. \\
\poeml A form appeared before my eyes; \\
\poemll    At first there was\fnote{\fbackref{4:16} The Heb. lacks \fbib{At first there was}} silence, and then this voice: \\
\poeml \v{17}`Can a mortal person\fnote{\fbackref{4:17} Lit. \fbib{a man}} be more righteous than God? \\
\poemll    Or can the purity of the valiant exceed that of his maker?'\fnote{\fbackref{4:17} The quotation possibly continues through v 21.} \\
\poeml \v{18}``Indeed, since he doesn't trust his servants,\fnote{\fbackref{4:18} Cf. Job 15:15} \\
\poemll    since he charges\fnote{\fbackref{4:18} Or \fbib{lay upon}} his angels with error, \\
\poeml \v{19}how much less confidence\fnote{\fbackref{4:19} The Heb. lacks \fbib{confidence}} does he have \\
\poemll    in those who dwell in houses of clay; \\
\poeml who were formed from a foundation in dust \\
\poemll    and can perish\fnote{\fbackref{4:19} Lit. \fbib{can crush them}} like a moth? \\
\poeml \v{20}They are defeated between morning and evening; \\
\poemll    they perish forever---and no one notices! \\
\poeml \v{21}Their wealth\fnote{\fbackref{4:21} Or \fbib{remnants}, \fbib{left over}} perishes with them, doesn't it? \\
\poemll    They die, and do so without having wisdom, don't they?''
\end{poetry}
\labelchapt{5}
\passage{Eliphaz: God Blesses those who Seek Him}

\chapt{5}
\v{1}``Cry out, won't you!

\begin{poetry}
\poemll    Is there anyone who will answer you? \\
\poemlll       To whom will you turn among the holy ones? \\
\poeml \v{2}For wrath will slay a fool; \\
\poemll    zealous anger will kill the na\"{i}ve. \\
\poeml \v{3}``I myself saw a fool becoming established, \\
\poemll    but I suddenly cursed where he lived.\fnote{\fbackref{5:3} Or \fbib{dwelling place}} \\
\poeml \v{4}His children are far from deliverance; \\
\poemll    they'll be maltreated before they leave home,\fnote{\fbackref{5:4} Lit. \fbib{be crushed in the gate}} \\
\poemlll       with no one to rescue them. \\
\poeml \v{5}Then the hungry will devour his harvest, \\
\poemll    snatching it even from the midst of thorns, \\
\poemlll       while the thirsty covet their wealth. \\
\poeml \v{6}For wickedness doesn't crop up from dust, \\
\poemll    nor does trouble sprout out of the ground; \\
\poeml \v{7}But mankind is born headed for trouble, \\
\poemll    just as sparks soar skyward.''
\passage{God Can be Trusted in Adversity}
\poeml \v{8}``Now as for me, I would seek God if I were you;\fnote{\fbackref{5:8} The Heb. lacks \fbib{if I were you}} \\
\poemll    I would commit my case to God. \\
\poeml \v{9}He is always doing great things that cannot be explained, \\
\poemll    countless awesome deeds. \\
\poeml \v{10}He sends rain on the surface of the earth, \\
\poemll    and waters the surface of the open country. \\
\poeml \v{11}He sets the lowly on high, \\
\poemll    and lifts those who mourn to safety.\fnote{\fbackref{5:11} Or \fbib{deliverance}} \\
\poeml \v{12}He frustrates the plans of the crafty; \\
\poemll    so that what they work for never succeeds. \\
\poeml \v{13}He captures the wise in their own craftiness, \\
\poemll    bringing a quick end to their cunning advice. \\
\poeml \v{14}They meet with darkness in broad daylight; \\
\poemll    at noonday they grope around as if it were night. \\
\poeml \v{15}So he delivers from the sword of their mouth--- \\
\poemll    the poor from the power\fnote{\fbackref{5:15} Lit. \fbib{hand}} of the mighty. \\
\poeml \v{16}Therefore there is hope for those who are poor, \\
\poemll    and iniquity shuts its mouth. \\
\poeml \v{17}``Indeed, how blessed is the person whom God reproves! \\
\poemll    So never disrespect the discipline of the Almighty, \\
\poeml \v{18}because though he wounds, but then applies bandages; \\
\poemll    though he strikes, his hands still heal. \\
\poeml \v{19}``He will deliver you through six calamities; \\
\poemll    and calamity won't touch you throughout the seventh. \\
\poeml \v{20}He will deliver you from death by famine; \\
\poemll    in war from the power\fnote{\fbackref{5:20} Lit. \fbib{mouth}} of the sword. \\
\poeml \v{21}You'll be protected from the accusing\fnote{\fbackref{5:21} Lit. \fbib{lash of the}} tongue; \\
\poemll    you need not fear destruction when it heads your way. \\
\poeml \v{22}You'll laugh at destruction and famine; \\
\poemll    and you need not fear the beasts of the earth. \\
\poeml \v{23}For you'll have a pact\fnote{\fbackref{5:23} Or \fbib{be in league}} with the stones in the field; \\
\poemll    and the beasts of the field will be at peace with you. \\
\poeml \v{24}You'll know that your home\fnote{\fbackref{5:24} Lit. \fbib{tent}} is secure; \\
\poemll    when you search your possessions, and nothing will be missing. \\
\poeml \v{25}You'll know that you'll have many children; \\
\poemll    and that your offspring will be like the grass of the earth. \\
\poeml \v{26}You'll go to your grave at a ripe old age; \\
\poemll    like a stack of grain that's harvested at just the right time. \\
\poeml \v{27}``Look! We have thought all this through, \\
\poemll    and what we've said is true;\fnote{\fbackref{5:27} Lit. \fbib{and thus it is so}} \\
\poemlll       So please listen and learn for your own good!''
\end{poetry}
\labelchapt{6}
\passage{Job's Suffering is Grave}

\chapt{6}
\v{1}In rebuttal, Job replied:

\begin{poetry}
\poeml \v{2}``If only my grief could be weighed; \\
\poemll    or my calamity piled together on a balance scale! \\
\poeml \v{3}It would weigh more than the sand on the seashore!\fnote{\fbackref{6:3} Lit. \fbib{sea}} \\
\poemll    Here's why I've talked so rashly: \\
\poeml \v{4}``The arrows of the Almighty have pierced me; \\
\poemll    my spirit absorbs\fnote{\fbackref{6:4} Lit. \fbib{drinks}} their poison;\fnote{\fbackref{6:4} Lit. \fbib{heat}} \\
\poemlll       God's terrors have been arranged just for me! \\
\poeml \v{5}``Will the wild donkey bray from hunger\fnote{\fbackref{6:5} The Heb. lacks \fbib{from hunger}} if fresh grass is beside him? \\
\poemll    Will the ox low from distress\fnote{\fbackref{6:5} The Heb. lacks \fbib{from distress}} if it is near its feed? \\
\poeml \v{6}Tasteless food isn't eaten without salt, is it? \\
\poemll    Is there any taste in an egg white? \\
\poeml \v{7}I cannot bring myself to touch them;\fnote{\fbackref{6:7} The Heb. lacks \fbib{them}} \\
\poemll    food like this makes me sick.''
\passage{Job Desires Death}
\poeml \v{8}``Who will grant my wish?\fnote{\fbackref{6:8} Or \fbib{Oh, that I might have my request;}} \\
\poemll    I wish God would grant what I'm hoping for: \\
\poeml \v{9}that God would just be willing\fnote{\fbackref{6:9} Lit. \fbib{pleased}} to crush me; \\
\poemll    that he would let loose\fnote{\fbackref{6:9} Lit. \fbib{loose his hand}} and eliminate me! \\
\poeml \v{10}At least I could still take comfort \\
\poemll    and rejoice in unceasing anguish, \\
\poemlll       for I didn't conceal what the Holy One has to say. \\
\poeml \v{11}``Do I have the strength to wait? \\
\poemll    And why\fnote{\fbackref{6:11} Lit. \fbib{And to what end}} should I be patient? \\
\poeml \v{12}Am I as strong as a rock? \\
\poemll    Am I some kind of iron man?\fnote{\fbackref{6:12} Lit. \fbib{Is my flesh bronze?}} \\
\poeml \v{13}There is no help within me, is there? \\
\poemll    My resources have been driven away from me, haven't they?
\passage{Job Accuses His Friends of Treachery}
\poeml \v{14}The friend shows gracious love for his friend, \\
\poemll    even if he has forsaken the fear of the Almighty. \\
\poeml \v{15}But my brothers have acted treacherously like a cascading river, \\
\poemll    like torrential rivers that overflow. \\
\poeml \v{16}Filled with waters made cold\fnote{\fbackref{6:16} Lit. \fbib{dark}} by ice, \\
\poemll    they are where the snow goes to hide. \\
\poeml \v{17}But then the snow melts, and they disappear; \\
\poemll    when warmed, they evaporate from their stream beds.\fnote{\fbackref{6:17} Lit. \fbib{their place}} \\
\poeml \v{18}Travelers divert\fnote{\fbackref{6:18} Lit. \fbib{twist}} in their route; \\
\poemll    they go into a wasteland and die. \\
\poeml \v{19}Travelers from Tema search intently; \\
\poemll    caravans from Sheba hope to find them. \\
\poeml \v{20}For all their expectations, they are doomed to disappointment; \\
\poemll    even though they have come and searched this far. \\
\poeml \v{21}``And now you're all just like them, aren't you?\fnote{\fbackref{6:21} Lit. \fbib{it}} \\
\poemll    You see my terror and are terrified. \\
\poeml \v{22}When did I ever ask you for anything, \\
\poemll    say `Offer a bribe for me from your wealth?' \\
\poeml \v{23}or say `Deliver me from my enemy's control,'\fnote{\fbackref{6:23} Lit. \fbib{hand} } \\
\poemll    or `Redeem me from the domination\fnote{\fbackref{6:23} Lit. \fbib{hand}} of ruthless people'?''
\passage{Job Requests Mercy from His Friends}
\poeml \v{24}``Instruct me, and I'll remain silent. \\
\poemll    Help me understand where I've gone astray. \\
\poeml \v{25}The truth\fnote{\fbackref{6:25} Lit. \fbib{Upright words}} can be painful, \\
\poemll    but what has your argument proven? \\
\poeml \v{26}Did you intend your words to reprove, \\
\poemll    even though the speech of a desperate person is just wind? \\
\poeml \v{27}Indeed, you would gamble to buy an orphan; \\
\poemll    and barter to buy your friend! \\
\poeml \v{28}Now be willing to face me, \\
\poemll    and I won't lie to your face. \\
\poeml \v{29}Repent! Let there be no injustice; \\
\poemll    Change your ways!\fnote{\fbackref{6:29} The Heb. lacks \fbib{your ways}} My vindication\fnote{\fbackref{6:29} Or \fbib{righteousness}} is at stake. \\
\poeml \v{30}Have I said anything that's unjust? \\
\poemll    I can discern\fnote{\fbackref{6:30} Lit. \fbib{taste}} evil, can't I?''
\end{poetry}
\labelchapt{7}
\passage{Job Acknowledges the Brevity of Life}

\chapt{7}
\v{1}``Men have harsh servitude on earth, do they not?

\begin{poetry}
\poemll    His days are like those of a hired laborer, are they not?\fnote{\fbackref{7:1} Or \fbib{hireling}} \\
\poeml \v{2}I'm like a servant who longs for the shade, \\
\poemll    like a hired laborer who is looking for his wages. \\
\poeml \v{3}Truly I've been allotted months of emptiness; \\
\poemll    nights of trouble have been appointed for me. \\
\poeml \v{4}``When I lie down I ask, \\
\poemll    `When will I wake up?' \\
\poeml But the night continues \\
\poemll    and I keep tossing and turning until dawn.\fnote{\fbackref{7:4} Or \fbib{twilight}} \\
\poeml \v{5}My skin\fnote{\fbackref{7:5} Or \fbib{flesh}} is covered with worms and clods of dirt; \\
\poemll    my skin becomes rough and then breaks out afresh. \\
\poeml \v{6}My days pass as swiftly as a hand-loom; \\
\poemll    they come to their conclusion without hope. \\
\poeml \v{7}Remember that my life is a breath; \\
\poemll    my eyes won't go back to seeing good things.\fnote{\fbackref{7:7} The Heb. lacks \fbib{things}} \\
\poeml \v{8}The eyes of the one who sees me won't see me anymore; \\
\poemll    your eyes will look\fnote{\fbackref{7:8} The Heb. lacks \fbib{will look}} for me \\
\poemlll       but I won't be around!\fnote{\fbackref{7:8} The Heb. lacks \fbib{around}} \\
\poeml \v{9}As a cloud fades away and vanishes, \\
\poemll    the one who descends to the afterlife\fnote{\fbackref{7:9} Lit. \fbib{Sheol}; i.e. the dwelling place of the dead} doesn't return.\fnote{\fbackref{7:9} Lit. \fbib{doesn't come back up}} \\
\poeml \v{10}He doesn't return again to his house, \\
\poemll    and his place won't recognize him anymore.''
\passage{Job Intends to Complain}
\poeml \v{11}``In addition, I won't keep my opinion\fnote{\fbackref{7:11} Lit. \fbib{mouth}} to myself; \\
\poemll    I'll speak from my distressed spirit; \\
\poemlll       I'll complain with my bitter soul. \\
\poeml \v{12}Am I the sea, or a sea monster, \\
\poemll    that you keep watching me? \\
\poeml \v{13}For I've said, `My bed will comfort me; \\
\poemll    my couch will ease my burdens\fnote{\fbackref{7:13} Or \fbib{carry}} while I complain.' \\
\poeml \v{14}But then you scared me with dreams; \\
\poemll    you terrified me with visions. \\
\poeml \v{15}I would rather die by strangulation \\
\poemll    than continue living.\fnote{\fbackref{7:15} Lit. \fbib{my bones}} \\
\poeml \v{16}I hate the thought of living forever! \\
\poemll    Leave me alone, because my days are pointless.''
\passage{Job Acknowledges Humankind's Insignificance}
\poeml \v{17}``What is a human being, that you make so much of him; \\
\poemll    that you set your affections on him, \\
\poeml \v{18}visit him every morning, \\
\poemll    and test him continually? \\
\poeml \v{19}Why won't you look away from me? \\
\poemll    Why don't you leave me alone so I can swallow my saliva? \\
\poeml \v{20}So what if I sin? What have I done against you, \\
\poemll    you observer of humankind? \\
\poeml Why have you made me your target? \\
\poemll    Why burden yourself with me? \\
\poeml \v{21}Why haven't you pardoned my transgression \\
\poemll    and taken away my iniquity? \\
\poeml Now I'm about to lie down in the dust. \\
\poemll    You will seek me diligently, \\
\poemlll       but I won't be around!''\fnote{\fbackref{7:21} The Heb. lacks \fbib{around}}
\end{poetry}
\labelchapt{8}
\passage{Bildad: God Rewards the Godly}

\chapt{8}
\v{1}Then in response, Bildad from Shuah said:

\begin{poetry}
\poeml \v{2}``How long will you keep talking like this? \\
\poemll    How long will you keep talking like a wind storm? \\
\poeml \v{3}Does God pervert justice? \\
\poemll    Does the Almighty pervert what's right? \\
\poeml \v{4}``If your children sin against him, \\
\poemll    he'll make them a prisoner\fnote{\fbackref{8:4} Lit. \fbib{he'll place them into the hand}} of their sins. \\
\poeml \v{5}If you seek God, \\
\poemll    if you ask the Almighty for mercy, \\
\poeml \v{6}if you are clean and upright, \\
\poemll    surely then, he'll act on your behalf \\
\poemlll       and restore your rightful\fnote{\fbackref{8:6} Lit. \fbib{and bring peace to your righteous}} place. \\
\poeml \v{7}Your beginning may be small, \\
\poemll    but later years\fnote{\fbackref{8:7} Lit. \fbib{days}} will be very great. \\
\poeml \v{8}``Inquire of the previous generation, won't you please? \\
\poemll    Consider what their forefathers searched out. \\
\poeml \v{9}Because we are of yesterday and we know nothing, \\
\poemll    for our time on earth is only a shadow. \\
\poeml \v{10}Won't they instruct you, and tell you, \\
\poemll    and bring out words from the heart? \\
\poeml \v{11}``Can papyrus grow where there's no marsh? \\
\poemll    Can reeds flourish without water? \\
\poeml \v{12}While they are still green \\
\poemll    and not yet ready to be harvested, \\
\poemlll       they wither before any plant. \\
\poeml \v{13}Such are the paths of everyone who forgets God--- \\
\poemll    the hope of the godless will be destroyed: \\
\poeml \v{14}his confidence is shattered; \\
\poemll    his trust is in a spider's web. \\
\poeml \v{15}He leans against his house, \\
\poemll    but it won't stand; \\
\poeml he grabs hold of it firmly, \\
\poemll    but it doesn't last. \\
\poeml \v{16}He is a fresh sapling planted in the sunlight, \\
\poemll    spreading out its branches over its garden. \\
\poeml \v{17}Its roots weave around a pile of stones, \\
\poemll    seeking to entrench itself among the rocks. \\
\poeml \v{18}If he is uprooted\fnote{\fbackref{8:18} Lit. \fbib{is swallowed up}} from his place,
\end{poetry}

then it will deny him:

\begin{poetry}
\poemlll       `I never saw you.' \\
\poeml \v{19}``Indeed, this is the benefit of God's\fnote{\fbackref{8:19} Lit. \fbib{his}} way: \\
\poemll    from the soil other plants\fnote{\fbackref{8:19} The Heb. lacks \fbib{plants}} will sprout. \\
\poeml \v{20}Surely God won't reject those who are blameless \\
\poemll    or hold hands with those who practice evil. \\
\poeml \v{21}He will soon fill your mouth with laughter, \\
\poemll    and your lips will shout with joy. \\
\poeml \v{22}Those who hate you will be clothed with shame, \\
\poemll    and the tent of the wicked will no longer exist.
\end{poetry}
\labelchapt{9}
\passage{Job Cannot Argue with His Creator}

\chapt{9}
\v{1}This was Job's response:

\begin{poetry}
\poeml \v{2}``Indeed, I'm fully aware that this is so, \\
\poemll    but how can a person become right with God? \\
\poeml \v{3}If one were to seek to argue with him, \\
\poemll    he won't be able to answer him even once in a thousand times. \\
\poeml \v{4}He is wise in heart and strong in will--- \\
\poemll    who can be stubborn against him and succeed? \\
\poeml \v{5}``He removes mountains without their knowledge, \\
\poemll    overthrowing them in his anger. \\
\poeml \v{6}He shakes the earth from its orbit,\fnote{\fbackref{9:6} Lit. \fbib{place}} \\
\poemll    so that its foundations shudder. \\
\poeml \v{7}He commands the sun so that it doesn't shine\fnote{\fbackref{9:7} Lit. \fbib{rise}} \\
\poemll    and seals up the stars. \\
\poeml \v{8}He alone spreads out the heavens, \\
\poemll    he walks on the waves\fnote{\fbackref{9:8} Lit. \fbib{high places}} of the sea. \\
\poeml \v{9}He created Bear, Orion, the Pleiades, \\
\poemll    and the southern constellations. \\
\poeml \v{10}He does great things that cannot be explained, \\
\poemll    and awesome deeds that cannot be counted. \\
\poeml \v{11}``If he were to pass near me, I wouldn't notice; \\
\poemll    if he moves by, I wouldn't perceive him. \\
\poeml \v{12}Indeed, if he snatches someone\fnote{\fbackref{9:12} The Heb. lacks \fbib{someone}} away, \\
\poemll    who could restrain him? \\
\poemlll       Who can say to him, `What are you doing?' \\
\poeml \v{13}``God doesn't restrain his anger. \\
\poemll    Rahab's assistants are humiliated under him. \\
\poeml \v{14}So how am I to answer him, \\
\poemll    choosing what I am to say to him? \\
\poeml \v{15}Even if I'm in the right, \\
\poemll    I cannot answer him. \\
\poemlll       I can only appeal for mercy. \\
\poeml \v{16}``Were I to be summoned, and he were to answer me, \\
\poemll    I wouldn't even believe \\
\poemlll       that he was listening to what I have to say.\fnote{\fbackref{9:16} Lit. \fbib{to my voice}} \\
\poeml \v{17}For he crushes me with a storm, \\
\poemll    and keeps on wounding me for no reason. \\
\poeml \v{18}He won't let me catch my breath; \\
\poemll    instead, he fills me with bitterness. \\
\poeml \v{19}``Is this a contest of strength? He is obviously stronger! \\
\poemll    Is this a matter of justice? Who can sue him? \\
\poeml \v{20}Though I'm in the right, my own mouth will condemn me; \\
\poemll    though I'm blameless, he'll pronounce me as guilty. \\
\poeml \v{21}``I'm blameless; \\
\poemll    I don't know myself; \\
\poemlll       I despise my life. \\
\poeml \v{22}I say it's all the same--- \\
\poemll    he destroys both the blameless and the guilty.\fnote{\fbackref{9:22} Or \fbib{wicked}} \\
\poeml \v{23}If a calamity\fnote{\fbackref{9:23} Or \fbib{scourge}} causes sudden death, \\
\poemll    he'll mock at the despair of the innocent. \\
\poeml \v{24}A land is given into the hands of a wicked person;\fnote{\fbackref{9:24} Lit. \fbib{man}} \\
\poemll    he covers the faces of its judges. \\
\poemlll       If it is not God,\fnote{\fbackref{9:24} Lit. \fbib{he}} then who is it?''
\passage{Job Argues that God Won't Acquit Him}
\poeml \v{25}``My days pass faster than a runner; \\
\poemll    but they pass quickly without seeing anything good. \\
\poeml \v{26}They pass by like a ship made of reeds, \\
\poemll    like an eagle swooping down on its prey. \\
\poeml \v{27}If I were to say, `Let me forget my complaint,' \\
\poemll    change\fnote{\fbackref{9:27} Lit. \fbib{forsake}} the expression on\fnote{\fbackref{9:27} The Heb. lacks \fbib{the expression on}} my face, and look cheerful, \\
\poeml \v{28}then I still dread all of my suffering; \\
\poemll    I know you still won't acquit me. \\
\poeml \v{29}I will be condemned, \\
\poemll    so why should I wear myself out with this futility? \\
\poeml \v{30}``If I wash myself with water from snow, \\
\poemll    and cleanse my hands with lye, \\
\poeml \v{31}you'll still drop me into the Pit,\fnote{\fbackref{9:31} I.e. the place of punishment in the afterlife} \\
\poemll    and my own clothes will despise me. \\
\poeml \v{32}He's not a man like me, \\
\poemll    so that I can answer him, \\
\poemlll       or that we can enter into litigation\fnote{\fbackref{9:32} Lit. \fbib{controversy}} with one another. \\
\poeml \v{33}There is not yet a mediator between us, \\
\poemll    who would set his hand on the two of us, \\
\poeml \v{34}removing his rod from me, \\
\poemll    and not letting terror of him overwhelm me. \\
\poeml \v{35}Otherwise, I would speak without being terrified of him, \\
\poemll    because I'm not like that inside myself.''
\end{poetry}
\labelchapt{10}
\passage{Job Asks God to Acquit Him}

\begin{poetry}
\poeml \chapt{10}
\v{1}``I am disgusted with living, \\
\poemll    so I'm going to talk about my complaint freely. \\
\poemlll       I'll speak out from the bitterness of my soul. \\
\poeml \v{2}I'll say to God, \\
\poemll    `Don't condemn me! \\
\poemlll       Let me know why you are fighting me. \\
\poeml \v{3}Does it delight you to oppress \\
\poemll    or despise what you have made, \\
\poemlll       while you smile at the plans of the wicked?\fnote{\fbackref{10:3} Lit. \fbib{you cause the plans of the wicked to shine}} \\
\poeml \v{4}Do you have eyes made of flesh? \\
\poemll    Can you look at things as humans do? \\
\poeml \v{5}Can you live only as long as a human being? \\
\poemll    Or live the years\fnote{\fbackref{10:5} Lit. \fbib{days}} of a mortal man? \\
\poeml \v{6}```For you seek out my iniquity \\
\poemll    and search for my sin. \\
\poeml \v{7}Although you know that I'm not guilty, \\
\poemll    there's no one to deliver me from you!\fnote{\fbackref{10:7} Lit. \fbib{from your hand}} \\
\poeml \v{8}Your hands formed and fashioned me, \\
\poemll    but then you have destroyed me all at once on all sides. \\
\poeml \v{9}```Please remember that you've made me like clay \\
\poemll    and you'll return me to dust. \\
\poeml \v{10}Didn't you pour me out like milk \\
\poemll    and let me congeal like cheese? \\
\poeml \v{11}You covered me with skin and flesh, \\
\poemll    weaving me together with bones and sinews. \\
\poeml \v{12}You gave life and gracious love to me; \\
\poemll    your providential care has preserved my spirit. \\
\poeml \v{13}But you've hidden these things in your heart--- \\
\poemll    I know this was your purpose:\fnote{\fbackref{10:13} Or \fbib{was in you}} \\
\poeml \v{14}If I sin, you watch me \\
\poemll    and won't acquit me for my iniquity. \\
\poeml \v{15}```Woe to me if I'm guilty! \\
\poemll    If I'm innocent, I cannot lift my head, \\
\poeml because I am filled with disgrace. \\
\poemll    Look at my affliction! \\
\poeml \v{16}But if I do lift up my head, \\
\poemll    you will hunt me like a lion! \\
\poemlll       You will perform miracles in order to fight against me. \\
\poeml \v{17}```You have brought new witnesses against me, \\
\poemll    you're even more angry with me--- \\
\poemlll       you've brought fresh troops to attack me! \\
\poeml \v{18}So why did you bring me out from the womb? \\
\poemll    I wish I had died, before anyone had seen me, \\
\poeml \v{19}as if I had never existed; \\
\poemll    carried from the womb to the grave. \\
\poeml \v{20}My days are so few, aren't they? \\
\poemll    So leave me alone, then, \\
\poemlll       so I can smile a little \\
\poeml \v{21}before I go, never to return, \\
\poemll    leaving for the land of deep darkness and shadow. \\
\poeml \v{22}It's a gloomy land, like deepest darkness; \\
\poemll    where there's no order, \\
\poemlll       and where even\fnote{\fbackref{10:22} The Heb. lacks \fbib{even}} the brightness is like darkness.'\,''
\end{poetry}
\labelchapt{11}
\passage{Zophar Accuses Job}

\chapt{11}
\v{1}Zophar from Naamath had this to say:

\begin{poetry}
\poeml \v{2}``Shouldn't a multitude of words be answered, \\
\poemll    or a person who talks too much\fnote{\fbackref{11:2} Or \fbib{a talker}} be vindicated? \\
\poeml \v{3}Will your irrational babble silence people, \\
\poemll    and when you mock them, \\
\poemlll       will you escape without being shamed?\fnote{\fbackref{11:3} MT has \fbib{without being humiliated}} \\
\poeml \v{4}You've said, `My teaching is flawless; \\
\poemll    I'm clean\fnote{\fbackref{11:4} Or \fbib{pure}} in God's\fnote{\fbackref{11:4} Lit. \fbib{your}} sight.' \\
\poeml \v{5}``But what if God were to speak? \\
\poemll    What if he were\fnote{\fbackref{11:5} The Heb. lacks \fbib{What if he were}} to talk\fnote{\fbackref{11:5} Lit. \fbib{open his lips against}} with you, \\
\poeml \v{6}and disclose his wise secrets? \\
\poeml After all, there's so much more\fnote{\fbackref{11:6} Lit. \fbib{double}} to understanding. \\
\poemll    So be aware that God will exact from you \\
\poemlll       less than your sin deserves.''
\passage{God's Wisdom is Unfathomable}
\poeml \v{7}``Can you search through God's complex things? \\
\poemll    Can you uncover the limits of the Almighty? \\
\poeml \v{8}These things are higher than the heavens, \\
\poemll    so what can you do? \\
\poeml They are deeper than Sheol,\fnote{\fbackref{11:8} I.e. the place where the dead are in the afterlife} \\
\poemll    so what can you know? \\
\poeml \v{9}They are longer than the earth's circumference,\fnote{\fbackref{11:9} Lit. \fbib{measure}} \\
\poemll    and broader than the ocean. \\
\poeml \v{10}``If he bypasses, or imprisons, or convenes a court,\fnote{\fbackref{11:10} The Heb. lacks \fbib{a court}} \\
\poemll    who can stop\fnote{\fbackref{11:10} Or \fbib{repel}} him? \\
\poeml \v{11}For he knows mankind's\fnote{\fbackref{11:11} Lit. \fbib{men}} deceitfulness; \\
\poemll    when he sees iniquity, won't he himself consider it? \\
\poeml \v{12}An empty-headed person will gain understanding \\
\poemll    when a wild donkey is born a human being!''
\passage{Zophar Counsels Job to Repent}
\poeml \v{13}``Now for you, if you will prepare your heart, \\
\poemll    spread out your hands to him. \\
\poeml \v{14}If you have any iniquity, throw it far away. \\
\poemll    Don't let evil\fnote{\fbackref{11:14} Or \fbib{wrong}} live in your residence.\fnote{\fbackref{11:14} Lit. \fbib{tents}} \\
\poeml \v{15}Then your confidence will be flawless, \\
\poemll    and your security will keep you from terror. \\
\poeml \v{16}You'll forget your suffering; \\
\poemll    you'll remember it like water that has evaporated.\fnote{\fbackref{11:16} Or \fbib{has flowed past}} \\
\poeml \v{17}Your life will be brighter than noonday. \\
\poemll    Even its darkness will be like dawn. \\
\poeml \v{18}You'll be secure, because there is hope; \\
\poemll    you'll see that you're at rest and safe. \\
\poeml \v{19}When you sleep, there'll be nothing to fear; \\
\poemll    and many will court your favor.\fnote{\fbackref{11:19} Lit. \fbib{face}} \\
\poeml \v{20}But what the wicked look for will fail; \\
\poemll    their way of escape will be taken away\fnote{\fbackref{11:20} Lit. \fbib{destroyed}} from them; \\
\poemlll       their only hope is to take their final breath.''\fnote{\fbackref{11:20} Lit. \fbib{is to breathe out their soul}}
\end{poetry}
\labelchapt{12}
\passage{Job Responds to Zophar}

\chapt{12}
\v{1}In response Job replied:

\begin{poetry}
\poeml \v{2}``Truly, you are the people \\
\poemll    and wisdom will die with you! \\
\poeml \v{3}Like you, I also have understanding.\fnote{\fbackref{12:3} Lit. \fbib{my heart is like yours}} \\
\poemll    I'm not inferior to you; \\
\poemlll       who doesn't know things\fnote{\fbackref{12:3} The Heb. lacks \fbib{things}} like this?''
\passage{Job Has Become a Laughingstock}
\poeml \v{4}``I'm a laughingstock to my friends, \\
\poemll    someone\fnote{\fbackref{12:4} The Heb. lacks \fbib{someone}} who called on God. \\
\poeml But then he answered this upright and blameless man, \\
\poemll    and I have become\fnote{\fbackref{12:4} The Heb. lacks \fbib{have become}} a laughingstock. \\
\poeml \v{5}The carefree are thinking, `I have contempt for misfortune,'
\end{poetry}

Those who are about to stumble deserve it.

\begin{poetry}
\poeml \v{6}The tents of robbers are at rest, \\
\poemll    and those who provoke God are secure, \\
\poemlll       that is, those who carry their god in their pocket.\fnote{\fbackref{12:6} Lit. \fbib{hand}}
\passage{Wisdom Can Be Found in God's Creation}
\poeml \v{7}``Ask the wild animals, and they'll teach you; \\
\poemll    the birds of the sky will tell you. \\
\poeml \v{8}Or ask the green plants of the earth and they'll teach you; \\
\poemll    let the fish in the sea tell you. \\
\poeml \v{9}Who among all of these doesn't know \\
\poemll    that the \divine{Lord}'s hand made them,\fnote{\fbackref{12:9} Lit. \fbib{this}} \\
\poeml \v{10}and that the life of every living thing\fnote{\fbackref{12:10} Lit. \fbib{all the living} } rests in his control, \\
\poemll    along with the breath of every living human being? \\
\poeml \v{11}The ear scrutinizes speech \\
\poemll    just as the palate tastes food.''
\passage{God is the All-Wise and All-Powerful Creator}
\poeml \v{12}``Wisdom may be found in the company of the aged. \\
\poemll    Understanding comes\fnote{\fbackref{12:12} The Heb. lacks \fbib{comes}} with longevity. \\
\poeml \v{13}With God\fnote{\fbackref{12:13} Lit. \fbib{him}} is wisdom and strength; \\
\poemll    counsel and understanding belongs to him. \\
\poeml \v{14}When he tears down, nobody rebuilds; \\
\poemll    when\fnote{\fbackref{12:14} Lit. \fbib{man}} he incarcerates, nobody escapes. \\
\poeml \v{15}When he withholds water, rivers\fnote{\fbackref{12:15} Lit. \fbib{they}} dry up; \\
\poemll    when he lets them loose, they'll flood\fnote{\fbackref{12:15} Lit. \fbib{overthrow}} the land. \\
\poeml \v{16}``With God\fnote{\fbackref{12:16} Lit. \fbib{him}} are strength and sound wisdom; \\
\poemll    both the deceived and those who deceive are responsible to him.\fnote{\fbackref{12:16} Or \fbib{are his}} \\
\poeml \v{17}He leads counselors away naked; \\
\poemll    he turns judges into fools. \\
\poeml \v{18}He strips away the authority of kings to punish \\
\poemll    and puts them in prison clothes instead. \\
\poeml \v{19}He leads away the priests naked \\
\poemll    and overthrows the ruling class.\fnote{\fbackref{12:19} Lit. \fbib{strong ones}} \\
\poeml \v{20}He keeps reliable advisors from speaking,\fnote{\fbackref{12:20} Lit. \fbib{deprives the lips of advisors}} \\
\poemll    and removes discernment from elders. \\
\poeml \v{21}He pours contempt on nobles \\
\poemll    and embarrasses\fnote{\fbackref{12:21} Lit. \fbib{and loosens the belt of}} the mighty. \\
\poeml \v{22}He uncovers the hidden dimensions from darkness, \\
\poemll    bringing what is in deep shadow to light. \\
\poeml \v{23}He makes nations great, and then destroys them; \\
\poemll    he enlarges nations, but then sends them away to captivity.\fnote{\fbackref{12:23} The Heb. lacks \fbib{to captivity}} \\
\poeml \v{24}He withdraws understanding\fnote{\fbackref{12:24} Lit. \fbib{heart}} from national leaders of the world,\fnote{\fbackref{12:24} Or \fbib{land}} \\
\poemll    causing them to wander through uncharted\fnote{\fbackref{12:24} Or \fbib{trackless}} wilderness. \\
\poeml \v{25}They grope in the dark without light; \\
\poemll    he causes them to stagger around like a drunkard.''
\end{poetry}
\labelchapt{13}
\passage{Job Begins to Argues His Case}

\begin{poetry}
\poeml \chapt{13}
\v{1}``Look, I've seen everything; \\
\poemll    I've listened carefully and I understand. \\
\poeml \v{2}What you know, I know, too; \\
\poemll    I'm not inferior to you. \\
\poeml \v{3}But I want to talk to the Almighty; \\
\poemll    and I'm determined to argue my case\fnote{\fbackref{13:3} The Heb. lacks \fbib{my case}} before God.''
\passage{Job Accuses His Friends}
\poeml \v{4}``But as for you, you whitewash with lies; \\
\poemll    all of you are worthless quacks.\fnote{\fbackref{13:4} Lit. \fbib{physicians}} \\
\poeml \v{5}I wish you'd all just shut up. \\
\poemll    Then at least you would appear to be wise. \\
\poeml \v{6}``Now listen to my dispute! \\
\poemll    Pay attention to my arguments.\fnote{\fbackref{13:6} Lit. \fbib{arguments of my lips}} \\
\poeml \v{7}Why do you speak falsely on God's behalf \\
\poemll    and speak deceitfully\fnote{\fbackref{13:7} Or \fbib{treachery}} about him? \\
\poeml \v{8}Will you show partiality to him?\fnote{\fbackref{13:8} Lit. \fbib{lift up his face}} \\
\poemll    Will you plead God's case? \\
\poeml \v{9}Will things go well for you under his cross-examination? \\
\poemll    Can you lie to him, as you would to a human being?\fnote{\fbackref{13:9} Lit. \fbib{mankind}} \\
\poeml \v{10}``He will be sure to rebuke you, \\
\poemll    if you show partiality\fnote{\fbackref{13:10} Lit. \fbib{you lift up the face}} in secret. \\
\poeml \v{11}His splendor will petrify you with terror, \\
\poemll    paralyzing you with fear, won't it? \\
\poeml \v{12}Your maxims are just worthless proverbs; \\
\poemll    your defensive arguments are made of clay.''
\passage{Job Resolves to Present His Case}
\poeml \v{13}``Don't talk to me! It's my turn to speak; \\
\poemll    then whatever happens, happens. \\
\poeml \v{14}Why shouldn't I bite my flesh \\
\poemll    or take my life in my hands? \\
\poeml \v{15}Even though he kills me, \\
\poemll    I'll continue to hope in him. \\
\poemlll       At least I'll be able to argue my case\fnote{\fbackref{13:15} Or \fbib{way}} to his face! \\
\poeml \v{16}I have this as my salvation: \\
\poemll    the godless person won't be appearing before him. \\
\poeml \v{17}Pay attention\fnote{\fbackref{13:17} Lit. \fbib{listen}, \fbib{to listen}} to what I have to say \\
\poemll    and listen to my testimony with your own ears.''
\passage{Job Presents His Conditions}
\poeml \v{18}``Look, now! I've prepared my case for court.\fnote{\fbackref{13:18} Or \fbib{judgment}} \\
\poemll    I know that I'm going to win.\fnote{\fbackref{13:18} Lit. \fbib{I'm in the right}} \\
\poeml \v{19}Who can oppose me? \\
\poemll    If they do, I'll be silent and die. \\
\poeml \v{20}Grant me only two things as you deal with me; \\
\poemll    then I won't hide from you.\fnote{\fbackref{13:20} Lit. \fbib{from your face}} \\
\poeml \v{21}Withdraw your hand far from me \\
\poemll    and keep me from being petrified with terror. \\
\poeml \v{22}Then call and I'll answer, \\
\poemll    or let me speak and then you reply to me!''
\passage{Job Presents Himself for Cross-Examination}
\poeml \v{23}``How many of my iniquities and sins have you counted? \\
\poemll    Show me my transgression and sin. \\
\poeml \v{24}Why do you hide your face \\
\poemll    and regard me as your enemy? \\
\poeml \v{25}Are you a god who would make a leaf tremble \\
\poemll    or who would prosecute a dry straw? \\
\poeml \v{26}You've accused me of bitter things; \\
\poemll    you've caused me to reap\fnote{\fbackref{13:26} Lit. \fbib{inherit}} the sins of my youth. \\
\poeml \v{27}You've locked my feet in stocks; \\
\poemll    you watch all my steps; \\
\poemlll       You've limited where I can walk.\fnote{\fbackref{13:27} Lit. \fbib{limited the soles of my feet}} \\
\poeml \v{28}So I'm a man who wears out like something rotten, \\
\poemll    like a garment that has become moth-eaten.''
\end{poetry}
\labelchapt{14}
\passage{Human Beings Live and Die}

\chapt{14}
\v{1}Human beings born by women

\begin{poetry}
\poemll    are short-lived\fnote{\fbackref{14:1} Lit. \fbib{is of short days}} and full of trouble. \\
\poeml \v{2}He springs up\fnote{\fbackref{14:2} Lit. \fbib{goes out}} like a flower and then withers.\fnote{\fbackref{14:2} Lit. \fbib{is cut off}} \\
\poemll    Like a shadow, he disappears\fnote{\fbackref{14:2} Lit. \fbib{flees}} and doesn't last. \\
\poeml \v{3}Indeed, have you opened your eyes on one like this--- \\
\poemll    to bring me into a legal fight with you? \\
\poeml \v{4}Who can produce a clean thing from an unclean thing? \\
\poemll    No one! \\
\poeml \v{5}Since his days have been determined, \\
\poemll    the number of his months is known to you. \\
\poeml You've set his limit \\
\poemll    and he cannot exceed it. \\
\poeml \v{6}Look away from him and leave him alone, \\
\poemll    so he can enjoy his time, like a hired worker.''
\passage{Death is Certain}
\poeml \v{7}``There is hope for the tree, if it is cut down, \\
\poemll    that it will sprout again, \\
\poemlll       and that its shoots won't stop growing. \\
\poeml \v{8}Even if its roots have grown ancient in the earth, \\
\poemll    and its stump begins to rot\fnote{\fbackref{14:8} Lit. \fbib{die}} in the ground, \\
\poeml \v{9}the presence\fnote{\fbackref{14:9} Lit. \fbib{scent}} of water will make it to bud \\
\poemll    so that it sprouts new branches like a young plant. \\
\poeml \v{10}``But when a person\fnote{\fbackref{14:10} Lit. \fbib{man}} dies and wastes away, \\
\poemll    when a person\fnote{\fbackref{14:10} Lit. \fbib{the valiant man}} breathes his last, where will he be? \\
\poeml \v{11}As water disappears from the sea, \\
\poemll    or water evaporates from a river, \\
\poeml \v{12}so also a person\fnote{\fbackref{14:12} Lit. \fbib{man}} lies down and does not get up; \\
\poemll    they won't awaken until the heavens are no more, \\
\poemlll       nor will they arise from their sleep.''
\passage{There is Life after Death}
\poeml \v{13}``Won't you keep me safe in the afterlife?\fnote{\fbackref{14:13} Lit. \fbib{in Sheol}; i.e. the realm of the dead} \\
\poemll    Conceal me until your anger subsides. \\
\poeml Set an appointment for me, \\
\poemll    then remember me. \\
\poeml \v{14}If a human being\fnote{\fbackref{14:14} Lit. \fbib{strong man}} dies, will he live again? \\
\poemll    I will endure the entire time of my assigned service, \\
\poemlll       until I am changed.\fnote{\fbackref{14:14} Lit. \fbib{until my change comes}; i.e. change in bodily state at the resurrection; cf. 1 Cor 15:51} \\
\poeml \v{15}You'll call and I'll answer you; \\
\poemll    you'll long for your creatures that your hands have made.\fnote{\fbackref{14:15} Lit. \fbib{for the work of your hands}} \\
\poeml \v{16}Then you'll certainly count every step I took, \\
\poemll    but you won't keep an inventory of my sin. \\
\poeml \v{17}My transgressions would be sealed up in a bag; \\
\poemll    you would cover over my sins. \\
\poeml \v{18}``Mountains fall and crumble; \\
\poemll    rocks are dislodged from their places. \\
\poeml \v{19}Water wears away stones; \\
\poemll    floods wash away topsoil from the land--- \\
\poemlll       but you destroy the hope of human beings just like that! \\
\poeml \v{20}You overpower him once and for all, and then he departs; \\
\poemll    you change his appearance and then send him away. \\
\poeml \v{21}``If his children are honored, he doesn't know it; \\
\poemll    if they become insignificant, he never perceives it. \\
\poeml \v{22}He feels only his own pain,\fnote{\fbackref{14:22} Lit. \fbib{flesh}} \\
\poemll    and grieves only for himself.''
\end{poetry}
\labelchapt{15}
\passage{Eliphaz Speaks Again}

\chapt{15}
\v{1}Then Eliphaz from Teman responded:

\begin{poetry}
\poeml \v{2}``Should a wise person respond with knowledge based on wind? \\
\poemll    Should he fill his stomach with a wind storm from the east? \\
\poeml \v{3}Should he engage in unprofitable argument, \\
\poemll    or give a speech that benefits no one? \\
\poeml \v{4}Yet you dispense with fear of God \\
\poemll    and hinder meditations before God. \\
\poeml \v{5}Because your sin dictates your speech,\fnote{\fbackref{15:5} Lit. \fbib{mouth}} \\
\poemll    you have chosen the language\fnote{\fbackref{15:5} Lit. \fbib{tongue}} of the crafty. \\
\poeml \v{6}Your own mouth is condemning you, not I; \\
\poemll    your own lips will testify against you.''
\passage{Eliphaz Claims that Job is Guilty}
\poeml \v{7}``Were you the first person\fnote{\fbackref{15:7} Lit. \fbib{man}}to be born? \\
\poemll    Were you brought forth before the hills were made? \\
\poeml \v{8}Have you listened in on God's secret council? \\
\poemll    Have you limited wisdom only to yourself? \\
\poeml \v{9}What do you know that we don't know, \\
\poemll    or that you understand and that isn't clear to us? \\
\poeml \v{10}``We have both the gray-haired and the aged with us, \\
\poemll    and they are far older\fnote{\fbackref{15:10} Lit. \fbib{are older by many days}} than your father. \\
\poeml \v{11}Are God's encouragements inconsequential to you, \\
\poemll    even a word that has been spoken\fnote{\fbackref{15:11} The Heb. lacks \fbib{spoken}} gently to you? \\
\poeml \v{12}Why have your emotions\fnote{\fbackref{15:12} Lit. \fbib{heart}} carried you away? \\
\poemll    And why do your eyes flash \\
\poeml \v{13}that you turn your anger\fnote{\fbackref{15:13} Lit. \fbib{spirit}} against God \\
\poemll    and speak words like this? \\
\poeml \v{14}``What is mankind, that he can be blameless? \\
\poemll    Or does being born of a woman mean he'll be in the right? \\
\poeml \v{15}Look, if God\fnote{\fbackref{15:15} Lit. \fbib{he}} doesn't trust his holy ones,\fnote{\fbackref{15:15} Cf. Job 4:18} \\
\poemll    if even the heavens aren't pure as he looks at them, \\
\poeml \v{16}then how much less is one who is abhorred and corrupted, \\
\poemll    such as a man who drinks injustice like water?''
\passage{Eliphaz Describes the Plight of the Wicked}
\poeml \v{17}``I'll tell you what, listen to me! \\
\poemll    Let me relate what I've seen, \\
\poeml \v{18}which is what wise men have explained, \\
\poemll    who didn't withhold anything from their ancestors. \\
\poeml \v{19}To them alone was the land given, \\
\poemll    when no invader\fnote{\fbackref{15:19} Or \fbib{foreigner}} passed through their midst. \\
\poeml \v{20}``The wicked person writhes in pain throughout his life, \\
\poemll    a number of years has been reserved for the ruthless. \\
\poeml \v{21}Terrifying sounds ring\fnote{\fbackref{15:21} The Heb. lacks \fbib{ring}} in his ears; \\
\poemll    when times are prosperous, the Destroyer will attack\fnote{\fbackref{15:21} Or \fbib{come upon him}} him. \\
\poeml \v{22}He does not believe he will escape\fnote{\fbackref{15:22} Or \fbib{turn aside}} darkness; \\
\poemll    he is destined for the sword. \\
\poeml \v{23}He wanders around for food---where is it? \\
\poemll    He knows that a time of darkness is near.\fnote{\fbackref{15:23} Lit. \fbib{is at hand}} \\
\poeml \v{24}Distress and pressure terrify him; \\
\poemll    they overwhelm him, like a king poised for attack. \\
\poeml \v{25}``For he has raised his fist against God, \\
\poemll    defying the Almighty. \\
\poeml \v{26}He defiantly ran against him \\
\poemll    carrying his thick, reinforced shield. \\
\poeml \v{27}Though he covered his face with fat, \\
\poemll    and is grossly overweight at the waist, \\
\poeml \v{28}He will live in devastated towns, \\
\poemll    in abandoned houses \\
\poemlll       that are about to become heaps of rubble. \\
\poeml \v{29}``He won't become rich and his wealth won't last; \\
\poemll    he won't expand his holdings to cover the land. \\
\poeml \v{30}He won't escape darkness; \\
\poemll    a flame will wither his shoots; \\
\poemlll       and he'll depart by the breath of God's\fnote{\fbackref{15:30} Lit. \fbib{his}} mouth. \\
\poeml \v{31}Let him not trust in a worthless speech. \\
\poemll    He leads only himself astray, \\
\poemlll       for emptiness will be his reward. \\
\poeml \v{32}This will be accomplished before his time;\fnote{\fbackref{15:32} Lit. \fbib{day}} \\
\poemll    his branches won't grow luxuriant. \\
\poeml \v{33}``He is like a vine that drops its unripe grapes; \\
\poemll    like an olive tree that loses its blossoms. \\
\poeml \v{34}For the company of the godless is desolation, \\
\poemll    and fire consumes the tents of those who\fnote{\fbackref{15:34} The Heb. lacks \fbib{those who}} bribe. \\
\poeml \v{35}For they conceive mischief and give birth to iniquity; \\
\poemll    their womb is pregnant\fnote{\fbackref{15:35} Lit. \fbib{womb fashions}; i.e., as a womb fashions a child} with deception.''
\end{poetry}
\labelchapt{16}
\passage{Job Reasons with Eliphaz}

\chapt{16}
\v{1}In response, Job said:

\begin{poetry}
\poeml \v{2}``I've heard many things like this. \\
\poemll    What miserable comforters you all are! \\
\poeml \v{3}Will windy words like yours never end? \\
\poemll    What is upsetting you that you keep on arguing? \\
\poeml \v{4}``I could also talk like you \\
\poemll    if only you were in my place! \\
\poeml Then I would put together an argument\fnote{\fbackref{16:4} Lit. \fbib{together words}} against you. \\
\poemll    I would shake my head at you \\
\poeml \v{5}and encourage you with what I have to say;\fnote{\fbackref{16:5} Lit. \fbib{with my mouth}} \\
\poemll    my words of comfort would lessen your pain. \\
\poeml \v{6}``But if I speak, my pain isn't assuaged; \\
\poemll    if I refrain from speaking, what do I have to lose?''
\passage{Job Claims of God's Mistreatment}
\poeml \v{7}``God\fnote{\fbackref{16:7} Lit. \fbib{He}} has certainly worn me out; \\
\poemll    you devastated my entire world.\fnote{\fbackref{16:7} Lit. \fbib{community}} \\
\poeml \v{8}You've arrested me, making me testify against myself! \\
\poemll    My leanness rises up to attack me, accusing\fnote{\fbackref{16:8} Lit. \fbib{testifying}} me to my face. \\
\poeml \v{9}His anger tears me in his persistent resentment against me; \\
\poemll    he gnashes his teeth at me. \\
\poemlll       My adversary glares\fnote{\fbackref{16:9} Lit. \fbib{sharpens his eyes}} at me. \\
\poeml \v{10}People gaped at me with mouths wide open; \\
\poemll    they slap me in their scorn \\
\poemlll       and gather together against me. \\
\poeml \v{11}God has delivered me over to the ungodly, \\
\poemll    throwing me into the control of the wicked. \\
\poeml \v{12}``He tore me apart when I was at ease; \\
\poemll    grabbing me by my neck, he shook me to pieces--- \\
\poemlll       then he really made me his target. \\
\poeml \v{13}His archers surround me, \\
\poemll    slashing open my kidneys without pity; \\
\poemlll       he pours out my gall on the ground. \\
\poeml \v{14}Attack follows attack as he breaks through my defenses! \\
\poemll    He runs over me like a mighty warrior. \\
\poeml \v{15}``I've even sewn sackcloth directly to my skin; \\
\poemll    I've buried my strength\fnote{\fbackref{16:15} Lit. \fbib{horn}} in the dust. \\
\poeml \v{16}My face is red from my tears, \\
\poemll    and dark shadows encircle my eyelids, \\
\poeml \v{17}even though violence is not my intention, \\
\poemll    and my prayer is pure.''
\passage{Job Appeals to Witnesses}
\poeml \v{18}``Listen, earth! Don't cover my blood, \\
\poemll    for my outcry has no place to rest. \\
\poeml \v{19}Even now, behold! I have a witness in heaven, \\
\poemll    my Advocate is on high. \\
\poeml \v{20}My friends mock me, \\
\poemll    while my eyes overflow with tears to God, \\
\poeml \v{21}crying for him to arbitrate between this\fnote{\fbackref{16:21} The Heb. lacks \fbib{this}} man and God; \\
\poemll    as a human being does with his fellow neighbor. \\
\poeml \v{22}For when only a few years have elapsed, \\
\poemll    I'll start down a path from which I'll never return.''
\end{poetry}
\labelchapt{17}
\passage{Job Laments and Prepares for Death}

\begin{poetry}
\poeml \chapt{17}
\v{1}``My spirit is crushed, \\
\poemll    my days are over;\fnote{\fbackref{17:1} Lit. \fbib{extinguished}} \\
\poemlll       it's the grave for me! \\
\poeml \v{2}Mockers surround me; \\
\poemll    I cannot stop staring at their hostility all through the night. \\
\poeml \v{3}Offer, then, some collateral on my behalf. \\
\poemll    Is there anyone who will be my guarantor? \\
\poeml \v{4}``Because you're the one who closed their hearts to compassion;\fnote{\fbackref{17:4} Lit. \fbib{understanding}} \\
\poemll    therefore, you won't let them triumph. \\
\poeml \v{5}Now as for the one who testifies against his friends \\
\poemll    to take their property,\fnote{\fbackref{17:5} Or \fbib{inheritance}} \\
\poemlll       even the eyes of his children will fail. \\
\poeml \v{6}``He has made me a byword among people; \\
\poemll    I'm being spit on in the face. \\
\poeml \v{7}My eyes have grown weak from grief; \\
\poemll    and my whole body is as thin as a shadow. \\
\poeml \v{8}The upright are appalled over this, \\
\poemll    and the innocent person is troubled by the godless. \\
\poeml \v{9}But the righteous person will hold to his way, \\
\poemll    and those with clean hands will grow stronger and stronger.''
\passage{Job Prepares for Death}
\poeml \v{10}``Come here now, all of you, \\
\poemll    and I won't find a wise person among you. \\
\poeml \v{11}My days are passed; \\
\poemll    my plans have been shattered; \\
\poemlll       along with my heart's desires. \\
\poeml \v{12}They have transformed night into day--- \\
\poemll    `The light,' they say, `is about to become dark.' \\
\poeml \v{13}``If my hope were that my house is the afterlife\fnote{\fbackref{17:13} Lit. \fbib{Sheol}; i.e. the realm of the afterlife} itself, \\
\poemll    if I were to make my bed in darkness, \\
\poeml \v{14}if I call out to the Pit,\fnote{\fbackref{17:14} I.e. the realm of punishment in the afterlife} `You're my father!' \\
\poemll    or say to the worm,\fnote{\fbackref{17:14} I.e. an agent of punishment in the afterlife} `My mother!' or `My sister!' \\
\poeml \v{15}where would my hope be? \\
\poeml ``And speaking of my hope, who would notice it? \\
\poeml \v{16}Will it go down to the bars that lock the doors\fnote{\fbackref{17:16} The Heb. lacks \fbib{that lock the doors}} of the afterlife?\fnote{\fbackref{17:16} Lit. \fbib{Sheol}; i.e. the realm of the dead} \\
\poemlll       Will we descend together into the dust?''
\end{poetry}
\labelchapt{18}
\passage{Bildad Speaks Again}

\chapt{18}
\v{1}Bildad from Shuah replied, saying:

\begin{poetry}
\poeml \v{2}``When are you going to stop your word hunt? \\
\poemll    Think first, and then we can talk. \\
\poeml \v{3}Why do you think we're like dumb animals? \\
\poemll    Do you think we're stupid? \\
\poeml \v{4}You're tearing yourself to pieces in your anger. \\
\poemll    Will the land be abandoned because of you, \\
\poemlll       or the rock be moved from its place?''
\passage{The Wicked are Trapped}
\poeml \v{5}``Indeed, the light of the wicked is extinguished; \\
\poemll    the flame from his fire doesn't shine. \\
\poeml \v{6}Light in his tent is dark, \\
\poemll    and his lamp goes out above him. \\
\poeml \v{7}His strong steps are restricted, \\
\poemll    and his own advice trips him up. \\
\poeml \v{8}``For he has stumbled into a net with his own feet; \\
\poemll    he walked right into the network! \\
\poeml \v{9}The trap seizes him by the heel; \\
\poemll    a snare tightens its hold on him. \\
\poeml \v{10}A rope lies hidden in the dirt; \\
\poemll    a trap lies\fnote{\fbackref{18:10} The Heb. lacks \fbib{lies}} waiting for him where he is walking.''
\passage{The Wicked Perish without Descendants}
\poeml \v{11}``He is petrified by terror that surrounds him on all sides; \\
\poemll    they chase at his heels. \\
\poeml \v{12}He is starved for strength; \\
\poemll    and is ripe for a fall. \\
\poeml \v{13}Something gnaws on his skin; \\
\poemll    a deadly disease\fnote{\fbackref{18:13} Lit. \fbib{a firstborn of death}} consumes his limbs. \\
\poeml \v{14}Torn from the security of his home,\fnote{\fbackref{18:14} Lit. \fbib{tent}} \\
\poemll    he is brought before the king of terrors. \\
\poeml \v{15}``There's nothing in his tent that belongs to him; \\
\poemll    sulfur is scattered all over his dwelling place. \\
\poeml \v{16}His roots wither underneath, \\
\poemll    while his branches above are being cut off. \\
\poeml \v{17}No one remembers him anywhere in the land; \\
\poemll    no one names streets in his honor. \\
\poeml \v{18}He is driven away from light to darkness, \\
\poemll    made to wander the landscape. \\
\poeml \v{19}He has no children or descendants within his own people; \\
\poemll    and no survivors where he once lived. \\
\poeml \v{20}People\fnote{\fbackref{18:20} The Heb. lacks \fbib{people}} who live west of him are appalled at his fate;\fnote{\fbackref{18:20} Lit. \fbib{at his day}} \\
\poemll    those who live east of him are seized with terror. \\
\poeml \v{21}Indeed, the residences of the wicked are like this; \\
\poemll    and so are the homes of those who don't know God.''
\end{poetry}
\labelchapt{19}
\passage{Job Responds to Bildad}

\chapt{19}
\v{1}In response, Job said:

\begin{poetry}
\poeml \v{2}``How long do you intend to keep torturing me \\
\poemll    and trying to break me by what you're saying? \\
\poeml \v{3}Ten times you've tried to humiliate me! \\
\poemll    You're not ashamed to wrong me! \\
\poeml \v{4}Even if it's true that I've erred, \\
\poemll    my error only affects me. \\
\poeml \v{5}If you really intend to vaunt yourselves over me, \\
\poemll    and make my problems the basis of your case against me, \\
\poeml \v{6}then at least you must know that God has accused me of wrong, \\
\poemll    and trapped me with his net.''
\passage{Job Accuses God of Being Angry}
\poeml \v{7}``Although I cried out `Violence!' \\
\poemll    I received no answer; \\
\poeml I cried for help, \\
\poemll    but there was no justice. \\
\poeml \v{8}He blocked my path, \\
\poemll    so I cannot pass; \\
\poemlll       and he turned out the lights on my pathways. \\
\poeml \v{9}``He has stripped me of my honor; \\
\poemll    he has stolen the crown off my head! \\
\poeml \v{10}He is breaking me down on every side, \\
\poemll    and now it's too late for me;\fnote{\fbackref{19:10} Lit. \fbib{and I'm gone}} \\
\poemlll       he has uprooted my hopes like a tree. \\
\poeml \v{11}His anger burns against me; \\
\poemll    he regards me as his adversary. \\
\poeml \v{12}His troops march\fnote{\fbackref{19:12} Or \fbib{proceed}} in a column\fnote{\fbackref{19:12} Or \fbib{together}} against me, \\
\poemll    erecting their siege ramps against me; \\
\poemlll       they surround my tent.''
\passage{Job's Family and Friends Abandoned Him}
\poeml \v{13}``My brothers are alienated from me; \\
\poemll    my acquaintances are estranged; \\
\poeml \v{14}my relatives have failed me; \\
\poemll    and my friends\fnote{\fbackref{19:14} Lit. \fbib{and those who know me}} have abandoned me. \\
\poeml \v{15}Those who live in my house--- \\
\poemll    and my maidservants, too!--- \\
\poeml treat me like a stranger; \\
\poemll    they think I'm a foreigner. \\
\poeml \v{16}``I call to my servant, \\
\poemll    but he doesn't respond, \\
\poemlll       even though I beg to him earnestly.\fnote{\fbackref{19:16} Lit. \fbib{him with my mouth}} \\
\poeml \v{17}My wife says my breath stinks; \\
\poemll    even my children say I smell bad! \\
\poeml \v{18}Even little children hate me; \\
\poemll    when I get up, they mock me. \\
\poeml \v{19}My closest friends\fnote{\fbackref{19:19} Or \fbib{circle of familiar friends}} detest me; \\
\poemll    even the ones I love have turned against me. \\
\poeml \v{20}I'm a pile of skin and bones; \\
\poemll    I have barely escaped by the skin of my teeth.''
\passage{Job Pleads with His Friends}
\poeml \v{21}``Be gracious to me, be gracious to me, my friends, \\
\poemll    because God's hand has struck me. \\
\poeml \v{22}Why are you chasing me, as God has been doing? \\
\poemll    Aren't you satisfied that I'm sick?\fnote{\fbackref{19:22} Lit. \fbib{satisfied with my flesh}} \\
\poeml \v{23}If only my words were written down; \\
\poemll    if only they were inscribed in a book \\
\poeml \v{24}using an iron stylus with lead for ink! \\
\poemll    Then they'd be engraved in rock forever. \\
\poeml \v{25}``As for me, I know that my Vindicator\fnote{\fbackref{19:25} Or \fbib{Redeemer}} is alive; \\
\poemll    And he, the Last One,\fnote{\fbackref{19:25} Lit. \fbib{And the Last}} will take his stand on the soil.\fnote{\fbackref{19:25} Or \fbib{dust}} \\
\poeml \v{26}Even after my skin has been destroyed, \\
\poemll    clothed in my flesh I will see God, \\
\poeml \v{27}whom I will see for myself. \\
\poeml My own eyes will look at him--- \\
\poemll    there won't be anyone else for me!--- \\
\poemlll       He is the culmination of my innermost desire.''
\passage{Job Reminds His Friends of Judgment}
\poeml \v{28}``When you're thinking about asking yourselves, \\
\poemll    `How will we pursue him, \\
\poemlll       since the root of the problem is with him?'\fnote{\fbackref{19:28} Lit \fbib{me}} \\
\poeml \v{29}Make sure that you remain wary of God's sword, \\
\poemll    for God's wrath brings with it the sword of punishment, \\
\poemlll       by which you'll know there's a judgment.''
\end{poetry}
\labelchapt{20}
\passage{Zophar Speaks the Second Time}

\chapt{20}
\v{1}Then Zophar from Naamath replied:

\begin{poetry}
\poeml \v{2}``Therefore my anxious thoughts cause me to answer \\
\poemll    because I'm agitated within me. \\
\poeml \v{3}Whenever I hear an insulting rebuke, \\
\poemll    I respond from my spirit because I understand.''
\passage{Destruction Awaits the Wicked}
\poeml \v{4}``Haven't you known this from ancient times, \\
\poemll    since mankind was placed on the earth? \\
\poeml \v{5}The wicked triumph only briefly; \\
\poemll    the joy of the godless is momentary. \\
\poeml \v{6}Though he grow as tall as the sky, \\
\poemll    or though his head touches the clouds, \\
\poeml \v{7}he'll perish forever, like his own excrement; \\
\poemll    those who saw him will ask, `Where is he?' \\
\poeml \v{8}He'll vanish\fnote{\fbackref{20:8} Lit. \fbib{He'll fly away}} like a dream, and no one will find him; \\
\poemll    he will be chased away like a night vision.'' \\
\poeml \v{9}``An eye that gazes at him won't do so again; \\
\poemll    and his place won't even recognize him. \\
\poeml \v{10}His sons will make amends to the poor; \\
\poemll    their hands will return his wealth. \\
\poeml \v{11}Though his bones were full of youthful vigor; \\
\poemll    yet they will lie down with him in the dust. \\
\poeml \v{12}Though evil tastes sweet in his mouth, \\
\poemll    though he conceals it under his tongue, \\
\poeml \v{13}though he savors it and delays swallowing it \\
\poemll    so he can taste it again and again in his mouth,\fnote{\fbackref{20:13} Lit. \fbib{can hold it in the middle of his palate}} \\
\poeml \v{14}his food will turn rancid in his stomach--- \\
\poemll    it will become a cobra's poison inside him. \\
\poeml \v{15}``Though he swallows wealth, he will vomit it; \\
\poemll    God will dislodge it from his stomach. \\
\poeml \v{16}He will suck the poison of cobras; \\
\poemll    the fangs\fnote{\fbackref{20:16} Or \fbib{tongue}} of a viper will slay him. \\
\poeml \v{17}He won't look at the rivers--- \\
\poemll    the torrents of honey and curd.\fnote{\fbackref{20:17} Or \fbib{butter}} \\
\poeml \v{18}``He will restore what he has attained from his work \\
\poemll    and won't consume it; \\
\poemlll       he won't enjoy the profits from his business transactions, \\
\poeml \v{19}because he has crushed and abandoned the poor; \\
\poemll    he has seized a house that he didn't build. \\
\poeml \v{20}``Since his appetite won't quit;\fnote{\fbackref{20:20} Lit. \fbib{his belly knew no contentment}} \\
\poemll    he won't let anything escape his lust.\fnote{\fbackref{20:20} Lit. \fbib{delight}} \\
\poeml \v{21}Because nothing was left for him to devour, \\
\poemll    therefore his prosperity won't last. \\
\poeml \v{22}Even though he is satiated and self-sufficient, he suffers--- \\
\poemll    everyone in any sort of trouble will attack him. \\
\poeml \v{23}``It will come about that, \\
\poemll    when he has filled himself to the full, \\
\poeml God\fnote{\fbackref{20:23} Lit. \fbib{he}} will vent his burning anger on him; \\
\poemll    he will pour it out on him and on his body. \\
\poeml \v{24}Though he dodges an iron weapon, \\
\poemll    a bronze arrow will pierce him. \\
\poeml \v{25}It will impale him and come out through his back; \\
\poemll    the point will glisten as it protrudes through his gall bladder, \\
\poemlll       and he will be terrified. \\
\poeml \v{26}``Total darkness has been reserved for his treasures; \\
\poemll    a fire that has no need to be kindled will devour him \\
\poemlll       and consume whatever remains in his possession.\fnote{\fbackref{20:26} Lit. \fbib{tent}} \\
\poeml \v{27}Heaven will reveal his iniquity, \\
\poemll    while the earth will rise up against him. \\
\poeml \v{28}A flood will wash away his house; \\
\poemll    dragging it away when God becomes angry. \\
\poeml \v{29}This is what the wicked person inherits from God; \\
\poemll    it is the inheritance that God appoints for him.''
\end{poetry}
\labelchapt{21}
\passage{Job Reasons with Zophar}

\chapt{21}
\v{1}In response, Job said:

\begin{poetry}
\poeml \v{2}``Listen carefully to my words; \\
\poemll    let this encourage all of you. \\
\poeml \v{3}Bear with me and let me speak! \\
\poemll    Then, after I've spoken, you'll be free to mock me. \\
\poeml \v{4}After all, isn't my complaint against a human being? \\
\poemll    If so, why shouldn't I be impatient? \\
\poeml \v{5}Look at me, be appalled, \\
\poemll    and then shut up! \\
\poeml \v{6}When I think about this,\fnote{\fbackref{21:6} The Heb. lacks \fbib{of this}} I'm petrified with terror \\
\poemll    and my body shudders uncontrollably.''
\passage{The Wicked Prospers}
\poeml \v{7}``Why do the wicked live to reach old age \\
\poemll    and increase in power and wealth, too? \\
\poeml \v{8}Their children grow up while they're alive, \\
\poemll    and they live to see their grandchildren. \\
\poeml \v{9}Their houses are safe from fear, \\
\poemll    and God's chastisement\fnote{\fbackref{21:9} Lit. \fbib{rod}} never visits them. \\
\poeml \v{10}Their bull breeds without fail, \\
\poemll    and their cows calve without miscarriages. \\
\poeml \v{11}They release their children to play like sheep; \\
\poemll    their young ones\fnote{\fbackref{21:11} Or \fbib{children}} dance about, \\
\poeml \v{12}singing\fnote{\fbackref{21:12} Lit. \fbib{they take up}} with tambourines and lyres \\
\poemll    as they rejoice to the sound of flutes. \\
\poeml \v{13}They grow old\fnote{\fbackref{21:13} Lit. \fbib{wear out their days}} in prosperity, \\
\poemll    as they descend peacefully into the afterlife.\fnote{\fbackref{21:13} Lit. \fbib{Sheol}; i.e. the abode of the dead} \\
\poeml \v{14}``They say to God, `Turn away from us! \\
\poemll    We have no desire to know your ways. \\
\poeml \v{15}Who is the Almighty, that we should serve him? \\
\poemll    Where's the profit in talking to him?' \\
\poeml \v{16}Behold! Their prosperity isn't in their control! \\
\poemll    The counsel of the wicked will remain far from me.''
\passage{God will Punish the Wicked}
\poeml \v{17}``How often do the wicked have their lights put out? \\
\poemll    Does calamity ever fall on them? \\
\poemlll       Will God\fnote{\fbackref{21:17} Lit. \fbib{he}} in his anger ever apportion their destruction? \\
\poeml \v{18}May they become like a straw, \\
\poemll    blown away before the wind; \\
\poemlll       like a chaff that's swept off by a storm. \\
\poeml \v{19}God stores up their iniquity to repay their children; \\
\poemll    making them\fnote{\fbackref{21:19} Lit. \fbib{him}} repay so that they may be aware. \\
\poeml \v{20}Their own eyes will see their destruction; \\
\poemll    and they'll drink the wrath of the Almighty. \\
\poeml \v{21}What will they care for their household after them, \\
\poemll    when the number of his months comes to an end?''
\passage{Death Levels Everyone}
\poeml \v{22}``Can God learn anything? \\
\poemll    After all, he will judge even the exalted ones. \\
\poeml \v{23}Such persons will die in their full vigor, \\
\poemll    completely prosperous and secure. \\
\poeml \v{24}His buckets are filled with milk, \\
\poemll    his bone marrow is healthy.\fnote{\fbackref{21:24} Lit. \fbib{moist}} \\
\poeml \v{25}Others die with a bitter soul, \\
\poemll    never having tasted the good life.\fnote{\fbackref{21:25} The Heb. lacks \fbib{life}} \\
\poeml \v{26}They both lie down in the dust; \\
\poemll    and worms\fnote{\fbackref{21:26} Lit. \fbib{and a worm}} cover them.''
\passage{Job Suspects His Friends of Treachery}
\poeml \v{27}``Look! I know your thoughts, \\
\poemll    your plans\fnote{\fbackref{21:27} Or \fbib{purposes}} are going to harm me. \\
\poeml \v{28}You ask, `Where is the noble person's house?' \\
\poemll    and `Where are the tents where the wicked live?' \\
\poeml \v{29}Haven't you asked travelers on the highway? \\
\poemll    Don't you accept their word \\
\poeml \v{30}that the wicked person is spared from times of calamity, \\
\poemll    that he is rescued on the day of wrath? \\
\poeml \v{31}Who will expose his conduct to his face? \\
\poemll    Who will repay him for what he has done \\
\poeml \v{32}when he is carried away to the cemetery \\
\poemll    and guardians are placed to watch his tomb? \\
\poeml \v{33}The runoff from the streams will seem sweet to him; \\
\poemll    everyone will follow after him; \\
\poemlll       countless crows march ahead of him. \\
\poeml \v{34}How then, can you console me so worthlessly? \\
\poemll    What is left of your answers is treachery.''
\end{poetry}
\labelchapt{22}
\passage{Eliphaz Speaks a Third Time}

\chapt{22}
\v{1}Then in response, Eliphaz from Teman said:

\begin{poetry}
\poeml \v{2}``Can a human being be useful to God, \\
\poemll    since he, who is wise, is sufficient to himself? \\
\poeml \v{3}Will it please the Almighty if you are innocent, \\
\poemll    or does he profit if your life is\fnote{\fbackref{22:3} Lit. \fbib{your ways are}} blameless? \\
\poeml \v{4}Will he acquit you just because you fear him, \\
\poemll    and render a verdict on your behalf? \\
\poeml \v{5}Your wickedness is great, isn't it? \\
\poemll    There's no limit to your iniquity, is there? \\
\poeml \v{6}``After all, you've taken collateral from your relatives for no reason; \\
\poemll    you stripped the naked of their clothing.\fnote{\fbackref{22:6} I.e. in exchange for a short-term loan} \\
\poeml \v{7}You've neglected to give water to the weary,\fnote{\fbackref{22:7} MT has \fbib{cause the weary to drink}} \\
\poemll    and you've withheld food from the hungry. \\
\poeml \v{8}The land belongs to the powerful, \\
\poemll    and the privileged\fnote{\fbackref{22:8} Lit. \fbib{who lifts the face}} thrive in it. \\
\poeml \v{9}You sent away widows empty-handed, \\
\poemll    and broke the outstretched arms of orphans. \\
\poeml \v{10}That's why disaster surrounds you, \\
\poemll    terror suddenly overwhelms you, \\
\poeml \v{11}you see nothing but darkness, \\
\poemll    and a flood of troubles\fnote{\fbackref{22:11} The Heb. lacks \fbib{of troubles}} drowns you.''
\passage{Eliphaz Acknowledges God but Issues an Imprecatory Prayer}
\poeml \v{12}``Isn't God in heaven above? \\
\poemll    Consider how far away the stars are, \\
\poemlll       and how lofty they are! \\
\poeml \v{13}You've asked, `What does God know? \\
\poemll    Can he sort through pitch black darkness?'\fnote{\fbackref{22:13} Or \fbib{deep darkness}} \\
\poeml \v{14}Thick clouds cover him so he can't see \\
\poemll    as he walks back and forth at heaven's horizon. \\
\poeml \v{15}``Will you keep walking on the traditional path \\
\poemll    that sinners\fnote{\fbackref{22:15} MT has \fbib{men of iniquity}} have tread, \\
\poeml \v{16}who were snatched away before their time; \\
\poemll    when their foundation was swept away by a river? \\
\poeml \v{17}They told God, `Get away from us!' \\
\poemll    and `What will the Almighty do to them?' \\
\poeml \v{18}``Though God\fnote{\fbackref{22:18} Lit. \fbib{he}} fills their houses with good things, \\
\poemll    the counsel of the wicked will remain far from me. \\
\poeml \v{19}The righteous will see this and rejoice; \\
\poemll    the innocent will insult him, saying,\fnote{\fbackref{22:19} The Heb. lacks \fbib{saying}} \\
\poeml \v{20}`Our enemies are sure to be destroyed, \\
\poemll    and fire will burn up what's left of their riches.''
\passage{Eliphaz Challenges Job to Repent}
\poeml \v{21}``Get to know God, and you'll be at peace with him, \\
\poemll    and then prosperity will come to you. \\
\poeml \v{22}Accept what he has to teach you, \\
\poemll    and treasure his words in your heart. \\
\poeml \v{23}``If you return to the Almighty you'll be restored, \\
\poemll    as you remove iniquity from your household.\fnote{\fbackref{22:23} Lit. \fbib{tent}} \\
\poeml \v{24}Bury your gold nuggets in the dust, \\
\poemll    and your source of gold\fnote{\fbackref{22:24} Lit. \fbib{Ophir}; i.e., an ancient source fine gold; cf. 1Chr 29:4} among the stones in a streambed, \\
\poeml \v{25}and then the Almighty will be your gold \\
\poemll    and your refined silver. \\
\poeml \v{26}``Then you'll take delight in the Almighty; \\
\poemll    and will turn your face toward God. \\
\poeml \v{27}You'll entreat him and he'll listen to you \\
\poemll    as you fulfill your vows. \\
\poeml \v{28}When you make a decision on something, \\
\poemll    it will be established for you, \\
\poemlll       and light will brighten\fnote{\fbackref{22:28} Or \fbib{enlighten}} your way. \\
\poeml \v{29}``For when they're humbled, you may respond;\fnote{\fbackref{22:29} Lit. \fbib{say back}} \\
\poemll    `It's their pride!' \\
\poemlll       but he delivers the humble. \\
\poeml \v{30}He'll even deliver the guilty, \\
\poemll    who will be delivered through your innocence.''\fnote{\fbackref{22:30} Lit. \fbib{through the cleanness of your hands}}
\end{poetry}
\labelchapt{23}
\passage{Job Responds to Eliphaz}

\chapt{23}
\v{1}Job's response was to say:

\begin{poetry}
\poeml \v{2}``I'm still complaining bitterly today; \\
\poemll    my hand is heavy because of groaning. \\
\poeml \v{3}If only I knew where to find him, \\
\poemll    I would visit him where he has taken his seat. \\
\poeml \v{4}I would lay out my case before him; \\
\poemll    and fill my mouth with arguments. \\
\poeml \v{5}I know how he would answer me; \\
\poemll    I understand what he'll tell me. \\
\poeml \v{6}``Would he use his great power to fight me? \\
\poemll    No, he'll pay attention to me. \\
\poeml \v{7}In that place, the upright can reason with him; \\
\poemll    and I'll be acquitted once and for all by my judge.''
\passage{Job Justifies His Innocence}
\poeml \v{8}``Look! If I go east,\fnote{\fbackref{23:8} Or \fbib{forward}} he isn't there! \\
\poemll    If I go to the west,\fnote{\fbackref{23:8} Or \fbib{back}} I don't perceive him. \\
\poeml \v{9}If he's working in the north,\fnote{\fbackref{23:9} Or \fbib{left}} I can't observe him;\fnote{\fbackref{23:9} The Heb. lacks \fbib{him}} \\
\poemll    If he turns south,\fnote{\fbackref{23:9} Or \fbib{right}} I can't see him.\fnote{\fbackref{23:9} The Heb. lacks \fbib{him}} \\
\poeml \v{10}Because he knows the road on which I travel, \\
\poemll    when he had tested me, I'll come out like gold. \\
\poeml \v{11}My feet stay where his footsteps lead; \\
\poemll    I kept on his pathway and haven't turned aside. \\
\poeml \v{12}I haven't wandered away from the commands that he has spoken;\fnote{\fbackref{23:12} Lit. \fbib{commands of his lips}} \\
\poemll    I've treasured what he has said\fnote{\fbackref{23:12} Lit. \fbib{treasured the words of his mouth}} more than my own meals.''
\passage{Job Stands Petrified Before God}
\poeml \v{13}``But he is One---who can change him? \\
\poemll    He does whatever he wants to do. \\
\poeml \v{14}He'll complete what he has planned for me; \\
\poemll    he has many things in mind for me! \\
\poeml \v{15}That's why I'm terrified at his presence! \\
\poemll    When I think about it, I'm afraid of him. \\
\poeml \v{16}God has caused me to faint;\fnote{\fbackref{23:16} Or \fbib{tender hearted}} \\
\poemll    the Almighty makes me terrified! \\
\poeml \v{17}Nevertheless, I haven't been silenced because of the darkness, \\
\poemll    even when thick darkness obscures my vision.''\fnote{\fbackref{23:17} Lit. \fbib{face}}
\end{poetry}
\labelchapt{24}
\passage{Job Describes Social Injustice}

\begin{poetry}
\poeml \chapt{24}
\v{1}Why doesn't the Almighty reserve time for judgment? \\
\poemll    and why don't those who know him perceive his days? \\
\poeml \v{2}They move boundary stones,\fnote{\fbackref{24:2} Or \fbib{borders}} \\
\poemll    steal flocks, and pasture them.\fnote{\fbackref{24:2} The Heb. lacks \fbib{them}} \\
\poeml \v{3}They drive away the orphan's donkey; \\
\poemll    they take the ox of the widow as security for a loan;\fnote{\fbackref{24:3} The Heb. lacks \fbib{for a loan}} \\
\poeml \v{4}They push the needy off the road, \\
\poemll    and force the poor of the land into hiding. \\
\poeml \v{5}``Look! Like wild donkeys in the wilderness, \\
\poemll    they work diligently as they seek wild game in the desert, \\
\poemlll       food for them and their young ones. \\
\poeml \v{6}They reap fodder in the field \\
\poemll    and glean in the vineyard of the wicked. \\
\poeml \v{7}They spend the night naked, without clothing, \\
\poemll    with no covering against the cold. \\
\poeml \v{8}They are wet from mountain rains; \\
\poemll    without shelter, they cling to a rock. \\
\poeml \v{9}``The fatherless are torn from the breast; \\
\poemll    the poor are taken away as security for a loan.\fnote{\fbackref{24:9} The Heb. lacks \fbib{for a loan}} \\
\poeml \v{10}They wander around naked, without clothes; \\
\poemll    hungry, though they carry sheaves of grain.\fnote{\fbackref{24:10} The Heb. lacks \fbib{grain}} \\
\poeml \v{11}They press oil between the olive groves owned by the wicked; \\
\poemll    they suffer from thirst, even while treading the winepress. \\
\poeml \v{12}From the city, dying men groan aloud, \\
\poemll    and the wounded cries out for help, \\
\poemlll       but God charges no one with wrong. \\
\poeml \v{13}``Then there are those who rebel against the light; \\
\poemll    they are not acquainted with its ways; \\
\poemlll       and they don't stay on its course.\fnote{\fbackref{24:13} Or \fbib{path}} \\
\poeml \v{14}The murderer rises at dawn to kill the poor and needy; \\
\poemll    at night, he is like a thief. \\
\poeml \v{15}The adulterer watches for twilight,\fnote{\fbackref{24:15} Lit. \fbib{twilight}} \\
\poemll    saying, `No eye is watching me' \\
\poemlll       while he veils his face. \\
\poeml \v{16}They break into houses in the dark; \\
\poemll    during the day they remained sealed in. \\
\poemlll       They don't know daylight. \\
\poeml \v{17}As a group, deep darkness is their morning time; \\
\poemll    fear that lives in darkness is their friend.''
\passage{Social Injustice will Be Punished}
\poeml \v{18}``They remain only a short time on the water's surface; \\
\poemll    their inheritance will be cursed in the land; \\
\poemlll       no one will work in their vineyards. \\
\poeml \v{19}As drought and heat evaporate melting snow, \\
\poemll    that's what Sheol\fnote{\fbackref{24:19} I.e. the realm of the afterlife} does with sinners. \\
\poeml \v{20}The womb will forget them. \\
\poemll    Maggots will find them to be a delicacy! \\
\poeml They won't be remembered anymore, \\
\poemll    their iniquity will be cut to pieces like firewood.\fnote{\fbackref{24:20} Lit. \fbib{like a tree}} \\
\poeml \v{21}``They prey on the barren woman, \\
\poemll    and do no favors for widows. \\
\poeml \v{22}God\fnote{\fbackref{24:22} Lit. \fbib{He}} prolongs the life of the strong by his power, \\
\poemll    but they get up in the morning\fnote{\fbackref{24:22} The Heb. lacks \fbib{in the morning}} without purpose in life. \\
\poeml \v{23}He gives them security and financial support, \\
\poemll    but he watches everything they do. \\
\poeml \v{24}They're exalted momentarily, but then they are gone; \\
\poemll    they are humbled,\fnote{\fbackref{24:24} Or \fbib{brought low}} just like all the others. \\
\poemlll       They are cut down like heads of corn. \\
\poeml \v{25}If this weren't so, who can prove that I'm a liar \\
\poemll    by showing that there's nothing to what I'm saying?''
\end{poetry}
\labelchapt{25}
\passage{Bildad Speaks a Third Time}

\chapt{25}
\v{1}Bildad from Shuah responded and said:

\begin{poetry}
\poeml \v{2}``Dominion and fear belong to God;\fnote{\fbackref{25:2} Lit. \fbib{him}} \\
\poemll    who fashions peace in his high heaven. \\
\poeml \v{3}Is there any limit to his armies? \\
\poemll    On whom does his light not shine?\fnote{\fbackref{25:3} Lit. \fbib{rise}} \\
\poeml \v{4}How can a human being\fnote{\fbackref{25:4} Lit. \fbib{man}} become right with God? \\
\poemll    How can a human being\fnote{\fbackref{25:4} Lit. \fbib{can one born of a woman}} be pure? \\
\poeml \v{5}Behold, even the moon isn't bright, \\
\poemll    and the stars aren't pure in his eyes. \\
\poeml \v{6}How much less is man, who is only a maggot, \\
\poemll    or a man's children, who are only worms!''
\end{poetry}
\labelchapt{26}
\passage{Job Reasons with Bildad}

\chapt{26}
\v{1}In reply, Job responded:

\begin{poetry}
\poeml \v{2}``What a help you are to the weak! \\
\poemll    How powerfully you deliver those without strength! \\
\poeml \v{3}What counsel you provide to the fool! \\
\poemll    What insight you provide so abundantly! \\
\poeml \v{4}Who helped you say all of this? \\
\poemll    Who inspired you?''
\passage{Job Acknowledges God's Power}
\poeml \v{5}``The ghosts of the dead\fnote{\fbackref{26:5} Lit. \fbib{Rephaim}; i.e., souls of the dead} writhe under the waters \\
\poemll    along with those who live there with them. \\
\poeml \v{6}Sheol\fnote{\fbackref{26:6} I.e. the realm of the afterlife} is naked before God\fnote{\fbackref{26:6} Lit. \fbib{him}} \\
\poemll    and Abaddon\fnote{\fbackref{26:6} I.e. the realm of punishment in the afterlife} has no clothes. \\
\poeml \v{7}He spreads out the north over empty space, \\
\poemll    suspending the earth over nothing. \\
\poeml \v{8}``He restricts the waters within clouds \\
\poemll    and the clouds don't burst open under them. \\
\poeml \v{9}He has enclosed the face of the full moon \\
\poemll    and spread his clouds over it. \\
\poeml \v{10}He has delimited a boundary\fnote{\fbackref{26:10} Lit. \fbib{statute}} over the surface of the oceans \\
\poemll    as a limit between light and darkness. \\
\poeml \v{11}The pillars of the heavens tremble \\
\poemll    and are astounded at his rebuke. \\
\poeml \v{12}By his power he disturbs the sea; \\
\poemll    and with his skill he shatters the sea monster.\fnote{\fbackref{26:12} Lit. \fbib{shattered Rahab}} \\
\poeml \v{13}he clears the skies with his wind; \\
\poemll    his hands have pierced the fleeing serpent. \\
\poeml \v{14}Indeed, these are the fringes of his ways, \\
\poemll    and how faint is the whisper we've heard of it! \\
\poemlll       But who can comprehend the thunder of his might?''
\end{poetry}
\labelchapt{27}
\passage{Job Asserts His Innocence}

\chapt{27}
\v{1}Job continued with his discussion and said:

\begin{poetry}
\poeml \v{2}``The living God has withheld justice from me; \\
\poemll    the Almighty has made my life\fnote{\fbackref{27:2} Or \fbib{soul}} bitter. \\
\poeml \v{3}As long as I can breathe; \\
\poemll    as long as God's breath is in my nostrils, \\
\poeml \v{4}I won't speak lies \\
\poemll    nor will I utter deceit. \\
\poeml \v{5}Far be it from me to admit that you're right! \\
\poemll    I intend to maintain my integrity\fnote{\fbackref{27:5} Cf. Job 2:9} even if it kills me! \\
\poeml \v{6}I'll retain my righteousness and not compromise it; \\
\poemll    my conscience won't rebuke me at any time. \\
\poeml \v{7}``May my enemy be like the wicked; \\
\poemll    my adversary like the unjust.\fnote{\fbackref{27:7} Or \fbib{unrighteous one}} \\
\poeml \v{8}For where is the hope of the godless when he is eliminated; \\
\poemll    when God takes away his life? \\
\poeml \v{9}Will God hear his cry \\
\poemll    when distress overtakes him? \\
\poeml \v{10}Will he take delight in the Almighty? \\
\poemll    Will he call on God at all times?''
\passage{On the Demise of the Wicked}
\poeml \v{11}``I'll teach you about the power\fnote{\fbackref{27:11} Lit. \fbib{hand}} of God, \\
\poemll    that which is with the Almighty I won't conceal. \\
\poeml \v{12}Look! All of you have been watching, \\
\poemll    so why have you become so completely worthless? \\
\poeml \v{13}``This is what a wicked person\fnote{\fbackref{27:13} Lit. \fbib{man}} inherits from God, \\
\poemll    and what the ruthless will receive from the Almighty: \\
\poeml \v{14}If he has many children, \\
\poemll    their destiny is to die by the sword, \\
\poemll    and his descendants won't have enough food. \\
\poeml \v{15}Those who do survive him disease will bury, \\
\poemll    and his widow won't even weep. \\
\poeml \v{16}``Though he hoards silver\fnote{\fbackref{27:16} Or \fbib{money}} like dust, \\
\poemll    and stores away garments like clay, \\
\poeml \v{17}whatever he stores up, the righteous will wear, \\
\poemll    and the innocent will inherit that silver! \\
\poeml \v{18}``He has built his house like a moth's cocoon,\fnote{\fbackref{27:18} The Heb. lacks \fbib{cocoon}} \\
\poemll    like a temporary\fnote{\fbackref{27:18} The Heb. lacks \fbib{temporary}} sunshade that a watchman makes. \\
\poeml \v{19}He will go to bed wealthy, \\
\poemll    but won't be doing that anymore! \\
\poemlll       When he opens his eyes, it will be gone! \\
\poeml \v{20}Terror will overtake him like a flood,\fnote{\fbackref{27:20} Lit. \fbib{water}} \\
\poemll    at night, a tornado will sweep him away. \\
\poeml \v{21}He'll be swept up by a storm\fnote{\fbackref{27:21} Lit. \fbib{east}} wind and carried away; \\
\poemll    he'll be whirled away from his place. \\
\poeml \v{22}It will toss him around without pity. \\
\poemll    He'll try to break free\fnote{\fbackref{27:22} Lit. \fbib{grip to flee}, \fbib{he will flee}} from its grip,\fnote{\fbackref{27:22} Lit. \fbib{hand}} \\
\poeml \v{23}but it will clap its hands over him, \\
\poemll    hissing at him as it lunges toward him.''\fnote{\fbackref{27:23} Lit. \fbib{him from its place}}
\end{poetry}
\labelchapt{28}
\passage{Priceless Wisdom is Sourced in God}

\begin{poetry}
\poeml \chapt{28}
\v{1}``Surely there are mines for silver \\
\poemll    and places where gold is refined. \\
\poeml \v{2}Iron is taken from the ground;\fnote{\fbackref{28:2} Or \fbib{dry earth}} \\
\poemll    and copper is smelted from ore. \\
\poeml \v{3}Mankind limits the darkness \\
\poemll    as they search the deepest depths \\
\poemlll       for ore\fnote{\fbackref{28:3} Lit. \fbib{for darkest stone}} in unfathomable darkness. \\
\poeml \v{4}He sinks his shaft far from human habitations, \\
\poemll    in a place\fnote{\fbackref{28:4} The Heb. lacks \fbib{in a place}} forgotten by explorers; \\
\poeml they hang on harnesses \\
\poemll    as they swing back and forth. \\
\poeml \v{5}``While the ground produces food, \\
\poemll    underneath it is torn up and burning hot,\fnote{\fbackref{28:5} Lit. \fbib{is turned up as by fire}} \\
\poeml \v{6}where stones are sapphire \\
\poemll    and gold dust can be found, \\
\poeml \v{7}a place where birds of prey never fly, \\
\poemll    and the eyes of the falcon have never seen. \\
\poeml \v{8}The proud beasts haven't walked there; \\
\poemll    lions have never passed over it. \\
\poeml \v{9}``Using a flint, he thrusts his hand, \\
\poemll    overturning mountains by the roots. \\
\poeml \v{10}He cuts a channel through the rocks, \\
\poemll    while his eyes search for anything of value. \\
\poeml \v{11}He dams up flowing rivers, \\
\poemll    bringing hidden things to light.''
\passage{Wisdom is of Greater Value than Precious Stones}
\poeml \v{12}``Where can wisdom be found? \\
\poemll    Where is understanding's home? \\
\poeml \v{13}Mankind doesn't appreciate their value; \\
\poemll    and you won't find it anywhere on earth.\fnote{\fbackref{28:13} Lit. \fbib{it in the land of the living}} \\
\poeml \v{14}The deepest ocean says, `It's not within me.' \\
\poemll    and the sea says, `You'll never find it with me.' \\
\poeml \v{15}You can't buy it with gold, \\
\poemll    and its value cannot be calculated in silver. \\
\poeml \v{16}It cannot be compared to gold from Ophir,\fnote{\fbackref{28:16} I.e. an ancient source of fine gold; cf. 1Chr 29:4} \\
\poemll    with precious onyx, or with sapphire. \\
\poeml \v{17}It cannot be compared to gold and fine glass\fnote{\fbackref{28:17} The Heb. lacks \fbib{fine glass}} crystal, \\
\poemll    nor can it be exchanged for gold-plated weaponry.\fnote{\fbackref{28:17} Or \fbib{for refined implements made of gold}} \\
\poeml \v{18}Don't even bother to mention coral and crystal--- \\
\poemll    wisdom is more valuable than a bag of rubies.\fnote{\fbackref{28:18} Or \fbib{pearls}} \\
\poeml \v{19}It can neither be compared with the topaz of Ethiopia \\
\poemll    nor valued in comparison to pure gold.''
\passage{Wisdom is from God}
\poeml \v{20}``From where, then, does wisdom originate? \\
\poemll    Where does understanding live?\fnote{\fbackref{28:20} Lit. \fbib{Where is its place?}} \\
\poeml \v{21}It has been concealed from the sight of every living creature \\
\poemll    and hidden even from the birds in the skies. \\
\poeml \v{22}Abaddon\fnote{\fbackref{28:22} I.e. the realm of eternal judgment in the afterlife} and death said, \\
\poemll    `We did hear a rumor about it.' \\
\poeml \v{23}God understands how to get there; \\
\poemll    he knows where they live.\fnote{\fbackref{28:23} Lit. \fbib{knows its place}} \\
\poeml \v{24}For he looks as far as the ends of the earth \\
\poemll    and sees everything under the sky.\fnote{\fbackref{28:24} Or \fbib{under heaven}} \\
\poeml \v{25}``He imparted weight to the wind; \\
\poemll    he regulated water by his measurement. \\
\poeml \v{26}He set in place ordinances for the rain; \\
\poemll    and determined the pathway for thunder that accompanies lightning.\fnote{\fbackref{28:26} \fbib{The sound of a thunderbolt}} \\
\poeml \v{27}Then he looked at wisdom, \\
\poemll    and fixed it in place; \\
\poeml he established it, \\
\poemll    and also examined it. \\
\poeml \v{28}He has commanded mankind: \\
\poemll    `To fear the Lord---that is wisdom; \\
\poemlll       to move away from evil---that is understanding.'\,''
\end{poetry}
\labelchapt{29}
\passage{Job Wishes for the Old Days}

\chapt{29}
\v{1}Then Job continued with his discourse:

\begin{poetry}
\poeml \v{2}``I wish I could go back to how things were a few months ago; \\
\poemll    when God used to watch over me, \\
\poeml \v{3}when his lamp used to shine over my head, \\
\poemll    so I could walk through the dark, \\
\poeml \v{4}like when I was in my prime \\
\poemll    and God trusted me with his secrets!\fnote{\fbackref{29:4} Lit. \fbib{God's counsel was over my tent}} \\
\poeml \v{5}``The Almighty was still with me back then, \\
\poemll    and my children were still around me. \\
\poeml \v{6}I was successful wherever I went,\fnote{\fbackref{29:6} Lit. \fbib{When my feet were bathed in cream;}} \\
\poemll    and even the rocks poured out streams of olive oil for me.''
\passage{Job Remembers His Respected Position}
\poeml \v{7}``Whenever I went out to the city gate, \\
\poemll    a seat had been reserved for me in the plaza.\fnote{\fbackref{29:7} Lit. \fbib{square}; i.e. he served as a ruling elder in his home city} \\
\poeml \v{8}The young men would see me and withdraw, \\
\poemll    and the aged would rise and stand. \\
\poeml \v{9}Nobles would refrain from speaking, \\
\poemll    covering their mouths with their hands. \\
\poeml \v{10}The voices of the commanders-in-chief\fnote{\fbackref{29:10} Lit. \fbib{Nagidim}; i.e. senior officers entrusted with dual roles of operational oversight and administrative authority} were hushed, \\
\poemll    and their tongues would cling to the roofs of their mouths.''
\passage{Job Remembers His Acts of Kindness}
\poeml \v{11}``When people heard me speak, they blessed me; \\
\poemll    when people saw me, they approved me, \\
\poeml \v{12}because I delivered the poor who were crying for help, \\
\poemll    along with orphans who had no one to help them. \\
\poeml \v{13}Those who were about to die blessed me, \\
\poemll    and I made widows sing for joy. \\
\poeml \v{14}I put on righteousness like clothing; \\
\poemll    my just decisions were like a robe and a turban. \\
\poeml \v{15}I served as eyes for the blind \\
\poemll    and feet for the lame. \\
\poeml \v{16}I was a father to the needy; \\
\poemll    I diligently inquired into the case of those I didn't know. \\
\poeml \v{17}I broke the fangs of the wicked, \\
\poemll    and made him drop the prey.''
\passage{Job Remembers His Previous Condition}
\poeml \v{18}``I used to say: `I will die in my home.\fnote{\fbackref{29:18} Lit. \fbib{nest}} \\
\poemll    I'm going to live as many days \\
\poemlll       as there are grains of sand on the shore.\fnote{\fbackref{29:18} The Heb. lacks \fbib{on the shore}} \\
\poeml \v{19}My roots have spread out and have found water, \\
\poemll    and dew settles at night on my branches. \\
\poeml \v{20}My glory renews for me \\
\poemll    and my bow is as good as new in my hand.' \\
\poeml \v{21}``They listened and waited for me, \\
\poemll    as they remained in silence for my counsel. \\
\poeml \v{22}After I spoke, they had nothing to say, \\
\poemll    when what I said hit them. \\
\poeml \v{23}They waited for me as one waits for rain, \\
\poemll    as one opens his mouth to drink in a spring rain shower. \\
\poeml \v{24}I smiled at them when they had no confidence, \\
\poemll    and no one could discourage me. \\
\poeml \v{25}I set an example of the way to live,\fnote{\fbackref{29:25} Lit. \fbib{I chose their way}} as a leader would; \\
\poemll    I lived like a king among his army; \\
\poemlll       like one who comforts mourners.''
\end{poetry}
\labelchapt{30}
\passage{Job Describes His Current Status in Life}

\chapt{30}
\v{1}``But now they mock me;

\begin{poetry}
\poemll    men who are far younger than I, \\
\poeml whose fathers I would have hated \\
\poemll    to entrust with my own sheep dogs. \\
\poeml \v{2}Furthermore, what could I have gained \\
\poemll    from men whose strength is gone? \\
\poeml \v{3}Unproductive due to poverty\fnote{\fbackref{30:3} Or \fbib{want}} and hunger, \\
\poemll    they could only scratch in parched soil, \\
\poemlll       devastated and desolated. \\
\poeml \v{4}``They would pluck off herbs from salt marshes to eat; \\
\poemll    and roots of the broom shrub\fnote{\fbackref{30:4} I.e. a desert bush native to Israel whose bitter roots could be harvested by the destitute and eaten when food was scarce} for food. \\
\poeml \v{5}Driven away from human company, \\
\poemll    they were shouted at as though they were thieves. \\
\poeml \v{6}They lived in the most dangerous of ravines, \\
\poemll    in holes in the ground, and among rocks. \\
\poeml \v{7}They bray like donkeys\fnote{\fbackref{30:7} The Heb. lacks \fbib{like donkeys}} among the bushes \\
\poemll    and huddle together under the desert weeds. \\
\poeml \v{8}Sons of fools and of uncertain reputation,\fnote{\fbackref{30:8} The Or \fbib{and without a name}} \\
\poemll    they have been driven from the land by scourging.''
\passage{Job Presents the Actions of the Mockers}
\poeml \v{9}``Now, I've become the object of their mocking melodies;\fnote{\fbackref{30:9} Lit. \fbib{their neginnoth}} \\
\poemll    I'm nothing but a fool's proverb to them! \\
\poeml \v{10}They abhor me---they keep their distance from me; \\
\poemll    but they don't refrain from spitting at the sight of me. \\
\poeml \v{11}But God\fnote{\fbackref{30:11} Lit. \fbib{he}} has loosened his cord and afflicted me; \\
\poemll    so they've cast off all restraints in my presence. \\
\poeml \v{12}``A wretched crowd ambushes me to my right; \\
\poemll    they trip my feet; \\
\poemlll       they build up their path of calamity for me. \\
\poeml \v{13}They tear up my pathways; \\
\poemll    they profit from my destruction, \\
\poemlll       and they need no help to do this! \\
\poeml \v{14}They come like those who breach through a wall; \\
\poemll    as everything crashes around me they'll roll on and on! \\
\poeml \v{15}My greatest fears have overcome me; \\
\poemll    my honor is assaulted as though by a wind storm; \\
\poemlll       my prosperity evaporates like a morning cloud.''
\passage{Job Accuses God of Mistreating Him}
\poeml \v{16}``Now, my soul pours itself out; \\
\poemll    the time of my affliction has taken control of me. \\
\poeml \v{17}The night racks my bones; \\
\poemll    and the pain that gnaws on me will not rest. \\
\poeml \v{18}My clothes are disheveled by his forceful treatment of me;\fnote{\fbackref{30:18} The Heb. lacks \fbib{of me}} \\
\poemll    he restricts my movement like the collar of my cloak. \\
\poeml \v{19}``He tossed me into the mire; \\
\poemll    I've become like dust and ashes. \\
\poeml \v{20}I cry for help to you, \\
\poemll    but you won't answer me; \\
\poeml I stand still, \\
\poemll    but you only look at me. \\
\poeml \v{21}You changed toward me, and now you're cruel to me; \\
\poemll    with your mighty hand you are persecuting me; \\
\poeml \v{22}you carried me off in a wind storm, \\
\poemll    making me ride on it \\
\poemlll       while you toss me about as the storm roars around me. \\
\poeml \v{23}I know that you're about to kill me, \\
\poemll    so I'm about to go to the house that's appointed for all the living.''
\passage{Job Lists His Hopes Despite His Deplorable Condition}
\poeml \v{24}``Surely he won't stretch his hand against the needy, will he, \\
\poemll    especially if they cry to him in their calamity? \\
\poeml \v{25}Haven't I wept for the one who is going through hard times? \\
\poemll    Haven't I grieved for the needy? \\
\poeml \v{26}I have hoped for good, but evil came instead; \\
\poemll    I have hoped for light, but darkness came. \\
\poeml \v{27}I'm boiling mad inside, and I won't remain silent; \\
\poemll    the time for my affliction to confront me has arrived. \\
\poeml \v{28}``In growing darkness, I walked without sunlight; \\
\poemll    I stood in the congregation to cry for help. \\
\poeml \v{29}I've become a brother to jackals, \\
\poemll    and a friend to ostriches. \\
\poeml \v{30}My skin turns black all over me; \\
\poemll    and my bones seem burned from the heat. \\
\poeml \v{31}But my harp is in mourning; \\
\poemll    my flute plays only songs for those who are weeping.''
\end{poetry}
\labelchapt{31}
\passage{Job Asserts His Moral Innocence}

\begin{poetry}
\poeml \chapt{31}
\v{1}``I made a covenant with my eyes; \\
\poemll    how, then, can I focus my attention on a virgin? \\
\poeml \v{2}What would I have\fnote{\fbackref{31:2} The Heb. lacks \fbib{would one have}} from God above, \\
\poemll    what heritage from the Almighty on high, \\
\poeml \v{3}if not calamity that is due the unjust, \\
\poemll    and misfortune that is due those who practice iniquity? \\
\poeml \v{4}He watches my life, \\
\poemll    observing every one of my actions,\fnote{\fbackref{31:4} Lit. \fbib{steps}} does he not?''
\passage{No Lies and Deception}
\poeml \v{5}``If I've lived my life in the company of vanity, \\
\poemll    or run quickly to embrace deception, \\
\poeml \v{6}let my righteousness be weighed in honest scales, \\
\poemll    and God will make known my integrity. \\
\poeml \v{7}If I have stepped away from the way, \\
\poemll    or if my heart covets whatever my eyes see, \\
\poemlll       or if some other blemish clings to my hands, \\
\poeml \v{8}what I've planted, let another eat \\
\poemll    or let my crops be uprooted.''
\passage{No Adultery}
\poeml \v{9}``If my heart has been seduced by a woman \\
\poemll    and I've laid in wait at my friend's door, \\
\poeml \v{10}then let my wife cook\fnote{\fbackref{31:10} Lit. \fbib{grind}} for another person \\
\poemll    and may someone else sleep with her, \\
\poeml \v{11}because something as lascivious as that \\
\poemll    is an iniquity that should be judged. \\
\poeml \v{12}The fires of Abaddon\fnote{\fbackref{31:12} Or \fbib{Destruction}; i.e. the realm of eternal punishment in the afterlife} will burn,\fnote{\fbackref{31:12} Lit. \fbib{consume}} \\
\poemll    disrupting every part of my eternal reward.''\fnote{\fbackref{31:12} Lit. \fbib{my harvest}}
\passage{No Abuse of Servants}
\poeml \v{13}``If I've refused to help my male and female servants \\
\poemll    when they complain against me, \\
\poeml \v{14}what will I do when God stands up to act? \\
\poemll    When he asks the questions, how will I answer him? \\
\poeml \v{15}The one who made me in the womb made them,\fnote{\fbackref{31:15} Lit. \fbib{him}} too, didn't he? \\
\poemll    Didn't the same one prepare each of us in the womb?''
\passage{No Injustice on the Poor}
\poeml \v{16}``If I refused to grant the desire of the poor \\
\poemll    or exhausted the eyes of the widow, \\
\poeml \v{17}if I ate my meals by myself \\
\poemll    without feeding orphans, \\
\poeml \v{18}(even a poor man had grown up with me as if I were his father, \\
\poemll    and even though I had guided the widow\fnote{\fbackref{31:18} Lit. \fbib{her}} \\
\poemlll       from the time I was born), \\
\poeml \v{19}if I've observed someone who is about to die for lack of clothes \\
\poemll    or if I have no clothing to give to the poor, \\
\poeml \v{20}if he hadn't thanked me from the bottom of his heart,\fnote{\fbackref{31:20} Lit. \fbib{hadn't blessed me from his loins}} \\
\poemll    if he had not been warmed by wool from my sheep, \\
\poeml \v{21}if I've raised my hand against an orphan \\
\poemll    when I thought I would against him in court,\fnote{\fbackref{31:21} Lit. \fbib{when I saw help for me at the gate,}} \\
\poeml \v{22}then let my arm\fnote{\fbackref{31:22} Lit. \fbib{side}} fall from its socket; \\
\poemll    and may my arm be torn off at the shoulder. \\
\poeml \v{23}For I'm terrified of what calamity God may have in store for me; \\
\poemll    and I cannot endure his grandeur.''
\passage{No Trust in Wealth and Heavenly Bodies}
\poeml \v{24}``If I've put my confidence in gold, \\
\poemll    if I've told gold, `You're my security,' \\
\poeml \v{25}if I've found joy in great wealth that I own, \\
\poemll    if I've earned a lot with my own hands, \\
\poeml \v{26}if I look at the sun\fnote{\fbackref{31:26} Lit. \fbib{light}} when it shines \\
\poemll    or the moon as it rises in steady splendor, \\
\poeml \v{27}so that in the depths of my deceived heart \\
\poemll    I worshipped them with my mouth and hands, \\
\poeml \v{28}this is also a sin that deserves to be judged, \\
\poemll    since I would have tried to deceive\fnote{\fbackref{31:28} Or \fbib{have denied}} God above.''
\passage{No Rejoicing over the Plight of Adversary}
\poeml \v{29}``Have I rejoiced in the destruction of those who hate me, \\
\poemll    or have I been happy that evil caught up with him? \\
\poeml \v{30}No, I haven't allowed my mouth to sin \\
\poemll    by asking for his life\fnote{\fbackref{31:30} Lit. \fbib{soul}} with a curse. \\
\poeml \v{31}People in my household have said, \\
\poemll    `We cannot find anyone who has not been satisfied with his meat,' haven't they? \\
\poeml \v{32}No stranger ever spent the night in the street, \\
\poemll    because I opened my doors to travelers.''
\passage{No Secret Sins}
\poeml \v{33}``Have I covered my transgression like other people, \\
\poemll    to conceal iniquity within myself?\fnote{\fbackref{31:33} Or \fbib{bosom}} \\
\poeml \v{34}Have I feared large crowds? \\
\poemll    Has my family's contempt ever terrified me \\
\poemlll       so that I remained silent and wouldn't go outside?''
\passage{Request for A Hearing}
\poeml \v{35}``Who will grant me a hearing? \\
\poemll    Here's my signature\fnote{\fbackref{31:35} Lit. \fbib{seal}}---let the Almighty answer! \\
\poeml Since my adversary indicted me, \\
\poeml \v{36}I'll wear it on my shoulder, \\
\poemlll       or tie it on my head for a crown! \\
\poeml \v{37}I'll give an account for every step I've taken; \\
\poemll    I'll approach him confidently like a Commander-in-Chief.''\fnote{\fbackref{31:37} Lit. \fbib{Nagid}; i.e. a senior officer entrusted with dual roles of operational oversight and administrative authority}
\passage{No Abuse of the Land}
\poeml \v{38}``If my land were to cry out against me \\
\poemll    or if all its furrows wept as one, \\
\poeml \v{39}If I've consumed its produce\fnote{\fbackref{31:39} Lit. \fbib{strength}} without paying for it \\
\poemll    and snuffed out the life of its owners; \\
\poeml \v{40}may thorns spring up instead of wheat, \\
\poemll    and obnoxious weeds instead of barley.''
\end{poetry}

With this, Job's discourse with his friends\fnote{\fbackref{31:40} The Heb. lacks \fbib{with his friends}} is completed.
\labelchapt{32}
\passage{Elihu Addresses Job and His Friends}

\chapt{32}
\v{1}These three men stopped responding to Job, because he was claiming to be righteous, in his own opinion.\fnote{\fbackref{32:1} Lit. \fbib{eyes}} \v{2}But then Barachel's son Elihu from Buz, one of Ram's descendants, got really angry. He was furious with Job because he had been declaring himself righteous instead of vindicating God. \v{3}Furthermore, he was furious with his three friends because they had not answered Job, but instead had condemned him. \v{4}Elihu waited to have a word with Job, since the others were older than he, \v{5}but when he saw that there had been no response\fnote{\fbackref{32:5} Lit. \fbib{mouth}} from those three, he got even more angry. \v{6}Barachel's son Elihu from Buz responded and said:

\begin{poetry}
\poeml I'm younger than you are. \\
\poemll    Because you're older,\fnote{\fbackref{32:6} Lit. \fbib{aged}} I was terrified \\
\poemlll       to tell you what I know. \\
\poeml \v{7}I thought, experience\fnote{\fbackref{32:7} Lit. \fbib{days}} should speak; \\
\poemll    abundance of years teaches wisdom. \\
\poeml \v{8}However, a spirit exists in mankind, \\
\poemll    and the Almighty's breath gives him insight.
\passage{There's No Fool Like an Old Fool}
\poeml \v{9}``The aged aren't always wise, \\
\poemll    nor do the elderly always understand justice. \\
\poeml \v{10}Therefore I'm saying, `Listen to me!' \\
\poemll    Then I'll declare what I know. \\
\poeml \v{11}``Look! I have waited to hear your speech, \\
\poemll    so I listened to your insights \\
\poemlll       while you searched for the right words to say.\fnote{\fbackref{32:11} The Heb. lacks \fbib{to say}} \\
\poeml \v{12}Indeed, I paid close attention to you all, \\
\poemll    but none of you were able to refute\fnote{\fbackref{32:12} Or \fbib{rebuke}} Job \\
\poemlll       or answer his arguments convincingly. \\
\poeml \v{13}``So that you cannot claim, `We have found wisdom!' \\
\poemll    let God do the rebuking, not man; \\
\poeml \v{14}let him not direct a rebuke toward me. \\
\poemll    I won't be responding to him with your arguments. \\
\poeml \v{15}``Job's friends\fnote{\fbackref{32:15} Lit. \fbib{They}} won't reason with him anymore; \\
\poemll    discouraged, words escape them. \\
\poeml \v{16}Shall I continue to wait, since they're no longer talking? \\
\poemll    After all, they're only standing there; \\
\poemlll       they're no longer responding. \\
\poeml \v{17}``I will contribute my arguments\fnote{\fbackref{32:17} Lit. \fbib{portion}} as an answer; \\
\poemll    I'll declare what I know, \\
\poeml \v{18}because I'm filled with things to say, \\
\poemll    and my spirit within me compels me to speak.\fnote{\fbackref{32:18} The Heb. lacks \fbib{to speak}} \\
\poeml \v{19}My insides feel like unvented wine, \\
\poemll    like it's about to burst like a new wineskin. \\
\poeml \v{20}``Let me speak! I need relief! \\
\poemll    Let me open my lips and respond. \\
\poeml \v{21}I won't discriminate against anyone, \\
\poemll    and I won't flatter any person, \\
\poeml \v{22}since I don't know the first thing about how to flatter; \\
\poemll    and the one who made me would sweep me away \\
\poemlll       as if I were nothing.''
\end{poetry}
\labelchapt{33}
\passage{Elihu Begins His Discourse}

\begin{poetry}
\poeml \chapt{33}
\v{1}``Now please listen to what I have to say, Job. \\
\poemll    Listen to every word! \\
\poeml \v{2}Look! I've begun to speak,\fnote{\fbackref{33:2} Lit. \fbib{I've opened my mouth}} \\
\poemll    and I'm fashioning my words.\fnote{\fbackref{33:2} Lit. \fbib{and my tongue speaks in my mouth}} \\
\poeml \v{3}I speak from the innocence\fnote{\fbackref{33:3} Or \fbib{integrity}} of my heart; \\
\poemll    and my lips will utter what I sincerely know. \\
\poeml \v{4}``The spirit of God fashioned me; \\
\poemll    and the breath of the Almighty gives me life. \\
\poeml \v{5}Answer me, if you can! \\
\poemll    Present your case! Take your stand in my presence! \\
\poeml \v{6}Look! As far as God is concerned,\fnote{\fbackref{33:6} Lit. \fbib{Look! Before God}} I'm just like you are--- \\
\poemll    I, too, have been pinched off from a piece of clay. \\
\poeml \v{7}Don't be afraid of me; \\
\poemll    I'll go easy\fnote{\fbackref{33:7} Lit. \fbib{my hand won't be heavy}} on you.''
\passage{Elihu Reviews Job's Claim for Innocence}
\poeml \v{8}``You spoke clearly so I could hear; \\
\poemll    I've heard what you've said: \\
\poeml \v{9}`I'm pure. I'm without sin; \\
\poemll    I'm innocent. I'm harboring no iniquity inside of me. \\
\poeml \v{10}Nevertheless, God\fnote{\fbackref{33:10} Lit. \fbib{he}} has found a pretext to attack me; \\
\poemll    he considers me his enemy. \\
\poeml \v{11}He has bound my feet in shackles, \\
\poemll    and keeps watching everything I do.'\,''\fnote{\fbackref{33:11} Lit. \fbib{watching all my paths}}
\passage{God Responds to Humanity's Need}
\poeml \v{12}``You aren't right about this; \\
\poemll    My response is that God is greater than human beings. \\
\poeml \v{13}Why are you arguing with him? \\
\poemll    He doesn't have to give explanations for what he does to you! \\
\poeml \v{14}``God speaks time and time again\fnote{\fbackref{33:14} Lit. \fbib{speaks once and twice}}--- \\
\poemll    but nobody notices--- \\
\poeml \v{15}in a dream or night vision, \\
\poemll    when a deep sleep falls on mankind \\
\poemlll       while they sleep on their beds. \\
\poeml \v{16}That's when he opens the ear of mankind, \\
\poemll    authenticating his messages\fnote{\fbackref{33:16} Lit. \fbib{mankind, sealing his instruction}} to them, \\
\poeml \v{17}turning a person from his actions, \\
\poemll    keeping him\fnote{\fbackref{33:17} Lit. \fbib{man}} from pride, \\
\poeml \v{18}sparing his soul from the Pit\fnote{\fbackref{33:18} I.e. the realm of punishment in the afterlife} \\
\poemll    and his life from violent death.\fnote{\fbackref{33:18} Lit. \fbib{from death by the sword}} \\
\poeml \v{19}``He is being reproved by painful bed rest, \\
\poemll    with continual aching in his bones. \\
\poeml \v{20}He cannot stand his food, \\
\poemll    and he\fnote{\fbackref{33:20} Lit. \fbib{his soul}} has no desire for appetizing food. \\
\poeml \v{21}His flesh wastes away; \\
\poemll    his bones, which once couldn't be seen, are visible. \\
\poeml \v{22}His soul is getting close to the Pit;\fnote{\fbackref{33:22} I.e. the realm of punishment in the afterlife} \\
\poemll    his life is approaching its executioner.''
\passage{God Delivers through His Ransom}
\poeml \v{23}``If there's a messenger\fnote{\fbackref{33:23} Or \fbib{an angel}} appointed to mediate for Job\fnote{\fbackref{33:23} Lit. \fbib{him}} \\
\poemll    ---one out of a thousand--- \\
\poemlll       to represent the man's integrity on his behalf, \\
\poeml \v{24}to show favor to him and to plead, \\
\poemll    `Deliver him from having to go down to the Pit\fnote{\fbackref{33:24} I.e. the realm of punishment in the afterlife}--- \\
\poemlll       I know where his ransom is!' \\
\poeml \v{25}Let his flesh be rejuvenated\fnote{\fbackref{33:25} Lit. \fbib{grew fresh}} as he was in his youth! \\
\poemll    Let him recover the strength of his youth. \\
\poeml \v{26}Let him pray to God \\
\poemll    and he will accept him; \\
\poemlll       he will appear before him with joyful shouts!''
\passage{The Song of the Ransomed}
\poeml \v{27}``He'll sing to mankind with these words: \\
\poeml `I've sinned. I have twisted what is right. \\
\poemll    Yet he has not repaid me like I deserve.\fnote{\fbackref{33:27} The Heb. lacks \fbib{like I deserve}} \\
\poeml \v{28}He has redeemed my soul from going down to the Pit;\fnote{\fbackref{33:28} I.e. the realm of punishment in the afterlife} \\
\poemll    my life will see the light.' \\
\poeml \v{29}Indeed God does all these things \\
\poemll    again and again\fnote{\fbackref{33:29} Lit. \fbib{things twice, three times}} with a person \\
\poeml \v{30}to bring back his soul from the Pit;\fnote{\fbackref{33:30} I.e. the realm of punishment in the afterlife} \\
\poemll    to light him with the light of life.''
\passage{Elihu Invites Job to Respond}
\poeml \v{31}``Job, pay attention! Listen to me! \\
\poemll    Be silent and let me speak. \\
\poeml \v{32}If you have anything to say, answer me; \\
\poemll    speak up, because I'd be happy to vindicate you. \\
\poeml \v{33}But if you have nothing to say, then at least listen to me! \\
\poemll    Be quiet and learn some wisdom from me.''
\end{poetry}
\labelchapt{34}
\passage{Elihu Continues Speaking}

\chapt{34}
\v{1}Elihu continued speaking, and said:

\begin{poetry}
\poeml \v{2}``Listen to what I have to say, you wise men! \\
\poemll    Pay attention to me, you educated people! \\
\poeml \v{3}Since the ear tests words \\
\poemll    like a palate tastes food, \\
\poeml \v{4}let's choose what's right for us. \\
\poemll    Let's consider among ourselves what is good.''
\passage{Elihu Reviews Job's Complaint against God's Injustice}
\poeml \v{5}Now this is Job's claim: \\
\poeml `Even though I'm innocent, \\
\poemll    God has stopped treating me righteously. \\
\poeml \v{6}Have I lied concerning the justice that I deserve?\fnote{\fbackref{34:6} Lit. \fbib{concerning my justice}} \\
\poemll    My wound\fnote{\fbackref{34:6} Or \fbib{cut}} is incurable, \\
\poemlll       though transgression cannot be attributed to me.' \\
\poeml \v{7}``What man is like Job, \\
\poemll    who drinks mockery like water, \\
\poeml \v{8}traffics in those who practice evil, \\
\poemll    and walks with wicked people? \\
\poeml \v{9}Because he says, `There's no profit \\
\poemll    for a man to find joy with God.'\,''\fnote{\fbackref{34:9} Cf. Mal 3:14}
\passage{God is Just}
\poeml \v{10}``Therefore you men of understanding,\fnote{\fbackref{34:10} Lit. \fbib{heart}} listen to me! \\
\poemll    Far be it for God to practice wickedness, \\
\poemlll       or the Almighty to do what is wrong, \\
\poeml \v{11}because he repays a person for his behavior; \\
\poemll    and according to a person's\fnote{\fbackref{34:11} Lit. \fbib{man}} conduct, \\
\poemlll       he lets it happen to\fnote{\fbackref{34:11} Lit. \fbib{it find}} him. \\
\poeml \v{12}Truly, God doesn't practice wickedness, \\
\poemll    and the Almighty doesn't pervert justice. \\
\poeml \v{13}Who entrusted the earth to him? \\
\poemll    Who made him responsible for the entire inhabited world? \\
\poeml \v{14}If he were to decide to do so, \\
\poemll    that is, to take back to himself\fnote{\fbackref{34:14} The Heb. lacks \fbib{to himself}} his spirit and breath of life,\fnote{\fbackref{34:14} The Heb. lacks \fbib{of life}} \\
\poeml \v{15}every living thing would die all at once,\fnote{\fbackref{34:15} Lit. \fbib{die together}} \\
\poemll    and mankind would return to dust.''
\passage{God's Rule is Just}
\poeml \v{16}If you have\fnote{\fbackref{34:16} The Heb. lacks \fbib{you have}} understanding, listen to this! \\
\poemll    Pay attention to what I have to say: \\
\poeml \v{17}Can one who hates justice really govern? \\
\poemll    And if God\fnote{\fbackref{34:17} Lit. \fbib{he}} is righteous and mighty, can you condemn him?\fnote{\fbackref{34:17} The Heb. lacks \fbib{him}} \\
\poeml \v{18}Can one say to a king, `You're vile!' \\
\poemll    or to nobles, `You're wicked!'? \\
\poeml \v{19}Who isn't partial to\fnote{\fbackref{34:19} Lit. \fbib{Who doesn't lift the faces of}} princes? \\
\poemll    Who doesn't give preference to the nobles over the poor? \\
\poemlll       Nevertheless, all of them are his handiwork. \\
\poeml \v{20}``They die suddenly, in the middle of the night; \\
\poemll    people suffer seizures and pass away; \\
\poeml even valiant men can be taken away--- \\
\poemll    and not by human hands. \\
\poeml \v{21}Yes, Job,\fnote{\fbackref{34:21} The Heb. lacks \fbib{Job}} his eyes constantly watch the behavior of human beings; \\
\poemll    he carefully observes their every step. \\
\poeml \v{22}There's no such thing as darkness to him--- \\
\poemll    not even deep darkness--- \\
\poemlll       that can conceal those who practice evil. \\
\poeml \v{23}He won't examine mankind further, \\
\poemll    that they would go before God to judgment. \\
\poeml \v{24}He shatters valiant men without a need to investigate, \\
\poemll    and he raises others in their place. \\
\poeml \v{25}Thus he acknowledges their behavior, and overcomes them; \\
\poemll    when night time comes, they are crushed. \\
\poeml \v{26}``He strikes\fnote{\fbackref{34:26} Or \fbib{slaps}} the wicked among them \\
\poemll    in a place where they can be seen \\
\poeml \v{27}because they've abandoned their pursuit of him \\
\poemll    and had no respect for any of his ways. \\
\poeml \v{28}As a result, the cries of the poor have reached him \\
\poemll    and he has heard the cry of the afflicted. \\
\poeml \v{29}``If he remains silent, who will condemn him? \\
\poemll    If he conceals his face, who can see him? \\
\poemlll       He watches over both nation and individual alike, \\
\poeml \v{30}to keep the godless man from reigning \\
\poemll    or laying a snare for the people.''
\passage{Elihu's Challenge to Job}
\poeml \v{31}``Has anyone ever really told God, \\
\poemll    `I've endured,\fnote{\fbackref{34:31} Or \fbib{carry}} and I won't act corruptly anymore. \\
\poeml \v{32}What I don't see, instruct me! \\
\poemll    If I've done anything evil, I won't repeat it!' \\
\poeml \v{33}``Should you not be paid back, \\
\poemll    since you have rejected him? \\
\poeml You do the choosing! I won't! \\
\poemll    Tell us what you know!
\passage{Elihu's Verdict: Job is not Wise}
\poeml \v{34}``Men of understanding, speak to me! \\
\poemll    Are any of you men wise? Then listen to me! \\
\poeml \v{35}Job has been speaking from his own ignorance, \\
\poemll    and what he has to say lacks insight! \\
\poeml \v{36}Oh, how Job needs to be given a full court trial, \\
\poemll    as a rebuke to those who practice evil, \\
\poeml \v{37}because he has been adding rebellion to his sin; \\
\poemll    he claps his hands among us,\fnote{\fbackref{34:37} I.e. as a gesture of disrespect} \\
\poemlll       and keeps on ranting against God.''
\end{poetry}
\labelchapt{35}
\passage{Elihu Speaks Again}

\chapt{35}
\v{1}In response, Elihu said:

\begin{poetry}
\poeml \v{2}``Are you saying that it's just for you to claim, \\
\poemll    `I'm more righteous than God?' \\
\poeml \v{3}After all, you've asked what your benefit will be: \\
\poemll    `What will I profit from refraining from sin?' \\
\poeml \v{4}I'm going to respond to that statement, \\
\poemll    and to your friends with you.''
\passage{God's Justice Remains Unsullied}
\poeml \v{5}``Observe the heavens! Take a look around! \\
\poemll    Look! The clouds are higher than you, aren't they? \\
\poeml \v{6}If you sin, what will that do to harm him? \\
\poemll    If you add transgression to transgression \\
\poemlll       what will it do to him? \\
\poeml \v{7}If you are righteous, what will you add to him? \\
\poemll    What can God receive from your efforts?\fnote{\fbackref{35:7} Lit. \fbib{hand}} \\
\poeml \v{8}Your wickedness affects only\fnote{\fbackref{35:8} The Heb. lacks \fbib{only}} yourself; \\
\poemll    and your righteousness, only human beings.\fnote{\fbackref{35:8} Lit. \fbib{only a son of man}} \\
\poeml \v{9}``They cry out because they have many oppressors; \\
\poemll    they cry for help because the powerful are abusing them.\fnote{\fbackref{35:9} Lit. \fbib{because of the arm of the powerful}} \\
\poeml \v{10}He never asks, `Where is God, my Creator, \\
\poemll    who gives me songs in the night, \\
\poeml \v{11}who teaches us more than the earth's wild animals, \\
\poemll    and makes us wiser than the birds of the sky?' \\
\poeml \v{12}``They cry out there, but he doesn't answer \\
\poemll    because of the arrogance of those who practice evil. \\
\poeml \v{13}Theirs is a useless plea--- \\
\poemll    God won't listen; \\
\poemlll       the Almighty won't pay any attention. \\
\poeml \v{14}Even though you complain that you can't perceive him, \\
\poemll    your case is already pending for judgment in his presence \\
\poemlll       so keep on placing your hope in him. \\
\poeml \v{15}``So now, if he doesn't inflict punishment in his anger, \\
\poemll    then he doesn't keep track of your many transgressions. \\
\poeml \v{16}When he began speaking, he communicated only worthlessness; \\
\poemll    he added words upon words without knowing anything.''
\end{poetry}
\labelchapt{36}
\passage{Elihu Concludes His Arguments}

\chapt{36}
\v{1}Elihu responded again and said:

\begin{poetry}
\poeml \v{2}``Be patient with me a moment longer, \\
\poemll    and I'll show you that there's more to say on God's behalf. \\
\poeml \v{3}I'll take what I know to its logical conclusion\fnote{\fbackref{36:3} Lit. \fbib{I'll bring my knowledge from a long ways away}} \\
\poemll    and ascribe righteousness to my Creator, \\
\poeml \v{4}because what I have to say isn't deceptive, \\
\poemll    and the one who has perfect knowledge is with you.''
\passage{God Disciplines}
\poeml \v{5}``Indeed God is mighty and he doesn't show disrespect; \\
\poemll    he is mighty and strong of heart. \\
\poeml \v{6}He doesn't let the wicked live; \\
\poemll    he grants justice to the afflicted. \\
\poeml \v{7}He won't stop looking at righteous people; \\
\poemll    he seats them on thrones with kings forever, \\
\poemlll       and they are exalted. \\
\poeml \v{8}``If they're bound in chains, \\
\poemll    caught in ropes of affliction, \\
\poeml \v{9}he'll reveal their actions to them, \\
\poemll    when their transgressions have become excessive. \\
\poeml \v{10}He opens their ears and instructs them, \\
\poemll    commanding them to repent from evil. \\
\poeml \v{11}If they listen and serve him,\fnote{\fbackref{36:11} The Heb. lacks \fbib{him}} \\
\poemll    they'll finish\fnote{\fbackref{36:11} Or \fbib{finish}} their lives in prosperity \\
\poemlll       and their years will be pleasant. \\
\poeml \v{12}``But if they won't listen, \\
\poemll    they'll perish\fnote{\fbackref{36:12} Lit. \fbib{by the sword they'll pass through} } by the sword \\
\poemlll       and die in their ignorance. \\
\poeml \v{13}The godless at heart cherish\fnote{\fbackref{36:13} Lit. \fbib{lay up}} anger; \\
\poemll    they won't cry out for help when God\fnote{\fbackref{36:13} Lit. \fbib{he}} afflicts\fnote{\fbackref{36:13} Lit. \fbib{binds}} them. \\
\poeml \v{14}They\fnote{\fbackref{36:14} Lit. \fbib{Their souls}} die in their youth; \\
\poemll    and their life will end\fnote{\fbackref{36:14} The Heb. lacks \fbib{will end}} among temple prostitutes. \\
\poeml \v{15}He'll deliver the afflicted through their afflictions \\
\poemll    and open their ears when they are oppressed.''
\passage{God is an All-Powerful and Just Teacher}
\poeml \v{16}``Indeed, he drew you away from the brink of distress \\
\poemll    to a spacious place without constraints, \\
\poemlll       filling your festive\fnote{\fbackref{36:16} Lit. \fbib{restful}} table with bountiful\fnote{\fbackref{36:16} Lit. \fbib{fat}} food. \\
\poeml \v{17}But now you are occupied with the case of the wicked; \\
\poemll    but justice and judgment will be served. \\
\poeml \v{18}So that no one entices you with riches, \\
\poemll    don't let a large ransom turn you astray. \\
\poeml \v{19}``Will your wealth sustain you when you're in distress, \\
\poemll    despite your most powerful efforts?\fnote{\fbackref{36:19} Or \fbib{your force of strength}} \\
\poeml \v{20}Don't long for night, \\
\poemll    when people vanish\fnote{\fbackref{36:20} Lit. \fbib{go up}} in their place. \\
\poeml \v{21}Be careful! Don't turn to evil, \\
\poemll    because of this you will be tried by more than affliction. \\
\poeml \v{22}``Indeed, God is exalted in his power. \\
\poemll    Who is like him as a teacher? \\
\poeml \v{23}Who ordained his path for him, \\
\poemll    and who has asked him, `You are wrong, aren't you?' \\
\poeml \v{24}Remember to magnify his awesome activities, \\
\poemll    about which mortal man has sung. \\
\poeml \v{25}All of mankind sees him; \\
\poemll    human beings observe him from afar off.''
\passage{God Controls the Weather}
\poeml \v{26}``God is truly awesome, beyond what we know; \\
\poemll    the number of his years is unknowable.\fnote{\fbackref{36:26} Or \fbib{unreachable}} \\
\poeml \v{27}He draws up drops of water, \\
\poemll    distilling it to rain and mist.\fnote{\fbackref{36:27} Or \fbib{distilling rain into mist}} \\
\poeml \v{28}When the clouds pour down;\fnote{\fbackref{36:28} Or \fbib{drops}} \\
\poemll    they drop their rain on all of humanity. \\
\poeml \v{29}``Furthermore, can anyone understand cloud patterns, \\
\poemll    or the thundering in his pavilion? \\
\poeml \v{30}He scatters his lightning above it, \\
\poemll    and covers the bottom\fnote{\fbackref{36:30} Or \fbib{root}} of the sea. \\
\poeml \v{31}He uses them to judge some people \\
\poemll    and give food to many. \\
\poeml \v{32}His hands are covered with lightning \\
\poemll    that he commands to strike his designated target. \\
\poeml \v{33}His thunder\fnote{\fbackref{36:33} Lit. \fbib{shout}} declares his presence; \\
\poemll    and tells the animals what is coming.''
\end{poetry}
\labelchapt{37}
\passage{Elihu Concludes His Argument}

\begin{poetry}
\poeml \chapt{37}
\v{1}``Now I'll conclude with this: \\
\poemll    my heart is trembling violently; \\
\poemlll       it feels like it's about to leap from my body! \\
\poeml \v{2}Listen carefully to his thundering voice; \\
\poemll    to the sound that rumbles from his mouth. \\
\poeml \v{3}He releases his lightning throughout the sky, \\
\poemll    to the ends\fnote{\fbackref{37:3} Lit. \fbib{wingtips}} of the earth. \\
\poeml \v{4}His thunder roars after it; \\
\poemll    his majestic voice will thunder; \\
\poeml and no one can trace them\fnote{\fbackref{37:4} Lit. \fbib{follow at the heel of their feet}} \\
\poemll    once his voice has been heard. \\
\poeml \v{5}``God thunders with his wondrous voice; \\
\poemll    he does awesome works that we don't comprehend. \\
\poeml \v{6}For he says to the snow, `Fall to the earth.' \\
\poemll    He tells the rain, `Pour down,' \\
\poemlll       then it rains profusely. \\
\poeml \v{7}``He puts a limit to the skill\fnote{\fbackref{37:7} Lit. \fbib{a seal on the hand}} of every person; \\
\poemll    to delineate all people from what they do. \\
\poeml \v{8}``Then a beast enters its lair \\
\poemll    and remains in its den. \\
\poeml \v{9}``From the south,\fnote{\fbackref{37:9} Or \fbib{From a storeroom}} a whirlwind proceeds, \\
\poemll    out of the icy north winds. \\
\poeml \v{10}From the breath of God ice is produced, \\
\poemll    and a wide body of water is frozen. \\
\poeml \v{11}He also loads the clouds with moisture, \\
\poemll    scattering his lightning with the clouds. \\
\poeml \v{12}It whirls about in circles at his direction \\
\poemll    to accomplish all that he commands \\
\poemlll       throughout the surface of the entire world, \\
\poeml \v{13}whether for discipline on his land \\
\poemll    or to demonstrate his gracious love, \\
\poemlll       he causes it to be realized.''
\passage{Elihu Challenges Job to Pay Attention}
\poeml \v{14}``Pay attention to this, Job! \\
\poemll    Stand still, \\
\poemlll       and consider the wondrous attributes of God. \\
\poeml \v{15}Do you know how God ordains them, \\
\poemll    and makes his lightning to flash throughout his clouds? \\
\poeml \v{16}Do you understand his wondrous work of balancing the clouds, \\
\poemll    the one\fnote{\fbackref{37:16} The Heb. lacks \fbib{the one}} whose knowledge is perfect, \\
\poeml \v{17}you whose garments are hot, \\
\poemll    even though the land is cooled by a south wind? \\
\poeml \v{18}Can you spread out the skies like he does; \\
\poemll    can you cast them as one might a mirror? \\
\poeml \v{19}Tell us! What are we to say to him? \\
\poemll    Can we prepare our case to face him \\
\poemlll       when our faces are in darkness? \\
\poeml \v{20}Has it been relayed to God\fnote{\fbackref{37:20} Lit. \fbib{him}} that I want to talk? \\
\poemll    Can a person\fnote{\fbackref{37:20} Lit. \fbib{man}} speak when he is confused?''
\passage{God is Revered}
\poeml \v{21}``So then, the sun\fnote{\fbackref{37:21} Or \fbib{light}} is too bright to gaze at, is it not? \\
\poemll    The sky is swept clean by the wind that blows,\fnote{\fbackref{37:21} Lit. \fbib{that passes through}} is it not? \\
\poeml \v{22}From the north he brings gold; \\
\poemll    around God is awesome splendor. \\
\poeml \v{23}We cannot find the Almighty--- \\
\poemll    he is majestic in power and justice, \\
\poeml and overflowing with righteousness; \\
\poemll    he never oppresses. \\
\poeml \v{24}Therefore humanity fears him, \\
\poemll    which none of the wise\fnote{\fbackref{37:24} Lit. \fbib{wise of heart}} can quite comprehend.''
\end{poetry}
\labelchapt{38}
\passage{The \divine{Lord} Speaks to Job}

\chapt{38}
\v{1}The \divine{Lord} responded to Job from the whirlwind and said:

\begin{poetry}
\poeml \v{2}``Who is this who keeps darkening my counsel \\
\poemll    without knowing what he's talking about? \\
\poeml \v{3}Stand up\fnote{\fbackref{38:3} Lit. \fbib{Gird up your loins}} like a man! \\
\poemll    I'll ask you some questions, \\
\poemlll       and you give me some answers!''
\passage{On the Natural World}
\poeml \v{4}``Where were you when I laid the foundation of my earth? \\
\poemll    Tell me,\fnote{\fbackref{38:4} Or \fbib{declare}} since you're so informed! \\
\poeml \v{5}Who set its measurement? Am I to assume you know? \\
\poemll    Who stretched a boundary line over it? \\
\poeml \v{6}On what were its bases set? \\
\poemll    Who laid its corner stone \\
\poeml \v{7}while the morning stars sang together \\
\poemll    and all the divine beings\fnote{\fbackref{38:7} Lit. \fbib{sons of God}} shouted joyfully? \\
\poeml \v{8}``Who\fnote{\fbackref{38:8} Lit. \fbib{and he}} enclosed the sea with limits\fnote{\fbackref{38:8} Lit. \fbib{doors}} \\
\poemll    when it gushed out of the womb, \\
\poeml \v{9}when I made clouds to be its clothes \\
\poemll    and thick darkness its swaddling blanket, \\
\poeml \v{10}when I proscribed a boundary for it, \\
\poemll    set in place bars and doors for it; \\
\poeml \v{11}and said, `You may come only this far and no more. \\
\poemll    Your majestic waves will stop here.'? \\
\poeml \v{12}``Have you ever commanded the morning at any time during your life?\fnote{\fbackref{38:12} Lit. \fbib{morning in your days}} \\
\poemll    Do you know where the dawn lives, \\
\poeml \v{13}where it seizes the edge of the earth \\
\poemll    and shakes the wicked out of it? \\
\poeml \v{14}Like clay is molded by a signet ring, \\
\poemll    the earth's hills and valleys\fnote{\fbackref{38:14} The Heb. lacks \fbib{the earth's hills and valleys}} then stand out \\
\poemlll       like the colors of a garment. \\
\poeml \v{15}Then from the wicked their light is withheld \\
\poemll    and their upraised arm is broken. \\
\poeml \v{16}``Have you been to the source of the sea \\
\poemll    and walked about in the recesses of the deepest ocean? \\
\poeml \v{17}Have the gates of death been revealed to you? \\
\poemll    Have you seen the gates of the deepest darkness? \\
\poeml \v{18}Do you understand the breadth of the earth? \\
\poemll    Tell me, since you know it all! \\
\poeml \v{19}``Where is the road to where the light lives? \\
\poemll    Or where does the darkness live? \\
\poeml \v{20}Can you take it to its homeland, \\
\poemll    since you know the path to his house? \\
\poeml \v{21}You should know! After all, you had been born back then, \\
\poemll    so the number of your days is great! \\
\poeml \v{22}``Have you entered the storehouses of the snow \\
\poemll    or seen where the hail is stored, \\
\poeml \v{23}which I've reserved for the tribulation to come, \\
\poemll    for the day of battle and war? \\
\poeml \v{24}Where is the lightning diffused \\
\poemll    or the east wind scattered around the earth? \\
\poeml \v{25}``Who cuts canals for storm floods, \\
\poemll    and paths for the lightning and thunder, \\
\poeml \v{26}to bring rain upon a land without inhabitants, \\
\poemll    a desert in which no human beings live, \\
\poeml \v{27}to satisfy a desolate and devastated desert, \\
\poemll    causing it to sprout vegetation? \\
\poeml \v{28}``Does the rain have a father? \\
\poemll    Who fathered the dew? \\
\poeml \v{29}Whose womb brings forth the ice? \\
\poemll    Who gives birth to frost out of an empty\fnote{\fbackref{38:29} The Heb. lacks \fbib{an empty}} sky, \\
\poeml \v{30}when water solidifies\fnote{\fbackref{38:30} Or \fbib{harden}} like stone \\
\poemll    and the surface of the deepest sea freezes?
\passage{On the Heavens}
\poeml \v{31}``Can you bind the chains of Pleiades \\
\poemll    or loosen the cords of Orion? \\
\poeml \v{32}Can you bring out constellations in their season? \\
\poemll    Can you guide the Bear with her cubs? \\
\poeml \v{33}Do you know the laws of the heavens? \\
\poemll    Can you regulate their authority over the earth? \\
\poeml \v{34}``Can you call out to the clouds, \\
\poemll    so that abundant water drenches you? \\
\poeml \v{35}Can you command the lightning, \\
\poemll    so that it goes forth and calls to you, `Look at us!'\fnote{\fbackref{38:35} Lit. \fbib{Here we are}} \\
\poeml \v{36}``Who sets wisdom within you, \\
\poemll    or imbues your mind with understanding? \\
\poeml \v{37}Who has the wisdom to be able to count the clouds, \\
\poemll    or to empty\fnote{\fbackref{38:37} Lit. \fbib{cause to rest}, \fbib{lie down}} the water jars of heaven, \\
\poeml \v{38}when dust dries into a mass \\
\poemll    and then breaks apart into clods?
\passage{On the Animal World}
\poeml \v{39}``Can you hunt prey for the lioness \\
\poemll    to satisfy young lions \\
\poeml \v{40}when they crouch in their dens \\
\poemll    and lie in ambush in their lairs? \\
\poeml \v{41}Who prepares food for the raven, \\
\poemll    when its offspring cry out to God \\
\poemlll       as they wander for lack of food?''
\end{poetry}
\labelchapt{39}
\passage{On the Birth of Young}

\begin{poetry}
\poeml \chapt{39}
\v{1}``Do you know when the mountain goat gives birth? \\
\poemll    Do you watch the doe as it calves its young? \\
\poeml \v{2}Can you count the months of their gestation? \\
\poemll    Do you know the time when they give birth, \\
\poeml \v{3}when they crouch down\fnote{\fbackref{39:3} Or \fbib{bow down}} to give birth\fnote{\fbackref{39:3} Lit. \fbib{cleave open}} to their offspring, \\
\poemll    and let go\fnote{\fbackref{39:3} Lit. \fbib{send}} of their birth pangs? \\
\poeml \v{4}Their young are strong; \\
\poemll    they grow up in the open field; \\
\poeml then they go off \\
\poemll    and don't return to them.''
\passage{On Wild Animals}
\poeml \v{5}``Who sets the wild donkey free? \\
\poemll    Who loosens the bonds of the wild donkey \\
\poeml \v{6}to whom I've given the Arabah\fnote{\fbackref{39:6} I.e. the desert wilderness of southern Israel} for a home; \\
\poemll    the salt plain for his dwelling place? \\
\poeml \v{7}He despises city noises;\fnote{\fbackref{39:7} Or \fbib{sound}} \\
\poemll    he ignores the shouts\fnote{\fbackref{39:7} Or \fbib{noise}} of the driver. \\
\poeml \v{8}He ranges the mountains that are his pasture \\
\poemll    to search for anything green. \\
\poeml \v{9}Is the wild ox willing to serve you? \\
\poemll    Will he sleep at night near your feeding trough? \\
\poeml \v{10}Can you bind the ox to plow a furrow with a rope? \\
\poemll    Will he harrow after you in the valley? \\
\poeml \v{11}Will you trust him because of his great strength \\
\poemll    and entrust your labor to him? \\
\poeml \v{12}Will you trust him that he'll bring in your grain, \\
\poemll    and gather it to your threshing floor?''
\passage{On the Ostrich}
\poeml \v{13}``The wings of the ostrich flap joyously, \\
\poemll    but aren't its pinions and feathers like the stork? \\
\poeml \v{14}She abandons her eggs on the ground \\
\poemll    and lets them be warmed in the sand, \\
\poeml \v{15}but she forgets that a foot might crush them \\
\poemll    or any wild animal might trample them. \\
\poeml \v{16}She mistreats her young as though they're not hers, \\
\poemll    and she has no fear that her labor may be in vain, \\
\poeml \v{17}because God didn't grant her wisdom \\
\poemll    and never gave her understanding. \\
\poeml \v{18}And yet when she gets ready to run, \\
\poemll    she laughs at the horse and its rider.''
\passage{On the Horse}
\poeml \v{19}Do you instill the horse with strength? \\
\poemll    Do you clothe its neck with a mane? \\
\poeml \v{20}Can you make him leap like the locust, \\
\poemll    and make the splendor of his snorting terrifying? \\
\poeml \v{21}He paws the ground\fnote{\fbackref{39:21} The Heb. lacks \fbib{the ground}} in the valley \\
\poemll    and rejoices in his strength; \\
\poemlll       he goes out to face weapons. \\
\poeml \v{22}He scoffs at fear \\
\poemll    and is never scared; \\
\poemlll       he never retreats from a sword. \\
\poeml \v{23}A quiver of arrows rattles against his side, \\
\poemll    along with a flashing spear and a lance. \\
\poeml \v{24}Leaping in his excitement, he takes in\fnote{\fbackref{39:24} Lit. \fbib{swallows}} the ground; \\
\poemll    he cannot stand still when the trumpets sound! \\
\poeml \v{25}When the trumpet blasts he'll neigh, `Aha! Aha!' \\
\poemll    From a distance he can sense war, \\
\poemlll       the war cry of generals,\fnote{\fbackref{39:25} Or \fbib{officers}} and their shouting.''
\passage{On Raptors}
\poeml \v{26}``Is it by your understanding that the hawk flies, \\
\poemll    spreading its wings toward the south? \\
\poeml \v{27}Does the eagle soar high at your command\fnote{\fbackref{39:27} Lit. \fbib{mouth}} \\
\poemll    and build its nest on the highest crags? \\
\poeml \v{28}He dwells on the crags where he makes his home, \\
\poemll    there on the rocky crag is his stronghold. \\
\poeml \v{29}From there he searches for prey, \\
\poemll    and his eyes recognize it from a distance. \\
\poeml \v{30}His young ones feast\fnote{\fbackref{39:30} Lit. \fbib{suck up}} on blood; \\
\poemll    he'll be found wherever there's a carcass.''\fnote{\fbackref{39:30} Or \fbib{slain}}
\end{poetry}
\labelchapt{40}
\passage{The \divine{Lord} Challenges Job Again}

\chapt{40}
\v{1}The \divine{Lord} continued his response to Job by saying:

\begin{poetry}
\poeml \v{2}``Should the one who is fighting the Almighty find fault with him?\fnote{\fbackref{40:2} The Heb. lacks \fbib{him}} \\
\poemll    Let God's accuser answer.''
\end{poetry}
\passage{Job Acknowledges His Limitations}

\v{3}Then Job replied to the \divine{Lord}. He said:

\begin{poetry}
\poeml \v{4}``I must look insignificant to you! \\
\poemll    How can I answer you? \\
\poemlll       I'm speechless.\fnote{\fbackref{40:4} Lit. \fbib{I put my hand over my mouth}} \\
\poeml \v{5}I spoke once, \\
\poemll    but I can't answer; \\
\poeml I tried\fnote{\fbackref{40:5} The Heb. lacks \fbib{tried}} a second time, \\
\poemll    but I won't do so anymore.''
\end{poetry}
\passage{The \divine{Lord} Continues to Interrogate Job}

\v{6}The \divine{Lord} answered Job from the wind storm and told him:

\begin{poetry}
\poeml \v{7}``Stand up\fnote{\fbackref{40:7} Lit. \fbib{Gird up your loins}} like a man! \\
\poemll    I'll ask you some questions, \\
\poemlll       and you give me some answers! \\
\poeml \v{8}Indeed would you annul my justice and condemn me, \\
\poemll    just so you can claim that you're righteous? \\
\poeml \v{9}Do you have strength\fnote{\fbackref{40:9} Lit. \fbib{have an arm}} like God? \\
\poemll    Can you create thunder with a sound\fnote{\fbackref{40:9} Lit. \fbib{voice}} like he can?''
\passage{Can You Save Yourself?}
\poeml \v{10}``When you have adorned yourself with exalted majesty, \\
\poemll    clothed yourself with splendor and dignity,\fnote{\fbackref{40:10} Lit. \fbib{splendor} and \fbib{majesty}} \\
\poeml \v{11}dispensed the fury of your anger, \\
\poemll    made sure\fnote{\fbackref{40:11} Lit. \fbib{see}} that you have humbled every proud person, \\
\poeml \v{12}stared down and subdued every proud person, \\
\poemll    trampled the wicked right where they are, \\
\poeml \v{13}buried\fnote{\fbackref{40:13} MT has \fbib{hide}} them in the dust together, \\
\poemll    and sent them bound to that secret place,\fnote{\fbackref{40:13} I.e. the afterlife} \\
\poeml \v{14}then I will applaud you myself! \\
\poemll    I'll admit that you can deliver yourself by your own efforts!''
\passage{On Behemoth}
\poeml \v{15}``Please observe\fnote{\fbackref{40:15} Lit. \fbib{look}} Behemoth,\fnote{\fbackref{40:15} I.e. an ancient, gigantic herbivore, living in Job's time but now apparently extinct} which I made along with you. \\
\poemll    He eats grass like an ox. \\
\poeml \v{16}Now take a look at the strength that he has in his loins, \\
\poemll    and in the muscles of his abdomen. \\
\poeml \v{17}His tail protrudes stiffly, like cedar;\fnote{\fbackref{40:17} I.e. a genus of coniferous evergreen in the family \fbib{Pinaceae}} \\
\poemll    the sinews of his thigh interlink for strength. \\
\poeml \v{18}His bones are conduits\fnote{\fbackref{40:18} Or \fbib{channels}} of bronze;\fnote{\fbackref{40:18} Or \fbib{copper}} \\
\poemll    his strong bones are like bars of iron. \\
\poeml \v{19}He is the grandest\fnote{\fbackref{40:19} Or \fbib{first}} of God's undertakings,\fnote{\fbackref{40:19} Lit. \fbib{ways}} \\
\poemll    yet his creator is approaching him with his sword.\fnote{\fbackref{40:19} I.e. God was about to make Behemoth extinct} \\
\poeml \v{20}Mountains produce food for him, \\
\poemll    where all the wild animals frolic. \\
\poeml \v{21}He lies under the lotus trees, \\
\poemll    hiding under reeds and marshes.\fnote{\fbackref{40:21} Lit. reed and marsh} \\
\poeml \v{22}The lotus trees cover him with their shade, \\
\poemll    and willows that line the wadis\fnote{\fbackref{40:22} I.e. seasonal streams that channel water during rain seasons but are dry at other times} surround him. \\
\poeml \v{23}What you see as a raging river doesn't alarm him; \\
\poemll    he is confident when the Jordan overflows. \\
\poeml \v{24}Are your eyes looking to capture him, \\
\poemll    or to pierce his snout with a bridle?''
\end{poetry}
\labelchapt{41}
\passage{On Leviathan}

\begin{poetry}
\poemll    \chapt{41}
\v{1}\fnote{\fbackref{41:1} This v. is 40:25 in MT, v2 is 40:26 in MT, and so through v8.}``Can you draw Leviathan\fnote{\fbackref{41:1} I.e. an ancient, gigantic sea creature, living in Job's time but now apparently extinct} out of the water\fnote{\fbackref{41:1} The Heb. lacks \fbib{of the water}} with a hook, \\
\poemll    or tie down\fnote{\fbackref{41:1} Lit. \fbib{or sink}} his tongue with a rope? \\
\poeml \v{2}Can you attach a bridle\fnote{\fbackref{41:2} Lit. \fbib{rope}} to his snout, \\
\poemll    or pierce his jaw with a hook? \\
\poeml \v{3}Will he make many supplications to you, \\
\poemll    or will he beg you for mercy?\fnote{\fbackref{41:3} Lit. \fbib{you with gentle words}} \\
\poeml \v{4}Will he try to make a deal with you, \\
\poemll    so that you may take him in servitude forever? \\
\poeml \v{5}``Will you play with him like a pet bird? \\
\poemll    Will you put a leash on him for your little girls? \\
\poeml \v{6}Will your business be able to buy him, \\
\poemll    Will you divide him among your merchant friends? \\
\poeml \v{7}Will you fill his flesh with harpoons, \\
\poemll    or his head with lances? \\
\poeml \v{8}Lay your hand on him, \\
\poemll    and you'll remember the struggle. \\
\poemlll       You'll never do that again! \\
\poeml \v{9}``Look! Anyone's hope to capture him\fnote{\fbackref{41:9} The Heb. lacks \fbib{to capture him}} will prove itself false; \\
\poemll    anyone would be terrified\fnote{\fbackref{41:9} Or \fbib{subdued}} just by looking at him. \\
\poeml \v{10}No one is fierce enough to dare to arouse him.
\end{poetry}

\begin{poetry}
\poeml ``Who, then, can stand in my presence and face me? \\
\poeml \v{11}Who can take me to court and be reconciled to me? \\
\poemlll       All of heaven is mine. \\
\poeml \v{12}``I won't be silent concerning his limbs, \\
\poemll    his mighty strength, and orderly frame. \\
\poeml \v{13}Who can strip off his outer armor?\fnote{\fbackref{41:13} Lit. \fbib{clothing}} \\
\poemll    Who can approach him with a bridle? \\
\poeml \v{14}Who dares to open his mouth,\fnote{\fbackref{41:14} Lit. \fbib{door of his face}} \\
\poemll    since it is ringed with his terrible teeth! \\
\poeml \v{15}His protective scales are his pride, \\
\poemll    they lie sealed tightly together. \\
\poeml \v{16}Each one is so close to the other \\
\poemll    that not even air comes in between them. \\
\poeml \v{17}Each is attached to the other,\fnote{\fbackref{41:17} Lit. \fbib{with his brother}} \\
\poemll    grasping each other so they cannot be separated. \\
\poeml \v{18}``His snorting releases flashes of light; \\
\poemll    his eyes are like the rays\fnote{\fbackref{41:18} Lit. \fbib{eyelids}} of the dawn. \\
\poeml \v{19}Flames blaze from his mouth; \\
\poemll    streams of sparking fire fly out. \\
\poeml \v{20}Smoke billows from his nostrils; \\
\poemll    like a boiling pot or burning reeds. \\
\poeml \v{21}His breath can ignite coal; \\
\poemll    and flames proceed from his mouth. \\
\poeml \v{22}``His neck is so powerful \\
\poemll    that all who meet him are terrified. \\
\poeml \v{23}There is no flaw in his body's armor; \\
\poemll    it is firmly fixed on him and unbreachable. \\
\poeml \v{24}His heart is as strong as stone, \\
\poemll    it is as hard as a lower millstone. \\
\poeml \v{25}When he rears up, the mighty are terrified; \\
\poemll    they are bewildered as he thrashes about. \\
\poeml \v{26}``Thrusting at him with a sword won't be effective, \\
\poemll    nor will spears, darts, or javelins. \\
\poeml \v{27}He regards iron like straw, \\
\poemll    and hardened bronze like a dead tree. \\
\poeml \v{28}Arrows won't make him flee; \\
\poemll    stones from a sling are only pebbles to him. \\
\poeml \v{29}Clubs are like twigs;\fnote{\fbackref{41:29} Lit. \fbib{stubble}} \\
\poemll    he laughs at the whoosh of the javelin. \\
\poeml \v{30}``Beneath him he is armored as with sharp potsherds; \\
\poemll    he tears through muddy ground \\
\poemlll       like a threshing sledge through grain.\fnote{\fbackref{41:30} The Heb. lacks \fbib{through grain}} \\
\poeml \v{31}He causes the deep to boil like water in\fnote{\fbackref{41:31} The Heb. lacks \fbib{water in}} a pot, \\
\poemll    and churns the sea like one stirs ointment. \\
\poeml \v{32}The sea is luminescent behind him; \\
\poemll    his wake turns the sea white, like those with gray hair. \\
\poeml \v{33}``There's nothing like him on earth; \\
\poemll    he was created without the ability to fear. \\
\poeml \v{34}He looks down on everything that is high; \\
\poemll    he rules over every kind\fnote{\fbackref{41:34} Lit. \fbib{son}} of pride.''
\end{poetry}
\labelchapt{42}
\passage{Job Repents and is Restored}

\chapt{42}
\v{1}Job replied to the \divine{Lord}:

\begin{poetry}
\poeml \v{2}``I know\fnote{\fbackref{42:2} Or \fbib{You know that I know}} that you can do anything \\
\poemll    and nothing that you plan is impossible. \\
\poeml \v{3}You asked,\fnote{\fbackref{42:3} The Heb. lacks \fbib{You asked}} `Who is this that darkens counsel without knowledge?' \\
\poemll    Well now, I have talked about what I don't understand--- \\
\poemlll       awesome things beyond me that I don't know. \\
\poeml \v{4}Listen now, and I will speak for myself; \\
\poemll    I'll interrogate you and then inform me. \\
\poeml \v{5}I've heard you with my ears; \\
\poemll    and now I've seen you with my eyes. \\
\poeml \v{6}As a result, I despise myself and repent \\
\poemll    in dust and ashes.''
\end{poetry}
\passage{Job's Friends are Restored}

\v{7}After these words had been spoken by the \divine{Lord} to Job, the \divine{Lord} spoke to Eliphaz from Teman: ``My anger is burning against you along with your two friends, since you haven't spoken correctly about me, as did my servant Job. \v{8}So take seven bulls and seven rams and bring them to my servant Job. And bring a whole burnt offering for yourselves and my servant Job will pray for you. I'll encourage him\fnote{\fbackref{42:8} Lit. \fbib{I'll lift his face}} by not responding as your disgraceful folly deserves, since you didn't speak about me correctly as did my servant Job. \v{9}So Eliphaz from Teman, Bildad from Shuah, and Zophar from Naamath did precisely as the \divine{Lord} had spoken to them, because the \divine{Lord} showed favor to\fnote{\fbackref{42:9} Lit. \fbib{lift his face}} Job.
\passage{Job's Prosperity Returns}

\v{10}The \divine{Lord} restored Job's prosperity after he prayed for his friends. The \divine{Lord} doubled everything that Job had once possessed. \v{11}Then all his brothers and sisters and all those who knew him before arrived. They ate food with him in his house, mourned for him, and consoled him for all the trouble that the \divine{Lord} had brought and placed on him. Some\fnote{\fbackref{42:11} Lit. \fbib{A man}} gave him gold bullion\fnote{\fbackref{42:11} Lit. \fbib{him one kesitah}; a unit of gold weight, the value of which is unknown today} and some brought\fnote{\fbackref{42:11} Lit. \fbib{and a man}} gold earrings.

\v{12}The \divine{Lord} blessed Job during the latter part of his life\fnote{\fbackref{42:12} The Heb. lacks \fbib{part of his life}} more than the former, since he owned 14,000 sheep, 6,000 camels, 1,000 teams of oxen\fnote{\fbackref{42:12} Or \fbib{1,000 pairs of cattle}} and 1,000 female donkeys. \v{13}He also had seven sons and three daughters. \v{14}He named the first daughter Jemima,\fnote{\fbackref{42:14} The name means \fbib{day by day}} the second Keziah,\fnote{\fbackref{42:14} The name means \fbib{cinnamon}} and the name of the third was Keren-happuch.\fnote{\fbackref{42:14} The name means \fbib{power of antimony}; i.e. an element valued for medicinal uses} \v{15}No one could find more beautiful women in the whole land than Job's daughters. Their father gave them their inheritance along with their brothers. \v{16}Job lived 140 years after this, and saw his children and grandchildren to the fourth generation. \v{17}Then Job died at an old age, having lived a full life.\fnote{\fbackref{42:17} Lit. \fbib{died old and full of days}}

\addcontentsline{toc}{chapter}{Poetry}
\renewcommand{\verseone}{\textsuperscript{1}}
\bookheader{Psalms}
\labelbook{Ps}

\bookpretitle{The Book of}
\booktitle{Psalms}

\booksection{BOOK I (Psalms 1-41)}
\labelpsalm{1}
\passage{The Righteous and the Wicked\passagenote{Note: (1) Verse numbers may be different from MT because the titles of many psalms in MT are part of the first verse. (2) The phrase \fbib{A song of}, which appears in many psalm titles, may also be translated \fbib{A song by, A song for,} or \fbib{A song to}. (3) Psalm title terminologies ``a psalm of David,'' ``a song of David,'' etc., may connote---but do not necessarily connote---authorship by Israel's King David. They are rendered herein as ``a Davidic psalm,'' ``a Davidic song,'' etc. (4) The traditionally unpronounced literary term \fbib{Selah}, which may indicate that the oral reader or cantor is to pause briefly after reading the line in which the term appears, is rendered herein as \fbib{Interlude}.}}

\begin{poetry}
\poeml \v{1}How blessed is the person, \\
\poemll    who does not take\fnote{Lit. \fbib{not walk by}} the advice of the wicked, \\
\poeml who does not stand on the path with sinners, \\
\poemll    and who does not sit in the seat of mockers. \\
\poeml \v{2}But he delights in the \divine{Lord}'s instruction,\fnote{Or \fbib{Law}} \\
\poemll    and meditates in his instruction\fnote{Or \fbib{Law}} day and night. \\
\poeml \v{3}He will be like a tree planted by streams of water, \\
\poemll    yielding its fruit in its season, \\
\poemlll       and whose leaf does not wither. \\
\poeml He will prosper in everything he does. \\
\poeml \v{4}But this is not the case with the wicked. \\
\poemll    They are like chaff that the wind blows away. \\
\poeml \v{5}Therefore the wicked will not escape\fnote{Lit. \fbib{stand in the}} judgment, \\
\poemll    nor will sinners have a place\fnote{The Heb. lacks \fbib{have a place in}} in the assembly of the righteous. \\
\poeml \v{6}For the \divine{Lord} knows the way of the righteous, \\
\poemll    but the way of the wicked will be destroyed.
\end{poetry}
\labelpsalm{2}
\passage{The Nations and God's Anointed}

\begin{poetry}
\poeml \v{1}Why are the nations in an uproar, \\
\poemll    and their people involved in a vain plot? \\
\poeml \v{2}As the kings of the earth take their stand \\
\poemll    and the rulers conspire together against the \divine{Lord} and his anointed one, they say,\fnote{The Heb. lacks \fbib{say}} \\
\poeml \v{3}``Let us tear off their shackles from us,\fnote{The Heb. lacks \fbib{from us}} \\
\poemll    and cast off their chains.'' \\
\poeml \v{4}He who sits in the heavens laughs; \\
\poemll    the Lord scoffs at them. \\
\poeml \v{5}In his anger he rebukes them, \\
\poemll    and in his wrath he terrifies them: \\
\poeml \v{6}``I have set my king on Zion, \\
\poemll    my holy mountain.''
\passage{The Anointed King Speaks}
\poeml \v{7}Let me announce the decree of the \divine{Lord} \\
\poemll    that he told me: \\
\poeml ``You are my son, \\
\poemll    today I have become your father. \\
\poeml \v{8}Ask of me, and I will give you \\
\poemll    the nations as your inheritance, \\
\poemlll       the ends of the earth as your possession. \\
\poeml \v{9}You will break them with an iron rod, \\
\poemll    you will shatter them like pottery.'' \\
\poeml \v{10}Therefore, kings, act wisely! \\
\poemll    Earthly rulers, be warned! \\
\poeml \v{11}Serve the \divine{Lord} with fear, \\
\poemll    and rejoice with trembling. \\
\poeml \v{12}Kiss\fnote{Or \fbib{Worship}} the son before he becomes\fnote{Or \fbib{son lest he become}} angry, \\
\poemll    and you die where you stand.\fnote{Lit. \fbib{you perish in the way}} \\
\poeml Indeed, his wrath can flare up quickly. \\
\poeml How blessed are those who take refuge in him.
\end{poetry}
\labelpsalm{3}
\psalminfo{A Davidic Psalm, when he fled from his son Absalom.}
\passage{God Delivers His Servants}

\begin{poetry}
\poeml \v{1}\divine{Lord}, I have so many persecutors! \\
\poemll    Many are rising up against me! \\
\poeml \v{2}Many are saying about me, \\
\poemll    ``God will never deliver him!''
\end{poetry}
\interlude{Interlude}

\begin{poetry}
\poeml \v{3}But you, \divine{Lord}, are a shield around me, \\
\poemll    my glory, and the one who lifts up my head. \\
\poeml \v{4}I cry aloud\fnote{Lit. \fbib{with my voice}} to the \divine{Lord}, \\
\poemll    and he answers me from his holy mountain.
\end{poetry}
\interlude{Interlude}

\begin{poetry}
\poeml \v{5}I lie down and sleep, \\
\poemll    I wake up, because the \divine{Lord} sustains me. \\
\poeml \v{6}I will not fear multitudes of\fnote{Or \fbib{ten thousand}} people, \\
\poemll    who set themselves against me on every side. \\
\poeml \v{7}Arise, \divine{Lord}! \\
\poemll    Deliver me, my God! \\
\poeml For you strike the jaw of all my enemies, \\
\poemll    and you break the teeth of the wicked. \\
\poeml \v{8}Deliverance comes from the \divine{Lord}! \\
\poemll    May your blessing be on your people.
\end{poetry}
\interlude{Interlude}
\labelpsalm{4}
\psalminfo{To the Director: With stringed instruments. A Davidic Psalm}
\passage{Trust God under Adversity}

\begin{poetry}
\poeml \v{1}When I call, answer me, \\
\poemll    my righteous God!\fnote{Or \fbib{God of my righteousness}} \\
\poeml When I was in distress, you set me free. \\
\poemll    Be gracious to me and hear my prayer. \\
\poeml \v{2}You people, \\
\poemll    how long will you malign my reputation? \\
\poeml How long will you love what is vain\fnote{I.e. what has no substance} \\
\poemll    and what is false?
\end{poetry}
\interlude{Interlude}

\begin{poetry}
\poeml \v{3}But understand this:\fnote{The Heb. lacks \fbib{this}} \\
\poemll    the \divine{Lord} has set apart the godly for himself! \\
\poemlll       The \divine{Lord} will hear me when I cry out to him! \\
\poeml \v{4}Be angry, yet do not sin.\fnote{Cf. Eph 4:26} \\
\poemll    Think about this\fnote{The Heb. lacks \fbib{this}} when upon your beds, \\
\poemlll       and be silent.
\end{poetry}
\interlude{Interlude}

\begin{poetry}
\poeml \v{5}Offer sacrifices that are righteous, \\
\poemll    and put your confidence in the \divine{Lord}. \\
\poeml \v{6}Many are asking, ``Who will help us to see better days?''\fnote{Lit. \fbib{cause us to see good}} \\
\poemll    \divine{Lord}, may the light of your favor\fnote{Lit. \fbib{face}} shine upon us. \\
\poeml \v{7}You have given me more joy in my heart than at harvest times, \\
\poemll    when grain and wine abound. \\
\poeml \v{8}I will lie down and sleep in peace, \\
\poemll    for you alone, \divine{Lord}, enable me to live securely.
\end{poetry}
\labelpsalm{5}
\psalminfo{To the Director: For flutes. A Davidic Psalm}
\passage{A Prayer for God's Help}

\begin{poetry}
\poeml \v{1}\divine{Lord}, listen to my words, \\
\poemll    consider my groaning. \\
\poeml \v{2}Pay attention to my cry for help,\fnote{Lit. \fbib{the sound of my cry for help}} \\
\poemll    my king and my God, \\
\poemlll       for unto you will I pray. \\
\poeml \v{3}\divine{Lord}, in the morning you will hear my voice; \\
\poemll    in the morning I will pray\fnote{Lit. \fbib{arrange my prayer}} to you, \\
\poemll    and I will watch for your answer.\fnote{The Heb. lacks \fbib{for your answer}} \\
\poeml \v{4}Indeed, you aren't a God who delights in wickedness; \\
\poemll    evil will never dwell with you. \\
\poeml \v{5}Boastful ones will not stand before you; \\
\poemll    you hate all those who practice wickedness. \\
\poeml \v{6}You will destroy those who speak lies. \\
\poemll    The \divine{Lord} abhors the person of bloodshed and deceit. \\
\poeml \v{7}But I, because of the abundance of your gracious love, \\
\poemll    may come into your house. \\
\poemlll       In awe of you, I will worship in your holy Temple. \\
\poeml \v{8}\divine{Lord}, lead me in your righteousness because of my enemies. \\
\poemll    Make your path straight before me. \\
\poeml \v{9}But as for the wicked,\fnote{The Heb. lacks \fbib{as for the wicked}} \\
\poemll    they do not speak truth at all. \\
\poemlll       Inside them there is only wickedness. \\
\poeml Their throat is an open grave, \\
\poemll    on their tongue is deceitful flattery. \\
\poeml \v{10}Declare them guilty, God! \\
\poemll    Let them fall by their own schemes. \\
\poeml Drive them away because of their many transgressions, \\
\poemll    for they have rebelled against you. \\
\poeml \v{11}Let all those who take refuge in you rejoice! \\
\poemll    Let them shout for joy forever, \\
\poeml and may you protect them. \\
\poemll    Let those who love your name exult in you. \\
\poeml \v{12}Indeed, you will bless the righteous one, \divine{Lord}, \\
\poemll    like a large shield, you will surround him with favor.
\end{poetry}
\labelpsalm{6}
\psalminfo{To the Director: With stringed instruments. On an eight-stringed harp.\fnote{T Or \fbib{On a lower octave}} A Davidic Psalm}
\passage{A Prayer in Times of Trouble}

\begin{poetry}
\poeml \v{1}\divine{Lord}, in your anger, do not rebuke me, \\
\poemll    in your wrath, do not discipline me. \\
\poeml \v{2}Be gracious to me, \divine{Lord}, \\
\poemll    because I am fading away. \\
\poeml Heal me, \\
\poemll    because my body\fnote{Or \fbib{bones}} is distressed. \\
\poeml \v{3}And my soul\fnote{Or \fbib{And I am}} is deeply distressed. \\
\poemll    But you, \divine{Lord}, how long do I wait?\fnote{The Heb. lacks \fbib{do I wait}} \\
\poeml \v{4}Return, \divine{Lord}, \\
\poemll    save my life! \\
\poemlll       Deliver me, because of your gracious love. \\
\poeml \v{5}In death, there is no memory of you. \\
\poemll    Who will give you thanks where the dead are?\fnote{Lit. \fbib{thanks in Sheol}; a reference to the realm of the dead} \\
\poeml \v{6}I am weary from my groaning. \\
\poemll    Every night my couch is drenched with tears, \\
\poemlll       my bed is soaked through. \\
\poeml \v{7}My eyesight has faded because of grief, \\
\poemll    it has dimmed because of all my enemies. \\
\poeml \v{8}Get away from me, all of you who practice evil, \\
\poemll    for the \divine{Lord} has heard the sound of my weeping. \\
\poeml \v{9}The \divine{Lord} has heard my plea; \\
\poemll    the \divine{Lord} receives my prayer. \\
\poeml \v{10}As for all my enemies, they will be put to shame; \\
\poemll    they will be greatly frightened \\
\poemlll       and suddenly turn away ashamed.
\end{poetry}
\labelpsalm{7}
\psalminfo{A Davidic psalm,\fnote{T Heb. \fbib{Shiggaion}} which he sang to the \divine{Lord}, because of the words of Cush the descendant of Benjamin.}
\passage{A Prayer for Vindication}

\begin{poetry}
\poeml \v{1}\divine{Lord}, my God, \\
\poemll    I seek refuge in you. \\
\poeml Deliver me from those who persecute me! \\
\poemll    Rescue me! \\
\poeml \v{2}Otherwise, they will rip me to shreds like a lion, \\
\poemll    tearing me\fnote{The Heb. lacks \fbib{me}} apart with no one to rescue me.\fnote{The Heb. lacks \fbib{me}} \\
\poeml \v{3}\divine{Lord}, my God, if I have done this thing, \\
\poemll    if there is injustice on my hands, \\
\poeml \v{4}if I have rewarded those who did me good with evil, \\
\poemll    if I have plundered my enemy without justification, \\
\poeml \v{5}then, let my enemy pursue me, \\
\poemll    let him overtake me, \\
\poemlll       and let him trample my life to the ground.
\end{poetry}
\interlude{Interlude}

\begin{poetry}
\poeml Let him put my honor into the dust. \\
\poeml \v{6}Get up, \divine{Lord}, in your anger! \\
\poemll    Rise up, because of the fury of my enemies; \\
\poeml Arouse yourself for me; \\
\poemll    you have ordained justice. \\
\poeml \v{7}Let the assembly of the peoples gather around you, \\
\poemll    and you will sit\fnote{Lit. \fbib{return}} high above them. \\
\poeml \v{8}For the \divine{Lord} will judge the peoples. \\
\poemll    Judge me according to my righteousness, \divine{Lord}, \\
\poemlll       and according to my integrity, Exalted One. \\
\poeml \v{9}Let the evil of the wicked come to an end, \\
\poemll    but establish the righteous. \\
\poeml For you are the righteous God \\
\poemll    who discerns the inner thoughts.\fnote{Lit. \fbib{hearts and innards}} \\
\poeml \v{10}God is my shield,\fnote{Lit. \fbib{My shield is on God}} \\
\poemll    the one who delivers the upright in heart. \\
\poeml \v{11}God is a righteous judge, \\
\poemll    a God who is angry with sinners\fnote{The Heb. lacks \fbib{sinners}} every day. \\
\poeml \v{12}If the ungodly one\fnote{Lit. \fbib{If he}} doesn't repent, \\
\poemll    God will sharpen his sword; \\
\poemlll       he will string his bow and prepare it. \\
\poeml \v{13}He prepares weapons of death for himself, \\
\poemll    he makes his arrows into fiery shafts. \\
\poeml \v{14}But the wicked one\fnote{Lit. \fbib{But he}} travails with evil, \\
\poemll    he conceives malice and gives birth to lies. \\
\poeml \v{15}He digs a pit, even excavates it; \\
\poemll    then he fell into the hole that he had made. \\
\poeml \v{16}The trouble\fnote{Lit. \fbib{His trouble}} he planned will return on his own head, \\
\poemll    and his violence will descend on his skull. \\
\poeml \v{17}But as for me, \\
\poemll    I will praise the \divine{Lord} for his righteousness, \\
\poemlll       and I will sing to the name of the \divine{Lord} Most High.
\end{poetry}
\labelpsalm{8}
\psalminfo{To the Director: On a stringed instrument.\fnote{T Or \fbib{according to a Gittite melody}} A Davidic Psalm.}
\passage{Divine Glory and Human Dignity}

\begin{poetry}
\poeml \v{1}\divine{Lord}, our Lord, \\
\poemll    how excellent is your name in all the earth! \\
\poemlll       You set your glory above the heavens! \\
\poeml \v{2}Out of the mouths of infants and nursing babies \\
\poemll    you have established strength\fnote{LXX reads \fbib{praise}} \\
\poemlll       on account of your adversaries, \\
\poeml in order to silence the enemy and vengeful foe. \\
\poeml \v{3}When I look at the heavens, \\
\poemll    the work of your fingers, \\
\poemlll       the moon and the stars that you established--- \\
\poeml \v{4}what is man that you take notice of him, \\
\poemll    or the son of man\fnote{A title of Messiah (cf. Dan 7:13-14) or a Heb. synonym for a human being (cf. Dan 8:17)} that you pay attention to him? \\
\poeml \v{5}You made him a little less than divine,\fnote{Or \fbib{God}; or \fbib{gods}; or \fbib{than heavenly beings}} \\
\poemll    but you crowned him with glory and honor. \\
\poeml \v{6}You gave him dominion over the work of your hands, \\
\poemll    you put all things under his feet: \\
\poeml \v{7}Sheep and cattle---all of them, \\
\poemll    wild creatures of the field, \\
\poeml \v{8}birds in the sky, \\
\poemll    fish in the sea--- \\
\poemlll       whatever moves through the currents of the oceans. \\
\poeml \v{9}\divine{Lord}, our Lord, \\
\poemll    how excellent is your name in all the earth!
\end{poetry}
\labelpsalm{9}
\psalminfo{To the Director: Accompanied by female voices.\fnote{T Or \fbib{according to the tune `Death of a Son'}} A Davidic Psalm.}
\passage{A Cry for God's Justice}

\begin{poetry}
\poeml \v{1}\fnote{Psalms 9 \& 10 constitute a single psalm in the LXX.}I will give thanks to the \divine{Lord} with all my heart, \\
\poemll    I will declare all your wonderful deeds. \\
\poeml \v{2}I will be glad and exult in you; \\
\poemll    I will sing praises to your name, Most High! \\
\poeml \v{3}When my enemies turn back, \\
\poemll    they will stumble and perish before you. \\
\poeml \v{4}For you have brought about justice for me and my cause; \\
\poemll    you sit on the throne judging righteously. \\
\poeml \v{5}You rebuked the nations, \\
\poemll    you destroyed the wicked, \\
\poemlll       you wiped out their name forever and ever. \\
\poeml \v{6}The enemy has perished, \\
\poemll    reduced to ruins forever. \\
\poeml You uprooted their cities, \\
\poemll    the very memory of them vanished. \\
\poeml \v{7}But the \divine{Lord} sits on his throne\fnote{The Heb. lacks \fbib{on his throne}} forever; \\
\poemll    his throne is established for judgment. \\
\poeml \v{8}He will judge the world righteously \\
\poemll    and make just decisions for the people. \\
\poeml \v{9}The \divine{Lord} is a refuge for the oppressed, \\
\poemll    a refuge in times of distress. \\
\poeml \v{10}Those who know your name will trust you, \\
\poemll    for you have not forsaken those who seek you, \divine{Lord}. \\
\poeml \v{11}Sing praises to the \divine{Lord} who dwells in Zion; \\
\poemll    declare his mighty deeds among the peoples. \\
\poeml \v{12}As an avenger of blood, he remembers them; \\
\poemll    he has not forgotten the cry of the afflicted. \\
\poeml \v{13}Be gracious to me, \divine{Lord}, \\
\poemll    take note of my affliction, \\
\poemlll       because of those who hate me. \\
\poeml You snatch me away from the gates of death, \\
\poeml \v{14}so I may declare everything for which you should be praised\fnote{Lit. \fbib{declare all your praise}} \\
\poeml in the gates of the daughter of Zion,\fnote{I.e. Jerusalem} \\
\poemll    so I will rejoice in your deliverance. \\
\poeml \v{15}The nations have sunk down into the pit they made, \\
\poemll    their feet are ensnared in the trap\fnote{Lit. \fbib{net}} they set. \\
\poeml \v{16}The \divine{Lord} has made himself known, \\
\poemll    executing judgment. \\
\poeml The wicked are ensnared \\
\poemll    by what their hands have made.
\end{poetry}
\interlude{Interlude\fnote{Heb. \fbib{Higgaion Selah}}}

\begin{poetry}
\poeml \v{17}The wicked will turn back to where the dead are\fnote{Lit. \fbib{to Sheol}; a reference to the realm of the dead}--- \\
\poemll    all the nations that have forgotten God. \\
\poeml \v{18}For he will not always overlook the plight of the poor, \\
\poemll    nor will the hope of the afflicted perish forever. \\
\poeml \v{19}Rise up, \divine{Lord}, \\
\poemll    do not let man prevail! \\
\poemlll       The nations will be judged in your presence. \\
\poeml \v{20}Make them afraid, \divine{Lord}, \\
\poemll    Let the nations know that they are only human.\fnote{Or \fbib{men}}
\end{poetry}
\interlude{Interlude}
\labelpsalm{10}
\passage{A Prayer for Judging the Wicked}

\begin{poetry}
\poeml \v{1}\fnote{Psalms 9 \& 10 constitute a single psalm in the LXX.}Why do you stand far away, \divine{Lord}? \\
\poemll    Why do you hide in times of distress? \\
\poeml \v{2}The wicked one arrogantly pursues the afflicted,\fnote{Or \fbib{the poor}} \\
\poemll    who are trapped in the schemes he devises. \\
\poeml \v{3}For the wicked one boasts about his own desire; \\
\poemll    he blesses the greedy \\
\poemlll       and despises the \divine{Lord}. \\
\poeml \v{4}With haughty arrogance, the wicked thinks, \\
\poemll    ``God will not seek justice.''\fnote{The Heb. lacks \fbib{justice}} \\
\poemlll       He always presumes ``There is no God.'' \\
\poeml \v{5}Their ways always seem prosperous. \\
\poeml Your judgments are on high, \\
\poemlll       far away from them. \\
\poeml They scoff at all their enemies. \\
\poeml \v{6}They say to themselves, \\
\poemll    ``We will not be moved throughout all time, \\
\poemlll       and we will not experience adversity.'' \\
\poeml \v{7}Their mouth is full of curses, lies, and oppression, \\
\poemll    their tongues\fnote{Lit. \fbib{under his tongue}} spread trouble and iniquity. \\
\poeml \v{8}They wait\fnote{Lit. \fbib{sit}} in ambush in the villages, \\
\poemll    they kill the innocent in secret. \\
\poeml \v{9}Their eyes secretly watch the helpless, \\
\poemll    lying in wait like a lion in his den. \\
\poeml They lie in wait to catch the afflicted. \\
\poemll    They catch the afflicted when they pull him into their net. \\
\poeml \v{10}The victim\fnote{Lit. \fbib{He}} is crushed, \\
\poemll    and he sinks down; \\
\poemlll       the helpless fall by their might. \\
\poeml \v{11}The wicked say to themselves, \\
\poemll    ``God has forgotten, \\
\poeml he has hidden his face, \\
\poemll    he will never see it.'' \\
\poeml \v{12}Rise up, \divine{Lord}! \\
\poemll    Raise your hand, God. \\
\poemlll       Don't forget the afflicted! \\
\poeml \v{13}Why do the wicked despise God \\
\poemll    and say to themselves, ``God\fnote{Lit. \fbib{He}} will not seek justice.''?\fnote{The Heb. lacks \fbib{justice}} \\
\poeml \v{14}But you do see! \\
\poemll    You take note of trouble and grief \\
\poemlll       in order to take the matter into your own hand. \\
\poeml The helpless one commits himself\fnote{The Heb. lacks \fbib{himself}} to you; \\
\poemll    you have been the orphan's helper. \\
\poeml \v{15}Break the arm of the wicked and evil man; \\
\poemll    so that when you seek out his wickedness \\
\poemlll       you will find it no more. \\
\poeml \v{16}The \divine{Lord} is king forever and ever; \\
\poemll    nations will perish from his land. \\
\poeml \v{17}\divine{Lord}, you heard the desire of the afflicted; \\
\poemll    you will strengthen them,\fnote{Lit. \fbib{strengthen their heart}} \\
\poemlll       you will listen carefully, \\
\poeml \v{18}to do justice for the orphan\fnote{Or \fbib{fatherless}} and the oppressed, \\
\poemll    so that men of the earth may cause terror no more.
\end{poetry}
\labelpsalm{11}
\psalminfo{To the Director: A Davidic Song.\fnote{T The Heb. lacks \fbib{A song}}}
\passage{Confident Trust in God}

\begin{poetry}
\poeml \v{1}I take refuge in the \divine{Lord}. \\
\poemll    So how can you say to me, \\
\poemlll       ``Flee like a bird to the mountains.''? \\
\poeml \v{2}Look, the wicked have bent their bow \\
\poemll    and placed their arrow\fnote{So MT DSS 5/6HevPs; DSS 4QCatena A LXX read \fbib{arrows}} on the string,\fnote{So MT; LXX reads \fbib{arrows for the quiver}} \\
\poemlll       to shoot from the darkness\fnote{So MT DSS; LXX reads \fbib{shoot on a moonless night}} at the upright in heart. \\
\poeml \v{3}When the foundations are destroyed, \\
\poemll    what can the righteous do? \\
\poeml \v{4}The \divine{Lord} is in his holy Temple; \\
\poemll    the \divine{Lord}'s throne is in the heavens. \\
\poeml His eyes see, \\
\poemll    his glance\fnote{Lit. \fbib{eyelids}} examines humanity.\fnote{Lit. \fbib{examines the children of men}} \\
\poeml \v{5}The \divine{Lord} examines the righteous, \\
\poemll    but the wicked and those who love violence, he hates. \\
\poeml \v{6}He rains on the wicked burning coals and sulfur; \\
\poemll    a scorching wind is their destiny.\fnote{Lit. \fbib{the portion of their cup}} \\
\poeml \v{7}Indeed, the \divine{Lord} is righteous; \\
\poemll    he loves righteousness; \\
\poemlll       the upright will see him face-to-face.
\end{poetry}
\labelpsalm{12}
\psalminfo{To the Director: On an eight stringed harp.\fnote{T Or \fbib{on a lower octave}} A Davidic Psalm.}
\passage{Human and Divine Words Contrasted}

\begin{poetry}
\poeml \v{1}Help, \divine{Lord}, for godly people no longer exist; \\
\poemll    trustworthy people have disappeared from humanity.\fnote{Lit. \fbib{from among the children of men}} \\
\poeml \v{2}Everyone speaks lies to his neighbor; \\
\poemll    they speak with flattering lips and hidden motives.\fnote{Lit. \fbib{with slippery lips and a double heart}} \\
\poeml \v{3}The \divine{Lord} will cut off all slippery lips, \\
\poemll    and the tongue that boasts great things, \\
\poeml \v{4}those who say, \\
\poemll    ``By our tongues we will prevail; \\
\poemlll       our lips belong to us. \\
\poemll    Who is master\fnote{Or \fbib{lord}} over us?'' \\
\poeml \v{5}``Because the poor are being oppressed, \\
\poemll    because the needy are sighing, \\
\poemll    I will now arise,'' says the \divine{Lord}, \\
\poemlll       ``I will establish in safety those who yearn for it.'' \\
\poeml \v{6}The words of the \divine{Lord} are pure, \\
\poemll    like silver refined in an earthen furnace, \\
\poemlll       purified seven times over. \\
\poeml \v{7}You, \divine{Lord}, will keep them\fnote{So MT DSS 5/6HevPs 11QPs\textsuperscript{c}; LXX reads \fbib{us}} safe, \\
\poemll    you will guard them\fnote{So MT DSS 5/6HevPs 11QPs\textsuperscript{c}; LXX reads \fbib{us}} from this generation forever. \\
\poeml \v{8}The wicked, however,\fnote{The Heb. lacks \fbib{however}} keep walking around, \\
\poemll    exalting the vileness of human beings.\fnote{Lit. \fbib{of children of men}}
\end{poetry}
\labelpsalm{13}
\psalminfo{To the Director: A Davidic Psalm.}
\passage{A Prayer for Deliverance}

\begin{poetry}
\poeml \v{1}How long? \divine{Lord}, will you forget me forever?\fnote{Or \fbib{How long, \divine{Lord}, will you forget me? Forever?}} \\
\poemll    How long will you hide your face from me? \\
\poeml \v{2}How long must I struggle in my soul at night \\
\poemll    and have sorrow in my heart during the day? \\
\poemlll       How long will my enemy rise up against me? \\
\poeml \v{3}Look at me! \\
\poemll    Answer me, \divine{Lord}, my God! \\
\poeml Give light to my eyes! \\
\poemll    Otherwise, I will sleep in death; \\
\poeml \v{4}Otherwise, my enemy will say, \\
\poemll    ``I have overcome him;'' \\
\poeml Otherwise, my persecutor will rejoice \\
\poemll    when I am shaken. \\
\poeml \v{5}As for me, I have trusted in your gracious love, \\
\poemll    my heart will rejoice in your deliverance. \\
\poeml \v{6}I will sing to the \divine{Lord}, \\
\poemll    for he has dealt bountifully with me.
\end{poetry}
\labelpsalm{14}
\psalminfo{To the Director: A Davidic Psalm.}
\passage{The Fool and God's Response}

\begin{poetry}
\poeml \v{1}Fools say to themselves, ``There is no God.'' \\
\poemll    They are corrupt and commit evil deeds; \\
\poemlll       not one of them practices what is good. \\
\poeml \v{2}The \divine{Lord} looks down from the heavens upon humanity\fnote{Lit. \fbib{upon the sons of Adam}} \\
\poemll    to see if anyone shows discernment as he searches for God. \\
\poeml \v{3}All have turned away, \\
\poemll    together they have become corrupt; \\
\poemlll       no one practices what is good, not even one. \\
\poeml \v{4}Will those who do evil ever learn? \\
\poemll    They devour my people like they devour bread, \\
\poemlll       and never call on the \divine{Lord}. \\
\poeml \v{5}There they are seized with terror, \\
\poemll    because God is with those who are\fnote{Lit. \fbib{with the generation of the}} righteous. \\
\poeml \v{6}You would frustrate the plans of the oppressed,\fnote{Or \fbib{the poor}} \\
\poemll    but the \divine{Lord} is their refuge. \\
\poeml \v{7}May Israel's deliverance come from Zion! \\
\poemll    When the \divine{Lord} restores the fortunes of his people, \\
\poemlll       Jacob will rejoice, and Israel will be glad.\fnote{Cf. Ps 53:1-6}
\end{poetry}
\labelpsalm{15}
\psalminfo{A Davidic Psalm.}
\passage{Welcomed into God's Presence}

\begin{poetry}
\poeml \v{1}\divine{Lord}, who may stay in your tent? \\
\poemll    Who may dwell on your holy mountain? \\
\poeml \v{2}The one who lives with integrity, \\
\poemll    who does righteous deeds, \\
\poemlll       and who speaks truth to himself. \\
\poeml \v{3}The one who does not slander with his tongue, \\
\poemll    who does no evil to his neighbor, \\
\poemlll       and who does not destroy his friend's reputation. \\
\poeml \v{4}The one who despises those who are utterly wicked, \\
\poemll    but who honors the one who fears the \divine{Lord}, \\
\poeml who keeps his word even when it hurts and does not change, \\
\poeml \v{5}who does not loan his money with interest, \\
\poemlll       and who does not take a bribe against those who are innocent. \\
\poeml The one who does these things will stand firm\fnote{Lit. \fbib{won't be shaken}} forever.
\end{poetry}
\labelpsalm{16}
\psalminfo{A special Davidic Psalm.\fnote{T Heb. \fbib{miktam}}}
\passage{Trust in the Face of Death}

\begin{poetry}
\poeml \v{1}Keep me safe, God, \\
\poemll    for I take refuge in you. \\
\poeml \v{2}I told the \divine{Lord}, \\
\poemll    ``You are my master,\fnote{Heb. \fbib{Adonai}} \\
\poemlll       I have nothing good apart from you.'' \\
\poeml \v{3}As for the saints that are in the land, \\
\poemll    they are noble, and all my delight is in them. \\
\poeml \v{4}Those who hurry after another god\fnote{The Heb. lacks \fbib{god}} will have many sorrows; \\
\poemll    I will not present\fnote{Lit. \fbib{pour out}} their drink offerings of blood, \\
\poemlll       nor will my lips speak\fnote{Lit. \fbib{lift up on my lips}} their names. \\
\poeml \v{5}The \divine{Lord} is my inheritance and my cup; \\
\poemll    you support my lot. \\
\poeml \v{6}The boundary lines have fallen in pleasant places for me; \\
\poemll    truly, I have a beautiful heritage. \\
\poeml \v{7}I will bless the \divine{Lord} who has counseled me; \\
\poemll    indeed, my conscience instructs\fnote{Lit. \fbib{thoughts instruct}} me during the night. \\
\poeml \v{8}I have set the \divine{Lord} before me continuously; \\
\poemll    because he stands at my right hand, I will stand firm.\fnote{Lit. \fbib{not be shaken}} \\
\poeml \v{9}Therefore, my heart is glad, \\
\poemll    my whole being\fnote{Lit. \fbib{glory}} rejoices, \\
\poemlll       and my body will dwell securely. \\
\poeml \v{10}For you will not leave my soul in Sheol,\fnote{I.e. the realm of the dead} \\
\poemll    you will not allow your holy one to experience corruption.\fnote{Lit. \fbib{to see the Pit}; i.e. the realm of punishment in the afterlife} \\
\poeml \v{11}You cause me to know the path of life; \\
\poemll    in your presence is joyful abundance, \\
\poemlll       at your right hand there are pleasures forever.
\end{poetry}
\labelpsalm{17}
\psalminfo{A Davidic Prayer.}
\passage{A Cry for Justice}

\begin{poetry}
\poeml \v{1}\divine{Lord}, hear my just plea! \\
\poemll    Pay attention to my cry! \\
\poeml Listen to my prayer, \\
\poemll    since it does not come from lying lips. \\
\poeml \v{2}Justice for me will come from your presence; \\
\poemll    your eyes see what is right. \\
\poeml \v{3}When you probe my heart, \\
\poemll    and examine me at night; \\
\poeml when you refine me, \\
\poemll    you will find nothing wrong,\fnote{The Heb. lacks \fbib{wrong}} \\
\poemlll       for I have determined that I will not transgress with my mouth. \\
\poeml \v{4}As for the ways of mankind, \\
\poemll    I have, according to the words of your lips, \\
\poemlll       avoided the ways of the violent. \\
\poeml \v{5}Because my steps have held fast to your paths, \\
\poemll    my footsteps have not faltered. \\
\poeml \v{6}I call upon you, for you will answer me, God. \\
\poemll    Listen closely to me \\
\poemlll       and hear my prayer. \\
\poeml \v{7}Show forth your gracious love, \\
\poemll    save those who take refuge in you \\
\poemlll       from those who rebel against your sovereign power.\fnote{Lit. \fbib{against your right hand}} \\
\poeml \v{8}Protect me as the most precious part of the eye;\fnote{Lit. \fbib{as the pupil of the daughter of the eye}} \\
\poemll    hide me under the shadow of your wings \\
\poeml \v{9}from the wicked\fnote{Lit. \fbib{face of the wicked}} who have afflicted me, \\
\poemll    from my enemies who have surrounded me. \\
\poeml \v{10}They are imprisoned by their own prosperity,\fnote{Lit. \fbib{fat}} \\
\poemll    they have boasted proudly with their mouth. \\
\poeml \v{11}Now they have encircled our paths\fnote{So MT; DSS 11QPs\textsuperscript{c} LXX read \fbib{have expelled me}} \\
\poemll    and are determined\fnote{Lit. \fbib{and have set their eyes}} to cast us down to the ground. \\
\poeml \v{12}Like a lion they desire to rip us to pieces, \\
\poemll    like a young lion waiting in ambush. \\
\poeml \v{13}Arise, \divine{Lord}, \\
\poemll    confront them, \\
\poemlll       bring them to their knees! \\
\poeml Deliver me from the wicked by your sword--- \\
\poeml \v{14}from men, \divine{Lord}, by your hand--- \\
\poeml from men who belong to this world, \\
\poemll    whose reward is only\fnote{The Heb. lacks \fbib{only}} in this\fnote{So MT; DSS 11QPs\textsuperscript{c} LXX read \fbib{their}} life. \\
\poeml But as for your treasured ones, \\
\poemll    may their stomachs be full, \\
\poeml may their children have an abundance, \\
\poemll    and may they leave wealth to their offspring. \\
\poeml \v{15}But as for me, justified, I will behold your face; \\
\poemll    when I awake, your presence\fnote{Lit. \fbib{form, likeness}} will satisfy me.
\end{poetry}
\labelpsalm{18}
\psalminfo{To the Director: By the servant of the \divine{Lord}, David, who spoke the words of this song to the \divine{Lord} on the day when the \divine{Lord} delivered him from the hands of all his enemies and from the hand of Saul.}
\passage{Gratitude for Victory}

\begin{poetry}
\poeml \v{1}He said: \\
\poemll    ``I love you, Lord, my strength. \\
\poeml \v{2}The \divine{Lord} is my rock, my fortress, my deliverer, my God, \\
\poemll    my stronghold\fnote{Lit. \fbib{rock}} in whom I take refuge, my shield, the glory\fnote{Lit. \fbib{horn}} \\
\poemlll       of my salvation, and my high tower.'' \\
\poeml \v{3}I cried out to the \divine{Lord}, who is worthy to be praised, \\
\poemll    and I was delivered from my enemies. \\
\poeml \v{4}The cords of death entangled me; \\
\poemll    the rivers of Belial\fnote{I.e. the forces of death and destruction} made me afraid. \\
\poeml \v{5}The cords of Sheol\fnote{I.e. the realm of the dead} surrounded me; \\
\poemll    the snares of death confronted me. \\
\poeml \v{6}In my distress I cried to the \divine{Lord}; \\
\poemll    to my God I cried for help. \\
\poeml From his Temple he heard my voice; \\
\poemll    my cry reached his ears. \\
\poeml \v{7}The world shook and trembled; \\
\poemll    the foundations of the mountains quaked, \\
\poemlll       they shook because he was angry. \\
\poeml \v{8}In his anger smoke poured out of his nostrils, \\
\poemll    and consuming fire from his mouth; \\
\poemlll       coals were lit from it. \\
\poeml \v{9}He bent the sky and descended, \\
\poemll    and darkness was under his feet. \\
\poeml \v{10}He rode upon a cherub and flew; \\
\poemll    he soared upon the wings of the wind. \\
\poeml \v{11}He made darkness his hiding place, \\
\poemll    his canopy surrounding him was dark waters and thick clouds. \\
\poeml \v{12}The brightness before him scattered the thick clouds, \\
\poemll    with hail stones and flashes of fire. \\
\poeml \v{13}Then the \divine{Lord} thundered in\fnote{So MT DSS 4QPs\textsuperscript{c}; LXX Targ Vg (cf. 2 Sam 22:14) reads \fbib{from}} the heavens, \\
\poemll    and the Most High sounded aloud, \\
\poemlll       calling for hail stones and flashes of fire.\fnote{So MT DSS 4QPs\textsuperscript{c}; LXX lacks \fbib{calling for hail stones and flashes of fire}} \\
\poeml \v{14}He shot his arrows and scattered them; \\
\poemll    with many lightning bolts he frightened them. \\
\poeml \v{15}Then the channels of the sea could be seen, \\
\poemll    and the foundations of the earth were uncovered \\
\poeml because of your rebuke, \divine{Lord}, \\
\poemll    because of the blast from the breath of your nostrils. \\
\poeml \v{16}He reached down and took me; \\
\poemll    he drew me from many waters. \\
\poeml \v{17}He delivered me from my strong enemies, \\
\poemll    from those who hated me because \\
\poemlll       they were stronger than I. \\
\poeml \v{18}They confronted me in the day of my calamity, \\
\poemll    but the \divine{Lord} was my support. \\
\poeml \v{19}He brought me out to a spacious place; \\
\poemll    he delivered me, for in me he takes delight.
\passage{God's Reward to the Righteous}
\poeml \v{20}The \divine{Lord} will reward me because I am righteous; \\
\poemll    because my hands are clean he will restore me; \\
\poeml \v{21}because I have kept the ways of the \divine{Lord}, \\
\poemll    and I have not wickedly departed from my God; \\
\poeml \v{22}because all his judgments were always before me, \\
\poemll    and I did not cast off his statutes. \\
\poeml \v{23}I was upright\fnote{Or \fbib{blameless}} before him, \\
\poemll    and I kept myself from iniquity. \\
\poeml \v{24}So the \divine{Lord} restored me according to my righteousness, \\
\poemll    because my hands were clean in his sight. \\
\poeml \v{25}To the holy, you show your gracious love, \\
\poemll    to the upright, you show yourself upright; \\
\poeml \v{26}to the pure, you show yourself pure, \\
\poemll    and to the morally corrupt, you appear to be perverse. \\
\poeml \v{27}Indeed, you deliver the oppressed,\fnote{Or \fbib{poor}} \\
\poemll    but you bring down those who exalt themselves \\
\poemlll       in their own eyes. \\
\poeml \v{28}For you, \divine{Lord}, make my lamp shine; \\
\poemll    my God enlightens my darkness. \\
\poeml \v{29}With your help\fnote{The Heb. lacks \fbib{help}} I will run through an army, \\
\poemll    with help from\fnote{The Heb. lacks \fbib{help from}} my God I leap over walls. \\
\poeml \v{30}As for God, his way is upright;\fnote{Or \fbib{blameless}} \\
\poemll    the word of God is pure; \\
\poemlll       he is a shield to all those who take refuge in him.
\passage{The Acts of God for the Righteous}
\poeml \v{31}For who is God but the \divine{Lord}, \\
\poemll    and who is a Rock other than our God?--- \\
\poeml \v{32}the God who clothes me with strength, \\
\poemll    and who makes my way upright;\fnote{Or \fbib{blameless}} \\
\poeml \v{33}who makes my feet swift as the deer; \\
\poemll    who makes me stand on high places; \\
\poeml \v{34}who teaches my hands to make war, \\
\poemll    and my arms to bend a bronze bow. \\
\poeml \v{35}You have given to me the shield of your deliverance, \\
\poemll    and your right hand holds me up; \\
\poemlll       your gentleness made me great. \\
\poeml \v{36}You make a broad place for my steps, \\
\poemll    so my feet\fnote{Lit. \fbib{ankle}} won't slip. \\
\poeml \v{37}I pursued my enemies and overtook them; \\
\poemll    I did not turn around until they were utterly defeated. \\
\poeml \v{38}I struck them down, \\
\poemll    so they are not able to rise up; \\
\poemlll       they fell under my feet. \\
\poeml \v{39}You clothed me with strength for war; \\
\poemll    you will subdue under me those who rise up against me. \\
\poeml \v{40}You have made my enemies turn their back to me, \\
\poemll    and I will destroy those who hate me. \\
\poeml \v{41}They cried out for deliverance, \\
\poemll    but there was no one to deliver; \\
\poeml they cried out\fnote{The Heb. lacks \fbib{they cried out}} to the \divine{Lord}, \\
\poemll    but he did not answer them. \\
\poeml \v{42}I ground them like wind-swept dust; \\
\poemll    I emptied them out\fnote{So MT DSS 5/6HevPs; LXX reads \fbib{I will grind them down}} like dirt in the street. \\
\poeml \v{43}You rescued me from conflict with the people; \\
\poemll    you made me head of the nations. \\
\poemlll       People who did not know me will serve me. \\
\poeml \v{44}When they hear of me,\fnote{Lit. \fbib{At the hearing of the ear}} they will obey me; \\
\poemll    foreigners will submit to me. \\
\poeml \v{45}Foreigners will wilt away; \\
\poemll    they will come trembling out of their stronghold. \\
\poeml \v{46}The \divine{Lord} lives! \\
\poemll    Blessed be my Rock! \\
\poemlll       May the God of my deliverance be exalted! \\
\poeml \v{47}He is the God who executes vengeance on my behalf; \\
\poemll    who destroys peoples under me; \\
\poeml \v{48}who delivers me from my enemies. \\
\poemll    Truly you will exalt me above those who oppose me; \\
\poemlll       you will deliver me from the violent person. \\
\poeml \v{49}Therefore, I will give thanks to you among the nations, \divine{Lord}; \\
\poemll    I will sing praises to your name. \\
\poeml \v{50}He is the one who gives victories to his king; \\
\poemll    who shows gracious love to his anointed, \\
\poemlll       to David and his seed forever.
\end{poetry}
\labelpsalm{19}
\psalminfo{To the Director: A Davidic Psalm.}
\passage{God's Revelation in the Heavens}

\begin{poetry}
\poeml \v{1}The heavens are declaring the glory of God, \\
\poemll    and their expanse shows the work of his hands. \\
\poeml \v{2}Day after day they pour forth speech, \\
\poemll    night after night they reveal knowledge. \\
\poeml \v{3}There is no speech nor are there words--- \\
\poemll    their voice is not heard--- \\
\poeml \v{4}yet their message\fnote{Or \fbib{sound}; so LXX; MT DSS 11QPs\textsuperscript{c} Syr read \fbib{line}} goes out into all the world, \\
\poemll    and their words to the ends of the earth. \\
\poeml He has set up a tent for the sun in the heavens,\fnote{Lit. \fbib{them}} \\
\poeml \v{5}which is like a bridegroom coming out of his chamber, \\
\poemlll       or like a champion who rejoices at the beginning of a race. \\
\poeml \v{6}Its circuit is from one end of the sky to the other, \\
\poemll    and nothing is hidden from its heat.
\passage{God's Revelation in the Law}
\poeml \v{7}The Law of the \divine{Lord} is perfect, \\
\poemll    restoring life. \\
\poeml The testimony of the \divine{Lord} is steadfast, \\
\poemll    making foolish people wise. \\
\poeml \v{8}The precepts of the \divine{Lord} are upright, \\
\poemll    making the heart rejoice. \\
\poeml The commandment of the \divine{Lord} is pure, \\
\poemll    giving light to the eyes. \\
\poeml \v{9}The fear of the \divine{Lord} is clean, \\
\poemll    standing forever. \\
\poeml The judgments of the \divine{Lord} are true; \\
\poemll    they are altogether righteous. \\
\poeml \v{10}They are more desirable than gold, \\
\poemll    even much fine gold. \\
\poeml They are sweeter than honey, \\
\poemll    even the drippings from a honeycomb. \\
\poeml \v{11}Moreover your servant is warned by them; \\
\poemll    and there is great reward in keeping them. \\
\poeml \v{12}Who can detect his own\fnote{The Heb. lacks \fbib{his own}} mistake? \\
\poemll    Cleanse me from hidden sin. \\
\poeml \v{13}Preserve your servant from arrogant people;\fnote{Or \fbib{from presumptuous sins}} \\
\poemll    do not let them rule over me. \\
\poeml Then I will be upright\fnote{Or \fbib{perfect}, or \fbib{blameless}} \\
\poemll    and acquitted of great wickedness. \\
\poeml \v{14}May the words of my mouth and the meditations of my heart \\
\poemll    be acceptable in your sight, \divine{Lord}, my Rock and my Redeemer.
\end{poetry}
\labelpsalm{20}
\psalminfo{To the Director: A Davidic Psalm.}
\passage{A Prayer for Victory}

\begin{poetry}
\poeml \v{1}May the \divine{Lord} answer you in the day of distress; \\
\poemll    may the name of the God of Jacob\fnote{I.e. Israel} protect you. \\
\poeml \v{2}May he send you help from the sanctuary, \\
\poemll    and may he sustain you from Zion. \\
\poeml \v{3}May he remember all your gifts, \\
\poemll    and may he accept your burnt offerings.
\end{poetry}
\interlude{Interlude}

\begin{poetry}
\poeml \v{4}May he give you what your heart desires, \\
\poemll    and may he fulfill all your plans. \\
\poeml \v{5}May we shout for joy at your deliverance \\
\poemll    and unfurl our banners in the name of our God. \\
\poemlll       May the \divine{Lord} fulfill all your petitions. \\
\poeml \v{6}Now I know that the \divine{Lord} has delivered his anointed; \\
\poemll    he has answered him from his sanctuary \\
\poemlll       with the strength of his right hand of deliverance. \\
\poeml \v{7}Some boast\fnote{The Heb. lacks \fbib{Some boast}} in chariots, \\
\poemll    others in horses; \\
\poemlll       but we will boast in\fnote{Or \fbib{remember}} the name of the \divine{Lord} our God. \\
\poeml \v{8}While they bowed down and fell, \\
\poemll    we arose and stood upright. \\
\poeml \v{9}Deliver us, \divine{Lord}! \\
\poemll    Answer us, our King,\fnote{I.e. God} on the day we cry out!
\end{poetry}
\labelpsalm{21}
\psalminfo{To the Director: A Davidic Psalm.}
\passage{Praise for the \divine{Lord}'s Deliverance}

\begin{poetry}
\poeml \v{1}The king rejoices in your strength, \divine{Lord}. \\
\poemll    How greatly he rejoices in your deliverance. \\
\poeml \v{2}You have granted him the desire of his heart, \\
\poemll    and have not withheld what his lips requested.
\end{poetry}
\interlude{Interlude}

\begin{poetry}
\poeml \v{3}You go before him with wonderful blessings, \\
\poemll    and put a crown of fine gold on his head. \\
\poeml \v{4}He asked life from you, and you gave it to him--- \\
\poemll    a long life for ever and ever. \\
\poeml \v{5}His glory is great because of your deliverance, \\
\poemll    you have given him honor and majesty. \\
\poeml \v{6}Indeed, you have given him eternal blessings; \\
\poemll    you will make him glad with the joy of your presence. \\
\poeml \v{7}The king trusts in the \divine{Lord}; \\
\poemll    because of the gracious love of the Most High, \\
\poemlll       he will stand firm.\fnote{Lit. \fbib{will not be shaken}} \\
\poeml \v{8}Your hand will find all your enemies, \\
\poemll    your right hand will find those who hate you. \\
\poeml \v{9}When you appear, \\
\poemll    you will set them ablaze like a fire furnace. \\
\poeml In his wrath, the \divine{Lord} will consume them, \\
\poemll    and the fire will devour them. \\
\poeml \v{10}You will destroy their descendants\fnote{Lit. \fbib{his fruit}} from the earth, \\
\poemll    even their offspring from the ranks\fnote{Lit. \fbib{children}} of mankind. \\
\poeml \v{11}Though they plot evil against you and devise schemes, \\
\poemll    they will not succeed. \\
\poeml \v{12}Indeed, you will make them retreat,\fnote{Lit. \fbib{will turn the shoulder}} \\
\poemll    when you aim your bow\fnote{Lit. \fbib{when your bow string is ready}} at their faces. \\
\poeml \v{13}Rise up, \divine{Lord}, because you are strong; \\
\poemll    we will sing and praise your power.
\end{poetry}
\labelpsalm{22}
\psalminfo{To the Director: To the tune of\fnote{T Lit. \fbib{According to}} ``Doe of the Dawn''.\\ A Davidic Psalm.}
\passage{God Delivers His Suffering Servant}

\begin{poetry}
\poeml \v{1}My God! My God! \\
\poemll    Why have you abandoned me? \\
\poeml Why are you so far from delivering me--- \\
\poemll    from my groaning words? \\
\poeml \v{2}My God, I cry out to you throughout the day, \\
\poemll    but you do not answer; \\
\poeml and throughout the night, \\
\poemll    but I have no rest.\fnote{Lit. \fbib{but there is no silence for me}} \\
\poeml \v{3}You are holy, \\
\poemll    enthroned on the praises of Israel. \\
\poeml \v{4}Our ancestors trusted in you; \\
\poemll    they trusted and you delivered them. \\
\poeml \v{5}They cried out to you and escaped; \\
\poemll    they trusted in you and were not put to shame. \\
\poeml \v{6}But as for me, \\
\poemll    I am only a worm and not a man, \\
\poemlll       scorned by mankind and despised by people. \\
\poeml \v{7}Everyone who sees me mocks me; \\
\poemll    they gape at me with open mouths \\
\poemlll       and shake their heads at me. \\
\poeml \v{8}They say,\fnote{The Heb. lacks \fbib{They say}} ``Commit yourself to the \divine{Lord}; \\
\poemll    perhaps the \divine{Lord}\fnote{Lit. \fbib{he}} will deliver him, \\
\poeml perhaps he will cause him to escape, \\
\poemll    since he delights in him.'' \\
\poeml \v{9}Yet, you are the one who took me from the womb, \\
\poemll    and kept me safe on my mother's breasts. \\
\poeml \v{10}I was dependent on you from birth; \\
\poemll    from my mother's womb you have been my God. \\
\poeml \v{11}Do not be so distant from me, \\
\poemll    for trouble is at hand; \\
\poemlll       indeed, there is no deliverer. \\
\poeml \v{12}Many bulls have surrounded me; \\
\poemll    the vicious bulls of Bashan have encircled me. \\
\poeml \v{13}Their mouths are opened wide toward me, \\
\poemll    like roaring and attacking lions. \\
\poeml \v{14}I am poured out like water; \\
\poemll    all my bones are out of joint. \\
\poemlll       My heart is like wax, melting within me. \\
\poeml \v{15}My strength is dried up like broken pottery; \\
\poemll    my tongue sticks to the roof of my mouth,\fnote{Lit. \fbib{to my jaws}} \\
\poemlll       and you have brought me down to the dust of death. \\
\poeml \v{16}For dogs have surrounded me; \\
\poemll    a gang of those who practice of evil has encircled me. \\
\poemlll       They gouged\fnote{So LXX Syr DSS 5/6 HevPS XHev/Se4; MT reads \fbib{Like a lion}} my hands and my\fnote{So MT; LXX lacks \fbib{my}} feet. \\
\poeml \v{17}I can count all my bones. \\
\poemll    They look at me; \\
\poemlll       they stare at me. \\
\poeml \v{18}They divide my clothing among themselves; \\
\poemll    they cast lots for my clothing! \\
\poeml \v{19}But as for you, \divine{Lord}, do not be far away from me; \\
\poemll    My Strength, come quickly to help me. \\
\poeml \v{20}Deliver me from the sword; \\
\poemll    my precious life from the power of the dog. \\
\poeml \v{21}Deliver me from the mouth of the lion, \\
\poemll    from the horns of the wild oxen. \\
\poeml You have answered me. \\
\poeml \v{22}I will declare your name to my brothers; \\
\poemll    in the midst of the congregation, I will praise you, saying,\fnote{The Heb. lacks \fbib{saying}} \\
\poeml \v{23}``All who fear the \divine{Lord}, praise him! \\
\poemll    All the seed of Jacob, glorify him! \\
\poeml All the seed of Israel, fear him! \\
\poeml \v{24}For he does not despise nor detest the afflicted person; \\
\poeml he does not hide his face from him, \\
\poemll    but he hears him when he cries out to him.'' \\
\poeml \v{25}My praise in the great congregation is because of you; \\
\poemll    I will pay my vows before those who fear you.\fnote{Lit. \fbib{him}} \\
\poeml \v{26}The afflicted will eat and be satisfied; \\
\poemll    those who seek the \divine{Lord} will praise him, \\
\poemlll       ``May you\fnote{Lit. \fbib{your heart}} live forever!'' \\
\poeml \v{27}All the ends of the earth will remember and turn to the \divine{Lord}; \\
\poemll    all the families of the nations will bow in submission to the \divine{Lord}. \\
\poeml \v{28}Indeed, the kingdom belongs to the \divine{Lord}; \\
\poemll    he rules over the nations. \\
\poeml \v{29}All the prosperous people will eat and bow down in submission. \\
\poemll    All those who are about to go down to the grave\fnote{Lit. \fbib{dust}} \\
\poemlll       will bow down in submission, \\
\poemll    along with the one who can no longer keep himself alive. \\
\poeml \v{30}Our\fnote{The Heb. lacks \fbib{our}} descendants will serve him, \\
\poemll    and that generation will be told about the Lord. \\
\poeml \v{31}They will come and declare his righteousness \\
\poemll    to a people yet to be born; \\
\poemlll       indeed, he has accomplished it!
\end{poetry}
\labelpsalm{23}
\psalminfo{A Davidic Psalm.}
\passage{The \divine{Lord} Shepherds His People}

\begin{poetry}
\poeml \v{1}The \divine{Lord} is the one who is shepherding me; \\
\poemll    I lack nothing. \\
\poeml \v{2}He causes me to lie down in pastures of green grass; \\
\poemll    he guides me beside quiet waters. \\
\poeml \v{3}He revives my life; \\
\poemll    he leads me in pathways that are righteous \\
\poemlll       for the sake of his name.\fnote{I.e. his reputation} \\
\poeml \v{4}Even when I walk through a valley of deep darkness,\fnote{Or \fbib{valley of the shadow of death}} \\
\poemll    I will not be afraid \\
\poemlll       because you are with me. \\
\poeml Your rod and your staff---they comfort me. \\
\poeml \v{5}You prepare a table before me, \\
\poemll    even in the presence of my enemies. \\
\poeml You anoint my head with oil; \\
\poemll    my cup overflows. \\
\poeml \v{6}Truly, goodness and gracious love will pursue me \\
\poemll    all the days of my life, \\
\poemlll       and I will remain in\fnote{MT DSS 5/6HevPs read \fbib{will return to}; LXX reads \fbib{and my residing will be}} the \divine{Lord}'s Temple forever.\fnote{Lit. \fbib{for the length of days}}
\end{poetry}
\labelpsalm{24}
\psalminfo{A Davidic Psalm.}
\passage{A Song for the King of Glory}

\begin{poetry}
\poeml \v{1}The earth and everything in it exists for the \divine{Lord}--- \\
\poemll    the world and those who live in it. \\
\poeml \v{2}Indeed, he founded it upon the seas, \\
\poemll    he established it upon deep waters.\fnote{Lit. \fbib{rivers}; i.e. the subterranean waters} \\
\poeml \v{3}Who may ascend the mountain of the \divine{Lord}?\fnote{I.e. the temple mount} \\
\poemll    Who may stand in his Holy Place? \\
\poeml \v{4}The one who has innocent hands and a pure heart; \\
\poemll    the person who does not delight in what is false \\
\poemlll       and does not swear an oath deceitfully. \\
\poeml \v{5}This person\fnote{Lit. \fbib{he}} will receive blessing from the \divine{Lord} \\
\poemll    and righteousness from the God of his salvation. \\
\poeml \v{6}This is the generation that seeks him. \\
\poemll    Those who seek your face \\
\poemlll       are the true seed of\fnote{The Heb. lacks \fbib{the true seed of}} Jacob.
\end{poetry}
\interlude{Interlude}

\begin{poetry}
\poeml \v{7}Lift up your heads,\fnote{I.e. \fbib{Open}} gates! \\
\poemll    Be lifted up, ancient doors, \\
\poemlll       so the King of Glory may come in. \\
\poeml \v{8}Who is the King of Glory? \\
\poemll    The \divine{Lord} strong and mighty, \\
\poemlll       the \divine{Lord}, mighty in battle. \\
\poeml \v{9}Lift up your heads,\fnote{I.e. \fbib{Open}} gates! \\
\poemll    Be lifted up, ancient doors, \\
\poemlll       so the King of Glory may come in. \\
\poeml \v{10}Who is he, this King of Glory? \\
\poemll    The \divine{Lord} of the heavenly armies--- \\
\poemlll       He is the King of Glory.
\end{poetry}
\interlude{Interlude}
\labelpsalm{25}
\psalminfo{Davidic\fnote{T This acrostic psalm begins each verse with a consecutive letter of the Hebrew alphabet}}
\passage{A Prayer for Help and Forgiveness}

\begin{poetry}
\poeml \v{1}I will lift up my soul to you, \divine{Lord}. \\
\poeml \v{2}I trust in you, my God, \\
\poemll    do not let me be ashamed; \\
\poemlll       do not let my enemies triumph over me. \\
\poeml \v{3}Indeed, no one who waits on you will be ashamed, \\
\poemll    but those who offend for no reason will be put to shame. \\
\poeml \v{4}Cause me to understand your ways, \divine{Lord}; \\
\poemll    teach me your paths. \\
\poeml \v{5}Guide me in your truth and teach me; \\
\poemll    for you are the God who delivers me. \\
\poemlll       All day long I have waited for you. \\
\poeml \v{6}Remember, \divine{Lord}, your tender mercies and your gracious love; \\
\poemll    indeed, they are eternal! \\
\poeml \v{7}Do not remember my youthful sins and transgressions; \\
\poemll    but remember me in light of your gracious love, \\
\poemlll       in light of your goodness, \divine{Lord}. \\
\poeml \v{8}The \divine{Lord} is good and just; \\
\poemll    therefore he will teach sinners concerning the way. \\
\poeml \v{9}He will guide the humble\fnote{Or \fbib{afflicted}} to justice; \\
\poemll    he will teach the humble\fnote{Or \fbib{afflicted}} his way. \\
\poeml \v{10}All the paths of the \divine{Lord} lead to gracious love and truth \\
\poemll    for those who keep his covenant and his decrees.\fnote{Or \fbib{testimonies}} \\
\poeml \v{11}For the sake of your name,\fnote{I.e. reputation} \divine{Lord}, \\
\poemll    forgive my sin, for it is great. \\
\poeml \v{12}Who is the man who fears the \divine{Lord}? \\
\poemll    God\fnote{Lit. \fbib{He}} will teach him the path he should choose. \\
\poeml \v{13}He\fnote{Lit. \fbib{His soul}} will experience good things; \\
\poemll    his descendants will inherit the earth. \\
\poeml \v{14}The intimate counsel of the \divine{Lord} is for those who fear him \\
\poemll    so they may know his covenant. \\
\poeml \v{15}My eyes look to the \divine{Lord} continuously, \\
\poemll    because he's the one who releases my feet from the trap.\fnote{Lit. \fbib{net}} \\
\poeml \v{16}Turn toward me and have mercy on me, \\
\poemll    for I am lonely and oppressed. \\
\poeml \v{17}The troubles of my heart have increased; \\
\poemll    bring me out of my distress! \\
\poeml \v{18}Look upon my distress and affliction; \\
\poemll    forgive all my sins. \\
\poeml \v{19}Look how many enemies I have gained! \\
\poemll    They hate me with a vicious hatred. \\
\poeml \v{20}Preserve my life and deliver me; \\
\poemll    do not let me be ashamed, \\
\poemlll       because I take refuge in you. \\
\poeml \v{21}Integrity and justice will preserve me, \\
\poemll    because I wait on you. \\
\poeml \v{22}Redeem Israel, God, from all its troubles.
\end{poetry}
\labelpsalm{26}
\psalminfo{Davidic}
\passage{A Man of Integrity Pleads for Justice}

\begin{poetry}
\poeml \v{1}Vindicate me, \divine{Lord}, \\
\poemll    because I have walked in integrity; \\
\poemlll       I have trusted in the \divine{Lord} without wavering. \\
\poeml \v{2}Examine me, \divine{Lord}, and inspect me! \\
\poemll    Test my heart and mind.\fnote{Lit. \fbib{kidneys}; i.e. the center of emotions} \\
\poeml \v{3}For your gracious love precedes me, \\
\poemll    and I continuously walk according to your truth. \\
\poeml \v{4}I do not sit with those committed to what is false, \\
\poemll    nor do I travel with hypocrites. \\
\poeml \v{5}I hate the company of those who practice evil, \\
\poemll    nor do I sit with the wicked. \\
\poeml \v{6}I wash my hands innocently. \\
\poemll    I go around your altar, \divine{Lord}, \\
\poeml \v{7}so I may praise you loudly with thanksgiving \\
\poemll    and declare all your wondrous acts. \\
\poeml \v{8}\divine{Lord}, I love the dwelling place that is your house, \\
\poemll    the place where your glory resides. \\
\poeml \v{9}Do not group me\fnote{Lit. \fbib{my soul}} with sinners, \\
\poemll    nor include me\fnote{The Heb. lacks \fbib{include me}} with men who shed blood. \\
\poeml \v{10}Their hands are filled with wicked schemes, \\
\poemll    and their right hands with bribes. \\
\poeml \v{11}But as for me, \\
\poemll    I walk in my integrity. \\
\poemlll       Redeem me and be gracious to me! \\
\poeml \v{12}My feet stand on level ground; \\
\poemll    among the worshiping congregations \\
\poemlll       I will bless the \divine{Lord}.
\end{poetry}
\labelpsalm{27}
\psalminfo{Davidic}
\passage{Confidence in the \divine{Lord}}

\begin{poetry}
\poeml \v{1}The \divine{Lord} is my light and my salvation--- \\
\poemll    whom will I fear? \\
\poeml The \divine{Lord} is the strength of my life; \\
\poemll    of whom will I be afraid? \\
\poeml \v{2}When those who practice evil, my enemies, and my oppressors \\
\poemll    come near me to devour my flesh, \\
\poemlll       they stumble and fall. \\
\poeml \v{3}If an army encamps against me, \\
\poemll    my heart will not fear. \\
\poeml If a war is launched against me, \\
\poemll    I will even trust in that situation. \\
\poeml \v{4}I have asked one thing from the \divine{Lord}; \\
\poemll    it is what I really seek: \\
\poeml that I may remain in the \divine{Lord}'s Temple \\
\poemll    all the days of my life, \\
\poeml to gaze on the beauty of the \divine{Lord}; \\
\poemll    and to inquire in his Temple. \\
\poeml \v{5}For he will conceal me in his shelter on the day of evil; \\
\poemll    He will hide me in a secluded chamber within his tent; \\
\poemlll       He will place me on a high rock. \\
\poeml \v{6}Now my head will be lifted up above my enemies, \\
\poemll    even those who surround me. \\
\poeml I will sacrifice in his tent with shouts of joy; \\
\poemll    I will sing and make melodies to the \divine{Lord}. \\
\poeml \v{7}Hear my voice, \divine{Lord}, when I cry out! \\
\poemll    Be gracious to me and answer me. \\
\poeml \v{8}My mind recalls your word,\fnote{The Heb. lacks \fbib{your word}} \\
\poemll    ``Seek my face,'' \\
\poemlll       so your face, \divine{Lord}, I will seek. \\
\poeml \v{9}Do not hide your face from me; \\
\poemll    do not turn away in anger from your servant. \\
\poeml You have been my help, \\
\poemll    therefore do not abandon or forsake me, \\
\poemlll       God of my salvation. \\
\poeml \v{10}Though my father and my mother abandoned me, \\
\poemll    the \divine{Lord} gathers me up. \\
\poeml \v{11}Teach me your way, \divine{Lord}, \\
\poemll    and lead me on a level path because of my enemies. \\
\poeml \v{12}Do not hand me over to the desires of my enemies; \\
\poemll    for false witnesses have risen up against me; \\
\poemlll       even the one who breathes out violence. \\
\poeml \v{13}I believe that I will see the \divine{Lord}'s goodness \\
\poemll    in the land of the living. \\
\poeml \v{14}Wait on the \divine{Lord}. \\
\poemll    Be courageous, and he will strengthen your heart. \\
\poemlll       Wait on the \divine{Lord}!
\end{poetry}
\labelpsalm{28}
\psalminfo{Davidic}
\passage{A Prayer for Help}

\begin{poetry}
\poeml \v{1}To you, \divine{Lord}, I cry out! \\
\poemll    My Rock, do not refuse to answer me.\fnote{Lit. \fbib{do not be silent to me}} \\
\poeml If you remain silent, \\
\poemll    I will become like those who descend into the Pit.\fnote{I.e. the place of punishment in the afterlife} \\
\poeml \v{2}Hear the sound of my supplications when I cry to you for help, \\
\poemll    as I lift up my hands toward your most holy sanctuary. \\
\poeml \v{3}Do not drag me away with the wicked, \\
\poemll    with those who practice iniquity, \\
\poeml who speak peace to their neighbors \\
\poemll    while harboring evil in their hearts. \\
\poeml \v{4}Reward them according to their deeds; \\
\poemll    according to the evil of their actions. \\
\poeml Reward them based on what they do;\fnote{Lit. \fbib{them according to work of their hands}} \\
\poemll    give them what they deserve. \\
\poeml \v{5}Because they do not understand the deeds of the \divine{Lord} \\
\poemll    or the work of his hands, \\
\poemlll       He will tear them down and never build them up. \\
\poeml \v{6}Blessed be the \divine{Lord}! \\
\poemll    For he has heard the sound of my supplications. \\
\poeml \v{7}The \divine{Lord} is my strength and my shield; \\
\poemll    my heart trusts in him, \\
\poemlll       and I received help. \\
\poeml My heart rejoices, \\
\poemll    and I give thanks to him with my song. \\
\poeml \v{8}The \divine{Lord} is the strength of his people;\fnote{Lit. \fbib{of them}} \\
\poemll    he is a refuge of deliverance for his anointed. \\
\poeml \v{9}Deliver your people \\
\poemll    and bless your inheritance! \\
\poeml Shepherd them \\
\poemll    and lift them up forever!
\end{poetry}
\labelpsalm{29}
\psalminfo{A Davidic Psalm.}
\passage{Praise to the Majestic \divine{Lord}}

\begin{poetry}
\poeml \v{1}Ascribe to the \divine{Lord}, you heavenly beings; \\
\poemll    ascribe to the \divine{Lord} glory and strength. \\
\poeml \v{2}Ascribe to the \divine{Lord} the glory due his name; \\
\poemll    worship the \divine{Lord} wearing holy attire. \\
\poeml \v{3}The voice of the \divine{Lord} was heard\fnote{The Heb. lacks \fbib{heard}} above the waters; \\
\poemll    the God of glory thundered; \\
\poemlll       the \divine{Lord} was heard\fnote{The Heb. lacks \fbib{heard}} over many waters. \\
\poeml \v{4}The voice of the \divine{Lord} is powerful; \\
\poemll    the voice of the \divine{Lord} is majestic. \\
\poeml \v{5}The voice of the \divine{Lord} snaps the cedars;\fnote{I.e. a genus of coniferous evergreen in the family \fbib{Pinaceae}; and so throughout the book} \\
\poemll    the \divine{Lord} snaps the cedars of Lebanon. \\
\poeml \v{6}He makes them stagger like a calf, \\
\poemll    even Lebanon and Sirion\fnote{I.e. Mount Hermon; cf. Deut 3:9} like a young wild ox. \\
\poeml \v{7}The voice of the \divine{Lord} shoots out flashes of fire. \\
\poeml \v{8}The voice of the \divine{Lord} shakes the wilderness; \\
\poemlll       the voice of the \divine{Lord} shakes\fnote{The Heb. lacks \fbib{shakes}} the wilderness of Kadesh. \\
\poeml \v{9}The voice of the \divine{Lord} causes deer to give birth, \\
\poemll    and strips the forest bare. \\
\poemlll       In his Temple all of them shout, ``Glory!'' \\
\poeml \v{10}The \divine{Lord} sat enthroned over the flood, \\
\poemll    and the \divine{Lord} sits as king forever. \\
\poeml \v{11}The \divine{Lord} will give strength to his people; \\
\poemll    the \divine{Lord} will bless his people with peace.
\end{poetry}
\labelpsalm{30}
\psalminfo{A Davidic Psalm for the dedication of the Temple.}
\passage{Thanksgiving for Deliverance}

\begin{poetry}
\poeml \v{1}I exalt you, \divine{Lord}, \\
\poemll    for you have lifted me up, \\
\poemlll       and my enemies could not gloat over me. \\
\poeml \v{2}\divine{Lord}, my God! \\
\poemll    I cried out to you for help \\
\poemlll       and you healed me. \\
\poeml \v{3}\divine{Lord}, you brought me from death;\fnote{Lit. \fbib{Sheol}, a reference to the realm of the dead} \\
\poemll    you kept me alive so that I did not descend into the Pit.\fnote{I.e. the place of punishment in the afterlife} \\
\poeml \v{4}You, his godly ones, \\
\poemll    sing to the \divine{Lord}, \\
\poemlll       give thanks at the mention of his holiness. \\
\poeml \v{5}For his wrath is only momentary; \\
\poemll    yet his favor is for a lifetime. \\
\poeml Weeping may lodge for the night, \\
\poemll    but shouts of joy will come in the morning. \\
\poeml \v{6}As for me, \\
\poemll    I said in my prosperity, \\
\poemlll       ``I will never be moved.'' \\
\poeml \v{7}By your favor, \divine{Lord}, \\
\poemll    you established me as a strong mountain; \\
\poeml Then you hid your face, \\
\poemll    and I was dismayed. \\
\poeml \v{8}I cried out to you, \divine{Lord}, \\
\poemll    and I make supplication to the Lord: \\
\poeml \v{9}``What profit is there in my death\fnote{Lit. \fbib{my blood}} if I go down to the Pit?\fnote{I.e. the place of punishment in the afterlife} \\
\poemll    Can dust worship you? \\
\poemlll       Can it proclaim your faithfulness?'' \\
\poeml \v{10}Hear me, \divine{Lord}, \\
\poemll    and have mercy on me! \\
\poemlll       \divine{Lord}, help me! \\
\poeml \v{11}You have turned my mourning into dancing; \\
\poemll    you took off my sackcloth \\
\poemlll       and clothed me with a garment of joy, \\
\poeml \v{12}so that I may sing praise to you \\
\poemll    and not remain silent. \\
\poeml \divine{Lord}, my God, \\
\poemll    I will give you thanks forever!
\end{poetry}
\labelpsalm{31}
\psalminfo{To the Director: A Davidic Psalm.}
\passage{Prayer and Thanksgiving}

\begin{poetry}
\poeml \v{1}In you, \divine{Lord}, I have taken refuge. \\
\poemll    Let me never be ashamed. \\
\poemlll       Because you are righteous, deliver me! \\
\poeml \v{2}Listen to me, \\
\poemll    and deliver me quickly. \\
\poeml Become a rock of safety for me, \\
\poemll    a fortified citadel to deliver me; \\
\poeml \v{3}For you are my rock and my fortress; \\
\poemll    for the sake of your name guide me and lead me. \\
\poeml \v{4}Rescue me from the net that they concealed to trap me; \\
\poemll    for you are my strength. \\
\poeml \v{5}Into your hands I commit my spirit; \\
\poemll    for you have redeemed me, \\
\poemlll       \divine{Lord} God of truth. \\
\poeml \v{6}I despise those who trust vain idols; \\
\poemll    but I have trusted in the \divine{Lord}. \\
\poeml \v{7}I will rejoice and be glad in your gracious love, \\
\poemll    for you see my affliction \\
\poemlll       and take note that my soul is distressed. \\
\poeml \v{8}You have not delivered me into the hand of the enemy, \\
\poemll    but you have set my feet in a sturdy\fnote{Lit. \fbib{broad}} place. \\
\poeml \v{9}Be gracious to me, \divine{Lord}, \\
\poemll    for I am in distress. \\
\poeml My eyes have been consumed by my grief \\
\poemll    along with my soul and my body. \\
\poeml \v{10}My life is consumed by sorrow, \\
\poemll    my years with groaning. \\
\poeml My strength has faltered because of my iniquity;\fnote{So MT DSS 5/6HevPs; LXX reads \fbib{strength grew weak in poverty}} \\
\poemll    my bones have been consumed. \\
\poeml \v{11}I have become an object of reproach to all my enemies, \\
\poemll    especially to my neighbors. \\
\poeml I have become an object of fear to my friends, \\
\poemll    and whoever sees me outside runs away from me. \\
\poeml \v{12}Like a dead man, I am forgotten in their thoughts\fnote{Lit. \fbib{hearts}}--- \\
\poemll    like broken pottery. \\
\poeml \v{13}I have heard the slander of many; \\
\poemll    it is like terror all around me, \\
\poemlll       as they conspire together and plot to take my life. \\
\poeml \v{14}But I trust in you, \divine{Lord}. \\
\poemll    I say, ``You are my God.'' \\
\poeml \v{15}My times are in your hands. \\
\poemll    Deliver me from the hands of my enemies \\
\poemlll       and from those who pursue me. \\
\poeml \v{16}May your face shine on your servant; \\
\poemll    in your gracious love, deliver me. \\
\poeml \v{17}Let me not be ashamed, \divine{Lord}, \\
\poemll    for I have called upon you. \\
\poeml Let the wicked be put to shame, \\
\poemll    let them be silent in the next life.\fnote{Lit. \fbib{in Sheol}; i.e. the realm of the dead} \\
\poeml \v{18}Let the lying lips be made still, \\
\poemll    especially those who speak arrogantly \\
\poemlll       against the righteous with pride and contempt. \\
\poeml \v{19}How great is your goodness \\
\poemll    that you have reserved for those who fear you, \\
\poeml that you have set in place for those who take refuge in you, \\
\poemll    in the presence of the children of men. \\
\poeml \v{20}You will hide them in the secret place of your presence, \\
\poemll    away from the conspiracies of men. \\
\poeml You will hide them in your tent, \\
\poemll    away from their contentious tongues. \\
\poeml \v{21}Blessed be the \divine{Lord}! \\
\poemll    In a marvelous way he demonstrated his gracious love to me, \\
\poemlll       when I was in a city under siege. \\
\poeml \v{22}When I said in my panic, \\
\poemll    ``I have been cut off in your sight,'' \\
\poeml then you surely heard the voice of my prayer \\
\poemll    in my plea to you for help. \\
\poeml \v{23}Love the \divine{Lord}, all his godly ones! \\
\poemll    The \divine{Lord} preserves the faithful \\
\poemlll       and repays those who act with proud motives. \\
\poeml \v{24}Be strong, \\
\poemll    and let your heart be courageous, \\
\poemlll       all you who put your hope in the \divine{Lord}.
\end{poetry}
\labelpsalm{32}
\psalminfo{A Davidic instruction.\fnote{T Lit. \fbib{maskil}}}
\passage{The Blessings of Forgiveness}

\begin{poetry}
\poeml \v{1}How blessed is the one whose transgression is forgiven, \\
\poemll    whose sin is covered. \\
\poeml \v{2}How blessed is the person against whom the \divine{Lord} does not charge iniquity, \\
\poemll    and in whose spirit there is no deceit. \\
\poeml \v{3}When I kept silent about my sin,\fnote{The Heb. lacks \fbib{about my sin}} \\
\poemll    my body\fnote{Lit. \fbib{bones}} wasted away \\
\poemlll       by my groaning all day long. \\
\poeml \v{4}For your hand was heavy upon me day and night; \\
\poemll    my strength was exhausted \\
\poemlll       as in a summer drought.
\end{poetry}
\interlude{Interlude}

\begin{poetry}
\poeml \v{5}My sin I acknowledged to you; \\
\poemll    my iniquity I did not hide. \\
\poeml I said, ``I will confess my transgressions to the \divine{Lord}.'' \\
\poemll    And you forgave the guilt of my sin!
\end{poetry}
\interlude{Interlude}

\begin{poetry}
\poeml \v{6}Therefore every godly person should pray to you at such a time.\fnote{Lit. \fbib{at a time of finding}} \\
\poemll    Surely a flood of great waters will not reach him. \\
\poeml \v{7}You are my hiding place; \\
\poemll    you will deliver me from trouble \\
\poemlll       and surround me with shouts of deliverance.
\end{poetry}
\interlude{Interlude}

\begin{poetry}
\poeml \v{8}I will instruct you and teach you \\
\poemll    concerning the path you should walk; \\
\poemlll       I will direct you with my eye. \\
\poeml \v{9}Don't be like a horse or mule, \\
\poemll    without understanding. \\
\poeml They are held in check by a bit and bridle in their mouths; \\
\poemll    otherwise they will not remain near you. \\
\poeml \v{10}The wicked have many sorrows, \\
\poemll    but gracious love surrounds those who trust in the \divine{Lord}. \\
\poeml \v{11}Righteous ones, be glad in the \divine{Lord} and rejoice! \\
\poemll    Shout for joy, all of you who are upright in heart!
\end{poetry}
\labelpsalm{33}
\passage{Praise to the Creator and Deliverer}

\begin{poetry}
\poeml \v{1}Rejoice in the \divine{Lord}, righteous ones; \\
\poemll    for the praise of the upright is beautiful. \\
\poeml \v{2}With the lyre, give thanks to the \divine{Lord}; \\
\poemll    with the ten stringed harp, play music to him; \\
\poeml \v{3}with a new song, sing to him; \\
\poemll    with shouts of joy, play skillfully. \\
\poeml \v{4}For the word of the \divine{Lord} is upright; \\
\poemll    and all his works are done in faithfulness. \\
\poeml \v{5}He loves righteousness and justice; \\
\poemll    the world is filled with the gracious love of the \divine{Lord}. \\
\poeml \v{6}By the word of the \divine{Lord} the heavens were made; \\
\poemll    all the heavenly bodies\fnote{Lit. \fbib{all their host}} by the breath of his mouth. \\
\poeml \v{7}He gathered the oceans into a single place; \\
\poemll    he put the deep water into storehouses. \\
\poeml \v{8}Let all the world fear the \divine{Lord}; \\
\poemll    let all the inhabitants of the world stand in awe of him; \\
\poeml \v{9}because he spoke and it came to be, \\
\poemll    because he commanded, it stood firm. \\
\poeml \v{10}The \divine{Lord} makes void the counsel of nations; \\
\poemll    he frustrates the plans of peoples. \\
\poeml \v{11}But the \divine{Lord}'s counsel stands firm forever, \\
\poemll    the plans in his mind for all generations. \\
\poeml \v{12}How blessed is the nation whose God is the \divine{Lord}, \\
\poemll    the people he has chosen as his own inheritance. \\
\poeml \v{13}When the \divine{Lord} looks down from heaven, \\
\poemll    he observes every human being. \\
\poeml \v{14}From his dwelling place, \\
\poemll    he looks down on all the inhabitants of the earth. \\
\poeml \v{15}He formed the hearts of them all; \\
\poemll    he understands everything they do. \\
\poeml \v{16}A king is not saved by a large army; \\
\poemll    a mighty soldier is not delivered by his great strength. \\
\poeml \v{17}It is vain to trust in a horse for deliverance, \\
\poemll    even with its great strength, it cannot deliver. \\
\poeml \v{18}Indeed, the \divine{Lord} watches those who fear him; \\
\poemll    those who trust in his gracious love \\
\poeml \v{19}to deliver them from death; \\
\poemll    to keep them alive in times of famine. \\
\poeml \v{20}We wait on the \divine{Lord}; \\
\poemll    he is our help and our shield. \\
\poeml \v{21}Indeed, our heart will rejoice in him, \\
\poemll    because we have placed our trust in his holy name. \\
\poeml \v{22}\divine{Lord}, may your gracious love be upon us, \\
\poemll    even as we hope in you.
\end{poetry}
\labelpsalm{34}
\psalminfo{By David, when he pretended to be insane before Abimelech, who drove him away. So David\fnote{T Lit. \fbib{he}} left.}
\passage{Learning about God's Deliverance}

\begin{poetry}
\poeml \v{1}\fnote{This Psalm is an acrostic poem.}I will bless the \divine{Lord} at all times; \\
\poemll    his praise will be in my mouth continuously. \\
\poeml \v{2}My soul will glorify the \divine{Lord}; \\
\poemll    the humble will hear about it and rejoice. \\
\poeml \v{3}Magnify the \divine{Lord} with me! \\
\poeml Let us lift up his name together! \\
\poeml \v{4}I sought the \divine{Lord} and he answered me; \\
\poemll    he delivered me from all of my fears. \\
\poeml \v{5}Look to him and be radiant; \\
\poemll    and you\fnote{Lit. \fbib{their faces}} will not be ashamed. \\
\poeml \v{6}This poor man cried out, and the \divine{Lord} heard \\
\poemll    and delivered him from all of his distress. \\
\poeml \v{7}The angel of the \divine{Lord} surrounds those who fear him, \\
\poemll    and he delivers them. \\
\poeml \v{8}Taste and see that the \divine{Lord} is good! \\
\poemll    How blessed is the person who trusts in him! \\
\poeml \v{9}Fear the \divine{Lord}, you holy ones of his; \\
\poemll    for those who fear him lack nothing. \\
\poeml \v{10}Young lions lack and go hungry, \\
\poemll    but those who seek the \divine{Lord} will never lack any good thing. \\
\poeml \v{11}Come, children, listen to me, \\
\poemll    and I will teach you the fear of the \divine{Lord}. \\
\poeml \v{12}Who among you\fnote{Lit. \fbib{Who is the person who}} desires life, \\
\poemll    and wants long life in order to see good? \\
\poeml \v{13}Then keep your tongue from doing evil \\
\poemll    and your lips from spreading lies. \\
\poeml \v{14}Avoid evil and do good! \\
\poemll    Seek peace and pursue it! \\
\poeml \v{15}The\fnote{Lit. \fbib{The eyes of the}} \divine{Lord} looks on the righteous, \\
\poemll    and he listens to their cries. \\
\poeml \v{16}The face of the \divine{Lord} is set against those who do evil, \\
\poemll    and he will remove people's recollection of them from the earth. \\
\poeml \v{17}The \divine{Lord} hears those who cry out, \\
\poemll    and he delivers them from all their distress. \\
\poeml \v{18}The \divine{Lord} is close to the brokenhearted, \\
\poemll    and he delivers those whose spirit has been crushed. \\
\poeml \v{19}A righteous person will have many troubles, \\
\poemll    but the \divine{Lord} will deliver him from them all. \\
\poeml \v{20}God\fnote{Lit. \fbib{He}} protects all his bones; \\
\poemll    not one of them will be broken. \\
\poeml \v{21}Evil will kill the wicked; \\
\poemll    those who hate the righteous will be held guilty. \\
\poeml \v{22}The \divine{Lord} redeems the lives of his servants; \\
\poemll    and none of those who trust in him will be held guilty.
\end{poetry}
\labelpsalm{35}
\psalminfo{Davidic}
\passage{A Prayer for Deliverance}

\begin{poetry}
\poeml \v{1}Argue my case,\fnote{The Heb. lacks \fbib{my case}} \divine{Lord}, \\
\poemll    against those who argue against me. \\
\poemlll       Fight against those who fight against me. \\
\poeml \v{2}Take up the buckler\fnote{I.e. a small shield} and the shield, \\
\poemll    and rise up to help me. \\
\poeml \v{3}Take out the spear and the ax to confront the one who pursues me; \\
\poemll    say to me, ``I am your deliverer!'' \\
\poeml \v{4}Let those who seek my life be ashamed and disgraced; \\
\poemll    let those who plot evil against me be driven back and confounded. \\
\poeml \v{5}Make them like the chaff before the wind, \\
\poemll    as the messenger of the \divine{Lord} pushes them aside. \\
\poeml \v{6}May their path be dark and slippery, \\
\poemll    as the messenger of the \divine{Lord} tracks them down. \\
\poeml \v{7}Without justification they laid a snare for me; \\
\poemll    without justification they dug a pit to trap me. \\
\poeml \v{8}Let destruction come upon them\fnote{Lit. \fbib{him}} unawares, \\
\poemll    and let the net that he hid catch him; \\
\poemlll       let him fall into destruction. \\
\poeml \v{9}My soul will rejoice in the \divine{Lord} \\
\poemll    and be glad in his deliverance. \\
\poeml \v{10}All my bones will say, \\
\poemll    ``\divine{Lord}, who is like you? \\
\poeml Who delivers the weak from the one who is stronger than he, \\
\poemll    and the weak and the needy from the one who wants to rob him?'' \\
\poeml \v{11}False witnesses stepped forward \\
\poemll    and questioned me concerning things \\
\poemlll       about which I knew nothing. \\
\poeml \v{12}They paid me back evil for good; \\
\poemll    my soul mourns. \\
\poeml \v{13}But when they were sick, \\
\poemll    I wore sackcloth, humbled myself with fasting, \\
\poemlll       and prayed from my heart repeatedly for them.\fnote{The Heb. lacks \fbib{for them}} \\
\poeml \v{14}I paced about as for my friend or my brother, \\
\poemll    and fell down mourning as one weeps for one's mother. \\
\poeml \v{15}But when I stumbled, \\
\poemll    they rejoiced and gathered together. \\
\poeml They gathered together against me--- \\
\poemll    attackers whom I did not know. \\
\poemlll       They tore me apart and would not stop. \\
\poeml \v{16}Malicious mockers\fnote{So LXX; DSS 4QPs\textsuperscript{a} read \fbib{They mocked me viciously}; MT reads \fbib{Mockers of cake}}--- \\
\poemll    they gnashed\fnote{So DSS 4QPs\textsuperscript{a} LXX; MT reads \fbib{gnashing}} their teeth against me. \\
\poeml \v{17}Lord, how long will you just watch? \\
\poemll    Rescue me from their destruction, \\
\poemlll       my precious life from these young lions. \\
\poeml \v{18}Then I will give you thanks in front of the great congregation; \\
\poemll    in the midst of the mighty throng I will praise you. \\
\poeml \v{19}Do not let my deceitful enemies gloat over me, \\
\poemll    nor let those who hate me without justification mock me with their eyes. \\
\poeml \v{20}For they do not speak peace; \\
\poemll    they devise clever lies against the peaceful people of the land. \\
\poeml \v{21}They open their mouth wide against me, \\
\poemll    claiming, ``Yes! Yes! We saw him do\fnote{The Heb. lacks \fbib{him do}} it with our own eyes!'' \\
\poeml \v{22}You see this, \divine{Lord}, \\
\poemll    so do not be silent. \\
\poemlll       Lord, do not be far from me! \\
\poeml \v{23}Wake up! Arouse yourself to vindicate me \\
\poemll    and argue my case, my God and my Lord. \\
\poeml \v{24}Judge me according to your righteousness, \divine{Lord} my God! \\
\poemll    But do not let them gloat over me. \\
\poeml \v{25}Don't let them say in their hearts, \\
\poemll    ``Yes! We got what we wanted.'' \\
\poeml Don't let them say, \\
\poemll    ``We have swallowed him up.'' \\
\poeml \v{26}Instead, let those who gloat over the evil directed against me \\
\poemll    be ashamed and confounded together; \\
\poeml Let those who exalt themselves over me \\
\poemll    be clothed with shame and dishonor. \\
\poeml \v{27}Let those who delight in my vindication \\
\poemll    shout for joy and rejoice! \\
\poeml Let them continuously say, \\
\poemll    ``Magnify the \divine{Lord}, who delights in giving peace to\fnote{So MT; DSS 4QPsa LXX read \fbib{\divine{Lord}, you who delight in the welfare of}} his servant.'' \\
\poeml \v{28}My tongue will declare your righteousness \\
\poemll    and praise you all day long.
\end{poetry}
\labelpsalm{36}
\psalminfo{To the Director: By the servant of the \divine{Lord}, David.}
\passage{An Oracle from the \divine{Lord}}

\begin{poetry}
\poeml \v{1}An oracle that came to me\fnote{So MT DSS 4QPs\textsuperscript{a}; lit. \fbib{oracle in the midst of my heart}; Syr Origen read \fbib{of his heart}} about the transgressions of the wicked: \\
\poeml There is no fear of God before his eyes. \\
\poeml \v{2}He flatters himself\fnote{Lit. \fbib{himself in his own eyes}} too much\fnote{The Heb. lacks \fbib{too much}} to discover his transgression and hate it. \\
\poeml \v{3}The words from his mouth are vain and deceptive. \\
\poemll    He has abandoned behaving wisely and doing good. \\
\poeml \v{4}He devises iniquity on his bed \\
\poemll    and is determined to follow a path that is not good. \\
\poemlll       He does not resist evil.
\passage{Praise to the \divine{Lord}}
\poeml \v{5}Your gracious love, \divine{Lord}, reaches to the heavens; \\
\poemll    your truth\fnote{Or \fbib{faithfulness}} extends to the skies.\fnote{Or \fbib{clouds}} \\
\poeml \v{6}Your righteousness is like the mountains of God; \\
\poemll    your justice is like the great depths of the sea.\fnote{The Heb. lacks \fbib{of the sea}} \\
\poemlll       You deliver both\fnote{The Heb. lacks \fbib{both}} people and animals, \divine{Lord}. \\
\poeml \v{7}How precious is your gracious love, God! \\
\poemll    The children of men take refuge in the shadow of your wings. \\
\poeml \v{8}They are refreshed from the abundance of your house; \\
\poemll    You cause them to drink from the river of your pleasures. \\
\poeml \v{9}For with you is a fountain of life, \\
\poemll    and in your light we will see light. \\
\poeml \v{10}Send forth your gracious love to those who know you, \\
\poemll    and your righteousness to those who are upright in heart. \\
\poeml \v{11}Do not let the foot of the proud crush me; \\
\poemll    and do not let the hand of the wicked dissuade me. \\
\poeml \v{12}There, those who do evil have fallen; \\
\poemll    They have been thrown down, \\
\poemlll       and they cannot get up.
\end{poetry}
\labelpsalm{37}
\psalminfo{Davidic\fnote{T This acrostic psalm begins each verse with a consecutive letter of the Hebrew alphabet}}
\passage{Patiently Trust in God}

\begin{poetry}
\poeml \v{1}Don't be angry because of those who do evil, \\
\poemll    do not be jealous because of those who commit iniquity. \\
\poeml \v{2}Indeed, they soon will wither like grass, \\
\poemll    and like green herbs they will fade away. \\
\poeml \v{3}Trust in the \divine{Lord} and do good. \\
\poemll    Dwell in the land and feed on faithfulness. \\
\poeml \v{4}Delight yourself in the \divine{Lord}, \\
\poemll    and he will give you the desires of your heart. \\
\poeml \v{5}Commit your way to the \divine{Lord}; \\
\poemll    Trust him, and he will act. \\
\poeml \v{6}He will bring forth your righteousness as a light, \\
\poemll    and your justice as the noonday sun.\fnote{The Heb. lacks \fbib{sun}} \\
\poeml \v{7}Be silent in the \divine{Lord}'s presence \\
\poemll    and wait patiently for him. \\
\poeml Don't be angry because of the one whose way prospers \\
\poemll    or the one who implements evil schemes. \\
\poeml \v{8}Calm your anger and abandon wrath. \\
\poemll    Don't be angry--- \\
\poemlll       it only leads to evil. \\
\poeml \v{9}Those who do evil will perish. \\
\poemll    But those who wait\fnote{I.e. \fbib{trust}} on the \divine{Lord} will inherit the land. \\
\poeml \v{10}Yet a little while longer, \\
\poemll    and the wicked will be no more. \\
\poeml You will search for his place, \\
\poemll    but he will not be there. \\
\poeml \v{11}The humble will inherit the land; \\
\poemll    they will take in abundant peace. \\
\poeml \v{12}The wicked person plots against the righteous, \\
\poemll    and grinds his teeth at him. \\
\poeml \v{13}But the Lord laughs at him \\
\poemll    because he sees that his day is coming! \\
\poeml \v{14}The wicked take out a sword and bend the bow, \\
\poemll    to bring down the humble and the poor \\
\poemlll       to slay those who are righteous in conduct. \\
\poeml \v{15}But their sword will pierce their own heart, \\
\poemll    and their bows will be broken! \\
\poeml \v{16}Better is the little that the righteous have \\
\poemll    than the abundance of many wicked people. \\
\poeml \v{17}For the arms of the wicked will be broken, \\
\poemll    but the \divine{Lord} upholds the righteous. \\
\poeml \v{18}The \divine{Lord} knows the day of the blameless, \\
\poemll    and their inheritance will last forever. \\
\poeml \v{19}They will not experience shame in times of trouble; \\
\poemll    in times of famine they will have plenty. \\
\poeml \v{20}Indeed, the wicked will perish. \\
\poemll    The \divine{Lord}'s enemies will be consumed like flowers\fnote{Lit. \fbib{like glorious things}} in the fields. \\
\poemlll       They will vanish like\fnote{So LXX DSS 4QpPs\textsuperscript{a}; MT reads \fbib{in}} smoke. \\
\poeml \v{21}The wicked borrow but never pay back; \\
\poemll    but the righteous are generous and give. \\
\poeml \v{22}For those blessed by God\fnote{Lit. \fbib{him}} will inherit the land, \\
\poemll    but those cursed by him will be cut off. \\
\poeml \v{23}A man's steps are established by the \divine{Lord}, \\
\poemll    and the \divine{Lord}\fnote{Lit. \fbib{he}} delights in his way. \\
\poeml \v{24}Though he stumbles, \\
\poemll    he will not fall down flat, \\
\poemlll       for the \divine{Lord} will hold up his hand. \\
\poeml \v{25}I once was young and now I am old, \\
\poemll    but I have not seen a righteous person forsaken \\
\poemlll       or his descendants begging for bread. \\
\poeml \v{26}Every day he is generous, lending freely, \\
\poemll    and his descendants are blessed. \\
\poeml \v{27}Depart from evil, and do good, \\
\poemll    and you will live in the land\fnote{The Heb. lacks \fbib{in the land}} forever. \\
\poeml \v{28}Indeed, the \divine{Lord} loves justice, \\
\poemll    and he will not abandon his godly ones. \\
\poeml They are kept safe forever, \\
\poemll    but the lawless will be chased away,\fnote{So LXX DSS 4QpPs\textsuperscript{a}; the Heb. lacks this line} \\
\poemlll       and the descendants of the wicked will be cut off. \\
\poeml \v{29}The righteous will inherit the land, \\
\poemll    and they will dwell in it forever. \\
\poeml \v{30}The mouth of the righteous one produces wisdom; \\
\poemll    his tongue speaks justice. \\
\poeml \v{31}The instruction\fnote{Or \fbib{law}} of his God is in his heart; \\
\poemll    his steps will not slip. \\
\poeml \v{32}The wicked stalks the righteous person, seeking to kill him, \\
\poeml \v{33}but the \divine{Lord} will not let him fall into his hands. \\
\poemlll       He will not be condemned when he is put on trial. \\
\poeml \v{34}Wait on the \divine{Lord}, \\
\poemll    Keep faithful to his way, \\
\poemlll       and he will exalt you to possess the land. \\
\poeml You will see the wicked cut off. \\
\poeml \v{35}I once observed a wicked and oppressive person, \\
\poemll    flourishing like a green tree in native soil. \\
\poeml \v{36}But then he\fnote{So MT; LXX 4QpPs\textsuperscript{a} read \fbib{I}} passed away;\fnote{So MT; LXX reads \fbib{I passed by}; Syr Hieronymus DSS 4QpPs\textsuperscript{a} read \fbib{I passed by in front of him}} \\
\poemll    in fact, he simply was not there. \\
\poeml When I looked for him, \\
\poemll    he could not be found. \\
\poeml \v{37}Observe the blameless! \\
\poemll    Take note of the upright! \\
\poemlll       Indeed, the future of that man is peace. \\
\poeml \v{38}Sinners will be destroyed together; \\
\poemll    the future of the wicked will be cut off. \\
\poeml \v{39}But deliverance for the righteous one comes from the \divine{Lord}; \\
\poemll    he is their strength in times of distress. \\
\poeml \v{40}The \divine{Lord} helps and delivers them; \\
\poemll    he will deliver them from the wicked, \\
\poemlll       and he will save them because they have sought refuge in him.
\end{poetry}
\labelpsalm{38}
\psalminfo{A Davidic Psalm: As a Reminder.}
\passage{The Outcast Cries Out}

\begin{poetry}
\poeml \v{1}\divine{Lord}! Do not rebuke me in your anger; \\
\poemll    do not correct me in your wrath, \\
\poeml \v{2}because your arrows have sunk deep into me, \\
\poemll    and your hand has come down hard on me. \\
\poeml \v{3}My body is unhealthy due to your anger, \\
\poemll    and my bones have no rest due to my sin. \\
\poeml \v{4}My iniquities loom over my head; \\
\poemll    like a cumbersome burden, they are too heavy for me. \\
\poeml \v{5}My wounds have putrefied and festered \\
\poemll    because of my foolishness. \\
\poeml \v{6}I am bent over and walk about greatly bowed down; \\
\poemll    all day long I go around mourning. \\
\poeml \v{7}My insides\fnote{Lit. \fbib{loins}} are burning \\
\poemll    and my body is unhealthy. \\
\poeml \v{8}I am weak and utterly crushed; \\
\poemll    I cry out in distress because of my heart's anguish. \\
\poeml \v{9}Lord, all my longings are before you, \\
\poemll    and my groaning is not hidden from you. \\
\poeml \v{10}My heart pounds, \\
\poemll    my strength fails me, \\
\poemlll       even the gleam in my eye is gone. \\
\poeml \v{11}As for my friends and my neighbors, \\
\poemll    they stand aloof from my distress; \\
\poemlll       even my close relatives stand at a distance. \\
\poeml \v{12}Those who seek my life lay snares for me; \\
\poemll    those who seek to do me harm brag all day long about their wicked planning. \\
\poeml \v{13}I am like the deaf, who cannot hear, \\
\poemll    and like the mute, who cannot open his mouth. \\
\poeml \v{14}Indeed, I have become like a man who hears nothing, \\
\poemll    and in whose mouth there is no rebuke. \\
\poeml \v{15}Because I have placed my hope in you, \divine{Lord}, \\
\poemll    you will answer, Lord, my God. \\
\poeml \v{16}For I said, ``Do not let them gloat over me, \\
\poemll    as they congratulate themselves when my foot slips.'' \\
\poeml \v{17}Indeed, I am being set up for a fall, \\
\poemll    and I am continuously reminded of my pain. \\
\poeml \v{18}I confess my iniquity, \\
\poemll    and my sin troubles me. \\
\poeml \v{19}But my enemies are alive and well;\fnote{So MT LXX; DSS 4QPs\textsuperscript{a} lack this line} \\
\poemll    those who hate me\fnote{So MT LXX; DSS 4QPs\textsuperscript{a} read \fbib{Those who are my enemies}} for no reason are numerous.\fnote{DSS 4QPs\textsuperscript{a} read \fbib{numerous, and many are those who hate me by deceiving me}; cf. Ps 35:19; 69:5} \\
\poeml \v{20}They\fnote{So LXX DSS 4QPs\textsuperscript{a}; MT reads \fbib{And they}} reward my good with evil, \\
\poemll    opposing me because I seek to do good.\fnote{So MT; DSS 4QPs\textsuperscript{a} read \fbib{evil plunder me instead of a good thing}} \\
\poeml \v{21}Don't forsake me, \divine{Lord}. \\
\poemll    My God, do not be so distant from me. \\
\poeml \v{22}Come quickly and help me, \\
\poemll    Lord, my deliverer.
\end{poetry}
\labelpsalm{39}
\psalminfo{To the Director: To Jeduthun. A Davidic Psalm.}
\passage{A Prayer about Life's Priorities}

\begin{poetry}
\poeml \v{1}I told myself, ``I will keep watch over my tongue to keep from sinning. \\
\poemll    I will muzzle my mouth when the wicked are around.'' \\
\poeml \v{2}I was as silent as a mute person; \\
\poemll    I said nothing, not even something good, \\
\poemlll       and my distress deepened. \\
\poeml \v{3}My heart within me became incensed;\fnote{Lit. \fbib{hot}} \\
\poemll    as I thought about it, the fire burned. \\
\poeml Then I\fnote{Lit. \fbib{Then my mouth}} spoke out: \\
\poeml \v{4}``\divine{Lord}, let me know how my life ends,\fnote{Lit. \fbib{my end}} \\
\poemll    and the standard by which you will measure\fnote{Lit. \fbib{the measure of}} my days, whatever it is! \\
\poemlll       Then I will know how transient my life is. \\
\poeml \v{5}Look, you have made my life span fit in your hand; \\
\poemll    It is nothing compared to yours. \\
\poemlll       Surely every person at their best is a puff of wind.
\end{poetry}
\interlude{Interlude}

\begin{poetry}
\poeml \v{6}In fact, people walk around as shadows. \\
\poemll    Surely, they busy themselves for nothing, \\
\poemlll       heaping up possessions but not knowing who will get them. \\
\poeml \v{7}How long, \divine{Lord}, will I wait expectantly? \\
\poemll    I have placed my hope in you. \\
\poeml \v{8}Deliver me from all my transgressions, \\
\poemll    and do not let fools scorn me.'' \\
\poeml \v{9}I remain silent; \\
\poemll    I do not open my mouth, \\
\poemlll       for you are the one who acted. \\
\poeml \v{10}Stop scourging me, \\
\poemll    since I have been crushed by your heavy hand. \\
\poeml \v{11}You rebuke by chastening a man with the consequence of iniquities; \\
\poemll    you destroy what is attractive to him, as one would treat a moth. \\
\poemlll       Indeed, every person is a puff of wind.
\end{poetry}
\interlude{Interlude}

\begin{poetry}
\poeml \v{12}Hear my prayer, \divine{Lord}, \\
\poemll    pay attention to my cry, \\
\poemlll       and do not ignore my tears. \\
\poeml I am an alien in your presence, \\
\poemll    a stranger just like my ancestors were. \\
\poeml \v{13}Stop looking at me with chastisement,\fnote{The Heb. lacks \fbib{with chastisement}} so I can smile again, \\
\poemll    before I depart and am no more.
\end{poetry}
\labelpsalm{40}
\psalminfo{To the Director: A Davidic Psalm.}
\passage{Prayer for Help and Praise to God}

\begin{poetry}
\poeml \v{1}I waited expectantly\fnote{Or \fbib{eagerly}} for the \divine{Lord}, \\
\poemll    and he took notice of me \\
\poemlll       and heard my cry. \\
\poeml \v{2}He plucked me out of a pit of confusion,\fnote{Or \fbib{destruction}} \\
\poemll    even out of the quicksand; \\
\poeml he placed my feet on a rock \\
\poemll    and established my steps. \\
\poeml \v{3}He put a new song in my mouth, \\
\poemll    praise to our God! \\
\poeml Many will watch and be in awe, \\
\poemll    and they will place their trust in the \divine{Lord}. \\
\poeml \v{4}How blessed is that strong person \\
\poemll    who places his trust in the \divine{Lord}, \\
\poemll    and who has not acknowledged the proud \\
\poemlll       nor resorted to lies. \\
\poeml \v{5}\divine{Lord}, my God, \\
\poemll    You have done great things: \\
\poemlll       marvelous works and your thoughts toward us. \\
\poeml There is no one who compares to you! \\
\poemll    I will try to recite your actions,\fnote{Lit. \fbib{recite them}} \\
\poemlll       even though there are too many to number. \\
\poeml \v{6}You take no delight in sacrifices and offerings--- \\
\poemll    you have prepared my ears to listen---\fnote{The Heb. lacks \fbib{to listen}} \\
\poemlll       you require no burnt offerings or sacrifices for sin. \\
\poeml \v{7}Then I said, ``Here I am! I have come! \\
\poemll    In the scroll of the book it is written about me. \\
\poeml \v{8}I delight to do your will, my God. \\
\poemll    Your Law is part of my inner being.'' \\
\poeml \v{9}In the great congregation I have proclaimed the righteous good news. \\
\poemll    Behold, I did not seal my lips, \divine{Lord}, as you know. \\
\poeml \v{10}I have not ignored\fnote{Lit. \fbib{not covered over}} your righteousness in my heart; \\
\poemll    instead, I have proclaimed your faithfulness and deliverance. \\
\poeml I have not concealed your gracious love and truthfulness \\
\poemll    from the great congregation. \\
\poeml \v{11}\divine{Lord}, do not withhold your mercy\fnote{Lit. \fbib{mercies}} from me, \\
\poemll    for your gracious love and truthfulness will keep me safe continuously. \\
\poeml \v{12}Innumerable evils have surrounded me; \\
\poemll    my iniquities have overtaken me so that I cannot see. \\
\poeml They are more in number than the hair on my head, \\
\poemll    and my courage\fnote{Lit. \fbib{heart}} has forsaken me. \\
\poeml \v{13}Be pleased, \divine{Lord}, to deliver me; \\
\poemll    \divine{Lord}, hurry up and help me! \\
\poeml \v{14}May those who seek to destroy my life be ashamed and confounded; \\
\poemll    let them be driven backwards and humiliated, \\
\poemlll       particularly those who wish me evil. \\
\poeml \v{15}Let shame be the reward for those who say to me, ``Aha! Aha!'' \\
\poeml \v{16}Let all who seek you shout for joy and be glad in you. \\
\poeml May those who love your deliverance say, \\
\poemll    ``The \divine{Lord} be magnified!'' continuously. \\
\poeml \v{17}But I am poor and needy; \\
\poemll    may the Lord think about me. \\
\poeml You are my help and deliverer. \\
\poemll    My God, do not tarry too long!
\end{poetry}
\labelpsalm{41}
\psalminfo{To the Director: A Davidic Psalm.}
\passage{When Things Go Wrong}

\begin{poetry}
\poeml \v{1}Blessed is the one who is considerate of the destitute;\fnote{Or \fbib{poor}} \\
\poemll    the \divine{Lord} will deliver him when the times are evil. \\
\poeml \v{2}The \divine{Lord} will protect him and keep him alive; \\
\poemll    he will be blessed in the land; \\
\poemlll       and he will not be handed over to the desires of his enemies. \\
\poeml \v{3}The \divine{Lord} will uphold him even on his sickbed; \\
\poemll    you will transform his bed of illness into health. \\
\poeml \v{4}As for me, I said, \\
\poemll    ``\divine{Lord}, be gracious to me! \\
\poemlll       Heal me, for I have sinned against you!'' \\
\poeml \v{5}As for my enemies, with malice they said, \\
\poemll    ``When will he die and memory of\fnote{The Heb. lacks \fbib{memory of}} his name perish?'' \\
\poeml \v{6}The one who comes to visit me speaks lies; \\
\poemll    in his heart he thinks slanderous things about me \\
\poemlll       and goes around spreading them. \\
\poeml \v{7}As for all who hate me, \\
\poemll    they whisper together against me; \\
\poemlll       they desire to do me harm. \\
\poeml \v{8}They say, ``Wickedness is entrenched in him. \\
\poemll    Once he is brought low, \\
\poemlll       he will not rise again.'' \\
\poeml \v{9}As for my best friend, \\
\poemll    the one in whom I trusted, \\
\poeml the one who ate my bread, \\
\poemll    even he has insulted\fnote{Lit. \fbib{has lifted up his heel against}} me! \\
\poeml \v{10}But you, \divine{Lord}, be gracious to me and raise me up \\
\poemll    so that I may pay them back! \\
\poeml \v{11}In this way I will know that you are pleased with me, \\
\poemll    and that my enemies will not shout in triumph over me. \\
\poeml \v{12}As for me, you will maintain my just cause, \\
\poemll    and you will cause me to stand in your presence forever. \\
\poeml \v{13}Blessed be the \divine{Lord} God of Israel, \\
\poemll    from eternity to eternity. \\
\poemlll       Amen and amen!
\end{poetry}
\booksection{BOOK II (Psalms 42-72)}
\labelpsalm{42}
\psalminfo{To the Director: An instruction\fnote{T Lit. \fbib{maskil}} of the Sons of Korah.}
\passage{Hope in God When Times of Trouble Come}

\begin{poetry}
\poeml \v{1}As an antelope pants for streams of water, \\
\poemll    so my soul pants for you, God. \\
\poeml \v{2}My soul thirsts for God, for the living God. \\
\poemll    When may I come and appear in God's presence? \\
\poeml \v{3}My tears have been my food day and night, \\
\poemll    while people\fnote{The Heb. lacks \fbib{people}} keep asking me all day long, \\
\poemlll       ``Where is your God?'' \\
\poeml \v{4}These things I will recall as I pour out my troubles\fnote{Lit. \fbib{soul}} within me: \\
\poemll    I used to go with the crowd in a procession to the house of God, \\
\poemlll       accompanied with shouts of joy and thanksgiving. \\
\poeml \v{5}Why are you in despair, my soul? \\
\poemll    Why are you disturbed within me? \\
\poeml Hope in God, \\
\poemll    for once again I will praise him, \\
\poemlll       since his presence saves me. \\
\poeml \v{6}My God, my soul feels depressed\fnote{Lit. \fbib{soul is bowed down}} within me; \\
\poemll    therefore I will remember you from the land of Jordan, \\
\poeml from the heights of Hermon, \\
\poemll    even from the foothills.\fnote{Or \fbib{from Mount Mizar}} \\
\poeml \v{7}Deep waters call out to what is deeper still;\fnote{Lit. \fbib{Deep calls to deep}} \\
\poemll    at the roar of your waterfalls \\
\poemlll       all your breakers and your waves swirled over me. \\
\poeml \v{8}By day the \divine{Lord} will command his gracious love, \\
\poemll    and by night his song is with me--- \\
\poemlll       a prayer to the God of my life. \\
\poeml \v{9}I will ask God, my Rock, ``Why have you forsaken me? \\
\poemll    Why do I go around mourning under the enemy's oppression?'' \\
\poeml \v{10}Like the shattering of my bones are the taunts of my oppressors, \\
\poemll    saying to me all day long, \\
\poemlll       ``Where is your God?'' \\
\poeml \v{11}Why are you in despair, my soul? \\
\poemll    Why are you disturbed within me? \\
\poeml Hope in God, \\
\poemll    for once again I will praise him, \\
\poeml since his presence saves me \\
\poemll    and he is my God.
\end{poetry}
\labelpsalm{43}
\passage{God is my Hope during Times of Trouble}

\begin{poetry}
\poeml \v{1}\fnote{Some Heb. MSS constitute Psalms 42 and 43 as a single psalm.}You be my judge,\fnote{Lit. \fbib{Judge me}} God, \\
\poemll    and plead my case against an unholy nation; \\
\poemlll       rescue me from the deceitful and unjust man. \\
\poeml \v{2}Since you are the God who strengthens me, \\
\poemll    why have you forsaken me? \\
\poeml Why do I go around mourning under the enemy's oppression?'' \\
\poeml \v{3}Send forth your light and your truth \\
\poemll    so they may guide me. \\
\poeml Let them bring me to your holy mountain and to your dwelling places.\fnote{Or \fbib{tents}} \\
\poeml \v{4}Then I will approach the altar of God, \\
\poemll    even to God in whom my joy finds its source.\fnote{Lit. \fbib{God who is the gladness of my joy}} \\
\poeml Then I will praise you with the lyre, \\
\poemll    God, my God, \\
\poeml \v{5}Why are you in despair, my soul? \\
\poemll    Why are you disturbed within me? \\
\poeml Hope in God, \\
\poemll    because I will praise him once again, \\
\poeml since his presence saves me \\
\poemll    and he is my God.
\end{poetry}
\labelpsalm{44}
\psalminfo{To the Director: An instruction\fnote{T Lit. \fbib{maskil}} of the Sons of Korah.}
\passage{A Prayer in Times of Defeat}

\begin{poetry}
\poeml \v{1}God, we heard it with our ears; \\
\poemll    our ancestors told us about what you did in their day--- \\
\poemlll       a long time ago. \\
\poeml \v{2}With your hand you expelled the nations \\
\poemll    and established our ancestors.\fnote{Lit. \fbib{them}} \\
\poeml You afflicted nations \\
\poemll    and cast them out. \\
\poeml \v{3}It was not with their sword that they inherited the land, \\
\poemll    nor did their own arm deliver them. \\
\poeml But it was by your power,\fnote{Lit. \fbib{right hand}} your strength, \\
\poemll    and by the light of your face; \\
\poemlll       because you were pleased with them. \\
\poeml \v{4}You are my king, God, \\
\poemll    command\fnote{So MT DSS 11QPs\textsuperscript{c}; LXX reads \fbib{truly my king and my God, who commands}} victories\fnote{Lit. \fbib{deliverances}} for Jacob. \\
\poeml \v{5}Through you we will knock down our oppressors; \\
\poemll    through your name we will tread down those who rise up against us. \\
\poeml \v{6}For I place no confidence in my bow, \\
\poemll    nor will my sword deliver me. \\
\poeml \v{7}For you delivered us from our oppressors \\
\poemll    and put to shame those who hate us. \\
\poeml \v{8}We will praise God all day long; \\
\poemll    and to your name we will give thanks forever.
\end{poetry}
\interlude{Interlude}

\begin{poetry}
\poeml \v{9}However, you cast us off and made us ashamed! \\
\poemll    You did not even march with our armies! \\
\poeml \v{10}You made us retreat from our oppressors. \\
\poemll    Our enemies ransacked us. \\
\poeml \v{11}You handed us over to be slaughtered like sheep \\
\poemll    and you scattered us among the nations. \\
\poeml \v{12}You sold out your people for nothing, \\
\poemll    and made no profit at that price. \\
\poeml \v{13}You made us a laughing stock to our neighbors, \\
\poemll    a source of mockery and derision to those around us. \\
\poeml \v{14}You made us an object lesson among the nations; \\
\poemll    people shake their heads at us.\fnote{The Heb. lacks \fbib{at us}} \\
\poeml \v{15}My dishonor tortures\fnote{Lit. \fbib{dishonor remains before}} me continuously;\fnote{Lit. \fbib{all the day}} \\
\poemll    the shame on my face overwhelms\fnote{Lit. \fbib{covers}} me \\
\poeml \v{16}because of the voice of the one who mocks and reviles, \\
\poemll    because of the enemy and the avenger. \\
\poeml \v{17}All this came upon us, \\
\poemll    yet we did not forsake you, \\
\poemlll       and we have not dealt falsely with your covenant; \\
\poeml \v{18}Our hearts have not turned away; \\
\poemll    our steps have not swerved from your path. \\
\poeml \v{19}Nevertheless, you crushed us in the lair of jackals, \\
\poemll    and covered us in deep darkness.\fnote{Or \fbib{in the shadow of death}} \\
\poeml \v{20}If we had forgotten the name of our God \\
\poemll    or lifted our hands to a foreign god, \\
\poeml \v{21}wouldn't God find out \\
\poemll    since he knows the secrets of the heart? \\
\poeml \v{22}For your sake we are being killed all day long. \\
\poemll    We are thought of as sheep to be slaughtered. \\
\poeml \v{23}Wake up! Why are you asleep, Lord? \\
\poemll    Get up! Don't cast us off forever! \\
\poeml \v{24}Why are you hiding your face? \\
\poemll    Why are you ignoring our affliction and oppression? \\
\poeml \v{25}For we\fnote{Lit. \fbib{our souls}} have collapsed in the dust; \\
\poemll    our bodies cling to the ground. \\
\poeml \v{26}Arise! Deliver us! \\
\poemll    Redeem us according to your gracious love!
\end{poetry}
\labelpsalm{45}
\psalminfo{To the Director: An instruction\fnote{T Lit. \fbib{maskil}} by the Sons of Korah. A love song to the tune of\fnote{T The Heb. lacks \fbib{the tune of}} ``Lilies''.}
\passage{A Royal Wedding Song}

\begin{poetry}
\poeml \v{1}My heart is overflowing with good news; \\
\poemll    I speak what I have composed to the king; \\
\poemlll       my tongue is like the pen of an articulate scribe. \\
\poeml \v{2}You are the most handsome of Adam's descendants; \\
\poemll    grace has anointed your lips; \\
\poemlll       therefore God has blessed you forever. \\
\poeml \v{3}Strap your sword to your side, \\
\poemll    mighty warrior, along with your honor and majesty. \\
\poeml \v{4}In your majesty ride forth for the cause of truth, humility, and righteousness; \\
\poemll    and your strong right hand will teach you awesome things. \\
\poeml \v{5}Your arrows are sharpened \\
\poemll    to penetrate the hearts of the king's enemies. \\
\poemlll       People will fall under you. \\
\poeml \v{6}Your throne, God, exists forever and ever, \\
\poemll    and the scepter of your kingdom is a righteous scepter. \\
\poeml \v{7}You love justice and hate wickedness. \\
\poemll    That is why God, even your God, has anointed you \\
\poemlll       rather than your companions with the oil of gladness. \\
\poeml \v{8}All your clothes are scented with\fnote{The Heb. lacks \fbib{are scented with}} myrrh, aloes, and cassia. \\
\poemll    From ivory palaces stringed instruments have made you glad. \\
\poeml \v{9}The king's daughters are among your honorable women; \\
\poemll    the queen, dressed in gold from Ophir, has taken her place at your right hand.'' \\
\poeml \v{10}Listen, daughter! Consider and pay attention. \\
\poemll    Forget your people and your father's house, \\
\poeml \v{11}and the king will greatly desire your beauty. \\
\poemll    Because he is your lord, you should bow in respect before him. \\
\poeml \v{12}The daughter\fnote{I.e. The people} of Tyre will come with\fnote{The Heb. lacks \fbib{will come with}} a wedding gift; \\
\poemll    wealthy people will entreat your favor. \\
\poeml \v{13}In her chamber,\fnote{The Heb. lacks \fbib{her chamber}} the king's daughter is glorious; \\
\poemll    her clothing is embroidered with gold thread. \\
\poeml \v{14}In embroidered garments \\
\poemll    she is presented to the king. \\
\poeml Her virgin companions who follow her train \\
\poemll    will be presented to you. \\
\poeml \v{15}Filled with joy and gladness, they are presented \\
\poemll    when they enter the king's palace. \\
\poeml \v{16}Your sons will take the place of your ancestors, \\
\poemll    and you will set them up as princes in all the earth. \\
\poeml \v{17}From generation to generation, \\
\poemll    I will cause your name to be remembered. \\
\poemlll       Therefore people will thank you forever and ever.
\end{poetry}
\labelpsalm{46}
\psalminfo{To the Director: A song by the Sons of Korah, to the tune of\fnote{T Lit. \fbib{according to}} ``The Maidens''.}
\passage{God is the Refuge of His People}

\begin{poetry}
\poeml \v{1}God is our refuge and strength, \\
\poemll    a great help in times of distress. \\
\poeml \v{2}Therefore we will not be frightened \\
\poemll    when the earth roars, \\
\poeml when the mountains shake in the depths of the seas, \\
\poeml \v{3}when its waters roar and rage, \\
\poemlll       when the mountains tremble despite their pride.\fnote{Or \fbib{tumult}}
\end{poetry}
\interlude{Interlude}

\begin{poetry}
\poeml \v{4}Look! There is a river \\
\poemll    whose streams make the city of God rejoice, \\
\poemlll       even the Holy Place of the Most High. \\
\poeml \v{5}Since God is in her midst, \\
\poemll    she will not be shaken. \\
\poeml God will help her \\
\poemll    at the break of dawn. \\
\poeml \v{6}The nations roared; \\
\poemll    the kingdoms were shaken. \\
\poeml His voice boomed; \\
\poemll    the earth melts. \\
\poeml \v{7}The \divine{Lord} of the heavenly armies is with us; \\
\poemll    our refuge is the God of Jacob.
\end{poetry}
\interlude{Interlude}

\begin{poetry}
\poeml \v{8}Come, observe the mighty works of the \divine{Lord}, \\
\poemll    who causes desolation in the earth. \\
\poeml \v{9}He causes wars to cease all over\fnote{Lit. \fbib{cease to the end of}} the earth, \\
\poemll    he causes the bow to break, the spear to snap, \\
\poemlll       the chariots to ignite and burn. \\
\poeml \v{10}Be in awe and know that I am God. \\
\poemll    I will be exalted among the nations. \\
\poemlll       I will be exalted throughout the earth. \\
\poeml \v{11}The \divine{Lord} of the heavenly armies is with us; \\
\poemll    the God of Jacob is our refuge.
\end{poetry}
\interlude{Interlude}
\labelpsalm{47}
\psalminfo{To the Director: A song by the Sons of Korah.}
\passage{The Ruler over the Nations}

\begin{poetry}
\poeml \v{1}Clap your hands, all you peoples! \\
\poemll    Shout to God with a loud cry of joy! \\
\poeml \v{2}For the \divine{Lord}, the Most High, is to be feared, \\
\poemll    a great king over all the earth. \\
\poeml \v{3}He subdued peoples under us, \\
\poemll    and nations under our feet. \\
\poeml \v{4}He chose our inheritance for us, \\
\poemll    even the pride of Jacob whom he loved.
\end{poetry}
\interlude{Interlude}

\begin{poetry}
\poeml \v{5}God has ascended on high with a shout, \\
\poemll    the \divine{Lord} has ascended\fnote{The Heb. lacks \fbib{has ascended}} with the blast of a trumpet. \\
\poeml \v{6}Sing songs to God! \\
\poemll    Sing songs! \\
\poeml Sing songs to our King! \\
\poemll    Sing songs! \\
\poeml \v{7}Indeed, God is king over all the earth; \\
\poemll    sing a song of praise. \\
\poeml \v{8}God is king over the nations; \\
\poemll    God is seated on his holy throne. \\
\poeml \v{9}The nobles among the nations \\
\poemll    have joined the people of the God of Abraham. \\
\poeml For the shields of the earth belong to God; \\
\poemll    he is greatly exalted.
\end{poetry}
\labelpsalm{48}
\psalminfo{A song: Lyrics\fnote{T Or \fbib{A song: A song}} by the Sons of Korah.}
\passage{Zion, City of God}

\begin{poetry}
\poeml \v{1}Great is the \divine{Lord}! \\
\poemll    For he is to be praised greatly, \\
\poeml even in the city of our God, \\
\poemll    his holy mountain. \\
\poeml \v{2}Beautifully situated, \\
\poemll    the joy of all the earth, \\
\poeml Mount Zion towards the north,\fnote{Or \fbib{on the northern side}} \\
\poemll    the city of the great King. \\
\poeml \v{3}Within her citadels \\
\poemll    God is known as a place of refuge. \\
\poeml \v{4}Behold, when the kings assembled together, \\
\poemll    when they traveled together, \\
\poeml \v{5}they looked and were awestruck; \\
\poemll    they became afraid and ran away. \\
\poeml \v{6}Trembling seized them there, \\
\poemll    pains like those of a woman in labor, \\
\poeml \v{7}as when an east wind destroyed the ships of Tarshish. \\
\poeml \v{8}Just as we have heard, \\
\poemll    so have we seen; \\
\poeml in the city of the \divine{Lord} of the heavenly armies--- \\
\poemll    even in the city of our God--- \\
\poemlll       God will establish her forever.
\end{poetry}
\interlude{Interlude}

\begin{poetry}
\poeml \v{9}God, we have meditated on your gracious love \\
\poemll    in the midst of your Temple. \\
\poeml \v{10}God, according to your name, \\
\poemll    so is your praise to the ends of the earth. \\
\poemlll       Your right hand is filled with righteousness. \\
\poeml \v{11}Mount Zion will be glad; \\
\poemll    the towns\fnote{Lit. \fbib{daughters}} of Judah will rejoice because of your judgments. \\
\poeml \v{12}March around Zion; \\
\poemll    encircle her; \\
\poemlll       count her towers. \\
\poeml \v{13}Take note of her ramparts; \\
\poemll    investigate her citadels; \\
\poemlll       that you may speak about them to the next generation. \\
\poeml \v{14}For this God is our God forever and ever. \\
\poemll    He will guide us until death.
\end{poetry}
\labelpsalm{49}
\psalminfo{To the Director: A song by the Sons of Korah.}
\passage{The Destiny of the Wicked and the Upright}

\begin{poetry}
\poeml \v{1}Listen to this, all you people! \\
\poemll    Pay attention, all you who live in the world, \\
\poeml \v{2}both average people and those of means,\fnote{Lit. \fbib{both sons of Adam and sons of men}} \\
\poemll    the rich and the poor together. \\
\poeml \v{3}My mouth will speak wisely, \\
\poemll    and I will understand what I think about. \\
\poeml \v{4}I will focus my attention on\fnote{Lit. \fbib{will incline my ear to}} a proverb; \\
\poemll    I will use the harp to expound my riddle. \\
\poeml \v{5}Why should I be afraid when evil days come my way, \\
\poemll    when the wickedness of those who deceive me surrounds me--- \\
\poeml \v{6}those who put confidence in their wealth \\
\poemll    and boast about their great riches? \\
\poeml \v{7}No man can redeem the life of another,\fnote{Lit. \fbib{of a brother}} \\
\poemll    nor can he give to God a sufficient payment for him--- \\
\poeml \v{8}for it would cost too much to redeem his life, \\
\poemll    and the payments would go on forever--- \\
\poeml \v{9}that he should go on living \\
\poemll    and not see corruption. \\
\poeml \v{10}Indeed, he will see wise people die; \\
\poemll    the stupid and the senseless will meet their doom \\
\poemlll       and leave their wealth to others. \\
\poeml \v{11}Their inner thoughts are on\fnote{So MT DSS 4QPs\textsuperscript{c} 4QPs\textsuperscript{j}; LXX reads \fbib{Their graves are}} their homes forever; \\
\poemll    their dwellings from generation to generation. \\
\poemlll       They even name their lands after themselves. \\
\poeml \v{12}But humanity cannot last, despite its conceit;\fnote{So MT; DSS 4QPs\textsuperscript{c} Syr LXX read \fbib{Humans, held in honor, had no understanding;}} \\
\poemll    it will pass away just like the animals.\fnote{So MT; LXX reads \fbib{they resembled senseless animals, and became like them}; DSS 4QPs\textsuperscript{c} read \fbib{they are like animals that perish}} \\
\poeml \v{13}This is the fate of those who are foolish \\
\poemll    and of those who correct their words after they speak.
\end{poetry}
\interlude{Interlude}

\begin{poetry}
\poeml \v{14}Like sheep, they are destined for the realm of the dead,\fnote{Lit. \fbib{for Sheol}; i.e. the realm of the dead} \\
\poemll    with death as their shepherd. \\
\poeml The upright will have dominion over them in the morning; \\
\poemll    their strength will be consumed in the afterlife,\fnote{Lit. \fbib{in Sheol}; i.e. the realm of the dead} \\
\poemlll       so that they have no home. \\
\poeml \v{15}God will truly redeem me from the power\fnote{Lit. \fbib{hand}} of Sheol.\fnote{I.e. the realm of the dead} \\
\poemll    He will surely receive me!
\end{poetry}
\interlude{Interlude}

\begin{poetry}
\poeml \v{16}Don't be afraid when someone gets rich, \\
\poemll    when the glory of his household increases. \\
\poeml \v{17}When he dies, he will not be able to take it all with him\fnote{The Heb. lacks \fbib{with him}}--- \\
\poemll    his possessions\fnote{Or \fbib{glory}} will not follow him to the grave,\fnote{The Heb. lacks \fbib{to the grave}} \\
\poeml \v{18}although he considers himself blessed while he's alive. \\
\poeml Though people praise you for doing well, \\
\poeml \v{19}you will end up like your\fnote{Lit. \fbib{his}} ancestors' generation, \\
\poemlll       never again to see the light of day! \\
\poeml \v{20}Humanity, despite its conceit, does not understand \\
\poemll    that it will perish, just like the animals.
\end{poetry}
\labelpsalm{50}
\psalminfo{A song of Asaph.}
\passage{The Acceptable Sacrifice}

\begin{poetry}
\poeml \v{1}God, the \divine{Lord},\fnote{Or \fbib{The mighty God}} has spoken. \\
\poemll    He has summoned the earth \\
\poemlll       from the rising of the sun to its setting place. \\
\poeml \v{2}From Zion, the perfection of beauty, \\
\poemll    God has shined forth. \\
\poeml \v{3}Our God has appeared and he has not been silent; \\
\poemll    a devouring fire blazed before him, \\
\poemlll       and a mighty storm swirled around him. \\
\poeml \v{4}He summoned the heavens above \\
\poemll    and the earth below,\fnote{The Heb. lacks \fbib{below}} \\
\poemlll       to sit in judgment on his people. \\
\poeml \v{5}``Assemble before me, my saints, \\
\poemll    who have entered into my covenant by sacrifice.'' \\
\poeml \v{6}The heavens revealed his justice, \\
\poemll    for God is himself the judge.
\end{poetry}
\interlude{Interlude}

\begin{poetry}
\poeml \v{7}``Listen, my people, \\
\poemll    for I am making a pronouncement: \\
\poemlll       Israel, I, God, your God, am testifying against you. \\
\poeml \v{8}I do not rebuke you because of your sacrifices; \\
\poemll    indeed, your burnt offerings are continuously before me. \\
\poeml \v{9}I will no longer accept a sacrificial\fnote{The Heb. lacks \fbib{sacrificial}} bull from your household; \\
\poemll    nor goats from your pens. \\
\poeml \v{10}Indeed, every animal of the forest is mine, \\
\poemll    even the cattle on a thousand hills. \\
\poeml \v{11}I know all the birds in the mountains; \\
\poemll    indeed, everything that moves in the field is mine. \\
\poeml \v{12}``If I were hungry, I would not tell you; \\
\poemll    for the world is mine along with everything in it. \\
\poeml \v{13}Why should I eat the flesh of oxen \\
\poemll    or drink the blood of goats? \\
\poeml \v{14}Offer to God a thanksgiving praise; \\
\poemll    pay your vows to the Most High. \\
\poeml \v{15}Call on me in the day of distress; \\
\poemll    I will deliver you, and you will glorify me.'' \\
\poeml \v{16}As for the wicked, God says, \\
\poeml ``How dare you recite my statutes \\
\poemll    or speak about my covenant with your lips! \\
\poeml \v{17}You hate instruction \\
\poemll    and toss my words behind you. \\
\poeml \v{18}When you see a thief, you befriend him, \\
\poemll    and you keep company with adulterers. \\
\poeml \v{19}You give your mouth free reign for evil, \\
\poemll    and your tongue devises deceit. \\
\poeml \v{20}You sit and speak against your brother; \\
\poemll    you slander your own mother's son. \\
\poeml \v{21}These things you did, and I kept silent, \\
\poemll    because you assumed that I was like you. \\
\poeml But now I am going to rebuke you, \\
\poemll    and I will set forth my case before your very own eyes.'' \\
\poeml \v{22}Consider this, you who have forgotten God--- \\
\poemll    Otherwise, I will tear you in pieces \\
\poemlll       and there will be no deliverer: \\
\poeml \v{23}Whoever offers a sacrifice of thanksgiving glorifies me, \\
\poemll    and I will reveal the salvation of God \\
\poemlll       to whomever continues in my way.''\fnote{Lit. \fbib{sets a way}}
\end{poetry}
\labelpsalm{51}
\psalminfo{To the Director: A Davidic Psalm. When the prophet Nathan came to him, after he had gone in to Bathsheba.}
\passage{A Prayer for Cleansing and Pardon}

\begin{poetry}
\poeml \v{1}Have mercy, God, according to your gracious love, \\
\poemll    according to your unlimited compassion, \\
\poemlll       erase my transgressions. \\
\poeml \v{2}Wash me from my iniquity, \\
\poemll    cleanse me from my sin. \\
\poeml \v{3}For I acknowledge my transgression; \\
\poemll    my sin remains continuously before me. \\
\poeml \v{4}Against you, you only, have I sinned, \\
\poemll    and done what was evil in your sight. \\
\poeml As a result, you are just in your pronouncement \\
\poemll    and clear in your judgment. \\
\poeml \v{5}Indeed, in iniquity I was brought forth; \\
\poemll    in sin my mother conceived me. \\
\poeml \v{6}Indeed, you are pleased with truth in the inner person, \\
\poemll    and you will teach me wisdom in my\fnote{The Heb. lacks \fbib{my}} innermost parts. \\
\poeml \v{7}Purge me with hyssop, \\
\poemll    and I will be clean. \\
\poeml Wash me, \\
\poemll    and I will be whiter than snow. \\
\poeml \v{8}Let me know\fnote{Lit. \fbib{hear}} joy and gladness; \\
\poemll    let the bones that you have broken rejoice. \\
\poeml \v{9}Hide your countenance from my sins \\
\poemll    and erase the record of my iniquities. \\
\poeml \v{10}God, create a pure heart in me, \\
\poemll    and renew a right attitude within me. \\
\poeml \v{11}Do not cast me from your presence; \\
\poemll    do not take your Holy Spirit from me. \\
\poeml \v{12}Restore to me the joy of your salvation, \\
\poemll    and let a willing attitude control me. \\
\poeml \v{13}Then I will teach transgressors about your ways, \\
\poemll    and sinners will turn to you. \\
\poeml \v{14}Deliver me from the guilt of shedding blood,\fnote{Lit. \fbib{from bloods}} \\
\poemll    God, God of my salvation. \\
\poemlll       Then my tongue will sing about your righteousness. \\
\poeml \v{15}Lord, open my lips, \\
\poemll    and my mouth will declare your praise. \\
\poeml \v{16}Indeed, you do not delight in sacrifices, \\
\poemll    or I would give them, \\
\poemlll       nor do you desire burnt offerings. \\
\poeml \v{17}True sacrifice to God\fnote{Lit. \fbib{The sacrifice of God}} is a broken spirit. \\
\poemll    A broken and chastened heart, God, \\
\poemlll       you will not despise. \\
\poeml \v{18}Show favor to Zion in your good pleasure; \\
\poemll    and rebuild the walls of Jerusalem. \\
\poeml \v{19}Then you will be pleased with right sacrifices, \\
\poemll    with burnt offerings, and with whole burnt offerings. \\
\poemlll       Then they will offer bulls on your altar.
\end{poetry}
\labelpsalm{52}
\psalminfo{To the Director: A Davidic instruction\fnote{T Lit. \fbib{maskil}} about Doeg, the Edomite, when he went to Saul and told him, ``David went to the house of Abimelech.''}
\passage{A Rebuke to the Deceitful}

\begin{poetry}
\poeml \v{1}Why do you make evil \\
\poemll    the foundation of your boasting, mighty one?\fnote{Or \fbib{O warrior}} \\
\poemlll       God's gracious love never ceases.\fnote{Lit. \fbib{love is all the day}}
\end{poetry}

\begin{poetry}
\poeml \v{2}Your tongue, like a sharp razor, devises wicked things \\
\poemll    and crafts treachery. \\
\poeml \v{3}You love evil rather than good, \\
\poemll    falsehood rather than speaking uprightly.
\end{poetry}
\interlude{Interlude}

\begin{poetry}
\poeml \v{4}You love all words that destroy, you deceitful tongue! \\
\poeml \v{5}But God will tear you down forever; \\
\poemll    he will take you away, \\
\poemlll       even snatching you out of your tent! \\
\poeml He will uproot you from the land of the living.
\end{poetry}
\interlude{Interlude}

\begin{poetry}
\poeml \v{6}The righteous will fear when they see this, \\
\poemll    but then they will laugh at him, saying, \\
\poeml \v{7}``Look, here is a young man who refused to make God his strength; \\
\poemll    instead, he trusted in his great wealth \\
\poemlll       and made his wickedness his strength. \\
\poeml \v{8}But I am like a green olive tree in the house of God; \\
\poemll    I trust in the gracious love of God forever and ever. \\
\poeml \v{9}Therefore I will praise you forever \\
\poemll    because of what you did; \\
\poeml I will proclaim that your name is good \\
\poemll    in the midst of your faithful ones.
\end{poetry}
\labelpsalm{53}
\psalminfo{To the Director: Upon machalath.\fnote{T A Heb. musical term} A Davidic instruction.\fnote{T Lit. \fbib{maskil}}}
\passage{The Fool and God's Response}

\begin{poetry}
\poeml \v{1}Fools say to themselves ``There is no God.'' \\
\poemll    They are corrupt and commit iniquity; \\
\poemlll       not one of them practices what is good. \\
\poeml \v{2}God looks down from the heavens upon humanity\fnote{Lit. \fbib{upon the sons of Adam}} \\
\poemll    to see if anyone shows discernment as he searches for God. \\
\poeml \v{3}All of them\fnote{So MT; DSS 4QPs\textsuperscript{a} lack \fbib{of them}} have fallen away; \\
\poemll    together they have become corrupt; \\
\poemlll       no one does what is good, not even one. \\
\poeml \v{4}Will those who do evil ever learn? \\
\poemll    They devour my people like they devour bread, \\
\poemlll       and never call on God. \\
\poeml \v{5}There the Israelis\fnote{Lit. \fbib{they}} were seized with terror, \\
\poemll    when there was nothing to fear. \\
\poeml For God scattered the bones of those who laid siege against you\fnote{So MT DSS 4QPs\textsuperscript{a}; LXX reads \fbib{of men pleasers}}--- \\
\poemll    you put them to shame,\fnote{So MT DSS 4QPs\textsuperscript{a}; LXX reads \fbib{they were put to shame}} \\
\poemlll       for God rejected them. \\
\poeml \v{6}Would that Israel's deliverance come out of Zion! \\
\poemll    When God restores the fortunes of his people, \\
\poemlll       Jacob will rejoice and Israel will be glad.\fnote{Cf. Ps 14:1-7}
\end{poetry}
\labelpsalm{54}
\psalminfo{To the Director: With stringed instruments. A Davidic instruction,\fnote{T Lit. \fbib{maskil}} when the Ziphites came and told Saul, ``David is hiding among us, is he not?''}
\passage{A Prayer in Times of Trouble}

\begin{poetry}
\poeml \v{1}God, by your name deliver me, \\
\poemll    and by your power vindicate me. \\
\poeml \v{2}God, listen to my prayer, \\
\poemll    and pay attention to the words of my mouth. \\
\poeml \v{3}For the arrogant have arisen against me; \\
\poemll    oppressors have sought to take my life. \\
\poeml They do not keep God in mind!\fnote{Lit. \fbib{before them}}
\end{poetry}
\interlude{Interlude}

\begin{poetry}
\poeml \v{4}Look, God is my helper; \\
\poemll    the Lord is with those who are guarding my life. \\
\poeml \v{5}He will turn the evil upon those who lie in wait for me. \\
\poemll    Cut them off with your truth. \\
\poeml \v{6}With a free will offering I will sacrifice to you; \\
\poemll    I will give thanks to your name, \divine{Lord}, \\
\poemlll       because it is good, \\
\poeml \v{7}for he has delivered me from every trouble, \\
\poemll    and my eyes have seen the end of\fnote{The Heb. lacks \fbib{the end of}} my enemies.
\end{poetry}
\labelpsalm{55}
\psalminfo{To the Director: With stringed instruments. A Davidic instruction.\fnote{T Lit. \fbib{maskil}}}
\passage{Betrayal by a Friend}

\begin{poetry}
\poeml \v{1}Pay attention to my prayer, God, \\
\poemll    and do not hide yourself from my appeal. \\
\poeml \v{2}Pay attention to me and answer me. \\
\poemll    I moan and groan in my thoughts, \\
\poeml \v{3}because of the voice of the enemy, \\
\poeml and because of the oppression of the wicked. \\
\poeml They bring down evil upon me, \\
\poemll    and in anger they hate me. \\
\poeml \v{4}My heart is trembling within me, \\
\poemll    and the terrors of death have assaulted me. \\
\poeml \v{5}Fear and trembling have overwhelmed me, \\
\poemll    and horror has covered me. \\
\poeml \v{6}I said, ``O, who will give me the wings of a dove, \\
\poemll    so that I could fly away and live somewhere else? \\
\poeml \v{7}Look, I want to flee far away; \\
\poemll    I want to settle down in the wilderness.
\end{poetry}
\interlude{Interlude}

\begin{poetry}
\poeml \v{8}I want to deliver myself quickly \\
\poemll    from this windstorm and tempest.'' \\
\poeml \v{9}Confound them, Lord, \\
\poemll    and confuse their speech, \\
\poemlll       because I have seen violence and strife in the city. \\
\poeml \v{10}Day and night they prowl around its walls; \\
\poemll    evil and iniquity are within it. \\
\poeml \v{11}Wickedness is at the center of it; \\
\poemll    fraud and lies never leave its streets. \\
\poeml \v{12}For it is not an enemy who insults me--- \\
\poemll    I could have handled that--- \\
\poeml nor is it someone who hates me and who now arises against me--- \\
\poemll    I could have hidden myself from him--- \\
\poeml \v{13}but it is you--- \\
\poemll    a man whom I treated as my equal--- \\
\poeml my personal confidant, \\
\poemll    my close friend! \\
\poeml \v{14}We had good fellowship together; \\
\poemll    and we even walked together in the house of God! \\
\poeml \v{15}Let death seize them! \\
\poemll    May they be plunged alive into the afterlife,\fnote{Lit. \fbib{into Sheol}; a reference to the realm of the dead} \\
\poeml for wicked things are in their homes \\
\poemll    and among them. \\
\poeml \v{16}I call upon God, \\
\poemll    and the \divine{Lord} will deliver me. \\
\poeml \v{17}Morning, noon, and night, \\
\poemll    I mulled over these things \\
\poeml and cried out in my distress, \\
\poemll    and he heard my voice. \\
\poeml \v{18}He calmly ransomed my soul from the war waged against me, \\
\poemll    for there was a vast crowd who stood against me. \\
\poeml \v{19}God, who is enthroned from long ago, \\
\poemll    will hear me and humble them.
\end{poetry}
\interlude{Interlude}

\begin{poetry}
\poeml Because they do not repent, \\
\poemll    they do not fear God. \\
\poeml \v{20}Each of my friends\fnote{Lit. \fbib{Each one}} raises his hand against his allies; \\
\poemll    each of my friends\fnote{Lit. \fbib{Each one}} breaks his word.\fnote{Lit. \fbib{covenant}} \\
\poeml \v{21}His mouth is as smooth as butter, \\
\poemll    while war is in his heart. \\
\poeml His words were as smooth as olive oil, \\
\poemll    while his sword is drawn. \\
\poeml \v{22}Cast on the \divine{Lord} whatever he sends your way, \\
\poemll    and he will sustain you. \\
\poemlll       He will never allow the righteous to be shaken. \\
\poeml \v{23}But you, God, bring them down to the Pit of corruption;\fnote{I.e. the place of punishment in the afterlife} \\
\poemll    bloodthirsty and deceitful people will not live out half their days. \\
\poemlll       But I put my full confidence in you.
\end{poetry}
\labelpsalm{56}
\psalminfo{To the Director: A special Davidic psalm\fnote{T Heb. \fbib{miktam}} to the tune of\fnote{T Lit. \fbib{David according to}} ``A Silent Dove Far Away,'' when the Philistines seized him in Gath.}
\passage{A Prayer about Trust in God}

\begin{poetry}
\poeml \v{1}Have mercy on me, God, \\
\poemll    because men have harassed me. \\
\poemlll       Those who oppress me have fought against me all day long. \\
\poeml \v{2}Those who watch me all day have harassed me, \\
\poemll    for there are many who fight against me out of conceit. \\
\poeml \v{3}On days when I am afraid, \\
\poemll    I put my trust in you. \\
\poeml \v{4}In God, whose word I praise, \\
\poemll    in God I put my trust. \\
\poemlll       I will not fear what mortal man\fnote{Lit. \fbib{what flesh}} can do to me. \\
\poeml \v{5}All day long people\fnote{Lit. \fbib{they}} distort what I say; \\
\poemll    all their schemes against me are for evil purposes. \\
\poeml \v{6}They gather together \\
\poemll    and hide in ambush. \\
\poeml They watch my every step \\
\poemll    as they lie in wait for my life. \\
\poeml \v{7}Cast them away because of their wickedness. \\
\poemll    In wrath, God, cast down these\fnote{The Heb. lacks \fbib{these}} people! \\
\poeml \v{8}You have kept count of my wanderings. \\
\poemll    Put my tears in your bottle--- \\
\poemlll       have not you recorded them in your book? \\
\poeml \v{9}My enemies will retreat when I call on you.\fnote{The Heb. lacks \fbib{on you}} \\
\poemll    This has been my experience, \\
\poemlll       because God is with me. \\
\poeml \v{10}In God, whose word I praise, \\
\poemll    in the \divine{Lord}, whose word I praise, \\
\poeml \v{11}in God I will put my trust. \\
\poemll    I will not fear what mortal man can do to me. \\
\poeml \v{12}God, I have taken vows before you;\fnote{Lit. \fbib{your vows are upon me}} \\
\poemll    therefore I will offer thanksgiving sacrifices to you. \\
\poeml \v{13}For you have delivered me\fnote{Or \fbib{my soul}} from death \\
\poemll    and my feet from stumbling, \\
\poemlll       so that I may walk before God in the light of the living!
\end{poetry}
\labelpsalm{57}
\psalminfo{To the Director: A special Davidic psalm\fnote{T Heb. \fbib{miktam}} to the tune of\fnote{T Lit. \fbib{David according to}} ``Do Not Destroy,'' when he fled from Saul into a cave.}
\passage{A Prayer for Deliverance}

\begin{poetry}
\poeml \v{1}Have mercy on me, God, have mercy, \\
\poemll    for in you I\fnote{Or \fbib{my soul}} have placed my trust. \\
\poeml Even in the shadow of your wings \\
\poemll    will I find my refuge until this calamity passes. \\
\poeml \v{2}I call upon the God Most High; \\
\poemll    to the God who completes what he began\fnote{The Heb. lacks \fbib{what he began}} in me. \\
\poeml \v{3}He will send help from heaven to deliver me \\
\poemll    from those who harass and despise me.
\end{poetry}
\interlude{Interlude}

\begin{poetry}
\poemlll       God will send his gracious love and truth. \\
\poeml \v{4}I am\fnote{Or \fbib{My soul is}} surrounded by lions. \\
\poemll    I lie down with those who burn with fire--- \\
\poeml that is, with people whose teeth are like spears and arrows--- \\
\poemll    whose tongues are like sharp swords. \\
\poeml \v{5}Be exalted above the heavens, God! \\
\poemll    May your glory cover the earth! \\
\poeml \v{6}They have set a snare for my feet, \\
\poemll    which makes me\fnote{Lit. \fbib{my soul}} depressed. \\
\poeml They dug a pit in front of me, \\
\poemll    but they are the ones who fell into it!
\end{poetry}
\interlude{Interlude}

\begin{poetry}
\poeml \v{7}My heart is committed, God, \\
\poemll    my heart is committed, \\
\poemlll       so I will sing and play music. \\
\poeml \v{8}Wake up, my soul,\fnote{Lit. \fbib{glory}} \\
\poemll    wake up, lyre and harp! \\
\poemlll       I will awaken at dawn. \\
\poeml \v{9}I will exalt you among the peoples, Lord. \\
\poemll    I will play music among the nations. \\
\poeml \v{10}For your gracious love is great, \\
\poemll    extending even to the heavens, \\
\poemlll       and your truth even to the skies. \\
\poeml \v{11}Be exalted above the heavens, God! \\
\poemll    May your glory cover the earth!
\end{poetry}
\labelpsalm{58}
\psalminfo{To the Director: A special Davidic psalm\fnote{T Heb. \fbib{miktam}} to the tune of\fnote{T Lit. \fbib{David according to}} ``Do Not Destroy''.}
\passage{A Prayer for Justice}

\begin{poetry}
\poeml \v{1}How is it that by remaining silent you can speak righteously? \\
\poemll    How can you judge people fairly? \\
\poeml \v{2}As a matter of fact, in your heart you plan iniquities! \\
\poemll    In the land your hands are violent! \\
\poeml \v{3}The wicked go astray from the womb; \\
\poemll    they go astray, telling lies even from birth. \\
\poeml \v{4}Their venom is like a poisonous snake; \\
\poemll    even like a deaf serpent that shuts its ears, \\
\poeml \v{5}refusing to hear the voice of the snake charmer, \\
\poemll    the cunning enchanter. \\
\poeml \v{6}God, shatter their teeth in their mouths; \\
\poemll    \divine{Lord}, break the fangs of the young lions! \\
\poeml \v{7}May they flow away like rain water that runs off, \\
\poemll    may they become like someone who shoots broken arrows. \\
\poeml \v{8}May they be like a snail that dries up as it crawls; \\
\poemll    like a woman's stillborn baby, who never saw the sun. \\
\poeml \v{9}Before your clay pots are placed on a fire of burning\fnote{The Heb. lacks \fbib{a fire of burning}} thorns--- \\
\poemll    whether green or ablaze--- \\
\poemlll       wrath will sweep them away like a storm. \\
\poeml \v{10}The righteous person will rejoice when he sees your\fnote{The Heb. lacks \fbib{your}} vengeance; \\
\poemll    when he washes his feet in the blood of the wicked. \\
\poeml \v{11}A person will say, \\
\poemll    ``Certainly, the righteous are rewarded; \\
\poemlll       certainly there is a God who judges the earth.''
\end{poetry}
\labelpsalm{59}
\psalminfo{To the Director: A special Davidic psalm\fnote{T Heb. \fbib{miktam}} to the tune of\fnote{T Lit. \fbib{David according to}} ``Do Not Destroy,'' when Saul sent men to watch the house in order to kill him.}
\passage{A Prayer for Deliverance and Justice}

\begin{poetry}
\poeml \v{1}Save me from my enemies, my God! \\
\poemll    Keep me safe from those who rise up against me. \\
\poeml \v{2}Save me from those who practice evil; \\
\poemll    deliver me from bloodthirsty men. \\
\poeml \v{3}Look, they lie in ambush for my life; \\
\poemll    these violent men gather together against me, \\
\poemlll       but not because of any transgression or sin of mine, \divine{Lord}. \\
\poeml \v{4}Without any fault on my part, \\
\poemll    they rush together and prepare themselves. \\
\poeml Get up! \\
\poemll    Come help me! \\
\poemlll       Pay attention! \\
\poeml \v{5}You, \divine{Lord} God of the Heavenly Armies, God of Israel, \\
\poemll    stir yourself up to punish all the nations. \\
\poemlll       Show no mercy to those wicked transgressors.
\end{poetry}
\interlude{Interlude}

\begin{poetry}
\poeml \v{6}At night they return like howling dogs; \\
\poemll    they prowl around the city. \\
\poeml \v{7}Look what pours out of their mouths! \\
\poemll    They use their lips like swords, \\
\poemlll       saying\fnote{The Heb. lacks \fbib{saying}} ``Who will hear us?'' \\
\poeml \v{8}But you, \divine{Lord}, will laugh at them; \\
\poemll    you will mock all the nations. \\
\poeml \v{9}My Strength, I will watch for you, \\
\poemll    for God is my fortress. \\
\poeml \v{10}My God of Gracious Love will meet me; \\
\poemll    God will enable me to see what happens\fnote{The Heb. lacks \fbib{what happens}} to my enemies. \\
\poeml \v{11}Don't kill them! \\
\poemll    Otherwise, my people may forget. \\
\poeml By your power make them stumble around; \\
\poemll    bring them down low, \\
\poemlll       Lord, our Shield. \\
\poeml \v{12}The sin of their mouth is the word on their lips. \\
\poemll    They will be caught in their own conceit; \\
\poemlll       for they speak curses and lies. \\
\poeml \v{13}Go ahead and destroy them in anger! \\
\poemll    Wipe them out, \\
\poeml and they will know to the ends of the earth \\
\poemll    that God rules over Jacob.\fnote{Or \fbib{know that God rules over Jacob to the ends of the earth}}
\end{poetry}
\interlude{Interlude}

\begin{poetry}
\poeml \v{14}At night they return like howling dogs; \\
\poemll    they prowl around the city. \\
\poeml \v{15}They scavenge for food. \\
\poemll    If they find nothing, \\
\poemlll       they become hungry and growl. \\
\poeml \v{16}But I will sing of your power \\
\poemll    and in the morning I will shout for joy about your gracious love. \\
\poeml For you have been a fortress for me; \\
\poemll    and a refuge when I am distressed.\fnote{Lit. \fbib{refuge in the day of my distress}} \\
\poeml \v{17}My Strength, I will sing praises to you, \\
\poemll    for you, God of Gracious Love, are my fortress.
\end{poetry}
\labelpsalm{60}
\psalminfo{To the Director: A special Davidic psalm to the tune of\fnote{T Lit. \fbib{David according to}} ``Lily of The Covenant,'' for teaching about his battle with Aram-naharaim and Aram-zobah, when Joab returned and attacked 12,000 Edomites in the Salt Valley.\fnote{T I.e. Dead Sea region}}
\passage{A Prayer for God's Help against Adversaries}

\begin{poetry}
\poeml \v{1}God, you have cast us off; \\
\poemll    you have breached our defenses \\
\poeml and you have become enraged. \\
\poemll    Return to us! \\
\poeml \v{2}You made the earth quake; \\
\poemll    you broke it open. \\
\poeml Repair its fractures, \\
\poemll    because it has shifted. \\
\poeml \v{3}You made your people go through hard times; \\
\poemll    you had us drink wine that makes us stagger. \\
\poeml \v{4}But you have given a banner to those who fear you, \\
\poemll    so they may display it in honor of truth.\fnote{Or \fbib{display it because of the archer}}
\end{poetry}
\interlude{Interlude}

\begin{poetry}
\poeml \v{5}So your loved ones may be delivered, \\
\poemll    save us by your power\fnote{Lit. \fbib{right hand}} \\
\poemlll       and answer us quickly! \\
\poeml \v{6}Then God spoke in his holiness, \\
\poeml ``I will rejoice--- \\
\poemll    I will divide Shechem; \\
\poemlll       I will portion out the Succoth Valley. \\
\poeml \v{7}Gilead belongs to me, \\
\poemll    and Manasseh is mine. \\
\poeml Ephraim is my helmet, \\
\poemll    and Judah my scepter. \\
\poeml \v{8}Moab is my wash basin; \\
\poemll    over Edom I will throw my shoes; \\
\poemlll       over Philistia I will celebrate my triumph.'' \\
\poeml \v{9}Who will lead me to the fortified city? \\
\poemll    Who will lead me to Edom? \\
\poeml \v{10}Aren't you the one, God, who has cast us off? \\
\poemll    Didn't you refuse, God, to accompany our armies? \\
\poeml \v{11}Help us in our distress, \\
\poemll    for human help is worthless. \\
\poeml \v{12}Through God we will fight\fnote{Lit. \fbib{will do}} valiantly; \\
\poemll    and it is he who will crush our enemies.\fnote{vv.5-12 is the same as Psalm 108:6-13.}
\end{poetry}
\labelpsalm{61}
\psalminfo{To the Director: A composition\fnote{T The Heb. lacks \fbib{A composition}} by David for stringed instruments.}
\passage{A Prayer for God's Protection}

\begin{poetry}
\poeml \v{1}God, hear my cry; \\
\poemll    pay attention to my prayer. \\
\poeml \v{2}From the end of the earth I will cry to you \\
\poemll    whenever my heart is overwhelmed. \\
\poemlll       Place me on the rock that's too high for me. \\
\poeml \v{3}For you have been a refuge for me, \\
\poemll    a tower of strength before the enemy. \\
\poeml \v{4}Let me make my home in your tent forever; \\
\poemll    let me hide under the shelter of your wings.
\end{poetry}
\interlude{Interlude}

\begin{poetry}
\poeml \v{5}For you, God, have heard my promises; \\
\poemll    you have assigned to me\fnote{The Heb. lacks \fbib{to me}} the heritage of those who fear your name. \\
\poeml \v{6}Add day after day to the king's life; \\
\poemll    may his years continue\fnote{The Heb. lacks \fbib{continue}} for many generations. \\
\poeml \v{7}May he be enthroned before God forever; \\
\poemll    Appoint your\fnote{The Heb. lacks \fbib{your}} gracious love and truth to guard him. \\
\poeml \v{8}So I will sing songs to your name forever; \\
\poemll    I will fulfill my promises day by day.
\end{poetry}
\labelpsalm{62}
\psalminfo{To the Director: According to Jeduthun's style. A Davidic Psalm.}
\passage{A Psalm of Trust in God}

\begin{poetry}
\poeml \v{1}My soul rests quietly only when it looks\fnote{The Heb. lacks \fbib{when it looks}} to God; \\
\poemll    from him comes my deliverance. \\
\poeml \v{2}He alone is my rock, my deliverance, and my high tower; \\
\poemll    nothing will shake me. \\
\poeml \v{3}How long will you rage against someone? \\
\poemll    Would you attack him \\
\poemlll       as if he were a leaning wall or a tottering fence? \\
\poeml \v{4}They plan to cast him down from his exalted position. \\
\poemll    They delight in lies; \\
\poeml their mouth utters blessings, \\
\poemll    while their heart is cursing.
\end{poetry}
\interlude{Interlude}

\begin{poetry}
\poeml \v{5}My soul, be quiet before God, \\
\poemll    for from him comes my hope. \\
\poeml \v{6}He alone is my rock, my deliverance, and my high tower; \\
\poemll    nothing will shake me. \\
\poeml \v{7}I rely on God who is my deliverance and my glory; \\
\poemll    he is my strong rock, \\
\poemlll       and my refuge is in God. \\
\poeml \v{8}People, in every situation put your trust in God;\fnote{Lit. \fbib{in him}} \\
\poemll    pour out your heart before him; \\
\poemlll       for God is a refuge for us.
\end{poetry}
\interlude{Interlude}

\begin{poetry}
\poeml \v{9}Human beings\fnote{Lit. \fbib{sons of Adam}} are a mere vapor, \\
\poemll    while people in high positions\fnote{Lit. \fbib{sons of man}} are not what they appear. \\
\poemll    When they are placed on the scales, they weigh nothing; \\
\poemlll       even when weighed together, they are less than nothing. \\
\poeml \v{10}Don't trust in oppression \\
\poemll    or put false hope in stealing; \\
\poeml if you become wealthy, \\
\poemll    do not set your heart on it. \\
\poeml \v{11}God spoke once, \\
\poemll    but I heard it twice, \\
\poemlll       ``Power belongs to God.'' \\
\poeml \v{12}Also to you, Lord, belongs gracious love, \\
\poemll    because you reward each person according to what he does.
\end{poetry}
\labelpsalm{63}
\psalminfo{A Davidic Psalm, while he was in the Judean wilderness.}
\passage{Joyful Trust in God}

\begin{poetry}
\poeml \v{1}God, you are my God! \\
\poemll    I will fervently seek you. \\
\poeml My soul thirsts for you; \\
\poemll    my flesh longs for you in a dry, weary, and parched land. \\
\poeml \v{2}So I have looked for you in the sanctuary, \\
\poemll    to behold your power and glory. \\
\poeml \v{3}Because your gracious love is better than life itself, \\
\poemll    my lips will praise you. \\
\poeml \v{4}So I will bless you as long as I live; \\
\poemll    I will lift up my hands in your name. \\
\poeml \v{5}Just as I am satisfied with the choicest of foods,\fnote{Lit. \fbib{with marrow and fatness}} \\
\poemll    so my lips will praise you joyfully. \\
\poeml \v{6}When I think of you in bed, \\
\poemll    I will meditate on you in the night watches. \\
\poeml \v{7}For you have been my strength, \\
\poemll    and in the shadow of your wings I will shout for joy. \\
\poeml \v{8}My soul clings to you, \\
\poemll    even as your right hand supports me. \\
\poeml \v{9}But as for those who seek to destroy me, \\
\poemll    they will go down to the depths of the earth; \\
\poeml \v{10}May they be given over to the power of\fnote{The Heb. lacks \fbib{to the power of}} the sword; \\
\poemll    may they become carrion for jackals. \\
\poeml \v{11}But as for the king, \\
\poemll    he will rejoice in God. \\
\poeml Indeed, everyone who swears by God\fnote{Lit. \fbib{him}} will exult, \\
\poemll    because the mouths of liars will be silenced.
\end{poetry}
\labelpsalm{64}
\psalminfo{To the Director: A Davidic Psalm.}
\passage{A Prayer for Protection}

\begin{poetry}
\poeml \v{1}Hear, God, as I express my concern; \\
\poemll    protect me\fnote{Lit. \fbib{my life}} from fear of the enemy. \\
\poeml \v{2}Hide me from the secret plots of the wicked, \\
\poemll    from the mob of those who practice evil, \\
\poeml \v{3}who sharpen their tongues like swords, \\
\poemll    and aim their bitter words like arrows, \\
\poeml \v{4}shooting at the innocent from concealment. \\
\poeml Suddenly they shoot, fearing nothing. \\
\poeml \v{5}They concoct an evil scheme for themselves; \\
\poeml they enumerate their hidden snares; \\
\poemll    they say, ``Who will see them?''\fnote{Lit. \fbib{see him}; or \fbib{see it}} \\
\poeml \v{6}They devise wicked schemes, saying, \\
\poemll    ``We have completed our plans, \\
\poemlll       hiding them deep in our hearts.'' \\
\poeml \v{7}But God shot an arrow at them, \\
\poemll    and they were wounded immediately. \\
\poeml \v{8}They tripped over their own tongues, \\
\poemll    and everyone who was watching ran away. \\
\poeml \v{9}Everyone was gripped with fear \\
\poemll    and acknowledged God's deeds, \\
\poemlll       understanding what he had done. \\
\poeml \v{10}The righteous rejoiced in the \divine{Lord}, \\
\poemll    because they had fled to him for refuge. \\
\poemlll       Let all the upright in heart exult.
\end{poetry}
\labelpsalm{65}
\psalminfo{To the Director: A song. Lyrics\fnote{T Lit. \fbib{A song. A song}} by David.}
\passage{A Song of Praise to God}

\begin{poetry}
\poeml \v{1}In Zion, God, praise silently awaits you, \\
\poemll    and vows will be paid to you. \\
\poeml \v{2}Since you hear prayer, \\
\poemll    everybody will come to you. \\
\poeml \v{3}My acts of iniquity---they overwhelm me! \\
\poemll    Our transgressions---you blot them out! \\
\poeml \v{4}How blessed is the one you choose, \\
\poemll    the one you cause to live in your courts. \\
\poeml We will be satisfied with the goodness of your house, \\
\poemll    yes, even with the holiness of your Temple. \\
\poeml \v{5}With awesome deeds of justice\fnote{Or \fbib{righteousness}} \\
\poemll    you will answer us, God our Deliverer; \\
\poeml you are\fnote{The Heb. lacks \fbib{you are}} the confidence for everyone at the ends of the earth, \\
\poemll    even for those far away overseas. \\
\poeml \v{6}The One who established the mountains by his strength \\
\poemll    is clothed with omnipotence. \\
\poeml \v{7}He calmed the roar of seas, \\
\poemll    the roaring of the waves, \\
\poemlll       and the turmoil of the peoples. \\
\poeml \v{8}Those living at the furthest ends of the earth\fnote{The Heb. lacks \fbib{of the earth}} are seized by fear because of your miraculous deeds. \\
\poeml You make the going forth of the morning and the evening shout for joy. \\
\poeml \v{9}You take care of the earth, \\
\poemll    you water it, \\
\poemlll       and you enrich it greatly with the river of God that overflows with water. \\
\poeml You provide grain for them, \\
\poemll    for you have ordained it this way. \\
\poeml \v{10}You fill the furrows of the field with water \\
\poemll    so that their ridges overflow. \\
\poeml You soften them with rain showers; \\
\poemll    their sprouts you have blessed. \\
\poeml \v{11}You crown the year with your goodness; \\
\poemll    your footsteps drop prosperity behind them. \\
\poeml \v{12}The wilderness pastures drip with dew,\fnote{The Heb. lacks \fbib{with dew}} \\
\poemll    and the hills wrap themselves with joy. \\
\poeml \v{13}The meadows are clothed with flocks of sheep, \\
\poemll    and the valleys are covered with grain. \\
\poeml They shout for joy; \\
\poemll    yes, they burst out in song!
\end{poetry}
\labelpsalm{66}
\psalminfo{To the Director: A song. A Psalm.}
\passage{A Song of Praise}

\begin{poetry}
\poeml \v{1}Shout praise to God all the earth! \\
\poeml \v{2}Sing praise about the glory of his name.\fnote{I.e. \fbib{reputation}; and so throughout the Psalms} \\
\poemll    Make his praise glorious. \\
\poeml \v{3}Say to God: ``How awesome are your works! \\
\poemll    Because of your great strength \\
\poemlll       your enemies cringe before you.'' \\
\poeml \v{4}The whole earth worships you. \\
\poemll    They sing praise to you. \\
\poemlll       They sing praise to your name.
\end{poetry}
\interlude{Interlude}

\begin{poetry}
\poeml \v{5}Come and see the awesome works of God \\
\poemll    on behalf of human beings: \\
\poeml \v{6}He turned the sea into dry land. \\
\poemll    Israel\fnote{Lit. \fbib{He}} crossed the river on foot; \\
\poemlll       let us rejoice in him. \\
\poeml \v{7}He rules by his power forever, \\
\poemll    his eyes watching over the nations. \\
\poemlll       Do not let the rebellious exalt themselves.
\end{poetry}
\interlude{Interlude}

\begin{poetry}
\poeml \v{8}Bless our God, people, \\
\poemll    and let the sound of his praise be heard. \\
\poeml \v{9}He gives us life \\
\poemll    and does not permit our feet to slip. \\
\poeml \v{10}For you, God, tested us, \\
\poemll    to purify us like fine silver. \\
\poeml \v{11}You have led us into a trap\fnote{Or \fbib{net}} \\
\poemll    and set burdens on our backs. \\
\poeml \v{12}You caused men to ride over us.\fnote{Lit. \fbib{over our head}} \\
\poemll    You brought us through fire and water, \\
\poemlll       but you led us to abundance. \\
\poeml \v{13}I will come to your house with burnt offerings. \\
\poemll    I will fulfill my vows to you \\
\poeml \v{14}that my lips uttered and that my mouth spoke \\
\poemll    when I was in trouble. \\
\poeml \v{15}I will offer to you burnt offerings of fat, \\
\poemll    along with the smoke of the sacrifice of rams. \\
\poemlll       I will offer bulls along with goats.
\end{poetry}
\interlude{Interlude}

\begin{poetry}
\poeml \v{16}Come and listen, all of you who fear God, \\
\poemll    and I will tell you what he did for me. \\
\poeml \v{17}I called aloud to him \\
\poemll    and praised him with my tongue. \\
\poeml \v{18}Were I to cherish iniquity in my heart, \\
\poemll    the Lord would not listen to me. \\
\poeml \v{19}Surely God has heard, \\
\poemll    and he paid attention to my\fnote{Lit. \fbib{to the voice of my}} prayers. \\
\poeml \v{20}Blessed be God, who did not turn away my prayers \\
\poemll    nor his gracious love from me.
\end{poetry}
\labelpsalm{67}
\psalminfo{To the Director of music: Accompanied by stringed instruments. A Psalm. A song.}
\passage{A Call to Thanksgiving}

\begin{poetry}
\poeml \v{1}May God show us favor and bless us; \\
\poemll    may he truly show us his favor.\fnote{Lit. \fbib{he cause his face to shine on us}}
\end{poetry}
\interlude{Interlude}

\begin{poetry}
\poeml \v{2}Let your ways be known by all the nations of the earth, \\
\poemll    along with your deliverance. \\
\poeml \v{3}Let the people thank you, God. \\
\poemll    Let all the people thank you. \\
\poeml \v{4}Let the nations rejoice and sing for joy, \\
\poemll    because you judge people with fairness \\
\poemlll       and you govern the people of the earth.
\end{poetry}
\interlude{Interlude}

\begin{poetry}
\poeml \v{5}Let the people thank you, God; \\
\poemll    let all the people thank you. \\
\poeml \v{6}May the earth yield its produce. \\
\poemll    May God, our God, bless us. \\
\poeml \v{7}May God truly bless us \\
\poemll    so that all the peoples\fnote{Lit. \fbib{ends}} of the earth will fear him.
\end{poetry}
\labelpsalm{68}
\psalminfo{To the Director of music: A Psalm. A song.}
\passage{A Song of Praise to God}

\begin{poetry}
\poeml \v{1}God arises, \\
\poemll    and his enemies are scattered. \\
\poemlll       Those who hate him flee from his presence.\fnote{Lit. \fbib{face}} \\
\poeml \v{2}As smoke is driven away, so you drive them away. \\
\poemll    As wax melts in the presence of fire, \\
\poemlll       so the wicked die in the presence of God. \\
\poeml \v{3}But the righteous rejoice and exult before God; \\
\poemll    they are overwhelmed with joy. \\
\poeml \v{4}Sing to God! \\
\poemll    Sing praise to his name! \\
\poemlll       Exalt the one who rides on the clouds. \\
\poeml The \divine{Lord} is his name. \\
\poemll    Be jubilant in his presence. \\
\poeml \v{5}A father to orphans and an advocate for widows \\
\poemll    is God in his holy dwelling place. \\
\poeml \v{6}God causes the lonely to dwell in families.\fnote{Lit. \fbib{in a house}} \\
\poemll    He leads prisoners into prosperity, \\
\poemlll       but rebels live on parched land. \\
\poeml \v{7}God, when you led out your people, \\
\poemll    when you marched through the desert,
\end{poetry}
\interlude{Interlude}

\begin{poetry}
\poeml \v{8}the land quaked. \\
\poeml Indeed, the heavens poured down rain \\
\poemll    from the presence of God, \\
\poemlll       this God of Sinai, \\
\poemll    from the presence of God, \\
\poemlll       the God of Israel. \\
\poeml \v{9}God, you poured out abundant rain on your inheritance. \\
\poemll    When Israel\fnote{Lit. \fbib{it}} was weary, you sustained her. \\
\poeml \v{10}Your people live\fnote{Or \fbib{tribe lives}} there; \\
\poemll    you sustain the needy\fnote{Or \fbib{afflicted}} with your goodness, God. \\
\poeml \v{11}The Lord issues a command. \\
\poemll    Numerous are the women who announce the news: \\
\poeml \v{12}``Kings of armies retreat and flee, \\
\poemll    while the lady of the house divides the spoil. \\
\poeml \v{13}When you men lie down among the sheepfolds, \\
\poemll    you are like\fnote{The Heb. lacks \fbib{you are like}} the wings of the dove covered with silver, \\
\poemlll       with its feathers in glittering gold.'' \\
\poeml \v{14}When the Almighty scattered the kings there, \\
\poemll    there was snow on Mt. Zalmon. \\
\poeml \v{15}The mountain of God is as the mountain of Bashan; \\
\poemll    a mountain of many peaks is Mount Bashan. \\
\poeml \v{16}You mountains of many peaks, why do you watch with envy \\
\poemll    the mountain in which God has chosen to dwell? \\
\poemlll       Indeed, the \divine{Lord} will live there forever. \\
\poeml \v{17}God's chariots were many thousands. \\
\poemll    The Lord was there with them at Sinai in holiness. \\
\poeml \v{18}You ascended to the heights, \\
\poemll    you took captives. \\
\poeml You received gifts among mankind, \\
\poemll    even the rebellious, \\
\poemlll       so the \divine{Lord} God may live there.\fnote{The Heb. lacks \fbib{there}} \\
\poeml \v{19}Blessed be the Lord who daily carries us. \\
\poemll    God is our deliverer. \\
\poeml \v{20}God is for us the God of our deliverance. \\
\poemll    The Lord \divine{God} rescues us from death. \\
\poeml \v{21}God surely strikes the heads of his enemies, \\
\poemll    even the hairy heads of those who continue in their guilt. \\
\poeml \v{22}The Lord says, ``From Bashan I will bring them, \\
\poemll    I will bring them from the depths of the sea, \\
\poeml \v{23}that your feet may wade through blood. \\
\poeml The tongues of your dogs will have their portions \\
\poemll    from your enemies.'' \\
\poeml \v{24}They have observed your processions, God, \\
\poemll    the processions of my God, \\
\poemlll       my king, in the sanctuary. \\
\poeml \v{25}The singers are in front, \\
\poemll    the musicians follow, \\
\poemlll       strumming their stringed instruments \\
\poeml among the maidens who are playing their tambourines. \\
\poeml \v{26}Bless God in the great congregation, \\
\poemll    the \divine{Lord} who is the fountain of Israel. \\
\poeml \v{27}Little Benjamin is there, leading them, \\
\poemll    and the princes of Judah all together \\
\poemlll       with the princes of Zebulun and the princes of Naphtali. \\
\poeml \v{28}Summon the power of your God, \\
\poemll    the power, God, that you have shown us. \\
\poeml \v{29}Because of your Temple in Jerusalem, \\
\poemll    kings bring tribute to you. \\
\poeml \v{30}Rebuke the wildlife that lives among the reeds, \\
\poemll    the nations that congregate like bulls and cows, \\
\poeml humbling themselves with pieces of silver, \\
\poemll    for God\fnote{Lit. \fbib{he}} scatters the nations that delight in battle. \\
\poeml \v{31}Envoys will come from Egypt. \\
\poemll    Let the Ethiopians stretch out their hands to God. \\
\poeml \v{32}You kingdoms of the earth, sing to God! \\
\poemll    Sing praises to the Lord,
\end{poetry}
\interlude{Interlude}

\begin{poetry}
\poeml \v{33}to the one who rides the heavens, the ancient heavens. \\
\poemll    Behold! He thunders with a mighty voice. \\
\poeml \v{34}Ascribe power to God, whose glory is over Israel, \\
\poemll    whose power is in the skies. \\
\poeml \v{35}You are awesome, God, from your sanctuaries. \\
\poemll    The God of Israel is the one \\
\poemlll       who gives strength and power to the people. \\
\poeml Blessed be God!
\end{poetry}
\labelpsalm{69}
\psalminfo{To the Director: To the tune of\fnote{T Lit. \fbib{According to}} ``The Lilies''. Davidic.}
\passage{When God Seems Distant}

\begin{poetry}
\poeml \v{1}Deliver me, God, \\
\poemll    because the waters are up to my neck.\fnote{Lit. \fbib{soul}} \\
\poeml \v{2}I am sinking in deep mire, \\
\poemll    and there is no solid ground.\fnote{Or \fbib{foothold}} \\
\poeml I have come into deep water, \\
\poemll    and the flood overwhelms me. \\
\poeml \v{3}I am exhausted from calling for help. \\
\poemll    My throat is parched. \\
\poemlll       My eyes are strained from looking for God. \\
\poeml \v{4}Those who hate me without cause \\
\poemll    are more than the hairs of my head. \\
\poeml My persecutors are mighty, \\
\poemll    and they want to destroy me. \\
\poemlll       Must I be forced to return what I did not steal? \\
\poeml \v{5}God, you know my sins, \\
\poemll    and my guilt is not hidden from you. \\
\poeml \v{6}Do not let those who look up to you be ashamed \\
\poemll    because of me, \\
\poemlll       Lord God of the Heavenly Armies. \\
\poeml Let not those who seek you be humiliated \\
\poemll    because of me, \\
\poemlll       God of Israel. \\
\poeml \v{7}I am being mocked because of you. \\
\poemll    Dishonor overwhelms me. \\
\poeml \v{8}I am a stranger to my brothers, \\
\poemll    a foreigner to my mother's sons. \\
\poeml \v{9}Zeal for your house consumes me, \\
\poemll    and the mockeries of those who insult you fall on me. \\
\poeml \v{10}I weep and fast, \\
\poemll    and I am mocked for it. \\
\poeml \v{11}When I dressed in sackcloth, \\
\poemll    I became an object of gossip among them. \\
\poeml \v{12}The prominent people mock me, \\
\poemll    composing drinking songs.
\passage{Seeking God for Deliverance}
\poeml \v{13}As for me, \divine{Lord}, may my prayer to you come at a favorable time. \\
\poemll    God, in the abundance of your gracious love, \\
\poemlll       answer me with your sure deliverance. \\
\poeml \v{14}Rescue me from the mud \\
\poemll    and do not let me sink. \\
\poeml Rescue me from those who hate me, \\
\poemll    and from the deep waters. \\
\poeml \v{15}Let neither the floodwaters overwhelm me \\
\poemll    nor let the deep swallow me up, \\
\poemlll       nor the mouth of the well close over me. \\
\poeml \v{16}Answer me, \divine{Lord}, for your gracious love is good; \\
\poemll    Turn to me in keeping with your great compassion, \\
\poeml \v{17}and\fnote{So MT; DSS 4QPs\textsuperscript{a} lack \fbib{and}} do not ignore your servant, \\
\poemll    because I am in distress. \\
\poemlll       Hurry to answer me! \\
\poeml \v{18}Draw near and redeem me; \\
\poemll    ransom me because of my enemies. \\
\poeml \v{19}Truly you know my reproach, shame, and disgrace. \\
\poemll    All my enemies are known to\fnote{Lit. \fbib{are before}} you. \\
\poeml \v{20}Insults broke my heart. \\
\poemll    I despaired and looked for sympathy; \\
\poeml but there was none, \\
\poemll    for comforters, but I found none. \\
\poeml \v{21}They put poison in my food, \\
\poemll    in my thirst they forced me to drink vinegar. \\
\poeml \v{22}May their dining\fnote{The Heb. lacks \fbib{dining}} tables entrap them, \\
\poemll    and become a snare for their allies. \\
\poeml \v{23}May their eyes be blinded \\
\poemll    and may their bodies tremble continuously. \\
\poeml \v{24}May you pour out your fury on them. \\
\poemll    May your burning anger overtake them. \\
\poeml \v{25}May their camp become desolate \\
\poemll    and their tents remain unoccupied. \\
\poeml \v{26}For they persecute those whom you have struck, \\
\poemll    and they brag about the pain of those you have wounded. \\
\poeml \v{27}May you punish them for their crimes; \\
\poemll    may they receive no verdict of innocence\fnote{Lit. \fbib{no righteousness}} from you. \\
\poeml \v{28}May they be erased from the Book of Life, \\
\poemll    and their names not be written with the righteous. \\
\poeml \v{29}As for me, I am afflicted and hurting; \\
\poemll    may your deliverance, God, establish me on high. \\
\poeml \v{30}Let me praise the name of God with a song \\
\poemll    that I may magnify him with thanksgiving. \\
\poeml \v{31}That will please the \divine{Lord} \\
\poemll    more than oxen and bulls with horns and hooves. \\
\poeml \v{32}The afflicted will watch and rejoice. \\
\poemll    May you who seek God take courage. \\
\poeml \v{33}For the \divine{Lord} listens to the needy \\
\poemll    and doesn't despise those in bondage. \\
\poeml \v{34}Let the heavens and earth praise him, \\
\poemll    along with the sea and its swarming creatures.\fnote{The Heb. lacks \fbib{creatures}} \\
\poeml \v{35}For God will deliver Zion \\
\poemll    and will rebuild the cities of Judah \\
\poemlll       so they may live there and possess them. \\
\poeml \v{36}The descendants of his servants will inherit it, \\
\poemll    and those who cherish his name will live there.
\end{poetry}
\labelpsalm{70}
\psalminfo{To the Music director. Davidic. As a memorial.}
\passage{A Call for Help}

\begin{poetry}
\poeml \v{1}God, come to my rescue. \\
\poemll    \divine{Lord}, hurry to help me. \\
\poeml \v{2}May those who seek to kill me be publicly humiliated. \\
\poemll    May those who take pleasure in my harm \\
\poemlll       be turned back in humiliation. \\
\poeml \v{3}May those who say ``Aha! Aha!'' \\
\poemll    be turned back because of their shameful deeds.\fnote{The Heb. lacks \fbib{deeds}} \\
\poeml \v{4}Let those who seek you greatly rejoice in you. \\
\poemll    Let those who love your deliverance say, \\
\poemlll       ``May God be continuously exalted.'' \\
\poeml \v{5}As for me, I am poor and needy. \\
\poemll    God, come quickly to me. \\
\poeml You are my helper and my deliverer. \\
\poemll    \divine{Lord}, please do not delay.
\end{poetry}
\labelpsalm{71}
\passage{A Prayer for Deliverance}

\begin{poetry}
\poeml \v{1}In you, \divine{Lord}, I take refuge; \\
\poemll    let me never be humiliated. \\
\poeml \v{2}Rescue and deliver me,\fnote{So LXX DSS 4QPs\textsuperscript{a}; MT reads \fbib{In your righteousness you are delivering me and rescuing me}} because you are righteous. \\
\poemll    Turn your ear to me and save me. \\
\poeml \v{3}Be my sheltering refuge where I may go continuously; \\
\poemll    command my deliverance \\
\poemlll       for you are my rock and fortress. \\
\poeml \v{4}My God, deliver me from the power of the wicked \\
\poemll    and the grasp of ruthless practicers of evil. \\
\poeml \v{5}For you are my hope, Lord \divine{God}, \\
\poemll    my security since I was young. \\
\poeml \v{6}I depended on you since birth,\fnote{Lit. \fbib{you from the womb}} \\
\poemll    when you brought me\fnote{So MT; LXX reads \fbib{birth, it was you who sheltered me}; DSS 4QPs\textsuperscript{a} reads \fbib{birth, you are my protector}} from my mother's womb; \\
\poemlll       I praise you continuously. \\
\poeml \v{7}I have become an example to many \\
\poemll    that you are my strong refuge. \\
\poeml \v{8}My mouth is filled with your praise \\
\poemll    and your splendor daily. \\
\poeml \v{9}Don't throw me away when I am old; \\
\poemll    do not abandon me when my strength fails. \\
\poeml \v{10}For my enemies talk against me; \\
\poemll    those who seek to kill me plot together \\
\poeml \v{11}and say, ``God has abandoned him. \\
\poemll    Run after him and seize him, \\
\poemlll       because there's no deliverer.'' \\
\poeml \v{12}God, do not be distant from me. \\
\poemll    My God, come quickly to help me. \\
\poeml \v{13}Let my adversaries be ashamed and consumed;\fnote{So MT; LXX reads \fbib{and let them expire}; DSS 4QPs\textsuperscript{a} reads \fbib{and let them be consumed}} \\
\poemll    let those who seek my destruction \\
\poemlll       be covered with scorn and disgrace. \\
\poeml \v{14}As for me, I will hope continuously \\
\poemll    and will praise you more and more. \\
\poeml \v{15}I\fnote{Lit. \fbib{My mouth}} will declare your righteousness \\
\poemll    and your salvation every day, \\
\poeml though I do not fully understand \\
\poemll    what the outcome will be.\fnote{Lit. \fbib{understand the sum}} \\
\poeml \v{16}Lord \divine{God}, I will come in the power of\fnote{The Heb. lacks \fbib{the power of}} your mighty acts, \\
\poemll    remembering your righteousness---yours alone. \\
\poeml \v{17}God, you taught me from my youth, \\
\poemll    so I am still declaring your awesome deeds. \\
\poeml \v{18}Also, when I reach old age and have gray hair, \\
\poemll    God, do not forsake me, \\
\poeml until I have declared your power \\
\poemll    to this generation \\
\poemlll       and your might to the next one. \\
\poeml \v{19}Your many righteous deeds,\fnote{Lit. \fbib{righteous deeds as far as the height}} God, are great, \\
\poeml \v{20}God, who can compare to you, \\
\poeml who caused me to experience\fnote{Lit. \fbib{see}} troubles \\
\poeml that were numerous and disastrous? \\
\poeml You will return to revive me \\
\poemll    and lift me up from the depths of the earth. \\
\poeml \v{21}You will increase my honor \\
\poemll    and comfort me once again. \\
\poeml \v{22}I also will praise you with the harp; \\
\poemll    because of your faithfulness, my God, \\
\poeml I will praise you with the lyre--- \\
\poemll    Holy One of Israel. \\
\poeml \v{23}My lips will shout for joy when I sing praise to you, \\
\poemll    whose life you have redeemed. \\
\poeml \v{24}Moreover, my tongue will speak all day about your justice; \\
\poemll    for those who seek my destruction will be utterly humiliated.
\end{poetry}
\labelpsalm{72}
\psalminfo{Solomonic}
\passage{A Prayer for the King}

\begin{poetry}
\poeml \v{1}God, endow the king with ability to render\fnote{The Heb. lacks \fbib{to render}} your justice, \\
\poemll    and the king's son to render your right decisions. \\
\poeml \v{2}May he rule your people with right decisions \\
\poemll    and your oppressed ones with justice. \\
\poeml \v{3}May the mountains bring prosperity to the people \\
\poemll    and the hills bring righteousness. \\
\poeml \v{4}May he defend the afflicted of the people \\
\poemll    and deliver the children of the poor, \\
\poemlll       but crush the oppressor. \\
\poeml \v{5}May they fear you as long as the sun and moon shine\fnote{The Heb. lacks \fbib{shine}}--- \\
\poemll    from generation to generation. \\
\poeml \v{6}May he be like the rain that descends on mown grass, \\
\poemll    like showers sprinkling on the ground. \\
\poeml \v{7}The righteous will flourish at the proper time \\
\poemll    and peace will prevail until the moon is no more. \\
\poeml \v{8}May he rule from sea to sea, \\
\poemll    from the Euphrates River\fnote{The Heb. lacks \fbib{Euphrates}} to the ends of the earth. \\
\poemll    \v{9}May the nomads bow down before him, \\
\poemll    and his enemies lick the dust. \\
\poeml \v{10}May the kings of Tarshish and of distant shores bring gifts, \\
\poemll    and may the kings of Sheba and Seba offer tribute. \\
\poeml \v{11}May all kings bow down to him, \\
\poemll    and all nations serve him. \\
\poeml \v{12}For he will deliver the needy when they cry out for help, \\
\poemll    and the poor when there is no deliverer. \\
\poeml \v{13}He will have compassion on the poor and the needy, \\
\poemll    and he will save the lives of the needy. \\
\poeml \v{14}He will redeem them\fnote{Lit. \fbib{redeem their souls}} from oppression and violence, \\
\poemll    since their lives are\fnote{Lit. \fbib{their blood is}} precious in his sight.
\passage{Prayer for the King}
\poeml \v{15}May he live long and be given gold from Sheba, \\
\poemll    and may prayer be offered for him continuously, \\
\poemlll       and may he be blessed every day. \\
\poeml \v{16}May grain be abundant in the land \\
\poemll    all the way\fnote{The Heb. lacks \fbib{all the way}} to the mountain tops; \\
\poeml may its fruits flourish \\
\poemll    like the forests of Lebanon, \\
\poeml and may the cities sprout \\
\poemll    like the grass of the earth.
\passage{Praising the God of Israel}
\poeml \v{17}May his fame\fnote{Lit. \fbib{name}} be eternal--- \\
\poemll    as long as the sun--- \\
\poeml may his name endure, \\
\poemll    and may they be blessed through him, \\
\poemlll       and may all nations call him blessed. \\
\poeml \v{18}Blessed be the \divine{Lord} God, the God of Israel, \\
\poemll    who alone does awesome deeds. \\
\poeml \v{19}And blessed be his glorious name forever, \\
\poemll    and may the whole earth be filled with his glory. \\
\poemlll       Amen and amen! \\
\poeml \v{20}This ends the prayers of Jesse's son David.
\end{poetry}
\booksection{BOOK III (Psalms 73-89)}
\labelpsalm{73}
\psalminfo{A song of Asaph.}
\passage{A Plea for Deliverance}

\begin{poetry}
\poeml \v{1}God is indeed good to Israel, \\
\poemll    to those pure in heart. \\
\poeml \v{2}Now as for me, my feet nearly stumbled, \\
\poemll    as I almost lost my step. \\
\poeml \v{3}For I was envious of the proud \\
\poemll    when I observed the prosperity of the wicked. \\
\poeml \v{4}For there is no struggle at their deaths, \\
\poemll    and their bodies are healthy. \\
\poeml \v{5}They do not experience problems common to ordinary people; \\
\poemll    they aren't afflicted as others\fnote{Lit. \fbib{human beings}} are. \\
\poeml \v{6}Therefore pride is their necklace \\
\poemll    and violence covers them like a garment. \\
\poeml \v{7}Their eyes bulge from obesity \\
\poemll    and the imaginations of their mind cross the border into sin.\fnote{The Heb. lacks \fbib{into sin}} \\
\poeml \v{8}In their mockery they speak evil; \\
\poemll    from their arrogant position they speak oppression. \\
\poeml \v{9}They choose to speak\fnote{Lit. \fbib{They set their mouth}} against heaven; \\
\poemll    while they talk about things on earth. \\
\poeml \v{10}Therefore God's\fnote{Lit. \fbib{his}} people return there \\
\poemll    and drink it all in like water until they're satiated. \\
\poeml \v{11}Then they say, \\
\poemll    ``How can God know? \\
\poemlll       Does the Most High have knowledge?'' \\
\poeml \v{12}Just look at these wicked people! \\
\poemll    They're perpetually carefree \\
\poemlll       as they increase their wealth. \\
\poeml \v{13}I kept my heart pure for nothing \\
\poemll    and kept my hands clean from guilt. \\
\poeml \v{14}For I suffer all day long \\
\poemll    and I am punished every morning. \\
\poeml \v{15}If I say, ``I will talk like this,'' \\
\poemll    I would betray a generation of your children. \\
\poeml \v{16}When I tried to understand this, \\
\poemll    it was too difficult for me \\
\poeml \v{17}until I entered the sanctuaries of God. \\
\poemll    Then I understood their destiny. \\
\poeml \v{18}You have certainly set them in slippery places; \\
\poemll    you will make them fall to their ruin. \\
\poeml \v{19}How desolate they quickly become, \\
\poemll    completely destroyed by calamities. \\
\poeml \v{20}Like a dream when one awakens, Lord, \\
\poemll    you will despise their image when you arise. \\
\poeml \v{21}When I chose to be bitter \\
\poemll    I was emotionally pained. \\
\poeml \v{22}Then, I was too stupid \\
\poemll    and didn't realize I was acting like\fnote{The Heb. lacks \fbib{acting like}} a wild animal with you. \\
\poeml \v{23}But now I am always with you, \\
\poemll    for you keep holding my right hand. \\
\poeml \v{24}You will guide me with your wise advice, \\
\poemll    and later you will receive me with honor. \\
\poeml \v{25}Whom do I have in heaven but you? \\
\poemll    I desire nothing on this\fnote{The Heb. lacks \fbib{this}}earth. \\
\poeml \v{26}My body and mind may fail, \\
\poemll    but God is my strength\fnote{Lit. \fbib{is the rock of my heart}} and my portion forever. \\
\poeml \v{27}Those far from you will perish; \\
\poemll    you will destroy those who are unfaithful to you. \\
\poeml \v{28}As for me, how good for me it is that God is near! \\
\poemll    I have made the Lord \divine{God} my refuge \\
\poemlll       so I can tell about all your deeds.
\end{poetry}
\labelpsalm{74}
\psalminfo{An instruction\fnote{T Lit. \fbib{maskil}} of Asaph}
\passage{A Plea for Deliverance}

\begin{poetry}
\poeml \v{1}Why, God? Have you rejected us forever? \\
\poemll    Your anger is burning against the sheep of your pasture. \\
\poeml \v{2}Remember your community, \\
\poeml whom you purchased long ago, \\
\poeml the tribe whom you redeemed \\
\poemll    for your possession. \\
\poeml Remember\fnote{The Heb. lacks \fbib{Remember}} Mount Zion, \\
\poemll    where you live. \\
\poemlll       \v{3}Hurry! Look at the permanent ruins---
\end{poetry}

\begin{poetry}
\poemll    every calamity the enemy brought upon the Holy Place. \\
\poeml \v{4}Those who are opposing you roar \\
\poemll    where we were meeting with you; \\
\poemlll       they unfurl their war banners as signs. \\
\poeml \v{5}As one blazes a trail \\
\poemll    through a forest with an ax, \\
\poeml \v{6}now they're tearing down all its carved work \\
\poemll    with hatchets and hammers. \\
\poeml \v{7}They burned your sanctuary to the ground, \\
\poemll    desecrating your dwelling place. \\
\poeml \v{8}They say to themselves, \\
\poemll    ``We'll crush them completely;'' \\
\poemlll       They burned down all the meeting places of God in the land. \\
\poeml \v{9}We see no signs for us; \\
\poemll    there is no longer a prophet, \\
\poemlll       and no one among us knows the future.\fnote{Lit. \fbib{knows when}} \\
\poeml \v{10}God, how long will the adversary scorn \\
\poemll    while the enemy despises your name endlessly? \\
\poeml \v{11}Why do you not withdraw your hand--- \\
\poemll    your right hand---from your bosom \\
\poemlll       and destroy them?\fnote{The Heb. lacks \fbib{them}} \\
\poeml \v{12}But God is my king from ancient times, \\
\poemll    who brings acts of deliverance throughout the earth. \\
\poeml \v{13}You split the sea by your own power. \\
\poemll    You shattered the heads of sea monsters in the water. \\
\poeml \v{14}You crushed the heads of Leviathan. \\
\poemll    You set it as food for desert creatures.\fnote{Or \fbib{people}} \\
\poeml \v{15}You opened both the spring and the river; \\
\poemll    you dried up flowing rivers. \\
\poeml \v{16}Yours is the day, and yours is the night; \\
\poemll    you established the moon and the sun. \\
\poeml \v{17}You set all the boundaries of the earth; \\
\poemll    you made summer and winter. \\
\poeml \v{18}Remember this: The enemy scorns the \divine{Lord} \\
\poemll    and a foolish people despises your name. \\
\poeml \v{19}Don't hand over the life of your dove to beasts; \\
\poemll    do not continuously forget your afflicted ones. \\
\poeml \v{20}Pay attention to your covenant, \\
\poemll    for the dark regions of the earth are full of violence. \\
\poeml \v{21}Don't let the oppressed return in humiliation. \\
\poemll    The poor and needy will praise your name. \\
\poeml \v{22}Get up, God, and prosecute your case--- \\
\poemll    remember that you're being scorned \\
\poemlll       by fools all day long. \\
\poeml \v{23}Don't ignore the shout of those opposing you, \\
\poemll    The uproar of those who rebel against you continuously.
\end{poetry}
\labelpsalm{75}
\psalminfo{To the Director: To the tune of\fnote{T The Heb. lacks \fbib{the tune of}} ``Do not Destroy!'' \\ A psalm of Asaph. A song.}
\passage{Praise to God for Justice}

\begin{poetry}
\poeml \v{1}We praise you, God! \\
\poemll    We praise you\fnote{The Heb. lacks \fbib{you}}---your presence\fnote{Lit. \fbib{name}} draws near--- \\
\poemlll       as we declare your wonderful deeds. \\
\poeml \v{2}``At the time that I choose \\
\poemll    I will judge the righteous.\fnote{Or \fbib{judge righteously}} \\
\poeml \v{3}While the earth and all its inhabitants melt away, \\
\poemll    it is I who keep its pillars firm.''
\end{poetry}
\interlude{Interlude}

\begin{poetry}
\poeml \v{4}I will say to the proud, ``Don't brag,'' \\
\poemll    and to the wicked, \\
\poemlll       ``Don't vaunt your strength.\fnote{Lit. \fbib{Don't lift up your horn}} \\
\poeml \v{5}Don't use your strength to fight heaven\fnote{Lit. \fbib{Don't lift your horns to the height}} \\
\poemll    or speak from stubborn arrogance.''\fnote{Lit. \fbib{speak with a stiff neck}} \\
\poeml \v{6}For exaltation comes not from the east, \\
\poemll    the west, or the wilderness, \\
\poeml \v{7}since God is the judge. \\
\poemll    This one he will debase or that one he will exalt. \\
\poeml \v{8}For there is a cup in the hand of the \divine{Lord}, \\
\poemll    foaming with well-mixed wine \\
\poeml that he will pour out, leaving only the dregs, \\
\poemll    from which all the wicked of the earth will drink. \\
\poeml \v{9}But as for me, I will declare forever, \\
\poemll    singing praise to the God of Jacob. \\
\poeml \v{10}I will cut down the strength\fnote{Lit. \fbib{horn}} of the wicked, \\
\poemll    but the strength\fnote{Lit. \fbib{horn}} of the righteous will be lifted up.
\end{poetry}
\labelpsalm{76}
\psalminfo{To the Director: With stringed instruments. A psalm of Asaph. A song.}
\passage{The Awesome God}

\begin{poetry}
\poeml \v{1}God is known in Judah; \\
\poemll    in Israel his reputation is great. \\
\poeml \v{2}His abode is in Salem,\fnote{I.e. Jerusalem} \\
\poemll    his dwelling place in Zion. \\
\poeml \v{3}There he shattered sharp arrows, \\
\poemll    shields, swords, and weapons of\fnote{The Heb. lacks \fbib{weapons of}} war.
\end{poetry}
\interlude{Interlude}

\begin{poetry}
\poeml \v{4}You are enveloped by light; \\
\poemll    more majestic than mountains filled with game. \\
\poeml \v{5}Brave men were plundered \\
\poemll    while they slumbered in their sleep. \\
\poemlll       All the men of the army were immobilized. \\
\poeml \v{6}At the sound of your battle cry, God of Jacob, \\
\poemll    both horse and chariot rider fell into deep sleep. \\
\poeml \v{7}You are awesome! \\
\poemll    who can stand in your presence when you're angry? \\
\poeml \v{8}From heaven you declared judgment. \\
\poemll    The earth stands in awe and is quiet \\
\poeml \v{9}when God arose to execute justice \\
\poemll    and to deliver all the afflicted of the earth.
\end{poetry}
\interlude{Interlude}

\begin{poetry}
\poeml \v{10}Even human anger praises you; \\
\poemll    you will wear the survivors of your wrath as an ornament.\fnote{The Heb. lacks \fbib{as an ornament}} \\
\poeml \v{11}Let everyone who surrounds the \divine{Lord} your God \\
\poemll    make a vow and fulfill it to the Awesome One.\fnote{Or \fbib{to the one whom they fear}} \\
\poeml \v{12}He will humble the arrogant\fnote{Lit. \fbib{the spirit of}} commanders-in-chief,\fnote{Lit. \fbib{Nagidim}; i.e. senior officers entrusted with dual roles of operational oversight and administrative authority} \\
\poemll    instilling fear among the kings of the earth.
\end{poetry}
\labelpsalm{77}
\psalminfo{To the director: To Jeduthun. A psalm of Asaph.}
\passage{Remembering God in Times of Trouble}

\begin{poetry}
\poeml \v{1}I cry out to God! \\
\poemll    I cry out to God and he hears me. \\
\poeml \v{2}When I was in distress, I sought the Lord; \\
\poemll    my hands were raised at night \\
\poeml and they did not grow weary. \\
\poemlll       I refused to be comforted. \\
\poeml \v{3}I remember God, and I groan; \\
\poemll    I meditate, while my spirit grows faint.
\end{poetry}
\interlude{Interlude}

\begin{poetry}
\poeml \v{4}You kept my eyes open; \\
\poemll    I was troubled and couldn't speak. \\
\poeml \v{5}I thought of ancient times, \\
\poemll    considering years long past. \\
\poeml \v{6}During the night I remembered my song. \\
\poemll    I meditate in my heart, \\
\poemlll       and my spirit ponders. \\
\poeml \v{7}Will the Lord reject me\fnote{The Heb. lacks \fbib{me}} forever \\
\poemll    and not show favor again? \\
\poeml \v{8}Has his gracious love ceased forever? \\
\poemll    Will his promise be unfulfilled in future generations? \\
\poeml \v{9}Has God forgotten to be gracious? \\
\poemll    Has he in anger withheld his compassion?
\end{poetry}
\interlude{Interlude}

\begin{poetry}
\poeml \v{10}So I say: ``It causes me pain \\
\poemll    that the right hand of the Most High has changed.'' \\
\poeml \v{11}I will remember the \divine{Lord}'s deeds; \\
\poemll    indeed, I will remember your awesome deeds from long ago. \\
\poeml \v{12}As I meditate on all your works, \\
\poemll    I will consider your awesome deeds. \\
\poeml \v{13}God, your way is holy. \\
\poemll    What god is like our great God? \\
\poeml \v{14}God, you are the one performing awesome deeds. \\
\poemll    You reveal your might among the nations. \\
\poeml \v{15}You delivered\fnote{Or \fbib{redeemed}} your people--- \\
\poemll    the descendants of Jacob and Joseph--- \\
\poemlll       with your power.
\end{poetry}
\interlude{Interlude}

\begin{poetry}
\poeml \v{16}The waters saw you, God; \\
\poemll    the waters saw you and writhed. \\
\poemlll       Indeed, the depths of the sea quaked. \\
\poeml \v{17}The clouds poured rain; \\
\poemll    the skies rumbled. \\
\poemlll       Indeed, your lightning bolts flashed.\fnote{Lit. \fbib{your fierce arrows traveled}} \\
\poeml \v{18}Your thunderous sound was in a whirlwind; \\
\poemll    your lightning lights up the world; \\
\poemlll       the earth becomes agitated and quakes. \\
\poeml \v{19}Your way was through the sea, \\
\poemll    and your path through mighty waters, \\
\poemlll       but your footprints cannot be traced.\fnote{Lit. \fbib{steps are not recognized}} \\
\poeml \v{20}You have led your people like a flock \\
\poemll    by the hands of Moses and Aaron.
\end{poetry}
\labelpsalm{78}
\psalminfo{An instruction\fnote{T Lit. \fbib{maskil}} of Asaph}
\passage{Remembering God in Times of Trouble}

\begin{poetry}
\poeml \v{1}Listen, my people, to my instruction. \\
\poemll    Hear\fnote{Lit. \fbib{Stretch out your ear}} the words of my mouth. \\
\poeml \v{2}I will tell\fnote{Lit. \fbib{will open my mouth in}} a parable, \\
\poemll    speaking riddles from long ago--- \\
\poeml \v{3}things that we have heard and known \\
\poemll    and that our ancestors related to us. \\
\poeml \v{4}We will not withhold them from their descendants; \\
\poemll    we'll declare to the next generation the praises of the \divine{Lord}--- \\
\poemlll       his might and awesome deeds that he has performed. \\
\poeml \v{5}He established a decree in Jacob, \\
\poemll    and established the Law in Israel, \\
\poeml that he commanded our ancestors \\
\poemll    to reveal to their children \\
\poeml \v{6}in order that the next generation--- \\
\poemll    children yet to be born--- \\
\poeml will know them and \\
\poemll    in turn teach them to their children. \\
\poeml \v{7}Then they will put their trust in God \\
\poemll    and they will not forget his awesome deeds. \\
\poemlll       Instead, they will keep his commandments. \\
\poeml \v{8}They will not be like the rebellious generation of their ancestors, \\
\poemll    a rebellious generation, \\
\poeml whose heart was not steadfast, \\
\poemll    and whose spirits were unfaithful to God. \\
\poeml \v{9}The descendants of Ephraim were sharp shooters with the bow, \\
\poemll    but they retreated in the day of battle. \\
\poeml \v{10}They did not keep God's covenant, \\
\poemll    and refused to live by his Law. \\
\poeml \v{11}They have forgotten what he has done, \\
\poemll    his awesome deeds that they witnessed. \\
\poeml \v{12}He performed marvelous things \\
\poemll    in the presence of their ancestors \\
\poeml in the land of Egypt--- \\
\poemll    in the fields of Zoan. \\
\poeml \v{13}He divided the sea so that they were able to cross; \\
\poemll    he caused the water to stand in a single location. \\
\poeml \v{14}He led them with a cloud during the day, \\
\poemll    and during the night with light from the fire. \\
\poeml \v{15}He caused the rocks to split in the wilderness, \\
\poemll    and gave them water\fnote{Lit. \fbib{drink}} as from an abundant sea. \\
\poeml \v{16}He brought streams from rock, \\
\poemll    causing water to flow like a river. \\
\poeml \v{17}But time and again, they sinned against him, \\
\poemll    rebelling against the Most High in the desert. \\
\poeml \v{18}To test God was in their minds, \\
\poemll    when they demanded food to satisfy their cravings.\fnote{Lit. \fbib{food for their soul}} \\
\poeml \v{19}They spoke against God by asking, \\
\poemll    ``Is God able to prepare a feast\fnote{Or \fbib{table}} in the desert? \\
\poeml \v{20}It's true that\fnote{Lit. \fbib{Indeed,}} Moses\fnote{Lit. \fbib{he}} struck the rock so that water flowed forth \\
\poemll    and torrents of water gushed out, \\
\poeml but is he also able to give bread \\
\poemll    or to supply meat for his people?'' \\
\poeml \v{21}Therefore, when the \divine{Lord} heard this, he was angry, \\
\poemll    and fire broke out against Jacob. \\
\poeml Moreover, his anger flared against Israel, \\
\poeml \v{22}because they didn't believe in God \\
\poemlll       and didn't trust in his deliverance. \\
\poeml \v{23}Yet he commanded the skies above \\
\poemll    and the doors of the heavens to open, \\
\poeml \v{24}so that manna rained down on them for food \\
\poemll    and he sent them the grain of heaven. \\
\poeml \v{25}Mortal men\fnote{Lit. \fbib{A man}} ate the food of angels; \\
\poemll    he sent provision to them in abundance. \\
\poeml \v{26}He stirred up the east wind in the heavens \\
\poemll    and drove the south wind by his might. \\
\poeml \v{27}He caused meat to rain on them like dust \\
\poemll    and winged birds as the sand of the sea. \\
\poeml \v{28}He caused these to fall in the middle of the camp \\
\poemll    and all around their tents. \\
\poeml \v{29}So they ate and were very satisfied, \\
\poemll    because he granted their desire. \\
\poeml \v{30}However, before they had fulfilled their desire, \\
\poemll    while their food was still in their mouths, \\
\poeml \v{31}the anger of God flared against them, \\
\poemll    and he killed the strongest men \\
\poemlll       and humbled Israel's young men. \\
\poeml \v{32}In spite of all of this, they kept on sinning \\
\poemll    and didn't believe in his marvelous deeds. \\
\poeml \v{33}So he made their days end in futility, \\
\poemll    and their years with sudden terror. \\
\poeml \v{34}When he struck them, they sought him; \\
\poemll    they repented, and eagerly sought God. \\
\poeml \v{35}Then they remembered that God was their rock, \\
\poemll    and the Most High God was their deliverer. \\
\poeml \v{36}But they deceived him with their mouths; \\
\poemll    they lied to him with their tongues. \\
\poeml \v{37}For their hearts weren't committed to him, \\
\poemll    and they weren't faithful to his covenant. \\
\poeml \v{38}But he, being merciful, forgave their iniquity \\
\poemll    and didn't destroy them; \\
\poeml He restrained his anger \\
\poemll    and didn't vent all his wrath. \\
\poeml \v{39}For he remembered that they were only flesh, \\
\poemll    a passing wind that doesn't return. \\
\poeml \v{40}How they rebelled against him in the desert, \\
\poemll    grieving him in the wilderness! \\
\poeml \v{41}They tested God again and again, \\
\poemll    provoking the Holy One of Israel. \\
\poeml \v{42}They did not remember his power--- \\
\poemll    the day he delivered them from their adversary, \\
\poeml \v{43}when he set his signs in Egypt \\
\poemll    and his wonders in the plain of Zoan. \\
\poeml \v{44}He turned their rivers into blood \\
\poemll    and made their streams undrinkable. \\
\poeml \v{45}He sent swarms of insects to bite them \\
\poemll    and frogs to destroy them. \\
\poeml \v{46}He gave their crops to caterpillars \\
\poemll    and what they worked for to locusts. \\
\poeml \v{47}He destroyed their vines with hail \\
\poemll    and their sycamore\fnote{The sycamore fruit tree native to Israel bears figs} trees with frost. \\
\poeml \v{48}He delivered their beasts to hail \\
\poemll    and their livestock to lightning bolts. \\
\poeml \v{49}He inflicted his burning anger, \\
\poemll    wrath, indignation, and distress, \\
\poemlll       sending destroying angels among them. \\
\poeml \v{50}He blazed a path for his anger; \\
\poemll    he did not stop short from killing them, \\
\poemlll       but handed them over to pestilence. \\
\poeml \v{51}He struck every firstborn in Egypt, \\
\poemll    the first fruits of their manhood in the tents of Ham. \\
\poeml \v{52}Yet he led out his people like sheep, \\
\poemll    guiding them like a flock in the desert. \\
\poeml \v{53}He led them to safety so they would not fear. \\
\poemll    As for their enemies, the sea covered them. \\
\poeml \v{54}He brought the people\fnote{Lit. \fbib{brought them}} to the border of his holy mountain, \\
\poemll    which he acquired by his might. \\
\poeml \v{55}He drove out nations before them \\
\poemll    and allotted their tribal inheritance, \\
\poemlll       settling the tribes of Israel in their tents. \\
\poeml \v{56}But they tested the Most High God by rebelling against him, \\
\poemll    and they did not obey his statutes. \\
\poeml \v{57}They fell away and were as disloyal as their ancestors. \\
\poemll    They became unreliable, like a defective bow; \\
\poeml \v{58}they angered him with their high places \\
\poemll    and with their carved images they made him jealous. \\
\poeml \v{59}God heard and became furious, \\
\poemll    and he completely rejected Israel. \\
\poeml \v{60}He abandoned the tent at Shiloh, \\
\poemll    the tent that he established among mankind. \\
\poeml \v{61}Then he sent his might\fnote{I.e. the Ark of the Covenant} into captivity \\
\poemll    and his glory into the control of the adversary. \\
\poeml \v{62}He delivered his people over to the sword \\
\poemll    and was angry with his possession. \\
\poeml \v{63}The young men were consumed by fire, \\
\poemll    and the virgins had no marriage celebrations.\fnote{Lit. \fbib{virgins sang no wedding song}} \\
\poeml \v{64}The priests fell by the sword, \\
\poemll    yet their widows couldn't weep. \\
\poeml \v{65}The \divine{Lord} awoke as though from sleep, \\
\poemll    like a mighty warrior stimulated by wine. \\
\poeml \v{66}He beat back his adversaries, \\
\poemll    permanently disgracing them. \\
\poeml \v{67}He rejected the clan\fnote{Lit. \fbib{tent}} of Joseph; \\
\poemll    and the tribe of Ephraim he did not choose. \\
\poeml \v{68}But he chose the tribe of Judah, \\
\poemll    the mountain of Zion, which he loves. \\
\poeml \v{69}He built his sanctuary, high as the heavens, \\
\poemll    like the earth that he established forever. \\
\poeml \v{70}Then he chose his servant David, \\
\poemll    whom he took from the sheepfold. \\
\poeml \v{71}He brought him from birthing sheep \\
\poemll    to care for Jacob, his people, \\
\poemlll       Israel, his possession. \\
\poeml \v{72}David\fnote{Lit. \fbib{He}} shepherded them with a devoted heart, \\
\poemll    and led them with skillful hands.
\end{poetry}
\labelpsalm{79}
\psalminfo{A Psalm of Asaph}
\passage{A Prayer for Jerusalem}

\begin{poetry}
\poeml \v{1}God, nations have invaded your land\fnote{Lit. \fbib{your possession}; or \fbib{your inheritance}} \\
\poemll    to desecrate your holy Temple, \\
\poemlll       to destroy Jerusalem, \\
\poeml \v{2}to give the corpses of your servants \\
\poemll    as food for the birds of the skies \\
\poeml and the flesh of your godly ones \\
\poemll    to the beasts of the earth; \\
\poeml \v{3}to make their blood flow like water around Jerusalem, \\
\poemll    with no one being buried. \\
\poeml \v{4}We have become a reproach to our neighbors, \\
\poemll    a mockery and a derision to those around us. \\
\poeml \v{5}How long, \divine{Lord}, will you be angry? Forever? \\
\poemll    Will your jealousy burn like fire? \\
\poeml \v{6}Pour out your wrath upon the nations \\
\poemll    that do not acknowledge you, \\
\poeml and over the kingdoms \\
\poemll    that do not call on your name. \\
\poemll    \v{7}For they consumed Jacob, \\
\poemll    making his dwelling place desolate. \\
\poeml \v{8}Don't charge\fnote{Lit. \fbib{remember}} us for previous iniquity, \\
\poemll    but let your compassion come quickly to us, \\
\poemlll       for we have been brought very low. \\
\poeml \v{9}Help us, God, our deliverer, \\
\poemll    on account of your glorious name, \\
\poeml deliver us and forgive\fnote{Lit. \fbib{cover}} our sins \\
\poemll    on account of your name. \\
\poeml \v{10}Why should the nations say, ``Where is their God?'' \\
\poemll    Let vengeance for the blood of your servants be meted\fnote{Lit. \fbib{spilled}} out \\
\poemlll       before our eyes and among the nations. \\
\poeml \v{11}Let the cries of the prisoners reach you. \\
\poemll    With the strength of your power, \\
\poemlll       release those condemned to death.\fnote{Lit. \fbib{the sons of death}} \\
\poeml \v{12}Pay back our neighbors seven times\fnote{Or \fbib{seven-fold}} \\
\poemll    the reproach with which they reproached you, \divine{Lord}. \\
\poemll    \v{13}Then we, your people, the sheep of your pasture, \\
\poemll    will praise you always, from generation to generation. \\
\poemlll       We will declare your praise.
\end{poetry}
\labelpsalm{80}
\psalminfo{For the Director of Music: According to ``The Lilies''. A testimony of Asaph. A psalm.}
\passage{A Prayer for Jerusalem}

\begin{poetry}
\poeml \v{1}Shepherd of Israel, listen! \\
\poemll    The one who leads Joseph like a flock, \\
\poeml the one enthroned on the cherubim, \\
\poemll    display your glory.\fnote{The Heb. lacks \fbib{your glory}} \\
\poeml \v{2}Reveal\fnote{Or \fbib{rouse}, \fbib{stir up}} your power before Ephraim, Benjamin, and Manasseh, \\
\poemll    then come to our rescue. \\
\poeml \v{3}God, restore us, \\
\poemll    show your favor\fnote{Lit. \fbib{cause your face to shine}} and deliver us. \\
\poeml \v{4}\divine{Lord} God of the Heavenly Armies, \\
\poemll    when will your smoldering anger\fnote{Lit. \fbib{Until when will you burn in anger}} \\
\poemlll       toward your people's prayers cease?\fnote{The Heb. lacks \fbib{cease}} \\
\poeml \v{5}You fed them tears as their food, \\
\poemll    and caused them to drink a full measure of tears. \\
\poeml \v{6}You have set us at strife against our neighbors \\
\poemll    and our enemies deride us. \\
\poeml \v{7}God of the Heavenly Armies, restore us \\
\poemll    and show your favor,\fnote{Lit. \fbib{cause your face to shine}} \\
\poemlll       so we may be delivered. \\
\poeml \v{8}You uprooted a vine from Egypt, \\
\poemll    and drove out nations to transplant it. \\
\poeml \v{9}You cleared the ground\fnote{The Heb. lacks \fbib{ground}} so that its roots grew \\
\poemll    and filled the land. \\
\poeml \v{10}Mountains were covered by its shadows, \\
\poemll    and the mighty cedars by its branches. \\
\poeml \v{11}Its branches spread out to the Mediterranean\fnote{The Heb. lacks \fbib{Mediterranean}} Sea \\
\poemll    and its shoots to the Euphrates\fnote{The Heb. lacks \fbib{Euphrates}} River. \\
\poeml \v{12}Why did you break down its walls \\
\poemll    so that those who pass by pluck its fruits?\fnote{Heb. lacks \fbib{its fruits}} \\
\poeml \v{13}Wild boars of the forest gnaw at it, \\
\poemll    and creatures of the field feed on it. \\
\poeml \v{14}God of the Heavenly Armies, return! \\
\poemll    Look down from heaven and see. \\
\poemlll       Show care\fnote{Lit. \fbib{Visit}} toward this vine. \\
\poeml \v{15}The root\fnote{Or \fbib{stock}} that your right hand planted, \\
\poemll    the shoot\fnote{Lit. \fbib{son}} that you tended for yourself, \\
\poeml \v{16}was burned with fire, cut off, \\
\poemll    and destroyed on account of your rebuke. \\
\poeml \v{17}May you support the man at your right hand; \\
\poemll    the son of man whom you have raised for yourself. \\
\poeml \v{18}Then we will not turn away from you. \\
\poemll    Restore us, so we can call upon your name. \\
\poeml \v{19}God of hosts, restore to us the light of your favor.\fnote{Lit. \fbib{face}} \\
\poemll    Then we'll be delivered.
\end{poetry}
\labelpsalm{81}
\psalminfo{For the Director: On the Gittith. By Asaph.}
\passage{Celebrating and Remembering God}

\begin{poetry}
\poeml \v{1}Sing joyfully to God, our strength. \\
\poemll    Raise a shout to the God of Jacob. \\
\poeml \v{2}Sing a song and play the tambourine, \\
\poemll    the pleasant-sounding lyre along with the harp. \\
\poeml \v{3}Blow the ram's horn when there is a New Moon, \\
\poemll    when there is a full moon, \\
\poemlll       on our festival day, \\
\poeml \v{4}because it is a statute in Israel, \\
\poemll    an ordinance by the God of Jacob, \\
\poeml \v{5}a decree that he prescribed for Joseph \\
\poemll    when he went throughout the land of Egypt, \\
\poemlll       speaking a language I did not recognize.\fnote{Lit. \fbib{hear}} \\
\poeml \v{6}I removed the burden from your\fnote{Lit. \fbib{his}} shoulder; \\
\poemll    your\fnote{Lit. \fbib{his}} hands were freed of the burdensome basket.\fnote{Lit. \fbib{hands let pass through the basket}} \\
\poeml \v{7}In a time of need you called out and I delivered you; \\
\poemll    I answered you from the dark thundercloud; \\
\poemlll       I tested you at the waters of Meribah.
\end{poetry}
\interlude{Interlude}

\begin{poetry}
\poeml \v{8}Listen, My people and I will warn you. \\
\poemll    Israel, if only you would obey me! \\
\poeml \v{9}You must neither have a foreign god over you \\
\poemll    or worship a strange god. \\
\poeml \v{10}I am the \divine{Lord} your God, \\
\poemll    who brought you out of the land of Egypt, \\
\poemlll       open your mouth that I may fill it. \\
\poeml \v{11}Yet my people didn't obey my voice; \\
\poemll    Israel didn't submit to me. \\
\poeml \v{12}So I allowed them\fnote{Or \fbib{it} / \fbib{her}} to continue in their stubbornness, \\
\poemll    living by their own advice. \\
\poeml \v{13}If only my people would obey me, \\
\poemll    if only Israel would walk in my ways! \\
\poeml \v{14}Then I would quickly subdue their enemies. \\
\poemll    I would turn against their foes. \\
\poeml \v{15}Those who hate the \divine{Lord} will cringe before him; \\
\poemll    their punishment will be permanent. \\
\poeml \v{16}But I will feed Israel\fnote{Lit. \fbib{him}} with the finest wheat, \\
\poemll    satisfying you with honey from the rock.
\end{poetry}
\labelpsalm{82}
\psalminfo{A Psalm of Asaph}
\passage{Asking God for Justice}

\begin{poetry}
\poeml \v{1}God takes his stand in the divine assembly; \\
\poemll    among the divine\fnote{Or \fbib{angelic}} beings\fnote{Or \fbib{the gods}} he renders judgment: \\
\poeml \v{2}``How long will you judge partially \\
\poemll    by showing favor on the wicked?\fnote{Lit. \fbib{you lift the face}}
\end{poetry}
\interlude{Interlude}

\begin{poetry}
\poeml \v{3}``Defend the poor and the fatherless. \\
\poemll    Vindicate the afflicted and the poor. \\
\poeml \v{4}Rescue the poor and the needy, \\
\poemll    delivering them from the power of the wicked. \\
\poeml \v{5}They neither know nor understand; \\
\poemll    they walk about in the dark \\
\poemlll       while all the foundations of the earth are shaken. \\
\poeml \v{6}``Indeed I said, `You are gods, \\
\poemll    and all of you are sons of the Most High. \\
\poeml \v{7}However, as all human beings do, you will die, \\
\poemll    and like other rulers, you will fall.' \\
\poeml \v{8}Arise, God, to judge the earth, \\
\poemll    for all nations belong to you.
\end{poetry}
\labelpsalm{83}
\psalminfo{A song. A Psalm of Asaph}
\passage{A Plea for Judgment}

\begin{poetry}
\poeml \v{1}God, do not rest! \\
\poemll    Don't be silent! \\
\poemlll       Don't stay inactive, God! \\
\poeml \v{2}See! Your enemies rage; \\
\poemll    those who hate you issue threats.\fnote{Lit. \fbib{you lift their head}} \\
\poeml \v{3}They plot against\fnote{Lit. \fbib{they make shrewd secret counsel}} your people \\
\poemll    and conspire against your cherished ones. \\
\poeml \v{4}They say, ``Let us go and erase them as a nation \\
\poemll    so the name of Israel will not be remembered anymore.'' \\
\poeml \v{5}Indeed, they shrewdly planned together, \\
\poemll    forming an alliance against you--- \\
\poeml \v{6}the tents of Edom, the Ishmaelites, \\
\poemll    Moab, the Hagrites, \\
\poeml \v{7}Gebal, Ammon, Amalek, Philistia, \\
\poemll    and the inhabitants of Tyre. \\
\poeml \v{8}Even Assyria joined them \\
\poemll    to strengthen the descendants of Lot.
\end{poetry}
\interlude{Interlude}

\begin{poetry}
\poeml \v{9}Deal with them as you did to Midian,\fnote{Cf. Judg 7:1-24} \\
\poemll    Sisera, and Jabin at the Kishon Brook.\fnote{Cf. Judg 4:7, 15, 21-24} \\
\poeml \v{10}They were destroyed at En-dor \\
\poemll    and became as dung on the ground. \\
\poeml \v{11}Punish their nobles like Oreb and Zeeb,\fnote{Cf. Judg 7:25} \\
\poemll    and all their princes like Zebah and Zalmunna,\fnote{Cf. Judg 8:12, 21} \\
\poeml \v{12}who said, ``Let us possess the pastures of God.'' \\
\poeml \v{13}God, set them up like dried thistles, \\
\poemll    like straw before the wind. \\
\poeml \v{14}Like a fire burning a forest, \\
\poemll    and a flame setting mountains ablaze. \\
\poeml \v{15}Pursue them with your storm and \\
\poemll    terrify them with your whirlwind. \\
\poeml \v{16}Fill their faces with shame \\
\poemll    until they seek your name, God. \\
\poeml \v{17}Let them be humiliated and terrified permanently \\
\poemll    until they die in shame.\fnote{Lit. \fbib{they are abased and destroyed}} \\
\poeml \v{18}Then they will know that you alone--- \\
\poemll    whose name is \divine{Lord}--- \\
\poemlll       are the Most High over all the earth.
\end{poetry}
\labelpsalm{84}
\psalminfo{To the Director: On the Gittith. \\ A Psalm by the descendants of Korah.}
\passage{Longing for God}

\begin{poetry}
\poeml \v{1}How lovely are your dwelling places, \\
\poemll    \divine{Lord} of the Heavenly Armies. \\
\poeml \v{2}I desire and long \\
\poemll    for the Temple\fnote{The Heb. lacks \fbib{temple}} courts of the \divine{Lord}. \\
\poeml My heart and body\fnote{Lit. \fbib{flesh}} sing for joy \\
\poemll    to the living God.\fnote{Or \fbib{the God of life}} \\
\poeml \v{3}Even the sparrow found a house for herself \\
\poemll    and the swallow a nest \\
\poeml to lay\fnote{Or \fbib{to set up}} her young at your altar, \\
\poemll    \divine{Lord} of the Heavenly Armies, \\
\poemlll       my king and God. \\
\poeml \v{4}How happy are those who live in your Temple, \\
\poemll    for they can praise you continuously.
\end{poetry}
\interlude{Interlude}

\begin{poetry}
\poeml \v{5}How happy are those whose strength is in you, \\
\poemll    whose heart is on your path. \\
\poeml \v{6}They will pass through the Baca Valley \\
\poemll    where he will prepare a spring for them; \\
\poemlll       even the early rain will cover it with blessings. \\
\poeml \v{7}They will walk from strength to strength; \\
\poemll    each will appear before God in Zion. \\
\poeml \v{8}\divine{Lord} God of the Heavenly Armies, hear my prayer! \\
\poemll    Listen, God of Jacob!
\end{poetry}
\interlude{Interlude}

\begin{poetry}
\poeml \v{9}God, look at our shield, \\
\poemll    and show favor to your anointed, \\
\poeml \v{10}for a day in your Temple\fnote{The Heb. lacks \fbib{temple}} courts is better \\
\poemll    than a thousand elsewhere; \\
\poeml I would rather stand \\
\poemll    at the entrance of God's house \\
\poemlll       than live in the tent of wickedness. \\
\poeml \v{11}For the \divine{Lord} God is a sun and shield; \\
\poemll    the \divine{Lord} grants grace and favor; \\
\poeml the \divine{Lord} will not withhold any good thing \\
\poemll    from those who walk blamelessly. \\
\poeml \v{12}\divine{Lord} of Heavenly Armies, \\
\poemll    how happy are those who trust in you.
\end{poetry}
\labelpsalm{85}
\psalminfo{To the Director: A Psalm by the descendants of Korah.}
\passage{Restore Us, God}

\begin{poetry}
\poeml \v{1}\divine{Lord} you have favored your land \\
\poemll    and restored the fortunes of Jacob. \\
\poeml \v{2}You took away the iniquity of your people, \\
\poemll    forgiving all their sins
\end{poetry}
\interlude{Interlude}

\begin{poetry}
\poeml \v{3}You withdrew all your wrath \\
\poemll    and turned away from your burning anger. \\
\poeml \v{4}Restore us, God of our salvation, \\
\poemll    and stop being angry with us. \\
\poeml \v{5}Will you be angry with us forever? \\
\poemll    Will you prolong your anger from generation to generation? \\
\poeml \v{6}Will you restore our lives again \\
\poemll    so that your people may rejoice in you? \\
\poeml \v{7}\divine{Lord}, show your gracious love \\
\poemll    and deliver us. \\
\poeml \v{8}Let me listen to what God, the \divine{Lord}, says; \\
\poemll    for the \divine{Lord} will promise peace \\
\poeml to his people, to his holy ones; \\
\poemll    may they not return to foolishness. \\
\poeml \v{9}Surely, he will soon deliver those who fear him, \\
\poemll    for his glory will live in our land. \\
\poeml \v{10}Gracious love and truth meet; \\
\poemll    righteousness and peace kiss. \\
\poeml \v{11}Truth sprouts up from the ground, \\
\poemll    while righteousness looks down from the sky. \\
\poeml \v{12}The \divine{Lord} will also provide what is good, \\
\poemll    and our land will yield its produce. \\
\poeml \v{13}Righteousness will go before him \\
\poemll    to prepare a path for his steps.
\end{poetry}
\labelpsalm{86}
\psalminfo{A Davidic prayer}
\passage{Help Us, God}

\begin{poetry}
\poeml \v{1}\divine{Lord}, listen and answer me, \\
\poemll    for I am afflicted and needy. \\
\poeml \v{2}Protect me, for I am faithful;\fnote{Or \fbib{righteous}} \\
\poemll    My God, deliver your servant who trusts in you. \\
\poeml \v{3}Have mercy on me Lord, \\
\poemll    for I call on you all day long. \\
\poeml \v{4}Your servant rejoices, \\
\poemll    because, Lord, I set my hope on\fnote{Or \fbib{I lift my soul to}} you. \\
\poeml \v{5}Indeed you, Lord, are kind and forgiving, \\
\poemll    overflowing with gracious love to everyone who calls on you. \\
\poeml \v{6}Hear my prayer, \divine{Lord}; \\
\poemll    attend to my prayer of supplication. \\
\poeml \v{7}In my troubled times I will call on you, \\
\poemll    for you will answer me. \\
\poeml \v{8}No one can compare with you among the gods, Lord; \\
\poemll    No one can accomplish\fnote{The Heb. lacks \fbib{can accomplish}} your work. \\
\poeml \v{9}All the nations that you have established will come \\
\poemll    and worship you, my Lord. \\
\poemlll       They will honor your name. \\
\poeml \v{10}For you are great, \\
\poemll    and you are doing awesome things; \\
\poemlll       you alone are God. \\
\poeml \v{11}Teach me your ways, \divine{Lord}, \\
\poemll    that I may walk in your truth; \\
\poemlll       let me wholeheartedly\fnote{Lit. \fbib{my heart be undivided}} revere your name. \\
\poeml \v{12}I will praise you, Lord my God, with my whole being; \\
\poemll    and I will honor your name continuously. \\
\poeml \v{13}For great is your gracious love to me; \\
\poemll    you've delivered me from the depths of Sheol.\fnote{I.e. the realm of the dead} \\
\poeml \v{14}God, arrogant men rise up against me, \\
\poemll    while a company of ruthless individuals want to kill me. \\
\poemlll       They do not have regard for you.\fnote{Lit. \fbib{don't set you before them}} \\
\poeml \v{15}But you, Lord, are a compassionate God, \\
\poemll    merciful and patient,\fnote{Or \fbib{slow to anger}} \\
\poemlll       with unending gracious love and faithfulness. \\
\poeml \v{16}Return to me and have mercy on me; \\
\poemll    clothe your servant with your strength \\
\poemlll       and deliver the son of your maid servant. \\
\poeml \v{17}Show me a sign of your goodness, \\
\poemll    so that those who hate me will see it and be ashamed. \\
\poemlll       For you, \divine{Lord}, will help and comfort me.
\end{poetry}
\labelpsalm{87}
\psalminfo{A psalm by the descendants of Korah. A song.}
\passage{The Holy City for All People}

\begin{poetry}
\poeml \v{1}God's\fnote{Lit. \fbib{His}} foundation is in the holy mountains. \\
\poeml \v{2}The \divine{Lord} loves the gates of Zion \\
\poemll    more than the dwellings of Jacob. \\
\poeml \v{3}Glorious things are spoken about you, \\
\poemll    city of God.
\end{poetry}
\interlude{Interlude}

\begin{poetry}
\poeml \v{4}I will mention Rahab and Babylon \\
\poemll    among those who acknowledge me--- \\
\poeml including Philistia, Tyre, and Ethiopia\fnote{Lit. \fbib{Cush}}--- \\
\poemll    ``This one was born there,'' they say.\fnote{The Heb. lacks \fbib{they say}} \\
\poeml \v{5}Indeed, about Zion it will be said: \\
\poemll    ``More than one person\fnote{Lit. \fbib{a man and a man}} was born in it,'' and \\
\poemlll       ``The Most High himself did\fnote{Or \fbib{secured}} it.'' \\
\poeml \v{6}The \divine{Lord} will record, \\
\poemll    as he registers the peoples,\fnote{Lit. \fbib{record, in a registry of people}} \\
\poemlll       ``This one was born there.''
\end{poetry}
\interlude{Interlude}

\begin{poetry}
\poeml \v{7}Then singers, as they play their instruments,\fnote{Or \fbib{singers and flute players}} will declare, \\
\poemll    ``All my roots\fnote{Lit. \fbib{springs}} are in you.''
\end{poetry}
\labelpsalm{88}
\psalminfo{A song. A psalm by the descendants of Korah. According to Machalath Leannoth. An instruction\fnote{T Lit. \fbib{maskil}} by Heman the Ezrahite.}
\passage{A Cry for Help}

\begin{poetry}
\poeml \v{1}\divine{Lord}, God of my salvation, \\
\poemll    by day and by night I cry out before you. \\
\poeml \v{2}Let my prayer come before you; \\
\poemll    listen\fnote{Lit. \fbib{stretch your ears}} to my cry. \\
\poeml \v{3}For my life is filled with troubles \\
\poemll    as I approach Sheol.\fnote{I.e. the realm of the dead} \\
\poeml \v{4}I am considered as one of those descending into the Pit,\fnote{I.e. the place of punishment in the afterlife} \\
\poemll    like a mighty man without strength, \\
\poeml \v{5}released to remain\fnote{The Heb. lacks \fbib{to remain}} with the dead, \\
\poemll    lying in a grave like a corpse, \\
\poeml remembered no longer, \\
\poemll    and cut off from your power. \\
\poeml \v{6}You have assigned me to the lowest part of the Pit,\fnote{I.e. the place of punishment in the afterlife} \\
\poemll    to the darkest depths. \\
\poeml \v{7}Your anger lies heavily upon me; \\
\poemll    you pound\fnote{Lit. \fbib{oppress}} me with all your waves.
\end{poetry}
\interlude{Interlude}

\begin{poetry}
\poeml \v{8}You caused my acquaintances to shun me;\fnote{Lit. \fbib{to be distant}} \\
\poemll    you make me extremely abhorrent to them. \\
\poemlll       Restrained, I am unable to go out. \\
\poeml \v{9}My eyes languish on account of my affliction; \\
\poemll    all day long I call out to you, \divine{Lord}, \\
\poemlll       I spread out my hands to you. \\
\poeml \v{10}Can you perform wonders for the dead? \\
\poemll    Can departed spirits stand up to praise you?
\end{poetry}
\interlude{Interlude}

\begin{poetry}
\poeml \v{11}Can your gracious love be declared in the grave \\
\poemll    or your faithfulness in Abaddon?\fnote{I.e. the realm of destruction in the afterlife} \\
\poeml \v{12}Can your awesome deeds be known in darkness \\
\poemll    or your righteousness in the land of oblivion? \\
\poeml \v{13}As for me, I cry out to you \divine{Lord}, \\
\poemll    and in the morning my prayer greets you. \\
\poeml \v{14}Why, \divine{Lord}, have you rejected me? \\
\poemll    Why have you hidden your face from me? \\
\poeml \v{15}Since my youth I have been oppressed \\
\poemll    and in danger of death. \\
\poeml I bear your dread \\
\poemll    and am overwhelmed. \\
\poeml \v{16}Your burning anger overwhelms me; \\
\poemll    your terrors destroy me. \\
\poeml \v{17}Like waters, they engulf me all day long; \\
\poemll    they surround me on all sides. \\
\poeml \v{18}You caused my friend and neighbor to shun me;\fnote{Lit. \fbib{be distant from}} \\
\poemll    and my acquaintances are confused.\fnote{Lit. \fbib{are in darkness}}
\end{poetry}
\labelpsalm{89}
\psalminfo{An instruction\fnote{T Lit. \fbib{maskil}}. By Ethan, the Ezrahite}
\passage{God's Covenant with David}

\begin{poetry}
\poeml \v{1}I will sing forever about the gracious love of the \divine{Lord}; \\
\poemll    from generation to generation \\
\poemlll       I will declare your faithfulness with my mouth. \\
\poeml \v{2}I will declare that your gracious love was established forever; \\
\poemll    in the heavens itself, you have established your faithfulness. \\
\poeml \v{3}I have made a covenant with my chosen one; \\
\poemll    I have made a promise to David, my servant. \\
\poeml \v{4}``I will establish your dynasty forever, \\
\poemll    and I will lift up one who will build\fnote{Or \fbib{confirm}} your throne \\
\poemlll       from generation to generation.''
\end{poetry}
\interlude{Interlude}

\begin{poetry}
\poeml \v{5}Even the heavens praise your awesome deeds, \divine{Lord}, \\
\poemll    your faithfulness in the assembly of the holy ones. \\
\poeml \v{6}For who in the skies compares to the \divine{Lord}? \\
\poemll    Who is like the \divine{Lord} among the divine beings? \\
\poeml \v{7}God is feared in the council of the holy ones, \\
\poemll    revered by all those around him. \\
\poeml \v{8}\divine{Lord} God of the Heavenly Armies, \\
\poemll    who is as mighty as you, \divine{Lord}? \\
\poemlll       Your faithfulness surrounds you. \\
\poeml \v{9}You rule over the majestic\fnote{Lit. \fbib{roaring}} sea; \\
\poemll    when its waves surge, \\
\poemlll       you calm them. \\
\poeml \v{10}You crushed the proud one\fnote{Lit. \fbib{Rahab}} to death; \\
\poemll    with your powerful arm \\
\poemlll       you scattered your enemies. \\
\poeml \v{11}Heaven and the earth belong to you, \\
\poemll    the world and everything it contains--- \\
\poemlll       you established them. \\
\poeml \v{12}The north and south---you created them; \\
\poemll    Tabor and Hermon joyously praise your name. \\
\poeml \v{13}Your arm is strong; \\
\poemll    your hand is mighty; \\
\poemlll       indeed, your right hand is victorious.\fnote{Lit. \fbib{lifted up}} \\
\poeml \v{14}Righteousness and justice make up \\
\poemll    the foundation of your throne; \\
\poemlll       gracious love and truth meet before you. \\
\poeml \v{15}How happy are the people who can worship joyfully!\fnote{Lit. \fbib{who know the joyful shout}} \\
\poemll    \divine{Lord}, they walk in the light of your presence. \\
\poeml \v{16}In your name they rejoice all day long; \\
\poemll    they exult in your justice.\fnote{Or \fbib{righteousness}} \\
\poeml \v{17}For you are their strength's grandeur; \\
\poemll    by your favor you exalted our power.\fnote{Lit. \fbib{horn}} \\
\poeml \v{18}Indeed, our shield belongs to the \divine{Lord}, \\
\poemll    and our king to the Holy One of Israel.
\end{poetry}
\passage{God's Describes His Anointed}

\begin{poetry}
\poeml \v{19}You spoke to your faithful\fnote{Or \fbib{godly}; so MT LXX; DSS 4Q98\textsuperscript{g} reads \fbib{chosen}} ones through a vision:\fnote{So MT LXX; DSS 4Q98\textsuperscript{g} reads \fbib{vision; you said}}
\end{poetry}

\begin{poetry}
\poeml ``I will set a helper over\fnote{So MT LXX; DSS 4Q98\textsuperscript{g} reads \fbib{I have lent support to}} a warrior. \\
\poemll    I will raise up a chosen one from the people. \\
\poeml \v{20}I have found my servant David; \\
\poemll    I have anointed him with my sacred oil, \\
\poeml \v{21}with whom my power\fnote{Lit. \fbib{hand}} will be firmly established; \\
\poemll    for my arm will strengthen him. \\
\poeml \v{22}No enemy will deceive him; \\
\poemll    no wicked person\fnote{Lit. \fbib{no son of iniquity}} will afflict him. \\
\poemll    \v{23}I will crush his enemies before him \\
\poemll    and strike those who hate him. \\
\poemll    \v{24}My faithfulness and gracious love will be with him, \\
\poemll    and in my name his power\fnote{Lit. \fbib{horn}} will be exalted. \\
\poeml \v{25}I will place his hand\fnote{I.e. his authority} over the sea, \\
\poemll    and his right hand\fnote{I.e. his authority} over the rivers. \\
\poeml \v{26}He will announce to me \\
\poemll    `You are my father, \\
\poemlll       my God, and the rock of my salvation.' \\
\poeml \v{27}``Indeed, I myself made him the firstborn, \\
\poemll    the highest of the kings of the earth. \\
\poeml \v{28}I will show\fnote{Lit. \fbib{keep}} my gracious love toward him forever, \\
\poemll    since my covenant is securely established with him. \\
\poeml \v{29}I will establish his dynasty\fnote{Lit. \fbib{seed}} forever, \\
\poemll    and his throne as long as heaven endures.\fnote{Lit. \fbib{as the days of the heavens}} \\
\poeml \v{30}``But if his sons abandon my laws and \\
\poemll    do not follow my ordinances, \\
\poeml \v{31}if they profane my statutes; \\
\poemll    and do not keep my commands, \\
\poeml \v{32}then I will punish their disobedience with a rod \\
\poemll    and their iniquity with lashes. \\
\poeml \v{33}But I will not cut off\fnote{Lit. \fbib{break}} my gracious love from him, \\
\poemll    and I will not stop being faithful. \\
\poeml \v{34}I will not dishonor my covenant, \\
\poemll    because I will not change what I have spoken.\fnote{Lit. \fbib{what goes out of my lips}} \\
\poeml \v{35}I have sworn by my holiness once for all: \\
\poemll    I will not lie to David. \\
\poeml \v{36}His dynasty\fnote{Lit. \fbib{seed}} will last forever \\
\poemll    and his throne will be like the sun before me. \\
\poeml \v{37}It will be established forever like the moon, \\
\poemll    a faithful witness in the sky.''
\end{poetry}
\interlude{Interlude}
\passage{A Commitment to Persevere}

\begin{poetry}
\poeml \v{38}But you have spurned, rejected, \\
\poemll    and became angry with your anointed one. \\
\poeml \v{39}You have dishonored the covenant with your servant; \\
\poemll    you have defiled his crown on the ground. \\
\poeml \v{40}You have broken through all his\fnote{Or \fbib{its}} walls; \\
\poemll    you have laid his fortresses in ruin. \\
\poeml \v{41}All who pass by on their way plunder him; \\
\poemll    he has become a reproach to his neighbors. \\
\poeml \v{42}You have exalted the right hand of his adversaries; \\
\poemll    you have caused all of his enemies to rejoice. \\
\poeml \v{43}Moreover, you have turned back the edge of his sword \\
\poemll    and did not support him in battle. \\
\poeml \v{44}You have caused his splendor\fnote{Or \fbib{luster}} to cease \\
\poemll    and cast down his throne to the ground. \\
\poeml \v{45}You have caused the days of his youth to be cut short; \\
\poemll    you have covered him with shame.
\end{poetry}
\interlude{Interlude}

\begin{poetry}
\poeml \v{46}How long, \divine{Lord}, will you hide yourself? Forever? \\
\poemll    Will your anger continuously burn like fire? \\
\poeml \v{47}Remember how short my lifetime is! \\
\poemll    How powerless have you created all human beings!\fnote{Lit. \fbib{all sons of Adam}} \\
\poeml \v{48}What valiant man can live and not see death? \\
\poemll    Who can deliver himself\fnote{Lit. \fbib{deliver his soul}} from the power\fnote{Lit. \fbib{hand}} of Sheol.\fnote{I.e. the realm of the dead}
\end{poetry}
\interlude{Interlude}

\begin{poetry}
\poeml \v{49}Where is your gracious love of old, Lord, \\
\poemll    that in your faithfulness you promised to David? \\
\poeml \v{50}Remember, Lord, the reproach of your servant! \\
\poemll    I carry inside me all the insults of many people, \\
\poeml \v{51}when your enemies reproached you, \divine{Lord}, \\
\poemll    when they reproached the footsteps\fnote{Lit. \fbib{the hind part}} of your anointed. \\
\poeml \v{52}Blessed is the \divine{Lord} forever! \\
\poemll    Amen and amen!
\end{poetry}
\booksection{\divine{BOOK IV} (Psalms 90-106)}
\labelpsalm{90}
\psalminfo{A prayer by Moses, the godly man}
\passage{Life is Short}

\begin{poetry}
\poeml \v{1}Lord, you've been our refuge\fnote{Or \fbib{our dwelling place}} \\
\poemll    from generation to generation. \\
\poeml \v{2}Before the mountains were formed \\
\poemll    or the earth and the world were brought forth, \\
\poemlll       you are God from eternity to eternity. \\
\poeml \v{3}You return people to dust \\
\poemll    merely by\fnote{The Heb. lacks \fbib{merely by}} saying, ``Return, you mortals!'' \\
\poeml \v{4}One thousand years in your sight are but a single day \\
\poemll    that passes by, just like a night watch. \\
\poeml \v{5}You will sweep them away while they are asleep--- \\
\poemll    by morning they are like growing grass. \\
\poeml \v{6}In the morning it blossoms and is renewed, \\
\poemll    but toward evening, it fades and withers. \\
\poeml \v{7}Indeed, we are consumed\fnote{Lit. \fbib{finished}} by your anger \\
\poemll    and terrified by your wrath. \\
\poeml \v{8}You have set our iniquities before you, \\
\poemll    what we have concealed in the light of your presence. \\
\poeml \v{9}All our days pass\fnote{Or \fbib{turn}} away in your wrath; \\
\poemll    our years fade away\fnote{Lit. \fbib{are finished}} and end like a sigh. \\
\poeml \v{10}We live for 70 years, \\
\poemll    or 80 years if we're healthy,\fnote{Lit. \fbib{strong}} \\
\poeml yet even in the prime years\fnote{Lit. \fbib{the pride}} there are troubles and sorrow. \\
\poemll    They pass by quickly and we fly away. \\
\poeml \v{11}Who can know the intensity of your anger? \\
\poemll    Because our fear of you matches your wrath, \\
\poeml \v{12}teach us to keep account of our days \\
\poemll    so we may develop inner wisdom. \\
\poeml \v{13}Please return, \divine{Lord}! When will it be? \\
\poemll    Comfort your servants. \\
\poeml \v{14}Satisfy us in the morning with your gracious love \\
\poemll    so we may sing for joy \\
\poemlll       and rejoice every day. \\
\poeml \v{15}Cause us to rejoice throughout the time when you have afflicted us, \\
\poemll    the years when we have known\fnote{Lit. \fbib{seen}} trouble. \\
\poeml \v{16}May your awesome deeds be revealed to your servants, \\
\poemll    as well as your splendor to their children. \\
\poeml \v{17}May your favor be on us, Lord our God; \\
\poemll    make our endeavors successful; \\
\poemlll       yes, make our endeavors secure!
\end{poetry}
\labelpsalm{91}
\psalminfo{A Davidic Psalm\fnote{T So LXX; DSS 11QPs\textsuperscript{a} lacks \fbib{Psalm}; the Heb. lacks this line}}
\passage{God is My Refuge}

\begin{poetry}
\poeml \v{1}The one who lives in the shelter of the Most High, \\
\poemll    who rests in the shadow of the Almighty, \\
\poeml \v{2}will say to the \divine{Lord}, \\
\poemll    ``You are my refuge, my fortress, \\
\poemlll       and my God in whom I trust!'' \\
\poeml \v{3}He will surely deliver you from the hunter's snare \\
\poemll    and from the destructive plague. \\
\poeml \v{4}With his feathers he will cover you, \\
\poemll    under his wings you will find safety. \\
\poemlll       His truth is your shield and armor. \\
\poeml \v{5}You need not fear terror that stalks\fnote{The Heb. lacks \fbib{that stalks}} in the night, \\
\poemll    the arrow that flies in the day, \\
\poeml \v{6}plague that strikes in the darkness, \\
\poemll    or calamity that destroys at noon. \\
\poeml \v{7}If a thousand fall at your side \\
\poemll    or ten thousand at your right hand, \\
\poemlll       it will not overcome you. \\
\poeml \v{8}Only observe\fnote{Or \fbib{Only you will observe}} it with your eyes, \\
\poemll    and you will see how the wicked are paid back. \\
\poeml \v{9}``\divine{Lord}, you are my refuge!'' \\
\poeml Because you chose the Most High as your dwelling place, \\
\poeml \v{10}no evil will fall upon you, \\
\poemlll       and no affliction will approach your tent, \\
\poeml \v{11}for he will command his angels \\
\poemll    to protect you in all your ways. \\
\poeml \v{12}With their hands they will lift you up \\
\poemll    so you will not trip over a stone. \\
\poeml \v{13}You will stomp on lions and snakes; \\
\poemll    you will trample young lions and serpents.
\passage{The \divine{Lord} Speaks}
\poeml \v{14}Because he has focused his love on me, \\
\poemll    I will deliver him. \\
\poeml I will protect him\fnote{Or \fbib{will set him on high}} \\
\poemll    because he knows my name. \\
\poeml \v{15}When he calls out to me, \\
\poemll    I will answer him. \\
\poeml I will be with him in his\fnote{The Heb. lacks \fbib{his}} distress. \\
\poemll    I will deliver him, \\
\poemlll       and I will honor him. \\
\poeml \v{16}I will satisfy him with long life; \\
\poemll    I will show him my deliverance.
\end{poetry}
\labelpsalm{92}
\psalminfo{A Psalm. A song for the Sabbath Day}
\passage{Praise and Thanksgiving to God}

\begin{poetry}
\poeml \v{1}It is good to give thanks to the \divine{Lord} \\
\poemll    and to sing praise to your name, Most High; \\
\poeml \v{2}to proclaim your gracious love in the morning \\
\poemll    and your faithfulness at night, \\
\poeml \v{3}accompanied by a ten-stringed instrument and a lyre, \\
\poemll    and the contemplative sound of a harp. \\
\poeml \v{4}Because you made me glad \\
\poemll    with your awesome deeds, \divine{Lord}, \\
\poemlll       I will sing for joy at the works of your hands. \\
\poeml \v{5}How great are your works, \divine{Lord}! \\
\poemll    Your thoughts are unfathomable.\fnote{Lit. \fbib{very deep}} \\
\poeml \v{6}A stupid man doesn't know, \\
\poemll    and a fool can't comprehend this: \\
\poeml \v{7}Though the wicked sprout like grass; \\
\poemll    and all who practice iniquity flourish, \\
\poemlll       it is they who will be eternally destroyed. \\
\poeml \v{8}But you are exalted forever, \divine{Lord}. \\
\poeml \v{9}Look at your enemies, \divine{Lord}! \\
\poemll    Look at your enemies, for they are destroyed; \\
\poemlll       everyone who practices iniquity will be scattered.\fnote{Lit. \fbib{divided}; or \fbib{separated}} \\
\poeml \v{10}You've grown my strength\fnote{Lit. \fbib{horn}} like the horn of a wild ox; \\
\poemll    I was anointed with fresh oil. \\
\poeml \v{11}My eyes gloated over those who lie in wait for me;\fnote{The Heb. lacks \fbib{for me}} \\
\poemll    when those of evil intent attack me, my ears will hear. \\
\poeml \v{12}The righteous will flourish like palm trees; \\
\poemll    they will grow like a cedar in Lebanon. \\
\poeml \v{13}Planted in the \divine{Lord}'s Temple, \\
\poemll    they will flourish in the courtyard of our God. \\
\poeml \v{14}They will still bear fruit even in old age;\fnote{Lit. \fbib{Even with gray hair}} \\
\poemll    they will be luxuriant and green. \\
\poeml \v{15}They will proclaim: ``The \divine{Lord} is upright; \\
\poemll    my rock, in whom there is no injustice.''
\end{poetry}
\labelpsalm{93}
\passage{God Reigns}

\begin{poetry}
\poeml \v{1}The \divine{Lord} reigns! He is clothed in majesty; \\
\poemll    the \divine{Lord} is clothed, \\
\poemlll       and he is girded\fnote{So MT; DSS 11QPsa reads \fbib{is robed in power and girded himself}} with strength. \\
\poeml Indeed, the world is well established, \\
\poemll    and cannot be shaken. \\
\poeml \v{2}Your throne has been established since time immemorial; \\
\poemll    you are king from eternity. \\
\poeml \v{3}The rivers have flooded, \divine{Lord}; \\
\poemll    the rivers have spoken aloud, \\
\poemlll       the rivers have lifted up their crushing waves. \\
\poeml \v{4}More than the sound of surging waters--- \\
\poemll    the majestic waves of the sea--- \\
\poemlll       the \divine{Lord} on high is majestic. \\
\poeml \v{5}Your decrees are very trustworthy, \\
\poemll    and holiness always befits your house, \divine{Lord}.
\end{poetry}
\labelpsalm{94}
\passage{God Avenges His Own}

\begin{poetry}
\poeml \v{1}God of vengeance, \\
\poemll    \divine{Lord} God of vengeance, \\
\poemlll       display your splendor!\fnote{The Heb. lacks \fbib{your splendor}} \\
\poeml \v{2}Stand up, judge of the earth, \\
\poemll    and repay the proud. \\
\poeml \v{3}How long will the wicked, \divine{Lord}, \\
\poemll    how long will the wicked continue to triumph? \\
\poeml \v{4}When they speak, they spew arrogance. \\
\poemll    Everyone who practices iniquity brags about it.\fnote{The Heb. lacks \fbib{about it}} \\
\poeml \v{5}\divine{Lord}, they have crushed your people, \\
\poemll    afflicting your heritage. \\
\poeml \v{6}The wicked\fnote{Lit. \fbib{They}} kill widows and foreigners; \\
\poemll    they murder orphans. \\
\poeml \v{7}They say, ``The \divine{Lord} cannot see, \\
\poemll    and the God of Jacob will not notice.'' \\
\poeml \v{8}Pay attention, you dull ones among the crowds! \\
\poemll    You fools! Will you ever become wise? \\
\poeml \v{9}The one who formed\fnote{Lit. \fbib{planted}} the ear can hear, can he not? \\
\poemll    The one who made the eyes can see, can he not? \\
\poeml \v{10}The one who disciplines nations can rebuke them, can he not? \\
\poemll    The one who teaches mankind can discern, can he not? \\
\poeml \v{11}The \divine{Lord} knows the thoughts of human beings--- \\
\poemll    that they are futile. \\
\poeml \v{12}How blessed is the man whom you instruct, \divine{Lord}, \\
\poemll    whom you teach from your Law, \\
\poeml \v{13}keeping him calm when times are troubled \\
\poemll    until a pit has been dug for the wicked. \\
\poeml \v{14}For the \divine{Lord} will not forsake his people; \\
\poemll    he will not abandon his heritage. \\
\poeml \v{15}Righteousness will be restored with justice, \\
\poemll    and all the pure of heart will follow it. \\
\poeml \v{16}Who will rise up for me against the wicked? \\
\poemll    Who will stand for me against those who practice iniquity? \\
\poeml \v{17}If the \divine{Lord} had not been my helper, \\
\poemll    I would have quickly become silent. \\
\poeml \v{18}When I say that my foot is shaking, \\
\poemll    your gracious love, \divine{Lord}, will sustain me. \\
\poeml \v{19}When my anxious inner thoughts become overwhelming, \\
\poemll    your comfort encourages me. \\
\poeml \v{20}Will destructive national leaders,\fnote{Lit. \fbib{destructive throne}} \\
\poemll    who plan wicked things through misuse of the Law, \\
\poemlll       be allied with you? \\
\poeml \v{21}They gather together against the righteous, \\
\poemll    condemning the innocent to death. \\
\poeml \v{22}But the \divine{Lord} is my stronghold, \\
\poemll    and my God, the rock, is my refuge. \\
\poeml \v{23}He will repay them for their sin; \\
\poemll    he will annihilate them because of their evil. \\
\poemlll       The \divine{Lord} our God will annihilate them.
\end{poetry}
\labelpsalm{95}
\passage{Worship and Obedience}

\begin{poetry}
\poeml \v{1}Come! Let us sing joyfully to the \divine{Lord}! \\
\poemll    Let us shout for joy to the rock of our salvation. \\
\poeml \v{2}Let us come into his presence with thanksgiving; \\
\poemll    let us shout with songs of praise to him. \\
\poeml \v{3}For the \divine{Lord} is an awesome God; \\
\poemll    a great king above all divine beings.\fnote{Or \fbib{all gods}} \\
\poeml \v{4}He holds in his hand the lowest parts of the earth \\
\poemll    and the mountain peaks belong to him. \\
\poeml \v{5}The sea that he made belongs to him, \\
\poemll    along with the dry land that his hands formed. \\
\poeml \v{6}Come! Let us worship and bow down; \\
\poemll    let us kneel in the presence of the \divine{Lord}, who made us. \\
\poeml \v{7}For he is our God, and we are the people of his pasture \\
\poemll    and the flock in his care.\fnote{Lit. \fbib{flock of his hand}} \\
\poeml If only you would listen to his voice today, \\
\poeml \v{8}do not be stubborn like your ancestors were\fnote{Lit. \fbib{stubborn as at}} at Meribah, \\
\poeml as on that day at Massah, in the wilderness, \\
\poeml \v{9}where your ancestors tested me. \\
\poeml They tested me, \\
\poemll    even though they had seen my awesome deeds. \\
\poeml \v{10}For forty years I loathed that generation, so I said, \\
\poemll    ``They are a people whose hearts continuously err, \\
\poemlll       and they have not understood my ways.'' \\
\poeml \v{11}So in my anger I declared an oath: \\
\poemll    ``They are not to enter my place of rest.''
\end{poetry}
\labelpsalm{96}
\passage{Give Glory to the \divine{Lord}}

\begin{poetry}
\poeml \v{1}Sing a new song to the \divine{Lord}! \\
\poemll    Sing to the \divine{Lord}, all the earth! \\
\poeml \v{2}Sing to the \divine{Lord}! \\
\poemll    Bless his name! \\
\poemlll       Proclaim his deliverance every day! \\
\poeml \v{3}Declare his glory among the nations \\
\poemll    and his awesome deeds among all the peoples! \\
\poeml \v{4}For the \divine{Lord} is great, \\
\poemll    and greatly to be praised; \\
\poemlll       he is awesome above all gods. \\
\poeml \v{5}For all the gods of the peoples are worthless idols, \\
\poemll    but the \divine{Lord} made the heavens. \\
\poeml \v{6}Splendor and majesty are before him; \\
\poemll    might and beauty are in his sanctuary. \\
\poeml \v{7}Ascribe to the \divine{Lord}, you families of peoples, \\
\poemll    ascribe to the \divine{Lord} glory and strength! \\
\poeml \v{8}Ascribe to the \divine{Lord} the glory due his name, \\
\poemll    bring an offering and enter his courts! \\
\poeml \v{9}Worship the \divine{Lord} in holy splendor; \\
\poemll    tremble before him, all the earth. \\
\poeml \v{10}Declare among the nations, ``The \divine{Lord} reigns!'' \\
\poemll    Indeed, he established the world so that it will not falter. \\
\poemlll       He will judge people fairly. \\
\poeml \v{11}The heavens will be glad \\
\poemll    and the earth will rejoice; \\
\poemlll       even the sea and everything that fills it will roar.\fnote{Or \fbib{thunder}} \\
\poeml \v{12}The field and all that is in it will rejoice; \\
\poemll    then all the trees of the forest will sing for joy \\
\poeml \v{13}in the \divine{Lord}'s presence, \\
\poeml because he is coming; \\
\poemll    indeed, he will come to judge the earth. \\
\poeml He will judge the world fairly \\
\poemll    and its people reliably.
\end{poetry}
\labelpsalm{97}
\passage{The \divine{Lord} is King}

\begin{poetry}
\poeml \v{1}The \divine{Lord} reigns! \\
\poemll    Let the earth rejoice! \\
\poemlll       May many islands be glad! \\
\poeml \v{2}Thick clouds are all around him; \\
\poemll    righteousness and justice are his throne's foundation. \\
\poeml \v{3}Fire goes out from his presence \\
\poemll    to consume his enemies on every side. \\
\poeml \v{4}His lightning bolts light the world; \\
\poemll    the earth sees and shakes. \\
\poeml \v{5}Mountains melt like wax in the \divine{Lord}'s presence--- \\
\poemll    In the presence of the \divine{Lord} of all the earth. \\
\poeml \v{6}The heavens declare his righteousness \\
\poemll    so that all the nations see his glory. \\
\poeml \v{7}All who serve carved images--- \\
\poemll    and those who praise idols---will be humiliated. \\
\poemlll       Worship him, all you ``gods''! \\
\poeml \v{8}Zion hears and rejoices; \\
\poemll    the towns\fnote{Lit. \fbib{daughters}} of Judah rejoice \\
\poemlll       on account of your justice, \divine{Lord}. \\
\poeml \v{9}For you, \divine{Lord}, are the Most High above all the earth; \\
\poemll    you are exalted high above all divine beings.\fnote{Or \fbib{all gods}} \\
\poeml \v{10}Hate evil, you who love the \divine{Lord}! \\
\poemll    He guards the lives of those who love him,\fnote{Or \fbib{his saints}} \\
\poemlll       delivering them from domination by\fnote{Lit. \fbib{from the hand of}} the wicked. \\
\poeml \v{11}Light shines on the righteous; \\
\poemll    gladness on the morally upright.\fnote{Lit. \fbib{the upright of heart}} \\
\poeml \v{12}Rejoice in the \divine{Lord}, you righteous ones! \\
\poemll    Give thanks at the mention of his holiness!
\end{poetry}
\labelpsalm{98}
\psalminfo{A psalm}
\passage{Sing Praise to the King}

\begin{poetry}
\poeml \v{1}Sing to the \divine{Lord} a new song, \\
\poemll    for he has done awesome deeds! \\
\poeml His right hand and powerful\fnote{Lit. \fbib{holy}} arm\fnote{I.e. \fbib{the Messiah}} \\
\poemll    have brought him victory. \\
\poeml \v{2}The \divine{Lord} has made his deliverance known; \\
\poemll    he has disclosed his justice before the nations. \\
\poeml \v{3}He has remembered his gracious love; \\
\poemll    his faithfulness toward the house of Israel; \\
\poemlll       all the ends of the earth saw our God's deliverance. \\
\poeml \v{4}Make a joyful noise to the \divine{Lord}, all the earth! \\
\poemll    Break forth into joyful songs of praise! \\
\poeml \v{5}Sing praises to the \divine{Lord} with a lyre--- \\
\poemll    with a lyre and a melodious song! \\
\poeml \v{6}With trumpets and the sound of a ram's horn \\
\poemll    shout in the presence of the \divine{Lord}, the king! \\
\poeml \v{7}Let the sea and everything in it shout,\fnote{Lit. \fbib{thunder}} \\
\poemll    along with the world and its inhabitants; \\
\poeml \v{8}let the rivers clap their hands in unison; \\
\poemll    and let the mountains sing for joy \\
\poeml \v{9}in the \divine{Lord}'s presence, who comes to judge the earth; \\
\poeml He'll judge the world righteously; \\
\poemll    and its people fairly.
\end{poetry}
\labelpsalm{99}
\passage{The \divine{Lord} is Holy}

\begin{poetry}
\poeml \v{1}The \divine{Lord} reigns--- \\
\poemll    let people tremble; \\
\poeml he is seated above the cherubim--- \\
\poemll    let the earth quake. \\
\poeml \v{2}The \divine{Lord} is great in Zion \\
\poemll    and is exalted above all peoples. \\
\poeml \v{3}Let them praise your great and awesome name. \\
\poemll    He is holy! \\
\poeml \v{4}A mighty king who loves justice, \\
\poemll    you have established fairness. \\
\poeml You have exercised justice \\
\poemll    and righteousness over Jacob. \\
\poeml \v{5}Exalt the \divine{Lord} our God; \\
\poemll    worship and bow down at his footstool; \\
\poemlll       He is holy! \\
\poeml \v{6}Moses and Aaron were among his priests; \\
\poemll    Samuel also was among those who invoked his name. \\
\poeml When they called on the \divine{Lord}, \\
\poemll    he answered them. \\
\poeml \v{7}In a pillar of cloud he spoke to them; \\
\poemll    they obeyed his decrees \\
\poemlll       and the Law that he gave them. \\
\poeml \v{8}\divine{Lord} our God, you answered them; \\
\poemll    you were their God who forgave\fnote{Lit. \fbib{carried}} them, \\
\poemlll       but also avenged their evil deeds. \\
\poeml \v{9}Exalt the \divine{Lord} our God and worship at his holy mountain, \\
\poemll    for the \divine{Lord} our God is holy!
\end{poetry}
\labelpsalm{100}
\psalminfo{A psalm of thanksgiving}
\passage{Give Thanks to the \divine{Lord}}

\begin{poetry}
\poeml \v{1}Shout to the \divine{Lord} all the earth! \\
\poeml \v{2}Serve the \divine{Lord} with joy. \\
\poemlll       Come before him with a joyful shout! \\
\poeml \v{3}Acknowledge that the \divine{Lord} is God. \\
\poemll    He made us and we belong to him; \\
\poeml we are his people \\
\poemll    and the sheep of his pasture. \\
\poeml \v{4}Enter his gates with thanksgiving \\
\poemll    and his courts with praise. \\
\poeml Thank him and bless his name, \\
\poeml \v{5}for the \divine{Lord} is good \\
\poemlll       and his gracious love stands forever. \\
\poeml His faithfulness remains from generation to generation.
\end{poetry}
\labelpsalm{101}
\psalminfo{A Davidic Psalm}
\passage{Remembering God's Love}

\begin{poetry}
\poeml \v{1}I will sing about gracious love and justice; \\
\poemll    \divine{Lord}, I will sing praise to you. \\
\poeml \v{2}I will pay attention to living a life of integrity--- \\
\poemll    when will I attain it? \\
\poemlll       I will live with integrity of heart in my house. \\
\poeml \v{3}I will not even think about doing anything lawless; \\
\poemll    I hate to do evil deeds; \\
\poemlll       I will have none of it. \\
\poeml \v{4}I will not allow anyone with a perverted mind in my presence; \\
\poemll    I will not be involved with\fnote{Lit. \fbib{not know}} anything evil. \\
\poeml \v{5}I will destroy the one who secretly slanders a friend. \\
\poemll    I will not allow the proud and haughty to prevail. \\
\poeml \v{6}My eyes are looking at the faithful of the land, \\
\poemll    so they may live with me; \\
\poemlll       The one who lives a life of integrity will serve me. \\
\poeml \v{7}A deceitful person will not sit in my house; \\
\poemll    A liar will not remain in my presence. \\
\poeml \v{8}Every morning I will destroy all the wicked of the land, \\
\poemll    eliminating everyone who practices iniquity from the \divine{Lord}'s city.
\end{poetry}
\labelpsalm{102}
\psalminfo{A prayer by the afflicted man who is overwhelmed and talks about his troubles with the \divine{Lord}.}
\passage{A Prayer for Help}

\begin{poetry}
\poeml \v{1}\divine{Lord}, hear my prayer! \\
\poemll    May my cry for help come to you. \\
\poeml \v{2}Do not hide your face from me when I am in trouble. \\
\poemll    Listen to me. \\
\poeml When I call to out you, \\
\poemll    hurry to answer me! \\
\poeml \v{3}For my days are vanishing like smoke; \\
\poemll    my bones are charred as in a fireplace. \\
\poeml \v{4}Withered like grass, my heart is overwhelmed, \\
\poemll    and I have even forgotten to eat my food. \\
\poeml \v{5}Because of the sound of my sighing, \\
\poemll    my bones cling to my skin. \\
\poeml \v{6}I resemble a pelican in the wilderness \\
\poemll    or an owl in a desolate land. \\
\poeml \v{7}I lie awake, \\
\poemll    yet I am like a bird isolated on a rooftop. \\
\poeml \v{8}My enemies revile me all day long; \\
\poemll    those who ridicule me use my name to curse. \\
\poeml \v{9}I have eaten ashes as food \\
\poemll    and mixed my drink with tears \\
\poeml \v{10}because of your indignation and wrath, \\
\poemll    when you lifted and threw me away. \\
\poeml \v{11}My life is\fnote{Lit. \fbib{My days are}} like a declining shadow, \\
\poemll    and I am withering like a plant. \\
\poeml \v{12}But you, \divine{Lord}, are enthroned forever; \\
\poemll    You are remembered throughout all generations. \\
\poeml \v{13}You will arise to extend compassion on Zion, \\
\poemll    for it is time to show her favor--- \\
\poemlll       the appointed time has come. \\
\poeml \v{14}Your servants take pleasure in its stones \\
\poemll    and delight in its debris. \\
\poeml \v{15}Nations will fear the name of the \divine{Lord}, \\
\poemll    and all the kings of the earth, your splendor. \\
\poeml \v{16}When the \divine{Lord} rebuilds Zion, \\
\poemll    he will appear in his glory. \\
\poeml \v{17}He will turn to the prayer of the destitute, \\
\poemll    not despising their prayer. \\
\poeml \v{18}Write this for the next generation, \\
\poemll    that a people yet to be created will praise the \divine{Lord}. \\
\poeml \v{19}For when he looked down from his holy heights--- \\
\poemll    the \divine{Lord} looked over the earth from heaven--- \\
\poeml \v{20}to listen to the groans of prisoners, \\
\poemll    to set free those condemned to death, \\
\poeml \v{21}so they would declare the name of the \divine{Lord} in Zion \\
\poemll    and his praise in Jerusalem, \\
\poeml \v{22}when people and kingdoms gather together \\
\poemll    to serve the \divine{Lord}. \\
\poeml \v{23}He has weakened my\fnote{So MT Qere (oral reading) DSS 4QPsb; Symmachus, Syr, Targ, and Hieronymous; MT \fbib{Qetiv} (written) reads \fbib{his}} strength along the way.\fnote{Or \fbib{strength in mid-course}} \\
\poemll    He has cut short my days. \\
\poeml \v{24}I say, ``My God, whose years continue through all generations, \\
\poemll    do not take me in the middle of my life. \\
\poeml \v{25}You established the earth long ago; \\
\poemll    the heavens are the work\fnote{So MT DSS 4QPs\textsuperscript{b}; LXX Targ DSS 11QPs\textsuperscript{a} read \fbib{works}} of your hands. \\
\poeml \v{26}They will perish, \\
\poemll    but you will remain; \\
\poeml and they all will become worn out,\fnote{So MT DSS 4QPs\textsuperscript{b} 11QPs\textsuperscript{a}; LXX reads \fbib{will grow old}} like a garment. \\
\poemll    You\fnote{So MT; LXX DSS 4QPs\textsuperscript{b} 11QPs\textsuperscript{a} read \fbib{And you}} will change them like clothing, \\
\poemlll       and they will pass away. \\
\poeml \v{27}But you remain the same; \\
\poemll    your years never end. \\
\poeml \v{28}May the descendants of your servants live securely, \\
\poemll    and may their children be established in your presence.''
\end{poetry}
\labelpsalm{103}
\psalminfo{Davidic}
\passage{Praise God, who Forgives}

\begin{poetry}
\poeml \v{1}Bless the \divine{Lord}, my soul, \\
\poemll    and all that is within me, bless\fnote{The Heb. lacks \fbib{bless}} his holy name. \\
\poeml \v{2}Bless the \divine{Lord}, my soul, \\
\poemll    and never forget any of his benefits: \\
\poeml \v{3}He continues to forgive all your sins, \\
\poemll    he continues to heal all your diseases, \\
\poeml \v{4}he continues to redeem your life from the Pit,\fnote{I.e. the place of punishment in the afterlife} \\
\poemll    and he continuously surrounds you \\
\poemlll       with gracious love and compassion. \\
\poeml \v{5}He keeps satisfying you with good things, \\
\poemll    and he keeps renewing your youth like the eagle's. \\
\poeml \v{6}The \divine{Lord} continuously does what is right, \\
\poemll    executing justice for all who are being oppressed. \\
\poeml \v{7}He revealed his plans\fnote{Lit. \fbib{ways}} to Moses \\
\poemll    and his deeds to the people of Israel. \\
\poeml \v{8}The \divine{Lord} is compassionate and gracious, \\
\poemll    patient,\fnote{Lit. \fbib{slow of anger}} and abundantly rich in gracious love. \\
\poeml \v{9}He does not maintain a dispute\fnote{Or \fbib{not rebuke}} continuously \\
\poemll    or remain angry for all time. \\
\poeml \v{10}He neither deals with us according to our sins, \\
\poemll    nor repays us equivalent to our iniquity. \\
\poeml \v{11}As high as heaven rises above earth, \\
\poemll    so his gracious love strengthens\fnote{MT phrase \fbib{high as} sounds like MT verb \fbib{strengthens}} those who fear him. \\
\poeml \v{12}As distant as the east is from the west, \\
\poemll    that is how far he has removed our sins from us. \\
\poeml \v{13}As a father has compassion for his children, \\
\poemll    so the \divine{Lord} has compassion for those who fear him. \\
\poeml \v{14}For he knows how we were formed, \\
\poemll    aware that we were made from dust. \\
\poeml \v{15}A person's life is like grass--- \\
\poemll    it blossoms like wild flowers, \\
\poeml \v{16}but when the wind blows through it, \\
\poemll    it withers away and no one remembers where it was. \\
\poeml \v{17}Yet the \divine{Lord}'s gracious love remains \\
\poemll    throughout eternity for those who fear him \\
\poemlll       and his righteous acts extend to their children's children, \\
\poeml \v{18}to those who keep his covenant \\
\poemll    and to those who remember to observe his precepts. \\
\poeml \v{19}The \divine{Lord} has established his throne in heaven \\
\poemll    and his kingdom rules over all. \\
\poeml \v{20}Bless the \divine{Lord}, you angels who belong to him, \\
\poemll    you mighty warriors who carry out his commands, \\
\poemlll       who are obedient to the sound of his words.\fnote{So LXX 4QPs\textsuperscript{b}; MT LXX read \fbib{word}} \\
\poeml \v{21}Bless the \divine{Lord}, all his heavenly armies, \\
\poemll    his ministers who do his will. \\
\poeml \v{22}Bless the \divine{Lord}, all his creation,\fnote{Lit. \fbib{works}; or \fbib{deeds}} \\
\poemll    in all the places of his dominion. \\
\poeml Bless the \divine{Lord}, my soul.
\end{poetry}
\labelpsalm{104}
\psalminfo{Davidic\fnote{T So LXX DSS 4QPs\textsuperscript{a} 11QPs\textsuperscript{a}; MT DSS 4QPs\textsuperscript{d} lack this line}}
\passage{Praise God, who Creates}

\begin{poetry}
\poeml \v{1}Bless the \divine{Lord}, my soul; \\
\poemll    \divine{Lord}, my God, you are very great. \\
\poeml You are clothed in splendor and majesty; \\
\poeml \v{2}you are wrapped in light like a garment, \\
\poemlll       stretching out the sky like a curtain. \\
\poeml \v{3}He lays the beams of his roof loft on the water above,\fnote{The Heb. lacks \fbib{above}} \\
\poemll    making clouds his chariot, \\
\poemlll       walking on the wings of the wind. \\
\poeml \v{4}He makes the winds his messengers, \\
\poemll    blazing fires his servants. \\
\poeml \v{5}He established the earth on its foundations, \\
\poemll    so that it never falters. \\
\poeml \v{6}You covered the primeval ocean like a garment; \\
\poemll    the water stood above the mountains. \\
\poeml \v{7}They flee at your rebuke; \\
\poemll    they rush away at the sound of your thunders. \\
\poeml \v{8}Mountains rise up and valleys sink \\
\poemll    to the place you have ordained for them. \\
\poeml \v{9}You have set a boundary they cannot cross; \\
\poemll    they will never again cover the earth. \\
\poeml \v{10}He causes springs to gush forth into rivers \\
\poemll    that flow between the\fnote{So LXX DSS 4QPs\textsuperscript{d}; the Heb. lacks \fbib{the}} mountains. \\
\poeml \v{11}They give water\fnote{The Heb. lacks \fbib{water}} for animals of the field to drink; \\
\poemll    the wild donkeys quench their thirst. \\
\poeml \v{12}Birds of the sky live beside them \\
\poemll    and chirp a song\fnote{Lit. \fbib{and they give a voice}} among the foliage. \\
\poeml \v{13}He waters the mountains from his heavenly rooms; \\
\poemll    the earth is satisfied from the fruit of your work. \\
\poeml \v{14}He causes grass to sprout for the cattle \\
\poemll    and plants for people to cultivate, \\
\poemlll       to produce food from the land, \\
\poeml \v{15}like wine that makes the heart of people\fnote{Lit. \fbib{man}} happy, \\
\poemll    oil that makes the face glow, \\
\poemlll       and food\fnote{Or \fbib{bread}} that sustains people.\fnote{Lit. \fbib{heart of man}} \\
\poeml \v{16}The loftiest trees\fnote{Lit. \fbib{trees of the \divine{Lord}}} are satisfied, \\
\poemll    the cedars of Lebanon that he planted, \\
\poeml \v{17}the birds build their nests there, \\
\poemll    and the heron builds\fnote{The Heb. lacks \fbib{builds}} its nest among the evergreen. \\
\poeml \v{18}The high mountains are for wild goats; \\
\poemll    the cliffs are a refuge for the rock badger. \\
\poeml \v{19}He made the moon to mark time;\fnote{Lit. \fbib{for an appointed time}} \\
\poemll    the sun knows its setting time. \\
\poeml \v{20}You bring darkness and it becomes night; \\
\poemll    when every beast of the forest prowls. \\
\poeml \v{21}Young lions roar for prey, \\
\poemll    seeking their food from God. \\
\poeml \v{22}When the sun rises, they\fnote{So MT; LXX DSS 4QPs\textsuperscript{d} 11QPs\textsuperscript{a} read \fbib{and they}} gather \\
\poemll    and lie down in their dens. \\
\poeml \v{23}People go out to their work \\
\poemll    and labor until evening. \\
\poeml \v{24}How numerous are your works, \divine{Lord}! \\
\poemll    You have made them all wisely; \\
\poemlll       the earth is filled with your creations.\fnote{Lit. \fbib{acquisitions}} \\
\poeml \v{25}There is the deep and wide sea, \\
\poemll    teeming with numberless creatures, \\
\poemlll       living things small and great. \\
\poeml \v{26}There, the ships pass through; \\
\poemll    Leviathan, which you created, frolics in it. \\
\poeml \v{27}All of them look to you \\
\poemll    to provide them\fnote{So LXX DSS 11QPs\textsuperscript{a}; the Heb. lacks \fbib{them}} their food at the proper time. \\
\poeml \v{28}They receive what you give them; \\
\poemll    when you open your hand, \\
\poemlll       they are filled with good things. \\
\poeml \v{29}When you withdraw your favor,\fnote{Lit. \fbib{you conceal your face}} \\
\poemll    they are disappointed; \\
\poeml Take away their breath, \\
\poemll    and\fnote{So MT; LXX DSS 11QPs\textsuperscript{a} read \fbib{then}} they die\fnote{So MT DSS 11QPs\textsuperscript{a}; LXX reads \fbib{they will fail}} and return to dust. \\
\poeml \v{30}When you send your spirit,\fnote{Or \fbib{breath}} they are\fnote{So MT; LXX reads \fbib{they will be}; DSS 11QPs\textsuperscript{a} reads \fbib{then they are}} created, \\
\poemll    and you replenish the surface of the earth. \\
\poeml \v{31}May the glory of the \divine{Lord} last forever; \\
\poemll    may the \divine{Lord} rejoice in his works! \\
\poeml \v{32}He looks at the earth and it shakes; \\
\poemll    he touches the mountains and they smoke. \\
\poeml \v{33}I will sing to the \divine{Lord} with my whole being;\fnote{Lit. \fbib{with my life}} \\
\poemll    I will sing to my God continually! \\
\poeml \v{34}May my thoughts be pleasing to him; \\
\poemll    indeed, I will rejoice in the \divine{Lord}! \\
\poeml \v{35}May sinners disappear from the land \\
\poemll    and the wicked live no longer. \\
\poeml Bless the \divine{Lord}, my soul! Hallelujah!
\end{poetry}
\labelpsalm{105}
\passage{Thanksgiving for God's Deliverance}

\begin{poetry}
\poeml \v{1}Give thanks to the \divine{Lord}, \\
\poemll    call on his name, \\
\poemlll       and make his deeds known among the people. \\
\poeml \v{2}Sing to him! Praise him! \\
\poemll    Declare all his awesome deeds! \\
\poeml \v{3}Exult in his holy name; \\
\poemll    let all\fnote{Lit. \fbib{Let the heart of}} those who seek the \divine{Lord} rejoice! \\
\poeml \v{4}Seek the \divine{Lord} and his strength; \\
\poemll    seek his face continually. \\
\poeml \v{5}Remember his awesome deeds that he has done, \\
\poemll    his wonders and the judgments he declared. \\
\poeml \v{6}You descendants of Abraham, his servant, \\
\poemll    You children of Jacob, his chosen ones. \\
\poeml \v{7}He is the \divine{Lord} our God; \\
\poemll    his judgments extend to the entire earth. \\
\poeml \v{8}He remembers his eternal covenant--- \\
\poemll    every promise he made\fnote{Lit. \fbib{every word he commanded}} for a thousand generations, \\
\poeml \v{9}like the covenant he made\fnote{Lit. \fbib{like he cut}} with Abraham, \\
\poemll    and his promise to Isaac. \\
\poeml \v{10}He presented it to Jacob as a decree, \\
\poemll    to Israel as an everlasting covenant. \\
\poeml \v{11}He said: ``I will give Canaan to you \\
\poemll    as the allotted portion that is your inheritance.'' \\
\poeml \v{12}When the Hebrews\fnote{Lit. \fbib{When they}} were few in number---so very few--- \\
\poemll    and were sojourners in it, \\
\poeml \v{13}they wandered from nation to nation, \\
\poemll    from one kingdom to another.\fnote{Lit. \fbib{one kingdom to another nation}} \\
\poeml \v{14}He did not allow anyone to oppress them, \\
\poemll    or any kings to reprove them. \\
\poeml \v{15}``Don't touch my anointed \\
\poemll    or hurt my prophets!'' \\
\poeml \v{16}He declared a famine on the land; \\
\poemll    destroying the entire food supply.\fnote{Lit. \fbib{every staff of bread}} \\
\poeml \v{17}He sent a man before them--- \\
\poemll    Joseph, who had been sold as a slave. \\
\poeml \v{18}They bound his feet with fetters \\
\poemll    and placed an iron collar on his neck,\fnote{Lit. \fbib{soul}} \\
\poeml \v{19}until the time his prediction came true, \\
\poemll    as the word of the \divine{Lord} refined him. \\
\poeml \v{20}He sent a king who released him, \\
\poemll    a ruler of people who set him free. \\
\poeml \v{21}He made him the master over his household, \\
\poemll    the manager of all his possessions--- \\
\poeml \v{22}to discipline his rulers at will \\
\poemll    and make his elders wise. \\
\poeml \v{23}Then Israel came to Egypt; \\
\poemll    indeed, Jacob lived in the land of Ham.\fnote{I.e. Egypt} \\
\poeml \v{24}He caused his people to multiply greatly; \\
\poemll    and be more numerous than their enemies. \\
\poeml \v{25}He caused them\fnote{Lit. \fbib{He turned their hearts}} to hate his people \\
\poemll    and to deceive his servants. \\
\poeml \v{26}He sent his servant Moses, along with Aaron, \\
\poemll    whom he had chosen. \\
\poeml \v{27}They performed his signs among them, \\
\poemll    his wonders in the land of Ham.\fnote{I.e. Egypt} \\
\poeml \v{28}He sent darkness, and it became dark. \\
\poemll    Did they not rebel against\fnote{So MT DSS 11QPs\textsuperscript{a}; LXX reads \fbib{not embitter}} his words? \\
\poeml \v{29}He turned their water into blood, \\
\poemll    so that the fish died. \\
\poeml \v{30}Their land swarmed with frogs \\
\poemll    even to the chambers of their kings. \\
\poeml \v{31}He spoke, \\
\poemll    and a swarm of insects invaded their land.\fnote{Or \fbib{borders}} \\
\poeml \v{32}He sent hail instead of rain, \\
\poemll    and lightning throughout their land. \\
\poeml \v{33}It destroyed their vines and their figs, \\
\poemll    breaking trees throughout their country.\fnote{Or \fbib{borders}} \\
\poeml \v{34}Then he commanded the locust to come--- \\
\poemll    grasshoppers without number. \\
\poeml \v{35}They consumed every green plant in their land, \\
\poemll    and devoured the fruit of their soil. \\
\poeml \v{36}He struck down every firstborn in their land, \\
\poemll    the first fruits of all their progeny. \\
\poeml \v{37}Then he brought Israel\fnote{Lit. \fbib{them}} out with silver and gold, \\
\poemll    and no one among his tribes stumbled. \\
\poeml \v{38}The Egyptians rejoiced when they left, \\
\poemll    because fear of Israel\fnote{Lit. \fbib{them}} descended on them. \\
\poeml \v{39}He spread out a cloud for a cover, \\
\poemll    and fire for light at night. \\
\poeml \v{40}Israel\fnote{The Heb. lacks \fbib{Israel}} asked, and quail came; \\
\poemll    food from heaven satisfied them. \\
\poeml \v{41}He opened a rock, and water gushed out \\
\poemll    flowing like a river in the desert. \\
\poeml \v{42}Indeed, he remembered his sacred promise \\
\poemll    to his servant Abraham. \\
\poeml \v{43}He led his people out with gladness, \\
\poemll    his chosen ones with shouts of joy. \\
\poeml \v{44}He gave to them the land of nations; \\
\poemll    they inherited the labor of other\fnote{The Heb. lacks \fbib{other}} people \\
\poeml \v{45}so they might keep his statutes \\
\poemll    and observe his laws. \\
\poemlll       Hallelujah!
\end{poetry}
\labelpsalm{106}
\passage{The Unfaithfulness of God's People}

\begin{poetry}
\poeml \v{1}Hallelujah! \\
\poeml Give thanks to the \divine{Lord}, \\
\poemll    since he is good, \\
\poemlll       for his gracious love exists forever. \\
\poeml \v{2}Who can fully describe the mighty acts of the \divine{Lord} \\
\poemll    or proclaim all his praises? \\
\poeml \v{3}How happy are those who enforce justice, \\
\poemll    who live righteously all the time. \\
\poeml \v{4}Remember me, \divine{Lord}, \\
\poemll    when you show favor to your people. \\
\poeml Visit us with your deliverance, \\
\poeml \v{5}to witness the prosperity of your chosen ones, \\
\poeml to rejoice in your nation's joy, \\
\poemll    to glory in your inheritance. \\
\poeml \v{6}We have sinned, along with our ancestors; \\
\poemll    we have committed iniquity and wickedness. \\
\poeml \v{7}In Egypt, our ancestors neither comprehended your awesome deeds \\
\poemll    nor remembered your abundant gracious love. \\
\poemlll       Instead, they rebelled beside the sea, the Reed\fnote{So MT; LXX reads \fbib{Red}} Sea. \\
\poeml \v{8}He delivered for the sake of his name,\fnote{Or \fbib{reputation}} \\
\poemll    to make his power known. \\
\poeml \v{9}He shouted at the Reed\fnote{So MT; LXX reads \fbib{Red}} Sea and it dried up; \\
\poemll    and led them through the sea as though through a desert. \\
\poeml \v{10}He delivered them from the power of their foe; \\
\poemll    redeeming them from the power of their enemy. \\
\poeml \v{11}The water overwhelmed their enemies, \\
\poemll    so that not one of them survived.\fnote{Or \fbib{remained}} \\
\poeml \v{12}Then they believed his word \\
\poemll    and sung his praise. \\
\poeml \v{13}But they quickly forgot his deeds \\
\poemll    and did not wait for his counsel. \\
\poeml \v{14}They were overwhelmed with craving in the wilderness, \\
\poemll    so God tested them in the wasteland. \\
\poeml \v{15}God granted them their request, \\
\poemll    but sent leanness into their lives. \\
\poeml \v{16}They were envious of Moses in the camp, \\
\poemll    and of Aaron, the holy one of the \divine{Lord}. \\
\poeml \v{17}The earth opened and swallowed Dathan, \\
\poemll    closing over Abiram's clan. \\
\poeml \v{18}Then a fire burned among their company, \\
\poemll    a flame that set the wicked ablaze. \\
\poeml \v{19}They fashioned a calf at Horeb \\
\poemll    and worshipped a carved image. \\
\poeml \v{20}They exchanged their glory\fnote{I.e. their God} \\
\poemll    with the image of a grass-eating bull. \\
\poeml \v{21}They forgot God their Savior, \\
\poemll    who performed great things in Egypt--- \\
\poeml \v{22}awesome deeds in the land of Ham,\fnote{I.e. Egypt} \\
\poemll    astonishing deeds at the Reed\fnote{So MT; LXX reads \fbib{Red}} Sea. \\
\poeml \v{23}He would have destroyed them \\
\poemll    but for Moses, his chosen one, \\
\poeml who stood in the breach before him \\
\poemll    to avert\fnote{Or \fbib{turn back}} his destructive wrath. \\
\poeml \v{24}They rejected the desirable land, \\
\poemll    and they didn't trust his promise. \\
\poeml \v{25}They murmured in their tents, \\
\poemll    and didn't listen to the voice of the \divine{Lord}. \\
\poeml \v{26}So he swore an oath concerning them--- \\
\poemll    that he would cause them to die in the wilderness, \\
\poeml \v{27}to cause their children to perish among the nations \\
\poemll    and be scattered among many\fnote{The Heb. lacks \fbib{many}} lands. \\
\poeml \v{28}For they adopted the worship\fnote{Lit. \fbib{they attached themselves with Baal Peor}} of Baal Peor \\
\poemll    and ate sacrifices offered to the dead. \\
\poeml \v{29}They had provoked anger by their deeds, \\
\poemll    so that a plague broke out against them. \\
\poeml \v{30}But Phinehas intervened and prayed \\
\poemll    so that the plague was restrained. \\
\poeml \v{31}And it was credited to him as a righteous act, \\
\poemll    from generation to generation---to eternity. \\
\poeml \v{32}They provoked wrath at the waters of Meribah, \\
\poemll    and Moses suffered\fnote{Lit. \fbib{and it was evil for Moses}} on account of them. \\
\poeml \v{33}For they rebelled against him,\fnote{Lit. \fbib{against his spirit}} \\
\poemll    so that he spoke thoughtlessly with his lips. \\
\poeml \v{34}They never destroyed the people, \\
\poemll    as the \divine{Lord} had commanded them. \\
\poeml \v{35}Instead, they mingled among the nations \\
\poemll    and learned their ways.\fnote{Lit. \fbib{deeds}} \\
\poeml \v{36}They worshipped\fnote{Lit. \fbib{served}} their idols, \\
\poemll    and this became a trap for them. \\
\poeml \v{37}They sacrificed their sons and daughters to demons. \\
\poeml \v{38}They shed innocent blood--- \\
\poemll    the blood of their sons and daughters--- \\
\poeml whom they sacrificed to the idols of Canaan, \\
\poemll    thereby polluting the land with blood. \\
\poeml \v{39}Therefore, they became unclean because of what they did; \\
\poemll    they have acted like whores by their evil deeds. \\
\poeml \v{40}The \divine{Lord}'s anger burned against his people, \\
\poemll    so that he despised his own inheritance. \\
\poeml \v{41}He turned them over to domination by nations \\
\poemll    where those who hated them ruled over them. \\
\poeml \v{42}Their enemies oppressed them, \\
\poemll    so that they were humiliated by their power. \\
\poeml \v{43}He delivered them many times, \\
\poemll    but they demonstrated rebellion by their evil plans; \\
\poemlll       therefore they sunk deep in their sins. \\
\poeml \v{44}Yet when he saw their distress \\
\poemll    and heard their cries for help,\fnote{The Heb. lacks \fbib{help}} \\
\poeml \v{45}he remembered his covenant with them, \\
\poemll    and so relented \\
\poemlll       according to the greatness of his gracious love. \\
\poeml \v{46}He caused all their captors to show compassion toward them. \\
\poeml \v{47}Deliver us, \divine{Lord} our God, \\
\poemll    gather us from among the nations \\
\poeml so we may praise your holy name \\
\poemll    and rejoice in praising you. \\
\poeml \v{48}Blessed are you, \divine{Lord} God of Israel, \\
\poemll    from eternity to eternity; \\
\poeml Let all the people say, ``Amen!'' \\
\poemll    Hallelujah!
\end{poetry}
\booksection{\divine{BOOK V} (Psalms 107-150)}
\labelpsalm{107}
\passage{Gratitude for God's Deliverance}

\begin{poetry}
\poeml \v{1}Give thanks to the \divine{Lord}, for he is good! \\
\poemll    His gracious love exists forever. \\
\poeml \v{2}Let those who have been redeemed by the \divine{Lord} declare it--- \\
\poemll    those whom he redeemed \\
\poemlll       from the power\fnote{Lit. \fbib{hand}} of the enemy, \\
\poeml \v{3}those whom he gathered from other lands--- \\
\poemll    from the east, west, north, and south.\fnote{Lit. \fbib{and the sea}; i.e. the Reed Sea} \\
\poeml \v{4}They wandered in desolate wilderness; \\
\poemll    they found no road to a city where they could live. \\
\poeml \v{5}Hungry and thirsty, \\
\poemll    their spirits\fnote{Lit. \fbib{soul}} failed. \\
\poeml \v{6}Then they cried out to the \divine{Lord} in their trouble, \\
\poemll    and he delivered them from their distress. \\
\poeml \v{7}He led them in a straight way \\
\poemll    to find a city where they could live. \\
\poeml \v{8}Let them give thanks to the \divine{Lord} \\
\poemll    for his gracious love \\
\poemlll       and his awesome deeds for mankind. \\
\poeml \v{9}He has satisfied the one who thirsts, \\
\poemll    filling the hungry with what is good. \\
\poeml \v{10}Some sat in deepest darkness, \\
\poemll    shackled with cruel iron, \\
\poeml \v{11}because they had rebelled against the command of God, \\
\poemll    despising the advice of the Most High. \\
\poeml \v{12}He humbled them\fnote{Lit. \fbib{humbled their hearts}} through suffering, \\
\poemll    as they stumbled without a helper. \\
\poeml \v{13}Then they cried out to the \divine{Lord} in their trouble; \\
\poemll    he delivered them from their distress. \\
\poeml \v{14}And he\fnote{So LXX DSS 4QPs\textsuperscript{f}; MT reads \fbib{He}} brought them out from darkness and the shadow of death,\fnote{So LXX; MT reads \fbib{and gloom}} \\
\poemll    shattering their chains. \\
\poeml \v{15}Let them give\fnote{So MT LXX; DSS 4QPs\textsuperscript{f} reads \fbib{Give}} thanks to the \divine{Lord} for his gracious love, \\
\poemll    and for his awesome deeds to mankind. \\
\poeml \v{16}For he shattered bronze gates \\
\poemll    and cut through iron bars. \\
\poeml \v{17}Because of their rebellious ways, \\
\poemll    fools suffered for their iniquities. \\
\poeml \v{18}They\fnote{Lit. \fbib{their souls}} loathed all food, \\
\poemll    and even reached the gates of death. \\
\poeml \v{19}Yet when they cried out to the \divine{Lord} in their trouble, \\
\poemll    he delivered them from certain destruction. \\
\poeml \v{20}He issued his command\fnote{Lit. \fbib{word}} and healed them; \\
\poemll    he delivered them from their destruction. \\
\poeml \v{21}Let them give thanks to the \divine{Lord} for his gracious love, \\
\poemll    and for his awesome deeds for mankind. \\
\poeml \v{22}Let them offer sacrifices of thanksgiving \\
\poemll    and talk about his works with shouts of joy. \\
\poeml \v{23}Those who go down to the sea in ships, \\
\poemll    who work in the great waters, \\
\poeml \v{24}witnessed the works of the \divine{Lord}--- \\
\poemll    his awesome deeds in the ocean's depth. \\
\poeml \v{25}He spoke and stirred up a windstorm \\
\poemll    that made its waves surge. \\
\poeml \v{26}The people\fnote{Lit. \fbib{They}} ascended skyward and descended to the depths, \\
\poemll    their courage\fnote{Lit. \fbib{souls}} melting away in their peril. \\
\poeml \v{27}They reeled and staggered like a drunkard, \\
\poemll    as all their wisdom became useless. \\
\poeml \v{28}Yet when they cried out to the \divine{Lord} in their trouble, \\
\poemll    the \divine{Lord} brought them out of their distress. \\
\poeml \v{29}He calmed the storm \\
\poemll    and its waves\fnote{So MT LXX; DSS 4QPs\textsuperscript{f} reads \fbib{and the waves of the sea}; cf. Psa 107:25} quieted down. \\
\poeml \v{30}So they rejoiced that the waves\fnote{Lit. \fbib{they}} became quiet, \\
\poemll    and he led them to their desired haven. \\
\poeml \v{31}Let them give thanks to the \divine{Lord} for his gracious love \\
\poemll    and for his awesome deeds on behalf of mankind. \\
\poeml \v{32}Let them exalt him in the assembly of the people \\
\poemll    and praise him in the counsel of the elders. \\
\poeml \v{33}He turns rivers into a desert, \\
\poemll    springs of water into dry ground, \\
\poeml \v{34}and a fruitful land into a salty waste, \\
\poemll    due to the wickedness of its inhabitants. \\
\poeml \v{35}He turns a desert into a pool of water, \\
\poemll    dry land into springs of water. \\
\poeml \v{36}There he settled the hungry, \\
\poemll    where they built a city to live in. \\
\poeml \v{37}They sowed fields and planted vineyards \\
\poemll    that yielded a productive harvest. \\
\poeml \v{38}Then he blessed them, and they became numerous; \\
\poemll    he multiplied their cattle.\fnote{Or \fbib{he didn't let their cattle become few}} \\
\poeml \v{39}But they became few in number, and humiliated \\
\poemll    by continued oppression, agony, and sorrow. \\
\poeml \v{40}Having poured contempt on their nobles, \\
\poemll    causing them to err aimlessly in the way. \\
\poeml \v{41}Yet he lifted the needy from affliction \\
\poemll    and made them families like a flock. \\
\poeml \v{42}The upright see it and rejoice, \\
\poemll    but the mouth of an evil person is shut. \\
\poeml \v{43}Let whoever is wise observe these things, \\
\poemll    that they may comprehend the gracious love of the \divine{Lord}.
\end{poetry}
\labelpsalm{108}
\psalminfo{A song. A Davidic psalm.}
\passage{A Plea for Victory}

\begin{poetry}
\poeml \v{1}My heart is firm, God; \\
\poemll    I will sing and praise you with my whole being. \\
\poeml \v{2}Awake, harp and lyre! \\
\poemll    I will wake up at dawn. \\
\poeml \v{3}I will give thanks to you among the peoples, \divine{Lord}! \\
\poemll    I will sing praise to you among the nations. \\
\poeml \v{4}For your gracious love extends to the sky,\fnote{Or is \fbib{great above the heavens}} \\
\poemll    and your faithfulness reaches to the clouds. \\
\poeml \v{5}May you be exalted above the heavens, God, \\
\poemll    and your glory be over all the earth. \\
\poeml \v{6}In order that those you love may be rescued, \\
\poemll    deliver with your power\fnote{Lit. \fbib{right hand}} and answer me! \\
\poeml \v{7}God had promised in his sanctuary: \\
\poeml ``I will triumph and divide Shechem, \\
\poemll    then I will measure the valley of Succoth! \\
\poeml \v{8}Gilead and Manasseh belong to me, \\
\poemll    while Ephraim is my chief stronghold \\
\poemlll       and Judah is my scepter. \\
\poeml \v{9}Moab is my washbasin; \\
\poemll    I will fling my shoe on Edom \\
\poemlll       and shout over Philistia.'' \\
\poeml \v{10}Who will lead me to the fortified city? \\
\poemll    Who will lead me as far as Edom? \\
\poeml \v{11}God, you have rejected us, have you not, \\
\poemll    since you did not march out with our army, God? \\
\poeml \v{12}Give us help against the enemy, \\
\poemll    because human help is useless.\fnote{Or \fbib{vain}} \\
\poeml \v{13}I will find strength in God, \\
\poemll    for he will trample on our foes.
\end{poetry}
\labelpsalm{109}
\psalminfo{To the Director. A Davidic psalm.}
\passage{A Prayer against the Evil One}

\begin{poetry}
\poeml \v{1}God, whom I praise, \\
\poemll    do not be silent, \\
\poeml \v{2}for the mouths of wicked and deceitful people \\
\poemll    are opened against me; \\
\poemlll       they speak against me with lying tongues. \\
\poeml \v{3}They surround me with hate-filled words, \\
\poemll    attacking me for no reason. \\
\poeml \v{4}Instead of receiving\fnote{The Heb. lacks \fbib{receiving}} my love, they accuse me, \\
\poemll    though I continue in prayer. \\
\poeml \v{5}They devise evil against me instead of good, \\
\poemll    and hatred in place of my love. \\
\poeml \v{6}Appoint an evil person over him; \\
\poemll    may an accuser stand at his right side.\fnote{Lit. \fbib{hand}} \\
\poeml \v{7}When he is judged, may he be found guilty; \\
\poemll    may his prayer be regarded as sin. \\
\poeml \v{8}May his days be few; \\
\poemll    may another take over his position.\fnote{Or \fbib{office}} \\
\poeml \v{9}May his children become fatherless, \\
\poemll    and his wife a widow. \\
\poeml \v{10}May his children roam around begging, \\
\poemll    seeking food\fnote{The Heb. lacks \fbib{food}} while driven far\fnote{So LXX; the Heb. lacks \fbib{while driven far}} from their ruined homes. \\
\poeml \v{11}May creditors seize all his possessions, \\
\poemll    and may foreigners loot the property he has acquired.\fnote{Or \fbib{the result of his labor}} \\
\poeml \v{12}May no one extend gracious love to him, \\
\poemll    or show favor to his fatherless children. \\
\poeml \v{13}May his descendants\fnote{Lit. \fbib{May those after him}} be eliminated, \\
\poemll    and their memory\fnote{Or \fbib{their name}} be erased from the next generation. \\
\poeml \v{14}May his ancestors' guilt be remembered in the \divine{Lord}'s presence, \\
\poemll    and may his mother's guilt not be erased. \\
\poeml \v{15}May what\fnote{The Heb. lacks \fbib{what}} they have done\fnote{The Heb. lacks \fbib{have done}} be continually in the \divine{Lord}'s presence; \\
\poemll    and may their memory be excised from the earth. \\
\poeml \v{16}For he didn't think to extend gracious love; \\
\poemll    he harassed to death the poor, the needy, and the broken hearted.\fnote{Or \fbib{downhearted}} \\
\poeml \v{17}He loved to curse---may his curses\fnote{Lit. \fbib{may it}} return upon him! \\
\poemll    He took no delight in blessing others\fnote{The Heb. lacks \fbib{others}}--- \\
\poemlll       so may blessings\fnote{Lit. \fbib{it}} be far from him. \\
\poeml \v{18}He wore curses like a garment--- \\
\poemll    may they\fnote{Lit. \fbib{it}} enter his inner being like water \\
\poemlll       and his bones like oil. \\
\poeml \v{19}May those curses\fnote{Lit. \fbib{may it}} wrap around him like a garment, \\
\poemll    or like a belt that one always wears. \\
\poeml \v{20}May this be the way the \divine{Lord} repays my accuser, \\
\poemll    those who speak evil against me. \\
\poeml \v{21}Now you, \divine{Lord} my God, defend\fnote{Lit. \fbib{God, do to}} me for your name's sake; \\
\poemll    because your gracious love is good, deliver me! \\
\poeml \v{22}Indeed, I am poor and needy, \\
\poemll    and my heart is wounded within me. \\
\poeml \v{23}I am fading\fnote{Or \fbib{walking}} away like a shadow late in the day; \\
\poemll    I am shaken off like a locust. \\
\poeml \v{24}My knees give way\fnote{Or \fbib{knees stagger}} from fasting, \\
\poemll    and my skin is lean, deprived of oil. \\
\poeml \v{25}I have become an object of derision to them--- \\
\poemll    they shake their heads whenever they see me. \\
\poeml \v{26}Help me, \divine{Lord} my God! \\
\poemll    Deliver me in accord with your gracious love! \\
\poeml \v{27}Then they will realize that your hand is in this--- \\
\poemll    that you, \divine{Lord}, have accomplished it. \\
\poeml \v{28}They will curse, \\
\poemll    but you will bless. \\
\poeml When they attack,\fnote{Lit. \fbib{arise}} they will\fnote{So MT DSS 4QPs\textsuperscript{f} 11QPs\textsuperscript{a}; LXX reads \fbib{arise, let my opponents}} be humiliated, \\
\poemll    while your servant rejoices. \\
\poeml \v{29}May my accusers be clothed with shame \\
\poemll    and wrapped in their humiliation as with a robe. \\
\poeml \v{30}I will give many thanks to the \divine{Lord} with my mouth, \\
\poemll    praising him publicly, \\
\poeml \v{31}for he stands\fnote{So MT; LXX DSS 11QPs\textsuperscript{a} read \fbib{he has stood}} at the right hand of the needy one, \\
\poemll    to deliver him from his accusers.\fnote{Or \fbib{from those who condemn him}}
\end{poetry}
\labelpsalm{110}
\psalminfo{A Davidic psalm}
\passage{A Priestly Ruler}

\begin{poetry}
\poeml \v{1}A declaration from the \divine{Lord}\fnote{So MT; LXX reads \fbib{The \divine{Lord} said}} to my Lord:
\end{poetry}

\begin{poetry}
\poemll    ``Sit at my right hand \\
\poemlll       until I make your enemies your footstool.'' \\
\poeml \v{2}When the \divine{Lord} extends your mighty scepter from Zion, \\
\poemll    rule in the midst of your enemies. \\
\poeml \v{3}Your soldiers\fnote{Lit. \fbib{people}} are willing volunteers on your day of battle; \\
\poemll    in majestic holiness, from the womb, \\
\poemlll       from the dawn, the dew of your youth belongs to you.
\end{poetry}

\begin{poetry}
\poeml \v{4}The \divine{Lord} took an oath and will never recant: \\
\poemll    ``You are a priest forever, \\
\poemlll       after the manner of Melchizedek.'' \\
\poeml \v{5}The Lord is at your right hand; \\
\poemll    he will utterly destroy kings in the time of his wrath. \\
\poeml \v{6}He will execute judgment against the nations, \\
\poemll    filling graves\fnote{The Heb. lacks \fbib{graves}} with corpses. \\
\poemlll       He will utterly destroy leaders far and wide. \\
\poeml \v{7}He will drink from a stream on the way, \\
\poemll    then hold his head high.
\end{poetry}
\labelpsalm{111}
\passage{Praise for God's Amazing Deeds\fnote{T In this acrostic psalm each line begins with a consecutive letter of the Heb. alphabet.}}

\begin{poetry}
\poeml \v{1}Hallelujah!
\end{poetry}

\begin{poetry}
\poeml I will give thanks to the \divine{Lord} with all of my heart \\
\poemll    in the assembled congregation of the upright. \\
\poeml \v{2}Great are the acts of the \divine{Lord}; \\
\poemll    they are within reach of\fnote{Or \fbib{are sought by}} all who desire them. \\
\poeml \v{3}Splendid and glorious are his awesome deeds, \\
\poemll    and his righteousness endures forever. \\
\poeml \v{4}He is remembered for his awesome deeds; \\
\poemll    the \divine{Lord} is gracious and compassionate. \\
\poeml \v{5}He prepares food\fnote{Or \fbib{prey}} for those who fear him; \\
\poemll    he is ever mindful of his covenant. \\
\poeml \v{6}He revealed his mighty deeds to his people \\
\poemll    by giving them a country of their own.\fnote{Lit. \fbib{an inheritance of nations}} \\
\poeml \v{7}Whatever he does is\fnote{Lit. \fbib{The works of his hands are}} reliable and just, \\
\poemll    and all his precepts are trustworthy, \\
\poeml \v{8}sustained through all eternity, \\
\poemll    and fashioned in both truth and righteousness. \\
\poeml \v{9}He sent deliverance to his people; \\
\poemll    he ordained his covenant to last forever; \\
\poemlll       his name is holy and awesome. \\
\poeml \v{10}The fear of the \divine{Lord} is the beginning of wisdom; \\
\poemll    sound understanding belongs to those who practice it. \\
\poeml Praise of God\fnote{Lit. \fbib{him}} endures forever.
\end{poetry}
\labelpsalm{112}
\passage{The Gracious Person\fnote{T In this acrostic psalm each line begins with a consecutive letter of the Heb. alphabet.}}

\begin{poetry}
\poeml \v{1}Hallelujah!
\end{poetry}

\begin{poetry}
\poeml How happy is the person who fears the \divine{Lord}, \\
\poemll    who truly delights in his commandments. \\
\poeml \v{2}His descendants will be powerful in the land, \\
\poemll    a generation of the upright who will be blessed. \\
\poeml \v{3}Wealth and riches are in his house, \\
\poemll    and his righteousness endures forever. \\
\poeml \v{4}A light shines in the darkness for the upright, \\
\poemll    for the one who is gracious, compassionate, and just. \\
\poeml \v{5}It is good for the person who lends generously, \\
\poemll    conducting his affairs with fairness. \\
\poeml \v{6}He will never be shaken; \\
\poemll    the one who is just will always be remembered. \\
\poeml \v{7}He need not fear a bad report, \\
\poemll    for his heart is unshaken, since he trusts in the \divine{Lord}. \\
\poeml \v{8}His heart is steadfast, he will not fear. \\
\poemll    In the end he will look in triumph over his enemy. \\
\poeml \v{9}He gives generously to the poor; \\
\poemll    his righteousness endures forever; \\
\poemlll       his horn is exalted in honor. \\
\poeml \v{10}The wicked person sees this and flies into a rage; \\
\poemll    his teeth gnash and wear away. \\
\poeml The desire of the wicked will amount to nothing.
\end{poetry}
\labelpsalm{113}
\passage{Praise to the Loving God}

\begin{poetry}
\poeml \v{1}Hallelujah!
\end{poetry}

\begin{poetry}
\poeml Give praise, you servants of the \divine{Lord}. \\
\poemll    Praise the name of the \divine{Lord}! \\
\poeml \v{2}May the name of the \divine{Lord} be blessed \\
\poemll    from now to eternity. \\
\poeml \v{3}From rising\fnote{Lit. \fbib{from the eastern}} to setting\fnote{Lit. \fbib{to the western}} sun, \\
\poemll    may the name of the \divine{Lord} be praised. \\
\poeml \v{4}The \divine{Lord} is exalted high above all the nations; \\
\poemll    his glory beyond the heavens. \\
\poeml \v{5}Who is like the \divine{Lord} our God, \\
\poemll    enthroned on high, \\
\poeml \v{6}yet stooping low to observe \\
\poemll    the sky and the earth? \\
\poeml \v{7}He lifts the poor person from the dust, \\
\poemll    raising the needy from the trash pile \\
\poeml \v{8}and giving him a seat among nobles--- \\
\poemll    with the nobles of his people. \\
\poeml \v{9}He makes the barren woman among her household \\
\poemll    a happy mother of joyful children. \\
\poeml Hallelujah!
\end{poetry}
\labelpsalm{114}
\passage{Deliverance of Israel from Egypt}

\begin{poetry}
\poeml \v{1}When Israel came out of Egypt--- \\
\poemll    the household of Jacob from a people of foreign speech--- \\
\poeml \v{2}Judah became his sanctuary \\
\poemll    and Israel his place of dominion. \\
\poeml \v{3}The sea saw this\fnote{The Heb. lacks \fbib{this}} and fled, \\
\poemll    the Jordan River\fnote{The Heb. lacks \fbib{River}} ran backwards, \\
\poeml \v{4}the mountains skipped like rams, \\
\poemll    and the hills like lambs. \\
\poeml \v{5}What happened to you, sea, that you fled? \\
\poemll    Jordan, that you ran backwards? \\
\poeml \v{6}Mountains, that you skipped like rams? \\
\poemll    and you hills, that you skipped\fnote{The Heb. lacks \fbib{that you skipped}} like lambs? \\
\poeml \v{7}Tremble then, earth, at the presence of the Lord, \\
\poemll    at the presence of the God of Jacob, \\
\poeml \v{8}who turned the rock into a pool of water, \\
\poemll    the flinty rock into flowing springs.
\end{poetry}
\labelpsalm{115}
\passage{The Impotence of Idols}

\begin{poetry}
\poeml \v{1}Not to us, \divine{Lord}, not to us, \\
\poemll    but to your name be given glory \\
\poemll    on account of your gracious love and faithfulness. \\
\poeml \v{2}Why should the nations ask \\
\poemll    ``Where now is their God?'' \\
\poeml \v{3}when our God is in the heavens \\
\poemll    and he does whatever he desires? \\
\poeml \v{4}Their idols are silver and gold, \\
\poemll    crafted by human hands. \\
\poeml \v{5}They have mouths, but cannot speak; \\
\poemll    they have eyes, but cannot see. \\
\poeml \v{6}They have ears, but cannot hear; \\
\poemll    they have noses, but cannot smell. \\
\poeml \v{7}They have hands, but cannot touch; \\
\poemll    feet, but cannot walk; \\
\poemlll       they cannot even groan with their throats. \\
\poeml \v{8}Those who craft them will become like them, \\
\poemll    as will all those who trust in them. \\
\poeml \v{9}Israel, trust in the \divine{Lord}! \\
\poemll    He is their helper and shield. \\
\poeml \v{10}House of Aaron, trust in the \divine{Lord}! \\
\poemll    He is their helper and shield. \\
\poeml \v{11}You who fear the \divine{Lord}, trust in the \divine{Lord}! \\
\poemll    He is their helper and shield. \\
\poeml \v{12}The \divine{Lord} remembers and blesses us. \\
\poemll    He will indeed bless the house of Israel; \\
\poemlll       he will bless the house of Aaron. \\
\poeml \v{13}He will bless those who fear the \divine{Lord}, \\
\poemll    both the important and the insignificant together. \\
\poeml \v{14}May the \divine{Lord} add to your numbers--- \\
\poemll    to you and to your descendants. \\
\poeml \v{15}May you be blessed by the \divine{Lord}, \\
\poemll    who made the heavens and the earth. \\
\poeml \v{16}The highest heavens\fnote{Lit. \fbib{heaven of heaven}} belong to the \divine{Lord}, \\
\poemll    but he gave the earth to human beings. \\
\poeml \v{17}Neither can the dead praise the \divine{Lord}, \\
\poemll    nor those who go down into the silence of death.\fnote{The Heb. lacks \fbib{of death}} \\
\poeml \v{18}But we will bless the \divine{Lord} \\
\poemll    from now to eternity. \\
\poeml Hallelujah!
\end{poetry}
\labelpsalm{116}
\passage{God, My Deliverer}

\begin{poetry}
\poeml \v{1}I love the \divine{Lord} \\
\poemll    because he has heard my prayer for mercy;\fnote{Lit. \fbib{the voice of my supplication}} \\
\poeml \v{2}for he listens to me whenever I call. \\
\poeml \v{3}The ropes of death were wound around me \\
\poemll    and the anguish of Sheol\fnote{I.e. the realm of the dead} came upon me; \\
\poemlll       I encountered distress and sorrow. \\
\poeml \v{4}Then I called on the name of the \divine{Lord}, \\
\poemll    ``\divine{Lord}, please deliver me!''\fnote{Lit. \fbib{deliver my soul}} \\
\poeml \v{5}The \divine{Lord} is gracious and righteous; \\
\poemll    our God is compassionate; \\
\poeml \v{6}the \divine{Lord} watches over the innocent;\fnote{Or \fbib{naive}} \\
\poemll    I was brought low, and he delivered me. \\
\poeml \v{7}Return to your resting place, my soul, \\
\poemll    for the \divine{Lord} treated you generously. \\
\poeml \v{8}Indeed, you delivered my soul from death, \\
\poemll    my eyes from crying,\fnote{Lit. \fbib{tears}} \\
\poemlll       and my feet from stumbling. \\
\poeml \v{9}I will walk in the \divine{Lord}'s presence \\
\poemll    in the lands of the living. \\
\poeml \v{10}I will continue to believe, even when I say, \\
\poemll    ``I am greatly afflicted'' \\
\poeml \v{11}and speak hastily, \\
\poemll    ``All people are liars!'' \\
\poeml \v{12}What will I return to the \divine{Lord} \\
\poemll    for all his benefits to me? \\
\poeml \v{13}I will raise my cup of deliverance \\
\poemll    and invoke the \divine{Lord}'s name. \\
\poeml \v{14}I will fulfill my vows to the \divine{Lord} \\
\poemll    in the presence of all his people. \\
\poeml \v{15}In the sight of the \divine{Lord}, \\
\poemll    the death of his faithful ones is valued. \\
\poeml \v{16}\divine{Lord}, I am indeed your servant. \\
\poemll    I am your servant. \\
\poeml I am the son of your handmaid. \\
\poemll    You have released my bonds. \\
\poeml \v{17}I will bring you a thanksgiving offering \\
\poemll    and call on the name of the \divine{Lord}! \\
\poeml \v{18}I will fulfill my vows to the \divine{Lord} \\
\poemll    in the presence of all his people, \\
\poeml \v{19}in the courts of the \divine{Lord}'s house, \\
\poemll    in your midst, Jerusalem. \\
\poeml Hallelujah!
\end{poetry}
\labelpsalm{117}
\passage{A Call to Praise the \divine{Lord}}

\begin{poetry}
\poeml \v{1}Praise the \divine{Lord}, all you nations! \\
\poemll    Exalt him, all you peoples! \\
\poeml \v{2}For great is his gracious love toward us, \\
\poemll    and the \divine{Lord}'s faithfulness is eternal. \\
\poeml Hallelujah!
\end{poetry}
\labelpsalm{118}
\passage{Thanksgiving to God}

\begin{poetry}
\poeml \v{1}Give thanks to the \divine{Lord}, \\
\poemll    for he is good; \\
\poemlll       his gracious love is eternal. \\
\poeml \v{2}Let Israel now say, \\
\poemll    ``His gracious love is eternal.'' \\
\poeml \v{3}Let the house of Aaron now say, \\
\poemll    ``His gracious love is eternal.'' \\
\poeml \v{4}Let those who fear the \divine{Lord} now say, \\
\poemll    ``His gracious love is eternal.'' \\
\poeml \v{5}I called on the \divine{Lord} in my distress; \\
\poemll    the \divine{Lord} answered me openly.\fnote{Lit. \fbib{in a wide open place}} \\
\poeml \v{6}The \divine{Lord} is with me. \\
\poemll    I will not be afraid. \\
\poemlll       What can people do to me? \\
\poeml \v{7}With the \divine{Lord} beside me as my helper, \\
\poemll    I will triumph over those who hate me. \\
\poeml \v{8}It is better to take shelter\fnote{Or \fbib{refuge}; LXX DSS 4QPs\textsuperscript{b} 11QPs\textsuperscript{a} read \fbib{to trust}} in the \divine{Lord} \\
\poemll    than to trust in people. \\
\poeml \v{9}It is better to take shelter\fnote{Or \fbib{refuge}} in the \divine{Lord} \\
\poemll    than to trust in princes. \\
\poeml \v{10}All the nations surrounded me; \\
\poemll    but in the name of the \divine{Lord} I will defeat them. \\
\poeml \v{11}They surrounded me, they are around me; \\
\poemll    but in the name of the \divine{Lord} I will defeat them. \\
\poeml \v{12}They surrounded me like bees; \\
\poemll    but they will be extinguished like\fnote{So MT DSS 4QPs\textsuperscript{b}; LXX reads \fbib{bees; they blazed like a fire among}} burning thorns. \\
\poemlll       In the name of the \divine{Lord} I will defeat them. \\
\poeml \v{13}Indeed, you\fnote{I.e. the enemy} oppressed me so much that I nearly fell, \\
\poemll    but the \divine{Lord} helped me. \\
\poeml \v{14}The \divine{Lord} is my strength and protector,\fnote{Or \fbib{might}} \\
\poemll    for he has become my deliverer.\fnote{Or \fbib{salvation}} \\
\poeml \v{15}There's exultation\fnote{Lit. \fbib{sound of}} for deliverance in the tents of the righteous:
\end{poetry}

\begin{poetry}
\poeml ``The right hand of the \divine{Lord} is victorious!\fnote{Lit. \fbib{\divine{Lord} acted valiantly}} \\
\poeml \v{16}The right hand of the \divine{Lord} is exalted! \\
\poemlll       The right hand of the \divine{Lord} is victorious!''\fnote{MT reads \fbib{\divine{Lord} acted valiantly}; LXX DSS 11QPs\textsuperscript{a} read \fbib{\divine{Lord} acted powerfully}} \\
\poeml \v{17}I will not die, but I will live \\
\poemll    to recount the deeds of the \divine{Lord}. \\
\poeml \v{18}The \divine{Lord} will discipline me severely, \\
\poemll    but he won't hand me over to die. \\
\poeml \v{19}Open for me the righteous gates \\
\poemll    so I may enter through them to give thanks to the \divine{Lord}. \\
\poeml \v{20}This is the \divine{Lord}'s gate--- \\
\poemll    The righteous will enter through it. \\
\poeml \v{21}I will praise you because you have answered me \\
\poemll    and have become my deliverer. \\
\poeml \v{22}The stone that the builders rejected \\
\poemll    has become the cornerstone. \\
\poeml \v{23}This is from the \divine{Lord}--- \\
\poemll    it is awesome in our sight. \\
\poeml \v{24}This is the day that the \divine{Lord} has made; \\
\poemll    let's rejoice and be glad in it. \\
\poeml \v{25}Please \divine{Lord}, deliver us! \\
\poemll    Please \divine{Lord}, hurry\fnote{Or \fbib{rush}} and bring success now! \\
\poeml \v{26}Blessed is the one who comes in the name of the \divine{Lord}! \\
\poemll    Let us bless you from the \divine{Lord}'s house. \\
\poeml \v{27}The \divine{Lord} is God---he will be our light! \\
\poemll    Bind the festival sacrifice with ropes \\
\poemlll       to the horn at the altar. \\
\poeml \v{28}You are my God, and I will praise you; \\
\poemll    my God, and I will exalt you. \\
\poeml \v{29}Give thanks to the \divine{Lord}, for he is good \\
\poemll    and his gracious love is eternal.
\end{poetry}
\labelpsalm{119}
\psalminfo{Alef\fnote{T This Psalm is an acrostic in which all verses in each eight-verse section begin with the letter of the Heb. alphabet indicated.}}
\passage{Living in the Law of God}

\begin{poetry}
\poeml \v{1}How blessed are those whose life\fnote{Lit. \fbib{way}} is blameless, \\
\poemll    who walk in the Law of the \divine{Lord}! \\
\poeml \v{2}How blessed are those who observe his decrees, \\
\poemll    who seek him with all of their heart, \\
\poeml \v{3}who practice no evil \\
\poemll    while they walk in his ways. \\
\poeml \v{4}You have commanded concerning your precepts, \\
\poemll    that they be guarded with diligence. \\
\poeml \v{5}Oh, that my ways were steadfast, \\
\poemll    so I may keep your statutes. \\
\poeml \v{6}Then I will not be ashamed, \\
\poemll    since my eyes will be fixed on all of your commands. \\
\poeml \v{7}I will praise you with an upright heart, \\
\poemll    as I learn your righteous decrees. \\
\poeml \v{8}I will keep your statutes; \\
\poemll    do not ever abandon me.
\end{poetry}
\psalminfo{Bet}
\passage{The Benefits of the Word}

\begin{poetry}
\poeml \v{9}How can a young man keep his behavior pure? \\
\poemll    By guarding it in accordance with your word. \\
\poeml \v{10}I have sought you with all of my heart; \\
\poemll    do not let me drift away from your commands. \\
\poeml \v{11}I have stored what you have said\fnote{So MT DSS 4QPs\textsuperscript{h}; LXX Syr read \fbib{stored your oracles}} in my heart, \\
\poemll    so I won't sin against you. \\
\poeml \v{12}Blessed are you, \divine{Lord}! \\
\poemll    Teach me your statutes. \\
\poeml \v{13}I have spoken with my lips \\
\poemll    about all your decrees that you have announced.\fnote{Lit. \fbib{decrees of your mouth}} \\
\poeml \v{14}I find joy in the path of your decrees, \\
\poemll    as if I owned all kinds of riches. \\
\poeml \v{15}I will meditate on your precepts, \\
\poemll    and I will respect your ways. \\
\poeml \v{16}I am delighted with your statutes; \\
\poemll    I will not forget your word.\fnote{So MT; LXX Syr DSS 11QPs\textsuperscript{a} read \fbib{words}}
\end{poetry}
\psalminfo{Gimmel}
\passage{Living and Keeping God's Word}

\begin{poetry}
\poeml \v{17}Deal kindly with your servant \\
\poemll    so I may live and keep your word.\fnote{So MT; LXX DSS 11QPs\textsuperscript{a} read \fbib{words}} \\
\poeml \v{18}Open my eyes \\
\poemll    so that I will observe amazing things from your instruction.\fnote{Or \fbib{Law}} \\
\poeml \v{19}Since I am a stranger on the earth, \\
\poemll    do not hide your commands from me. \\
\poeml \v{20}My soul is consumed with longing \\
\poemll    for your decrees at all times. \\
\poeml \v{21}You rebuke the accursed ones, \\
\poemll    who wander from your commands. \\
\poeml \v{22}Remove scorn and disrespect from me, \\
\poemll    for I observe your decrees. \\
\poeml \v{23}Though nobles take their seat and gossip about me, \\
\poemll    your servant will meditate on your statutes. \\
\poeml \v{24}I take joy in your decrees, \\
\poemll    for they are my counselors.
\end{poetry}
\psalminfo{Daleth}
\passage{Strength Comes from the Word}

\begin{poetry}
\poeml \v{25}My soul clings to the dust; \\
\poemll    revive me according to your word. \\
\poeml \v{26}I have talked about my ways, \\
\poemll    and you have answered me; \\
\poemlll       Teach me your statutes. \\
\poeml \v{27}Help me understand how your precepts function,\fnote{Lit. \fbib{understand the ways of your precepts}} \\
\poemll    and I will meditate on your wondrous acts. \\
\poeml \v{28}I weep because of sorrow; \\
\poemll    fortify me according to your word. \\
\poeml \v{29}Remove false paths from me; \\
\poemll    and graciously give me your instruction.\fnote{Or \fbib{Law}} \\
\poeml \v{30}I have chosen the faithful way; \\
\poemll    I have firmly placed your ordinances before me.\fnote{The Heb. lacks \fbib{before me}} \\
\poeml \v{31}I cling to your decrees; \\
\poemll    \divine{Lord}, do not put me to shame. \\
\poeml \v{32}I eagerly race along the way of your commands, \\
\poemll    for you enable me to do so.\fnote{Lit. \fbib{will enlarge my heart}}
\end{poetry}
\psalminfo{He}
\passage{Instructed by the Word}

\begin{poetry}
\poeml \v{33}Teach me, \divine{Lord}, about the way of your statutes, \\
\poemll    and I will observe them without fail.\fnote{Or \fbib{them to the end}} \\
\poeml \v{34}Give me understanding \\
\poemll    and I will observe your instruction.\fnote{Or \fbib{Law}} \\
\poemlll       I will keep it with all of my heart. \\
\poeml \v{35}Help me live my life by your commands, \\
\poemll    because my joy is in them. \\
\poeml \v{36}Turn my heart to your decrees \\
\poemll    and away from unjust gain. \\
\poeml \v{37}Turn my eyes away from gazing at worthless things, \\
\poemll    and revive me by your ways. \\
\poeml \v{38}Confirm your promise to your servant, \\
\poemll    which is for those who fear you. \\
\poeml \v{39}Turn away the shame that I dread, \\
\poemll    because your ordinances are good. \\
\poeml \v{40}Look, I long for your precepts; \\
\poemll    revive me through your righteousness.
\end{poetry}
\psalminfo{Vav}
\passage{A Song of Praise}

\begin{poetry}
\poeml \v{41}May your gracious love come to me, \divine{Lord}, \\
\poemll    your salvation, just as you said. \\
\poeml \v{42}Then I can answer the one who insults me, \\
\poemll    for I place my trust in your word. \\
\poeml \v{43}Never take your truthful words from me, \\
\poemll    For I wait for\fnote{Or \fbib{place my hope in}} your ordinances. \\
\poeml \v{44}Then I will always keep your Law, \\
\poemll    forever and ever, \\
\poeml \v{45}I will walk in liberty, \\
\poemll    for I seek your precepts. \\
\poeml \v{46}Then I will speak of your decrees before kings \\
\poemll    and not be ashamed. \\
\poeml \v{47}I will take delight in your commands, \\
\poemll    which I love. \\
\poeml \v{48}I will lift up my hands to your commands, \\
\poemll    which I love, \\
\poemlll       and I will meditate on your statutes.
\end{poetry}
\psalminfo{Zayin}
\passage{Remembering What God Has Said}

\begin{poetry}
\poeml \v{49}Remember what you said\fnote{Lit. \fbib{Remember the word}} to your servant, \\
\poemll    by which you caused me to hope. \\
\poeml \v{50}This is what comforts me in my troubles: \\
\poemll    that what you say revives me. \\
\poeml \v{51}Even though the arrogant utterly deride me, \\
\poemll    I do not turn away from your instruction.\fnote{Or \fbib{Law}} \\
\poeml \v{52}I have remembered your ancient ordinances, \divine{Lord}, \\
\poemll    and I take comfort in them. \\
\poeml \v{53}I burn with indignation because of the wicked \\
\poemll    who forsake your instruction.\fnote{Or \fbib{Law}} \\
\poeml \v{54}Your statutes are my songs, \\
\poemll    no matter where I make my home.\fnote{Lit. \fbib{songs in the house of my sojourn}} \\
\poeml \v{55}In the night I remember your name, \divine{Lord}, \\
\poemll    and keep your instruction.\fnote{Or \fbib{Law}} \\
\poeml \v{56}I have made it my personal responsibility \\
\poemll    to keep your precepts.
\end{poetry}
\psalminfo{Cheth}
\passage{Keeping God's Word}

\begin{poetry}
\poeml \v{57}The \divine{Lord} is my inheritance; \\
\poemll    I have given my promise to keep your word. \\
\poeml \v{58}I have sought your favor with all of my heart; \\
\poemll    be gracious to me according to your promise. \\
\poeml \v{59}I examined my lifestyle \\
\poemll    and set my feet in the direction of your decrees. \\
\poeml \v{60}I hurried and did not procrastinate \\
\poemll    to keep your commands. \\
\poeml \v{61}Though the ropes of the wicked have ensnared me, \\
\poemll    I have not forgotten your instruction.\fnote{Or \fbib{Law}} \\
\poeml \v{62}At midnight I will get up to thank you \\
\poemll    for your righteous ordinances. \\
\poeml \v{63}I am united with all who fear you, \\
\poemll    and with everyone who keeps your precepts. \\
\poeml \v{64}\divine{Lord}, the earth overflows with your gracious love! \\
\poemll    Teach me your statutes.
\end{poetry}
\psalminfo{Teth}
\passage{Praise for God's Word}

\begin{poetry}
\poeml \v{65}\divine{Lord}, you have dealt well with your servant, \\
\poemll    according to your word. \\
\poeml \v{66}Teach me both knowledge and appropriate discretion, \\
\poemll    because I believe in your commands. \\
\poeml \v{67}Before I was humbled, I wandered away, \\
\poemll    but now I observe your words. \\
\poeml \v{68}\divine{Lord},\fnote{So LXX Syr DSS 11QPs\textsuperscript{a}; the Heb. lacks \fbib{\divine{Lord}}} you are good\fnote{So MT; LXX reads \fbib{kind}}, and do what is good; \\
\poemll    teach me your statutes. \\
\poeml \v{69}The arrogant have accused me falsely; \\
\poemll    but I will observe your precepts wholeheartedly. \\
\poeml \v{70}Their minds are clogged as with greasy fat, \\
\poemll    but I find joy in your instruction.\fnote{Or \fbib{Law}} \\
\poeml \v{71}It was for my good that I was humbled;\fnote{So MT; LXX reads \fbib{that you humbled me}; DSS 11QPs\textsuperscript{a} reads \fbib{that you afflicted me}} \\
\poemll    so that I would learn your statutes. \\
\poeml \v{72}Instruction\fnote{Or \fbib{Law}} that comes from you\fnote{Lit. \fbib{from your mouth}} is better for me \\
\poemll    than thousands of gold and silver coins.\fnote{Lit. \fbib{pieces}}
\end{poetry}
\psalminfo{Yod}
\passage{Prayer for God's Grace}

\begin{poetry}
\poeml \v{73}Your hands made and formed me; \\
\poemll    give me understanding, \\
\poemlll       that I may learn your commands. \\
\poeml \v{74}May those who fear you see me and be glad, \\
\poemll    for I have hoped in your word. \\
\poeml \v{75}I know, \divine{Lord}, that your decrees are just, \\
\poemll    and that you have rightfully humbled me. \\
\poeml \v{76}May your gracious love comfort me \\
\poemll    in accordance with your promise to your servant. \\
\poeml \v{77}May your mercies come to me that I may live, \\
\poemll    for your instruction\fnote{Or \fbib{Law}} is my delight. \\
\poeml \v{78}May the arrogant become ashamed, \\
\poemll    because they have subverted me with deceit; \\
\poemlll       as for me, I will meditate on your precepts. \\
\poeml \v{79}May those who fear you turn to me, \\
\poemll    along with those who know your decrees. \\
\poeml \v{80}May my heart be blameless with respect to your statutes \\
\poemll    so that I may not become ashamed.
\end{poetry}
\psalminfo{Kaf}
\passage{On Obeying God's Word}

\begin{poetry}
\poeml \v{81}I long for your deliverance; \\
\poemll    I have looked to your word, \\
\poemlll       placing my hope in it. \\
\poeml \v{82}My eyes grow weary \\
\poemll    with respect to what you have promised--- \\
\poemlll       I keep asking, ``When will you comfort me?'' \\
\poeml \v{83}Though I have become like a water skin dried by\fnote{The Heb. lacks \fbib{dried by}} smoke, \\
\poemll    I have not forgotten your statutes. \\
\poeml \v{84}How many days must your servant endure this?\fnote{The Heb. lacks \fbib{this}} \\
\poemll    When will you judge those who persecute me? \\
\poeml \v{85}The arrogant have dug pitfalls for me \\
\poemll    disobeying your instruction.\fnote{Or \fbib{Law}} \\
\poeml \v{86}All of your commands are reliable. \\
\poemll    I am persecuted without cause---help me! \\
\poeml \v{87}Though the arrogant\fnote{Lit. \fbib{they}} nearly destroyed me on earth, \\
\poemll    I did not abandon your precepts. \\
\poeml \v{88}Revive me according to your gracious love; \\
\poemll    and I will keep the decrees that you have proclaimed.
\end{poetry}
\psalminfo{Lamed}
\passage{Pay Attention to God's Word}

\begin{poetry}
\poeml \v{89}Your word is forever, \divine{Lord}; \\
\poemll    it is firmly established in heaven. \\
\poeml \v{90}Your faithfulness continues from generation to generation. \\
\poemll    You established the earth, and it stands firm. \\
\poeml \v{91}To this day they stand by means of your rulings, \\
\poemll    for all things serve you. \\
\poeml \v{92}Had your instruction\fnote{Or \fbib{Law}} not been my pleasure, \\
\poemll    I would have died in my affliction. \\
\poeml \v{93}I will never forget your precepts, \\
\poemll    for you have revived me with them. \\
\poeml \v{94}I am yours, so save me, \\
\poemll    since I have sought your precepts. \\
\poeml \v{95}The wicked lay in wait to destroy me, \\
\poemll    while I ponder your decrees. \\
\poeml \v{96}I have observed that all things have their limit, \\
\poemll    but your commandment is very broad.
\end{poetry}
\psalminfo{Mem}
\passage{Loving God's Word}

\begin{poetry}
\poeml \v{97}How I love your instruction!\fnote{Or \fbib{Law}} \\
\poemll    Every day it is my meditation. \\
\poeml \v{98}Your commands make me wiser than my adversaries, \\
\poemll    since they are always with me. \\
\poeml \v{99}I am more insightful than my teachers, \\
\poemll    because your decrees are my meditations. \\
\poeml \v{100}I have more common sense than the elders, \\
\poemll    for I observe your precepts. \\
\poeml \v{101}I keep away from every evil choice\fnote{Lit. \fbib{way}} \\
\poemll    so that I may keep your word.\fnote{So MT DSS 5QPs; LXX reads \fbib{words}} \\
\poeml \v{102}I do not avoid your judgments, \\
\poemll    for you pointed them out to me. \\
\poeml \v{103}How pleasing is what you have to say to me--- \\
\poemll    tasting better than honey. \\
\poeml \v{104}I obtain understanding from your precepts; \\
\poemll    therefore I hate every false way.
\end{poetry}
\psalminfo{Nun}
\passage{God's Word a Light}

\begin{poetry}
\poeml \v{105}Your word is\fnote{So MT and LXX; the DSS 11QPs\textsuperscript{a} reads \fbib{words are}} a lamp for my feet, \\
\poemll    a light for my pathway. \\
\poeml \v{106}I have given my word and affirmed it, \\
\poemll    to keep your righteous judgments. \\
\poeml \v{107}I am severely afflicted. \\
\poemll    Revive me, \divine{Lord}, according to your word. \\
\poeml \v{108}\divine{Lord}, please accept my voluntary offerings of praise,\fnote{Lit. \fbib{of my mouth}} \\
\poemll    and teach me your judgments. \\
\poeml \v{109}Though I constantly take my life in my hands, \\
\poemll    I do not forget your instruction.\fnote{Or \fbib{Law}} \\
\poeml \v{110}Though the wicked lay a trap for me, \\
\poemll    I haven't wandered away from your precepts. \\
\poeml \v{111}I have inherited your decrees forever, \\
\poemll    because they are the joy of my heart. \\
\poeml \v{112}As a result, I am determined \\
\poemll    to carry out your statutes forever.
\end{poetry}
\psalminfo{Samek}
\passage{Loving God's Law}

\begin{poetry}
\poeml \v{113}I despise the double-minded, \\
\poemll    but I love your instruction.\fnote{Or \fbib{Law}} \\
\poeml \v{114}You are my fortress and shield; \\
\poemll    I hope in your word. \\
\poeml \v{115}Leave me, you who practice evil, \\
\poemll    that I may observe the commands of my God. \\
\poeml \v{116}Sustain me, God,\fnote{The Heb. lacks \fbib{God}} as you have promised, \\
\poemll    and I will live. \\
\poemlll       Do not let me be ashamed of my hope. \\
\poeml \v{117}Support me, that I may be saved, \\
\poemll    and I will carry out your statutes consistently. \\
\poeml \v{118}You reject all who wander from your statutes, \\
\poemll    since their deceitfulness is vain. \\
\poeml \v{119}You remove\fnote{So MT; Hieronymus Aquila Symmachus read \fbib{You consider}; LXX reads \fbib{I considered}; DSS 11QPs\textsuperscript{a} reads \fbib{I consider}} all the wicked of the earth like\fnote{The Heb. lacks \fbib{like}} dross; \\
\poemll    therefore I love your decrees. \\
\poeml \v{120}My flesh trembles out of fear of you, \\
\poemll    and I am in awe of\fnote{Or \fbib{I fear}} your judgments.
\end{poetry}
\psalminfo{Ayin}
\passage{Praying for God's Deliverance}

\begin{poetry}
\poeml \v{121}I have acted with justice and righteousness; \\
\poemll    do not abandon me to my oppressors. \\
\poeml \v{122}Back up your servant in a positive way; \\
\poemll    do not let the arrogant oppress me. \\
\poeml \v{123}My eyes fail as I look\fnote{The Heb. lacks \fbib{as I look}} for your salvation \\
\poemll    and for your righteous promise. \\
\poeml \v{124}Act toward your servant consistent with your gracious love, \\
\poemll    and teach me your statutes. \\
\poeml \v{125}Since I am your servant, give me understanding, \\
\poemll    so I will know your decrees. \\
\poeml \v{126}It is time for the \divine{Lord} to act, \\
\poemll    since they have violated your instruction.\fnote{Or \fbib{Law}} \\
\poeml \v{127}I truly love your commands more than gold, \\
\poemll    including fine gold. \\
\poeml \v{128}I truly consider all of your precepts---all of them---to be just, \\
\poemll    while I despise every false way.
\end{poetry}
\psalminfo{Peyh}
\passage{Living in God's Word}

\begin{poetry}
\poeml \v{129}Your decrees are wonderful--- \\
\poemll    that's why I observe them. \\
\poeml \v{130}The disclosure of your words illuminates, \\
\poemll    providing understanding to the simple. \\
\poeml \v{131}I open my mouth and pant \\
\poemll    as I long for your commands. \\
\poeml \v{132}Turn in my direction and show mercy to me, \\
\poemll    as you have decreed regarding those who love your name. \\
\poeml \v{133}Direct my footsteps by your promise, \\
\poemll    and do not let any kind of iniquity rule over me. \\
\poeml \v{134}Deliver me from human oppression \\
\poemll    and I will keep your precepts. \\
\poeml \v{135}Show favor to\fnote{Lit. \fbib{Make your face shine on}} your servant, \\
\poemll    and teach me your statutes. \\
\poeml \v{136}My eyes shed rivers of tears, \\
\poemll    when others do not obey your instruction.\fnote{Or \fbib{Law}}
\end{poetry}
\psalminfo{Tsade}
\passage{God's Righteous Decrees}

\begin{poetry}
\poeml \v{137}\divine{Lord}, you are righteous, \\
\poemll    and your judgments are right. \\
\poeml \v{138}You have ordered your decrees to us rightly, \\
\poemll    and they are very faithful. \\
\poeml \v{139}My zeal consumes me \\
\poemll    because my enemies forget your words. \\
\poeml \v{140}Your word is very pure, \\
\poemll    and your servant loves it. \\
\poeml \v{141}Though I may be small and despised, \\
\poemll    I do not neglect your precepts. \\
\poeml \v{142}Your righteousness is an eternal righteousness, \\
\poemll    and your instruction\fnote{Or \fbib{Law}} is true. \\
\poeml \v{143}Though trouble and anguish overwhelm me, \\
\poemll    your commands remain my delight. \\
\poeml \v{144}Your righteous decrees are eternal; \\
\poemll    give me understanding, and I will live.
\end{poetry}
\psalminfo{Qof}
\passage{Waiting in Hope}

\begin{poetry}
\poeml \v{145}I have cried out with all of my heart. \\
\poemll    Answer me, \divine{Lord}! \\
\poemlll       I will observe your statutes. \\
\poeml \v{146}I have called out to you, ``Save me, \\
\poemll    so I may keep your decrees.'' \\
\poeml \v{147}I get up before dawn and cry for help; \\
\poemll    I place my hope in your word. \\
\poeml \v{148}I look forward to the night watches, \\
\poemll    when I may meditate on what you have said. \\
\poeml \v{149}Hear my voice according to your gracious love. \\
\poemll    \divine{Lord}, revive me in keeping with your justice. \\
\poeml \v{150}Those who pursue wickedness draw near; \\
\poemll    they remain far from your instruction.\fnote{Or \fbib{Law}} \\
\poeml \v{151}You are near, \divine{Lord}, \\
\poemll    and all of your commands are true. \\
\poeml \v{152}I discovered long ago about your decrees \\
\poemll    that you have confirmed them forever.
\end{poetry}
\psalminfo{Resh}
\passage{God's Word is Truth}

\begin{poetry}
\poeml \v{153}Look on my misery, and rescue me, \\
\poemll    for I do not ignore your instruction.\fnote{Or \fbib{Law}} \\
\poeml \v{154}Defend my case and redeem me; \\
\poemll    revive me according to your promise. \\
\poeml \v{155}Deliverance remains remote from the wicked, \\
\poemll    for they do not seek your statutes. \\
\poeml \v{156}Your mercies are magnificent, \divine{Lord}; \\
\poemll    revive me according to your judgments. \\
\poeml \v{157}Though my persecutors and adversaries are numerous, \\
\poemll    I do not turn aside from your decrees. \\
\poeml \v{158}I watch the treacherous, and despise them, \\
\poemll    because they do not do what you have said. \\
\poeml \v{159}Look how I love your precepts, \divine{Lord}; \\
\poemll    revive me according to your gracious love. \\
\poeml \v{160}The sum\fnote{So MT; LXX reads \fbib{beginning}} of your word\fnote{So MT; LXX Hieronymous DSS 11QPs\textsuperscript{a} read \fbib{words}} is truth, \\
\poemll    and each righteous ordinance of yours is everlasting.
\end{poetry}
\psalminfo{Sin/Shin}
\passage{Loving God's Instruction}

\begin{poetry}
\poeml \v{161}Though nobles persecute me for no reason, \\
\poemll    my heart stands in awe of your words. \\
\poeml \v{162}I find joy at what you have said \\
\poemll    like one who has discovered a great treasure. \\
\poeml \v{163}I despise and hate falsehood, \\
\poemll    but\fnote{So LXX Syr DSS 11QPs\textsuperscript{a}; the Heb. lacks \fbib{but}} I love your instruction.\fnote{Or \fbib{Law}} \\
\poeml \v{164}I praise you seven times a day \\
\poemll    because of your righteous ordinances. \\
\poeml \v{165}Great peace belongs to those who love your instruction,\fnote{Or \fbib{Law}} \\
\poemll    and nothing makes them stumble. \\
\poeml \v{166}I am looking in hope for your deliverance, \divine{Lord}, \\
\poemll    as I carry out your commands. \\
\poeml \v{167}My soul treasures\fnote{Lit. \fbib{guards}} your decrees, \\
\poemll    and I love them deeply. \\
\poeml \v{168}I keep your precepts and your decrees \\
\poemll    because all of my ways are before you.
\end{poetry}
\psalminfo{Tav}
\passage{The Joy of God's Word}

\begin{poetry}
\poeml \v{169}May my cry arise before you, \divine{Lord}; \\
\poemll    give me understanding according to your word. \\
\poeml \v{170}Let my request come before you; \\
\poemll    deliver me, as you have promised. \\
\poeml \v{171}May my lips utter praise, \\
\poemll    for you teach me your statutes. \\
\poeml \v{172}May my tongue sing about your promise, \\
\poemll    for all of your commands are right. \\
\poeml \v{173}May your hand stand ready to assist me, \\
\poemll    for I have chosen your precepts. \\
\poeml \v{174}I am longing for your deliverance, \divine{Lord}, \\
\poemll    and your instruction\fnote{Or \fbib{Law}} is my joy. \\
\poeml \v{175}Let me live, and I will praise you; \\
\poemll    let your ordinances\fnote{So LXX Targ DSS 11QPs\textsuperscript{a} (original); MT reads \fbib{ordinance}} help me. \\
\poeml \v{176}I have wandered away like a lost sheep; \\
\poemll    come find your servant, \\
\poemlll       for I do not forget your commands.
\end{poetry}
\labelpsalm{120}
\psalminfo{A Song of Ascents\fnote{T Or \fbib{Degrees}; and so through Psalm 134}}
\passage{A Prayer for Deliverance}

\begin{poetry}
\poeml \v{1}I cried to the \divine{Lord} in my distress, \\
\poemll    and he responded to me. \\
\poeml \v{2}``\divine{Lord}, deliver me\fnote{Lit. \fbib{my soul}} from lips that lie \\
\poemll    and tongues that deceive.'' \\
\poeml \v{3}What will be given to you, \\
\poemll    and what will be done to you, \\
\poemlll       you treacherous tongue? \\
\poeml \v{4}Like a\fnote{The Heb. lacks \fbib{Like a}} sharp arrow from a warrior, \\
\poemll    along with fiery coals from juniper trees! \\
\poeml \v{5}How terrible for me, \\
\poemll    that I am an alien in Meshech, \\
\poemlll       that I reside among the tents of Kedar! \\
\poeml \v{6}I have resided too long \\
\poemll    with those who hate peace. \\
\poeml \v{7}I am in favor of peace; \\
\poemll    but when I speak, \\
\poemlll       they are in favor of war.
\end{poetry}
\labelpsalm{121}
\psalminfo{A Song of Ascents}
\passage{The Guardian of God's People}

\begin{poetry}
\poeml \v{1}I lift up my eyes toward the mountains--- \\
\poemll    from where will my help come? \\
\poeml \v{2}My help is from the \divine{Lord}, \\
\poemll    maker of heaven and earth. \\
\poeml \v{3}He will never let\fnote{So MT; LXX reads \fbib{Do not let}} your foot slip, \\
\poemll    nor\fnote{So LXX Syr Hieronymous DSS 11QPs\textsuperscript{a}; the Heb. lacks \fbib{nor}} will\fnote{So MT; LXX reads \fbib{nor let}} your guardian become drowsy. \\
\poeml \v{4}Look! The one who is guarding Israel \\
\poemll    never sleeps and does not take naps. \\
\poeml \v{5}The \divine{Lord} is your guardian; \\
\poemll    the \divine{Lord} is your shade at your right side. \\
\poeml \v{6}The sun will not ravage you by day, \\
\poemll    nor the moon by night. \\
\poeml \v{7}The \divine{Lord} will guard you from all evil, \\
\poemll    preserving\fnote{Or \fbib{guarding}} your life. \\
\poeml \v{8}The \divine{Lord} will guard your goings and comings,\fnote{Cf. Deut 28:6} \\
\poemll    from this time on and forever.
\end{poetry}
\labelpsalm{122}
\psalminfo{A Davidic Song of Ascents}
\passage{Up to Jerusalem}

\begin{poetry}
\poeml \v{1}I rejoiced when they kept on asking me, \\
\poemll    ``Let us go to the \divine{Lord}'s Temple.'' \\
\poeml \v{2}Our feet are standing \\
\poemll    inside your gates, Jerusalem. \\
\poeml \v{3}Jerusalem stands built up, \\
\poemll    a city knitted together. \\
\poeml \v{4}To it the tribes ascend--- \\
\poemll    the tribes of the \divine{Lord}--- \\
\poeml as decreed to Israel, \\
\poemll    to give thanks to the name of the \divine{Lord}. \\
\poeml \v{5}For thrones are established there for judgment, \\
\poemll    thrones of the house of David. \\
\poeml \v{6}Pray for peace for Jerusalem: \\
\poemll    ``May those who love you be at peace!\fnote{Or \fbib{you prosper}} \\
\poeml \v{7}May peace be within your ramparts, \\
\poemll    and\fnote{So LXX Syr DSS 11QPs\textsuperscript{a}; the Heb. lacks \fbib{and}} prosperity\fnote{Or \fbib{peacefulness}; LXX reads \fbib{abundance}} within your fortresses.'' \\
\poeml \v{8}For the sake of my relatives and friends \\
\poemll    I will now say, ``May there be peace within you.'' \\
\poeml \v{9}For the sake of the Temple of the \divine{Lord} our God, \\
\poemll    I will seek your welfare.
\end{poetry}
\labelpsalm{123}
\psalminfo{A Song of Ascents}
\passage{A Prayer for Relief}

\begin{poetry}
\poeml \v{1}To you, who sit enthroned in heaven, \\
\poemll    I lift up my eyes. \\
\poeml \v{2}Consider this: as the eyes of a servant focus \\
\poemll    on what his master provides,\fnote{Lit. \fbib{on the hand of his master}} \\
\poeml and as the eyes of a female servant focus\fnote{The Heb. lacks \fbib{focus}} \\
\poemll    on what her mistress provides,\fnote{Lit. \fbib{on the hand of her mistress}} \\
\poeml so our eyes focus on the \divine{Lord} our God, \\
\poemll    until he has mercy on us. \\
\poeml \v{3}Have mercy on us, \divine{Lord}, have mercy, \\
\poemll    for we have had more than enough of contempt. \\
\poeml \v{4}Our lives overflow \\
\poemll    with scorn from those who live at ease, \\
\poemlll       with contempt from those who are proud.
\end{poetry}
\labelpsalm{124}
\psalminfo{A Davidic Song of Ascents}
\passage{God is for Us}

\begin{poetry}
\poeml \v{1}If the \divine{Lord} had not been on our side--- \\
\poemll    let Israel now say--- \\
\poeml \v{2}if the \divine{Lord} had not been on our side, \\
\poemll    when men came against us, \\
\poeml \v{3}then they would have devoured us alive, \\
\poemll    when their anger burned against us. \\
\poeml \v{4}Then the flood waters would have overwhelmed us, \\
\poemll    the torrent would have flooded over us; \\
\poeml \v{5}the swollen waters would have swept us away. \\
\poeml \v{6}Blessed be the \divine{Lord}, \\
\poemll    who did not give us as prey to their teeth. \\
\poeml \v{7}We have escaped like a bird from the hunter's trap. \\
\poemll    The trap has been broken, \\
\poemlll       and we have escaped. \\
\poeml \v{8}Our help is in the name of the \divine{Lord}, \\
\poemll    the maker of heaven and earth.
\end{poetry}
\labelpsalm{125}
\psalminfo{A Song of Ascents}
\passage{God is Secure}

\begin{poetry}
\poeml \v{1}Those who are trusting in the \divine{Lord} \\
\poemll    are like Mount Zion, which cannot be overthrown. \\
\poemlll       They remain forever. \\
\poeml \v{2}Just as mountains encircle Jerusalem, \\
\poemll    so the \divine{Lord} encircles his people, \\
\poemlll       from now to eternity. \\
\poeml \v{3}For evil's scepter will not rest \\
\poemll    on the land that has been allotted to the righteous, \\
\poeml and so the righteous will not direct themselves\fnote{Lit. \fbib{will not set their hands}} to do wrong. \\
\poeml \v{4}\divine{Lord}, do good to those who are good, \\
\poemll    and to those who are upright in heart.\fnote{So LXX DSS 4QPs\textsuperscript{e} 11QPs\textsuperscript{a}; MT reads \fbib{in their hearts}} \\
\poeml \v{5}But for those who choose their own devious paths, \\
\poemll    the \divine{Lord} will lead them away, \\
\poemlll       along with those who practice evil. \\
\poeml Peace be upon Israel.
\end{poetry}
\labelpsalm{126}
\psalminfo{A Song of Ascents}
\passage{The Exiles Restored}

\begin{poetry}
\poeml \v{1}When the \divine{Lord} brought back Zion's exiles,\fnote{Or \fbib{fortunes}} \\
\poemll    we were like dreamers.\fnote{So MT; LXX DSS11QPsa read \fbib{were restored}} \\
\poeml \v{2}Then our mouths were filled with laughter, \\
\poemll    and our tongues formed joyful shouts. \\
\poeml Then it was said among the nations, \\
\poemll    ``The \divine{Lord} has done great things for them.'' \\
\poeml \v{3}The great things that the \divine{Lord} has done for us \\
\poemll    gladden us. \\
\poeml \v{4}Restore our exiles,\fnote{Or \fbib{fortunes}} \divine{Lord}, \\
\poemll    like the streams of the Negev.\fnote{I.e. the southern regions of the Sinai peninsula; cf. Josh 10:40} \\
\poeml \v{5}Those who weep while they plant \\
\poemll    will sing for joy while they harvest. \\
\poeml \v{6}The one who goes out weeping,\fnote{So MT and DSS 11QPs\textsuperscript{a} (corrected); LXX DSS 11QPs\textsuperscript{a} (original) read \fbib{out and weeps}} \\
\poemll    carrying a bag of seeds, \\
\poeml will surely return with a joyful song, \\
\poemll    bearing sheaves from his harvest.\fnote{The Heb. lacks \fbib{harvest}}
\end{poetry}
\labelpsalm{127}
\psalminfo{A Solomonic Song of Ascents}
\passage{God's Blessing in the Family}

\begin{poetry}
\poeml \v{1}Unless the \divine{Lord} builds the house, \\
\poemll    its builders labor uselessly. \\
\poeml Unless the \divine{Lord} guards the city, \\
\poemll    its security forces keep watch uselessly. \\
\poeml \v{2}It is useless to get up early \\
\poemll    and to stay up late,\fnote{Lit. \fbib{delay sitting}} \\
\poeml eating the food of exhausting labor--- \\
\poemll    truly he gives sleep to those he loves. \\
\poeml \v{3}Children\fnote{Lit. \fbib{Sons}} are a gift\fnote{Lit. \fbib{heritage}} from the \divine{Lord}; \\
\poemll    a productive womb, the \divine{Lord}'s\fnote{The Heb. lacks \fbib{\divine{Lord}'s}} reward. \\
\poeml \v{4}As arrows in the hand of a warrior, \\
\poemll    so also are children\fnote{Lit. \fbib{sons}} born during one's\fnote{The Heb. lacks \fbib{born during one's}} youth. \\
\poeml \v{5}How blessed\fnote{Or \fbib{happy}} is the man whose quiver is full of them! \\
\poemll    He\fnote{Lit. \fbib{They}} will not be ashamed \\
\poemlll       as they confront their enemies at the city gate.
\end{poetry}
\labelpsalm{128}
\psalminfo{A Song of Ascents}
\passage{The Blessings of Fearing God}

\begin{poetry}
\poeml \v{1}How blessed\fnote{Or \fbib{happy}} are all who fear the \divine{Lord} \\
\poemll    as they follow in his ways. \\
\poeml \v{2}You will eat from the work of your hands; \\
\poemll    you will be happy, and it will go well for you. \\
\poeml \v{3}Your wife will be like a fruitful vine within your house; \\
\poemll    your children\fnote{Lit. \fbib{sons}} like olive shoots surrounding your table. \\
\poeml \v{4}See how the man will be blessed \\
\poemll    who fears the \divine{Lord}. \\
\poeml \v{5}May the \divine{Lord} bless you from Zion, \\
\poemll    and may you observe the prosperity of Jerusalem \\
\poemlll       every day that you live! \\
\poeml \v{6}And may you see your children's children! \\
\poemll    Peace be on Israel!
\end{poetry}
\labelpsalm{129}
\psalminfo{A Song of Ascents}
\passage{God Defeats Israel's Enemies}

\begin{poetry}
\poeml \v{1}``Since my youth they have often persecuted me,'' \\
\poemll    let Israel repeat it, \\
\poeml \v{2}``Since my youth they have often persecuted me, \\
\poemll    yet they haven't defeated me. \\
\poeml \v{3}Wicked people\fnote{So LXX DSS 11QPs\textsuperscript{a}; MT reads \fbib{The ploughman}} plowed over my back, \\
\poemll    creating long-lasting wounds.''\fnote{Or \fbib{long furrows}; LXX reads \fbib{back; they prolonged their lawlessness}} \\
\poeml \v{4}The \divine{Lord} is righteous--- \\
\poemll    he has cut me free from the cords of the wicked. \\
\poeml \v{5}Let all who hate Zion \\
\poemll    be turned away and be ashamed. \\
\poeml \v{6}May they become like a tuft of grass on a roof top, \\
\poemll    that withers before it takes root--- \\
\poeml \v{7}not enough to fill one's hand \\
\poemll    or to bundle in one's arms. \\
\poeml \v{8}And may those who pass by never tell them, \\
\poemll    ``May the \divine{Lord}'s blessing be upon you. \\
\poemlll       We bless you in the name of the \divine{Lord}.''
\end{poetry}
\labelpsalm{130}
\psalminfo{A Song of Ascents}
\passage{A Prayer for Mercy}

\begin{poetry}
\poeml \v{1}I cry to you from the depths, \divine{Lord}, \\
\poeml \v{2}Lord, listen to my voice; \\
\poeml let your ears pay attention \\
\poemll    to what I ask of you!\fnote{Lit. \fbib{to the voice of my supplications}} \\
\poeml \v{3}\divine{Lord},\fnote{Lit. \fbib{Yah}} if you were to record iniquities, \\
\poemll    Lord, who could remain standing? \\
\poeml \v{4}But with you there is forgiveness, \\
\poemll    so that you may be feared. \\
\poeml \v{5}I wait for the \divine{Lord}; \\
\poemll    my soul waits, \\
\poemlll       and I will hope in his word. \\
\poeml \v{6}My soul looks to the Lord \\
\poemll    more than watchmen look for the morning--- \\
\poemlll       more, indeed, than\fnote{The Heb. lacks \fbib{more indeed, than}} watchmen for the morning. \\
\poeml \v{7}Israel, hope in the \divine{Lord}! \\
\poemll    For with the \divine{Lord} there is gracious love, \\
\poemlll       along with abundant redemption. \\
\poeml \v{8}And he will redeem Israel \\
\poemll    from all its sins.
\end{poetry}
\labelpsalm{131}
\psalminfo{A Davidic Song of Ascents}
\passage{Hope in the \divine{Lord}}

\begin{poetry}
\poeml \v{1}\divine{Lord}, my heart is not arrogant, \\
\poemll    nor do I look haughty. \\
\poeml I do not aspire\fnote{Lit. \fbib{walk}} to great things, \\
\poemll    nor concern myself with things beyond my ability. \\
\poeml \v{2}Instead, I have composed and quieted myself \\
\poemll    like a weaned child with its mother; \\
\poemlll       I am like a weaned child. \\
\poeml \v{3}Place your hope in the \divine{Lord}, Israel, \\
\poemll    both now and forever.
\end{poetry}
\labelpsalm{132}
\psalminfo{A Song of Ascents}
\passage{The \divine{Lord} Lives in Zion}

\begin{poetry}
\poeml \v{1}\divine{Lord}, remember in David's favor \\
\poemll    all of his troubles; \\
\poeml \v{2}how he swore an oath to the \divine{Lord}, \\
\poemll    vowing to the Mighty One of Jacob, \\
\poeml \v{3}``I will not enter\fnote{Lit. \fbib{enter the tent that is}} my house, \\
\poemll    or lie down on\fnote{Lit. \fbib{on the couch that is}} my bed, \\
\poeml \v{4}or let myself go to sleep\fnote{Lit. \fbib{or give sleep to my eyes}} \\
\poemll    or even take a nap,\fnote{Lit. \fbib{or let my eyelids slumber}} \\
\poeml \v{5}until I locate a place for the \divine{Lord}, \\
\poemll    a dwelling place for the Mighty One of Jacob.'' \\
\poeml \v{6}We heard about it\fnote{I.e. the Ark of the Covenant} in Ephrata;\fnote{I.e. the region of Bethlehem} \\
\poemll    we found it in the fields of Jaar.\fnote{Cf. 1Sam 7:1-2; 1Chr 16:5-6} \\
\poeml \v{7}Let's go to his dwelling place \\
\poemll    and worship at his footstool. \\
\poeml \v{8}Arise, \divine{Lord}, \\
\poemll    and go to your resting place, \\
\poemlll       you and the ark of your strength. \\
\poeml \v{9}May your priests be clothed with righteousness \\
\poemll    and may your godly ones shout for joy. \\
\poeml \v{10}For the sake of your servant David, \\
\poemll    don't turn away the face of your anointed one. \\
\poeml \v{11}The \divine{Lord} made an oath to David \\
\poemll    from which he will not retreat: \\
\poeml ``One of your sons \\
\poemll    I will set in place on your throne. \\
\poeml \v{12}If your sons keep my covenant \\
\poemll    and my statutes that I will teach them, \\
\poemlll       then their sons will also sit on your throne forever.'' \\
\poeml \v{13}For the \divine{Lord} has chosen Zion, \\
\poemll    desiring it as his dwelling place. \\
\poeml \v{14}``This is my resting place forever. \\
\poemll    Here I will live, \\
\poemlll       because I desire to do so. \\
\poeml \v{15}I will bless its provisions abundantly; \\
\poemll    I will satiate its poor with food.\fnote{Lit. \fbib{bread}} \\
\poeml \v{16}I will clothe its priests with salvation \\
\poemll    and its godly ones will shout for joy. \\
\poeml \v{17}There I will create a power base\fnote{Lit. \fbib{will cause a horn to sprout}} for David--- \\
\poemll    I have prepared a lamp for my anointed one. \\
\poeml \v{18}I will clothe his enemies with disgrace, \\
\poemll    but on him his crown will shine.''
\end{poetry}
\labelpsalm{133}
\psalminfo{A Davidic Song of Ascents}
\passage{The Significance of Unity}

\begin{poetry}
\poeml \v{1}Look how good and how pleasant it is \\
\poemll    when brothers live together in unity! \\
\poeml \v{2}It is like precious oil on the head, \\
\poemll    descending to the beard--- \\
\poeml even to Aaron's beard--- \\
\poemll    and flowing down to the edge of his robes. \\
\poeml \v{3}It is like the dew of Hermon \\
\poemll    falling on Zion's mountains. \\
\poeml For there the \divine{Lord} commanded his blessing--- \\
\poemll    life everlasting.
\end{poetry}
\labelpsalm{134}
\psalminfo{A Song of Ascents}
\passage{Praise to the Creator}

\begin{poetry}
\poeml \v{1}Now bless the \divine{Lord}, \\
\poemll    all you servants of the \divine{Lord} \\
\poemlll       who serve\fnote{Lit. \fbib{stand}} nightly in the \divine{Lord}'s Temple. \\
\poeml \v{2}Lift up your hands to the Holy Place \\
\poemll    and bless the \divine{Lord}. \\
\poeml \v{3}May the \divine{Lord} who fashions heaven and earth \\
\poemll    bless you from Zion.
\end{poetry}
\labelpsalm{135}
\passage{Praising God for His Graciousness}

\begin{poetry}
\poeml \v{1}Hallelujah! \\
\poemll    Praise the name of the \divine{Lord}! \\
\poeml Give praise, you servants of the \divine{Lord}, \\
\poeml \v{2}you who are standing in the \divine{Lord}'s Temple, \\
\poemlll       in the courtyards of the house of our God. \\
\poeml \v{3}Praise the \divine{Lord}, \\
\poemll    because the \divine{Lord} is good; \\
\poeml Sing to his name, \\
\poemll    for he is gracious. \\
\poeml \v{4}It is Jacob whom the \divine{Lord} chose for himself--- \\
\poemll    Israel as his personal possession. \\
\poeml \v{5}Indeed, I know that the \divine{Lord} is great, \\
\poemll    and that our Lord\fnote{So MT LXX; DSS 11QPs\textsuperscript{a} reads \fbib{God}} surpasses all gods. \\
\poeml \v{6}The \divine{Lord} does whatever pleases him \\
\poemll    in heaven and on earth, \\
\poemlll       in the seas and all its\fnote{So DSS 11QPs\textsuperscript{a}; MT LXX lack \fbib{its}} deep regions. \\
\poeml \v{7}He makes the clouds rise from the ends of the earth, \\
\poemll    fashioning lightning for the rain, \\
\poemlll       bringing the wind from his storehouses. \\
\poeml \v{8}It was the \divine{Lord}\fnote{Lit. \fbib{was he}} who struck down the firstborn of Egypt, \\
\poemll    including both men and animals. \\
\poeml \v{9}He sent signs and wonders among you, Egypt, \\
\poemll    before\fnote{Or \fbib{among}} Pharaoh and all his servants. \\
\poeml \v{10}He struck down many nations, \\
\poemll    killing many kings--- \\
\poeml \v{11}Sihon, king of the Amorites, \\
\poemll    Og, king of Bashan, \\
\poemlll       and every kingdom of Canaan--- \\
\poeml \v{12}and he gave their land as an inheritance, \\
\poemll    an inheritance to his people Israel. \\
\poeml \v{13}Your name, \divine{Lord}, exists forever, \\
\poemll    and your reputation, \divine{Lord}, throughout the ages. \\
\poeml \v{14}For the \divine{Lord} will vindicate his people, \\
\poemll    and he will show compassion on his servants. \\
\poeml \v{15}The idols of the nations are silver and gold, \\
\poemll    worked by\fnote{So MT LXX; DSS 4QPs\textsuperscript{k} reads \fbib{gold, products of}} the hands of human beings. \\
\poeml \v{16}Mouths are attributed to them, \\
\poemll    but they cannot speak; \\
\poeml sight is attributed to them, \\
\poemll    but they cannot see; \\
\poeml \v{17}ears are attributed to them, \\
\poemll    but they do not hear, \\
\poemlll       and there is no breath in their mouths. \\
\poeml \v{18}Those who craft them--- \\
\poemll    and all\fnote{So LXX DSS 11QPs\textsuperscript{a}; the lacks \fbib{and}} who trust in them--- \\
\poemlll       will become like them. \\
\poeml \v{19}House of Israel, bless the \divine{Lord}! \\
\poemll    House of Aaron, bless the \divine{Lord}! \\
\poeml \v{20}House of Levi, bless the \divine{Lord}! \\
\poemll    You who fear the \divine{Lord}, bless the \divine{Lord}! \\
\poeml \v{21}Blessed be the \divine{Lord} from Zion, \\
\poemll    he who lives in Jerusalem. \\
\poeml Hallelujah!
\end{poetry}
\labelpsalm{136}
\passage{God's Gracious Love}

\begin{poetry}
\poeml \v{1}Give thanks to the \divine{Lord}, for he is good, \\
\poemll    for his gracious love is everlasting. \\
\poeml \v{2}Give thanks to the God of gods, \\
\poemll    for his gracious love is everlasting. \\
\poeml \v{3}Give thanks to the Lord of lords, \\
\poemll    for his gracious love is everlasting--- \\
\poeml \v{4}To the one who alone does great and wondrous things, \\
\poemll    for his gracious love is everlasting--- \\
\poeml \v{5}to the one who by wisdom made the heavens, \\
\poemll    for his gracious love is everlasting--- \\
\poeml \v{6}to the one who spread out the earth over the waters, \\
\poemll    for his gracious love is everlasting--- \\
\poeml \v{7}to the one who made the great lights, \\
\poemll    for his gracious love is everlasting--- \\
\poeml \v{8}the sun to illumine\fnote{Lit. \fbib{govern}; cf. Gen 1:16} the day, \\
\poemll    for his gracious love is everlasting--- \\
\poeml \v{9}and the moon and stars to illumine\fnote{Lit. \fbib{govern}; cf. Gen 1:16} the night, \\
\poemll    for his gracious love is everlasting--- \\
\poeml \v{10}to the one who struck the firstborn of Egypt, \\
\poemll    for his gracious love is everlasting--- \\
\poeml \v{11}and brought Israel out from among them, \\
\poemll    for his gracious love is everlasting--- \\
\poeml \v{12}with a strong hand and an active\fnote{Lit. \fbib{outstretched}} arm, \\
\poemll    for his gracious love is everlasting. \\
\poeml \v{13}To the one who split the Reed\fnote{So MT; LXX reads \fbib{Red}} Sea in two \\
\poemll    for his gracious love is everlasting--- \\
\poeml \v{14}and made Israel pass through the middle of it, \\
\poemll    for his gracious love is everlasting--- \\
\poeml \v{15}and cast Pharaoh and his armies into the Reed\fnote{So MT; LXX reads \fbib{Red}} Sea, \\
\poemll    for his gracious love is everlasting. \\
\poeml \v{16}To the one who led his people into the wilderness, \\
\poemll    for his gracious love is everlasting--- \\
\poeml \v{17}to the one who struck down great kings, \\
\poemll    for his gracious love is everlasting--- \\
\poeml \v{18}and killed famous kings, \\
\poemll    for his gracious love is everlasting--- \\
\poeml \v{19}including Sihon king of the Amorites, \\
\poemll    for his gracious love is everlasting--- \\
\poeml \v{20}and Og king of Bashan, \\
\poemll    for his gracious love is everlasting--- \\
\poeml \v{21}and gave their land as an inheritance, \\
\poemll    for his gracious love is everlasting--- \\
\poeml \v{22}to Israel his servant as a possession, \\
\poemll    for his gracious love is everlasting--- \\
\poeml \v{23}He it is who remembered us in our lowly circumstances, \\
\poemll    for his gracious love is everlasting--- \\
\poeml \v{24}and rescued us from our enemies, \\
\poemll    for his gracious love is everlasting. \\
\poeml \v{25}He gives food to all creatures, \\
\poemll    for his gracious love is everlasting. \\
\poeml \v{26}Give thanks to the God of Heaven, \\
\poemll    for his gracious love is everlasting.
\end{poetry}
\labelpsalm{137}
\passage{Remembering Jerusalem}

\begin{poetry}
\poeml \v{1}There we sat down and cried--- \\
\poemll    by the rivers of Babylon--- \\
\poemlll       as we remembered Zion. \\
\poeml \v{2}On the willows there \\
\poemll    we hung our harps, \\
\poeml \v{3}for it was there that our captors \\
\poemll    asked us for songs \\
\poeml and our torturers demanded joy from us, \\
\poemll    ``Sing us one of the songs about Zion!'' \\
\poeml \v{4}How are we to sing the song of the \divine{Lord} \\
\poemll    on foreign soil? \\
\poeml \v{5}If I forget you, Jerusalem, \\
\poemll    may my right hand cease to function.\fnote{Lit. \fbib{remember}} \\
\poeml \v{6}May my tongue stick to the roof of my mouth \\
\poemll    if I don't remember you, \\
\poeml if I don't consider Jerusalem \\
\poemll    to be more important than my highest joy. \\
\poeml \v{7}Remember the day of Jerusalem's fall,\fnote{The Heb. lacks \fbib{fall}} \divine{Lord}, \\
\poemll    because of\fnote{Lit. \fbib{against}} the Edomites, \\
\poeml who kept saying, ``Tear it down! \\
\poemll    Tear it right down to its foundations!'' \\
\poeml \v{8}Daughter of Babylon! You devastator! \\
\poemll    How blessed will be the one who pays you back \\
\poemlll       for what you have done to us. \\
\poeml \v{9}How blessed will be the one who seizes your young children \\
\poemll    and pulverizes them against the cliff!
\end{poetry}
\labelpsalm{138}
\passage{Thanksgiving to God}

\begin{poetry}
\poeml \v{1}\divine{Lord},\fnote{So LXX DSS 11QPs\textsuperscript{a}; MT and Aquilla lack \fbib{\divine{Lord}}} I thank\fnote{So MT; LXX reads \fbib{acknowledge}} you with all of my heart; \\
\poemll    because you heard the words that I spoke,\fnote{So LXX; MT DSS lack this line} \\
\poemlll       I will sing your praise before the heavenly beings.\fnote{Or \fbib{the gods}; LXX reads \fbib{the angels}} \\
\poeml \v{2}I will bow down in worship toward your holy Temple \\
\poemll    and give thanks to your name for your gracious love and truth, \\
\poeml for you have done great things \\
\poemll    to carry out your word \\
\poemlll       consistent with your name. \\
\poeml \v{3}When\fnote{Lit. \fbib{In the day}} I called out, you answered me; \\
\poemll    you strengthened me. \\
\poeml \v{4}\divine{Lord}, all the kings of the earth will give you thanks, \\
\poemll    for they have heard what you have spoken.\fnote{Lit. \fbib{heard the words of your mouth}} \\
\poeml \v{5}They will sing about the ways of the \divine{Lord}, \\
\poemll    for great is the glory of the \divine{Lord}! \\
\poeml \v{6}Though the \divine{Lord} is highly exalted, \\
\poemll    yet he pays attention to those who are lowly regarded, \\
\poemlll       but he is aware of the arrogant from afar. \\
\poeml \v{7}Though I walk straight into trouble, \\
\poemll    you preserve my life, \\
\poeml stretching out your hand \\
\poemll    to fight the vehemence of my enemies, \\
\poemlll       and your right hand delivers me. \\
\poeml \v{8}The \divine{Lord} will complete what his purpose is for me. \\
\poemll    \divine{Lord}, your gracious love is eternal; \\
\poemlll       do not abandon your personal work in me.\fnote{Lit. \fbib{abandon the work of your hand}}
\end{poetry}
\labelpsalm{139}
\psalminfo{To the Music Director: A Davidic Song}
\passage{God's Knowledge and Presence}

\begin{poetry}
\poeml \v{1}\divine{Lord}, you have examined me; \\
\poemll    you have known me. \\
\poeml \v{2}You know when I rest\fnote{Lit. \fbib{know my sitting}} \\
\poemll    and when I am active.\fnote{Lit. \fbib{and my rising}} \\
\poeml You understand what I am thinking \\
\poemll    when I am distant from you.\fnote{Or \fbib{thinking from a distance}} \\
\poeml \v{3}You scrutinize my life and my rest;\fnote{Or \fbib{death}; Lit. \fbib{my path and my lying down}} \\
\poemll    you are familiar with all of my ways. \\
\poeml \v{4}Even before I have formed a word with my tongue, \\
\poemll    you, \divine{Lord}, know it completely! \\
\poeml \v{5}You encircle me from back to front, \\
\poemll    placing your hand upon me. \\
\poeml \v{6}Knowledge like this is too amazing for me. \\
\poemll    It is beyond my reach, \\
\poemlll       and I cannot fathom it.
\passage{The Magnitude of God}
\poeml \v{7}Where can I flee from your spirit? \\
\poemll    Or where will I run from your presence? \\
\poeml \v{8}If I rise to heaven, there you are! \\
\poemll    If I lay down with the dead,\fnote{Lit. \fbib{to Sheol}; i.e. the realm of the dead} there you are! \\
\poeml \v{9}If I take wings with the dawn \\
\poemll    and settle down on the western horizon\fnote{Lit. \fbib{the end of the sea}} \\
\poeml \v{10}your hand will guide me there, too, \\
\poemll    while your right hand keeps a firm grip on me. \\
\poeml \v{11}If I say, ``Darkness will surely conceal me, \\
\poemll    and the light around me will become night,''\fnote{So MT LXX; DSS 11QPs\textsuperscript{a} reads \fbib{And let me say, ``Surely darkness conceals and night has girded me about.''}} \\
\poeml \v{12}even darkness isn't dark to you, \\
\poemll    darkness and light are the same to you.\fnote{The Heb. lacks \fbib{to you}} \\
\poeml \v{13}It was you who formed my internal organs,\fnote{Lit. \fbib{my kidneys}} \\
\poemll    fashioning me within my mother's womb. \\
\poeml \v{14}I praise you, \\
\poemll    because you are fearful and wondrous!\fnote{So DSS 11QPsa Syr Hieronymus; MT LXX read \fbib{because I am fearfully and wonderfully made}} \\
\poeml Your work is wonderful, \\
\poemll    and I am fully aware of it. \\
\poeml \v{15}My frame was not hidden from you \\
\poemll    while I was being crafted in a hidden place, \\
\poemlll       knit together in the depths of the earth. \\
\poeml \v{16}Your eyes looked upon my embryo, \\
\poemll    and everything was recorded in your book. \\
\poeml The days scheduled\fnote{The Heb. lacks \fbib{scheduled}} for my formation were inscribed, \\
\poemll    even though not one of them had come yet.\fnote{The Heb. lacks \fbib{had come yet}} \\
\poeml \v{17}How deep\fnote{Or \fbib{precious}} are your thoughts, God! \\
\poemll    How great is their number! \\
\poeml \v{18}Were I to count them, \\
\poemll    they would number more than the sand. \\
\poemlll       When I awake, I will be with you. \\
\poeml \v{19}God, if only you would execute the wicked, \\
\poemll    so that\fnote{So LXX DSS 11QPsa; MT reads \fbib{and so that}} the men guilty of bloodshed would get away from me, \\
\poeml \v{20}who speak against you with evil motives, \\
\poemll    your enemies who are acting in vain. \\
\poeml \v{21}I hate those who hate you, \divine{Lord}, do I not? \\
\poemll    I loathe those who rebel against you, do I not ? \\
\poeml \v{22}With consummate hatred I hate them; \\
\poemll    I consider them my enemies. \\
\poeml \v{23}Examine me, God, and know my mind, \\
\poemll    test me, and know my thoughts. \\
\poeml \v{24}See if there is any offensive tendency\fnote{Lit. \fbib{way}} in me, \\
\poemll    and lead me in the eternal way.
\end{poetry}
\labelpsalm{140}
\psalminfo{To the Music Director: A Davidic Song}
\passage{A Prayer for Deliverance}

\begin{poetry}
\poeml \v{1}\fnote{V.1 is v. 2 in MT, and so throughout the chapter.}Deliver me, \divine{Lord}, from evil people, \\
\poemll    preserve me from violent men, \\
\poeml \v{2}who craft evil plans in their minds, \\
\poemll    inciting wars every day.\fnote{Lit. \fbib{all day}; LXX DSS 11QPs\textsuperscript{a} read \fbib{all the day}} \\
\poeml \v{3}They sharpen their tongues like a serpent; \\
\poemll    the venom of vipers is on their lips.
\end{poetry}
\interlude{Interlude}

\begin{poetry}
\poeml \v{4}Protect me, \divine{Lord}, from the control of evil people, \\
\poemll    from violent men who have planned to trip me. \\
\poeml \v{5}The arrogant have laid a trap for me; \\
\poemll    they have spread a net with ropes, \\
\poemlll       lining it with snares along the way.
\end{poetry}
\interlude{Interlude}

\begin{poetry}
\poeml \v{6}So I say to the \divine{Lord}, ``You are my God; \\
\poemll    listen to my voice \\
\poemlll       as I plead for mercy, \divine{Lord}. \\
\poeml \v{7}\divine{Lord}, my Lord, my strong deliverer, \\
\poemll    you have protected my head in the time\fnote{Lit. \fbib{day}} of battle. \\
\poeml \v{8}Never grant, \divine{Lord}, the desires of the wicked; \\
\poemll    never condone their plans \\
\poemlll       so they cannot exalt themselves.
\end{poetry}
\interlude{Interlude}

\begin{poetry}
\poeml \v{9}May those who surround me discover \\
\poemll    that the trouble they talk about falls on their own head! \\
\poeml \v{10}May burning coals fall on them; \\
\poemll    may they be cast into fire, \\
\poemlll       and into miry pits, never to rise again. \\
\poeml \v{11}Let not the slanderer\fnote{Lit. \fbib{the man of tongue}} become established in the land. \\
\poemll    May evil quickly hunt down the violent man. \\
\poeml \v{12}I know that the \divine{Lord} will act on behalf of the tormented, \\
\poemll    providing justice for the needy. \\
\poeml \v{13}Surely the righteous will give thanks to your name, \\
\poemll    while the upright live in your presence.
\end{poetry}
\labelpsalm{141}
\psalminfo{A Davidic Song}
\passage{A Prayer for Maturity}

\begin{poetry}
\poeml \v{1}\divine{Lord}, I call to you, \\
\poemll    be quick to listen to me when I cry out! \\
\poeml \v{2}Let my prayer be like incense offered before you, \\
\poemll    and my uplifted hands like the evening sacrifice. \\
\poeml \v{3}\divine{Lord}, set a guard over my mouth; \\
\poemll    keep watch over the door to my lips. \\
\poeml \v{4}Don't let my heart turn toward evil \\
\poemll    or involve itself in wicked activities \\
\poeml with men who practice iniquity. \\
\poemll    Let me not feast on their delicacies. \\
\poeml \v{5}Let one who is righteous strike me; \\
\poemll    It is an act of gracious love. \\
\poeml Let him rebuke me, \\
\poemll    because it is oil for my head; \\
\poemll    do not let my head refuse it. \\
\poeml My prayers continuously will be \\
\poemll    against their wicked activities. \\
\poeml \v{6}When their judges are thrown off the cliff, \\
\poemll    the people\fnote{Lit. \fbib{they}} will hear my words, \\
\poemlll       for they are appropriate. \\
\poeml \v{7}Just as one plows and breaks up the earth, \\
\poemll    our\fnote{So MT LXX; DSS 11QPs\textsuperscript{a} reads \fbib{my}; Syr reads \fbib{their}} bones are scattered \\
\poemlll       near the entrance to the place of the dead.\fnote{Lit. \fbib{to Sheol}; i.e. the realm of the dead} \\
\poeml \v{8}Nevertheless, my eyes are on you, Lord \divine{God}, \\
\poemll    as I seek protection in you. \\
\poemlll       Don't leave me defenseless! \\
\poeml \v{9}Protect me from the trap laid for me \\
\poemll    and from the snares of those who practice evil. \\
\poeml \v{10}Let the wicked fall into their own nets, \\
\poemll    while I come through.
\end{poetry}
\labelpsalm{142}
\psalminfo{A Davidic Song, when he was in the cave.\fnote{T cf. 1Sam 24:3-4} A prayer.}
\passage{A Call to God for Help}

\begin{poetry}
\poeml \v{1}My voice cries out to the \divine{Lord}; \\
\poemll    my voice pleads for mercy to the \divine{Lord}. \\
\poeml \v{2}I pour out my complaint to him, \\
\poemll    telling him all of my troubles. \\
\poeml \v{3}Though my spirit grows faint within me, \\
\poemll    you are aware of my path. \\
\poeml Wherever I go, \\
\poemll    they have hidden a trap for me. \\
\poeml \v{4}I look to my right\fnote{So LXX and DSS 11QPs\textsuperscript{a}; MT reads \fbib{Look to the right}} and observe--- \\
\poemll    no one is concerned about me. \\
\poeml There is nowhere I can go for refuge, \\
\poemll    and no one cares for me. \\
\poeml \v{5}So I cry to you, Lord, \\
\poemll    declaring, ``You are my refuge, \\
\poemlll       my only\fnote{The Heb. lacks \fbib{only}} possession while I am on this earth.''\fnote{Lit. \fbib{possession in the land of the living}} \\
\poeml \v{6}Pay attention to my cry, \\
\poemll    for I have been brought very low. \\
\poeml Deliver me from my tormentors, \\
\poemll    for they are far too strong for me. \\
\poeml \v{7}Break me out of this prison, \\
\poemll    so I can give thanks to your name. \\
\poeml The righteous will surround me, \\
\poemll    for you will deal generously with me.
\end{poetry}
\labelpsalm{143}
\psalminfo{A Davidic Song}
\passage{Longing for God}

\begin{poetry}
\poeml \v{1}\divine{Lord}, hear my prayer; \\
\poemll    pay attention to my request, because you are faithful; \\
\poemlll       answer me in your righteousness. \\
\poeml \v{2}Do not enter into judgment with your servant, \\
\poemll    for no living person is righteous in your sight. \\
\poeml \v{3}For those who oppose me are pursuing my life, \\
\poemll    crushing me to the ground, \\
\poeml making me sit in darkness \\
\poemll    like those who died long ago. \\
\poeml \v{4}As a result, my spirit is desolate within me, \\
\poemll    and my mind within me is appalled. \\
\poeml \v{5}I remember the former times, \\
\poemll    meditating on everything you have done. \\
\poemlll       I think about the work\fnote{So MT; LXX DSS 11QPs\textsuperscript{a} read \fbib{works}} of your hands. \\
\poeml \v{6}I stretch out my hands toward you, \\
\poemll    longing for you like a parched land.
\end{poetry}
\interlude{Interlude}

\begin{poetry}
\poeml \v{7}Answer me quickly, \divine{Lord}; \\
\poemll    my spirit is failing. \\
\poeml Do not hide your face from me; \\
\poemll    otherwise, I will become like those who descend to the Pit,\fnote{I.e. the place of punishment in the afterlife} \\
\poeml \v{8}In the morning let me hear of your gracious love, \\
\poemll    for in you I trust. \\
\poeml Cause me to know the way I should take, \\
\poemll    because I have set my hope on you. \\
\poeml \v{9}Deliver me from my enemies, \divine{Lord}. \\
\poemll    I have taken refuge in you. \\
\poeml \v{10}Teach me to do your will, \\
\poemll    for you are my God. \\
\poemlll       Let your good Spirit lead me on level ground. \\
\poeml \v{11}For the sake of your name, \divine{Lord}, \\
\poemll    preserve my life. \\
\poeml Because you are righteous, \\
\poemll    bring me out of trouble. \\
\poeml \v{12}Because of your gracious love, \\
\poemll    you will cut off my enemies. \\
\poeml You will destroy all who oppose me, \\
\poemll    for I am your servant.
\end{poetry}
\labelpsalm{144}
\psalminfo{Davidic}
\passage{A Song for God's Provision}

\begin{poetry}
\poeml \v{1}Blessed be the \divine{Lord}, my rock, \\
\poemll    who trains my hands for battle \\
\poemlll       and my fingers for warfare, \\
\poeml \v{2}he is my gracious love and my fortress, \\
\poemll    my strong tower and my deliverer, \\
\poeml my shield and the one in whom I find refuge, \\
\poemll    who subdues\fnote{So LXX; the Heb. lacks \fbib{subdues}} peoples\fnote{So DSS 11QPs\textsuperscript{a} Sebir Aquila Syr Targ Hieronymus; cf. Psa 18:48; 2 Sam 22:48; LXX reads \fbib{subdues my people}} under me. \\
\poeml \v{3}\divine{Lord}, what are human beings, \\
\poemll    that you should care about them, \\
\poeml or mortal man, \\
\poemll    that you should think about him? \\
\poeml \v{4}The human person is a mere empty breath; \\
\poemll    his days are like a fading shadow. \\
\poeml \v{5}Bow your heavens, \divine{Lord}, and descend;\fnote{So MT (imperfect verb); LXX, DSS 11QPs\textsuperscript{a} read \fbib{and descend!}; i.e. a verb of command} \\
\poemll    touch the mountains, and they will smolder. \\
\poeml \v{6}Send forth lightning and scatter the enemy,\fnote{Lit. \fbib{scatter them}} \\
\poemll    shoot your arrows and confuse them. \\
\poeml \v{7}Reach down your hand from your high place; \\
\poemll    rescue me and deliver me from mighty waters, \\
\poemlll       from the control of foreigners.\fnote{Lit. \fbib{the hand of the sons of strangers}} \\
\poeml \v{8}Their mouths speak lies, \\
\poemll    and their right hand deceives,\fnote{I.e. they swear to false oaths} \\
\poeml \v{9}God, I will sing a new song to you. \\
\poemll    On a harp of ten strings I will play to you--- \\
\poeml \v{10}to you who gives victory to kings, \\
\poemll    rescuing his servant David from cruel swords. \\
\poeml \v{11}Rescue me and deliver me \\
\poemll    from the control of foreigners,\fnote{Lit. \fbib{the hand of the sons of strangers}} \\
\poeml whose mouths speak lies, \\
\poemll    and whose right hand deceives.\fnote{I.e. they swear to false oaths} \\
\poeml \v{12}May our sons in their youth be like full-grown plants, \\
\poemll    and our daughters like pillars \\
\poemlll       destined to decorate a palace. \\
\poeml \v{13}May our granaries be filled, \\
\poemll    storing produce in abundance; \\
\poeml may our sheep bring forth thousands, \\
\poemll    even tens of thousands in our fields. \\
\poeml \v{14}May our cattle grow heavy with young, \\
\poemll    with no damage or loss. \\
\poeml May there be no cry of anguish in our streets! \\
\poeml \v{15}Happy are the people to whom these things come; \\
\poemll    happy are the people whose God is the \divine{Lord}.
\end{poetry}
\labelpsalm{145}
\psalminfo{A Davidic Psalm\fnote{T In this acrostic psalm each verse begins with a successive letter of the Heb. alphabet, except that v13b, corresponding to the Heb. Letter nun, is missing from MT}}
\passage{Praising God for His Works}

\begin{poetry}
\poeml \v{1}I will speak highly of you, my God and King, \\
\poemll    and I will bless your name forever and ever. \\
\poeml \v{2}I will bless you every day \\
\poemll    and I will praise your name forever and ever. \\
\poeml \v{3}The \divine{Lord} is great, \\
\poemll    and to be praised highly, \\
\poemlll       though his greatness is indescribable. \\
\poeml \v{4}One generation will acclaim your works to another \\
\poemll    and will describe your mighty actions. \\
\poeml \v{5}I\fnote{So MT; LXX Syr DSS 11QPs\textsuperscript{a} read \fbib{They}} will speak about the glorious splendor of your majesty \\
\poemll    as well as\fnote{So LXX DSS 11QPs\textsuperscript{a}; the Heb. lacks \fbib{as}} your awesome actions. \\
\poeml \v{6}People\fnote{Lit. \fbib{They}} will speak about the might of your great deeds, \\
\poemll    and I will announce your greatness. \\
\poeml \v{7}They will extol the fame of your abundant goodness, \\
\poemll    and will sing out loud about your righteousness. \\
\poeml \v{8}Gracious and merciful is the \divine{Lord}, \\
\poemll    slow to become angry, \\
\poemlll       and overflowing with gracious love. \\
\poeml \v{9}The \divine{Lord} is good to everyone \\
\poemll    and his mercies extend to everything he does. \\
\poeml \v{10}\divine{Lord}, everything you have done will praise you, \\
\poemll    and your holy ones will bless you. \\
\poeml \v{11}They will speak about the glory of your kingdom, \\
\poemll    and they will talk about your might, \\
\poeml \v{12}in order to make known your mighty acts to mankind\fnote{Lit. \fbib{the children of the Man}} \\
\poemll    as well as the majestic splendor of your kingdom. \\
\poeml \v{13}Your kingdom is an everlasting kingdom, \\
\poemll    and your authority endures from one generation to another. \\
\poeml 13bGod\fnote{So DSS 11QPs\textsuperscript{a}; LXX Vg Syr read \fbib{The \divine{Lord};} MT lacks this v.} is faithful about everything he says \\
\poemll    and merciful in everything he does. \\
\poeml \v{14}The \divine{Lord} supports everyone who falls \\
\poemll    and raises up those who are bowed down. \\
\poeml \v{15}Everyone's eyes are on you, \\
\poemll    as you give them their food in due time. \\
\poeml \v{16}You\fnote{So MT; LXX DSS 11QPs\textsuperscript{a} read \fbib{You yourself}} open your hand \\
\poemll    and keep on satisfying the desire of every living thing. \\
\poeml \v{17}The \divine{Lord} is righteous in all of his ways \\
\poemll    and graciously loving in all of his activities. \\
\poeml \v{18}The \divine{Lord} remains near to all who call out to him, \\
\poemll    to everyone who calls out to him sincerely.\fnote{Or \fbib{truthfully}} \\
\poeml \v{19}He fulfills the desire of those who fear him, \\
\poemll    hearing their cry and saving them. \\
\poeml \v{20}The \divine{Lord} preserves everyone who loves him, \\
\poemll    but he will destroy all of the wicked. \\
\poeml \v{21}My mouth will praise the \divine{Lord}, \\
\poemll    and all creatures will bless his holy name forever and ever.
\end{poetry}
\labelpsalm{146}
\passage{Praise to God the Help of Israel}

\begin{poetry}
\poeml \v{1}Hallelujah! \\
\poemll    Praise the \divine{Lord}, my soul! \\
\poeml \v{2}I will praise the \divine{Lord} as long as I live, \\
\poemll    singing praises to my God while I exist. \\
\poeml \v{3}Do not look to nobles, \\
\poemll    nor to mere human beings who cannot save. \\
\poeml \v{4}When they stop breathing, \\
\poemll    they return to the ground; \\
\poemlll       on that very day their plans evaporate! \\
\poeml \v{5}Happy is the one whose help is the God of Jacob, \\
\poemll    whose hope is in the \divine{Lord} his God, \\
\poeml \v{6}maker of heaven and earth, \\
\poemll    the seas and everything in them, \\
\poemlll       forever the guardian of truth, \\
\poeml \v{7}who brings justice for the oppressed, \\
\poemll    and who gives food to the hungry. \\
\poeml The \divine{Lord} frees the prisoners; \\
\poeml \v{8}the \divine{Lord} gives sight to the blind. \\
\poeml The \divine{Lord} lifts up those who are weighed down. \\
\poemll    The \divine{Lord} loves the righteous. \\
\poeml \v{9}The \divine{Lord} stands guard over the stranger; \\
\poemll    he supports both widows and orphans, \\
\poemlll       but makes the path of the wicked slippery.\fnote{Or \fbib{treacherous}} \\
\poeml \v{10}The \divine{Lord} will reign forever, \\
\poemll    your God, Zion, for all generations! \\
\poeml Hallelujah!
\end{poetry}
\labelpsalm{147}
\passage{Praise for God's Provision}

\begin{poetry}
\poeml \v{1}Hallelujah! \\
\poemll    It is good to sing praise to our God, \\
\poemlll       and it is fitting to sing glorious praise. \\
\poeml \v{2}The \divine{Lord} rebuilds Jerusalem; \\
\poemll    he gathers together the outcasts of Israel. \\
\poeml \v{3}He heals the brokenhearted, \\
\poemll    binding up their injuries. \\
\poeml \v{4}He keeps track of the number of stars, \\
\poemll    assigning names to all of them. \\
\poeml \v{5}Our Lord is great, \\
\poemll    and rich in power; \\
\poemlll       his understanding has no limitation. \\
\poeml \v{6}The \divine{Lord} supports the afflicted \\
\poemll    while he casts the wicked to the ground. \\
\poeml \v{7}Sing to the \divine{Lord} with thanksgiving, \\
\poemll    and compose music to our God with the lyre. \\
\poeml \v{8}He shields the heavens with clouds, \\
\poemll    preparing rain for the earth \\
\poemlll       and making grass grow on the hills. \\
\poeml \v{9}He gives wild animals their food, \\
\poemll    including the young ravens when they cry. \\
\poeml \v{10}He takes no delight in the strength of a horse, \\
\poemll    and gains no pleasure in the runner's swiftness.\fnote{Lit. \fbib{the legs of a man}} \\
\poeml \v{11}But the \divine{Lord} is pleased with those who fear him, \\
\poemll    with those who depend on his gracious love. \\
\poeml \v{12}Glorify the \divine{Lord}, Jerusalem! \\
\poemll    Praise your God, Zion! \\
\poeml \v{13}For he has strengthened the bars of your gates, \\
\poemll    blessing your children within you. \\
\poeml \v{14}He grants peace within your borders, \\
\poemll    satisfying\fnote{So MT; LXX DSS 4QPs\textsuperscript{d} read \fbib{borders, and satisfies}} you with the finest of wheat. \\
\poeml \v{15}He sends out his command to the earth, \\
\poemll    making\fnote{The Heb. lacks \fbib{making}} his word go forth quickly. \\
\poeml \v{16}He supplies snow like wool, \\
\poemll    scattering frost like ashes. \\
\poeml \v{17}He casts down his ice crystals like bread\fnote{The Heb. lacks \fbib{bread}} fragments. \\
\poemll    Who can endure his freezing cold? \\
\poeml \v{18}He sends out his word \\
\poemll    and melts them. \\
\poeml He makes his wind blow \\
\poemll    and the water flows. \\
\poeml \v{19}He declares his words to Jacob, \\
\poemll    his statutes and decrees to Israel. \\
\poeml \v{20}He has not dealt with any other nation like this; \\
\poemll    they never knew\fnote{So MT; LXX reads \fbib{he did not explain to them}; Syr Targ DSS 11QPs\textsuperscript{a} read \fbib{he has not revealed to them}} his decrees. \\
\poeml Hallelujah!
\end{poetry}
\labelpsalm{148}
\passage{Let All the Earth Praise the \divine{Lord}}

\begin{poetry}
\poeml \v{1}Hallelujah! \\
\poemll    Praise the \divine{Lord} from heaven; \\
\poemlll       praise him in the highest places. \\
\poeml \v{2}Praise him, all his angels; \\
\poemll    praise him, all his armies! \\
\poeml \v{3}Praise him, sun and moon; \\
\poemll    praise him, all you shining stars.\fnote{Lit. \fbib{you stars of light}} \\
\poeml \v{4}Praise him, you heaven of heavens, \\
\poemll    and you waters above the heavens. \\
\poeml \v{5}Let them praise the name of the \divine{Lord}, \\
\poemll    for he himself gave the command that they be created. \\
\poeml \v{6}He set them in place to last forever and ever; \\
\poemll    he gave the command and will not rescind it. \\
\poeml \v{7}Praise the \divine{Lord}, you from the earth, \\
\poemll    you creatures of the sea \\
\poemlll       and all you depths, \\
\poeml \v{8}fire, hail, snow, fog, and wind storm \\
\poemll    that carry out his command,\fnote{Or \fbib{word}} \\
\poeml \v{9}mountains and every hill, \\
\poemll    fruit trees and cedars, \\
\poeml \v{10}living creatures and livestock, \\
\poemll    insects and flying birds, \\
\poeml \v{11}earthly kings and all peoples, \\
\poemll    nobles and all court officials of the earth, \\
\poeml \v{12}young men and young women alike, \\
\poemll    along with older people and children. \\
\poeml \v{13}Let them praise the name of the \divine{Lord}, \\
\poemll    for his name alone is lifted up; \\
\poemlll       his majesty transcends earth and heaven. \\
\poeml \v{14}He has raised up a source of strength\fnote{Lit. \fbib{a horn}} for his people, \\
\poemll    an object of praise for all of his holy ones, \\
\poemlll       that is, for the people of Israel who are near him. \\
\poeml Hallelujah!
\end{poetry}
\labelpsalm{149}
\passage{A Song About Rejoicing in God}

\begin{poetry}
\poeml \v{1}Hallelujah! \\
\poemll    Sing a new song to the \divine{Lord}, \\
\poemlll       praising him where the godly gather together. \\
\poeml \v{2}May Israel rejoice in its Maker, \\
\poemll    and Zion's descendants in their King! \\
\poeml \v{3}May they praise his name with dancing, \\
\poemll    chanting songs to him with tambourines and lyres. \\
\poeml \v{4}For the \divine{Lord} is pleased with his people; \\
\poemll    he beautifies the afflicted with salvation. \\
\poeml \v{5}May those he loves be exalted, \\
\poemll    singing for joy on their couches. \\
\poeml \v{6}Let high praises to God be heard\fnote{The Heb. lacks \fbib{heard}} in their throats, \\
\poemll    while they wield two-edged swords in their hands \\
\poeml \v{7}as they bring retribution to nations \\
\poemll    and punishment to peoples, \\
\poeml \v{8}binding their kings with chains, \\
\poemll    their officials with iron bands, \\
\poeml \v{9}and executing the judgment written against them. \\
\poeml This is honor for all the ones he loves. \\
\poeml Hallelujah!
\end{poetry}
\labelpsalm{150}
\passage{A Psalm of Praise}

\begin{poetry}
\poeml \v{1}Hallelujah! \\
\poeml Praise God in his Holy Place. \\
\poemll    Praise him in his great expanse. \\
\poeml \v{2}Praise him for his mighty works. \\
\poemll    Praise him according to his excellent greatness. \\
\poeml \v{3}Praise him with trumpet sounding. \\
\poemll    Praise him with stringed instrument and harp. \\
\poeml \v{4}Praise him with tambourine and dancing. \\
\poemll    Praise him with stringed and wind instruments. \\
\poeml \v{5}Praise him with loud cymbals. \\
\poemll    Praise him with reverberating cymbals. \\
\poeml \v{6}Let everyone who breathes praise the \divine{Lord}. \\
\poeml Hallelujah!
\end{poetry}

\renewcommand{\verseone}{}
\bookheader{Proverbs}
\labelbook{Prov}

\bookpretitle{The Book of}
\booktitle{Proverbs}

\labelchapt{1}
\passage{Introduction and Purpose}

\chapt{1}
\v{1}The proverbs of David's son Solomon, king of Israel.

\begin{poetry}
\poeml \v{2}These proverbs are\fnote{The Heb. lacks \fbib{These proverbs are}} for gaining\fnote{Or \fbib{knowing}} wisdom and discipline;\fnote{Or \fbib{instruction}} \\
\poemll    for understanding words of insight; \\
\poeml \v{3}for acquiring the discipline\fnote{Or \fbib{instruction}} that produces wise behavior, \\
\poemll    righteousness, justice, and upright living;\fnote{Lit. \fbib{and uprightness}} \\
\poeml \v{4}for giving prudence to the na\"{i}ve, \\
\poemll    and knowledge and discretion to the young. \\
\poeml \v{5}Let the wise listen and increase their\fnote{The Heb. lacks \fbib{their}} learning; \\
\poemll    let the person of understanding receive guidance \\
\poeml \v{6}in understanding proverbs, clever sayings, \\
\poemll    words of the wise, and their riddles.
\passage{The Major Theme}
\poeml \v{7}The fear of the \divine{Lord} is the beginning of knowledge, \\
\poemll    but fools despise wisdom and discipline.\fnote{Or \fbib{instruction}}
\passage{The Minor Theme}
\poeml \v{8}My son, listen to your father's instruction, \\
\poemll    and do not let go of your mother's teaching. \\
\poeml \v{9}They will be a graceful wreath for your head \\
\poemll    and a chain for your neck.
\passage{Avoid Evil Counsel}
\poeml \v{10}My son, if sinners entice you, \\
\poemll    do not consent. \\
\poeml \v{11}If they say, ``Come with us! \\
\poemll    Let's lie in wait for blood; \\
\poemlll       let's ambush some innocent person for no reason at all. \\
\poeml \v{12}Let's swallow them alive like Sheol,\fnote{I.e. the realm of the dead; possibly an allusion to the rebellion of Korah (cf. Num 16:33)} \\
\poemll    and whole like those who go down into the Pit.\fnote{I.e. the place of punishment in the afterlife} \\
\poeml \v{13}We'll find all kinds of valuable wealth, \\
\poemll    and we'll fill our houses with spoil. \\
\poeml \v{14}Throw your lot in with us, \\
\poemll    and all of us will have one purse.'' \\
\poeml \v{15}My son, do not go along with them,\fnote{Lit. \fbib{in the way with them}} \\
\poemll    and keep your feet away from their paths! \\
\poeml \v{16}For they\fnote{Lit. \fbib{For their feet}} run toward evil; \\
\poemll    these enticers\fnote{Lit. \fbib{they}} shed blood without hesitation.\fnote{Lit. \fbib{blood quickly}} \\
\poeml \v{17}Look, it is useless to spread a net in full view of\fnote{Lit. \fbib{in the eyes of}} all the birds, \\
\poeml \v{18}but these people\fnote{Lit. \fbib{they}} lie in wait for their own blood.\fnote{The Heb. lacks \fbib{their own}} \\
\poemlll       They ambush only themselves. \\
\poeml \v{19}Such is the way of all those who seek illicit gain--- \\
\poemll    it takes away the lives of those who possess it.
\passage{The Benefits of Choosing Wisdom}
\poeml \v{20}Wisdom cries out in the street; \\
\poemll    she raises her voice in the public squares. \\
\poeml \v{21}She calls out at the busiest part\fnote{Lit. \fbib{head}} of the noisy streets,\fnote{So MT; LXX Syr Targ read \fbib{and on top of the walls}} \\
\poemll    and at the entrance to the gates of the city she utters her words: \\
\poeml \v{22}``You na\"{i}ve ones, how long will you love naivet\'{e}? \\
\poemll    And how long will scoffers delight in scoffing \\
\poemlll       or fools hate knowledge?'' \\
\poeml \v{23}Return to my correction! \\
\poemll    Look, I will pour out my spirit on you, \\
\poemlll       and I will make my words known to you.
\passage{The Consequences of Refusing Wisdom}
\poeml \v{24}``Because I called out to you and you refused to respond---\fnote{Lit. \fbib{you refused}} \\
\poemll    I appealed,\fnote{Lit. \fbib{I stretched out my hand}} but no one paid attention--- \\
\poeml \v{25}because\fnote{The Heb. lacks \fbib{because}} you neglected all my advice \\
\poemll    and did not want my correction, \\
\poeml \v{26}I will laugh at your calamity. \\
\poemll    I will mock when what you fear\fnote{Lit. \fbib{when your fear}} comes, \\
\poeml \v{27}when what you dread comes like a storm, \\
\poemll    and your calamity comes on like a whirlwind, \\
\poemlll       when distress and anguish come upon you. \\
\poeml \v{28}``Then they will call out to me, \\
\poemll    but I will not answer; \\
\poeml they will seek me diligently, \\
\poemll    but they will not find me. \\
\poeml \v{29}``Because they hated knowledge \\
\poemll    and did not choose the fear of the \divine{Lord}; \\
\poeml \v{30}they did not want my advice, \\
\poemll    and they rejected all my correction. \\
\poeml \v{31}They will eat the fruit\fnote{I.e. experience the consequences} of their way, \\
\poemll    and they will be filled with their own devices. \\
\poeml \v{32}Indeed, the waywardness\fnote{So MT; DSS 4QProv\textsuperscript{a} reads \fbib{narrow-mindedness}; lit. \fbib{the pull of}; LXX reads \fbib{Because they would wrong the na\"{i}ve, they will be murdered}} of the na\"{i}ve will kill them, \\
\poemll    and the complacency of fools will destroy them. \\
\poeml \v{33}``But the person who listens to me will live safely \\
\poemll    and will be secure from the fear of evil.''
\end{poetry}
\labelchapt{2}
\passage{The Benefits of Embracing Wisdom}

\begin{poetry}
\poeml \chapt{2}
\v{1}My son, if you accept my words, \\
\poeml and treasure my instructions\fnote{Lit. \fbib{instructions within you}}--- \\
\poeml \v{2}making your ear attentive to wisdom, \\
\poemll    and turning your heart to understanding--- \\
\poeml \v{3}if, indeed, you call out for insight \\
\poemll    and raise your voice for understanding, \\
\poeml \v{4}if you seek it like silver \\
\poemll    and search for it like hidden treasure, \\
\poeml \v{5}then you will understand the fear of the \divine{Lord} \\
\poemll    and learn to know God. \\
\poeml \v{6}For the \divine{Lord} gives wisdom, \\
\poemll    and from his mouth come knowledge and understanding. \\
\poeml \v{7}He stores up sound wisdom for the upright \\
\poemll    and is a shield to those who walk in integrity--- \\
\poeml \v{8}guarding the paths of the just \\
\poemll    and protecting the way of his faithful ones. \\
\poeml \v{9}Then you will understand what is right, just, \\
\poemll    and upright---every good path. \\
\poeml \v{10}For wisdom will enter your heart, \\
\poemll    and knowledge will be pleasant to your soul. \\
\poeml \v{11}Discretion\fnote{Or \fbib{Wise planning}} will protect you; \\
\poemll    understanding will watch over you, \\
\poeml \v{12}delivering you from the way of evil, \\
\poemll    from men who speak perverse things, \\
\poeml \v{13}and from those who abandon the right\fnote{Lit. \fbib{straight} or \fbib{upright}} path \\
\poemll    to travel along the ways of darkness; \\
\poeml \v{14}who delight in doing evil, \\
\poemll    and rejoice in the perverseness of evil; \\
\poeml \v{15}whose paths are crooked \\
\poemll    and who are devious in their ways, \\
\poeml \v{16}delivering you from the adulteress, \\
\poemll    from the immoral\fnote{Lit. \fbib{foreign}; i.e. one whose values are foreign to God's Law} woman with her seductive words, \\
\poeml \v{17}someone who abandoned the companion of her youth \\
\poemlll       and forgot the covenant of her God. \\
\poeml \v{18}For her house leads down to death, \\
\poemll    and her paths down to the realm of the dead. \\
\poeml \v{19}None who go to her return, \\
\poeml nor do they reach the paths of life. \\
\poeml \v{20}This is how you will walk in the way of good men \\
\poemll    and will keep to the paths of the righteous. \\
\poeml \v{21}For the upright will live in the land, \\
\poemll    and people of integrity will remain in it. \\
\poeml \v{22}But the wicked will be cut off from the land, \\
\poemll    and the treacherous will be uprooted from it.
\end{poetry}
\labelchapt{3}
\passage{The Blessings of Trusting God}

\begin{poetry}
\poeml \chapt{3}
\v{1}My son, don't forget my instruction, \\
\poeml and keep my commandments carefully in mind.\fnote{Lit. \fbib{Let your heart keep my commandments}} \\
\poeml \v{2}For they will add length to your days, years to your life, \\
\poemll    and abundant peace to you. \\
\poeml \v{3}Do not let gracious love and truth leave you. \\
\poemll    Bind them around your neck, \\
\poemlll       write them on the tablet of your heart, \\
\poeml \v{4}and find favor and a good reputation\fnote{Lit. \fbib{good judgment} or \fbib{sense}} with God and men. \\
\poeml \v{5}Trust in the \divine{Lord} with all your heart, \\
\poemll    and do not depend on your own understanding. \\
\poeml \v{6}In all your ways acknowledge\fnote{Or \fbib{know}} him, \\
\poemll    and he will make your paths straight. \\
\poeml \v{7}Do not be wise in your own opinion. \\
\poemll    Fear the \divine{Lord} and turn away from evil. \\
\poeml \v{8}This will bring healing to your body, \\
\poemll    and refreshment to your bones. \\
\poeml \v{9}Honor the \divine{Lord} with your wealth \\
\poemll    and with the first\fnote{Or \fbib{best}} of all your produce, \\
\poeml \v{10}so your barns will be filled with abundance, \\
\poemll    and your vats will burst open with new wine. \\
\poeml \v{11}My son, do not reject the \divine{Lord}'s discipline, \\
\poemll    and do not despise his correction, \\
\poeml \v{12}because the \divine{Lord} corrects the person he loves, \\
\poemll    just as a father corrects\fnote{The Heb. lacks \fbib{corrects}} the son he delights in.\fnote{So MT; LXX reads \fbib{loves, and he punishes every son he accepts}}
\passage{Wisdom More Valuable than Riches}
\poeml \v{13}How joyful is the man who finds wisdom, \\
\poemll    and the man who gains understanding, \\
\poeml \v{14}because her profit is better than the profit of silver, \\
\poemll    and her yield than fine gold. \\
\poeml \v{15}She is more precious than rubies, \\
\poemll    and nothing you desire compares with her. \\
\poeml \v{16}Long life is in her right hand, \\
\poemll    and in her left are riches and honor. \\
\poeml \v{17}Her ways are pleasant ways, \\
\poemll    and all her paths are peaceful. \\
\poeml \v{18}She is a tree of life for those who embrace her, \\
\poemll    and whoever clutches her tightly will be joyful. \\
\poeml \v{19}By wisdom the \divine{Lord} laid the earth's foundations, \\
\poemll    and by understanding he set the heavens in place. \\
\poeml \v{20}By his knowledge the depths broke open, \\
\poemll    and the clouds drip with dew.
\passage{Benefits of Wisdom}
\poeml \v{21}My son, do not let wisdom\fnote{The Heb. lacks \fbib{wisdom}} leave your sight. \\
\poemll    Carefully observe sound judgment and discernment, \\
\poeml \v{22}and they will be life to you \\
\poemll    and a graceful ornament\fnote{Lit. \fbib{grace}} for your neck. \\
\poeml \v{23}Then you will travel safely on your way, \\
\poemll    and your foot will not stumble. \\
\poeml \v{24}When you sit\fnote{So LXX; MT reads \fbib{lie}} down, you will not be afraid; \\
\poemll    when you lie down, your sleep will be pleasant.\fnote{Or \fbib{sweet}} \\
\poeml \v{25}Do not be afraid of sudden disaster,\fnote{Lit. \fbib{terror}} \\
\poemll    or the devastation that comes to the wicked. \\
\poeml \v{26}Indeed, the \divine{Lord} will be your confidence, \\
\poemll    and he will keep your foot from being caught.
\passage{Wisdom in Action}
\poeml \v{27}Do not withhold good from those to whom it is due, \\
\poemll    when it is in your power to act. \\
\poeml \v{28}Do not say to your neighbor, \\
\poemll    ``Go, and come back. \\
\poemlll       I will pay you\fnote{Lit. \fbib{it}} tomorrow,'' \\
\poeml when you have cash\fnote{The Heb. lacks \fbib{cash}} with you. \\
\poeml \v{29}Do not plan to harm your neighbor, \\
\poemll    when he is living peacefully\fnote{Or \fbib{securely}} beside you. \\
\poeml \v{30}Do not bring a lawsuit against a person for no reason, \\
\poemll    when he has done you no harm. \\
\poeml \v{31}Do not envy a violent man, \\
\poemll    and do not emulate his lifestyle.\fnote{Lit. \fbib{ways}} \\
\poeml \v{32}Indeed, a perverse man is utterly disgusting\fnote{Lit. \fbib{an abomination}} to the \divine{Lord}, \\
\poemll    but he takes the upright into his confidence.\fnote{Lit. \fbib{but his secret counsel is with the upright}} \\
\poeml \v{33}The \divine{Lord}'s curse is on the house of the wicked, \\
\poemll    but he blesses the dwelling of the righteous. \\
\poeml \v{34}Though God\fnote{Lit. \fbib{he}} scoffs at scoffers, \\
\poemll    he gives grace to the humble. \\
\poeml \v{35}The wise will inherit honor, \\
\poemll    but he holds fools up for ridicule.
\end{poetry}
\labelchapt{4}
\passage{Diligently Pursue Wisdom}

\begin{poetry}
\poeml \chapt{4}
\v{1}Listen, children,\fnote{Lit. \fbib{sons}} to your father's instruction, \\
\poeml and pay attention in order to gain understanding. \\
\poeml \v{2}I give you sound teaching, \\
\poeml so do not abandon my instruction.\fnote{Or \fbib{law}} \\
\poeml \v{3}When I was a son to my father, \\
\poemll    not yet strong\fnote{Lit. \fbib{delicate}} and an only son to my mother, \\
\poeml \v{4}he taught me and told me, \\
\poemll    ``Let your heart fully embrace what I have to say;\fnote{Lit. \fbib{embrace my words}} \\
\poemlll       keep my commandments and live! \\
\poeml \v{5}Get wisdom! Get understanding! \\
\poemll    Do not forget or turn aside from the words of my mouth! \\
\poeml \v{6}Do not abandon her, and she will protect you. \\
\poemll    Love her, and she will watch over you. \\
\poeml \v{7}Wisdom is of utmost importance, therefore get wisdom, \\
\poemll    and with all your effort work to acquire understanding. \\
\poeml \v{8}Prize her and she will exalt you. \\
\poemll    Indeed, if you embrace her, she will honor you. \\
\poeml \v{9}She will place on your head a graceful garland; \\
\poemll    she will present to you a crown of beauty.'' \\
\poeml \v{10}Listen, my son: accept my words, \\
\poemll    and you'll live a long, long time.\fnote{Lit. \fbib{and the years of your life will be many}} \\
\poeml \v{11}I have directed you in the way of wisdom, \\
\poemll    and I have led you along straight\fnote{Or \fbib{upright}} paths. \\
\poeml \v{12}When you walk, your step will not be hindered, \\
\poemll    and when you run, you will not stumble. \\
\poeml \v{13}Hold on to instruction, do not let it go! \\
\poemll    Guard wisdom,\fnote{Lit. \fbib{her}} because she is your life!
\passage{Avoiding the Ways of the Wicked}
\poeml \v{14}Do not enter the path of the wicked, \\
\poemll    or go along the way of evil men. \\
\poeml \v{15}Avoid it! Don't travel on it! \\
\poemll    Turn away from it, and pass on by. \\
\poeml \v{16}For they cannot sleep unless they are doing evil, \\
\poemll    and they are robbed of their sleep unless they cause someone to stumble. \\
\poeml \v{17}For they eat the bread of wickedness, \\
\poemll    and they drink the wine of violence. \\
\poeml \v{18}The path of the righteous is like the light of dawn \\
\poemll    that grows brighter until the full light of day. \\
\poeml \v{19}But the way of the wicked is like deep darkness, \\
\poemll    and they do not know what they are stumbling over.
\passage{Remembering the Counsel of a Wise Father}
\poeml \v{20}My son, pay attention to my words, \\
\poemll    and listen closely\fnote{Lit. \fbib{turn your ear}} to what I say. \\
\poeml \v{21}Do not let them out of your sight; \\
\poemll    keep them within your heart. \\
\poeml \v{22}For they are life to those who find them, \\
\poemll    and healing to their whole body.\fnote{Lit. \fbib{flesh}} \\
\poeml \v{23}Above everything else\fnote{Lit. \fbib{Above all watching}} guard your heart, \\
\poemll    because from it flow the springs of life. \\
\poeml \v{24}Never talk deceptively \\
\poemll    and don't keep company with people whose speech is corrupt.\fnote{Lit. \fbib{keep corrupt lips far from you}} \\
\poeml \v{25}Let your eyes look directly ahead; \\
\poemll    fix your gaze straight in front of you. \\
\poeml \v{26}Carefully measure\fnote{Lit. \fbib{Weigh}} the paths for your feet, \\
\poemll    and all your ways will be established. \\
\poeml \v{27}Do not turn to the right or to the left; \\
\poemll    turn your foot away from evil.
\end{poetry}
\labelchapt{5}
\passage{Warning against Sexual Immorality}

\begin{poetry}
\poeml \chapt{5}
\v{1}My son, pay attention to my wisdom, \\
\poeml and listen closely to my insight, \\
\poeml \v{2}so you may carefully practice\fnote{Lit. \fbib{guard}} discretion \\
\poemll    and your lips preserve knowledge. \\
\poeml \v{3}For the lips of an adulteress drip honey, \\
\poemll    and her speech\fnote{Lit. \fbib{palate}} is smoother than oil. \\
\poeml \v{4}But in the end she is as bitter as wormwood,\fnote{\fbib{Wormwood} is a plant with an extremely bitter taste.} \\
\poemll    and as sharp as a double-edged sword. \\
\poeml \v{5}Her feet go down to death; \\
\poemll    her steps lead to Sheol.\fnote{I.e. the realm of the dead} \\
\poeml \v{6}You aren't thinking about\fnote{Or \fbib{She does not consider}} where her life is headed; \\
\poemll    her steps wander, but you do not realize\fnote{Or \fbib{she does not realize}} it. \\
\poeml \v{7}Now, children,\fnote{Or \fbib{sons}} listen to me. \\
\poemll    Don't turn away from what I am saying.\fnote{Lit. \fbib{from the words of my mouth}} \\
\poeml \v{8}Keep\fnote{Lit. \fbib{Keep your path}} far away from her, \\
\poemll    and don't go near the entrance to her house, \\
\poeml \v{9}so that you don't give your honor to others, \\
\poemll    and waste your best years;\fnote{Lit. \fbib{and your years to the cruel}} \\
\poeml \v{10}so that strangers don't enrich themselves at your expense,\fnote{Lit. \fbib{don't satisfy themselves with your strength}} \\
\poemll    and your work won't end up the possession of foreigners.\fnote{Lit. \fbib{won't go into a foreigner's house}} \\
\poeml \v{11}You will cry out in anguish when your end comes, \\
\poemll    when your flesh and body are consumed, \\
\poeml \v{12}and you will say, ``How I hated instruction,\fnote{Or \fbib{discipline}} \\
\poemll    and my heart rejected correction! \\
\poeml \v{13}I did not obey my teachers \\
\poemll    and did not listen\fnote{Lit. \fbib{incline my ear}} to my instructors. \\
\poeml \v{14}Now I am at the point of utter disaster \\
\poemll    in\fnote{Lit. \fbib{in the midst of}} the assembly and in the congregation.''
\passage{The Delights of Marital Faithfulness}
\poeml \v{15}Drink water from your own cistern, \\
\poemll    and fresh\fnote{Lit. \fbib{flowing}} water from your own well. \\
\poeml \v{16}Should your springs flow outside, \\
\poemll    or streams of water in the street? \\
\poeml \v{17}They should be for you alone \\
\poemll    and not for strangers who are with you. \\
\poeml \v{18}Let your fountain be blessed \\
\poemll    and enjoy the wife of your youth. \\
\poeml \v{19}Like a loving deer, a beautiful doe, \\
\poemll    let her breasts satisfy you all the time. \\
\poemlll       Be constantly intoxicated by her love. \\
\poeml \v{20}Why should you be intoxicated by an adulteress, my son, \\
\poemll    and embrace the bosom of a foreign woman? \\
\poeml \v{21}Indeed, what a man does is\fnote{Lit. \fbib{Indeed, a man's ways are}} always in the \divine{Lord}'s presence,\fnote{Lit. \fbib{in front of the \divine{Lord}'s eyes}} \\
\poemll    and he weighs all his paths. \\
\poeml \v{22}The wicked person's iniquities will capture him, \\
\poemll    and he will be held with the cords of his sin. \\
\poeml \v{23}He will die for lack of discipline, \\
\poemll    and he goes astray because of his great folly.
\end{poetry}
\labelchapt{6}
\passage{The Folly of Guaranteeing Loans}

\begin{poetry}
\poeml \chapt{6}
\v{1}My son, if you guarantee a loan for your neighbor, \\
\poeml if you have agreed to a deal\fnote{Lit. \fbib{have clapped your hands}; i.e. have shaken hands} with a stranger, \\
\poeml \v{2}trapped by your own words, \\
\poemll    and caught by your own words, \\
\poeml \v{3}then do this, my son, and deliver yourself, \\
\poemll    because you have come under your neighbor's control.\fnote{Lit. \fbib{into the hands of your neighbor}} \\
\poeml Go, humble yourself! \\
\poemll    Plead passionately with your neighbor! \\
\poeml \v{4}Don't allow yourself to sleep \\
\poemll    or even to close your eyes. \\
\poeml \v{5}Deliver yourself like a gazelle from a hunter's hand,\fnote{So MT; LXX Syr Targ read \fbib{from the hunter}; or \fbib{a noose}} \\
\poemll    or like a bird from a fowler's hand.
\passage{The Folly of Laziness}
\poeml \v{6}Go to the ant, you lazy man! \\
\poemll    Observe its ways and become wise. \\
\poeml \v{7}It has no commander, \\
\poemll    officer, or ruler, \\
\poeml \v{8}but prepares its provisions in the summer \\
\poemll    and gathers its food in the harvest. \\
\poeml \v{9}How long will you lie down, lazy man? \\
\poemll    When will you get up from your sleep? \\
\poeml \v{10}A little sleep, a little slumber, \\
\poemll    a little folding of the hands to rest, \\
\poeml \v{11}and your poverty will come on you like a bandit \\
\poemll    and your desperation like an armed man.
\passage{The Folly of Causing Strife}
\poeml \v{12}A worthless man, a wicked man, \\
\poemll    goes around with devious speech, \\
\poeml \v{13}winking with his eyes, making signs\fnote{Lit. \fbib{scraping}} with\fnote{The Heb. lacks \fbib{with}} his feet, \\
\poemll    pointing with his fingers, \\
\poeml \v{14}planning evil with a perverse mind,\fnote{Or \fbib{heart}} \\
\poemll    continually stirring up discord. \\
\poeml \v{15}Therefore, disaster will overtake him suddenly. \\
\poemll    He will be broken in an instant, \\
\poemlll       and he will never recover.
\passage{What God Hates}
\poeml \v{16}Here are six things that the \divine{Lord} hates--- \\
\poemll    seven, in fact,\fnote{The Heb. lacks \fbib{in fact}} are detestable to him:\fnote{Lit. \fbib{to his soul}} \\
\poeml \v{17}Arrogant eyes, \\
\poemll    a lying tongue, \\
\poemlll       and hands shedding innocent blood; \\
\poeml \v{18}a heart crafting evil plans, \\
\poemll    feet running swiftly to wickedness, \\
\poeml \v{19}a false witness snorting lies, \\
\poemll    and someone sowing quarrels between brothers.
\passage{Parental Counsel about Immorality}
\poeml \v{20}Keep your father's commands, my son, \\
\poemll    and never forsake your mother's rules,\fnote{Or \fbib{laws}} \\
\poeml \v{21}by binding them to your heart continuously, \\
\poemll    fastening them around your neck. \\
\poeml \v{22}During your travels wisdom\fnote{Lit. \fbib{wisdom; i}.e. wisdom personified as a woman} will lead you; \\
\poemll    she will watch over you while you rest; \\
\poeml and when you are startled from your sleep, \\
\poemll    she will commune with you. \\
\poeml \v{23}Because the command is a lamp \\
\poemll    and the Law a light, \\
\poemlll       rebukes that discipline are a way of life--- \\
\poeml \v{24}to protect you from the evil\fnote{So MT; LXX reads \fbib{married}} woman, \\
\poemll    from the words of the seductive woman. \\
\poeml \v{25}Do not focus on her beauty in your mind, \\
\poemll    nor allow her to take you prisoner with her flirting eyes, \\
\poeml \v{26}because the price of a whore is a loaf of bread, \\
\poemll    and an adulterous woman stalks a man's precious life. \\
\poeml \v{27}Can a man scoop fire into his bosom \\
\poemll    without burning his clothes? \\
\poeml \v{28}Can a man walk on hot coals \\
\poemll    without scorching his feet? \\
\poeml \v{29}So also is it with someone who has sex with his neighbor's wife; \\
\poemll    anyone touching her will not remain unpunished. \\
\poeml \v{30}A thief isn't despised \\
\poemll    if he steals to meet his needs\fnote{Lit. \fbib{to refresh his soul}} when he is hungry, \\
\poeml \v{31}but when he is discovered, \\
\poemll    he must restore seven-fold, \\
\poemlll       forfeiting the entire value of his house. \\
\poeml \v{32}Whoever commits adultery with a woman is out of his mind; \\
\poemll    by doing so he corrupts his own soul. \\
\poeml \v{33}He will receive a beating and dishonor, \\
\poemll    and his shame won't disappear, \\
\poeml \v{34}because jealousy incites\fnote{The Heb. lacks \fbib{incites}} a strong man's rage, \\
\poemll    and he will show no mercy when it's time for revenge. \\
\poeml \v{35}He will not consider any payment, \\
\poemll    nor will he be willing to accept it,\fnote{The Heb. lacks \fbib{to accept it}} \\
\poemlll       no matter how large the bribe.
\end{poetry}
\labelchapt{7}
\passage{On Avoiding the Immoral Woman}

\begin{poetry}
\poeml \chapt{7}
\v{1}My son, guard what I say \\
\poeml and treasure my commands. \\
\poeml \v{2}Keep my commands and you'll live. \\
\poemll    Guard\fnote{The Heb. lacks \fbib{Guard}} my teaching as you do your eyesight. \\
\poeml \v{3}Strap them to your fingers \\
\poemll    and engrave them on the tablet of your heart. \\
\poeml \v{4}Say to wisdom, ``You're my sister!'' \\
\poemll    and call understanding your close relative, \\
\poeml \v{5}so they can keep you from an adulterous woman, \\
\poemll    from the immoral woman with her seductive words.
\passage{A Father's Warning}
\poeml \v{6}For from a window in my house \\
\poemll    I peered through the lattice work, \\
\poeml \v{7}and I noticed among the na\"{i}ve--- \\
\poemll    that is, I discerned among the youths--- \\
\poemlll       a senseless young man. \\
\poeml \v{8}Proceeding down the street near her corner, \\
\poemll    he makes his way toward her house \\
\poeml \v{9}at twilight, during the evening, \\
\poemll    even during the darkest part of the night. \\
\poeml \v{10}Look! A woman makes her way to meet him, \\
\poemll    dressed as a prostitute \\
\poemlll       and intending to entrap him. \\
\poeml \v{11}She is brazen and defiant--- \\
\poemll    her feet don't remain at home. \\
\poeml \v{12}Now she is in the street, now in the plazas, \\
\poemll    she lurks near every corner. \\
\poeml \v{13}So she grabs hold of him and kisses him, \\
\poemll    with a brazen face she speaks to him, \\
\poeml \v{14}``I have given\fnote{The Heb. lacks \fbib{given}} my peace offerings, \\
\poemll    and today I fulfilled my vows. \\
\poeml \v{15}Therefore, I've come out to meet you, \\
\poemll    I've looked just for you, \\
\poemlll       and I found you! \\
\poeml \v{16}I've decorated my bed with new coverings--- \\
\poemll    embroidered linen from Egypt. \\
\poeml \v{17}I've perfumed my bed \\
\poemll    with myrrh, aloes, and cinnamon. \\
\poeml \v{18}Come, let's make love until dawn; \\
\poemll    let's comfort ourselves with love, \\
\poeml \v{19}because my husband isn't home. \\
\poemll    He left on a long trip. \\
\poeml \v{20}He took a fist full of cash \\
\poemll    and he'll return home in a month.'' \\
\poeml \v{21}She leads him astray with great persuasion; \\
\poemll    with flattering lips she seduces him. \\
\poeml \v{22}All of a sudden he follows her \\
\poemll    like an ox fit for slaughter \\
\poemlll       or like a fool fit for a trap\fnote{So MT; LXX reads \fbib{a dog fit for chains}} \\
\poeml \v{23}until an arrow pierces his liver. \\
\poemll    As a bird darts into a snare, \\
\poemlll       he doesn't realize his fatal decision.\fnote{Lit. \fbib{realize it is his life}} \\
\poeml \v{24}So listen to me, my sons, \\
\poemll    and pay attention to what I have to say. \\
\poeml \v{25}Don't be led astray by her lifestyle,\fnote{Lit. \fbib{ways}} \\
\poemll    and don't imitate her behavior.\fnote{Lit. \fbib{paths}} \\
\poeml \v{26}For many are the victims whom she has conquered, \\
\poemll    and many are her slain. \\
\poeml \v{27}Her house leads to Sheol,\fnote{I.e. the realm of the dead} \\
\poemll    descending to death's catacombs.
\end{poetry}
\labelchapt{8}
\passage{Wisdom Calls for an Audience}

\begin{poetry}
\poeml \chapt{8}
\v{1}Isn't wisdom calling out; \\
\poeml isn't understanding raising her voice? \\
\poeml \v{2}On top of the highest places along the road \\
\poemll    she stands where the roads meet. \\
\poeml \v{3}Beside the gates, at the city entrance--- \\
\poemll    at the entrance to the portals she cries aloud: \\
\poeml \v{4}``I'm calling to you, men! \\
\poemll    What I have to say pertains\fnote{Lit. \fbib{My voice is}} to all mankind! \\
\poeml \v{5}Understand prudence, you na\"{i}ve people; \\
\poemll    and gain an understanding heart, you foolish ones. \\
\poeml \v{6}Listen, because I have noble things to say, \\
\poemll    and what I have to say\fnote{Lit. \fbib{my open lips}} will reveal what is right. \\
\poeml \v{7}For my mouth speaks the truth--- \\
\poemll    wickedness is detestable to me. \\
\poeml \v{8}Everything I have to say is just; \\
\poemll    there isn't anything corrupt or perverse in my speech.\fnote{Lit. \fbib{words}} \\
\poeml \v{9}Everything I say is sensible to someone who understands, \\
\poemll    and correct to those who have acquired knowledge. \\
\poeml \v{10}Grab hold of my instruction in lieu of money \\
\poemll    and knowledge instead of the finest gold, \\
\poeml \v{11}because wisdom is better than precious gems\fnote{Or \fbib{rubies}} \\
\poemll    and nothing you desire can compare to it.''
\passage{The Way of Wisdom}
\poeml \v{12}``I, wisdom, am related to\fnote{Lit. \fbib{wisdom, live with}} prudence. \\
\poemll    I know how to be discreet. \\
\poeml \v{13}The fear of the \divine{Lord} is to hate evil. \\
\poemll    Pride, arrogance, an evil lifestyle, \\
\poemlll       and perverted speech I despise. \\
\poeml \v{14}Counsel belongs to me, \\
\poemll    along with sound judgment. \\
\poeml I am understanding. \\
\poemll    Power belongs to me. \\
\poeml \v{15}Kings reign by me, \\
\poemll    and rulers dispense justice through me. \\
\poeml \v{16}By me leaders rule, as do noble officials \\
\poemll    and all who govern justly.\fnote{So MT; LXX reads \fbib{and tyrants rule the earth}} \\
\poeml \v{17}I love those who love me, \\
\poemll    and those who seek me will find me. \\
\poeml \v{18}Wealth and honor accompany me, \\
\poemll    as do enduring wealth and righteousness. \\
\poeml \v{19}My fruit is better than gold, \\
\poemll    better\fnote{The Heb. lacks \fbib{better}} than even refined gold, \\
\poemlll       and my benefit surpasses the purest silver. \\
\poeml \v{20}I walk on the way of righteousness, \\
\poemll    along paths that are just, \\
\poeml \v{21}I bequeath wealth to those who love me, \\
\poemll    and I will fill their treasuries.''
\passage{The Agelessness of Wisdom}
\poeml \v{22}``The \divine{Lord} made me as he began his planning,\fnote{Lit. \fbib{ways}} \\
\poemll    before his ancient activity commenced. \\
\poeml \v{23}From eternity I was appointed, \\
\poemll    from the beginning, \\
\poemlll       from before there was land. \\
\poeml \v{24}When there were no ocean depths, \\
\poemll    I brought them\fnote{The Heb. lacks \fbib{them}} to birth \\
\poemlll       at a time when there were no springs. \\
\poeml \v{25}Before the mountains were shaped, \\
\poemll    before there were hills, \\
\poemlll       I was bringing them\fnote{The Heb. lacks \fbib{them}} to birth. \\
\poeml \v{26}Even though he had not made the earth, nor the fields, \\
\poemll    nor the world's first grains of dust, \\
\poeml \v{27}when he crafted the heavens, \\
\poemll    I was there--- \\
\poemlll       when he marked out a circle on the face of the deep, \\
\poeml \v{28}when he made the clouds from above, \\
\poemll    when the springs of the depths were established, \\
\poeml \v{29}when he set a boundary for the sea \\
\poemll    so the waters would not exceed his limits,\fnote{Lit. \fbib{command}} \\
\poemlll       when he marked out the foundations of the earth. \\
\poeml \v{30}Then I was with him, his master craftsman--- \\
\poemll    I was his delight\fnote{So LXX; MT reads \fbib{was filled with delight}} daily, \\
\poemlll       continuously rejoicing in his presence, \\
\poeml \v{31}rejoicing in his inhabitable world \\
\poemll    and taking delight in mankind.''
\passage{The Exhortation of Wisdom}
\poeml \v{32}``So listen to me, children! \\
\poemll    Blessed are those who obey me. \\
\poeml \v{33}Listen to instruction and be wise. \\
\poemll    Don't ignore it. \\
\poeml \v{34}Blessed is the person who listens to me, \\
\poemll    watching daily at my gates, \\
\poemlll       waiting at my doorways--- \\
\poeml \v{35}because those who find me find life \\
\poemll    and gain favor from the \divine{Lord}. \\
\poeml \v{36}But whoever sins against me destroys himself; \\
\poemll    everyone who hates me loves death.''
\end{poetry}
\labelchapt{9}
\passage{Wisdom's Invitation}

\begin{poetry}
\poeml \chapt{9}
\v{1}Wisdom\fnote{I.e. \fbib{wisdom} personified as a woman} has built her house; \\
\poeml she has hewn out her seven pillars. \\
\poeml \v{2}She has prepared her food,\fnote{Or \fbib{meat}} \\
\poemll    she has spiced\fnote{Or \fbib{mixed}} her wine, \\
\poemlll       and she also has set her dining table. \\
\poeml \v{3}She has sent out her young women, \\
\poemll    while calling out from the heights of the city, \\
\poeml \v{4}``Let whoever is na\"{i}ve, turn in here.'' \\
\poemll    To anyone lacking sense, she says, \\
\poeml \v{5}``Come! Eat my food, \\
\poemll    and drink the wine that I have mixed. \\
\poeml \v{6}Leave your na\"{i}ve ways, and live. \\
\poemll    Walk in the path of understanding.''
\passage{Wisdom Extends Life}
\poeml \v{7}Whoever corrects a mocker invites only insult,\fnote{Lit. \fbib{insult to himself}} \\
\poemll    and whoever rebukes the wicked will himself become stained. \\
\poeml \v{8}Don't rebuke a mocker or he will hate you. \\
\poemll    Rebuke a wise person, and he will love you. \\
\poeml \v{9}Counsel a wise man, \\
\poemll    and he will be wiser still; \\
\poeml teach a righteous man, \\
\poemll    and he will add to his learning. \\
\poeml \v{10}The fear of the \divine{Lord} is where wisdom begins, \\
\poemll    and knowing holiness\fnote{Or \fbib{knowing holy ones}} demonstrates understanding. \\
\poeml \v{11}For because of me you will live a long life, \\
\poemll    and years will be added to your life. \\
\poeml \v{12}If you are wise, \\
\poemll    your wisdom will assist you. \\
\poeml If you mock, \\
\poemll    you alone will be held responsible.
\passage{Folly's Entrapment}
\poeml \v{13}The foolish woman is loud, \\
\poemll    undisciplined, and without knowledge. \\
\poeml \v{14}She sits at the entrance of her house, \\
\poemll    on a seat high above the city. \\
\poeml \v{15}She calls out to those passing by on the road, \\
\poemll    who are minding their own business,\fnote{Or \fbib{are going straight on their way}} \\
\poeml \v{16}``Whoever is na\"{i}ve, turn in here!'' \\
\poemll    And to anyone lacking sense, she says, \\
\poeml \v{17}``Stolen waters are sweet, \\
\poemll    and food eaten in secret is delicious.'' \\
\poeml \v{18}But he does not realize that the dead lurk there, \\
\poemll    and her invited guests wind up in the depths of Sheol.\fnote{I.e. the realm of the dead}
\end{poetry}
\labelchapt{10}
\passage{Solomon's Sayings}

\chapt{10}
\v{1}The proverbs of Solomon.

\begin{poetry}
\poeml A wise son brings joy to his father, \\
\poemll    but a foolish son grieves his mother. \\
\poeml \v{2}Nothing good comes from ill-gotten wealth, \\
\poemll    but righteousness delivers from death. \\
\poeml \v{3}The \divine{Lord} won't cause the righteous to hunger, \\
\poemll    but he will reject what the wicked crave. \\
\poeml \v{4}Lazy hands bring poverty, \\
\poemll    but hard-working hands lead to wealth. \\
\poeml \v{5}Whoever harvests during summer acts wisely, \\
\poemll    but the son who sleeps during harvest is disgraceful.
\passage{The Righteous and Wicked Compared}
\poeml \v{6}Blessings come\fnote{The Heb. lacks \fbib{come}} upon the head of the righteous, \\
\poemll    but the words\fnote{Lit. \fbib{mouth}} of the wicked conceal violence. \\
\poeml \v{7}The reputation\fnote{Lit. \fbib{memorial}} of the righteous leads to blessing, \\
\poemll    but the name of the wicked will rot. \\
\poeml \v{8}The wise person\fnote{Lit. \fbib{wise in heart}} accepts commands, \\
\poemll    but the chattering fool will be brought down. \\
\poeml \v{9}Whoever walks in integrity lives prudently,\fnote{Lit. \fbib{lives in safety}} \\
\poemll    but whoever perverts his way of life will be exposed. \\
\poeml \v{10}Those who wink their eyes\fnote{I.e. Those whose looks communicate insincerity} are trouble makers, \\
\poemll    and the mocking fool will be brought down.\fnote{So MT; LXX reads \fbib{makers, but the one who reproves publicly makes peace}} \\
\poeml \v{11}What the righteous say\fnote{Lit. \fbib{The mouth of the righteous}} is a flowing fountain,\fnote{Lit. \fbib{a fountain of life}} \\
\poemll    but what the wicked say\fnote{Lit. \fbib{but the mouth of the wicked}} conceals violence. \\
\poeml \v{12}Hatred awakens contention, \\
\poemll    but love covers all transgressions. \\
\poeml \v{13}Wisdom characterizes the speech\fnote{Lit. \fbib{Wisdom is found on the lips}} of the discerning, \\
\poemll    but the rod is for the backs of those lacking discernment. \\
\poeml \v{14}Those who are wise store up knowledge, \\
\poemll    but when the fool speaks,\fnote{Lit. \fbib{but the mouth of the fool}} destruction is near. \\
\poeml \v{15}The rich hide within the fortress that is their wealth, \\
\poemll    but the poor are dismayed due to their poverty. \\
\poeml \v{16}Honorable wages lead\fnote{The Heb. lacks \fbib{lead}} to life; \\
\poemll    the salaries of the wicked, to retribution. \\
\poeml \v{17}Whoever heeds correction is on the pathway to life, \\
\poemll    but someone who ignores exhortation goes astray. \\
\poeml \v{18}Whoever conceals hatred is a deceitful liar, \\
\poemll    and whoever spreads slander is a fool. \\
\poeml \v{19}Transgression is at work where people talk too much, \\
\poemll    but anyone who holds his tongue is prudent. \\
\poeml \v{20}What the righteous person says\fnote{Lit. \fbib{The tongue of the righteous}} is like precious silver; \\
\poemll    the thoughts of the wicked are compared to small things. \\
\poeml \v{21}What the righteous person says\fnote{Lit. \fbib{The lips of the righteous}} nourishes many, \\
\poemll    but fools die because they lack discerning\fnote{The Heb. lacks \fbib{discerning}} hearts. \\
\poeml \v{22}The blessing of the \divine{Lord} establishes wealth, \\
\poemll    and difficulty does not accompany it. \\
\poeml \v{23}Just as the fool considers wickedness his joy, \\
\poemll    so is wisdom to the discerning man. \\
\poeml \v{24}What the wicked fears will come about, \\
\poemll    but the longing of the righteous will be granted. \\
\poeml \v{25}When the storm ends, the wicked vanish,\fnote{Lit. \fbib{wicked are no more}} \\
\poemll    but the righteous person is forever firm. \\
\poeml \v{26}As vinegar is to the mouth\fnote{Lit. \fbib{teeth}} and smoke to the eyes, \\
\poemll    so is the lazy person to those who send him. \\
\poeml \v{27}Fearing the \divine{Lord} prolongs life, \\
\poemll    but the wicked will not live long. \\
\poeml \v{28}What the righteous hope for brings joy, \\
\poemll    but the expectation of the wicked dies. \\
\poeml \v{29}To the upright, the way of the \divine{Lord} is a place of safety, \\
\poemll    but it's a place of ruin to those who practice evil. \\
\poeml \v{30}The righteous will never be overthrown, \\
\poemll    but the wicked will never inhabit the land. \\
\poeml \v{31}The words of the righteous overflow with wisdom, \\
\poemll    but the perverse tongue will be cut out. \\
\poeml \v{32}Righteous lips know what is prudent, \\
\poemll    but the words of the wicked are perverse.
\end{poetry}
\labelchapt{11}
\passage{The Value of Righteousness}

\begin{poetry}
\poeml \chapt{11}
\v{1}The \divine{Lord} hates false scales, \\
\poeml but he delights in accurate weights. \\
\poeml \v{2}When pride appears, disgrace accompanies it, \\
\poemll    but humility is present with wisdom. \\
\poeml \v{3}The integrity of the righteous guides them, \\
\poemll    but the hypocrisy of the treacherous destroys them. \\
\poeml \v{4}Wealth won't help in the time of judgment,\fnote{Lit. \fbib{the day of wrath}} \\
\poemll    but righteousness will deliver from death. \\
\poeml \v{5}The righteousness of the innocent creates a level path, \\
\poemll    but the wicked fall by their wickedness. \\
\poeml \v{6}The righteousness of the upright delivers them, \\
\poemll    but the treacherous are trapped by their evil desires. \\
\poeml \v{7}When a wicked person dies, his hope vanishes;\fnote{So MT; LXX reads \fbib{When a righteous man dies, his hope does not perish}} \\
\poemll    and what he\fnote{So MT; LXX reads \fbib{what the ungodly}} expected from his scheming comes to nothing. \\
\poeml \v{8}The righteous person is delivered from trouble; \\
\poemll    it comes upon the wicked instead. \\
\poeml \v{9}By what he says, the godless person can destroy his neighbor, \\
\poemll    but through knowledge the righteous escape. \\
\poeml \v{10}The city rejoices when the righteous prosper, \\
\poemll    and when the wicked perish there is jubilation. \\
\poeml \v{11}Through the blessing of the righteous a city is built up, \\
\poemll    but what the wicked say tears it down. \\
\poeml \v{12}Whoever belittles his neighbor lacks sense, \\
\poemll    but the discerning man controls his comments. \\
\poeml \v{13}Whoever spreads gossip betrays secrets, \\
\poemll    but the trustworthy person\fnote{Lit. \fbib{trustworthy in spirit}} keeps a confidence. \\
\poeml \v{14}A nation falls through a lack of guidance, \\
\poemll    but victory comes through the counsel of many.\fnote{Or \fbib{through much planning}} \\
\poeml \v{15}Securing a loan for a stranger will bring suffering, \\
\poemll    but by refusing to do so, one remains safe. \\
\poeml \v{16}A gracious woman attains honor,\fnote{So MT; LXX reads \fbib{honor for her husband}} \\
\poemll    but\fnote{So MT; LXX reads \fbib{but a seat of dishonor is for the woman who hates justice}} ruthless men attain\fnote{So MT; LXX reads \fbib{justice. The deficient shrink from wealth, but the diligent support themselves with}} wealth. \\
\poeml \v{17}A gracious man benefits himself, \\
\poemll    but the cruel person damages himself. \\
\poeml \v{18}Evil people earn deceptive wages, \\
\poemll    but those who plant righteousness are truly rewarded. \\
\poeml \v{19}Genuine righteousness leads to life, \\
\poemll    but whoever pursues evil will die. \\
\poeml \v{20}Devious minds are abhorrent to the \divine{Lord}, \\
\poemll    but those whose ways are innocent are his delight. \\
\poeml \v{21}Be sure of this:\fnote{Lit. \fbib{Hand to hand}} the wicked will not go unpunished, \\
\poemll    but the descendants of the righteous will go free. \\
\poeml \v{22}Like a gold ring in a pig's snout \\
\poemll    is a beautiful woman without discretion. \\
\poeml \v{23}The desire of the righteous is to seek good, \\
\poemll    but the hope of the wicked results in wrath. \\
\poeml \v{24}Those who give freely gain even more; \\
\poemll    others hold back what they owe, becoming even poorer. \\
\poeml \v{25}A generous person will prosper, \\
\poemll    and anyone who gives water will receive a flood in return. \\
\poeml \v{26}People will curse whoever withholds grain, \\
\poemll    but blessing will come to whoever is selling. \\
\poeml \v{27}The person seeking good will find favor, \\
\poemll    but anyone who searches for evil---it will find him! \\
\poeml \v{28}The person who trusts in his wealth will fall, \\
\poemll    but the righteous will flourish like green leaves. \\
\poeml \v{29}Whoever troubles his household will inherit the wind, \\
\poemll    and the fool will be a servant to the wise. \\
\poeml \v{30}The fruit of the righteous is\fnote{So MT; LXX reads \fbib{From the fruit of righteousness grows}} a tree of life, \\
\poemll    and the one who wins people is wise.\fnote{So MT; LXX reads \fbib{life, but the souls of those who practice evil are cut off prematurely}} \\
\poeml \v{31}If the righteous receive what they are due here on earth, \\
\poemll    how much more will the wicked and the sinner.
\end{poetry}
\labelchapt{12}
\passage{Wisdom and Wickedness Contrasted}

\begin{poetry}
\poeml \chapt{12}
\v{1}The person who loves correction loves knowledge, \\
\poeml but anyone who hates a rebuke is stupid. \\
\poeml \v{2}The good person will gain favor from the \divine{Lord}, \\
\poemll    but the man who plots evil will be condemned by him. \\
\poeml \v{3}A person doesn't gain security by wickedness, \\
\poemll    but the righteous won't be uprooted. \\
\poeml \v{4}A virtuous woman is a crown to her husband, \\
\poemll    but a wife\fnote{Lit. \fbib{but she}} who puts him to shame is like bone cancer.\fnote{Lit. \fbib{decay}} \\
\poeml \v{5}The plans of the righteous are just, \\
\poemll    but the advice of the wicked is deceitful. \\
\poeml \v{6}The words of the wicked lead to\fnote{Lit. \fbib{wicked lie in wait for}} bloodshed, \\
\poemll    but the speech of the upright delivers them. \\
\poeml \v{7}After they're overthrown, the wicked won't be found, \\
\poemll    but the house of the righteous stands firm. \\
\poeml \v{8}A man is praised because of his wise words, \\
\poemll    but the perverted mind\fnote{Lit. \fbib{heart}} will be despised. \\
\poeml \v{9}It's better to be unimportant, yet have a servant, \\
\poemll    than to pretend to be important, but lack food. \\
\poeml \v{10}The righteous person looks out for the welfare of his livestock, \\
\poemll    but even\fnote{The Heb. lacks \fbib{even}} the compassion of the wicked is cruel. \\
\poeml \v{11}Whoever tills his soil will have a lot to eat, \\
\poemll    but anyone who pursues fantasies lacks sense.\fnote{Lit. \fbib{heart}} \\
\poeml \v{12}The wicked desires what evil people gain, \\
\poemll    but the foundation\fnote{Or \fbib{root}} of the righteous is productive. \\
\poeml \v{13}An evil man's sinful speech ensnares him, \\
\poemll    but the righteous person escapes from trouble. \\
\poeml \v{14}By his fruitful speech a man can remain satisfied, \\
\poemll    and a man's handiwork will reward him. \\
\poeml \v{15}The lifestyle of the fool is right in his own opinion, \\
\poemll    but wise is the man who listens to advice. \\
\poeml \v{16}The anger of a fool becomes readily apparent, \\
\poemll    but the prudent person overlooks an insult. \\
\poeml \v{17}The truth teller speaks what is right, \\
\poemll    but the false witness speaks what is\fnote{The Heb. lacks \fbib{speaks what is}} deceitful. \\
\poeml \v{18}Some speak rashly like the cutting of a sword, \\
\poemll    but what the wise say promotes healing. \\
\poeml \v{19}A truthful saying\fnote{Lit. \fbib{lips}} is trusted forever, \\
\poemll    but the liar\fnote{Lit. \fbib{the lying tongue}} only for a moment. \\
\poeml \v{20}Deceit is at home\fnote{The Heb. lacks \fbib{at home}} in the heart of those who plan evil, \\
\poemll    but those who promote peace rejoice. \\
\poeml \v{21}No harm overwhelms the righteous, \\
\poemll    but the wicked overflow with trouble. \\
\poeml \v{22}Deceitful speech is reprehensible to the \divine{Lord}, \\
\poemll    but those who act faithfully are his delight. \\
\poeml \v{23}A prudent man keeps what he knows to himself,\fnote{The Heb. lacks \fbib{to himself}} \\
\poemll    but the hearts of fools shout forth their foolishness. \\
\poeml \v{24}The diligent will take control, \\
\poemll    but the lazy will be put to forced labor. \\
\poeml \v{25}A person's anxiety weighs down his heart, \\
\poemll    but an appropriate word is encouraging. \\
\poeml \v{26}The righteous person is cautious with respect to his neighbor, \\
\poemll    but the lifestyle of the wicked leads them astray. \\
\poeml \v{27}The lazy person does not roast what he has hunted, \\
\poemll    but diligence is one's most important possession. \\
\poeml \v{28}In the pathway to righteousness there is life, \\
\poemll    and in that lifestyle there is no death.
\end{poetry}
\labelchapt{13}
\passage{Who is a Wise Son?}

\begin{poetry}
\poeml \chapt{13}
\v{1}A wise son heeds\fnote{The Heb. lacks \fbib{heeds}} a father's correction, \\
\poeml but a mocker does not listen to rebuke. \\
\poeml \v{2}From the fruit of his words a man receives benefit,\fnote{Lit. \fbib{man eats good things}} \\
\poemll    but the treacherous crave violence. \\
\poeml \v{3}Anyone who guards his words protects his life; \\
\poemll    anyone who talks too much\fnote{Lit. \fbib{who opens wide his lips}} is ruined. \\
\poeml \v{4}The lazy person craves, yet receives nothing, \\
\poemll    but the desires of the diligent are satisfied. \\
\poeml \v{5}A righteous person hates deceit, \\
\poemll    but the wicked person is shameful and disgraceful. \\
\poeml \v{6}Righteousness protects the blameless, \\
\poemll    but wickedness brings down\fnote{So MT DSS 4QProv\textsuperscript{b}; LXX reads \fbib{but sins ruin the wicked}} the sinner. \\
\poeml \v{7}One person pretends to be wealthy, but has nothing; \\
\poemll    another pretends to be poor, yet is rich. \\
\poeml \v{8}The life of a wealthy man may be held for ransom, \\
\poemll    but whoever is poor receives no threats. \\
\poeml \v{9}The light of the righteous shines, \\
\poemll    but the lamp of the wicked is extinguished. \\
\poeml \v{10}Arrogance only brings quarreling, \\
\poemll    but those receiving advice are wise. \\
\poeml \v{11}Wealth gained dishonestly dwindles away, \\
\poemll    but whoever works diligently increases his prosperity.\fnote{The Heb. lacks \fbib{his prosperity}} \\
\poeml \v{12}Delayed hope makes the heart ill, \\
\poemll    but fulfilled longing is a tree of life. \\
\poeml \v{13}Anyone who despises a word of advice will pay for it, \\
\poemll    but whoever heeds a command will be rewarded. \\
\poeml \v{14}What the wise have to teach is a fountain of life \\
\poemll    and causes someone to avoid the snares of death. \\
\poeml \v{15}Good understanding produces grace, \\
\poemll    but the lifestyle of the treacherous never changes.\fnote{So MT; LXX Syr read \fbib{grace, and to know the Law is the sign of a sound mind, but the path of scorners ends in destruction}} \\
\poeml \v{16}Every sensible person acts from knowledge, \\
\poemll    but a fool demonstrates folly. \\
\poeml \v{17}An evil messenger stumbles into trouble, \\
\poemll    but a faithful envoy brings healing. \\
\poeml \v{18}Poverty and shame are for those who ignore correction, \\
\poemll    but whoever listens to instruction gains honor. \\
\poeml \v{19}Fulfilled longing is sweet to the soul, \\
\poemll    but avoiding evil is detestable to the fool. \\
\poeml \v{20}Whoever keeps company with the wise becomes wise, \\
\poemll    but the companion of fools suffers harm. \\
\poeml \v{21}Disaster pursues the sinful, \\
\poemll    but good will reward the righteous. \\
\poeml \v{22}A good person leaves an inheritance to his grandchildren, \\
\poemll    but the wealth of the wicked is reserved for the righteous. \\
\poeml \v{23}The field of the poor may produce much food, \\
\poemll    but it can be swept away through injustice. \\
\poeml \v{24}Whoever does not discipline\fnote{Lit. \fbib{Whoever spares the rod}} his son hates him, \\
\poemll    but whoever loves him is diligent to correct him. \\
\poeml \v{25}A righteous person eats to his heart's content, \\
\poemll    but the stomach of the wicked remains hungry.
\end{poetry}
\labelchapt{14}
\passage{How Wise People Live}

\begin{poetry}
\poeml \chapt{14}
\v{1}Every wise woman builds up her household, \\
\poeml but the foolish one tears it down with her own hands. \\
\poeml \v{2}Someone whose conduct is upright fears the \divine{Lord}, \\
\poemll    but whoever is devious in his ways despises him. \\
\poeml \v{3}What a fool says brings\fnote{Lit. \fbib{The mouth of the fool}} a rod to his back, \\
\poemll    but the words of the wise protect them. \\
\poeml \v{4}Where there are no oxen, the feeding trough is clean, \\
\poemll    but profits come through the strength of the ox. \\
\poeml \v{5}A trustworthy witness does not deceive, \\
\poemll    but a false witness spews lies. \\
\poeml \v{6}A mocker seeks wisdom and finds\fnote{The Heb. lacks \fbib{finds}} none, \\
\poemll    but learning comes easily to someone who understands. \\
\poeml \v{7}Stay away from a foolish man, \\
\poemll    for you will not find competent advice. \\
\poeml \v{8}The wisdom of the prudent helps him know how to live, \\
\poemll    but a fool's stupidity deceives him. \\
\poeml \v{9}Fools make fun of guilt, \\
\poemll    but among the upright there are good intentions. \\
\poeml \v{10}The heart knows its own bitterness--- \\
\poemll    an outsider cannot share in its joy. \\
\poeml \v{11}The house of the wicked will be destroyed, \\
\poemll    but the tent of the upright will flourish. \\
\poeml \v{12}There is a pathway that seems right to a man, \\
\poemll    but in the end it's a road to death. \\
\poeml \v{13}Even in laughter there may be heartache, \\
\poemll    and at the end of joy there may be grief. \\
\poeml \v{14}The faithless one will pay for his behavior,\fnote{Lit. \fbib{ways}} \\
\poemll    but a good man will be rewarded\fnote{The Heb. lacks \fbib{will be rewarded}} for his. \\
\poeml \v{15}An unthinking person believes everything, \\
\poemll    but the prudent one thinks before acting.\fnote{Lit. \fbib{one considers his steps}} \\
\poeml \v{16}The wise person fears and turns away from evil, \\
\poemll    but a fool is reckless and overconfident. \\
\poeml \v{17}A quick tempered person does foolish things, \\
\poemll    and a devious man is hated. \\
\poeml \v{18}The na\"{i}ve inherit folly, \\
\poemll    but the careful are crowned with knowledge. \\
\poeml \v{19}Evil men will bow down in the presence of good men \\
\poemll    and the wicked at the gates of the righteous. \\
\poeml \v{20}The poor person is shunned by his neighbor, \\
\poemll    but many are the friends of the wealthy. \\
\poeml \v{21}Whoever despises his neighbor sins, \\
\poemll    but whoever shows kindness to the poor will be happy. \\
\poeml \v{22}Won't those who plot evil go astray? \\
\poemll    But gracious love and truth are for those who plan what is good. \\
\poeml \v{23}In hard work there is always profit, \\
\poemll    but too much chattering\fnote{Lit. \fbib{word of lips}} leads to poverty. \\
\poeml \v{24}The crown of the wise is their wealth, \\
\poemll    but the stupidity of fools is just that---stupidity! \\
\poeml \v{25}A truthful witness saves lives, \\
\poemll    but the person who lies is deceitful. \\
\poeml \v{26}Rock-solid security is found\fnote{The Heb. lacks \fbib{is found}} in the fear of the \divine{Lord}, \\
\poemll    and within it one's children find refuge. \\
\poeml \v{27}The fear of the \divine{Lord} is a fountain of life, \\
\poemll    enabling anyone to escape the snares of death. \\
\poeml \v{28}A large population is a king's glory, \\
\poemll    but a shortage of people is a ruler's ruin. \\
\poeml \v{29}Being slow to get angry compares to great understanding \\
\poemll    as being quick-tempered compares to stupidity. \\
\poeml \v{30}A tranquil mind brings life to one's body, \\
\poemll    but jealousy causes one's bones to rot. \\
\poeml \v{31}Whoever oppresses the poor defies their Creator, \\
\poemll    but whoever is kind to the needy honors them. \\
\poeml \v{32}The wicked person is thrown down by his own wrongdoing, \\
\poemll    but the righteous person has a place of safety in death.\fnote{So MT DSS 4QProv\textsuperscript{b}; LXX reads \fbib{in his own piety}} \\
\poeml \v{33}Wisdom is at rest in the mind of the discerning--- \\
\poemll    even fools know this.\fnote{So MT; LXX reads \fbib{but in the heart of fools it is not discerned}} \\
\poeml \v{34}Righteousness makes a nation great, \\
\poemll    but sin diminishes\fnote{So DSS 4QPro\textsuperscript{b} LXX; MT reads \fbib{sin is a disgrace to}} any people. \\
\poeml \v{35}The king approves the wise servant, \\
\poemll    but he is angry at anyone who acts shamefully.
\end{poetry}
\labelchapt{15}
\passage{How to Live Wisely}

\begin{poetry}
\poeml \chapt{15}
\v{1}A gentle response diverts anger, \\
\poeml but a harsh statement incites fury. \\
\poeml \v{2}The wise speak, presenting\fnote{Lit. \fbib{The tongues of the wise present}} knowledge appropriately, \\
\poemll    but fools spout foolishness. \\
\poeml \v{3}The eyes of the \divine{Lord} are in every place, \\
\poemll    observing both the evil and the good. \\
\poeml \v{4}A gentle statement\fnote{Lit. \fbib{tongue}} is a tree of life, \\
\poemll    but perverted speech shatters the spirit. \\
\poeml \v{5}A fool rejects his father's instructions, \\
\poemll    but anyone who respects\fnote{Lit. \fbib{keeps}} reproof acts sensibly. \\
\poeml \v{6}The righteous house is itself\fnote{The Heb. lacks \fbib{itself}} a great treasure, \\
\poemll    but within the revenue of the wicked calamity is at work. \\
\poeml \v{7}What the wise have to say disseminates\fnote{Lit. \fbib{The lips of the wise spread}} knowledge, \\
\poemll    but it's not in the heart of fools to do so. \\
\poeml \v{8}The sacrifice of the wicked is detestable to the \divine{Lord}, \\
\poemll    but the prayer of the upright is his delight. \\
\poeml \v{9}The lifestyle of the wicked is detestable to the \divine{Lord}, \\
\poemll    but he loves those who ardently pursue righteousness. \\
\poeml \v{10}Severe punishment awaits anyone who wanders off the path--- \\
\poemll    anyone who despises reproof will die. \\
\poeml \v{11}Since Sheol\fnote{I.e. the realm of the dead} and Abaddon\fnote{I.e. the realm of destruction in the afterlife} lie open in the \divine{Lord}'s presence, \\
\poemll    how much more the hearts of human beings! \\
\poeml \v{12}The arrogant mocker never loves the one who corrects him; \\
\poemll    he will not inquire of\fnote{Lit. \fbib{not go to}} the wise. \\
\poeml \v{13}A happy heart enlightens the face, \\
\poemll    but a sad heart reflects a broken spirit. \\
\poeml \v{14}A discerning mind seeks knowledge, \\
\poemll    but the mouth of fools feeds on stupidity. \\
\poeml \v{15}The entire life\fnote{Lit. \fbib{All the days}} of the afflicted seems disastrous, \\
\poemll    but a good heart feasts continuously.
\passage{On Contentment and Other Good Things of Life}
\poeml \v{16}Better is a little accompanied by fear of the \divine{Lord} \\
\poemll    than abundant wealth with turmoil. \\
\poeml \v{17}A vegetarian meal\fnote{Lit. \fbib{A meal of herbs}} served with love is better \\
\poemll    than a big, thick steak\fnote{Lit. \fbib{a fattened ox}} with a plateful of\fnote{The Heb. lacks \fbib{a plateful of}} animosity. \\
\poeml \v{18}The quickly angered man stirs up contention, \\
\poemll    but anyone who controls his temper calms a dispute. \\
\poeml \v{19}The lifestyle of the lazy is like a thorny hedge, \\
\poemll    but the path taken by the upright is an open highway. \\
\poeml \v{20}A wise son makes a father glad, \\
\poemll    but a foolish man despises his mother. \\
\poeml \v{21}Stupidity is the delight of the senseless, \\
\poemll    but an understanding man walks uprightly. \\
\poeml \v{22}Plans fail without advice, \\
\poemll    but with many counselors they are confirmed. \\
\poeml \v{23}An appropriate answer brings joy to a person, \\
\poemll    and a well-timed word is a good thing. \\
\poeml \v{24}The way of life leads upward for the wise \\
\poemll    so he may avoid Sheol\fnote{I.e. the realm of the dead} below. \\
\poeml \v{25}The house of the proud the \divine{Lord} will demolish, \\
\poemll    but he will protect the widow's boundary line. \\
\poeml \v{26}To the \divine{Lord} evil plans are detestable, \\
\poemll    but pleasant words are pure. \\
\poeml \v{27}Those who are greedy for unjust gain bring trouble into their homes, \\
\poemll    but the person who hates bribes will live. \\
\poeml \v{28}The mind of the righteous thinks before speaking, \\
\poemll    but the wicked person spews out evil. \\
\poeml \v{29}The \divine{Lord} is far away from the wicked, \\
\poemll    but he hears the prayers of the righteous. \\
\poeml \v{30}Bright eyes\fnote{Or \fbib{A cheerful look}} encourage the heart; \\
\poemll    good news nourishes the body.\fnote{Lit. \fbib{bones}} \\
\poeml \v{31}Whoever listens to a life-giving rebuke \\
\poemll    will be at home among the wise. \\
\poeml \v{32}Whoever ignores instruction hates himself, \\
\poemll    but anyone who heeds reproof gains understanding.\fnote{Lit. \fbib{heart}} \\
\poeml \v{33}The fear of the \divine{Lord} teaches wisdom, \\
\poemll    and humility precedes honor.
\end{poetry}
\labelchapt{16}
\passage{Wisdom's Blessings}

\begin{poetry}
\poeml \chapt{16}
\v{1}People do the planning,\fnote{Lit. \fbib{Preparations of the heart belong to human beings}} \\
\poeml but the end result\fnote{Or \fbib{the response of the tongue}} is from the \divine{Lord}. \\
\poeml \v{2}Everything a person does seems pure in his own opinion, \\
\poemll    but the \divine{Lord} weighs intentions. \\
\poeml \v{3}Entrust your work to the \divine{Lord}, \\
\poemll    and your planning will succeed. \\
\poeml \v{4}The \divine{Lord} made everything answerable to him, \\
\poemll    including the wicked at the time of trouble.\fnote{Lit. \fbib{evil}} \\
\poeml \v{5}The \divine{Lord} detests those who are proud; \\
\poemll    truly they will not go unpunished. \\
\poeml \v{6}Iniquity is atoned for by gracious love and truth, \\
\poemll    and through fear of the \divine{Lord} people\fnote{The Heb. lacks \fbib{people}} turn from evil. \\
\poeml \v{7}When a person's ways please the \divine{Lord}, \\
\poemll    even his enemies will be at peace with him. \\
\poeml \v{8}A little gain\fnote{The Heb. lacks \fbib{gain}} with righteousness is better \\
\poemll    than great income without justice. \\
\poeml \v{9}A person plans his way, \\
\poemll    but the \divine{Lord} directs his steps. \\
\poeml \v{10}When a king is ready to speak officially,\fnote{Lit. \fbib{king speaks an oracle}} \\
\poemll    what he says should not err with respect to justice. \\
\poeml \v{11}Honest scales and balances are from the \divine{Lord}; \\
\poemll    he made all the weights in the bag. \\
\poeml \v{12}Kings detest wrongdoing, \\
\poemll    for through righteousness the throne is established. \\
\poeml \v{13}Kings take pleasure in righteous speech; \\
\poemll    they treasure a person who speaks what is upright. \\
\poeml \v{14}The king's wrath results in a death sentence, \\
\poemll    but whoever is wise will appease him. \\
\poeml \v{15}When a king is pleased,\fnote{Lit. \fbib{a king's face lightens}} there is life, \\
\poemll    and his favor is like a cloud that brings spring rain. \\
\poeml \v{16}How much better than gaining gold is the acquisition of wisdom, \\
\poemll    the attainment of wisdom better than silver! \\
\poeml \v{17}The road of the upright circumvents evil, \\
\poemll    and whoever watches how he lives\fnote{Lit. \fbib{watches his path}} preserves his life. \\
\poeml \v{18}Pride precedes destruction; \\
\poemll    an arrogant spirit appears before a fall. \\
\poeml \v{19}Better to be humble among the poor, \\
\poemll    than to share what is stolen with the proud. \\
\poeml \v{20}Whoever listens to a word of instruction prospers, \\
\poemll    and anyone who trusts in the \divine{Lord} is blessed. \\
\poeml \v{21}The wise-hearted person is told to be discerning, \\
\poemll    and that pleasant speech promotes instruction. \\
\poeml \v{22}Anyone who has understanding is a fountain of life, \\
\poemll    but foolishness brings punishment to fools. \\
\poeml \v{23}A wise person's thoughts\fnote{Lit. \fbib{heart}} control his words, \\
\poemll    and his speech promotes instruction. \\
\poeml \v{24}Pleasant words are honey from a honeycomb--- \\
\poemll    sweet to the soul and healing for the body.\fnote{Lit. \fbib{bone}}
\passage{Advice to the Wise}
\poeml \v{25}There is a road that seems right for a man to travel,\fnote{The Heb. lacks \fbib{to travel}} \\
\poemll    but in the end it's the road to death. \\
\poeml \v{26}The appetite of the laborer motivates him; \\
\poemll    indeed, his hunger drives him on. \\
\poeml \v{27}A worthless person concocts evil gossip\fnote{The Heb. lacks \fbib{gossip}}--- \\
\poemll    his lips are like a burning fire. \\
\poeml \v{28}A deceitful man stirs dissension, \\
\poemll    and anyone who gossips separates friends. \\
\poeml \v{29}A violent man entices his companion \\
\poemll    and leads him on a path that is not good. \\
\poeml \v{30}Whoever winks knowingly\fnote{Lit. \fbib{with his eyes}} is plotting\fnote{So MT; LXX Syr Targ Vg read \fbib{winks with his eyes considers}} deceit; \\
\poemll    anyone who purses his lips is bent towards evil. \\
\poeml \v{31}Gray hair is a crown of glory; \\
\poemll    it is obtained by following\fnote{The Heb. lacks \fbib{following}} a righteous path. \\
\poeml \v{32}Whoever controls his temper is better than a warrior, \\
\poemll    and anyone who has control of his spirit is better \\
\poemlll       than someone who captures a city. \\
\poeml \v{33}The dice is cast into someone's lap, \\
\poemll    but the outcome is from the \divine{Lord}.
\end{poetry}
\labelchapt{17}
\passage{More Words of Wisdom}

\begin{poetry}
\poeml \chapt{17}
\v{1}Dry crumbs in peace\fnote{Lit. \fbib{quiet}} are better \\
\poeml than a full meal\fnote{Lit. \fbib{house full of meat}} with strife. \\
\poeml \v{2}A prudent servant will rule in place of a disgraceful son \\
\poemll    and will share in the inheritance among brothers. \\
\poeml \v{3}The crucible is for silver \\
\poemll    and the furnace for gold--- \\
\poemlll       but the \divine{Lord} assays hearts. \\
\poeml \v{4}Whoever practices evil pays attention to wicked speech, \\
\poemll    and the liar listens to malicious talk. \\
\poeml \v{5}Whoever mocks the poor shows contempt for their maker, \\
\poemll    and whoever is happy about disaster \\
\poemlll       will not go unpunished. \\
\poeml \v{6}Grandchildren are the crown of the aged, \\
\poemll    and the pride of children is their parents. \\
\poeml \v{7}Appropriate speech is inconsistent with the fool; \\
\poemll    how much more are deceitful statements\fnote{Lit. \fbib{lips}} with a prince! \\
\poeml \v{8}A bribe works wonders\fnote{Lit. \fbib{A gift is a stone of favor}} in the eyes of its giver; \\
\poemll    wherever he turns he prospers. \\
\poeml \v{9}Anyone who overlooks\fnote{Lit. \fbib{covers}} an offense promotes love, \\
\poemll    but someone who gossips separates close friends. \\
\poeml \v{10}A rebuke is more effective with a man of understanding \\
\poemll    than a hundred lashes to a fool. \\
\poeml \v{11}A rebellious person seeks evil; \\
\poemll    a cruel emissary will be sent to oppose him. \\
\poeml \v{12}It's better to meet a mother bear who has lost her cubs \\
\poemll    than a fool in his stupidity. \\
\poeml \v{13}The person who repays good with evil \\
\poemll    will never see\fnote{The Heb. lacks \fbib{will see}} evil leave his home. \\
\poeml \v{14}Starting a quarrel is like spilling water--- \\
\poemll    so drop the dispute before it escalates. \\
\poeml \v{15}Exonerating the wicked and condemning the righteous \\
\poemll    are both detestable to the \divine{Lord}. \\
\poeml \v{16}What is this? A fool has enough money to buy wisdom, \\
\poemll    but is senseless?\fnote{Lit. \fbib{but has no heart}} \\
\poeml \v{17}A friend loves at all times, \\
\poemll    and a brother is there\fnote{Lit. \fbib{born}} for times of trouble. \\
\poeml \v{18}A man who lacks sense\fnote{Lit. \fbib{heart}} cosigns a loan,\fnote{Lit. \fbib{sense strikes the palm}} \\
\poemll    becoming a guarantor for his neighbor. \\
\poeml \v{19}The person who loves transgression loves strife; \\
\poemll    the person who builds a high gate invites destruction. \\
\poeml \v{20}The person whose mind\fnote{Lit. \fbib{heart}} is perverse does not find good, \\
\poemll    and anyone with perverted speech falls into trouble. \\
\poeml \v{21}The man who fathers a fool does so to his sorrow--- \\
\poemll    the father of a fool has no joy. \\
\poeml \v{22}A joyful heart is good medicine, \\
\poemll    but a broken spirit drains one's strength.\fnote{Lit. \fbib{spirit dries the bones}} \\
\poeml \v{23}The wicked man takes a bribe in secret \\
\poemll    in order to pervert the course of justice. \\
\poeml \v{24}A person with understanding has wisdom as his objective, \\
\poemll    but a fool looks only\fnote{The Heb. lacks \fbib{only}} to earthly goals. \\
\poeml \v{25}A foolish son brings grief to his father \\
\poemll    and bitterness to his mother.\fnote{Lit. \fbib{to the one who bore him}} \\
\poeml \v{26}Furthermore, it isn't good to fine the righteous, \\
\poemll    or to beat an official because of his uprightness. \\
\poeml \v{27}Whoever controls what he says is knowledgeable; \\
\poemll    anyone who has a calm spirit is a man of understanding. \\
\poeml \v{28}Even a fool is thought to be wise when he remains silent; \\
\poemll    he is thought to be prudent when he keeps his mouth shut.
\end{poetry}
\labelchapt{18}
\passage{How Fools Talk}

\begin{poetry}
\poeml \chapt{18}
\v{1}Whoever isolates himself pursues selfish ends; \\
\poeml he resists all sound advice. \\
\poeml \v{2}A fool finds no satisfaction in trying to understand, \\
\poemll    for he would rather express his own opinion. \\
\poeml \v{3}When an evil person comes, contempt also comes, \\
\poemll    along with dishonor and disgrace. \\
\poeml \v{4}The words a man says are as deep waters--- \\
\poemll    a fountain of wisdom is an overflowing stream. \\
\poeml \v{5}It's not good to be partial towards an evil person, \\
\poemll    thereby depriving the righteous of justice. \\
\poeml \v{6}A fool's words\fnote{Lit. \fbib{lips}} bring strife, \\
\poemll    and his mouth invites fighting. \\
\poeml \v{7}A fool's mouth is his unraveling, \\
\poemll    and his lips entrap himself. \\
\poeml \v{8}The words of a gossip are like choice morsels \\
\poemll    as they descend to the innermost parts of the body.
\passage{Avoiding Fools and Their Foolishness}
\poeml \v{9}Whoever is lazy regarding his work \\
\poemll    is also a brother to the master of destruction. \\
\poeml \v{10}The name of the \divine{Lord} is a strong tower; \\
\poemll    a righteous person rushes to it and is lifted up above the danger.\fnote{The Heb. lacks \fbib{above the danger}} \\
\poeml \v{11}The wealth of a rich person is his fortified city; \\
\poemll    in his own imagination, it is like a high wall. \\
\poeml \v{12}Before a man's downfall, his mind\fnote{Lit. \fbib{heart}} is arrogant, \\
\poemll    but humility precedes honor. \\
\poeml \v{13}Whoever answers before listening \\
\poemll    is both foolish and shameful. \\
\poeml \v{14}A man's spirit can sustain him during his illness, \\
\poemll    but who can bear a crushed spirit? \\
\poeml \v{15}The mind\fnote{Lit. \fbib{heart}} of a discerning person gains knowledge, \\
\poemll    while the ears of wise people seek out knowledge. \\
\poeml \v{16}A person's gift opens doors for him, \\
\poemll    bringing him access to important people. \\
\poeml \v{17}The first to put forth his case seems right, \\
\poemll    until someone else steps forward and cross-examines him. \\
\poeml \v{18}Casting dice settles a dispute, \\
\poemll    deciding between strong contenders. \\
\poeml \v{19}An offended brother is more unyielding than a fortified city, \\
\poemll    and his disputes are like the bars of a fortress. \\
\poeml \v{20}The positive words that a man speaks\fnote{Lit. \fbib{words from a man's mouth}} fill his stomach; \\
\poemll    he will be satisfied with what his lips produce. \\
\poeml \v{21}The power of the tongue is life and death--- \\
\poemll    those who love to talk\fnote{Lit. \fbib{love it}} will eat what it produces. \\
\poeml \v{22}Whoever finds a wife finds what is good, \\
\poemll    and receives favor from the \divine{Lord}. \\
\poeml \v{23}The poor person pleads for mercy, \\
\poemll    but the wealthy man responds harshly. \\
\poeml \v{24}A man with many friends can still be ruined, \\
\poemll    but a true friend sticks closer than a brother.
\end{poetry}
\labelchapt{19}
\passage{The Priorities of Life Contrasted}

\begin{poetry}
\poeml \chapt{19}
\v{1}A poor man who walks blamelessly is better \\
\poeml than a fool who speaks perversely. \\
\poeml \v{2}Furthermore, it isn't good to be ignorant,\fnote{Lit. \fbib{good for an ignorant soul}} \\
\poemll    and whoever rushes into things\fnote{Lit. \fbib{whoever hurries with his feet}} misses the mark. \\
\poeml \v{3}A man's foolishness ruins his life,\fnote{Lit. \fbib{way}} \\
\poemll    yet his heart rages against the \divine{Lord}. \\
\poeml \v{4}Wealth brings many friends, \\
\poemll    but a poor man is deserted by his friend. \\
\poeml \v{5}A witness to lies will not go unpunished; \\
\poemll    the teller of falsehoods will not escape. \\
\poeml \v{6}Many curry favor of an official; \\
\poemll    everyone is a friend of the gift giver. \\
\poeml \v{7}All the relatives of a poor person shun him--- \\
\poemll    how much more do his friends avoid him! \\
\poeml Though he runs after them pleading, \\
\poemll    they aren't around. \\
\poeml \v{8}Whoever obtains wisdom loves himself, \\
\poemll    and whoever treasures understanding will prosper. \\
\poeml \v{9}A witness to lies will not go unpunished; \\
\poemll    the teller of falsehoods will perish. \\
\poeml \v{10}It's not fitting for a fool to live in luxury; \\
\poemll    neither is it for a servant to rule over princes. \\
\poeml \v{11}A person's discretion makes him slow to anger, \\
\poemll    and it is to his credit that he ignores an offence. \\
\poeml \v{12}The king's anger is like the roaring of a lion, \\
\poemll    but his goodwill is like dew on the grass. \\
\poeml \v{13}A father's ruin is a foolish son, \\
\poemll    and a wife's quarreling is like\fnote{The Heb. lacks \fbib{like}} dripping water that never stops. \\
\poeml \v{14}A house and self-sufficiency are a father's inheritance, \\
\poemll    but from the \divine{Lord} comes an insightful wife. \\
\poeml \v{15}Laziness puts one to sleep, \\
\poemll    and an idle person will go hungry. \\
\poeml \v{16}Whoever obeys a commandment keeps himself safe,\fnote{Lit. \fbib{keeps his soul}} \\
\poemll    but someone who is contemptuous in conduct\fnote{Lit. \fbib{in his way}} will die. \\
\poeml \v{17}Whoever is kind to the poor is lending to the \divine{Lord}--- \\
\poemll    the benefit of his gift will return to him in abundance. \\
\poeml \v{18}Discipline your son while there is still hope--- \\
\poemll    but don't set your heart on his destruction. \\
\poeml \v{19}The person who has great anger must pay the consequences, \\
\poemll    because if you rescue him, you will have to do it again. \\
\poeml \v{20}Listen to advice and accept discipline, \\
\poemll    and you'll be wise for the rest of your life.\fnote{The Heb. lacks \fbib{of your life}} \\
\poeml \v{21}Many plans occupy the mind\fnote{Lit. \fbib{heart}} of a man, \\
\poemll    but the \divine{Lord}'s purposes will prevail.\fnote{Or \fbib{will be established}} \\
\poeml \v{22}Human beings long for grace, \\
\poemll    and it's better to be poor than a man of deceit. \\
\poeml \v{23}The fear of the \divine{Lord} leads\fnote{The Heb. lacks \fbib{leads}} to life; \\
\poemll    whoever is satisfied with it will rest, \\
\poemlll       untouched by evil. \\
\poeml \v{24}The lazy person buries his hand in his dish \\
\poemll    and doesn't bother to bring it back to his mouth. \\
\poeml \v{25}If you scourge a scoffer, \\
\poemll    the simple person may learn to be discreet; \\
\poeml rebuke a discerning man \\
\poemll    and he will gain understanding. \\
\poeml \v{26}Whoever mistreats his father \\
\poemll    and alienates his mother \\
\poemlll       is a son who brings both shame and disrespect. \\
\poeml \v{27}My son, if you stop listening to instruction, \\
\poemll    you will stray from the principles of knowledge. \\
\poeml \v{28}A corrupt witness\fnote{I.e. a worthless person} mocks justice, \\
\poemll    and the wicked person feeds on iniquity. \\
\poeml \v{29}Condemnation is appropriate for mockers, \\
\poemll    just as beatings are for the backs of fools.
\end{poetry}
\labelchapt{20}
\passage{Advice on How to Live}

\begin{poetry}
\poeml \chapt{20}
\v{1}Wine causes mocking, and beer causes fights; \\
\poeml everyone led astray by them lacks wisdom. \\
\poeml \v{2}A king's anger is like a lion's roar; \\
\poemll    anyone who angers him forfeits his life. \\
\poeml \v{3}Avoiding strife brings a man honor, \\
\poemll    but every fool is quarrelsome. \\
\poeml \v{4}A lazy person doesn't plow in the proper\fnote{The Heb. lacks \fbib{proper}} season; \\
\poemll    he looks for a harvest, but there is nothing. \\
\poeml \v{5}The intentions of a person's heart are deep waters, \\
\poemll    but a discerning person reveals them. \\
\poeml \v{6}Many claim ``I'm a loyal person!''\fnote{Lit. \fbib{claim to be people of gracious love}} \\
\poemll    but who can find someone who truly is? \\
\poeml \v{7}The righteous person lives a life of integrity; \\
\poemll    happy are his children who follow him! \\
\poeml \v{8}A king sits on a throne of justice, \\
\poemll    sifting out all sorts of evil with his glance. \\
\poeml \v{9}Who can say, ``My intentions are pure; \\
\poemll    I am clean from any sin?'' \\
\poeml \v{10}False\fnote{Or \fbib{Diverse}} weights and measures--- \\
\poemll    the \divine{Lord} surely detests both of them. \\
\poeml \v{11}Even a child is known by his actions, \\
\poemll    whether his deeds are pure and right. \\
\poeml \v{12}The ear that hears and the eye that sees--- \\
\poemll    the \divine{Lord} surely made them both. \\
\poeml \v{13}Do not love sleep or you'll become poor, \\
\poemll    keep your eyes open and you'll have plenty of food. \\
\poeml \v{14}``This is bad, bad,'' says whoever is buying--- \\
\poemll    but then he brags as he walks away after the sale.\fnote{The Heb. lacks \fbib{after the sale}} \\
\poeml \v{15}There is an abundance of gold and precious stones, \\
\poemll    but lips of knowledge are a rare jewel. \\
\poeml \v{16}Take the garment of anyone who puts up collateral for a stranger; \\
\poemll    hold it in pledge if he does it for an unfamiliar woman. \\
\poeml \v{17}Bread gained by deceit is sweet to a man, \\
\poemll    but later his mouth will be full of gravel. \\
\poeml \v{18}Make plans by seeking advice; \\
\poemll    make war by obtaining guidance. \\
\poeml \v{19}Whoever spreads gossip betrays confidences; \\
\poemll    so don't get involved with someone who talks too much. \\
\poeml \v{20}Whoever curses his father or mother, \\
\poemll    his lamp will be extinguished in the deepest darkness. \\
\poeml \v{21}An inheritance quickly obtained at the beginning \\
\poemll    will not be blessed at the end. \\
\poeml \v{22}Don't say ``I'll avenge that wrong!'' \\
\poemll    Wait on the \divine{Lord} and he will deliver you. \\
\poeml \v{23}The \divine{Lord} detests differing weights, \\
\poemll    and dishonest scales are not good. \\
\poeml \v{24}A man's steps are directed by the \divine{Lord}; \\
\poemll    how then can anyone understand his own way? \\
\poeml \v{25}It is a trap for a person to declare quickly, ``This is sacred,'' \\
\poemll    and only later to have second thoughts about the vows. \\
\poeml \v{26}A wise king sifts the wicked, \\
\poemll    crushing them with the threshing wheel. \\
\poeml \v{27}A person's spirit is the lamp of the \divine{Lord}; \\
\poemll    it searches throughout one's innermost being. \\
\poeml \v{28}Gracious love and truth preserve a king; \\
\poemll    through love his throne is made secure. \\
\poeml \v{29}The glory of young men is their strength; \\
\poemll    and the splendor of elders is their gray hair. \\
\poeml \v{30}Blows that wound clean away evil; \\
\poemll    such beatings cleanse\fnote{The Heb. lacks \fbib{cleanse}} the innermost being.
\end{poetry}
\labelchapt{21}
\passage{Thoughts on the Sovereignty of God}

\begin{poetry}
\poeml \chapt{21}
\v{1}A king's heart is a water stream that the \divine{Lord} controls; \\
\poeml he directs it wherever he pleases. \\
\poeml \v{2}Every man's lifestyle is proper in his own view, \\
\poemll    but the \divine{Lord} weighs the heart. \\
\poeml \v{3}To do what is right and just \\
\poemll    is more acceptable to the \divine{Lord} than sacrifice.
\end{poetry}

\v{4}A proud attitude,\fnote{Lit. \fbib{heart}} accompanied by\fnote{Lit. \fbib{proud heart and}} a haughty look, is sin;

\begin{poetry}
\poemll    they reveal\fnote{Lit. \fbib{sin; the lamp of}} wicked people. \\
\poeml \v{5}Plans of the persistent surely lead to productivity, \\
\poemll    but all who are hasty will surely become poor. \\
\poeml \v{6}A fortune gained by deceit\fnote{Lit. \fbib{by a lying tongue}} \\
\poemll    is a fleeting vapor and a deadly snare.\fnote{So MT; LXX reads \fbib{is pursuing worthlessness into deadly snares}} \\
\poeml \v{7}Devastation caused by the wicked will drag them away \\
\poemll    because they refuse to do what is just. \\
\poeml \v{8}The conduct\fnote{Lit. \fbib{way}} of a guilty man is perverse, \\
\poemll    but the behavior of the pure is upright. \\
\poeml \v{9}It's better to live in a corner on the roof \\
\poemll    than to share a house with a contentious woman. \\
\poeml \v{10}The soul of the wicked craves evil; \\
\poemll    he extends no mercy to his neighbor. \\
\poeml \v{11}When a mocker is punished, the fool gains wisdom; \\
\poemll    but when the wise is instructed, he receives knowledge. \\
\poeml \v{12}The righteous God\fnote{The Heb. lacks \fbib{God}} considers the house of the wicked, \\
\poemll    bringing the wicked to ruin. \\
\poeml \v{13}Whoever refuses to hear the cry of the poor \\
\poemll    will also cry himself, but he won't be answered. \\
\poeml \v{14}Privately given gifts pacify wrath, \\
\poemll    and payments made secretly\fnote{Lit. \fbib{made under the cloak}} appease\fnote{The Heb. lacks \fbib{appease}} great anger. \\
\poeml \v{15}Administering justice brings joy to the righteous, \\
\poemll    but terror to those who practice iniquity. \\
\poeml \v{16}Whoever wanders from the path of understanding \\
\poemll    will end up where the dead\fnote{Lit. \fbib{the departed spirits}} are gathered. \\
\poeml \v{17}Pleasure lovers become poor; \\
\poemll    loving wine and oil doesn't bring riches. \\
\poeml \v{18}The wicked are ransom for the righteous, \\
\poemll    and the unfaithful for the upright. \\
\poeml \v{19}It's better to live in the wilderness \\
\poemll    than to live with a contentious and irritable woman. \\
\poeml \v{20}Precious treasures and oil are found\fnote{So MT; LXX reads \fbib{A desirable treasure will rest}} where the wise live, \\
\poemll    but a foolish man devours them. \\
\poeml \v{21}Whoever pursues righteousness and gracious love \\
\poemll    finds life, righteousness, and honor. \\
\poeml \v{22}A wise man attacks the city of the mighty, \\
\poemll    bringing down the fortress in which they trust. \\
\poeml \v{23}Whoever watches his mouth and tongue \\
\poemll    keeps himself from trouble. \\
\poeml \v{24}The names ``Proud,'' ``Arrogant,'' and ``Mocker'' \\
\poemll    fit whoever acts with presumptuous conceit. \\
\poeml \v{25}What the lazy person craves will kill him, \\
\poemll    because his hands refuse to work. \\
\poeml \v{26}All day long he continues to crave, \\
\poemll    while the righteous person gives without holding back. \\
\poeml \v{27}What the wicked person sacrifices is detestable--- \\
\poemll    how much more when he offers it with vile motives! \\
\poeml \v{28}A false witness will perish, \\
\poemll    but whoever listens will testify successfully.\fnote{Lit. \fbib{testify forever}} \\
\poeml \v{29}The wicked man puts up a bold appearance, \\
\poemll    but the upright thinks about what he is doing.\fnote{Lit. \fbib{about his ways}} \\
\poeml \v{30}No wisdom, insight, or counsel \\
\poemll    can prevail\fnote{The Heb. lacks \fbib{can prevail}} against the \divine{Lord}. \\
\poeml \v{31}The horse may be prepared for the day of battle, \\
\poemll    but to the \divine{Lord} goes the victory.
\end{poetry}
\labelchapt{22}
\passage{Advice for Everyday Life}

\begin{poetry}
\poeml \chapt{22}
\v{1}A good reputation is more desirable than great wealth, \\
\poeml and favorable acceptance more than silver and gold. \\
\poeml \v{2}The rich and the poor have this in common--- \\
\poemll    the \divine{Lord} created both of them. \\
\poeml \v{3}The prudent person sees trouble ahead and hides, \\
\poemll    but the na\"{i}ve continue on and suffer the consequences. \\
\poeml \v{4}The reward of humility is the fear of the \divine{Lord}, \\
\poemll    along with wealth, honor, and life. \\
\poeml \v{5}Thorns and snares lie in the path of the perverse person, \\
\poemll    but whoever is cautious stays far away from them. \\
\poeml \v{6}Train a child in the way appropriate for him, \\
\poemll    and when he becomes older, he will not turn from it. \\
\poeml \v{7}The wealthy rule over the poor, \\
\poemll    and anyone who borrows is a slave to the lender. \\
\poeml \v{8}Whoever sows wickedness reaps trouble, \\
\poemll    and the anger he uses for a weapon\fnote{Lit. \fbib{rod}} will be destroyed. \\
\poeml \v{9}Whoever is generous\fnote{Lit. \fbib{A good eye}} will be blessed, \\
\poemll    for he shares his food with the poor. \\
\poeml \v{10}Throw out the mocker and strife departs, too;\fnote{The Heb. lacks \fbib{too}} \\
\poemll    furthermore, quarrels\fnote{Or \fbib{litigation}} and discord will end. \\
\poeml \v{11}Whoever loves purity\fnote{Lit. \fbib{purity of heart}} and gracious speech \\
\poemll    will gain the king as his friend. \\
\poeml \v{12}The \divine{Lord} watches over anyone with knowledge, \\
\poemll    but he ruins the plans\fnote{Lit. \fbib{words}} of the unfaithful. \\
\poeml \v{13}The lazy person says, ``There is a lion outside! \\
\poemll    I will be killed in the street!'' \\
\poeml \v{14}The mouth of an immoral woman is a deep pit; \\
\poemll    a man experiencing the \divine{Lord}'s wrath will fall into it. \\
\poeml \v{15}A child's heart has a tendency to do wrong, \\
\poemll    but the rod of discipline removes it far away from him. \\
\poeml \v{16}Whoever oppresses the poor to enrich himself \\
\poemll    and whoever gives gifts to the wealthy \\
\poemlll       will yield only loss.
\passage{Sayings of the Wise}
\poeml \v{17}Pay attention and listen to the words of the wise, \\
\poemll    and apply your heart to my teaching, \\
\poeml \v{18}for it is pleasant when you treasure them within you \\
\poemll    and have them ready on your lips. \\
\poeml \v{19}As a result, your trust will be in the \divine{Lord}, \\
\poemll    that's why I'm teaching you today, even you. \\
\poeml \v{20}Have I not written for you 30 sayings \\
\poemll    containing counsel and knowledge, \\
\poeml \v{21}to teach you true and reliable advice, \\
\poemll    so you can give truthful answers to those who sent you? \\
\poeml \v{22}Don't rob the poor person because he is poor, \\
\poemll    and don't crush the helpless in court,\fnote{Lit. \fbib{gate}} \\
\poeml \v{23}for the \divine{Lord} will plead their case \\
\poemll    and ruin the lives of those who ruin them. \\
\poeml \v{24}Don't make friends with a hot-tempered man, \\
\poemll    and do not associate with someone who is easily angered, \\
\poeml \v{25}or you may learn his ways \\
\poemll    and find yourself caught in a trap. \\
\poeml \v{26}Don't be one of those who make promises \\
\poemll    to guarantee loans for debts. \\
\poeml \v{27}If you don't have the ability to pay, \\
\poemll    why should your very bed be taken from under you? \\
\poeml \v{28}Don't remove an ancient boundary stone \\
\poemll    that was set up by your ancestors. \\
\poeml \v{29}Do you see a man skilled in his work? \\
\poemll    He will work for kings, not unimportant people.
\end{poetry}
\labelchapt{23}
\passage{Things to Avoid in Life}

\begin{poetry}
\poeml \chapt{23}
\v{1}Whenever you sit down to dine with a ruler, \\
\poeml carefully think about what is before you. \\
\poeml \v{2}Put a knife to your own throat, \\
\poemll    if you have a big appetite.\fnote{Lit. \fbib{a master of an appetite}} \\
\poeml \v{3}Don't crave his delicacies, \\
\poemll    because the meal is deceptive. \\
\poeml \v{4}Don't exhaust yourself acquiring wealth; \\
\poemll    be smart enough to stop. \\
\poeml \v{5}When you fix your gaze on it, it's gone, \\
\poemll    for it sprouts wings for itself \\
\poemlll       and flies to the sky like an eagle. \\
\poeml \v{6}Don't consume food provided by a miserly\fnote{Lit. \fbib{by the evil eyed}} person, \\
\poemll    and don't desire his delicacies, \\
\poeml \v{7}for as he thinks within himself, so he is. \\
\poemll    ``Eat and drink!'' he'll say to you, \\
\poemlll       but his heart won't be with you. \\
\poeml \v{8}You'll vomit up what little you've eaten, \\
\poemll    and your compliments will have been wasted. \\
\poeml \v{9}Don't speak when a fool is listening, \\
\poemll    because he'll despise your wise words. \\
\poeml \v{10}Don't move ancient boundaries \\
\poemll    or invade fields belonging to orphans; \\
\poeml \v{11}for strong is their Redeemer \\
\poemll    who will take up their case against you. \\
\poeml \v{12}Learn diligently, \\
\poemll    and listen to words of knowledge. \\
\poeml \v{13}Don't withhold discipline from a child; \\
\poemll    if you punish him with a rod, \\
\poemlll       he won't die. \\
\poeml \v{14}Punish him with a rod, \\
\poemll    and you will rescue his soul from Sheol.\fnote{I.e. the realm of the dead.}
\passage{On Listening to Your Parents}
\poeml \v{15}My son, if your heart is wise, \\
\poemll    my own heart will greatly rejoice. \\
\poeml \v{16}My innermost being will be glad \\
\poemll    when your lips speak what is right. \\
\poeml \v{17}Never let yourself envy sinners; \\
\poemll    instead, remain\fnote{The Heb. lacks \fbib{remain}} in fear of the \divine{Lord} every day, \\
\poeml \v{18}for there is surely a future life, \\
\poemll    and what you hope for will not be cut off. \\
\poeml \v{19}Listen, my son, and be wise, \\
\poemll    commit yourself to live God's\fnote{Lit. \fbib{live in the}} way. \\
\poeml \v{20}Don't associate with heavy drinkers \\
\poemll    or dine with gluttons, \\
\poeml \v{21}because drunks and gluttons tend to become poor, \\
\poemll    and drowsiness will clothe them in rags. \\
\poeml \v{22}Listen to the one who fathered you, \\
\poemll    and don't despise your mother in her old age. \\
\poeml \v{23}Purchase truth, but don't sell it; \\
\poemll    store up\fnote{The Heb. lacks \fbib{store up}} wisdom, instruction, and understanding. \\
\poeml \v{24}The father of a righteous person will greatly rejoice; \\
\poemll    whoever fathers a wise son will be glad because of him. \\
\poeml \v{25}Let your father and mother rejoice; \\
\poemll    make the one who gave birth to you happy. \\
\poeml \v{26}Give me your heart, my son, \\
\poemll    and keep your eyes fixed on my ways, \\
\poeml \v{27}because a prostitute is a deep pit, \\
\poemll    and the adulterous\fnote{Lit. \fbib{foreign}} woman a narrow well. \\
\poeml \v{28}Surely she lies in wait like a bandit, \\
\poemll    increasing those who are faithless among mankind.
\passage{On Sobriety}
\poeml \v{29}Who has woe? Who has grief? \\
\poemll    Who has contention? Who has complaints? \\
\poeml Who has wounds without cause? \\
\poemll    Who has bloodshot eyes? \\
\poeml \v{30}Those who linger over their wine, \\
\poemll    who consume mixed drinks. \\
\poeml \v{31}Don't stare into red wine, \\
\poemll    when it sparkles in the cup \\
\poemlll       and goes down smoothly. \\
\poeml \v{32}Eventually it will bite like a snake \\
\poemll    and sting like a serpent. \\
\poeml \v{33}Your eyes will see strange things, \\
\poemll    and with slurred words you'll speak what you really believe. \\
\poeml \v{34}You will be like someone who lies down in the sea, \\
\poemll    or like someone who sleeps on top of a mast. \\
\poeml \v{35}``They struck me,'' you will say,\fnote{The Heb. lacks \fbib{you will say}} \\
\poemll    ``but I never felt it. \\
\poeml They beat me, \\
\poemll    but I never knew it \\
\poeml When will I wake up? \\
\poemll    I want another drink.''
\end{poetry}
\labelchapt{24}
\passage{Benefits of Wisdom}

\begin{poetry}
\poeml \chapt{24}
\v{1}Don't be envious of wicked men \\
\poeml or wish you were with them, \\
\poeml \v{2}because they\fnote{Lit. \fbib{because their hearts}} plan violence, \\
\poemll    and they are always talking\fnote{Lit. \fbib{and their lips talk}} about trouble. \\
\poeml \v{3}By wisdom a house is built; \\
\poemll    it is made secure through understanding. \\
\poeml \v{4}By knowledge its rooms are furnished \\
\poemll    with all sorts of expensive and beautiful goods. \\
\poeml \v{5}A wise man is strong,\fnote{So MT; LXX reads \fbib{Being wise is better than being strong}} \\
\poemll    and a knowledgeable man grows in strength. \\
\poeml \v{6}For through wise counsel you will wage your war, \\
\poemll    and victory lies in an abundance of advisors. \\
\poeml \v{7}Wisdom lies beyond reach of the fool; \\
\poemll    he has nothing to say in court.\fnote{Lit. \fbib{in the gate}} \\
\poeml \v{8}The person who plans on doing evil \\
\poemll    will be called a schemer. \\
\poeml \v{9}To devise folly is sin, \\
\poemll    and people detest a scoffer. \\
\poeml \v{10}If you grow weary when times are troubled, \\
\poemll    your strength is limited.\fnote{Or \fbib{undersized}} \\
\poeml \v{11}Rescue those who are being led away to death, \\
\poemll    and save those who stumble toward slaughter. \\
\poeml \v{12}If you say, ``Look here, we didn't know about this,'' \\
\poemll    doesn't God,\fnote{Lit. \fbib{he}} who examines motives,\fnote{Lit. \fbib{examines the heart}} discern it? \\
\poeml Doesn't the one who guards your soul \\
\poemll    know about it? \\
\poeml Won't he repay each person \\
\poemll    according to what he has done? \\
\poeml \v{13}My son, eat honey, because it's good for you;\fnote{The Heb. lacks \fbib{for you}} \\
\poemll    indeed, drippings from the honeycomb are sweet to your taste; \\
\poeml \v{14}Keep in mind that wisdom is like that for your soul; \\
\poemll    if you find it, there will be a future for you, \\
\poemlll       and what you hope for won't be cut short. \\
\poeml \v{15}Don't lie in wait like an outlaw \\
\poemll    to attack where the righteous live; \\
\poeml \v{16}for though a righteous man falls seven times, \\
\poemll    he will rise again, \\
\poemlll       but the wicked stumble into calamity. \\
\poeml \v{17}Don't rejoice when your enemy falls; \\
\poemll    don't let yourself be glad when he stumbles. \\
\poeml \v{18}Otherwise the \divine{Lord} will observe and disapprove, \\
\poemll    and he will turn his anger away from him. \\
\poeml \v{19}Don't be anxious about those who practice evil, \\
\poemll    and don't be envious of the wicked. \\
\poeml \v{20}For the wicked man has no future; \\
\poemll    the lamp of the wicked will be extinguished. \\
\poeml \v{21}My son, fear both the \divine{Lord} and the king, \\
\poemll    and don't keep company with rebels. \\
\poeml \v{22}They will be destroyed suddenly, \\
\poemll    and who knows what kind of punishment will come from these two?
\passage{Sayings of the Wise}
\poeml \v{23}Here are some more proverbs from wise people: \\
\poeml It isn't good to show partiality in judgment. \\
\poeml \v{24}Whoever says to the wicked, ``You're in the right,'' \\
\poemlll       will be cursed by people and hated by nations. \\
\poeml \v{25}But as for people who rebuke the wicked;\fnote{The Heb. lacks \fbib{the wicked}} \\
\poemll    a good blessing will fall upon them. \\
\poeml \v{26}A kiss on the lips--- \\
\poemll    that's what someone who gives an honest answer deserves.\fnote{The Heb. lacks \fbib{deserves}} \\
\poeml \v{27}First do your outside work, \\
\poemll    preparing your land for yourself. \\
\poemlll       After that, build your house. \\
\poeml \v{28}Don't testify against your neighbor without a cause, \\
\poemll    and don't lie when you speak.\fnote{Lit. \fbib{don't deceive with your lips}} \\
\poeml \v{29}Don't say, ``I'll do to him like he did to me, \\
\poemll    I'll be sure to pay him back for what he did.'' \\
\poeml \v{30}I went by the field belonging to a lazy man, \\
\poemll    by a vineyard belonging to a senseless person. \\
\poeml \v{31}There it was, overgrown with thistles, \\
\poemll    the ground covered with thorns, \\
\poemlll       its stone wall collapsed. \\
\poeml \v{32}As I observed, I thought about it; \\
\poemll    I watched, and learned a lesson: \\
\poeml \v{33}``A little sleep! A little slumber! \\
\poemll    A little folding of my hands to rest!'' \\
\poeml \v{34}Then your poverty will come upon you like a robber, \\
\poemll    your need like an armed bandit.
\end{poetry}
\labelchapt{25}
\passage{More Proverbs from Solomon}

\chapt{25}
\v{1}Here are some more proverbs by Solomon, which the men of Hezekiah, king of Judah, transcribed.

\begin{poetry}
\poeml \v{2}It is the glory of God to conceal a matter, \\
\poemll    and the glory of kings to investigate a matter. \\
\poeml \v{3}Just as the heavens are high \\
\poemll    and earth is deep, \\
\poemlll       so the heart of a king is unfathomable. \\
\poeml \v{4}Purge the dross from the silver, \\
\poemll    and material for\fnote{The Heb. lacks \fbib{material for}} a vessel comes forth for the silversmith. \\
\poeml \v{5}Purge the wicked from the king's presence, \\
\poemll    and his throne will be established in righteousness. \\
\poeml \v{6}Don't magnify yourself in the presence of a king, \\
\poemll    and don't pretend to be in the company of famous men, \\
\poeml \v{7}for it is better that it be told you, ``Come up here,'' \\
\poemll    than for you to be placed lower \\
\poemlll       in the presence of an official. \\
\poeml What you've seen with your own eyes, \\
\poeml \v{8}don't be in a hurry to argue in court. \\
\poeml Otherwise, what will you do later on \\
\poemll    when your neighbor humiliates you? \\
\poeml \v{9}Instead, take up the matter with your neighbor, \\
\poemll    and don't betray another person's confidence. \\
\poeml \v{10}Otherwise, anyone who hears will make you ashamed, \\
\poemll    and your bad reputation will never leave you. \\
\poeml \v{11}Like golden apples set in silver \\
\poemll    is a word spoken at the right time. \\
\poeml \v{12}Like a gold earring and a necklace of pure gold \\
\poemll    is a wise reprover to a listening ear. \\
\poeml \v{13}Like cold snow during harvest time \\
\poemll    is a faithful messenger to those who send him; \\
\poemlll       he refreshes his masters. \\
\poeml \v{14}Like clouds and winds without rain \\
\poemll    is the man who brags \\
\poemlll       about gifts he never gave. \\
\poeml \v{15}Through patience a ruler may be persuaded; \\
\poemll    a gentle word\fnote{Lit. \fbib{tongue}} can break a bone. \\
\poeml \v{16}If you find some honey, \\
\poemll    eat only what you need. \\
\poeml Take too much, \\
\poemll    and you'll vomit. \\
\poeml \v{17}Seldom set foot in your neighbor's home; \\
\poemll    otherwise, he'll grow weary and hate you. \\
\poeml \v{18}A club, a sword, and a sharp arrow--- \\
\poemll    that's what a man is who lies about his neighbor. \\
\poeml \v{19}A bad tooth and an unsteady foot--- \\
\poemll    that's what confidence in an unreliable man is like \\
\poemlll       in a time of trouble. \\
\poeml \v{20}Taking your coat off when it's cold \\
\poemll    or pouring vinegar on soda--- \\
\poemlll       that's what singing songs does to a heavy heart. \\
\poeml \v{21}If your enemy hungers, give him food to eat; \\
\poemll    and if he thirsts, give him water to drink. \\
\poeml \v{22}For you'll be piling burning coals of shame\fnote{The Heb. lacks \fbib{of shame}} on his head \\
\poemll    and the \divine{Lord} will reward you. \\
\poeml \v{23}The north wind brings rain, \\
\poemll    and a backbiting tongue an angry look. \\
\poeml \v{24}It's better to live in a corner on the roof \\
\poemll    than in a house with a contentious woman. \\
\poeml \v{25}Cold water to someone who is thirsty\fnote{Or \fbib{tired}}--- \\
\poemll    that's what good news from a distant land is. \\
\poeml \v{26}A muddied spring or a polluted well--- \\
\poemll    that's what a righteous person is \\
\poemlll       who compromises with the wicked. \\
\poeml \v{27}To eat too much honey isn't good; \\
\poemll    and neither is it honorable to seek one's own glory. \\
\poeml \v{28}Like a city with breached walls \\
\poemll    is a man without self-control.
\end{poetry}
\labelchapt{26}
\passage{On Fools}

\begin{poetry}
\poeml \chapt{26}
\v{1}Like snowfall in summer or rain at harvest time, \\
\poeml so honor is inappropriate for a fool. \\
\poeml \v{2}Like a fluttering sparrow \\
\poemll    or a swallow in flight, \\
\poemlll       a curse without cause will not alight. \\
\poeml \v{3}A whip is for the horses, \\
\poemll    a bridle is for the donkey, \\
\poemlll       a rod is for the back of fools. \\
\poeml \v{4}Don't answer a fool according to his foolishness, \\
\poemll    or you will be just like him. \\
\poeml \v{5}Answer a fool according to his foolishness, \\
\poemll    or he will think himself to be wise. \\
\poeml \v{6}Whoever sends a message by the hand of a fool \\
\poemll    cuts off his own\fnote{The Heb. lacks \fbib{his own}} feet and drinks violence. \\
\poeml \v{7}Useless legs to the lame--- \\
\poemll    that's what a proverb quoted by a fool is. \\
\poeml \v{8}Tying a stone to a sling--- \\
\poemll    that's what giving honor to a fool is. \\
\poeml \v{9}A thorn in the hand of a drunkard--- \\
\poemll    that's what a proverb quoted by a fool is. \\
\poeml \v{10}An archer who shoots at anyone--- \\
\poemll    is like someone who hires a fool or anyone who passes by. \\
\poeml \v{11}A dog that returns to its vomit \\
\poemll    is like a fool who reverts to his folly. \\
\poeml \v{12}Do you see a man who is wise in his own opinion? \\
\poemll    There's more hope for a fool than for him.
\passage{On Laziness}
\poeml \v{13}The lazy person claims, ``There is a lion in the road! \\
\poemll    There's a lion in the streets!'' \\
\poeml \v{14}The door turns on its hinges--- \\
\poemll    as does the lazy person on his bed. \\
\poeml \v{15}The lazy person buries his hand in the dish, \\
\poemll    but he's too tired to bring it to his mouth again. \\
\poeml \v{16}The lazy person is wiser in his own opinion \\
\poemll    than seven men who can give an appropriate response. \\
\poeml \v{17}Picking up a dog by the ears--- \\
\poemll    that's what someone is like who\fnote{Lit. \fbib{who, as he is passing by,}} meddles in another's fight. \\
\poeml \v{18}Like the maniac who shoots \\
\poemll    fiery darts and deadly arrows--- \\
\poeml \v{19}that's what someone is like who lies to his neighbor \\
\poemll    and then says, ``I was joking, wasn't I?''
\passage{On Gossip and Backbiting}
\poeml \v{20}Without wood, the fire goes out. \\
\poemll    Without a gossip, contention stops. \\
\poeml \v{21}Charcoal is to hot coals \\
\poemll    as wood is to fire; \\
\poemlll       so also a quarrelsome man fuels strife. \\
\poeml \v{22}The words of a gossip are like delicate morsels; \\
\poemll    they sink down deep within. \\
\poeml \v{23}A clay vessel plated with a thin veneer of silver--- \\
\poemll    that's what smooth\fnote{So LXX; MT reads \fbib{burning}} lips with a wicked heart are. \\
\poeml \v{24}Someone who hates hides behind his words, \\
\poemll    harboring deceit within himself. \\
\poeml \v{25}Though he speaks graciously, don't believe him, \\
\poemll    for there are seven detestable things in his heart. \\
\poeml \v{26}Though malice disguises itself with deception, \\
\poemll    its evil will be exposed publicly. \\
\poeml \v{27}Whoever digs a pit will fall into it, \\
\poemll    and the stone will come back \\
\poemlll       on whoever starts it rolling. \\
\poeml \v{28}A lying tongue hates its victims, \\
\poemll    and a flattering mouth causes ruin.
\end{poetry}
\labelchapt{27}
\passage{General Counsel}

\begin{poetry}
\poeml \chapt{27}
\v{1}Never brag about the day to come, \\
\poeml because you don't know what it\fnote{Lit. \fbib{what a day}} might bring. \\
\poeml \v{2}Let someone else praise you, \\
\poemll    not your own mouth; \\
\poemlll       a stranger, and never your own lips. \\
\poeml \v{3}Rocks are heavy, \\
\poemll    and sand is weighty, \\
\poemlll       but a fool's provocation outweighs them both. \\
\poeml \v{4}Wrath can be fierce and anger overwhelms \\
\poemll    but who can stand up to jealousy? \\
\poeml \v{5}An open rebuke is better \\
\poemll    than unspoken love. \\
\poeml \v{6}Wounds from someone who loves are trustworthy, \\
\poemll    but kisses from an enemy speak volumes.\fnote{Lit. \fbib{enemy are profuse}} \\
\poeml \v{7}The person\fnote{Lit. \fbib{soul}} who is full spurns honey, \\
\poemll    but to a hungry person even the bitter seems sweet. \\
\poeml \v{8}Like a bird that strays from its nest \\
\poemll    is a man who wanders away from his home.\fnote{Lit. \fbib{place}} \\
\poeml \v{9}Ointments and perfume encourage the heart; \\
\poemll    in a similar way, a friend's advice is sweet to the soul.\fnote{So MT; LXX reads \fbib{heart; but through misfortune the soul is torn apart}} \\
\poeml \v{10}Never abandon your friend nor your father's friend, \\
\poemll    and don't go to your brother's house in times of trouble. \\
\poeml A neighbor who is near is better \\
\poemll    than a brother who lives far away. \\
\poeml \v{11}Be wise, my son, and make me happy, \\
\poemll    so I can reply to anyone who insults me. \\
\poeml \v{12}Those who are prudent see danger and take refuge, \\
\poemll    but the na\"{i}ve continue on and suffer the consequences. \\
\poeml \v{13}Take the coat of anyone who puts up security for a stranger; \\
\poemll    hold it in pledge if he cosigns for an immoral woman. \\
\poeml \v{14}A friend's loud blessing early in the morning \\
\poemll    will be thought of as a curse. \\
\poeml \v{15}A continual dripping on a rainy day \\
\poemll    and a contentious wife are alike. \\
\poeml \v{16}Trying to keep her in check is like stopping a wind storm \\
\poemll    or grabbing oil with your right hand. \\
\poeml \v{17}Iron sharpens iron; \\
\poemll    so a man sharpens a friend's character.\fnote{Lit. \fbib{countenance}} \\
\poeml \v{18}Whoever nurtures the fig tree will eat its fruit, \\
\poemll    and whoever obeys\fnote{Lit. \fbib{guards}} his master will be honored. \\
\poeml \v{19}Just as water reflects the face, \\
\poemll    so the heart reflects the person. \\
\poeml \v{20}Sheol\fnote{I.e. the realm of the dead} and Abaddon\fnote{I.e. the realm of destruction in the afterlife} are never satiated, \\
\poemll    and neither are human eyes. \\
\poeml \v{21}As the crucible tests\fnote{The Heb. lacks \fbib{tests}} silver, \\
\poemll    and the furnace assays\fnote{The Heb. lacks \fbib{assays}} gold; \\
\poemlll       so praise received tests\fnote{The Heb. lacks \fbib{tests}} a man. \\
\poeml \v{22}Though you crush a fool in a mortar and pestle \\
\poemll    as someone might crush grain, \\
\poemlll       his stupidity still won't leave him. \\
\poeml \v{23}Keep well informed of the condition of your flocks \\
\poemll    and pay attention to your herds, \\
\poeml \v{24}because riches don't endure forever, \\
\poemll    and crowns don't last from one generation to the next. \\
\poeml \v{25}When the grass disappears, \\
\poemll    and new growth appears, \\
\poemlll       the mountain spices will be harvested, \\
\poeml \v{26}the lambs will supply your clothing, \\
\poemll    and your goats the price of a field. \\
\poeml \v{27}You will have enough goat's milk to drink \\
\poemll    and to supply your household needs, \\
\poemlll       as well as sustenance for your servant girls.
\end{poetry}
\labelchapt{28}
\passage{Contrasting Good and Evil}

\begin{poetry}
\poeml \chapt{28}
\v{1}The wicked flee, though no one pursues, \\
\poeml but the righteous are bold like a lion. \\
\poeml \v{2}When a land transgresses, \\
\poemll    it gains a succession of leaders, \\
\poeml but with an understanding and knowledgeable man, \\
\poemll    its stability endures. \\
\poeml \v{3}A poor man who oppresses the weak \\
\poemll    is like a rainstorm that destroys all\fnote{Lit. \fbib{that leaves no}} the crops. \\
\poeml \v{4}Those who forsake the Law praise the wicked, \\
\poemll    but whoever keeps it\fnote{Lit. \fbib{keeps the Law}} fights them. \\
\poeml \v{5}Evil men don't understand justice, \\
\poemll    but whoever seeks the \divine{Lord} understands it all. \\
\poeml \v{6}It's better to be poor and live a blameless life \\
\poemll    than to be rich but crooked in one's lifestyle. \\
\poeml \v{7}Whoever keeps the Law is a discerning son, \\
\poemll    but whoever keeps company with gluttons \\
\poemlll       brings shame to his father. \\
\poeml \v{8}Whoever gains wealth by charging exorbitant\fnote{Lit. \fbib{charging interest upon}} interest \\
\poemll    collects it for someone who is kind to the poor. \\
\poeml \v{9}If someone quits\fnote{Lit. \fbib{turns away from}} listening to the Law \\
\poemll    even his prayer is detestable. \\
\poeml \v{10}Whoever misleads the upright along an evil way \\
\poemll    will himself fall into his own pit, \\
\poemlll       but the blameless will inherit what is good. \\
\poeml \v{11}The rich man may be wise in his own opinion; \\
\poemll    but a discerning, poor man sees through him. \\
\poeml \v{12}When the righteous are victorious, there is great glory, \\
\poemll    but when the wicked arise, men hide themselves. \\
\poeml \v{13}Whoever hides his transgressions will not succeed, \\
\poemll    but whoever confesses and forsakes them will find mercy. \\
\poeml \v{14}Blessed is the man who always fears the \divine{Lord},\fnote{The Heb. lacks \fbib{the \divine{Lord}}} \\
\poemll    but whoever hardens his heart will fall into disaster. \\
\poeml \v{15}A roaring lion and a charging bear--- \\
\poemll    that's what a wicked tyrant is over poor people. \\
\poeml \v{16}A Commander-in-Chief\fnote{Lit. \fbib{Nagid}; i.e. a senior officer entrusted with dual roles of operational oversight and administrative authority} who is a cruel oppressor lacks understanding, \\
\poemll    but whoever hates unjust gain will live longer.\fnote{Lit. \fbib{will lengthen his days}} \\
\poeml \v{17}A guilty man tormented by bloodshed \\
\poemll    will be a lifelong fugitive; \\
\poemlll       let no one support him. \\
\poeml \v{18}Whoever lives blamelessly will be delivered, \\
\poemll    but whoever is perverted will fall without warning. \\
\poeml \v{19}Whoever works his farmland will have abundant food, \\
\poemll    but whoever chases fantasies will become very poor. \\
\poeml \v{20}The faithful man will prosper with blessings, \\
\poemll    but whoever is in a hurry to get rich \\
\poemlll       will not escape punishment. \\
\poeml \v{21}To show partiality isn't good, \\
\poemll    yet for a piece of bread the valiant will transgress. \\
\poeml \v{22}The miser\fnote{Lit. \fbib{The man with an evil eye}} is in a hurry to get wealthy, \\
\poemll    but he isn't aware that poverty will catch up with him. \\
\poeml \v{23}Whoever rebukes a man will later on find more favor \\
\poemll    than someone who flatters with his words.\fnote{Lit. \fbib{tongue}} \\
\poeml \v{24}Whoever steals from his father or mother \\
\poemll    but claims, ``It's no sin,'' \\
\poemlll       is a companion to someone who demolishes. \\
\poeml \v{25}An arrogant\fnote{Or \fbib{greedy}} man stirs up dissension, \\
\poemll    but anyone who trusts in the \divine{Lord} prospers. \\
\poeml \v{26}Whoever trusts in himself is foolish, \\
\poemll    but whoever lives wisely will be kept safe. \\
\poeml \v{27}Whoever gives to the poor will never lack, \\
\poemll    but whoever shuts his eyes to their poverty\fnote{The Heb. lacks \fbib{to their poverty}} will be cursed. \\
\poeml \v{28}When the wicked rise to power, people hide themselves, \\
\poemll    but when the wicked\fnote{Lit. \fbib{when they}} perish, the righteous increase.
\end{poetry}
\labelchapt{29}
\passage{Advice on Life and Justice}

\begin{poetry}
\poeml \chapt{29}
\v{1}After many rebukes, the stiff-necked man \\
\poeml will be broken incurably, without any warning. \\
\poeml \v{2}As the righteous grow powerful,\fnote{The Heb. lacks \fbib{powerful}} people rejoice; \\
\poemll    but when the wicked rule, people groan. \\
\poeml \v{3}The man who loves wisdom brings joy to his father, \\
\poemll    but anyone who consorts with immoral women squanders his wealth. \\
\poeml \v{4}A king brings stability to a land through justice, \\
\poemll    but a man who takes bribes brings it to ruin. \\
\poeml \v{5}A strong man who flatters his neighbor \\
\poemll    is setting a trap where he walks.\fnote{Lit. \fbib{trap for his footsteps}}
\passage{The Wicked and Righteous Contrasted}
\poeml \v{6}An evil man is trapped by transgression, \\
\poemll    but the righteous person sings and rejoices. \\
\poeml \v{7}The righteous person is concerned about the poor; \\
\poemll    but the wicked don't understand what they need to know.\fnote{Lit. \fbib{understand knowledge}} \\
\poeml \v{8}Scornful men enflame a city, \\
\poemll    but the wise defuse anger. \\
\poeml \v{9}When a wise man has a dispute with a fool, \\
\poemll    the fool either rages or laughs without relief. \\
\poeml \v{10}Bloodthirsty men hate the innocent person, \\
\poemll    but the upright show concern for his life. \\
\poeml \v{11}The fool vents all his feelings,\fnote{Lit. \fbib{spirit}} \\
\poemll    but the wise person keeps them to himself.\fnote{The Heb. lacks \fbib{to himself}} \\
\poeml \v{12}When a ruler is listening to lies, \\
\poemll    all of his officials tend to become wicked. \\
\poeml \v{13}The poor man and the oppressor have this in common:\fnote{Lit. \fbib{oppressor meet together}} \\
\poemll    the \divine{Lord} gave them both eyes with which to see.\fnote{Lit. \fbib{\divine{Lord} lights the eyes of both}} \\
\poeml \v{14}When a king faithfully administers justice to the poor, \\
\poemll    his throne will be established forever. \\
\poeml \v{15}The rod and rebuke bestow wisdom, \\
\poemll    but an undisciplined child\fnote{Lit. \fbib{but a child left alone}} brings shame to his mother. \\
\poeml \v{16}As the wicked grow powerful,\fnote{The Heb. lacks \fbib{powerful}} transgression increases, \\
\poemll    but the righteous will observe their downfall. \\
\poeml \v{17}Correct your son and he will comfort you; \\
\poemll    he will also delight your soul. \\
\poeml \v{18}Without prophetic vision, people abandon restraint, \\
\poemll    but those who obey the Law are happy.
\passage{Dangerous Behaviors}
\poeml \v{19}By mere words a servant will not be corrected; \\
\poemll    even though he understands, \\
\poemlll       there will be no response. \\
\poeml \v{20}Do you see a man who speaks hastily? \\
\poemll    There is more hope for a fool than for him. \\
\poeml \v{21}If you pamper a servant from his childhood, \\
\poemll    later on he'll become ungrateful. \\
\poeml \v{22}An angry man stirs up arguments, \\
\poemll    and a hot-tempered man causes many transgressions. \\
\poeml \v{23}A person's pride will bring about his downfall, \\
\poemll    but the humble in spirit will gain honor. \\
\poeml \v{24}A thief's accomplice hates himself; \\
\poemll    though testifying under oath,\fnote{Lit. \fbib{though he hears the oath}} he reveals nothing. \\
\poeml \v{25}Fearing any human being is a trap, \\
\poemll    but confiding in the \divine{Lord} keeps anyone safe. \\
\poeml \v{26}Many seek a ruler's favor,\fnote{Lit. \fbib{face}} \\
\poemll    but justice for a man comes from the \divine{Lord}. \\
\poeml \v{27}The unjust man is detestable to the righteous, \\
\poemll    and whoever lives blamelessly is detestable to the wicked.
\end{poetry}
\labelchapt{30}
\passage{The Oracle}

\chapt{30}
\v{1}A discourse by the faithful collector.\fnote{Or \fbib{by Jakeh's son Agur}}

\begin{poetry}
\poeml This is what this valiant man declared to the God with me, \\
\poemll    to the God with me, who then prevailed:\fnote{Or \fbib{declared to Ithiel, to Ithiel, and Ucal}} \\
\poeml \v{2}Surely I am beyond the senselessness of any man; \\
\poemll    I do not perceive things\fnote{The Heb. lacks \fbib{things}} the way human beings do. \\
\poeml \v{3}I never acquired wisdom, \\
\poemll    but I know what the Holy One knows. \\
\poeml \v{4}Who has ascended to heaven, \\
\poemll    and then descended? \\
\poemlll       Who has collected the wind in his hands? \\
\poeml Who has wrapped up waters in a garment? \\
\poemll    Who has established all the farthest points of the earth? \\
\poeml What is his name, \\
\poemll    and what is his son's name? \\
\poemlll       Surely you know! \\
\poeml \v{5}Everything God says is pure; \\
\poemll    he is a shield for those who take refuge in him. \\
\poeml \v{6}Don't add to his words, \\
\poemll    or he will rebuke you, \\
\poemlll       and you will be shown to be a liar.
\passage{On Contentment in Life}
\poeml \v{7}God,\fnote{The Heb. lacks \fbib{God}} I have asked you for two things--- \\
\poemll    don't refuse me before I die--- \\
\poeml \v{8}Keep deception and lies far away from me, \\
\poemll    and give me neither poverty nor wealth. \\
\poeml Feed me with food that I need for today,\fnote{Or \fbib{that is appropriate for me}} \\
\poeml \v{9}so that I don't become overfed and deny you by saying, \\
\poemlll       ``Who is the \divine{Lord}?'' \\
\poeml or so that I don't become poor and steal, \\
\poemll    and then misuse the name of my God.
\passage{On Different Kinds of People}
\poeml \v{10}Don't lie about a servant to his master, \\
\poemll    or that servant\fnote{Lit. \fbib{or he}} will curse you and you will pay for it. \\
\poeml \v{11}Some people\fnote{Lit. \fbib{A} \fbib{generation}} curse their fathers \\
\poemll    and won't bless their mothers. \\
\poeml \v{12}Some people\fnote{Lit. \fbib{A generation}} view themselves as pure, \\
\poemll    but haven't been cleansed from their own filth. \\
\poeml \v{13}Some people\fnote{Lit. \fbib{A generation}}---what an arrogant look they have!--- \\
\poemll    raise their eyebrows haughtily. \\
\poeml \v{14}Some people\fnote{Lit. \fbib{A generation}} have swords for teeth \\
\poemll    and knives for fangs \\
\poeml to devour the afflicted from the earth \\
\poemll    and the needy from among mankind. \\
\poeml \v{15}The leech has two daughters: \\
\poemll    ``Give'' and ``Give''. \\
\poeml Three things will never be satisfied; \\
\poemll    four will never say ``Enough''--- \\
\poeml \v{16}The afterlife,\fnote{Lit. \fbib{Sheol}; i.e. the realm of the dead} the barren womb, \\
\poemll    earth that still demands water, \\
\poemlll       and fire---they never say, ``Enough''. \\
\poeml \v{17}The eye that mocks a father \\
\poemll    and looks with a disobedient attitude at\fnote{Lit. \fbib{and despises obedience to}} a mother--- \\
\poeml the valley ravens will pluck it out; \\
\poemll    and vultures will eat it.
\passage{What Causes Wonder}
\poeml \v{18}Three things cause wonder for me; \\
\poemll    four are beyond my understanding: \\
\poeml \v{19}The way an eagle flies in the sky, \\
\poemll    the way of a serpent on a rock, \\
\poeml the way of a ship on the high seas, \\
\poemll    and the way of a man with a young woman. \\
\poeml \v{20}This is what an immoral woman is like: \\
\poemll    she eats, wipes her mouth, then says \\
\poemlll       ``I've done nothing wrong.'' \\
\poeml \v{21}Under three things the earth trembles, \\
\poemll    under four it cannot remain steady: \\
\poeml \v{22}Under a slave when he becomes a king, \\
\poemll    a fool when he is overfed, \\
\poeml \v{23}an unloved woman when she finds a husband, \\
\poemll    and a servant girl who inherits from her mistress. \\
\poeml \v{24}Four things on earth are small, \\
\poemll    but they are very, very wise: \\
\poeml \v{25}Ants aren't a strong species,\fnote{Lit. \fbib{people}} \\
\poemll    yet they prepare their food in the summer. \\
\poeml \v{26}The rock badgers aren't a strong species\fnote{Lit. \fbib{people}} either, \\
\poemll    yet they build their dens in the rocks. \\
\poeml \v{27}Locusts have no king, \\
\poemll    but they all swarm in ranks. \\
\poeml \v{28}Spiders can be caught by the hand, \\
\poemll    yet they're found in kings' palaces. \\
\poeml \v{29}Three things are stately in procession, \\
\poemll    four which are stately in their gait: \\
\poeml \v{30}The lion, mighty among the beasts, \\
\poemll    retreats before nothing. \\
\poeml \v{31}The strutting rooster, as well as the goat, \\
\poemll    and a king with his army. \\
\poeml \v{32}If you've foolishly exalted yourself \\
\poemll    or if you've plotted evil, \\
\poemlll       put your hand over your mouth. \\
\poeml \v{33}Just as whipping milk produces butter, \\
\poemll    and twisting the nose causes bleeding, \\
\poemlll       so also stirring up anger produces contention.
\end{poetry}
\labelchapt{31}
\passage{Counsel from King Lemuel's Mother}

\begin{poetry}
\poeml \chapt{31}
\v{1}The words of King Lemuel--- \\
\poeml a pronouncement with which his mother encouraged him. \\
\poeml \v{2}No,\fnote{Or \fbib{What}} my son! \\
\poemll    No,\fnote{Or \fbib{What}} my son whom I conceived!\fnote{Lit. \fbib{son of my womb?}} \\
\poemlll       No,\fnote{Or \fbib{What}} my son to whom I gave birth!\fnote{Lit. \fbib{son of my vows}} \\
\poeml \v{3}Never devote all your energy to sex,\fnote{Lit. \fbib{women}} \\
\poemll    or dedicate your life\fnote{Lit. \fbib{ways}} to destroying kings. \\
\poeml \v{4}It is not for kings, Lemuel--- \\
\poemll    Not for kings to drink wine \\
\poemlll       or for rulers to desire liquor. \\
\poeml \v{5}Otherwise, they may drink and forget what has been ordained, \\
\poemll    perverting justice for all the oppressed. \\
\poeml \v{6}Give liquor to someone who is perishing, \\
\poemll    and wine to someone who is deeply depressed.\fnote{Lit. \fbib{one whose soul is bitter}} \\
\poeml \v{7}Let him drink, forget his poverty, \\
\poemll    and remember his troubles no more. \\
\poeml \v{8}Speak for those who cannot speak; \\
\poemll    seek justice for all those on the verge of destruction.\fnote{Lit. \fbib{all sons of destruction}} \\
\poeml \v{9}Speak up, judge righteously, \\
\poemll    and defend the rights of the afflicted and oppressed.
\passage{The Honorable Woman}
\poeml \v{10}Who can find a capable wife? \\
\poemll    Her value far exceeds the finest jewels. \\
\poeml \v{11}Her husband has full confidence in her; \\
\poemll    as a result, he lacks nothing of value. \\
\poeml \v{12}She will bring good to him---never evil--- \\
\poemll    every day of her life. \\
\poeml \v{13}She seeks out wool and flax, \\
\poemll    working with eager hands. \\
\poeml \v{14}She is like a seagoing ship \\
\poemll    that brings her food from far away. \\
\poeml \v{15}She rises while it is still night, \\
\poemll    preparing meals for her family \\
\poemlll       and providing for her women servants. \\
\poeml \v{16}She evaluates a field and purchases it; \\
\poemll    from the proceeds she plants a vineyard. \\
\poeml \v{17}She clothes herself with fortitude, \\
\poemll    and fortifies her arms with strength. \\
\poeml \v{18}She is confident that her profits are sufficient. \\
\poemll    Her lamp does not go out at night. \\
\poeml \v{19}She works with her own hands on her clothes\fnote{Lit. \fbib{staff}}--- \\
\poemll    her hands work the sewing spindle. \\
\poeml \v{20}She reaches out to the poor, \\
\poemll    opening her hands to those in need. \\
\poeml \v{21}She is unafraid of winter's effect on\fnote{Lit. \fbib{of the snow for}} her household, \\
\poemll    because all of them are warmly clothed.\fnote{Lit. \fbib{are clothed in red}} \\
\poeml \v{22}She creates her own bedding, \\
\poemll    using fine linen and violet cloth. \\
\poeml \v{23}Her husband is well known;\fnote{Lit. \fbib{is known in the gates}} \\
\poemll    he sits among the leaders of the land. \\
\poeml \v{24}She designs and sells linen garments, \\
\poemll    supplying accessories\fnote{Or \fbib{belts}} to clothiers. \\
\poeml \v{25}Strength and dignity are her garments; \\
\poemll    she smiles about the future. \\
\poeml \v{26}She speaks wisely, \\
\poemll    teaching with gracious love. \\
\poeml \v{27}She looks discretely to the affairs of her household, \\
\poemll    and she is never lazy.\fnote{Lit. \fbib{she does not eat the food of laziness}} \\
\poeml \v{28}Her children stand up and encourage her, \\
\poemll    as does her husband, who praises her: \\
\poeml \v{29}``Many women have done wonderful things,'' he says,\fnote{The Heb. lacks \fbib{he says}} \\
\poemll    ``but you surpass all of them!'' \\
\poeml \v{30}Charm is deceitful and beauty fades; \\
\poemll    but a woman who fears the \divine{Lord} will be praised. \\
\poeml \v{31}Reward her for her work--- \\
\poemll    let her actions result in public praise.\fnote{Lit. \fbib{in praise in the gates}}\end{poetry}

\bookheader{Ecclesiastes}
\labelbook{Eccl}

\bookpretitle{The Book of}
\booktitle{Ecclesiastes}

\labelchapt{1}
\passage{The Theme}

\chapt{1}
\v{1}The words of the Teacher,\fnote{\fbackref{1:1} Or \fbib{Speaker}, Or \fbib{Philosopher}, and so throughout the book} the son of David, king in Jerusalem.

\begin{poetry}
\poeml \v{2}``Utterly pointless,''\fnote{\fbackref{1:2} Or \fbib{Utter vanity}; and so throughout the book} \\
\poemlll       says the Teacher. \\
\poeml ``Absolutely pointless; \\
\poemll    everything is pointless.'' \\
\poeml \v{3}What does a man gain \\
\poemll    from all of the work that he undertakes on earth?\fnote{\fbackref{1:3} Lit. \fbib{under the sun}; i.e. from an earthly perspective; and so throughout the book}
\passage{The Predictability of Life}
\poeml \v{4}A generation goes, \\
\poemll    a generation comes, \\
\poemlll       but the earth remains forever. \\
\poeml \v{5}The sun rises, \\
\poemll    the sun sets, \\
\poemlll       then rushes back to where it arose. \\
\poeml \v{6}The wind blows southward, \\
\poemll    then northward, constantly circulating, \\
\poemlll       and the wind comes back again in its courses. \\
\poeml \v{7}All the rivers flow toward the sea, \\
\poemll    but the sea is never full; \\
\poemlll       then rivers return to the headwaters\fnote{\fbackref{1:7} Lit. \fbib{place}} where they began. \\
\poeml \v{8}Everything is wearisome, \\
\poemll    more than man is able to express. \\
\poeml The eye is never satisfied by seeing, \\
\poemll    nor the ear by hearing. \\
\poeml \v{9}Whatever has happened, will happen again; \\
\poemll    whatever has been done, will be done again. \\
\poemlll       There is nothing new on earth. \\
\poeml \v{10}Does anything exist about which someone might say, \\
\poemll    ``Look at this! Is this new?'' \\
\poeml It happened ages ago; \\
\poemll    it existed before we did. \\
\poeml \v{11}No one remembers those in the past, \\
\poemll    nor will they be remembered \\
\poemlll       by those who come after them.
\end{poetry}
\passage{A Testimony to an Unwise Search}

\v{12}I, the Teacher, have been king over Israel in Jerusalem. \v{13}I dedicated myself to using wisdom for study and discovery of everything that is done under heaven.\fnote{\fbackref{1:13} I.e. from a heavenly perspective} God uses terrible things so human beings will struggle with life.\fnote{\fbackref{1:13} The Heb. lacks \fbib{with life}} \v{14}I observed every activity done on earth. My conclusion: all of it is pointless---like chasing after the wind.

\begin{poetry}
\poeml \v{15}What is crooked cannot be made straight; \\
\poemll    what is not there cannot be counted.
\end{poetry}

\v{16}I told myself, ``I have become greater and wiser than anyone who ruled before me in Jerusalem---yes, I have acquired a great deal of wisdom and knowledge.'' \v{17}So I dedicated myself to learn about wisdom and knowledge, and about insanity and foolishness. And I discovered that this is also like chasing after the wind.

\begin{poetry}
\poeml \v{18}For with much wisdom there is much sorrow; \\
\poemll    the more someone adds to knowledge, \\
\poemlll       the more someone adds to grief.
\end{poetry}
\labelchapt{2}
\passage{A Testimony to Self-Indulgence}

\chapt{2}
\v{1}I told myself, ``I will test you with pleasure, so enjoy yourself.'' But this was pointless. \v{2}``Senseless,'' said I concerning laughter and pleasure, ``How practical is this?'' \v{3}I decided to indulge in wine, while still remaining committed to wisdom. I also tried to indulge in foolishness, just enough to determine whether it was good for human beings under heaven given the short time of their lives.
\passage{A Testimony to Extravagant Works}

\v{4}With respect to my extravagant works, I built houses for myself; I planted vineyards for myself. \v{5}I constructed gardens and orchards for myself, and within them I planted all kinds of fruit trees. \v{6}I built for myself water reservoirs to irrigate forests that produce trees.
\passage{A Testimony to Extravagant Possessions}

\v{7}I acquired male and female slaves, and had other slaves born in my house. I also acquired for myself increasing numbers of herds and flocks---more than anyone who had lived before me in Jerusalem. \v{8}I also accumulated silver, gold, and the wealth of kings and their kingdoms. I gathered around me both male and female singers, along with what delights a man---all sorts of mistresses.
\passage{A Testimony to Extravagant Position}

\v{9}So I became great, greater than anyone who had lived before me in Jerusalem. Throughout all of this, I remained wise. \v{10}Whenever I wanted something I had seen, I never refused that desire. Instead, I enjoyed everything I did, and this became the reward in what I had undertaken. \v{11}Then I examined all of my accomplishments that I had brought about by my own efforts, including the work that I had labored so hard to complete---and it was all pointless, like chasing after the wind, and there was nothing to be gained on earth.

\v{12}Next I turned to examine wisdom, insanity, and foolishness, because what can a person do who succeeds the king except what has already been accomplished? \v{13}I concluded that wisdom is more useful than foolishness, just as light is more useful than darkness. \v{14}The wise use their eyes, but the fool walks in darkness. I also perceived that the same outcome affects them all.
\passage{The Pointlessness of Life}

\v{15}Then I told myself, ``Whatever happens to the fool will happen also to me. Therefore what's the point in being so wise?'' And I told myself that this also is pointless. \v{16}For neither the wise nor the fool will be long remembered, since in days to come everything will be forgotten. The wise man dies the same way as the fool, does he not? \v{17}So I hated life, because whatever is done on earth causes me trouble---it's all pointless, like chasing after the wind.
\passage{The Pointlessness of Labor}

\v{18}Then I despised everything I had worked for on earth, that is, the things that I will leave to the person who will succeed me. \v{19}And who knows whether he will be wise or foolish? Either way, he will take possession of everything that I have done on earth, especially where I have excelled. This also is pointless. \v{20}So I came to be in despair about everything I had accomplished on earth. \v{21}For sometimes people who strive to obtain wisdom, knowledge, and equity leave everything as an inheritance to a person who never worked for it. This, too, is pointless and greatly troublesome.

\v{22}For what does a person gain from everything that he accomplishes and from his inner life struggles that he undergoes while working on earth? \v{23}Indeed, all of his days are filled with sorrow, and his struggles bring grief. In fact, his mind remains restless throughout the night. This is pointless, too!
\passage{The Central Point of Life}

\v{24}The only worthwhile thing for a human being is to eat, drink, and enjoy life's goodness that he finds in what he accomplishes. This, I observed, is also from the hand of God himself, \v{25}for who can eat or enjoy life apart from him? \v{26}After all, to the person who is good in God's sight, he gives wisdom, knowledge, and joy, but to the sinner he gives the troublesome task of acquiring and accumulating in order to leave it to someone who is good in the sight of God. This also is pointless and chasing after the wind.
\labelchapt{3}
\passage{The Purposes in God's Timing}

\begin{poetry}
\poeml \chapt{3}
\v{1}There is a season for everything, \\
\poeml and a time for every event under heaven:\fnote{\fbackref{3:1} I.e. from a heavenly perspective} \\
\poeml \v{2}a time to be born, and a time to die; \\
\poemll    a time to plant, and a time to uproot what was planted; \\
\poeml \v{3}a time to kill, and a time to heal; \\
\poemll    a time to tear down, and a time to build up; \\
\poeml \v{4}a time to weep, and a time to laugh; \\
\poemll    a time to mourn, and a time to dance; \\
\poeml \v{5}a time to scatter stones, and a time to gather stones; \\
\poemll    a time to embrace, and a time to refrain from embracing; \\
\poeml \v{6}a time to search, and a time to give up searching;\fnote{\fbackref{3:6} The Heb. lacks \fbib{searching}} \\
\poemll    a time to keep, and a time to discard; \\
\poeml \v{7}a time to tear, and a time to mend; \\
\poemll    a time to be silent, and a time to speak; \\
\poeml \v{8}a time to love, and a time to hate; \\
\poemll    a time for war, and a time for peace.
\end{poetry}
\passage{The Purpose of Life}

\v{9}What benefit does the worker gain from what he undertakes? \v{10}I have observed the burdens placed by God on human beings in order to perfect them. \v{11}He made everything appropriate in its time. He also placed eternity within them---yet, no person can fully comprehend what God is doing from beginning to end.

\v{12}I have concluded that the only worthwhile thing for them is to take pleasure in doing good in life; \v{13}moreover, every person should eat, drink, and enjoy the benefits of everything that he undertakes, since it is a gift from God.

\v{14}I have concluded that everything that God undertakes will last for eternity---nothing can be added to it nor taken away from it---and that God acts this way so that people will fear him. \v{15}That which was, now is; and that which will be, already is; and God examines what has already taken place.
\passage{From Dust to Dust}

\v{16}I also examined on earth:

\begin{poetry}
\poeml where the halls of justice were supposed to be, \\
\poemll    there was lawlessness; \\
\poeml and where the righteous were supposed to be,\fnote{\fbackref{3:16} Lit. \fbib{and the place of judgment}} \\
\poemll    there was lawlessness.
\end{poetry}

\v{17}I told myself, ``God will judge both the righteous and the wicked, because there is a time set to judge\fnote{\fbackref{3:17} The Heb. lacks \fbib{to judge}} every event and every work.''

\v{18}``As for human beings,'' I told myself, ``God puts them to the test, that they might see themselves as mere animals.'' \v{19}For what happens to people also happens to animals---a single event happens to them: just as someone dies, so does the other. In fact, they all breathe the same way, so that a human being has no superiority over an animal. All of this is pointless. \v{20}All of them go to one place: all of them originate from dust, and all of them return to dust.

\v{21}Who knows whether\fnote{\fbackref{3:21} So LXX. The Heb. lacks \fbib{whether}} the spirit of human beings ascends, and whether\fnote{\fbackref{3:21} So LXX. The Heb. lacks \fbib{whether}} the spirit of animals descends to the earth? \v{22}I concluded that it is worthwhile for people to find joy in their accomplishments, because that is their inheritance, since who can see what will exist after them?
\labelchapt{4}
\passage{On the Abuse of Authority}

\chapt{4}
\v{1}Next I turned to consider all kinds of oppression that exists on earth.

\begin{poetry}
\poeml Look at the tears of the oppressed--- \\
\poemll    there is no one to comfort them. \\
\poeml Power is on the side of their oppressors; \\
\poemll    so they have no comforters.
\end{poetry}

\v{2}So I commended the dead who had already died as being happier than the living who are still alive. \v{3}Better than both of them is someone who has not yet been born,\fnote{\fbackref{4:3} The Heb. lacks \fbib{born}} because he hasn't experienced evil on earth. \v{4}Then I examined all sorts of work, including all kinds of excellent achievements that create envy in others.\fnote{\fbackref{4:4} Lit. \fbib{envy of a man by his neighbor}} This also is pointless and chasing after the wind. \v{5}The fool crosses his arms\fnote{\fbackref{4:5} Lit. \fbib{folds his hands}} and starves himself.\fnote{\fbackref{4:5} Lit. \fbib{eats his own flesh}} \v{6}It's better to have one handful of tranquility than to have two handfuls of trouble and to chase after the wind.
\passage{On Aloneness and Companionship}

\v{7}Then I turned to re-examine something else that is pointless on earth: \v{8}Consider someone who is alone, having neither son nor brother. There is no end to all of his work, and he is\fnote{\fbackref{4:8} Lit. \fbib{and his eyes are}} never satisfied with wealth. ``So for whom do I work,'' he asks,\fnote{\fbackref{4:8} The Heb. lacks \fbib{he asks}} ``and deprive myself of pleasure?'' This, too, is pointless and a terrible tragedy.

\v{9}Two are better than one, because they have a good return for their labor. \v{10}If they stumble, the first will lift up his friend---but woe to anyone who is alone when he falls and there is no one to help him get up. \v{11}Again, if two lie close together, they will keep warm, but how can only one stay warm? \v{12}If someone attacks one of them, the two of them together will resist. Furthermore, the tri-braided cord is not soon broken.
\passage{There's No Fool Like an Old Fool}

\begin{poetry}
\poeml \v{13}A poor but wise youth is better \\
\poemll    than an old but foolish king \\
\poemlll       who will no longer accept correction. \\
\poeml \v{14}The former can come out of prison to reign, \\
\poemll    while the latter, even if born to\fnote{\fbackref{4:14} Lit. \fbib{to his}} kingship, may become poor.
\end{poetry}

\v{15}I observed everyone who lives and walks on earth, along with the youth\fnote{\fbackref{4:15} Lit. \fbib{second child}} who will take the king's\fnote{\fbackref{4:15} Lit. \fbib{take his}} place. \v{16}There was no end to all of his subjects\fnote{\fbackref{4:16} Lit. \fbib{of the people}} or to all of the people who had come before them. But those who come along afterward will not be happy with him. This is also pointless and a chasing after wind.
\labelchapt{5}
\passage{Advice in Worship}

\chapt{5}
\v{1}\fnote{\fbackref{5:1} This v. is 4:17 in MT}Watch your step whenever you visit God's house, and come more ready to listen than to offer a fool's sacrifice, since fools\fnote{\fbackref{5:1} Lit. \fbib{they}} never think they're doing evil.

\begin{poetry}
\poeml \v{2}\fnote{\fbackref{5:2} This v. is 5:1 in MT, and so throughout the chapter.}Don't be impulsive with your mouth \\
\poemll    nor be in a hurry to talk in God's presence. \\
\poeml Since God is in heaven \\
\poemll    and you're on earth, \\
\poemlll       keep your speech short. \\
\poeml \v{3}Too many worries lead to nightmares, \\
\poemll    and a fool is known from talking too much.
\end{poetry}
\passage{Keep Your Promises to God}

\v{4}When you make a promise to God, don't fail to keep it, since he isn't pleased with fools. Keep what you promise--- \v{5}it's better that you don't promise than that you do promise and not follow through.\fnote{\fbackref{5:5} Or \fbib{not pay}} \v{6}Never let your mouth cause you\fnote{\fbackref{5:6} Lit. \fbib{cause your body}} to sin and don't proclaim in the presence of the angel,\fnote{\fbackref{5:6} LXX reads \fbib{of God}} ``My promise\fnote{\fbackref{5:6} Lit. \fbib{It}} was a mistake,'' for why should God be angry at your excuse\fnote{\fbackref{5:6} Lit. \fbib{voice}} and destroy what you've undertaken? \v{7}In spite of many daydreams, pointless actions, and empty words, it is more important to fear God.
\passage{The Use and Abuse of Wealth}

\v{8}Don't be surprised when you see the poor oppressed and the violent perverting both justice and verdicts\fnote{\fbackref{5:8} Or \fbib{judgment}} in a province, for one high official watches another, and there are ones higher still over them. \v{9}Also, the increase of the land belongs to everyone; the king himself is served by his\fnote{\fbackref{5:9} The Heb. lacks \fbib{his}} field.

\begin{poetry}
\poeml \v{10}Whoever loves money will never have enough money. \\
\poemll    Whoever loves luxury will not be content with abundance. \\
\poemlll       This also is pointless. \\
\poeml \v{11}When possessions increase, \\
\poemll    so does the number of consumers; \\
\poeml therefore what good are they to their owners, \\
\poemll    except to look at them? \\
\poeml \v{12}Sweet is the sleep of a working man, \\
\poemll    whether he eats a little or a lot, \\
\poeml but the excess wealth of the rich \\
\poemll    will not allow him to rest.
\end{poetry}

\v{13}I have observed a painful tragedy on earth:

\begin{poetry}
\poeml Wealth hoarded by its owner harms him, \\
\poeml \v{14}and that wealth is lost in troubled circumstances. \\
\poeml Then a son is born, \\
\poemll    but there is nothing left for him.\fnote{\fbackref{5:14} Lit. \fbib{nothing in his hand}} \\
\poeml \v{15}Just as he came naked from his mother's womb, \\
\poemll    he will leave\fnote{\fbackref{5:15} Lit. \fbib{return}} as naked as he came; \\
\poeml he will receive no profit from his efforts--- \\
\poemll    he cannot carry away even a handful.
\end{poetry}

\v{16}This is also a painful tragedy:

\begin{poetry}
\poeml However a person comes, he also departs; \\
\poemll    so what does he gain as he labors after the wind? \\
\poeml \v{17}Furthermore, all his days he lives\fnote{\fbackref{5:17} Lit. \fbib{eats}} in darkness \\
\poemll    with great sorrow, anger, and affliction.
\end{poetry}
\passage{The Use and Abuse of Accomplishment}

\v{18}Look! I observed that it is good and prudent to eat, drink, and enjoy all that is good of a person's\fnote{\fbackref{5:18} Lit. \fbib{of his}} work that he does on earth during the limited days of his life, which God gives him, for this is his allotment. \v{19}Furthermore, for every person to whom God has given wealth, riches, and the ability to enjoy them, to accept this allotment, and to rejoice in his work---this is a gift from God. \v{20}For he will not brood much over the days of his life, since God will keep him occupied with the joys of his heart.
\labelchapt{6}
\passage{Enjoyment of Life as a Gift from God}

\chapt{6}
\v{1}There exists another misfortune that I have observed on earth, and it is a heavy burden upon human beings: \v{2}a man to whom God gives wealth, riches, and honor, so that he lacks none of his heart's desires---but God does not give him the capability to enjoy them. Instead, a stranger consumes them. This is pointless and a grievous affliction.

\v{3}A man might father a hundred children,\fnote{\fbackref{6:3} The Heb. lacks \fbib{children}} and live for many years, so that the length of his life\fnote{\fbackref{6:3} Lit. \fbib{years}} is long---but if his life does not overflow with goodness, and he doesn't receive a proper\fnote{\fbackref{6:3} The Heb. lacks \fbib{proper}} burial, I maintain that stillborn children\fnote{\fbackref{6:3} Lit. \fbib{child}; and so through v. 5} are better off than he is, \v{4}because stillborn children\fnote{\fbackref{6:3} Lit. \fbib{because he}} arrive in pointlessness, leave in darkness, and their names are covered in darkness. \v{5}Furthermore, though they never saw the sun nor learned anything,\fnote{\fbackref{6:5} The Heb. lacks \fbib{anything}} they are more content than the other. \v{6}Even if he lives a thousand years twice over without experiencing the best---aren't all of them going to the same place?

\begin{poetry}
\poeml \v{7}Every person works for his own self-interests,\fnote{\fbackref{6:7} Lit. \fbib{for his mouth}} \\
\poemll    but his desires remain unsatisfied. \\
\poeml \v{8}For what advantage has the wise person over the fool? \\
\poemll    What advantage does the poor man have \\
\poemlll       in knowing how to face life?\fnote{\fbackref{6:8} Lit. \fbib{knows to walk before the living}} \\
\poeml \v{9}It is better to focus on what you can see \\
\poemll    than to meander after your self-interest; \\
\poemlll       this also is pointless and a chasing after wind. \\
\poeml \v{10}Whatever exists has been named already;\fnote{\fbackref{6:10} I.e. its destiny is known} \\
\poemll    people know what it means\fnote{\fbackref{6:10} Lit. \fbib{already; it is known}} to be human--- \\
\poemlll       and a person cannot defeat one who is more powerful than he. \\
\poeml \v{11}Because many words lead to pointlessness, \\
\poemll    how do people benefit from this?
\end{poetry}

\v{12}Who knows what is best for people in this life, every day of their pointless lives that they pass through\fnote{\fbackref{6:12} Or \fbib{they spend}} like a shadow? Who informs people on earth what will come along after them?
\labelchapt{7}
\passage{Lessons for Life}

\begin{poetry}
\poeml \chapt{7}
\v{1}A good name exceeds the value of fine perfume, \\
\poemll    and the day of someone's death exceeds the value of\fnote{\fbackref{7:1} Lit. \fbib{death than}} the day of his birth. \\
\poeml \v{2}It's better to attend a funeral\fnote{\fbackref{7:2} Lit. \fbib{house of mourning}} \\
\poemll    than to attend a banquet,\fnote{\fbackref{7:2} Lit. \fbib{house of feasting}} \\
\poeml for everyone dies eventually, \\
\poemll    and the living will take this to heart. \\
\poeml \v{3}Sorrow is better than laughter, \\
\poemll    because the heart is made better through trouble. \\
\poeml \v{4}For the wise person thinks carefully when in mourning, \\
\poemll    but fools focus their thoughts on pleasure. \\
\poeml \v{5}It is better to listen to a wise person's rebuke \\
\poemll    than to listen to the praise\fnote{\fbackref{7:5} Lit. \fbib{song}} of fools. \\
\poeml \v{6}For as thorns burn to heat a pot, \\
\poemll    so also is the laughter of the fool--- \\
\poemlll       even this is pointless.
\passage{Avoiding the Evils of Life}
\poeml \v{7}Unjust gain makes the wise foolish, \\
\poemll    and a bribe corrupts the heart. \\
\poeml \v{8}The conclusion of something is better than its beginning, \\
\poemll    and a patient attitude\fnote{\fbackref{7:8} Lit. \fbib{spirit}} is more valuable than a proud one.\fnote{\fbackref{7:8} Lit. \fbib{spirit}} \\
\poeml \v{9}Never be in a hurry to become internally angry, \\
\poemll    since anger settles down in the lap of fools. \\
\poeml \v{10}Never ask ``Why does the past\fnote{\fbackref{7:10} Lit. \fbib{the former days}} seem so much better than now?''\fnote{\fbackref{7:10} Lit. \fbib{than these}} \\
\poemll    because this question does not come from wisdom. \\
\poeml \v{11}Wise use of possessions is good; \\
\poemll    it brings benefit to the living.\fnote{\fbackref{7:11} Lit. \fbib{to those who see the sun}} \\
\poeml \v{12}Indeed, wisdom gives protection,\fnote{\fbackref{7:12} Or \fbib{shade}} just like money does, \\
\poemll    but it's better to know that wisdom gives life, \\
\poemlll       to those who have mastered\fnote{\fbackref{7:12} Or \fbib{acquired}} it.
\end{poetry}
\passage{The Works of God}

\v{13}Consider the work of God:

\begin{poetry}
\poeml Who is able to straighten \\
\poemll    what he has bent? \\
\poeml \v{14}When times are good, be joyful; \\
\poemll    when times are bad, consider this: \\
\poeml God made the one as well as the other, \\
\poemll    so people won't seek anything outside of his best.
\end{poetry}

\v{15}I have seen it all\fnote{\fbackref{7:15} Lit. \fbib{seen in pointlessness}} during my pointless life:

\begin{poetry}
\poeml both a righteous person who dies \\
\poemll    while he is righteous, \\
\poeml and a wicked person who lives to an old age, \\
\poemll    while remaining wicked.\fnote{\fbackref{7:15} Lit. \fbib{lives long in his evil}}
\passage{Practical Wisdom}
\poeml \v{16}Do not be overly righteous, \\
\poemll    nor be overly wise. \\
\poemlll       Why be self-destructive? \\
\poeml \v{17}Do not excel at wickedness, \\
\poemll    nor be a fool. \\
\poemlll       Why die before your time? \\
\poeml \v{18}It is good for you to grab hold of this and not let go, \\
\poemll    because whoever fears God will escape \\
\poemlll       all of these extremes.\fnote{\fbackref{7:18} The Heb. lacks \fbib{extremes}} \\
\poeml \v{19}Wisdom given as strength to a wise person \\
\poemll    is better than having ten powerful men in the city. \\
\poeml \v{20}For there is not a single righteous man on earth \\
\poemll    who practices good and does not sin. \\
\poeml \v{21}Don't listen to everything that is spoken--- \\
\poemll    you may hear your servant cursing you, \\
\poeml \v{22}since you also know how often \\
\poemll    you have cursed others.
\end{poetry}

\v{23}I used my wisdom to test all of this.

\begin{poetry}
\poeml I said, ``I want to be wise,'' \\
\poemll    but it was beyond me. \\
\poeml \v{24}Whatever it is, \\
\poemll    it's far off and most profound. \\
\poemlll       Who can attain it? \\
\poeml \v{25}I committed myself to understand, \\
\poemlll       to learn, to search for wisdom and explanations, \\
\poeml and to understand both the evil that is foolishness \\
\poemll    and the stupidity that is delusion. \\
\poeml \v{26}I discovered for myself a bitterness \\
\poemll    that surpasses that of death: \\
\poeml the woman whose heart is full of\fnote{\fbackref{7:26} The Heb. lacks \fbib{full of}} snares and nets, \\
\poemll    whose hands are chains of bondage. \\
\poeml Whoever pleases God will escape from her, \\
\poemll    but the transgressor will be trapped by her.
\end{poetry}

\v{27}``Look at this,'' says the Teacher. ``Linking one thing to another, I reached this conclusion:

\begin{poetry}
\poeml \v{28}Among the things I seek but have not found: \\
\poemll    one man among a thousand I did find, \\
\poemlll       but I have not found one woman to be wise\fnote{\fbackref{7:28} The Heb. lacks \fbib{to be wise}} among all these. \\
\poeml \v{29}I have discovered only this: \\
\poemll    God made human beings for righteousness, \\
\poemlll       but they seek many alternatives.''
\end{poetry}
\labelchapt{8}
\passage{The Wise Use of Power}

\begin{poetry}
\poeml \chapt{8}
\v{1}Who is really wise? \\
\poeml Who knows how to interpret this saying: \\
\poeml ``A person's wisdom improves his appearance, \\
\poemll    softening a harsh countenance.''
\end{poetry}
\passage{The Wisdom of Pleasing Leaders}

\v{2}I advise\fnote{\fbackref{8:2} The Heb. lacks \fbib{advise}} doing what the king says, especially regarding an oath to God. \v{3}Don't be in a hurry to leave him, and don't persist in evil, for he does whatever he pleases. \v{4}Since a king's command is powerful, who is able to challenge him, asking, ``What are you doing?''

\v{5}Whoever obeys his commands will not experience harm, and the wise in heart will discern both the appropriate time and response.\fnote{\fbackref{8:5} Lit. \fbib{judgment}} \v{6}Indeed, there is an appropriate time and a response\fnote{\fbackref{8:6} Lit. \fbib{judgment}} for every circumstance, since human misery weighs heavily upon him. \v{7}For he has absolutely no knowledge what will happen, since who can declare to him when it will come about? \v{8}Just as no human being has control over the wind\fnote{\fbackref{8:8} Or \fbib{spirit}} to restrain it, so also no human being has control over the day of his death. Just as no one is discharged during war, so wickedness will not release those who practice\fnote{\fbackref{8:8} Or \fbib{serve}} it.

\v{9}I observed all this, and carefully considered everything that is undertaken on earth, especially the time when someone dominates another to his detriment. \v{10}So I watched the wicked being entombed. They used to come in and out of the Holy Place,\fnote{\fbackref{8:10} I.e. the Temple} but now they are forgotten in the city, where they used to work. This, too, is pointless.
\passage{The Wisdom of Fearing God}

\v{11}Whenever a sentence for a crime is not carried out swiftly, the human mind\fnote{\fbackref{8:11} Lit. \fbib{the heart of the sons of men within them}} becomes determined to commit evil. \v{12}Even though a sinner does what is wrong a hundred times and still survives, nevertheless I also know that things will go well for those who fear God, who fear in his presence. \v{13}But things will not go well for the wicked person: he will not lengthen his life\fnote{\fbackref{8:13} Lit. \fbib{days}} like a shadow, since he has no fear before God.
\passage{Fruitless Righteousness, Fruitful Evil}

\v{14}Here is a pointless thing that happens on earth: A righteous man receives what happens to the wicked, and a wicked man receives what happens to the righteous. I concluded that this, too, is pointless. \v{15}So then I recommended enjoyment of life, because it is better on earth for a man to eat, drink, and be happy, since this will stay with him throughout his struggle all the days of his life, which God grants him on earth.

\v{16}When I dedicated myself to experience wisdom and to observe what is undertaken on earth---even going without sleep day and night--- \v{17}I saw all of it as the activity of God. Frankly, a human being cannot understand what happens on earth, because however hard a man works to discover it, he will not find out. Despite what he thinks he knows, he will not be able to figure it out.
\labelchapt{9}
\passage{God's Sovereignty}

\chapt{9}
\v{1}In light of all of this, I committed myself to explain it this way: the righteous and the wise, along with everything they do, are in the hands of God. Furthermore, as to love and hate, no human being knows everything concerning them. \v{2}Everyone shares the same experience: a single event affects the righteous, the wicked, the good, the clean, the unclean, whoever sacrifices, and whoever does not sacrifice.

\begin{poetry}
\poeml As it is with the good person, \\
\poemll    so also it is with the sinner; \\
\poeml as it is with someone who takes an oath, \\
\poemll    so also it is with someone who fears taking an oath.
\end{poetry}
\passage{The Universality of Death}

\v{3}There is a tragedy that infects everything that happens on earth: a common event happens to everyone. In fact, the hearts of human beings are full of evil. Madness remains in their hearts while they live, and afterwards they join the dead. \v{4}``While someone is among the living, hope remains,'' because ``it is better to be a living dog than to be a dead lion.''\fnote{\fbackref{9:4} These are ancient proverbs.}

\begin{poetry}
\poeml \v{5}At least the living know they will die, \\
\poemll    but the dead know nothing; \\
\poeml they no longer have a reward, \\
\poemll    since memory about them has been forgotten. \\
\poeml \v{6}Furthermore, their love, their hate, and their envy \\
\poemll    have been long lost. \\
\poeml Never again will they have a part \\
\poemll    in what happens on earth.
\end{poetry}
\passage{The Fine Art of Enjoying Life}

\v{7}Go ahead and enjoy your meals as you eat. Drink your wine with a joyful attitude, because God already has approved your actions. \v{8}Always keep your garments white, and don't fail to anoint your head. \v{9}Find joy in living with your wife whom you love every day of your pointless life that God\fnote{\fbackref{9:9} Lit. \fbib{he}} gave you on earth, because this is your life assignment and your work to do on earth. \v{10}Whatever the activity in which you engage, do it with all your ability, because there is no work, no planning, no learning, and no wisdom in the next world\fnote{\fbackref{9:10} Lit. \fbib{in Sheol}; i.e. the realm of the dead} where you're going.

\v{11}I considered and observed on earth the following:

\begin{poetry}
\poeml The race doesn't go to the swift, \\
\poemll    nor the battle to the strong, \\
\poeml nor food to the wise, \\
\poemll    nor wealth to the smart, \\
\poeml nor recognition to the skilled. \\
\poemll    Instead, timing and circumstances meet them all.
\end{poetry}

\v{12}In addition, no human being knows his time:

\begin{poetry}
\poeml Like fish captured in a cruel net, \\
\poemll    or as birds caught in a snare, \\
\poeml so also are human beings caught by bad timing \\
\poemll    that surprises them.
\end{poetry}
\passage{Wisdom Surpasses Foolishness}

\v{13}I also observed this example of\fnote{\fbackref{9:13} The Heb. lacks \fbib{example of}} wisdom on earth, and it seemed important to me: \v{14}There was a little city with few men in it. A great king came against it, surrounded it, and built massive siege ramps against it. \v{15}Now there was found within it a poor, but wise man. He delivered the city by his wisdom, but not one person remembered that poor man.

\v{16}So I concluded,\fnote{\fbackref{9:16} Lit. \fbib{said}} ``Wisdom is better than strength. Nevertheless, the wisdom of the poor is rejected---his words are never heard.''

\begin{poetry}
\poeml \v{17}The softly spoken words of the wise are to be heard \\
\poemll    rather than the shouts of a ruler of fools. \\
\poeml \v{18}Wisdom is better than weapons of war, \\
\poemll    and a single sinner can destroy a lot of good.
\end{poetry}
\labelchapt{10}
\passage{Proverbs about Wisdom and Foolishness}

\begin{poetry}
\poeml \chapt{10}
\v{1}As dead flies cause the perfumer's ointment to stink, \\
\poeml so also does a little foolishness to one's reputation of wisdom and honor. \\
\poeml \v{2}A wise man's heart tends toward his right, \\
\poemll    but a fool's heart tends toward his left. \\
\poeml \v{3}Furthermore, the way a fool lives shows he has no sense; \\
\poemll    he proclaims to everyone that he's a fool. \\
\poeml \v{4}If your overseer gets angry at you, don't resign, \\
\poemll    because calmness pacifies great offenses. \\
\poeml \v{5}Here's another tragedy that I've observed on earth, \\
\poemll    a kind of error that comes from an overseer: \\
\poeml \v{6}Foolishness is given great honor, \\
\poemll    while the prosperous sit in lowly places.\fnote{\fbackref{10:6} The Heb. lacks \fbib{places}} \\
\poeml \v{7}And I have observed servants riding\fnote{\fbackref{10:7} The Heb. lacks \fbib{riding}} on horses, \\
\poemll    while princes walk on the ground like servants. \\
\poeml \v{8}Whoever digs a pit may fall into it, \\
\poemll    and whoever breaks through a wall \\
\poemlll       may suffer a snake bite. \\
\poeml \v{9}Someone who quarries stone might be injured; \\
\poemll    someone splitting logs can fall into danger. \\
\poeml \v{10}If someone's ax is blunt---the edge isn't sharpened--- \\
\poemll    then more strength will be needed. \\
\poemlll       Putting wisdom to work will bring success. \\
\poeml \v{11}If a serpent strikes despite being charmed, \\
\poemll    there's no point in being a snake charmer. \\
\poeml \v{12}The words spoken by the wise are gracious, \\
\poemll    but the lips of a fool will devour him. \\
\poeml \v{13}He begins his speech with foolishness, \\
\poemll    and concludes it with evil madness. \\
\poeml \v{14}The fool overflows with words, \\
\poemll    and no one can predict what will happen. \\
\poeml As to what will happen after him, \\
\poemll    who can explain it? \\
\poeml \v{15}The work of a fool so wears him out \\
\poemll    that he can't even find his way to town. \\
\poeml \v{16}Woe to the land whose king is a youth \\
\poemll    and whose princes feast in the morning. \\
\poeml \v{17}That land is blessed whose king is of noble birth, \\
\poemll    whose princes feast at the right time, \\
\poemlll       for strength, and not to become drunk. \\
\poeml \v{18}Through slothfulness the roof deteriorates, \\
\poemll    and a house leaks because of idleness. \\
\poeml \v{19}Festivals are for laughter, \\
\poemll    wine makes life pleasant, \\
\poemlll       and money speaks to everything. \\
\poeml \v{20}Do not curse the king, \\
\poemll    even in your thoughts. \\
\poeml Do not curse the rich, \\
\poemll    even in your bedroom. \\
\poeml For a bird will fly by and tell what you say, \\
\poemll    or something with wings may talk about it.
\end{poetry}
\labelchapt{11}
\passage{Preparing for the Future}

\begin{poetry}
\poeml \chapt{11}
\v{1}Spread your bread on the water--- \\
\poeml after a while you will find it. \\
\poeml \v{2}Apportion what you have into seven, or even eight parts, \\
\poemll    because you don't know what disaster might befall the land. \\
\poeml \v{3}If the clouds are full of rain, \\
\poemll    they will pour out on the earth; \\
\poeml if a tree falls toward the south or the north, \\
\poemll    wherever it falls, there it will lay. \\
\poeml \v{4}Whoever keeps staring at the wind won't sow; \\
\poemll    whoever daydreams\fnote{\fbackref{11:4} Lit. \fbib{who stares at clouds}} won't reap. \\
\poeml \v{5}Just as you do not understand the way of the spirit \\
\poemll    in the\fnote{\fbackref{11:5} Lit. \fbib{the bones in the} } womb of a pregnant mother, \\
\poeml so also you do not know \\
\poemll    what God is doing in everything he makes. \\
\poeml \v{6}Sow your seed in the morning, \\
\poemll    and don't stop working\fnote{\fbackref{11:6} Lit. \fbib{then give your hand no rest}} until evening, \\
\poeml since you don't know which of your endeavors will do well, \\
\poemll    whether this one or that, \\
\poemlll       or even if both will do equally well.
\end{poetry}
\passage{Preparing for Old Age}

\v{7}How sweet is the daylight, and how pleasant it is for someone's eyes to behold the sunshine! \v{8}Even if a person lives many years, let him enjoy them all, recalling that there will be many days of darkness to come---all of which are pointless. \v{9}So enjoy yourself in your youth, young man, and be encouraged during your younger days. Live as you like, consistent with your world view, but keep in mind that God will bring you to account for everything. \v{10}Banish sorrow from your heart, and evil from your body, since both childhood and the prime of life\fnote{\fbackref{11:10} Lit. \fbib{dark hair}} are pointless.
\labelchapt{12}
\passage{Remember Your Creator}

\begin{poetry}
\poeml \chapt{12}
\v{1}So remember your Creator during your youth! \\
\poeml Otherwise, troublesome days will come \\
\poeml and years will creep up on you when you'll say, \\
\poemlll       ``I find no pleasure in them,'' \\
\poeml \v{2}Otherwise, when the sun, daylight, moon, or stars turn dark, \\
\poemll    or when clouds fail to return after the rain--- \\
\poeml \v{3}when that day comes, the palace guards will tremble, \\
\poemll    strong men will stoop down, \\
\poeml women grinders will cease because they are few, \\
\poemll    and the sight of\fnote{\fbackref{12:3} The Heb. lacks \fbib{the sight of}} those who peer through the lattice will grow dim. \\
\poeml \v{4}The doors to the street will be shut \\
\poemll    when the sound of grinding decreases, \\
\poeml when one wakes up at the song of a bird, \\
\poemll    and all of the singing women are silenced. \\
\poeml \v{5}At that time they will fear climbing\fnote{\fbackref{12:5} The Heb. lacks \fbib{climbing}} heights \\
\poemll    and dangers along the road \\
\poeml while the almond tree will blossom, \\
\poemll    and the grasshopper is weighed down. \\
\poeml Desire will cease,\fnote{\fbackref{12:5} Lit. \fbib{The caper berry will have no effect}} \\
\poemll    because the person goes to his eternal home, \\
\poemlll       and mourners will gather in the marketplace. \\
\poeml \v{6}When the silver cord is severed, \\
\poemll    the golden vessel is broken, \\
\poeml the pitcher is shattered at the fountain, \\
\poemll    and the wheel is broken at the cistern, \\
\poeml \v{7}then man's\fnote{\fbackref{12:7} The Heb. lacks \fbib{man's}} dust will go back to the earth, \\
\poemll    returning to what it was, \\
\poemlll       and the spirit\fnote{\fbackref{12:7} Or \fbib{the breath of life}} will return to the God who gave it. \\
\poeml \v{8}``Utterly pointless,'' says the Teacher. \\
\poemll    ``Everything is pointless.''
\end{poetry}
\passage{Conclusions}

\v{9}Moreover, besides being wise himself, the Teacher taught people what he had learned by listening, making inquiries, and composing many proverbs. \v{10}The Teacher searched to find appropriate expressions, and what is written here\fnote{\fbackref{12:10} The Heb. lacks \fbib{here}} is right and truthful.

\v{11}Sayings from the wise are like cattle prods and well fastened nails; this\fnote{\fbackref{12:11} The Heb. lacks \fbib{this}} masterful collection was given by one shepherd. \v{12}So learn from them, my son. There is no end to the crafting of many books, and too much study wearies the body.

\begin{poetry}
\poeml \v{13}Let the conclusion of all of these thoughts be heard: \\
\poeml Fear God and obey his commandments, \\
\poemll    for this is what it means to be human. \\
\poeml \v{14}For God will judge every deed, \\
\poemll    along with every secret, \\
\poemlll       whether good or evil.\end{poetry}

\bookheader{Song of Songs}
\labelbook{Song}

\bookpretitle{The Most Beautiful}
\booktitle{Song of Songs}

\labelchapt{1}
\passage{Title}

\chapt{1}
\v{1}The Most Beautiful Song by Solomon.
\passage{The Loved One}

\begin{poetry}
\poeml \v{2}Let him kiss me over and over again!\fnote{Lit. \fbib{me with the kisses of his mouth}} \\
\poemll    Your love is better than wine. \\
\poeml \v{3}The fragrance of your perfumed oil is wonderful. \\
\poemll    Your name is perfume poured out. \\
\poemlll       Therefore the young women love you. \\
\poeml \v{4}Take me with\fnote{Lit. \fbib{Draw me after}} you! Let's run away! \\
\poemll    Let the king bring me into his private chambers.
\passage{The Young Women}
\poeml The daughters\fnote{Or \fbib{The young women}} of Jerusalem\fnote{The Heb. lacks \fbib{daughters of Jerusalem} (cf. v. 5)} will rejoice and be happy for you. \\
\poemll    We will value your love more than wine. \\
\poemlll       They love you appropriately.
\passage{The Loved One}
\poeml \v{5}The daughters\fnote{Or \fbib{The young women}} of Jerusalem, I'm dark and lovely \\
\poemll    like the tents of Kedar, \\
\poemlll       like the curtains of Solomon. \\
\poeml \v{6}Don't stare at me because I'm dark; \\
\poemll    the sun has tanned me. \\
\poeml My mother's sons were angry with me. \\
\poemll    They made me the caretaker of the vineyards, \\
\poemlll       but I didn't take care of my own vineyard. \\
\poeml \v{7}Tell me, you whom I love, \\
\poemll    where do you graze your flock? \\
\poemlll       Where do you make your flock lie down\fnote{Lit. \fbib{where do you make to lie down}} at noon? \\
\poeml Why should I be considered a veiled woman\fnote{I.e. a prostitute} \\
\poemll    beside the flocks of your companions?
\passage{The Lover}
\poeml \v{8}If you don't know, most beautiful of women, \\
\poemll    go out after the flock and graze your young goats beside the shepherd's tents. \\
\poeml \v{9}My darling, I compare you to a\fnote{Lit. \fbib{my}} mare \\
\poemll    among Pharaoh's chariots. \\
\poeml \v{10}Your cheeks are lovely with ornaments, \\
\poemll    your neck with strings of jewels.
\passage{The Young Women}
\poeml \v{11}We will make ornaments of gold for you, \\
\poemll    accented with silver.
\passage{The Loved One}
\poeml \v{12}While the king was sitting at his table, \\
\poemll    my perfume sent forth its fragrance. \\
\poeml \v{13}My beloved is to me a pouch of myrrh\fnote{I.e. a fragrant spice used as a perfume} \\
\poemll    that lies between my breasts all night. \\
\poeml \v{14}My beloved is to me a cluster of henna\fnote{I.e. an aromatic shrub used cosmetically} blossoms \\
\poemll    in the vineyards of En-gedi.
\passage{The Lover}
\poeml \v{15}Look at you! You are beautiful, my darling. \\
\poemll    Look at you! You are so beautiful. \\
\poemlll       Your eyes are doves.
\passage{The Loved One}
\poeml \v{16}Look at you! You are handsome, my beloved, truly lovely. \\
\poemll    How lush is our couch. \\
\poeml \v{17}The beams of our house are cedar,\fnote{I.e. a genus of coniferous evergreen in the family \fbib{Pinaceae}; and so throughout the book} \\
\poemll    our rafters are pine.
\labelchapt{2}
\poeml \chapt{2}
\v{1}I'm a flower\fnote{Or \fbib{crocus}} from Sharon, \\
\poeml a lily of the valleys.
\passage{The Lover}
\poeml \v{2}Like a lily among thorns, \\
\poemll    so is my darling among the young women.
\passage{The Loved One}
\poeml \v{3}Like an apple\fnote{Or \fbib{apricot}} tree among the trees of the forest, \\
\poemll    so is my beloved among the young men. \\
\poeml In his shade I take delight and sit down, \\
\poemll    and his fruit is sweet to my taste.\fnote{Lit. \fbib{palate}} \\
\poeml \v{4}He has brought me to the banquet hall, \\
\poemll    and his banner over me is love. \\
\poeml \v{5}Sustain me with raisin cakes, \\
\poemll    refresh me with apples,\fnote{Or \fbib{apricots}} \\
\poemlll       for I'm weak with love. \\
\poeml \v{6}I wish that his left hand were under my head, \\
\poemll    and that his right hand were embracing me! \\
\poeml \v{7}Swear to me, young women of Jerusalem, \\
\poemll    by the gazelles or by the does of the field, \\
\poeml that you won't awaken or arouse love \\
\poemll    before its proper time!\fnote{Lit. \fbib{until it pleases}} \\
\poeml \v{8}The voice of my beloved! \\
\poeml Look! He's coming, \\
\poemll    leaping over the mountains, \\
\poemlll       bounding over the hills. \\
\poeml \v{9}My beloved is like a gazelle or a young stag. \\
\poeml Look, there he stands behind our wall, \\
\poemll    looking through the windows, \\
\poemlll       gazing through the lattice.
\passage{The Lover}
\poeml \v{10}My beloved spoke to me: \\
\poemll    ``Get up, my darling, my beautiful one, and come on. \\
\poeml \v{11}Look! Winter is past. \\
\poemll    The rain is over and gone. \\
\poeml \v{12}Blossoms have appeared in the land. \\
\poemll    The season of songbirds\fnote{Or \fbib{for pruning}} has arrived, \\
\poemlll       and cooing of turtledoves is heard in our land. \\
\poeml \v{13}The fig tree has produced its fruit,\fnote{Lit. \fbib{has ripened}} \\
\poemll    the grapevines have blossomed and exude their fragrance. \\
\poeml ``Get up, my darling, my beautiful one, and come on. \\
\poeml \v{14}My dove, in the hidden places of the rocks, \\
\poemll    in the secret places of the cliffs, \\
\poeml show me your form, and let me hear your voice. \\
\poemll    For your voice is pleasant, \\
\poemlll       and your shape is lovely. \\
\poeml \v{15}Catch the foxes for us, \\
\poemll    the little foxes that destroy the vineyards, \\
\poemlll       our vineyards that are in bloom.''
\passage{The Loved One}
\poeml \v{16}My beloved belongs to me and I belong to him. \\
\poemll    He is the one who shepherds his flock among the lilies. \\
\poeml \v{17}Until the day breaks\fnote{Or \fbib{until the cool of the day}; lit. \fbib{until the day breathes}} and the shadows flee, \\
\poemll    turn around, my beloved, \\
\poeml and be like a gazelle or a young stag \\
\poemll    on the rugged mountains.\fnote{Or \fbib{the mountains of Bether}}
\end{poetry}
\labelchapt{3}
\passage{The Loved One}

\begin{poetry}
\poeml \chapt{3}
\v{1}Night after night on my bed, \\
\poeml I sought the one I love; \\
\poeml I sought him, but didn't find him. \\
\poeml \v{2}I'll get up and go all around the city, \\
\poemll    throughout the streets, and in the squares. \\
\poeml I'll seek the one I love. \\
\poemll    I sought him, but didn't find him. \\
\poeml \v{3}The watchmen who go all around the city found me. \\
\poemll    I asked,\fnote{The Heb. lacks \fbib{I asked}} ``Have you seen the one I love?'' \\
\poeml \v{4}I had just passed them \\
\poemll    when I found the one I love. \\
\poeml I held him and wouldn't let him go \\
\poemll    until I brought him to my mother's house, \\
\poemlll       to the room of the one who conceived me. \\
\poeml \v{5}Swear to me, young women of Jerusalem, \\
\poemll    by the gazelles or by the does of the field, \\
\poeml that you won't awaken or arouse love \\
\poemll    before its proper time!\fnote{Lit. \fbib{until it pleases}}
\passage{The Lover Arrives}
\poeml \v{6}What\fnote{Or \fbib{Who}} is this coming up from the desert \\
\poemll    like columns of smoke, \\
\poeml perfumed with myrrh\fnote{Myrrh was a fragrant spice used as a perfume.} and incense\fnote{Or \fbib{frankincense}} \\
\poemll    from all the fragrant powders of the merchant? \\
\poeml \v{7}Look! It's Solomon's sedan chair,\fnote{Or \fbib{palanquin}; i.e. a portable compartment for carrying important people.} \\
\poemll    with 60 of the best soldiers in Israel\fnote{Lit. \fbib{mighty men from the mighty men of Israel}} surrounding it. \\
\poeml \v{8}All of them are wearing swords and are \\
\poemll    experienced in battle. \\
\poeml Each has his sword on his thigh, \\
\poemll    ready for the terror of the night.\fnote{Lit. \fbib{thigh, from the terror of the night}} \\
\poeml \v{9}King Solomon made the sedan chair for himself \\
\poemll    from the trees of Lebanon. \\
\poeml \v{10}He made its posts of silver, \\
\poemll    its back of gold. \\
\poeml Its seat was purple, \\
\poemll    and its interior was lovingly inlaid \\
\poemlll       by the young women of Jerusalem. \\
\poeml \v{11}Come out, young women of Zion, \\
\poemll    and see King Solomon with the crown \\
\poeml with which his mother crowned him \\
\poemll    on his wedding day--- \\
\poemlll       his day of great delight.\fnote{Lit. \fbib{day of the rejoicing of his heart}}
\end{poetry}
\labelchapt{4}
\passage{The Lover}

\begin{poetry}
\poeml \chapt{4}
\v{1}Look at you! You are beautiful, my darling. \\
\poeml Look at you! You are so beautiful. \\
\poeml Your eyes behind your veil are doves, \\
\poemll    your hair is like a flock of goats \\
\poemlll       coming down from Mt. Gilead. \\
\poeml \v{2}Your teeth are like a flock of sheep about to be sheared,\fnote{Or \fbib{like sheared sheep}} \\
\poemll    who are coming up from being washed.\fnote{Lit. \fbib{from washing}} \\
\poeml All of them are twins, \\
\poemll    not one has lost\fnote{Lit. \fbib{been bereaved of}; i.e. her teeth match} her young. \\
\poeml \v{3}Your lips are like a scarlet thread, \\
\poemll    and your mouth is lovely. \\
\poeml Behind your veil, \\
\poemll    your temple is like a slice of pomegranate. \\
\poeml \v{4}Your neck is like the tower of David, \\
\poemll    built with rows of stones. \\
\poeml A thousand shields are hung upon it, \\
\poemll    all the shields of the warriors. \\
\poeml \v{5}Your two breasts are like two fawns, \\
\poemll    twins of a gazelle grazing among the lilies. \\
\poeml \v{6}Until the day breaks\fnote{Or \fbib{until the cool of the day;} lit. \fbib{until the day breathes};} and the shadows flee, \\
\poemll    I'll go to the mountain of myrrh\fnote{Myrrh was a fragrant spice used as a perfume.} \\
\poemlll       and to the hill of frankincense.\fnote{Frankincense was a fragrant spice used to make incense and perfume.} \\
\poeml \v{7}My darling, you are altogether beautiful \\
\poemll    and there is no blemish in you. \\
\poeml \v{8}Come with me from Lebanon, my bride, \\
\poemll    come with me from Lebanon. \\
\poeml May you journey from the top of Amana, \\
\poemll    from the tops of Senir and Hermon, \\
\poeml from the dens of lions, \\
\poemll    from the mountain lairs of leopards. \\
\poeml \v{9}You have made my heart beat faster, my sister, my\fnote{The Heb. lacks \fbib{my}} bride. \\
\poemll    You have made my heart beat faster \\
\poemlll       with one glance of your eyes, \\
\poemll    with one strand of your necklace. \\
\poeml \v{10}How beautiful is your love, my sister, my\fnote{The Heb. lacks \fbib{my}} bride. \\
\poemll    How much better is your love than wine, \\
\poeml and the fragrance of your perfume \\
\poemll    than all kinds of spices. \\
\poeml \v{11}Your lips drip honey, my\fnote{The Heb. lacks \fbib{my}} bride; \\
\poemll    milk and honey are under your tongue. \\
\poeml The scent of your garments \\
\poemll    is like the fragrance of Lebanon. \\
\poeml \v{12}My sister, my\fnote{The Heb. lacks \fbib{my}} bride, is a locked garden \\
\poemll    a locked rock garden, a sealed up spring. \\
\poeml \v{13}Your shoots are an orchard\fnote{Or \fbib{a park}} of pomegranates, \\
\poemll    with choice fruit, henna with nard, \\
\poeml \v{14}nard and saffron, \\
\poemll    calamus and cinnamon, \\
\poeml with all the trees of frankincense, \\
\poemll    along with myrrh and aloes, and all the finest spices.\fnote{All the spices listed in verses 13 and 14 were used for perfume.} \\
\poeml \v{15}You are a garden spring, \\
\poemll    a well of fresh water, \\
\poemlll       streams flowing from Lebanon.
\passage{The Loved One}
\poeml \v{16}Awake, north wind, and come, south wind. \\
\poemll    Make my garden breathe out, \\
\poemlll       let its fragrance\fnote{Or \fbib{its spices}} flow. \\
\poeml Let my beloved come into his garden, \\
\poemll    and let him eat its choicest fruits.
\end{poetry}
\labelchapt{5}
\passage{The Lover}

\begin{poetry}
\poeml \chapt{5}
\v{1}I've come into my garden, my sister, my\fnote{The Heb. lacks \fbib{my}} bride; \\
\poeml I've gathered my myrrh with my spices. \\
\poeml I've eaten my honeycomb with my honey. \\
\poeml I've drunk my wine with my milk. \\
\poeml Eat, friends! \\
\poemll    Drink and become drunk with love.
\passage{The Loved One}
\poeml \v{2}I was asleep, but my heart was awake. \\
\poemll    There's a sound! My beloved is knocking.
\passage{The Lover}
\poeml ``Open up for me, my sister, my darling, \\
\poemll    my dove, my perfect one. \\
\poeml For my head is drenched with dew, \\
\poemll    my hair with the dampness of the night.''
\passage{The Loved One}
\poeml \v{3}``I've taken off my clothes\fnote{Or \fbib{my tunic}}--- \\
\poemll    am I supposed to put them on again?\fnote{Lit. \fbib{How shall I put them on again?}} \\
\poeml I've washed my feet--- \\
\poemll    am I supposed to\fnote{Lit. \fbib{How shall I}} get them dirty again?'' \\
\poeml \v{4}My beloved reached out his hand for the latch.\fnote{Or \fbib{keyhole}} \\
\poemll    My feelings for him were aroused. \\
\poeml \v{5}I got up to open the door,\fnote{The Heb. lacks \fbib{the door}} \\
\poemll    and my hands dripped with myrrh, \\
\poeml my fingers with liquid myrrh, \\
\poemll    on the handle of the lock. \\
\poeml \v{6}I opened the door\fnote{The Heb. lacks \fbib{the door}} for my beloved, \\
\poemll    but my beloved had turned away; he was gone! \\
\poeml My very life went out when he departed.\fnote{Or \fbib{when he spoke}} \\
\poemll    I searched for him, \\
\poemlll       but couldn't find him. \\
\poemll    I called out to him, \\
\poemlll       but he didn't answer. \\
\poeml \v{7}The watchmen making their\fnote{The Heb. lacks \fbib{their}} rounds \\
\poemll    through the city found me. \\
\poeml They beat me, they bruised me. \\
\poemll    Those guarding the walls took my cloak\fnote{Or \fbib{shawl}} from me. \\
\poeml \v{8}I charge you, young women of Jerusalem, \\
\poemll    ``If you find my beloved, what are you to tell him? \\
\poemlll       Tell him\fnote{The Heb. lacks \fbib{Tell him}} that I'm weak with love.''
\passage{The Young Women}
\poeml \v{9}What is so special about your beloved,\fnote{Or \fbib{How is your beloved better than other beloveds}} \\
\poemll    most beautiful of women? \\
\poeml What is so special about your beloved, \\
\poemll    that you charge us like this?
\passage{The Loved One}
\poeml \v{10}My beloved is dazzling, \\
\poemll    with a dark and healthy complexion, \\
\poemlll       outstanding among ten thousand. \\
\poeml \v{11}His head is pure gold, \\
\poemll    his hair is wavy, black like a raven. \\
\poeml \v{12}His eyes are like doves \\
\poemll    by streams of water, \\
\poeml bathed in milk, \\
\poemll    mounted like jewels. \\
\poeml \v{13}His cheeks are like beds of spices, \\
\poemll    like towers of perfume. \\
\poeml His lips are lilies, \\
\poemll    dripping with liquid myrrh. \\
\poeml \v{14}His hands are rods of gold, \\
\poemll    set with beryl. \\
\poeml His stomach is carved ivory, \\
\poemll    inlaid with sapphires.\fnote{Or \fbib{lapis lazuli}} \\
\poeml \v{15}His legs are pillars of marble \\
\poemll    set on bases of pure gold. \\
\poeml His appearance is like Lebanon, \\
\poemll    choice like its cedars. \\
\poeml \v{16}His mouth\fnote{Lit. \fbib{palate}} is sweetness, \\
\poemll    and all of him is desirable. \\
\poeml This is my beloved, this is my friend, \\
\poemll    young women of Jerusalem!
\end{poetry}
\labelchapt{6}
\passage{The Young Women}

\begin{poetry}
\poeml \chapt{6}
\v{1}Where did your beloved go, \\
\poeml most beautiful of women? \\
\poeml Where did your beloved turn, \\
\poeml so we may look for him with you?
\passage{The Loved One}
\poeml \v{2}My beloved has gone down to his garden, \\
\poemll    to the beds of spices, \\
\poeml to graze his flock\fnote{The Heb. lacks \fbib{his flock}} in the gardens \\
\poemll    and gather lilies. \\
\poeml \v{3}I belong to my beloved, and my beloved belongs to me. \\
\poemll    He is the one who grazes his flock\fnote{The Heb. lacks \fbib{his flock}} among the lilies.
\passage{The Lover}
\poeml \v{4}You are beautiful, my darling, like Tirzah,\fnote{I.e. a prominent city in the northern kingdom of Israel} \\
\poemll    lovely like Jerusalem, \\
\poemlll       as awesome as an army with banners. \\
\poeml \v{5}Turn your eyes from me, \\
\poemll    for they excite me.\fnote{Or \fbib{overwhelm me}} \\
\poeml Your hair is like a flock of goats \\
\poemll    coming down from Mt. Gilead. \\
\poeml \v{6}Your teeth are like a flock of ewes \\
\poemll    coming up from being washed.\fnote{Lit. \fbib{from washing}} \\
\poeml All of them are twins, \\
\poemll    not one has lost\fnote{Lit. \fbib{been bereaved of}; i.e. her teeth match} her young. \\
\poeml \v{7}Your temple\fnote{Or \fbib{brow}} behind your veil \\
\poemll    is like a slice of pomegranate. \\
\poeml \v{8}There are sixty queens and eighty mistresses,\fnote{Or \fbib{concubines}; i.e. secondary wives} \\
\poemll    and too many young women to count, \\
\poeml \v{9}but my dove, my perfect one, is unique. \\
\poeml She's unique to her mother, \\
\poemll    she's pure to the one who gave birth to her. \\
\poeml Young women see her and call her blessed, \\
\poemll    queens and mistresses praise her.
\passage{The Young Women}
\poeml \v{10}Who is this who appears like the dawn, \\
\poemll    beautiful as the moon, \\
\poeml bright as the sun, \\
\poemll    awesome as an army with banners?
\passage{The Loved One}
\poeml \v{11}I went down to the walnut orchard, \\
\poemll    to look at the green sprouts in the valley, \\
\poeml to see whether the vine had budded, \\
\poemll    whether the pomegranates had blossomed. \\
\poeml \v{12}Before I knew it,\fnote{Lit. \fbib{I did not know}; i.e. was daydreaming} I imagined myself \\
\poemll    among the chariots of my noble people.\fnote{Or \fbib{of Amminadib}}
\passage{The Young Women}
\poeml \v{13}\fnote{This v. is 7:1 in MT}Return, return, Shulammite, \\
\poemll    return, return, so we may look at you!
\passage{The Lover}
\poeml Why should you look at the Shulammite,\fnote{Or \fbib{look at Shulamit}} \\
\poemll    like you watch\fnote{The Heb. lacks \fbib{you watch}} the dance of the two camps?\fnote{Or \fbib{the dance of Mahanaim}}
\labelchapt{7}
\poeml \chapt{7}
\v{1}\fnote{This v. is 7:2 in MT}How beautiful are your feet in sandals, \\
\poeml noble lady.\fnote{Or \fbib{prince's daughter}} \\
\poeml The curves of your thighs are like ornaments, \\
\poemll    the work of a skilled artist's hands. \\
\poeml \v{2}Your navel is a rounded goblet \\
\poemll    that never lacks mixed wine. \\
\poeml Your abdomen is a bundle of wheat, \\
\poemll    enclosed by lilies. \\
\poeml \v{3}Your two breasts are like two fawns, \\
\poemll    twins of a gazelle. \\
\poeml \v{4}Your neck is like a tower of ivory. \\
\poemll    Your eyes are like the\fnote{The Heb. lacks \fbib{like the}} pools in Heshbon, \\
\poemlll       beside the gate of Beth-rabbim. \\
\poeml Your nose is like the tower of Lebanon, \\
\poemll    which faces Damascus. \\
\poeml \v{5}Your head crowns you\fnote{Lit. \fbib{is on you}} like Mount Carmel. \\
\poemll    Your flowing locks\fnote{Lit. \fbib{the hair of your head}} are like purple, \\
\poemlll       and a king could be captured in the dangling tresses. \\
\poeml \v{6}How beautiful and lovely you are, \\
\poemll    you are love with its exquisite delights.\fnote{Or \fbib{my love with exquisite delights}; Lit. \fbib{Love with exquisite delights}} \\
\poeml \v{7}Your stature\fnote{Lit. \fbib{This stature of yours}} is like a palm tree, \\
\poemll    and your breasts are like clusters of fruit.\fnote{The Heb. lacks \fbib{of fruit}} \\
\poeml \v{8}I told myself, ``I'll go up the palm tree, \\
\poemll    and take hold of its fruit. \\
\poeml May your breasts be like clusters of the vine,\fnote{I.e. grapes} \\
\poemll    and the smell of your breath like apples.\fnote{Or \fbib{apricots}} \\
\poeml \v{9}May your mouth\fnote{Lit. \fbib{palate}} be like good wine.
\passage{The Loved One}
\poeml May it go down smoothly to my beloved, \\
\poemll    gliding gently over the lips of the sleeping ones. \\
\poeml \v{10}I belong to my beloved, \\
\poemll    and his desire is for me. \\
\poeml \v{11}Come, my beloved. \\
\poemll    Let us go out to the country, \\
\poemlll       let us spend the night in the villages.\fnote{Or \fbib{night among the henna blossoms}} \\
\poeml \v{12}Let us go early to the vineyards, \\
\poemll    let us see whether the vine has budded, \\
\poeml whether the blossom has opened, \\
\poemll    whether the pomegranates have bloomed. \\
\poemlll       There I'll give you my love. \\
\poeml \v{13}The mandrakes give off their\fnote{The Heb. lacks \fbib{their}} fragrance, \\
\poemll    and over our doors are all the choice fruits, \\
\poeml both old and new, \\
\poemll    that I've stored up for you, my beloved.
\labelchapt{8}
\poeml \chapt{8}
\v{1}If only you were like a brother to me, \\
\poeml like\fnote{The Heb. lacks \fbib{like}} one who nursed at my mother's breasts. \\
\poeml If I found you outside I would kiss you, \\
\poeml and no one would view me with contempt.\fnote{I.e. think badly of me} \\
\poeml \v{2}I would lead you, \\
\poemll    I would bring you \\
\poeml to the house of my mother \\
\poemll    who used to teach me. \\
\poeml I would give you some spiced wine to drink, \\
\poemll    from the juice of my pomegranates.\fnote{Or \fbib{some of my sweet pomegranate wine}} \\
\poeml \v{3}Let his left hand be under my head, \\
\poemll    and let his right hand embrace me. \\
\poeml \v{4}Swear to me, young women of Jerusalem, \\
\poemll    that you won't awaken or arouse love \\
\poemlll       before its proper time!\fnote{Lit. \fbib{until it pleases}}
\passage{The Young Women}
\poeml \v{5}Who is this coming up from the desert, \\
\poemll    leaning on her beloved?
\passage{The Loved One}
\poeml Under the apple tree\fnote{Or \fbib{apricot tree}} I awakened you. \\
\poemll    There your mother had gone into labor with you; \\
\poemlll       there she went into labor and gave birth to you. \\
\poeml \v{6}Set me like a seal\fnote{I.e. a cylinder seal or a signet ring used to indicate ownership or authority} over your heart, \\
\poemll    like a seal on your arm. \\
\poeml For love is as strong as death, \\
\poemll    passion\fnote{Or \fbib{jealousy}} as intense as Sheol.\fnote{I.e. the realm of the dead} \\
\poeml The flames of love\fnote{Lit. \fbib{its flames}} are flames of fire, \\
\poemll    a blaze that comes from the \divine{Lord}.\fnote{Or \fbib{an intense blaze}} \\
\poeml \v{7}Mighty bodies of water cannot extinguish love, \\
\poemll    rivers cannot put it out. \\
\poeml If a man were to give all the wealth of his house for love, \\
\poemll    he would surely be viewed with contempt.
\passage{The Loved One's Brothers}
\poeml \v{8}We\fnote{I.e. the bride's brothers} have a little sister, \\
\poemll    and she has not yet reached maturity.\fnote{Lit. \fbib{has no breasts}} \\
\poeml What will we do for our sister \\
\poemll    to prepare her for\fnote{Lit. \fbib{sister on the day when she's spoken for}} her engagement?\fnote{Or \fbib{marriage}} \\
\poeml \v{9}If she's a wall, \\
\poemll    on her we will build a battlement of silver. \\
\poeml If she's a door, \\
\poemll    we will enclose her with planks of cedar.
\passage{The Loved One}
\poeml \v{10}I was a wall and my breasts were like towers. \\
\poemll    Then I became like one who finds peace\fnote{Or \fbib{contentment}} in his eyes. \\
\poeml \v{11}Solomon has a vineyard in Baal-hamon. \\
\poemll    He entrusted the vineyard to caretakers--- \\
\poeml each one is to bring 1,000 pieces of silver \\
\poemll    in exchange for its fruit. \\
\poeml \v{12}My vineyard belongs to me and is at my disposal.\fnote{Lit. \fbib{is before me}} \\
\poemll    The 1,000 are for you, Solomon, \\
\poemlll       and 200 are for those who take care of its fruit.
\passage{The Lover}
\poeml \v{13}You who sit in the gardens, \\
\poemll    companions are listening for your voice, \\
\poemlll       but let me hear it.
\passage{The Loved One}
\poeml \v{14}Come quickly, my beloved, and be like a gazelle \\
\poemll    or a young stag on the mountains of spices.\end{poetry}

\addcontentsline{toc}{chapter}{Longer Prophetic Writings}
\bookheader{Isaiah}
\labelbook{Isa}

\bookpretitle{The Book of the Prophet}
\booktitle{Isaiah}

\labelchapt{1}
\passage{The Vision of Isaiah}

\chapt{1}
\v{1}This\fnote{This book has been translated from the Great Isaiah Scroll (1QIsa\textsuperscript{a}) of the DSS. The MT, LXX, Syr, Targ, and other ancient texts are compared in footnotes where they may vary from 1QIsa\textsuperscript{a} and other DSS mss. Some of these ancient readings were incorporated instead of the DSS Isaiah text.} is the vision that Amoz's son Isaiah\fnote{\fbackref{1:1} The Heb. name \fbib{Isaiah} means \fbib{The \divine{Lord} has saved}} had about Judah and Jerusalem during the reigns\fnote{\fbackref{1:1} Lit. \fbib{days}} of Uzziah, Jotham, Ahaz, and Hezekiah, kings of Judah.
\passage{Rebellious Judah}

\begin{poetry}
\poeml \v{2}Listen, you heavens, \\
\poemll    and let the\fnote{\fbackref{1:2} So 1QIsa\textsuperscript{a}; the Heb. lacks \fbib{the}} earth pay attention, \\
\poemlll       because the \divine{Lord} has spoken: \\
\poeml ``I reared children \\
\poemll    and brought them to adulthood, \\
\poemlll       but then they rebelled against me. \\
\poeml \v{3}The ox knows its owner, \\
\poemll    and the donkey its master's feeding trough, \\
\poeml but\fnote{\fbackref{1:3} So 4QIsa\textsuperscript{j}; 1QIsa\textsuperscript{a} MT lack \fbib{but}} Israel doesn't know, \\
\poemll    and\fnote{\fbackref{1:3} So 4QIsa\textsuperscript{j}; 1QIsa\textsuperscript{a} MT lack \fbib{and}} my people don't understand. \\
\poeml \v{4}``Oh, you sinful nation! \\
\poemll    You people burdened down by iniquity! \\
\poeml You offspring of those who keep practicing what is evil! \\
\poemll    You corrupt children! \\
\poeml ``They've abandoned the \divine{Lord}; \\
\poemll    they've despised the Holy One of Israel; \\
\poemlll       in their estrangement, they've walked away from me.\fnote{\fbackref{1:4} Lit. \fbib{they've gone back}} \\
\poeml \v{5}``Why will you still be struck down? \\
\poemll    Why will you continue to rebel? \\
\poeml Your whole head is sick, \\
\poemll    and your whole heart is faint. \\
\poeml \v{6}From the sole of your foot to the top of your head, \\
\poemll    there's no soundness evident\fnote{\fbackref{1:6} Lit. \fbib{soundness in it}}--- \\
\poeml only bruises, sores, \\
\poemll    and festering wounds \\
\poeml that haven't been cleaned out, \\
\poemll    bandaged, or treated\fnote{\fbackref{1:6} Or \fbib{softened}} with oil.''
\passage{God's Diagnosis}
\poeml \v{7}``Your country lies desolate; \\
\poemll    your cities have been incinerated. \\
\poeml Before your very eyes, \\
\poemll    foreigners are devouring your land--- \\
\poeml they've brought devastation on it,\fnote{\fbackref{1:7} So 1QIsa\textsuperscript{a}; MurIsa MT LXX read \fbib{Devastated}} \\
\poemll    while the land is\fnote{\fbackref{1:7} DSS MT lack \fbib{the land is}} overthrown by foreigners. \\
\poeml \v{8}``The daughter of Zion is left abandoned, \\
\poemll    like a booth in a vineyard, \\
\poeml like a hut in a cucumber field, \\
\poemll    or like a city under siege. \\
\poeml \v{9}If the Lord of the Heavenly Armies \\
\poemll    hadn't left us a few survivors, \\
\poeml we would be like Sodom; \\
\poemll    we would be like Gomorrah. \\
\poeml \v{10}``Listen to what the \divine{Lord} says, \\
\poemll    you rulers of Sodom, \\
\poeml and\fnote{\fbackref{1:10} So 1:10 1QIsa\textsuperscript{a} Syr; MT LXX Targ Vulgate lack \fbib{and}} pay attention to the teaching of our God, \\
\poemll    you people of Gomorrah! \\
\poeml \v{11}``How do your voluminous sacrifices benefit me?'' \\
\poemll    the \divine{Lord} is asking. \\
\poeml ``I've had enough of burnt offerings of rams \\
\poemll    and the fat of well-fed beasts. \\
\poeml I don't enjoy the blood of bulls, \\
\poemll    lambs, or goats. \\
\poeml \v{12}``When you come to present yourselves in my presence,\fnote{\fbackref{1:12} Lit. \fbib{come for my face to appear}} \\
\poemll    who has required you \\
\poemlll       to trample on my courts? \\
\poeml \v{13}Stop bringing useless offerings! \\
\poemll    Incense is detestable to me, \\
\poeml as are your New Moons, Sabbaths, and calling of convocations. \\
\poemll    I cannot stand iniquity within\fnote{\fbackref{1:13} Lit. \fbib{and}} a solemn assembly. \\
\poeml \v{14}As for your New Moons and your appointed festivals, \\
\poemll    I abhor\fnote{\fbackref{1:14} Lit. \fbib{festivals, my soul abhors}} them. \\
\poeml They've become a burden to me; \\
\poemll    I've grown weary of carrying that burden.\fnote{\fbackref{1:14} Lit. \fbib{carrying them}} \\
\poeml \v{15}When you spread out your hands in prayer,\fnote{\fbackref{1:15} DSS MT lack \fbib{in prayer}} \\
\poemll    I'll hide my eyes from you. \\
\poeml Even though you pray repeatedly, \\
\poemll    I won't listen. \\
\poeml Your hands are full of blood, \\
\poemll    your fingers drenched\fnote{\fbackref{1:15} DSS lack \fbib{drenched}} with iniquity.''\fnote{\fbackref{1:15} MT 4QIsa\textsuperscript{f} lack this line}
\passage{An Invitation to Reconciliation}
\poeml \v{16}``Wash yourselves, \\
\poemll    and\fnote{\fbackref{1:16} So 1QIsa\textsuperscript{a}; 4QIsa\textsuperscript{f} MT lack \fbib{and}} make yourselves clean; \\
\poeml remove your evil behavior \\
\poemll    from my presence; \\
\poemlll       stop practicing what is evil. \\
\poeml \v{17}Learn to practice what is good; \\
\poemll    seek justice, \\
\poeml alleviate oppression,\fnote{\fbackref{1:17} Or \fbib{rescue the oppressed}} \\
\poemll    defend orphans\fnote{\fbackref{1:17} Or \fbib{defend the fatherless}} in court, \\
\poemlll       and\fnote{\fbackref{1:17} DSS MT lack \fbib{and}} plead the widow's case. \\
\poeml \v{18}``Please come, \\
\poemll    and let's reason together,'' implores the \divine{Lord}. \\
\poeml ``Even though your\fnote{\fbackref{1:18} Lit. \fbib{your} (pl.)} sins are like scarlet, \\
\poemll    they'll be white like snow. \\
\poeml Though they're like crimson,\fnote{\fbackref{1:18} So 1QIsa\textsuperscript{a} LXX; 4QIsa\textsuperscript{f} MT read \fbib{they're red like crimson}} \\
\poemll    they'll become like wool. \\
\poeml \v{19}If you're willing and obedient, \\
\poemll    you'll eat the best that the land produces; \\
\poeml \v{20}but\fnote{\fbackref{1:20} So 1QIsa\textsuperscript{a} LXX Targ Syr; 4QIsa\textsuperscript{f} lacks \fbib{but}} if you refuse and rebel, \\
\poemll    you'll be devoured by the sword,\fnote{\fbackref{1:20} So 1QIsa\textsuperscript{a} Targ Syr; LXX reads \fbib{the sword will devour you}} \\
\poemlll       because the \divine{Lord}\fnote{\fbackref{1:20} Lit. \fbib{\divine{Lord}'s mouth}} has spoken.''
\passage{Diagnosis and Judgment}
\poeml \v{21}``How the faithful city \\
\poemll    has become a whore, \\
\poemlll       she who used to be filled with justice! \\
\poeml Righteousness used to reside within her, \\
\poemll    but now only murderers live there. \\
\poeml \v{22}Your silver has\fnote{\fbackref{1:22} So MT; 1QIsa\textsuperscript{a} reads \fbib{have}} become dross, \\
\poemll    your best wine is diluted with water. \\
\poeml \v{23}Your princes are rebels \\
\poemll    and companions of thieves. \\
\poeml All of them are lovers of\fnote{\fbackref{1:23} So 1QIsa\textsuperscript{a} LXX; MT reads \fbib{Everyone loves}} bribes \\
\poemll    and are runners\fnote{\fbackref{1:23} So 1QIsa\textsuperscript{a} LXX; MT reads \fbib{and run}} after gifts. \\
\poeml They don't bring justice to orphans,\fnote{\fbackref{1:23} Or \fbib{to the fatherless}} \\
\poemll    and the widow's case never comes up for review in their court.''\fnote{\fbackref{1:23} Lit. \fbib{comes before them}}
\end{poetry}
\passage{Restoration and Redemption}

\begin{poetry}
\poeml \v{24}Therefore this is what the Lord \divine{God} of the Heavenly Armies, \\
\poemll    the one who is Israel's Mighty One, declares: \\
\poeml ``Now I'll get relief from his\fnote{\fbackref{1:24} So 1QIsa\textsuperscript{a}; 4QIsa\textsuperscript{f} MT LXX Vulgate read \fbib{my}} enemies \\
\poemll    and avenge myself on his\fnote{\fbackref{1:24} So 1QIsa\textsuperscript{a}; LXX MT read \fbib{my}} foes. \\
\poeml \v{25}When I turn\fnote{\fbackref{1:25} So 1QIsa\textsuperscript{a}; LXX MT read \fbib{And let me turn}} my attention to you,\fnote{\fbackref{1:25} Lit. \fbib{my hand against you}} \\
\poemll    I'll refine your dross as in a furnace.\fnote{\fbackref{1:25} 1QIsa\textsuperscript{a} lacks \fbib{as in a furnace}; MT reads \fbib{as lye}} \\
\poemlll       I'll remove\fnote{\fbackref{1:25} So 1QIsa\textsuperscript{a} 4QIsa\textsuperscript{f}; MT reads \fbib{Let me remove}} all your alloy. \\
\poeml \v{26}Let me restore\fnote{\fbackref{1:26} 1QIsa\textsuperscript{a} MT; 4QIsa\textsuperscript{f} reads \fbib{I will restore}} your judges as at the first, \\
\poemll    and your counselors as at the beginning. \\
\poeml Afterward you'll be called `The Righteous City' \\
\poemll    and `The Faithful City of Zion'.\fnote{\fbackref{1:26} So LXX; the Heb. lacks \fbib{Zion}} \\
\poeml \v{27}``Zion\fnote{\fbackref{1:27} So MT; 1QIsa\textsuperscript{a} reading unclear} will be redeemed by justice, \\
\poemll    and her repentant ones\fnote{\fbackref{1:27} So 1QIsa\textsuperscript{a} MT 4QIsa\textsuperscript{f}; LXX reads \fbib{her captivity}} by righteousness. \\
\poeml \v{28}Rebels and sinners will be broken together, \\
\poemll    and those who forsake the \divine{Lord} will be consumed. \\
\poeml \v{29}They'll be ashamed of the oak trees \\
\poemll    that you desired; \\
\poeml and you'll blush because of the gardens \\
\poemll    that you have chosen. \\
\poeml \v{30}You'll be like an oak \\
\poemll    whose leaf is withering, \\
\poemlll       like an unwatered garden. \\
\poeml \v{31}Your\fnote{\fbackref{1:31} Lit. \fbib{Your} (pl.); so 1QIsa\textsuperscript{a} Vulgate. \fbib{The}} strong one\fnote{\fbackref{1:31} LXX MT read \fbib{Their strength}} will be like tinder, \\
\poemll    and your work\fnote{\fbackref{1:31} So 1QIsa\textsuperscript{a}; MT reads \fbib{their work}; LXX reads \fbib{their deeds}} a spark; \\
\poeml both of them will burn together, \\
\poemll    with no one to quench the flames that burn\fnote{\fbackref{1:31} DSS MT lack \fbib{the flames that burn}} them.''
\end{poetry}
\labelchapt{2}
\passage{A Message for Judah and Jerusalem}

\chapt{2}
\v{1}The message that Amoz's son Isaiah received\fnote{\fbackref{2:1} Lit. \fbib{observed}} concerning Judah and Jerusalem:

\begin{poetry}
\poeml \v{2}``It will come about in the last days \\
\poemll    that the mountain that is the \divine{Lord}'s Temple will be established \\
\poemlll       as the highest of mountains,\fnote{\fbackref{2:2} So 4QIsa\textsuperscript{e} MT LXX; 1QIsa\textsuperscript{a} lacks \fbib{of mountains}} \\
\poeml and will be raised above the hills; \\
\poemll    all the nations will stream to\fnote{\fbackref{2:2} Or \fbib{will flow over}; so 1QIsa\textsuperscript{a} LXX; 4QIsa\textsuperscript{e} 4QIsa\textsuperscript{f} MT LXX read \fbib{will travel to}; cf. Mic 4:1} it. \\
\poeml \v{3}Many groups of people\fnote{\fbackref{2:3} Lit. \fbib{Many peoples}} will come, commenting, \\
\poemll    ``Come! Let's go up to\fnote{\fbackref{2:3} So 1QIsa\textsuperscript{a} 4QIsa\textsuperscript{f}; 4QIsa\textsuperscript{e} MT read \fbib{to the \divine{Lord}'s mountain, to}} the Temple of the God of Jacob, \\
\poeml that they\fnote{\fbackref{2:3} So 1QIsa\textsuperscript{a}; cf. LXX, Mic 4:2; 4QIsa\textsuperscript{e} MT LXX read \fbib{he}} may teach us his ways. \\
\poemll    Then let's walk in his paths.''
\passage{The Rule of God from Jerusalem}
\poeml ``Instruction\fnote{\fbackref{2:3} Or \fbib{For the law}} will proceed from Zion, \\
\poemll    and the word of the \divine{Lord} from Jerusalem. \\
\poeml \v{4}He will judge between the nations, \\
\poemll    and will render verdicts\fnote{\fbackref{2:4} Or \fbib{judgment}} for the benefit of many.\fnote{\fbackref{2:4} Lit. \fbib{many peoples}} \\
\poeml ``They will beat their swords into plowshares, \\
\poemll    and their spears into pruning hooks; \\
\poeml nations will not raise swords against nations, \\
\poemll    and they will not learn warfare anymore. \\
\poeml \v{5}``You house of Jacob!
\end{poetry}

Come! Let's live\fnote{\fbackref{2:5} Lit. \fbib{walk}} in the \divine{Lord}'s light.

\begin{poetry}
\poeml \v{6}For you have rejected your people, \\
\poeml the house of Jacob, \\
\poeml because they are filled with practices learned\fnote{\fbackref{2:6} 1QIsa\textsuperscript{a} MT lack \fbib{with practices learned}} from the East \\
\poeml and they are fortune-tellers like the Philistines. \\
\poeml They cut deals\fnote{\fbackref{2:6} Lit. \fbib{They shake hands}} with foreigners.\fnote{\fbackref{2:6} Lit. \fbib{with descendants of foreigners}} \\
\poeml \v{7}Their land is filled with silver and gold, \\
\poeml and there is no end to their treasures; \\
\poeml their land is filled with horses, \\
\poeml and there is no end to their chariots. \\
\poeml \v{8}Their land is filled with idols; \\
\poeml they bow down to the work of their hands, \\
\poemlll       to what their own fingers have made. \\
\poeml \v{9}``So mankind is humbled, \\
\poemll    each human being is brought low, \\
\poemlll       and\fnote{\fbackref{2:9} So 1QIsa\textsuperscript{a} 4QIsa\textsuperscript{b}; the Heb. lacks \fbib{and}} you won't forgive.''\fnote{\fbackref{2:9} So 1QIsa\textsuperscript{a} 4QIsa\textsuperscript{b}; MT reads \fbib{forgive them}}
\passage{The Coming Day of the \divine{Lord}}
\poeml \v{10}\fnote{\fbackref{2:10} This v. is missing from 1QIsa\textsuperscript{a}} ``Go into the rocks! \\
\poemll    Hide in the dust \\
\poeml to escape\fnote{\fbackref{2:10} Lit. \fbib{from}} the terror of the \divine{Lord} \\
\poemll    and to escape\fnote{\fbackref{2:10} Lit. \fbib{and from}} the glory of his majesty!\fnote{\fbackref{2:10} So MT} \\
\poeml \v{11}The\fnote{\fbackref{2:11} Lit. \fbib{And the}} haughty looks of mankind will be brought low,\fnote{\fbackref{:11} 1QIsa\textsuperscript{a} MT LXX read \fbib{mankind are low}} \\
\poemll    the lofty pride of human beings will be humbled, \\
\poemlll       and the \divine{Lord} alone will be exalted at that time.\fnote{\fbackref{2:11} Lit. \fbib{in that day}} \\
\poeml \v{12}``For the \divine{Lord} of the Heavenly Armies has reserved\fnote{\fbackref{2:12} 1QIsa\textsuperscript{a} MT LXX lack \fbib{reserved}} a time\fnote{\fbackref{2:12} Lit. \fbib{day}} \\
\poemll    to oppose\fnote{\fbackref{2:12} Lit. \fbib{against}} all who are proud and haughty, \\
\poeml and the\fnote{\fbackref{2:12} So 1QIsa\textsuperscript{a}; MT LXX read \fbib{oppose all of the}} self-exalting--- \\
\poemll    they will be humbled. \\
\poeml \v{13}He will take his stand\fnote{\fbackref{2:13} DSS MT LXX lack \fbib{He will take his stand}} against all the cedars\fnote{\fbackref{2:13} I.e. a genus of coniferous evergreen in the family \fbib{Pinaceae}; and so throughout the book} of Lebanon, \\
\poemll    against the proud and self-exalting; \\
\poemlll       and against all the oaks of Bashan; \\
\poeml \v{14}against all the high mountains, \\
\poemll    and against all the lofty hills; \\
\poeml \v{15}against every high tower, \\
\poemll    and against every fortified wall; \\
\poeml \v{16}against all the ships from Tarshish, \\
\poemll    and against all their impressive watercraft. \\
\poeml \v{17}``Humanity's haughtiness will be humbled, \\
\poemll    male arrogance will be brought low, \\
\poeml and the \divine{Lord} alone will be exalted in that day. \\
\poeml \v{18}Their\fnote{\fbackref{2:18} DSS MT LXX lack \fbib{Their}} idols will utterly vanish.\fnote{\fbackref{2:18} So 1QIsa\textsuperscript{a} LXX; MT reads \fbib{He will abolish the idols}} \\
\poeml \v{19}``They will enter caverns in the rocks \\
\poemll    and holes in the ground \\
\poeml to escape\fnote{\fbackref{2:19} Lit. \fbib{from}} the presence of the terror of the \divine{Lord}, \\
\poemll    to escape\fnote{\fbackref{2:19} Lit. \fbib{from}} the splendor of his majesty \\
\poemlll       when he arises to terrify the earth. \\
\poeml \v{20}At that time,\fnote{\fbackref{2:20} Lit. \fbib{day}} mankind will throw \\
\poemll    their silver and gold idols\fnote{\fbackref{2:20} Lit. \fbib{silver idols and gold idols}} \\
\poeml that their fingers have made\fnote{\fbackref{2:20} So 1QIsa\textsuperscript{a}; MT reads \fbib{that they made for themselves}; LXX reads \fbib{that they made}} as objects of worship \\
\poemll    to the moles and to the bats. \\
\poeml \v{21}They will enter caverns in the rocks \\
\poemll    and clefts in the cliffs, \\
\poeml to escape\fnote{\fbackref{2:21} Lit. \fbib{from}} the terror of the \divine{Lord} \\
\poemll    and to escape\fnote{\fbackref{2:21} Lit. \fbib{from}} the splendor of his majesty, \\
\poemlll       when he arises to terrorize the earth. \\
\poeml \v{22}``Stop trusting in human beings, \\
\poemll    whose life breath is in their nostrils, \\
\poemlll       for what are they\fnote{\fbackref{2:22} Lit. \fbib{what is he}} really worth?''\fnote{\fbackref{2:22} LXX lacks this verse}
\end{poetry}
\labelchapt{3}
\passage{Judgment Comes to Judah's Leaders}

\begin{poetry}
\poeml \chapt{3}
\v{1}``Note this! The Lord \divine{God} of the Heavenly Armies \\
\poemll    is taking away from Jerusalem and Judah \\
\poemlll       everything that your society needs---\fnote{\fbackref{3:1} Lit. \fbib{Judah both supply and support}} \\
\poeml all food supplies \\
\poemll    and all water supplies, \\
\poeml \v{2}the mighty man \\
\poemll    and the warrior, \\
\poeml the judge \\
\poemll    and the prophet, \\
\poeml the fortune-teller \\
\poemll    and the elder, \\
\poeml \v{3}the commander of fifty \\
\poemll    and the man of rank, \\
\poeml and the counselor, the expert magician, \\
\poemll    and the medium. \\
\poeml \v{4}``I will make youths their princes, \\
\poemll    and infants will rule over them. \\
\poeml \v{5}People will oppress one another--- \\
\poemll    It will be man against man \\
\poemlll       and neighbor against neighbor. \\
\poeml The young will be disrespectful to the old, \\
\poemll    and the worthless to the honorable. \\
\poeml \v{6}``For a man will grab his brother \\
\poemll    in his own father's house, \\
\poeml and say, `You have a cloak, \\
\poemll    so you be our leader, \\
\poeml and this heap of ruins \\
\poemll    will be under your rule!' \\
\poeml \v{7}``But\fnote{\fbackref{3:7} So 1QIsa\textsuperscript{a} LXX; the Heb. lacks \fbib{But}} at that time,\fnote{\fbackref{3:7} Lit. \fbib{day}} he'll protest!\fnote{\fbackref{3:7} Lit. \fbib{he'll cry out}} \\
\poemll    He'll say, `I won't be your healer. \\
\poeml I have neither food nor clothing in my house! \\
\poemll    You're not going to make me a leader of the people!' \\
\poeml \v{8}``For Jerusalem has stumbled, \\
\poemll    and Judah has fallen, \\
\poeml because what they say and do opposes\fnote{\fbackref{3:8} So 1QIsa\textsuperscript{a}; MT LXX read \fbib{do is towards}} the \divine{Lord}; \\
\poemll    they keep defying him.\fnote{\fbackref{3:8} Lit. \fbib{defying his glorious presence}} \\
\poeml \v{9}``The expressions on their faces give them away.\fnote{\fbackref{3:9} Lit. \fbib{faces bears witness against them}} \\
\poemll    They parade their sin around like Sodom; \\
\poemlll       they don't even try to\fnote{\fbackref{3:9} 1QIsa\textsuperscript{a} MT lack \fbib{try to}} hide it. \\
\poeml How horrible it will be for them, \\
\poemll    because they have brought disaster on themselves!''
\passage{Encouragement to the Righteous}
\poeml \v{10}``Tell\fnote{\fbackref{3:10} So 1QIsa\textsuperscript{a}; the Heb. lacks \fbib{Tell}} the righteous that things will go well, \\
\poemll    because they will enjoy\fnote{\fbackref{3:10} Lit. \fbib{eat}} the fruit of their actions.''
\passage{Warning to the Wicked}
\poeml \v{11}``How terrible it will be for the wicked! \\
\poemll    Disaster is headed their way, \\
\poemlll       because what they did with their hand\fnote{\fbackref{3:11} So 1QIsa\textsuperscript{a}; MT LXX read \fbib{hands}} will be repaid\fnote{\fbackref{3:11} So 1QIsa\textsuperscript{a}; MT reads \fbib{done}} to them. \\
\poeml \v{12}``As for my people, children\fnote{\fbackref{3:12} Or \fbib{youths}} are their oppressors, \\
\poemll    and women rule over them. \\
\poeml My people, your leaders are misleading you--- \\
\poemll    they're giving you confusing directions.''\fnote{\fbackref{3:12} So MT; 1QIsa\textsuperscript{a} reads \fbib{they're devouring your paths}}
\passage{When God Goes to Court}
\poeml \v{13}The \divine{Lord} is taking his place to argue his case; \\
\poemll    he's standing up to judge the people. \\
\poeml \v{14}The \divine{Lord} will go to court\fnote{\fbackref{:14} Lit. \fbib{go into judgment}} \\
\poemll    to oppose\fnote{\fbackref{3:14} Lit. \fbib{with}} the elders and princes of his people: \\
\poeml ``You're the ones who have been devouring the vineyard, \\
\poemll    the plunder of the poor is in your own houses! \\
\poeml \v{15}How dare you crush\fnote{\fbackref{3:15} Lit. \fbib{What do you mean by crushing}} my people \\
\poemll    as you grind down the face of the poor?'' \\
\poemlll       declares the Lord \divine{God} of the Heavenly Armies.\fnote{\fbackref{3:15} So 1QIsa\textsuperscript{a} MT; LXX lacks this line}
\passage{Judgment of Jerusalem's Women}
\poeml \v{16}The \divine{Lord} also says: \\
\poeml ``Because Zion's women are so haughty, \\
\poemll    and walk with outstretched necks, \\
\poeml flirting with their eyes, \\
\poemll    prancing\fnote{\fbackref{3:16} Or \fbib{mincing}} along as they walk, \\
\poemlll       and making tinkling noises with their ankle bracelets,\fnote{\fbackref{3:16} Lit. \fbib{their feet}} \\
\poeml \v{17}therefore the \divine{Lord}\fnote{\fbackref{3:17} So 1QIsa\textsuperscript{a} corrector; 1QIsa\textsuperscript{a} 4QIsa\textsuperscript{b} MT read \fbib{my Lord}; LXX reads \fbib{God}} will afflict sores \\
\poemll    on the heads of Zion's women, \\
\poemlll       and the \divine{Lord}\fnote{\fbackref{3:17} So 1QIsa\textsuperscript{a} corrector MT; 1QIsa\textsuperscript{a} reads \fbib{my Lord}} will expose their private parts.
\end{poetry}

\v{18}``At that time,\fnote{\fbackref{3:18} Lit. \fbib{In that day}} the \divine{Lord}\fnote{\fbackref{3:18} So 1QIsa\textsuperscript{a} LXX; MT 1QIsa\textsuperscript{a} corrector read \fbib{my Lord}} will take away the finery of the ankle bracelets, headbands, crescents, \v{19}pendants, bracelets, veils, \v{20}headdresses, armlets, sashes, perfume boxes, charms, \v{21}signet rings, nose rings, \v{22}fine robes, capes,\fnote{\fbackref{3:22} So 1QIsa\textsuperscript{a}; 4QIsa\textsuperscript{b} MT read \fbib{capes and cloaks}} purses, \v{23}mirrors, linen garments, tiaras, and veils.

\begin{poetry}
\poeml \v{24}``And it will come about that instead of fragrance \\
\poemll    there will be\fnote{\fbackref{3:24} The 1QIsa\textsuperscript{a} lacks \fbib{will be a stench}} a stench; \\
\poeml instead of a belt, a rope; \\
\poemll    instead of well-set hair, baldness; \\
\poeml instead of a fine robe, sackcloth; \\
\poemll    and instead of beauty, shame.\fnote{\fbackref{3:24} So 1QIsa\textsuperscript{a}; MT reads \fbib{burning instead of beauty}; LXX lacks this line} \\
\poeml \v{25}Your men will die violently,\fnote{\fbackref{3:25} Lit. \fbib{will fall by the sword}} \\
\poemll    while your forces\fnote{\fbackref{3:25} So 1QIsa\textsuperscript{a}; MT reads \fbib{force}} fall\fnote{\fbackref{3:25} 1QIsa\textsuperscript{a} MT lack \fbib{fall}} in battle \\
\poeml \v{26}and her gates lament and mourn. \\
\poemll    Ravaged, she will sit on the ground.''
\end{poetry}
\labelchapt{4}

\chapt{4}
\v{1}``At that time,\fnote{\fbackref{4:1} Lit. \fbib{day}} seven women will cling tightly to one man and will make him this offer:\fnote{\fbackref{4:1} Lit. \fbib{will say}} `We'll provide\fnote{\fbackref{4:1} Lit. \fbib{eat}} our own bread. We'll provide our own clothes. Just let us marry you\fnote{\fbackref{4:1} Lit. \fbib{let your name be upon us}} so we won't be stigmatized anymore.'\,''\fnote{\fbackref{4:1} I.e. by appearing to be part of a family}
\passage{The Future Glory of Jerusalem}

\v{2}``At that time,\fnote{\fbackref{4:2} Lit. \fbib{In that day}} the \divine{Lord}'s branch will be beautiful and glorious, and the fruit of the land will be the pride and glory of the survivors of Israel and Judah.\fnote{\fbackref{4:2} So 1QIsa\textsuperscript{a}; MT LXX lack \fbib{and Judah}} \v{3}Whoever\fnote{\fbackref{4:3} Lit. \fbib{It will come about that whoever}} survives in Zion and whoever remains in Jerusalem will be called holy---everyone who has been appointed to survive in Jerusalem--- \v{4}when the \divine{Lord} will have washed away the filth of the women\fnote{\fbackref{4:4} Lit. \fbib{daughters}} of Zion, cleaning up Jerusalem's guilt\fnote{\fbackref{4:4} Lit. \fbib{blood}; i.e. guilt incurred by shedding innocent blood} by a spirit of judgment and a spirit of tempest.\fnote{\fbackref{4:4} So 1QIsa\textsuperscript{a}; MT reads \fbib{of burning}} \v{5}Then the \divine{Lord} will create over the entire site of Mount Zion---including over those who assemble there---a cloud by day\fnote{\fbackref{4:5} So 1QIsa\textsuperscript{a}; MT LXX reads \fbib{day, accompanied by smoke, as well as the brilliance of a flaming fire by night, because over the entire glorious area there will be a canopy and} \fbib{\v{6}a shelter to protect from the heat of the day, and}} \v{6}and also to serve as a refuge and shelter from storms and rain.''
\labelchapt{5}
\passage{The \divine{Lord}'s Vineyard}

\begin{poetry}
\poeml \chapt{5}
\v{1}I will sing\fnote{\fbackref{5:1} So 1QIsa\textsuperscript{a}; MT reads \fbib{Please, let me sing}} for my beloved \\
\poemll    my love-song concerning his vineyard: \\
\poeml ``The one I love had a vineyard \\
\poemll    on a very fertile hill. \\
\poeml \v{2}He plowed its land\fnote{\fbackref{5:2} 1QIsa\textsuperscript{a} MT lack \fbib{its land}} and cleared it of stones. \\
\poemll    Then he planted it with the choicest vines, \\
\poeml built a watchtower in the middle of it, \\
\poemll    and dug a wine vat in it; \\
\poeml He expected\fnote{\fbackref{5:2} Or \fbib{waited for}} it to produce good\fnote{\fbackref{5:2} 1QIsa\textsuperscript{a} MT lack \fbib{good}} grapes, \\
\poemll    but it produced only wild ones.''\fnote{\fbackref{5:2} I.e. grapes unsuitable for wine making} \\
\poeml \v{3}``So now, you inhabitants\fnote{\fbackref{5:3} So 1QIsa\textsuperscript{a}; MT reads \fbib{inhabitant}} of Jerusalem \\
\poemll    and men of Judah, \\
\poeml judge, won't you please, \\
\poemll    between me and my vineyard. \\
\poeml \v{4}What more could I do in\fnote{\fbackref{5:4} So 1QIsa\textsuperscript{a}; MT reads \fbib{for}} my vineyard, \\
\poemll    that I haven't already done? \\
\poeml When I expected it to produce good\fnote{\fbackref{5:4} 1QIsa\textsuperscript{a} MT lack \fbib{good}} grapes, \\
\poemll    why did it yield\fnote{\fbackref{5:4} So 1QIsa\textsuperscript{a}; MT LXX read \fbib{produce}} wild ones?\fnote{\fbackref{5:4} I.e. grapes unsuitable for wine making} \\
\poeml \v{5}``Now, let me tell you, won't you please, \\
\poemll    what I'm going to do to my vineyard. \\
\poeml ``I'm going to take away its protective hedge, \\
\poemll    and it will be devoured;\fnote{\fbackref{5:5} So 1QIsa\textsuperscript{a}; MT \fbib{will be for devouring}; LXX \fbib{will be for plundering}} \\
\poeml I'll break down its wall, \\
\poemll    and it will be trampled. \\
\poeml \v{6}I'll make it a wasteland, \\
\poemll    and it won't be pruned or cultivated. \\
\poeml Instead, briers and thorns will grow up. \\
\poemll    I'll also issue commands to the clouds, \\
\poemlll       that they drop no rain upon it.'' \\
\poeml \v{7}For the vineyard of the \divine{Lord} of the Heavenly Armies \\
\poemll    is the house of Israel, \\
\poeml and the men of Judah \\
\poemll    are the garden in which he delights.\fnote{\fbackref{5:7} So 1QIsa\textsuperscript{a}; MT \fbib{his delightful garden}} \\
\poeml He looked for justice, \\
\poemll    but saw only bloodshed; \\
\poeml he searched\fnote{\fbackref{5:7} 1QIsa\textsuperscript{a} MT lack \fbib{he searched}} for righteousness, \\
\poemll    but heard only an outcry!
\passage{Judgment on Land Barons}
\poeml \v{8}``How terrible it will be for you who join house to house, \\
\poemll    who add field to field, \\
\poeml until there is no more room, \\
\poemll    and you have settled yourselves alone\fnote{\fbackref{5:8} So 1QIsa\textsuperscript{a}; MT reads \fbib{you are made to live alone}} \\
\poemlll       in the middle of the land!'' \\
\poeml \v{9}The \divine{Lord} of the Heavenly Armies has declared this so I could hear it: \\
\poeml ``Surely many houses will become desolate, \\
\poemll    great and beautiful houses, \\
\poemlll       without occupants. \\
\poeml \v{10}For ten acres of vineyard will produce only one bath,\fnote{\fbackref{5:10} I.e. about six gallons} \\
\poemll    and one omer\fnote{\fbackref{5:10} I.e. about ten bushels} of seed\fnote{\fbackref{5:10} 1QIsa\textsuperscript{a} MT lack \fbib{of seed}} will produce only one ephah.''\fnote{\fbackref{5:10} I.e. about one tenth of what was sown}
\passage{Judgment on Alcoholics}
\poeml \v{11}``How terrible it will be for those who rise at dawn \\
\poemll    in order to grab\fnote{\fbackref{5:11} So 1QIsa\textsuperscript{a}; MT LXX \fbib{may run after}} a stiff drink, \\
\poeml for those who stay up late at night \\
\poemll    as wine inflames them! \\
\poeml \v{12}They have the lyre and harp, \\
\poemll    the tambourine and flute, \\
\poemlll       as well as wine at their festivals, \\
\poeml but they don't respect what the \divine{Lord} is doing, \\
\poemll    nor do they consider his actions.\fnote{\fbackref{5:12} Lit. \fbib{consider the work of his hands}} \\
\poeml \v{13}Therefore my people go into exile \\
\poemll    because they lack understanding; \\
\poeml my\fnote{\fbackref{5:13} So 1QIsa\textsuperscript{a}; MT reads \fbib{their}} honored men go hungry, \\
\poemll    and the crowd is parched with thirst. \\
\poeml \v{14}Therefore Sheol's\fnote{\fbackref{5:14} I.e. the realm of the dead} appetite has grown; \\
\poemll    it has opened its mouth beyond limit. \\
\poeml Jerusalem's nobility and her multitudes will go there, \\
\poemll    along with her brawlers and whoever is reveling within her. \\
\poeml \v{15}Humanity is brought low, \\
\poemll    and each one is humbled, \\
\poemlll       while the eyes of the self-exalting are brought low. \\
\poeml \v{16}But the \divine{Lord} of the Heavenly Armies is exalted in justice, \\
\poemll    and the Holy God proves himself to be righteously holy. \\
\poeml \v{17}Then the lambs will graze in their pasture; \\
\poemll    fatlings and foreigners will eat \\
\poemlll       among the waste places of the rich.''
\passage{Judgment on Mockers}
\poeml \v{18}``How terrible it will be for those who parade iniquity with cords of falsehood, \\
\poemll    who draw sin along as\fnote{\fbackref{5:18} 1QIsa\textsuperscript{a} MT lack \fbib{as}} with a cart rope; \\
\poeml \v{19}who say: `Let God\fnote{\fbackref{5:19} Lit. \fbib{him}} be quick, \\
\poemll    let him speed up\fnote{\fbackref{5:19} So 1QIsa\textsuperscript{a}; MT reads \fbib{hurry}} his work \\
\poemlll       so we may see it! \\
\poeml Let it happen! \\
\poemll    let the plan of the Holy One of Israel draw near, \\
\poemlll       so we may recognize it!'\,''
\passage{Judgment on Moral Relativists}
\poeml \v{20}``How terrible it will be for those who call evil good \\
\poemll    and good evil, \\
\poeml who substitute darkness for light \\
\poemll    and light for darkness, \\
\poeml who substitute what is bitter for what is sweet \\
\poemll    and what is sweet for what is bitter!''
\passage{Judgment on the Arrogant}
\poeml \v{21}``How terrible it will be for those who are wise in their own opinion, \\
\poemll    and clever in their own reckoning! \\
\poeml \v{22}``How terrible it will be for those who are heroes at drinking wine, \\
\poemll    and champions in mixing strong drink, \\
\poeml \v{23}who acquit the guilty for a bribe, \\
\poemll    and deprive the innocent of justice!''
\passage{The Effects of Divine Judgment}
\poeml \v{24}Therefore, as flames of fire devour straw, \\
\poemll    as dry grass\fnote{\fbackref{5:24} So MT; 1QIsa\textsuperscript{a} reads \fbib{fire}} collapses in flames, \\
\poeml so their root will be rotten, \\
\poemll    and their blossom will blow away like dust, \\
\poeml because they have rejected the instruction\fnote{\fbackref{5:24} Or \fbib{law}} of the \divine{Lord} of the Heavenly Armies, \\
\poemll    and have despised the word of the Holy One of Israel. \\
\poeml \v{25}Therefore\fnote{\fbackref{5:25} So 1QIsa\textsuperscript{a} MT; LXX reads \fbib{And}} the anger of the \divine{Lord}\fnote{\fbackref{5:25} So 1QIsa\textsuperscript{a} MT; 4QIsa\textsuperscript{b} LXX read \fbib{\divine{Lord} of the Heavenly Armies}} burned against his people, \\
\poemll    so he stretched out his hands\fnote{\fbackref{5:25} So 1QIsa\textsuperscript{a}; MT LXX read \fbib{hand}} against them \\
\poemlll       and afflicted them. \\
\poeml The mountains quaked, \\
\poemll    and their corpses were like refuse \\
\poemlll       in the middle of the streets. \\
\poeml Throughout all of this, his anger has not turned away, \\
\poemll    and his hands are\fnote{\fbackref{5:25} So 1QIsa\textsuperscript{a}; MT LXX read \fbib{his hand is}} still stretched out to attack.\fnote{\fbackref{5:25} DSS MT lack \fbib{to attack}} \\
\poeml \v{26}The \divine{Lord}\fnote{\fbackref{5:26} Lit. \fbib{He}} will signal\fnote{\fbackref{5:26} Lit. \fbib{will send up a signal}} for nations far away, \\
\poemll    whistling for them to come\fnote{\fbackref{5:26} 1QIsa\textsuperscript{a} MT lack \fbib{to come}} \\
\poemlll       from the ends of the earth. \\
\poeml Look how quickly \\
\poemll    and how swiftly they come! \\
\poeml \v{27}No one is weary, no one stumbles,\fnote{\fbackref{5:27} So 1QIsa\textsuperscript{a}; MT reads \fbib{stumbles among them}} \\
\poemll    and no one slumbers or sleeps. \\
\poeml No belt around their waists will come undone, \\
\poemll    nor will their sandal straps be broken. \\
\poeml \v{28}Their arrows are sharp, \\
\poemll    all their bows ready for action.\fnote{\fbackref{5:28} Lit. \fbib{bows bent}} \\
\poeml Their horses' hooves seem like flint, \\
\poemll    and their chariot wheels spin\fnote{\fbackref{5:28} DSS MT lack \fbib{spin}} like a whirlwind. \\
\poeml \v{29}With a roar like a lion, they snarl, \\
\poemll    and like young lions, they growl;\fnote{\fbackref{5:29} So 1QIsa\textsuperscript{a}; MT reads \fbib{Their roaring is like a lion; like young lions they roar. They growl}} \\
\poeml they seize their prey \\
\poemll    and then carry it off, \\
\poemlll       with no one to rescue. \\
\poeml \v{30}They will roar over it\fnote{\fbackref{5:30} I.e. over conquered Judah} at that time,\fnote{\fbackref{5:30} Lit. \fbib{it on that day}} \\
\poemll    like the sea waves roar. \\
\poeml If one surveys the land, watch out! \\
\poemll    There's darkness and distress; \\
\poemlll       even the daylight is darkened by its clouds.
\end{poetry}
\labelchapt{6}
\passage{Holy is the \divine{Lord}}

\chapt{6}
\v{1}In the year that King Uzziah died, I saw the Lord sitting upon his\fnote{\fbackref{6:1} So 1QIsa\textsuperscript{a}; MT LXX read \fbib{a}} throne, high and exalted. The train of his robe filled the Temple. \v{2}The seraphim stood above him. Each had six wings:\fnote{\fbackref{6:2} So 1QIsa\textsuperscript{a}; MT reads \fbib{six wings, six wings}; LXX reads \fbib{six wings and six wings}} with two he covered his face, and with two he covered his feet, and with two he was flying. \v{3}They kept on calling to each other:\fnote{\fbackref{6:3} So 1QIsa\textsuperscript{a}; MT LXX read \fbib{calling and saying}}

\begin{poetry}
\poeml ``Holy, holy, holy\fnote{\fbackref{6:3} So MT LXX; 1QIsa\textsuperscript{a} reads \fbib{Holy, holy}} is the \divine{Lord} of the Heavenly Armies! \\
\poemll    The whole earth is full of his glory!''
\end{poetry}

\v{4}The foundations of the thresholds quaked at the sound of those who kept calling out,\fnote{\fbackref{6:4} So 1QIsa\textsuperscript{a} LXX; MT reads \fbib{of him who called out}} and the Temple was filled with smoke.

\v{5}``How terrible it will be for me!'' I cried, ``because I am ruined! I'm a man with unclean lips, and I live among a people with unclean lips! And my eyes have seen the King, the \divine{Lord} of the Heavenly Armies!''
\passage{The Calling of Isaiah}

\v{6}Then one of the seraphim flew to me, carrying a burning coal in his hand that he had taken from the altar with tongs. \v{7}He touched my mouth and said, ``Look! Now that this has touched your\fnote{\fbackref{6:7} So 1QIsa\textsuperscript{a} MT LXX; 4QIsa\textsuperscript{f} reads \fbib{the}} lips, your guilt is taken away, and your sins\fnote{\fbackref{6:7} So 1QIsa\textsuperscript{a} LXX; MT reads \fbib{sin}} atoned for.''

\v{8}Then I heard the voice of the \divine{Lord} as he was asking, ``Whom will I send? Who will go for us?''

``Here I am!'' I replied. ``Send me.''

\v{9}``Go!'' he responded. ``Tell this people:

\begin{poetry}
\poeml ```Keep on hearing, but do not understand; \\
\poemll    keep\fnote{\fbackref{6:9} So 1QIsa\textsuperscript{a}; MT reads \fbib{and keep}} on seeing, but do not perceive.' \\
\poeml \v{10}Dull the mind\fnote{\fbackref{6:10} Lit. \fbib{Fatten the heart}} of this people, \\
\poemll    deafen their ears, \\
\poemlll       and blind their eyes. \\
\poeml By doing so, they won't see with their eyes, \\
\poemll    hear with their ears, \\
\poeml understand with their minds, \\
\poemll    turn back, \\
\poemlll       and be healed.''
\end{poetry}

\v{11}Then I asked, ``For how long, \divine{Lord}?''\fnote{\fbackref{6:11} So 1QIsa\textsuperscript{a}; MT reads \fbib{Lord}}

He replied:

\begin{poetry}
\poeml ``Until cities lie waste, \\
\poemll    without inhabitants, \\
\poeml and houses without people; \\
\poemll    and the land becomes utterly desolate. \\
\poeml \v{12}Until\fnote{\fbackref{6:12} Lit. \fbib{And}} the \divine{Lord} removes people far away, \\
\poemll    and there are many empty places \\
\poemlll       in the middle of the land. \\
\poeml \v{13}Even though a tenth of its people remain\fnote{\fbackref{6:13} Lit. \fbib{tenth remains}} in it, \\
\poemll    it will once again be burned,\fnote{\fbackref{6:13} Or \fbib{devastated}} \\
\poeml like a terebinth\fnote{\fbackref{6:13} I.e. a sacred tree used for idolatry; cf. Hos 4:13} or an oak tree,\fnote{\fbackref{6:13} Or \fbib{Asherah pole}; i.e. felled oaks used for making idols; cf. Hos 4:13, Isa 44:14} \\
\poemll    the stump of which, though the tree has been\fnote{\fbackref{6:13} 1QIsa\textsuperscript{a} lacks \fbib{though the tree has been}; MT LXX read \fbib{which, when}} felled, \\
\poemlll       still contains holy seed.''\fnote{\fbackref{6:13} So 1QIsa\textsuperscript{a} MT; LXX lacks this line}
\end{poetry}
\labelchapt{7}
\passage{The Message to Ahaz}

\chapt{7}
\v{1}During the reign of Jotham's son Ahaz, Uzziah's grandson, king of Judah, King Rezin of Aram and Remaliah's son Pekah, king of Israel, approached Jerusalem and waged war against it, but they\fnote{\fbackref{7:1} So 1QIsa\textsuperscript{a} LXX; MT reads \fbib{he}} could not mount an attack against it. \v{2}When it was reported to the house of David, ``Aram has joined forces with Ephraim!'' the\fnote{\fbackref{7:2} So 1QIsa\textsuperscript{a}; MT LXX read \fbib{his}} heart of the people of Ahaz\fnote{\fbackref{7:2} So 1QIsa\textsuperscript{a}; MT LXX read \fbib{his heart and the heart of his people}} trembled like forest\fnote{\fbackref{7:2} So 1QIsa\textsuperscript{a}; the Heb. lacks \fbib{forest}} trees in a windstorm.

\v{3}So the \divine{Lord} told Isaiah, ``Go out to meet Ahaz, you and your son Shear-jashub, at the end of the aqueduct of the Upper Pool that proceeds along the highway to Launderer's Field. \v{4}Tell him, `Be careful, be calm, don't be afraid, and don't lose heart because of these two smoldering stumps of torches, that is, because of\fnote{\fbackref{7:4} So 1QIsa\textsuperscript{a}; the Heb. lacks \fbib{because of}} the fierce anger of Rezin, from Aram, and Remaliah's son. \v{5}Aram, Ephraim, and Remaliah's son have plotted this evil against you: \v{6}``Let's go attack Judah, let's terrorize it, and let's conquer it for ourselves. Then we'll install Tabeel's son as king!''\,'

\v{7}`But this is what the Lord \divine{God} has to say:

\begin{poetry}
\poeml ```It won't take place. \\
\poemll    It won't ever happen. \\
\poeml \v{8}Because Aram's head is Damascus, \\
\poemll    and Rezin is its king,\fnote{\fbackref{7:8} Lit. \fbib{is head of Damascus}} \\
\poeml within sixty-five years \\
\poemll    Ephraim will be shattered as a people. \\
\poeml \v{9}Furthermore, Ephraim's head is Samaria, \\
\poemll    and Remaliah's son is its king.\fnote{\fbackref{7:9} Lit. \fbib{is head of Samaria}} \\
\poeml If all of you don't keep on believing,
\end{poetry}

you'll never remain loyal.'\,''\fnote{\fbackref{7:9} Or \fbib{never keep on enduring}}
\passage{God with Us}

\v{10}Later on, the \divine{Lord} spoke to Ahaz again: \v{11}``Ask a sign from the \divine{Lord} your God. Make it as deep as Sheol\fnote{\fbackref{7:11} I.e. the realm of the dead} or as high as heaven above.''

\v{12}But Ahaz replied, ``I won't ask! I won't put the \divine{Lord} to the test.''

\v{13}In reply, the \divine{Lord}\fnote{\fbackref{7:13} Lit. \fbib{reply, he}} announced, ``Please listen, you household of David. Is it such a minor thing for you to try the patience of\fnote{\fbackref{7:13} Lit. \fbib{to wear out}} men? Must you also try the patience of\fnote{\fbackref{7:13} Lit. \fbib{also wear out}} my God?

\v{14}``Therefore the \divine{Lord}\fnote{\fbackref{7:14} So 1QIsa\textsuperscript{a}; MT reads \fbib{Lord}} himself will give you a sign. Watch! The virgin\fnote{\fbackref{7:14} So LXX; 1QIsa\textsuperscript{a} MT read \fbib{The young woman}} is conceiving a child, and will give birth to a son, and his name will be called\fnote{\fbackref{7:14} So 1QIsa\textsuperscript{a}; MT LXX read \fbib{she will name him}; MT alt. reading \fbib{and you will name him}} Immanuel.\fnote{\fbackref{7:14} The Heb. name \fbib{Immanuel} means \fbib{God with us}} \v{15}He'll eat cheese\fnote{\fbackref{7:15} Or \fbib{curds}} and honey, when he knows enough to reject what's wrong and choose what's right. \v{16}However, before the youth knows enough to reject what's wrong and choose what's right, the land whose two kings you dread will be devastated.''
\passage{Conquest by Assyria}

\v{17}``The \divine{Lord} will bring to you, to your people, and to your ancestor's house such a time\fnote{\fbackref{7:17} Lit. \fbib{such days}} as has never been since Ephraim broke away from Judah---the king of Assyria will come.\fnote{\fbackref{7:17} 1QIsa\textsuperscript{a} MT lack \fbib{will come}}

\v{18}``At that time,\fnote{\fbackref{7:18} Lit. \fbib{On that day}} the \divine{Lord} will call\fnote{\fbackref{7:18} Lit. \fbib{whistle}} for flies that will come from far away---from the headwaters of Egypt's rivers---and for bees that are in the land of Assyria. \v{19}They will all come and settle in the steep ravines, in the rocky crevices, in all the thorn bushes, and in all the pastures.\fnote{\fbackref{7:19} Or \fbib{the watering places}} \v{20}At that time,\fnote{\fbackref{7:20} Lit. \fbib{On that day}} the \divine{Lord} will hire a barber\fnote{\fbackref{7:20} Lit. \fbib{razor}} to come\fnote{\fbackref{7:20} 1QIsa\textsuperscript{a} MT lack \fbib{to come}} from beyond the Euphrates\fnote{\fbackref{7:20} 1QIsa\textsuperscript{a} MT lack \fbib{Euphrates}} River---that is, the king of Assyria---and he will shave your heads, your leg\fnote{\fbackref{7:20} Or \fbib{feet}} hair, and your beards, too.

\v{21}``At that time,\fnote{\fbackref{7:21} Lit. \fbib{On that day}} a man will keep alive a heifer and two sheep, \v{22}and because of the abundance of milk that they give, he will have cheese\fnote{\fbackref{7:22} Or \fbib{curds}} to eat, since whoever remains in the land will be eating cheese\fnote{\fbackref{7:22} Or \fbib{curds}} and honey.

\v{23}``At that time,\fnote{\fbackref{7:23} Lit. \fbib{On that day}} every place where once there were a thousand vines worth a thousand shekels\fnote{\fbackref{7:23} I.e. about 400 ounces; a shekel weighed about 0.4 ounces} of silver, only briars and thorns will grow.

\v{24}``People will come there armed with bows\fnote{\fbackref{7:24} So 1QIsa\textsuperscript{a}; MT LXX read \fbib{bow}} and arrows, because the entire land will be nothing but briers and thorns. \v{25}As for all the hills that used to be cultivated with a hoe, you won't go there, because you'll fear iron\fnote{\fbackref{7:25} So 1QIsa\textsuperscript{a}; MT LXX lack \fbib{iron}} briars and thorns. Nevertheless, those hills\fnote{\fbackref{7:25} Lit. \fbib{but they}} will be reserved as a pasture where cattle will feed and where sheep will graze.''\fnote{\fbackref{7:25} Lit. \fbib{tread}}
\labelchapt{8}
\passage{Isaiah's Son is Born}

\chapt{8}
\v{1}The \divine{Lord} also told me, ``Take a large tablet and write on it with a stylus\fnote{\fbackref{8:1} Or \fbib{with an ordinary}} pen, `For Maher-shalal-hash-baz'.\fnote{\fbackref{8:1} The Heb. name \fbib{Maher-shalal-hash-baz} means \fbib{Hurry to the plunder, quick to the loot}} \v{2}Then I will call\fnote{\fbackref{8:2} So 1QIsa\textsuperscript{a} LXX; MT reads \fbib{call in}} Uriah the priest and Jeberechiah's son Zechariah as reliable witnesses to testify on my behalf.''

\v{3}After this, I was intimate with the prophetess and she conceived. Later, she bore a son, and then the \divine{Lord} told me,\fnote{\fbackref{8:3} So 1QIsa\textsuperscript{a} MT LXX; 4QIsa\textsuperscript{e} lacks \fbib{me}} ``Call him\fnote{\fbackref{8:3} Lit. \fbib{Call his name}} `Maher-shalal-hash-baz,' \v{4}for before the young lad knows how to call out to his father or mother,\fnote{\fbackref{8:4} So 1QIsa\textsuperscript{a}; MT reads \fbib{out `My father!' or `My mother!';} LXX reads `\fbib{Father!'} or `\fbib{Mother!'}} the wealth of Damascus and the plunder of Samaria will be carried off by the king of Assyria.''
\passage{Invasion by Assyria}

\v{5}The \divine{Lord} spoke to me again: \v{6}``Because this people have rejected the gently-flowing waters of Shiloah, and because\fnote{\fbackref{8:6} 1QIsa\textsuperscript{a} MT lack \fbib{because}} they keep rejoicing in Rezin and Remaliah's son, \v{7}watch out! The \divine{Lord} God\fnote{\fbackref{8:7} So 1QIsa\textsuperscript{a}; MT reads \fbib{the Lord}} is about to bring the flood waters of the Euphrates\fnote{\fbackref{8:7} 1QIsa\textsuperscript{a} MT lack \fbib{Euphrates}} River against them, mighty and strong.\fnote{\fbackref{8:7} So 4QIsa\textsuperscript{f} MT LXX; 1QIsa\textsuperscript{a} lacks \fbib{mighty and strong}}

``It's the king of Assyria and all of his arrogance! He will rise over all of the river's channels and run over all of its banks. \v{8}He will sweep on into Judah, overflowing as he passes through, like flood waters\fnote{\fbackref{8:8} DSS MT lack \fbib{like flood waters}} reaching up to a person's neck. His outstretched wings will flow as wide as your land, O Immanuel!''

\begin{poetry}
\poeml \v{9}``Band together,\fnote{\fbackref{8:9} So 1QIsa\textsuperscript{a} MT; 4QIsa\textsuperscript{e} 4QIsa\textsuperscript{f} LXX read \fbib{Learn this}; or \fbib{Know this}} you peoples, \\
\poemll    but be shattered! \\
\poemlll       Listen, all you distant countries! \\
\poeml Strap on your armor, \\
\poemlll       but be shattered.\fnote{\fbackref{8:9} So 1QIsa\textsuperscript{a}; MT adds a second \fbib{strap on your armor but be shattered}; cf. LXX} \\
\poeml \v{10}Take counsel together, \\
\poemll    but it will all be for nothing; \\
\poeml go ahead and talk, \\
\poemll    but\fnote{\fbackref{8:10} So 1QIsa\textsuperscript{a} MT; 4QIsa\textsuperscript{e} LXX lack \fbib{but}} it will all be for nothing,\fnote{\fbackref{8:10} Lit. \fbib{it won't stand}} \\
\poemlll       for God is with us.''\fnote{\fbackref{8:10} I.e. a word play on the name \fbib{Immanuel}; cf. 7:14, 8:8}
\end{poetry}
\passage{Waiting on God}

\v{11}For\fnote{\fbackref{8:11} So 1QIsa\textsuperscript{a} 4QIsa\textsuperscript{e} MT; 4QIsa\textsuperscript{f} LXX Syr lack \fbib{For}} this is what the \divine{Lord} spoke to me, as his forceful hand was resting on me, and as he was warning me not to live the way this people were living:\fnote{\fbackref{8:11} Lit. \fbib{not to walk in the way of this people}}

\begin{poetry}
\poeml \v{12}``Don't call conspiracy everything \\
\poemll    that this people calls conspiracy, \\
\poeml and don't fear what they fear, \\
\poemll    or live in terror. \\
\poeml \v{13}The \divine{Lord} of the Heavenly Armies--- \\
\poemll    he's the one you are to regard as holy. \\
\poeml Let him be the one whom you fear, \\
\poemll    and let him be the one before whom you stand in terror! \\
\poeml \v{14}Then he will be a sanctuary, \\
\poemll    but for both houses of Israel \\
\poeml he'll also be a stone with which someone strikes himself, \\
\poemll    a rock one stumbles over, \\
\poemlll       a trap and a snare to those who live in Jerusalem. \\
\poeml \v{15}Many will stumble on them; \\
\poemll    They'll fall and be broken; \\
\poemlll       They'll be snared and captured. \\
\poeml \v{16}``Bind up the testimony, \\
\poemll    and seal up the teaching among my disciples. \\
\poeml \v{17}I'll wait for the \divine{Lord}, \\
\poemll    who is hiding his face from the house of Jacob, \\
\poemlll       and I'll put my trust in him. \\
\poeml \v{18}Watch out! I and the children \\
\poemll    whom the \divine{Lord} has given me \\
\poeml are a sign and a wonder\fnote{\fbackref{8:18} So 1QIsa\textsuperscript{a}; MT LXX read \fbib{are signs and wonders}} in Israel \\
\poemll    from the \divine{Lord} of the Heavenly Armies, \\
\poemlll       who resides on Mount Zion.''
\passage{Rejecting Occultic Wisdom}
\poeml \v{19}``So when they advise you, \\
\poeml `Ask the mediums your questions, \\
\poemll    and quiz the spiritists who chirp and mutter,' \\
\poeml shouldn't a people instead be consulting their God---\fnote{\fbackref{8:19} So 1QIsa\textsuperscript{a} LXX; MT reads \fbib{gods}} \\
\poemll    and not the dead--- \\
\poeml on behalf of those who are living \\
\poeml \v{20}for instruction and for testimony? \\
\poemll    Surely they are speaking like this \\
\poemlll       because the truth\fnote{\fbackref{8:20} 1QIsa\textsuperscript{a} MT lack \fbib{the truth}} hasn't dawned on them. \\
\poeml \v{21}``They'll pass through the land,\fnote{\fbackref{8:21} Lit. \fbib{through it}} \\
\poemll    while\fnote{\fbackref{8:21} So 1QIsa\textsuperscript{a}; the Heb. lacks \fbib{while}} greatly distressed and hungry. \\
\poeml When they are hungry, \\
\poemll    they'll become enraged, \\
\poeml and they'll curse their king and their god.\fnote{\fbackref{8:21} So 1QIsa\textsuperscript{a}; MT reads \fbib{gods}; LXX reads \fbib{idols}} \\
\poemll    They'll turn their faces upwards, \\
\poeml \v{22}or they'll look toward the\fnote{\fbackref{8:22} So 1QIsa\textsuperscript{a} LXX; the Heb. lacks \fbib{the}} earth, \\
\poemll    but they'll see only distress and darkness, \\
\poeml the gloom that comes from anguish, \\
\poemll    and then they'll be thrown into total darkness.''
\end{poetry}
\labelchapt{9}
\passage{The Prince of Peace}

\chapt{9}
\v{1}\fnote{\fbackref{9:1} This v. is 8:23 in the MT}But there will be no gloom for her who was\fnote{\fbackref{9:1} So DSS; MT reads \fbib{those who were}} in distress. Formerly, he brought contempt to the region of Zebulun and the region of Naphtali, but in the future\fnote{\fbackref{9:1} So DSS; MT reads \fbib{the latter time}} he will have made glorious the way of the sea, the territory beyond the Jordan---Galilee of the nations.\fnote{\fbackref{9:1} Or \fbib{gentiles}}

\begin{poetry}
\poeml \v{2}\fnote{\fbackref{9:2} This v. is 9:1 in the MT}The people who walked in darkness \\
\poemll    have seen a great light; \\
\poeml for those living in a land of deep darkness, \\
\poemll    a light has shined upon them. \\
\poeml \v{3}You have increased the nation; \\
\poemll    you have increased its joy; \\
\poeml they rejoice in your presence \\
\poemll    as they rejoice at the harvest, \\
\poeml as they are glad \\
\poemll    when they're dividing the spoils of war.\fnote{\fbackref{9:3} 1QIsa\textsuperscript{a} MT lack \fbib{of war}} \\
\poeml \v{4}Now as to the yoke that has been\fnote{\fbackref{9:4} Lit. \fbib{yoke of}} his burden, \\
\poemll    and the bar laid\fnote{\fbackref{9:4} DSS The Heb. lacks \fbib{laid}} on his shoulder--- \\
\poeml the rod of his oppressor--- \\
\poemll    you have broken it\fnote{\fbackref{9:4} 1QIsa\textsuperscript{a} MT lack \fbib{it}} as on the day of Midiam.\fnote{\fbackref{9:4} So 1QIsa\textsuperscript{a} LXX; MT reads \fbib{Midian}; cf. Judg 7:8-25 2King 15:19; 16:8} \\
\poeml \v{5}For every boot of the tramping soldier in battle tumult \\
\poemll    and every garment rolled in blood \\
\poemlll       will be used for burning as fuel for a fire. \\
\poeml \v{6}For to us a child is born, \\
\poemll    to us a son is given; \\
\poeml and the government will be upon his shoulder, \\
\poemll    and his name is\fnote{\fbackref{9:6} So 1QIsa\textsuperscript{a}; MT 4QIsa\textsuperscript{c} read \fbib{name will be}} called \\
\poeml Wonderful Counselor, Mighty God, \\
\poeml Everlasting Father, Prince of Peace. \\
\poeml \v{7}Of the growth of his government and peace \\
\poemll    there will be no end. \\
\poeml He will rule\fnote{\fbackref{9:7} DSS MT lack \fbib{He will rule}} over his kingdom, \\
\poemll    sitting on the throne of David, \\
\poeml to establish it and to uphold it\fnote{\fbackref{9:7} So 1QIsa\textsuperscript{a}, referring to the throne; MT reads \fbib{it}, referring to the kingdom} \\
\poemll    with justice and righteousness \\
\poemlll       from this time onward and forevermore. \\
\poeml The zeal of the \divine{Lord} of the Heavenly Armies will accomplish this.
\passage{A Rebuke to Jacob and Israel}
\poeml \v{8}``The \divine{Lord}\fnote{\fbackref{9:8} So 1QIsa\textsuperscript{a}; MT reads \fbib{Lord}} has sent a plague\fnote{\fbackref{9:8} So LXX; MT reads \fbib{word}; 1QIsa\textsuperscript{a} can mean \fbib{plague} or \fbib{word}.} against Jacob, \\
\poemll    and it will fall on Israel; \\
\poeml \v{9}and all of the people were evil\fnote{\fbackref{9:9} So 1QIsa\textsuperscript{a}; MT LXX read \fbib{knew}}--- \\
\poemll    Ephraim and the inhabitants of Samaria--- \\
\poemlll       saying proudly with arrogant hearts: \\
\poeml \v{10}`The bricks have fallen, \\
\poemll    but we will build with dressed\fnote{\fbackref{9:10} Or \fbib{quarried}} stones; \\
\poeml the sycamore\fnote{\fbackref{9:10} The sycamore fruit tree native to Israel bears figs} trees have been cut down, \\
\poemll    but we will replace them with cedars.'\fnote{\fbackref{9:10} I.e. a genus of coniferous evergreen in the family \fbib{Pinaceae}} \\
\poeml \v{11}But the \divine{Lord} has raised adversaries\fnote{\fbackref{9:11} So 1QIsa\textsuperscript{a} MT LXX; other MT mss. read \fbib{princes}} from Rezin\fnote{\fbackref{9:11} So 1QIsa\textsuperscript{a} MT; LXX lacks \fbib{from Rezin}} against him, \\
\poemll    and he stirs up his enemies--- \\
\poeml \v{12}Arameans from the east \\
\poemll    and Philistines from the west--- \\
\poeml and they devour Israel with open mouths! \\
\poeml ``Yet\fnote{\fbackref{9:12} So 1QIsa\textsuperscript{a} 4QIsa\textsuperscript{c}; the Heb. lacks \fbib{Yet}} for all this, his anger has not turned away, \\
\poemll    and his hand is still stretched out, ready to strike.''\fnote{\fbackref{9:12} DSS MT lack \fbib{ready to strike}}
\passage{Judgment for Not Repenting}
\poeml \v{13}``But the people have not returned to rely\fnote{\fbackref{9:13} 1QIsa\textsuperscript{a} MT LXX lack \fbib{to rely}} on\fnote{\fbackref{9:13} So 1QIsa\textsuperscript{a}; MT reads \fbib{toward}; LXX reads \fbib{until}} him who struck them, \\
\poemll    nor have they sought the \divine{Lord} of the Heavenly Armies. \\
\poeml \v{14}So the \divine{Lord} has cut off from Israel head and tail, \\
\poemll    palm branch and reed \\
\poemlll       in\fnote{\fbackref{9:14} So 1QIsa\textsuperscript{a} LXX; the The Heb. lacks \fbib{in}} a single day--- \\
\poeml \v{15}the elder and the dignitary is the head, \\
\poemll    and the prophet who teaches lies is the tail. \\
\poeml \v{16}For those who guide this people have been leading them astray, \\
\poemll    and those who are guided by them are swallowed up. \\
\poeml \v{17}Therefore the Lord does not have pity on\fnote{\fbackref{9:17} So 1QIsa\textsuperscript{a};MT LXX read \fbib{rejoice over}} their young men, \\
\poemll    and has no compassion on their orphans\fnote{\fbackref{9:17} Or \fbib{fatherless}} and widows, \\
\poeml because each of them was godless and an evildoer, \\
\poemll    and every mouth spoke folly. \\
\poeml ``Yet\fnote{\fbackref{9:17} So 1QIsa\textsuperscript{a} 4QIsa\textsuperscript{c}; the Heb. lacks \fbib{Yet}} for all this, his anger has not turned away, \\
\poemll    and his hand is still stretched out, ready to strike.\fnote{\fbackref{9:17} DSS MT lack \fbib{ready to strike}} \\
\poeml \v{18}``For wickedness has burned like a blaze \\
\poemll    that consumes briers and thorns; \\
\poeml it sets thickets of the forest on fire, \\
\poemll    and skyward\fnote{\fbackref{9:18} Lit. \fbib{upward}} they swirl \\
\poemlll       in a column of smoke. \\
\poeml \v{19}From\fnote{\fbackref{9:19} So 1QIsa\textsuperscript{a}; MT reads \fbib{By}} the wrath of the \divine{Lord} of the Heavenly Armies \\
\poemll    the land has been scorched, \\
\poeml and the people have become like fuel for the fire; \\
\poemlll       no one will spare his neighbor. \\
\poeml \v{20}They cut meat on the right, \\
\poemll    but they're still hungry, \\
\poeml and they devour also\fnote{\fbackref{9:20} So 1QIsa\textsuperscript{a}; MT LXX lacks \fbib{also}} on the left, \\
\poemll    but they're not satisfied; \\
\poemlll       each devours the flesh of his own children.\fnote{\fbackref{9:20} So 4QIsa\textsuperscript{e}; or \fbib{arms}; 1QIsa\textsuperscript{a} MT read \fbib{offsprin} g or \fbib{arm}; LXX reads \fbib{arm}} \\
\poeml \v{21}Manasseh devours Ephraim, \\
\poemll    and Ephraim devours Manasseh; \\
\poeml together they are against Judah. \\
\poeml ``Yet\fnote{\fbackref{9:21} So 1QIsa\textsuperscript{a}; MT LXX lack \fbib{Yet}} for all this, his anger has not turned away, \\
\poemll    and his hand is still stretched out, ready to strike.''\fnote{\fbackref{9:21} DSS MT lack \fbib{ready to strike}}
\end{poetry}
\labelchapt{10}
\passage{Judgment on Unjust Lawmakers}

\begin{poetry}
\poeml \chapt{10}
\v{1}``How terrible it will be for the one\fnote{\fbackref{10:1} So 1QIsa\textsuperscript{a}; MT reads \fbib{the ones}} who enacts unjust decrees, \\
\poemll    for those who write oppressive laws \\
\poemll    that they have prescribed \\
\poeml \v{2}to deprive the needy of justice \\
\poemll    and to rob the poor of my people of their rights,\fnote{\fbackref{10:2} Lit. \fbib{right}} \\
\poeml so that widows may become their spoil \\
\poemll    and so that they may plunder orphans!\fnote{\fbackref{10:2} Or \fbib{plunder the fatherless}} \\
\poeml \v{3}What will you do on the day of Judgment,\fnote{\fbackref{10:3} Lit. \fbib{reckoning}} \\
\poemll    in the calamity that will come from far away? \\
\poeml To whom will you run for help, \\
\poemll    and where will you leave your wealth, \\
\poeml \v{4}so you won't have to crouch among those in chains\fnote{\fbackref{10:4} So 1QIsa\textsuperscript{a}; MT LXX read \fbib{beneath prisoners}} \\
\poemll    or fall among the slain? \\
\poeml ``Yet\fnote{\fbackref{10:4} So 1QIsa\textsuperscript{a}; MT LXX lack \fbib{Yet}} for all this, his anger has not turned away, \\
\poemll    and his hand is still stretched out, ready to strike.''\fnote{\fbackref{10:4} DSS MT lack \fbib{ready to strike}}
\passage{Assyria is an Instrument of Judgment}
\poeml \v{5}``How terrible it will be \\
\poemll    for Assyria, the rod of my anger! \\
\poemlll       The club is in their hands!\fnote{\fbackref{10:5} So 1QIsa\textsuperscript{a} LXX; MT reads \fbib{is their fury!}} \\
\poeml \v{6}I'm sending my fury\fnote{\fbackref{10:6} So 1QIsa\textsuperscript{a} LXX; MT reads \fbib{sending him}} against a godless nation, \\
\poemll    and I'll command him against the people with whom I'm angry \\
\poeml to seize loot and snatch plunder, \\
\poemll    and to trample them down \\
\poemlll       like mud in the streets. \\
\poeml \v{7}But this is not what he intends, \\
\poemll    and this is not what he thinks in his mind; \\
\poeml but it is in his mind to destroy, \\
\poemll    and to cut down\fnote{\fbackref{10:7} Lit. \fbib{off}} many nations. \\
\poeml \v{8}``Because this is what he is saying: \\
\poemll    `My commanders are all kings, are they not? \\
\poeml \v{9}Isn't Calno like Carchemish? \\
\poemll    Isn't Hamath like Arpad? \\
\poemlll       Isn't Samaria like Damascus? \\
\poeml \v{10}As my hand has reached to the idolatrous kingdoms\fnote{\fbackref{10:10} So 1QIsa\textsuperscript{a}; MT reads \fbib{the idol}} \\
\poemll    whose carved images were greater than those of Jerusalem and Samaria, \\
\poeml \v{11}will I not deal with Jerusalem and her idols \\
\poemll    as I have dealt with Samaria and her images?'\,''
\end{poetry}
\passage{Assyria will be Judged}

\v{12}``For\fnote{\fbackref{10:12} So 1QIsa\textsuperscript{a}; MT LXX read \fbib{And when}} the Lord has finished all his work against Mount Zion and against Jerusalem; he will punish the speech that comes from that willful\fnote{\fbackref{10:12} Lit. \fbib{the fruit of the arrogant}} heart of Assyria's king and the haughty look in his eyes. \v{13}He keeps bragging:\fnote{\fbackref{10:13} Lit. \fbib{saying}; so 1QIsa\textsuperscript{a}; MT reads \fbib{He said}}

\begin{poetry}
\poeml `I've done it by the strength of my hand, \\
\poemll    and by my wisdom, \\
\poemlll       because I'm so clever.\fnote{\fbackref{10:13} Lit. \fbib{I have understanding}} \\
\poeml I removed the boundaries of peoples, \\
\poemll    and plundered their treasures; \\
\poeml like a bull I brought down \\
\poemll    those who sat on thrones. \\
\poeml \v{14}My hand has found, as if in a nest, \\
\poemll    the wealth of the people; \\
\poeml and as one gathers eggs that have been abandoned, \\
\poemll    so I have gathered all the inhabitants of the\fnote{\fbackref{10:14} 1QIsa\textsuperscript{a} MT lack \fbib{inhabitants of the}} earth. \\
\poeml Nothing moved a wing, \\
\poemll    opened its mouth, \\
\poemlll       or chirped.' \\
\poeml \v{15}``Does the ax exalt itself \\
\poemll    over the one who swings it? \\
\poeml Or does the saw magnify itself \\
\poemll    in opposition to the one who wields it? \\
\poeml As if a rod were to wield those who lift\fnote{\fbackref{10:15} So 1QIsa\textsuperscript{a} MT; LXX reads \fbib{the one who lifts}} it, \\
\poemll    or as if a club were to brandish the one who is not wood! \\
\poeml \v{16}Therefore, the Lord \divine{God}\fnote{\fbackref{10:16} So 1QIsa\textsuperscript{a} MT; other LXX MT mss read \fbib{}\divine{Lord}} of the Heavenly Armies will send a wasting disease \\
\poemll    among Assyria's\fnote{\fbackref{10:16} Lit. \fbib{his}} sturdy warriors, \\
\poeml and under its glory a conflagration will be kindled, \\
\poemll    like a blazing bonfire. \\
\poeml \v{17}``The light of Israel will become a fire, \\
\poemll    and its Holy One a flame, \\
\poeml and it will burn \\
\poemll    and consume Assyria's\fnote{\fbackref{10:17} Lit. \fbib{its}} thorns and briers \\
\poemlll       in a single day. \\
\poeml \v{18}The splendor of its forest and its fruitful land \\
\poemll    the \divine{Lord} will destroy--- \\
\poemlll       both soul and body--- \\
\poeml and Assyria\fnote{\fbackref{10:18} Lit. \fbib{it}} will be \\
\poemll    as when a dying man wastes away. \\
\poeml \v{19}What survives of the trees in his forest will be so few \\
\poemll    that a child can count them.''\fnote{\fbackref{10:19} Lit. \fbib{can write them down}}
\end{poetry}
\passage{The Remnant Returns}

\v{20}At that time, the remnant of Israel and the survivors of the house of Jacob will no longer rely on the one who struck them down, but will truly rely on the \divine{Lord}, the Holy One of Israel. \v{21}A remnant will return---a remnant of Jacob---to the Mighty God. \v{22}For even if your people of Israel number as many as the sand of the sea, only a remnant of them will return. Overwhelming, righteous destruction is decreed, \v{23}because the Lord \divine{God} of the Heavenly Armies\fnote{\fbackref{10:23} So 1QIsa\textsuperscript{a}; LXX, MT mss lack \fbib{Lord of the Heavenly Armies}} will bring about destruction, as has been decreed, throughout\fnote{\fbackref{10:23} Lit. \fbib{in the midst of}} the entire region.\fnote{\fbackref{10:23} Lit. \fbib{land}}

\v{24}Therefore this is what the Lord \divine{God} of the Heavenly Armies says: ``My people, you who live in Zion, don't be afraid of the Assyrians, of the rod that beats you,\fnote{\fbackref{10:24} So 1QIsa\textsuperscript{a}; MT reads \fbib{Assyrians, when they strike you with a rod}} who lift up their club against you as the Egyptians did. \v{25}In just a little while, my fury will come to an end, and my anger then will be directed to their destruction.\fnote{\fbackref{10:25} So 1QIsa\textsuperscript{a} MT; MT mss. read \fbib{end}; LXX reads \fbib{counsel}} \v{26}The \divine{Lord} of the Heavenly Armies will brandish a whip against them, as when he struck Midian at the rock of Oreb;\fnote{\fbackref{10:26} Cf. Judg 7:25} and as his staff was stretched out\fnote{\fbackref{10:26} 1QIsa\textsuperscript{a} MT lack \fbib{stretched out}} over the sea,\fnote{\fbackref{10:26} Cf. Exod 14:16,26} so he will lift it up as he did in Egypt. \v{27}At that time,\fnote{\fbackref{10:27} Lit. \fbib{On that day}; so 1QIsa\textsuperscript{a} MT LXX; 4QIsa\textsuperscript{c} reads \fbib{On a day}} his burden will depart from your shoulder and his yoke from your neck. Indeed, the yoke will be broken, because you've become obese.''\fnote{\fbackref{10:27} So 1QIsa\textsuperscript{a} MT; LXX lacks \fbib{because you've become obese}.}
\passage{The Coming Judgment of God}

\begin{poetry}
\poeml \v{28}``The Assyrian commander\fnote{\fbackref{10:28} Lit. \fbib{He}} has come upon\fnote{\fbackref{10:28} So 1QIsa\textsuperscript{a} MT; 4QIsa\textsuperscript{c} LXX read \fbib{to}} Aiath \\
\poemll    and has passed through Migron; \\
\poemlll       he stores his supplies at Michmash. \\
\poeml \v{29}He has\fnote{\fbackref{10:29} So 1QIsa\textsuperscript{a}; MT reads \fbib{They have}} crossed over by\fnote{\fbackref{10:29} So 1QIsa\textsuperscript{a}; the Heb. lacks \fbib{by}} the pass; \\
\poemll    his overnight lodging is at Geba. \\
\poeml Ramah trembles; \\
\poemll    Gibeah of Saul has fled. \\
\poeml \v{30}Cry aloud, you daughter of Gallim! \\
\poemll    Pay attention, Laish!\fnote{\fbackref{10:30} So 1QIsa\textsuperscript{a}; MT LXX read \fbib{Laishah}} \\
\poemlll       Poor Anathoth! \\
\poeml \v{31}Marmenah\fnote{\fbackref{10:31} So 1QIsa\textsuperscript{a} Syr.; MT LXX read \fbib{Madmenah}} is in flight; \\
\poemll    the inhabitants of Gebim take cover. \\
\poeml \v{32}This very day he will halt at Nob;\fnote{\fbackref{10:32} I.e. city where the ephod was stored during the reign of Saul; cf. 1Sam 22:13-20} \\
\poemll    he will shake\fnote{\fbackref{10:32} So 1QIsa\textsuperscript{a}; MT reads \fbib{brandish}} his fists\fnote{\fbackref{10:32} So 1QIsa\textsuperscript{a}; MT reads \fbib{fist}} \\
\poeml at the mountain that is the Daughter of Zion, \\
\poemll    at Jerusalem's hill. \\
\poeml \v{33}Behold, the Lord \divine{God} of the Heavenly Armies \\
\poemll    will lop off its\fnote{\fbackref{10:33} Lit. \fbib{the}} boughs with terrifying power; \\
\poeml the tallest in height will be cut down, \\
\poemll    and the lofty will be brought low. \\
\poeml \v{34}He will cut down the thickets of the forest \\
\poemll    with an ax, \\
\poeml and Lebanon will fall \\
\poemll    by the Majestic One.''\fnote{\fbackref{10:34} Or \fbib{fall, along with its majestic trees}}
\end{poetry}
\labelchapt{11}
\passage{The Reign of the Davidic King}

\begin{poetry}
\poeml \chapt{11}
\v{1}``A shoot will come out \\
\poemll    from the stump of Jesse, \\
\poeml and a branch will bear fruit \\
\poemll    from his roots. \\
\poeml \v{2}The Spirit of the \divine{Lord} will rest upon him, \\
\poemll    the Spirit of wisdom and understanding, \\
\poeml the Spirit of counsel and power, \\
\poemll    the Spirit of knowledge and fear of the \divine{Lord}. \\
\poeml \v{3}His delight will be in the fear of the \divine{Lord}. \\
\poemll    He won't judge by what his eyes see, \\
\poemlll       nor decide disputes by what his ears hear, \\
\poeml \v{4}but with righteousness he will judge the needy, \\
\poemll    and decide with equity for\fnote{\fbackref{11:4} So 1QIsa\textsuperscript{a}; MT LXX read \fbib{for the}} earth's poor.\fnote{\fbackref{11:4} So 1QIsa\textsuperscript{a}; MT LXX read \fbib{humble}} \\
\poeml He will strike the earth with the rod of his mouth,\fnote{\fbackref{11:4} I.e. by pronouncing judgment} \\
\poemll    and the wicked will be killed\fnote{\fbackref{11:4} So 1QIsa\textsuperscript{a}; 1QIsa\textsuperscript{a} corrector MT LXX read \fbib{he will kill the wicked}} with the breath of his lips. \\
\poeml \v{5}Righteousness will be the sash around his loins, \\
\poemll    and faithfulness the belt around his waist.''
\passage{A Transformed Ecology}
\poeml \v{6}``The wolf will live with the lamb; \\
\poemll    the leopard will lie down with the young goat. \\
\poeml The calf and the lion will graze\fnote{\fbackref{11:6} So 1QIsa\textsuperscript{a} LXX; MT reads \fbib{lion and the fattened calf}} together, \\
\poemll    and a little child will lead them. \\
\poeml \v{7}The cow and the bear will graze, \\
\poemll    and\fnote{\fbackref{11:7} So 1QIsa\textsuperscript{a} LXX; the Heb. lacks \fbib{and}} their young will lie down together, \\
\poemlll       and the lion will eat straw like the ox. \\
\poeml \v{8}The nursing child will play \\
\poemll    over the hole of the cobra, \\
\poemlll       and the weaned child will put his hand on vipers' dens.\fnote{\fbackref{11:8} So 1QIsa\textsuperscript{a}; MT reads \fbib{a viper's den}; LXX reads \fbib{a den of vipers}} \\
\poeml \v{9}They will neither harm nor destroy \\
\poemll    on\fnote{\fbackref{11:9} So 1QIsa\textsuperscript{a} LXX; 4QIsa\textsuperscript{c} MT read \fbib{on all}} my holy mountain; \\
\poeml for the earth will be full \\
\poemll    of the knowledge\fnote{\fbackref{11:9} So 1QIsa\textsuperscript{a} MT; 4QIsa\textsuperscript{c} LXX read \fbib{to know}} of the \divine{Lord},\fnote{\fbackref{11:9} So 1QIsa\textsuperscript{a} MT LXX; 4QIsa\textsuperscript{c} reads \fbib{of glory}; cf. Hab 2:14} \\
\poemlll       as the waters cover the sea.''
\end{poetry}
\passage{Israel Regathered}

\v{10}At that time,\fnote{\fbackref{11:10} Lit. \fbib{day}} as to\fnote{\fbackref{11:10} 1QIsa\textsuperscript{a} MT lack \fbib{as to}} the root of Jesse, who will be standing as a banner for the peoples, the nations will rally to him, and his resting place is\fnote{\fbackref{11:10} So 1QIsa\textsuperscript{a}; 4QIsa\textsuperscript{c} MT LXX read \fbib{place will be}} glorious.

\v{11}At that time,\fnote{\fbackref{11:11} Lit. \fbib{day}} the \divine{Lord} will reach out his hand yet a second time to recover the remnant that is left of his people, from Assyria, from Lower Egypt, from Upper Egypt,\fnote{\fbackref{11:11} Lit. \fbib{from Egypt, from Pathros}} from Cush, from Elam, from Shinar, from Hamath, and from the islands\fnote{\fbackref{11:11} Or \fbib{coastlands}} of the sea.

\begin{poetry}
\poeml \v{12}He will raise a banner for the nations \\
\poemll    and will assemble the dispersed of Israel; \\
\poeml he will gather the scattered people of Judah \\
\poemll    from the corners\fnote{\fbackref{11:12} So 1QIsa\textsuperscript{a}; 4QIsa\textsuperscript{a} MT LXX read \fbib{four corners}} of the earth.
\passage{Israel's Victory over Its Enemies}
\poeml \v{13}Ephraim's jealousy will vanish,\fnote{\fbackref{11:13} Lit. \fbib{depart}} \\
\poemll    and those who are hostile to Judah will be eliminated;\fnote{\fbackref{11:13} Lit. \fbib{be cut off}} \\
\poeml Ephraim will no longer be jealous of Judah, \\
\poemll    and Judah will not be hostile to Ephraim. \\
\poeml \v{14}But they\fnote{\fbackref{11:14} So 1QIsa\textsuperscript{a} MT LXX; 4QIsa\textsuperscript{a} Targ read \fbib{he}} will swoop down \\
\poemll    on the slopes\fnote{\fbackref{11:14} Lit. \fbib{backs}} of the Philistines to the west, \\
\poemlll       and they will plunder\fnote{\fbackref{11:14} So pap4QIsa\textsuperscript{e} 1QIsa\textsuperscript{a}; MT LXX read \fbib{the west; together they will plunder}} the people to\fnote{\fbackref{11:14} Lit. \fbib{of}} the east. \\
\poeml They'll lay their hands on Edom and Moab, \\
\poemll    and the Ammonites will be subject to them. \\
\poeml \v{15}The \divine{Lord} will totally destroy \\
\poemll    the gulf\fnote{\fbackref{11:15} Lit. \fbib{tongue}} of the Sea of Egypt. \\
\poeml He will sweep his hand \\
\poemll    over the Euphrates River \\
\poemlll       with a violent wind,\fnote{\fbackref{11:15} So 1QIsa\textsuperscript{a}; MT LXX read \fbib{his violent wind}} \\
\poeml breaking it up into seven streams, \\
\poemll    and making a way for people to cross on foot. \\
\poeml \v{16}And there will be a highway \\
\poemll    for the remnant that is left of his people out of Assyria, \\
\poeml as there was for Israel \\
\poemll    when they came up \\
\poemlll       from the land of Egypt.
\end{poetry}
\labelchapt{12}
\passage{Israel's Praise to the \divine{Lord}}

\begin{poetry}
\poeml \chapt{12}
\v{1}At that time,\fnote{\fbackref{12:1} Lit. \fbib{day}} you will say:
\end{poetry}

\begin{poetry}
\poeml ``I will praise you, \divine{Lord}, \\
\poemll    for although you were angry with me, \\
\poeml your anger has turned away, \\
\poemll    and you have comforted me. \\
\poeml \v{2}``Look! God---yes God---is\fnote{\fbackref{12:2} So 1QIsa\textsuperscript{a}; MT reads \fbib{Look! God is}; LXX reads \fbib{Look! The \divine{Lord} is the God of}} my salvation; \\
\poemll    I will trust, and not be afraid. \\
\poeml For the \divine{Lord}\fnote{\fbackref{12:2} So 1QIsa\textsuperscript{a} MT\textsuperscript{mss} LXX; MT reads \fbib{Lord \divine{God}}} is my strength and my song,\fnote{\fbackref{12:2} So 1QIsa\textsuperscript{a} MT\textsuperscript{mss} LXX; MT reads \fbib{a song}} \\
\poemll    and he has become my salvation.''
\end{poetry}

\v{3}You will draw water joyfully from the wells of salvation. And you will say at that time:\fnote{\fbackref{12:3} Lit. \fbib{say in that day}}

\begin{poetry}
\poeml \v{4}``Give thanks to the \divine{Lord}; \\
\poemll    call on his name. \\
\poeml Make known his actions \\
\poemll    among the nations. \\
\poemlll       Proclaim that his name is exalted. \\
\poeml \v{5}``Sing praises to the \divine{Lord},\fnote{\fbackref{12:5} So 1QIsa\textsuperscript{a}; MT reads \fbib{to the \divine{Lord}}; LXX reads \fbib{to the name of the \divine{Lord}}} \\
\poemll    because he has acted gloriously, \\
\poemlll       being made\fnote{\fbackref{12:5} So Isa\textsuperscript{a} MT\textsuperscript{qere} Syr Targ; the Heb. lacks \fbib{made}} known in all the world. \\
\poeml \v{6}Shout aloud, and sing for joy, \\
\poemll    you who live in Zion, \\
\poeml because great in your midst \\
\poemll    is the Holy One of Israel.''
\end{poetry}
\labelchapt{13}
\passage{The Destruction of Babylon}

\chapt{13}
\v{1}A message\fnote{\fbackref{13:1} Lit. \fbib{An oracle}} that Amoz's son Isaiah received\fnote{\fbackref{13:1} Lit. \fbib{saw}} about Babylon:

\begin{poetry}
\poeml \v{2}``Raise a banner on a bare hilltop! \\
\poemll    Cry out loud to them! \\
\poeml Give a wave of the hand, \\
\poemll    signaling\fnote{\fbackref{13:2} The Heb. lacks \fbib{signaling}} for them to enter\fnote{\fbackref{13:2} So 1QIsa\textsuperscript{a}; MT reads \fbib{for them to enter}; LXX lacks \fbib{to enter}} \\
\poemlll       the gates of the nobles. \\
\poeml \v{3}I myself have commanded my consecrated ones; \\
\poemll    I have also summoned my warriors, \\
\poeml those who rejoice in my triumph, \\
\poemll    to carry out my angry judgments.\fnote{\fbackref{13:3} Lit. \fbib{my anger}} \\
\poeml \v{4}``Listen! There's a noise on the mountains \\
\poemll    like that of a great multitude! \\
\poeml Listen! There's an uproar among the kingdoms, \\
\poemll    like that of nations massing together! \\
\poeml The \divine{Lord} of the Heavenly Armies is mustering \\
\poemll    an army for battle. \\
\poeml \v{5}They're coming from a faraway land, \\
\poemll    from the distant horizon---\fnote{\fbackref{13:5} Lit. \fbib{end of the heavens}} \\
\poeml the \divine{Lord} and the weapons of his anger--- \\
\poemll    to destroy the entire land.''\fnote{\fbackref{13:5} Or \fbib{earth}}
\passage{The Day of the \divine{Lord}}
\poeml \v{6}Wail out loud, because the Day of the \divine{Lord} is near. \\
\poemll    It will come like destruction from the Almighty! \\
\poeml \v{7}Because of this, every hand\fnote{\fbackref{13:7} So 1QIsa\textsuperscript{a}; MT reads \fbib{all hands}} will go limp, \\
\poemll    and every man's courage\fnote{\fbackref{13:7} Lit. \fbib{heart}} will melt. \\
\poeml \v{8}They will be terrified; \\
\poemll    pain and anguish will seize them; \\
\poeml they'll writhe like a woman in labor. \\
\poemll    They'll look aghast at one another; \\
\poemlll       and\fnote{\fbackref{13:8} So 1QIsa\textsuperscript{a}; cf. LXX; 4QIsa\textsuperscript{a} 4QIsa\textsuperscript{b} MT lack \fbib{and}} their faces will be ablaze with fear.\fnote{\fbackref{13:8} DSS MT lack \fbib{with fear}} \\
\poeml \v{9}Watch out! The Day of the \divine{Lord} is coming--- \\
\poemll    cruel, with wrath and fierce anger--- \\
\poeml to turn the entire inhabited\fnote{\fbackref{13:9} So LXX; 1QIsa\textsuperscript{a} lacks \fbib{the entire inhabited}; the Heb. lacks \fbib{entire inhabited}} earth\fnote{\fbackref{13:9} LXX lacks \fbib{earth}} into a desolation \\
\poemll    and to annihilate sinners\fnote{\fbackref{13:9} So 1QIsa\textsuperscript{a} LXX; 4QIsa\textsuperscript{a} 4QIsa\textsuperscript{b} MT read \fbib{its sinners}} from it. \\
\poeml \v{10}For the stars of the heavens and their constellations \\
\poemll    won't shine\fnote{\fbackref{13:10} So 1QIsa\textsuperscript{a}; MT reads \fbib{beam}} their light; \\
\poeml the sun will be dark when it rises, \\
\poemll    and the moon won't shine its light. \\
\poeml \v{11}I'll punish the world for its evil, \\
\poemll    and the wicked for their iniquity; \\
\poeml I'll put an end to the pomposity of the arrogant, \\
\poemll    and overthrow the insolence of tyrants. \\
\poeml \v{12}I'll make people scarcer\fnote{\fbackref{13:12} Lit. \fbib{people more precious}} than pure gold, \\
\poemll    and mankind rarer\fnote{\fbackref{13:12} 1QIsa\textsuperscript{a} MT lack \fbib{rarer}} than gold from Ophir. \\
\poeml \v{13}Therefore I'll make the heavens tremble. \\
\poemll    The earth will shake from its place \\
\poeml at the wrath of the \divine{Lord} of the Heavenly Armies, \\
\poemll    at the time\fnote{\fbackref{13:13} Lit. \fbib{in the day}} of his burning anger.\fnote{\fbackref{13:13} Lit. \fbib{nostrils}} \\
\poeml \v{14}They\fnote{\fbackref{13:14} So 1QIsa\textsuperscript{a} LXX; MT reads \fbib{it}} will be like a hunted gazelle, \\
\poemll    or like sheep with no one to gather them,\fnote{\fbackref{13:14} So 1QIsa\textsuperscript{a} MT LXX; 4QIsa\textsuperscript{a} reads \fbib{banished}} \\
\poeml each will turn to his own people, \\
\poemll    and each will flee to his own land. \\
\poeml \v{15}Whoever is captured will be thrust through, \\
\poemll    and whoever is caught will fall dead, killed\fnote{\fbackref{13:15} So 1QIsa\textsuperscript{a}; MT lacks \fbib{dead, killed}} by the sword. \\
\poeml \v{16}Their infants will be dashed to pieces \\
\poemll    before their eyes, \\
\poeml and\fnote{\fbackref{13:16} So 1QIsa\textsuperscript{a} Syr; cf. LXX; 4QIsa\textsuperscript{a} MT lack \fbib{and}} their houses will be looted, \\
\poemll    and their wives slept with.\fnote{\fbackref{13:16} So 1QIsa\textsuperscript{a} MT\textsuperscript{qere}; 4QIsa\textsuperscript{a} MT read \fbib{raped}; LXX reads \fbib{they will take}; cf. Deut 28:30}
\passage{Babylon Falls}
\poeml \v{17}Watch out! I'm stirring up the Medes against them, \\
\poemll    who care nothing for silver \\
\poemlll       and take no delight in gold. \\
\poeml \v{18}Their bows will dash the young men to pieces; \\
\poemll    they'll show no pity on those not yet born,\fnote{\fbackref{13:18} Lit. \fbib{on the fruit of the womb}} \\
\poemlll       and\fnote{\fbackref{13:18} So 1QIsa\textsuperscript{a} MT\textsuperscript{mss} LXX; the Heb. lacks \fbib{and}} their eyes will not spare children. \\
\poeml \v{19}Babylon, that jewel of kingdoms, \\
\poemll    the splendor and pride of the Chaldeans, \\
\poeml will be like Sodom and Gomorrah, \\
\poemll    when God overthrew them--- \\
\poeml \v{20}It will never be inhabited \\
\poemll    or lived in through all generations; \\
\poeml no Bedouin\fnote{\fbackref{13:20} I.e. Middle Eastern nomadic herders; or \fbib{Arab}} will pitch his tent there; \\
\poemll    no shepherds will make their flocks lie down there. \\
\poeml \v{21}But desert beasts will lie down there, \\
\poemll    and their houses will be full of howling creatures; \\
\poeml there owls\fnote{\fbackref{13:21} Or \fbib{ostriches}} will dwell, \\
\poemll    and goat-demons\fnote{\fbackref{13:21} Or \fbib{satyrs}; or \fbib{wild goats}} will dance there. \\
\poeml \v{22}Hyenas will howl in its strongholds,\fnote{\fbackref{13:22} Lit. \fbib{desolate places}} \\
\poemll    and jackals will make their dens in its citadels.\fnote{\fbackref{13:22} So 1QIsa\textsuperscript{a} LXX; MT reads \fbib{in the citadels of luxury}} \\
\poeml Its\fnote{\fbackref{13:22} So 1QIsa\textsuperscript{a} LXX; MT reads \fbib{And its}} time is close at hand, \\
\poemll    and its days will not be extended any further.\fnote{\fbackref{13:22} So 1QIsa\textsuperscript{a}; MT LXX lack \fbib{any further}}
\end{poetry}
\labelchapt{14}
\passage{Israel Mocks Babylon's King}

\chapt{14}
\v{1}However, the \divine{Lord} will have compassion on Jacob and will once again choose Israel. He will settle them in their own land, and foreigners will join them, affiliating themselves with the house of Jacob. \v{2}Many\fnote{\fbackref{14:2} So 1QIsa\textsuperscript{a}; MT LXX lack \fbib{Many}} nations will take them and bring them to their land and\fnote{\fbackref{14:2} So 1QIsa\textsuperscript{a}; MT LXX read \fbib{and to}} their own place. The house of Israel will put those nations\fnote{\fbackref{14:2} Lit. \fbib{put them}} to conscripted labor\fnote{\fbackref{14:2} Lit. \fbib{to male and female slavery}} in the \divine{Lord}'s land. They will take captive those who were their captors, and will rule continually\fnote{\fbackref{14:2} So 1QIsa\textsuperscript{a}; 4QIsa\textsuperscript{c} LXX MT read \fbib{rule}} over those who oppressed them.

\v{3}At the time,\fnote{\fbackref{14:3} Lit. \fbib{day}} when the \divine{Lord} gives you rest from your suffering, turmoil, and the cruel bondage which they forced you to serve,\fnote{\fbackref{14:3} So 1QIsa\textsuperscript{a} 4QIsa\textsuperscript{e}; MT LXX read \fbib{which you were forced to serve}} \v{4}you will lift up this song of mockery against\fnote{\fbackref{14:4} So 1QIsa\textsuperscript{a} MT LXX; 4QIsa\textsuperscript{e} reads \fbib{to}} the king of Babylon:

\begin{poetry}
\poeml ``How the oppressor has come to an end! \\
\poemll    How the attacker\fnote{\fbackref{14:4} 1QIsa\textsuperscript{a} LXX; MT reads. \fbib{the golden city}} has ceased! \\
\poeml \v{5}The \divine{Lord} has broken the staff of the wicked, \\
\poemll    the scepter of rulers, \\
\poeml \v{6}that struck down peoples in anger \\
\poemll    with unceasing blows, \\
\poeml that oppressed nations in fury \\
\poemll    with relentless persecution. \\
\poeml \v{7}The entire earth is at rest and peace; \\
\poemll    its inhabitants\fnote{\fbackref{14:7} Lit. \fbib{they}} break into song. \\
\poeml \v{8}Even the cypresses rejoice over you, \\
\poemll    as do\fnote{\fbackref{14:8} So 1QIsa\textsuperscript{a}; MT LXX lack \fbib{as do}} the cedars of Lebanon, saying, \\
\poeml `Now that you've been laid low, \\
\poemll    no woodcutter comes up against us.'\fnote{\fbackref{14:8} So 1QIsa\textsuperscript{a} MT LXX; 4QIsa\textsuperscript{e} reads \fbib{against them}} \\
\poeml \v{9}``The afterlife\fnote{\fbackref{14:9} Lit. \fbib{Sheol}, i.e. the realm of the dead} below is all astir \\
\poemll    to meet you when you arrive;\fnote{\fbackref{14:9} Lit. \fbib{at your coming}} \\
\poeml it rouses up the spirits of the dead to greet you--- \\
\poemll    everyone who used to be world leaders. \\
\poeml It has raised up from their thrones \\
\poemll    all who used to be kings of the nations. \\
\poeml \v{10}In answer, all of them\fnote{\fbackref{14:10} So 1QIsa\textsuperscript{a} MT LXX; 4QIsa\textsuperscript{e} lacks \fbib{of them}} will tell you, \\
\poemll    `You've also become as weak as we are! \\
\poeml You have become just like us!' \\
\poeml \v{11}Your\fnote{\fbackref{14:11} 1QIsa\textsuperscript{a} reads \fbib{The;} MT LXX lack \fbib{Your}} pomp has been brought down to Sheol,\fnote{\fbackref{14:11} I.e. the realm of the dead} \\
\poemll    along with the noise of your harps. \\
\poeml Maggots are spread out beneath you, \\
\poemll    and worms are your covering.''\fnote{\fbackref{14:11} So LXX; MT reads \fbib{and a worm covers you}}
\passage{The Fall of the Day Star}
\poeml \v{12}``How you have fallen from heaven, \\
\poemll    Day Star, son of the Dawn!\fnote{\fbackref{14:12} I.e. Lucifer} \\
\poeml How you have been thrown down to earth, \\
\poemll    you who laid low the nation!\fnote{\fbackref{14:12} So 1QIsa\textsuperscript{a}; 4QIsa\textsuperscript{e} MT read \fbib{the nations}; LXX reads \fbib{all the nations}} \\
\poeml \v{13}You said in your heart, \\
\poemll    `I'll ascend to heaven, \\
\poemlll       above the stars of God. \\
\poeml I'll erect my throne; \\
\poemll    I'll sit\fnote{\fbackref{14:13} So 1QIsa\textsuperscript{a} LXX; MT reads \fbib{and I will sit}} on the Mount of Assembly \\
\poemlll       in the far reaches of the north;\fnote{\fbackref{14:13} Lit. \fbib{Zaphon}; or \fbib{the Sacred Mountain}} \\
\poeml \v{14}I'll ascend above the tops of the clouds; \\
\poemll    I'll make myself like the Most High.' \\
\poeml \v{15}But you are brought down to join the dead,\fnote{\fbackref{14:15} Lit. \fbib{to Sheol}, i.e. the realm of the dead} \\
\poemll    to the far reaches of the Pit.\fnote{\fbackref{14:15} I.e. the realm of punishment in the afterlife} \\
\poeml \v{16}``Those who see you will stare at you. \\
\poemll    They will wonder about you: \\
\poeml `Is this the man who\fnote{\fbackref{14:16} So 1QIsa\textsuperscript{a} LXX; the Heb. lacks \fbib{who}} made the earth tremble, \\
\poemll    who made kingdoms quake, \\
\poeml \v{17}who made the world like a desert, \\
\poemll    who\fnote{\fbackref{14:17} So 1QIsa\textsuperscript{a}; MT LXX read \fbib{and who}} destroyed its cities, \\
\poemlll       who would not open the jails for his prisoners?' \\
\poeml \v{18}All the kings of the nations lie\fnote{\fbackref{14:18} So 1QIsa\textsuperscript{a} LXX; MT reads \fbib{nations, every one of them lies}} in state, \\
\poemll    each in his own tomb. \\
\poeml \v{19}But you are cast away from your grave, \\
\poemll    like a repulsive branch, \\
\poeml your clothing is the slain, \\
\poemll    those pierced by the sword; \\
\poemlll       those who go down to the Pit.\fnote{\fbackref{14:19} So 1QIsa\textsuperscript{a}; i.e. to the realm of punishment in the afterlife; LXX reads \fbib{to Hades}; 1QIsa\textsuperscript{a} corrector MT read \fbib{to the stones of the Pit}} \\
\poeml Like a dead body trampled underfoot, \\
\poeml \v{20}you will not be united\fnote{\fbackref{14:20} So 1QIsa\textsuperscript{a}; Lit. \fbib{under}; MT reads \fbib{joined}} with them\fnote{\fbackref{14:20} I.e. with the dead} in burial, \\
\poeml for you have destroyed your land, \\
\poemll    you have slain your people. \\
\poeml People will never mention \\
\poemll    the descendants of those who practice evil again!\fnote{\fbackref{14:20} So 1QIsa\textsuperscript{a}; MT reads \fbib{May the descendants of those who practice of evil never be mentioned again!}; LXX reads \fbib{May you not remain forever, you evil seed!}} \\
\poeml \v{21}Prepare a massacre for his sons \\
\poemll    because of the guilt of their forefathers! \\
\poeml They are not to rise and inherit the earth, \\
\poemll    and cover\fnote{\fbackref{14:21} Lit. \fbib{fill}} the surface of the world with cities.''
\end{poetry}
\passage{Babylon's Desolation}

\v{22}``I will rise up against them,'' declares the \divine{Lord} of the Heavenly Armies, ``and I will eliminate from Babylon her name and survivors, her offspring and descendants,''\fnote{\fbackref{14:22} So 1QIsa\textsuperscript{a}; MT reads \fbib{and her offspring and descendants}; LXX lacks \fbib{and descendants}} declares the \divine{Lord}. \v{23}``And\fnote{\fbackref{14:23} So 1QIsa\textsuperscript{a}; MT LXX lack \fbib{And}} I'm going to make it a possession of the hedgehog---pools\fnote{\fbackref{14:23} So 1QIsa\textsuperscript{a}; MT reads \fbib{and pools}} of water---and I'll sweep\fnote{\fbackref{14:23} So 1QIsa\textsuperscript{a}; MT reads \fbib{sweep it}} with the broom of destruction,'' declares the \divine{Lord} of the Heavenly Armies.

\begin{poetry}
\poeml \v{24}The \divine{Lord} of the Heavenly Armies has sworn: \\
\poeml ``Surely as I have planned, \\
\poemll    that's what she\fnote{\fbackref{14:24} I.e. Babylon} will become;\fnote{\fbackref{14:24} So 1QIsa\textsuperscript{a} LXX; MT reads \fbib{so has she been}} \\
\poeml and just as I have determined, \\
\poemll    so will it remain--- \\
\poeml \v{25}to crush the Assyrian in my land, \\
\poemll    and on my mountains I will trample him down. \\
\poeml His yoke\fnote{\fbackref{14:25} I.e. Assyria's oppressive domination} will turn away from you,\fnote{\fbackref{14:25} So 1QIsa\textsuperscript{a}; MT LXX read \fbib{from them}} \\
\poemll    and his burden from your\fnote{\fbackref{14:25} So 1QIsa\textsuperscript{a}; MT reads \fbib{their}} shoulders.'' \\
\poeml \v{26}``This is what I've planned \\
\poemll    for the whole earth, \\
\poeml and this is the hand that is stretched out \\
\poemll    over all the nations. \\
\poeml \v{27}For the \divine{Lord} of the Heavenly Armies has planned, \\
\poemll    and who can thwart him? \\
\poeml His hand is stretched out, \\
\poemll    and who can turn it back?''
\end{poetry}
\passage{A Pronouncement against Philistia}

\v{28}In the year that King Ahaz died this message\fnote{\fbackref{14:28} Lit. \fbib{oracle}} came:

\begin{poetry}
\poeml \v{29}``Don't rejoice, all of you Philistines, \\
\poemll    that the rod that struck you is broken, \\
\poeml because from the snake's root a viper will spring up, \\
\poemll    and its offspring\fnote{\fbackref{14:29} Lit. \fbib{fruit}} will be a darting, poisonous serpent. \\
\poeml \v{30}The firstborn of the poor will find pasture, \\
\poemll    and the needy will lie down in safety; \\
\poeml but I'll kill your root\fnote{\fbackref{14:30} I.e. the source of their strengths} by famine, \\
\poemll    and I'll\fnote{\fbackref{14:30} So 1QIsa\textsuperscript{a}; MT LXX read \fbib{he}} execute your survivors. \\
\poeml \v{31}Wail, you gate! \\
\poemll    Cry out, you city! \\
\poemlll       Melt away,\fnote{\fbackref{14:31} Or \fbib{melt in fear}} all you Philistines! \\
\poeml For smoke comes from the north, \\
\poemll    and there's no one to take measure\fnote{\fbackref{14:31} So 1QIsa\textsuperscript{a}; 4QIsa\textsuperscript{o} MT read \fbib{no straggler}} in its festivals.\fnote{\fbackref{14:31} So 1QIsa\textsuperscript{a}; 4QIsa\textsuperscript{o} reads \fbib{ranks}} \\
\poeml \v{32}How will they\fnote{\fbackref{14:32} So 1QIsa\textsuperscript{a} LXX; MT reads \fbib{one}} answer the messengers of the nation? \\
\poemll    ``The \divine{Lord} has founded Zion, \\
\poemlll       and in it\fnote{\fbackref{14:32} So 1QIsa\textsuperscript{a} LXX; MT reads \fbib{her}} the afflicted among his people will find refuge.''
\end{poetry}
\labelchapt{15}
\passage{Moab's Pending Judgment}

\chapt{15}
\v{1}A message\fnote{\fbackref{15:1} Lit. \fbib{An oracle}} concerning Moab:

\begin{poetry}
\poeml ``For Ir\fnote{\fbackref{15:1} Or \fbib{For the city}; so 1QIsa\textsuperscript{a}; MT reads \fbib{Ar}; LXX lacks \fbib{For Ir}} in Moab is destroyed in a night, \\
\poemll    and Moab\fnote{\fbackref{15:1} So 1QIsa\textsuperscript{a}; MT reads \fbib{it;} LXX lacks \fbib{it}} is ruined! \\
\poeml Because Ir\fnote{\fbackref{15:1} Or \fbib{Because the city}; so 1QIsa\textsuperscript{a}; 4QIsa\textsuperscript{o} MT LXX read \fbib{Because the wall} or \fbib{Because Kir}} in Moab is destroyed in a single night, \\
\poemll    Moab is ruined! \\
\poeml \v{2}He has gone up to the temple, and to Dibon, \\
\poemll    to the high places to weep; \\
\poeml over Nebo and over Medeba \\
\poemll    Moab wails. \\
\poeml His head is completely\fnote{\fbackref{15:2} So 1QIsa\textsuperscript{a}; MT reads \fbib{all its heads are}; MT\textsuperscript{mss} LXX read \fbib{Over every head}; cf. Jer 48:37} bald, \\
\poemll    and\fnote{\fbackref{15:2} So 1QIsa\textsuperscript{a} Mt\textsuperscript{mss} LXX; the Heb. lacks \fbib{and}} every beard is shaved off. \\
\poeml \v{3}In its streets they wear sackcloth; \\
\poemll    on its rooftops and in its squares \\
\poemlll       everyone wails and falls down\fnote{\fbackref{15:3} So 1QIsa\textsuperscript{a}; MT reads \fbib{falling down}; LXX lacks \fbib{and falls down}} weeping. \\
\poeml \v{4}Heshbon and Elealeh cry out, \\
\poemll    their voices are heard as far as Jahaz; \\
\poeml therefore the loins\fnote{\fbackref{15:4} So 1QIsa\textsuperscript{a}; cf. LXX; MT reads \fbib{armed men}} of Moab cry aloud; \\
\poemll    its heart quakes for itself. \\
\poeml \v{5}My heart cries out over Moab; \\
\poemll    her fugitives flee as far as Zoar, \\
\poemlll       as far as Eglath-shelishiyah. \\
\poeml For at the ascent to Luhith \\
\poemll    they go up weeping; \\
\poeml on the road to Horonaim \\
\poemll    they raise a cry of destruction. \\
\poeml \v{6}The Nimrim waters are desolate; \\
\poemll    the grass is withered, \\
\poeml its vegetation gone; \\
\poemlll       there is\fnote{\fbackref{15:6} So 1QIsa\textsuperscript{a}; MT reads \fbib{was} or \fbib{there has been}; LXX reads \fbib{there will be}} no foliage left. \\
\poeml \v{7}Therefore the wealth they have acquired \\
\poemll    and what they have stored up--- \\
\poeml they carry them away \\
\poemll    over the Arab\fnote{\fbackref{15:7} So 1QIsa\textsuperscript{a}; cf. LXX; MT reads \fbib{Willow}} Wadi.\fnote{\fbackref{15:7} I.e. a seasonal stream or river that channels water during rain seasons but is dry at other times} \\
\poeml \v{8}For the cry has gone out \\
\poemll    along the border of Moab; \\
\poeml her wailing reaches as far as Eglaim, \\
\poemll    her wailing reaches as far as Beer-elim. \\
\poeml \v{9}The Dibon\fnote{\fbackref{15:9} So 1QIsa\textsuperscript{a}; MT reads \fbib{Dimon}; LXX reads \fbib{Remmon}} streams are full of blood; \\
\poemll    but I will bring upon Dibon\fnote{\fbackref{15:9} So 1QIsa\textsuperscript{a}; MT reads \fbib{Dimon}; LXX reads \fbib{Remmon}} even more--- \\
\poeml a lion will pounce\fnote{\fbackref{15:9} 1QIsa\textsuperscript{a} LXX MT lack \fbib{will pounce}} upon those of Moab who escape, \\
\poeml upon\fnote{\fbackref{15:9} So 1QIsa\textsuperscript{a} LXX; the Heb. lacks \fbib{upon}} those who remain in the land.''
\end{poetry}
\labelchapt{16}
\passage{Moab's Destruction}

\begin{poetry}
\poeml \chapt{16}
\v{1}``Send a lamb to the ruler of the land, \\
\poemll    from Selah,\fnote{\fbackref{16:1} So 1QIsa\textsuperscript{a}; MT reads \fbib{Sela}; LXX reads \fbib{not the rock}} by way of the desert, \\
\poemlll       to the mountain of the Daughter of Zion.\fnote{\fbackref{16:1} I.e. Mt. Zion} \\
\poeml \v{2}Like fluttering birds, \\
\poemll    like an abandoned nest, \\
\poeml so are the daughters of Moab \\
\poemll    at the fords of the Arnon River.\fnote{\fbackref{16:2} So 1QIsa\textsuperscript{a}; the Heb. lacks \fbib{River}} \\
\poeml \v{3}``Give us advice; \\
\poemll    reach a decision! \\
\poeml Cast your shadow as if night had come \\
\poemll    at high noon. \\
\poeml Shelter the fugitives, \\
\poemll    And don't betray a single refugee. \\
\poeml \v{4}Let the fugitives\fnote{\fbackref{16:4} So 1QIsa\textsuperscript{a}; MT reads \fbib{my fugitives}} from Moab \\
\poemll    settle among you; \\
\poeml be a shelter to them \\
\poemll    from the destroyer. \\
\poeml When the oppressor comes to an end, \\
\poemll    and destruction has\fnote{\fbackref{16:4} So 1QIsa\textsuperscript{a}; MT reads \fbib{have}} ceased, \\
\poemlll       and the marauder\fnote{\fbackref{16:4} Lit. \fbib{the one who tramples underfoot}} has\fnote{\fbackref{16:4} So 1QIsa\textsuperscript{a}; MT reads \fbib{have}} vanished from the land, \\
\poeml \v{5}then a throne will be established in gracious love, \\
\poemll    and there will sit in faithfulness--- \\
\poemlll       in the Tent of David--- \\
\poeml one who judges, seeks justice, \\
\poemll    and is swift to do what is right.'' \\
\poeml \v{6}``We've heard about Moab's pride--- \\
\poemll    so very proud he became!---\fnote{\fbackref{16:6} So 1QIsa\textsuperscript{a} MT\textsuperscript{mss}; cf. Jer 48:29; MT reads \fbib{how very proud he was}} \\
\poeml his arrogance, his pride, and his insolence; \\
\poemll    therefore he is alone.\fnote{\fbackref{16:6} So 1QIsa\textsuperscript{a}; MT LXX read \fbib{but his boasts mean nothing}} \\
\poeml \v{7}Therefore, let Moab not\fnote{\fbackref{16:7} So 1QIsa\textsuperscript{a}; the Heb. lacks \fbib{not}} wail, \\
\poemll    let everyone wail for Moab. \\
\poeml Lament and grieve deeply \\
\poemll    for the ruined remains\fnote{\fbackref{16:7} Or \fbib{for the raisin cakes}} of Kir-hareseth. \\
\poeml \v{8}For the fields of Heshbon wither, \\
\poemll    as well as the vines of Sibmah.\fnote{\fbackref{16:8} So 1QIsa\textsuperscript{a}; MT LXX include the rest of v. 8: \fbib{The rulers {\ldots} vine of Sibmah} in v. 9} \\
\poeml The rulers of the nations \\
\poeml have struck down its choicest vines, \\
\poeml which once reached Jazer \\
\poeml and pushed to the desert. \\
\poeml Its shoots spread out \\
\poeml and passed over the sea.''
\passage{Isaiah Weeps for Moab}
\poeml \v{9}``Therefore I weep with the tears of Jazer \\
\poemll    for the vines of Sibmah. \\
\poeml I drench you with my tears, \\
\poemll    O Heshbon and Elealeh--- \\
\poeml for the shouts of joy over your summer fruit \\
\poemll    and your grain harvest have ended. \\
\poeml \v{10}Joy and gladness are taken away from the orchards, \\
\poemll    in the vineyards people will sing no songs,\fnote{\fbackref{16:10} So 1QIsa\textsuperscript{a} LXX; MT reads \fbib{no songs are sung}} \\
\poeml and\fnote{\fbackref{16:10} So 1QIsa\textsuperscript{a} MT\textsuperscript{mss}; the Heb. lacks \fbib{and}} no cheers are raised. \\
\poemll    No vintner\fnote{\fbackref{16:10} Lit. \fbib{treader}} treads out wine in the presses, \\
\poemlll       because I've put an end to the shouting. \\
\poeml \v{11}Therefore my insides\fnote{\fbackref{16:11}Or \fbib{heart}; lit. \fbib{bowels}} moan like a lyre for Moab, \\
\poemll    and my innermost being\fnote{\fbackref{16:11} Or \fbib{my very soul}} for Kir-hareseth. \\
\poeml \v{12}When Moab appears, \\
\poemll    when he arrives\fnote{\fbackref{16:12} So 1QIsa\textsuperscript{a}; MT LXX read \fbib{tires himself}} upon the high place \\
\poeml and comes to his sanctuary to pray, \\
\poemll    he will not prevail.''
\end{poetry}

\v{13}This was the message that the \divine{Lord} spoke concerning Moab in the past. \v{14}But now the \divine{Lord} has spoken again: ``Within three years, like the years of a contract worker,\fnote{\fbackref{16:14} I.e. as if carefully counted pursuant to an employment contract; cf. Isa 21:16} Moab's glory will be brought into contempt, in spite of all its great multitude, and its survivors will be very few and\fnote{\fbackref{16:14} So 1QIsa\textsuperscript{a} LXX; the Heb. lacks \fbib{and}} of no importance.''
\labelchapt{17}
\passage{A Rebuke to Damascus}

\chapt{17}
\v{1}A message\fnote{\fbackref{17:1} Lit. \fbib{An oracle}} about Damascus:\fnote{\fbackref{17:1} So MT LXX; 1QIsa\textsuperscript{a} reads \fbib{Dramascus}}

\begin{poetry}
\poeml ``Look! Damascus\fnote{\fbackref{17:1} So MT LXX; 1QIsa\textsuperscript{a} reads \fbib{Dramascus}} will cease to be a city. \\
\poemll    Instead, it will become a pile of ruins. \\
\poeml \v{2}The cities of Oraru\fnote{\fbackref{17:2} So 1QIsa\textsuperscript{a}; MT reads \fbib{Aroer}, a pun on the Heb. word for \fbib{ruins}; LXX reads \fbib{forever}} will be deserted--- \\
\poemll    they will be devoted to herds that will lay at rest, \\
\poemlll       and terrorism will be no more.\fnote{\fbackref{17:2} Lit. \fbib{and no one will make them afraid}} \\
\poeml \v{3}The fortress will disappear from Ephraim, \\
\poemll    and royal authority from Damascus;\fnote{\fbackref{17:3} So MT LXX; 1QIsa\textsuperscript{a} reads \fbib{Dramascus}} \\
\poeml the survivors\fnote{\fbackref{17:3} I.e. believing Jews who return} from Aram\fnote{\fbackref{17:3} I.e. \fbib{Syria}} will be like the glory of the Israelis,'' \\
\poemll    declares the \divine{Lord} of the Heavenly Armies.
\passage{A Time of Weakness for Israel}
\poeml \v{4}``At that time,\fnote{\fbackref{17:4} Lit. \fbib{On that day}} Jacob's glory will have become weakened, \\
\poemll    and his strong\fnote{\fbackref{17:4} Lit. \fbib{fat}} flesh will turn gaunt; \\
\poeml \v{5}it will be as if harvesters gather standing grain, \\
\poemll    reaping the ears by hand,\fnote{\fbackref{17:5} Lit. \fbib{ears with his arm}} \\
\poeml or it will be as if grain is harvested \\
\poemll    in the valley of Rephaim.\fnote{\fbackref{17:5} Lit. \fbib{Giants}} \\
\poeml \v{6}Nevertheless, gleanings will remain in Israel,\fnote{\fbackref{17:6} Lit. \fbib{it}} \\
\poemll    as when an olive tree is beaten---\fnote{\fbackref{17:6} Or \fbib{harvested}} \\
\poeml two or three ripe olives left in the topmost branches, \\
\poemll    four or five left among the branches of a fruit-filled tree,''\fnote{\fbackref{17:6} So 1QIsa\textsuperscript{a}; MT reads \fbib{among its branches}} \\
\poemlll       declares the \divine{Lord} God of Israel.
\end{poetry}
\passage{Revival to Come to Israel}

\v{7}At that time, men will look upon\fnote{\fbackref{17:7} So 1QIsa\textsuperscript{a}; cf. LXX; MT reads \fbib{to}} their Maker, and their eyes will honor the Holy One of Israel. \v{8}They will not look upon\fnote{\fbackref{17:8} So 1QIsa\textsuperscript{a}; cf. LXX; MT reads \fbib{to}} the altars, the products\fnote{\fbackref{17:8} So 1QIsa\textsuperscript{a}; MT LXX read \fbib{the product}} that their own fingers\fnote{\fbackref{17:8} So 1QIsa\textsuperscript{a}; MT LXX read \fbib{hands}} have made, and they will have no regard for Asherah poles\fnote{\fbackref{17:8} I.e. images of the Babylonian-Canaanite goddess of fortune} or incense altars.\fnote{\fbackref{17:8} So 1QIsa\textsuperscript{a} MT; LXX reads \fbib{for trees or for their abominations}}
\passage{Desolation to the Nations}

\begin{poetry}
\poeml \v{9}``At that time,\fnote{\fbackref{17:9} Lit. \fbib{On that day,}} their fortified cities \\
\poemll    that they abandoned because of the Israelis \\
\poeml will be like desolate places\fnote{\fbackref{17:9} So 1QIsa\textsuperscript{a}; cf. LXX; MT reads \fbib{place}} of the forests and hilltops---\fnote{\fbackref{17:9} Or \fbib{the Hivites and Amorites}} \\
\poemll    there will be desolation. \\
\poeml \v{10}For you have forgotten the God of your salvation \\
\poemll    and have not remembered the Rock \\
\poemlll       that is your strength. \\
\poeml Therefore even though you plant delightful plants, \\
\poemll    sowing them with imported vine-seedlings, \\
\poeml \v{11}at the time that you plant them, \\
\poemll    carefully making them grow, \\
\poeml the very morning you make your seed to sprout, \\
\poemll    your harvest will be ruined\fnote{\fbackref{17:11} Lit. \fbib{become a pile}} \\
\poemlll       in a time of grief and unbearable pain.''\fnote{\fbackref{17:11} Lit. \fbib{and sorrow}} \\
\poeml \v{12}``How terrible it will be for many peoples, \\
\poemll    who rage like the roaring sea! \\
\poeml Oh, how the uproar of nations \\
\poemll    is like the sound of rushing, mighty water--- \\
\poemll    How they roar! \\
\poeml \v{13}The nations roar like the rushing of many waters,\fnote{\fbackref{17:13} So 1QIsa\textsuperscript{a} MT; cf. LXX; MT\textsuperscript{ms} Syr lack this line} \\
\poemll    but the \divine{Lord}\fnote{\fbackref{17:13} Lit. \fbib{but he}} will rebuke them, \\
\poeml and they will run far away, \\
\poemll    chased like chaff blown down from the mountains \\
\poeml or like thick dust\fnote{\fbackref{17:13} Lit. \fbib{like something}} that rolls along, \\
\poemll    blown along by a wind storm. \\
\poeml \v{14}When the evening arrives, watch out---sudden terror! \\
\poemll    By morning they will be there no longer! \\
\poeml So it will be for those who plunder us \\
\poemll    and what will happen to those who rob us.''
\end{poetry}
\labelchapt{18}
\passage{A Rebuke to Cush}

\begin{poetry}
\poeml \chapt{18}
\v{1}Woe to the land of whirring wings \\
\poemll    that is beyond the rivers of Cush,\fnote{\fbackref{18:1} I.e. Nubia, south of Egypt (modern northern Sudan)} \\
\poeml \v{2}which sends envoys by the sea,\fnote{\fbackref{18:2} Or \fbib{Nile}} \\
\poemll    in papyrus boats over the water! \\
\poeml Go, swift messengers, \\
\poemll    to a tall, smooth-skinned nation, \\
\poeml to a people feared far and wide, \\
\poemll    a nation that metes out\fnote{\fbackref{18:2} Or \fbib{nation of strange speech}; so 1QIsa\textsuperscript{a} MT; LXX reads \fbib{nation without hope}} punishment\fnote{\fbackref{18:2} 1QIsa\textsuperscript{a} MT LXX lack \fbib{punishment}} and oppresses, \\
\poemlll       whose land the rivers divide. \\
\poeml \v{3}All you inhabitants of the world, \\
\poemll    you who live on the earth, \\
\poeml when a banner is raised on the mountains, \\
\poemll    you'll see it. \\
\poeml When a trumpet sounds, \\
\poemll    you'll hear it! \\
\poeml \v{4}For this is what the \divine{Lord} told me: \\
\poeml ``I will remain quiet and watch in my dwelling place \\
\poemll    like dazzling heat in sunshine, \\
\poemlll       like a cloud of dew in the heat\fnote{\fbackref{18:4} So 1QIsa\textsuperscript{a} MT; MT\textsuperscript{mss} LXX read \fbib{on the day}} of harvest.'' \\
\poeml \v{5}For before the harvest, when the season of\fnote{\fbackref{18:5} 1QIsa\textsuperscript{a} MT lack \fbib{season of}} budding is over, \\
\poemll    and sour grapes ripen into mature grapes,\fnote{\fbackref{18:5} Lit. \fbib{flowers}} \\
\poeml he cuts off the shoots with pruning knives, \\
\poemll    clearing away the spreading branches \\
\poemlll       as he lops them off. \\
\poeml \v{6}And\fnote{\fbackref{18:6} So 1QIsa\textsuperscript{a} MT LXX; 4QIsa\textsuperscript{b} lacks \fbib{And}} they will all be left \\
\poemll    for birds of prey that live on the mountains\fnote{\fbackref{18:6} So 4QIsa\textsuperscript{b}; 1QIsa\textsuperscript{a} MT read \fbib{of mountains}; LXX reads \fbib{mountains of heaven}} \\
\poemlll       and for wild animals.\fnote{\fbackref{18:6} Lit. \fbib{for beasts of the earth}; i.e. non-domesticated animals, as opposed to domesticated \fbib{livestock;}} \\
\poeml Birds of prey will pass the summer feeding on them, \\
\poemll    and all the wild animals\fnote{\fbackref{18:6} Lit. \fbib{the beasts of the earth}; so 1QIsa\textsuperscript{a}; i.e. non-domesticated animals, as opposed to domesticated \fbib{livestock;} MT LXX read \fbib{every beast of the field}} will pass the winter feeding\fnote{\fbackref{18:6} 1QIsa\textsuperscript{a} MT lack \fbib{feeding}} on them. \\
\poeml \v{7}At that time tribute will be brought to the \divine{Lord} of the Heavenly Armies \\
\poemll    from\fnote{\fbackref{18:7} So 1QIsa\textsuperscript{a} LXX; 4QIsa\textsuperscript{b} MT lack \fbib{from}} a tall and smooth-skinned people, \\
\poeml from a people feared far and wide, \\
\poemll    a nation that metes out\fnote{\fbackref{18:7} Or \fbib{nation of strange speech}; so 1QIsa\textsuperscript{a} MT; LXX reads \fbib{nation with hope}} punishment\fnote{\fbackref{18:7} 1QIsa\textsuperscript{a} MT LXX lack \fbib{punishment}} and oppresses, \\
\poeml whose land the rivers divide, \\
\poemll    to Mount Zion, \\
\poemlll       the place that bears\fnote{\fbackref{18:7} Lit. \fbib{place of}} the name of the \divine{Lord}.\fnote{\fbackref{18:7} So 1QIsa\textsuperscript{a}; 4QIsa\textsuperscript{b} MT LXX read \fbib{the \divine{Lord} of the Heavenly Armies}}
\end{poetry}
\labelchapt{19}
\passage{A Rebuke to Egypt}

\chapt{19}
\v{1}A message\fnote{\fbackref{19:1} Lit. \fbib{An oracle}} about Egypt:

\begin{poetry}
\poeml ``Watch out! The \divine{Lord} rides on a swift cloud, \\
\poemll    and is coming to Egypt. \\
\poeml The idols of Egypt tremble before him, \\
\poemll    and the hearts of the Egyptians melt within them. \\
\poeml \v{2}I will stir up Egyptians against Egyptians, \\
\poemll    and everyone will fight against his brother, \\
\poeml everyone against his neighbor, \\
\poemll    city\fnote{\fbackref{19:2} So 1QIsa\textsuperscript{a}; MT LXX lack \fbib{and}} against city, \\
\poemlll       kingdom against kingdom. \\
\poeml \v{3}The spirits of the Egyptians within them will be drained of courage,\fnote{\fbackref{19:3} 1QIsa\textsuperscript{a} MT lack \fbib{of courage}} \\
\poemll    and I will bring their plans to nothing. \\
\poeml They will consult idols\fnote{\fbackref{19:3} So 1QIsa\textsuperscript{a}; MT reads \fbib{consult the idols}; LXX reads \fbib{consult their idols}} and spirits of the dead, \\
\poemll    and mediums and spiritists. \\
\poeml \v{4}I will hand the Egyptians over \\
\poemll    to the power of a cruel master, \\
\poemlll       and a fierce king will rule over them,'' \\
\poeml declares the Lord \divine{God} of the Heavenly Armies.
\passage{A Rebuke to Egypt's Ecology and Industry}
\poeml \v{5}``The water sources\fnote{\fbackref{19:5} 1QIsa\textsuperscript{a} MT lack \fbib{sources}} of the Nile\fnote{\fbackref{19:5} Or \fbib{the sea}} will be dried up, \\
\poemll    and the river\fnote{\fbackref{19:5} I.e. the Nile} will become dry and parched. \\
\poeml \v{6}The canals will stink, \\
\poemll    and\fnote{\fbackref{19:6} So 1QIsa\textsuperscript{a} LXX; the Heb. lacks \fbib{and}} the tributaries of Egypt will dwindle and dry up. \\
\poemlll       Reeds and rushes will wither away.\fnote{\fbackref{19:6} So 4QIsa\textsuperscript{b} MT; 1QIsa\textsuperscript{a} lacks this line} \\
\poeml \v{7}And the bulrushes along the Nile,\fnote{\fbackref{19:7} So 1QIsa\textsuperscript{a} MT; LXX lacks this line} \\
\poemll    along the mouth of the Nile,\fnote{\fbackref{19:7} Lit. \fbib{the River}} will wither away. \\
\poeml All the sown fields of the Nile will become parched, \\
\poemll    and\fnote{\fbackref{19:7} So 1QIsa\textsuperscript{a}; the Heb. lacks \fbib{and}} they will be blown away; \\
\poemlll       there will be nothing left.\fnote{\fbackref{19:7} So 1QIsa\textsuperscript{a}; 4QIsa\textsuperscript{b} MT read \fbib{and will be no more}; LXX lacks this line} \\
\poeml \v{8}The fishermen will groan, \\
\poemll    and all who cast hooks into the Nile will lament; \\
\poeml those who spread nets upon the water \\
\poemll    will become weaker and weaker. \\
\poeml \v{9}The workers\fnote{\fbackref{19:9} So 1QIsa\textsuperscript{a} 4QIsa\textsuperscript{b}; MT LXX read \fbib{And the workers}} in combed flax \\
\poemll    and the weavers of white linen \\
\poemlll       will be in despair. \\
\poeml \v{10}Egypt's\fnote{\fbackref{19:10} Lit. \fbib{Its}} workers in cloth\fnote{\fbackref{19:10} Lit. \fbib{Its weavers}} will be crushed, \\
\poemll    and all who work for wages will be sick at heart.''
\passage{A Rebuke to Egypt's Leaders}
\poeml \v{11}Zoan's princes are nothing but fools; \\
\poemll    the wisest advisors of Pharaoh give stupid advice. \\
\poeml How can you say to Pharaoh, \\
\poemll    ``I'm a descendant of wise men, \\
\poemlll       a descendant of ancient kings''? \\
\poeml \v{12}Where are your wise men now? \\
\poemll    Let them tell you, \\
\poeml let them make known \\
\poemll    what the \divine{Lord}\fnote{\fbackref{19:12} So 1QIsa\textsuperscript{a}; 1QIsa\textsuperscript{a} corrector MT LXX read \fbib{\divine{Lord} of the Heavenly Armies}} has planned against Egypt. \\
\poeml \v{13}The princes of Zoan have become fools, \\
\poemll    and the princes of Memphis deluded; \\
\poeml the leaders\fnote{\fbackref{19:13} Or \fbib{cornerstones}} of its tribes \\
\poemll    have led Egypt astray. \\
\poeml \v{14}The \divine{Lord} has mixed\fnote{\fbackref{19:14} So 1QIsa\textsuperscript{a} MT LXX; 4QIsa\textsuperscript{b} reads \fbib{has poured}} within them\fnote{\fbackref{19:14} Lit. \fbib{it}} a spirit of confusion; \\
\poemll    so they make Egypt stagger in all that it does, \\
\poemlll       like a drunkard staggers around in his vomit. \\
\poeml \v{15}As a result, there will be nothing for Egypt \\
\poemll    that head or tail, palm branch or reed, can do.\fnote{\fbackref{19:15} So 1QIsa\textsuperscript{a} MT LXX; 4QIsa\textsuperscript{b} reads \fbib{do at that time}}
\end{poetry}
\passage{Egypt and Syria Will Worship God}

\v{16}At that time,\fnote{\fbackref{19:16} Lit. \fbib{On that day}; so 1QIsa\textsuperscript{a} MT LXX; 4QIsa\textsuperscript{b} lacks \fbib{At that time}} the Egyptians will be like women---they\fnote{\fbackref{19:16} So 1QIsa\textsuperscript{a}; 4QIsa\textsuperscript{b} MT read \fbib{he}} will shudder and be afraid before the uplifted hand of the \divine{Lord} of the Heavenly Armies, when he brandishes his hand against her.\fnote{\fbackref{19:16} I.e. Egypt; so 1QIsa\textsuperscript{a}; MT reads \fbib{Armies, which he brandishes against it}; LXX reads \fbib{Armies, which he brandishes against them}} \v{17}And the land of Judah will become a terror to the Egyptians. Everyone to whom it is mentioned will be afraid, because of the uplifted hand\fnote{\fbackref{19:17} Lit. \fbib{plotting}; so 4QIsa\textsuperscript{b}; 1QIsa\textsuperscript{a} MT LXX read \fbib{plan}} of the \divine{Lord} of the Heavenly Armies that\fnote{\fbackref{19:17} So 4QIsa\textsuperscript{b}; 1QIsa\textsuperscript{a} MT LXX read \fbib{hand that the \divine{Lord} of the Heavenly Armies}} is turning in their direction.

\v{18}At that time,\fnote{\fbackref{19:18} Lit. \fbib{On that day}} there will be five cities in the land of Egypt that speak the language of Canaan and swear allegiance to the \divine{Lord} of the Heavenly Armies. One of them will be called the City of the Sun.\fnote{\fbackref{19:18} So 1QIsa\textsuperscript{a} 4QIsa\textsuperscript{b} MT\textsuperscript{mss}; MT reads \fbib{Destruction}; LXX reads \fbib{Asedek City}; i.e. \fbib{City of Righteousness}}

\v{19}At that time,\fnote{\fbackref{19:19} Lit. \fbib{On that day}} there will be an altar to the \divine{Lord} of the Heavenly Armies\fnote{\fbackref{19:19} So 4QIsa\textsuperscript{b}; 1QIsa\textsuperscript{a} MT LXX lack \fbib{of the Heavenly Armies}} in the heart of the land of Egypt, and a monument to the \divine{Lord}\fnote{\fbackref{19:19} So 1QIsa\textsuperscript{a} MT LXX; 4QIsa\textsuperscript{b} reads \fbib{\divine{Lord} of hosts}} at its border. \v{20}It will be a sign and a witness to the \divine{Lord} of the Heavenly Armies in the land of Egypt; when they cry out to the \divine{Lord} because of their oppressors, he will send them a savior, and he will come down\fnote{\fbackref{19:20} So 1QIsa\textsuperscript{a}; MT LXX read \fbib{will defend}} and rescue them. \v{21}So the \divine{Lord} will make himself known to the Egyptians, and the Egyptians will acknowledge the \divine{Lord}.

At that time,\fnote{\fbackref{19:21} Lit. \fbib{On that day}} they will worship\fnote{\fbackref{19:21} So 1QIsa\textsuperscript{a}; MT LXX read \fbib{\divine{Lord} on that day. And they will worship}} with sacrifices and offerings, and they will make vows to the \divine{Lord} and carry them out. \v{22}The \divine{Lord} will strike Egypt with a plague, striking but then healing. Then they will turn to the \divine{Lord}, and he will respond to their pleas and heal them.

\v{23}At that time,\fnote{\fbackref{19:23} Lit. \fbib{On that day}} there will be a highway from Egypt to Assyria. The Assyrians will come into Egypt, and the Egyptians into Assyria, and they\fnote{\fbackref{19:23} So 1QIsa\textsuperscript{a}; 4QIsa\textsuperscript{b} MT LXX read \fbib{and the Egyptians}} will worship with the Assyrians.

\v{24}At that time,\fnote{\fbackref{19:24} Lit. \fbib{On that day}} Israel will be in a triple alliance\fnote{\fbackref{19:24} Lit. \fbib{will be the third}} with Egypt and Assyria; they will be\fnote{\fbackref{19:24} DSS MT lack \fbib{they will be}} a blessing in the midst of the earth. \v{25}The \divine{Lord} of the Heavenly Armies has blessed them, saying, ``Blessed be Egypt my people, Assyria the work of my hands, and Israel my inheritance.''
\labelchapt{20}
\passage{The Conquest of Egypt and Cush}

\chapt{20}
\v{1}In the year that the supreme commander, sent by Sargon the king of Assyria, came to Ashdod, attacked it, and captured it--- \v{2}at that time the \divine{Lord} spoke through Amoz's son Isaiah: ``Go loosen the sackcloth that's around your waist,\fnote{\fbackref{20:2} Lit. \fbib{your hips and lower back}} and take your sandals\fnote{\fbackref{20:2} So 1QIsa\textsuperscript{a} LXX; MT reads \fbib{sandal}} off your feet.'' So that's what he did: he went around naked and barefoot.

\v{3}Then the \divine{Lord} said, ``Just as my servant Isaiah has walked around naked and barefoot for three years as a sign and a warning for Egypt and Ethiopia,\fnote{\fbackref{20:3} I.e. Nubia, south of Egypt (modern northern Sudan)} \v{4}so the king of Assyria will lead away the Egyptian captives and exiles from Cush,\fnote{\fbackref{20:4} I.e. Nubia, south of Egypt (modern northern Sudan)} both the young and the old, naked and barefoot---with even their buttocks uncovered---to the shame\fnote{\fbackref{20:4} Or \fbib{nakedness}} of Egypt. \v{5}Then they will be dismayed and put to shame because of Cush,\fnote{\fbackref{20:5} I.e. Nubia, south of Egypt (modern northern Sudan)} their hope, and Egypt, their jewel.\fnote{\fbackref{20:5} Or \fbib{pride}} \v{6}At that time, the inhabitants of this coastland will say, `See, this is what has happened to those on whom we counted and relied\fnote{\fbackref{20:6} So 1QIsa\textsuperscript{a}; MT LXX read \fbib{and to whom we fled}} for help and deliverance from the king of Assyria! How, then, can we escape?'\,''
\labelchapt{21}
\passage{Elam and Media are Rebuked}

\chapt{21}
\v{1}A message\fnote{\fbackref{21:1} Lit. \fbib{An oracle}} concerning the pasture\fnote{\fbackref{21:1} Or \fbib{plague}; cf. Isa 5:17; 1King 8:37; Jer 14:12; MT LXX read \fbib{wilderness}} by the Sea.

\begin{poetry}
\poeml ``Like whirlwinds in the Negev\fnote{\fbackref{21:1} I.e. southern regions of the Sinai peninsula; cf. Josh 10:40} sweep on, \\
\poemll    it comes from the desert, \\
\poemlll       from a distant\fnote{\fbackref{21:1} So 1QIsa\textsuperscript{a}; 1QIsa\textsuperscript{a} corrector MT LXX read \fbib{terrible}} land. \\
\poeml \v{2}A dire vision has been announced to me: \\
\poemll    the traitor betrays, \\
\poemlll       and the plunderer takes loot. \\
\poeml Get up, Elam! \\
\poemll    Attack, Media! \\
\poeml I am putting a stop \\
\poemll    to all the groaning she has caused. \\
\poeml \v{3}Therefore my body is\fnote{\fbackref{21:3} Or \fbib{waist}; lit. \fbib{my hips and lower back are}} racked with pain; \\
\poemll    pangs have seized me, \\
\poemlll       like the pangs of a woman in labor; \\
\poeml I am so upset that I cannot hear; \\
\poemll    I am so frightened that I cannot see \\
\poemll    while I'm reeling around.\fnote{\fbackref{21:3} So 1QIsa\textsuperscript{a} 4QIsa\textsuperscript{a}; MT LXX begin v. 4 with this line} \\
\poeml \v{4}And as for my heart,\fnote{\fbackref{21:4} So 1QIsa\textsuperscript{a} 4QIsa\textsuperscript{a}; MT LXX read \fbib{\v{4}My mind reels}} horror has terrified me; \\
\poemll    the twilight I longed for \\
\poemlll       has started to make me tremble. \\
\poeml \v{5}They set the tables; \\
\poemll    they spread the carpets;\fnote{\fbackref{21:5} So 1QIsa\textsuperscript{a} MT; LXX lacks this line} \\
\poemlll       they eat, they drink! \\
\poeml Get up, you officers! \\
\poemll    Oil the shields!''
\passage{The Fall of Babylon}
\poeml \v{6}For this is what the \divine{Lord} told me: \\
\poeml ``Go post a lookout. \\
\poemlll       Have him report what he sees. \\
\poeml \v{7}When he sees chariots, each man\fnote{\fbackref{21:7} So 1QIsa\textsuperscript{a} 4QIsa\textsuperscript{a}; cf. v. 9; MT LXX lack \fbib{each man}} with a pair of horses, \\
\poemll    riders on donkeys or riders on\fnote{\fbackref{21:7} So 1QIsa\textsuperscript{a} LXX; MT reads \fbib{train of donkeys or train of}} camels, \\
\poeml let him pay attention, \\
\poemll    full attention.'' \\
\poeml \v{8}Then the lookout\fnote{\fbackref{21:8} So 1QIsa\textsuperscript{a} Syr; MT reads \fbib{Then a lion}} shouted: \\
\poemll    ``Upon a watchtower I stand, O Lord, \\
\poemlll       continually by day, \\
\poeml and I am stationed at my post \\
\poemll    throughout the night. \\
\poeml \v{9}Look! Here come riders,\fnote{\fbackref{21:9} So 1QIsa\textsuperscript{a} LXX; MT reads \fbib{chariot}; cf. v. 7} \\
\poemll    each man with a pair of horses!'' \\
\poemlll       They're shouting out the answer: \\
\poeml ``Babylon has fallen, has fallen, \\
\poemll    and they have shattered \\
\poemlll       all the images of her gods\fnote{\fbackref{21:9} So 1QIsa\textsuperscript{a}; LXX reads \fbib{all the images of her gods are shattered}; MT reads \fbib{He has shattered all the images of her gods}} on the ground! \\
\poeml \v{10}O my downtrodden people,\fnote{\fbackref{21:10} 1QIsa\textsuperscript{a} lacks \fbib{people}} my wall!\fnote{\fbackref{21:10} So 1QIsa\textsuperscript{a}; MT reads \fbib{my threshing floor}} \\
\poemll    I'll tell you what I have heard \\
\poemlll       from the \divine{Lord} of the Heavenly Armies, the God of Israel.''
\passage{A Message about Dumah}
\poeml \v{11}A message\fnote{\fbackref{21:11} Lit. \fbib{An oracle}} concerning Dumah. \\
\poeml ``Someone is calling to me from Seir: \\
\poemll    `Watchman, what is left of the night?\fnote{\fbackref{21:11} Or \fbib{What time of night?}} \\
\poemlll       Watchman, what is left of the night?'\fnote{\fbackref{21:11} Or \fbib{What time of night?}} \\
\poeml \v{12}The watchman replies: \\
\poemll    `Morning is coming, but also the night. \\
\poeml If you want to ask, then ask; \\
\poemll    come back again.'\,''
\passage{A Message about Arabia}
\poeml \v{13}A message\fnote{\fbackref{21:13} Lit. \fbib{An oracle}} concerning Arabia. \\
\poeml ``You will camp in the thickets in Arabia, \\
\poemll    you caravans of the Dedanites. \\
\poeml \v{14}Bring water for the thirsty, \\
\poemll    you who live in the land of Tema. \\
\poemlll       Meet the fugitive with bread,\fnote{\fbackref{21:14} So 1QIsa\textsuperscript{a} LXX; MT reads \fbib{with his bread}; 4QIsa\textsuperscript{a} reads \fbib{and with his bread}} \\
\poeml \v{15}For he has fled\fnote{\fbackref{21:15} So 1QIsa\textsuperscript{a}; MT reads \fbib{they have fled}} from swords, \\
\poemll    from the drawn sword, \\
\poeml from the bent bow, \\
\poemll    and from the heat of battle.''
\end{poetry}

\v{16}For this is what the \divine{Lord}\fnote{\fbackref{21:16} So 1QIsa\textsuperscript{a} 4QIsa\textsuperscript{a}; MT reads \fbib{Lord}} is saying to me: ``Within three years,\fnote{\fbackref{21:16} So 1QIsa\textsuperscript{a}; MT LXX read \fbib{Within a year}} according to the years of a contract worker,\fnote{\fbackref{21:16} I.e. as if carefully counted pursuant to an employment contract; cf. Isa 16:14} the pomp\fnote{\fbackref{21:16} So 1QIsa\textsuperscript{a} LXX; MT reads \fbib{years, all the pomp}} of Kedar will come to an end. \v{17}And there will be few archers, those who are descendants of Kedar, who survive, because the \divine{Lord}, the God of Israel, has spoken.''
\labelchapt{22}
\passage{Jerusalem is Rebuked}

\chapt{22}
\v{1}A message\fnote{\fbackref{22:1} Lit. \fbib{An oracle}} concerning the Valley of Vision.\fnote{\fbackref{22:1} I.e. a poetic allusion to the Hinnom Valley in Jerusalem}

\begin{poetry}
\poeml ``What troubles you, \\
\poemll    now that you've all gone up \\
\poemlll       to the rooftops, \\
\poeml \v{2}you who are full of commotion, \\
\poemll    you passionate city, \\
\poemlll       you rollicking town? \\
\poeml Your slain weren't killed by the sword, \\
\poemll    nor are they dead in battle. \\
\poeml \v{3}All your leaders have fled together; \\
\poemll    she is captured\fnote{\fbackref{22:3} So 1QIsa\textsuperscript{a}; MT reads \fbib{they were captured}} without using bows. \\
\poeml All of you who were caught were captured together, \\
\poeml although they had fled \\
\poemlll       while the enemy was still\fnote{\fbackref{22:3} The Heb. lacks \fbib{while the enemy was still}} far away. \\
\poeml \v{4}Therefore I said: \\
\poemll    ``Look away from me; \\
\poemlll       and let me weep bitter tears; \\
\poeml don't try to console\fnote{\fbackref{22:4} Lit. \fbib{don't insist on consoling}} me \\
\poemll    over the destruction of the daughter of my people.''\fnote{\fbackref{22:4} I.e. \fbib{the \divine{Lord}'s beloved people}} \\
\poeml \v{5}For to the Lord \divine{God} of the Heavenly Armies \\
\poemll    belongs the day of tumult, trampling, and confusion \\
\poemlll       in the Valley of Vision,\fnote{\fbackref{22:5} I.e. a poetic allusion to the Hinnom Valley in Jerusalem} \\
\poeml and the pulling down of his Temple on\fnote{\fbackref{22:5} Or \fbib{his Holy Place on}; so 1QIsa\textsuperscript{a}; MT reads \fbib{and a crying out for help to}} its mountain. \\
\poeml \v{6}Elam takes up the quiver \\
\poemll    with chariots and cavalry, \\
\poemlll       while Kir unsheathes the shield. \\
\poeml \v{7}And it will come about\fnote{\fbackref{22:7} So 1QIsa\textsuperscript{a}; MT reads \fbib{it came about}} \\
\poemll    that your choicest valleys will be filled with chariots, \\
\poemlll       and horsemen will take their positions at the gates. \\
\poeml \v{8}He has uncovered the defenses of Judah.''
\end{poetry}

At that time,\fnote{\fbackref{22:8} Lit. \fbib{On that day}} you looked at the arsenal of the Palace of the Forest,\fnote{\fbackref{22:8} Cf. 1King 10:16-17} \v{9}and saw that there were many breaches in the City of David. So you stored up water from the Lower Pool, \v{10}counted the houses of Jerusalem, tore down certain houses to strengthen the wall, \v{11}and built a reservoir between the walls to store water from the Old Pool. But you did not look at\fnote{\fbackref{22:11} So 1QIsa\textsuperscript{a}; MT LXX 4QIsa\textsuperscript{c} read \fbib{to}} the One who did it, nor did you see the One who planned it long ago.

\begin{poetry}
\poeml \v{12}On that day the Lord \divine{God} of the Heavenly Armies \\
\poemll    called for weeping and mourning, \\
\poemlll       for shaving heads\fnote{\fbackref{22:12} Lit. \fbib{for baldness}} and wearing sackcloth. \\
\poeml \v{13}But look! \\
\poemll    There is joy and festivity, \\
\poeml slaughtering of cattle \\
\poemll    and killing of sheep, \\
\poeml eating meat \\
\poemll    and drinking\fnote{\fbackref{22:13} So 1QIsa\textsuperscript{a} MT; 1QIsa\textsuperscript{c} reads \fbib{and they drink}} wine. \\
\poeml ``Let us eat and drink, you say, \\
\poemll    because we die tomorrow.'' \\
\poeml \v{14}``Nevertheless, the \divine{Lord} of the Heavenly Armies has revealed himself to my hearing: \\
\poeml ```Surely because of you\fnote{\fbackref{22:14} So 1QIsa\textsuperscript{a}; 4QIsa\textsuperscript{c} MT LXX lack \fbib{because of you}} \\
\poemll    this iniquity will not be forgiven \\
\poemlll       you until you die,' \\
\poeml says the Lord \divine{God} of the Heavenly Armies.''
\end{poetry}
\passage{The \divine{Lord} Rebukes Shebna}

\v{15}This is what the Lord \divine{God} of the Heavenly Armies\fnote{\fbackref{22:15} So 1QIsa\textsuperscript{a} MT; MT\textsuperscript{mss} LXX read \fbib{\divine{Lord} of the Heavenly Armies}} says:

``Come, go to this steward, to Shebna who is in charge of the household, and ask him: \v{16}`What are you doing here, and who are your relatives here\fnote{\fbackref{22:16} Lit. \fbib{whom do you have here}} that you could carve out a grave for yourself here---cutting out a tomb at the choicest location,\fnote{\fbackref{22:16} Lit. \fbib{at the height}} chiseling out a resting place for yourself out of solid rock? \v{17}Look Out! The \divine{Lord} is about to hurl you away violently, my strong fellow! He\fnote{\fbackref{22:17} So 1QIsa\textsuperscript{a}; 4QIsa\textsuperscript{a} 4QIsa\textsuperscript{b} MT LXX read \fbib{And he}} will fold you up completely, \v{18}rolling you up tightly like a ball and throwing you into a large country. There\fnote{\fbackref{22:18} So 4QIsa\textsuperscript{f}; 1QIsa\textsuperscript{a} 1QIsa\textsuperscript{b} 4QIsa\textsuperscript{a} MT read \fbib{To there}; cf. LXX} you will die, and there\fnote{\fbackref{22:18} So 4QIsa\textsuperscript{f}; 1QIsa\textsuperscript{a} 1QIsa\textsuperscript{b} 4QIsa\textsuperscript{a} MT read \fbib{and to there}; cf. LXX} your splendid chariots will lie. You're a disgrace to your master's house! \v{19}I will depose you from your office, ousting you\fnote{\fbackref{22:19} Lit. \fbib{he has ousted you}; so 1QIsa\textsuperscript{a}; 4QIsa\textsuperscript{f} MT LXX\textsuperscript{ms} read \fbib{he will oust you}; LXX lacks \fbib{he has ousted you}} from your position.

\v{20}``At that time,\fnote{\fbackref{22:20} Lit. \fbib{On that day}} I'll call for my servant, Hilkiah's\fnote{\fbackref{22:20} Lit. \fbib{Hilkyah}; so 1QIsa\textsuperscript{a} 4QIsa\textsuperscript{f}} son Eliakim, \v{21}and I'll clothe him with your robe and fasten your sash around him. I'll transfer your authority to him,\fnote{\fbackref{22:21} Lit. \fbib{to his hand}} and he'll be a father to those who live in Jerusalem and to the house of Judah.

\v{22}``I'll place on his shoulder the key to the house of David---what he opens, no one will shut, and what he shuts, no one will open. \v{23}I'll set him like a peg into a secure place; he will become a throne of honor to his father's house. \v{24}The entire reputation of his father's house will hang on him: its offspring and offshoots---all its smaller vessels, from the cups to all the jars. \v{25}At that time,''\fnote{\fbackref{22:25} Lit. \fbib{On that day}} declares the \divine{Lord}\fnote{\fbackref{22:25} So 1QIsa\textsuperscript{a} 4QIsa\textsuperscript{a} MT LXX; 4QIsa\textsuperscript{f} reads \fbib{the Lord \divine{God}}} of the Heavenly Armies, ``the peg that was driven into a secure place will give way; it will be sheared and will fall, and the load hanging on it will be cut down.''

The \divine{Lord} has spoken.
\labelchapt{23}
\passage{Tyre is Rebuked}

\chapt{23}
\v{1}A message\fnote{\fbackref{23:1} Lit. \fbib{An oracle}} concerning Tyre.

\begin{poetry}
\poeml ``Wail, you ships of Tarshish, \\
\poeml for Tyre is destroyed \\
\poemlll       and is without house or harbor! \\
\poeml From the land of Cyprus\fnote{\fbackref{23:1} Lit. \fbib{of the Kittim}} \\
\poeml it was revealed to them. \\
\poeml \v{2}``Be silent,\fnote{\fbackref{23:2} So 1QIsa\textsuperscript{a} 4QIsa\textsuperscript{a} MT; LXX reads \fbib{To whom are they like?}} you inhabitants of the coast, \\
\poemll    you merchants of Sidon, \\
\poemlll       whose messengers crossed over the sea,\fnote{\fbackref{23:2} So 1QIsa\textsuperscript{a} 4QIsa\textsuperscript{a} LXX; MT reads \fbib{you whom the merchants of Sidon, passing over the sea}, \fbib{have replenished}} \\
\poeml \v{3}and were on mighty waters. \\
\poemll    Her revenue was the grain of Shihor, \\
\poeml the harvest of the Nile; \\
\poemll    and she became the marketplace of nations. \\
\poeml \v{4}Be ashamed, Sidon, because the sea\fnote{\fbackref{23:4} So 1QIsa\textsuperscript{a}; 4QIsa\textsuperscript{a} MT read \fbib{for he}} has spoken, \\
\poemll    the fortress of the sea: \\
\poeml I have neither been in labor nor given birth, \\
\poemll    I have neither reared young men \\
\poemlll       nor brought up young women.'' \\
\poeml \v{5}When the news reaches Egypt, \\
\poemll    they will be in anguish \\
\poemlll       at the report about Tyre. \\
\poeml \v{6}``You who are crossing over\fnote{\fbackref{23:6} So 1QIsa\textsuperscript{a}; MT LXX read \fbib{Cross over}} to Tarshish--- \\
\poemll    Wail, you inhabitants of the coast! \\
\poeml \v{7}Is this your exciting\fnote{\fbackref{23:7} Or \fbib{happy}} city, \\
\poemll    that was founded long ago, \\
\poeml whose feet carried her \\
\poemll    to settle in far-off lands? \\
\poeml \v{8}Who has planned this \\
\poemll    against Tyre, \\
\poemlll       that bestower of crowns, \\
\poeml whose merchants were princes,\fnote{\fbackref{23:8} So 1QIsa\textsuperscript{a} corrector} \\
\poemll    whose traders were the most renowned on earth? \\
\poeml \v{9}The \divine{Lord} of the Heavenly Armies has planned it--- \\
\poemll    to neutralize all the hubris of grandeur,\fnote{\fbackref{23:9} So 1QIsa\textsuperscript{a} LXX; MT reads \fbib{the hubris of all grandeur}} \\
\poemlll       to discredit all the renowned men of earth. \\
\poeml \v{10}``Cultivate\fnote{\fbackref{23:10} Or \fbib{Worship}; so 1QIsa\textsuperscript{a} LXX; 4QIsa\textsuperscript{c} MT read \fbib{Pass through}} your land like the Nile, \\
\poemll    you daughter of Tarshish; \\
\poemlll       for there is no longer a harbor. \\
\poeml \v{11}He has stretched out his hand over the sea; \\
\poemll    he has made\fnote{\fbackref{23:11} So 1QIsa\textsuperscript{a} 4QIsa\textsuperscript{a} MT LXX; 4QIsa\textsuperscript{c} reads \fbib{to make}} kingdoms tremble. \\
\poeml The \divine{Lord} has issued orders concerning Canaan \\
\poemll    to destroy its strongholds. \\
\poeml \v{12}And he said: \\
\poeml `You will revel no longer,\fnote{\fbackref{23:12} So 1QIsa\textsuperscript{a} MT LXX; 4QIsa\textsuperscript{c} reads \fbib{won't take refuge to revel}; or \fbib{won't revel with gusto}} \\
\poemll    you virgin daughter of Sidon, \\
\poemlll       now crushed. \\
\poeml Get up, cross over to Cyprus--- \\
\poemll    but even there you will find no rest.'\,''
\end{poetry}
\passage{Tyre's Desolation and Restoration}

\begin{poetry}
\poeml \v{13}``Look at the land of the Chaldeans! \\
\poemll    This is a people that no longer exist; \\
\poeml Assyria destined her\fnote{\fbackref{23:13} I.e. Tyre} for desert creatures.\fnote{\fbackref{23:13} Or \fbib{demons}} \\
\poemll    They raised up her\fnote{\fbackref{23:13} So 1QIsa\textsuperscript{a}; MT reads \fbib{his}} siege towers, \\
\poeml they stripped her fortresses bare \\
\poemll    and turned her into a ruin. \\
\poeml \v{14}Wail, you ships of Tarshish, \\
\poemll    because your\fnote{\fbackref{23:14} So 1QIsa\textsuperscript{a} (sing.); MT LXX (pl.)} stronghold is destroyed!''
\end{poetry}

\v{15}It will happen at that time that Tyre will be forgotten for 70 years, the span of a king's life. Then, at the end of those 70 years, it will turn out for Tyre as in the prostitute's song:\fnote{\fbackref{23:15} So 1QIsa\textsuperscript{c} MT LXX; 1QIsa\textsuperscript{a} lacks \fbib{that Tyre will be forgotten for 70 years, the span of a king's life. Then, at the end of those 70 years}}

\begin{poetry}
\poeml \v{16}``Take a harp; \\
\poemll    walk around the city, \\
\poemlll       you forgotten whore! \\
\poeml Make sweet melody; \\
\poemll    sing many songs, \\
\poemlll       and perhaps you'll be remembered.''
\end{poetry}

\v{17}At the end of 70 years, the \divine{Lord} will deal with Tyre, at which time she'll return to her courtesan's trade, and prostitute herself with the kingdoms\fnote{\fbackref{23:17} So 1QIsa\textsuperscript{a}; MT LXX read \fbib{all the kingdoms}} of the world on the surface of the earth. \v{18}Nevertheless, her profits and her earnings will be dedicated\fnote{\fbackref{23:18} Or \fbib{given}} to the \divine{Lord}; they will not be stored up or hoarded---but her profits will go to those who live in the \divine{Lord}'s presence,\fnote{\fbackref{23:18} So 1QIsa\textsuperscript{a} MT; 4QIsa\textsuperscript{c} reads \fbib{but will be for those who live in the \divine{Lord}'s presence. And her profits will be}} for abundant food and choice clothing.
\labelchapt{24}
\passage{The Earth is Judged}

\begin{poetry}
\poeml \chapt{24}
\v{1}``Watch out! The \divine{Lord}\fnote{\fbackref{24:1} So 1QIsa\textsuperscript{a} MT; 4QIsa\textsuperscript{c} reads \fbib{Lord}} is about to depopulate the land \\
\poemll    and devastate it; \\
\poeml he will turn it upside down\fnote{\fbackref{24:1} Or \fbib{distort its surface}} \\
\poemll    and scatter its inhabitants. \\
\poeml \v{2}It will be the same for the lay people as for priests, \\
\poemll    the same for servants as for their masters, \\
\poeml for female servants as for their mistresses, \\
\poemll    for buyers as for sellers, \\
\poeml for lenders as for borrowers, \\
\poemll    and for creditors as for debtors. \\
\poeml \v{3}The earth will be utterly depopulated \\
\poemll    and completely laid waste --- \\
\poemlll       for the \divine{Lord} has spoken this message.\fnote{\fbackref{24:3} Lit. \fbib{word}} \\
\poeml \v{4}``The earth dries up and withers; \\
\poemll    the world languishes and fades away; \\
\poeml heaven fades away, \\
\poemll    along with the earth.\fnote{\fbackref{24:4} So 1QIsa\textsuperscript{a} 4QIsa\textsuperscript{c}; MT reads \fbib{the heavens fade away}; cf. LXX reads \fbib{the exalted ones of the earth mourn}} \\
\poeml \v{5}The earth lies defiled \\
\poemll    beneath its inhabitants; \\
\poeml because they have transgressed the laws,\fnote{\fbackref{24:5} So 1QIsa\textsuperscript{a} MT; 4QIsa\textsuperscript{c} LXX read \fbib{the Law}} \\
\poeml violated the statutes, \\
\poeml and broken the everlasting covenant. \\
\poeml \v{6}Therefore the curse keeps on consuming,\fnote{\fbackref{24:6} So 1QIsa\textsuperscript{a}; 4QIsa\textsuperscript{c} MT LXX read \fbib{consuming the earth}} \\
\poemll    and its inhabitants are declared guilty. \\
\poeml Furthermore, the inhabitants of earth are ablaze, \\
\poemll    and few people are left. \\
\poeml \v{7}The new wine evaporates; \\
\poemll    the vine and the oil\fnote{\fbackref{24:7} So 4QIsa\textsuperscript{c}; 1QIsa\textsuperscript{a} MT LXX lack \fbib{and the oil}} dry up; \\
\poemlll       all the merrymakers groan. \\
\poeml \v{8}``The celebrations of the tambourine have ended, \\
\poemll    the noise of the jubilant has stopped, \\
\poemlll       and the mirth that the harp produces has ended. \\
\poeml \v{9}No longer do they drink wine \\
\poemll    accompanied by singing; \\
\poemlll       even beer\fnote{\fbackref{24:9} Or \fbib{and strong drink}} tastes bitter to those who drink it. \\
\poeml \v{10}The chaotic city lies broken down; \\
\poemll    every house is closed up \\
\poemlll       so that no one can enter them.\fnote{\fbackref{24:10} 1QIsa\textsuperscript{a} MT lack \fbib{them}} \\
\poeml \v{11}There is an outcry in the streets over wine; \\
\poemll    all cheer turns to gloom; \\
\poemlll       the fun times of the earth are banished. \\
\poeml \v{12}Desolation remains in the city \\
\poemll    whose gates lie battered into ruins. \\
\poeml \v{13}So it will be on the earth \\
\poemll    and among the nations--- \\
\poeml as when an olive tree is beaten, \\
\poemll    or as gleanings when the grape harvest has ended.''
\passage{Glorifying God}
\poeml \v{14}``They raise their voices; \\
\poemll    they shout for joy;\fnote{\fbackref{24:14} So 1QIsa\textsuperscript{a} MT; 4QIsa\textsuperscript{c} reads \fbib{and they shout}} \\
\poeml from the west\fnote{\fbackref{24:14} Lit. \fbib{sea}; so 1QIsa\textsuperscript{a} MT; cf. LXX; 4QIsa\textsuperscript{c} reads \fbib{day}} they shout aloud\fnote{\fbackref{24:14} So 1QIsa\textsuperscript{a} MT; 4QIsa\textsuperscript{c} reads \fbib{And they cry out}} \\
\poemll    over the \divine{Lord}'s majesty. \\
\poeml \v{15}Therefore, you in the east,\fnote{\fbackref{24:15} So 1QIsa\textsuperscript{a} MT; 4Q1sa\textsuperscript{c} reads \fbib{in the east, in Aram}; LXX lacks \fbib{in the east}} \\
\poemll    give glory to the \divine{Lord}! \\
\poeml You in the coastlands of the sea, \\
\poemll    give glory to the name of the \divine{Lord} God of Israel! \\
\poeml \v{16}From the ends of the earth \\
\poemll    we hear songs of praise: \\
\poemlll       `Glory to the Righteous One!' \\
\poeml ``But I say, `I am pining away, \\
\poemll    I'm pining away. \\
\poemlll       How terrible things are for me! \\
\poeml For treacherous people betray--- \\
\poemll    treacherous people are betraying with treachery!'\,''
\passage{The Universal Impact of Judgment}
\poeml \v{17}``Terror and pit and snare are coming in your direction, \\
\poemll    you inhabitants of the earth! \\
\poeml \v{18}Whoever flees at the sound of terror \\
\poemll    will fall into a pit, \\
\poeml and whoever climbs out of the pit \\
\poemll    will be caught in a snare. \\
\poeml For the windows of judgment\fnote{\fbackref{24:18} 1QIsa\textsuperscript{a} MT lack \fbib{of judgment}} from above are opened, \\
\poemll    and the foundations of the earth are shaken. \\
\poeml \v{19}The earth is utterly shattered, \\
\poemll    the earth is split apart, \\
\poemlll       the earth is violently shaken. \\
\poeml \v{20}The earth\fnote{\fbackref{24:20} So 1QIsa\textsuperscript{a}; MT LXX read \fbib{Earth}} reels to and fro like a drunkard; \\
\poemll    it sways like a hut;\fnote{\fbackref{24:20} So MT; 1QIsa\textsuperscript{a} lacks \fbib{like a hut}} \\
\poeml its transgression lies so heavy upon it, \\
\poemll    that it falls, never to rise again. \\
\poeml \v{21}``And it will come about at that time,\fnote{\fbackref{24:21} Lit. \fbib{about on that day}} \\
\poemll    the \divine{Lord} will punish \\
\poeml the armies of the exalted ones in the heavens,\fnote{\fbackref{24:21} Or \fbib{ones on high}} \\
\poemll    and the rulers\fnote{\fbackref{24:21} Lit. \fbib{kings}} of the earth on earth. \\
\poeml \v{22}They\fnote{\fbackref{24:22} So 1QIsa\textsuperscript{a}; 4QIsa\textsuperscript{c} MT LXX read \fbib{And they}} will be herded together\fnote{\fbackref{24:22} So 1QIsa\textsuperscript{a} LXX; 4QIsa\textsuperscript{c} MT read \fbib{together like prisoners}} \\
\poemll    into the Pit;\fnote{\fbackref{24:22} I.e., the place of punishment in the afterlife; or \fbib{into a dungeon};} \\
\poeml they will be shut up in prison, \\
\poemll    and after many days they will be punished. \\
\poeml \v{23}Then the moon will be embarrassed \\
\poemll    and the sun ashamed, \\
\poeml for the \divine{Lord} of the Heavenly Armies will reign \\
\poemll    on Mount Zion and in Jerusalem; \\
\poeml and in the presence of its elders \\
\poemll    there will be glory.''
\end{poetry}
\labelchapt{25}
\passage{Praise to the Victorious God}

\begin{poetry}
\poeml \chapt{25}
\v{1}\divine{Lord}, you are my God; \\
\poemll    I will exalt you and praise your name, \\
\poeml for you have done marvelous things, \\
\poemlll       plans made long ago in faithfulness and truth. \\
\poeml \v{2}For you have made the city a heap of rubble, \\
\poemll    the fortified city into a ruin; \\
\poeml the foreigners' citadel\fnote{\fbackref{25:2} So 1QIsa\textsuperscript{a} MT; MT\textsuperscript{mss} LXX read \fbib{citadel of arrogant people}} is no longer a city--- \\
\poemll    it will never be rebuilt! \\
\poeml \v{3}Therefore strong peoples will glorify you; \\
\poemll    cities of ruthless nations will revere you. \\
\poeml \v{4}For you have been a stronghold for the poor, \\
\poemll    a stronghold for the needy in distress, \\
\poeml a shelter from the storm \\
\poemll    and a shade from the heat--- \\
\poeml for the blistering attack\fnote{\fbackref{25:4} 1QIsa\textsuperscript{a} MT lack \fbib{attack}} from the ruthless \\
\poemll    is like a rainstorm beating against a wall, \\
\poeml \v{5}and the noise of foreigners is like the heat of the desert. \\
\poemll    Just as you subdue heat by the shade of clouds, \\
\poemlll       so the victory songs of violent men will be stilled.
\passage{Celebration of the Righteous}
\poeml \v{6}``On this mountain,\fnote{\fbackref{25:6} I.e. Mount Zion; cf. 24:23} the \divine{Lord} of the Heavenly Armies will prepare for all peoples \\
\poemll    a banquet of rich food, \\
\poeml a banquet of well-aged wines--- \\
\poemll    rich food full of marrow, \\
\poemlll       and refined wines of the finest\fnote{\fbackref{25:6} 1QIsa\textsuperscript{a} MT lack \fbib{the finest}} vintage \\
\poeml \v{7}And on this mountain,\fnote{\fbackref{25:7} I.e. Mount Zion; cf. 24:23} he will swallow up \\
\poemll    the burial\fnote{\fbackref{25:7} So 1QIsa\textsuperscript{a}; the Heb. lacks \fbib{burial}} shroud that enfolds all peoples, \\
\poeml the veil that is spread over all nations--- \\
\poeml \v{8}he has swallowed up\fnote{\fbackref{25:8} So 1QIsa\textsuperscript{a} MT\textsuperscript{ms}; MT reads \fbib{And he will swallow up}; cf. LXX Syr Theodotian 1Cor 15:54} death forever! \\
\poeml Then the Lord \divine{God} will wipe away the tears from all faces, \\
\poemll    and he will take away the disgrace of his people \\
\poemlll       from the entire earth.'' \\
\poeml for the \divine{Lord} has spoken. \\
\poeml \v{9}``And you\fnote{\fbackref{25:9} So 1QIsa\textsuperscript{a} Syriac; 4QIsa\textsuperscript{c} MT read \fbib{he}; LXX reads \fbib{they}} will say at that time,\fnote{\fbackref{25:9} Lit. \fbib{say on that day}} \\
\poemll    `Look! It's the \divine{Lord}!\fnote{\fbackref{25:9} So 1QIsa\textsuperscript{a}; MT LXX lack \fbib{It's the \divine{Lord}}} This is our God! \\
\poeml We waited for him, \\
\poemll    and he saved us. \\
\poeml This is the \divine{Lord}! \\
\poemll    We waited for him, \\
\poeml so let us rejoice, \\
\poemll    and we will\fnote{\fbackref{25:9} So 1QIsa\textsuperscript{a}; MT reads \fbib{and let us}} be glad that he has saved us.''
\passage{The Misery of Moab}
\poeml \v{10}For the \divine{Lord}'s power\fnote{\fbackref{25:10} Lit. \fbib{hand}} will rest on this mountain,\fnote{\fbackref{25:10} I.e. Mount Zion; cf. 24:23} \\
\poemll    but the Moabites will be trodden down beneath him, \\
\poeml just as straw is trodden down \\
\poemll    in the slime of\fnote{\fbackref{25:10} Lit. \fbib{in the water of}; so 1QIsa\textsuperscript{a} MT; MT\textsuperscript{qere} reads \fbib{in}; LXX reads \fbib{in wagons}} a manure pit. \\
\poeml \v{11}They will spread out their hands in the thick of it, \\
\poemll    just as swimmers spread out their hands to swim, \\
\poeml but the \divine{Lord} will bring down their pride, \\
\poemll    together with the cleverness of their hands. \\
\poeml \v{12}He brings down the high fortifications of your walls \\
\poemll    and lays them low; \\
\poeml he will raze them\fnote{\fbackref{25:12} Or \fbib{reach}; so 1QIsa\textsuperscript{a}; 4QIsa\textsuperscript{c} MT read \fbib{he casts them}} to the ground, \\
\poemll    right down to the dust.
\end{poetry}
\labelchapt{26}
\passage{The Song of Redeemed Judah}

\begin{poetry}
\poeml \chapt{26}
\v{1}At that time,\fnote{\fbackref{26:1} Lit. \fbib{On that day}} people will sing this song\fnote{\fbackref{26:1} So 1QIsa\textsuperscript{a}; 1QIsa\textsuperscript{b} 4QIsa\textsuperscript{c} MT read \fbib{time, this song will be sung}; LXX reads \fbib{time, they will sing that song}} in the land of Judah: \\
\poeml ``We have a strong city; \\
\poemll    God crafts victory, \\
\poemlll       its walls and ramparts.\fnote{\fbackref{26:1} So 4QIsa\textsuperscript{c}; 1QIsa\textsuperscript{a} MT read \fbib{walls and ramparts}; LXX reads \fbib{wall and rampart}} \\
\poeml \v{2}Open your\fnote{\fbackref{26:2} So 1QIsa\textsuperscript{a}; MT LXX read \fbib{the}} gates, \\
\poemll    so the righteous nation that safeguards its faith may enter. \\
\poeml \v{3}You will keep perfectly peaceful\fnote{\fbackref{26:3} Lit. \fbib{peace, peace}; so 1QIsa\textsuperscript{a} MT; LXX Syr read \fbib{peace}} \\
\poemll    the one whose mind remains focused on you, \\
\poemlll       because he remains\fnote{\fbackref{26:3} So 1QIsa\textsuperscript{a} 1QIsa\textsuperscript{b} LXX; 4QIsa\textsuperscript{c} MT read \fbib{trusts}} in you. \\
\poeml \v{4}``Trust in the \divine{Lord} forever, \\
\poemll    for in the \divine{Lord God}\fnote{\fbackref{26:4} So 1QIsa\textsuperscript{a} MT; 4QIsa\textsuperscript{b} reads \fbib{the \divine{Lord} God}; LXX reads \fbib{the Lord, the Lord}} you have an everlasting rock. \\
\poeml \v{5}For he has made drunk\fnote{\fbackref{26:5} So MT 1QIsa\textsuperscript{a}; 1QIsa\textsuperscript{b} 4QIsa\textsuperscript{b} 4QIsa\textsuperscript{c} MT read \fbib{has brought low}; LXX reads \fbib{has humbled and brought down}} \\
\poemll    the inhabitants of the height, \\
\poemlll       the lofty city. \\
\poeml He lays it low\fnote{\fbackref{26:5} So 1QIsa\textsuperscript{a} LXX; MT reads \fbib{He levels it, he levels it}} to the ground \\
\poeml casting it down to the dust, \\
\poeml \v{6}by the feet of the oppressed who trample it,\fnote{\fbackref{26:6} So 1QIsa\textsuperscript{a} LXX; MT reads \fbib{The foot tramples it}} \\
\poeml by the footsteps of the needy.\fnote{\fbackref{26:6} So 1QIsa\textsuperscript{a} LXX; MT reads \fbib{oppressed}} \\
\poeml \v{7}``The path of the righteous is level; \\
\poemll    O Upright One,\fnote{\fbackref{26:7} So 1QIsa\textsuperscript{a} MT; 4QIsa\textsuperscript{c} reads \fbib{they go straight ahead}; LXX lacks this line} \\
\poemlll       you make safe\fnote{\fbackref{26:7} So 1QIsa\textsuperscript{a}; MT LXX read \fbib{smooth} or \fbib{you prepare}} the way of justice.\fnote{\fbackref{26:7} So 1QIsa\textsuperscript{a} 4QIsa\textsuperscript{c}; MT LXX read \fbib{of the righteous ones}} \\
\poeml \v{8}Yes, \divine{Lord}, in the path of your judgments we wait;\fnote{\fbackref{26:8} So 1QIsa\textsuperscript{a} LXX; MT reads \fbib{we wait for you}} \\
\poemll    your name and your Law\fnote{\fbackref{26:8} So 1QIsa\textsuperscript{a}; 4QIsa\textsuperscript{c} MT read \fbib{your renown}; cf. LXX} are the\fnote{\fbackref{26:8} So 1QIsa\textsuperscript{a} MT; 4QIsa\textsuperscript{b} reads \fbib{my}} soul's desire. \\
\poeml \v{9}My soul yearns for you in the night; \\
\poemll    my spirit within me searches for you. \\
\poeml For when your judgments come upon the earth, \\
\poemll    the world's inhabitants learn righteousness. \\
\poeml \v{10}If favor is shown to the wicked, \\
\poemll    they don't learn righteousness; \\
\poeml even in a land of uprightness they act perversely \\
\poemll    and do not perceive the majesty of the \divine{Lord}. \\
\poeml \v{11}``\divine{Lord}, your hand is lifted up, \\
\poemll    but they do not see it. \\
\poeml And\fnote{\fbackref{26:11} So 1QIsa\textsuperscript{a}; cf. LXX; the Heb. lacks \fbib{And}} let them see your zeal for your\fnote{\fbackref{26:11} Lit. \fbib{the}; so 1QIsa\textsuperscript{a}; the Heb. lacks \fbib{your}} people \\
\poemll    and be put to shame--- \\
\poemlll       yes, let the fire reserved for your enemies consume them! \\
\poeml \v{12}\divine{Lord}, you will decide\fnote{\fbackref{26:12} So 1QIsa\textsuperscript{a}; MT reads \fbib{prepare} or \fbib{give}; LXX reads \fbib{\divine{Lord}, give}} peace for us, \\
\poemll    for you have indeed accomplished \\
\poemlll       all our achievements for us. \\
\poeml \v{13}O \divine{Lord} our God, \\
\poemll    other lords besides you have ruled over us, \\
\poeml but through you alone \\
\poemll    we acknowledge your name. \\
\poeml \v{14}The dead won't live, \\
\poemll    and\fnote{\fbackref{26:14} So 1QIsa\textsuperscript{a} LXX; 1QIsa\textsuperscript{a} corrector MT lack \fbib{and}} the departed spirits won't rise--- \\
\poeml to that end, you punished and destroyed them, \\
\poemll    then locked away\fnote{\fbackref{26:14} So 1QIsa\textsuperscript{a}; MT LXX read \fbib{wiped out}} all memory of them. \\
\poeml \v{15}``But you have enlarged the nation,\fnote{\fbackref{26:15} So 1QIsa\textsuperscript{a} 4QIsa\textsuperscript{b}} \divine{Lord}; \\
\poemll    you have enlarged the nation.\fnote{\fbackref{26:15} So 1QIsa\textsuperscript{a}; MT\textsuperscript{ms} lacks this line} \\
\poeml You have gained honor; \\
\poemll    you have extended all the borders of the land. \\
\poeml \v{16}\divine{Lord}, they\fnote{\fbackref{26:16} So 1QIsa\textsuperscript{a} MT; MT\textsuperscript{mss} LXX\textsuperscript{mss} read \fbib{we;} LXX reads \fbib{I remembered you}} came to you in distress; \\
\poemll    they poured out their secret\fnote{\fbackref{26:16} So 1QIsa\textsuperscript{a}; 4QIsa\textsuperscript{b} MT read \fbib{out a magical}} prayer \\
\poemlll       when your chastenings were\fnote{\fbackref{26:16} So 1QIsa\textsuperscript{a}; MT LXX read \fbib{your chastening was}} afflicting\fnote{\fbackref{26:16} Lit. \fbib{upon}} them. \\
\poeml \v{17}Just as a pregnant woman writhes \\
\poemll    and cries out during her labor \\
\poeml when she is about to give birth, \\
\poemll    so were we because of you, \divine{Lord}. \\
\poeml \v{18}We were pregnant, writhing in pain, \\
\poemll    but we gave birth only to wind. \\
\poeml We have not won your\fnote{\fbackref{26:18} So 1QIsa\textsuperscript{a} LXX; MT lacks \fbib{your}} victory on earth, \\
\poeml nor have the inhabitants of the world been born.''
\passage{The Resurrection of the Dead}
\poeml \v{19}``But your dead will live; their bodies will rise. \\
\poeml Those who live in the dust will wake up and shout for joy!\fnote{\fbackref{26:19} So 1QIsa\textsuperscript{a}; MT reads \fbib{Wake up and shout for joy}, \fbib{you}; LXX reads \fbib{Those in the dust will rejoice, for}} \\
\poeml For your dew is like the dew of dawn, \\
\poeml and the earth will give birth to the dead. \\
\poeml \v{20}Come, my people, enter your rooms \\
\poemll    and shut your doors\fnote{\fbackref{26:20} So 1QIsa\textsuperscript{a} MT; MT\textsuperscript{qere} reads \fbib{door}} behind you. \\
\poeml Hide yourselves\fnote{\fbackref{26:20} So 1QIsa\textsuperscript{a}; MT reads \fbib{yourself}} for a little while \\
\poemll    until the fury has passed by. \\
\poeml \v{21}For see, the \divine{Lord} is coming from his place \\
\poemll    to punish the inhabitants of the earth for their sins; \\
\poeml the earth will reveal the blood that has been shed on it, \\
\poemll    and will no longer conceal its slain.''
\end{poetry}
\labelchapt{27}
\passage{Israel's Deliverance}

\chapt{27}
\v{1}At that time,\fnote{\fbackref{27:1} Lit. \fbib{On that day}} with his fierce, mighty, and powerful sword, the \divine{Lord} will punish the gliding serpent Leviathan---the coiling serpent Leviathan---and he will kill the dragon that's in the sea.

\begin{poetry}
\poeml \v{2}At that time,\fnote{\fbackref{27:2} Lit. \fbib{On that day}} \\
\poemll    ``A fermenting\fnote{\fbackref{27:2} So 1QIsa\textsuperscript{a}; MT LXX read \fbib{pleasant}} vineyard--- \\
\poemlll       sing about it! \\
\poeml \v{3}I, the \divine{Lord}, watch over it \\
\poemll    And I water it continuously. \\
\poeml I guard it night and day \\
\poemll    so no one can harm it. \\
\poeml \v{4}I am not angry. \\
\poemll    If only the vineyard\fnote{\fbackref{27:4} Lit. \fbib{only it}} could give me briers and thorns\fnote{\fbackref{27:4} So 1QIsa\textsuperscript{a}; MT reads \fbib{thorns}} to battle, \\
\poeml I would march against it, \\
\poemll    and\fnote{\fbackref{27:4} So 1QIsa\textsuperscript{a}; the Heb. lacks \fbib{and}} I would burn it all up. \\
\poeml \v{5}Or else let it lay claim to my protection; \\
\poemll    let it make peace with me, \\
\poemlll       yes, let it make peace with me.'' \\
\poeml \v{6}In times to come, Jacob will take root, \\
\poemll    and\fnote{\fbackref{27:6} So 1QIsa\textsuperscript{a} LXX; the Heb. lacks \fbib{and}} Israel will blossom, sprout shoots, \\
\poemlll       and fill the whole world with fruit. \\
\poeml \v{7}Has the \divine{Lord}\fnote{\fbackref{27:7} Lit. \fbib{Has he}} struck them down, \\
\poemll    just as he struck down those who struck them? \\
\poeml Or have they been killed, \\
\poemll    just as their killers were killed? \\
\poeml \v{8}Measure by measure,\fnote{\fbackref{27:8} Or \fbib{With war cries}} \\
\poemll    in their exile you contended with them; \\
\poeml with his fierce blast he removed them, \\
\poemll    as on a day when the east wind blows. \\
\poeml \v{9}By this, then, Jacob's guilt will be atoned for, \\
\poemll    and this will be the full harvest \\
\poemlll       that comes from the removal of his sin: \\
\poeml when he makes all the altar stones \\
\poemll    like pulverized chalkstones, \\
\poeml no Asherah\fnote{\fbackref{27:9} Or \fbib{sacred}} poles or incense altars will be left standing. \\
\poeml \v{10}For the fortified city stands desolate, \\
\poemll    a settlement abandoned and forsaken like the desert; \\
\poeml calves graze there, \\
\poemll    and there they lie down \\
\poemlll       and strip bare its branches. \\
\poeml \v{11}When its branches are dry, \\
\poemll    they are broken off, \\
\poeml and women come and kindle fires with them, \\
\poemll    since this is a people who show no consideration. \\
\poeml That is why the One who made them shows them no compassion; \\
\poemll    the One who created them shows them no mercy.
\end{poetry}
\passage{Assyria and Egypt Exiles Redeemed}

\v{12}At that time,\fnote{\fbackref{27:12} Lit. \fbib{On that day}} the \divine{Lord} will winnow grain from the Euphrates River\fnote{\fbackref{27:12} DSS MT lack \fbib{River}} channel to the Wadi\fnote{\fbackref{27:12} I.e. a seasonal stream or river that channels water during rain seasons but is dry at other times} of Egypt,\fnote{\fbackref{27:12} I.e. the southwestern-most border of ancient Philistia} and you will be gathered in one by one, O people of Israel. \v{13}Furthermore, at that time,\fnote{\fbackref{27:13} Lit. \fbib{Furthermore, on that day}} a great trumpet will be sounded, and those who were perishing in the land of Assyria and those who had been expelled\fnote{\fbackref{27:13} Or \fbib{exiled}} to the land of Egypt will come and worship the \divine{Lord} on his holy mountain at Jerusalem.
\labelchapt{28}
\passage{The Captivity of Ephraim}

\begin{poetry}
\poeml \chapt{28}
\v{1}How terrible it will be for that arrogant garland--- \\
\poeml the drunks of Ephraim! \\
\poeml How terrible it will be for that fading flower of his glorious beauty, \\
\poemll    which sits on the heads of people bloated with food,\fnote{\fbackref{28:1} Lit. \fbib{the valley of those grown fat}} \\
\poemlll       of people overcome with wine! \\
\poeml \v{2}Look! The \divine{Lord}\fnote{\fbackref{28:2} So 1QIsa\textsuperscript{a}; MT reads \fbib{Lord}} has one who is mighty and strong, \\
\poemll    like a hailstorm and destructive tempest, \\
\poeml like a storm of mighty, overflowing water--- \\
\poemll    and\fnote{\fbackref{28:2} So 1QIsa\textsuperscript{a}; the Heb. lacks \fbib{and}} he will give rest to the land. \\
\poeml \v{3}With hands\fnote{\fbackref{28:3} So 1QIsa\textsuperscript{a}; cf. LXX; or \fbib{He will throw them forcefully down to the ground}; cf. MT} and feet, that proud garland\fnote{\fbackref{28:3} So 1QIsa\textsuperscript{a} LXX; or \fbib{\v{3}Underfoot that proud garland}; cf. MT}--- \\
\poemll    those drunks of Ephraim---will be trampled. \\
\poeml \v{4}And that fading flower, his glorious beauty, \\
\poemll    which sits on the heads of people bloated with food, \\
\poeml will be like an early fig before summer--- \\
\poemll    whenever someone sees it, he swallows it \\
\poemlll       as soon as it's in his hand. \\
\poeml \v{5}At that time,\fnote{\fbackref{28:5} Lit. \fbib{On that day}} the \divine{Lord} of the Heavenly Armies will become a glorious crown, \\
\poemll    a beautiful diadem for the remnant of his people, \\
\poeml \v{6}and a spirit of justice to the one who sits in judgment, \\
\poemll    a source of strength to those who turn back the battle at the gate. \\
\poeml \v{7}These people also\fnote{\fbackref{28:7} 1QIsa\textsuperscript{a} lacks \fbib{people also}; MT lacks \fbib{people}} stagger from wine \\
\poemll    and reel from strong drink. \\
\poeml Priests and prophets stagger from strong drink; \\
\poemll    they're drunk from\fnote{\fbackref{28:7} Lit. \fbib{are devoured by}} wine; \\
\poeml they reel from strong drink, \\
\poemll    waver when seeing visions, \\
\poemlll       and stumble when rendering decisions. \\
\poeml \v{8}For all the tables are covered in vomit and filth, \\
\poemll    with no clean\fnote{\fbackref{28:8} 1QIsa\textsuperscript{a} MT lack \fbib{clean}} space left.
\passage{Misuse of God's Word}
\poeml \v{9}To whom will he teach knowledge, \\
\poemll    and to whom will he explain the message? \\
\poeml To children just weaned from milk? \\
\poemll    To those just taken from the breast? \\
\poeml \v{10}For it is: ``Do this and do that, \\
\poemll    do this and do that, \\
\poeml Line upon line, line upon line, \\
\poemll    a little here, a little there.'' \\
\poeml \v{11}Very well, then, through the mouths of foreigners\fnote{\fbackref{28:11} Or \fbib{through foreign lips}} \\
\poemll    and foreign languages \\
\poemlll       the \divine{Lord} will speak to this people \\
\poeml \v{12}to whom he said, \\
\poemll    ``This is the resting place, \\
\poemlll       so give rest to the weary''\,' \\
\poeml and, \\
\poemll    ``This is the place of repose''--- \\
\poemlll       but they would not listen. \\
\poeml \v{13}So, then, the message from the \divine{Lord} to them will become: \\
\poemll    ``Do this and do that, \\
\poemlll       do this and do that, \\
\poemll    line upon line, \\
\poemlll       line upon line, \\
\poemll    a little here, \\
\poemlll       a little there,'' \\
\poeml so that they will go, \\
\poemll    but fall backward, \\
\poemlll       and be injured, snared, and captured.
\passage{God's Precious Cornerstone}
\poeml \v{14}``Therefore hear\fnote{\fbackref{28:14} So 1QIsa\textsuperscript{a}; 4QIsa\textsuperscript{c} MT LXX read \fbib{hear} (pl.)} the message from the \divine{Lord}, you scoffers \\
\poemll    who rule this people that are in Jerusalem. \\
\poeml \v{15}Because you said: \\
\poemll    `We have entered into a covenant with death, \\
\poemlll       and we have an agreement with Sheol,\fnote{\fbackref{28:15} I.e. the place where the dead dwell in the afterlife} \\
\poemll    so when the overwhelming scourge makes its choice,\fnote{\fbackref{28:15} So 1QIsa\textsuperscript{a}; MT LXX read \fbib{sweeps by}} \\
\poemlll       it cannot reach us, \\
\poemll    since we have made lies our refuge \\
\poemlll       and have concealed ourselves inside falsehood' \\
\poeml \v{16}therefore this is what the \divine{Lord} God\fnote{\fbackref{28:16} So 1QIsa\textsuperscript{a} corrector} says: \\
\poemll    ``Look! I am laying\fnote{\fbackref{28:16} So 1QIsa\textsuperscript{a} 1QIsa\textsuperscript{b}; MT reads \fbib{I have laid}} a foundation stone in Zion, a tested stone, \\
\poemlll       a precious cornerstone for a sure\fnote{\fbackref{28:16} So 1QIsa\textsuperscript{a} MT; MT\textsuperscript{mss} LXX lack \fbib{sure}} foundation: \\
\poemll    Whoever believes firmly will not act hastily. \\
\poeml \v{17}And I will make justice the measuring line, \\
\poemll    and righteousness the plumb line; \\
\poeml hail will sweep away your refuge of lies, \\
\poemll    and floods\fnote{\fbackref{28:17} Lit. \fbib{waters}} will overflow your hiding place. \\
\poeml \v{18}``Then your covenant with death will be annulled, \\
\poemll    and your agreement with Sheol\fnote{\fbackref{28:18} I.e. the place where the dead dwell in the afterlife} will not stand; \\
\poeml when the overwhelming scourge sweeps by, \\
\poemll    you will be trampled by it. \\
\poeml \v{19}As often as it sweeps through, \\
\poemll    it will carry you away, \\
\poeml for it will sweep by morning after morning in the day; \\
\poemll    but understanding this message will bring sheer terror at night,\fnote{\fbackref{28:19} So 1QIsa\textsuperscript{a}; MT reads \fbib{by day and by night; and there will be sheer terror.}} \\
\poeml \v{20}because the bed is too short to stretch out on, \\
\poemll    and its blankets too narrow to wrap around oneself!
\passage{God is on Mount Perazim}
\poeml \v{21}For the \divine{Lord} will stand upon\fnote{\fbackref{28:21} So 1QIsa\textsuperscript{a} LXX; MT reads \fbib{as}} Mount Perazim,\fnote{\fbackref{28:21} I.e. a mountain near Jerusalem, perhaps the Mount of Olives; cf. 2Sam 6:8} \\
\poemll    he will rouse himself in\fnote{\fbackref{28:21} So 1QIsa\textsuperscript{a}; MT reads \fbib{as}} the Valley of Gibeon; \\
\poeml to carry out his work--- \\
\poemll    his strange deed, \\
\poeml and to perform his task--- \\
\poemll    his alien task! \\
\poeml \v{22}But as for you,\fnote{\fbackref{28:22} So 1QIsa\textsuperscript{a} LXX; MT reads \fbib{So now}} don't start mocking, \\
\poemll    or your chains will become tighter; \\
\poeml for I have heard from the \divine{Lord}\fnote{\fbackref{28:22} So 1QIsa\textsuperscript{a} MT\textsuperscript{mss} LXX Syr; MT reads \fbib{the \divine{Lord God}} of the Heavenly Armies about destruction, \\
\poemll    and it is decreed against the whole land.
\passage{The God who Plows and Harvests}
\poeml \v{23}``Pay attention! \\
\poemll    Listen to what I have to say; \\
\poeml Pay attention, \\
\poemll    and hear my speech. \\
\poeml \v{24}Does he who plows for sowing plow all the time? \\
\poemll    Does he keep on breaking up and harrowing his field? \\
\poeml \v{25}When he has leveled its surface, \\
\poemll    he scatters caraway\fnote{\fbackref{28:25} Or \fbib{scatters black cumin}} and sows cumin, doesn't he? \\
\poeml He plants wheat in rows, \\
\poemll    barley in its designated place, \\
\poeml and feed for livestock\fnote{\fbackref{28:25} Lit. \fbib{and spelt}; i.e. a grass grown and used as fodder} around its borders,\fnote{\fbackref{28:25} So 1QIsa\textsuperscript{a}; MT reads \fbib{its border}; LXX reads \fbib{your borders}} doesn't he? \\
\poeml \v{26}His God instructs him regarding the correct way, \\
\poemll    directing him how to plant.\fnote{\fbackref{28:26} DSS MT lack \fbib{how to plant}} \\
\poeml \v{27}For caraway is not threshed with a sharp sledge, \\
\poemll    nor is a cart wheel rolled over cumin. \\
\poeml Instead, caraway is winnowed with a stick, \\
\poemll    and cumin with a rod. \\
\poeml \v{28}It\fnote{\fbackref{28:28} I.e. \fbib{grain}; so 1QIsa\textsuperscript{a} MT; 4QIsa\textsuperscript{k} reads \fbib{And it}} must be ground;\fnote{\fbackref{28:28} So 1QIsa\textsuperscript{a}; 4QIsa\textsuperscript{k} MT read \fbib{must be ground for bread}} \\
\poemll    one cannot keep threshing it forever. \\
\poeml Even if he drives his cart\fnote{\fbackref{8:28} So 1QIsa\textsuperscript{a}; 1QIsa\textsuperscript{a} corrector MT read \fbib{the wheel of his cart}} and horses over it, \\
\poemll    he cannot crush it. \\
\poeml \v{29}This insight also comes from the \divine{Lord} of the Heavenly Armies, \\
\poemll    who is distinguished\fnote{\fbackref{28:29} So 1QIsa\textsuperscript{a}; MT reads \fbib{wonderful}} in practical advice \\
\poemlll       and\fnote{\fbackref{28:29} So 1QIsa\textsuperscript{a}; the Heb. lacks \fbib{and}} magnificent in sound wisdom.''
\end{poetry}
\labelchapt{29}
\passage{Judgment is Coming to Jerusalem}

\begin{poetry}
\poeml \chapt{29}
\v{1}``How terrible it will be for you, Aruel, Aruel,\fnote{\fbackref{29:1} So 1QIsa\textsuperscript{a}; MT LXX read \fbib{Ariel, Ariel}; i.e. a nickname assigned by the prophet for Jerusalem} \\
\poemll    the city where David encamped! \\
\poeml Year after year, \\
\poemll    let your festivals run their cycle. \\
\poeml \v{2}Then I'll besiege Aruel,\fnote{\fbackref{29:2} So 1QIsa\textsuperscript{a}; MT LXX read \fbib{Ariel, Ariel}; i.e. an allusion to Jerusalem} \\
\poemll    and there will be sorrow and mourning; \\
\poemlll       she will become to me like an altar fireplace.\fnote{\fbackref{29:2} Lit. \fbib{an Ariel}; i.e. perhaps a pun on the name \fbib{Aruel}} \\
\poeml \v{3}Then I'll encamp against you like David,\fnote{\fbackref{29:3} So 4QIsa\textsuperscript{k} MT\textsuperscript{mss} LXX; 1QIsa\textsuperscript{a} MT read \fbib{you on all sides}} \\
\poemll    and I'll lay siege to you with towers, \\
\poeml raise siege works against you, \\
\poeml \v{4}and you will be brought down. \\
\poeml You will speak from the ground, \\
\poemll    and your speech will mumble from the dust. \\
\poeml Your voice will come ghostlike from the ground, \\
\poemll    and your speech will whisper from the dust. \\
\poeml \v{5}``But the hordes of your enemies\fnote{\fbackref{29:5} So 1QIsa\textsuperscript{a}; MT reads \fbib{foreigners}; LXX reads \fbib{the ungodly}} \\
\poemll    will become like fine dust, \\
\poemlll       and the hordes of tyrants like flying chaff. \\
\poeml Then suddenly, in an instant, \\
\poeml \v{6}you will be visited by the \divine{Lord} of the Heavenly Armies--- \\
\poeml with thunder, an earthquake, and great noise, \\
\poemll    with a windstorm, a tempest, \\
\poemlll       and flames from a devouring fire. \\
\poeml \v{7}Then the hordes of all the nations that fight against Aruel,\fnote{\fbackref{29:7} So 1QIsa\textsuperscript{a}; MT LXX read \fbib{Ariel, Ariel}; i.e. an allusion to Jerusalem} \\
\poemll    all that attack her and her fortification\fnote{\fbackref{29:7} So 1QIsa\textsuperscript{a}; MT reads \fbib{her mountain stronghold}; LXX reads \fbib{Jerusalem}} and besiege her, \\
\poemlll       will become like a dream, with its visions in the night--- \\
\poeml \v{8}as when a hungry man dreams--- \\
\poemll    he eats, but wakes up still hungry; \\
\poeml or when a thirsty man dreams--- \\
\poemll    he drinks, but wakes up faint, \\
\poemlll       with his thirst unquenched. \\
\poeml So will it be with the hordes of all the nations \\
\poemll    that fight against Mount Zion.
\passage{Blind to God's Words}
\poeml \v{9}``Act stupid! \\
\poemll    Be astonished! \\
\poeml Act blind, \\
\poemll    and be blind! \\
\poeml Be drunk,\fnote{\fbackref{29:9} So 1QIsa\textsuperscript{a}; MT reads \fbib{They have become drunk}} but not from\fnote{\fbackref{29:9} So 1QIsa\textsuperscript{a} LXX; MT lacks \fbib{from}} wine; \\
\poemll    stagger around,\fnote{\fbackref{29:9} So 1QIsa\textsuperscript{a}; MT reads \fbib{They stagger around}} but not from strong drink. \\
\poeml \v{10}For the \divine{Lord} has poured out upon you \\
\poemll    a spirit of deep sleep--- \\
\poeml he has closed your eyes, you prophets, \\
\poemll    he has covered your heads, you seers!''
\end{poetry}

\v{11}``And this entire vision has become for you like the words of a sealed book. When people give it to someone who can read, and say, `Read this, please,' he answers,\fnote{\fbackref{29:11} So 1QIsa\textsuperscript{a}; MT LXX read \fbib{he will answer}} `I cannot, because it is sealed.' \v{12}Or when they give the book\fnote{\fbackref{29:12} So 1QIsa\textsuperscript{a}; MT LXX read \fbib{the book will be given}} to someone who cannot read, and say, `Read this, please,' he answers,\fnote{\fbackref{29:12} So 1QIsa\textsuperscript{a}; MT LXX read \fbib{he will answer}} `I don't know how to read.'\,''
\passage{A Rebuke of Hypocritical Worship}

\begin{poetry}
\poeml \v{13}Then the Lord said: \\
\poeml ``Because these people draw near with their mouths \\
\poemll    and honor me with their lips, \\
\poemlll       but their hearts are far from me, \\
\poeml worship of me\fnote{\fbackref{29:13} So 1QIsa\textsuperscript{a}; MT reads \fbib{Their worship of me}} has become \\
\poemll    merely like\fnote{\fbackref{29:13} So 1QIsa\textsuperscript{a}; MT LXX lack \fbib{like}} rules taught by human beings. \\
\poeml \v{14}Therefore, watch out! \\
\poeml ``As for me,\fnote{\fbackref{29:14} So 1QIsa\textsuperscript{a} LXX; the Heb. lacks \fbib{as for me}} I will once again \\
\poemll    do amazing things with this people, \\
\poemlll       wonder upon wonder. \\
\poeml The wisdom of their wise men will perish, \\
\poemll    and the insights\fnote{\fbackref{29:14} So 1QIsa\textsuperscript{a}; MT LXX read \fbib{insight}} of their discerning men will stay hidden.''
\passage{A Rebuke to the Deceptive}
\poeml \v{15}``How terrible it will be for you who go to great depths \\
\poemll    to hide your plans from the \divine{Lord}, \\
\poeml you whose deeds have been\fnote{\fbackref{29:15} So 1QIsa\textsuperscript{a}; MT LXX read \fbib{deeds are} (or \fbib{will be})} done\fnote{\fbackref{29:15} DSS MT lack \fbib{done}} in the dark, \\
\poemll    and who say, `Who can see us? \\
\poemlll       Who has recognized\fnote{\fbackref{29:15} So 1QIsa\textsuperscript{a}; MT LXX read \fbib{recognizes}} us?' \\
\poeml \v{16}He has turned the tables on you\fnote{\fbackref{29:16} So 1QIsa\textsuperscript{a}; MT reads \fbib{You turn things upside down!}}--- \\
\poemll    as if the potter were thought to be like heat.\fnote{\fbackref{29:16} I.e. the fire in a kiln; so 1QIsa\textsuperscript{a}; MT LXX read \fbib{clay}} \\
\poeml Can what is made say of the one who made it, \\
\poemll    `He did not make me?' \\
\poeml Or can what is formed say of the ones\fnote{\fbackref{29:16} So 1QIsa\textsuperscript{a}; MT LXX read \fbib{one}} who formed it, \\
\poemll    `He has no skill?' \\
\poeml \v{17}``In a very little while, \\
\poemll    will not Lebanon be turned into a garden of fruit,\fnote{\fbackref{29:17} Lit. \fbib{into Carmel}} \\
\poemlll       and the garden of fruit\fnote{\fbackref{29:17} Lit. \fbib{and Carmel}} seem like a forest? \\
\poeml \v{18}On that day the deaf will hear \\
\poemll    the words of a scroll, \\
\poeml and out of gloom and darkness \\
\poemll    the eyes of the blind will see. \\
\poeml \v{19}The humble will again experience joy in the \divine{Lord}, \\
\poemll    and the poorest people will rejoice in the Holy One of Israel. \\
\poeml \v{20}For the ruthless will vanish, \\
\poemll    and mockers will disappear, \\
\poemlll       and all who have an eye for evil will be cut down--- \\
\poeml \v{21}those who make a person appear to be the offender in a lawsuit, \\
\poemll    who set a trap for someone \\
\poemlll       who is making his defense in court,\fnote{\fbackref{29:21} Lit. \fbib{in the gate}} \\
\poeml and push aside the innocent \\
\poemll    with specious arguments.
\end{poetry}

\v{22}``Therefore, this is what the \divine{Lord}, who redeemed Abraham, says concerning the house of Jacob:

\begin{poetry}
\poeml `No longer will Jacob be ashamed; \\
\poemll    no longer will his face grow pale. \\
\poeml \v{23}For when he sees in his midst his children, \\
\poemll    the work of my hands, \\
\poeml they will keep my name holy; \\
\poemll    they will sanctify the Holy One of Jacob \\
\poemlll       and stand in awe of the God of Israel. \\
\poeml \v{24}Moreover, those who go astray in spirit will gain\fnote{\fbackref{29:24} Lit. \fbib{discover}} understanding, \\
\poemll    and those who complain will accept instruction.'\,''
\end{poetry}
\labelchapt{30}
\passage{Foolish Trust in Egypt}

\begin{poetry}
\poeml \chapt{30}
\v{1}``Oh, you stubborn children,'' declares the \divine{Lord}, \\
\poeml ``who carry out plans--- \\
\poemll    but they are not mine, \\
\poeml and who make alliances--- \\
\poemll    but not by my Spirit, \\
\poemlll       piling sin upon sin. \\
\poeml \v{2}They set out to go down to Egypt, \\
\poemll    without asking my advice; \\
\poeml taking refuge in Pharaoh's protection, \\
\poemll    and seeking shelter in Egypt's shadow. \\
\poeml \v{3}But Pharaoh's protection will become your shame, \\
\poemll    and sheltering in Egypt's shadow your longing.\fnote{\fbackref{30:3} So 1QIsa\textsuperscript{a}; MT LXX read \fbib{disgrace}} \\
\poeml \v{4}And it will turn out that\fnote{\fbackref{30:4} So 1QIsa\textsuperscript{a}; MT LXX read \fbib{For even though}} his officials are at Zoan, \\
\poemll    and his envoys will reach Hanes. \\
\poeml \v{5}There is only loathsome destruction\fnote{\fbackref{30:5} So 1QIsa\textsuperscript{a}; MT reads \fbib{Everyone comes to shame}} \\
\poemll    through a people that cannot benefit them, \\
\poeml who bring neither help nor profit, \\
\poemll    but only shame and disgrace.''
\passage{The Animals of the Negev}
\poeml \v{6}An oracle about the animals of the Negev:\fnote{\fbackref{30:6} I.e. southern regions of the Sinai peninsula; cf. Josh 10:40} \\
\poeml ``Through a land of trouble, dryness,\fnote{\fbackref{30:6} So 1QIsa\textsuperscript{a}; MT LXX lack \fbib{dryness}; cf. Isa 41:18} and distress, \\
\poemll    of lionesses and roaring lions, \\
\poemlll       where there is no water,\fnote{\fbackref{30:6} So 1QIsa\textsuperscript{a}; MT LXX read \fbib{from whence come}} \\
\poeml a land of vipers and darting snakes, \\
\poemll    he carries\fnote{\fbackref{30:6} So 1QIsa\textsuperscript{a}; MT LXX read \fbib{they carry}} their riches on donkeys' backs, \\
\poeml and their treasures on the humps of camels, \\
\poemll    to a nation that cannot benefit them, \\
\poeml \v{7}to Egypt, which gives help that is worthless and useless. \\
\poemll    Therefore I call her, \\
\poemlll       `Rahab,\fnote{\fbackref{30:7} The Heb. word \fbib{Rahab} means \fbib{The One who Storms}; i.e. Egypt; cf. Isa 51:9; Ps 87:4} who just sits still.'\,''
\passage{The Illusions of False Prophecy}
\poeml \v{8}``Go now, and write it down\fnote{\fbackref{30:8} So 1QIsa\textsuperscript{a} MT; 4QIsa\textsuperscript{c} LXX read \fbib{write down}} on a tablet in their presence, \\
\poemll    inscribing it in a book, \\
\poeml so that for times to come \\
\poemll    it may be an everlasting witness. \\
\poeml \v{9}For they are a rebellious people, \\
\poemll    deceitful children, \\
\poeml children unwilling to hear \\
\poemll    the \divine{Lord}'s instruction. \\
\poeml \v{10}They say to the seers, \\
\poemll    `Don't see visions,' \\
\poeml and to the prophets, \\
\poemll    `Don't give us visions of what is right! \\
\poemlll       Instead, tell us welcome things, prophesy illusions, \\
\poeml \v{11}get out of the way, \\
\poemll    turn aside from the path, \\
\poemlll       and stop confronting us with the Holy One of Israel.''\fnote{\fbackref{30:11} Lit. \fbib{bring to an end the Holy One from before us}.}
\passage{Rejecting God's Message}
\poeml \v{12}Therefore, this is what the Holy One of Israel says: \\
\poeml ``Because you reject this message, \\
\poemll    and put your trust in oppression and enjoy it,\fnote{\fbackref{30:12} Apparent meaning 1QIsa\textsuperscript{a}; MT reads \fbib{and are perverse}} \\
\poemlll       and since you rely on it, \\
\poeml \v{13}therefore, for you this sin will become \\
\poemll    like a breach in a high wall that is about to collapse, \\
\poemlll       bulging out, \\
\poemll    and whose crash comes suddenly---in an instant. \\
\poeml \v{14}Its breaking will be like when potters' vessels are broken, \\
\poemll    shattered so ruthlessly\fnote{\fbackref{30:14} Lit. \fbib{broken---they do not take pity}; so 1QIsa\textsuperscript{a}; MT reads \fbib{broken---he does not take pity}} \\
\poeml that among its fragments not even a broken sliver will be found \\
\poemll    for taking fire from a hearth \\
\poemlll       or scooping water out of a cistern.'' \\
\poeml \v{15}For this is what the \divine{Lord}\fnote{\fbackref{30:15} So 1QIsa\textsuperscript{a}} \divine{God},\fnote{\fbackref{30:15} So 1QIsa\textsuperscript{a} corrector} the Holy One of Israel, says: \\
\poeml ``In repentance and rest you will be saved; \\
\poemll    in staying calm and trusting will be your strength. \\
\poemlll       But you refused. \\
\poeml \v{16}Instead, you said, \\
\poemll    `No! We'll escape on horses!' \\
\poemlll       Therefore, you'll flee away. \\
\poeml And you said, \\
\poemll    `We'll ride off on swift steeds!' \\
\poemlll       Therefore your pursuers will be swift. \\
\poeml \v{17}A thousand will flee at the threat of one; \\
\poemll    and run away, pursued by\fnote{\fbackref{30:17} So 1QIsa\textsuperscript{a}; MT LXX read \fbib{away at the threat of}} five, \\
\poeml until you are left \\
\poemll    like a flagpole on a mountaintop,\fnote{\fbackref{30:17} So 1QIsa\textsuperscript{a} LXX; MT reads \fbib{the mountaintop}} \\
\poemlll       like a banner on a hill.''
\passage{Restoration is Promised to Israel}
\poeml \v{18}``Nevertheless, the \divine{Lord} will wait \\
\poemll    so he can be gracious to you; \\
\poemlll       and thus he will rise up to show you mercy. \\
\poeml For the \divine{Lord} is a God of justice. \\
\poemll    How blessed are all those who wait for him.''
\end{poetry}

\v{19}Indeed, you people who live in Zion and in Jerusalem,\fnote{\fbackref{30:19} So 1QIsa\textsuperscript{a}; cf. LXX; MT reads \fbib{at Jerusalem}} you\fnote{\fbackref{30:19} So 1QIsa\textsuperscript{a} (pl.); MT (sing.)} will weep no more. How gracious the \divine{Lord}\fnote{\fbackref{30:19} So 1QIsa\textsuperscript{a}; MT LXX read \fbib{he}} will be to you at the sound of your cry! As soon as he hears it, he will answer you. \v{20}And although the \divine{Lord} gives you the bread of adversity and the water\fnote{\fbackref{30:20} So 1QIsa\textsuperscript{a}; MT lacks the correct Heb. construct} of affliction, your teachers won't hide themselves\fnote{\fbackref{30:20} So 1QIsa\textsuperscript{a}; MT reads \fbib{himself}} anymore, but your own eyes will see your teachers. \v{21}And whether you turn to the right or turn to the left, your ears will hear a message behind you: ``This is the way, walk in it.'' \v{22}Then you will defile your carved idols that are overlaid with silver and your images plated with gold. You'll throw them away like disgusting objects\fnote{\fbackref{30:22} Lit. \fbib{like menstrual rags}} and say to them, ``Away with you!''

\v{23}He will also provide rain for your seed that you sow in the ground, and the food that comes from the ground will be\fnote{\fbackref{30:23} So 1QIsa\textsuperscript{a} LXX; MT reads \fbib{and it will be}} rich and abundant. At that time,\fnote{\fbackref{30:23} Lit. \fbib{On that day}} your cattle will graze in broad meadows, \v{24}and oxen and donkeys that work the ground will eat seasoned\fnote{\fbackref{30:24} Lit. \fbib{salted}} fodder that workers will winnow with shovels and forks. \v{25}And on every lofty mountain and every high hill there will be brooks and canals\fnote{\fbackref{30:25} So 1QIsa\textsuperscript{a}; MT reads \fbib{streams}} running with water on the day of the great slaughter, when the towers fall.

\v{26}Moreover, the light of the moon will be like the light of the sun, and the sun's light will be seven times brighter, like the light of seven full days,\fnote{\fbackref{30:26} So 1QIsa\textsuperscript{a} MT; LXX lacks \fbib{like the light of seven full days}} when the \divine{Lord} binds up the bruises of his people and heals the wounds inflicted by his blow.

\begin{poetry}
\poeml \v{27}See, the name of the \divine{Lord} comes from far away, \\
\poemll    burning with his anger, and in thick rising smoke; \\
\poeml his lips are full of fury, \\
\poemll    and his tongue is like a devouring fire. \\
\poeml \v{28}His breath is like an overflowing torrent, \\
\poemll    and it rises right up to the neck, \\
\poeml to shake\fnote{\fbackref{30:28} So 1QIsa\textsuperscript{a}; MT reads \fbib{to sift}; LXX reads \fbib{to confuse}} the nations in the sieve of destruction, \\
\poemll    and to place in the jaws of the peoples a bit that leads them astray.
\end{poetry}

\v{29}You will have songs as on nights when people celebrate a holy festival,\fnote{\fbackref{30:29} So 1QIsa\textsuperscript{a}; MT reads \fbib{one celebrates a holy festival}} and gladness of heart, as when they set out with flutes to go to the \divine{Lord}'s mountain, to the Rock of Israel.
\passage{God's Judgment on Assyria}

\v{30}And the \divine{Lord} will make heard---yes, he will make heard\fnote{\fbackref{30:30} So 1QIsa\textsuperscript{a}; MT LXX read \fbib{heard only once}}---his majestic voice, and make his arm\fnote{\fbackref{30:30} I.e. \fbib{the Messiah}} seen descending in raging anger and in a flame of consuming fire, with a cloudburst, thunderstorm and hailstones. \v{31}Indeed, the Assyrians will be shattered at the \divine{Lord}'s voice, when he strikes them with his scepter. \v{32}And every stroke of his punishing rod\fnote{\fbackref{30:32} So MT\textsuperscript{mss}; 1QIsa\textsuperscript{a} reads \fbib{the rod of his foundation}; MT reads \fbib{the rod of foundation}} that the \divine{Lord} brings down on them will be to the sound of tambourines and harps, as he fights against her\fnote{\fbackref{30:32} So 1QIsa\textsuperscript{a} MT; MT\textsuperscript{qere, mss} read \fbib{against them}} in battle with a brandished arm.

\v{33}For the Fire Pit\fnote{\fbackref{30:33} Lit. \fbib{the Topheth}; i.e. a fire pit near Jerusalem where the Canaanite deity Molech was worshipped} has long been prepared; truly it is for the king; it will indeed be made ready.\fnote{\fbackref{30:33} So 1QIsa\textsuperscript{a}; MT reads \fbib{it is made ready for the king}; cf. LXX} And\fnote{\fbackref{30:33} So 1QIsa\textsuperscript{a}; MT lacks \fbib{And}} its pyre will be deep and wide, with abundant fire and wood. Like a stream of burning sulfur, the breath of the \divine{Lord} will set it ablaze.
\labelchapt{31}
\passage{Only the \divine{Lord} can Help}

\begin{poetry}
\poeml \chapt{31}
\v{1}``How terrible it will be for those who go down to\fnote{\fbackref{31:1} So 1QIsa\textsuperscript{a} LXX; the Heb. lacks \fbib{to}} Egypt for help, \\
\poemll    who rely on horses, \\
\poeml who trust in the chariot, \\
\poemll    because there are so many, \\
\poeml and in charioteers,\fnote{\fbackref{31:1} Or \fbib{horsemen}} \\
\poemll    because they are so strong--- \\
\poeml but do not look to\fnote{\fbackref{31:1} So 1QIsa\textsuperscript{a}; MT LXX read \fbib{upon}} the Holy One of Israel \\
\poemll    or seek the \divine{Lord}! \\
\poeml \v{2}Yet he is also wise and can bring disaster; \\
\poemll    he does not take back his words, \\
\poeml but will rise up against the house of those who practice evil \\
\poemll    and against anyone who assists people who work iniquity. \\
\poeml \v{3}The Egyptians are men, not God, \\
\poemll    and their horses are physical,\fnote{\fbackref{31:3} Or \fbib{flesh}} not spirit. \\
\poeml When the \divine{Lord} stretches out his hand, \\
\poemll    anyone who assists will stumble, \\
\poeml and the one who is helped will fall; \\
\poemll    and they will all perish together.''
\passage{The \divine{Lord} will Defend Jerusalem}
\poeml \v{4}For this is what the \divine{Lord} told me: \\
\poeml ``Just as a lion or a young lion growls over his objects of prey,\fnote{\fbackref{31:4} So 1QIsa\textsuperscript{a}; MT LXX read \fbib{his prey} (sing.)}--- \\
\poemll    even when a whole band of shepherds is called out against it, \\
\poeml it is not alarmed at their shouting \\
\poemll    or disturbed by their clamor--- \\
\poeml so the \divine{Lord} of the Heavenly Armies will come down \\
\poemll    to do battle on Mount Zion and on its hill. \\
\poeml \v{5}Like birds hovering overhead, \\
\poemll    so the \divine{Lord} of the Heavenly Armies will protect Jerusalem; \\
\poeml he will shield and deliver it; \\
\poemll    and\fnote{\fbackref{31:5} So 1QIsa\textsuperscript{a}; cf. LXX; the Heb. lacks \fbib{and}} he will pass over\fnote{\fbackref{31:5} I.e. as the Angel of Death passed over the Israelis; cf. Exod 12:13, 23, 27} and bring it to safety.\fnote{\fbackref{31:5} So 1QIsa\textsuperscript{a}; MT reads \fbib{rescue it}}
\end{poetry}

\v{6}Turn back to him, yes to him whom\fnote{\fbackref{31:6} So 1QIsa\textsuperscript{a}; MT reads \fbib{back to him whom}} your people\fnote{\fbackref{31:6} Lit. \fbib{whom they}} have so greatly betrayed, you people of Israel. \v{7}For at that time,\fnote{\fbackref{31:7} Lit. \fbib{on that day}} everyone will throw away their\fnote{\fbackref{31:7} Lit. \fbib{his}} idols of silver and their\fnote{\fbackref{31:7} Lit. \fbib{his}} idols of gold that your hands have sinfully made for yourselves.

\begin{poetry}
\poeml \v{8}``Then Assyria will fall by a sword \\
\poemll    that is not from human beings only\fnote{\fbackref{31:8} Lit. \fbib{not of man}}--- \\
\poemlll       a sword not wielded by mortal beings will devour them. \\
\poeml They will flee from the sword, \\
\poemll    and their young men will be put to forced labor. \\
\poeml \v{9}Their stronghold will vanish by reason of terror, \\
\poemll    and their commanders will be filled with alarm \\
\poemlll       because of the battle standard,'' \\
\poeml declares the \divine{Lord}, whose fire is in Zion \\
\poemll    and whose furnace is in Jerusalem.
\end{poetry}
\labelchapt{32}
\passage{The Government of Justice}

\begin{poetry}
\poeml \chapt{32}
\v{1}``Look, a king will reign in righteousness, \\
\poemll    and rulers will rule with justice. \\
\poeml \v{2}Each one will be like a shelter from the wind \\
\poemll    and a hiding place from\fnote{\fbackref{32:2} So 1QIsa\textsuperscript{a}; MT reads \fbib{of}} storms, \\
\poeml like streams of water in the desert, \\
\poemll    in\fnote{\fbackref{32:2} So 1QIsa\textsuperscript{a}; MT reads \fbib{like}} the shadow of a great rock in an exhausted\fnote{\fbackref{32:2} Or \fbib{thirsty}} land. \\
\poeml \v{3}Then the eyes of those who can see won't turn away, \\
\poemll    and the ears of those who can hear will listen. \\
\poeml \v{4}The hearts of reckless people will understand sound judgment, \\
\poemll    and the tongues of those who stammer will be ready to speak clearly. \\
\poeml \v{5}People will no longer call a fool\fnote{\fbackref{32:5} So 1QIsa\textsuperscript{a} LXX; MT reads \fbib{No longer will a fool will be called}} noble, \\
\poemll    nor will a bad person be declared honorable. \\
\poeml \v{6}For fools utter contempt, \\
\poemll    and their minds plot\fnote{\fbackref{32:6} So 1QIsa\textsuperscript{a} LXX; MT reads \fbib{work}} wrong things: \\
\poeml practicing ungodliness, \\
\poemll    spreading lies about the \divine{Lord}, \\
\poeml leaving the pangs of hungry people unsatisfied, \\
\poemll    and depriving thirsty people of drink. \\
\poeml \v{7}Furthermore, the crimes of bad people are evil; \\
\poemll    and\fnote{\fbackref{32:7} So 1QIsa\textsuperscript{a} LXX; the Heb. lacks \fbib{and}} they devise wicked schemes, \\
\poeml destroying the poor\fnote{\fbackref{32:7} 1QIsa\textsuperscript{a} and MT use two different synonyms} with lying words, \\
\poemll    even when needy people plead\fnote{\fbackref{32:7} So 1QIsa\textsuperscript{a}; cf. LXX; MT reads \fbib{a needy one pleads}} a just cause. \\
\poeml \v{8}But those who are decent plan noble things, \\
\poemll    and by noble deeds they stand.''
\passage{A Rebuke for Complacent Women}
\poeml \v{9}``As for you ladies of leisure--- \\
\poemll    Get up and listen to my voice! \\
\poeml You daughters who feel so complacent--- \\
\poemll    hear what I have to say! \\
\poeml \v{10}In little more than a year, \\
\poemll    you complacent women will shudder; \\
\poeml for the grape harvest will fail, \\
\poemll    and the fruit harvest will not\fnote{\fbackref{32:10} So 1QIsa\textsuperscript{a}; MT reads \fbib{without}} come. \\
\poeml \v{11}So tremble, you ladies of leisure! \\
\poemll    Shudder, you daughters who feel so complacent! \\
\poeml Strip down and make yourselves naked down to the waist!\fnote{\fbackref{32:11} Lit. \fbib{the loins}; so 1QIsa\textsuperscript{a} LXX; MT reads \fbib{to loins}} \\
\poemll    Then wrap yourself in\fnote{\fbackref{32:11} 1QIsa\textsuperscript{a} MT LXX lack \fbib{yourself in}} sackcloth and beat your breasts.\fnote{\fbackref{32:11} So 1QIsa\textsuperscript{a}; MT LXX lack \fbib{and beat your breasts}} \\
\poeml \v{12}For people will be beating their breasts \\
\poemll    in mourning\fnote{\fbackref{32:12} 1QIsa\textsuperscript{a} MT lack \fbib{mourning}} over the pleasant fields, \\
\poemlll       over the fruitful vines, \\
\poeml \v{13}and over the land of my people \\
\poemll    overgrown with thorns and\fnote{\fbackref{32:13} So 1QIsa\textsuperscript{a} LXX; the Heb. lacks \fbib{and}} briers--- \\
\poeml yes, over all the houses of merriment \\
\poemll    and over this city of revelry. \\
\poeml \v{14}``For the palace will be abandoned, \\
\poemll    the noisy city deserted; \\
\poeml the citadel and watchtower \\
\poemll    will become barren wastes forever, \\
\poeml the delight of wild donkeys, \\
\poemll    and a pasture for\fnote{\fbackref{32:14} So 1QIsa\textsuperscript{a}; MT reads \fbib{of}; cf. LXX} flocks, \\
\poeml \v{15}until the Spirit from on high is poured upon us, \\
\poemll    and the desert becomes a fertile field, \\
\poemlll       and the fertile field seems like a forest.''
\passage{Restoration of God's Reign}
\poeml \v{16}``Then justice will live in the wilderness, \\
\poemll    and righteousness will dwell in the fertile field. \\
\poeml \v{17}The effect of righteousness will be peace, \\
\poemll    and the result of righteousness will be quietness and confidence forever. \\
\poeml \v{18}My people will live in peaceful dwellings, \\
\poemll    in secure homes and in undisturbed resting places. \\
\poeml \v{19}But it will hail when the forest comes down, \\
\poemll    and the wood\fnote{\fbackref{32:19} So 1QIsa\textsuperscript{a}; MT reads \fbib{the city}; LXX lacks \fbib{the wood}} will be leveled completely. \\
\poeml \v{20}How happy you will be, sowing your seed beside every stream, \\
\poemll    and\fnote{\fbackref{32:20} So 1QIsa\textsuperscript{a}; the Heb. lacks \fbib{and}} letting your\fnote{\fbackref{32:20} Lit. \fbib{letting the feet of your}} cattle and donkeys range freely!''
\end{poetry}
\labelchapt{33}
\passage{God's Judgment}

\begin{poetry}
\poeml \chapt{33}
\v{1}``How terrible it will be for you, destroyer, \\
\poemll    you who have not been destroyed yourself! \\
\poeml And how terrible it will be for you, traitor, \\
\poemll    one whom\fnote{\fbackref{33:1} So 1QIsa\textsuperscript{a} MT; MT\textsuperscript{mss} read \fbib{you whom}} people have not betrayed! \\
\poeml When you have sunk so low in\fnote{\fbackref{33:1} So 1QIsa\textsuperscript{a}; MT reads \fbib{stopped}} destroying others, \\
\poemll    you will be destroyed; \\
\poeml and when you have finished betraying, \\
\poemll    they will betray you.''
\passage{A Prayer for Grace}
\poeml \v{2}``\divine{Lord}, be gracious to us; we long for you; \\
\poemll    and\fnote{\fbackref{33:2} So 1QIsa\textsuperscript{a}; the Heb. lacks \fbib{and}} be our strength\fnote{\fbackref{33:2} Lit. \fbib{arm}} every morning, \\
\poemlll       our salvation in times of trouble. \\
\poeml \v{3}At the thunder of your voice, the peoples flee; \\
\poemll    at your silence,\fnote{\fbackref{33:3} So 1QIsa\textsuperscript{a}; MT reads \fbib{when you rise up}; LXX reads \fbib{from fear of you}} the nations scatter. \\
\poeml \v{4}Your plunder is gathered as when grasshoppers gather; \\
\poemll    just like\fnote{\fbackref{33:4} So 1QIsa\textsuperscript{a}; the Heb. lacks \fbib{just like}} locusts pounce, people have pounced\fnote{\fbackref{33:4} So 1QIsa\textsuperscript{a}; MT reads \fbib{people pounce}} on it. \\
\poeml \v{5}``The \divine{Lord} is exalted, for he lives on high; \\
\poemll    he has filled Zion with justice and righteousness. \\
\poeml \v{6}He will be a sure foundation for your times, \\
\poemll    abundance and salvation,\fnote{\fbackref{33:6} So 1QIsa\textsuperscript{a}; MT reads \fbib{of salvation}} wisdom and knowledge --- \\
\poemlll       the fear of the \divine{Lord} is Zion's treasure.''
\passage{Israel's Unenviable Plight}
\poeml \v{7}``Listen! Their brave men cry out in the streets; \\
\poemll    the envoys of peace weep bitterly. \\
\poeml \v{8}The highways are deserted; \\
\poemll    travelers have quit the road. \\
\poeml The enemy\fnote{\fbackref{33:8} Lit. \fbib{He}} has broken treaties; \\
\poemll    he despises their witnesses,\fnote{\fbackref{33:8} So 1QIsa\textsuperscript{a}; MT reads \fbib{cities}} \\
\poemlll       and respects no one. \\
\poeml \v{9}The land mourns and wastes away; \\
\poemll    Lebanon feels ashamed and withers. \\
\poeml Sharon is like a desert; \\
\poemll    Bashan\fnote{\fbackref{33:9} So 1QIsa\textsuperscript{a}; MT reads \fbib{and Bashan}} and Carmel shake off their leaves.''
\passage{God is Exalted}
\poeml \v{10}``Now I'll rise up,'' the \divine{Lord} has said,\fnote{\fbackref{33:10} So 1QIsa\textsuperscript{a}; MT LXX read \fbib{says the \divine{Lord}}} \\
\poemll    ``now I'll exalt myself; \\
\poemlll       now I'll be lifted up. \\
\poeml \v{11}You conceive dried grass, you give birth to stubble; \\
\poemll    your breath is a fire that will consume you. \\
\poeml \v{12}And the peoples will be burned as if to ashes; \\
\poemll    like cut thorn bushes, they will be set ablaze. \\
\poeml \v{13}``Those who are far away have heard\fnote{\fbackref{33:13} So 1QIsa\textsuperscript{a}; cf. LXX; MT reads \fbib{You who are far away, hear}} what I've done; \\
\poemll    and those that are near have acknowledged\fnote{\fbackref{33:13} So 1QIsa\textsuperscript{a} LXX; MT reads \fbib{you that are near, acknowledge}} my power. \\
\poeml \v{14}The sinners in Zion are terrified; \\
\poemll    trembling grips the godless: \\
\poeml ``Who among us can live with the consuming fire? \\
\poemll    Who among us can live with everlasting flames?'' \\
\poeml \v{15}The one who walks righteously and has spoken\fnote{\fbackref{33:15} So 1QIsa\textsuperscript{a}; MT LXX read \fbib{and who speaks}} sincere words, \\
\poemll    who rejects gain from extortion \\
\poeml and waves his hand, \\
\poemll    rejecting bribes, \\
\poemll    who blocks his ears from hearing plots of murder \\
\poemlll       and shuts his eyes against seeing evil--- \\
\poeml \v{16}this is the one who will live on the heights; \\
\poemll    his refuge will be a mountain fortress. \\
\poeml His food will be supplied, \\
\poemll    and his water will be guaranteed. \\
\poeml \v{17}``Your eyes will see the king in his elegance, \\
\poemll    and will view a land that stretches afar. \\
\poeml \v{18}Your mind will ponder at that time of terror: \\
\poemll    `Where is the king's accountant? \\
\poeml Where is the one who weighed the revenue? \\
\poemll    Where is the officer who supervises\fnote{\fbackref{33:18} Lit. \fbib{counts}} the towers?' \\
\poeml \v{19}No longer will you\fnote{\fbackref{33:19} So 1QIsa\textsuperscript{a} (pl.); MT (sing.)} see those arrogant people, \\
\poemll    those people with their obscure speech you cannot comprehend, \\
\poemll    stammering in a language you cannot understand. \\
\poeml \v{20}``Look at Zion, city of our festivals!\fnote{\fbackref{33:20} So 1QIsa\textsuperscript{a} MT\textsuperscript{mss}; MT reads \fbib{our festival}} \\
\poemll    Your eyes will see Jerusalem, \\
\poemlll       an undisturbed abode, an immovable tent; \\
\poeml its stakes will never be pulled up, \\
\poemll    nor will any of its ropes be broken. \\
\poeml \v{21}But there the \divine{Lord} in majesty will be for us \\
\poemll    our source\fnote{\fbackref{33:21} Lit. \fbib{us a place}} of broad rivers and streams, \\
\poeml where no galley with oars can go, \\
\poemll    where no stately ship can sail. \\
\poeml \v{22}For the \divine{Lord} is our judge, \\
\poemll    and the \divine{Lord} is our lawgiver; \\
\poeml and the \divine{Lord} is our king, \\
\poemll    and it is he who will save us. \\
\poeml \v{23}``Your rigging hangs loose; \\
\poemll    it cannot reliably\fnote{\fbackref{33:23} So 1QIsa\textsuperscript{a}; MT reads \fbib{firmly}} hold the mast in its place, \\
\poemlll       and the sail cannot spread out.\fnote{\fbackref{33:23} So 1QIsa\textsuperscript{a}; MT reads \fbib{they cannot spread the sail}} \\
\poeml Then an abundance of spoils will be divided --- \\
\poemll    even the lame will carry off plunder. \\
\poeml \v{24}And no one living there will say, `I am ill.' \\
\poemll    The people living there will have their sins forgiven.''
\end{poetry}
\labelchapt{34}
\passage{Judgment of the Nations}

\begin{poetry}
\poeml \chapt{34}
\v{1}``Come near, you nations, to listen, \\
\poemll    and pay attention, you peoples! \\
\poeml Let the earth hear, and all that fills it; \\
\poemll    the world, and all that comes out of it. \\
\poeml \v{2}For the \divine{Lord} is angry against all the nations, \\
\poemll    and furious against all their armies. \\
\poeml He has doomed them to destruction, \\
\poemll    and\fnote{\fbackref{34:2} So 1QIsa\textsuperscript{a}; the Heb. lacks \fbib{and}} given them up to be slaughtered.\fnote{\fbackref{34:2} So 1QIsa\textsuperscript{a}; MT LXX read \fbib{for slaughter}} \\
\poeml \v{3}Their slain\fnote{\fbackref{34:3} Or \fbib{mortally wounded}} will be thrown out; \\
\poemll    and as for their dead bodies--- \\
\poeml their stench will ascend; \\
\poemll    the\fnote{\fbackref{34:3} So 1QIsa\textsuperscript{a}; the Heb. lacks \fbib{the}} mountains will be soaked with their blood. \\
\poeml \v{4}The valleys will be split, \\
\poemll    all the stars\fnote{\fbackref{34:4} Lit. \fbib{host}} in the heavens will fall down,\fnote{\fbackref{34:4} So 1QIsa\textsuperscript{a}; MT reads \fbib{All the stars of the heavens will rot away} ; LXX lacks this line} \\
\poemlll       and the skies will be rolled up like a scroll. \\
\poeml All their starry host will fade away \\
\poemll    like leaves withering on a vine, \\
\poemlll       or fruit withering on a fig tree. \\
\poeml \v{5}For my sword will be seen\fnote{\fbackref{34:5} So 1QIsa\textsuperscript{a}; MT reads \fbib{has drunk its fill}; cf. LXX} in the heavens. \\
\poemll    Look! It descends in judgment on Edom, \\
\poemlll       on the people I have doomed to destruction. \\
\poeml \v{6}The \divine{Lord} has a sword bathed in blood; \\
\poemll    it's covered\fnote{\fbackref{34:6} Or \fbib{satiated}} with fat, \\
\poeml with the blood of lambs and goats, \\
\poemll    and with fat from the kidneys of rams.''
\passage{Judgment on Bozrah and Edom}
\poeml ``For the \divine{Lord} holds a sacrifice in Bozrah, \\
\poemll    and a great slaughter in the land of Edom. \\
\poeml \v{7}Wild oxen will fall together with them--- \\
\poemll    young steers and mighty bulls. \\
\poeml Their land will be drenched\fnote{\fbackref{34:7} So 1QIsa\textsuperscript{a} LXX; MT reads \fbib{drench}} with blood, \\
\poemll    and their soil will be swollen with fat. \\
\poeml \v{8}For the \divine{Lord} has a day of vengeance, \\
\poemll    a year of recompense for Zion's cause. \\
\poeml \v{9}Edom's\fnote{\fbackref{34:9} Lit. \fbib{Its}} streams will be turned into burning sulfur, \\
\poemll    and its dust into sulfur; \\
\poemlll       its land will become pitch. \\
\poeml \v{10}It will burn night and day, \\
\poemll    and will never be extinguished. \\
\poeml Its smoke will rise from generation to generation, \\
\poemll    and it will lie desolate forever and ever. \\
\poemlll       And no one will pass through it.\fnote{\fbackref{34:10} So 1QIsa\textsuperscript{a}; MT reads \fbib{blazing pitch}. \fbib{\v{10}Night and day it will not be extinguished. Its smoke will go up forever; from generation to generation it will lie desolate. No one will ever pass through it again}. LXX lacks the last line} \\
\poeml \v{11}``But hawks and hedgehogs will possess it; \\
\poemll    owls and ravens will nest in it. \\
\poeml God\fnote{\fbackref{34:11} Lit. \fbib{He}} will stretch out over it a measuring line, and chaos,\fnote{\fbackref{34:11} So 1QIsa\textsuperscript{a}; MT reads \fbib{a measuring line of chaos}} \\
\poemll    and plumb lines of emptiness, and its nobles.\fnote{\fbackref{34:11} So 1QIsa\textsuperscript{a}; MT reads \fbib{And plumb lines of emptiness are} \fbib{\v{12}its} nobles; LXX reads \fbib{and satyrs will live in it}. \fbib{\v{12}Its nobles}} \\
\poeml \v{12}They will name it ``No Kingdom There,'' \\
\poemll    and all its princes will come to nothing. \\
\poeml \v{13}Thorns will grow over its palaces, \\
\poemll    nettles and brambles its fortresses. \\
\poeml It will become a haunt for jackals, \\
\poemll    a home for ostriches. \\
\poeml \v{14}And desert creatures will meet with hyenas, \\
\poemll    and goat-demons will call out to each other. \\
\poeml There also Liliths\fnote{\fbackref{34:14} I.e. desert demons of the night} will settle, \\
\poemll    and find for themselves\fnote{\fbackref{34:14} So 1QIsa\textsuperscript{a}; MT reads \fbib{The Lilith will settle, and find for itself}} a resting place. \\
\poeml \v{15}Owls\fnote{\fbackref{34:15} Or \fbib{tree-snakes}; LXX reads \fbib{Hedgehogs}} will nest there, lay eggs, \\
\poemll    hatch them, and care for their young \\
\poemlll       under the shadow of their wings;\fnote{\fbackref{34:15} Lit. \fbib{in her shadow}} \\
\poeml yes\fnote{\fbackref{34:15} So 1QIsa\textsuperscript{a}; MT lacks \fbib{yes}} indeed, vultures will gather there, \\
\poemll    each one with its mate.''
\passage{The Certainty of God's Deliverance}
\poeml \v{16}``Study and read from the book of the \divine{Lord}: \\
\poemll    And not one\fnote{\fbackref{34:16} So 1QIsa\textsuperscript{a}; MT reads \fbib{Not one of them}} will be missing, \\
\poemlll       each will not long for its mate.\fnote{\fbackref{34:16} So MT; 1QIsa\textsuperscript{a} reads \fbib{each its mate}} \\
\poeml For it is the mouth of the \divine{Lord} that has issued the order, \\
\poemll    and it is his Spirit that has gathered them. \\
\poeml \v{17}It is he who has allotted their portions,\fnote{\fbackref{34:17} Lit. \fbib{has cast the lot for them}} \\
\poemll    and his hand has divided it for them with a measuring line forever.\fnote{\fbackref{34:17} So 1QIsa\textsuperscript{a}; the Heb. lacks \fbib{forever}} \\
\poeml They will possess it forever;\fnote{\fbackref{34:17} So MT; 1QIsa\textsuperscript{a} lacks \fbib{Forever}} \\
\poemll    from generation to generation they will live in it.''\fnote{\fbackref{34:17} So MT; 1QIsa\textsuperscript{a} corrector reads \fbib{will they possess {\ldots} live in it}.}
\end{poetry}
\labelchapt{35}
\passage{The Future of Israel's Land}

\begin{poetry}
\poeml \chapt{35}
\v{1}``The desert and the dry land will rejoice; \\
\poemll    the desert will celebrate and blossom. Like crocuses, \\
\poeml \v{2}it will burst into bloom, \\
\poemll    and rejoice with gladness and shouts of joy. \\
\poeml The glory of Lebanon will be given to it, \\
\poemll    the splendor of Carmel and Sharon. \\
\poeml They will see the glory of the \divine{Lord}, \\
\poemll    the splendor of our God.\fnote{\fbackref{35:2} So MT LXX 1QIsa\textsuperscript{a} corrector; 1QIsa\textsuperscript{a} lacks vss. 1-2} \\
\poeml \v{3}Strengthen the feeble hands, \\
\poemll    and support the stumbling knees. \\
\poeml \v{4}Say to those with anxious hearts, \\
\poemll    `Be strong, do not be afraid! \\
\poeml Here is your God--- \\
\poemll    he will bring\fnote{\fbackref{35:4} So 1QIsa\textsuperscript{a} LXX; MT reads \fbib{he will come}} vengeance, \\
\poeml he will bring\fnote{\fbackref{35:4} So 1QIsa\textsuperscript{a}; MT LXX read \fbib{he will come}} divine retribution, \\
\poemll    and he will save you.' \\
\poeml \v{5}``Then the eyes of the blind will be opened, \\
\poemll    and the ears of the deaf unblocked; \\
\poeml \v{6}then the lame will leap like deer, \\
\poemll    and the tongues of speechless people will sing for joy. \\
\poeml Yes, waters will gush forth in the wilderness, \\
\poemll    and streams will run\fnote{\fbackref{35:6} So 1QIsa\textsuperscript{a}; MT LXX lack \fbib{will run}} through the desert; \\
\poeml \v{7}the burning sands will become a pool, \\
\poemll    and the thirsty ground fountains of water. \\
\poeml In the haunts of jackals there will be \\
\poemll    a verdant resting place with\fnote{\fbackref{35:7} So 1QIsa\textsuperscript{a}; MT reads \fbib{is her resting place; the grass will become}} reeds and rushes.''
\passage{God's Holy Highway}
\poeml \v{8}``A highway will be there---yes, there---\fnote{\fbackref{35:8} So 1QIsa\textsuperscript{a}; MT LXX lack \fbib{yes, there}} \\
\poemll    and people will call it\fnote{\fbackref{35:8} So 1QIsa\textsuperscript{a}; MT LXX read \fbib{it will be called}} `The Holy Way'.\fnote{\fbackref{35:8} So 1QIsa\textsuperscript{a} LXX; MT reads \fbib{Way, yes, Way}} \\
\poeml As for unclean people, \\
\poemll    they will not journey on it, \\
\poeml but it will be for whomever\fnote{\fbackref{35:8} So 1QIsa\textsuperscript{a}; MT reads \fbib{but it will be for the one}; cf. LXX} is traveling on that Way--- \\
\poemlll       not even fools will get lost. \\
\poeml \v{9}No lions will be there--- \\
\poemll    no---\fnote{\fbackref{35:9} So 1QIsa\textsuperscript{a}; MT LXX lack \fbib{no}} nor will any ferocious beasts get up on it, \\
\poemlll       and\fnote{\fbackref{35:9} So 1QIsa\textsuperscript{a} LXX; the Heb. lacks \fbib{and}} they will not be found there. \\
\poeml ``But the redeemed will walk there, \\
\poeml \v{10}and the \divine{Lord}'s ransomed ones will return \\
\poemlll       and enter Zion with singing. \\
\poeml Everlasting joy will rest upon their heads, \\
\poemll    gladness and joy will overtake them,\fnote{\fbackref{35:10} So 1QIsa\textsuperscript{a}; 1QIsa\textsuperscript{a} corrector lacks \fbib{them}; MT reads \fbib{they will attain gladness and joy}} \\
\poemlll       and sorrow and mourning will flee away.''
\end{poetry}
\labelchapt{36}
\passage{Sennacherib Attacks}

\chapt{36}
\v{1}In the fourteenth year of King Hezekiah,\fnote{\fbackref{36:1} The Heb. name \fbib{Hezekiah} is usu. spelled \fbib{Hizqiyah} in 1QIsa\textsuperscript{a}; 4QIsa\textsuperscript{b} MT spell the name \fbib{Hizqiyahu}.} King Sennacherib of Assyria attacked all the fortified cities of Judah and captured them. \v{2}Then the king of Assyria sent his field commander,\fnote{\fbackref{36:2} Or \fbib{sent Rab-shakeh}} along with a very\fnote{\fbackref{36:2} So 1QIsa\textsuperscript{a}; MT LXX lack \fbib{very}} large army, from Lachish to King Hezekiah at Jerusalem. When the field commander stopped at the aqueduct at the Upper Pool on the road to Laundryman's Field, \v{3}Hilkiah's son Eliakim, who was in charge of the palace, Shebna the secretary, and Asaph's son Joah, the recorder, went out to him.

\v{4}The field commander told them:

\begin{poetry}
\poeml ``Tell Hezekiah, king of Judah,\fnote{\fbackref{36:4} So 1QIsa\textsuperscript{a}; 1QIsa\textsuperscript{a} corrector deleted \fbib{king of Judah}; MT LXX lack \fbib{king of Judah}} `This is what the mighty king, the king of Assyria, has to say: What is this ``guarantee'' that makes you yourself\fnote{\fbackref{36:4} So 1QIsa\textsuperscript{a}; MT LXX lack \fbib{yourself}} rely on it?\fnote{\fbackref{36:4} So 1QIsa\textsuperscript{a}; MT LXX lack \fbib{on it}} \v{5}Do you really think that guarantees alone can withstand\fnote{\fbackref{36:5} Lit. \fbib{that words alone equal}} strategy and military strength? On whom are you now depending, that you're rebelling against me? \v{6}Take note: you're relying on Egypt, that splintered reed of a staff, which pierces the palm of anyone who leans on it. This is what Pharaoh king of Egypt is like to everybody who depends on him! \\
\poeml \v{7}But if you all\fnote{\fbackref{36:7} So 1QIsa\textsuperscript{a} LXX; MT reads \fbib{you} (sing.)} say to me, ``We are depending on the \divine{Lord} our God''---isn't he the one whose high places and altars Hezekiah removed, while he kept on telling Judah and Jerusalem, `You are to worship in front of this altar in\fnote{\fbackref{36:7} So 1QIsa\textsuperscript{a} MT; LXX lacks \fbib{while he kept on telling Judah and Jerusalem, `You are to worship in front of this altar in Jerusalem'}} Jerusalem'?\fnote{\fbackref{36:7} So 1QIsa\textsuperscript{a}; 1QIsa\textsuperscript{a} corrector deleted \fbib{in Jerusalem}; the Heb. lacks \fbib{in Jerusalem}} \v{8}Come now, all of you,\fnote{\fbackref{36:8} So 1QIsa\textsuperscript{a} LXX; MT reads \fbib{you} (sing.)} make a bet with my master, the king of Assyria: I will give you two thousand horses, if you can furnish riders for them! \v{9}How, then, can you repulse even one officer from\fnote{\fbackref{36:9} So 1QIsa\textsuperscript{a}; MT reads \fbib{one of}} the least of my master's officials, when you are depending for yourselves\fnote{\fbackref{36:9} So 1QIsa\textsuperscript{a}; MT reads \fbib{yourself}} on Egypt for chariots and horsemen? \v{10}One other thing: have I really marched against this country to destroy it apart from the \divine{Lord}'s direction?\fnote{\fbackref{36:10} 1QIsa\textsuperscript{a} MT lack `\fbib{s direction}} The \divine{Lord} himself ordered me, `March against this country to\fnote{\fbackref{36:10} So 1QIsa\textsuperscript{a}; MT reads \fbib{and}} destroy it.'\,''\fnote{\fbackref{36:10} So 1QIsa\textsuperscript{a} MT; LXX lacks \fbib{The \divine{Lord} himself ordered me, `March against this country to destroy it.'}}
\end{poetry}

\v{11}Then Eliakim, Shebna, and Joah replied to him,\fnote{\fbackref{36:11} So 1QIsa\textsuperscript{a} LXX; MT reads \fbib{to the field commander}} ``Please speak with\fnote{\fbackref{36:11} So 1QIsa\textsuperscript{a}; MT reads \fbib{to}} your servants---with us\fnote{\fbackref{6:11} So 1QIsa\textsuperscript{a}; MT LXX lack ---\fbib{with us---}}---in Aramaic, since we understand it. Don't speak to us in Hebrew\fnote{\fbackref{36:11} Lit. \fbib{in these words}; \fbib{s} o 1QIsa\textsuperscript{a}; MT LXX read \fbib{in the Judean language}} where the people sitting on\fnote{\fbackref{36:11} So 1QIsa\textsuperscript{a}; the Heb. lacks \fbib{sitting}; cf. LXX} the wall can hear.''

\v{12}But the field commander asked, ``Was it only to all of you and to your\fnote{\fbackref{36:12} So 1QIsa\textsuperscript{a} (pl.); MT reads \fbib{your} (sing.) \fbib{master and to you} (sing.)} master that my master sent me to speak these things? Wasn't it also to the men sitting on the wall---who, like you, will have to eat their own excrement and drink their own urine?''

\v{13}Then the\fnote{\fbackref{36:13} So 1QIsa\textsuperscript{a}; the Heb. lacks \fbib{the}} commander stood up and shouted out loud in Hebrew:\fnote{\fbackref{36:13} Or \fbib{the Judean language}}

\begin{poetry}
\poeml ``Hear the words of the great king, the king of Assyria! \v{14}This is what the king of Assyria\fnote{\fbackref{36:14} So 1QIsa\textsuperscript{a}; MT LXX lack \fbib{of Assyria}} says: `Don't let Hezekiah deceive you---for he cannot save you! \v{15}Don't let Hezekiah persuade you to rely on the \divine{Lord} when he says, ``The \divine{Lord} will really deliver\fnote{\fbackref{36:15} Or \fbib{save}} us!'' and\fnote{\fbackref{36:15} So 1QIsa\textsuperscript{a} LXX; MT lacks \fbib{and}} ``This city will never be handed over to the king of Assyria!'' \v{16}Don't listen to Hezekiah, because this is what the king of Assyria says: `Make your peace with me and come out to me. Then everyone will eat from his own vine and from his own fig tree, and everyone will drink water from his own cistern, \v{17}until I come and take you away to a land like your own land---to\fnote{\fbackref{36:17} So 1QIsa\textsuperscript{a}; the Heb. lacks \fbib{to}} a land of grain and new wine, a land of bread and vineyards.' \v{18}Be careful not to let Hezekiah mislead you when he says, ``The \divine{Lord} will save us.'' Has any god of any nation ever delivered\fnote{\fbackref{36:18} Or \fbib{saved}} his country from the\fnote{\fbackref{36:18} Lit. \fbib{the hand of the}} king of Assyria? \v{19}Where are the gods of Hamath and Arpad? Where are the gods of Sephar-vaim? Have they saved Samaria from me?\fnote{\fbackref{36:19} Lit. \fbib{from my hand}} \v{20}Who among all the gods of these countries has delivered\fnote{\fbackref{36:20} Or \fbib{saved}} their land from me?\fnote{\fbackref{36:20} Lit. \fbib{from my hand}} How then can the \divine{Lord} deliver\fnote{\fbackref{36:20} Or \fbib{saved}} Jerusalem from me?'\,''\fnote{\fbackref{36:20} Lit. \fbib{from my hand}}
\end{poetry}

\v{21}But the people remained silent and didn't respond to him with so much as a single word, because the king had commanded, ``Don't answer him.''

\v{22}Then Hilkiah's son Eliakim, who was in charge of the palace, Shebna the secretary, and Asaph's son Joah, the recorder, approached Hezekiah with their clothes torn,\fnote{\fbackref{36:22} I.e. as a symbol of pending disaster} and let him know what the field commander had said.
\labelchapt{37}
\passage{Hezekiah Seeks Isaiah's Counsel}

\chapt{37}
\v{1}As soon as Hezekiah the king\fnote{\fbackref{37:1} So 1QIsa\textsuperscript{a}; MT LXX read \fbib{the king Hezekiah}} heard this, he tore his clothes, dressed himself in sackcloth, and went into the \divine{Lord}'s Temple. \v{2}Then he sent Eliakim, who was in charge of the palace, Shebna the secretary, and the senior priests, all wearing sackcloth, to Amoz's son, the prophet Isaiah. \v{3}``Here is what Hezekiah says,'' they told him. ``This day is a day of trouble, rebuke, and disgrace, as when children come to the point of birth and there is no energy to deliver them. \v{4}Perhaps the \divine{Lord} your God will hear the words of the field commander, whom his master, the king of Assyria, sent to mock the living God, and perhaps he will rebuke the words that the \divine{Lord} your God has heard. So lift up a prayer for the remnant that still survives in this city.''\fnote{\fbackref{37:4} So 1QIsa\textsuperscript{a}; MT LXX lack \fbib{in this city}} \v{5}That's why King Hezekiah's officials came to Isaiah.
\passage{Isaiah Responds to Hezekiah}

\v{6}``Here is what to tell your master,'' Isaiah told them. ``This is what the \divine{Lord} says: `Don't be afraid of the words you've heard---those words with which the underlings of the king of Assyria have insulted me. \v{7}Watch this! I'm going to place an attitude\fnote{\fbackref{37:7} Or \fbib{to put a spirit}} within him,\fnote{\fbackref{37:7} So 1QIsa\textsuperscript{a}; MT LXX read \fbib{put a spirit in him}} so that when he hears a certain report, he'll return to his own country. Then I'll have him cut down by the sword in his own land.''\fnote{\fbackref{37:5-7} So MT LXX 1QIsa\textsuperscript{a} corrector; 1QIsa\textsuperscript{a} lacks vss. 5-7}
\passage{Sennacherib Retreats}

\v{8}So the field commander returned and found the king of Assyria fighting against Libnah, since he had heard that the king of Assyria\fnote{\fbackref{37:8} Lit. \fbib{that he}} had left Lachish. \v{9}Now King Sennacherib\fnote{\fbackref{37:9} Lit. \fbib{Now he}} had received this report concerning King Tirhakah of Cush: ``He has marched out to fight against you.''

When he heard it, he returned and\fnote{\fbackref{37:9} So 1QIsa\textsuperscript{a} LXX; cf. 2Kgs 19:9 MT; the Heb. lacks \fbib{returned and}} sent messengers to Hezekiah: \v{10}``Say this to Hezekiah king of Judah: `Don't let your God on whom you depend deceive you when he says, ``Jerusalem will not be handed over to the king of Assyria.'' \v{11}Surely you have heard what the kings of Assyria have done to all countries, dooming them to destruction. So do you think you will be saved? \v{12}Did the gods of the nations that were destroyed by my ancestors save them---the nations of Gozan, Haran, Rezeph, and the people of Eden, who were in Tel-assar? \v{13}Where is the king of Hamath, the king of Arpad, the king of the city of Sephar-vaim, or of Hena, or of Ivvah, or of Samaria?'\,''\fnote{\fbackref{37:13} So 1QIsa\textsuperscript{a}; MT LXX lack \fbib{or of Samaria}}
\passage{Hezekiah Prays}

\v{14}Hezekiah received the letters from the messengers, and read them.\fnote{\fbackref{37:14} So 1QIsa\textsuperscript{a}; MT LXX read \fbib{it}} Then he\fnote{\fbackref{37:14} Lit. \fbib{Hezekiah}} went up to the \divine{Lord}'s Temple and spread the letters\fnote{\fbackref{37:14} Lit. \fbib{it}} in front of the \divine{Lord}. \v{15}Hezekiah prayed to the \divine{Lord}:

\begin{poetry}
\poeml \v{16}``O \divine{Lord} of the Heavenly Armies, God of Israel, enthroned above the cherubim, you alone are the God of all the kingdoms of the earth. You made heaven and earth. \v{17}Extend your ear, \divine{Lord}, and listen! Open your eyes, \divine{Lord}, and look! Listen to all the words Sennacherib has sent to mock the living God. \v{18}It is true, \divine{Lord}, that Assyrian kings have devastated all these countries,\fnote{\fbackref{37:18} So 1QIsa\textsuperscript{a}; MT reads \fbib{countries and their land}; some MT\textsuperscript{mss} read \fbib{nations and their land}} \v{19}and have thrown their gods into the fire---but they are not gods, but rather the products\fnote{\fbackref{37:19} So 1QIsa\textsuperscript{a} LXX; MT reads \fbib{work}} of human hands, mere wood and stone. So the Assyrians\fnote{\fbackref{37:19} Lit. \fbib{So they}} destroyed them. \v{20}So now, \divine{Lord} our God, save us from his oppressive\fnote{\fbackref{37:20} 1QIsa\textsuperscript{a} LXX MT lack \fbib{oppressive}} hand, so that all kingdoms on earth may know that you alone, \divine{O Lord}, are God.''\fnote{\fbackref{37:20} So 1QIsa\textsuperscript{a}; MT reads \fbib{alone are \divine{Lord}}; LXX reads \fbib{alone are God}}
\end{poetry}
\passage{God's Answer}

\v{21}Then Amoz's son Isaiah sent this message to Hezekiah: ``This is what the \divine{Lord}, the God of Israel, says, to whom you prayed\fnote{\fbackref{37:21} So 1QIsa\textsuperscript{a}; MT reads \fbib{because you prayed to me}; cf. LXX} concerning Sennacherib king of Assyria. \v{22}This is the message that the \divine{Lord} has spoken in opposition to him:

\begin{poetry}
\poeml ```The Virgin Daughter of Zion \\
\poemll    despises and mocks you; \\
\poeml the Daughter of Jerusalem--- \\
\poemll    she tosses her head behind you as you flee. \\
\poeml \v{23}Whom have you insulted and reviled? \\
\poemll    Against whom have you raised your voice \\
\poeml and lifted your eyes in pride? \\
\poemll    Against the Holy One of Israel! \\
\poeml \v{24}By your messengers\fnote{\fbackref{37:24} Lit. \fbib{servants}} you have insulted the \divine{Lord}, \\
\poemll    and you have said, \\
\poeml ``With my many chariots \\
\poemll    I have climbed the heights of mountains, \\
\poemlll       the utmost heights of Lebanon. \\
\poeml I cut down its tallest cedars, \\
\poemll    the choicest of its pines; \\
\poeml I reached its remotest heights, \\
\poemll    the most verdant of its forests. \\
\poeml \v{25}I myself dug wells\fnote{\fbackref{37:25} So 1QIsa\textsuperscript{a}; MT reads \fbib{dug}; LXX reads \fbib{appointed}} \\
\poemll    and drank foreign\fnote{\fbackref{37:25} So 1QIsa\textsuperscript{a}; MT LXX lack \fbib{foreign}} waters; \\
\poeml with the soles of my feet \\
\poemll    I dried up all the streams of Egypt.'' \\
\poeml \v{26}```Didn't you hear \\
\poemll    how in the distant past I decided to do it, \\
\poemlll       how\fnote{\fbackref{37:26} So 1QIsa\textsuperscript{a}; MT reads \fbib{and how}} I planned from days of old? \\
\poeml Now I've made it happen--- \\
\poemll    that fortified cities become devastated, besieged heaps.\fnote{\fbackref{37:26} So 1QIsa\textsuperscript{a}; MT reads \fbib{you should make fortified cities crash into ruined heaps}} \\
\poeml \v{27}Their inhabitants are devoid of power, \\
\poemll    and are terrified and put to shame. \\
\poeml They've become like plants in the field, \\
\poemll    like\fnote{\fbackref{37:27} So 1QIsa\textsuperscript{a}; MT reads \fbib{and like}} green shoots, \\
\poeml like grass on rooftops, \\
\poemll    scorched by the east wind.\fnote{\fbackref{37:27} So 1QIsa\textsuperscript{a}; MT reads \fbib{and a field before the standing grain}} \\
\poeml \v{28}```I know when you rise up \\
\poemll    and\fnote{\fbackref{37:28} So 1QIsa\textsuperscript{a}; MT LXX lack \fbib{when you rise up and}} when you sit down, \\
\poeml your comings and goings--- \\
\poemll    and how you've become enraged at me. \\
\poeml \v{29}Your insolence\fnote{\fbackref{37:29} So 1QIsa\textsuperscript{a}; MT reads \fbib{because your raging against me and your insolence}; cf. LXX} has reached my ears, \\
\poemll    so I'll put my hook in your nose \\
\poemlll       and my bit in your mouth,\fnote{\fbackref{37:29} Lit. \fbib{lips}; so 1QIsa\textsuperscript{a} LXX; MT reads \fbib{lip}} \\
\poeml and I'll make you turn back on the road \\
\poemll    by which you came.
\end{poetry}

\v{30}``And this will be your sign, Hezekiah:\fnote{\fbackref{37:30} So 1QIsa\textsuperscript{a}; the Heb. lacks \fbib{Hezekiah}} Eat this year what grows on its own, and in the second year what springs from that. But in the third year sow, reap, plant vineyards, and eat their fruit. \v{31}Then the ones belonging to the house of Judah who have escaped will gather,\fnote{\fbackref{37:31} So 1QIsa\textsuperscript{a}; MT reads \fbib{be increased}} and those who are found\fnote{\fbackref{37:31} So 1QIsa\textsuperscript{a}; MT reads \fbib{and the remainder}; cf. LXX} will take root downward and bear fruit upward. \v{32}For a remnant will come out of Zion,\fnote{\fbackref{37:32} So 1QIsa\textsuperscript{a}; 4QIsa\textsuperscript{b} MT LXX read \fbib{Jerusalem}} and a band of survivors from Jerusalem.\fnote{\fbackref{37:32} So 1QIsa\textsuperscript{a}; 4QIsa\textsuperscript{b} MT LXX read \fbib{Mount Zion}} The zeal of the \divine{Lord} of the Heavenly Armies will accomplish this.

\v{33}``Therefore this what the \divine{Lord} says concerning the king of Assyria: `He won't enter this city, build up a siege ramp against it, shoot an arrow here, or threaten it with a shield.\fnote{\fbackref{37:33} So 1QIsa\textsuperscript{a}; MT reads \fbib{or shoot an arrow here, or threaten it with a shield, or build up a siege ramp against it}} \v{34}By the same way that he came, he will return; he won't enter this city,' declares the \divine{Lord}, \v{35}`because I will defend this city and deliver\fnote{\fbackref{37:35} Or \fbib{save}} it, for my own sake and for the sake of my servant David!'\,''
\passage{Sennacherib is Defeated}

\v{36}After this, the angel of the \divine{Lord} went out and put to death 185,000 men in the Assyrian camp. When Hezekiah's army\fnote{\fbackref{37:36} Lit. \fbib{When the people}} awakened in the morning---there were all the dead bodies!

\v{37}King Sennacherib broke camp, retreated, returned home to Nineveh, and remained there. \v{38}Later, while he was worshiping in\fnote{\fbackref{37:38} So 1QIsa\textsuperscript{a} LXX; the Heb. lacks \fbib{in}} the house of his god Nisroch, his sons Adrammelech and Sharezer cut him down with swords and escaped to the land of Ararat. Then Sennacherib's\fnote{\fbackref{37:38} Lit. \fbib{his}} son Esar-haddon reigned in his place.
\labelchapt{38}
\passage{Hezekiah's Illness and Recovery}

\chapt{38}
\v{1}During that time,\fnote{\fbackref{38:1} Lit. \fbib{During those days}} Hezekiah became ill and was at the point of death. Then Amoz's son Isaiah the prophet came to him and told him, ``This is what the \divine{Lord} says: `Put your house in order, because you are going to die. You won't recover.'\,''

\v{2}Then Hezekiah turned his face to the wall and prayed to the \divine{Lord}. \v{3}``Please, \divine{Lord},'' he said, ``Remember how I have walked before you faithfully and with a true heart, and I have done what pleases you.''\fnote{\fbackref{38:3} Lit. \fbib{done what is good in your eyes}} And Hezekiah wept bitterly.

\v{4}Then this message\fnote{\fbackref{38:4} Lit. \fbib{Then the word}} from the \divine{Lord} came to Isaiah: \v{5}``Go tell Hezekiah, `This is what the \divine{Lord} God of your ancestor David has to say: ``I've heard your prayer and\fnote{\fbackref{38:5} So 1QIsa\textsuperscript{a} LXX; the Heb. lacks \fbib{and}} I've seen your tears; so I will add fifteen years to your life. \v{6}I'll save you and this city from the\fnote{\fbackref{38:6} Lit. \fbib{the hand of the}} king of Assyria, and I'll defend this city, for my own sake and for my servant David's sake.\fnote{\fbackref{38:6} So 1QIsa\textsuperscript{a}; MT LXX lack \fbib{for my own sake and for my servant David's sake}} \v{7}This is the \divine{Lord}'s sign to you that the \divine{Lord} will carry out this thing he has promised: \v{8}Watch! I will make the shadow on the steps of the upper\fnote{\fbackref{38:8} So 1QIsa\textsuperscript{a}; the Heb. lacks \fbib{upper}} dial of Ahaz that marks the sun go ten steps backwards.''\,'\,''

Then the sunlight turned back on the dial the ten steps by which it had gone down.
\passage{Hezekiah's Prayer}

\v{9}A composition by King Hezekiah of Judah, following his illness and recovery:

\begin{poetry}
\poeml \v{10}I said, ``Must I leave in the prime of my life? \\
\poemll    Must I be consigned to the control\fnote{\fbackref{38:10} Lit. \fbib{gates}; i.e. the place where legal cases were adjudicated} of Sheol?\fnote{\fbackref{38:10} I.e. the realm of the afterlife} \\
\poemlll       Bitter are\fnote{\fbackref{38:10} So 1QIsa\textsuperscript{a}; MT LXX read \fbib{the rest of}} my years!'' \\
\poeml \v{11}I said, ``I won't see the \divine{Lord}\fnote{\fbackref{38:11} Lit. \fbib{Yah}; So 1QIsa\textsuperscript{a}; MT reads \fbib{Yah Yah}; MT\textsuperscript{mss} read \fbib{Lord}} in the land of the living; \\
\poemll    and\fnote{\fbackref{38:11} So 1QIsa\textsuperscript{a}; the Heb. lacks \fbib{and}} I'll no longer observe human beings \\
\poemlll       among the denizens of the grave.\fnote{\fbackref{38:11} Lit. \fbib{cessation}; or \fbib{the end}; So 1QIsa\textsuperscript{a} MT; MT\textsuperscript{mss} read \fbib{the world}} \\
\poeml \v{12}My house has been plucked up and vanishes\fnote{\fbackref{38:12} So 1QIsa\textsuperscript{a}; 1QIsa\textsuperscript{b} MT read \fbib{and has been taken away}} from me \\
\poemll    like a shepherd's tent; \\
\poeml like a weaver, I've taken account of\fnote{\fbackref{38:12} So 1QIsa\textsuperscript{a}; MT reads \fbib{have rolled up}} my life, \\
\poemll    and he cuts me off from the loom--- \\
\poemlll       day and night you make an end of me. \\
\poeml \v{13}I've been swept bare\fnote{\fbackref{8:13} So 1QIsa\textsuperscript{a}; or \fbib{I cried for help}; MT reads \fbib{I was composed}; cf. Targ} until morning; \\
\poemll    just like a lion, he breaks all my bones--- \\
\poemlll       day and night you make an end of me. \\
\poeml \v{14}Like a swallow or a crane I chirp, \\
\poemll    I moan like a dove. \\
\poeml My eyes look weakly upward. \\
\poemll    O Lord,\fnote{\fbackref{38:14} So 1QIsa\textsuperscript{a} MT; 1QIsa\textsuperscript{b} reads \fbib{\divine{Lord}}} I am oppressed, so\fnote{\fbackref{38:14} So 1QIsa\textsuperscript{a}; the Heb. lacks \fbib{so}} stand up for me! \\
\poeml \v{15}What can I say, so I tell myself,\fnote{\fbackref{38:15} So 1QIsa\textsuperscript{a}; MT reads \fbib{for he has spoken to me}} \\
\poemll    since he has done this to me?\fnote{\fbackref{38:15} So 1QIsa\textsuperscript{a}; MT reads \fbib{and it is he who has done it}} \\
\poeml I will walk slowly all my years \\
\poemll    because of my soul's anguish. \\
\poeml \v{16}``My Lord is against them, yet they live, \\
\poemll    and among all of them who live is his spirit.\fnote{\fbackref{38:16} So 1QIsa\textsuperscript{a}; MT reads \fbib{is the life of my spirit}} \\
\poeml Now you have restored me to health, \\
\poemll    so let me live! \\
\poeml \v{17}Yes, it was for my own good \\
\poemll    that I suffered extreme anguish.\fnote{\fbackref{38:17} So 1QIsa\textsuperscript{a}; 1QIsa\textsuperscript{b} MT read \fbib{bitter, bitter}} \\
\poeml But in love you have held back\fnote{\fbackref{38:17} So 1QIsa\textsuperscript{a}; cf. LXX; MT reads \fbib{you have loved}} my life \\
\poemll    from the Pit\fnote{\fbackref{38:17} I.e. the realm of punishment in the afterlife} in which it has been confined;\fnote{\fbackref{38:17} So 1QIsa\textsuperscript{a}; MT reads \fbib{pit of destruction}} \\
\poeml you have tossed all my sins \\
\poemll    behind your back. \\
\poeml \v{18}For Sheol\fnote{\fbackref{38:18} I.e. the realm of the afterlife} cannot thank you, \\
\poemll    death cannot\fnote{\fbackref{38:18} So 1QIsa\textsuperscript{a} LXX; implied in 1QIsa\textsuperscript{b} MT} sing your praise; \\
\poeml and\fnote{\fbackref{38:18} So 1QIsa\textsuperscript{a}; the Heb. lacks \fbib{and}} those who go down to the Pit\fnote{\fbackref{38:18} I.e. the realm of punishment in the afterlife} cannot hope \\
\poemll    for your faithfulness. \\
\poeml \v{19}The living---yes the living---they thank you, \\
\poemll    just as I am doing today; \\
\poeml fathers will tell their children \\
\poemll    about your faithfulness. \\
\poeml \v{20}The \divine{Lord} will save me,\fnote{\fbackref{38:20} At this point a later scribe inserted into 1QIsa\textsuperscript{a} a repetition of v. 19 and the beginning of v. 20, but with some different spellings and a word missing.} \\
\poemll    and we will play my music on strings \\
\poeml all the days of our lives \\
\poemll    in the \divine{Lord}'s Temple.\fnote{\fbackref{38:20} The same second scribe continued with the rest of this verse; not originally in 1QIsa\textsuperscript{a}.}
\end{poetry}

\v{21}Now Isaiah had said, ``Let them prepare\fnote{\fbackref{38:21} So MT; LXX reads \fbib{Take}; 1QIsa\textsuperscript{a} lacks \fbib{Let them prepare}} a poultice of figs and apply it to the boil, so that he may recover.''

\v{22}Hezekiah also had asked, ``What will be the sign for me to go up to the \divine{Lord}'s Temple?''\fnote{\fbackref{38:21-22} So 1QIsa\textsuperscript{b} MT LXX; 1QIsa\textsuperscript{a} lacks vs. 21-22; a later, third scribe, includes vs. 21-22}
\labelchapt{39}
\passage{The Visit by Merodach-baladan}

\chapt{39}
\v{1}At that time Merodach-baladan, the son of Baladan, king of Babylon, sent letters and a gift to Hezekiah, when\fnote{\fbackref{39:1} So 1QIsa\textsuperscript{a} 1QIsa\textsuperscript{b} MT; 4QIsa\textsuperscript{b} LXX read \fbib{because}} he heard he had been sick and had survived.\fnote{\fbackref{39:1} So 1QIsa\textsuperscript{a}; 1QIsa\textsuperscript{b} MT read \fbib{had recovered}} \v{2}Hezekiah was delighted with them, and showed them everything in\fnote{\fbackref{39:2} So 1QIsa\textsuperscript{a} MT\textsuperscript{mss}; the Heb. lacks \fbib{in}} his treasure-houses\fnote{\fbackref{39:2} So 1QIsa\textsuperscript{a}; MT LXX read \fbib{treasure-house}; MT\textsuperscript{qere} reads \fbib{his treasure-house}}---the silver, the gold, the spices, the precious oils, his entire armory, and everything found in his treasuries. There was nothing in his palace or in all his kingdom\fnote{\fbackref{39:2} So 1QIsa\textsuperscript{a}; MT reads \fbib{realm}; LXX lacks \fbib{kingdom}} that Hezekiah did not show them.
\passage{Isaiah Rebukes Hezekiah}

\v{3}Then the prophet Isaiah came to King Hezekiah and asked him, ``What did these men have to say? And from where did they come to you?''

Hezekiah replied, ``From a distant land---they came to me from Babylon.''

\v{4}``What did they see in your palace?'' he asked.

``They saw everything in my palace,'' Hezekiah replied. ``There is nothing in my treasuries that I did not show them.''

\v{5}Then Isaiah told Hezekiah, ``Listen to this message\fnote{\fbackref{39:5} Lit. \fbib{word}} from the \divine{Lord} of the Heavenly Armies: \v{6}`The days are surely coming when everything in your palace and all that your ancestors have stored up to this day will be carried off\fnote{\fbackref{39:6} So 1QIsa\textsuperscript{a} (pl.); cf. LXX; 1QIsa\textsuperscript{b} MT (sing.)} to Babylon. They will come in, and\fnote{\fbackref{39:6} So 1QIsa\textsuperscript{a} LXX; the Heb. lacks \fbib{and}} nothing will be left,' says the \divine{Lord}. \v{7}`Then some of your own sons, who will come from your loins,\fnote{\fbackref{39:7} So 1QIsa\textsuperscript{a}; 4QIsa\textsuperscript{b} MT read \fbib{from you}} whom you will father, will be taken away to become eunuchs in the palace of the king of Babylon.'\,''

\v{8}``The message from the \divine{Lord} that you have spoken is good,'' Hezekiah replied to Isaiah, since he was thinking, ``{\ldots}at least there will be peace and security in my lifetime.''
\labelchapt{40}
\passage{God Comforts His People}

\begin{poetry}
\poeml \chapt{40}
\v{1}``Comfort! Yes, comfort my people,'' \\
\poemll    says your God. \\
\poeml \v{2}``Speak tenderly to Jerusalem, \\
\poemll    and proclaim to her \\
\poeml that her heavy service has been completed, \\
\poemll    that her penalty has been paid, \\
\poeml that she has received from the \divine{Lord}'s hand \\
\poemll    double for all her sins.'' \\
\poeml \v{3}A voice cries out: \\
\poemll    `In the wilderness prepare the way for the \divine{Lord}; \\
\poemlll       and\fnote{\fbackref{40:3} So1QIsa\textsuperscript{a}; MT LXX lack \fbib{and}} in the desert a straight highway for our God.' \\
\poeml \v{4}Every valley will be lifted up, \\
\poemll    and every mountain and hill will be lowered; \\
\poeml the rough ground will become level, \\
\poemll    and the mountain ridges made a plain. \\
\poeml \v{5}Then the glory of the \divine{Lord} will be revealed, \\
\poemll    and all humanity will see it at once; \\
\poemlll       for the mouth of the \divine{Lord} has spoken.''
\passage{The Word of God Endures Forever}
\poeml \v{6}A voice says, ``Cry out!'' \\
\poemll    So I\fnote{\fbackref{40:6} So 1QIsa\textsuperscript{a} LXX; MT reads \fbib{he}} asked, ``What am I to cry out?'' \\
\poeml ``All humanity is grass, \\
\poemll    and all its loyalty\fnote{\fbackref{40:6} Or \fbib{glory}} is like the flowers of the field. \\
\poeml \v{7}Grass withers and flowers fade away \\
\poemll    when the \divine{Lord}'s breath blows on them; \\
\poemlll       surely the people are like grass.\fnote{\fbackref{40:7} So MT; 1QIsa\textsuperscript{a} LXX lack this v.} \\
\poeml \v{8}Grass withers and flowers fade away, \\
\poemll    when the \divine{Lord}'s breath blows on them, \\
\poemlll       but the word of\fnote{\fbackref{40:8} So MT LXX; 1QIsa\textsuperscript{a} lacks \fbib{the word of}} our God will stand forever. ``
\passage{Here is Your God}
\poeml \v{9}``Climb up a high mountain, \\
\poemll    you messenger of good news to Zion! \\
\poeml Lift up your voice with strength, \\
\poemll    you messenger to Jerusalem! \\
\poeml Lift it up! \\
\poemll    Don't be afraid! \\
\poeml Say to the towns of Judah, \\
\poemll    `Here is your God!' \\
\poeml \v{10}Look! The Lord \divine{God} comes with strength, \\
\poemll    and his arm\fnote{\fbackref{40:10} I.e. the Messiah} rules for him. \\
\poeml Look! His reward is with him, \\
\poemll    and his payment accompanies him. \\
\poeml \v{11}Like a shepherd, he tends his flock. \\
\poemll    He gathers the lambs in his arms, \\
\poeml carries them close to his heart, \\
\poemll    and gently leads the mother sheep.''
\passage{Who is Like the \divine{Lord}?}
\poeml \v{12}``Who has measured the waters of the sea\fnote{\fbackref{40:12} Lit. \fbib{from}} \\
\poemll    in the hollow of his hand \\
\poeml and marked off the heavens \\
\poemll    by the width of his hand?\fnote{\fbackref{40:12} So 1QIsa\textsuperscript{a} Syr; MT LXX read \fbib{by a hand's width}} \\
\poeml Who has enclosed the dust of the earth \\
\poemll    in a measuring bowl, \\
\poeml or weighed the mountains \\
\poemll    in scales \\
\poemlll       and the hills in a balance? \\
\poeml \v{13}Who has fathomed the Spirit of the \divine{Lord}, \\
\poemll    or as his counselor has taught him?\fnote{\fbackref{40:13} I.e. the Spirit; so 1QIsa\textsuperscript{a}; MT reads \fbib{the \divine{Lord}}} \\
\poeml \v{14}With whom did he consult \\
\poemll    to enlighten and instruct him on the path of justice? \\
\poeml Or who taught him knowledge \\
\poemll    and showed him the way of wisdom? \\
\poeml \v{15}``Look! The nations are like a drop in a bucket, \\
\poemll    and are reckoned as dust on the scales. \\
\poeml Look! He even lifts up the islands like powder! \\
\poeml \v{16}Lebanon would not provide enough fuel, \\
\poemll    nor are its animals enough for a burnt offering.\fnote{\fbackref{40:14}b-16 So MT LXX; 1QIsa\textsuperscript{a} includes these lines by a later scribe} \\
\poeml \v{17}All the nations are as nothing before him--- \\
\poemll    they are reckoned by him as\fnote{\fbackref{40:17} So 1QIsa\textsuperscript{a} LXX; MT reads \fbib{as less than}} nothing and chaos. \\
\poeml \v{18}``To whom, then, will you compare me,\fnote{\fbackref{40:18} So 1QIsa\textsuperscript{a}; MT LXX lack \fbib{me}} the One who is\fnote{\fbackref{40:18} 1QIsa\textsuperscript{a} LXX MT lack \fbib{the One who is}} God? \\
\poemll    Or to what image will you liken me?\fnote{\fbackref{40:18} So 1QIsa\textsuperscript{a}; MT LXX read \fbib{him}} \\
\poeml \v{19}To an idol? A craftsman makes\fnote{\fbackref{40:19} So 1QIsa\textsuperscript{a} LXX; MT reads \fbib{casts}} the image, \\
\poemll    and a goldsmith overlays it with gold \\
\poemlll       and casts silver chains. \\
\poeml \v{20}To the impoverished person? \\
\poemll    He prepares\fnote{\fbackref{40:20} 1QIsa\textsuperscript{a} MT LXX lack \fbib{prepares}} an offering---\fnote{\fbackref{40:20} So MT; later 1QIsa\textsuperscript{a} scribe includes this line} \\
\poemlll       wood that won't rot--- \\
\poeml Or to the one who chooses a skilled craftsman\fnote{\fbackref{40:20} So 1QIsa\textsuperscript{a}; MT LXX read \fbib{chooses wood}} \\
\poemll    and\fnote{\fbackref{40:20} So 1QIsa\textsuperscript{a}; MT lacks \fbib{and}} seeks\fnote{\fbackref{40:20} So 1QIsa\textsuperscript{a}; MT reads \fbib{seeks a skilled craftsman}} to erect an idol that won't topple?''
\passage{The Majesty of the \divine{Lord}}
\poeml \v{21}``You know, don't you? \\
\poemll    You have heard, haven't you? \\
\poeml Hasn't it been told you from the beginning? \\
\poemll    Haven't you understood from the foundations of the\fnote{\fbackref{40:21} So MT; implied in 1QIsa\textsuperscript{a}} earth? \\
\poeml \v{22}He's the one who sits above the disk of the earth, \\
\poemll    and its inhabitants are like grasshoppers. \\
\poeml He's the one who stretches out the heavens like a curtain, \\
\poemll    and spreads them like a tent to live in, \\
\poeml \v{23}who brings princes to nothing, \\
\poemll    and makes void the rulers of the earth. \\
\poeml \v{24}No sooner are they planted, \\
\poemll    no sooner are they sown, \\
\poemlll       no sooner have\fnote{\fbackref{40:24} So 1QIsa\textsuperscript{a}; 4QIsa\textsuperscript{b} MT LXX read \fbib{has}} their stems taken root in the earth, \\
\poeml than\fnote{\fbackref{40:24} So 1QIsa\textsuperscript{a}; 4QIsa\textsuperscript{b} MT read \fbib{and then}; LXX lacks \fbib{than}} he blows on them, and they wither, \\
\poemll    and the tempest sweeps them away like stubble. \\
\poeml \v{25}``To\fnote{\fbackref{40:25} So 1QIsa\textsuperscript{a}; 4QIsa\textsuperscript{b} MT read \fbib{And to}} whom, then, will you compare me, \\
\poemll    and to whom should I be equal?'' \\
\poemlll       asks the Holy One. \\
\poeml \v{26}``Lift your eyes up to heaven and see \\
\poemll    who created all these--- \\
\poeml the one who leads out their vast array of stars by number, \\
\poemll    calling them all by name--- \\
\poeml because of his great might \\
\poemll    and his\fnote{\fbackref{40:26} So 1QIsa\textsuperscript{a}; MT LXX lack \fbib{his}} powerful strength\fnote{\fbackref{40:26} So 1QIsa\textsuperscript{a} LXX; MT reads \fbib{strong}}--- \\
\poemlll       and\fnote{\fbackref{40:26} So 1QIsa\textsuperscript{a}; MT LXX lack \fbib{and}} not one is missing.''
\passage{The \divine{Lord} Watches Israel}
\poeml \v{27}``Jacob, why do you say--- \\
\poemll    and Israel, why do you complain--- \\
\poeml `My predicament is hidden from the \divine{Lord}, \\
\poemll    and my cause is ignored by my God.'? \\
\poeml \v{28}Don't you know? \\
\poemll    Haven't you heard? \\
\poeml The \divine{Lord} is the eternal God, \\
\poemlll       the Creator of the ends of the earth. \\
\poeml He does not grow tired or weary; \\
\poemll    and\fnote{\fbackref{40:28} So 1QIsa\textsuperscript{a} LXX; the Heb. lacks \fbib{and}} his understanding cannot be fathomed. \\
\poeml \v{29}He's the\fnote{\fbackref{40:29} So 1QIsa\textsuperscript{a}; the Heb. lacks \fbib{The}} one who gives might to the faint, \\
\poemll    renewing strength for the powerless. \\
\poeml \v{30}Even boys grow tired and weary, \\
\poemll    and young men collapse and fall, \\
\poeml \v{31}but those who keep waiting for the \divine{Lord} will renew their strength. \\
\poemll    Then\fnote{\fbackref{40:31} So 1QIsa\textsuperscript{a}; MT LXX lack \fbib{Then}} they'll soar on wings like eagles; \\
\poeml they'll run and not grow weary; \\
\poemll    they'll walk and not grow tired.''
\end{poetry}
\labelchapt{41}
\passage{The \divine{Lord} Comes as Judge}

\begin{poetry}
\poeml \chapt{41}
\v{1}``Be silent before me, you coastlands, \\
\poemll    and let the people renew their strength! \\
\poeml Let them come forward, \\
\poemll    then let them speak together--- \\
\poemlll       let's draw near for a ruling. \\
\poeml \v{2}Who has aroused victory from the east, \\
\poemll    and\fnote{\fbackref{41:2} So 1QIsa\textsuperscript{a}; MT LXX lack \fbib{and}} has summoned it to his service, \\
\poemlll       and\fnote{\fbackref{41:2} So 1QIsa\textsuperscript{a}; the Heb. lacks \fbib{and}} has handed over nations to him? \\
\poeml Who brings down kings, \\
\poemll    and\fnote{\fbackref{41:2} So 1QIsa\textsuperscript{a}; the Heb. lacks \fbib{and}} turns them into dust with his sword, \\
\poemlll       into windblown stubble with his bow? \\
\poeml \v{3}And\fnote{\fbackref{41:3} So 1QIsa\textsuperscript{a} LXX; the Heb. lacks \fbib{and}} who pursues them \\
\poemll    and\fnote{\fbackref{41:3} So 1QIsa\textsuperscript{a}; MT LXX lack \fbib{and}} moves on unscathed \\
\poemlll       by a path that his feet don't know?\fnote{\fbackref{41:3} So 1QIsa\textsuperscript{a}; MT reads \fbib{travel}} \\
\poeml \v{4}Who has performed and carried this out, \\
\poemll    calling the generations from the beginning? \\
\poeml I, the \divine{Lord}---the first \\
\poemll    and will be with the last \\
\poemlll       ---I am the One!''
\passage{Idolaters Encourage Each Other}
\poeml \v{5}``The coastlands have looked and are afraid; \\
\poemll    the ends of the earth have drawn near together\fnote{\fbackref{41:5} So 1QIsa\textsuperscript{a} LXX; MT reads \fbib{earth tremble}} \\
\poemlll       and come forward. \\
\poeml \v{6}Each helps his neighbor, \\
\poemll    saying\fnote{\fbackref{41:6} So 1QIsa\textsuperscript{a}; 1QIsa\textsuperscript{b} MT read \fbib{and saying}; cf. LXX} to each other, `Be strong!' \\
\poeml \v{7}The craftsman encourages the goldsmith, \\
\poemll    and the hammersmith\fnote{\fbackref{41:7} Lit. \fbib{he who smoothes with the hammer}} encourages the one who strikes the anvil. \\
\poeml He says\fnote{\fbackref{41:7} So 1QIsa\textsuperscript{a}; MT reads \fbib{saying}} about the welding, `It's good!' \\
\poemll    and he reinforces it with nails so that it won't topple.''
\passage{The \divine{Lord} Encourages Israel}
\poeml \v{8}``But as for you, Israel, my servant, \\
\poemll    Jacob, whom I've chosen, \\
\poemlll       the offspring of my friend Abraham--- \\
\poeml \v{9}you whom I encouraged from the ends of the earth \\
\poemll    and called from its farthest corners, \\
\poeml and told you, `You're my servant, \\
\poemll    I've chosen you \\
\poemlll       and haven't cast you aside.' \\
\poeml \v{10}Don't be afraid, \\
\poemll    because I'm with you; \\
\poeml don't be anxious, \\
\poemll    because I am your God. \\
\poeml I keep on strengthening you; \\
\poemll    I'm truly helping you. \\
\poeml I'm surely upholding you \\
\poemll    with my victorious right hand.''
\passage{The Coming Defeat of God's Enemies}
\poeml \v{11}``Look! All who are enraged at you \\
\poemll    will be put to shame and disgraced; \\
\poeml those who contend with you \\
\poemll    will all die.\fnote{\fbackref{41:11} So 1QIsa\textsuperscript{a}; MT reads \fbib{will become nothing and die}; cf. LXX} \\
\poeml \v{12}Those who quarrel with you\fnote{\fbackref{41:12} So 1QIsa\textsuperscript{a}; 1QIsa\textsuperscript{b} MT LXX read \fbib{You will seek them and not find those who quarrel with you}} \\
\poemll    will be as nothing; \\
\poeml those who fight you \\
\poemll    like nothing at all!''
\passage{A Call to Courage}
\poeml \v{13}``For I am the \divine{Lord} your God, \\
\poemll    who takes hold of your right hand, \\
\poeml who says to you, `Don't be afraid. \\
\poemll    I'll help you. \\
\poeml \v{14}Don't be afraid, you little worm Jacob, \\
\poemll    and\fnote{\fbackref{41:14} So 1QIsa\textsuperscript{a}; the Heb. lacks \fbib{and}} you insects of Israel! \\
\poeml I myself will help you,'\fnote{\fbackref{41:14} 1QIsa\textsuperscript{a} is masculine; MT is feminine} declares the \divine{Lord}, \\
\poemll    your\fnote{\fbackref{41:14} 1QIsa\textsuperscript{a} is masculine; MT is feminine} Redeemer, the Holy One of Israel.''
\passage{A Promise of Victory}
\poeml \v{15}``See, I'm making you\fnote{\fbackref{41:15} 1QIsa\textsuperscript{a} is masculine; MT is feminine} into \\
\poemll    a new, sharp, and multi-tooth threshing sledge. \\
\poeml You'll thresh and crush the mountains, \\
\poemll    and make the hills like chaff. \\
\poeml \v{16}You'll winnow them, and the wind will lift them up, \\
\poemll    and a tempest will blow them away. \\
\poeml Then you'll rejoice in the \divine{Lord}, \\
\poemll    and\fnote{\fbackref{41:16} So 1QIsa\textsuperscript{a}; the Heb. lacks \fbib{and}} you'll make your boast in the Holy One of Israel.'' \\
\poeml \v{17}``As for the poor, the needy, those seeking\fnote{\fbackref{41:17} So 1QIsa\textsuperscript{a}; MT LXX read \fbib{poor and the needy seeking}} water--- \\
\poemll    when there is none \\
\poemlll       and their tongues are parched from thirst--- \\
\poeml I, the \divine{Lord}, will answer them. \\
\poemll    I, the God of Israel, won't abandon them. \\
\poeml \v{18}I'll open up rivers on the barren heights, \\
\poemll    and fountains in the midst of the valleys. \\
\poeml I'll turn the\fnote{\fbackref{41:18} So 1QIsa\textsuperscript{a}; MT implies \fbib{the}} desert into a pool of water, \\
\poemll    and the parched land into springs of water. \\
\poeml \v{19}I'll put cedar trees in the wilderness, \\
\poemll    along with acacia, myrtle, and olive trees. \\
\poeml I'll plant cypresses in the desert--- \\
\poemll    box\fnote{\fbackref{41:19} Or \fbib{elm}} trees, and pine trees together--- \\
\poeml \v{20}all so that people may see and recognize, \\
\poemll    perceive,\fnote{\fbackref{41:20} So 1QIsa\textsuperscript{a}; 1QIsa\textsuperscript{a} corrector MT LXX read \fbib{to consider}} consider, and comprehend at the same time, \\
\poeml that the hand of the \divine{Lord} has done this, \\
\poemll    and that the Holy One of Israel has created it.''
\passage{Can God's Enemies Predict the Future?}
\poeml \v{21}``Put forward your case!'' says the \divine{Lord}. \\
\poemll    ``Submit your arguments!'' says Jacob's King. \\
\poeml \v{22}Let them approach and ask us, \\
\poemll    `What will happen? \\
\poeml As to the former things, what were they? \\
\poemll    Tell us, so that we may consider them and know. \\
\poeml Or\fnote{\fbackref{41:22} So 1QIsa\textsuperscript{a}; MT reads \fbib{them. And we may know}} the latter things or the things to come--- \\
\poemlll       let us hear. \\
\poeml \v{23}Tell us what the future holds, \\
\poemll    so we may know that you are gods! \\
\poeml Yes, do something good or something bad, \\
\poemll    so we may hear\fnote{\fbackref{41:23} So 1QIsa\textsuperscript{a}; MT reads \fbib{stare at one another}; cf. LXX} and gaze at it together.'\,'' \\
\poeml \v{24}``Look! You and your work are less than nothing;\fnote{\fbackref{41:24} So 1QIsa\textsuperscript{a}; MT reads \fbib{you are nothing, and your work means nothing}} \\
\poemll    whoever finds you pleasing is disgusting.'' \\
\poeml \v{25}``You are stirring up\fnote{\fbackref{41:25} So 1QIsa\textsuperscript{a}; MT LXX read \fbib{I am raising}} one from the north, \\
\poemll    and they are coming\fnote{\fbackref{41:25} So 1QIsa\textsuperscript{a}; MT reads \fbib{I am raising}} from the rising of the sun; \\
\poemlll       and\fnote{\fbackref{41:25} So 1QIsa\textsuperscript{a}; MT LXX lack \fbib{and}} he will be called by his\fnote{\fbackref{41:25} So 1QIsa\textsuperscript{a}; MT LXX read \fbib{my}} name. \\
\poeml Rulers will arrive like mud;\fnote{\fbackref{41:25} I.e. like an overflowing river; so 1QIsa\textsuperscript{a} LXX; MT reads \fbib{He will come, rulers like mud}} \\
\poemll    just\fnote{\fbackref{41:25} So 1QIsa\textsuperscript{a}; MT reads \fbib{and}} like a potter, he will trample the clay. \\
\poeml \v{26}Who told of this from the beginning, \\
\poemll    so we could know, \\
\poeml or beforehand, so we could ask, \\
\poemll    `Is it right?'\fnote{\fbackref{41:26} So 1QIsa\textsuperscript{a} LXX; MT reads \fbib{He was right}} \\
\poeml Indeed, no one told of this, \\
\poemll    no one made an announcement, \\
\poemlll       and no one heard your words: \\
\poeml \v{27}First, to Zion: ``There is slumber.''\fnote{\fbackref{41:27} So 1QIsa\textsuperscript{a}; MT reads \fbib{look, there they are}} \\
\poemll    And to Jerusalem: ``I'll send a messenger with good news.'' \\
\poeml \v{28}But when I look, there is no one--- \\
\poemll    among them there's no one to give counsel, \\
\poemlll       no one to give an answer when I ask them. \\
\poeml \v{29}See, none of them exist, and their deeds are nothing.\fnote{\fbackref{41:29} So 1QIsa\textsuperscript{a}; MT reads \fbib{all of their works are utterly nothing}} \\
\poemll    Their metal images are only wind and confusion.'\,''
\end{poetry}
\labelchapt{42}
\passage{The Servant of the \divine{Lord}}

\begin{poetry}
\poeml \chapt{42}
\v{1}``Here is my servant, whom I support, \\
\poemll    my chosen one, in whom I delight.\fnote{\fbackref{42:1} Lit. \fbib{whom my soul delights}} \\
\poeml I've placed my Spirit upon him; \\
\poemll    and\fnote{\fbackref{42:1} So 1QIsa\textsuperscript{a}; MT LXX lack \fbib{and}} he'll deliver his\fnote{\fbackref{42:1} So 1QIsa\textsuperscript{a}; MT LXX lack \fbib{his}} justice throughout the world.\fnote{\fbackref{42:1} Lit. \fbib{justice to the nations}} \\
\poeml \v{2}He won't shout, \\
\poemll    or raise his voice, \\
\poemlll       or make it heard in the street. \\
\poeml \v{3}A crushed reed he will not break, \\
\poemll    and a fading candle\fnote{\fbackref{42:3} Lit. \fbib{fading linen wick}} he won't snuff out.\fnote{\fbackref{42:3} So 1QIsa\textsuperscript{a} LXX; MT reads \fbib{quench it}} \\
\poemlll       He'll bring forth\fnote{\fbackref{42:3} I.e. \fbib{will demonstrate}} justice for the truth. \\
\poeml \v{4}And\fnote{\fbackref{42:4} So 1QIsa\textsuperscript{a}; 4QIsa\textsuperscript{h} MT lack \fbib{And}} he won't grow faint or be crushed \\
\poemll    until he establishes justice on the mainland, \\
\poemlll       and the coastlands take ownership of\fnote{\fbackref{42:4} So 1QIsa\textsuperscript{a}; MT LXX read \fbib{the islands wait for} (cf. 4QIsa\textsuperscript{h})} his Law.''
\passage{God Speaks about the Servant}
\poeml \v{5}This is what God says--- \\
\poemll    the God\fnote{\fbackref{42:5} So 1QIsa\textsuperscript{a}; 1QIsa\textsuperscript{a} corrector reads \fbib{and}; MT reads \fbib{the \divine{Lord}}} who created the heavens \\
\poemlll       and stretched them out, \\
\poeml who spread out the earth and its produce, \\
\poemll    who gives breath to the people on it \\
\poemlll       and life\fnote{\fbackref{42:5} Lit. \fbib{spirit}} to those who walk in it: \\
\poeml \v{6}``I've\fnote{\fbackref{42:6} So 1QIsa\textsuperscript{a}; 4QIsa\textsuperscript{h} MT read \fbib{I, the \divine{Lord}, have}; LXX reads \fbib{I, the \divine{Lord} God, have}} called you in righteousness. \\
\poemll    I'll take hold of your hand. \\
\poeml I'll preserve you and appoint you \\
\poemll    as a covenant to the people,\fnote{\fbackref{42:6} So 1QIsa\textsuperscript{a} MT LXX; 4QIsa\textsuperscript{h} reads \fbib{as an everlasting covenant}} \\
\poemlll       as a light for the nations, \\
\poeml \v{7}to open blind eyes \\
\poemll    and to bring out those who are bound\fnote{\fbackref{42:7} So 1QIsa\textsuperscript{a} LXX; 4QIsa\textsuperscript{h} MT read \fbib{out prisoners}} from their cells, \\
\poemlll       and\fnote{\fbackref{42:7} So 1QIsa\textsuperscript{a}; 4QIsa\textsuperscript{h} MT lack \fbib{and}} those sitting in darkness from prison. \\
\poeml \v{8}I, the \divine{Lord}, am the one, \\
\poemll    and I won't give my name and\fnote{\fbackref{42:8} So 1QIsa\textsuperscript{a}; 4QIsa\textsuperscript{h} MT LXX read \fbib{I am the \divine{Lord}; that is my name. I won't give my}} glory to another, \\
\poemlll       nor my praise to idols. \\
\poeml \v{9}See, the former things have taken place, \\
\poemll    and I'm announcing the\fnote{\fbackref{42:9} So 1QIsa\textsuperscript{a}; 4QIsa\textsuperscript{b} 4QIsa\textsuperscript{h} MT LXX lack \fbib{the}} new things--- \\
\poeml before they spring into being
\end{poetry}

I'm telling you about them.''
\passage{Praise in Song to God}

\begin{poetry}
\poeml \v{10}Sing to the \divine{Lord} a new song, \\
\poemll    and\fnote{\fbackref{42:10} So 1QIsa\textsuperscript{a}; 4QIsa\textsuperscript{h} MT lack \fbib{and}} his praise from the ends of the earth, \\
\poeml you who sail down the sea and by everything in it, \\
\poemll    you coastlands and their inhabitants. \\
\poeml \v{11}Let the desert cry out,\fnote{\fbackref{42:11} So 1QIsa\textsuperscript{a} 4QIsa\textsuperscript{h} LXX (sing.); MT (pl.)} \\
\poemll    its towns and the\fnote{\fbackref{42:11} So 1QIsa\textsuperscript{a}; MT LXX read \fbib{and its towns}} villages where Kedar lives; \\
\poeml and\fnote{\fbackref{42:11} So 1QIsa\textsuperscript{a}; the Heb. lacks \fbib{and}} let those who live in Sela sing for joy. \\
\poemll    Let them shout aloud\fnote{\fbackref{42:11} So 1QIsa\textsuperscript{a} LXX; MT reads \fbib{them cry joyfully}} from the mountaintops. \\
\poeml \v{12}Let them give glory to the \divine{Lord}, \\
\poemll    and declare his praise in the islands. \\
\poeml \v{13}The \divine{Lord} marches out like a warrior; \\
\poemll    he stirs up his rage like a man of war; \\
\poeml he makes his anger heard; \\
\poemll    he shouts aloud;\fnote{\fbackref{42:13} So 1QIsa\textsuperscript{a}; MT reads \fbib{He makes a war cry and shouts out his anger}} \\
\poemlll       he declares his mastery over his enemies: \\
\poeml \v{14}``I have certainly\fnote{\fbackref{42:14} So 1QIsa\textsuperscript{a}; the Heb. lacks \fbib{certainly}} stayed silent for a long time; \\
\poemll    I've kept still and held myself back. \\
\poeml Now, like a woman giving birth, I'll cry out. \\
\poemll    All of a sudden I'll gasp and pant. \\
\poeml \v{15}I'll devastate the mountains and hills, \\
\poemll    and dry up all their vegetation; \\
\poeml I'll turn rivers into islands, \\
\poemll    and dry up the ponds. \\
\poeml \v{16}I'll help the blind walk, \\
\poemll    even\fnote{\fbackref{42:16} Or \fbib{and}; so 1QIsa\textsuperscript{a}; the Heb. lacks \fbib{even}} on a road they do not know; \\
\poeml I'll guide them \\
\poemll    in directions\fnote{\fbackref{42:16} Lit. \fbib{paths}} they do not know. \\
\poeml I'll turn the dark places\fnote{\fbackref{42:16} So 1QIsa\textsuperscript{a} (misspelling \fbib{places}); MT LXX read \fbib{darkness}} into light in front of them, \\
\poemll    and the rough places into level ground. \\
\poeml These are the things I will do, \\
\poemll    and I won't abandon them. \\
\poeml \v{17}Those who trust in carved idols \\
\poemll    will turn back and\fnote{\fbackref{42:17} So 1QIsa\textsuperscript{a}; the Heb. lacks \fbib{and}} be completely disappointed,\fnote{\fbackref{42:17} So 1QIsa\textsuperscript{a} MT LXX; MT\textsuperscript{ms} lacks \fbib{disappointed}} \\
\poeml along with those\fnote{\fbackref{42:17} 1QIsa\textsuperscript{a} MT LXX lack \fbib{along with those}} who say to metal images, \\
\poemll    `You are our gods.'\,''
\passage{God Rebukes Israel}
\poeml \v{18}``Listen, you deaf people, \\
\poemll    and look up, you blind people, so you may see! \\
\poeml \v{19}Who is blind except my servant, \\
\poemll    or deaf like my messenger I am sending? \\
\poeml Who is blind like the one committed to me, \\
\poemll    or blind like the \divine{Lord}'s servant? \\
\poeml \v{20}You've seen\fnote{\fbackref{42:20} So 1QIsa\textsuperscript{a} MT; MT\textsuperscript{qere} reads \fbib{To see} (or \fbib{He sees} )} many things, but you pay no\fnote{\fbackref{42:20} So 1QIsa\textsuperscript{a} MT LXX; MT\textsuperscript{mss} read \fbib{he pays no}} attention. \\
\poemll    His ears are open,\fnote{\fbackref{42:20} So 1QIsa\textsuperscript{a}; MT reads \fbib{to open}; or \fbib{are open}} but he doesn't listen. \\
\poeml \v{21}The \divine{Lord} was pleased, for the sake of his vindication, \\
\poemll    that he should magnify his Law and make it glorious. \\
\poeml \v{22}But this is a people who have been robbed and plundered, \\
\poemll    all of them trapped in pits \\
\poemlll       or hidden away in prisons.\fnote{\fbackref{42:22} So 1QIsa\textsuperscript{a} MT LXX; 4QIsa\textsuperscript{g} reads \fbib{prison}} \\
\poeml They have become prey, with no one to rescue them; \\
\poemll    they have been made loot, with no one to say, `Send them back!' \\
\poeml \v{23}``Who among you will listen, \\
\poemll    and\fnote{\fbackref{42:23} So 1QIsa\textsuperscript{a}; MT LXX lack \fbib{and}} pay attention, \\
\poemlll       and listen for the time to come?''
\passage{God Punishes Israel}
\poeml \v{24}``Who handed Jacob over to looters, \\
\poemll    and Israel to robbers? \\
\poeml Was it not the \divine{Lord}, against whom we have sinned? \\
\poemll    After all, they weren't willing to walk in his ways, \\
\poemlll       and they wouldn't obey\fnote{\fbackref{42:24} Or \fbib{wouldn't listen to}} his instruction, \\
\poeml \v{25}so he drenched him with\fnote{\fbackref{42:25} Lit. \fbib{he poured out on him}} the heat that is his anger,\fnote{\fbackref{42:25} So 1QIsa\textsuperscript{a} LXX; 4QIsa\textsuperscript{g} MT read \fbib{the heat, his anger}} \\
\poemll    the violence of war. \\
\poeml It enveloped him in flames, \\
\poemll    but still he had no insight. \\
\poeml It burned him up, \\
\poemll    but he didn't take it to heart.''
\end{poetry}
\labelchapt{43}
\passage{The Future Redemption of Israel}

\chapt{43}
\v{1}But now this is what the \divine{Lord} says,

\begin{poetry}
\poemll    the one who created you, Jacob, \\
\poemlll       the one who formed you, Israel: \\
\poeml ``Do not be afraid, because I've redeemed you. \\
\poemll    I've called you by name; \\
\poemlll       you are mine. \\
\poeml \v{2}When you pass through the waters, I'll be with you; \\
\poemll    and through the rivers, they won't sweep over you. \\
\poeml when you walk through fire you won't be scorched, \\
\poemll    and the flame won't set you ablaze. \\
\poeml \v{3}``I\fnote{\fbackref{43:3} So 1QIsa\textsuperscript{a}; 1QIsa\textsuperscript{b} 4QIsa\textsuperscript{g} MT LXX read \fbib{For I}} am the \divine{Lord} your God, \\
\poemll    the Holy One of Israel, your Redeemer.\fnote{\fbackref{43:3} So 1QIsa\textsuperscript{a}; MT LXX read \fbib{Savior}} \\
\poeml And\fnote{\fbackref{43:3} So 1QIsa\textsuperscript{a}; the Heb. lacks \fbib{And}} I've given Egypt as your ransom,\fnote{\fbackref{43:3} So 1QIsa\textsuperscript{a}; 1QIsa\textsuperscript{b} MT LXX read \fbib{as your ransom Egypt}} \\
\poemll    Cush and the people of Seba\fnote{\fbackref{43:3} So 1QIsa\textsuperscript{a}; 1QIsa\textsuperscript{b} 4QIsa\textsuperscript{g} MT read \fbib{and Seba}} in exchange for you. \\
\poeml \v{4}Since you're precious in my sight \\
\poemll    and honored, \\
\poeml and because I love you, \\
\poemll    I'm giving up people in your place, \\
\poemll    and nations in exchange for your life.''
\passage{The Regathering of Israel}
\poeml \v{5}``Don't be afraid, for I am with you; \\
\poemll    I'll bring your children from the east, \\
\poemlll       and gather you from the west. \\
\poeml \v{6}I'll say to the north, \\
\poemll    `Give them up'! \\
\poeml and to the south, \\
\poemll    `Don't keep them back!' \\
\poeml Bring\fnote{\fbackref{43:6} 1QIsa\textsuperscript{a} employs masculine pl.; MT employs feminine pl.} my sons from far away \\
\poemll    and my daughters from the ends of the earth--- \\
\poeml \v{7}everyone who is called by my name, \\
\poemll    whom I created for my glory, \\
\poemlll       whom\fnote{\fbackref{43:7} So 1QIsa\textsuperscript{a} MT; 1QIsa\textsuperscript{b} reads \fbib{and whom}} I formed and made. \\
\poeml \v{8}``Bring out the people who are blind, yet still have eyes, \\
\poemll    who are deaf, yet still have ears! \\
\poeml \v{9}Let all the nations be gathered together, \\
\poemll    and let the peoples be assembled. \\
\poeml ``Who is there among them who\fnote{\fbackref{43:9} So 1QIsa\textsuperscript{a}; MT LXX lack \fbib{who}} can declare this, \\
\poemll    or announce\fnote{\fbackref{43:9} So 1QIsa\textsuperscript{a}; MT reads \fbib{announce to us}; LXX reads \fbib{announce to you}} the former things? \\
\poeml Let them produce their witnesses to prove them right, \\
\poemll    and let them proclaim\fnote{\fbackref{43:9} So 1QIsa\textsuperscript{a}; MT reads \fbib{hear}} so people will say, `It's true.' \\
\poeml \v{10}``You are my witnesses,'' declares the \divine{Lord}, \\
\poemll    ``and my servant whom I have chosen, \\
\poeml so that you may know and trust me \\
\poemll    and understand that I am the One.\fnote{\fbackref{43:10} Or \fbib{am he}} \\
\poeml Before me no God was formed, \\
\poemll    nor will there be one after me. \\
\poeml \v{11}I, yes I, am the \divine{Lord}, \\
\poemll    and apart from me there is no savior. \\
\poeml \v{12}I've revealed and saved and proclaimed, \\
\poemll    when there was no foreign god among you --- \\
\poemlll       and you are my witnesses,'' declares the \divine{Lord}. \\
\poeml \v{13}``I am God; also\fnote{\fbackref{43:13} So 1QIsa\textsuperscript{a}; MT reads \fbib{God---Yes,}; cf. LXX} from ancient days\fnote{\fbackref{43:13} Or \fbib{from this day on,}} I am the one. \\
\poemll    And there is no one who can deliver out of my hand; \\
\poemlll       when I act, who can reverse it?'' \\
\poeml \v{14}This is what the \divine{Lord} says, \\
\poemll    your Redeemer, the Holy One of Israel: \\
\poeml ``For your sake I will send to Babylon,\fnote{\fbackref{43:14} So 1QIsa\textsuperscript{a}; 4QIsa\textsuperscript{b} MT each spell this line differently} \\
\poemll    and bring them all down as fugitives. \\
\poeml Now as for the Babylonians, \\
\poemll    their ringing cry will become lamentation. \\
\poeml \v{15}I am the \divine{Lord}, your Holy One, \\
\poemll    Creator of Israel, and your King.''
\passage{Something New for Israel}
\poeml \v{16}This is what the \divine{Lord} says --- \\
\poemll    who makes a way through the sea, \\
\poemlll       a path through the mighty waters, \\
\poeml \v{17}who brings out chariots and horsemen, \\
\poemll    and\fnote{\fbackref{43:17} So 1QIsa\textsuperscript{a} LXX; the Heb. lacks \fbib{and}} armies and warriors at the same time. \\
\poeml They lay there, never to rise again, \\
\poemll    extinguished, snuffed out like a candle:\fnote{\fbackref{43:17} Lit. \fbib{linen wick}} \\
\poeml \v{18}``Don't remember\fnote{\fbackref{43:18} 1QIsa\textsuperscript{a} employs second person sing.; MT LXX employ second person pl.} the former things; \\
\poemll    don't dwell on things past. \\
\poeml \v{19}Watch! I'm about to carry out something new! \\
\poemll    And\fnote{\fbackref{43:19} So 1QIsa\textsuperscript{a}; the Heb. lacks \fbib{And}} now it's springing up--- \\
\poemlll       don't you recognize it? \\
\poeml I'm making a way in the wilderness \\
\poemll    and paths\fnote{\fbackref{43:19} So 1QIsa\textsuperscript{a}; MT LXX read \fbib{streams}} in the desert. \\
\poeml \v{20}Wild animals, jackals, and owls\fnote{\fbackref{43:20} Or \fbib{ostriches}} will honor me \\
\poemll    because I provide\fnote{\fbackref{43:20} So 1QIsa\textsuperscript{a}; MT reads \fbib{have provided}} water in the desert \\
\poeml and streams in the wilderness \\
\poemll    to give drink to my people, my chosen ones,\fnote{\fbackref{43:20} So 1QIsa\textsuperscript{a}; MT reads \fbib{my chosen people}} \\
\poeml \v{21}the people whom I formed for myself \\
\poemll    and\fnote{\fbackref{43:21} So 1QIsa\textsuperscript{a}; 4QIsa\textsuperscript{g} lacks \fbib{my chosen ones}} so that they may speak\fnote{\fbackref{43:21} So 1QIsa\textsuperscript{a}; 4QIsa\textsuperscript{g} MT LXX read \fbib{recount}} my praise.''
\passage{God is Weary of Israel}
\poeml \v{22}``And\fnote{\fbackref{43:22} So 1QIsa\textsuperscript{a} MT; MT\textsuperscript{mss} lack \fbib{And}} yet you didn't call upon me, Jacob; \\
\poemll    indeed, you are tired of me, Israel! \\
\poeml \v{23}You haven't brought me your sheep for a burnt offering,\fnote{\fbackref{43:23} So 1QIsa\textsuperscript{a}; MT reads \fbib{for your burnt-offerings}; or \fbib{your burnt-offering}} \\
\poemll    nor have you honored me with\fnote{\fbackref{43:23} So 1QIsa\textsuperscript{a} LXX; implicit in MT} your sacrifices, \\
\poeml nor have you made meal offerings for me\fnote{\fbackref{43:23} So 1QIsa\textsuperscript{a}; 4QIsa\textsuperscript{g} MT read \fbib{I have not burdened you with grain offerings}; LXX lacks this line}--- \\
\poemll    yet I have not tired you about incense! \\
\poeml \v{24}You\fnote{\fbackref{43:24} So 1QIsa\textsuperscript{a} MT; 4QIsa\textsuperscript{g} reads \fbib{And you}} haven't bought me sweet cane with money, \\
\poemll    nor have you satisfied me with the fat of your sacrifices. \\
\poeml You have only burdened me with your sins \\
\poemll    and made me tired with your iniquities. \\
\poeml \v{25}``I, I am the one \\
\poemll    who blots out your transgression\fnote{\fbackref{43:25} So 1QIsa\textsuperscript{a}; MT reads \fbib{transgressions}; cf. LXX} for my own sake, \\
\poemlll       and I'll remember your sins no more.\fnote{\fbackref{43:25} So 1QIsa\textsuperscript{a}; MT reads \fbib{not remember your sins}} \\
\poeml \v{26}Recount the brief! \\
\poemll    Let's argue the matter together; \\
\poeml Present your case, \\
\poemll    so that you may be proved right. \\
\poeml \v{27}Your first ancestor sinned, \\
\poemll    and your mediators rebelled against me. \\
\poeml \v{28}So I'll disgrace the leaders of the Temple, \\
\poemll    and I'll consign Jacob to total destruction\fnote{\fbackref{43:28} The Heb. term refers to involuntary dedication to God of the thing destroyed} \\
\poemlll       and Israel to contempt.
\end{poetry}
\labelchapt{44}
\passage{God's Blessing on Israel}

\begin{poetry}
\poeml \chapt{44}
\v{1}``But now listen, Jacob my servant \\
\poemll    and Israel whom I have chosen: \\
\poeml \v{2}This what the \divine{Lord} says, the one who made you, \\
\poemll    formed you from the womb, \\
\poemlll       and who will help\fnote{\fbackref{44:2} So 1QIsa\textsuperscript{a}; MT reads \fbib{he will help}} you: \\
\poeml ``Don't be afraid, Jacob my servant, \\
\poemll    and Jeshurun,\fnote{\fbackref{44:2} I.e. a poetic term for national Israel; the Heb. name means \fbib{Upright One}; cf. Deut 32:15; 33:5, 26} whom I have chosen. \\
\poeml \v{3}For I'll pour water upon thirsty ground \\
\poemll    and streams on parched land. \\
\poeml So\fnote{\fbackref{44:3} So 1QIsa\textsuperscript{a}; MT lacks \fbib{So}} will I pour my Spirit upon your offspring, \\
\poemll    and my blessing upon your descendants. \\
\poeml \v{4}They'll\fnote{\fbackref{44:4} So 1QIsa\textsuperscript{a}; MT LXX read \fbib{And they}} spring up as among\fnote{\fbackref{44:4} So 1QIsa\textsuperscript{a} MT\textsuperscript{mss} LXX Targ; MT reads \fbib{up among}} the green grass, \\
\poemll    like willows by flowing streams. \\
\poeml \v{5}One will say, `I belong to the \divine{Lord},' \\
\poemll    and another will call himself by the name of Jacob; \\
\poeml still another will have written on his hand, `the \divine{Lord}'s,' \\
\poemll    and will adopt the name of Israel.''
\passage{I am the First and the Last}
\poeml \v{6}This is what the \divine{Lord} says, the King of Israel \\
\poemll    and its Redeemer--- \\
\poemlll       the \divine{Lord} of the Heavenly Armies is his name---\fnote{\fbackref{44:6} So 1QIsa\textsuperscript{a}; MT LXX lack \fbib{is his name}} \\
\poeml ``I am the first and I am the last, \\
\poemll    and apart from me there is no God. \\
\poeml \v{7}Who is like me? Let him proclaim \\
\poemll    and declare it, and lay it out for himself---\fnote{\fbackref{44:7} So 1QIsa\textsuperscript{a}; MT LXX read \fbib{me}} \\
\poeml since he made\fnote{\fbackref{44:7} Or \fbib{himself, making him}; so 1QIsa\textsuperscript{a}; MT LXX read \fbib{since I made}} an ancient people. \\
\poemll    And let him speak\fnote{\fbackref{44:7} So 1QIsa\textsuperscript{a}; 4QIsa\textsuperscript{c} MT LXX lack \fbib{let him speak}} future events; \\
\poemlll       let them tell him what\fnote{\fbackref{44:7} So 1QIsa\textsuperscript{a}; 4QIsa\textsuperscript{c} MT read \fbib{and what}} will happen. \\
\poeml \v{8}Don't tremble, and don't be afraid.\fnote{\fbackref{44:8} So 1QIsa\textsuperscript{a} MT; LXX lacks \fbib{and do not be afraid}} \\
\poemll    Didn't I tell you and announce it long ago? \\
\poeml You are my witnesses. \\
\poemll    Is there any God besides me? \\
\poeml There is no other Rock--- \\
\poemll    I don't know of any.''
\end{poetry}
\passage{A Rebuke to Idol Worship}

\v{9}Now,\fnote{\fbackref{44:9} So 1QIsa\textsuperscript{a}; MT LXX lack \fbib{Now}} all the forming of\fnote{\fbackref{44:9} So 1QIsa\textsuperscript{a}; MT LXX read \fbib{those who form}} images means nothing, and the things they treasure are worthless. Their own witnesses cannot see, and they\fnote{\fbackref{44:9} So MT LXX; 1QIsa\textsuperscript{a} reads \fbib{They}} know nothing. So they will be put to shame.

\v{10}Who would shape a god or cast an image that profits nothing? \v{11}To be sure, all who associate with it will be put to shame; and as for the craftsmen, they are only human. Let them all gather together and\fnote{\fbackref{44:11} So 1QIsa\textsuperscript{a} LXX; MT lacks \fbib{and}} take their stand. Then\fnote{\fbackref{44:11} So 1QIsa\textsuperscript{a}; MT LXX lack \fbib{Then}} let them be terrified---they will be humiliated together.

\v{12}The blacksmith prepares a tool and works in the coals, then\fnote{\fbackref{44:12} So 1QIsa\textsuperscript{a}; MT LXX lack \fbib{then}} fashions an idol with hammers, working by the strength of his arm. He even becomes hungry and loses his strength; he drinks no water and grows faint.

\v{13}The carpenter measures it\fnote{\fbackref{44:13} I.e. the idol; so 1QIsa\textsuperscript{a} LXX; MT lacks \fbib{it}} with a line; he traces its shape with a stylus, then fashions it with planes and shapes it with a compass. He makes the idol like a human figure, with human beauty, to be at home\fnote{\fbackref{44:13} Lit. \fbib{to rest}} in a shrine. \v{14}He cuts down cedars, or chooses a cypress tree or an oak, and lets it grow strong among the trees of the forest. Or he plants a cedar, and the rain makes it grow. \v{15}He divides it up\fnote{\fbackref{44:15} So 1QIsa\textsuperscript{a}; MT reads \fbib{It is}; LXX reads \fbib{so that it is}} for people to burn. Taking part of it, he warms himself, makes a fire, and bakes bread. Or perhaps\fnote{\fbackref{44:15} So 1QIsa\textsuperscript{a}; MT reads \fbib{Also}} he constructs a god and worships it. He makes it an idol and bows down to it. \v{16}Half the wood he burns in the fire, and\fnote{\fbackref{44:16} So 1QIsa\textsuperscript{a} LXX; MT lacks \fbib{and}} over\fnote{\fbackref{44:16} 1QIsa\textsuperscript{a} lacks \fbib{over}, but inserts it above the line} that half he places\fnote{\fbackref{44:16} Lit. \fbib{half is}} meat so he can eat. He sits by its coals, warms himself,\fnote{\fbackref{44:16} So 1QIsa\textsuperscript{a}; MT LXX read \fbib{eats meat he roasted as a roast and is satisfied. He also warms himself}} and says, ``Ah! I am warm in front of\fnote{\fbackref{44:16} So 1QIsa\textsuperscript{a}; MT reads \fbib{I see}; LXX reads \fbib{and I have seen}} the fire.'' \v{17}And the rest of it he makes into a god. To blocks\fnote{\fbackref{44:17} Or \fbib{to his Baals} (i.e. to Canaanite deities); so 1QIsa\textsuperscript{a} copyist error; MT LXX read \fbib{god, for his idol}} of wood he bows down, worships, prays, and says, ``Save me, since you are my god.''

\v{18}They don't realize; they don't understand, because their eyes are plastered over so they cannot see, and their minds, too, so they cannot understand. \v{19}No one stops to think. No one has the knowledge or understanding to think---yes to think!---\fnote{\fbackref{44:19} So 1QIsa\textsuperscript{a}; MT lacks ---\fbib{yes to think!---}}``Half of it I burned in the fire. I even baked bread on its coals, and\fnote{\fbackref{44:19} So 1QIsa\textsuperscript{a}; MT lacks \fbib{and}} I roasted meat and ate it. And\fnote{\fbackref{44:19} So 1QIsa\textsuperscript{a} MT; 4QIsa\textsuperscript{b} lacks \fbib{And}} am I about to make detestable things\fnote{\fbackref{44:19} So 1QIsa\textsuperscript{a}; 4QIsa\textsuperscript{b} MT LXX read \fbib{make a detestable thing}} from what is left? Am I about to bow down to blocks\fnote{\fbackref{44:19} So 1QIsa\textsuperscript{a}; MT reads \fbib{to a block}} of wood?'' \v{20}He tends ashes. A deceived mind has lead him astray. It cannot be his life,\fnote{\fbackref{44:20} So 1QIsa\textsuperscript{a}; 4QIsa\textsuperscript{b} MT LXX read \fbib{He does not save his life}} nor can he say, ``There's a lie in my right hand.''\fnote{\fbackref{44:20} So 1QIsa\textsuperscript{a}; 4QIsa\textsuperscript{b} MT LXX read ``\fbib{There's no lie in my right hand, is there?''}}
\passage{A Call to Remembrance and Joy}

\begin{poetry}
\poeml \v{21}``Remember these things, Jacob, \\
\poemll    Israel,\fnote{\fbackref{44:21} So 1QIsa\textsuperscript{a}; 4QIsa\textsuperscript{b} MT LXX read \fbib{and Israel}} for you are my servant. \\
\poeml I have formed you; \\
\poemll    you are a servant to me. \\
\poemlll       Israel,\fnote{\fbackref{44:21} So 1QIsa\textsuperscript{a} MT; 4QIsa\textsuperscript{b} LXX read \fbib{and Israel}} you must not mislead me.\fnote{\fbackref{44:21} So 1QIsa\textsuperscript{a}; 4QIsa\textsuperscript{b} MT read \fbib{you won't be forgotten by me}; LXX reads \fbib{don't forget me}} \\
\poeml \v{22}I've wiped away your transgressions like a cloud \\
\poemll    and your sins like mist. \\
\poeml Return to me; \\
\poemll    because I've redeemed you. \\
\poeml \v{23}``Shout for joy, you heavens, \\
\poemll    for the \divine{Lord} has done it! \\
\poeml Shout aloud, you depths of the earth! \\
\poemll    Burst out with singing, you mountains, \\
\poemlll       you forest, and all your trees! \\
\poeml For the \divine{Lord} has redeemed Jacob \\
\poemll    and will display his glory in Israel. \\
\poeml \v{24}``This is what the \divine{Lord} says, your Redeemer \\
\poemll    and the one who formed you in the womb: \\
\poeml `I am the \divine{Lord}, who has made everything, \\
\poemll    who alone stretched out the heavens, \\
\poeml who spread out the earth--- \\
\poemll    Who was with me at that time?---\fnote{\fbackref{44:24} 1QIsa\textsuperscript{a} 4QIsa\textsuperscript{b} MT LXX lack \fbib{at that time}} \\
\poeml \v{25}who frustrates the omens of idle talkers, \\
\poemll    and drives diviners mad, \\
\poeml who turns back the wise, \\
\poemll    and makes their knowledge foolish;\fnote{\fbackref{44:25} So 1QIsa\textsuperscript{a} 1QIsa\textsuperscript{b} 4QIsa\textsuperscript{b} LXX; MT reads \fbib{wise}} \\
\poeml \v{26}who carries out the words of his servants, \\
\poemll    and fulfills the predictions of his messengers, \\
\poeml who says of Jerusalem, ``It will be inhabited,'' \\
\poemll    and of Judah's cities, ``They will be rebuilt,'' \\
\poeml and of her ruins, ``I'll raise them up''; \\
\poeml \v{27}who says to the watery deep, ``Be dry--- \\
\poemll    I will dry up your rivers;'' \\
\poeml \v{28}who says about Cyrus, ``He's my shepherd, \\
\poemll    and he'll carry out everything that I please: \\
\poeml He'll say of Jerusalem, `Let it be rebuilt,' \\
\poemll    and of my\fnote{\fbackref{44:28} So 1QIsa\textsuperscript{a} LXX; 4QIsa\textsuperscript{b} MT read \fbib{the}} Temple, `Let its foundations be laid again.'\,''\,'\,''
\end{poetry}
\labelchapt{45}
\passage{Cyrus: God's Deliverer}

\begin{poetry}
\poeml \chapt{45}
\v{1}This is what the \divine{Lord} says to his anointed, Cyrus, \\
\poemll    whose right hand I have grasped \\
\poeml to subdue nations before him, \\
\poemll    as I strip kings of their armor,\fnote{\fbackref{45:1} Lit. \fbib{I expose the loins of kings}} \\
\poeml to open doors\fnote{\fbackref{45:1} So 1QIsa\textsuperscript{a} (pl.); MT (dual)} before him \\
\poemll    and gates that cannot keep closed: \\
\poeml \v{2}``I myself will go before you, \\
\poemll    and he\fnote{\fbackref{45:2} So 1QIsa\textsuperscript{a}; MT LXX read \fbib{I}} will make the mountains\fnote{\fbackref{45:2} So 1QIsa\textsuperscript{a} 1QIsa\textsuperscript{b} LXX; MT reads \fbib{hills}} level; \\
\poeml I'll shatter bronze doors \\
\poemll    and cut through iron bars. \\
\poeml \v{3}I'll give you concealed treasures\fnote{\fbackref{45:3} Lit. \fbib{treasures of darkness}} \\
\poemll    and riches hidden in secret places, \\
\poeml so that you'll know that it is I, the \divine{Lord}, \\
\poemll    the God of Israel, who calls you by name. \\
\poeml \v{4}For the sake of Jacob my servant, \\
\poemll    Israel\fnote{\fbackref{45:4} So 1QIsa\textsuperscript{a}; MT LXX read \fbib{and Israel}} my chosen, \\
\poeml I've called you, \\
\poemll    and he has established you with a name,\fnote{\fbackref{45:4} So 1QIsa\textsuperscript{a}; MT reads \fbib{I have called you by your name, given you a title}; LXX reads \fbib{I will call you by your name, and receive you}} \\
\poemlll       although you have not acknowledged me. \\
\poeml \v{5}I am the \divine{Lord}, and there is no other besides me: \\
\poemll    and there are no gods.\fnote{\fbackref{45:5} So 1QIsa\textsuperscript{a}; MT 1QIsa\textsuperscript{b} read \fbib{there is no other: besides me there are no gods}} \\
\poeml I'm strengthening you, \\
\poemll    although you have not acknowledged me, \\
\poeml \v{6}so that from the sun's rising\fnote{\fbackref{45:6} I.e. \fbib{the east}} to the west \\
\poemll    people may know that there is none besides me. \\
\poeml ``I am the \divine{Lord}, and there is no other.''
\passage{God is Sovereign}
\poeml \v{7}``I form light and create darkness, \\
\poeml I make goodness\fnote{\fbackref{45:7} So 1QIsa\textsuperscript{a}; MT reads \fbib{well-being}; LXX reads \fbib{peace}} and create disaster. \\
\poemll    I am the \divine{Lord}, who does all these things. \\
\poeml \v{8}``Shout,\fnote{\fbackref{45:8} So 1QIsa\textsuperscript{a} LXX; MT reads \fbib{Shower}} you skies above, and you clouds, \\
\poemll    and let righteousness stream down.\fnote{\fbackref{45:8} So 1QIsa\textsuperscript{a}; 1QIsa\textsuperscript{b} MT LXX read \fbib{Shower, you skies above, and let the clouds stream down righteousness}} \\
\poeml I am\fnote{\fbackref{45:8} 1QIsa\textsuperscript{a} MT LXX lack \fbib{I am}} the one who says to the earth, `Let salvation blossom, \\
\poemll    and let righteousness sprout forth.'\fnote{\fbackref{45:8} So 1QIsa\textsuperscript{a}; 1QIsa\textsuperscript{b} 1QIsa\textsuperscript{c} MT read \fbib{Let the earth open up, let them bear the fruit of salvation, and let righteousness sprout forth also. I the \divine{Lord} have created it}} \\
\poeml \v{9}``Woe to the one who quarrels with his makers,\fnote{\fbackref{45:9} So 1QIsa\textsuperscript{a}; MT reads \fbib{maker}} \\
\poemll    a mere potsherd with the potsherds of the\fnote{\fbackref{45:9} So 1QIsa\textsuperscript{a}; MT lacks \fbib{the}} earth! \\
\poeml Woe to the one who says\fnote{\fbackref{45:9} So 1QIsa\textsuperscript{a}; 1QIsa\textsuperscript{b} MT LXX read \fbib{Will clay say}} to the one forming him, \\
\poemll    `What are you making?' \\
\poemlll       or `Your work has no human\fnote{\fbackref{45:9} So 1QIsa\textsuperscript{a}; MT LXX lack \fbib{human}} hands?'! \\
\poeml \v{10}Woe to the\fnote{\fbackref{45:10} So 1QIsa\textsuperscript{a}; MT lacks \fbib{the}} one who says to his father, \\
\poemll    `What are you begetting?' \\
\poemlll       or to a woman, `To what are you giving birth?'!'' \\
\poeml \v{11}This is what the Lord\fnote{\fbackref{45:11} So 1QIsa\textsuperscript{a}; MT LXX 1QIsa\textsuperscript{a} corrector reads \fbib{the \divine{Lord}, the Holy One of Israel}} says, \\
\poemll    the Creator of the signs: \\
\poeml ``Question me about my children?\fnote{\fbackref{45:11} So 1QIsa\textsuperscript{a} LXX; Lit. \fbib{my sons and daughters}; MT reads \fbib{says, and its creator: Question me of things to come about my children}} \\
\poemll    Or give me orders about the work of my hands? \\
\poeml \v{12}I myself made the earth \\
\poemll    and personally created humankind upon it. \\
\poeml My own hands stretched out the skies; \\
\poemll    I marshaled all their starry hosts.''
\passage{God will Bless Cyrus}
\poeml \v{13}``I have aroused him\fnote{\fbackref{45:13} I.e. Cyrus, king of Persia} in righteousness, \\
\poemll    and I'll make all his pathways smooth. \\
\poeml It is he who will rebuild my city \\
\poemll    and set my exiles free, \\
\poeml but not for a price nor reward, '' \\
\poemll    says the \divine{Lord} of the Heavenly Armies. \\
\poeml \v{14}This is what the \divine{Lord} says: \\
\poeml ``The wealth of Egypt, and the merchandise of Ethiopia, \\
\poemll    those\fnote{\fbackref{45:14} Lit. \fbib{and those}; so 1QIsa\textsuperscript{a}; MT LXX read \fbib{and the}} Sabeans, men of great heights.\fnote{\fbackref{45:14} So 1QIsa\textsuperscript{a}; MT reads \fbib{height}} \\
\poeml They'll come over to you and will be yours; \\
\poemll    They'll trudge behind you--- \\
\poemlll       coming over in chains, they'll bow down to you. \\
\poeml They'll plead with you, \\
\poemll    `Surely God is in you; \\
\poemlll       and there is no other God at all.'\,''
\passage{God as Savior of Israel}
\poeml \v{15}``Truly you are a God who hides himself, \\
\poemll    O God of Israel, the Savior. \\
\poeml \v{16}All of them will be put to shame---indeed, disgraced! \\
\poemll    The makers of idols will go off in disgrace together. \\
\poeml \v{17}But Israel will be saved by the \divine{Lord} \\
\poemll    with everlasting salvation; \\
\poeml you won't be put to shame or disgraced ever again.'' \\
\poeml \v{18}For this is what the \divine{Lord} says, \\
\poemll    who created the heavens--- \\
\poeml he is God, \\
\poemll    and\fnote{\fbackref{45:18} So 1QIsa\textsuperscript{a}; MT LXX lack \fbib{and}} the one who formed the earth and made it, \\
\poeml and\fnote{\fbackref{45:18} So 1QIsa\textsuperscript{a}; MT lacks \fbib{and}} he is the one who established it; \\
\poemll    he didn't create it for\fnote{\fbackref{45:18} Or \fbib{it to remain in a state of}; so 1QIsa\textsuperscript{a} LXX; MT lacks \fbib{for}} chaos, \\
\poemlll       but formed it to be inhabited--- \\
\poeml ``I am the \divine{Lord} and there is no other. \\
\poeml \v{19}I didn't speak in secret, \\
\poemlll       from somewhere in a land of darkness; \\
\poeml I didn't say to Jacob's descendants, \\
\poemll    `Seek me in chaos.' \\
\poeml I, the \divine{Lord}, speak truth, \\
\poemll    declaring what is right. \\
\poeml \v{20}``Gather together and come; \\
\poemll    draw near and enter,\fnote{\fbackref{45:20} So 1QIsa\textsuperscript{a}; MT LXX read \fbib{together}} \\
\poemlll       your fugitives from the nations. \\
\poeml Those who carry around their wooden idols \\
\poemll    know nothing, \\
\poeml nor do those who keep praying to a god \\
\poemll    that cannot save. \\
\poeml \v{21}Explain and present a case! \\
\poemll    Yes, let them take counsel together. \\
\poeml Who announced this long ago, \\
\poemll    who declared it from the distant past? \\
\poemlll       Was it not I, the \divine{Lord}? \\
\poeml And there is no other God besides me, \\
\poemll    a righteous God and Savior; \\
\poemlll       and\fnote{\fbackref{45:21} So 1QIsa\textsuperscript{a}; MT LXX lack \fbib{and}} there is none besides me. \\
\poeml \v{22}Turn to me and be saved, \\
\poemll    all you ends of the earth. \\
\poemlll       For I am God, and there is no other.
\passage{Every Knee will Bow}
\poeml \v{23}By myself I have sworn--- \\
\poemll    from my mouth has gone out integrity, \\
\poemlll       a promise\fnote{\fbackref{45:23} Lit. \fbib{word}} that won't be revoked: \\
\poeml `To me every knee will bow, \\
\poemll    and\fnote{\fbackref{45:23} So 1QIsa\textsuperscript{a}; MT lacks \fbib{and}} every tongue will swear. ' \\
\poeml \v{24}One will say of me,\fnote{\fbackref{45:24} So 1QIsa\textsuperscript{a}; MT reads \fbib{one said of me}; LXX reads \fbib{saying}} \\
\poemll    `Only in the \divine{Lord} are victories and might.' \\
\poeml All who raged against him will come\fnote{\fbackref{45:24} So 1QIsa\textsuperscript{a} MT\textsuperscript{mss} LXX (pl.); MT (sing.)} to him \\
\poemll    and will be put to shame. \\
\poeml \v{25}In the \divine{Lord} all the descendants of Israel \\
\poemll    will triumph and make their boast.''
\end{poetry}
\labelchapt{46}
\passage{God is Unique and Eternal}

\begin{poetry}
\poeml \chapt{46}
\v{1}``Bel\fnote{\fbackref{46:1} I.e. the Babylonian sun god Marduk} bows down, Nebo\fnote{\fbackref{46:1} I.e. Nabu, the Babylonian god of astronomy and learning, son of Marduk} stoops low. \\
\poemll    Their idols are on beasts, on\fnote{\fbackref{46:1} So 1QIsa\textsuperscript{a}; MT LXX read \fbib{and on}} livestock. \\
\poemlll       Your loads are more burdensome than their reports.\fnote{\fbackref{46:1} So 1QIsa\textsuperscript{a}; 4QIsa\textsuperscript{b} MT LXX read \fbib{burdensome, a load for the weary}} \\
\poeml \v{2}They stoop, they bow down together, \\
\poemll    and\fnote{\fbackref{46:2} So 1QIsa\textsuperscript{a}; MT lacks \fbib{and}} they are not able to rescue the burden, \\
\poemlll       but they themselves go off\fnote{\fbackref{46:2} So 1QIsa\textsuperscript{a} LXX (pl.); 4QIsa\textsuperscript{b} MT (sing.)} into captivity. \\
\poeml \v{3}``Listen\fnote{\fbackref{46:3} So 1QIsa\textsuperscript{a} (sing.); MT LXX (pl.)} to me, house of Jacob, \\
\poemll    and all you remnant of the house of Israel, \\
\poeml who have been upheld from before your birth, \\
\poemll    and who have been carried from the womb. \\
\poeml \v{4}Even\fnote{\fbackref{46:4} So 1QIsa\textsuperscript{a}; MT reads \fbib{and even}} until your\fnote{\fbackref{46:4} 1QIsa\textsuperscript{a} MT LXX lack \fbib{your}} old age, I am the one, \\
\poemll    and I'll carry you even until your gray hairs come.\fnote{\fbackref{46:4} 1QIsa\textsuperscript{a} MT LXX lack \fbib{come}} \\
\poeml It is I who have created,\fnote{\fbackref{46:4} Or \fbib{made}} and I who will carry, \\
\poemll    and it is I who will bear and save. \\
\poeml \v{5}``To whom will you compare me, \\
\poemll    count me equal,\fnote{\fbackref{46:5} So 1QIsa\textsuperscript{a} (sing.); 1QIsa\textsuperscript{b} MT read \fbib{consider equal} (pl.); LXX reads \fbib{see}} or liken me, \\
\poemlll       so that I\fnote{\fbackref{46:5} So 1QIsa\textsuperscript{a}; MT reads \fbib{we}} may be compared? \\
\poeml \v{6}Those who pour out gold in\fnote{\fbackref{46:6} So 1QIsa\textsuperscript{a}; MT LXX read \fbib{from}} a purse, \\
\poemll    weigh silver in a balance, \\
\poeml hire a goldsmith in order to make\fnote{\fbackref{46:6} So 1QIsa\textsuperscript{a} LXX; 4QIsa\textsuperscript{b} MT read \fbib{make it}} a god, \\
\poemll    and then\fnote{\fbackref{46:6} Lit. \fbib{and}; so 1QIsa\textsuperscript{a} 1QIsa\textsuperscript{b} LXX; MT lacks \fbib{then}} they bow down and even worship it. \\
\poeml \v{7}And\fnote{\fbackref{46:7} So 1QIsa\textsuperscript{a}; 1QIsa\textsuperscript{b} MT LXX lack \fbib{And}} they lift it on their shoulders, carry it, \\
\poemll    set it up in its place, and there it stands. \\
\poemlll       It cannot move\fnote{\fbackref{46:7} So 1QIsa\textsuperscript{a} LXX; MT reads \fbib{one cannot remove it}} from that spot. \\
\poeml One may even call to\fnote{\fbackref{46:7} So 1QIsa\textsuperscript{a}; MT reads \fbib{may cry out}} it, but it cannot answer \\
\poemll    nor save him from his distress. \\
\poeml \v{8}``Remember this, and stand firm; \\
\poemll    take it again to heart, you rebels. \\
\poeml \v{9}Remember the former things from long ago, \\
\poemll    Because I am God, and there is no one else; \\
\poemlll       I am God, and there is none like me. \\
\poeml \v{10}I declare from the beginning things to follow,\fnote{\fbackref{46:10} So 1QIsa\textsuperscript{a} LXX 4QIsa\textsuperscript{c}; MT reads \fbib{the future}} \\
\poemll    and from ancient times things that have not yet been done; \\
\poeml saying, `My purpose will stand, \\
\poemll    and he\fnote{\fbackref{46:10} So 1QIsa\textsuperscript{a}; 1QIsa\textsuperscript{b} 4QIsa\textsuperscript{c} MT LXX read \fbib{I}} will accomplish everything that I please.' \\
\poeml \v{11}I am calling a bird of prey from the east, \\
\poemll    and from a far country a man with his\fnote{\fbackref{46:11} So 1QIsa\textsuperscript{a} 4QIsa\textsuperscript{d} MT; 1QIsa\textsuperscript{b} MT (vocalization) LXX read \fbib{of my}} purpose. \\
\poeml Indeed, I've spoken; \\
\poemll    I will certainly make it happen; \\
\poeml I've planned it;\fnote{\fbackref{46:11} So 1QIsa\textsuperscript{a}; 1QIsa\textsuperscript{b} 4QIsa\textsuperscript{c} MT LXX lack \fbib{it}} \\
\poemll    and I will certainly carry it out. \\
\poeml \v{12}``Listen to me, you hard-hearted, \\
\poemll    you who are far removed from righteousness: \\
\poeml \v{13}My righteousness is brought near\fnote{\fbackref{46:13} So 1QIsa\textsuperscript{a}; MT LXX 4QIsa\textsuperscript{c} read \fbib{I have brought near;}} and\fnote{\fbackref{46:13} So 1QIsa\textsuperscript{a}; MT lacks \fbib{and}} it's not far off, \\
\poemll    and my salvation won't delay. \\
\poeml I'll\fnote{\fbackref{46:13} So 1QIsa\textsuperscript{a} LXX; 1QIsa\textsuperscript{b} 4QIsa\textsuperscript{c} MT read \fbib{And I}} grant salvation in Zion, \\
\poemll    and\fnote{\fbackref{46:13} So 1QIsa\textsuperscript{a} 4QIsa\textsuperscript{c}; 1QIsa\textsuperscript{b} 4QIsa\textsuperscript{d} MT LXX lack \fbib{and}} to Israel, my glory.''
\end{poetry}
\labelchapt{47}
\passage{The Fall of Babylon}

\begin{poetry}
\poeml \chapt{47}
\v{1}``Come down and sit in the dust, \\
\poemll    Virgin Daughter of Babylon. \\
\poeml Sit on\fnote{\fbackref{47:1} So 1QIsa\textsuperscript{a}; 1QIsa\textsuperscript{b} MT use another preposition} the ground without a chair, \\
\poemll    Daughter of the Chaldeans! \\
\poeml For no longer will they call you \\
\poemll    tender and attractive. \\
\poeml \v{2}Take millstones and grind flour. \\
\poemll    Remove your veil, \\
\poeml strip off your robes,\fnote{\fbackref{47:2} So 1QIsa\textsuperscript{a}; 1QIsa\textsuperscript{b} 4QIsa\textsuperscript{d} MT read \fbib{skirt}} \\
\poemll    bare your legs, \\
\poemlll       and wade through the rivers. \\
\poeml \v{3}Your nakedness will be\fnote{\fbackref{47:3} So 1QIsa\textsuperscript{a}; 1QIsa\textsuperscript{b} MT read \fbib{let it be}} exposed, \\
\poemll    and your disgrace will also be seen. \\
\poeml I'll take vengeance, \\
\poemll    and I will spare no mortal. \\
\poeml \v{4}``Our Redeemer--- \\
\poemll    the \divine{Lord} of the Heavenly Armies is his name--- \\
\poemlll       is the Holy One of Israel. \\
\poeml \v{5}``Sit silent,\fnote{\fbackref{47:5} 1QIsa\textsuperscript{a} and MT use alternate forms of the same term} and enter into the darkness, \\
\poemll    you daughter of the Chaldeans; \\
\poeml for no more will they call you \\
\poemll    Queen of Kingdoms. \\
\poeml \v{6}I was angry with my people, \\
\poemll    and\fnote{\fbackref{47:6} So 1QIsa\textsuperscript{a}; 1QIsa\textsuperscript{b} MT lack \fbib{and}} I desecrated my heritage, \\
\poeml and gave them into your control. \\
\poemll    You showed them no mercy; \\
\poemlll       even on the aged you laid your yoke most heavily. \\
\poeml \v{7}You said, `I will always continue---Queen forever!' \\
\poemll    You didn't take these things into your thinking, \\
\poemlll       nor did you think about their consequences.\fnote{\fbackref{47:7} 1QIsa\textsuperscript{a} and MT use alternate forms; LXX reads \fbib{the last things}} \\
\poeml \v{8}``Now hear this, you wanton creature, \\
\poemll    lounging with no cares, \\
\poeml and saying to herself: \\
\poemll    `I am the one, and there will be none besides me; \\
\poeml I won't live as a widow, \\
\poemll    nor will I see\fnote{\fbackref{47:8} So 1QIsa\textsuperscript{a}; 4QIsa\textsuperscript{d} MT LXX read \fbib{know}} the loss of children.' \\
\poeml \v{9}Both of these things will overtake you \\
\poemll    suddenly on a single day: \\
\poeml loss of children and widowhood. \\
\poemll    They will come upon you in full measure, \\
\poeml despite the multitude of your incantations \\
\poemll    and the great power of your spells.''
\passage{Self-Deception of the Babylonians}
\poeml \v{10}``You trusted in your own knowledge.\fnote{\fbackref{47:10} So 1QIsa\textsuperscript{a}; MT LXX read \fbib{evil}} \\
\poemll    You said, `No one sees me.' \\
\poeml Your wisdom and knowledge have misled you. \\
\poemll    You said in your heart, \\
\poemlll       `I am the one, and there will be none besides me.' \\
\poeml \v{11}``But disaster will come\fnote{\fbackref{47:11} So 1QIsa\textsuperscript{a} (feminine); MT (masculine [incorrectly])} upon you, \\
\poemll    and\fnote{\fbackref{47:11} So 1QIsa\textsuperscript{a} LXX; MT lacks \fbib{and}} you will not know how to charm it away. \\
\poeml A calamity will befall you \\
\poemll    that you will not be able to\fnote{\fbackref{47:11} So 1QIsa\textsuperscript{a}; MT lacks \fbib{to}} ward off; \\
\poeml and devastation will come upon you suddenly, \\
\poemll    and\fnote{\fbackref{47:11} So 1QIsa\textsuperscript{a} LXX; MT lacks \fbib{and}} you won't anticipate it. \\
\poeml \v{12}``But\fnote{\fbackref{47:12} So 1QIsa\textsuperscript{a}; 1QIsa\textsuperscript{b} MT LXX lack \fbib{But}} stand up now with your spells \\
\poemll    and your many incantations, \\
\poeml at which you have labored from your childhood until today,\fnote{\fbackref{47:12} So 1QIsa\textsuperscript{a}; \fbib{childhood. Perhaps you can gain some profit; perhaps you may inspire fear}. LXX reads \fbib{if you can gain some profit}.} \\
\poeml \v{13}according to\fnote{\fbackref{47:12} So 1QIsa\textsuperscript{a}; MT LXX read \fbib{You are wearied by}} your multiple schemes. \\
\poeml Let them stand up now--- \\
\poemll    those who conjure\fnote{\fbackref{47:12} So 1QIsa\textsuperscript{a} LXX; MT reads \fbib{divide}} the heavens \\
\poeml and\fnote{\fbackref{47:12} So 1QIsa\textsuperscript{a}; MT LXX lack \fbib{and}} gaze at the stars, \\
\poemll    predicting at the new moons--- \\
\poemlll       and save you from what is about to happen to them.\fnote{\fbackref{47:12} So 1QIsa\textsuperscript{a}; MT reads \fbib{what things are about to happen to you}; LXX reads \fbib{what is about to happen to you}} \\
\poeml \v{14}``See, they are just like stubble; \\
\poemll    fire burns them up. \\
\poeml They could not\fnote{\fbackref{47:14} So 1QIsa\textsuperscript{a}; MT reads \fbib{cannot}} even save themselves \\
\poemll    from the power of the flame. \\
\poeml There will be no coals for warming oneself, \\
\poemll    no fire to sit by. \\
\poeml \v{15}So will they be to you---those with whom you toiled \\
\poemll    and did business since your childhood--- \\
\poeml they wander about, each in his own direction; \\
\poemll    there is not one who can save you.
\end{poetry}
\labelchapt{48}
\passage{God the Creator and Redeemer}

\begin{poetry}
\poeml \chapt{48}
\v{1}``Listen to this, house of Jacob, \\
\poemll    you who are called by the name of Israel, \\
\poemlll       and who have come forth from Judah's loins;\fnote{\fbackref{48:1} Lit. \fbib{waters}; this word is misspelled in both 1QIsa\textsuperscript{a} and MT} \\
\poeml you who swear oaths in the name of the \divine{Lord} \\
\poemll    and invoke the God of Israel--- \\
\poemlll       but not in truth, nor in good faith. \\
\poeml \v{2}For they name themselves after the holy city, \\
\poemll    and rely on the God of Israel--- \\
\poemlll       the \divine{Lord} of the Heavenly Armies is his name. \\
\poeml \v{3}I foretold the former things long ago; \\
\poemll    it\fnote{\fbackref{48:3} So 1QIsa\textsuperscript{a}; MT LXX read \fbib{they}} went forth from my mouth, \\
\poemlll       and I disclosed them; \\
\poeml Suddenly, I acted, \\
\poemll    and they came to pass. \\
\poeml \v{4}Because I knew\fnote{\fbackref{48:4} Lit. \fbib{Because of my knowledge}} that you are obstinate, \\
\poemll    and because your neck is an iron sinew, \\
\poemlll       and your forehead is bronze, \\
\poeml \v{5}I told you these things long ago; \\
\poemll    I announced them to you before they happened \\
\poeml so that you couldn't say, `My idol did them; \\
\poemll    my\fnote{\fbackref{48:5} So 1QIsa\textsuperscript{a}; MT reads \fbib{and my}} carved image or metal idol ordained them.' \\
\poeml \v{6}``You have heard---now look at them all! \\
\poemll    How\fnote{\fbackref{48:6} Lit. \fbib{And how}} can you not admit them? \\
\poeml From now on, I'll make you hear new things, \\
\poemll    hidden things that you have not known. \\
\poeml \v{7}They are created now, and not long ago; \\
\poemll    you didn't hear them before today, \\
\poemlll       so you cannot say, `Yes, I knew them.' \\
\poeml \v{8}And\fnote{\fbackref{48:8} So 1QIsa\textsuperscript{a}; 4QIsa\textsuperscript{b} MT LXX lack \fbib{And}} neither had you heard, nor did you understand, \\
\poemll    nor did you open\fnote{\fbackref{48:8} So 1QIsa\textsuperscript{a} Targ; MT reads \fbib{did your ear open itself}; CaiGen Syr Vulg read \fbib{was your ear uncovered}} your ear long ago.\fnote{\fbackref{48:8} Lit. \fbib{ear from of old}} \\
\poeml Indeed, I knew that\fnote{\fbackref{48:8} So 1QIsa\textsuperscript{a} LXX; MT lacks \fbib{that}} you would act very deceitfully, \\
\poemll    and they would call\fnote{\fbackref{48:8} So 1QIsa\textsuperscript{a}; LXX Targ read \fbib{and you would be called}; 4QIsa\textsuperscript{d} MT read \fbib{deceitfully, calling}} you a rebel from birth. \\
\poeml \v{9}I defer my anger for my name's sake, \\
\poemll    and as my first act\fnote{\fbackref{48:9} Lit. \fbib{and for my commencement}; or \fbib{and for my profanation}; so 1QIsa\textsuperscript{a}; 4QIsa\textsuperscript{d} MT read \fbib{for my praise}; LXX reads \fbib{for my glorious deeds}} I'm restraining it for you, \\
\poemlll       so as not to cut you off. \\
\poeml \v{10}Look, I have refined you, but not like silver; \\
\poemll    I have purified\fnote{\fbackref{48:10} So 1QIsa\textsuperscript{a}; MT LXX read \fbib{chosen}} you in the furnace of affliction. \\
\poeml \v{11}For my own sake---Yes, for my own sake!---I'm doing it; \\
\poemll    indeed, how can I be profaned?\fnote{\fbackref{48:11} Or \fbib{can I wait}; so 1QIsa\textsuperscript{a}; 4QIsa\textsuperscript{d} MT read \fbib{can it be profaned}; 4QIsa\textsuperscript{d} MT read \fbib{can it wait}; LXX reads \fbib{can my name be profaned}} \\
\poemlll       Furthermore, I won't give my glory to another.''
\passage{The \divine{Lord} Calls Israel}
\poeml \v{12}``Listen to these things,\fnote{\fbackref{48:12} So 1QIsa\textsuperscript{a}; MT LXX read \fbib{me}} Jacob, \\
\poemll    and Israel, whom I have called. \\
\poeml I am the One: I am the first, \\
\poemll    I am even\fnote{\fbackref{48:12} So 1QIsa\textsuperscript{a} MT; 4QIsa\textsuperscript{d} reads \fbib{also}} the last. \\
\poeml \v{13}Moreover, my hands laid\fnote{\fbackref{48:13} So 1QIsa\textsuperscript{a}; MT LXX read \fbib{hand laid}} the earth's foundation, \\
\poemll    and my right hand spread out the heavens. \\
\poeml I call out to them, \\
\poemll    and\fnote{\fbackref{48:13} So 1QIsa\textsuperscript{a} 4QIsa\textsuperscript{c} 4QIsa\textsuperscript{d} LXX; MT lacks \fbib{and}} they stand up together. \\
\poeml \v{14}Let all of them come together and listen:\fnote{\fbackref{48:14} So 1QIsa\textsuperscript{a} LXX; 4QIsa\textsuperscript{d} MT read \fbib{Come together, all of you, and listen!}} \\
\poemll    Who is there among them that could declare\fnote{\fbackref{48:14} So 1QIsa\textsuperscript{a}; 4QIsa\textsuperscript{d} MT LXX read \fbib{has declared}} these things? \\
\poeml ``The \divine{Lord} loves me,\fnote{\fbackref{48:14} So 1QIsa\textsuperscript{a}; 4QIsa\textsuperscript{d} MT read \fbib{him}; LXX reads \fbib{you}} \\
\poemll    and he will accomplish\fnote{\fbackref{48:14} So 1QIsa\textsuperscript{a} (misspelling \fbib{carry out}); 4QIsa\textsuperscript{d} MT read \fbib{He will carry out}; LXX reads \fbib{I have carried out}} my purpose\fnote{\fbackref{48:14} So 1QIsa\textsuperscript{a}; 4QIsa\textsuperscript{d} MT LXX\textsuperscript{ms} read \fbib{his purpose}; LXX reads \fbib{your purpose}} against Babylon; \\
\poemlll       his arm\fnote{\fbackref{48:14} I.e. the Messiah; so 1QIsa\textsuperscript{a}; MT reads \fbib{and his arm}; LXX reads \fbib{to do away with the offspring}} will be against the Chaldeans. \\
\poeml \v{15}I---Yes, I!---have spoken; \\
\poemll    indeed, I've called and\fnote{\fbackref{48:15} So 1QIsa\textsuperscript{a}; 4QIsa\textsuperscript{d} MT read \fbib{called him}; 4QIsa\textsuperscript{c} LXX read \fbib{brought him}} I've brought him, \\
\poemlll       and he will make his path successful.\fnote{\fbackref{48:15} Or \fbib{his path will be successful}; so 1QIsa\textsuperscript{a} (feminine) and 4QIsa\textsuperscript{d} MT (masculine); 4QIsa\textsuperscript{c} LXX read \fbib{I will make his path successful}} \\
\poeml \v{16}Draw near to me, and\fnote{\fbackref{48:16} So 1QIsa\textsuperscript{a} LXX; MT lacks \fbib{and}} listen to this: \\
\poemll    `From the beginning I haven't spoken in secret; \\
\poeml at\fnote{\fbackref{48:16} So 1QIsa\textsuperscript{a} LXX; MT reads \fbib{from}} the time it happened, I was there.' \\
\poemll    And now the \divine{Lord} God, and his Spirit, has sent me.\fnote{\fbackref{48:16} Or \fbib{has sent me and his Spirit}.} \\
\poeml \v{17}``This is what the \divine{Lord} says, \\
\poemll    your Redeemer, the Holy One of Israel: \\
\poeml ``I am the \divine{Lord} your God, \\
\poemll    who teaches you how to succeed, \\
\poemlll       who directs you\fnote{\fbackref{48:17} 1QIsa\textsuperscript{a} misspells this word} in the path by which\fnote{\fbackref{48:17} So 1QIsa\textsuperscript{a} LXX; 1QIsa\textsuperscript{b} 4QIsa\textsuperscript{c} 4QIsa\textsuperscript{d} MT lack \fbib{which}} you should go. \\
\poeml \v{18}Now\fnote{\fbackref{48:18} So 1QIsa\textsuperscript{a} 1QIsa\textsuperscript{b} 4QIsa\textsuperscript{c} LXX; MT lacks \fbib{Now}} if only you had paid attention to my commandments! \\
\poemll    Then your peace would have been like a river, \\
\poemlll       and your success like the waves of the sea. \\
\poeml \v{19}Your descendants would've been like the sand, \\
\poemll    and your offspring\fnote{\fbackref{48:19} So 1QIsa\textsuperscript{a}; 1QIsa\textsuperscript{b} MT LXX read \fbib{the offspring of your loins}} like its numberless grains. \\
\poeml Their name wouldn't have been cut off \\
\poemll    or annihilated out of my reach. \\
\poeml \v{20}``Go out from Babylon, flee from the Chaldeans! \\
\poemll    With happy shouts, announce \\
\poeml and\fnote{\fbackref{48:20} So 1QIsa\textsuperscript{a} LXX; 1QIsa\textsuperscript{b} MT lack \fbib{and}} proclaim this\fnote{\fbackref{48:20} So 1QIsa\textsuperscript{a}; 1QIsa\textsuperscript{b} 4QIsa\textsuperscript{d} MT LXX read \fbib{this. Send it forth}} to the ends\fnote{\fbackref{48:20} So 1QIsa\textsuperscript{a}; 4QIsa\textsuperscript{d} MT LXX read \fbib{end}} of the earth: \\
\poemll    Say, `The \divine{Lord} has redeemed his servant Jacob!' \\
\poeml \v{21}They didn't thirst when he led him\fnote{\fbackref{48:21} So 1QIsa\textsuperscript{a} LXX\textsuperscript{mss}; 4QIsa\textsuperscript{d} MT LXX read \fbib{them}} through the desolate places. \\
\poemll    He made water gush\fnote{\fbackref{48:21} So 1QIsa\textsuperscript{a} Syr; cf. Ps 78:20 and 105:41; 1QIsa\textsuperscript{b} 4QIsa\textsuperscript{d} MT read \fbib{flow;} LXX reads \fbib{he will bring forth}} from a rock for them; \\
\poemlll       he split open the rock, and water gushed out. \\
\poeml \v{22}``But\fnote{\fbackref{48:22} So 1QIsa\textsuperscript{a}; 1QIsa\textsuperscript{b} MT LXX lack \fbib{But}} there is no peace,'' says the \divine{Lord}, ``for the wicked.''
\end{poetry}
\labelchapt{49}
\passage{The Servant of the \divine{Lord}}

\begin{poetry}
\poeml \chapt{49}
\v{1}``Listen to me, you coastlands! \\
\poemll    Pay\fnote{\fbackref{49:1} So 1QIsa\textsuperscript{a} LXX; 1QIsa\textsuperscript{b} MT read \fbib{and pay}} attention, you people\fnote{\fbackref{49:1} Lit. \fbib{peoples}; i.e. non-Israelis then in the land} from far away! \\
\poeml The \divine{Lord} called me from the womb; \\
\poemll    while I was still in my mother's body, \\
\poemlll       he pronounced my name. \\
\poeml \v{2}He made my mouth like a sharp sword; \\
\poemll    he hid me in the shadow of his hands.\fnote{\fbackref{49:2} So 1QIsa\textsuperscript{a}; 1QIsa\textsuperscript{b} MT LXX read \fbib{hand}} \\
\poeml He made me like a polished arrow \\
\poemll    and hid me away in his quivers.\fnote{\fbackref{49:2} So 1QIsa\textsuperscript{a}; 1QIsa\textsuperscript{b} 4QIsa\textsuperscript{d} MT LXX read \fbib{quiver}} \\
\poeml \v{3}He said to me: `You are my servant, \\
\poemll    Israel, in whom I will glorify myself.' \\
\poeml \v{4}``I\fnote{\fbackref{49:4} So 1QIsa\textsuperscript{a}; 4QIsa\textsuperscript{d} MT LXX read \fbib{But I}} said: `I've labored for nothing. \\
\poemll    I've exhausted my strength on futility and on\fnote{\fbackref{49:4} So 1QIsa\textsuperscript{a} LXX; 1QIsa\textsuperscript{b} MT lack \fbib{on}} emptiness.' \\
\poeml Yet surely my recompense is with the \divine{Lord}, \\
\poemll    and my reward is with my God. \\
\poeml \v{5}``And now, says the \divine{Lord}, \\
\poemll    who formed you\fnote{\fbackref{49:5} So 1QIsa\textsuperscript{a}; 1QIsa\textsuperscript{b} MT LXX \fbib{me}} from the womb as his servant \\
\poeml to bring Jacob back to him \\
\poemll    so that Israel might be gathered\fnote{\fbackref{49:5} So 1QIsa\textsuperscript{a} MT\textsuperscript{qere} LXX; 4QIsa\textsuperscript{d} MT \fbib{might not be gathered} (misspelling)} to him--- \\
\poeml and I am honored in the \divine{Lord}'s sight \\
\poemll    and my God has been my help\fnote{\fbackref{49:5} So 1QIsa\textsuperscript{a}; 1QIsa\textsuperscript{b} MT LXX \fbib{strength}}--- \\
\poeml \v{6}he says: ``It is too small a thing for you to be my servant, \\
\poemll    to raise up the tribes of Israel\fnote{\fbackref{49:6} So 1QIsa\textsuperscript{a}; 1QIsa\textsuperscript{b} MT LXX \fbib{Jacob}} \\
\poemlll       and bring back those of Jacob\fnote{\fbackref{49:6} So 1QIsa\textsuperscript{a}; 1QIsa\textsuperscript{b} MT LXX \fbib{Israel}} I have preserved. \\
\poeml I'll also make you as a light to the nations, \\
\poemll    to be my salvation to the ends\fnote{\fbackref{49:6} So 1QIsa\textsuperscript{a}; 1QIsa\textsuperscript{b} MT LXX \fbib{end}} of the earth. \\
\poeml \v{7}``This is what my \divine{Lord}\fnote{\fbackref{49:7} So 1QIsa\textsuperscript{a} 1QIsa\textsuperscript{b}; MT LXX lacks \fbib{my Lord}} says--- \\
\poemll    the \divine{Lord} your Redeemer, O Israel,\fnote{\fbackref{49:7} So 1QIsa\textsuperscript{a} LXX; 1QIsa\textsuperscript{b} MT read \fbib{the Redeemer of Israel}} \\
\poemlll       and his Holy One--- \\
\poeml to one despised by people,\fnote{\fbackref{49:7} So 1QIsa\textsuperscript{a} 4QIsa\textsuperscript{d}; MT CaiGen LXX reads \fbib{to one people despise}} \\
\poemll    to those abhorred\fnote{\fbackref{49:7} So 1QIsa\textsuperscript{a}; MT LXX reads \fbib{one abhorred}} as a nation, \\
\poemlll       to the servant of rulers: \\
\poeml ``Kings see\fnote{\fbackref{49:7} So 1QIsa\textsuperscript{a}; 1QIsa\textsuperscript{b} MT read \fbib{Kings will see}; LXX reads \fbib{Kings will see him}} and arise, \\
\poemll    and princes\fnote{\fbackref{49:7} So 1QIsa\textsuperscript{a}; 1QIsa\textsuperscript{b} reads \fbib{They will rise}; MT LXX read \fbib{princes will rise, and they}} will bow down, \\
\poeml because of the \divine{Lord} who is faithful, \\
\poemll    the Holy One of Israel, \\
\poemlll       the one who has\fnote{\fbackref{49:7} So 1QIsa\textsuperscript{a}; MT reads \fbib{and the one who has}; LXX reads \fbib{and I have}} chosen you.''
\passage{The Restoration of Israel}
\poeml \v{8}``This what the \divine{Lord} says: \\
\poeml ``I'll answer\fnote{\fbackref{49:8} So 1QIsa\textsuperscript{a}; MT LXX reads \fbib{I have answered}} you in a time of favor, \\
\poemll    and on a day of salvation I'll help\fnote{\fbackref{49:8} So 1QIsa\textsuperscript{a}; 1QIsa\textsuperscript{b} MT LXX read \fbib{I have helped}} you. \\
\poeml I have watched over you, \\
\poemll    and given you as a covenant for the people, \\
\poeml to restore the land, \\
\poemll    to reassign the inheritances that have been devastated; \\
\poeml \v{9}saying to captives, `Come out!' \\
\poemll    and\fnote{\fbackref{49:9} So 1QIsa\textsuperscript{a} LXX; MT lack \fbib{and}} to those who are in darkness, `Be free!'\fnote{\fbackref{49:9} Or \fbib{darkness,} `\fbib{Show yourselves!'}} \\
\poeml ``They will feed on all the mountains,\fnote{\fbackref{49:9} So 1QIsa\textsuperscript{a}; MT reads \fbib{by the roads}; LXX reads \fbib{in all their roads}} \\
\poemll    and their pasture will be on all the barren hills. \\
\poeml \v{10}They won't hunger or thirst, \\
\poemll    nor will the desert heat or sun beat upon them; \\
\poeml for the one who has compassion on them will drive them \\
\poemll    and guide them alongside springs of water. \\
\poeml \v{11}I'll turn all my mountains into a road, \\
\poemll    and my highways will be raised up. \\
\poeml \v{12}``Watch! They'll come from far away--- \\
\poemll    some from the north and from the west, \\
\poemlll       and others from the Aswan region.''\fnote{\fbackref{49:12} Lit. \fbib{Syenes}; so 1QIsa\textsuperscript{a}; MT reads \fbib{Syene}; LXX reads \fbib{Persians}} \\
\poeml \v{13}Shout with joy, you heavens, \\
\poemll    and rock with glee, you earth! \\
\poemlll       Break out in song, you mountains!\fnote{\fbackref{49:13} So 1QIsa\textsuperscript{a}; MT LXX read \fbib{Let the mountains break out}; MT\textsuperscript{qere, mss} read \fbib{And break out, you mountains}} \\
\poeml The \divine{Lord} is comforting\fnote{\fbackref{49:13} So 1QIsa\textsuperscript{a}; MT LXX read \fbib{has comforted}} his people \\
\poemll    and will have compassion on his afflicted ones.
\passage{Zion Not Forgotten}
\poeml \v{14}``But Zion said, `The \divine{Lord} has abandoned me, \\
\poemll    and my God\fnote{\fbackref{49:14} So 1QIsa\textsuperscript{a} corrector; Lit. \fbib{my God} written above \fbib{my Lord}; MT LXX read \fbib{my Lord}} has forgotten me.' \\
\poeml \v{15}``Can a woman forget her nursing child, \\
\poemll    or have no compassion for the child of her womb? \\
\poeml Even these mothers may forget; \\
\poemll    But as for me, I'll never forget you! \\
\poeml \v{16}Look! I've inscribed you on the palms of my hands, \\
\poemll    and\fnote{\fbackref{49:16} So 1QIsa\textsuperscript{a}; MT LXX read \fbib{and}} your walls are forever before me. \\
\poeml \v{17}Your builders\fnote{\fbackref{49:17} So 1QIsa\textsuperscript{a} Aquila Vulg LXX; MT reads \fbib{sons}} are working faster than your destroyers, \\
\poemll    and those who devastated you go away from you. \\
\poeml \v{18}Lift up your eyes and look around--- \\
\poemll    they have all gathered together \\
\poemlll       and are coming to you. \\
\poeml ``As surely as I live,'' says the \divine{Lord}, \\
\poemll    ``you will clothe yourself with all of them like ornaments, \\
\poemlll       and tie them on like a bride. \\
\poeml \v{19}Indeed, your ruins, your desolate places, \\
\poemll    and your devastated land \\
\poeml will now be too crowded for your inhabitants, \\
\poemll    while those who swallowed you up will be far away. \\
\poeml \v{20}``The children who are grieving at present\fnote{\fbackref{49:20} Lit. \fbib{children of your bereavement}} \\
\poemll    will yet say in your hearing, \\
\poeml `This place is too crowded for me; \\
\poemll    make room for me, \\
\poemlll       so I may have a place to live.' \\
\poeml \v{21}Then you'll ask\fnote{\fbackref{49:21} Lit. \fbib{say}} in your heart, \\
\poemll    `Who bore these children for me, \\
\poeml although I was childless and barren, \\
\poemll    and\fnote{\fbackref{49:21} So 1QIsa\textsuperscript{a}; MT LXX lack \fbib{and}} an exile\fnote{\fbackref{49:21} LXX lacks \fbib{and an exile}} and cast aside? \\
\poeml Who\fnote{\fbackref{49:21} So 1QIsa\textsuperscript{a}. MT LXX read \fbib{And who}} brought these up? \\
\poemll    Look!\fnote{\fbackref{49:21} So 1QIsa\textsuperscript{a} MT; LXX lacks \fbib{Look!}} For my part I was left all alone; \\
\poemlll       but as for these, where have they come from?' \\
\poeml \v{22}``For\fnote{\fbackref{49:22} So 1QIsa\textsuperscript{a}; MT LXX lack \fbib{For}} this what the \divine{Lord}\fnote{\fbackref{9:22} So 1QIsa\textsuperscript{a} LXX; MT reads \fbib{my Lord the \divine{Lord}}} says, \\
\poemll    `Watch! I'll lift up my hand to the nations \\
\poeml and raise my banner to the\fnote{\fbackref{49:22} So 1QIsa\textsuperscript{a}; MT lacks \fbib{the}} peoples.\fnote{\fbackref{49:22} So 1QIsa\textsuperscript{a} MT; LXX reads \fbib{islands}} \\
\poemll    They will bring your sons in their arms, \\
\poemlll       and your daughters will be carried on their shoulders.' \\
\poeml \v{23}``Oh, yes!\fnote{\fbackref{49:23} Or \fbib{And aha!}; so 1QIsa\textsuperscript{a} cf. Isa 55:1; MT LXX read \fbib{And it will happen that}} Kings will be your foster fathers, \\
\poemll    and their queens will be your nursing mothers. \\
\poeml They will bow to you with their faces to the ground, \\
\poemll    and lick the dust from your feet. \\
\poeml Then you will know that I am the \divine{Lord}; \\
\poemll    those who hope in me will not be disappointed. \\
\poeml \v{24}``Can they seize plunder\fnote{\fbackref{49:24} So 1QIsa\textsuperscript{a} LXX; MT reads \fbib{plunder be seized}} from warriors, \\
\poemll    or\fnote{\fbackref{49:24} So 1QIsa\textsuperscript{a} LXX; MT lacks \fbib{or}} can the captives of tyrants\fnote{\fbackref{49:24} So 1QIsa\textsuperscript{a} LXX Targ Vulg; MT reads \fbib{righteous ones}} be rescued? \\
\poeml \v{25}But this is what the \divine{Lord} says: \\
\poeml ``He will seize\fnote{\fbackref{49:25} So 1QIsa\textsuperscript{a} LXX; MT reads \fbib{will be seized};} even the warriors' plunder,\fnote{\fbackref{49:25} So 1QIsa\textsuperscript{a}; MT LXX reads \fbib{captives}} \\
\poemll    and the captives\fnote{\fbackref{49:25} So 1QIsa\textsuperscript{a}; MT LXX reads \fbib{plunder}} of tyrants will be rescued. \\
\poeml I myself will quarrel with those who have a quarrel with you,\fnote{\fbackref{49:25} So 1QIsa\textsuperscript{a}; MT Vulg read \fbib{with your contender}; LXX reads \fbib{your cause}} \\
\poemll    and I myself will save your children. \\
\poeml \v{26}``I'll make those who mistreat you\fnote{\fbackref{49:26} So 1QIsa\textsuperscript{a} probably misspells this word as \fbib{I will eat}.} eat their own flesh, \\
\poemll    and they will get drunk on their own blood, as with new wine. \\
\poeml ``Then all mankind will know that I am the \divine{Lord} \\
\poemll    your Savior and your Redeemer, \\
\poemlll       the Mighty One of Jacob.''
\end{poetry}
\labelchapt{50}
\passage{A Call to Return to God}

\begin{poetry}
\poeml \chapt{50}
\v{1}This is what the \divine{Lord} says: \\
\poeml ``Where is your mother's certificate of divorce \\
\poemll    with which I sent her away? \\
\poeml Or to which of my creditors did I sell you? \\
\poemll    Look! It's because of your sins that you were sold, \\
\poemlll       and because of your transgressions that your mother was sent away. \\
\poeml \v{2}Why is it that when I came, no one was there? \\
\poemll    Why was there no answer when I called? \\
\poeml Was my arm\fnote{\fbackref{50:2} Lit. \fbib{my hand}} too short to redeem you? \\
\poemll    Do\fnote{\fbackref{50:2} So 1QIsa\textsuperscript{a}; MT LXX read \fbib{Or do}} I lack the strength to rescue you? \\
\poeml Look! By my mere rebuke I dry up the sea, \\
\poemll    I turn rivers into a desert. \\
\poeml Their fish stink for lack of water \\
\poemll    and die of thirst. \\
\poeml \v{3}I clothe the skies with darkness \\
\poemll    and make sackcloth their covering.''
\passage{The Servant's Obedience}
\poeml \v{4}``The Lord \divine{God} has given me \\
\poemll    a learned tongue, so that I may know \\
\poemlll       how to sustain the weary with words. \\
\poeml And\fnote{\fbackref{50:4} So 1QIsa\textsuperscript{a}; MT LXX lack \fbib{And}} morning after morning he wakens, \\
\poemll    and\fnote{\fbackref{50:4} So 1QIsa\textsuperscript{a}; MT LXX lack \fbib{and}} he wakens my ear to \\
\poemlll       listen like those who are being taught. \\
\poeml \v{5}My Lord \divine{God}\fnote{\fbackref{50:5} So 1QIsa\textsuperscript{a}; MT reads \fbib{the Lord}} has opened my ears, \\
\poemll    and I did not rebel; \\
\poemlll       I did not shrink back. \\
\poeml \v{6}I gave my back to those who beat me \\
\poemll    and my cheeks to those who pulled out my beard.\fnote{\fbackref{50:6} So MT; 1QIsa\textsuperscript{a} employs an incorrect reading; LXX reads \fbib{to blows}} \\
\poeml I did not turn away\fnote{\fbackref{50:6} So 1QIsa\textsuperscript{a} LXX; MT reads \fbib{hide}} my face \\
\poemll    from insults and spitting. \\
\poeml \v{7}For the Lord \divine{God} helps me, \\
\poemll    so I won't be disgraced. \\
\poeml Therefore I've made my face like flint, \\
\poemll    and I know that I won't be put to shame.''
\passage{The Servant's Vindication}
\poeml \v{8}The one who vindicates me is near. \\
\poemll    Who, then, will bring a charge against me? \\
\poemlll       Let's face each other! \\
\poeml Who has a case against me? \\
\poemll    Let him confront me! \\
\poeml \v{9}See! It is the Lord \divine{God} who will help me. \\
\poemll    Who is it that will declare me guilty? \\
\poeml See! They will all wear out like a garment; \\
\poemll    moths will eat them up. \\
\poeml \v{10}Who among you fears\fnote{\fbackref{50:10} So 1QIsa\textsuperscript{a} (pl.); MT LXX read \fbib{fear} (sing.)} the \divine{Lord}, \\
\poemll    obeying the voice of his servant, \\
\poeml who among you\fnote{\fbackref{50:10} 1QIsa\textsuperscript{a} MT LXX lack \fbib{among you}} walks\fnote{\fbackref{50:10} So 1QIsa\textsuperscript{a} LXX (pl.); MT reads \fbib{walks} (sing.)} in darkness\fnote{\fbackref{50:10} So 1QIsa\textsuperscript{a}; MT employs a different form} \\
\poemll    and has no light? \\
\poeml Let him trust in the name of the \divine{Lord}, \\
\poemll    and rely upon his God. \\
\poeml \v{11}Look! All those\fnote{\fbackref{50:11} So 1QIsa\textsuperscript{a}; MT LXX read \fbib{you}} who light a fire, \\
\poemll    who surround yourselves with flaming torches--- \\
\poeml walk by the light of your fire, \\
\poemll    and by the torches that you have set ablaze! \\
\poeml This is what you will receive from my hand: \\
\poemll    you will lie down in torment.
\end{poetry}
\labelchapt{51}
\passage{Deliverance for Zion}

\begin{poetry}
\poeml \chapt{51}
\v{1}``Listen to me, you who pursue righteousness, \\
\poemll    you who seek the \divine{Lord}! \\
\poeml Look to the rock from which you were cut, \\
\poemll    to the quarry from which you were hewn. \\
\poeml \v{2}Look to Abraham your father, \\
\poemll    and to Sarah who gave you birth. \\
\poeml For when he was only one person I called him, \\
\poemll    but I made him fruitful\fnote{\fbackref{51:2} So 1QIsa\textsuperscript{a}; MT reads \fbib{blessed him}; LXX reads \fbib{blessed him and loved him}} and made him many. \\
\poeml \v{3}For the \divine{Lord} will have compassion on Zion, \\
\poemll    have compassion on all her ruins. \\
\poeml He will make her wilderness like Eden, \\
\poemll    and her deserts like the garden of the \divine{Lord}. \\
\poeml Joy and gladness will be found in her, \\
\poemll    thanksgiving, and the sound of singing.\fnote{\fbackref{51:3} Or \fbib{music}} \\
\poemlll       Sorrow and sighing will flee away.\fnote{\fbackref{51:3} So 1QIsa\textsuperscript{a}; MT LXX lack this line; cf Isa 51:11} \\
\poeml \v{4}``Pay attention to me, my people! \\
\poemll    Listen to me, my nation! \\
\poeml For instruction\fnote{\fbackref{51:4} Or \fbib{For the Law}} will go out from me, \\
\poemll    and my justice will become a light for the nations.\fnote{\fbackref{51:4} Lit. \fbib{peoples}} \\
\poeml I will quickly bring \v{5}my deliverance near; \\
\poemll    my salvation is on the way. \\
\poeml His arm\fnote{\fbackref{51:5} So 1QIsa\textsuperscript{a}; 1QIsa\textsuperscript{b} reads \fbib{My arms}; or \fbib{My arm}; MT LXX read \fbib{My arms}} will bring justice to\fnote{\fbackref{51:5} The verb is pl. in 1QIsa\textsuperscript{a} MT} the nations;\fnote{\fbackref{51:5} Lit. \fbib{peoples}} \\
\poemll    the coastlands will hope for him,\fnote{\fbackref{51:5} So 1QIsa\textsuperscript{a}; 1QIsa\textsuperscript{b} MT LXX read \fbib{me}} \\
\poemlll       and they will wait for his arm.\fnote{\fbackref{51:5} So 1QIsa\textsuperscript{a}; 1QIsa\textsuperscript{b} MT LXX read \fbib{my arm}} \\
\poeml \v{6}``Lift up your eyes, you\fnote{\fbackref{51:6} So 1QIsa\textsuperscript{a}; MT LXX read \fbib{to the}} heavens \\
\poemll    and look to the earth beneath; \\
\poemlll       and see who created these.\fnote{\fbackref{51:6} So 1QIsa\textsuperscript{a}; 1QIsa\textsuperscript{b} MT LXX read \fbib{for the heavens will vanish like smoke, and the earth will wear out like a garment}} \\
\poeml Its inhabitants will die just like this;\fnote{\fbackref{51:6} Or \fbib{like gnats}} \\
\poemll    but my salvation will be forever, \\
\poemlll       and my deliverance will never fail. \\
\poeml \v{7}Listen to me, you who know righteousness, \\
\poemll    you people who have my instruction\fnote{\fbackref{51:7} Or \fbib{Law}} in their hearts. \\
\poeml Don't fear the insults of mortals, \\
\poemll    and don't be dismayed at their hateful words.\fnote{\fbackref{51:7} So 1QIsa\textsuperscript{a} 1QIsa\textsuperscript{b}; spelling differs from MT; LXX reads \fbib{contempt}} \\
\poeml \v{8}For moths will eat them up just like a garment, \\
\poemll    and worms will devour them like wool; \\
\poeml but my deliverance will last\fnote{\fbackref{51:8} Lit. \fbib{be}} forever, \\
\poemll    and my salvation to all generations. \\
\poeml \v{9}``Awake! Awake! Clothe yourself with strength, \\
\poemll    you arm\fnote{\fbackref{51:9} I.e. \fbib{the Messiah}} of the \divine{Lord}! \\
\poeml Awake, as in days gone by, \\
\poemll    as in generations of long ago. \\
\poeml Was it not you who split apart\fnote{\fbackref{51:9} So 1QIsa\textsuperscript{a} 4QIsa\textsuperscript{c} Vulg (cf. Job 26:12). MT LXX\textsuperscript{mss} read \fbib{cut in pieces}} Rehob,\fnote{\fbackref{51:9} So 1QIsa\textsuperscript{a} MT qere reads \fbib{Rahab}} \\
\poemll    who pierced that sea monster through?\fnote{\fbackref{51:9} So 1QIsa\textsuperscript{a} 4QIsa\textsuperscript{c} MT Vulg; LXX lacks \fbib{Was it{\ldots}through?}} \\
\poeml \v{10}Was it not you who dried up the sea, \\
\poemll    the waters of the great deep, \\
\poeml who made a road in\fnote{\fbackref{51:10} So 1QIsa\textsuperscript{a}; MT LXX lack \fbib{in}} the depths of the sea \\
\poemll    so that the redeemed could cross over?''
\passage{A Promise of Return to the Land}
\poeml \v{11}``The scattered ones\fnote{\fbackref{51:11} So 1QIsa\textsuperscript{a}; MT LXX read \fbib{ransomed ones}; 1QIsa\textsuperscript{a} corrector wrote \fbib{redeemed} then erased and wrote \fbib{scattered ones}} of the \divine{Lord} will return, \\
\poemll    and they will enter Zion with singing. \\
\poeml Everlasting joy will be upon their heads; \\
\poemll    they will attain joy and gladness, \\
\poemlll       and\fnote{\fbackref{51:11} So 1QIsa\textsuperscript{a} 4QIsa\textsuperscript{c} LXX; MT LXX lack \fbib{and}} sorrow and sighing will flee away.\fnote{\fbackref{51:11} So 1QIsa\textsuperscript{a} MT\textsuperscript{ms} (sing.); cf Isa 51:3; MT (pl.)} \\
\poeml \v{12}``I---yes, I---am the one who comforts you. \\
\poemll    Who are you, that you are so afraid of humans who will die, \\
\poemlll       descendants of mere\fnote{\fbackref{51:12} 1QIsa\textsuperscript{a} MT LXX lack \fbib{mere}} men, who have been made\fnote{\fbackref{51:12} So 1QIsa\textsuperscript{a}; MT reads \fbib{are made}; LXX reads \fbib{will be dried up}} like grass? \\
\poeml \v{13}As a result, you have forgotten the \divine{Lord} who made you, \\
\poemll    who stretched out the heavens \\
\poeml and laid the earth's foundations, \\
\poemll    and you live in constant fear every day \\
\poeml because of the oppressor's fury, \\
\poemll    since he's ready to destroy. \\
\poemlll       Now where is the\fnote{\fbackref{51:13} So 1QIsa\textsuperscript{a}; 1QIsa\textsuperscript{a} omitted \fbib{oppressor's{\ldots}the} then inserted the missing lines at the top of column 43} oppressor's fury? \\
\poeml \v{14}Distress\fnote{\fbackref{51:14} So 1QIsa\textsuperscript{a}; MT reads \fbib{The cowering one}; LXX lacks \fbib{Distress}} will quickly be set free. \\
\poemll    He won't die in the Pit,\fnote{\fbackref{51:14} I.e. the realm of punishment in the afterlife} \\
\poemlll       nor will he lack food.''
\passage{A Promise of Restoration}
\poeml \v{15}``For I am the \divine{Lord} your God, \\
\poemll    who churns up the sea, so that its waves roar, \\
\poemlll       `The \divine{Lord} of the Heavenly Armies is his name.' \\
\poeml \v{16}I have put my words in your mouth \\
\poemll    and have covered you with the shadow of my hand, \\
\poeml so that I could plant the heavens \\
\poemll    and lay the earth's foundations, \\
\poemlll       to say to Zion, `You are my people.' \\
\poeml \v{17}``Awake, Awake! \\
\poemll    Stand up, Jerusalem, \\
\poeml you who have drunk from the \divine{Lord}'s hand \\
\poemll    from the cup that is\fnote{\fbackref{51:17} Lit. \fbib{hand, the cup of}} his anger. \\
\poeml You have drunk to the dregs \\
\poemll    the cup that makes you stagger,\fnote{\fbackref{51:17} Lit. \fbib{the cup of staggering}} \\
\poemlll       and have drained it. \\
\poeml \v{18}There is no one to guide you\fnote{\fbackref{51:18} So 1QIsa\textsuperscript{a}; MT reads \fbib{no one to guide her}; LXX reads \fbib{no one who comforted you}} \\
\poemll    out of all the children she bore, \\
\poeml no one to take her by the hand \\
\poemll    out of all the children she brought up. \\
\poeml \v{19}``These twin things have come upon you \\
\poemll    (who can feel sorry for you?): \\
\poeml ruin and destruction, \\
\poemll    famine and the sword--- \\
\poemlll       who can console you? \\
\poeml \v{20}Your children have fainted. \\
\poemll    They lie at the head of every street, \\
\poemlll       like antelope caught in a trap, \\
\poeml filled with the anger of the \divine{Lord} \\
\poemll    and the rebuke of your God. \\
\poeml \v{21}Now listen to this, you afflicted one, \\
\poemll    made drunk, but not with wine: \\
\poeml \v{22}This is what your \divine{Lord}, the \divine{Lord},\fnote{\fbackref{51:21} 1QIsa\textsuperscript{a} MT; MT\textsuperscript{ms} lacks \fbib{}\divine{Lord}} says, \\
\poemll    your\fnote{\fbackref{51:22} So 1QIsa\textsuperscript{a}; MT reads \fbib{and your}} God, who defends his people's cause: \\
\poeml ``See, I have taken from your hand the cup that made you stagger.\fnote{\fbackref{51:22} Lit. \fbib{the cup of staggering}} \\
\poemll    And you will never again drink to the dregs the cup that is my anger. \\
\poeml \v{23}But I will put it into the hands of those who tormented and oppressed you,\fnote{\fbackref{51:23} So 1QIsa\textsuperscript{a}; MT reads \fbib{tormented you}; LXX reads \fbib{harmed you and humiliated you}} \\
\poemll    those who said to you, \\
\poeml `Lie down,\fnote{\fbackref{51:23} \fbib{Lit.bow down}} so we can step over you,' \\
\poemll    so that you had to make your back like the ground, \\
\poemlll       and like a street for them to walk over.''
\end{poetry}
\labelchapt{52}
\passage{The Redemption of Zion}

\chapt{52}
\v{1}Awake, awake!

\begin{poetry}
\poemll    Clothe yourself with strength,\fnote{\fbackref{52:1} So 1QIsa\textsuperscript{a}; MT LXX read \fbib{with your strength}} O Zion! \\
\poeml Put on your beautiful garments, \\
\poemll    O Jerusalem, the holy city, \\
\poemlll       for the uncircumcised and the unclean won't enter you.\fnote{\fbackref{52:1} So 1QIsa\textsuperscript{a}; MT LXX read \fbib{you any more}} \\
\poeml \v{2}Shake yourself from the dust and\fnote{\fbackref{52:2} So 1QIsa\textsuperscript{a} LXX; 4QIsa\textsuperscript{b} MT lack \fbib{and}} arise, \\
\poemll    and\fnote{\fbackref{52:2} So 1QIsa\textsuperscript{a}; MT LXX lack \fbib{and}} sit on your throne, O Jerusalem! \\
\poeml Loosen the bonds from your neck, \\
\poemll    O captive daughter of Zion.
\end{poetry}

\v{3}For this is what the \divine{Lord} says: ``You were sold for nothing, and you'll be redeemed without money.''

\v{4}For this is what the \divine{Lord}\fnote{\fbackref{52:4} So 1QIsa\textsuperscript{a} LXX; MT reads \fbib{the \divine{Lord} \divine{God}}} says: ``My people went down long ago into Egypt to live\fnote{\fbackref{52:4} Or \fbib{sojourn}} there; the Assyrian, too, has oppressed them without cause.

\v{5}``Now therefore, what\fnote{\fbackref{52:5} So 1QIsa\textsuperscript{a} LXX; MT reads \fbib{who}} am I doing here,'' asks\fnote{\fbackref{52:5} Lit. \fbib{declares}} the \divine{Lord}, ``seeing that my people are taken away without cause? Those who rule over them are deluded,''\fnote{\fbackref{52:5} So 1QIsa\textsuperscript{a}; MT LXX read \fbib{them wail}; or \fbib{them taunt}} says the \divine{Lord}, ``and continuously, all the day long, my name is blasphemed. \v{6}Therefore my people will know my name; in that day\fnote{\fbackref{52:6} So 1QIsa\textsuperscript{a} LXX. MT reads \fbib{therefore in that day}} they'll know that it is I who speaks, `Here I am!'

\begin{poetry}
\poeml \v{7}``How beautiful\fnote{\fbackref{52:7} Lit. \fbib{they are beautiful}; so 1QIsa\textsuperscript{a} MTT LXX\textsuperscript{mss}; 4QIsa\textsuperscript{b} reads \fbib{It is beautiful}} on the mountains \\
\poemll    are the feet of the one who brings news of peace,\fnote{\fbackref{52:7} So 1QIsa\textsuperscript{a}; MT LXX read \fbib{who brings good news}} \\
\poeml who announces good things, \\
\poemll    who announces salvation,\fnote{\fbackref{52:7} So 1QIsa\textsuperscript{a}; MT reads \fbib{who announces peace, who brings news of good things, who announces salvation}; cf. LXX} \\
\poemlll       who says to Zion, `Your God reigns!' \\
\poeml \v{8}Listen! Your watchmen lift up their voices,\fnote{\fbackref{52:8} Lit. \fbib{their voice}; so 1QIsa\textsuperscript{a}; MT reads \fbib{the voice}} \\
\poemll    together they sing for joy; \\
\poeml for they will see in plain sight \\
\poemll    the return of the \divine{Lord} to Zion with compassion.\fnote{\fbackref{52:8} So 1QIsa\textsuperscript{a} LXX; MT lacks \fbib{with compassion}} \\
\poeml \v{9}``Break forth together into singing,\fnote{\fbackref{52:9} Lit. \fbib{sing for joy} (sing.); so 1QIsa\textsuperscript{a}; MT reads \fbib{sing for joy} (pl.)} \\
\poemll    you ruins of Jerusalem; \\
\poeml for the \divine{Lord} has comforted his people, \\
\poemll    and\fnote{\fbackref{52:9} So 1QIsa\textsuperscript{a} LXX; MT lacks \fbib{and}} he has redeemed Jerusalem. \\
\poeml \v{10}The \divine{Lord} has bared his holy arm\fnote{\fbackref{52:10} I.e. \fbib{the Messiah}} \\
\poemll    in the eyes of all the nations; \\
\poeml and all the ends of the earth will see \\
\poemll    the salvation of our God. \\
\poeml \v{11}``Depart! Depart! Go out from there; \\
\poemll    touch no unclean thing; \\
\poeml go out from the midst of her; \\
\poemll    purify yourselves, \\
\poemlll       you who carry the vessels of the \divine{Lord}. \\
\poeml \v{12}For you won't go out in haste, \\
\poemll    nor will you go in flight; \\
\poeml for the \divine{Lord} will go before you; \\
\poemll    and the God of Israel will be your rear guard. \\
\poemlll       He is called the God of all the earth.''\fnote{\fbackref{52:12} So 1QIsa\textsuperscript{a}; MT LXX lack \fbib{this line}}
\passage{The Suffering Servant}
\poeml \v{13}``Look! My servant will prosper, \\
\poemll    and\fnote{\fbackref{52:13} So 1QIsa\textsuperscript{a}; 1QIsa\textsuperscript{b} 4QIsa\textsuperscript{c} MT lack \fbib{and}; LXX lacks \fbib{and he will be exalted}} he will be exalted and lifted up, \\
\poemlll       and will be very high. \\
\poeml \v{14}Just as many were astonished at you\fnote{\fbackref{52:14} So 1QIsa\textsuperscript{a} MT LXX; MT\textsuperscript{mss} Syriac read \fbib{at him}}--- \\
\poemll    so was he marred in\fnote{\fbackref{52:14} Or \fbib{was my marring}; so 1QIsa\textsuperscript{a}; MT reads \fbib{was the marring of}} his appearance, more than any human, \\
\poeml and his form beyond that of human semblance\fnote{\fbackref{52:14} Lit. \fbib{of the descendants of humans}; so 1QIsa\textsuperscript{a}; LXX reads \fbib{of the humans}; MT reads \fbib{of the human}}--- \\
\poeml \v{15}so will he startle\fnote{\fbackref{52:15} Or \fbib{sprinkle}} many nations. \\
\poeml Kings will shut their mouths at him; \\
\poemll    for what had not been told them they will see, \\
\poemlll       and what they had not heard they will understand.
\end{poetry}
\labelchapt{53}

\chapt{53}
\v{1}``Who\fnote{\fbackref{53:1} So 1QIsa\textsuperscript{a} 1QIsa\textsuperscript{b} MT; LXX reads \fbib{Lord, who}} has believed our message,

\begin{poetry}
\poemll    and\fnote{\fbackref{53:1} So 1QIsa\textsuperscript{a} 1QIsa\textsuperscript{b}; MT lacks \fbib{and}} to\fnote{\fbackref{53:1} So 1QIsa\textsuperscript{a} 1QIsa\textsuperscript{b}; MT reads \fbib{upon}} whom has the arm\fnote{\fbackref{53:1} I.e. \fbib{the Messiah}} of the \divine{Lord} been revealed? \\
\poeml \v{2}For he grew up before him like a tender plant, \\
\poemll    and like a root out of a dry ground; \\
\poeml he had no form and he had\fnote{\fbackref{53:2} So 1QIsa\textsuperscript{a}; 1QIsa\textsuperscript{b} MT LXX lack \fbib{he had}} no majesty that we should look at him,\fnote{\fbackref{53:2} So 1QIsa\textsuperscript{a} 1QIsa\textsuperscript{b} MT LXX; 1QIsa\textsuperscript{a} may read \fbib{at ourselves}} \\
\poemll    and there is no attractiveness that we should desire him.\fnote{\fbackref{53:2} So 1QIsa\textsuperscript{a} MT LXX; 1QIsa\textsuperscript{a} may read \fbib{desire ourselves}} \\
\poeml \v{3}``He was despised and rejected by others, \\
\poemll    and\fnote{\fbackref{53:3} So 1QIsa\textsuperscript{a}; 1QIsa\textsuperscript{b} MT LXX lack \fbib{and}} a man of sorrows, \\
\poemlll       intimately familiar with\fnote{\fbackref{53:3} So 1QIsa\textsuperscript{a} LXX; MT reads \fbib{and acquainted with}; 1QIsa\textsuperscript{b} reads \fbib{and knowing}} suffering; \\
\poeml and like one from whom people hide their faces; \\
\poemll    and\fnote{\fbackref{53:3} So 1QIsa\textsuperscript{a} 1QIsa\textsuperscript{b}; MT LXX lacks \fbib{and}} we despised him\fnote{\fbackref{53:3} So 1QIsa\textsuperscript{a}; MT LXX read \fbib{he was despised}} \\
\poemlll       and did not value him. \\
\poeml \v{4}``Surely he has borne our sufferings \\
\poemll    and carried our sorrows; \\
\poeml yet we considered him stricken, \\
\poemll    and\fnote{\fbackref{53:4} So 1QIsa\textsuperscript{a} LXX; MT lacks \fbib{and}} struck down by God, \\
\poemlll       and afflicted. \\
\poeml \v{5}But he was wounded for our transgressions, \\
\poemll    and\fnote{\fbackref{53:5} So 1QIsa\textsuperscript{a} LXX; MT lacks \fbib{and}} he was crushed for our iniquities, \\
\poeml and\fnote{\fbackref{53:5} So 1QIsa\textsuperscript{a} 1QIsa\textsuperscript{b}; MT LXX lack \fbib{and}} the punishment that made us whole was upon him, \\
\poemll    and by his bruises we are healed. \\
\poeml \v{6}All we like sheep have gone astray, \\
\poemll    we have turned, each of us, to his own way; \\
\poeml and the \divine{Lord} has laid on him \\
\poemll    the iniquity of us all. \\
\poeml \v{7}He was oppressed and he was afflicted, \\
\poemll    yet he didn't open his mouth; \\
\poeml like a lamb that is led to the slaughter, \\
\poemll    as\fnote{\fbackref{53:7} So 1QIsa\textsuperscript{a}; MT LXX read \fbib{and as}} a sheep that before its shearers is silent, \\
\poemlll       so he did not open\fnote{\fbackref{53:7} So 1QIsa\textsuperscript{a}; MT reads \fbib{does not open}} his mouth. \\
\poeml \v{8}``From detention and\fnote{\fbackref{53:8} So 1QIsa\textsuperscript{a} MT; 1QIsa\textsuperscript{b} lacks \fbib{and}} judgment he was taken away\fnote{\fbackref{53:8} So 1QIsa\textsuperscript{a} MT LXX; 1QIsa\textsuperscript{b} reads \fbib{judgment they took him away}}--- \\
\poemll    and who can even think about his descendants?\fnote{\fbackref{53:8} Or \fbib{future}} \\
\poeml For he was cut off from the land of the living, \\
\poemll    he was stricken\fnote{\fbackref{53:8} So 1QIsa\textsuperscript{a}; MT reads \fbib{living, an affliction}} for the transgression of my people. \\
\poeml \v{9}Then they made\fnote{\fbackref{53:9} So 1QIsa\textsuperscript{a}; 4QIsa\textsuperscript{d} MT read \fbib{he made}; LXX reads \fbib{I will give}} his grave with the wicked, \\
\poemll    and with rich people\fnote{\fbackref{53:9} So 1QIsa\textsuperscript{a}; 1QIsa\textsuperscript{a} corrector MT read \fbib{with a rich man}} in his death,\fnote{\fbackref{53:9} So 1QIsa\textsuperscript{a} LXX; MT reads \fbib{deaths}} \\
\poeml although he had committed no violence, \\
\poemll    nor was there any deceit in his mouth.''
\passage{The Exaltation of the Servant}
\poeml \v{10}``Yet the \divine{Lord} was willing to crush him, \\
\poemll    and he made him suffer.\fnote{\fbackref{53:10} So 1QIsa\textsuperscript{a}; 4QIsa\textsuperscript{d} MT read \fbib{he made him suffer}; LXX reads \fbib{with a blow}} \\
\poeml Although you make his soul an offering for sin, \\
\poemll    he\fnote{\fbackref{53:10} So MT LXX; 1QIsa\textsuperscript{a} reads \fbib{And he}} will see his offspring, \\
\poeml and\fnote{\fbackref{53:10} So 1QIsa\textsuperscript{a} 4QIsa\textsuperscript{d}; 1QIsa\textsuperscript{b} MT lack \fbib{and}} he will prolong his days, \\
\poemll    and the will of the \divine{Lord} will triumph in his hand. \\
\poeml \v{11}Out of the suffering of his soul he will see light\fnote{\fbackref{53:11} So 1QIsa\textsuperscript{a} 1QIsa\textsuperscript{b} 4QIsa\textsuperscript{d} LXX; MT reads \fbib{He will see of the suffering of his soul}} \\
\poemll    and\fnote{\fbackref{53:11} So 1QIsa\textsuperscript{a} 4QIsa\textsuperscript{d}; MT lacks \fbib{and}} find satisfaction. \\
\poeml And\fnote{\fbackref{53:11} So 1QIsa\textsuperscript{a}; 4QIsa\textsuperscript{d} MT lacks \fbib{And}} through his knowledge his servant,\fnote{\fbackref{53:11} So 1QIsa\textsuperscript{a}; 4QIsa\textsuperscript{d} MT read \fbib{my servant}} the righteous one, \\
\poemll    will make many righteous, \\
\poemlll       and he will bear their iniquities. \\
\poeml \v{12}Therefore I will allot him a portion with the great,\fnote{\fbackref{53:12} I.e. an allusion to the resurrection} \\
\poemll    and he will divide the spoils with the strong; \\
\poeml because he poured out his life to death, \\
\poemll    and was numbered with the transgressors; \\
\poeml yet he carried the sins\fnote{\fbackref{53:12} So 1QIsa\textsuperscript{a} 1QIsa\textsuperscript{b} 4QIsa\textsuperscript{d} LXX; MT reads \fbib{the sin}} of many, \\
\poemll    and made intercession for their transgressions.''\fnote{\fbackref{53:12} So 1QIsa\textsuperscript{a} 1QIsa\textsuperscript{b} 4QIsa\textsuperscript{d} LXX; MT reads \fbib{for the transgressors}}
\end{poetry}
\labelchapt{54}
\passage{The Coming Glory of Israel}

\begin{poetry}
\poeml \chapt{54}
\v{1}``Sing, you barren woman, \\
\poemll    even the\fnote{\fbackref{54:1} Or \fbib{and}; so 1QIsa\textsuperscript{a} 4QIsa\textsuperscript{d}; 1QIsa\textsuperscript{b} MT LXX lack \fbib{even}} one who never bore a child! \\
\poeml Burst into song and shout for joy, \\
\poemll    even\fnote{\fbackref{54:1} Or \fbib{and}; so 1QIsa\textsuperscript{a}; 4QIsa\textsuperscript{d} MT LXX lack \fbib{even}} you who were never in labor! \\
\poeml For the children of the desolate woman will be more \\
\poemll    than the children of her that is married,'' \\
\poemlll       says the \divine{Lord}. \\
\poeml \v{2}``Enlarge the location\fnote{\fbackref{54:2} Or \fbib{place}} of your tent, \\
\poemll    let the curtains of your dwellings be stretched wide, \\
\poemlll       and\fnote{\fbackref{54:2} So 1QIsa\textsuperscript{a}; MT LXX lack \fbib{and}} don't hold back. \\
\poeml Lengthen your cords; \\
\poemlll       strengthen your stakes. \\
\poeml \v{3}For you will spread out to the right hand and to the left, \\
\poemll    and your descendants\fnote{\fbackref{54:3} Lit. \fbib{seed}} will possess\fnote{\fbackref{54:3} 1QIsa\textsuperscript{a} is pl.; MT is sing.} the nations \\
\poemlll       and will populate the deserted towns. \\
\poeml \v{4}``Don't be afraid, \\
\poemll    because you won't be ashamed; \\
\poeml don't fear shame, \\
\poemll    for you won't be humiliated--- \\
\poeml because you will forget the disgrace of your youth, \\
\poemll    and the reproach of your widowhood you will remember no more. \\
\poeml \v{5}For your Maker is your husband; \\
\poemll    the \divine{Lord} of the Heavenly Armies is his name, \\
\poeml and the Holy One of Israel is your Redeemer; \\
\poemll    he is called the God of the whole earth. \\
\poeml \v{6}For the \divine{Lord} has called you back \\
\poemll    like a wife deserted and grieved in spirit, \\
\poeml like the wife of a man's youth when she is cast off,'' \\
\poemll    says the \divine{Lord} your God.\fnote{\fbackref{54:6} So 1QIsa\textsuperscript{a}; MT LXX read \fbib{says your God}} \\
\poeml \v{7}``For a brief moment I abandoned you; \\
\poemll    but I'll gather you with great compassion. \\
\poeml \v{8}I hid my face from you for a moment in a surge of anger, \\
\poemll    but I will have compassion on you with my\fnote{\fbackref{54:8} So 1QIsa\textsuperscript{a} 4QIsa\textsuperscript{c}; MT LXX lack \fbib{my}} everlasting gracious love,'' \\
\poemlll       says the \divine{Lord} your Redeemer.
\passage{God's Reconciliation with Israel}
\poeml \v{9}``For this is like the waters of Noah to me, \\
\poemll    when I swore that the waters of Noah \\
\poemlll       would never again spread over the earth; \\
\poeml so have I sworn that I won't be angry with you again\fnote{\fbackref{54:9} So 1QIsa\textsuperscript{a} LXX; MT lacks \fbib{again}} \\
\poemll    and that I\fnote{\fbackref{54:9} 1QIsa\textsuperscript{a} LXX MT lack \fbib{that I}} won't rebuke you. \\
\poeml \v{10}For the mountains may collapse \\
\poemll    and the hills may reel, \\
\poeml but my gracious love will not depart from you, \\
\poemll    neither will my covenant of peace totter,'' \\
\poemlll       says the \divine{Lord}, who has compassion on you. \\
\poeml \v{11}``O afflicted one,\fnote{\fbackref{54:11} I.e. the city of Jerusalem} passed back and forth,\fnote{\fbackref{54:11} So 1QIsa\textsuperscript{a}; 4QIsa\textsuperscript{d} MT read \fbib{storm-tossed}; LXX reads \fbib{unsteady}} and not comforted, \\
\poemll    Look! I am about to set your stones in antimony, \\
\poemlll       and lay your foundations with sapphires. \\
\poeml \v{12}And I'll make your battlements of rubies, \\
\poemll    and your gates of jewels, \\
\poemlll       and all your walls of precious stones. \\
\poeml \v{13}Then all your children will be taught by the \divine{Lord}, \\
\poemll    and great will be your children's prosperity. \\
\poeml \v{14}``In righteousness you'll be established; \\
\poemll    you will be far from tyranny, \\
\poemlll       for you won't be afraid, \\
\poemll    and from terror, \\
\poemlll       for it won't come near you. \\
\poeml \v{15}Watch! If anyone does attack you, \\
\poemll    it will not be from me; \\
\poeml whoever may attack\fnote{\fbackref{54:15} So 1QIsa\textsuperscript{a}; MT reads \fbib{whoever attacks}} you will fall\fnote{\fbackref{54:15} So 1QIsa\textsuperscript{a} 4QIsa\textsuperscript{c}; MT reads \fbib{he will fall}; LXX reads \fbib{they will flee}} because of you. \\
\poeml \v{16}Look! It is I who have created the blacksmith \\
\poemll    who fans coals in the fire, \\
\poemlll       and produces a weapon for his purpose. \\
\poeml It\fnote{\fbackref{54:16} So 1QIsa\textsuperscript{a}; MT reads \fbib{And it}; cf. LXX} is I who have created the ravager to wreak havoc; \\
\poeml \v{17}no weapon that is forged against you will be effective.\fnote{\fbackref{54:17} So 1QIsa\textsuperscript{a}; 4QIsa\textsuperscript{c} MT LXX read \fbib{effective, and you will refute every tongue that rises against you in judgment}} \\
\poeml This is the heritage of the \divine{Lord}'s servants, \\
\poemll    and their righteousness from me,'' \\
\poemlll       says the \divine{Lord}.
\end{poetry}
\labelchapt{55}
\passage{An Invitation to Life}

\chapt{55}
\v{1}``Come, everyone who is thirsty,

\begin{poetry}
\poemll    come to the waters! \\
\poeml Also, you that have no money, come, \\
\poemll    buy, and eat! \\
\poeml Come! Buy\fnote{\fbackref{55:1} So MT LXX; 1QIsa\textsuperscript{a} skips from the first \fbib{come, buy} to the second \fbib{come, buy,} omitting the words in between} wine and milk \\
\poemll    without money and without price. \\
\poeml \v{2}Why spend your money on what is not bread, \\
\poemll    and your labor on what does not satisfy?\fnote{\fbackref{55:2} Lit. \fbib{what is not satisfaction}; so 1QIsa\textsuperscript{a}; MT reads \fbib{what is not for satisfaction}} \\
\poeml Listen carefully to me, \\
\poemll    and eat what is good, \\
\poemlll       and let your soul delight itself in rich food. \\
\poeml \v{3}Pay attention\fnote{\fbackref{55:3} Lit. \fbib{Turn your ear}} to me, \\
\poemll    come to me; \\
\poemlll       and\fnote{\fbackref{55:3} So 1QIsa\textsuperscript{a}; MT LXX lacks \fbib{and}} listen, so that you may live; \\
\poeml then I'll make\fnote{\fbackref{55:3} So 1QIsa\textsuperscript{a}; 4QIsa\textsuperscript{c} MT read \fbib{then let me make}} an everlasting covenant with you, \\
\poemll    as promised by\fnote{\fbackref{55:3} 1QIsa\textsuperscript{a} 4QIsa\textsuperscript{c} MT LXX lack \fbib{as promised by}} my faithful, sure love for David. \\
\poeml \v{4}``Look! I have made him a witness to the peoples, \\
\poemll    a leader and commander of the peoples. \\
\poeml \v{5}``Look! You will call a nation that you do not know, \\
\poemll    and a nation that does\fnote{\fbackref{55:5} So 1QIsa\textsuperscript{a}; MT reads \fbib{a nation} (pl.) \fbib{that do}; LXX reads \fbib{nations that do}} not know you will run\fnote{\fbackref{55:5} So 1QIsa\textsuperscript{a} (sing.); MT LXX (pl.)} to you, \\
\poeml because of the \divine{Lord} your God, even\fnote{\fbackref{55:5} So 1QIsa\textsuperscript{a}; MT 1QIsa\textsuperscript{a} corrector read \fbib{and because of}} the Holy One of Israel, \\
\poemll    for he has glorified you.''
\passage{Steps to Reconciliation}
\poeml \v{6}``Seek the \divine{Lord} while he\fnote{\fbackref{55:6} So 1QIsa\textsuperscript{a} 1QIsa\textsuperscript{b} MT LXX; implied in 4QIsa\textsuperscript{c}} may be found, \\
\poemll    call upon him while he is near. \\
\poeml \v{7}Let the wicked forsake his way, \\
\poemll    and the unrighteous person his thoughts. \\
\poeml Let him return to the \divine{Lord}, \\
\poemll    So he'll have mercy upon him, \\
\poeml and to our God, \\
\poemll    for he'll pardon abundantly. \\
\poeml \v{8}For my thoughts are not your thoughts, \\
\poemll    nor are your ways my ways,'' declares the \divine{Lord}. \\
\poeml \v{9}``For just as\fnote{\fbackref{55:9} So 1QIsa\textsuperscript{a} LXX; MT lacks \fbib{just as}} the heavens are higher than the earth, \\
\poemll    so are my ways higher than your ways, \\
\poemlll       and my thoughts than your thoughts. \\
\poeml \v{10}``For just as the rain and snow come down from heaven, \\
\poemll    and do not return there without watering the earth, \\
\poeml making it bring forth and sprout, \\
\poemll    yielding seed for the sower and bread for eating,\fnote{\fbackref{55:10} So 1QIsa\textsuperscript{a}; MT reads \fbib{for the eater}} \\
\poeml \v{11}so will my message be that goes out of my mouth--- \\
\poemll    it won't return to me empty. \\
\poeml Instead, it will accomplish what I desire, \\
\poemll    and achieve the purpose for which I sent it. \\
\poeml \v{12}``For you will go out in joy, \\
\poemll    and come back\fnote{\fbackref{55:12} So 1QIsa\textsuperscript{a}. MT reads \fbib{and be led back} cf. LXX} with peace; \\
\poeml the mountains and the hills \\
\poemll    will burst into song before you, \\
\poemlll       and all the trees in\fnote{\fbackref{55:12} Or \fbib{of}} the fields\fnote{\fbackref{55:12} Or \fbib{orchards}} will clap their hands. \\
\poeml \v{13}Instead of thornbushes, pine trees will grow, \\
\poemll    and\fnote{\fbackref{55:13} So 1QIsa\textsuperscript{a} MT\textsuperscript{mss}; MT LXX lacks \fbib{and}} instead of briers, myrtles will grow; \\
\poeml and they\fnote{\fbackref{55:13} So 1QIsa\textsuperscript{a}; MT LXX read \fbib{it}} will be a sign for the \divine{Lord}, \\
\poemll    and an everlasting name\fnote{\fbackref{55:13} So 1QIsa\textsuperscript{a}; MT LXX read \fbib{a name, an everlasting sign}} that will not be cut off.''
\end{poetry}
\labelchapt{56}
\passage{The Covenant Extended to the Righteous}

\chapt{56}
\v{1}For\fnote{\fbackref{56:1} So 1QIsa\textsuperscript{a}; MT LXX lack \fbib{For}} this is what the \divine{Lord} says:

\begin{poetry}
\poemll    ``Maintain justice, and do what is right, \\
\poeml for soon my salvation will come, \\
\poemll    and soon my deliverance will be revealed. \\
\poeml \v{2}Blessed is the one who does this, \\
\poemll    and the person that holds it fast, \\
\poeml who observes the Sabbath without profaning it, \\
\poemll    and restrains his hands\fnote{\fbackref{56:2} So 1QIsa\textsuperscript{a} LXX 1QIsa\textsuperscript{b} MT read \fbib{hand}} from practicing any evil. \\
\poeml \v{3}``Let\fnote{\fbackref{56:3} So 1QIsa\textsuperscript{a} LXX; 1QIsa\textsuperscript{b} MT read \fbib{And let}} no foreigner who has joined himself to the \divine{Lord} say: \\
\poemll    `The \divine{Lord} will surely exclude me from his people.' \\
\poeml Furthermore, let no eunuch say, \\
\poemll    `Look!\fnote{\fbackref{56:3} So 1QIsa\textsuperscript{a} 1QIsa\textsuperscript{b} MT; LXX lacks \fbib{Look}!} I am just a dry tree.'\,'' \\
\poeml \v{4}For this is what the \divine{Lord} says: \\
\poeml ``To the eunuchs who observe my Sabbaths, \\
\poemll    who choose the things that please me, \\
\poemlll       and who hold fast my covenant--- \\
\poeml \v{5}to them I will give in my house\fnote{\fbackref{56:5} I.e. God's Temple} and within my walls \\
\poemll    a monument and a name \\
\poemlll       better than sons and daughters. \\
\poeml I will give them\fnote{\fbackref{56:5} So 1QIsa\textsuperscript{a} LXX; 1QIsa\textsuperscript{b} MT read \fbib{him}} an everlasting name \\
\poemll    that will not be cut off.\fnote{\fbackref{56:5} The Heb. verb is a word play on the Heb. word \fbib{eunuch}} \\
\poeml \v{6}``Also, the foreigners who join themselves to\fnote{\fbackref{56:6} So 1QIsa\textsuperscript{a} (cf. v 3); 1QIsa\textsuperscript{b} MT read \fbib{upon}} the \divine{Lord}, \\
\poemll    to minister to him, \\
\poemlll       to love the name of the \divine{Lord},\fnote{\fbackref{56:6} So 1QIsa\textsuperscript{b} MT LXX; 1QIsa\textsuperscript{a} lacks this line} \\
\poeml to be his servants, \\
\poemll    and to bless the \divine{Lord}'s name, \\
\poeml observing\fnote{\fbackref{56:6} So 1QIsa\textsuperscript{a}; 1QIsa\textsuperscript{b} MT LXX read \fbib{all who observe}} the Sabbath without profaning it, \\
\poemll    and who hold fast my covenant--- \\
\poeml \v{7}these I will bring to my holy mountain, \\
\poemll    and make them joyful in my house of prayer. \\
\poeml Their burnt offerings and their sacrifices \\
\poemll    will rise up to be accepted\fnote{\fbackref{56:7} 1QIsa\textsuperscript{a}; 1QIsa\textsuperscript{b} 4QIsa\textsuperscript{i} MT LXX read \fbib{will be accepted}} on my altar; \\
\poeml for my house will be called a house of prayer \\
\poemll    for everyone.''\fnote{\fbackref{56:7} Lit. \fbib{for all peoples}}
\passage{A Rebuke to Israel's Guardians}
\poeml \v{8}This is what the Lord \divine{God} says, \\
\poemll    the one who gathers the outcasts of Israel: \\
\poeml ``I'll gather still others to them \\
\poemll    besides those already gathered.\fnote{\fbackref{56:8} Lit. \fbib{besides their gathered ones}} \\
\poeml \v{9}``All you wild animals,\fnote{\fbackref{56:9} So 1QIsa\textsuperscript{a} LXX; 1QIsa\textsuperscript{b} MT read \fbib{Every wild animal}} come and devour--- \\
\poemll    even\fnote{\fbackref{56:9} So 1QIsa\textsuperscript{a}; 1QIsa\textsuperscript{b} MT LXX lack \fbib{even}} all of you wild animals.\fnote{\fbackref{56:9} So 1QIsa\textsuperscript{a} LXX; 1QIsa\textsuperscript{b} MT read \fbib{every wild animal}} \\
\poeml \v{10}His\fnote{\fbackref{56:10} I.e. Israel's; so 1QIsa\textsuperscript{a} MT\textsuperscript{q}} watchmen\fnote{\fbackref{56:10} MT reads \fbib{His watchman}; LXX reads \fbib{Look! They all}} are blind; \\
\poemll    they are all without knowledge. \\
\poeml They are all dumb dogs--- \\
\poemll    they cannot bark. \\
\poeml They keep on dreaming and lying around, \\
\poemll    and they're lovers of sleep!\fnote{\fbackref{56:10} So 1QIsa\textsuperscript{a} LXX; MT reads \fbib{sleeping}} \\
\poeml \v{11}Meanwhile,\fnote{\fbackref{56:11} Lit. \fbib{And}} the dogs have a mighty appetite--- \\
\poemll    they can never get enough. \\
\poeml And as for them, they are the shepherds\fnote{\fbackref{56:11} So 1QIsa\textsuperscript{a}; MT reads \fbib{shepherds}; LXX reads \fbib{evil}} who lack understanding; \\
\poemll    they have all turned to their own way, \\
\poeml each one to his gain, \\
\poemll    each and every one. \\
\poeml \v{12}```Come!' they say, `let's\fnote{\fbackref{56:12} So 1QIsa\textsuperscript{a} MT\textsuperscript{ms}; 1QIsa\textsuperscript{b} reads \fbib{I will}; MT reads \fbib{let me}} have some wine, \\
\poemll    and let's fill ourselves with strong drink! \\
\poeml Then,\fnote{\fbackref{56:12} Or \fbib{And}} tomorrow will be like today,\fnote{\fbackref{56:12} 1QIsa\textsuperscript{a} reads \fbib{this the day}; 1QIsa\textsuperscript{b} MT read \fbib{this day}} \\
\poemll    or even much better!'\,''
\end{poetry}
\labelchapt{57}
\passage{Israel's Idolatry}

\chapt{57}
\v{1}``Also\fnote{\fbackref{57:1} So 1QIsa\textsuperscript{a}; 1QIsa\textsuperscript{b} MT lacks \fbib{Also}; cf. LXX} the righteous are perishing,\fnote{\fbackref{57:1} So 1QIsa\textsuperscript{a}; 1QIsa\textsuperscript{b} MT LXX read \fbib{righteous person has perished}}

\begin{poetry}
\poemll    but no one takes it to heart; \\
\poeml devout people\fnote{\fbackref{57:1} Lit. \fbib{people of the mercy}; so 1QIsa\textsuperscript{a} MT reads \fbib{people of mercy}; LXX reads \fbib{just men}} are taken away, \\
\poemll    while no one understands \\
\poemlll       that the righteous person is taken away from calamity. \\
\poeml \v{2}Then\fnote{\fbackref{57:2} Or \fbib{and}; so 1QIsa\textsuperscript{a}; MT LXX lack \fbib{Then}} he enters into peace, \\
\poemll    and\fnote{\fbackref{57:2} So 1QIsa\textsuperscript{a}; MT lacks \fbib{and}} they'll rest on his\fnote{\fbackref{57:2} So 1QIsa\textsuperscript{a}; 1QIsa\textsuperscript{b} MT read \fbib{their}} couches, \\
\poemlll       each one living righteously.\fnote{\fbackref{57:2} Lit. \fbib{one walking in his uprightness}; so 1QIsa\textsuperscript{a} 1QIsa\textsuperscript{b}; MT reads \fbib{her uprightness}} \\
\poeml \v{3}``But as for you, come here, \\
\poemll    you children of a sorceress, \\
\poemlll       you offspring of adulterers and prostitutes!\fnote{\fbackref{57:3} So LXX (cf. Syriac); 1QIsa\textsuperscript{a} MT read \fbib{she has practiced prostitution}} \\
\poeml \v{4}Whom are you mocking? \\
\poemll    And\fnote{\fbackref{57:4} So 1QIsa\textsuperscript{a} LXX; 1QIsa\textsuperscript{b} MT lack \fbib{And}} against whom do you make a wide mouth \\
\poemlll       and stick out your tongue? \\
\poeml Are you not children of transgression, \\
\poemll    the offspring of lies, \\
\poeml \v{5}you who burn with lust among the oaks, \\
\poemll    under every spreading tree, \\
\poeml who slaughter your children in the ravines, \\
\poemll    under the clefts of the rocks? \\
\poeml \v{6}``Among the smooth stones\fnote{\fbackref{57:6} I.e. among the idols} of the ravines is your portion--- \\
\poemll    there they are as\fnote{\fbackref{57:6} So 1QIsa\textsuperscript{a}; 4QIsa\textsuperscript{i} MT reads \fbib{they---yes, they!---are}; LXX reads \fbib{there this is}} your lot. \\
\poeml To them you have poured out drink offerings; \\
\poemll    you have brought grain offerings. \\
\poemlll       Should I be lenient over such things? \\
\poeml \v{7}``You have made your bed \\
\poemll    on a high and lofty mountain, \\
\poemlll       and you went up to offer sacrifice there. \\
\poeml \v{8}Behind the doors and the doorposts \\
\poemll    you have set up your pagan sign.'' \\
\poeml For in deserting me you have uncovered your bed--- \\
\poemll    you have climbed up into it \\
\poemlll       and have opened it wide. \\
\poeml And you\fnote{\fbackref{57:8} 1QIsa\textsuperscript{a} reads sing.; MT reads pl.} have made a pact for yourself with them; \\
\poemll    you have loved their bed, \\
\poemlll       you have looked on their private parts.\fnote{\fbackref{57:8} Lit. \fbib{their hand}; i.e. a euphemism for the male sex organ} \\
\poeml \v{9}You went to Molech\fnote{\fbackref{57:9} I.e. to the Canaanite deity; or \fbib{to the king}} with olive oil \\
\poemll    and increased your perfumes; \\
\poeml you sent your ambassadors far away, \\
\poemll    you sent them down even to Sheol\fnote{\fbackref{57:9} I.e. the afterlife} itself! \\
\poeml \v{10}You grew tired with your many wanderings,\fnote{\fbackref{57:10} So 1QIsa\textsuperscript{a} LXX; MT reads \fbib{your wandering}} \\
\poemll    but you wouldn't say: `It is hopeless.' \\
\poeml You found new strength for your desire, \\
\poemll    and so you did not falter. \\
\poeml \v{11}``Whom did you so dread--- \\
\poemll    and while you feared me\fnote{\fbackref{57:11} So 1QIsa\textsuperscript{a}; 4QIsa\textsuperscript{d} MT LXX read \fbib{and fear}}--- \\
\poeml that you lied, \\
\poemll    and you did not remember me, \\
\poemlll       and\fnote{\fbackref{57:11} So 1QIsa\textsuperscript{a} 4QIsa\textsuperscript{d} LXX; MT lacks \fbib{and}} did not lay to heart these things?\fnote{\fbackref{57:11} So 1QIsa\textsuperscript{a}; 4QIsa\textsuperscript{d} MT lack \fbib{thing}; LXX reads \fbib{lay me to heart}} \\
\poeml Haven't I remained silent for a long time, \\
\poemll    and still you don't fear me?'' \\
\poeml \v{12}``I will denounce your righteousness\fnote{\fbackref{57:12} 1QIsa\textsuperscript{a} MT; 4QIsa\textsuperscript{d} reads \fbib{justice}} and your works, \\
\poemll    for your collections of idols\fnote{\fbackref{57:12} 1QIsa\textsuperscript{a} lacks \fbib{of idols}; 4QIsa\textsuperscript{d} MT LXX read \fbib{for they}} will not benefit you. \\
\poeml \v{13}When you cry out, let your collection deliver you! \\
\poemll    The wind will carry them all off, \\
\poemlll       and\fnote{\fbackref{57:13} So 1QIsa\textsuperscript{a} LXX; MT lacks \fbib{and}} a mere breath will sweep them all away.''
\passage{God's Reward for the Faithful}
\poeml ``But whoever\fnote{\fbackref{57:13} Lit. \fbib{But the one}; so 4QIsa\textsuperscript{d} MT; 1QIsa\textsuperscript{a} reads \fbib{But one} (i.e. without article)} takes refuge in me will possess the land, \\
\poemll    and will inherit my holy mountain. \\
\poeml \v{14}And one has said:\fnote{\fbackref{57:14} So 1QIsa\textsuperscript{a}; MT reads \fbib{one will say}; or \fbib{I will say}; LXX reads \fbib{they will say}} \\
\poemll    `Build up! Build up the road!\fnote{\fbackref{57:14} So 1QIsa\textsuperscript{a} LXX; MT lacks \fbib{the road}} \\
\poemlll       Prepare the highway! \\
\poeml Remove every obstacle from my people's way.' \\
\poeml \v{15}``For this is what the high and lofty One says, \\
\poemll    who inhabits eternity, whose name is Holy: \\
\poeml ``He lives\fnote{\fbackref{57:15} So 1QIsa\textsuperscript{a} 4QIsa\textsuperscript{d}; MT reads \fbib{I live}} in the height and in holiness,\fnote{\fbackref{57:15} So 1QIsa\textsuperscript{a}; 4QIsa\textsuperscript{d} MT read \fbib{in the high and holy place}} \\
\poemll    and also with the one who is of a contrite and humble spirit, \\
\poeml to revive the spirit of the humble, \\
\poemll    and to revive the heart of the contrite. \\
\poeml \v{16}For I won't accuse forever, \\
\poemll    nor will I always be angry; \\
\poeml for then the human spirit would grow faint before me--- \\
\poemll    even the souls that I have created. \\
\poeml \v{17}Because of his wicked greed I was angry, \\
\poemll    so I punished him; \\
\poeml and\fnote{\fbackref{57:17} So 1QIsa\textsuperscript{a} 4QIsa\textsuperscript{d} LXX; 1QIsa\textsuperscript{b} MT lack \fbib{and}} I hid my face, and was angry--- \\
\poemll    but he kept turning back to his stubborn will.\fnote{\fbackref{57:17} Lit. \fbib{into the way of his heart}} \\
\poeml \v{18}I've seen his ways,\fnote{\fbackref{57:18} So 1QIsa\textsuperscript{a} MT LXX; 4QIsa\textsuperscript{d} reads \fbib{way}} yet I will heal him,\fnote{\fbackref{57:18} So 1QIsa\textsuperscript{a}; MT reads \fbib{him, and I will guide him}; LXX reads \fbib{him, and I will exhort him}} \\
\poemll    and restore for him\fnote{\fbackref{57:18} So 1QIsa\textsuperscript{a}; the reading is probably an error: MT LXX lack \fbib{for him}} comfort\fnote{\fbackref{57:18} So 1QIsa\textsuperscript{a}; 1QIsa\textsuperscript{b} MT use different but related words} to him \\
\poemlll       and for those who mourn for him\fnote{\fbackref{57:18} Lit. \fbib{for his mourners}} \\
\poeml \v{19}when\fnote{\fbackref{57:19} So 1QIsa\textsuperscript{a}; 1QIsa\textsuperscript{b} 4QIsa\textsuperscript{d} MT lack \fbib{when}} I create the fruit of the lips: \\
\poemll    Peace\fnote{\fbackref{57:19} So 1QIsa\textsuperscript{a}; 1QIsa\textsuperscript{b} MT read \fbib{Peace, peace} cf. LXX} to the one who is far away or near,'' says the \divine{Lord}, \\
\poemlll       ``and I'll heal him. \\
\poeml \v{20}But the wicked are tossed like the sea;\fnote{\fbackref{57:20} So 1QIsa\textsuperscript{a} LXX; 4QIsa\textsuperscript{d} MT read \fbib{are like the tossing sea}} \\
\poemll    for it is not able to\fnote{\fbackref{57:20} So 1QIsa\textsuperscript{a}; 4QIsa\textsuperscript{d} MT lack \fbib{able to}} keep still, \\
\poemlll       and its waters toss up mire and mud. \\
\poeml \v{21}``Yet\fnote{\fbackref{57:21} So 1QIsa\textsuperscript{a}; MT LXX lack \fbib{Yet}} there is no peace,'' says my God, ``for the wicked.''
\end{poetry}
\labelchapt{58}
\passage{False and True Worship}

\chapt{58}
\v{1}``Shout aloud!

\begin{poetry}
\poemll    Don't hold back! \\
\poemlll       Lift up your voice like a trumpet! \\
\poeml Declare to my people their rebellions,\fnote{\fbackref{58:1} So 1QIsa\textsuperscript{a} LXX; 1QIsa\textsuperscript{b} MT read \fbib{rebellion}} \\
\poemll    and to the house of Jacob their sins. \\
\poeml \v{2}They\fnote{\fbackref{58:2} So 1QIsa\textsuperscript{a} 1QIsa\textsuperscript{b} 4QIsa\textsuperscript{d} LXX; MT reads \fbib{And they}} seek me day after day,\fnote{\fbackref{58:2} Lit. \fbib{me day and day}; so 1QIsa\textsuperscript{a}; 1QIsa\textsuperscript{b} 4QIsa\textsuperscript{d} MT read \fbib{me day, day}} \\
\poemll    and are eager to know my ways, \\
\poeml as if they were a nation that practices righteousness \\
\poemll    and has not forsaken the justice of their God. \\
\poeml ``They ask me to reveal just decisions; \\
\poemll    they are eager to draw near to God. \\
\poeml \v{3}`Why have we fasted,' they ask,\fnote{\fbackref{58:3} 1QIsa\textsuperscript{a} 1QIsa\textsuperscript{b} 4QIsa\textsuperscript{d} LXX MT lack \fbib{they ask}} \\
\poemll    `but you do not see? \\
\poeml `Why have we humbled ourselves,'\fnote{\fbackref{58:3} So 1QIsa\textsuperscript{a} 1QIsa\textsuperscript{b} LXX; MT reads \fbib{ourself}} they ask,\fnote{\fbackref{58:3} 1QIsa\textsuperscript{a} 1QIsa\textsuperscript{b} 4QIsa\textsuperscript{d} LXX MT lack \fbib{they ask}} \\
\poemll    `but you take no notice?'\,''
\passage{Fasting that God Approves}
\poeml ``Look! On your fast day you serve your own interest \\
\poemll    and oppress all your workers. \\
\poeml \v{4}``Look! You fast only for quarreling, and for\fnote{\fbackref{58:4} So 1QIsa\textsuperscript{b} 1QIsa\textsuperscript{a}; MT LXX lack \fbib{for}} fighting, \\
\poemll    and for hitting with wicked fists. \\
\poeml You cannot fast as you do today \\
\poemll    and have your voice heard on high. \\
\poeml \v{5}``Is this the kind of fast that I have chosen, \\
\poemll    merely a day for a person to humble himself? \\
\poeml Is it merely for bowing down one's head like a bulrush, \\
\poemll    for lying\fnote{\fbackref{58:5} So 1QIsa\textsuperscript{a} 1QIsa\textsuperscript{b}; MT LXX read \fbib{and for lying}} on sackcloth and ashes? \\
\poeml Is this what you\fnote{\fbackref{58:5} So 1QIsa\textsuperscript{a} 4QIsa\textsuperscript{d} LXX (pl.); 1QIsa\textsuperscript{b} MT (sing.)} call a fast, \\
\poemll    an\fnote{\fbackref{58:5} So 1QIsa\textsuperscript{a} 1QIsa\textsuperscript{b}; MT reads \fbib{and an}} acceptable day to the \divine{Lord}? \\
\poeml \v{6}Isn't this the\fnote{\fbackref{58:6} So 1QIsa\textsuperscript{a}; 1QIsa\textsuperscript{b} MT LXX lack \fbib{the}} fast that\fnote{\fbackref{58:6} So 1QIsa\textsuperscript{a}; 1QIsa\textsuperscript{b} MT LXX lack \fbib{that}} I have been choosing: \\
\poemll    to loose the bonds of injustice, \\
\poeml and\fnote{\fbackref{58:6} So 1QIsa\textsuperscript{a}; 1QIsa\textsuperscript{b} MT LXX lack \fbib{and}} to untie the cords of the yoke, \\
\poemll    and\fnote{\fbackref{58:6} So 1QIsa\textsuperscript{a} MT; 1QIsa\textsuperscript{b} 4QIsa\textsuperscript{d} LXX lack \fbib{and}} to let the oppressed go free, \\
\poemlll       and to break every yoke? \\
\poeml \v{7}Isn't it to share your bread with the hungry, \\
\poemll    and to bring the homeless poor into your house; \\
\poeml when you see the naked, \\
\poemll    to cover him with clothing,\fnote{\fbackref{58:7} So 1QIsa\textsuperscript{a}; 1QIsa\textsuperscript{b} MT LXX lack \fbib{with clothing}} \\
\poemlll       and not to raise yourself up\fnote{\fbackref{58:7} So 1QIsa\textsuperscript{a}; 1QIsa\textsuperscript{b} MT read \fbib{to hide yourself}; LXX reads \fbib{to disregard}} from your own flesh and blood?''
\passage{God's Reward}
\poeml \v{8}``Then your light will break forth like the dawn, \\
\poemll    and your healing will spring up quickly; \\
\poeml and your vindication will go before you, \\
\poemll    and\fnote{\fbackref{58:8} So 1QIsa\textsuperscript{a} 1QIsa\textsuperscript{b} LXX; MT lacks \fbib{and}} the glory of the \divine{Lord} will guard your back. \\
\poeml \v{9}Then you'll call, \\
\poemll    and the \divine{Lord} will answer; \\
\poeml you'll cry for help, \\
\poemll    and he'll respond, `Here I am.' \\
\poeml ``If you do away with the yoke among you, \\
\poemll    and\fnote{\fbackref{58:9} So 1QIsa\textsuperscript{a} LXX; 1QIsa\textsuperscript{b} MT lack \fbib{and}} pointing fingers and malicious talk; \\
\poeml \v{10}if you pour yourself out for the hungry \\
\poemll    and satisfy the needs of afflicted souls, \\
\poeml then your light will rise in darkness, \\
\poemll    and your night will be like noonday. \\
\poeml \v{11}And the \divine{Lord} will guide you continually, \\
\poemll    and satisfy your soul in parched places,\fnote{\fbackref{58:11} 1QIsa\textsuperscript{a} spells the word \fbib{places} incorrectly} \\
\poemlll       and they\fnote{\fbackref{58:11} So 1QIsa\textsuperscript{a} 1QIsa\textsuperscript{b}; MT reads \fbib{he}; LXX reads \fbib{and your bones will be strengthened}} will strengthen your bones; \\
\poeml and you'll be like a watered garden, \\
\poemll    like a spring of water, \\
\poemlll       whose waters never fail. \\
\poeml \v{12}And your people will rebuild the ancient ruins; \\
\poemll    You'll raise up the age-old foundations,\fnote{\fbackref{58:12} Lit. \fbib{the foundations of many generations}} \\
\poeml and people will call you\fnote{\fbackref{58:12} So 1QIsa\textsuperscript{a}; 1QIsa\textsuperscript{b} MT LXX read \fbib{you will be called}} `Repairer of Broken Walls,' \\
\poemll    `Restorer of Streets to Live In.' \\
\poeml \v{13}``If you keep your feet from trampling the Sabbath, \\
\poemll    from\fnote{\fbackref{58:13} So 1QIsa\textsuperscript{a} 4QIsa\textsuperscript{n}; 1QIsa\textsuperscript{b} MT lack \fbib{from}} pursuing your own interests on my holy day, \\
\poeml if you call the Sabbath a delight \\
\poemll    and\fnote{\fbackref{58:13} So 1QIsa\textsuperscript{a} 1QIsa\textsuperscript{b} 4QIsa\textsuperscript{n}; MT LXX lack \fbib{and}} the \divine{Lord}'s holy day honorable; \\
\poeml and if you honor it by not going your own ways\fnote{\fbackref{58:13} So 1QIsa\textsuperscript{a} 4QIsa\textsuperscript{n} MT; 1QIsa\textsuperscript{b} reads \fbib{way}} \\
\poemll    and\fnote{\fbackref{58:13} So 1QIsa\textsuperscript{a}; 1QIsa\textsuperscript{b} MT lack \fbib{and}} seeking your own pleasure or speaking merely idle\fnote{\fbackref{58:13} 1QIsa\textsuperscript{a} 1QIsa\textsuperscript{b} 4QIsa\textsuperscript{n} MT LXX lack \fbib{merely idle}} words, \\
\poeml \v{14}then you will take delight in the \divine{Lord}, \\
\poemll    and he\fnote{\fbackref{58:14} So 1QIsa\textsuperscript{a} 1QIsa\textsuperscript{b} 4QIsa\textsuperscript{n} LXX; MT reads \fbib{and I}} will make you ride upon the heights of the earth; \\
\poeml and he\fnote{\fbackref{58:14} So 1QIsa\textsuperscript{a} LXX; 1QIsa\textsuperscript{b} 4QIsa\textsuperscript{n} MT read \fbib{and I}} will make you feast on the inheritance of your ancestor Jacob, your father. \\
\poeml ``Yes! The mouth of the \divine{Lord} has spoken.''
\end{poetry}
\labelchapt{59}
\passage{Sins that Separate from God}

\chapt{59}
\v{1}``See, the \divine{Lord}'s hand is not too short to save,

\begin{poetry}
\poemll    nor are his ears\fnote{\fbackref{59:1} So 1QIsa\textsuperscript{a}; MT LXX read \fbib{ear}} too dull to hear. \\
\poeml \v{2}Instead, your iniquities have been barriers \\
\poemll    between you and your God, \\
\poeml and your sins have concealed his face from you \\
\poemll    so that he won't listen. \\
\poeml \v{3}For your hands are defiled with blood, \\
\poemll    and your fingers with iniquity; \\
\poemlll       your tongue\fnote{\fbackref{59:3} So 1QIsa\textsuperscript{a}; MT LXX read \fbib{your lips have spoken lies, your tongue}} mutters wickedness. \\
\poeml \v{4}No one brings a lawsuit fairly, \\
\poemll    and no one goes to law honestly; \\
\poeml they have relied\fnote{\fbackref{59:4} So 1QIsa\textsuperscript{a} 1QIsa\textsuperscript{b}; MT reads \fbib{they rely}} on empty arguments \\
\poemll    and they tell lies; \\
\poeml they conceive\fnote{\fbackref{59:4} So 1QIsa\textsuperscript{a}; 1QIsa\textsuperscript{b} MT reads \fbib{to conceive}} trouble \\
\poemll    and give birth\fnote{\fbackref{59:4} So 1QIsa\textsuperscript{a} 1QIsa\textsuperscript{b}; MT reads \fbib{and to give birth}} to iniquity. \\
\poeml \v{5}They hatch\fnote{\fbackref{59:5} So 1QIsa\textsuperscript{a}; 1QIsa\textsuperscript{b} MT reads \fbib{They have hatched}; LXX reads \fbib{hatched}} adders' eggs\fnote{\fbackref{59:5} So 1QIsa\textsuperscript{a} LXX; 1QIsa\textsuperscript{b} MT read \fbib{an adder's eggs}} \\
\poemll    and weave\fnote{\fbackref{59:5} So 1QIsa\textsuperscript{a}; MT reads \fbib{weave}, but with a different Heb. word} a spider's web; \\
\poeml whoever eats their eggs dies, \\
\poemll    and any crushed egg hatches out futility.\fnote{\fbackref{59:5} Lit. \fbib{a viper}; so 1QIsa\textsuperscript{a}; 1QIsa\textsuperscript{b} MT LXX utilize feminine form} \\
\poeml \v{6}Their cobwebs cannot become clothing, \\
\poemll    they cannot cover themselves with what they make. \\
\poeml Their deeds are deeds of iniquity, \\
\poemll    and acts of violence fill their hands. \\
\poeml \v{7}Their feet rush to evil, \\
\poemll    and they are quick to shed innocent blood. \\
\poeml Their thoughts are thoughts of iniquity; \\
\poemll    ruin, destruction, and violence\fnote{\fbackref{59:7} So 1QIsa\textsuperscript{a}; MT LXX lack \fbib{and violence}} are in their paths. \\
\poeml \v{8}The pathway of peace they do not know, \\
\poemll    and there is no justice in their courses. \\
\poeml They have made their roads crooked; \\
\poemll    no one who walks in them will know peace.''
\passage{A Commitment to Wait on God}
\poeml \v{9}``So justice is far from us, \\
\poemll    and righteousness does not reach us. \\
\poeml We wait for light, but look---there is darkness; \\
\poemll    we wait for brightness, but we walk in deep darkness.\fnote{\fbackref{59:9} So 1QIsa\textsuperscript{a} LXX; MT reads \fbib{darknesses}} \\
\poeml \v{10}Let's grope\fnote{\fbackref{59:10} So 1QIsa\textsuperscript{a}; MT reads \fbib{We grope} LXX reads \fbib{They grope}} along the wall like the blind; \\
\poemll    let us grope like those who have no eyes. \\
\poeml We stumble at midday as if it were twilight, \\
\poemll    in desolate places\fnote{\fbackref{59:10} Or \fbib{among vigorous people}} like dead people. \\
\poeml \v{11}We all growl like bears; \\
\poemll    we\fnote{\fbackref{59:11} So 1QIsa\textsuperscript{a}; MT reads \fbib{and we}} sigh mournfully like doves. \\
\poeml We look for justice, but there is none, \\
\poemll    and\fnote{\fbackref{59:11} So 1QIsa\textsuperscript{a}; Not in MT LXX} for deliverance, but it's far from us. \\
\poeml \v{12}``For our transgressions before you are many, \\
\poemll    and our sins testify\fnote{\fbackref{59:12} 1QIsa\textsuperscript{a} cf. LXX; MT reads \fbib{sin testifies}} against us; \\
\poeml for our transgressions are with us, \\
\poemll    and as for our iniquities, \\
\poemlll       we acknowledge them: \\
\poeml \v{13}they've rebelled\fnote{\fbackref{59:13} So 1QIsa\textsuperscript{a}; MT reads \fbib{rebellion} LXX reads \fbib{we have sinned}} in\fnote{\fbackref{59:13} Lit. \fbib{and}} treachery against the \divine{Lord}, \\
\poemll    and are turning away from following our God; \\
\poeml and they've spoken\fnote{\fbackref{59:13} So 1QIsa\textsuperscript{a}; MT reads \fbib{speaking} LXX reads \fbib{we have spoken}} oppression and revolt, \\
\poemll    and are conceiving\fnote{\fbackref{59:13} So 1QIsa\textsuperscript{a}; MT reads \fbib{conceiving and uttering}; LXX reads \fbib{we have conceived and thought about}} lying words from the heart. \\
\poeml \v{14}I'll drive back justice,\fnote{\fbackref{59:14} So 1QIsa\textsuperscript{a}; MT reads \fbib{Justice is driven back} LXX reads \fbib{We withdrew justice}} \\
\poemll    and righteousness stands at a distance; \\
\poeml for truth has fallen in the public square, \\
\poemll    and honesty cannot enter. \\
\poeml \v{15}Truth is missing, \\
\poemll    and whoever turns away from evil becomes a prey.''
\passage{God Brings His Own Salvation}
\poeml ``Then the \divine{Lord} looked, and it displeased him \\
\poemll    that there was no justice. \\
\poeml \v{16}He saw that there was no one, \\
\poemll    and was appalled that there was no one to intervene; \\
\poeml so his own arm\fnote{\fbackref{59:16} I.e. \fbib{the Messiah}} brought him victory, \\
\poemll    and his righteous acts upheld him. \\
\poeml \v{17}He put on righteousness like a breastplate, \\
\poemll    and a helmet of salvation on his head; \\
\poeml he put on garments of vengeance for clothing, \\
\poemll    and wrapped himself in fury like a cloak. \\
\poeml \v{18}So he will repay according to their action: \\
\poemll    Anger to his enemies, retribution to his foes; \\
\poemlll       to the coastlands he will render their due. \\
\poeml \v{19}So people will fear the name of the \divine{Lord} from the west, \\
\poemll    and his glories\fnote{\fbackref{59:19} So 1QIsa\textsuperscript{a}; MT reads \fbib{glory} LXX reads \fbib{his glorious name}} from the rising of the sun; \\
\poeml for he will come as a pent-up stream \\
\poemll    that the breath of the \divine{Lord} drives along. \\
\poeml \v{20}``And a Redeemer will come to Zion, \\
\poemll    to those in Jacob who turn from transgression,'' says the \divine{Lord}.
\end{poetry}

\v{21}``As for me, this is my covenant with them,''\fnote{\fbackref{59:21} So 1QIsa\textsuperscript{a} MT\textsuperscript{mss} LXX; 1QIsa\textsuperscript{b} MT read \fbib{them}, but the reading contains a grammatical object error} says the \divine{Lord}. ``And\fnote{\fbackref{59:21} So 1QIsa\textsuperscript{a}; 1QIsa\textsuperscript{b} MT LXX lack \fbib{And}} my Spirit that is upon you, and my words that I have put in your mouth, won't depart from your mouth, or from the mouths of your children, or from the mouths of your children's children,\fnote{\fbackref{59:21} So 1QIsa\textsuperscript{a}; 1QIsa\textsuperscript{b} MT LXX read \fbib{children, says the \divine{Lord},}} from now on and forever.''
\labelchapt{60}
\passage{The Light of God's Deliverance}

\begin{poetry}
\poeml \chapt{60}
\v{1}``Arise, shine! \\
\poemll    For your light has come; \\
\poemlll       the\fnote{\fbackref{60:1} So 1QIsa\textsuperscript{a}; 1QIsa\textsuperscript{b} MT LXX read \fbib{and the glory}} glory of the \divine{Lord} has risen upon you. \\
\poeml \v{2}For look! Darkness will cover the earth \\
\poemll    and thick darkness is over the people,\fnote{\fbackref{60:2} Lit. \fbib{peoples}} \\
\poeml but the \divine{Lord} will arise upon you, \\
\poemll    and his glory will appear over you. \\
\poeml \v{3}Nations will come to your light, \\
\poemll    and kings before\fnote{\fbackref{60:3} So 1QIsa\textsuperscript{a}; MT reads \fbib{kings to the brightness of}; cf. LXX} your dawn. \\
\poeml \v{4}``Lift up your eyes and look around: \\
\poemll    They all gather together, they come to you; \\
\poeml your sons will come from far away, \\
\poemll    and your daughters will be carried on the hip.''\fnote{\fbackref{60:4} Or \fbib{arm}} \\
\poeml \v{5}Then you will look and be radiant; \\
\poemll    your heart will swell with joy,\fnote{\fbackref{60:5} So 1QIsa\textsuperscript{a}; 1QIsa\textsuperscript{b} MT read \fbib{will throb and swell with joy} LXX reads \fbib{you will be amazed in your heart}} \\
\poeml because the abundance of the seas will be diverted to you, \\
\poemll    and the riches of the nations will come to you. \\
\poeml \v{6}Throngs of camels will blanket you: \\
\poemll    the young camels of Midian and Ephu;\fnote{\fbackref{60:6} So 1QIsa\textsuperscript{a}; 1QIsa\textsuperscript{b} MT read \fbib{Ephah}} \\
\poemlll       all those from Shebu\fnote{\fbackref{60:6} So 1QIsa\textsuperscript{a}; 1QIsa\textsuperscript{b} MT read \fbib{Sheba}} will come. \\
\poeml They'll carry gold and frankincense, \\
\poemll    and proclaim the praise of the \divine{Lord}. \\
\poeml \v{7}All Kedar's flocks will be gathered to you, \\
\poemll    the rams of Nebaioth will serve you. \\
\poeml and\fnote{\fbackref{60:7} So 1QIsa\textsuperscript{a} LXX; 1QIsa\textsuperscript{b} MT lack \fbib{and}} they'll come up with acceptance upon\fnote{\fbackref{60:7} So 1QIsa\textsuperscript{a} cf. LXX; 1QIsa\textsuperscript{b} MT read \fbib{upon the acceptance of}} my altar, \\
\poemll    and I'll glorify my glorious house.''
\passage{The Future Restoration of Zion}
\poeml \v{8}``Who are these that fly like clouds, \\
\poemll    and like doves to their windows?\fnote{\fbackref{60:8} I.e. \fbib{dovecotes}, cages in which pet doves are housed} \\
\poeml \v{9}For the coastlands will look to me, \\
\poemll    with the ships of Tarshish in the lead, \\
\poeml to bring my\fnote{\fbackref{60:9} So 1QIsa\textsuperscript{a}; 1QIsa\textsuperscript{b} MT LXX read \fbib{your}} children from far away, \\
\poemll    their silver and gold with them, \\
\poeml to the name of the \divine{Lord} your God, \\
\poemll    and to the Holy One of Israel, \\
\poemlll       because he has glorified you. \\
\poeml \v{10}``Foreigners will rebuild your walls, \\
\poemll    and their kings will serve you. \\
\poeml Though in my anger I struck you down, \\
\poemlll       in my favor I have shown you mercy. \\
\poeml \v{11}Your gates will always stand open \\
\poemll    by day or night, and\fnote{\fbackref{60:11} So 1QIsa\textsuperscript{a}; 1QIsa\textsuperscript{b} MT LXX lack \fbib{and}} they will not be shut, \\
\poeml so that nations will bring you their wealth, \\
\poemll    with their kings led in procession. \\
\poeml \v{12}For the nation or kingdom \\
\poemll    that will not serve you will perish; \\
\poemlll       those nations will be utterly ruined. \\
\poeml \v{13}``He has given you\fnote{\fbackref{60:13} So 1QIsa\textsuperscript{a}; 1QIsa\textsuperscript{b} MT LXX lack \fbib{He has given you}} the glory of Lebanon, \\
\poemll    and it will come\fnote{\fbackref{60:13} So 1QIsa\textsuperscript{a}; 1QIsa\textsuperscript{b} MT LXX read \fbib{Lebanon will come}} to you, \\
\poemlll       the cypress, and\fnote{\fbackref{60:13} So 1QIsa\textsuperscript{a}; 1QIsa\textsuperscript{b} MT lack \fbib{and}} the plane tree,\fnote{\fbackref{60:13} i.e. a species of trees that could readily be stripped of their bark; cf. Gen 30:37} and the pine, \\
\poeml to adorn the place of my sanctuary; \\
\poemll    and I will make the place of my feet glorious. \\
\poeml \v{14}``All\fnote{\fbackref{60:14} So 1QIsa\textsuperscript{a}; 1QIsa\textsuperscript{b} MT LXX lack \fbib{All}} the descendants of those who oppressed you \\
\poemll    will come bending low before you, \\
\poeml and all those who despised you \\
\poemll    will bow down at your feet. \\
\poeml They'll call you `The City of the \divine{Lord},' \\
\poemll    `Zion of the Holy One of Israel.'\,''
\passage{Israel: the Joy of Generations}
\poeml \v{15}``Although you have been forsaken and despised, \\
\poemll    with no one traveling through, \\
\poeml I will make you the everlasting pride, \\
\poemll    the joy of all generations. \\
\poeml \v{16}You'll suck the milk of nations, \\
\poemll    You'll suck the breasts of kings. \\
\poeml Then you will realize that I, the \divine{Lord}, am your Savior \\
\poemll    and your Redeemer, the Mighty One of Jacob. \\
\poeml \v{17}``Instead of bronze, I'll bring gold, \\
\poemll    and instead of iron, I'll bring silver; \\
\poeml instead of wood, bronze, \\
\poemll    and instead of stones, iron. \\
\poeml I'll appoint peace as your supervisor \\
\poemll    and righteousness as your taskmaster. \\
\poeml \v{18}Then\fnote{\fbackref{60:18} So 1QIsa\textsuperscript{a} LXX; 1QIsa\textsuperscript{b} MT lack \fbib{Then}} violence will no longer be heard in your land, \\
\poemll    nor devastation or destruction within your borders; \\
\poeml but you'll call your walls `Salvation', \\
\poemll    and your gates `Praise'. \\
\poeml \v{19}``The sun will no longer be your light by day, \\
\poemll    nor for brightness will the moon shine on you by night\fnote{\fbackref{60:19} So 1QIsa\textsuperscript{a} LXX. Not in 1QIsa\textsuperscript{b} MT}--- \\
\poeml for the \divine{Lord} will be your everlasting light,\fnote{\fbackref{60:19} 1QIsa\textsuperscript{b} lacks \fbib{and your God {\ldots} light}; 1QIsa\textsuperscript{a} MT LXX contain the longer reading} \\
\poemll    and your God will be your glory. \\
\poeml \v{20}Your sun won't\fnote{\fbackref{60:20} So 1QIsa\textsuperscript{a} LXX; MT reads \fbib{will no longer}} set, \\
\poemll    nor will your moon withdraw itself--- \\
\poeml for the \divine{Lord} will be your everlasting light, \\
\poemll    and your days of mourning will end. \\
\poeml \v{21}Then your people will all be righteous; \\
\poemll    They'll possess the land forever. \\
\poeml They are the shoot\fnote{\fbackref{60:21} So 1QIsa\textsuperscript{a} 4QIsa\textsuperscript{m} MT; 1QIsa\textsuperscript{a} MT\textsuperscript{ms} lacks \fbib{the shoot}; LXX reads \fbib{Guarding}} that the \divine{Lord} planted,\fnote{\fbackref{60:21} Lit. \fbib{of the plantings of the \divine{Lord}}; so 1QIsa\textsuperscript{a}; 1QIsa\textsuperscript{b} reads \fbib{of his plantings} MT reads \fbib{of his planting} MT\textsuperscript{qere} reads \fbib{of my planting} LXX reads \fbib{the planting}} \\
\poemll    the works\fnote{\fbackref{60:21} So 1QIsa\textsuperscript{a}; 1QIsa\textsuperscript{b} MT read \fbib{work}} of his hands, \\
\poemlll       so that I might be glorified. \\
\poeml \v{22}The least of them will become a thousand, \\
\poemll    and the smallest one a mighty nation. \\
\poeml ``I am the \divine{Lord}; \\
\poemll    When the time is right,\fnote{\fbackref{60:22} Lit. \fbib{In its time}} I will do this swiftly.''
\end{poetry}
\labelchapt{61}
\passage{Good News of Deliverance}

\begin{poetry}
\poeml \chapt{61}
\v{1}``The Spirit of the \divine{Lord}\fnote{\fbackref{61:1} So 1QIsa\textsuperscript{a} 1QIsa\textsuperscript{b} LXX; 4QIsa\textsuperscript{m} MT read \fbib{the Lord \divine{God}}} is upon me, \\
\poemll    because the \divine{Lord} has anointed me; \\
\poeml he has sent me to bring good news to the oppressed \\
\poemll    and\fnote{\fbackref{61:1} So 1QIsa\textsuperscript{a}; MT LXX lacks \fbib{and}} to bind up the brokenhearted, \\
\poeml to proclaim freedom for the captives, \\
\poemll    and release from darkness\fnote{\fbackref{61:1} Or \fbib{prison}; or \fbib{and opening of the eyes}} for the prisoners; \\
\poeml \v{2}to proclaim the year of the \divine{Lord}'s favor, \\
\poemll    the\fnote{\fbackref{61:2} So 1QIsa\textsuperscript{a} LXX\textsuperscript{ms}; 4QIsa\textsuperscript{b} MT LXX read \fbib{and the}} day of vengeance of our God; \\
\poemlll       to comfort all who mourn; \\
\poeml \v{3}to provide for those who grieve in Zion--- \\
\poemll    to bestow on them a crown of beauty instead of ashes, \\
\poeml the oil of gladness instead of mourning, \\
\poemll    a mantle of praise instead of a spirit of despair.'' \\
\poeml ``Then people will call them\fnote{\fbackref{61:3} So 1QIsa\textsuperscript{a}; MT LXX read \fbib{they will be called}} ``Oaks of Righteousness'', \\
\poemll    ``The Planting of the \divine{Lord}'', \\
\poemlll       in order to display his splendor. \\
\poeml \v{4}They will rebuild the ancient ruins; \\
\poemll    they will restore the places long devastated; \\
\poeml they will build again the ruined cities, \\
\poemll    they will build again\fnote{\fbackref{61:4} So 1QIsa\textsuperscript{a}; MT LXX lacks \fbib{again}} the places devastated for many generations. \\
\poeml \v{5}Strangers will stand and feed your flocks, \\
\poemll    and foreigners will work your land \\
\poemlll       and dress your vines. \\
\poeml \v{6}But as for you, you will be called priests of the \divine{Lord}, \\
\poemll    and\fnote{\fbackref{61:6} So 1QIsa\textsuperscript{a}; MT LXX lacks \fbib{and}} you will be named ministers of our God. \\
\poeml You will feed on the wealth of the nations, \\
\poemll    and you will boast about their riches. \\
\poeml \v{7}Instead of your shame you will receive double, \\
\poemll    and instead of disgrace people will shout with joy over your\fnote{\fbackref{61:7} So 1QIsa\textsuperscript{a}; MT reads \fbib{their}} inheritance; \\
\poeml therefore you\fnote{\fbackref{61:7} So 1QIsa\textsuperscript{a}; MT LXX reads \fbib{they}} will inherit a double portion in their land; \\
\poemll    everlasting joy will be yours.''\fnote{\fbackref{61:7} So 1QIsa\textsuperscript{a}; MT reads \fbib{theirs}; LXX reads \fbib{over their head}} \\
\poeml \v{8}``For I, the \divine{Lord}, love justice, \\
\poemll    and\fnote{\fbackref{61:8} So 1QIsa\textsuperscript{a} LXX; MT lacks \fbib{and}} I hate robbery and iniquity; \\
\poeml I will faithfully present your reward\fnote{\fbackref{61:8} So 1QIsa\textsuperscript{a}; MT LX read \fbib{present their reward}} \\
\poemll    and make an everlasting covenant with you.\fnote{\fbackref{61:8} So 1QIsa\textsuperscript{a}; MT LXX read \fbib{with them}} \\
\poeml \v{9}Your\fnote{\fbackref{61:9} So 1QIsa\textsuperscript{a}; MT LXX read \fbib{Their}} offspring will be known among the nations, \\
\poemll    and your\fnote{\fbackref{61:9} So 1QIsa\textsuperscript{a}; MT LXX read \fbib{their}} descendants among the people.\fnote{\fbackref{61:9} Lit. \fbib{peoples}} \\
\poeml All who see them will acknowledge them, \\
\poemll    that they are an offspring the \divine{Lord} has blessed.''
\passage{Rejoicing in God's Deliverance}
\poeml \v{10}``I will heartily rejoice in the \divine{Lord}, \\
\poemll    my soul will delight in my God; \\
\poeml for he has wrapped me in garments of salvation; \\
\poemll    he has arrayed me in a robe of righteousness, \\
\poeml just like a bridegroom, \\
\poemll    like a priest\fnote{\fbackref{61:10} So 1QIsa\textsuperscript{a}; MT reads \fbib{bridegroom decks himself like a priest}; LXX reads \fbib{bridegroom decks me}} with a garland, \\
\poemlll       and like a bride adorns herself with her jewels. \\
\poeml \v{11}For just as the soil brings forth its shoots, \\
\poemll    and as a garden makes what is sown within it spring up, \\
\poeml so the \divine{Lord} God\fnote{\fbackref{61:11} So 1QIsa\textsuperscript{a}; MT reads \fbib{the Lord \divine{God}}; LXX reads \fbib{the \divine{Lord}}} will make righteousness and praise \\
\poemll    spring up before all the nations for Zion's sake.''\fnote{\fbackref{61:11} So MT LXX; 1QIsa\textsuperscript{a} ends v. 11 with \fbib{for Zion's sake}}
\end{poetry}
\labelchapt{62}
\passage{The Vindication of Jerusalem}

\begin{poetry}
\poeml \chapt{62}
\v{1}``And\fnote{\fbackref{62:1} So 1QIsa\textsuperscript{a}; MT LXX lack \fbib{And}} I won't remain silent,\fnote{\fbackref{62:1} 1QIsa\textsuperscript{a} and MT use different Hebrew verbs for \fbib{silent}} \\
\poemll    and for Jerusalem's sake I won't stay quiet, \\
\poeml until her vindication shines out like brightness, \\
\poemll    and her salvation like a burning torch. \\
\poeml \v{2}The nations will see your vindication, \\
\poemll    and all the kings your glory; \\
\poeml and people will call you\fnote{\fbackref{62:2} So 1QIsa\textsuperscript{a}; MT reads \fbib{and you will be called}; LXX reads \fbib{he will call you}} by a new name \\
\poemll    that the mouth of the \divine{Lord} will bestow. \\
\poeml \v{3}You will be a crown of splendor in the \divine{Lord}'s hand, \\
\poemll    and a royal diadem in the hand of your God. \\
\poeml \v{4}And\fnote{\fbackref{62:4} So 1QIsa\textsuperscript{a} LXX; MT lacks \fbib{And}} you'll no longer be called `Deserted,' \\
\poemll    and your land will no longer be called `Desolate'; \\
\poeml but people will call you\fnote{\fbackref{62:4} So 1QIsa\textsuperscript{a}; 1QIsa\textsuperscript{b} MT lacks \fbib{will call you}} `Hephzibah,'\fnote{\fbackref{62:4} The Heb. word \fbib{Hephzibah} means \fbib{My Delight is in Her}} \\
\poemll    and your land `Beulah'\fnote{\fbackref{62:4} The Heb. word \fbib{Beulah} means \fbib{Married}}--- \\
\poeml for the \divine{Lord} will take delight in you, \\
\poemll    and your land will be married.'' \\
\poeml \v{5}``For just as\fnote{\fbackref{62:5} So 1QIsa\textsuperscript{a} LXX; 1QIsa\textsuperscript{b} MT lack \fbib{For just as}} a young man marries a maiden, \\
\poemll    so your sons will marry you; \\
\poeml and just as a bridegroom rejoices over his bride, \\
\poemll    so your God will rejoice over you. \\
\poeml \v{6}``Upon your walls, Jerusalem, \\
\poemll    I have posted watchmen; \\
\poeml all day and all night \\
\poemll    they won't\fnote{\fbackref{62:6} So 1QIsa\textsuperscript{a} 1QIsa\textsuperscript{b}; MT LXX read \fbib{never}} remain silent. \\
\poeml You who make mention of the \divine{Lord}, \\
\poemll    take no rest, \\
\poeml \v{7}and give him no rest \\
\poemll    until he prepares, establishes\fnote{\fbackref{62:7} So 1QIsa\textsuperscript{a}; 1QIsa\textsuperscript{b} MT LXX read \fbib{he establishes}} and makes Jerusalem \\
\poemlll       a song of praise throughout the earth. \\
\poeml \v{8}``The \divine{Lord} has sworn by his right hand \\
\poemll    and by his mighty arm:\fnote{\fbackref{62:8} I.e. \fbib{the Messiah}} \\
\poeml `I will never again give your grain\fnote{\fbackref{62:8} So 1QIsa\textsuperscript{a} 1QIsa\textsuperscript{b}; MT reads \fbib{I will never give your grain again}} \\
\poemll    as food for your enemies; \\
\poeml never\fnote{\fbackref{62:8} So 1QIsa\textsuperscript{a}; 1QIsa\textsuperscript{b} MT LXX read \fbib{and never}} again will foreigners drink your new wine \\
\poemll    for which you have toiled; \\
\poeml \v{9}but surely\fnote{\fbackref{62:9} So 1QIsa\textsuperscript{a} LXX; 1QIsa\textsuperscript{b} MT lack \fbib{but surely}} those who harvest it will eat it \\
\poemll    and praise the name of\fnote{\fbackref{62:9} So 1QIsa\textsuperscript{a}; 1QIsa\textsuperscript{b} MT LXX lack \fbib{the name of}} the \divine{Lord}, \\
\poeml and those who gather it will drink it \\
\poemll    in the courts of my sanctuary,' says your God.''\fnote{\fbackref{62:9} So 1QIsa\textsuperscript{a}; 1QIsa\textsuperscript{b} MT LXX lack \fbib{says your God}}
\passage{The Coming of God to Reign}
\poeml \v{10}``Pass through\fnote{\fbackref{62:10} So 1QIsa\textsuperscript{a} LXX; 1QIsa\textsuperscript{b} MT read \fbib{Pass through! Pass through}} the gates! \\
\poemll    prepare the way for the people! \\
\poeml Build up! Build up the highway! \\
\poemll    Clear it of stumbling stones,\fnote{\fbackref{62:10} So 1QIsa\textsuperscript{a}; cf. Isa 8:14; 1QIsa\textsuperscript{b} MT LXX read \fbib{of stones}} \\
\poemlll       speak among the peoples.\fnote{\fbackref{62:10} So 1QIsa\textsuperscript{a}; 1QIsa\textsuperscript{b} MT LXX read \fbib{raise a banner over the peoples}} \\
\poeml \v{11}Here is the \divine{Lord}! \\
\poemll    Proclaim\fnote{\fbackref{62:11} So 1QIsa\textsuperscript{a}; 1QIsa\textsuperscript{b} MT LXX read \fbib{See, the \divine{Lord} has proclaimed}} to the ends\fnote{\fbackref{62:11} So 1QIsa\textsuperscript{a}; 1QIsa\textsuperscript{b} MT LXX read \fbib{end}} of the earth, \\
\poeml say to the inhabitants\fnote{\fbackref{62:11} Lit. \fbib{daughter}} of Zion: \\
\poemll    `See, your salvation is coming! \\
\poeml See, his reward is with him, \\
\poemll    and his recompenses are\fnote{\fbackref{62:11} So 1QIsa\textsuperscript{a}; 1QIsa\textsuperscript{b} MT read \fbib{recompense is}; 1QIsa\textsuperscript{b} MT LXX read \fbib{work is}} before him.' \\
\poeml \v{12}People will call them, `The Holy People,' \\
\poemll    `The Redeemed of the \divine{Lord}'; \\
\poeml and they will call you,\fnote{\fbackref{62:12} So 1QIsa\textsuperscript{a}; 1QIsa\textsuperscript{b} MT LXX read \fbib{you will be called}} `Sought After,' \\
\poemll    `The City Not Deserted.'\,''
\end{poetry}
\labelchapt{63}
\passage{God's Day of Vengeance}

\chapt{63}
\v{1}``Who is this coming from Edom,

\begin{poetry}
\poemll    from Bozrah, in garments stained crimson? \\
\poeml Who is this, robed in such splendor, \\
\poemll    marching in his great might? \\
\poeml It is I, speaking in vindication, \\
\poemll    mighty to save. \\
\poeml \v{2}``Why is your clothing red, \\
\poemll    and your garments like those worn by the ones who tread in the winepress?\fnote{\fbackref{63:2} So 1QIsa\textsuperscript{a}; or \fbib{coriander} or \fbib{clothing}} \\
\poeml \v{3}``I have trodden the winepress alone, \\
\poemll    and from my people\fnote{\fbackref{63:3} So 1QIsa\textsuperscript{a}; 1QIsa\textsuperscript{b} MT LXX read \fbib{from the peoples}} no one was with me, \\
\poeml I trampled them in my anger \\
\poemll    and trod them down in my wrath; \\
\poeml their lifeblood spattered on my garments,\fnote{\fbackref{63:3} So 1QIsa\textsuperscript{b} MT LXX; 1QIsa\textsuperscript{a} lacks \fbib{I trampled{\ldots}my garment}} \\
\poemll    and I stained\fnote{\fbackref{63:3} So 1QIsa\textsuperscript{a} 1QIsa\textsuperscript{b}; MT verb \fbib{I stained} is problematic.} all my clothing. \\
\poeml \v{4}``For the day of vengeance was in my heart, \\
\poemll    and the year for my redeeming work had come. \\
\poeml \v{5}I looked, but there was no helper, \\
\poemll    I was appalled that there was no one to give support;\fnote{\fbackref{63:5} So 1QIsa\textsuperscript{a}; 1QIsa\textsuperscript{b} MT read \fbib{to support me}} \\
\poeml so my own arm\fnote{\fbackref{63:5} I.e. \fbib{the Messiah}} brought me victory, \\
\poemll    and as for my wrath, it supported me. \\
\poeml \v{6}I trampled people\fnote{\fbackref{63:6} Lit. \fbib{peoples}} in my anger; \\
\poemll    in my wrath I made them drunk \\
\poemlll       and I poured out their lifeblood on the ground.''
\passage{God's Grace to Israel}
\poeml \v{7}I will recount the gracious deeds of the \divine{Lord}, \\
\poemll    the praiseworthy acts of the \divine{Lord}, \\
\poeml according to all the \divine{Lord} has done for us--- \\
\poemll    yes, the great goodness to the house of Israel \\
\poeml that he has granted them according to his mercy, \\
\poemll    according to the abundance of his gracious love. \\
\poeml \v{8}For he said, ``Surely they are my people, \\
\poemll    children who won't act falsely.'' \\
\poemlll       And so he became their savior. \\
\poeml \v{9}In all their distress he wasn't distressed,\fnote{\fbackref{63:9} So 1QIsa\textsuperscript{a} MT; some MT\textsuperscript{mss} read \fbib{he was distressed}.} \\
\poemll    but the angel of his presence saved them; \\
\poeml in his acts of love\fnote{\fbackref{63:9} So 1QIsa\textsuperscript{a}; MT reads \fbib{in his love}; LXX reads \fbib{because of his love for them}} and in his acts\fnote{\fbackref{63:9} So 1QIsa\textsuperscript{a}; MT reads \fbib{act of pity}} of pity he redeemed them; \\
\poemll    he carried them and lifted them up\fnote{\fbackref{63:9} So 1QIsa\textsuperscript{a}; 1QIsa\textsuperscript{b} MT read \fbib{lifted them up and carried them}} all the days of old. \\
\poeml \v{10}Yet they rebelled \\
\poemll    and grieved his Holy Spirit; \\
\poeml so he changed and became their enemy, \\
\poemll    and\fnote{\fbackref{63:10} So 1QIsa\textsuperscript{a} LXX; MT lacks \fbib{and}} he himself fought against them. \\
\poeml \v{11}Then they\fnote{\fbackref{63:11} I.e. \fbib{his people}} remembered the days of old, \\
\poemll    of Moses his servant. \\
\poeml Where is the one who brought up\fnote{\fbackref{63:11} So 1QIsa\textsuperscript{a} LXX; MT reads \fbib{brought them up}} out of the sea \\
\poemll    the\fnote{\fbackref{63:11} So 1QIsa\textsuperscript{a}; MT reads \fbib{with the}} shepherds of his flock? \\
\poeml Where is the one who put his Holy Spirit among them, \\
\poeml \v{12}and\fnote{\fbackref{63:12} So 1QIsa\textsuperscript{a}; MT LXX lack \fbib{and}} who made his glorious arm\fnote{\fbackref{63:12} I.e. the Meessiah; lit. \fbib{arm of his glories}; so 1QIsa\textsuperscript{a}; MT reads \fbib{arm of his glory}} march at Moses' right hand, \\
\poeml who divided the waters in front of them \\
\poemll    to win\fnote{\fbackref{63:12} So 1QIsa\textsuperscript{a}; MT LXX read \fbib{to win for himself}} an everlasting name, \\
\poeml \v{13}who led them through the depths? \\
\poemll    Like a horse in the open desert, \\
\poemlll       they did not stumble; \\
\poeml \v{14}like cattle that go down into the valley, \\
\poemll    the Spirit of the \divine{Lord} gave them rest. \\
\poeml For\fnote{\fbackref{63:14} So 1QIsa\textsuperscript{a}; MT LXX read \fbib{This is how}} you led your people, \\
\poemll    to win for yourself a glorious name.
\passage{God the Father}
\poeml \v{15}Look down from heaven, and see \\
\poemll    from your holy and glorious dwelling. \\
\poeml Where are your zeal and your might? \\
\poemll    Where are the yearning of your heart and your compassion? \\
\poemlll       They are held back from me. \\
\poeml \v{16}But you are our Father, \\
\poemll    even\fnote{\fbackref{63:16} Lit. \fbib{and}; so 1QIsa\textsuperscript{a}; MT LXX read \fbib{but} or \fbib{although}} Abraham does not know us \\
\poemlll       and Israel has not acknowledged\fnote{\fbackref{63:16} So 1QIsa\textsuperscript{a} LXX; MT reads \fbib{Israel does not acknowledge}} us; \\
\poeml you are he,\fnote{\fbackref{63:16} So 1QIsa\textsuperscript{a}; 1QIsa\textsuperscript{b} MT LXX read \fbib{you}} O \divine{Lord}, our Father, \\
\poemll    from long ago, `Our Redeemer' is your name. \\
\poeml \v{17}Why, \divine{Lord}, do you make us wander\fnote{\fbackref{63:17} So 1QIsa\textsuperscript{a}; MT LXX read \fbib{do you make us wander, \divine{Lord}}} from your ways \\
\poemll    and harden our hearts, so that we do not fear you? \\
\poeml Turn back for the sake of your servants, \\
\poemll    for the sake of the tribes that are your heritage. \\
\poeml \v{18}Your holy people took possession\fnote{\fbackref{63:18} So 1QIsa\textsuperscript{a} (sing.); MT is pl.; LXX reads \fbib{so that we may take possession}} for a little while, \\
\poemll    but now our enemies have trampled down your sanctuary. \\
\poeml \v{19}For a long time we have been those you do not rule, \\
\poemll    those who are not called by your name.
\end{poetry}
\labelchapt{64}
\passage{A Prayer for God to Intervene}

\begin{poetry}
\poeml \chapt{64}
\v{1}\fnote{\fbackref{64:1} This v. is 63:19 in the MT}If only you would tear open the heavens and\fnote{\fbackref{64:1} So 1QIsa\textsuperscript{a}; MT lacks \fbib{and}} come down, \\
\poemll    so that the mountains would quake at your presence--- \\
\poeml \v{2}\fnote{\fbackref{64:2} This v. is 64:1 in the MT}just as fire sets twigs\fnote{\fbackref{64:2} Or \fbib{brushwood}} ablaze \\
\poemll    and the fire causes water to boil--- \\
\poeml to make known your name to your enemies, \\
\poemll    yes, to your enemies before you,\fnote{\fbackref{64:2} So 1QIsa\textsuperscript{a} LXX; MT reads \fbib{to make your name known to your adversaries}} \\
\poemlll       so that the nations might quake at your presence! \\
\poeml \v{3}When you did awesome deeds that we expected,\fnote{\fbackref{64:3} So 1QIsa\textsuperscript{a}; MT reads \fbib{did not expect}} \\
\poemll    you came down, \\
\poemlll       and the mountains shuddered before you. \\
\poeml \v{4}Since\fnote{\fbackref{64:4} So 1QIsa\textsuperscript{a} LXX; MT reads \fbib{And since}} ancient times no one has heard, \\
\poemll    and\fnote{\fbackref{64:4} So 1QIsa\textsuperscript{a}; MT lacks \fbib{and}} no ear has perceived, \\
\poeml and\fnote{\fbackref{64:4} So 1QIsa\textsuperscript{a} LXX; MT lacks \fbib{and}} no eye has seen any God besides you, \\
\poemll    who acts on behalf of those who wait for him. \\
\poeml \v{5}You come to the aid of those who gladly do what's right, \\
\poemll    To those who remember you in your ways. \\
\poeml See, you were angry, \\
\poemll    and we sinned against them for a long time, \\
\poemlll       but we will be saved. \\
\poeml \v{6}All of us have become like one who is unclean, \\
\poemll    and\fnote{\fbackref{64:6} So 1QIsa\textsuperscript{a} 4QIsa\textsuperscript{b} LXX; MT lacks \fbib{and}} all our righteous acts are like a filthy rag; \\
\poeml we all shrivel up like a leaf, \\
\poemll    and like the wind, our iniquities\fnote{\fbackref{64:6} The 1QIsa\textsuperscript{a} utilizes a masculine noun; 1QIsa\textsuperscript{b} MT utilize a feminine noun} sweep us away. \\
\poeml \v{7}There is no one who calls on your name \\
\poemll    or rouses himself to take hold of you; \\
\poeml for you have hidden your face from us, \\
\poemll    and have given us\fnote{\fbackref{64:7} So 1QIsa\textsuperscript{a}; MT reads \fbib{have melted us}; LXX reads \fbib{have delivered us}} into the control\fnote{\fbackref{64:7} Lit. \fbib{hand}} of our iniquity.
\passage{God, our Father, will Act}
\poeml \v{8}But as for you,\fnote{\fbackref{64:8} So 1QIsa\textsuperscript{a}; MT LXX read \fbib{But now}} O \divine{Lord}, you are our Father; \\
\poemll    and\fnote{\fbackref{64:8} So 1QIsa\textsuperscript{a} LXX; MT lacks \fbib{and}} we are clay,\fnote{\fbackref{64:8} So 1QIsa\textsuperscript{a} LXX; 1QIsa\textsuperscript{b} MT read \fbib{the clay}} \\
\poeml and you are our potter; \\
\poemll    we are all the work of your hands.\fnote{\fbackref{64:8} So 1QIsa\textsuperscript{a} LXX; MT reads \fbib{your hand}} \\
\poeml \v{9}Don't be angry beyond measure, \divine{Lord}, \\
\poemll    and don't remember our iniquity for a season.\fnote{\fbackref{64:9} So 1QIsa\textsuperscript{a} LXX; MT reads \fbib{for ever}} \\
\poemlll       Please look now, we are all your people. \\
\poeml \v{10}Your holy cities have become a desert; \\
\poemll    Zion has become like\fnote{\fbackref{64:10} So 1QIsa\textsuperscript{a} LXX; 1QIsa\textsuperscript{b} MT lack \fbib{like}} a desert, \\
\poemlll       Jerusalem a desolation. \\
\poeml \v{11}Our holy Temple and our splendor, \\
\poemll    where our ancestors praised you, \\
\poeml have become\fnote{\fbackref{64:11} So 1QIsa\textsuperscript{a}; MT LXX read \fbib{Our holy and glorious Temple {\ldots} has become}} a conflagration of fire, \\
\poemll    and all our dearest places have become\fnote{\fbackref{64:11} So 1QIsa\textsuperscript{a} LXX; MT reads \fbib{all our dearest places has become}; MT\textsuperscript{mss} read \fbib{our every dearest place has become}} ruins. \\
\poeml \v{12}\divine{Lord}, after all this, can you hold yourself back? \\
\poemll    Can you keep silent and punish us so severely?
\end{poetry}
\labelchapt{65}
\passage{God's Response}

\begin{poetry}
\poeml \chapt{65}
\v{1}``I let myself be sought by those who didn't ask for me;\fnote{\fbackref{65:1} So 1QIsa\textsuperscript{a} LXX; MT reads \fbib{ask}} \\
\poemll    I let myself be found by those who didn't seek me. \\
\poeml I said, `Here I am! Here I am!' \\
\poemll    to a nation that didn't call on my name. \\
\poeml \v{2}I held out my hands all day long \\
\poemll    to a disobedient\fnote{\fbackref{65:2} So 1QIsa\textsuperscript{a}; MT reads \fbib{an obstinate}} people, \\
\poeml who walk in a way that isn't good, \\
\poemll    following their own inclinations--- \\
\poeml \v{3}a people who continually provoke me to my face; \\
\poemll    they\fnote{\fbackref{65:3} So 1QIsa\textsuperscript{a} LXX; MT reads \fbib{who}} keep sacrificing in gardens \\
\poemlll       and waving their hands\fnote{\fbackref{65:3} So 1QIsa\textsuperscript{a}; MT LXX read \fbib{and offering incense}} over stone\fnote{\fbackref{65:3} So 1QIsa\textsuperscript{a}; MT LXX read \fbib{brick}} altars; \\
\poeml \v{4}who sit among graves, \\
\poemll    and spend the night in secret places; \\
\poeml who eat pigs' meat, \\
\poemll    with the broth\fnote{\fbackref{65:4} So 1QIsa\textsuperscript{a} MT\textsuperscript{qere} LXX Targ Vulg; MT reads \fbib{violence}; or \fbib{crumbs}} of detestable things in\fnote{\fbackref{65:4} So 1QIsa\textsuperscript{a}; MT LXX lack \fbib{in}} their pots; \\
\poeml \v{5}who say, `Keep to yourself!' \\
\poemll    `Don't touch\fnote{\fbackref{65:5} So 1QIsa\textsuperscript{a}; MT LXX read \fbib{come near to}} me!' and `I am\fnote{\fbackref{65:5} So 1QIsa\textsuperscript{a}; MT LXX read \fbib{for I am}} too holy for you!' \\
\poeml ``Such people are smoke in my nostrils, \\
\poemll    a fire that keeps burning all day long. \\
\poeml \v{6}Watch out! It stands written before me: \\
\poemll    `I won't keep silent, but I will pay back in full; \\
\poemlll       I'll indeed repay into\fnote{\fbackref{65:6} So 1QIsa\textsuperscript{a}; MT LXX read \fbib{upon}} their laps \\
\poeml \v{7}both your iniquities and your ancestors'\fnote{\fbackref{65:7} So 1QIsa\textsuperscript{a} MT; LXX Syr read \fbib{both their iniquities and their}} iniquities together,'\,' \\
\poemll    says the \divine{Lord}. \\
\poeml ``Because they offered incense on the mountains \\
\poemll    and insulted me on hills,\fnote{\fbackref{65:7} So 1QIsa\textsuperscript{a}; MT LXX read \fbib{the hills}} \\
\poeml I'll measure into\fnote{\fbackref{65:7} So 1QIsa\textsuperscript{a} MT\textsuperscript{qere}; MT LXX read \fbib{upon}} their laps \\
\poemll    full payment for their earlier actions.''
\passage{A Remnant will be Preserved}
\poeml \v{8}This is what the \divine{Lord} says: \\
\poemll    ``Just as new wine is found in the cluster, \\
\poeml and people have said,\fnote{\fbackref{65:8} So 1QIsa\textsuperscript{a}; MT LXX read \fbib{say}} `Don't destroy it, \\
\poemll    for there is a gift in it,' \\
\poeml so I'll act for my servants' sake, \\
\poemll    by not destroying them all. \\
\poeml \v{9}I'll bring forth descendants from Jacob, \\
\poemll    and from Judah they\fnote{\fbackref{65:9} Lit. \fbib{he}; so 1QIsa\textsuperscript{a} LXX} will inherit\fnote{\fbackref{65:9} So 1QIsa\textsuperscript{a} LXX; MT reads \fbib{Judah, the one about to inherit}} my mountains; \\
\poeml my chosen people will inherit it, \\
\poemll    and my servants will live there. \\
\poeml \v{10}Sharon will become a pasture for flocks, \\
\poemll    and the Valley of Achor a fold for herds,\fnote{\fbackref{65:10} So 1QIsa\textsuperscript{a} LXX; MT reads \fbib{a place for herds to lie down}} \\
\poemlll       for my people who have sought me. \\
\poeml \v{11}But as for you who forsake the \divine{Lord}, \\
\poemll    who forget my holy mountain, \\
\poeml who spread a table for Fortune\fnote{\fbackref{65:11} I.e. Fortune personified as a god} \\
\poemll    and\fnote{\fbackref{65:11} So 1QIsa\textsuperscript{a}; MT LXX reads \fbib{and who}} fill drink offerings\fnote{\fbackref{65:11} So 1QIsa\textsuperscript{a}; MT LXX read \fbib{cups of mixed wine}, or \fbib{mixing vessels}} for Destiny,\fnote{\fbackref{65:11} I.e. Destiny personified as a god} \\
\poeml \v{12}I'll consign\fnote{\fbackref{65:12} Lit. \fbib{destine}} you to the sword, \\
\poemll    and all of you will bend down for the slaughter--- \\
\poeml because when I called, you didn't answer, \\
\poemll    when I spoke, you didn't listen; \\
\poeml but you did what was evil in my sight, \\
\poemll    and chose what I took no pleasure in.''
\passage{The Righteous and Wicked Contrasted}
\poeml \v{13}Therefore, this is what the \divine{Lord}\fnote{\fbackref{65:13} So 1QIsa\textsuperscript{a} LXX; 1QIsa\textsuperscript{a} reads \fbib{Adonai}; MT reads \fbib{Lord \divine{God}}} says: \\
\poeml ``See, my servants will eat, \\
\poemll    but you'll go hungry; \\
\poeml my servants will drink, \\
\poemll    but you'll go thirsty; \\
\poeml my servants will rejoice, \\
\poemll    but you'll be put to shame. \\
\poeml \v{14}My servants will sing in gladness\fnote{\fbackref{65:14} So 1QIsa\textsuperscript{a} LXX; MT reads \fbib{out of gladness}} of heart, \\
\poemll    but you'll cry for help\fnote{\fbackref{65:14} So 1QIsa\textsuperscript{a}; MT LXX read \fbib{cry out}} from anguish of heart, \\
\poemlll       and you'll howl from brokenness of spirit. \\
\poeml \v{15}You'll leave your name to my chosen ones as a curse, \\
\poemll    and the Lord \divine{God} will put you to death permanently.\fnote{\fbackref{65:15} Or \fbib{for good}; so 1QIsa\textsuperscript{a}; MT LXX reads \fbib{but he will call his servants by a different name}} \\
\poeml \v{16}Then whoever takes an oath\fnote{\fbackref{65:16} So 1QIsa\textsuperscript{a}; MT LXX read \fbib{whoever invokes a blessing in the land will bless}} by the God of faithfulness, \\
\poemll    and whoever takes an oath in the land, \\
\poemlll       will swear by the God of faithfulness, \\
\poeml because the former troubles are forgotten \\
\poemll    and are hidden from my eyes.
\passage{A New Universe}
\poeml \v{17}Take notice! I'm about to create new heavens \\
\poemll    and a new earth; \\
\poeml the former things won't be remembered, \\
\poemll    nor will they come to mind. \\
\poeml \v{18}But be glad\fnote{\fbackref{65:18} Sing. 1QIsa\textsuperscript{a}; pl. MT} and rejoice\fnote{\fbackref{65:18} Sing. 1QIsa\textsuperscript{a}; pl. 1QIsa\textsuperscript{b} MT} forever \\
\poemll    in what I am creating, \\
\poeml for I am about to create Jerusalem as a joy, \\
\poemll    and its people as a delight. \\
\poeml \v{19}I'll rejoice over Jerusalem, \\
\poemll    and take delight in my people; \\
\poeml no longer will the sound of weeping be heard in it, \\
\poemll    nor the cry of distress. \\
\poeml \v{20}``And\fnote{\fbackref{65:20} So 1QIsa\textsuperscript{a} LXX; 1QIsa\textsuperscript{b} MT lack \fbib{And}} there will no longer be in it \\
\poemll    a young boy\fnote{\fbackref{65:20} So 1QIsa\textsuperscript{a}; 1QIsa\textsuperscript{b} MT read \fbib{an infant}; cf. 49:15} who lives only a few days, \\
\poemlll       or an old person who does not live out his days; \\
\poeml for one who dies at a hundred years will be thought a mere youth, \\
\poemll    and one who falls short of a hundred years will be considered accursed. \\
\poeml \v{21}People\fnote{\fbackref{65:21} Lit. \fbib{They}} will build houses and live in them; \\
\poemll    They'll plant vineyards and eat their fruit. \\
\poeml \v{22}They won't build for others to inhabit; \\
\poemll    they won't plant for others to eat--- \\
\poeml for like the lifetime\fnote{\fbackref{65:22} Lit. \fbib{days}} of a tree,\fnote{\fbackref{65:22} So 1QIsa\textsuperscript{a}; MT LXX read \fbib{the tree}} so will the lifetime\fnote{\fbackref{65:22} Lit. \fbib{days}} of my people be, \\
\poemll    and my chosen ones will long enjoy\fnote{\fbackref{65:22} Lit. \fbib{consume} or \fbib{wear out}} the work of their hands. \\
\poeml \v{23}They won't toil in vain \\
\poemll    nor bear children doomed to misfortune, \\
\poeml for they will be offspring blessed\fnote{\fbackref{65:23} Sing. 1QIsa\textsuperscript{a} LXX; pl. 1QIsa\textsuperscript{b} MT} by the \divine{Lord}, \\
\poemll    they and their descendants with them. \\
\poeml \v{24}Before they call, I will answer, \\
\poemll    while they are still speaking, I'll hear. \\
\poeml \v{25}``The wolf and the lamb will feed together, \\
\poemll    and the lion will eat straw like the ox; \\
\poeml but as for the serpent--- \\
\poemll    its food will be dust! \\
\poeml They won't harm or destroy \\
\poemll    on my entire holy mountain,''
\end{poetry}

says the \divine{Lord}.
\labelchapt{66}
\passage{The Worship that God Commands}

\chapt{66}
\v{1}This is what the \divine{Lord} says:

\begin{poetry}
\poeml ``Heaven is my throne, \\
\poemll    and the earth is my footstool. \\
\poeml Where is the house\fnote{\fbackref{66:1} I.e. a reconstructed Temple} that you would build for me, \\
\poemll    and where will my resting place be? \\
\poeml \v{2}All these things my hand has made, \\
\poemll    and so all these things came into being,'' \\
\poemlll       declares the \divine{Lord}. \\
\poeml ``But this is the one to whom I will look favorably: \\
\poemll    to the one who is humble and contrite in spirit, \\
\poemlll       and who\fnote{\fbackref{66:2} So 1QIsa\textsuperscript{a}; 1QIsa\textsuperscript{b} MT lack \fbib{who}} trembles at\fnote{\fbackref{66:2} So 1QIsa\textsuperscript{a}; 1QIsa\textsuperscript{b} MT use different Heb. prepositions} my message. \\
\poeml \v{3}``Whoever slaughters an ox \\
\poemll    is just like\fnote{\fbackref{66:3} So 1QIsa\textsuperscript{a} LXX; 1QIsa\textsuperscript{b} MT lack \fbib{just like}} one who kills a human being; \\
\poeml whoever sacrifices a lamb \\
\poemll    is just like one who breaks a dog's neck; \\
\poeml whoever makes a grain offering \\
\poemll    is just like one who offers pig's blood; \\
\poeml and whoever makes a memorial offering of frankincense \\
\poemll    is just like one who blesses an idol. \\
\poeml Yes, these have chosen their own ways, \\
\poemll    and they take delight in their contaminated actions. \\
\poeml \v{4}Therefore\fnote{\fbackref{66:4} Or \fbib{So}} I, too, will choose harsh treatment for them, \\
\poemll    and will bring upon them what they dread. \\
\poeml For when I called, no\fnote{\fbackref{66:4} So 1QIsa\textsuperscript{a}; MT LXX read \fbib{and no}} one answered; \\
\poemll    when I spoke, they didn't listen; \\
\poeml but they did what I consider to be evil, \\
\poemll    and chose what doesn't please me.''
\passage{The \divine{Lord} Vindicates Zion}
\poeml \v{5}``Hear this message from the \divine{Lord}, \\
\poemll    you who tremble at his words:\fnote{\fbackref{66:5} So 1QIsa\textsuperscript{a}; MT LXX read \fbib{his word}} \\
\poeml ``Your own brothers who hate you \\
\poemll    and exclude you because of my name \\
\poeml have said: `Let the \divine{Lord} be glorified; \\
\poemll    he will see\fnote{\fbackref{66:5} So 1QIsa\textsuperscript{a}; MT reads \fbib{so that we may see}; LXX reads \fbib{so that the name of the Lord may be glorified}} your joy,' \\
\poemlll       yet it is they who will be put to shame. \\
\poeml \v{6}``Listen to that uproar in\fnote{\fbackref{66:6} So 1QIsa\textsuperscript{a}; MT LXX read \fbib{from}} the city! \\
\poemll    Listen to that noise from the Temple! \\
\poeml It is the sound of the \divine{Lord} \\
\poemll    paying back retribution to his enemies! \\
\poeml \v{7}``Before she goes into labor she gives birth; \\
\poemll    before her pains come upon her \\
\poemlll       she has delivered\fnote{\fbackref{66:7} So 1QIsa\textsuperscript{a}; 1QIsa\textsuperscript{b} MT LXX read \fbib{and she has delivered}} a son. \\
\poeml \v{8}Who has ever heard of such a thing? \\
\poemll    And\fnote{\fbackref{66:8} So 1QIsa\textsuperscript{a} LXX; 1QIsa\textsuperscript{b} MT lack \fbib{And}} who ever sees\fnote{\fbackref{66:8} So 1QIsa\textsuperscript{a}. 1QIsa\textsuperscript{b} MT LXX read \fbib{has seen}} such things? \\
\poeml Can a country be born in a single day, \\
\poemll    or can a nation be brought forth in a single moment? \\
\poeml Yet no sooner was Zion in labor \\
\poemll    than she delivered her children. \\
\poeml \v{9}Am I to open the womb and not deliver?'' \\
\poemll    asks\fnote{\fbackref{66:9} Lit. \fbib{says}} the \divine{Lord}. \\
\poeml ``And when I bring to delivery, am I to close\fnote{\fbackref{66:9} So 1QIsa\textsuperscript{a} (imperfect); 1QIsa\textsuperscript{b} MT (perfect)} the womb?'' \\
\poemll    asks your God. \\
\poeml \v{10}``Rejoice with Jerusalem, and be happy for her, \\
\poemll    all you who love her; \\
\poeml rejoice with her in gladness, \\
\poemll    all you who mourn over her, \\
\poeml \v{11}so that you may nurse and be satisfied \\
\poemll    at her consoling breasts, \\
\poeml and so that you may drink deeply and take delight \\
\poemll    from her glorious bosom.''
\passage{The Rule of God}
\poeml \v{12}This\fnote{\fbackref{66:12} So 1QIsa\textsuperscript{a}; MT LXX read \fbib{For this}} is what the \divine{Lord} says: \\
\poeml ``See, I will extend prosperity to her like a river, \\
\poemll    and the wealth of nations like a flooding stream; \\
\poeml and you will nurse, \\
\poemll    and you\fnote{\fbackref{66:12} 1QIsa\textsuperscript{a} feminine pl.; MT masculine pl.} will be carried on her hip,\fnote{\fbackref{66:12} Or \fbib{arm}} \\
\poemlll       and bounced upon her knees. \\
\poeml \v{13}Like a child whom his mother comforts, \\
\poemll    so I will comfort you; \\
\poemlll       and you will be comforted in Jerusalem. \\
\poeml \v{14}And when you look, your hearts will rejoice \\
\poemll    and your bodies will flourish like grass; \\
\poeml and it will be made known \\
\poemll    that the \divine{Lord}'s hand is with his servants, \\
\poemlll       but his fury is with his enemies. \\
\poeml \v{15}``Take notice! The \divine{Lord} will come with fire, \\
\poemll    and his chariot\fnote{\fbackref{66:15} So 1QIsa\textsuperscript{a}; MT LXX read \fbib{his chariots}} will be\fnote{\fbackref{66:15} 1QIsa\textsuperscript{a} MT LXX lack \fbib{will be}} like a whirlwind, \\
\poeml to pay back his anger---yes, his anger!---\fnote{\fbackref{66:15} So 1QIsa\textsuperscript{a}; 1QIsa\textsuperscript{b} MT LXX lack \fbib{yes, his anger!}} in fury, \\
\poemll    and his menacing rebukes\fnote{\fbackref{66:15} So 1QIsa\textsuperscript{a}; 1QIsa\textsuperscript{b} MT read \fbib{his rebuke}} in flames of fire. \\
\poeml \v{16}For with fire and with his sword the \divine{Lord} will proceed to judgment\fnote{\fbackref{66:16} So 1QIsa\textsuperscript{a}; 1QIsa\textsuperscript{b} MT read \fbib{settle his claim}} on all humanity,\fnote{\fbackref{66:16} Lit. \fbib{on the humanity}; so 1QIsa\textsuperscript{a}; MT reads \fbib{on humanity}} \\
\poemll    and those slain by the \divine{Lord} will be many.''
\end{poetry}

\v{17}``Those who consecrate and purify themselves to enter the groves,\fnote{\fbackref{66:17} I.e. pagan sacred worship sites located in forested areas} following the one at the center of those who eat the meat of pigs, disgusting things,\fnote{\fbackref{66:17} Or \fbib{vermin}} and rats, are all alike,''\fnote{\fbackref{66:17} So 1QIsa\textsuperscript{a}; 1QIsa\textsuperscript{b} MT LXX read \fbib{alike}---\fbib{they will meet their end together}} says\fnote{\fbackref{66:17} So 1QIsa\textsuperscript{a}; MT reads \fbib{declares}} the \divine{Lord}. \v{18}``But as for me, I know their actions and their thoughts. Come\fnote{\fbackref{66:18} So IQIsa\textsuperscript{a}; MT LXX read \fbib{I am about to come}} and gather all nations and languages, and they will come and see my glory.

\v{19}``I will put up signs\fnote{\fbackref{66:19} So IQIsa\textsuperscript{a} LXX; 1QIsa\textsuperscript{b} MT read \fbib{a sign}} among them, and from them I will send survivors to the nations---to Tarshish, Libya,\fnote{\fbackref{66:19} Lit. \fbib{Put}} and Lydia,\fnote{\fbackref{66:19} Lit. \fbib{Lud}} (who draw the bow),\fnote{\fbackref{66:19} The Lydians were known for their skills at archery} to Tubal and Greece,\fnote{\fbackref{66:19} Lit. \fbib{Javan}} to the far off coastlands that have not heard of my fame or seen my glory. Then they will proclaim my glory among the nations. \v{20}They will bring all---yes, all!---\fnote{\fbackref{66:20} So 1QIsa\textsuperscript{a}; 1QIsa\textsuperscript{b} MT read \fbib{bring all}; LXX reads \fbib{bring}} of your kindred from all the nations to\fnote{\fbackref{66:20} So 1QIsa\textsuperscript{a} LXX; MT reads \fbib{upon}} my holy mountain Jerusalem as an offering to the \divine{Lord}---on horses, in chariots, in wagons, and on mules---yes, even on mules!---\fnote{\fbackref{66:20} So 1QIsa\textsuperscript{a}; 1QIsa\textsuperscript{b} MT LXX lack \fbib{yes, even on mules!}} and on camels,'' says the \divine{Lord}, ``just as the Israelis bring a grain offering in a clean vessel to the \divine{Lord}'s house. \v{21}Then I will also select some of them for myself\fnote{\fbackref{66:21} So 1QIsa\textsuperscript{a} LXX; 1QIsa\textsuperscript{b} MT LXX lack \fbib{for myself}} as priests and as Levites,''\fnote{\fbackref{66:21} I.e. the ministry formerly held by the descendants of Levi} says the \divine{Lord}.

\begin{poetry}
\poeml \v{22}``For as the new heavens and the new earth \\
\poemll    that I am about to make \\
\poeml will endure before me,'' says the \divine{Lord}, \\
\poemll    ``so will your descendants and your name endure. \\
\poeml \v{23}And from New Moon to New Moon, \\
\poemll    and from Sabbath to Sabbath,\fnote{\fbackref{66:23} So 1QIsa\textsuperscript{a} 4QIsa\textsuperscript{c}; lit. \fbib{to her Sabbath}; MT reads \fbib{to his Sabbath}} \\
\poeml all\fnote{\fbackref{66:23} Lit. \fbib{the}; so 1QIsa\textsuperscript{a}; 4QIsa\textsuperscript{b} MT LXX lack \fbib{all}} humanity will come to worship before me,'' \\
\poemll    says the \divine{Lord}.
\end{poetry}

\v{24}``Then they will go out and look upon the dead bodies of the people who rebelled against me. For their worm will not die, nor will their fire be extinguished, and they will remain an object of revulsion to all\fnote{\fbackref{66:24} Lit. \fbib{the}; so 1QIsa\textsuperscript{a}; 4QIsa\textsuperscript{b} MT LXX lack \fbib{all}} humanity.''

\bookheader{Jeremiah}
\labelbook{Jer}

\bookpretitle{The Book of the Prophet}
\booktitle{Jeremiah}

\labelchapt{1}
\passage{Introduction}

\chapt{1}
\v{1}The words of Hilkiah's son Jeremiah,\fnote{\fbackref{1:1} The Heb. name \fbib{Jeremiah} means \fbib{God has appointed me}} who was one of the priests at Anathoth in the territory of Benjamin. \v{2}This message from the \divine{Lord} came to him during the thirteenth year of the reign of\fnote{\fbackref{1:2} Lit. \fbib{him in the days of}} Ammon's son Josiah, the king of Judah, \v{3}and during the reign of\fnote{\fbackref{1:3} Lit. \fbib{and in the days of}} Josiah's son Jehoiakim, the king of Judah, and continued until the exile of Jerusalem in the fifth month, at\fnote{\fbackref{1:3} Lit. \fbib{until}} the end of the eleventh year of the reign of Josiah's son Zedekiah, the king of Judah.
\passage{Jeremiah's Call as a Prophet}

\v{4}This message from the \divine{Lord} came to me:

\begin{poetry}
\poeml \v{5}``I knew you before I formed you in the womb; \\
\poemll    I set you apart for me before you were born; \\
\poemlll       I appointed you to be a prophet to the nations.''
\end{poetry}

\v{6}I replied, ``Ah, \divine{Lord} God! Look, I don't know how to speak, because I'm only\fnote{\fbackref{1:6} The Heb. lacks \fbib{only}} a young man.''

\v{7}Then the \divine{Lord} told me, ``Don't say, `I'm only\fnote{\fbackref{1:7} The Heb. lacks \fbib{only}} a young man,' for you will go everywhere I send you, and you will speak everything I command you. \v{8}Don't be afraid of them, because I am with you to deliver you,'' declares the \divine{Lord}.

\v{9}The \divine{Lord} stretched out his hand, touched my mouth, and then told me, ``Look, I've put my words in your mouth. \v{10}See, today I've appointed you to prophesy about nations and kingdoms, to pull up and tear down, to destroy and overthrow, to build and to plant.''
\passage{The Visions of the Almond Branch and Boiling Pot}

\v{11}This message from the \divine{Lord} came to me, asking, ``What do you see, Jeremiah?''

I replied, ``I see an almond branch.''

\v{12}The \divine{Lord} told me, ``You have observed well, because I'm watching over\fnote{\fbackref{1:12} The Heb. word \fbib{almond} (\fbib{shaqed}) sounds like the Heb. word \fbib{watching} (\fbib{shoqed})} my message, to make sure it comes about.''

\v{13}This message from the \divine{Lord} came to me a second time: ``What do you see?''

I replied, ``I see a boiling pot, and its mouth is tilted away from the north.''

\v{14}Then the \divine{Lord} told me, ``From the north disaster will pour out on all who live in the land, \v{15}because I'm about to summon all the families and kingdoms from the north,'' declares the \divine{Lord}.

\begin{poetry}
\poeml ``They'll come and each one will set up his seat\fnote{\fbackref{1:15} Or \fbib{throne}} \\
\poemll    at the entrance of the gates of Jerusalem, \\
\poeml against all of its surrounding walls, \\
\poemll    and against all of the towns of Judah. \\
\poeml \v{16}``I'll pronounce my judgments against them \\
\poemll    because of all their wickedness. \\
\poeml They have forsaken me, \\
\poemll    they have burned incense to other gods, \\
\poeml and they have bowed down in worship \\
\poemll    to the works of their own\fnote{\fbackref{1:16} The Heb. lacks \fbib{own}} hands.''\fnote{\fbackref{1:16} I.e. idols}
\end{poetry}
\passage{The \divine{Lord}'s Assurance to Jeremiah}

\v{17}``As for you, get ready!\fnote{\fbackref{1:17} Lit. \fbib{gird up your loins}} Stand up and tell them everything that I've commanded you. Don't be frightened as you face them, or I'll frighten you right in front of them.

\v{18}``As for me, today I'm making you a fortified city, an iron pillar, and a bronze wall against the whole land---against the kings of Judah, against its princes, against its priests, and against the people of the land. \v{19}They'll fight against you, but they won't prevail against you, because I am with you,'' declares the \divine{Lord}, ``to deliver you.''
\labelchapt{2}
\passage{Israel's Initial Fidelity}

\chapt{2}
\v{1}This message from the \divine{Lord} came to me:

\begin{poetry}
\poeml \v{2}``Go and announce to Jerusalem:
\end{poetry}

\begin{poetry}
\poeml `This is what the \divine{Lord} says:
\end{poetry}

\begin{poetry}
\poeml ``I remember the loyal devotion of your youth, \\
\poemll    your love as a bride. \\
\poeml You followed me in the desert, \\
\poemll    in a land that was not planted. \\
\poeml \v{3}Israel was consecrated\fnote{\fbackref{2:3} Or \fbib{set apart}} to the \divine{Lord}, \\
\poemll    she was the first fruits\fnote{\fbackref{2:3} I.e. the first and best} of his produce. \\
\poeml All who devoured her became guilty \\
\poemll    and disaster came on them,'' \\
\poemlll       declares the \divine{Lord}.'\,''
\passage{Her Rejection of God's Love}
\poeml \v{4}Listen to this message from the \divine{Lord}, \\
\poemll    you descendants of Jacob \\
\poemlll       and all the families of the descendants of Israel. \\
\poeml \v{5}This is what the \divine{Lord} says:
\end{poetry}

\begin{poetry}
\poeml ``What did your ancestors find wrong with me \\
\poemll    that they left me, \\
\poeml and pursued worthless things,\fnote{\fbackref{2:5} I.e. idols or false gods} \\
\poemll    and so they became worthless? \\
\poeml \v{6}``They didn't ask, `Where is the \divine{Lord} \\
\poemll    who brought us up from the land of Egypt, \\
\poeml who led us through the wilderness, \\
\poemll    through the land of desert and pits, \\
\poeml through the land of dryness and deep darkness, \\
\poemll    a land that people don't pass through, \\
\poemlll       and where no one lives?' \\
\poeml \v{7}``I brought you into the fruitful land to eat its fruit \\
\poemll    and its good things. \\
\poeml But you came in, defiled my land, \\
\poemll    and made my inheritance into an abomination. \\
\poeml \v{8}``The priests didn't say, `Where is the \divine{Lord}?' \\
\poemll    and those handling the Law didn't know me. \\
\poeml The rulers transgressed against me, \\
\poemll    the prophets prophesied by Baal, \\
\poemlll       and they followed that which does not profit.\fnote{\fbackref{2:8} I.e. idols or false gods} \\
\poeml \v{9}``Therefore I'll again accuse you,'' \\
\poemll    declares the \divine{Lord}, \\
\poeml ``and I'll accuse your grandchildren.'' \\
\poeml \v{10}``Indeed, go over to the coasts of Cyprus and see, \\
\poemll    send to Kedar and pay very close attention. \\
\poemlll       See if there has ever been such a thing as this! \\
\poeml \v{11}Has a nation ever changed gods \\
\poemll    when they aren't even gods? \\
\poeml But my people have exchanged their glory \\
\poemll    for that which does not profit. \\
\poeml \v{12}Heavens, be appalled at this, \\
\poemll    be shocked, be utterly\fnote{\fbackref{2:12} Lit. \fbib{very}} devastated,'' \\
\poemlll       declares the \divine{Lord}. \\
\poeml \v{13}``Indeed, my people have committed two evils: \\
\poemll    they have forsaken me, the fountain of living water,\fnote{\fbackref{2:13} I.e. fresh water} \\
\poeml and they have dug cisterns for themselves, \\
\poemll    broken cisterns that cannot hold water.''
\passage{Consequences of Israel's Unfaithfulness}
\poeml \v{14}``Is Israel a slave, or was he born a servant?\fnote{\fbackref{2:14} Lit. \fbib{was he a home born servant} (cf. Exod 21:4)} \\
\poemll    Why then has he become plunder? \\
\poeml \v{15}Young lions roar at him, they cry out loudly. \\
\poemll    They have made his land into a wasteland, \\
\poeml and his cities are destroyed \\
\poemll    so they are without inhabitants. \\
\poeml \v{16}Also, people from Memphis and Tahpanhes\fnote{\fbackref{2:16} I.e. Egyptian cities} \\
\poemll    have broken\fnote{\fbackref{2:16} Or \fbib{shaved}} your skull. \\
\poeml \v{17}You have done this to yourselves, have you not, \\
\poemll    by forsaking the \divine{Lord} your God, when he \\
\poemlll       is the one who led you on the way? \\
\poeml \v{18}Now, what are you doing on the road to Egypt, \\
\poemll    to drink the waters of the Nile? \\
\poeml And what are you doing on the road to Assyria, \\
\poemll    to drink the waters of the Euphrates? \\
\poeml \v{19}Your wickedness will be punished, \\
\poemll    and you will be corrected due to your acts of apostasy. \\
\poeml Know and see that it's evil and bitter for you \\
\poemll    to forsake the \divine{Lord} your God, \\
\poeml but the fear of me is not in you,'' \\
\poemll    declares the Lord \divine{God} of the Heavenly Armies. \\
\poeml \v{20}``For long ago I broke your yoke \\
\poemll    and tore off your bonds, \\
\poeml But you said, `I won't serve you!' \\
\poemll    Instead, on every high hill \\
\poeml and under every green tree, \\
\poemll    you bend down to commit fornication. \\
\poeml \v{21}I planted you myself as a choice vine, \\
\poemll    from the very best seed.\fnote{\fbackref{2:21} Lit. \fbib{faithful seed}} \\
\poeml How did you turn against me \\
\poemll    into a degenerate and foreign vine? \\
\poeml \v{22}Though you wash yourself with lye \\
\poemll    and use much soap, \\
\poeml the stain of your guilt is still before me,'' \\
\poemlll       declares the Lord \divine{God}.
\passage{Israel's Passion for Sin}
\poeml \v{23}``How can you say, `I'm not defiled. \\
\poemll    I haven't gone after the Baals.'?\fnote{\fbackref{2:23} I.e. images of the Canaanite storm god} \\
\poeml Look at what you've done\fnote{\fbackref{2:23} Lit. \fbib{at your way}} in the valley. \\
\poemll    Know what you have done. \\
\poemlll       You are a swift young camel galloping aimlessly; \\
\poeml \v{24}a wild donkey accustomed to the desert, \\
\poemll    sniffing the wind in her passion. \\
\poeml When she's in heat, \\
\poemll    who can turn her away? \\
\poeml None of the males who pursue her need to tire themselves out, \\
\poemll    for in her month\fnote{\fbackref{2:24} I.e. at mating time} they'll find her.'' \\
\poeml \v{25}``Don't run until your feet are bare \\
\poemll    and your throat is dry.\fnote{\fbackref{2:25} Lit. \fbib{Hold back your feet from being bare and your throat from being dry}} \\
\poeml But you say, `It's hopeless! \\
\poemll    Because I love foreign gods,\fnote{\fbackref{2:25} Or \fbib{foreigners}} I'll go after them!'\,'' \\
\poeml \v{26}``As a thief is disgraced when he's caught, \\
\poemll    so the house of Israel is disgraced--- \\
\poemlll       they, their kings, their princes, their priests, and their prophets, \\
\poeml \v{27}who say to a tree, `You are my father,' \\
\poemll    and to a stone, `You gave birth to me.' \\
\poeml They have turned their back to me, \\
\poemll    but not their faces. \\
\poeml In the time of their trouble, they'll say, \\
\poemll    `Rise up! Deliver us!'\,'' \\
\poeml \v{28}``But where are your gods \\
\poemll    that you made for yourselves? \\
\poeml Let them rise up, if they can deliver you \\
\poemll    in the time of your trouble. \\
\poemlll       You have as many gods as you have towns, Judah. \\
\poeml \v{29}Why do you contend with me? \\
\poemll    You have rebelled against me,'' \\
\poemlll       declares the \divine{Lord}. \\
\poeml \v{30}``I've punished your children with no results,\fnote{\fbackref{2:30} Lit. \fbib{in vain}} \\
\poemll    they have accepted no discipline. \\
\poeml Your sword has devoured your prophets \\
\poemll    like a destroying lion.'' \\
\poeml \v{31}``You, generation, \\
\poemll    pay attention to\fnote{\fbackref{2:31} Or \fbib{see}} this message from the \divine{Lord}! \\
\poeml Am I the desert to Israel, \\
\poemll    or a land of gloom? \\
\poeml Why do my people say, `We're free to roam? \\
\poemll    We won't come to you anymore.' \\
\poeml \v{32}Will a young woman forget her wedding\fnote{\fbackref{2:32} The Heb. lacks \fbib{wedding}} ornaments, \\
\poemll    or a bride her attire? \\
\poeml But my people have forgotten me \\
\poemll    days without number. \\
\poeml \v{33}How well you perfect your techniques\fnote{\fbackref{2:33} Lit. \fbib{your way}} for seeking love. \\
\poemll    Therefore you can teach even the most immoral women\fnote{\fbackref{2:33} Lit. \fbib{the wicked women}} your techniques.\fnote{\fbackref{2:33} Lit. \fbib{your ways}} \\
\poeml \v{34}On your skirts is found the lifeblood of the innocent poor, \\
\poemll    even though you didn't catch them breaking in. \\
\poeml Yet despite all these things, \\
\poeml \v{35}you say, `I'm innocent. \\
\poemlll       Surely his anger has turned away from me.'\,'' \\
\poeml ``I'm about to bring charges against you\fnote{\fbackref{2:35} Lit. \fbib{enter into judgment with you}} \\
\poemll    because you say, `I haven't sinned.' \\
\poeml \v{36}Why do you go about changing your mind so much? \\
\poemll    You will also be disappointed\fnote{\fbackref{2:36} Or \fbib{put to shame}} by Egypt, \\
\poemlll       just as you were disappointed\fnote{\fbackref{2:36} Or \fbib{put to shame}} by Assyria. \\
\poeml \v{37}You will also go out from this place \\
\poemll    with your hands over your heads.\fnote{\fbackref{2:37} I.e. in a gesture indicating mourning} \\
\poeml For the \divine{Lord} has rejected those in whom you trust, \\
\poemll    and you won't prosper through them.''
\end{poetry}
\labelchapt{3}
\passage{God Contemplates Divorcing Israel}

\begin{poetry}
\poeml \chapt{3}
\v{1}``When a man divorces his wife, she leaves him and \\
\poeml becomes another man's wife, \\
\poeml will the first husband\fnote{\fbackref{3:1} Lit. \fbib{he}} return to her again? \\
\poemlll       The land would be deeply polluted, would it not? \\
\poeml Since you have committed fornication with many lovers, \\
\poemll    would you now return to me?'' \\
\poemlll       declares the \divine{Lord}. \\
\poeml \v{2}``Look up to the barren heights and see. \\
\poemll    Is there any place\fnote{\fbackref{3:2} The Heb. lacks \fbib{Is there anyplace}} where you have not been ravished? \\
\poeml You have sat beside the road, waiting\fnote{\fbackref{3:2} The Heb. lacks \fbib{waiting}} for them\fnote{\fbackref{3:2} I.e. as if waiting for a prostitute} \\
\poemll    like a nomad in the desert. \\
\poeml And you have polluted the land \\
\poemll    with your fornication and your wickedness. \\
\poeml \v{3}This is why the rain has been withheld \\
\poemll    and there are no spring showers. \\
\poeml Yet you have a harlot's look\fnote{\fbackref{3:3} Lit. \fbib{forehead}} \\
\poemll    and you refuse to be ashamed. \\
\poeml \v{4}Have you not just called out to me, \\
\poemll    `My father, you are the friend of my youth--- \\
\poeml \v{5}will he hold on to his anger forever, \\
\poemll    will he persist in his wrath to the end?' \\
\poeml Look, you have spoken and done evil things, \\
\poemll    and you have succeeded in it.''\fnote{\fbackref{3:5} The Heb. lacks \fbib{in it}}
\end{poetry}
\passage{The Example of Samaria}

\v{6}In the time of King Josiah the \divine{Lord} told me, ``Have you seen what unfaithful Israel did? She went up on every high hill and under every green tree, and she committed fornication there. \v{7}I thought,\fnote{\fbackref{3:7} Lit. \fbib{I said}} `After she has done all these things, she will return to me.' But she didn't return, and her treacherous sister Judah saw this. \v{8}I saw that even though I had sent unfaithful Israel away for all her adulteries and had given her a\fnote{\fbackref{3:8} The Heb. lacks \fbib{a}} divorce decree, her treacherous sister Judah didn't fear, and she, too, committed adultery. \v{9}She took her fornication so lightly that she polluted the land and committed adultery with stones and trees.\fnote{\fbackref{3:9} I.e. the Canaanite fertility gods were represented by stones and trees.} \v{10}Yet in all this her treacherous sister Judah didn't return to me with her whole heart, but rather deceptively,'' declares the \divine{Lord}.
\passage{A Call for Repentance}

\v{11}Then the \divine{Lord} told me, ``Unfaithful Israel has shown herself more righteous than treacherous Judah. \v{12}Go, proclaim these words to the north, and say,

\begin{poetry}
\poeml `Return, unfaithful Israel,' \\
\poemll    declares the \divine{Lord}. \\
\poeml `I won't look on you in anger, \\
\poemll    for I am gracious,'\fnote{\fbackref{3:12} I.e. characterized by gracious love} \\
\poemlll       declares the \divine{Lord}. \\
\poeml `I won't remain angry forever. \\
\poeml \v{13}`Only acknowledge your iniquity, \\
\poemll    that you have rebelled against the \divine{Lord} your God, \\
\poemll    and have scattered your favors to strangers \\
\poemlll       under every green tree. \\
\poeml But you haven't obeyed me,' \\
\poemlll       declares the \divine{Lord}.
\end{poetry}

\v{14}``Return, unfaithful people,''\fnote{\fbackref{3:14} Or \fbib{sons}} declares the \divine{Lord}, ``for I am your husband.\fnote{\fbackref{3:14} Or \fbib{master}} I'll take you, one from a city and two from a family, and I'll bring you to Zion. \v{15}I'll give you shepherds\fnote{\fbackref{3:15} I.e. leaders} after my own heart, and they'll shepherd you with knowledge and good sense.''

\v{16}``And in those days when you increase in numbers and multiply in the land,'' declares the \divine{Lord}, ``people will no longer say, `The Ark of the Covenant of the \divine{Lord},' and it won't come to mind, and they won't remember it or miss it, nor will it be made again. \v{17}At that time people will call Jerusalem, ``The Throne of the \divine{Lord},'' and all the nations will be gathered to it, to the name of the \divine{Lord}, to Jerusalem. They'll no longer stubbornly follow their own evil desires.\fnote{\fbackref{3:17} Lit. \fbib{follow the stubbornness of their evil hearts}} \v{18}In those days the house of Judah will walk with the house of Israel, and together they'll come to the land that I gave your ancestors as an inheritance.''
\passage{God's Desire for His People}

\v{19}``I said,

\begin{poetry}
\poeml `How I wanted to treat you like children, \\
\poemll    and give you a pleasant land, \\
\poemlll       the most beautiful inheritance of the nations.' \\
\poeml I said, `You will call me, my father, \\
\poemll    and won't turn back from following me.' \\
\poeml \v{20}Instead, like an unfaithful wife leaves her husband, \\
\poemll    so you have been unfaithful to me, house of Israel,'' \\
\poemlll       declares the \divine{Lord}.
\passage{Israel Cries for Help}
\poeml \v{21}``A voice is heard on the barren heights, \\
\poemll    the weeping and pleading of the children of Israel \\
\poeml because they have perverted their way. \\
\poemll    They have forgotten the \divine{Lord} their God.''
\passage{God Calls for Repentance}
\poeml \v{22}``Turn back, unfaithful people,\fnote{\fbackref{3:22} \fbib{Or sons}} \\
\poemll    and I'll heal your unfaithfulness.''
\passage{Israel Replies}
\poeml ``Look, we're coming to you \\
\poemll    because you are the \divine{Lord} our God. \\
\poeml \v{23}Truly the hills are a deception,\fnote{\fbackref{3:23} I.e. as a source of deliverance} \\
\poemll    and the mountains\fnote{\fbackref{3:23} I.e. where false gods were worshipped} are confusion. \\
\poemlll       Truly, in the \divine{Lord} our God is Israel's salvation.'' \\
\poeml \v{24}Since our youth the false gods have consumed \\
\poemll    the products of our ancestors' hard work, \\
\poeml their sheep and their cattle, \\
\poemll    their sons and their daughters. \\
\poeml \v{25}``Let us lie down in our shame, \\
\poemll    and let our humiliation cover us, \\
\poeml because both we and our ancestors have sinned \\
\poemll    against the \divine{Lord} our God from our youth \\
\poemlll       until this present time. \\
\poeml We haven't obeyed the \divine{Lord} our God.''
\end{poetry}
\labelchapt{4}
\passage{Instructions for True Repentance}

\begin{poetry}
\poeml \chapt{4}
\v{1}``Israel, if you return to me,'' \\
\poeml declares the \divine{Lord}, \\
\poeml ``Return to me, \\
\poemll    remove your detestable idols from my presence, \\
\poemlll       and don't waver. \\
\poeml \v{2}If you swear, `as surely as the \divine{Lord} lives,' \\
\poemll    in truth, in justice, and in righteousness, \\
\poeml then nations will be blessed\fnote{\fbackref{4:2} Or \fbib{bless themselves}} by him, \\
\poemll    and in him they will boast.'' \\
\poeml \v{3}For this is what the \divine{Lord} says \\
\poemll    to the men\fnote{\fbackref{4:3} Or \fbib{people}} of Judah and Jerusalem, \\
\poeml ``Break up your unplowed ground, \\
\poemll    and don't sow among thorns. \\
\poeml \v{4}Circumcise yourselves to the \divine{Lord} \\
\poemll    and remove the foreskin of your heart, \\
\poeml you men of Judah and residents of Jerusalem, \\
\poemll    or else my wrath will break out like fire \\
\poeml and burn with no one to put it out, \\
\poemll    because of your evil deeds.''
\passage{Warning of the Coming Disaster}
\poeml \v{5}Declare in Judah, make known in Jerusalem, by saying, \\
\poemll    ``Blow the trumpet in the land, cry out, and say, \\
\poeml `Gather together \\
\poemll    and let's go to the fortified cities!' \\
\poeml \v{6}Raise a standard in the direction of Zion. \\
\poemll    Flee! Don't stand around! \\
\poeml For I'm bringing calamity from the north, \\
\poemll    along with great destruction. \\
\poeml \v{7}A lion has gone up from his thicket, \\
\poemll    and a destroyer of nations has set out. \\
\poeml He has left his place \\
\poemll    to make your land a waste. \\
\poeml Your cities will be ruined, \\
\poemll    and without inhabitants. \\
\poeml \v{8}So, put on sackcloth, \\
\poemll    mourn and wail, \\
\poeml because the burning anger of the \divine{Lord} \\
\poemll    has not turned away from us.'' \\
\poeml \v{9}``On that day,'' declares the \divine{Lord}, \\
\poemll    ``the courage of the king and the leaders will fail. \\
\poeml The priests will be appalled \\
\poemll    and the prophets astounded.''
\end{poetry}

\v{10}Then I replied, ``Ah, Lord GOD, you have completely deceived this people and Jerusalem when you said, `You will have peace,' while the sword is at their\fnote{\fbackref{410} Lit. \fbib{the}} throat!''
\passage{The Scorching Wind of Judgment}

\v{11}At that time, it will be told this people and to Jerusalem, ``A scorching wind from the barren heights in the desert is coming\fnote{\fbackref{4:11} The Heb. lacks \fbib{is coming}} toward my people, and it's not for winnowing or cleansing. \v{12}A wind too strong for that is coming at my bidding.\fnote{\fbackref{4:12} Lit. \fbib{coming to me}} Now I'm judging them as I speak.''
\passage{The People's Response to Judgment}

\begin{poetry}
\poeml \v{13}Look, he comes up like clouds, \\
\poemll    and his chariots are like a whirlwind. \\
\poeml His horses are as swift as eagles. \\
\poemll    Woe to us---we're destroyed! \\
\poeml \v{14}Jerusalem, wash your evil from your heart \\
\poemll    so that you may be delivered. \\
\poeml How long will you harbor \\
\poemll    evil schemes within you? \\
\poeml \v{15}For a voice announces from Dan \\
\poemll    and declares disaster from Mount Ephraim.
\passage{The \divine{Lord} Speaks}
\poeml \v{16}``Tell the nations, `Here they come!'\fnote{\fbackref{4:16} The Heb. lacks \fbib{they come}} \\
\poemll    Proclaim to Jerusalem, \\
\poeml `The besieging forces are coming from a distant land. \\
\poemll    They cry out\fnote{\fbackref{4:16} I.e. battle cries} against the cities of Judah. \\
\poeml \v{17}They have surrounded her like those guarding a field \\
\poemll    because they have rebelled against me,'\,'' \\
\poemlll       declares the \divine{Lord}. \\
\poeml \v{18}``Your lifestyles and your actions \\
\poemll    have brought these things on you. \\
\poeml This is your calamity---it is indeed bitter, \\
\poemll    for it has reached your heart!''
\passage{Jeremiah's Lament for His People}
\poeml \v{19}``My anguish, my anguish! I writhe in pain. \\
\poemll    Oh, the aching\fnote{\fbackref{4:19} Lit. \fbib{walls}} of my heart! \\
\poeml My heart pounds within me; \\
\poemll    I cannot keep silent. \\
\poeml For I hear the sound of the trumpet,\fnote{\fbackref{4:19} I.e. the signal for the troops to attack} \\
\poemll    the alarm for war. \\
\poeml \v{20}Disaster upon disaster is proclaimed, \\
\poemll    for the entire land is devastated. \\
\poeml Suddenly, my tent is destroyed, \\
\poemll    in a moment my curtains. \\
\poeml \v{21}How long will I see the battle standard \\
\poemll    and hear the sound of the trumpet?
\passage{The \divine{Lord}'s Complaint about His People}
\poeml \v{22}``For my people are foolish, \\
\poemll    they don't know me. \\
\poeml They're stupid children, \\
\poemll    they have no understanding. \\
\poeml They're skilled at doing evil, \\
\poemll    but how to do good, they don't know.''
\passage{A Vision of Chaos}
\poeml \v{23}I looked at the earth, and it was formless and void,\fnote{\fbackref{4:23} Cf. Gen 1:2} \\
\poemll    at the heavens, and there was no light there. \\
\poeml \v{24}I looked at the mountains; they were quaking, \\
\poemll    and all the hills moved back and forth. \\
\poeml \v{25}I looked, and no people were there. \\
\poemll    All the birds of the sky had gone. \\
\poeml \v{26}I looked, and the fruitful land\fnote{\fbackref{4:26} Or \fbib{Carmel}} had become a desert. \\
\poemll    All its towns were broken down \\
\poeml because of the \divine{Lord}, \\
\poemll    because of his burning anger.
\end{poetry}

\begin{poetry}
\poeml \v{27}For this is what the \divine{Lord} says:
\end{poetry}

\begin{poetry}
\poeml ``The entire land will be devastated, \\
\poemll    but I won't completely destroy\fnote{\fbackref{4:27} Lit. \fbib{do}} it. \\
\poeml \v{28}Because of this, the land will mourn, \\
\poemll    and the heavens above will be dark. \\
\poeml Because I have spoken and decided, \\
\poemll    I won't turn back from doing it.''
\passage{A Lament for Zion}
\poeml \v{29}At the sound of the horseman and the archer \\
\poemll    the entire city flees. \\
\poeml Its residents go into the thickets and climb among the rocks. \\
\poemll    Every city is abandoned, and no one lives in them. \\
\poeml \v{30}You are ruined! What are you doing \\
\poemll    dressing in scarlet, \\
\poeml putting on golden ornaments, \\
\poemll    and highlighting your eyes with makeup? \\
\poeml You are making yourself beautiful in vain. \\
\poemll    Your lovers reject you--- \\
\poemlll       they're out to kill you. \\
\poeml \v{31}I heard a cry like that of a woman in labor, \\
\poemll    anguish like one giving birth to her firstborn, \\
\poeml the cry of the daughter of Zion gasping for air, \\
\poemll    stretching out her hand: \\
\poemll    ``Woe is me! I'm about to faint in front of killers!''
\end{poetry}
\labelchapt{5}
\passage{A Dialogue about Righteousness: The \divine{Lord} Speaks}

\chapt{5}
\v{1}``Wander through the streets of Jerusalem.

\begin{poetry}
\poeml Look and investigate;\fnote{\fbackref{5:1} Lit. \fbib{know}} \\
\poeml search through her squares \\
\poemll    and see whether you find anyone--- \\
\poeml even one person there---doing justice and seeking truth. \\
\poemll    Then I'll forgive them.\fnote{\fbackref{5:1} Lit. \fbib{her}; i.e. Judah} \\
\poeml \v{2}Although they say, `As surely as the \divine{Lord} lives,' \\
\poemll    still they are swearing falsely.''\fnote{\fbackref{5:2} Or \fbib{swearing to false gods}}
\passage{The Prophet Speaks}
\poeml \v{3}\divine{Lord}, don't your eyes look for truth? \\
\poemll    You struck\fnote{\fbackref{5:3} I.e. in discipline} them, but they didn't flinch.\fnote{\fbackref{5:3} Or \fbib{did not weaken}} \\
\poeml You brought them to an end, \\
\poemll    but they refused to receive discipline. \\
\poeml They made their faces harder than stone, \\
\poemll    and they refused to repent. \\
\poeml \v{4}Then I said, ``These are only the poor, \\
\poemll    they're foolish, \\
\poeml for they don't know the \divine{Lord}'s way, \\
\poemll    the requirement\fnote{\fbackref{5:4} Or \fbib{judgments}} of their God. \\
\poeml \v{5}Let me go to the leaders\fnote{\fbackref{5:5} Lit. \fbib{great ones}} and speak to them. \\
\poemll    For they know the \divine{Lord}'s way, \\
\poemlll       the requirement\fnote{\fbackref{5:5} Or \fbib{judgments}} of their God.''
\passage{The \divine{Lord} Answers}
\poeml ``But they, all together, have broken the yoke \\
\poemll    and torn off the restraints.\fnote{\fbackref{5:5} Or \fbib{cords}} \\
\poeml \v{6}Therefore a lion from the forest will attack them, \\
\poemll    a wolf from the desert will devastate them. \\
\poeml A leopard is watching their towns, \\
\poemll    and everyone who goes out of them \\
\poemlll       will be torn to pieces. \\
\poeml For their transgressions are many, \\
\poemll    and their apostasies numerous. \\
\poeml \v{7}Why should I forgive you? \\
\poemll    Your sons have forsaken me, \\
\poeml and you have sworn by those \\
\poemll    who aren't gods. \\
\poeml When I gave them enough food to satisfy them, \\
\poemll    they committed adultery \\
\poemll    and marched to the prostitute's house. \\
\poeml \v{8}They were well-fed, lusty stallions, \\
\poemll    each one neighing after his neighbor's wife. \\
\poeml \v{9}``Should I not punish them for these things?'' \\
\poemll    asks the \divine{Lord}, \\
\poeml ``And on a nation like this, \\
\poemll    should I not seek retribution?''
\passage{The People Reject God's Warning}
\poeml \v{10}``Go through her rows of vines and destroy them, \\
\poemll    but don't completely destroy them. \\
\poeml Strip away her branches, \\
\poemll    because they aren't the \divine{Lord}'s. \\
\poeml \v{11}For both the house of Israel and the house of Judah \\
\poemll    have been utterly unfaithful to me,'' \\
\poemlll       declares the \divine{Lord}. \\
\poeml \v{12}``They have lied about the \divine{Lord} \\
\poemll    by saying, `He wouldn't do that!\fnote{\fbackref{5:12} Lit. \fbib{Not he}} \\
\poeml Disaster won't come on us. \\
\poemll    We won't see sword and famine. \\
\poeml \v{13}The prophets are nothing but\fnote{\fbackref{5:13} The Heb. lacks \fbib{nothing but}} wind, \\
\poemll    and the word is not in them. \\
\poeml So may the disaster happen to them!'\,''\fnote{\fbackref{5:13} Lit. \fbib{so let it be done to them}}
\end{poetry}

\v{14}Therefore, this is what the \divine{Lord} God of the Heavenly Armies says:

\begin{poetry}
\poeml ``Because you people\fnote{\fbackref{5:14} Heb. \fbib{you} (pl.)} have said this, \\
\poemll    my words in your\fnote{\fbackref{5:14} Heb. \fbib{your} (masculine sing.); i.e. in Jeremiah's mouth} mouth will become a fire \\
\poemlll       and these people the wood. \\
\poemll    The fire\fnote{\fbackref{5:14} Lit. \fbib{It}} will destroy them. \\
\poeml \v{15}People of Israel, I'm now bringing \\
\poemll    a nation from far away to attack you,'' \\
\poemlll       declares the \divine{Lord}. \\
\poeml ``It is an enduring nation, \\
\poemll    an ancient nation, \\
\poeml a nation whose language you don't know. \\
\poemll    And you won't understand what they say. \\
\poeml \v{16}Their quiver is like an open grave, \\
\poemll    and all of them are powerful warriors. \\
\poeml \v{17}``They'll devour your harvest and your food. \\
\poemll    They'll devour your sons and your daughters. \\
\poemlll       They'll devour your vines and your fig trees. \\
\poeml With their swords they'll batter down \\
\poemll    your fortified cities in which you trust.
\end{poetry}

\v{18}``Yet even in those days,'' declares the \divine{Lord}, ``I won't destroy you completely. \v{19}When the people\fnote{\fbackref{5:19} Lit. \fbib{they}} ask, `Why has the \divine{Lord} our God done all this to us?' you are to say to them, `Just as you have forsaken me and served foreign gods in your land, so you will serve strangers in a land that is not yours.'\,''
\passage{The \divine{Lord} Warns a Stubborn People}

\begin{poetry}
\poeml \v{20}``Declare this to the descendants\fnote{\fbackref{5:20} Or \fbib{family}} of Jacob, \\
\poemll    and proclaim it in Judah: \\
\poeml \v{21}`Hear this, you foolish and stupid people: \\
\poemll    They have eyes, but don't see; \\
\poemlll       they have ears, but don't hear. \\
\poeml \v{22}`You don't fear me, do you?' declares the \divine{Lord}. \\
\poemll    `You don't tremble before me, do you? \\
\poeml I'm the one who put the sand as a boundary for the sea, \\
\poemll    a perpetual barrier that it cannot cross.\fnote{\fbackref{5:22} Or \fbib{statute that it cannot transgress}} \\
\poeml Though the waves toss, they cannot prevail against it, \\
\poemll    though they roar, they cannot cross it.' \\
\poeml \v{23}But these people have stubborn and rebellious hearts. \\
\poemll    They have turned aside and have gone away. \\
\poeml \v{24}They don't say to themselves, \\
\poemll    `Let's fear the \divine{Lord} our God, \\
\poeml who gives rain in its season, \\
\poemll    both the autumn and the spring rain. \\
\poeml He sets aside for us the weeks appointed \\
\poemll    for the harvest.' \\
\poeml \v{25}Your iniquities have turned these things away, \\
\poemll    and your sins have held back from you what is good. \\
\poeml \v{26}``Evil men are found among my people. \\
\poemll    They lie in wait like someone who traps birds. \\
\poeml They set a trap, \\
\poemll    but they do so to catch people. \\
\poeml \v{27}Like a cage full of birds, \\
\poemll    so their houses are filled with treachery. \\
\poeml This is how they have become prominent and rich, \\
\poeml \v{28}and have grown fat and sleek. \\
\poeml There is no limit\fnote{\fbackref{5:28} Lit. \fbib{pass over}; or \fbib{transgress}} to their evil deeds. \\
\poemll    They don't argue the case of the orphan to secure\fnote{\fbackref{5:28} Lit. \fbib{win}} justice. \\
\poemlll       They don't defend the rights of\fnote{\fbackref{5:28} Lit. \fbib{judge justly}} the poor. \\
\poeml \v{29}`Should I not punish them for this?'\fnote{\fbackref{5:29} Or \fbib{punish these people}} \\
\poemll    asks the \divine{Lord}. \\
\poeml `Should I not avenge myself \\
\poemll    on a nation like this?' \\
\poeml \v{30}``An appalling and horrible thing \\
\poemll    has happened in the land: \\
\poeml \v{31}The prophets prophesy falsely, \\
\poemll    the priests rule by their own authority, \\
\poeml and my people love it this way. \\
\poemll    But what will you do in the end?''
\end{poetry}
\labelchapt{6}
\passage{The Enemy Besieges Jerusalem}

\chapt{6}
\v{1}``Flee to safety, you people of Benjamin,

\begin{poetry}
\poeml leave Jerusalem. \\
\poeml Sound the trumpet in Tekoa, \\
\poemll    and raise a signal over Beth-haccerem! \\
\poeml For calamity and terrible destruction \\
\poemll    are turning toward you\fnote{\fbackref{6:1} The Heb. lacks \fbib{toward you}} from the north. \\
\poeml \v{2}I'll destroy the lovely and delicate \\
\poemll    Daughter of Zion.\fnote{\fbackref{6:2} I.e. Jerusalem} \\
\poeml \v{3}Shepherds and their flocks will come against her. \\
\poemll    They'll pitch their tents all around her, \\
\poemlll       and every one will tend his flock in his own place. \\
\poeml \v{4}Prepare for war against her. \\
\poemll    Get ready, let's attack at noon! \\
\poeml How terrible for us that the day is coming to an end,\fnote{\fbackref{6:4} Lit. \fbib{is turning}} \\
\poemll    and that the evening shadows are lengthening. \\
\poeml \v{5}Get ready, let's attack at night, \\
\poemll    and destroy her fortresses.''\fnote{\fbackref{6:5} Or \fbib{palaces}}
\end{poetry}
\passage{Instructions for the Attackers}

\begin{poetry}
\poeml \v{6}For this is what the \divine{Lord} of the Heavenly Armies says:
\end{poetry}

\begin{poetry}
\poeml ``Cut down trees and \\
\poemll    set up siege works against Jerusalem. \\
\poeml It is the city to be judged, \\
\poemll    and there is oppression throughout the entire city.\fnote{\fbackref{6:6} Lit. \fbib{through her}} \\
\poeml \v{7}As a well keeps its waters fresh,\fnote{\fbackref{6:7} Or \fbib{cool}} \\
\poemll    so the city\fnote{\fbackref{6:7} Lit. \fbib{she}} keeps her wickedness fresh.\fnote{\fbackref{6:7} Or \fbib{cool}} \\
\poeml Violence and destruction are heard in her, \\
\poemll    sickness and wounds are always before me. \\
\poeml \v{8}Be warned, Jerusalem, \\
\poemll    or I'll be alienated from you. \\
\poeml I'll make you desolate, \\
\poemll    a land not inhabited.''
\end{poetry}

\v{9}This is what the \divine{Lord} of the Heavenly Armies says:

\begin{poetry}
\poeml ``Let them glean the remnant of Israel \\
\poemll    as thoroughly as they would the vine. \\
\poeml Pass your hand over them like grape gatherers \\
\poemll    over the branches. \\
\poeml \v{10}To whom will I speak and give a warning \\
\poemll    so they'll listen? \\
\poeml Look, their ears are closed,\fnote{\fbackref{6:10} Lit. \fbib{uncircumcised}} \\
\poemll    and they cannot hear. \\
\poeml Look, this message from the \divine{Lord} is contemptible to them; \\
\poemll    they don't delight in it. \\
\poeml \v{11}I'm full of the wrath of the \divine{Lord}, \\
\poemll    and I'm tired of holding it back. \\
\poeml Pour it out on the children in the street \\
\poemll    and on the groups of young men gathered together. \\
\poeml Indeed, both husband and wife will be caught in it, \\
\poemll    the old and the very old. \\
\poeml \v{12}Their houses will be turned over to others--- \\
\poemll    their fields and wives together--- \\
\poeml when I stretch out my hand against \\
\poemll    those who live in the land,'' \\
\poemlll       declares the \divine{Lord}. \\
\poeml \v{13}``Indeed, from the least important to the most important, \\
\poemll    they're all greedy for dishonest gain. \\
\poeml From prophet to priest, \\
\poemll    they all act deceitfully. \\
\poeml \v{14}They treated my people's wound superficially, telling them, \\
\poemll    `Peace, peace,' but there is no peace. \\
\poeml \v{15}Were they ashamed because they did \\
\poemll    what was repugnant to God?\fnote{\fbackref{6:15} Lit. \fbib{committed an abomination}} \\
\poeml They were not ashamed at all--- \\
\poemll    they don't even know how to blush! \\
\poeml Therefore they'll fall with those who fall. \\
\poemll    When I punish them, they'll be brought down,'' \\
\poemlll       says the \divine{Lord}.
\end{poetry}
\passage{Israel Refuses to Repent}

\v{16}This is what the \divine{Lord} says:

\begin{poetry}
\poeml ``Stand beside the roads and watch. \\
\poemll    Ask for the ancient paths, where the good way is. \\
\poeml Walk in it and find rest for yourselves. \\
\poemll    But they said, `We won't walk in it!'\fnote{\fbackref{6:16} The Heb. lacks \fbib{in it}} \\
\poeml \v{17}I appointed watchmen over you. \\
\poemll    Listen for the sound of the trumpet. \\
\poeml But they said, `We won't listen!' \\
\poeml \v{18}Therefore, hear, nations, \\
\poemll    and know, congregation, \\
\poemlll       what will happen to them.\fnote{\fbackref{6:18} Or \fbib{what is among them}} \\
\poeml \v{19}Listen, earth! \\
\poemll    I'm about to bring calamity on this people, \\
\poemlll       on the fruit of their plans, \\
\poeml because they didn't listen to my words \\
\poemll    and they rejected my instruction.\fnote{\fbackref{6:19} Or \fbib{Law}} \\
\poeml \v{20}What good is frankincense \\
\poemll    that comes from Sheba\fnote{\fbackref{6:20} I.e. Yemen} to me, \\
\poemlll       or sweet cane from a distant country? \\
\poeml Your burnt offerings aren't acceptable, \\
\poemll    nor are your sacrifices pleasing to me.'' \\
\poeml \v{21}Therefore, this is what the \divine{Lord} says:
\end{poetry}

\begin{poetry}
\poeml ``I'm about to put stumbling blocks in front of this people, \\
\poemll    and fathers and sons will stumble over them together. \\
\poemlll       The neighbor and his friends will perish.''
\end{poetry}
\passage{The Invaders from the North}

\begin{poetry}
\poeml \v{22}This is what the \divine{Lord} says:
\end{poetry}

\begin{poetry}
\poeml ``Look, people are coming from a northern country. \\
\poemll    A great nation is stirring from the ends of the earth. \\
\poeml \v{23}They grab bow and spear; \\
\poemll    they're cruel and show no mercy. \\
\poeml Their sound roars like the sea \\
\poemll    as they ride on horses, \\
\poeml deployed like men ready for battle \\
\poemll    against you, daughter of Zion.'' \\
\poeml \v{24}We have heard the news about it, \\
\poemll    and our hands are limp. \\
\poeml Distress has seized us \\
\poemll    like a woman in labor. \\
\poeml \v{25}Don't go out into the field, \\
\poemll    and don't travel on the road, \\
\poeml because the enemy has a sword, \\
\poemll    and terror is on every side. \\
\poeml \v{26}Daughter of my people, put on sackcloth \\
\poemll    and roll in ashes. \\
\poeml Mourn with bitter wailing, \\
\poemll    as one mourns at the death of\fnote{\fbackref{6:26} The Heb. lacks \fbib{at the death of}} an only son. \\
\poeml For the destroyer will come on us suddenly.
\passage{People Rejected by the \divine{Lord}}
\poeml \v{27}``I've made you an assayer\fnote{\fbackref{6:27} I.e. one who tests metal for purity} of my people, \\
\poemll    as well as\fnote{\fbackref{6:27} The Heb. lacks \fbib{as well as} (cf. Jer 1:18)} a fortress. \\
\poeml You know how \\
\poemll    to test their way.'' \\
\poeml \v{28}All of them are very rebellious, \\
\poemll    going around as slanderers. \\
\poeml They're bronze and iron, \\
\poemll    and all of them are corrupt. \\
\poeml \v{29}The bellows blow fiercely to consume \\
\poemll    the lead with the fire. \\
\poeml The assayer\fnote{\fbackref{6:29} Lit. \fbib{He}} keeps on refining, \\
\poemll    but the impurities\fnote{\fbackref{6:29} Or \fbib{wicked}} aren't separated out. \\
\poeml \v{30}They're called reject silver, \\
\poemll    because the \divine{Lord} has rejected them.
\end{poetry}
\labelchapt{7}
\passage{Jeremiah's Temple Sermon: Judah's Idolatry}

\chapt{7}
\v{1}The message that came to Jeremiah from the \divine{Lord}: \v{2}``Stand at the gate of the \divine{Lord}'s Temple and proclaim this message there. Say, `Listen to this message from the \divine{Lord}, all you people of Judah who come through these gates to worship the \divine{Lord}.'\,''

\v{3}This is what the \divine{Lord} of the Heavenly Armies, the God of Israel, says:

\begin{poetry}
\poeml ``Change\fnote{\fbackref{7:3} Lit. \fbib{Make good}} your ways and your deeds, and I'll let you live in this place. \v{4}Don't trust deceptive words like these, and say, `The Temple of the \divine{Lord}, the Temple of the \divine{Lord}, the Temple of the \divine{Lord},' \v{5}but rather, truly change\fnote{\fbackref{7:5} Lit. \fbib{make good}} your ways and your deeds. If you truly practice justice between each person and his neighbor, \v{6}and if you don't oppress the alien, the orphan, and the widow, and don't shed an innocent person's blood in this place, and if you don't follow other gods to your own harm,\fnote{\fbackref{7:6} Or \fbib{disaster}} \v{7}then I'll let you dwell in this land, the land that I gave to your ancestors forever and ever. \\
\poeml \v{8}``Look, you're trusting in deceptive words that cannot benefit.\fnote{\fbackref{7:8} Or \fbib{profit}} \v{9}Will you steal, murder, commit adultery, swear by false gods, burn incense to Baal, follow other gods that you don't know, \v{10}and then come to stand before me in this house that is called by my name and say, `We're delivered' so we can continue to do all these things that are repugnant to God?\fnote{\fbackref{7:10} Lit. \fbib{all these abominations}} \v{11}Has this house that is called by my name become a hideout\fnote{\fbackref{7:11} Lit. \fbib{cave}} for bandits in your eyes? Look, I'm watching,'' declares the \divine{Lord}. \\
\poeml \v{12}``Go to my place that was in Shiloh, where I first caused my name to dwell. See what I did to it because of the evil of my people Israel. \v{13}Now, because you have done all these things,'' declares the \divine{Lord}, ``I spoke to you over and over again,\fnote{\fbackref{7:13} Lit. \fbib{getting up early and speaking}} but you didn't listen. I called to you, but you didn't answer. \v{14}Just as I did to Shiloh, I'll do to the house in which you trust and which is called by my name, the place that I gave to you and your ancestors. \v{15}I'll cast you out of my sight, just as I cast out all your brothers, all the descendants of Ephraim. \\
\poeml \v{16}``As for you, don't pray on behalf of this people, don't cry or offer a petition for them, and don't plead with me, for I won't listen to you. \v{17}Don't you see what they're doing in the cities of Judah and in the streets of Jerusalem? \v{18}The children gather wood, the fathers kindle the fire, and the women knead dough to make cakes for the Queen of Heaven,\fnote{\fbackref{7:18} I.e. the Near Eastern fertility goddess Ishtar} and they pour out liquid offerings to other gods in order to provoke me. \v{19}Are they provoking me?'' asks the \divine{Lord}. ``Is it not themselves, and to their own shame?'' \v{20}Therefore, this is what the Lord \divine{God} says: ``I'm about to pour out my anger and my wrath on this place, on people and animals, on the trees of the field, and on the fruit of the ground. It will burn, and it won't be put out.''
\end{poetry}

\v{21}This is what the \divine{Lord} of the Heavenly Armies, the God of Israel, says:

\begin{poetry}
\poeml ``Add your burnt offerings to your sacrifices and eat the meat. \v{22}Indeed, when I brought your ancestors out of the land of Egypt, I didn't speak or command them about burnt offering and sacrifice, \v{23}but I did give them this command:\fnote{\fbackref{7:23} Lit. \fbib{I commanded them this word}} `Obey me and I'll be your God, and you will be my people. Walk in all the ways that I command you so it will go well for you.' \v{24}But they didn't listen,\fnote{\fbackref{7:24} Or \fbib{obey}} nor did they pay attention.\fnote{\fbackref{7:24} Lit. \fbib{incline their ears}} They pursued their own plans,\fnote{\fbackref{7:24} Lit. \fbib{They walked in plans}} stubbornly following their own evil desires.\fnote{\fbackref{7:24} Lit. \fbib{following the stubbornness of their evil hearts}} They went backward and not forward. \v{25}From the day your ancestors left the land of Egypt to this present time, I've sent all my servants, the prophets, to you, again and again.\fnote{\fbackref{7:25} Lit. \fbib{daily getting up early and sending them}} \v{26}But they didn't listen to me, and they didn't pay attention.\fnote{\fbackref{7:26} Lit. \fbib{incline their ears}} They stiffened their necks, and they did more evil than their ancestors. \\
\poeml \v{27}``You will tell them all these things, but they won't listen to you. You will call out to them, but they won't answer you. \v{28}You will say to them, `This is the nation that wouldn't listen to the voice\fnote{\fbackref{7:28} I.e. wouldn't obey} of the \divine{Lord} its God and wouldn't accept correction. Truth has perished; it has been eliminated from their discussions.' \\
\poeml \v{29}``Cut off your hair and throw it away; \\
\poemll    let your lamentations rise on the barren heights, \\
\poeml because the \divine{Lord} has rejected and abandoned \\
\poemll    the generation that is subject to his wrath.\fnote{\fbackref{7:29} Lit. \fbib{generation of his wrath}}
\end{poetry}

\begin{poetry}
\poeml \v{30}``For the people of Judah have done evil in my eyes,'' declares the \divine{Lord}. ``They have put their detestable idols\fnote{\fbackref{7:30} Lit. \fbib{their detestable things}} in the house that is called by my name in order to defile it. \v{31}They have built high places at Topheth in the Valley of Ben-hinnom to burn their sons and daughters in the fire. I didn't command this, and it never entered my mind!
\end{poetry}

\v{32}``Therefore, the time is near,'' declares the \divine{Lord}, ``when it will no longer be called Topheth or the Valley of Ben-hinnom, but the Valley of Slaughter. They'll bury in Topheth because there is no other\fnote{\fbackref{7:32} The Heb. lacks \fbib{other}} place to do it.\fnote{\fbackref{7:32} The Heb. lacks \fbib{to do it}} \v{33}The dead bodies of these people will be food for the birds of the sky and for the animals of the land, and no one will disturb them. \v{34}In the towns of Judah and the streets of Jerusalem I'll bring an end to the sound of gladness and rejoicing, to the sounds of the bridegroom and bride, for the land will become a wasteland.''
\labelchapt{8}

\chapt{8}
\v{1}``At that time,'' declares the \divine{Lord}, ``the bones of the king of Judah, the bones of his officials, the bones of the priests, the bones of the prophets, and the bones of the residents of Jerusalem will be removed from their graves. \v{2}They'll be spread out to the sun, the moon, and all the stars of the heavens, which they loved and served,\fnote{\fbackref{8:2} Or \fbib{worshipped}} and which they followed, consulted, and worshipped. Their bones\fnote{\fbackref{8:2} Lit. \fbib{They}} won't be collected, nor will they be buried. They'll be like dung on the surface of the ground.

\v{3}``In all the places where the people\fnote{\fbackref{8:3} Lit. \fbib{they}} remain, where I've banished them, death will be chosen over life by all the remnant that remains of this evil family,'' declares the \divine{Lord} of the Heavenly Armies.
\passage{A Stubborn People}

\v{4}``You are to say to them, `This is what the \divine{Lord} says:

\begin{poetry}
\poeml ``Will a person fall down and then not get up? \\
\poemll    Will someone turn away\fnote{\fbackref{8:4} Or \fbib{repent}} and then not turn back again?\fnote{\fbackref{8:4} Or \fbib{not repent}} \\
\poeml \v{5}Why has this people turned away?\fnote{\fbackref{8:5} Lit. \fbib{people committed apostasy}} \\
\poemll    Why does Jerusalem continue in apostasy? \\
\poemlll       They hold on to deceit and refuse to repent. \\
\poeml \v{6}I've listened and I've heard, \\
\poemll    and what they say is not right. \\
\poeml No one repents of his evil and says, \\
\poemll    `What have I done?' \\
\poeml ``They all turn to their own course \\
\poemll    like a horse racing into battle. \\
\poeml \v{7}Even the stork in the sky knows its seasons, \\
\poemll    and the dove, the swallow, and the crane observe the time for migration. \\
\poeml But my people don't know \\
\poemll    the requirements\fnote{\fbackref{8:7} I.e. the behavior God expects of his people} of the \divine{Lord}. \\
\poeml \v{8}How can you say, `We're wise, \\
\poemll    and the Law of the \divine{Lord} is with us,' \\
\poeml when, in fact, the deceitful pen of the scribe has made it \\
\poemll    into something that deceives. \\
\poeml \v{9}The wise men will be put to shame. \\
\poemll    They'll be dismayed and taken captive. \\
\poeml Look, they have rejected the message from the \divine{Lord}! \\
\poemll    So what kind of wisdom do they have? \\
\poeml \v{10}Therefore, I'll give their wives to others, \\
\poemll    and their fields to new owners. \\
\poeml Indeed, from the least important to the most important, \\
\poemll    they're all greedy for dishonest gain. \\
\poeml From prophet to priest, \\
\poemll    they all act deceitfully. \\
\poeml \v{11}They have treated my people's\fnote{\fbackref{8:11} Lit. \fbib{of the daughter of my people}} wound \\
\poemll    superficially, telling them, `Peace, peace,' \\
\poemlll       when there is no peace. \\
\poeml \v{12}Are they ashamed because they have done \\
\poemll    what is repugnant to God?\fnote{\fbackref{8:12} Lit. \fbib{committed an abomination}} \\
\poeml They weren't ashamed at all; \\
\poemll    they don't even know how to blush! \\
\poeml Therefore they'll fall with those who fall. \\
\poemll    When I punish them, they'll be brought down,'' \\
\poemlll       says the \divine{Lord}. \\
\poeml \v{13}``I would have gathered them,'' \\
\poemll    declares the \divine{Lord}, \\
\poeml ``but there were no grapes on the vine, \\
\poemll    and no figs on the fig tree, \\
\poeml and their leaves were withered. \\
\poemll    What I've given them has been taken away.''\,'\,''
\passage{The People Respond}
\poeml \v{14}Why are we sitting here? \\
\poemll    Join together! Let's go to the fortified cities \\
\poemlll       and perish there! \\
\poeml For the \divine{Lord} our God has condemned us to perish \\
\poemll    and given us poisoned water to drink, \\
\poemlll       because we have sinned against him.\fnote{\fbackref{8:14} Lit. \fbib{the \divine{Lord}}} \\
\poeml \v{15}We waited for peace, but no good has come, \\
\poemll    for a time of healing, but instead there was terror.
\passage{The \divine{Lord}'s Warning}
\poeml \v{16}``The snorting of their horses is heard from Dan. \\
\poemll    At the neighing of their stallions, \\
\poemlll       the whole earth quakes. \\
\poeml They're coming to devour \\
\poemll    the land and all it contains, \\
\poemlll       the city and all who live in it. \\
\poeml \v{17}Look, I'll send snakes among you, \\
\poemll    vipers that cannot be charmed, \\
\poemlll       and they'll bite you.''
\passage{Jeremiah Mourns for His People}
\poeml \v{18}Incurable sorrow has overwhelmed me, \\
\poemll    my heart is sick within me. \\
\poeml \v{19}Listen! My people\fnote{\fbackref{8:19} Lit. \fbib{the daughter of my people}} cry \\
\poemll    from a distant land: \\
\poeml ``Is the \divine{Lord} no longer in Zion? \\
\poemll    Is her king no longer there?''
\passage{The \divine{Lord} Speaks}
\poeml ``Why did they provoke me to anger with their images, \\
\poemll    with their worthless foreign gods?''
\passage{The People Speak}
\poeml \v{20}The harvest is past, \\
\poemll    the summer has ended, \\
\poemlll       and we haven't been delivered.
\passage{The Prophet Mourns}
\poeml \v{21}Because my people\fnote{\fbackref{8:21} Lit. \fbib{the daughter of my people}} are crushed, I'm crushed. \\
\poemll    I mourn, and dismay overwhelms me. \\
\poeml \v{22}Is there no balm in Gilead? \\
\poemll    Is there no physician there? \\
\poemlll       So why is there no healing for my people?\fnote{\fbackref{8:22} Lit. \fbib{daughter of my people}}
\end{poetry}
\labelchapt{9}
\passage{The \divine{Lord}'s Sorrow for His People}

\begin{poetry}
\poeml \chapt{9}
\v{1}\fnote{\fbackref{9:1} Because this verse is 8:23 in MT, there is a one verse discrepancy between MT and the ISV throughout this chapter.}``Oh, that my head were a spring of water,\fnote{\fbackref{9:1} Lit. \fbib{were waters}} \\
\poeml and my eyes a fountain of tears, \\
\poeml for then I would cry day and night for those \\
\poemll    of my people\fnote{\fbackref{9:1} Lit. \fbib{for the daughter of my people}} who have been killed. \\
\poeml \v{2}Oh, that I had a lodging place for travelers in the desert, \\
\poemll    so that I could leave my people \\
\poemlll       and go away from them. \\
\poeml For all of them are adulterers, \\
\poemll    a band of traitors. \\
\poeml \v{3}They use their tongues like a bow. \\
\poemll    Lies rather than truth fly throughout\fnote{\fbackref{9:3} Lit. \fbib{prevail in}} the land. \\
\poeml They progress from one evil to another, \\
\poemll    and they don't know me,'' \\
\poemlll       declares the \divine{Lord}. \\
\poeml \v{4}``Beware of your neighbors, and don't trust \\
\poemll    any of your relatives. \\
\poeml For all of your relatives act deceitfully, \\
\poemll    and every friend goes around as a slanderer. \\
\poeml \v{5}People deceive their friends, \\
\poemll    and they don't tell the truth. \\
\poeml They have taught their tongues to tell lies. \\
\poemll    They exhaust themselves practicing evil.\fnote{\fbackref{9:5} Or \fbib{themselves with iniquity}} \\
\poeml \v{6}You yourself live in the midst of deception, \\
\poemll    and because they are deceived they do not know me,'' \\
\poemlll       declares the \divine{Lord}. \\
\poeml \v{7}Therefore, this is what the \divine{Lord} of the Heavenly Armies says:
\end{poetry}

\begin{poetry}
\poeml ``Look, I'm about to refine and test them. \\
\poemll    Because they're my people, what else can I do?\fnote{\fbackref{9:7} Lit. \fbib{because of the daughter of my people}} \\
\poeml \v{8}Their tongue is a deadly arrow \\
\poemll    that speaks deceit. \\
\poeml With his mouth a person says, `Peace,' to his friend, \\
\poemll    but inwardly he sets a trap for him. \\
\poeml \v{9}Should I not punish them for these things?''\fnote{\fbackref{9:9} Or \fbib{punish these people}} \\
\poemll    asks the \divine{Lord}, \\
\poemlll       ``and should I not avenge myself on a nation like this?'' \\
\poeml \v{10}I'll weep and mourn for the mountains, \\
\poemll    and lament for the desert pastures, \\
\poeml because they are desolate and no one passes through them. \\
\poemll    They don't hear the lowing of the cattle. \\
\poeml Both the birds of the sky and the animals have fled. \\
\poemll    They're gone! \\
\poeml \v{11}``I'll make Jerusalem a heap of ruins, \\
\poemll    a refuge for jackals. \\
\poeml I'll make the towns of Judah desolate, \\
\poemll    without inhabitants.''
\end{poetry}
\passage{The Reason for Judgment}

\v{12}Who is the wise person who understands this, and to whom has the \divine{Lord}\fnote{\fbackref{9:12} Lit. \fbib{the mouth of the \divine{Lord}}} spoken so that he may declare it? Why is the land destroyed, ruined like the desert, without anyone passing through it? \v{13}The \divine{Lord} said, ``It is because they have forsaken my Law that I gave them. They didn't obey me and didn't live according to it. \v{14}Instead, they followed their rebellious hearts and the Baals,\fnote{\fbackref{9:14} I.e. images of the Canaanite storm god} as their ancestors taught them.''

\v{15}Therefore, this is what the \divine{Lord} of the Heavenly Armies, the God of Israel, says: ``Look, I'll make these people eat wormwood\fnote{\fbackref{9:15} I.e. a bitter plant} and drink poisoned water. \v{16}I'll scatter them among nations that neither they nor their ancestors have known, and I'll pursue them with the sword until I've finished them off.''
\passage{A Call to Lament}

\v{17}This is what the \divine{Lord} of the Heavenly Armies says:

\begin{poetry}
\poeml ``Think about what I'm saying!\fnote{\fbackref{9:17} The Heb. lacks \fbib{about what I'm saying}} \\
\poemll    Indeed, call out the professional mourners!\fnote{\fbackref{9:17} Lit. \fbib{lamenting women}} \\
\poemlll       Send for the best of them to come. \\
\poeml \v{18}Let them hurry and lament for us. \\
\poemll    Let tears run down from our eyes, \\
\poemlll       and let our eyelids flow with water. \\
\poeml \v{19}For a sound of mourning is heard from Zion: \\
\poemll    `How we're ruined! \\
\poeml Our shame is very great, \\
\poemll    because we have left the land, \\
\poemlll       because our houses are torn down.'\,'' \\
\poeml \v{20}``Now, you women, hear the message from the \divine{Lord}; \\
\poemll    listen to what he has to say! \\
\poeml Teach your daughters how to mourn, \\
\poemll    let every woman teach\fnote{\fbackref{9:20} The Heb. lacks \fbib{teach}} her friend how to lament. \\
\poeml \v{21}For death comes up through our windows; \\
\poemll    it has come into our palaces \\
\poeml to eliminate children from the streets \\
\poemll    and young men from the town squares. \\
\poeml \v{22}Speak! `This is what the \divine{Lord} says: \\
\poeml ``The corpses of people will fall like dung \\
\poemll    on the surface of the field, \\
\poeml and like a row of cut grain behind \\
\poemll    the harvester when there is no one to gather it.''\,'\,''
\end{poetry}
\passage{True Wisdom and the Coming Judgment}

\v{23}This is what the \divine{Lord} says: ``The wise man is not to boast in his wisdom; the strong man is not to boast in his strength; and the rich man is not to boast in his riches. \v{24}Rather, let the one who boasts, boast in this: that he understands and knows me, for I am the \divine{Lord} who acts with gracious love, justice, and righteousness in the land. I delight in these things,'' declares the \divine{Lord}.

\v{25}``Look, days are coming,'' declares the \divine{Lord}, ``when I'll punish all who are circumcised only in the flesh:\fnote{\fbackref{9:25} Lit. \fbib{circumcised of foreskin}} \v{26}Egypt, Judah, Edom, the people of Ammon, Moab, all those who live in the desert and shave the corners of their beard;\fnote{\fbackref{9:26} Lit. \fbib{cut off of side}} indeed all the other\fnote{\fbackref{9:26} The Heb. lacks \fbib{other}} nations that are uncircumcised, and all the house of Israel that is uncircumcised of heart.''
\labelchapt{10}
\passage{The True God and Worthless Idols}

\chapt{10}
\v{1}Hear the message that the \divine{Lord} has spoken to you, house of Israel. \v{2}This is what the \divine{Lord} says:

\begin{poetry}
\poeml ``Don't learn the way of the nations, \\
\poemll    and don't be terrified by signs in the heavens, \\
\poemll    though the nations are terrified of them. \\
\poeml \v{3}For the practices\fnote{\fbackref{10:3} Or \fbib{customs, ordinances}} of the people are worthless. \\
\poemll    Indeed, a tree is cut down from the forest; \\
\poemlll       it's the work of the hands of a craftsman\fnote{\fbackref{10:3} Or \fbib{engraver}; i.e. a wood carver} with an ax. \\
\poeml \v{4}They decorate it with silver and gold. \\
\poemll    They secure it with nails and hammers \\
\poemlll       so it won't totter. \\
\poeml \v{5}Their idols\fnote{\fbackref{10:5} Lit. \fbib{They}} are like scarecrows in a cucumber field. \\
\poemll    They can't speak! \\
\poeml They must always be carried \\
\poemll    because they can't walk! \\
\poeml Don't be afraid of them \\
\poemll    because they can do no harm, \\
\poemlll       nor can they do any good.'' \\
\poeml \v{6}There is no one like you, \divine{Lord}. \\
\poemll    You are great, and your name is great and powerful. \\
\poeml \v{7}Who wouldn't fear you, king of the nations? \\
\poemll    This is what you deserve! \\
\poeml Indeed, among all the wise men of the nations, \\
\poemll    and throughout all their kingdoms, \\
\poemlll       there is no one like you! \\
\poeml \v{8}Everyone is stupid\fnote{\fbackref{10:8} I.e. like a beast} and senseless. \\
\poemll    They follow worthless instruction \\
\poemlll       from a piece of wood!\fnote{\fbackref{10:8} Lit. \fbib{it is worthless instruction from wood}} \\
\poeml \v{9}Beaten silver is brought from Tarshish, \\
\poemll    and gold from Uphaz. \\
\poeml The idols are\fnote{\fbackref{10:9} Lit. \fbib{It is}} the work of a craftsman\fnote{\fbackref{10:9} Or \fbib{engraver}; i.e. a wood carver} \\
\poemll    and of the hands of a goldsmith. \\
\poeml Their clothing is violet and purple. \\
\poemll    The idols\fnote{\fbackref{10:9} Lit. \fbib{They}} are all the work of skilled craftsmen. \\
\poeml \v{10}The \divine{Lord} is the true God; \\
\poemll    he's the living God and the everlasting king. \\
\poeml At his wrath the earth quakes, \\
\poemll    and the nations cannot endure his indignation.
\end{poetry}

\v{11}\fnote{\fbackref{10:11} This verse is in Aramaic, the language the exiles would speak in Babylon.}Tell this to them: ``The gods who\fnote{\fbackref{10:11} Lit. \fbib{the one who}} didn't make the heavens and the earth will perish from the earth and from these heavens.''
\passage{A Hymn of Praise to God}

\begin{poetry}
\poeml \v{12}The \divine{Lord} is\fnote{\fbackref{10:12} The Heb. lacks \fbib{The \divine{Lord} is}} the one who made \\
\poemll    the world by his power, \\
\poeml who established the earth by his wisdom \\
\poemll    and stretched out the heavens by his understanding. \\
\poeml \v{13}When his voice sounds there is thunder \\
\poemll    from the waters of heaven, \\
\poeml and he makes clouds rise up from \\
\poemll    the ends of the earth. \\
\poeml He makes lightning for the rain \\
\poemll    and brings wind out of his storehouses. \\
\poeml \v{14}Everyone is stupid\fnote{\fbackref{10:14} I.e. like a beast} and without knowledge. \\
\poemll    Every goldsmith is put to shame by his idols, \\
\poemlll       for his images are false.\fnote{\fbackref{10:14} Lit. \fbib{deception}} \\
\poeml There is no life\fnote{\fbackref{10:14} Or \fbib{breath}} in them. \\
\poeml \v{15}They're worthless, a work of mockery, \\
\poemll    and when the time of punishment comes,\fnote{\fbackref{10:15} Lit. \fbib{at the time of their punishment}} \\
\poemlll       they'll perish. \\
\poeml \v{16}The Portion of Jacob\fnote{\fbackref{10:16} I.e. \fbib{The Portion of Jacob} is a name for the \fbib{\divine{Lord}}} is not like these. \\
\poemll    He made everything, \\
\poeml and Israel is the tribe of his inheritance. \\
\poemll    The \divine{Lord} of the Heavenly Armies is his name.
\passage{The Coming Captivity of Judah}
\poeml \v{17}You who live under siege, \\
\poemll    Gather up your bundle\fnote{\fbackref{10:17} I.e. your possessions} from the ground.\fnote{\fbackref{10:17} Or \fbib{land}} \\
\poeml \v{18}For this is what the \divine{Lord} says: \\
\poeml ``I'm going to throw out the inhabitants \\
\poemll    of the land at this time, \\
\poeml and I'll bring distress on them \\
\poemll    so they'll experience\fnote{\fbackref{10:18} Lit. \fbib{find}} it.'' \\
\poeml \v{19}Woe is me because of my injury. \\
\poemll    My wound is severe. \\
\poeml I said, ``Truly this is my sickness, \\
\poemll    and I must bear it. \\
\poeml \v{20}My tent is destroyed, \\
\poemll    and all my tent cords are broken. \\
\poeml My sons have gone away from me, \\
\poemll    they no longer live. \\
\poeml There is no one to pitch my tent again \\
\poemll    and set up my curtains. \\
\poeml \v{21}Because the shepherds are stupid\fnote{\fbackref{10:21} I.e. like a beast} \\
\poemll    and don't seek\fnote{\fbackref{10:21} Or \fbib{inquire of}} the \divine{Lord}, \\
\poeml therefore, they don't prosper, \\
\poemll    and their flock is scattered. \\
\poeml \v{22}The sound of a report, it's coming now! \\
\poemll    There is a great commotion from a land in the north \\
\poeml to make the towns of Judah desolate, \\
\poemll    a refuge for jackals.''
\passage{Jeremiah's Prayer}
\poeml \v{23}\divine{Lord}, I know that a person's life is not his to control,\fnote{\fbackref{10:23} Or \fbib{does not belong to him}} \\
\poemll    nor does a person establish his way in life.\fnote{\fbackref{10:23} Or \fbib{step as he walks}} \\
\poeml \v{24}\divine{Lord}, correct me, but with justice, \\
\poemll    not with anger. \\
\poemlll       Otherwise, you'll bring me to nothing. \\
\poeml \v{25}Pour out your anger on the nations \\
\poemll    that don't acknowledge you, \\
\poemlll       and on the families that don't call on your name. \\
\poeml For they have devoured Jacob; \\
\poemll    they have devoured and consumed him; \\
\poemlll       they have devastated his habitation.
\end{poetry}
\labelchapt{11}
\passage{The Broken Covenant}

\chapt{11}
\v{1}This is the message that came to Jeremiah from the \divine{Lord}: \v{2}``Listen to the words of this covenant, and convey them to the people of Judah and the residents of Jerusalem. \v{3}You are to say to them, `This is what the \divine{Lord} God of Israel says: ``Cursed is the person who does not listen to the words of this covenant \v{4}which I commanded to your ancestors on the day I brought them out of the land of Egypt, from the iron furnace. I said, `Obey me and do everything\fnote{\fbackref{11:4} Lit. \fbib{according to all}} that I commanded you. Then you will be my people and I'll be your God.' \v{5}As a result, I'll fulfill the oath that I made with your ancestors to give them a land flowing with milk and honey, just as is the case today.''\,'\,''

Then I answered, ``So be it,\fnote{\fbackref{11:5} Heb. \fbib{amen}} \divine{Lord}.''

\v{6}The \divine{Lord} told me, ``Proclaim all these words in the towns of Judah and in the streets of Jerusalem. You are to say, `Listen to the words of this covenant and do them. \v{7}For I've diligently warned your ancestors from the day I brought them out of the land of Egypt until now, regularly warning them,\fnote{\fbackref{11:7} Lit. \fbib{getting up early and warning}} saying, ``Obey me!'' \v{8}But they would not listen or turn their ear, and each of them stubbornly followed his own evil desires.\fnote{\fbackref{11:8} Lit. \fbib{follow the stubbornness of their evil hearts}} So I brought on them all the consequences\fnote{\fbackref{11:8} Lit. \fbib{words}} of this covenant that I commanded them to fulfill, but they did not.'\,''

\v{9}The \divine{Lord} told me, ``Conspiracy has been found among the people of Judah and the residents of Jerusalem. \v{10}They have turned back to the iniquities of their ancestors of old\fnote{\fbackref{11:10} Lit. \fbib{their first ancestors}} who refused to listen to my words. They followed other gods to serve them. The house of Israel and the house of Judah broke my covenant which I made with their ancestors.''

\v{11}Therefore, this is what the \divine{Lord} says: ``I'm about to bring disaster on them from which they won't be able to escape. They'll cry out to me, but I won't listen to them. \v{12}The towns of Judah and the residents of Jerusalem will go and cry out to the gods to whom they burn incense, but they'll be no help at all to them\fnote{\fbackref{11:12} Or \fbib{won't save them at all}} in the time of their disaster. \v{13}Judah, you have as many gods as you have towns, and you have set up as many altars to the shameful idols as there are streets in Jerusalem. You burn incense to Baal on these altars.

\v{14}``Jeremiah,\fnote{\fbackref{11:14} Lit. \fbib{You}} don't pray for this people and don't cry or pray for them. I won't listen when they cry out to me because of their disaster.

\begin{poetry}
\poeml \v{15}``What right does my beloved have in my house, \\
\poemll    when she has carried out many evil schemes? \\
\poeml Can sacrificial\fnote{\fbackref{11:15} Lit. \fbib{holy}} flesh turn disaster away from you, \\
\poemll    so you can rejoice?'' \\
\poeml \v{16}The \divine{Lord} once called you a green olive tree, \\
\poemll    with beautiful shape and fruit. \\
\poeml With a great roaring sound, he has set fire to it \\
\poemll    and its branches will be destroyed.
\end{poetry}

\v{17}The \divine{Lord} of the Heavenly Armies who planted you has called for disaster on you because of the evil of the house of Israel and the house of Judah, has provoked me by burning incense to Baal.''
\passage{Jeremiah's Life is Threatened}

\begin{poetry}
\poeml \v{18}The \divine{Lord} made it known to me, \\
\poemll    and so I understood. \\
\poemlll       Then you showed me their malicious deeds. \\
\poeml \v{19}I was like a gentle lamb \\
\poemll    led to the slaughter. \\
\poeml I didn't know that they had devised schemes \\
\poemll    against me. They told themselves,\fnote{\fbackref{11:19} The Heb. lacks \fbib{They told themselves}} \\
\poeml ``Let's destroy the tree with its fruit. \\
\poemll    Let's eliminate him from the land of the living, \\
\poemlll       so his name won't be remembered again.'' \\
\poeml \v{20}\divine{Lord} of the Heavenly Armies, the righteous judge, \\
\poemll    the one who tests feelings and the heart, \\
\poeml let me see your vengeance on them, \\
\poemll    for I've committed my cause to you.
\end{poetry}

\v{21}Therefore, this is what the \divine{Lord} says about the men of Anathoth who seek to kill you, all the while threatening you, ``Don't prophesy in the name of the \divine{Lord} so you won't die by our hand!'' \v{22}Therefore, this is what the \divine{Lord} of the Heavenly Armies says: ``I'm about to punish them. The young men will die by the sword. Their sons and daughters will die by famine. \v{23}Not one of them will be left,\fnote{\fbackref{11:23} Lit. \fbib{A remnant won't be to them}} for I'll bring disaster on the men of Anathoth when I punish them.''
\labelchapt{12}
\passage{Jeremiah's Complaint about Justice}

\begin{poetry}
\poeml \chapt{12}
\v{1}You are righteous, \divine{Lord}, \\
\poeml even when I bring a complaint to you. \\
\poeml But I want to discuss justice with you. \\
\poemll    Why does the way of the wicked prosper, \\
\poemlll       while all who are treacherous are at ease? \\
\poeml \v{2}You plant them and they take root, \\
\poemll    they grow and bear fruit. \\
\poeml ``You are near to us,'' they say with their mouths, \\
\poemll    but the truth is that you're far from their hearts. \\
\poeml \v{3}You know me, \divine{Lord}. \\
\poemll    You see me and test my thoughts\fnote{\fbackref{12:3} Or \fbib{heart}} toward you. \\
\poeml Pull the wicked\fnote{\fbackref{12:3} Lit. \fbib{them}} out like sheep for slaughter; \\
\poemll    set them apart for the day of butchering.\fnote{\fbackref{12:3} Or \fbib{killing}} \\
\poeml \v{4}How long will the land mourn \\
\poemll    and the vegetation of every field dry up? \\
\poeml Because of the wickedness of those who live in it, \\
\poemll    animals and birds are swept away. \\
\poemlll       For they say, ``He does not see our future.''
\passage{God's Reply to Jeremiah}
\poeml \v{5}Indeed, if you run with others on foot, \\
\poemll    and they tire you out, \\
\poemlll       how can you compete with horses? \\
\poeml You are secure\fnote{\fbackref{12:5} Or \fbib{You trust}} in a land at peace, \\
\poemll    but how will you do in the thicket of the Jordan? \\
\poeml \v{6}Indeed, even your brothers and your father's family \\
\poemll    conspire against you. \\
\poeml Even they cry out after you loudly. \\
\poemll    Don't believe them, even though they speak friendly words to you. \\
\poeml \v{7}I'll forsake my house, \\
\poemll    I'll abandon my inheritance. \\
\poeml I'll give the beloved of my heart \\
\poemll    into the hand of her enemies. \\
\poeml \v{8}My inheritance has become like a lion in the forest to me. \\
\poemll    She roars at me; therefore, I hate her. \\
\poeml \v{9}Is my inheritance like a speckled bird of prey to me? \\
\poemll    Are the other\fnote{\fbackref{12:9} The Heb. lacks \fbib{other}} birds of prey all around her coming against her? \\
\poeml Go, gather all the wild animals and \\
\poemll    bring them to devour it. \\
\poeml \v{10}Many shepherds will destroy my vineyard. \\
\poemll    They'll trample down my portion. \\
\poeml They'll turn my pleasant portion \\
\poemll    into a desolate desert. \\
\poeml \v{11}They'll make it into a desolate place, \\
\poemll    and, desolate, it will cry out in mourning to me. \\
\poeml The whole land will be desolate \\
\poemll    because no one takes it to heart. \\
\poeml \v{12}On all the barren heights in the desert \\
\poemll    destroyers will come. \\
\poeml Indeed, a sword of the \divine{Lord} will devour from \\
\poemll    one end of the land to the other. \\
\poeml There will be no peace\fnote{\fbackref{12:12} Or \fbib{safety}} for any person.\fnote{\fbackref{12:12} Lit. \fbib{for all flesh}} \\
\poeml \v{13}They have sown wheat, \\
\poemll    but they have harvested thorns. \\
\poeml They have tired themselves out, \\
\poemll    but they don't show a profit. \\
\poeml Now be disappointed about your harvest \\
\poemll    because of the fierce anger of the \divine{Lord}.
\end{poetry}
\passage{God's Word about Judah's Neighbors}

\v{14}This is what the \divine{Lord} says about all the wicked neighbors who strike out against the land\fnote{\fbackref{12:14} Lit. \fbib{inheritance}} I've given to my people Israel as their inheritance:\fnote{\fbackref{12:14} Lit. \fbib{to inherit}} ``I'm about to uproot them from their land, and I'll uproot the house of Judah from among them. \v{15}After I've uprooted them, I'll again have compassion on them. I'll return each one of them to his inheritance, and each one to his own land. \v{16}If they have learned the ways of my people well, to swear by my name: `As surely as the \divine{Lord} lives,' just as they once taught my people to swear by Baal, then they'll be built up among my people. \v{17}But if they don't listen, then I'll completely uproot that nation and destroy it,'' declares the \divine{Lord}.
\labelchapt{13}
\passage{Jeremiah's Linen Belt}

\chapt{13}
\v{1}This is what the \divine{Lord} told me: ``Go and buy a linen belt for yourself, and put it around your waist.\fnote{\fbackref{13:1} Or \fbib{loins}} But don't let it get wet.'' \v{2}So I bought the belt according to the \divine{Lord}'s instruction, and put it around my waist.

\v{3}Then this message from the \divine{Lord} came to me a second time: \v{4}Take the belt that you bought and that is around your waist. Get up and go to the Euphrates,\fnote{\fbackref{13:4} Or \fbib{Perath}} and hide it there in a crevice in the rock.'' \v{5}So I went and hid it at the Euphrates,\fnote{\fbackref{13:5} Or \fbib{Perath}} just as the \divine{Lord} had commanded me.

\v{6}After a long time,\fnote{\fbackref{13:6} Lit. \fbib{At the end of many days}} the \divine{Lord} told me, ``Arise, go to the Euphrates,\fnote{\fbackref{13:6} Or \fbib{Perath}} and get the belt that I commanded you to hide there.'' \v{7}I went to the Euphrates and dug it up. I got the belt from the place where I had hidden it. The belt was ruined! It was not good for anything.

\v{8}Then this message from the \divine{Lord} came to me: \v{9}``This is what the \divine{Lord} says: `In the same way I'll ruin the pride of Judah and the pride of Jerusalem. \v{10}This evil people that refuses to listen to my words, that stubbornly pursues their own desires,\fnote{\fbackref{13:10} Lit. \fbib{that walks in the stubbornness of their hearts}} and that follows other gods to serve and worship them, will be like this belt that is not good for anything. \v{11}For just as the belt clings tightly to a person's waist, so I've made all the people\fnote{\fbackref{13:11} Lit. \fbib{house}} of Israel and all the people\fnote{\fbackref{13:11} Lit. \fbib{house}} of Judah cling tightly to me,' declares the \divine{Lord}. `I did this\fnote{\fbackref{13:11} The Heb. lacks \fbib{I did this}} so that they would be my people, name, praise, and glory. But they wouldn't listen.'
\passage{The Wineskins}

\v{12}``This is what you're to tell them: `This is what the \divine{Lord} God of Israel says: ``Every wineskin is to be filled with wine.''\,' When they say to you, `Don't we know very well that every wineskin is to be filled with wine?', \v{13}then say to them, `This is what the \divine{Lord} says: ``I'm about to make all the inhabitants of this land drunk---the kings who sit on David's throne, the priests, the prophets, and all the residents of Jerusalem. \v{14}I'll smash them against each other, even fathers against their sons,''\fnote{\fbackref{13:14} Lit. \fbib{fathers and sons together}} declares the \divine{Lord}. ``I'll have no pity, mercy, or compassion when I destroy them.''\,'\,''

\begin{poetry}
\poeml \v{15}Listen and pay attention!\fnote{\fbackref{13:15} Lit. \fbib{give ear}} \\
\poemll    Don't be proud, for the \divine{Lord} has spoken. \\
\poeml \v{16}Give glory to the \divine{Lord} your God \\
\poemll    before he brings darkness, \\
\poeml before your feet stumble on the \\
\poemll    mountains at twilight. \\
\poeml You hope for light, \\
\poemll    but he turns it into deep darkness. \\
\poemlll       He changes it into heavy gloom. \\
\poeml \v{17}If you don't listen, I'll cry secretly \\
\poemll    because of your pride. \\
\poeml My eyes will cry bitterly, flowing tears, \\
\poemll    because the \divine{Lord}'s flock has been taken captive. \\
\poeml \v{18}Say to the king and the queen mother,\fnote{\fbackref{13:18} I.e. the king's mother} \\
\poemll    ``Come take a lowly seat, \\
\poeml because your beautiful crowns have fallen off your heads.'' \\
\poeml \v{19}The towns in the Negev\fnote{\fbackref{13:19} I.e. the southern regions of the Sinai peninsula; cf. Josh 10:40} will be closed up, \\
\poemll    and there will be no one to open them. \\
\poeml All Judah will be taken into exile \\
\poemll    and be completely exiled. \\
\poeml \v{20}``Look up and see those who are coming from the north. \\
\poemll    Where is the flock that was given to you--- \\
\poemlll       your beautiful sheep? \\
\poeml \v{21}What will you say when the \divine{Lord}\fnote{\fbackref{13:21} Lit. \fbib{he}} \\
\poemll    appoints over you as your head \\
\poemlll       those whom you taught to be your allies?\fnote{\fbackref{13:21} I.e. the Babylonians} \\
\poeml Pain will seize you like that seizing a woman \\
\poemll    about to give birth, will it not? \\
\poeml \v{22}When you say to yourselves, \\
\poemll    `Why have all these things happened to me?' \\
\poeml It's because of the extent of your iniquity \\
\poemll    that your skirt has been lifted up, \\
\poemlll       and your heels have suffered violence.\fnote{\fbackref{13:22} I.e. you have been violated} \\
\poeml \v{23}Can an Ethiopian change his skin, \\
\poemll    or a leopard his spots? \\
\poeml Then you who are trained to do evil \\
\poemll    will also be able to do good. \\
\poeml \v{24}I'll scatter them like chaff \\
\poemll    blown away by a desert wind. \\
\poeml \v{25}``This is your fate, \\
\poemll    the portion I've measured out for you,'' \\
\poemlll       declares the \divine{Lord}, \\
\poeml ``because you have forgotten me \\
\poemll    and have trusted in false gods.\fnote{\fbackref{13:25} Or \fbib{deception}} \\
\poeml \v{26}I'll also pull your skirt up over your face, \\
\poemll    so your shame will be seen, \\
\poeml \v{27}I've seen your detestable behavior: \\
\poemll    your adulteries, your passionate neighing, \\
\poemlll       your lewd immorality on the hills in the field. \\
\poeml How terrible it will be for you, Jerusalem! \\
\poemll    You are unclean. How much longer will this go on?''
\end{poetry}
\labelchapt{14}
\passage{A Terrible Drought in the Land}

\chapt{14}
\v{1}This is\fnote{\fbackref{14:1} The Heb. lacks \fbib{This is}} this message from the \divine{Lord} that came\fnote{\fbackref{14:1} The Heb. lacks \fbib{that came}} to Jeremiah concerning the drought:

\begin{poetry}
\poeml \v{2}``Judah mourns, and her gates languish. \\
\poemll    The people\fnote{\fbackref{14:2} Lit. \fbib{They}} mourn for the land, \\
\poemll    and the cry of Jerusalem goes up. \\
\poeml \v{3}Their nobles send their young people for water. \\
\poemll    They go to the cisterns, but they find no water. \\
\poeml They return with their vessels empty. \\
\poemll    They're disappointed\fnote{\fbackref{14:3} Or \fbib{ashamed}} and dismayed, \\
\poemlll       and they cover their heads in shame.\fnote{\fbackref{14:3} The Heb. lacks \fbib{in shame}} \\
\poeml \v{4}The ground is cracked, \\
\poemll    because there has been no rain in the land. \\
\poeml The farmers are disappointed,\fnote{\fbackref{14:4} Or \fbib{ashamed}} \\
\poemll    and they cover their heads in shame.\fnote{\fbackref{14:4} The Heb. lacks \fbib{in shame}} \\
\poeml \v{5}Even the doe in the field gives birth \\
\poemll    and then abandons her young\fnote{\fbackref{14:5} The Heb. lacks \fbib{her young}} \\
\poemlll       because there is no grass. \\
\poeml \v{6}Wild donkeys stand on the barren hills. \\
\poemll    They pant for air like jackals. \\
\poeml Their eyesight fails \\
\poemll    because there is no vegetation.''
\passage{The People Cry for Help}
\poeml \v{7}\divine{Lord}, even though our iniquities testify against us, \\
\poemll    do something for the sake of your name. \\
\poeml Indeed, our apostasies are many, \\
\poemll    and we have sinned against you. \\
\poeml \v{8}Hope of Israel, \\
\poemll    its deliverer in time of trouble, \\
\poeml why are you like a stranger\fnote{\fbackref{14:8} Or \fbib{resident alien}} in the land, \\
\poemll    like a traveler who sets up his tent for a night? \\
\poeml \v{9}Why are you like a man taken by surprise, \\
\poemll    like a strong man who can't deliver? \\
\poeml You are among us, \divine{Lord}, \\
\poemll    and your name is the one by which we're called. \\
\poemlll       Don't abandon us!
\passage{God Responds to the Prophet}
\poeml \v{10}This is what the \divine{Lord} says to these people: \\
\poemll    ``Yes, they do love to wander, \\
\poemlll       and they haven't restrained their feet. \\
\poeml So the \divine{Lord} won't accept them now. \\
\poemll    He will remember their iniquity \\
\poemlll       and punish their sin.''
\end{poetry}

\v{11}Then the \divine{Lord} told me, ``Don't pray for the welfare of these people. \v{12}Although they fast, I won't listen to their cry, and although they offer burnt offerings and grain offerings, I won't accept them. Instead, I'll put an end to them with the sword, with famine, and with a plague.''

\v{13}Then I said, ``Ah, Lord GOD, look! The prophets are saying to them, `You won't see the sword and you won't experience famine. Rather, I'll give you lasting peace in this place.'\,''

\v{14}Then the \divine{Lord} told me, ``The prophets are prophesying lies\fnote{\fbackref{14:14} Or \fbib{deception}} in my name. I didn't send them, I didn't command them, and I didn't speak to them. They're proclaiming\fnote{\fbackref{14:14} Lit. \fbib{prophesying}} to you false visions, worthless predictions,\fnote{\fbackref{14:14} or \fbib{divination}} and the delusions of their own minds. \v{15}Therefore, this is what the \divine{Lord} says about the false prophets who prophesy in my name, `There will be no sword and famine in this land' (though I haven't sent them): `By the sword and by famine these prophets will be finished off! \v{16}The people to whom they have prophesied will be thrown out into the streets of Jerusalem because of the famine and the sword. There will be no one to bury them, their wives, their sons, or their daughters. I'll pour out on them the\fnote{\fbackref{14:16} Lit. \fbib{their}} judgment they deserve.'\,''\fnote{\fbackref{14:16} The Heb. lacks \fbib{they deserve}}

\v{17}``And deliver\fnote{\fbackref{14:17} Lit. \fbib{speak}} this message to them:

\begin{poetry}
\poeml `Let tears run down my face,\fnote{\fbackref{14:17} Lit. \fbib{Let my eyes run down with tears}} \\
\poemll    night and day, and don't let them stop, \\
\poeml because my virgin daughter---my people--- \\
\poemll    will be broken with a powerful blow, \\
\poemlll       with a severe wound. \\
\poeml \v{18}If I go out into the field, \\
\poemll    I see those slain by the sword! \\
\poeml If I go into the city, \\
\poemll    I see the ravages of the famine! \\
\poeml Indeed, both prophet and priest \\
\poemll    ply their trade in the land, \\
\poemlll       but they don't know anything.'\,''\fnote{\fbackref{14:18} The Heb. lacks \fbib{anything}}
\passage{The People Plead to the \divine{Lord}}
\poeml \v{19}Have you completely rejected Judah? \\
\poemll    Do you despise Zion? \\
\poeml Why have you struck us, \\
\poemll    so that there is no healing for us? \\
\poeml We hoped for peace, but no good came, \\
\poemll    for a time of healing, but there was only terror. \\
\poeml \v{20}We acknowledge, \divine{Lord}, our wickedness, \\
\poemll    the guilt of our ancestors. \\
\poeml Indeed, we have sinned against you. \\
\poeml \v{21}For the sake of your name\fnote{\fbackref{14:21} I.e. your reputation} don't despise us. \\
\poemll    Don't dishonor your glorious throne. \\
\poemlll       Remember, don't break your covenant with us! \\
\poeml \v{22}Can any of the worthless gods of the nations make it rain? \\
\poemll    Can the heavens themselves bring forth showers? \\
\poeml Aren't you the one who does this,\fnote{\fbackref{14:22} The Heb. lacks \fbib{who does this}} \\
\poemll    \divine{Lord} our God? \\
\poeml So we hope in you, \\
\poemll    for you are the one who does all these things.
\end{poetry}
\labelchapt{15}
\passage{The Destiny of the Judged}

\chapt{15}
\v{1}Then the \divine{Lord} told me, ``Even if Moses and Samuel were standing before me, I wouldn't be favorably disposed toward this people. Send them out of my presence! Let them go!

\v{2}``When they say to you, `Where can we go?', say to them, `This is what the \divine{Lord} says:

\begin{poetry}
\poeml ``Those destined for death, \\
\poemll    to death will go;\fnote{\fbackref{15:2} The Heb. lacks \fbib{will go}} \\
\poeml those destined for the sword, \\
\poemll    to the sword will go;\fnote{\fbackref{15:2} The Heb. lacks \fbib{will go}} \\
\poeml and those destined for captivity, \\
\poemll    to captivity will go.\fnote{\fbackref{15:2} The Heb. lacks \fbib{will go}}
\end{poetry}

\v{3}``I'll appoint four kinds of judgment for them,'' declares the \divine{Lord}: ``the sword to kill, the dogs to drag off, the birds of the sky to devour, and the animals of the land to destroy. \v{4}I'll make them a horrifying sight to all the kingdoms of the earth because of what Hezekiah's son Manasseh, king of Judah, did in Jerusalem.

\begin{poetry}
\poeml \v{5}``Who will have pity on you, Jerusalem, \\
\poemll    and who will grieve for you? \\
\poeml Who will go out of his way \\
\poemll    to ask about your welfare? \\
\poeml \v{6}You have deserted me,'' declares the \divine{Lord}. \\
\poemll    ``You keep going backward. \\
\poeml I'll reach out my hand and destroy you. \\
\poemll    I'm tired of showing compassion. \\
\poeml \v{7}I'll winnow\fnote{\fbackref{15:7} Winnowing is the process of separating husks from the grain.} them with a winnowing fork \\
\poemll    in the gates of the land. \\
\poeml I'll make them childless. \\
\poemll    I'll destroy my people, \\
\poemlll       for they didn't change their ways. \\
\poeml \v{8}I'll make their\fnote{\fbackref{15:8} Lit. \fbib{her}} widows more numerous \\
\poemll    than the sand of the sea. \\
\poeml At noontime I'll send a destroyer \\
\poemll    against the mother\fnote{\fbackref{15:8} Lit. \fbib{send against them}} of a young man. \\
\poeml I'll cause terror and anguish \\
\poemll    to come to her unexpectedly. \\
\poeml \v{9}``The woman who gave birth to seven will grow faint, \\
\poemll    her life will expire. \\
\poeml Her sun will set while it's still day. \\
\poemll    She will be disgraced and humiliated. \\
\poeml I'll kill the rest of them with swords \\
\poemll    in the presence of their enemies,'' \\
\poemlll       declares the \divine{Lord}.
\passage{Jeremiah's Complaint}
\poeml \v{10}How terrible for me, my mother, \\
\poemll    that you gave birth to me, \\
\poeml a man of strife and contention for the whole land! \\
\poemll    I've neither lent nor borrowed, \\
\poemlll       yet everyone curses me.
\end{poetry}
\passage{God's Answer to Jeremiah's Complaint}

\v{11}The \divine{Lord} said,

\begin{poetry}
\poeml ``Have I not set you free \\
\poemll    for a good purpose? \\
\poeml Have I not intervened for you with your enemies \\
\poemll    in times of trouble and times of distress? \\
\poeml \v{12}``Can anyone break iron--- \\
\poemll    iron from the north---or bronze? \\
\poeml \v{13}``I'll give away your wealth and your treasures \\
\poemll    as plunder, for free, \\
\poemlll       because of all your sins throughout your territory. \\
\poeml \v{14}I'll make you serve your enemies \\
\poemll    in a land you don't know, \\
\poeml for my anger has started a fire \\
\poemll    that will burn against you.''
\passage{Jeremiah's Revised Complaint}
\poeml \v{15}You are aware--- \\
\poemll    \divine{Lord}, remember me, \\
\poeml pay attention to me, \\
\poemll    and vindicate me in front of those who pursue me. \\
\poeml You are patient--- \\
\poemll    don't take me away. \\
\poemlll       Know that I suffer insult because of you! \\
\poeml \v{16}Your words were found, and I consumed them. \\
\poemll    Your words were joy and my hearts delight, \\
\poeml because I bear your name,\fnote{\fbackref{15:16} Lit. \fbib{your name is called on me}} \\
\poemll    \divine{Lord} God of the Heavenly Armies. \\
\poeml \v{17}I didn't sit in the company of those who have fun, \\
\poemll    and I didn't rejoice. \\
\poeml Because of your hand on me,\fnote{\fbackref{15:17} The Heb. lacks \fbib{on me}} I sat alone, \\
\poemll    for you filled me with indignation. \\
\poeml \v{18}Why is my pain unending and my wound incurable, \\
\poemll    refusing to be healed?
\passage{God's Answer to Jeremiah's Revised Complaint}
\poeml You are like a deceptive brook, \\
\poemll    whose waters cannot be depended on. \\
\poeml \v{19}Therefore, this is what the \divine{Lord} says: \\
\poemll    ``If you repent, I'll take you back \\
\poemlll       and you will stand before me. \\
\poeml If you speak what is worthwhile,\fnote{\fbackref{15:19} Lit. \fbib{if worthwhile things come out}} \\
\poemll    instead of what is worthless, \\
\poemlll       then you will be my spokesman.\fnote{\fbackref{15:19} Lit. \fbib{my mouth}} \\
\poeml People\fnote{\fbackref{15:19} Lit. \fbib{They}} will turn to you, \\
\poemll    but you aren't to turn to them. \\
\poeml \v{20}I'll make you a fortified wall of bronze to this people. \\
\poemll    They'll fight against you, \\
\poeml but they won't prevail against you, \\
\poemll    for I am with you to save you \\
\poemlll       and deliver you,'' \\
\poeml \v{21}So I'll deliver you from the hand of the wicked, \\
\poemll    and redeem you from the grasp of the ruthless.''
\end{poetry}
\labelchapt{16}
\passage{The \divine{Lord}'s Instruction to His Prophet}

\chapt{16}
\v{1}This message from the \divine{Lord} came to me: \v{2}``You are not to take a wife, nor are you to have sons or daughters in this place.''

\v{3}For this is what the \divine{Lord} says about the sons and daughters who are born in this place, about their mothers who give birth to them, and about their fathers who father them in this land: \v{4}``They'll die of deadly diseases. People won't mourn for them, nor will they be buried. They'll be dung on the surface of the ground, and they'll come to an end with the sword and with famine. Their dead bodies will be food for the birds of the sky and the animals of the land.''

\v{5}For this is what the \divine{Lord} says: ``Don't go to a house where there is mourning, don't go to lament, nor to express sorrow to them. For I've taken my peace away from this people,'' declares the \divine{Lord}, ``as well as gracious love and compassion. \v{6}Both the most and the least important people\fnote{\fbackref{16:6} Lit. \fbib{great and small}; i.e. adults and children} will die in this land, and they won't be buried. People won't mourn for them. They won't cut themselves,\fnote{\fbackref{16:6} A Canaanite mourning practice forbidden by Deut. 14:1} nor will they shave their heads for them.\fnote{\fbackref{16:6} A common mourning practice in the ancient world} \v{7}They won't break bread\fnote{\fbackref{16:7} The Heb. lacks \fbib{bread}} for the mourner to be consoled for the dead. They won't give anyone the cup of consolation to drink for his father or\fnote{\fbackref{16:7} The Heb. lacks \fbib{father or}} mother. \v{8}Don't go to a banquet to sit with people\fnote{\fbackref{16:8} Lit. \fbib{with them}} to eat and drink.'' \v{9}For this is what the \divine{Lord} of the Heavenly Armies, the God of Israel, says: ``In this place I'm about to bring an end to the sounds of happiness and rejoicing, the sounds of the bridegroom and the bride. I'll do it in front of your eyes and in your time.

\v{10}``When you speak all these words to this people, they'll say to you, `Why has the \divine{Lord} pronounced all this disaster against us? What is our iniquity, and what is the sin that we have committed against the \divine{Lord} our God?' \v{11}Then say to them, `It is because your ancestors abandoned me,' declares the \divine{Lord}. `They followed other gods, served them, worshipped them, abandoned me, and didn't keep my Law. \v{12}You have done even more evil than your ancestors, and each one of you is stubbornly following his own evil desires,\fnote{\fbackref{16:12} Lit. \fbib{following the stubbornness of their evil heart}} refusing to listen to me. \v{13}I'll throw you out of this land into a land neither you nor your ancestors have known. There you will serve other gods day and night, and I'll show you no favor.'

\v{14}``Therefore, days are coming,'' declares the \divine{Lord}, ``when it will no longer be said, `As surely as the \divine{Lord} lives, who brought up the Israelis from the land of Egypt.' \v{15}Rather it will be said, `As surely as the \divine{Lord} lives, who brought the Israelis up from the land of the north and from all the lands to which the \divine{Lord}\fnote{\fbackref{16:15} Lit. \fbib{he}} had banished them.' I'll bring them back to their land, which I gave to their ancestors.

\v{16}``I'm about to send many fishermen,'' declares the \divine{Lord}, ``and they'll catch them. Afterwards, I'll send for many hunters and they'll hunt for them on every mountain and hill and in the crevices of the rocks. \v{17}For I am watching all their ways; they are not hidden from my sight.\fnote{\fbackref{16:17} Lit. \fbib{from before me}} Their iniquity is not concealed from my eyes. \v{18}First I'll repay them double for their iniquity and their sin, because they have polluted my land with the dead bodies of their detestable images, and they have filled my inheritance with their abominations.''\fnote{\fbackref{16:18} Or \fbib{their abominable idols}}

\begin{poetry}
\poeml \v{19}\divine{Lord}, my strength and my stronghold, \\
\poemll    my refuge in a time of difficulty, \\
\poeml to you the nations will come, \\
\poemll    and from the ends of the earth they'll say, \\
\poeml ``Surely our ancestors inherited deception,\fnote{\fbackref{16:19} I.e. false gods or idols} \\
\poemll    things that are worthless, \\
\poemlll       and in which there is no profit.'' \\
\poeml \v{20}Can a person make a god for himself? \\
\poemll    They are not gods! \\
\poeml \v{21}Therefore, I'm about to make them understand; \\
\poemll    this time I'll make them understand \\
\poeml my power and strength, \\
\poemll    so they'll understand that my name is the \divine{Lord}.
\end{poetry}
\labelchapt{17}
\passage{Judah's Sin and Its Consequence}

\begin{poetry}
\poeml \chapt{17}
\v{1}The sin of Judah is engraved \\
\poeml with an iron stylus. \\
\poeml It is inscribed with a diamond point \\
\poeml on the tablet of their heart \\
\poemlll       and on the horns of their\fnote{\fbackref{17:1} Lit. \fbib{your}} altars. \\
\poeml \v{2}When their sons remember, \\
\poemll    they remember their altars\fnote{\fbackref{17:2} Lit. \fbib{it is their altars}} and their Asherah poles\fnote{\fbackref{17:2} I.e. sacred poles representing the goddess Asherah} \\
\poemlll       beside green trees on the high hills. \\
\poeml \v{3}My mountain in the field, your wealth and your treasures \\
\poemll    I'll give as spoil; \\
\poeml along with your high places as the price of your sin \\
\poemll    throughout your territory. \\
\poeml \v{4}You will let go of your inheritance \\
\poemll    which I gave you, \\
\poeml and I'll make you serve your enemies \\
\poemll    in a land that you don't know. \\
\poeml For with my anger you have started a fire \\
\poemll    that will burn forever.
\passage{Two Ways Contrasted}
\poeml \v{5}This is what the \divine{Lord} says: \\
\poeml ``Cursed is the person who trusts in mankind, \\
\poemll    who makes flesh his strength, \\
\poemlll       and whose heart turns away from the \divine{Lord}. \\
\poeml \v{6}He will be like a bush in the desert, \\
\poemll    and he won't see when good comes. \\
\poeml He will dwell in parched places in the wilderness,\fnote{\fbackref{17:6} Or \fbib{desert}} \\
\poemll    a land of salt, without inhabitants. \\
\poeml \v{7}Blessed is the person who trusts in the \divine{Lord}, \\
\poemll    making the \divine{Lord} his trust. \\
\poeml \v{8}He will be like a tree planted by the water \\
\poemll    that sends out its roots by a stream. \\
\poeml He won't fear when the heat comes, \\
\poemll    and his leaves will be green. \\
\poeml In a year of drought he won't be concerned, \\
\poemll    nor will he stop producing fruit.''
\passage{The Deceitfulness of the Human Heart}
\poeml \v{9}``The heart is more deceitful than anything. \\
\poemll    It is incurable--- \\
\poemlll       who can know it? \\
\poeml \v{10}I am the \divine{Lord} who searches the heart, \\
\poemll    who tests the inner depths \\
\poeml to give to each person \\
\poemll    according to what he deserves,\fnote{\fbackref{17:10} Lit. \fbib{according to his way}} \\
\poemlll       according to the fruit of his deeds. \\
\poeml \v{11}As a partridge gathers together eggs \\
\poemll    that it didn't lay, \\
\poeml so is a person who amasses wealth unjustly. \\
\poemll    In the middle of his life\fnote{\fbackref{17:11} Lit. \fbib{at half of his days}} it will leave him, \\
\poemlll       and in the end he will prove to be a fool.
\passage{The \divine{Lord}: The Hope of Israel}
\poeml \v{12}A glorious throne exalted from the beginning \\
\poemll    is the place of our sanctuary. \\
\poeml \v{13}\divine{Lord}, you are the hope of Israel; \\
\poemll    all who forsake you will be put to shame. \\
\poeml Those who turn aside from you\fnote{\fbackref{17:13} Lit. \fbib{me}} will be \\
\poemll    written in the dust,\fnote{\fbackref{17:13} Or \fbib{recorded in the underworld}} \\
\poeml because they have forsaken the \divine{Lord}, \\
\poemll    the spring of living water.
\passage{The Prophet's Call for Help and Justice}
\poeml \v{14}Heal me, \divine{Lord}, and I'll be healed; \\
\poemll    deliver me, and I'll be delivered, \\
\poemlll       because you are my praise. \\
\poeml \v{15}Look, they're saying to me, \\
\poemll    ``Where is the message from the \divine{Lord}? \\
\poemlll       Let it come about!'' \\
\poeml \v{16}I haven't run away from being your shepherd,\fnote{\fbackref{17:16} Lit. \fbib{from shepherding after you}} \\
\poemll    and I haven't longed for the day of sickness.\fnote{\fbackref{17:16} I.e. the day of judgment} \\
\poeml You know what comes out from my lips, \\
\poemll    it's open before you.\fnote{\fbackref{17:16} Lit. \fbib{it is in front of you}} \\
\poeml \v{17}Don't be a terror to me. \\
\poemll    You are my refuge in a day of trouble. \\
\poeml \v{18}Let those who pursue me be put to shame, \\
\poemll    but don't put me to shame. \\
\poeml Let them be terrified, \\
\poemll    but don't let me be terrified. \\
\poeml Bring the day of judgment\fnote{\fbackref{17:18} Or \fbib{disaster}} on them, \\
\poemll    and destroy them with double destruction!
\end{poetry}
\passage{A Test Case: Keeping the Sabbath}

\v{19}The \divine{Lord} told me, ``Go, stand in the gate of the people,\fnote{\fbackref{17:19} Lit. \fbib{gate of the sons of the people}} where the kings of Judah come in and go out, and in the other gates of Jerusalem as well. \v{20}Say to them, `Kings of Judah, all Judah, and all the residents of Jerusalem entering these gates, hear this message from the \divine{Lord}. \v{21}This is what the \divine{Lord} says: ``Be careful! On the Sabbath day, don't carry any load or bring anything through the gates of Jerusalem. \v{22}Don't bring any load out of your houses on the Sabbath day, nor are you to do any work. You are to consecrate\fnote{\fbackref{17:22} I.e. set it apart} the Sabbath day, just as I commanded your ancestors. \v{23}But they didn't listen, nor did they pay attention.\fnote{\fbackref{17:23} Lit. \fbib{incline the ear}} They were determined\fnote{\fbackref{17:23} Lit. \fbib{stiffened their necks}} not to listen and not to accept instruction.\fnote{\fbackref{17:23} Or \fbib{discipline}} \v{24}If you listen to me carefully,'' declares the \divine{Lord}, ``and don't bring a load through the gates of this city on the Sabbath day, and you consecrate the Sabbath day and don't do any work on it, \v{25}then kings and princes, sitting on the throne of David will come through the gates of this city. They, their princes, the men of Judah, and the residents of Jerusalem will come riding in chariots and on horses, and this city will be inhabited forever. \v{26}They'll come from the cities of Judah, from the places around Jerusalem, from the territory of Benjamin, from the Shephelah,\fnote{\fbackref{17:26} I.e. the verdant central lowlands of Israel; cf. Josh 10:40} from the hill country, and from the Negev,\fnote{\fbackref{17:26} I.e. the southern regions of the Sinai peninsula; cf. Josh 10:40} bringing burnt offerings, sacrifices, grain offerings, and incense, and bringing thanksgiving offerings to the \divine{Lord}'s Temple. \v{27}But if you don't listen to me, to consecrate the Sabbath day and not carry any load as you enter the gates of Jerusalem on the Sabbath day, then I'll start a fire in its gates. It will consume the palaces of Jerusalem and won't be extinguished.''\,'\,''
\labelchapt{18}
\passage{The Potter's House and the Ruined Vessel}

\chapt{18}
\v{1}The message that came to Jeremiah from the \divine{Lord}: \v{2}``Arise and go down to the potter's house, and there I'll allow you to hear my words.'' \v{3}So I went down to the potter's house, and there he was doing work at the potter's wheel. \v{4}But the vessel he was working on with the clay was ruined in the potter's hand. So he remade it into another vessel that seemed appropriate to him.

\v{5}Then this message from the \divine{Lord} came to me: \v{6}``Israel, can't I deal with you like this potter?'' declares the \divine{Lord}. ``Look, Israel, like clay in the potter's hand, so are you in my hand. \v{7}At one moment I may speak about a nation or a kingdom to uproot it, pull it down, or destroy it. \v{8}But if that nation about which I spoke turns from its evil way, I'll change my mind about the disaster that I had planned for it. \v{9}At another moment I may speak about a nation or kingdom to build it or plant it. \v{10}But if that nation does evil in my eyes by not obeying me, I'll change my mind about the good that I said I would bring on it.

\v{11}``Now say to the people of Judah and to the residents of Jerusalem, `This is what the \divine{Lord} says: ``Look, I'm designing a disaster just for you, and I'm making plans against you. Each one of you must repent from his evil way. Make your ways and deeds right.''\,' \v{12}But they'll say, `It's useless! We will follow our plans and each of us will pursue his own evil desires.'\fnote{\fbackref{18:12} Lit. \fbib{will do the stubbornness of his evil heart}}

\v{13}``Therefore, this is what the \divine{Lord} says:

\begin{poetry}
\poeml `Ask the nations. \\
\poemll    Who has ever heard of anything like this? \\
\poeml You have done a most horrible thing, \\
\poemll    virgin Israel. \\
\poeml \v{14}Does the snow of Lebanon \\
\poemll    ever vanish from its rocky slopes?\fnote{\fbackref{18:14} Or \fbib{Do rocks ever leave the slopes, or the snow from Lebanon?}} \\
\poeml Or does the cold water from a foreign land \\
\poemll    ever cease to flow? \\
\poeml \v{15}Yet my people have forgotten me, \\
\poemll    and they burn incense to worthless idols \\
\poeml that make them stumble in their journey \\
\poemll    on the ancient paths. \\
\poeml They walk on trails, \\
\poemll    on a way that is not built up. \\
\poeml \v{16}They make their land into a desolate place, \\
\poemll    an object of lasting scorn.\fnote{\fbackref{18:16} Lit. \fbib{hissing}} \\
\poeml All who pass by will be appalled \\
\poemll    and will shake their heads.\fnote{\fbackref{18:16} I.e. in surprise at the magnitude of the destruction} \\
\poeml \v{17}`Like the east wind, I'll scatter them \\
\poemll    before the enemy. \\
\poeml I'll show them my back and not my face, \\
\poemll    on the day of their downfall.'\,''
\end{poetry}
\passage{Jeremiah Reacts to the Plot against Him}

\v{18}Then they said, ``Come, let's make up a plot against Jeremiah. After all, the priest's instruction, the wise man's counsel, and the prophet's message won't be destroyed.\fnote{\fbackref{18:18} Lit. \fbib{perish}} So let's verbally attack him. Pay no attention to anything he says!''

\begin{poetry}
\poeml \v{19}\divine{Lord}, pay attention to me. \\
\poemll    Listen to the voice of my accusers! \\
\poeml \v{20}Should good be repaid with evil? \\
\poemll    Yet they have dug a pit to take my life.\fnote{\fbackref{18:20} Or \fbib{a pit for me}} \\
\poeml Remember! I stood before you \\
\poemll    and spoke good on their behalf \\
\poemlll       in order to turn your wrath away from them. \\
\poeml \v{21}Therefore, make their children undergo famine, \\
\poemll    and deliver them over to death in time of war.\fnote{\fbackref{18:21} Lit. \fbib{to the power of the sword.}} \\
\poeml May their women be childless widows! \\
\poemll    May their men be slaughtered!\fnote{\fbackref{18:21} Lit. \fbib{killed dead}} \\
\poeml May their young men be slain \\
\poemll    by the sword in battle! \\
\poeml \v{22}Let a cry be heard from their houses because you \\
\poemll    have brought a raiding party against them suddenly. \\
\poeml For they have dug a pit to capture me \\
\poemll    and have set\fnote{\fbackref{18:22} Or \fbib{hidden}} traps for my feet. \\
\poeml \v{23}But you, \divine{Lord}, know all their plots to kill me. \\
\poemll    Don't forgive their iniquity, \\
\poemll    and don't erase their sin from your sight. \\
\poeml Let them stumble before you. \\
\poemll    When it's time for you to be angry, act against them!
\end{poetry}
\labelchapt{19}
\passage{The Lesson of the Broken Jug}

\chapt{19}
\v{1}This is what the \divine{Lord} says: ``Go and buy a potter's clay jug. Take along\fnote{\fbackref{19:1} The Heb. lacks \fbib{along}} some of the elders of the people and some of the elders of the priests. \v{2}Go out to the Valley of Hinnom\fnote{\fbackref{19:2} Lit. \fbib{Valley of Hinnom's son}} at the entrance to the Potsherd Gate, and there proclaim the words that I'm telling you.

\v{3}``You are to say, `Hear this message from the \divine{Lord}, you kings of Judah and residents of Jerusalem!

```This is what the \divine{Lord} of the Heavenly Armies, the God of Israel, says: ``I'm about to bring a disaster on this place that will make the ears of all who hear about it tingle. \v{4}For they have forsaken me and have treated this place as foreign. In it they have burned incense to other gods that neither they, their ancestors, nor the kings of Judah knew. They have also filled this place with the blood of innocent people. \v{5}They built the high places\fnote{\fbackref{19:5} I.e. the places where Canaanite gods were worshipped} for Baal to burn their children in the fire as a burnt offering to Baal---something I didn't command, didn't say, nor did it ever enter my mind!

\v{6}`````Therefore, days are coming,'' declares the \divine{Lord}, ``when this place will no longer be called Topheth, or the Valley of Hinnom, but rather the Valley of Slaughter. \v{7}I'll shatter\fnote{\fbackref{19:7} Or \fbib{nullify}; MT word for \fbib{shatter} sounds like MT word for \fbib{jug} in v. 1} the counsel of Judah and Jerusalem in this place, and I'll make them fall by the sword before their enemies and at the hands of those seeking their lives. I'll give their dead bodies as food to the birds of the sky and to the animals of the land. \v{8}I'll make this city into a desolate place and an object of scorn.\fnote{\fbackref{19:8} Lit. \fbib{hissing}} All who pass by it will be astonished and will scoff\fnote{\fbackref{19:8} Lit. \fbib{hiss}; i.e. hissing was an expression of contempt} because of all its wounds. \v{9}I'll cause them to eat the flesh of their sons and daughters,\fnote{\fbackref{19:9} Lit. \fbib{sons and the flesh of their daughters}} and people will eat the flesh of their neighbors in the siege and in the distress to which their enemies and those seeking their lives will subject them.''\,'\,''

\v{10}``Then you are to break the jug in front of the men who have come with you, \v{11}and say to them, `This is what the \divine{Lord} of the Heavenly Armies says: ``In this same way I'll break this people and this city, just as someone breaks a potter's vessel which he then cannot put back together again. They'll bury corpses\fnote{\fbackref{19:11} The Heb. lacks \fbib{corpses}} in Topheth until there is no more room to bury anyone.\fnote{\fbackref{19:11} The Heb. lacks \fbib{anyone}} \v{12}This is what I'll do to this place and its residents,'' declares the \divine{Lord}, ``making this city like Topheth. \v{13}The houses of Jerusalem and the houses of the kings of Judah will be polluted like Topheth, as will be all the houses on whose roofs people\fnote{\fbackref{19:13} Lit. \fbib{they}} burned incense to all the host of heaven and poured out liquid offerings to other gods.''\,'\,''

\v{14}Then Jeremiah went from Topheth where the \divine{Lord} had sent him to prophesy. He stood in the courtyard of the \divine{Lord}'s Temple, saying to all the people, \v{15}``This is what the \divine{Lord} of the Heavenly Armies, the God of Israel, says: `I'm about to bring on this city and all its towns all the disaster that I declared against it because they were determined\fnote{\fbackref{19:15} Lit. \fbib{they stiffened their neck}} not to obey my message.'\,''
\labelchapt{20}
\passage{Jeremiah Denounced}

\chapt{20}
\v{1}When the priest Pashhur, Immer's son, who was the officer in charge\fnote{\fbackref{20:1} Lit. \fbib{the Nagid}; i.e. a senior officer entrusted with dual roles of operational oversight and administrative authority} of the \divine{Lord}'s Temple heard Jeremiah prophesying these words, \v{2}Pashhur struck Jeremiah the prophet and put him in the stocks that were at the upper Benjamin Gate of the Temple. \v{3}The next day, Pashhur released Jeremiah from the stocks, and Jeremiah told him, ``The \divine{Lord} has not named you Pashhur, but rather Magor-missabib.\fnote{\fbackref{20:3} The Heb. name \fbib{Magor-missabib} means \fbib{terror on every side}; cf. v. 10} \v{4}For this is what the \divine{Lord} says: `Look, I'm going to make you a terror to yourself and to all your loved ones. They'll fall by the sword of their enemies, and your eyes will see it. I'll give all Judah into the hand of the king of Babylon. He will take them into exile to Babylon, and he will execute them with swords. \v{5}I'll turn over all the wealth of this city, all its possessions, all its valuables, and all the treasures of the kings of Judah right into the hands of their enemies, and they'll plunder them, capture them, and take them to Babylon. \v{6}You, Pashhur, and all those living in your house will go into captivity. You will go to Babylon and there you will die. There you and all your loved ones\fnote{\fbackref{20:6} Or \fbib{friends, colleagues}} to whom you have falsely prophesied will be buried.'\,''
\passage{Jeremiah's Complaint to the \divine{Lord}}

\begin{poetry}
\poeml \v{7}You deceived me, \divine{Lord}, \\
\poemll    and I've been deceived. \\
\poeml You overpowered me, \\
\poemll    and you prevailed. \\
\poeml I've become a laughing stock all day long, \\
\poemll    and everyone mocks me. \\
\poeml \v{8}Indeed, as often as I speak, I cry out, \\
\poemll    and shout, ``Violence and destruction!'' \\
\poeml For this message from the \divine{Lord} has caused me \\
\poemll    constant\fnote{\fbackref{20:8} Lit. \fbib{all day long}} reproach and derision. \\
\poeml \v{9}When I say, ``I won't remember the \divine{Lord}\fnote{\fbackref{20:9} Lit. \fbib{him}}, \\
\poemll    nor will I speak in his name anymore, \\
\poeml then there is this burning fire in my heart. \\
\poemll    It is bound up in my bones, \\
\poeml I grow weary of trying to hold it in, \\
\poemll    and I cannot do it! \\
\poeml \v{10}Indeed, I hear many people whispering, \\
\poemll    ``Terror on every side.\fnote{\fbackref{20:10} I.e. in mockery of the prophet's statement \fbib{Magor-missabib} in v. 3} \\
\poeml Denounce him, let's denounce him!'' \\
\poemll    All my close friends watch my steps and say, \\
\poeml ``Perhaps he will be deceived, \\
\poemll    and we can prevail against him \\
\poemlll       and take vengeance on him.'' \\
\poeml \v{11}But the \divine{Lord} is with me like a fearsome warrior. \\
\poemll    Therefore, those who pursue me will stumble \\
\poemlll       and won't prevail. \\
\poeml They'll be put to great shame, \\
\poemll    when they don't succeed. \\
\poemlll       Their everlasting disgrace won't be forgotten. \\
\poeml \v{12}\divine{Lord} of the Heavenly Armies, \\
\poemll    who tests the righteous, \\
\poemll    who sees the inner motives\fnote{\fbackref{20:12} Lit. \fbib{the liver}} and the heart, \\
\poeml let me see you take vengeance on them, \\
\poemll    for I've committed my case to you. \\
\poeml \v{13}Sing to the \divine{Lord}, \\
\poemll    give praise to the \divine{Lord}! \\
\poeml For he saves the life of the poor \\
\poemll    from the hand of the wicked.
\passage{Jeremiah Curses the Day of His Birth}
\poeml \v{14}Let the day on which I was born be cursed. \\
\poemll    Don't let the day on which my mother gave birth to me be blessed. \\
\poeml \v{15}Cursed is the person who brought \\
\poemll    the good news to my father, \\
\poeml ``A baby boy has been born to you,'' \\
\poemll    making him very happy. \\
\poeml \v{16}May that man be like the cities that \\
\poemll    the \divine{Lord} overthrew without compassion. \\
\poeml Let him hear a cry in the morning, \\
\poemll    and a battle cry at noon, \\
\poeml \v{17}because he didn't kill me in the womb, \\
\poemll    so that my mother would have been my grave \\
\poemlll       and her womb forever pregnant. \\
\poeml \v{18}Why did I ever come out of the womb \\
\poemll    to see trouble and sorrow, \\
\poemlll       and to finish my life living in shame?
\end{poetry}
\labelchapt{21}
\passage{Zedekiah's Request for a Miracle}

\chapt{21}
\v{1}The message that came to Jeremiah from the \divine{Lord} when King Zedekiah sent to him Malchijah's son Pashhur and Maaseiah's son Zephaniah the priest: \v{2}Please inquire of the \divine{Lord} on our behalf, for Nebuchadnezzar king of Babylon is fighting against us. Perhaps the \divine{Lord} will do some of his miraculous acts\fnote{\fbackref{21:2} Lit. \fbib{according to all his miraculous acts}} for us, and Nebuchadnezzar\fnote{\fbackref{21:2} Lit. \fbib{he}} will depart from us.''

\v{3}Jeremiah told them, ``This is what you are to say to Zedekiah, \v{4}`This is what the \divine{Lord} God of Israel says: ``I'm about to turn against you the weapons of war that are in your hands and with which you are fighting the king of Babylon and the Chaldeans who are besieging you outside the walls. I'll gather them into the center of this city. \v{5}Because of my anger, wrath, and great fury, I'll fight against you myself with an outstretched hand and a strong arm. \v{6}I'll strike down the residents of this city, both people and animals, and they'll die from a terrible plague. \v{7}Afterwards,'' declares the \divine{Lord}, ``I'll give King Zedekiah of Judah, his officials,\fnote{\fbackref{21:7} Or \fbib{servants}} and the people---those who are left in this city from the plague, the sword, and the famine---into the control of Nebuchadnezzar king of Babylon, right into the hand of their enemies and the hand of those who want to kill them. He'll execute them with swords and won't pity them. He won't spare them, nor will he have compassion on them.''\,'

\v{8}``You are to say to this people, `This is what the \divine{Lord} says: ``I'm about to set before you the way of life and the way of death. \v{9}Whoever stays in this city will die by the sword, by famine, and by the plague. But whoever goes out and surrenders to the Chaldeans who are besieging you will live. He will save his life as a spoil of war.\fnote{\fbackref{21:9} I.e. his life will be spared} \v{10}Indeed, I'm firmly decided---I'm sending calamity to this city, not good,'' declares the \divine{Lord}. ``It will be given into the hand of the king of Babylon, and he will set it on fire.''\,'
\passage{The Guilt of Judah's King}

\v{11}``To the house of the king of Judah say, `Hear this message from the \divine{Lord}.

\begin{poetry}
\poeml \v{12}This is what the \divine{Lord} says, house of David: \\
\poeml ``Judge appropriately every morning, \\
\poemll    and deliver those who have been robbed \\
\poemlll       from the oppressor, \\
\poeml so my anger does not break out like fire \\
\poemll    and burn with no one to put it out \\
\poemlll       because of your evil deeds. \\
\poeml \v{13}``Look, I'm against you, \\
\poemll    city dwelling in the valley, \\
\poeml rock of the plain,'' \\
\poemll    declares the \divine{Lord}, \\
\poeml ``those of you who say, `Who can come down against us \\
\poemll    and who can enter our habitations?' \\
\poeml \v{14}But I'll punish you according to \\
\poemll    what you have done,''\fnote{\fbackref{21:14} Lit. \fbib{to the fruit of your deeds}} \\
\poemlll       declares the \divine{Lord}. \\
\poeml ``I'll start a fire in her forest, \\
\poemll    and it will consume everything around her.''\,'\,''
\end{poetry}
\labelchapt{22}
\passage{Instructions for the Kings of Judah}

\chapt{22}
\v{1}This is what the \divine{Lord} says: ``Go down to the house of the king of Judah and tell him this: \v{2}`Listen to this message from the \divine{Lord}, king of Judah, you who sit on the throne of David---you, your officials,\fnote{\fbackref{22:2} Or \fbib{your servants}} and your people who enter these gates. \v{3}This is what the \divine{Lord} says: ``Uphold justice and righteousness. Deliver from their oppressor those who have been robbed. Don't mistreat or do violence to the alien, the orphan, or the widow, or shed the blood of innocent people in this place. \v{4}Rather, carefully obey this message,\fnote{\fbackref{22:4} Or \fbib{do this thing}} and then kings sitting for David on his throne and riding in chariots and on horses will enter the gates of this house. The king will enter along with his officials\fnote{\fbackref{22:4} Lit. \fbib{house, he, his officials}} and his people. \v{5}But if you don't listen to these words, I swear,'' declares the \divine{Lord}, ``that this house will become a ruin.''\,'\,'' \v{6}For this is what the \divine{Lord} says about the house of the king of Judah,

\begin{poetry}
\poeml ``You are like Gilead to me, \\
\poemll    like the summit of Lebanon. \\
\poeml Yet I'll surely make you a desert, \\
\poemll    towns where no one lives. \\
\poeml \v{7}I'll appoint people to destroy you--- \\
\poemll    men with their weapons.
\end{poetry}

They'll cut down some of your choice cedars\fnote{\fbackref{22:7} I.e. a genus of coniferous evergreen in the family \fbib{Pinaceae}; and so throughout the book}

\begin{poetry}
\poemll    and incinerate them.
\end{poetry}

\v{8}``Many nations will pass by this city and say to one another, `Why did the \divine{Lord} do this to this great city?' \v{9}Then people\fnote{\fbackref{22:9} Lit. \fbib{they'll say}} will respond, `It is\fnote{\fbackref{22:9} The Heb. lacks \fbib{It is}} because they have forsaken the covenant of the \divine{Lord} their God and have bowed down to other gods and served them.'

\begin{poetry}
\poeml \v{10}``Don't cry for the dead \\
\poemll    or grieve for them. \\
\poeml Weep bitterly for the one going away, \\
\poemll    because he won't return again \\
\poemlll       nor see the land of his birth.
\end{poetry}

\v{11}``For this is what the \divine{Lord} says about Josiah's son Shallum,\fnote{\fbackref{22:11} Shallum (also known as Jehoahaz) succeeded his father Josiah, but was removed by the Egyptians after three months and exiled to Egypt.} king of Judah, who reigned in place of his father Josiah: `He went out from this place and won't return to it again. \v{12}He will die in the place where they exiled him, and he won't ever\fnote{\fbackref{22:12} The Heb. lacks \fbib{ever}} see this land again.'\,''
\passage{An Oracle against Jehoiakim}

\begin{poetry}
\poeml \v{13}``How terrible for him who builds his house \\
\poemll    without righteousness, \\
\poeml and its upper rooms without justice, \\
\poemll    who makes his neighbor work for nothing, \\
\poemll    and does not pay him his wage. \\
\poeml \v{14}How terrible for\fnote{\fbackref{22:14} The Heb. lacks \fbib{How terrible for}} him who says, `I'll build a large \\
\poemll    house for myself with spacious upper rooms, \\
\poeml who cuts out windows for it, \\
\poemll    paneling it with cedar and painting it red.' \\
\poeml \v{15}Are you a king because you try to outdo \\
\poemll    everyone with cedar? \\
\poeml Your father ate and drank and upheld \\
\poemll    justice and righteousness, did he not? \\
\poemlll       And then it went well for him. \\
\poeml \v{16}He judged the case of the poor and needy. \\
\poemll    And then it went well for him. \\
\poemlll       Isn't this what it means to know me? \\
\poeml \v{17}But your eyes and heart are on nothing but \\
\poemll    your dishonest gain, \\
\poeml shedding the blood of innocent people, \\
\poemll    and practicing oppression and extortion.''
\end{poetry}

\v{18}Therefore, this is what the \divine{Lord} says about Josiah's son Jehoiakim, king of Judah,

\begin{poetry}
\poeml ``They won't lament for him with these words:\fnote{\fbackref{22:18} The Heb. lacks \fbib{with these words}} \\
\poemll    `How terrible, my brother, \\
\poemlll       How terrible, my sister!' \\
\poeml They won't lament for him with these words:\fnote{\fbackref{22:18} The Heb. lacks \fbib{with these words}} \\
\poemll    `How terrible, lord, \\
\poemlll       How terrible, your\fnote{\fbackref{22:18} The Heb. lacks \fbib{your}} majesty!' \\
\poeml \v{19}He will receive\fnote{\fbackref{22:19} Lit. \fbib{be buried with}} a donkey's burial, \\
\poemll    dragged out and thrown outside the gates of Jerusalem.''
\passage{An Oracle against Jerusalem}
\poeml \v{20}Go up to Lebanon and cry out, \\
\poemll    to Bashan and lift up your voice. \\
\poeml Cry out from Abarim, for all your lovers\fnote{\fbackref{22:20} I.e. \fbib{your allies}} \\
\poemll    have been crushed. \\
\poeml \v{21}I spoke to you when you were secure,\fnote{\fbackref{22:21} Or \fbib{prosperous}} \\
\poemll    but you said, ``I won't listen!'' \\
\poeml This has been your way since your youth, \\
\poemll    for you haven't obeyed me. \\
\poeml \v{22}The wind will shepherd\fnote{\fbackref{22:22} I.e. round them up and blow them away} all your shepherds,\fnote{\fbackref{22:22} I.e. leaders} \\
\poemll    and your lovers\fnote{\fbackref{22:22} I.e. \fbib{your allies}} will go into exile. \\
\poeml Indeed, you will then be ashamed and humiliated \\
\poemll    because of all your wickedness. \\
\poeml \v{23}You who live in Lebanon, \\
\poemll    who build your nest in the cedars, \\
\poeml how you will groan when pains come upon you, \\
\poemll    pain like that of a woman giving birth.
\end{poetry}
\passage{An Oracle against Jehoiachin}

\v{24}``As certainly as I'm alive and living,'' declares the \divine{Lord}, ``even if Jehoiakim's son King Jehoiachin\fnote{\fbackref{22:24} Lit. \fbib{Coniah}} of Judah were a signet ring on my right hand, I would pull you off \v{25}and give you to those who are trying to kill you, whom you fear---that is, to King Nebuchadnezzar of Babylon and the Chaldeans. \v{26}I'll hurl you and the mother who gave birth to you into another land where you were not born, and there you will die. \v{27}As for the land to which you\fnote{\fbackref{22:27} Lit. \fbib{they}} want to return, you\fnote{\fbackref{22:27} Lit. \fbib{they}} won't return there!

\begin{poetry}
\poeml \v{28}``Is this man Jehoiachin\fnote{\fbackref{22:28} Lit. \fbib{Coniah}} a despised and shattered jar, \\
\poemll    a vessel no one wants? \\
\poeml Why were he and his descendants hurled away, \\
\poemll    thrown into a land that they didn't know? \\
\poeml \v{29}Land, land, land, \\
\poemll    listen to this message from the \divine{Lord}! \\
\poeml \v{30}This is what the \divine{Lord} says: \\
\poeml `Write this man off as childless, \\
\poemll    a man who does not prosper in his lifetime.\fnote{\fbackref{22:30} Lit. \fbib{in his days}} \\
\poeml None of his descendants will succeed \\
\poemll    in sitting on the throne of David, \\
\poemlll       or ever ruling in Judah again.'\,''
\end{poetry}
\labelchapt{23}
\passage{A Righteous King for God's People}

\chapt{23}
\v{1}``How terrible for the shepherds\fnote{\fbackref{23:1} I.e. leaders} who are destroying and scattering the sheep of my pasture!'' declares the \divine{Lord}. \v{2}Therefore, this is what the \divine{Lord} God of Israel says about the shepherds who are shepherding my people, ``You have scattered my flock and driven them away. You haven't taken care of them, and now I'm about to take care of you\fnote{\fbackref{23:2} I.e. in judgment} because of your evil deeds,'' declares the \divine{Lord}. \v{3}``I'll gather the remnant of my flock from all the countries where I've driven them, and bring them back to their pasture where they'll be fruitful and increase in numbers. \v{4}I'll raise up shepherds over them, and they'll shepherd them. My flock\fnote{\fbackref{23:4} Lit. \fbib{they}} will no longer be afraid or terrified, and none will be missing,'' declares the \divine{Lord}.

\begin{poetry}
\poeml \v{5}``The time is coming,'' declares the \divine{Lord}, \\
\poemll    ``when I'll raise up a righteous branch for David. \\
\poeml He will be a king who rules wisely, \\
\poemll    and he will administer justice and righteousness in the land. \\
\poeml \v{6}In his time\fnote{\fbackref{23:6} Lit. \fbib{days}} Judah will be delivered \\
\poemll    and Israel will dwell in safety. \\
\poeml This is the name by which he will be known: \\
\poemll    `The \divine{Lord} Our Righteousness.'
\end{poetry}

\v{7}``Therefore, the time is coming,'' declares the \divine{Lord}, ``when people will no longer say, `As surely as the \divine{Lord} lives who brought up the Israelis from the land of Egypt,' \v{8}but they'll say,\fnote{\fbackref{23:8} The Heb. lacks \fbib{they'll say}} `As surely as the \divine{Lord} lives who brought the descendants of the Israelis from the land of the north and from all the lands where I had driven them and brought them into the land.'\fnote{\fbackref{23:8} The Heb. lacks \fbib{the land}} Then they'll live in their own land.''
\passage{An Oracle about False Prophets}

\v{9}Concerning the prophets:

\begin{poetry}
\poeml My heart is broken within me, \\
\poemll    and all my bones shake. \\
\poeml I'm like a drunk man, \\
\poemll    like a person overcome with wine, \\
\poeml because of the \divine{Lord}, \\
\poemll    and because of his holy words. \\
\poeml \v{10}Indeed, the land is full of adulterers. \\
\poemll    Indeed, the land mourns because of the curse; \\
\poemlll       the pastures of the wilderness have dried up. \\
\poeml The adulterers'\fnote{\fbackref{23:10} Lit. \fbib{Their}} lifestyles are evil, \\
\poemll    and they use\fnote{\fbackref{23:10} The Heb. lacks \fbib{they use}} their strength for what\fnote{\fbackref{23:10} The Heb. lacks \fbib{for what}} is not right. \\
\poeml \v{11}Indeed, both priest and prophet are ungodly. \\
\poemll    Even in my house I find evil,'' declares the \divine{Lord}. \\
\poeml \v{12}Therefore their way will be slippery. \\
\poemll    They'll be driven out into the darkness, \\
\poemlll       where they'll fall. \\
\poeml For I'll bring disaster on them, \\
\poemll    the year of their judgment,'' \\
\poemlll       declares the \divine{Lord}. \\
\poeml \v{13}``Among the prophets of Samaria I saw a disgusting thing, \\
\poemll    for they prophesied by Baal \\
\poemlll       and led my people Israel astray. \\
\poeml \v{14}Among the prophets of Jerusalem I saw a horrible thing, \\
\poemll    for they commit adultery and live a lie. \\
\poeml They strengthen the hands of those who do evil, \\
\poemll    so that no one repents of his evil. \\
\poeml All of them are like Sodom to me, \\
\poemll    and her\fnote{\fbackref{23:14} I.e. Jerusalem's} residents like Gomorrah.''
\end{poetry}

\v{15}Therefore, this is what the \divine{Lord} God of the Heavenly Armies says about the prophets,

\begin{poetry}
\poeml ``I'm about to make them eat wormwood\fnote{\fbackref{23:15} \fbib{Wormwood} is a plant with an extremely bitter taste} \\
\poemll    and drink poisoned water, \\
\poeml because godlessness has spread from the \\
\poemll    prophets of Jerusalem throughout the land.'' \\
\poeml \v{16}This is what the \divine{Lord} of the Heavenly Armies says: \\
\poeml ``Don't listen to the words of the prophets \\
\poemll    who are prophesying to you; \\
\poemlll       they're giving you false hopes. \\
\poeml They declare visions from their own minds--- \\
\poemll    they don't come from the \divine{Lord}!\fnote{\fbackref{23:16} Lit. \fbib{not from the mouth of the \divine{Lord}}} \\
\poeml \v{17}They keep on saying to those who despise me, \\
\poemll    `The \divine{Lord} has said, ``You will have peace.''\,' \\
\poeml To all who stubbornly follow their own desires\fnote{\fbackref{23:17} Lit. \fbib{walk in the stubbornness of their heart}} they say, \\
\poemll    `Disaster won't come upon you.' \\
\poeml \v{18}But who has stood in the \divine{Lord}'s council \\
\poemll    to see and hear his message? \\
\poemlll       Who has paid attention to his message and obeyed it?\fnote{\fbackref{23:18} Or \fbib{listened to}} \\
\poeml \v{19}Look, the storm of the \divine{Lord}'s wrath has gone forth, \\
\poemll    a whirling tempest, \\
\poeml and it will swirl down \\
\poemll    around the head of the wicked. \\
\poeml \v{20}The \divine{Lord}'s anger won't turn back \\
\poemll    until he has accomplished \\
\poemlll       what he intended to do. \\
\poeml In the future \\
\poemll    you will clearly understand it. \\
\poeml \v{21}I didn't send these prophets,\fnote{\fbackref{23:21} Lit. \fbib{prophets}} \\
\poemll    but they ran anyway. \\
\poeml I didn't speak to them, \\
\poemll    but they prophesied. \\
\poeml \v{22}If they had stood in my council \\
\poemll    and had delivered my words to my people, \\
\poeml then they would have turned them back \\
\poemll    from their evil way, \\
\poemlll       from their evil deeds.'' \\
\poeml \v{23}``Am I a God who is near,'' declares the \divine{Lord}, \\
\poemll    ``rather than a God who is far away? \\
\poeml \v{24}If a person hides himself in secret places, \\
\poemll    will I not see him?'' \\
\poemlll       declares the \divine{Lord}. \\
\poeml ``I fill the heavens and the earth, do I not?'' \\
\poemll    declares the \divine{Lord}.
\end{poetry}

\v{25}``I've heard what the prophets who prophesy lies in my name have said: `I had a dream; I had a dream.' \v{26}How long will this go on?\fnote{\fbackref{23:26} The Heb. lacks \fbib{will this go on}} Is there anything\fnote{\fbackref{23:26} The Heb. lacks \fbib{anything}} in the hearts of the prophets who prophesy lies, and who prophesy from the deceit that is in their hearts? \v{27}With their dreams that they relate to one another,\fnote{\fbackref{23:27} Lit. \fbib{each to his colleague}} they plan to make my people forget my name just as their ancestors forgot my name by embracing\fnote{\fbackref{23:27} The Heb. lacks \fbib{embracing}} Baal. \v{28}Let the prophet who has a dream relate the dream, but let whoever receives my message\fnote{\fbackref{23:28} Lit. \fbib{my word is with him}} speak my message truthfully. What does straw have in common with wheat?'' declares the \divine{Lord}. \v{29}``My message is like fire or like a hammer that shatters rock, is it not?'' declares the \divine{Lord}.

\v{30}``Therefore, look, I'm against the prophets,'' declares the \divine{Lord}, ``who steal my words from each other. \v{31}Look, I'm against the prophets,'' declares the \divine{Lord}, ``who use their tongues to issue a declaration.\fnote{\fbackref{23:31} I.e. a message that they claim came from God} \v{32}Look, I'm against those who prophesy based on false dreams,'' declares the \divine{Lord}, ``and relate them and lead my people astray with their lies and their recklessness. I didn't send them; I didn't command them, and they provide no benefit at all to these people,'' declares the \divine{Lord}.
\passage{The Oracle-Burden\fnote{\fbackref{23:33} An \fbib{oracle} is a message that claims to be a revelation from the \fbib{\divine{Lord}}. The same Heb. word means both \fbib{oracle} and \fbib{burden}, and this entire section invokes a word play between the two meanings of this Heb. word.} of the \divine{Lord}}

\v{33}``Jeremiah,\fnote{\fbackref{23:33} The Heb. lacks \fbib{Jeremiah}} when these people, the prophet, or a priest ask you,\fnote{\fbackref{23:33} I.e. the prophet Jeremiah; MT is masculine sing.} `What is the oracle\fnote{\fbackref{23:33} Or \fbib{burden}} of the \divine{Lord}?' say to them, `You are the burden,\fnote{\fbackref{23:33} Or \fbib{oracle}; MT reads \fbib{What oracle?}} and I'll cast you out,'\,'' declares the \divine{Lord}. \v{34}``As for the prophet, the priest, or the people who say, `I have\fnote{\fbackref{23:34} The Heb. lacks \fbib{I have}} an oracle of the \divine{Lord},' I'll judge that person and his household. \v{35}This is what you should say to one another and among yourselves,\fnote{\fbackref{23:35} Lit. \fbib{each to his neighbor and each to his brother}} `What has the \divine{Lord} answered?' or `What has the \divine{Lord} said?' \v{36}But you are to no longer mention\fnote{\fbackref{23:36} Or \fbib{remember}} the oracle of the \divine{Lord}, because the oracle is only for the person to whom the \divine{Lord} gives his message,\fnote{\fbackref{23:36} Lit. \fbib{for the person of his word}} and you have overturned the words of the living God, the \divine{Lord} of the Heavenly Armies, our God. \v{37}This is what you should say to the prophet, `What has the \divine{Lord} answered?' or `What has the \divine{Lord} said?' \v{38}Since you're saying, `We have an oracle of the \divine{Lord},'\fnote{\fbackref{23:38} The Heb. lacks \fbib{of the \divine{Lord}}} therefore this is what the \divine{Lord} says: He will answer your message with this message, `Burden\fnote{\fbackref{23:38} Or \fbib{Oracle}} of the \divine{Lord},' and I'll send you away with these words: `Don't say, ``Oracle of the \divine{Lord}.''\,' \v{39}Therefore I'll surely forget you and cast you and the city I gave you and your ancestors out of my presence. \v{40}I'll bring on you everlasting reproach and everlasting humiliation that won't ever\fnote{\fbackref{23:40} The Heb. lacks \fbib{ever}} be forgotten.''
\labelchapt{24}
\passage{Two Baskets of Figs}

\chapt{24}
\v{1}After Nebuchadnezzar, king of Babylon, had taken Jehoiakim's son Jeconiah,\fnote{\fbackref{24:1} I.e. Jehoiachin} king of Judah, along with the officials\fnote{\fbackref{24:1} Or \fbib{princes}} of Judah, the craftsmen, and the smiths from Jerusalem into exile, and had brought them to Babylon, the \divine{Lord} showed me two baskets of figs placed right in front of the Temple of the \divine{Lord}. \v{2}One basket contained very good figs like the first figs that ripen on the tree. The other basket contained very bad figs that were too bad to be eaten. \v{3}The \divine{Lord} told me, ``What do you see?''

I replied, ``Figs. The good figs are very good, and the bad figs are very bad. They're too bad to be eaten.''

\v{4}Then this message from the \divine{Lord} came to me: \v{5}``This is what the \divine{Lord} God of Israel says: `Like these good figs, so I'll regard as good the exiles of Judah whom I sent from this place to the land of the Chaldeans. \v{6}I'll look at them with good intentions, and I'll bring them back to this land. I'll build them up. I won't tear them down; I'll plant them and not rip them up. \v{7}I'll give them the ability\fnote{\fbackref{24:7} Lit. \fbib{them a heart}} to know me, for I am the \divine{Lord}. They will be my people, and I will be their God when they return to me with all their heart.

\v{8}```Like the bad figs that are too bad to be eaten---for this is what the \divine{Lord} says---so I'll give up on Zedekiah king of Judah, along with his officials, the remnant of Jerusalem that is left in this land, and those living in the land of Egypt. \v{9}I'll make them into a horrifying sight to all the kingdoms of the earth; into a cause for contempt, into a byword, into a taunt, and into a curse in all the places to which I drive them. \v{10}I'll send the sword, famine, and plague against them until they're completely destroyed from the land which I gave them and their ancestors.'\,''
\labelchapt{25}
\passage{The Irrevocable Judgment on Judah}

\chapt{25}
\v{1}This message from the \divine{Lord} came to Jeremiah concerning all the people of Judah in the fourth year of Josiah's son Jehoiakim, king of Judah. (This was also the first year of the reign of\fnote{\fbackref{25:1} The Heb. lacks \fbib{the reign of}} King Nebuchadnezzar of Babylon.) \v{2}This is what Jeremiah the prophet told all the people of Judah and all the residents of Jerusalem: \v{3}``From the thirteenth year of the reign of\fnote{\fbackref{25:3} The Heb. lacks \fbib{the reign of}} Ammon's son Josiah, the king of Judah, until the present time, for 23 years this message from the \divine{Lord} has come to me, and I've spoken to you again and again,\fnote{\fbackref{25:3} Lit. \fbib{getting up early and speaking}} but you haven't listened. \v{4}Again and again,\fnote{\fbackref{25:4} Lit. \fbib{getting up early and sending}} the \divine{Lord} sent all his servants, the prophets, to you, but you wouldn't listen or even turn your ears in my direction to hear. \v{5}They said, `Turn, each one of you, from your\fnote{\fbackref{25:5} Lit. \fbib{his}} evil habits\fnote{\fbackref{25:5} Lit. \fbib{ways}} and evil deeds, and live in the land that the \divine{Lord} gave to you and your ancestors forever and ever. \v{6}Don't follow other gods to serve and worship them. Don't provoke me with the idols\fnote{\fbackref{25:6} Lit. \fbib{works}} you make with your hands, and I won't bring disaster on you.' \v{7}But you didn't listen to me,'' declares the \divine{Lord}, ``so as to provoke me with the idols\fnote{\fbackref{25:7} Lit. \fbib{works}} you make with your hands to your own harm.

\v{8}``Therefore, this is what the \divine{Lord} of the Heavenly Armies says: `Because you haven't listened to my message, \v{9}I'm now sending for all the tribes from the north, declares the \divine{Lord}, and for my servant Nebuchadnezzar king of Babylon. I'll bring them against this land, against its inhabitants, and against all these surrounding nations. I'll utterly destroy them and make them an object of horror and scorn,\fnote{\fbackref{25:9} Lit. \fbib{hissing}; i.e. a sign of mocking and contempt} and an everlasting desolation. \v{10}I'll destroy the sounds of gladness and rejoicing from them, the sounds of the bridegroom and the bride, the sound of the hand mill and also the light of the lamp. \v{11}This entire land will be a desolation and a waste, and these nations will serve the king of Babylon for seventy years.

\v{12}`Then when the seventy years have passed, I'll judge the king of Babylon and that nation, declares the \divine{Lord}, I'll judge\fnote{\fbackref{25:12} The Heb. lacks \fbib{I'll judge}} the land of the Chaldeans for their iniquity and I'll make it a desolation forever. \v{13}I'll bring on that land all the things I spoke against it, all that is written in this book, which Jeremiah prophesied about the nations. \v{14}Indeed many nations and great kings will make slaves even of them, and I'll repay them according to their deeds, according to what they have done.'\,''
\passage{Judgment on the Nations}

\v{15}For this is what the \divine{Lord} God of Israel says to me, ``Take this cup of the wine of burning anger from my hand and make all the nations to whom I send you drink it. \v{16}They'll drink, stagger, and act like madmen because of the sword I'm sending among them.'' \v{17}So I took the cup from the \divine{Lord}'s hand, and I made all the nations to whom the \divine{Lord} sent me drink it: \v{18}Jerusalem, the cities of Judah, its kings and officials\fnote{\fbackref{25:18} Or \fbib{princes}} to make them into a ruin, an object of horror and scorn,\fnote{\fbackref{25:18} Lit. \fbib{hissing}; i.e. hissing was a sign of ridicule and contempt} and a curse, as it is this day; \v{19}Pharaoh, king of Egypt, his officials,\fnote{\fbackref{25:19} Or \fbib{servants}} his princes, and all his people; \v{20}all the various people;\fnote{\fbackref{25:20} Or \fbib{the mixed company}} all the kings of the land of Uz, all the kings of the land of the Philistines, Ashkelon, Gaza, Ekron, and what remains of Ashdod; \v{21}Edom, Moab, and the people of Ammon; \v{22}all the kings of Tyre, all the kings of Sidon, and all the kings of the coast lands that are beyond the sea; \v{23}Dedan, Tema, Buz, and those who shave the corners of their beards;\fnote{\fbackref{25:23} Lit. \fbib{cut off the side}} \v{24}all the kings of Arabia and all the kings of the various people\fnote{\fbackref{25:24} Or \fbib{the mixed company}} who live in the desert; \v{25}all the kings of Zimri, all the kings of Elam, and all the kings of Media; \v{26}all the kings of the north near and far, one after another, and all the kingdoms of the world on the face of the earth. The king of Sheshak\fnote{\fbackref{25:26} \fbib{Sheshak} is a cryptogram for Babylon} will drink after all the others.\fnote{\fbackref{25:26} Lit. \fbib{after them}}

\v{27}``You are to say to them, `This is what the \divine{Lord} of the Heavenly Armies, the God of Israel, says: ``Drink, get drunk, and vomit! Fall down and don't get up because of the sword I'm sending among you.''\,' \v{28}And if they refuse to take the cup from your hand to drink it, say to them, `This is what the \divine{Lord} of the Heavenly Armies says: ``You will surely drink it! \v{29}Look, I'm beginning to bring disaster on the city that is called by my name, and do you actually think you will avoid punishment? You won't avoid punishment because I'm summoning the sword against all those who live in the land,'' declares the \divine{Lord} of the Heavenly Armies.'\,''
\passage{The \divine{Lord} will Judge the Nations}

\v{30}``You are to prophesy all these things against them, and you are to say to them,

\begin{poetry}
\poeml `The \divine{Lord} roars from his high place, \\
\poemll    from his holy dwelling he lifts his voice. \\
\poeml He roars loudly against his flock,\fnote{\fbackref{25:30} Or \fbib{habitation}} \\
\poemll    and against all who live on the earth; \\
\poemlll       he shouts like those treading grapes.\fnote{\fbackref{25:30} The Heb. lacks \fbib{grapes}} \\
\poeml \v{31}A tumult reaches to the ends of the earth \\
\poemll    because the \divine{Lord} is bringing an indictment against the nations. \\
\poeml He judges all flesh. \\
\poemll    He has given the wicked over to the sword,' \\
\poemlll       declares the \divine{Lord}. \\
\poeml \v{32}`This is what the \divine{Lord} of the Heavenly Armies says: \\
\poemll    ``Look, disaster is going from nation to nation, \\
\poeml a great storm is being stirred up \\
\poemll    from the most distant parts of the earth.
\end{poetry}

\v{33}``Those slain by the \divine{Lord} on that day will extend\fnote{\fbackref{25:33} Lit. \fbib{will be}} from one end of the earth to the other. They won't be mourned for or gathered up or buried. They'll be like dung on the surface of the ground.

\begin{poetry}
\poeml \v{34}``Scream, you shepherds! Cry out! \\
\poemll    Roll in the dust, you leaders of the flock! \\
\poeml Indeed, the time for your slaughter \\
\poemll    and your dispersion has arrived, \\
\poemlll       and you will break like a choice vessel. \\
\poeml \v{35}Flight will be impossible\fnote{\fbackref{25:35} Lit. \fbib{will perish}} for the shepherds, \\
\poemll    as will be escape for the leaders of the flock. \\
\poeml \v{36}A sound---it's the cry of the shepherds \\
\poemll    and the scream of the leaders of the flock--- \\
\poemlll       because the \divine{Lord} is destroying their pastures. \\
\poeml \v{37}The peaceful meadows are silent \\
\poemll    because of the \divine{Lord}'s fierce anger. \\
\poeml \v{38}Like a lion, he has left his den.\fnote{\fbackref{25:38} Or \fbib{thicket}} \\
\poemll    Indeed, their land has become a waste \\
\poeml because of the anger of the oppressor \\
\poemll    and because of the \divine{Lord}'s\fnote{\fbackref{25:38} Lit. \fbib{his}} fierce anger.''
\end{poetry}
\labelchapt{26}
\passage{Jeremiah is Arrested}

\chapt{26}
\v{1}In the beginning of the reign of Josiah's son Jehoiakim, king of Judah, this message came from the \divine{Lord}: \v{2}``This is what the \divine{Lord} says: `Stand in the courtyard of the \divine{Lord}'s Temple and tell those from all the cities\fnote{\fbackref{26:2} Lit. \fbib{speak to all the cities}} of Judah who are coming to worship at the \divine{Lord}'s Temple everything that I've commanded you to say to them. Don't leave out a word! \v{3}Perhaps they'll listen, and each of them will repent from his evil way. Then I'll change my mind about the disaster I'm planning to bring on\fnote{\fbackref{26:3} Lit. \fbib{do to}} them because of their evil deeds. \v{4}Say to them, ``This is what the \divine{Lord} says: `If you don't listen to me to follow my Law which I've set before you, \v{5}and listen to the words of my servants, the prophets, whom I've sent to you over and over\fnote{\fbackref{26:5} Lit. \fbib{getting up early to send}}---but you wouldn't listen--- \v{6}then I'll make this house like Shiloh and make this city into a curse to all the nations of the earth.'\,''\,'\,''
\passage{Jeremiah Threatened with Death}

\v{7}The priests, the prophets, and all the people listened as Jeremiah spoke these words at the \divine{Lord}'s Temple. \v{8}As soon as Jeremiah finished saying everything that the \divine{Lord} had commanded him to say to all the people, the priests, the prophets, and all the people seized him, telling him as they did: ``You must certainly die! \v{9}Why have you prophesied in the name of the \divine{Lord} that this house will be like Shiloh, and this city will be without an inhabitant?'' Then all the people gathered around Jeremiah at the \divine{Lord}'s Temple.

\v{10}When the Judean officials\fnote{\fbackref{26:10} Or \fbib{princes}} heard all these things, they came up from the king's house to the \divine{Lord}'s Temple and sat in the doorway of the New Gate of the \divine{Lord}'s Temple.\fnote{\fbackref{26:10} The Heb. lacks \fbib{temple}} \v{11}The priests and prophets told the officials and all the people, ``A death sentence for this man, because he prophesied against this city, as you heard with your own ears!''

\v{12}Then Jeremiah spoke to all the officials and to all the people: ``The \divine{Lord} has sent me to prophesy all the things you heard against this house and against this city. \v{13}Now, change your habits\fnote{\fbackref{26:13} Lit. \fbib{ways}} and your deeds and obey the \divine{Lord} your God, and the \divine{Lord} will change his mind about the disaster that he told you about. \v{14}Look, I'm in your hands, so do with me what you think is good and right. \v{15}But know for certain that if you kill me, you will bring innocent blood on yourselves and on this city and its residents because the \divine{Lord} really did send me to you to say all these things for you to hear.''
\passage{The Elders Remember Micah's Similar Message}

\v{16}The officials and all the people told the priests and the prophets, ``No death sentence for this man because he has spoken to us in the name of the \divine{Lord} our God.''

\v{17}Some of the elders of the land got up and told all the assembled people, \v{18}``Micah of Moresheth prophesied during the reign\fnote{\fbackref{26:18} Lit. \fbib{time}} of Hezekiah king of Judah to all the people of Judah, `This is what the \divine{Lord} of the Heavenly Armies says:

\begin{poetry}
\poeml ``Zion will be a plowed field, \\
\poemll    and Jerusalem a ruin. \\
\poemlll       The Temple Mount will be a wooded hill.''\,'\fnote{\fbackref{26:18} Or \fbib{a wooded high place}}
\end{poetry}

\v{19}``Did Hezekiah king of Judah or anyone in Judah kill him? Didn't he fear the \divine{Lord} and seek the \divine{Lord}'s favor, and so the \divine{Lord} changed his mind about the disaster that he had spoken to them about. We're bringing great disaster on ourselves. \v{20}There was also a man named Uriah, Shemaiah's son from Kiriath-jearim, who prophesied in the \divine{Lord}'s name. He prophesied about this city and this land in words similar to those of Jeremiah. \v{21}King Jehoiakim, all his troops, and all the officials heard his words, and the king sought to kill him. Uriah heard about this and was afraid, so he fled and went to Egypt. \v{22}King Jehoiakim sent men to Egypt. He sent\fnote{\fbackref{26:22} The Heb. lacks \fbib{He sent}} Achbor's son Elnathan, along with a contingent of men\fnote{\fbackref{26:22} Lit. \fbib{Achbor and men with him}} into Egypt. \v{23}They brought Uriah out of Egypt and brought him to King Jehoiakim, who killed him with a sword. Then they threw his body into a common grave.\fnote{\fbackref{26:23} Lit. \fbib{a grave of the sons of the people}}''

\v{24}Yet because Shaphan's son Ahikam supported Jeremiah,\fnote{\fbackref{26:24} Lit. \fbib{the hand of Shaphan's son Ahikam was with Jeremiah}} he was not handed over to the people for them to kill.
\labelchapt{27}
\passage{Jeremiah Tells the Nations to Submit to Babylon}

\chapt{27}
\v{1}At the beginning of the reign of Josiah's son Jehoiakim, king of Judah, this message came to Jeremiah from the \divine{Lord}: \v{2}this is what the \divine{Lord} says to me: ``Make restraints and yokes for yourself and put them on your neck. \v{3}Then send messengers\fnote{\fbackref{27:3} Lit. \fbib{them}} to the king of Edom, the king of Moab, the king of the Ammonites, the king of Tyre, and the king of Sidon through the envoys\fnote{\fbackref{27:3} Or \fbib{messengers}} who come to Jerusalem to king Zedekiah of Judah. \v{4}Give them this order for their masters: `This is what the \divine{Lord} of the Heavenly Armies, the God of Israel, says, and this is what you are to say to your masters, \v{5}``By my great power and outstretched arm I made the earth, mankind, and the animals that are on the face of the earth, and I give it to whomever I see fit.\fnote{\fbackref{27:5} Or \fbib{to whoever is upright in my eyes}} \v{6}Now I've given all these lands to my servant, Nebuchadnezzar king of Babylon, and I've even given him the wild animals to serve him. \v{7}All the nations will serve him, his son, and his grandson until his country's time also comes, and then many nations and great kings will use him as a slave. \v{8}If a nation and kingdom does not serve him---King Nebuchadnezzar of Babylon---and does not put its neck under the yoke of the king of Babylon, I'll judge that nation with the sword, with famine, and with plague,'' declares the \divine{Lord}, ``until I've completely destroyed it by his hand. \v{9}You aren't to listen to your prophets, your diviners, your dreamers,\fnote{\fbackref{27:9} Lit. \fbib{your dreams}} your soothsayers, and your sorcerers who say to you, `Don't serve the king of Babylon.' \v{10}They're prophesying a lie to you in order to remove you far away from your land. I'll drive you out and you will perish. \v{11}But I'll let the nation that brings its neck under the yoke of the king of Babylon and serves him remain in its own land,'' declares the \divine{Lord}, ``and they'll work it and remain in it.''\,'\,''
\passage{Zedekiah Told to Submit to Babylon}

\v{12}I spoke to Zedekiah king of Judah using words like these: ``Bring your neck under the yoke of the king of Babylon. Serve him and his people, and you will live! \v{13}Why should you and your people die by the sword, by famine, and by plague as the \divine{Lord} has decreed about the nation that does not serve the king of Babylon? \v{14}Don't listen to the words of the prophets who say to you, `You won't serve the king of Babylon.' Indeed, they're prophesying a lie to you. \v{15}For I didn't send them,'' declares the \divine{Lord}, ``and they're falsely prophesying in my name, so I will drive both you and the prophets who prophesy to you out of the land.''
\passage{The People and Priests Told to Submit to Babylon}

\v{16}Then I spoke to the priests and all this people: ``This is what the \divine{Lord} says: `Don't listen to the words of the prophets who prophesy to you: ``The vessels from the Temple are about to be returned from Babylon very soon now.'' Indeed, they're prophesying a lie to you. \v{17}Don't listen to them! Serve the king of Babylon and you'll live. Why should this city become a ruin? \v{18}If they're prophets, and if they have a message from the \divine{Lord}, let them plead with the \divine{Lord} of the Heavenly Armies so that the utensils that remain in the \divine{Lord}'s Temple, in the house of the king of Judah, and in Jerusalem might not be taken to Babylon. \v{19}For this is what the \divine{Lord} of the Heavenly Armies says about the pillars, the bronze sea, the stands, and the rest of the vessels that remain in this city \v{20}that Nebuchadnezzar king of Babylon didn't take when he took Jehoiakim's son Jeconiah, king of Judah, and all the nobles of Judah and Jerusalem from Jerusalem into exile to Babylon--- \v{21}For this is what the \divine{Lord} of the Heavenly Armies, the God of Israel says about the vessels that remain in the \divine{Lord}'s Temple, in the house of the king of Judah, and in Jerusalem, \v{22}``They'll go into Babylon and there they'll remain until the time I take note of them,'' declares the \divine{Lord}. ``Then I'll bring them up and return them to this place.''\,'\,''
\labelchapt{28}
\passage{Jeremiah Challenges a False Prophet}

\chapt{28}
\v{1}In that same year, in the beginning of the reign of Zedekiah, king of Judah, in the fourth year and the fifth month, Azzur's son Hananiah, the prophet from Gibeon, told me at the \divine{Lord}'s Temple in front of the priests and all the people, \v{2}``This is what the \divine{Lord} of the Heavenly Armies, the God of Israel, says: `I've broken the yoke of the king of Babylon, \v{3}and within two years I'll bring back to this place all the vessels of the \divine{Lord}'s Temple that Nebuchadnezzar king of Babylon took from this place and carried to Babylon. \v{4}I'll bring back Jehoiakim's son Jeconiah, king of Judah, and all the exiles of Judah who went to Babylon to this place,' declares the \divine{Lord}, `for I'll break the yoke of the king of Babylon.'\,''

\v{5}The prophet Jeremiah spoke to the prophet Hananiah in front of the priests and all\fnote{\fbackref{28:5} Lit. \fbib{and in front of}} the people who were standing in the \divine{Lord}'s Temple. \v{6}The prophet Jeremiah said, ``May the \divine{Lord} truly do this thing! May the \divine{Lord} fulfill the words\fnote{\fbackref{28:6} Lit. \fbib{your words}} that you prophesied to bring back the vessels of the \divine{Lord}'s Temple and all the exiles from Babylon to this place. \v{7}But please listen to what I'm saying in your hearing and in the hearing of all the people. \v{8}The prophets who came before us\fnote{\fbackref{28:8} Lit. \fbib{before me and before you}} from ancient times prophesied war, famine, and plague against many lands and great kingdoms. \v{9}When a prophet prophesies peace, and what the prophet speaks comes about, he will be known as the prophet whom the \divine{Lord} has truly sent.''

\v{10}Then the prophet Hananiah took the yoke\fnote{\fbackref{28:10} Lit. \fbib{the bar of the yoke}} from the neck of Jeremiah the prophet and broke it. \v{11}Hananiah, in front of all the people, said, ``This is what the \divine{Lord} says: `In the same way, within two years, I'll break the yoke of Nebuchadnezzar king of Babylon from the neck of all the nations.'\,'' Then Jeremiah the prophet went on his way.

\v{12}This message from the \divine{Lord} came to Jeremiah after the prophet Hananiah had broken the yoke\fnote{\fbackref{28:12} Lit. \fbib{the bar of the yoke}} from the neck of Jeremiah the prophet: \v{13}``Go and say to Hananiah, `This is what the \divine{Lord} says: ``You have broken wooden yokes,\fnote{\fbackref{28:13} Lit. \fbib{the bars of the yoke}} but you have made iron yokes\fnote{\fbackref{28:13} Lit. \fbib{the bars of the yoke}} in their place.'' \v{14}For this is what the \divine{Lord} of the Heavenly Armies, the God of Israel, says: ``I've put an iron yoke on the necks of all these nations to serve Nebuchadnezzar king of Babylon. They'll serve him, and I've even given the wild animals to him.''\,'\,''

\v{15}The prophet Jeremiah told the prophet Hananiah, ``Listen, Hananiah! The \divine{Lord} didn't send you, and you are causing these people to trust in a lie. \v{16}Therefore, this is what the \divine{Lord} says: `I'm about to remove\fnote{\fbackref{28:16} Lit. \fbib{send you away}} you from the face of the earth. This year you will die because you have preached rebellion against the \divine{Lord}.'\,''

\v{17}So the prophet Hananiah died in the seventh month of that year.
\labelchapt{29}
\passage{Jeremiah's Letter to the Exiles}

\chapt{29}
\v{1}These are the words of the letter that the prophet Jeremiah sent from Jerusalem to the remaining elders among the exiles, to the priests, to the prophets, and to all the people whom Nebuchadnezzar had taken into exile from Jerusalem to Babylon, \v{2}after King Jeconiah, the queen mother, the palace officials,\fnote{\fbackref{29:2} Or \fbib{eunuchs}} the officials\fnote{\fbackref{29:2} Or \fbib{princes}} of Judah and Jerusalem, the craftsmen, and the smiths left Jerusalem. \v{3}The letter was sent by Shaphan's son Elasah and by Hilkiah's son Gemariah, whom King Zedekiah of Judah sent to Nebuchadnezzar king of Babylon in Babylon, and it said, \v{4}``This is what the \divine{Lord} of the Heavenly Armies, the God of Israel, says to all the exiles who were taken from Jerusalem into exile to Babylon, \v{5}`Build houses and live in them.\fnote{\fbackref{29:5} The Heb. lacks \fbib{in them}} Plant gardens and eat their produce. \v{6}Take wives and father sons and daughters. Take wives for your sons and give your daughters in marriage, so they may have sons and daughters. Increase in numbers there, don't decrease. \v{7}Seek the welfare of the city to which I've exiled you and pray to the \divine{Lord} for it, for your welfare depends on its welfare.'\fnote{\fbackref{29:7} Lit. \fbib{for in its welfare is your welfare}} \v{8}For this is what the \divine{Lord} of the Heavenly Armies, the God of Israel, says: `Don't let the prophets and diviners\fnote{\fbackref{29:8} Lit. \fbib{your prophets and your diviners}} who are among you deceive you, and don't listen to them when they tell you their dreams.\fnote{\fbackref{29:8} Lit. \fbib{to your dreams that you cause to be dreamed}} \v{9}Indeed, they're prophesying lies to you in my name. I didn't send them,' declares the \divine{Lord}.

\v{10}``For this is what the \divine{Lord} says: `When Babylon's seventy years are completed, I'll take note of you and will fulfill my good promises to you by bringing you back to this place. \v{11}For I know the plans that I have for you,' declares the \divine{Lord}, `plans for well-being, and not for calamity, in order to give you a future and a hope. \v{12}When you call out to me and come and pray to me, I'll hear you. \v{13}You will seek me and find me when you search for me with all your heart. \v{14}I'll be found by you,' declares the \divine{Lord}, `and I'll restore your security\fnote{\fbackref{29:14} Or \fbib{captivity}} and gather you from all the nations and all the places to which I've driven you,' declares the \divine{Lord}. `I'll bring you back to the place from which I sent you into exile.'

\v{15}``Indeed, you have said, `The \divine{Lord} has raised up prophets for us in Babylon.'

\v{16}``But this is what the \divine{Lord} says about the king who sits on David's throne, and about the people who live in this city---your brothers who didn't go with you into exile: \v{17}This is what the \divine{Lord} says: `I'm about to send the sword, famine, and plague on them, and I'll make them like rotten figs that cannot be eaten because they're so bad. \v{18}I'll pursue them with the sword, with famine, and with plague, and I'll make them a horrifying sight to all the kingdoms of the earth. I'll make them\fnote{\fbackref{29:18} The Heb. lacks \fbib{I'll make them}} a curse, an object of horror, and scorn,\fnote{\fbackref{29:18} Lit. \fbib{hissing}; i.e. hissing was a sign of mocking and contempt} and a desolation in all the nations to which I've driven them, \v{19}because they didn't listen to my words,' declares the \divine{Lord}. `When I sent my servants, the prophets, to you again and again,\fnote{\fbackref{29:19} Lit. \fbib{getting up early and sending}} you didn't listen,' declares the \divine{Lord}.

\v{20}``Now, all you exiles whom I sent from Jerusalem to Babylon, listen to this message from the \divine{Lord}! \v{21}This is what the \divine{Lord} of the Heavenly Armies, the God of Israel, says about Kolaiah's son Ahab and Maaseiah's son Zedekiah, who are prophesying lies to you in my name, `I'm about to give them into the domination\fnote{\fbackref{29:21} Lit. \fbib{hand}} of Nebuchadnezzar king of Babylon, and he will kill them before your eyes. \v{22}What happens to them will be the basis for a curse\fnote{\fbackref{29:22} Lit. \fbib{From them a curse will be taken}} for all the Judean exiles who are in Babylon. People will say,\fnote{\fbackref{29:22} Lit. \fbib{Saying}} ``May the \divine{Lord} make you like Zedekiah and Ahab, whom the \divine{Lord} roasted\fnote{\fbackref{29:22} MT word for \fbib{roasted} sounds like MT word for \fbib{curse}} in the fire, \v{23}because they did something stupid\fnote{\fbackref{29:23} Lit. \fbib{they committed folly}} in Israel. They committed adultery with their neighbors' wives, and in my name they spoke lies that I didn't command them. I'm the one who knows, and I'm a witness,'' declares the \divine{Lord}.'\,''
\passage{A Rebuke to Shemaiah}

\v{24}``You are to say to Shemaiah of Nehelam: \v{25}`This is what the \divine{Lord} of the Heavenly Armies, the God of Israel, says: ``Because you sent letters in your own name to all the people who are in Jerusalem, to Maaseiah's son Zephaniah the priest and to all the priests--- \v{26}The \divine{Lord} made you a priest instead of Jehoiada the priest to serve in the \divine{Lord}'s Temple as an official against every crazy prophet, and to put him in stocks and restraints. \v{27}And now, why didn't you rebuke Jeremiah from Anathoth who prophesies to you? \v{28}So he sent a message\fnote{\fbackref{29:28} The Heb. lacks \fbib{a message}} to us in Babylon: `The exile\fnote{\fbackref{29:28} Lit. \fbib{It}} will be long, so build houses and live in them.\fnote{\fbackref{29:28} The Heb. lacks \fbib{in them}} Plant gardens and eat their produce.'\,''\,'\,''

\v{29}Then Zephaniah the priest read this letter to Jeremiah the prophet, \v{30}and this message from the \divine{Lord} came to Jeremiah: \v{31}``Send a message to all the exiles: `This is what the \divine{Lord} says about Shemaiah from Nehelam, ``Because Shemaiah has prophesied to you, even though I didn't send him, and has made you trust a lie,'' \v{32}therefore, this is what the \divine{Lord} says: ``I'm about to judge Shemaiah from Nehelam along with his descendants. He won't have anyone related to him\fnote{\fbackref{29:32} The Heb. lacks \fbib{related to him}} living among these people. Nor will he see the good that I'll do for my people,'' declares the \divine{Lord}, ``because he advocated rebellion against the \divine{Lord}.''\,'\,''
\labelchapt{30}
\passage{A Message of Consolation}

\chapt{30}
\v{1}This message came from the \divine{Lord} to Jeremiah: \v{2}``This is what the \divine{Lord} God of Israel says: `Write all the words that I've spoken to you in a book. \v{3}Indeed, the time\fnote{\fbackref{30:3} Lit. \fbib{days}} will come,' declares the \divine{Lord}, `when I'll restore the security of my people Israel and Judah,' says the \divine{Lord}. `I'll bring them back to the land that I gave to their ancestors, and they'll possess it.'\,''

\v{4}These are the words that the \divine{Lord} spoke about Israel and Judah:

\begin{poetry}
\poeml \v{5}``Indeed, this is what the \divine{Lord} says: \\
\poeml `We have heard a sound of terror \\
\poemll    and of fear, and there is no peace. \\
\poeml \v{6}Ask about this and think about it--- \\
\poemll    Can a man give birth to a child? \\
\poeml Why then do I see every strong man \\
\poemll    with his hands on his thighs \\
\poeml like a woman giving birth, \\
\poemll    and all their faces have turned pale? \\
\poeml \v{7}Oh how terrible! That time\fnote{\fbackref{30:7} Lit. \fbib{day}} will be worse \\
\poemll    than any like it. \\
\poeml It will be a time of trouble for Jacob, \\
\poemll    but he will be rescued from it. \\
\poeml \v{8}On that day,' declares the \divine{Lord} \\
\poemll    of the Heavenly Armies, \\
\poeml `I'll break the yoke\fnote{\fbackref{30:8} Lit. \fbib{his yoke}} from your neck \\
\poemll    and will tear off your restraints.\fnote{\fbackref{30:8} Or \fbib{cords}} \\
\poemlll       Foreigners will no longer make you\fnote{\fbackref{30:8} Lit. \fbib{him} (i.e. Jacob)} serve them.\fnote{\fbackref{30:8} I.e. enslave you} \\
\poeml \v{9}Rather, they will serve the \divine{Lord} their God \\
\poemll    and David their king, \\
\poemlll       whom I will raise up for them. \\
\poeml \v{10}`My servant Jacob, don't be afraid,' declares the \divine{Lord}, \\
\poemll    `and Israel, don't be dismayed. \\
\poeml For I'll deliver you from a distant place \\
\poemll    and your descendants from the land of their captivity. \\
\poeml Jacob will return. He will be undisturbed and secure, \\
\poemll    and no one will cause him to fear. \\
\poeml \v{11}For I'll be with you to save you,' \\
\poemll    declares the \divine{Lord}. \\
\poeml `For I'll put an end to all the nations \\
\poemll    where I scattered you; \\
\poemlll       but I won't make an end of you. \\
\poeml I'll discipline you justly, \\
\poemll    but I certainly won't leave you unpunished.'
\passage{The Healing of Zion's Wounds}
\poeml \v{12}``For this is what the \divine{Lord} says: \\
\poeml `Your injury won't heal; \\
\poemll    your wound is severe. \\
\poeml \v{13}There is no one to plead your cause. \\
\poemll    There is no medicine for your sore;\fnote{\fbackref{30:13} Lit. \fbib{for a sore}} \\
\poemlll       no healing for you. \\
\poeml \v{14}All your lovers have forgotten you; \\
\poemll    they don't seek you. \\
\poeml Indeed, I've struck you down \\
\poemll    with the blow of an enemy, \\
\poemlll       with the punishment of a cruel foe\fnote{\fbackref{30:14} Lit. \fbib{cruel one}} \\
\poeml because your wickedness is great, \\
\poemll    and your sins are numerous. \\
\poeml \v{15}Why do you cry out because of your injury? \\
\poemll    Your wound won't heal. \\
\poeml Because your wickedness is severe, \\
\poemll    and your sins are numerous, \\
\poemlll       I've done all these things to you. \\
\poeml \v{16}In addition, all who devour you will be devoured, \\
\poemll    and all your oppressors---all of them--- \\
\poemlll       will go into captivity. \\
\poeml Those who plunder you will become plunder, \\
\poemll    and all who spoil you will become spoil. \\
\poeml \v{17}Indeed, I'll bring you healing, \\
\poemll    and I'll heal you of your wounds,' \\
\poemlll       declares the \divine{Lord}, \\
\poeml `because they have called you an outcast \\
\poemll    and have said,\fnote{\fbackref{30:17} The Heb. lacks \fbib{and have said}} ``It is Zion, no one cares for her!''\,'\,''\fnote{\fbackref{30:17} Or \fbib{seeks her}}
\passage{Jacob's Restoration}
\poeml \v{18}``This is what the \divine{Lord} says: \\
\poeml `I'm going to restore the fortunes of the tents of Jacob \\
\poemll    and have compassion on his dwellings. \\
\poeml A city will be rebuilt on its ruins \\
\poemll    and a palace\fnote{\fbackref{30:18} Or \fbib{fortress}} will sit on its rightful place. \\
\poeml \v{19}Thanksgiving and the sounds of laughter \\
\poemll    will come out of them. \\
\poeml I'll cause them to increase in numbers and not decrease. \\
\poemll    I'll honor them and not make them insignificant. \\
\poeml \v{20}Their\fnote{\fbackref{30:20} Lit. \fbib{his children} and so through v. 21} children will be as they were before, \\
\poemll    and their congregation will be established before me. \\
\poeml I'll punish all who oppress them. \\
\poeml \v{21}Their leader will be one of their own,\fnote{\fbackref{30:21} Lit. \fbib{of them}} \\
\poemll    and their ruler will come from among them. \\
\poeml I'll bring him near, and he will approach me, \\
\poemll    for who would otherwise dare to approach me?' \\
\poemlll       declares the \divine{Lord}. \\
\poeml \v{22}`You will be my people, \\
\poemll    and I'll be your God.'\,''
\passage{The Coming Judgment}
\poeml \v{23}Look, the storm of the \divine{Lord}! \\
\poemll    His\fnote{\fbackref{30:23} The Heb. lacks \fbib{His}} wrath has gone forth, a twisting storm. \\
\poemlll       It will swirl around the head of the wicked. \\
\poeml \v{24}The fierce anger of the \divine{Lord} won't turn back \\
\poemll    until he has accomplished and established the plan of his heart. \\
\poeml In the days to come, you will understand this.
\end{poetry}
\labelchapt{31}

\chapt{31}
\v{1}``At that time,'' declares the \divine{Lord}, ``I'll be the God of all the families of Israel, and they will be my people.''
\passage{The \divine{Lord} Promises Restoration}

\begin{poetry}
\poeml \v{2}This is what the \divine{Lord} says: \\
\poeml ``The people who survived the sword, \\
\poemll    found favor in the desert \\
\poemlll       while Israel was seeking rest.\fnote{\fbackref{31:2} Lit. \fbib{while going to find rest for him, Israel}; cf. Deut 28:65} \\
\poeml \v{3}The \divine{Lord} appeared to Israel\fnote{\fbackref{31:3} Lit. \fbib{to me}; i.e. here Jeremiah personifies Israel} from far away and said,\fnote{\fbackref{31:3} The Heb. lacks \fbib{and said}} \\
\poemll    ``I've loved you with an everlasting love, \\
\poemlll       therefore I've drawn you with gracious love. \\
\poeml \v{4}I'll again build you, and you will be rebuilt, \\
\poemll    Virgin Israel! \\
\poeml You will again take up your tambourines \\
\poemll    and go out to dance with those who are filled with joy. \\
\poeml \v{5}You will again plant vineyards on the hills of Samaria \\
\poemll    where planters had planted and defiled the fruit.\fnote{\fbackref{31:5} I.e. had used the fruit for inappropriate purposes} \\
\poeml \v{6}For there will be a day when the watchmen \\
\poemll    will call out on the hills of Ephraim, \\
\poemll    `Arise, let's go up to Zion to the \divine{Lord} our God.'\,''
\end{poetry}
\passage{The Blessings of Returning from Exile}

\begin{poetry}
\poeml \v{7}For this is what the \divine{Lord} says:
\end{poetry}

\begin{poetry}
\poeml ``Cry out with joy for Jacob \\
\poemll    and shout for the chief among the nations. \\
\poeml Announce, give praise, and say, \\
\poemll    `Lord, save your people, the remnant of Israel.' \\
\poeml \v{8}Look, I'm bringing them from the northern region,\fnote{\fbackref{31:8} Lit. \fbib{the land of the north}} \\
\poemll    and I'll gather them from the farthest parts of the earth. \\
\poeml among them will be the blind and the lame, \\
\poemll    together with the pregnant woman \\
\poemlll       and the woman in labor. \\
\poeml A large group will return here. \\
\poeml \v{9}They'll come crying, \\
\poemll    and I'll lead them as they pray for mercy.\fnote{\fbackref{31:9} Lit. \fbib{I'll lead them with prayer for mercy}} \\
\poeml I'll make them walk by streams of water, \\
\poemll    along a straight path on which they won't stumble. \\
\poeml For I am Israel's father, \\
\poemll    and Ephraim is my firstborn.'' \\
\poeml \v{10}Nations, listen to this message from the \divine{Lord}, \\
\poemll    and declare it in the distant coastlands. \\
\poeml Say, ``The one who scattered Israel will gather him \\
\poemll    and keep him as a shepherd keeps his flock.'' \\
\poeml \v{11}For the \divine{Lord} will deliver Jacob \\
\poemll    and redeem him from the hand of one stronger than he. \\
\poeml \v{12}They'll come and cry out with joy \\
\poemll    on the heights of Zion. \\
\poeml They'll be radiant over the \divine{Lord}'s goodness, \\
\poemll    over the grain, the new wine, the fresh oil, \\
\poemlll       and over the young of the flocks and herds. \\
\poeml Their lives will be like a well-watered garden. \\
\poemll    They'll never again grow faint.\fnote{\fbackref{31:12} I.e. from lack of food and drink} \\
\poeml \v{13}The virgins will rejoice with dancing, \\
\poemll    together with young men and old men. \\
\poeml For I'll turn their mourning into joy, \\
\poemll    and I'll comfort them and give them gladness \\
\poemlll       instead of sorrow. \\
\poeml \v{14}I'll give the priests abundant provisions,\fnote{\fbackref{31:14} Lit. \fbib{fatness}} \\
\poemll    and my people will be satisfied with my goodness,'' \\
\poemlll       declares the \divine{Lord}.
\passage{The End of Rachel's Mourning}
\poeml \v{15}This is what the \divine{Lord} says: \\
\poeml ``A voice is heard in Ramah, \\
\poemll    lamentation and bitter crying. \\
\poeml Rachel is crying, \\
\poemll    and she refuses to be comforted for her children, \\
\poemlll       because they are no longer alive.'' \\
\poeml \v{16}This is what the \divine{Lord} says: \\
\poeml ``Restrain your voice from crying, \\
\poemll    and your eyes from tears, \\
\poeml for there is a reward for your work,'' \\
\poemll    declares the \divine{Lord}. \\
\poemlll       ``They'll return from the enemy's land. \\
\poeml \v{17}There is hope for your future,'' \\
\poemll    declares the \divine{Lord}. \\
\poeml ``Your\fnote{\fbackref{31:17} The Heb. lacks \fbib{Your}} children will return to their own territory.''
\passage{Ephraim's Prayer and Confession}
\poeml \v{18}``I've certainly heard Ephraim \\
\poemll    shuddering with grief as they said,\fnote{\fbackref{31:18} The Heb. lacks \fbib{as they said}} \\
\poeml `You have disciplined me, \\
\poemll    and I'm disciplined like an untrained calf. \\
\poeml Restore me, and let me return,\fnote{\fbackref{31:18} Or \fbib{repent}} \\
\poemll    for you are the \divine{Lord} my God. \\
\poeml \v{19}Indeed, after I turned away, then I repented. \\
\poemll    And after I came to understand, \\
\poemlll       I slapped my forehead.\fnote{\fbackref{31:19} Lit. \fbib{thigh}; i.e. as a sign of remorse} \\
\poeml I was both ashamed and humiliated \\
\poemll    because I bear the disgrace of my youth.'\,''
\passage{God's Gracious Response}
\poeml \v{20}``Is Ephraim my dear son? \\
\poemll    Is he a darling child? \\
\poeml Indeed, as often as I've spoken about him, \\
\poemll    I surely still remember him. \\
\poeml Therefore I deeply yearn for him. \\
\poemll    I'll surely have great compassion on him,'' \\
\poemlll       declares the \divine{Lord}. \\
\poeml \v{21}Set up markers for yourselves. \\
\poemll    Erect signposts for yourselves. \\
\poeml Pay attention to the highway, \\
\poemll    to the road you traveled. \\
\poeml Return, virgin Israel, \\
\poemll    return to these cities of yours. \\
\poeml \v{22}How long will you go this way and that, \\
\poemll    rebellious daughter? \\
\poeml Indeed, the \divine{Lord} will create a new thing on the earth; \\
\poemll    a woman will protect\fnote{\fbackref{31:22} Lit. \fbib{surround}} a man.
\end{poetry}

\v{23}This is what the \divine{Lord} of the Heavenly Armies, the God of Israel, says: ``They'll again speak this message in the land of Judah and its towns when I restore their fortunes:\fnote{\fbackref{31:23} Or \fbib{return them from captivity}} `The \divine{Lord} bless you, righteous dwelling, holy mountain.' \v{24}Judah and all its towns will live together in the land,\fnote{\fbackref{31:24} Lit. \fbib{in it}} along with farmers and those who follow the flock. \v{25}I'll provide abundance for those who are weary, and fill all who are faint.'' \v{26}Then I awoke and looked around, and I had had a pleasant sleep.
\passage{Restoration and Responsibility}

\v{27}``Look, days are coming,'' declares the \divine{Lord}, ``when I'll sow the house of Israel and the house of Judah using people and animals as seed.\fnote{\fbackref{31:27} Lit. \fbib{with the seed of people and the seed of animals}} \v{28}Just as I've watched over them to pull up, tear down, overthrow, destroy, and bring disaster, so I'll watch over them to build and to plant,'' declares the \divine{Lord}. \v{29}``In those days people will no longer say, `The fathers have eaten sour grapes, but the children's teeth have been set on edge.' \v{30}Instead, each person will die for his own iniquity. Everyone who eats sour grapes will have his own\fnote{\fbackref{31:30} The Heb. lacks \fbib{own}} teeth set on edge.''
\passage{The New Covenant}

\v{31}``Look, days are coming,'' declares the \divine{Lord}, ``when I'll make a new covenant with the house of Israel and the house of Judah. \v{32}It won't be like the covenant I made with their ancestors on the day I took them by the hand to bring them out of the land of Egypt. They broke my covenant, although I was a husband to them,'' declares the \divine{Lord}. \v{33}``Rather, this is the covenant that I'll make with the house of Israel after those days,'' declares the \divine{Lord}. ``I'll put my Law\fnote{\fbackref{31:33} Or \fbib{instruction}} within them and will write it on their hearts. I'll be their God and they will be my people. \v{34}No longer will a person teach his neighbor or his relative: `Know the \divine{Lord}.' Instead, they'll all know me, from the least to the greatest of them,'' declares the \divine{Lord}. ``Indeed, I'll forgive their iniquity, and I'll remember their sin no more.''

\begin{poetry}
\poeml \v{35}This is what the \divine{Lord} says, \\
\poemll    who gives the sun for light by day, \\
\poeml the laws that govern the moon and stars for light by night, \\
\poemll    and who stirs up the sea so that its waves roar. \\
\poemlll       The \divine{Lord} of the Heavenly Armies is his name: \\
\poeml \v{36}``If these laws cease to function in my presence,'' \\
\poemll    declares the \divine{Lord}, \\
\poeml ``then the descendants of Israel will cease to be \\
\poemll    a nation in my presence for all time!'' \\
\poeml \v{37}This is what the \divine{Lord} says: \\
\poeml ``If the heavens could be measured above, \\
\poemll    or the foundations of the earth be searched out below, \\
\poeml then I also would reject all the descendants of Israel \\
\poemll    because of everything they have done,'' \\
\poemlll       declares the \divine{Lord}.
\end{poetry}

\v{38}``Look, days are coming,'' declares the \divine{Lord}, ``when the city of the \divine{Lord} will be rebuilt\fnote{\fbackref{31:38} Or \fbib{the city will be rebuilt for the \divine{Lord}}} from the Tower of Hananel to the Corner Gate. \v{39}A measuring line will go straight out from there to the hill of Gareb, and then it will turn to Goah. \v{40}The whole valley of dead bodies and ashes and all the fields as far as the Brook Kidron to the corner of the Horse Gate toward the east will be holy to the \divine{Lord}. It won't be uprooted or overthrown again, forever.''
\labelchapt{32}
\passage{Jeremiah Purchases a Field}

\chapt{32}
\v{1}This is\fnote{\fbackref{32:1} The Heb. lacks \fbib{This is}} the message that came to Jeremiah from the \divine{Lord} in the tenth year of the reign of\fnote{\fbackref{32:1} The Heb. lacks \fbib{of the reign of}} Zedekiah, king of Judah. It was the eighteenth year of the reign of\fnote{\fbackref{32:1} The Heb. lacks \fbib{of the reign of}} Nebuchadnezzar. \v{2}The army of the king of Babylon was then besieging Jerusalem, and Jeremiah the prophet was confined in the courtyard of the guard at the palace of the king of Judah \v{3}where Zedekiah had confined him. Zedekiah had said,\fnote{\fbackref{32:3} Lit. \fbib{him, saying}} ``Why did you prophesy and say these things? You said,\fnote{\fbackref{32:3} Lit. \fbib{Why did you prophesy, saying,}} `This is what the \divine{Lord} says: ``I'm about to give this city into the hand of the king of Babylon, and he will capture it. \v{4}Zedekiah, king of Judah, won't escape from\fnote{\fbackref{32:4} Lit. \fbib{from the hand of}} the Chaldeans, for he has surely been given over to the king of Babylon. He will speak to him face to face and look at him eye to eye. \v{5}The king of Babylon\fnote{\fbackref{32:5} Lit. \fbib{He}} will take Zedekiah to Babylon and there he will stay until I judge him,'' declares the \divine{Lord}. ``If you fight against the Chaldeans, you won't succeed.''\,'\,''

\v{6}Jeremiah replied, ``This message from the \divine{Lord} came to me: \v{7}`Look, Hanamel, your cousin,\fnote{\fbackref{32:7} Lit. \fbib{uncle's son}; and so throughout the chapter} is coming to you and will say, ``Buy my field in Anathoth for yourself, because the right of redemption to buy it belongs to you.''\,'

\v{8}``Then my cousin Hanamel came to me in the courtyard of the guard just as the \divine{Lord} had said, and he told me, `Please buy my field in Anathoth in the territory of Benjamin because you have the right to possess it, and the right to redeem it belongs to you. Buy it for yourself.' So I knew that this was a message from the \divine{Lord}.

\v{9}``Then I bought the field in Anathoth from my cousin Hanamel. I weighed out the silver for him---seventeen shekels\fnote{\fbackref{32:9} I.e. about 6.8 ounces; a shekel weighed about 0.4 ounces} of silver. \v{10}I signed the deed and sealed it. I called in witnesses and used scales to weigh out the silver. \v{11}Then I took the deed of purchase---both the sealed one\fnote{\fbackref{32:11} I.e. a private copy} with the terms and conditions and the open one\fnote{\fbackref{32:11} I.e. a public copy}---\v{12}and I gave the deed of purchase to Neriah's son Baruch, the grandson of Mahseiah, in the presence of my cousin Hanamel, in the presence of the witnesses who signed the deed of purchase, and in the presence of all the Judeans sitting in the courtyard of the guard. \v{13}In their presence, I instructed Baruch as follows: \v{14}`This is what the \divine{Lord} of the Heavenly Armies, the God of Israel, says: ``Take these deeds---both this sealed deed of purchase and this open deed---and put them in a clay pot so they'll last for a long time. \v{15}For this is what the \divine{Lord} of the Heavenly Armies, the God of Israel, says: `Houses, fields, and vineyards will again be bought in this land.'\,''\,'\,''
\passage{Jeremiah's Prayer}

\v{16}``After I had given the deed of purchase to Neriah's son Baruch, I prayed to the \divine{Lord}: \v{17}`\divine{Lord}! Look, you made the heavens and the earth with your great power and your outstretched arm. Nothing is too difficult for you! \v{18}You, the great God, the mighty one, show gracious love to thousands and repay the parents' iniquity to their children after them. The \divine{Lord} of the Heavenly Armies is his name. \v{19}You are\fnote{\fbackref{32:19} The Heb. lacks \fbib{You are}} great in regards to your purposes and mighty in regards to your works. Your eyes are open to everything that people do, and will reward each one according to their ways and just as their actions deserve.\fnote{\fbackref{32:19} Lit. \fbib{and according to the fruit of his deeds}} \v{20}You are the one who performed signs and wonders in the land of Egypt and continue to do so\fnote{\fbackref{32:20} The Heb. lacks \fbib{and continue to do so}} until this day, both in Israel and among the rest of humanity. You made a reputation for yourself that continues to this day.\fnote{\fbackref{32:20} Lit. \fbib{yourself as this day}} \v{21}By your strong hand and outstretched arm, and with great terror, you brought your people Israel out of the land of Egypt with signs and wonders. \v{22}And you gave them this land which you had promised their ancestors that you would give them---a land flowing with milk and honey. \v{23}They came and took possession of it, but they didn't obey you or walk according to your Law.\fnote{\fbackref{32:23} Or \fbib{instruction}} They didn't do what you commanded them to do, so you caused all this calamity to happen to them. \v{24}Look, the siege ramps have reached the city to take it. Because of the sword, famine, and plague, the city has been given over to the Chaldeans who are fighting against it. What you said has happened, and you are watching it occur!\fnote{\fbackref{32:24} Lit. \fbib{and look, you are watching}} \v{25}Lord, you have told me, ``Buy the field for yourself with money and call in witnesses,\fnote{\fbackref{32:25} Lit. \fbib{have witnesses witness}}'' even though the city is being given over to the Chaldeans.'\,''
\passage{Prophecy of Jerusalem's Fall}

\v{26}Then this message from the \divine{Lord} came to Jeremiah: \v{27}``Look, I am the \divine{Lord}, the God who rules over all flesh. Is anything too difficult for me?'' \v{28}Therefore, this is what the \divine{Lord} says: ``I'm about to give this city into the hands of the Chaldeans and Nebuchadnezzar, king of Babylon, and he will capture it. \v{29}The Chaldeans who are fighting against this city will come, set this city on fire, and burn it along with the houses on whose roofs incense was burned to Baal and liquid offerings were poured out to other gods in order to provoke me. \v{30}Indeed, the Israelis\fnote{\fbackref{32:30} Lit. \fbib{sons of Israel}; and so throughout the book} and Judeans\fnote{\fbackref{32:30} Lit. \fbib{sons of Judah}} have been doing only evil in my presence since their youth. Indeed, the Israelis have done nothing but provoke me by what they have made with their hands,'' declares the \divine{Lord}.

\v{31}``Indeed, this city has provoked me to anger and wrath from the day they built it until now, and so I'll remove it\fnote{\fbackref{32:31} Lit. \fbib{to remove it}} from my sight \v{32}because of all the evil that the Israelis and Judeans have done to provoke me. They, their kings, their officials, their priests, their prophets, the people of Judah, and those living in Jerusalem have done these things.\fnote{\fbackref{32:32} The Heb. lacks \fbib{have done these things}} \v{33}They have turned their backs to me rather than their faces. Even though I taught them, teaching them again and again,\fnote{\fbackref{32:33} Lit. \fbib{getting up early and teaching}} they didn't listen to accept correction. \v{34}They put their detestable idols in the house that is called by my name and defiled it. \v{35}They built the high places of Baal that are in the Hinnom Valley\fnote{\fbackref{32:35} Lit. \fbib{Valley of Hinnom's son}} in order to sacrifice their sons and daughters to Molech---something that I didn't command, nor did it ever enter my mind for them to require this utterly repugnant thing---and lead Judah into sin.''
\passage{A Promise of Restoration}

\v{36}``Now therefore,'' says the \divine{Lord} God of Israel, ``concerning this city about which you are saying, `It is being given into the control of the king of Babylon by sword, famine, and plague,' \v{37}I'm about to gather my people\fnote{\fbackref{32:37} Lit. \fbib{them}} from all the lands where I've driven them in my anger, wrath, and great indignation. I'll bring them back to this place and let them live in safety. \v{38}They'll be my people, and I'll be their God. \v{39}I'll give them one heart and one lifestyle\fnote{\fbackref{32:39} Lit. \fbib{way}} so they'll fear me always for their own good and for the good of\fnote{\fbackref{32:39} The Heb. lacks \fbib{the good of}} their descendants after them. \v{40}I'll make an everlasting covenant with them that I won't turn away from doing good for them.\fnote{\fbackref{32:40} Or \fbib{from them to do good}} I'll put the fear of me in their hearts so they won't turn away from me. \v{41}I'll rejoice over them to do good for them, and I'll faithfully plant them in this land with all my heart and soul.'

\v{42}``Indeed, this is what the \divine{Lord} says: `Just as I'm bringing all this great disaster on this people, so I'll bring on them all the good things that I'm promising concerning them. \v{43}Fields will be bought in this land about which you will say, ``It is a desolate place without people or animals. It is given into the hands of the Chaldeans.'' \v{44}People\fnote{\fbackref{32:44} Lit. \fbib{They}} will buy fields for money, sign deeds, seal them, and call witnesses in the land of Benjamin, in the areas around Jerusalem, in the towns of Judah, the towns of the hill country, the towns of the Shephelah,\fnote{\fbackref{32:44} I.e. the verdant central lowlands of Israel; cf. Josh 10:40} and the towns of the Negev,\fnote{\fbackref{32:44} I.e. the southern regions of the Sinai peninsula; cf. Josh 10:40} for I'll restore their fortunes,' declares the \divine{Lord}.''
\labelchapt{33}
\passage{Restoration of Judah and Jerusalem}

\chapt{33}
\v{1}This message from the \divine{Lord} came to Jeremiah a second time while he was still confined in the courtyard of the guard: \v{2}``This is what the \divine{Lord} says who made the earth, the \divine{Lord} who formed it in order to establish it---whose name is the \divine{Lord}--- \v{3}`Call to me and I'll answer you, and will tell you about great and hidden\fnote{\fbackref{33:3} Or \fbib{inaccessible}} things that you don't know.' \v{4}For this is what the \divine{Lord} God of Israel says about the houses of this city and the houses of the kings of Judah that were torn down to defend\fnote{\fbackref{33:4} The Heb. lacks \fbib{to defend}} against the siege ramps and the sword, \v{5}`The Chaldeans are coming to fight and to fill those houses with the dead bodies of the people that I've struck down in my anger and wrath, for I've hidden my face from this city because of all their wickedness.

\v{6}```Look, I'll bring restoration and healing to it, and I'll heal them. I'll reveal to them an abundance of peace and faithfulness. \v{7}I'll restore the security of Judah and Israel\fnote{\fbackref{33:7} Lit. \fbib{fortunes of Israel}} and rebuild them as they were at first. \v{8}I'll cleanse them from all their sin that they have committed against me, and I'll forgive all their sins that they committed against me and by which they rebelled against me. \v{9}Jerusalem\fnote{\fbackref{33:9} Lit. \fbib{It}} will be for me a name of joy, praise, and glory to all the nations of the earth that hear about all the good that I'm doing for them. They'll fear and tremble because of all the good and because of all the peace that I'm bringing to Jerusalem.'\fnote{\fbackref{33:9} Lit. \fbib{for it}}

\v{10}``This is what the \divine{Lord} says: `You are saying about this place, ``It is a ruin without people and without animals.'' Yet in the cities of Judah and the streets of Jerusalem which are desolate places without inhabitants and without animals, there will again be heard \v{11}the sounds of rejoicing and gladness, the sounds of the bridegroom and the bride, and the sounds of those saying,

\begin{poetry}
\poeml ``Give thanks to the \divine{Lord} of the Heavenly Armies, \\
\poemll    for the \divine{Lord} is good, \\
\poemlll       and his gracious love lasts forever,''
\end{poetry}

as they bring thanksgiving offerings to the \divine{Lord}'s Temple. For I'll restore the fortunes of the land as they were at first,' declares the \divine{Lord}.

\v{12}``This is what the \divine{Lord} of the Heavenly Armies says: `In this place that is now a ruin without people or animals, and in all its towns there will again be pasture for shepherds resting their flocks. \v{13}In the towns of the hill country, in the towns of the Shephelah,\fnote{\fbackref{33:13} I.e. the verdant central lowlands of Israel; cf. Josh 10:40} in the towns of the Negev,\fnote{\fbackref{33:13} I.e. the southern regions of the Sinai peninsula; cf. Josh 10:40} in the territory of Benjamin, in the areas around Jerusalem, and in the towns of Judah flocks will again pass under the hands of the one who counts them,' says the \divine{Lord}.''
\passage{The Righteous Branch and the Davidic Covenant}

\v{14}```Look, the time is coming,' declares the \divine{Lord}, `when I'll fulfill the good promise that I spoke concerning the house of Israel and Judah. \v{15}In those days and at that time I'll cause a righteous Branch to spring up for David, and he will uphold justice and righteousness in the land. \v{16}At that time Judah will be delivered and Jerusalem will dwell in safety. And this is the name people\fnote{\fbackref{33:16} Lit. \fbib{they}} will call it, ``The \divine{Lord} is Our Righteousness.''\,' \v{17}For this is what the \divine{Lord} says: `David will never be without\fnote{\fbackref{33:17} Lit. \fbib{there will never be cut off for David}} a man sitting on the throne of the house of Israel, \v{18}nor will the Levitical priests be without\fnote{\fbackref{33:18} \fbib{there will never be cut off for the Levitical priests}} a man offering up burnt offerings, bringing in grain offerings, and offering sacrifices continuously\fnote{\fbackref{33:18} Lit. \fbib{all the days}} before me.'\,''

\v{19}This message from the \divine{Lord} came to Jeremiah: \v{20}``This is what the \divine{Lord} says: `If you could break my covenant with the day and night\fnote{\fbackref{33:20} Lit. \fbib{and my covenant with the night}} so that day and night wouldn't occur at the proper time,\fnote{\fbackref{33:20} Lit. \fbib{at their time}} \v{21}then my covenant with my servant David might also be broken so that he wouldn't have a son sitting on his throne, and so also with my servants the Levitical priests. \v{22}As the heavenly bodies\fnote{\fbackref{33:22} Lit. \fbib{the hosts of the heavens}} cannot be counted, and the sands of the sea cannot be measured, so I'll multiply the descendants\fnote{\fbackref{33:22} Lit. \fbib{seed}} of my servant David and the descendants of Levi who serve me.'\,''

\v{23}This message from the \divine{Lord} came to Jeremiah: \v{24}``Haven't you noticed what these people have been saying?---`The \divine{Lord} rejected the two families that he had chosen!' They have contempt for my people and no longer consider them a nation. \v{25}This is what the \divine{Lord} says: `If I had not established my covenant for day and night and the laws that govern\fnote{\fbackref{33:25} Lit. \fbib{the ordinances of}} the heavens and earth, \v{26}then I might reject the descendants\fnote{\fbackref{33:26} Lit. \fbib{seed}; and so throughout the verse} of Jacob and my servant David by not taking some of his descendants as rulers over the descendants of Abraham, Isaac, and Jacob. Indeed, I'll restore their fortunes, and I'll have compassion on them.'\,''
\labelchapt{34}
\passage{A Message to Zedekiah}

\chapt{34}
\v{1}This is\fnote{\fbackref{34:1} The Heb. lacks \fbib{This is}} the message that came to Jeremiah from the \divine{Lord} while king Nebuchadnezzar of Babylon, all his army, all the kingdoms of the earth that were under his authority, along with all the people were fighting against Jerusalem and all its towns: \v{2}``This is what the \divine{Lord} God of Israel says: `Go and speak to king Zedekiah of Judah. Say to him, ``This is what the \divine{Lord} says: `Look, I'm giving this city into the hand of the king of Babylon, and he will set it on fire. \v{3}You won't escape from him. You will certainly be captured and given into his control.\fnote{\fbackref{34:3} Lit. \fbib{hand}} You will see the king of Babylon eye to eye, he will speak to you face to face, and you will go to Babylon.'\,''\,' \v{4}Yet, hear this message from the \divine{Lord}, king Zedekiah of Judah. This is what the \divine{Lord} says to you, `You won't die by the sword. \v{5}You will die peacefully, and as they burned fires\fnote{\fbackref{34:5} I.e. a memorial fire} for your ancestors,\fnote{\fbackref{34:5} Lit. \fbib{like the burning for your ancestors}} the former kings who were before you, so they'll burn fires\fnote{\fbackref{34:5} The Heb. lacks \fbib{fires}} for you, wailing, ``Oh how terrible, your majesty!''\,' For I've spoken the message,'' declares the \divine{Lord}.

\v{6}Then Jeremiah the prophet spoke all of this in Jerusalem to king Zedekiah of Judah, \v{7}while the army of the king of Babylon was fighting against Jerusalem and all the cities of Judah that were left, namely Lachish and Azekah. (They were the only fortified cities that remained among the cities of Judah.)
\passage{A Broken Agreement with Hebrew Servants}

\v{8}This is\fnote{\fbackref{34:8} The Heb. lacks \fbib{This is}} this message from the \divine{Lord} that came to Jeremiah from the \divine{Lord} after Zedekiah had made a covenant with all the people in Jerusalem proclaiming release for them. \v{9}Each person was to set free his male and female slaves who were Hebrews, so that no Jewish person would enslave his brother.\fnote{\fbackref{34:9} I.e. another Jewish person} \v{10}All the officials and all the people who had entered into the covenant agreed\fnote{\fbackref{34:10} Or \fbib{obeyed}} that each would set his male and female slaves free so that they\fnote{\fbackref{34:10} Lit. \fbib{he}} would not enslave them any longer. They obeyed and they released them. \v{11}But afterward they turned around and took back the male and female slaves that they had set free, and they forced them to become male and female slaves.

\v{12}Then this message from the \divine{Lord} came to Jeremiah from the \divine{Lord}: \v{13}``This is what the \divine{Lord} God of Israel says: `I made a covenant with your ancestors on the day I brought them out of the land of Egypt, out of the house of slavery. I told them: \v{14}``At the end of seven years, each of you is to set free your fellow Hebrew who has sold himself to you and has served you for six years. You are to send him out from you with no further obligation.'' But your ancestors didn't obey me or pay attention.\fnote{\fbackref{34:14} Lit. \fbib{incline their ears}} \v{15}You recently repented and did what was right in my eyes by proclaiming release for one another, and you made a covenant before me in the house that is called by my name. \v{16}But then you turned around and profaned my name when each of you took back his male and female slaves whom you had set free according to their desire, and you forced them to become male and female slaves.''\,'

\v{17}``Therefore, this is what the \divine{Lord} says: `You haven't obeyed me by each of you proclaiming a release for your brothers and neighbors. Now I'm going to proclaim a release for you,' declares the \divine{Lord}, `a release\fnote{\fbackref{34:17} The Heb. lacks \fbib{a release}} to the sword, to plague, and to famine, and I'll make you a horrifying sight to all the kingdoms of the earth. \v{18}I'll give over the men who transgressed my covenant, who haven't fulfilled the terms of the covenant that they made before me when they cut the calf in two and passed between its parts---\v{19}the officials of Judah, the officials of Jerusalem, the eunuchs,\fnote{\fbackref{34:19} Or \fbib{palace officials}} the priests, and all the people of the land who passed between the parts of the calf. \v{20}I'll give them to their enemies who are seeking to kill them, and their dead bodies will be food for the birds of the sky and the animals of the land. \v{21}I'll give Zedekiah, king of Judah, and his officials into the domination of their enemies, to those\fnote{\fbackref{34:21} Lit. \fbib{to the hands of those}} who are seeking to kill them, and to\fnote{\fbackref{34:21} Lit. \fbib{to the hands of the}} the army of the king of Babylon that is coming against them. \v{22}Look, I'm in command of them,' declares the \divine{Lord}, `and I'll bring them back to this city. They'll capture it and burn it with fire, and I'll turn the towns of Judah into desolate places without inhabitants.'\,''
\labelchapt{35}
\passage{The Example of the Rechabites}

\chapt{35}
\v{1}This is the message that came to Jeremiah from the \divine{Lord} during the reign\fnote{\fbackref{35:1} Lit. \fbib{days}} of Josiah's son Jehoiakim, king of Judah: \v{2}``Go to the house of the Rechabites and speak to them. Bring them into the \divine{Lord}'s Temple, to one of the offices, and offer them wine to drink.'' \v{3}So I took Jeremiah's son Jaazaniah (a descendant of Habazziniah), his brothers, all his sons, and the whole family of the Rechabites. \v{4}I brought them to the \divine{Lord}'s Temple to the office of the descendants of Igdaliah's son Hanan, the man of God, which was next to the office of the officials, and which was above the office of Shallum's son Maaseiah, the keeper of the threshold.

\v{5}I put containers full of wine and cups in front of the members of the Rechabite clan\fnote{\fbackref{35:5} Lit. \fbib{the sons of the house of the Rechabites}} and told them, ``Drink the wine!''

\v{6}But they said, ``We won't drink wine, because our ancestor, Rechab's son Jonadab commanded us: `You and your descendants are never to drink wine! \v{7}You aren't to build houses, you aren't to sow seeds, and you aren't to plant vineyards, or own them. Instead, you are to live in tents all your lives,\fnote{\fbackref{35:7} Lit. \fbib{your days}} so you will enjoy a long life in the land where you reside.'\fnote{\fbackref{35:7} I.e. living as resident aliens} \v{8}We have obeyed everything that our ancestor, Rechab's son Jonadab, commanded us. So we, our wives, our sons, and our daughters have drunk no wine all our lives,\fnote{\fbackref{35:8} Lit. \fbib{all our days}} \v{9}and have built no houses to live in. We don't have vineyards, fields, or seed. \v{10}We have lived in tents. We have obeyed and have done everything that our ancestor Jonadab commanded us. \v{11}Now when Nebuchadnezzar king of Babylon came up against the land, we said, `Come on! Let's go to Jerusalem because of the army of the Chaldeans and the army of Aram. And now we're living in Jerusalem.'\,''

\v{12}This message from the \divine{Lord} came to Jeremiah: \v{13}``This is what the \divine{Lord} of the Heavenly Armies, the God of Israel says: `Go and say to the people of Judah and the inhabitants of Jerusalem, ``Will you not accept correction by listening to what I say?'' declares the \divine{Lord}. \v{14}``But what Rechab's son Jonadab commanded his sons about not drinking wine is observed, and they haven't drunk wine until this day. Indeed, they obey the commands of their ancestor. But I've spoken to you again and again,\fnote{\fbackref{35:14} Lit. \fbib{getting up early and speaking}} and you haven't obeyed me. \v{15}I've sent you all my servants, the prophets, sending them again and again.\fnote{\fbackref{35:15} Lit. \fbib{getting up early and sending}} I've said, `Each of you turn from his evil behavior\fnote{\fbackref{35:15} Lit. \fbib{way}} and make your deeds right. Don't follow other gods to serve them. Then you will remain in the land that I gave to you and to your ancestors.' But you haven't paid attention\fnote{\fbackref{35:15} Lit. \fbib{inclined your ear}} and you haven't obeyed me. \v{16}Indeed the descendants of Rechab's son Jonadab have carried out the command of their ancestor that he gave them, but this people has not obeyed me.'' \v{17}Therefore, this is what the \divine{Lord} God of the Heavenly Armies, the God of Israel says: ``Look, I'm bringing on Judah and all the residents of Jerusalem all the disaster that I pronounced against them, because I spoke to them, but they didn't listen, and I called out to them, but they didn't answer.''\,'\,''

\v{18}Then Jeremiah told the house of the Rechabites, ``This is what the \divine{Lord} God of the Heavenly Armies, the God of Israel says: `Because you obeyed the commandment of your ancestor Jonadab, have observed all his commandments, and have done everything that he commanded you,' \v{19}therefore, this is what the \divine{Lord} God of the Heavenly Armies, the God of Israel says: `Rechab's son Jonadab won't lack a descendant\fnote{\fbackref{35:19} Cf. Neh 3:14} who serves me\fnote{\fbackref{35:19} Lit. \fbib{there won't be cut off from Rechab's son Jonadab one standing before me}} always.'\,''
\labelchapt{36}
\passage{Jeremiah's Scroll Read in the Temple}

\chapt{36}
\v{1}In the fourth year of the reign of\fnote{\fbackref{36:1} The Heb. lacks \fbib{of the reign of}} Josiah's son King Jehoiakim of Judah, this message came to Jeremiah from the \divine{Lord}: \v{2}``Take a scroll and write on it all the words that I've spoken to you about Israel, about Judah, and about all the nations, since I first spoke to you\fnote{\fbackref{36:2} Lit. \fbib{from the day I spoke to you}} in the time of Josiah until the present time. \v{3}Perhaps the house of Judah will hear about all the calamity that I'm planning to bring on them, and so each of them will turn from his wicked way and I'll forgive their iniquities and sins.''

\v{4}Jeremiah summoned Neriah's son Baruch and at Jeremiah's dictation, Baruch wrote on the scroll all the words of the \divine{Lord} that he had spoken to him.

\v{5}Jeremiah instructed Baruch, ``I'm confined and can't go to the \divine{Lord}'s Temple. \v{6}You go and read the words of the \divine{Lord} that you wrote at my dictation from the scroll. Read them\fnote{\fbackref{36:6} The Heb. lacks \fbib{Read them}} to\fnote{\fbackref{36:6} Lit. \fbib{scroll in the hearing of}} the people at the \divine{Lord}'s Temple on the fast day. Also read them to all the people of Judah who are coming from their towns. \v{7}Perhaps their pleas for help will come to the \divine{Lord}'s attention, and each of them will turn from his evil lifestyle in light of the great anger and wrath that the \divine{Lord} has declared against this people.'' \v{8}So Neriah's son Baruch did just as Jeremiah the prophet instructed him, reading the words of the \divine{Lord} from the scroll at the \divine{Lord}'s Temple.

\v{9}In the ninth month of the fifth year of the reign of\fnote{\fbackref{36:9} The Heb. lacks \fbib{of the reign of}} Josiah's son Jehoiakim, king of Judah, a fast was proclaimed in the \divine{Lord}'s presence in Jerusalem for all the people of Jerusalem, as well as all the people who were coming from the towns of Judah. \v{10}Baruch read the words of Jeremiah from the scroll to\fnote{\fbackref{36:10} Lit. \fbib{book in the hearing of}} all the people at the \divine{Lord}'s Temple. He did this\fnote{\fbackref{36:10} Heb. lacks \fbib{He did this}} from the office of Shaphan's son Gemariah the scribe, in the upper court at the entrance of the New Gate of the \divine{Lord}'s Temple.
\passage{Jeremiah's Scroll Read in the Palace}

\v{11}When Gemariah's son Micaiah, the grandson of Shaphan, heard all the words of the \divine{Lord} from the scroll, \v{12}he went down to the palace, to the scribe's office, where all the officials were sitting. Elishama the scribe, Shemaiah's son Delaiah, Achbor's son Elnathan, Shaphan's son Gemariah, Hananiah's son Zedekiah, and all the other officials were there. \v{13}Micaiah told them all the things that he had heard when Baruch read from the scroll to the people. \v{14}Then all the officials sent Nethaniah's son Jehudi, (who was also the grandson of Shelemiah and Cushi's great-grandson), to Baruch, who said, ``Take the scroll that you read to\fnote{\fbackref{36:14} Lit. \fbib{read in the hearing of}} the people and come.'' Neriah's son Baruch took the scroll with him and went to them.

\v{15}They told him, ``Please sit down and read it to us.''\fnote{\fbackref{36:15} Lit. \fbib{in our hearing}} So Baruch read it to them. \v{16}When they heard all the words, they turned to one another in fear, saying to Baruch, ``We must report all these things to the king.'' \v{17}Then they asked Baruch, ``Please tell us how you wrote all the words. Did Jeremiah dictate them all?''\fnote{\fbackref{36:17} Lit. \fbib{from his mouth}}

\v{18}Baruch answered them, ``Yes, Jeremiah dictated all these words to me, and I wrote them in the scroll with ink.''

\v{19}Then the officials told Baruch, ``Go, hide yourself, both you and Jeremiah, and don't let anyone know where you are.''
\passage{The King Burns Jeremiah's Scroll}

\v{20}The officials\fnote{\fbackref{36:20} Lit. \fbib{They}} went to the king in the courtyard, but they deposited the scroll in the office of Elishama the scribe. Then they reported everything written on the scroll\fnote{\fbackref{36:20} Lit. \fbib{reported all the words}} to the king. \v{21}The king sent Jehudi to get the scroll, and he took it from the office of Elishama the scribe. Jehudi read it to the king\fnote{\fbackref{36:21} Lit. \fbib{in the hearing of the king}} and to all the officials who were standing beside the king. \v{22}The king was sitting in the winter palace in the ninth month and a stove\fnote{\fbackref{36:22} Or \fbib{brazier}} was burning in front of him.\fnote{\fbackref{36:22} Or \fbib{a fire was burning in the stove in front of him}} \v{23}As Jehudi would read three or four columns, the king\fnote{\fbackref{36:23} Lit. \fbib{he}} would cut it with a scribe's knife and throw it into the fire which was in the stove, until all the scroll was burned\fnote{\fbackref{36:23} Lit. \fbib{until the completion of the scroll}} in the fire in the stove. \v{24}The king and all his officials\fnote{\fbackref{36:24} Or \fbib{servants}} who were listening to these words were not afraid, nor did they tear their garments. \v{25}Even though Elnathan, Delaiah, and Gemariah urged the king not to burn the scroll, he would not listen to them. \v{26}The king ordered his\fnote{\fbackref{36:26} Lit. \fbib{the king's}} son Jerahmeel, Azriel's son Seraiah, and Abdeel's son Shelemiah to get Baruch the scribe and Jeremiah the prophet, but the \divine{Lord} had hidden them.
\passage{Jeremiah Rewrites the Scroll}

\v{27}This message from the \divine{Lord} came to Jeremiah after the king burned the scroll containing the words that Baruch had written at Jeremiah's dictation: \v{28}``Go back, take another scroll and write on it all the original\fnote{\fbackref{36:28} Lit. \fbib{first}} words which were on the scroll that Jehoiakim, king of Judah, burned. \v{29}Concerning Jehoiakim, king of Judah, you are to say, `This is what the \divine{Lord} says: ``You burned this scroll, all the while saying, `Why did you write on it that the king of Babylon will definitely come, destroy this land, and eliminate both people and animals from it?'\,'' \v{30}Therefore, this is what the \divine{Lord} says concerning Jehoiakim, king of Judah, ``He will have no one to sit on the throne of David, and his corpse will be thrown out to rot during the heat of the day and the frost of the night. \v{31}I'll punish him, his descendants, and his officials\fnote{\fbackref{36:31} Or \fbib{servants}} for their iniquity. I'll bring on them, on the residents of Jerusalem, and on the men of Judah all the calamity about which I've warned them, but they would not listen.''\,'\,''

\v{32}Then Jeremiah took another scroll and gave it to Neriah's son Baruch the scribe. He wrote on it, at Jeremiah's dictation, all the words of the book that Jehoiakim king of Judah burned in the fire. He also added to them many similar words.
\labelchapt{37}
\passage{Zedekiah Consults Jeremiah}

\chapt{37}
\v{1}Josiah's son King Zedekiah reigned in place of Jehoiakim's son Coniah,\fnote{\fbackref{37:1} I.e. Jehoiachin} whom Nebuchadnezzar king of Babylon had made king of the land of Judah. \v{2}But neither he nor his officials nor the people of the land listened to the words of the \divine{Lord} that were spoken by\fnote{\fbackref{37:2} Lit. \fbib{that were in the hand of}} Jeremiah the prophet.

\v{3}King Zedekiah sent Shelemiah's son Jehucal and Maaseiah's son Zephaniah the priest to Jeremiah the prophet, asking him, ``Please pray to the \divine{Lord} our God for us.'' \v{4}Now Jeremiah was still\fnote{\fbackref{37:4} The Heb. lacks \fbib{still}} going in and out among the people since he had not yet been put in prison. \v{5}Pharaoh's army had come out of Egypt, and when the Chaldeans who were besieging Jerusalem heard the report about them, they withdrew from Jerusalem.

\v{6}Then this message from the \divine{Lord} came to Jeremiah the prophet: \v{7}``This is what the \divine{Lord} God of Israel says: `This is what you are to say to the king of Judah who sent you to me to inquire of me, ``Look, Pharaoh's army that has come to help will go back to its own land of Egypt, \v{8}and then the Chaldeans will come back to fight against this city, to capture it, and burn it with fire.''\,' \v{9}``This is what the \divine{Lord} says: `Don't deceive yourselves by saying, ``The Chaldeans will surely go away from us,'' `for they won't go. \v{10}Indeed, even if you defeated the entire Chaldean army that is fighting against you, and they had only wounded men left in their tents, they would get up and burn this city with fire.'\,''\,'\,''
\passage{Jeremiah Arrested for Treason}

\v{11}When the Chaldean army was leaving Jerusalem because of Pharaoh's army, \v{12}Jeremiah left Jerusalem to go to the territory of Benjamin to take possession of his property\fnote{\fbackref{37:12} The Heb. lacks \fbib{of his property}} there among the people. \v{13}He was in the Gate of Benjamin, and chief officer Irijah, Shelemiah's son and the grandson of Hananiah, was there. He arrested Jeremiah the prophet, accusing him: ``You are going over to the Chaldeans!''

\v{14}Jeremiah said, ``It's a lie! I'm not going over to the Chaldeans.'' But Irijah\fnote{\fbackref{37:14} Lit. \fbib{he}} would not listen to him, and he\fnote{\fbackref{37:14} Lit. \fbib{Irijah}} arrested Jeremiah and brought him to the officials. \v{15}The officials were angry with Jeremiah and beat him. They put him in jail in the house of Jonathan the scribe because they had made it into a prison. \v{16}So Jeremiah came into the cells in the dungeon\fnote{\fbackref{37:16} Lit. \fbib{cistern-house}} and remained there for a long time.\fnote{\fbackref{37:16} Lit. \fbib{for many days}}

\v{17}Then King Zedekiah sent for Jeremiah\fnote{\fbackref{37:17} The Heb. lacks \fbib{for Jeremiah}} and received him. The king questioned him secretly in his house: ``Is there a message from the \divine{Lord}?''

Jeremiah said, ``There is,'' and then he said, ``You will be given into the hand of the king of Babylon.'' \v{18}Then Jeremiah asked King Zedekiah, ``What offense have I committed against you, your officials, or these people that you have put me in prison? \v{19}Where are your prophets who prophesied to you, telling you: `The king of Babylon won't come against you or against this land'? \v{20}Now, please listen, your majesty,\fnote{\fbackref{37:20} Lit. \fbib{my lord the king}} and pay attention to what I'm asking you. Don't make me go back to the house of Jonathan the scribe, so I don't die there.''

\v{21}So King Zedekiah gave the order, and they assigned Jeremiah to the courtyard of the guard. Each day they gave him a loaf of bread from the bakers' street until all the bread in the city was gone. So Jeremiah remained in the courtyard of the guard.
\labelchapt{38}
\passage{Jeremiah is Arrested and Imprisoned}

\chapt{38}
\v{1}Mattan's son Shephatiah, Pashhur's son Gedaliah, Shelemiah's son Jucal, and Malchijah's son Pashhur heard the words that Jeremiah was speaking to all the people: \v{2}``This is what the \divine{Lord} says: `Whoever stays in this city will die by the sword, by famine, and by the plague, but the one who goes over to the Chaldeans will live. His life will be spared,\fnote{\fbackref{38:2} Lit. \fbib{He will have his life as a spoil of war}} and he will live.' \v{3}This is what the \divine{Lord} says: `This city will surely be given to the army of the king of Babylon, and he will capture it.'\,''

\v{4}Then the officials told the king, ``Let this man be put to death because he's undermining the efforts\fnote{\fbackref{38:4} Lit. \fbib{weakening the hands}} of the soldiers who remain in this city and that of all the people by speaking words like these to them. Indeed, this man is not seeking the well-being of this people, but rather their harm.''

\v{5}King Zedekiah said, ``Look, he's in your hands, and the king can do nothing to you.'' \v{6}So they threw Jeremiah into a cistern that belonged to the king's son Malchijah and was located in the courtyard of the guard. When they let Jeremiah down with ropes, because there was no water in the cistern---only mud---Jeremiah sank into the mud.
\passage{Jeremiah Rescued from the Cistern}

\v{7}Ebed-melech the Ethiopian, a eunuch\fnote{\fbackref{38:7} Or \fbib{official}} in the king's house, heard that Jeremiah had been put in the cistern. The king was sitting in the Benjamin Gate, \v{8}so Ebed-melech went out of the palace and spoke to the king: \v{9}``Your majesty,\fnote{\fbackref{38:9} Lit. \fbib{My lord the king}} these men have acted wickedly in all they have done to the prophet Jeremiah by throwing him into the cistern. He will die where he is because of the famine since there is no more bread in the city.''

\v{10}Then the king ordered Ebed-melech the Ethiopian:\fnote{\fbackref{38:10} Lit. \fbib{Cushite}} ``Thirty men are at your disposal. Take them with you and bring up Jeremiah the prophet from the cistern before he dies.'' \v{11}So Ebed-melech took the men with him and went to the palace, underneath the storeroom. He took worn out rags and worn out clothes from there, and using ropes he lowered them down to Jeremiah in the cistern.

\v{12}Ebed-melech the Ethiopian told Jeremiah, ``Put the worn out rags and clothes under your armpits under the ropes,'' and Jeremiah did as he said.\fnote{\fbackref{38:12} Lit. \fbib{did thus}} \v{13}They pulled Jeremiah with the ropes and brought him up from the cistern, but Jeremiah remained in the courtyard of the guard.
\passage{Zedekiah Again Seeks Advice from Jeremiah}

\v{14}King Zedekiah sent for Jeremiah the prophet and had him brought to him\fnote{\fbackref{38:14} Lit. \fbib{sent and took Jeremiah the prophet to him}} at the third entrance to the \divine{Lord}'s Temple. The king told Jeremiah, ``I'm going to ask you something, and don't hide anything from me.''

\v{15}Jeremiah told Zedekiah, ``When I tell you, you will surely put me to death, won't you? And when I give you advice, you don't listen to me.''

\v{16}Then King Zedekiah, in secret, swore an oath to Jeremiah: ``As surely as the \divine{Lord} lives, who gave us this life to live, I won't have you put to death, nor will I hand you over to these men who are seeking to kill you.''

\v{17}So Jeremiah told Zedekiah, ``This is what the \divine{Lord} God of the Heavenly Armies, the God of Israel, says: `If you will immediately surrender\fnote{\fbackref{38:17} Lit. \fbib{go out}} to the officers\fnote{\fbackref{38:17} Or \fbib{officials}} of the king of Babylon, then you will live, and this city won't be burned with fire. Both you and your family will live. \v{18}But if you don't surrender to the officers of the king of Babylon, then this city will be given to the Chaldeans, and they'll burn it with fire. You won't escape from their hands.'\,''

\v{19}Then King Zedekiah told Jeremiah, ``I'm afraid of the Judeans who have gone over to the Chaldeans. The Chaldeans\fnote{\fbackref{38:19} Lit. \fbib{They}} may turn me over to them,\fnote{\fbackref{38:19} Lit. \fbib{may give me into their hands}} and they may treat me harshly.''

\v{20}Jeremiah said, ``They won't turn you over. Obey the \divine{Lord} in what I'm telling you, and it will go well for you and you will live. \v{21}But if you refuse to surrender,\fnote{\fbackref{38:21} Lit. \fbib{to go out}} this is what the \divine{Lord} has shown me: \v{22}Look, all the women who are left in the house of the king of Judah will be brought out to the officers of the king of Babylon, and will say,

\begin{poetry}
\poeml `These friends of yours have mislead you \\
\poemll    and overcome you. \\
\poeml Your feet have sunk down into the mire, \\
\poemll    but they have turned away.'
\end{poetry}

\v{23}``They'll bring all your women and children out to the Chaldeans, and you won't escape from their hand. Indeed, you will be seized by the hand of the king of Babylon, and this city will be burned with fire.''

\v{24}Then Zedekiah told Jeremiah, ``Don't let anyone know about these words and you won't die. \v{25}If the officials hear that I've spoken with you, and they come to you and say,\fnote{\fbackref{38:25} Lit. \fbib{say to you}} `Tell us what you told the king, and what the king told you; don't hide it from us, and we won't put you to death,' \v{26}then you are to say to them, `I was presenting my request to the king that I not be taken back to the house of Jonathan to die there.'\,''

\v{27}When all the officials came to Jeremiah and questioned him, he replied to them exactly as the king had ordered him.\fnote{\fbackref{38:27} Lit. \fbib{according to all these words that the king ordered him}} So they stopped speaking with him because the conversation had not been overheard. \v{28}Jeremiah remained in the courtyard of the guard until the day Jerusalem was captured.
\labelchapt{39}
\passage{The Fall of Jerusalem and the Capture of Zedekiah}

\chapt{39}
\v{1}This is how Jerusalem was captured:\fnote{\fbackref{39:1}a in English is 38:28b in Hebrew} In the tenth month of the ninth year of the reign of\fnote{\fbackref{39:1} The Heb. lacks \fbib{of the reign of}} Zedekiah king of Judah, Nebuchadnezzar king of Babylon and all his army came to Jerusalem and laid siege to it. \v{2}On the ninth day of the fourth month, in the eleventh year of the reign of\fnote{\fbackref{39:2} The Heb. lacks \fbib{of the reign of}} Zedekiah, the wall of\fnote{\fbackref{39:2} The Heb. lacks \fbib{the wall of}} the city was breached. \v{3}All the officials of the king of Babylon came and sat in the Middle Gate, including\fnote{\fbackref{39:3} The Heb. lacks \fbib{including}} Nergal-sarri-usur, governor\fnote{\fbackref{39:3} Or \fbib{high official}} of Sinmagir,\fnote{\fbackref{39:3} I.e. a province of Babylon} Nabu-sarrussu-ukin the high official,\fnote{\fbackref{39:3} Lit. \fbib{the Rab-sa-resi}; a Babylonian title for a royal official} Nergal-sarri-user, the chief official,\fnote{\fbackref{39:3} Lit. \fbib{the Rab-mugi}, a Babylonian title for a royal official} and\fnote{\fbackref{39:3} Or \fbib{Gate: Nergal-sarezer, Samgar-nebu, Sarsekim, the high official, Nergal-sarezer, the chief official, and}} all the rest of the officials of the king of Babylon.

\v{4}When Zedekiah king of Judah and all the soldiers saw them, they fled and went out of the city at night through the king's garden through the gate between the two walls. Then he went out on the road toward the Arabah. \v{5}The Chaldean army pursued them and overtook Zedekiah on the plains of Jericho. When they seized him they brought him to Nebuchadnezzar king of Babylon at Riblah in the land of Hamath, and he passed judgment on him. \v{6}At Riblah, the king of Babylon executed Zedekiah's sons right\fnote{\fbackref{39:6} The Heb. lacks \fbib{right}} before his eyes. He\fnote{\fbackref{39:6} Lit. \fbib{The king of Babylon}} also executed all the nobles of Judah. \v{7}Then he put out Zedekiah's eyes and bound him with bronze fetters to take him to Babylon.

\v{8}The Chaldeans burned the palace and the houses of the people with fire, and they broke down the walls of Jerusalem. \v{9}Nebuzaradan, the captain of the Babylonian guard, took into exile in Babylon the rest of the people who remained in the city, those who had deserted to Nebuchadnezzar, and the rest of the people who remained. \v{10}Nebuzaradan the captain of the guard left in the land of Judah some of the poor people who did not have anything, and he gave them vineyards and fields on that day.
\passage{Jeremiah's Release from Prison}

\v{11}Nebuchadnezzar king of Babylon gave orders concerning Jeremiah through Nebuzaradan, the captain of the guard: \v{12}``Take him, look after him, and don't do anything to harm him. Rather, do for him whatever he tells you.'' \v{13}So Nebuzaradan, the captain of the guard, Nebushazban, the high official, Nergal-sar-ezer, the chief official, and all the officials of the king of Babylon sent for Jeremiah.\fnote{\fbackref{39:13} The Heb. lacks \fbib{for Jeremiah}} \v{14}They sent for Jeremiah\fnote{\fbackref{39:14} The Heb. lacks \fbib{for Jeremiah}} and took\fnote{\fbackref{39:14} The Heb. lacks \fbib{took}} him from the courtyard of the guard. They handed him over to Ahikam's son Gedaliah, the grandson of Shaphan, to take him home. So he remained among the people.
\passage{Ebed-melech Rewarded}

\v{15}This message from the \divine{Lord} came to Jeremiah while he was confined in the courtyard of the guard: \v{16}``Go and speak to Ebed-melech the Ethiopian: `This is what the \divine{Lord} of the Heavenly Armies, the God of Israel, says: ``Look, I'm going to fulfill my promise against this city for disaster rather than for good, and on that day it will happen before your eyes. \v{17}But I'll deliver you on that day,'' declares the \divine{Lord}. ``You won't be given into the hands of the men you fear. \v{18}For I'll surely deliver you, and you won't fall by the sword. Your life will be spared\fnote{\fbackref{39:18} Lit. \fbib{You will have your life as a spoil of war}} because you trusted me,'' declares the \divine{Lord}.'\,''
\labelchapt{40}
\passage{Jeremiah Chooses to Remain in Judah}

\chapt{40}
\v{1}This is\fnote{\fbackref{40:1} The Heb. lacks \fbib{This is}} the message that came to Jeremiah from the \divine{Lord} after Nebuzaradan the captain of the guard had released him from Ramah, when he was bound in chains, along with all the exiles from Jerusalem and Judah who were being taken into exile in Babylon.

\v{2}The captain of the guard took Jeremiah and told him, ``The \divine{Lord} your God has predicted this disaster on this place. \v{3}And now the \divine{Lord} has brought it about and has done just as he said. Because you people sinned against the \divine{Lord} and didn't obey him, this has happened to you. \v{4}Now, look, I've freed you today from the chains that were on your hands. If you want\fnote{\fbackref{40:4} Lit. \fbib{it is good in your eyes}} to come with me to Babylon, come, and I'll look after you. But if you don't want\fnote{\fbackref{40:4} Lit. \fbib{it is bad in your eyes}} to come with me to Babylon, don't.\fnote{\fbackref{40:4} Lit. \fbib{stop!}} Look, the whole land lies before you, so go wherever it seems good and right for you to go.''

\v{5}When he still did not respond, Nebuzaradan said,\fnote{\fbackref{40:5} The Heb. lacks \fbib{Nebuzaradan said}} ``Return to Ahikam's son Gedaliah, whom the king of Babylon has appointed over the cities of Judah, and remain with him among the people---or go wherever it seems right for you to go.'' Then the captain of the guard gave him an allowance of food and a gift and sent him off. \v{6}Jeremiah came to Ahikam's son Gedaliah at Mizpah, and he remained with him among the people who were left in the land.
\passage{Gedaliah and the Community in Judah}

\v{7}All the leaders of the forces who were in the field along with their men heard that the king of Babylon had appointed Ahikam's son Gedaliah over the men, women, children, and the poor of the land who had not been taken into exile in Babylon. \v{8}Those who came to Gedaliah at Mizpah included Nethaniah's son Ishmael, Jonathan, Kareah's son Jonathan, Tanhumeth's son Seraiah, Ephai's sons from Netophah; and Jezaniah, the son of a man from Maacah. They came along with\fnote{\fbackref{40:8} The Heb. lacks \fbib{came along with}} their men.

\v{9}Ahikam's son Gedaliah, the grandson of Shaphan, swore an oath to them and their men: ``Don't be afraid to serve the Chaldeans. Remain in the land and serve the king of Babylon, and things will go well for you. \v{10}As for me, I'll remain at Mizpah to represent you before\fnote{\fbackref{40:10} Lit. \fbib{to stand before}} the Chaldeans who come to us. As for you, gather wine, summer fruit, and oil. Put it in your containers and live in your cities that you have taken over.''

\v{11}All the Judeans who were in Moab, those with the people in Ammon, those in Edom, and those in all the other\fnote{\fbackref{40:11} The Heb. lacks \fbib{other}} countries also heard that the king of Babylon had left a remnant for Judah and that he had appointed Ahikam's son Gedaliah, the grandson of Shaphan, over them. \v{12}So all the Judeans returned from all the countries where they had been scattered. They came to the land of Judah, to Gedaliah at Mizpah, and they gathered wine and summer fruit in great abundance.
\passage{A Plot against Gedaliah}

\v{13}Kareah's son Jonathan and all leaders of the forces who were in the field came to Gedaliah at Mizpah. \v{14}They told him, ``Are you aware that Baalis, the king of the people of Ammon, has sent Nethaniah's son Ishmael to take your life?'' But Ahikam's son Gedaliah did not believe them.

\v{15}Then Kareah's son Jonathan spoke privately to Gedaliah at Mizpah: ``Let me go kill Nethaniah's son Ishmael, and no one will know. Why should he take your life? Otherwise\fnote{\fbackref{40:15} Lit. \fbib{Then}} all the Judeans who have gathered around you will be scattered, and the remnant of Judah will perish.''

\v{16}Ahikam's son Gedaliah replied to Kareah's son Jonathan, ``Don't do this! You're lying about Ishmael!''
\labelchapt{41}
\passage{Gedaliah is Assassinated}

\chapt{41}
\v{1}In the seventh month, Nethaniah's son Ishmael, the grandson of Elishama, a member of the royal family and one of the chief officers of the king, came to Ahikam's son Gedaliah at Mizpah, along with ten men. While they were dining together there at Mizpah, \v{2}Nethaniah's son Ishmael and the ten men with him got up and killed Ahikam's son Gedaliah, the grandson of Shaphan, with swords and killed the man whom the king of Babylon had appointed over the land. \v{3}Ishmael also struck down all the Judeans who were with him (that is, with Gedaliah) at Mizpah, along with the Chaldean soldiers who were found there.

\v{4}Now on the day after Gedaliah was killed, when as yet no one knew about it,\fnote{\fbackref{41:4} The Heb. lacks \fbib{about it}} \v{5}eighty men from Shechem, from Shiloh, and from Samaria came with their beards shaved, their clothes torn, and their bodies slashed. They had grain offerings and incense with them to present at the \divine{Lord}'s Temple.

\v{6}Nethaniah's son Ishmael went out from Mizpah to meet them, crying as he went. As he met them he told them, ``Come meet with Ahikam's son Gedaliah.'' \v{7}When they reached the middle of the city, Nethaniah's son Ishmael and the men who were with him slaughtered them and threw them into a cistern.\fnote{\fbackref{41:7} Lit. \fbib{slaughtered them to the middle of the cistern}}

\v{8}Ten men who were among\fnote{\fbackref{41:8} Lit. \fbib{found among}} them told Ishmael, ``Don't kill us because we have stores of wheat, barley, oil, and honey hidden in the field.'' So Ishmael stopped and did not kill them or their companions. \v{9}Ishmael threw the bodies of the men he killed on account of Gedaliah into the cistern that King Asa had made for protection against\fnote{\fbackref{41:9} Lit. \fbib{made on account of}} King Baasha of Israel. That is the same one Nethaniah's son Ishmael filled with those he killed. \v{10}Then Ishmael took captive all the rest of the people who were in Mizpah, including the king's daughters and all the rest of the people in Mizpah over whom Nebuzaradan the captain of the guard had appointed Ahikam's son Gedaliah. Nethaniah's son Ishmael took them captive and then set out to cross over to the Ammonites.
\passage{The Captives Rescued; Ishmael Escapes}

\v{11}Kareah's son Jonathan and all the military leaders who were with him heard about all the terrible things that Nethaniah's son Ishmael had done. \v{12}So they took all the men and went to fight Nethaniah's son Ishmael, and they found him at the large pool that is at Gibeon. \v{13}When all the people who were with Ishmael saw Kareah's son Jonathan and all the military leaders who were with him, they were glad. \v{14}All the people whom Ishmael had taken captive from Mizpah turned around and went back to Kareah's son Jonathan. \v{15}But Nethaniah's son Ishmael and eight other\fnote{\fbackref{41:15} The Heb. lacks \fbib{other}} men escaped from Jonathan and went to the Ammonites. \v{16}Kareah's son Jonathan and all the military leaders who were with him took all the rest of the people from Mizpah whom he had rescued\fnote{\fbackref{41:16} Lit. \fbib{brought back}} from Nethaniah's son Ishmael after he had killed Ahikam's son Gedaliah, including the young men, the soldiers, women, children, and eunuchs\fnote{\fbackref{41:16} Or \fbib{officials}} whom he had rescued from Gibeon. \v{17}They traveled and then stopped at Geruth Chimham near Bethlehem on their way to Egypt \v{18}because of the Chaldeans. They were afraid of the Chaldeans\fnote{\fbackref{41:18} Lit. \fbib{them}} because Nethaniah's son Ishmael had killed Ahikam's son Gedaliah, whom the king of Babylon had appointed over the land.
\labelchapt{42}
\passage{Jeremiah Asked to Pray for the People}

\chapt{42}
\v{1}Then all the military leaders, Kareah's son Jonathan, Hoshaiah's son Jezaniah, and all the people from the least to the greatest approached Jeremiah.\fnote{\fbackref{42:1} The Heb. lacks \fbib{Jeremiah}} \v{2}They told Jeremiah the prophet, ``Please listen to what we have to ask of you. Pray to the \divine{Lord} your God for us and for all these survivors. Indeed, only a few of us remain out of many, as you can see.\fnote{\fbackref{42:2} Lit. \fbib{as your eyes see us}} \v{3}Pray that the \divine{Lord} your God may inform us as to how we should live\fnote{\fbackref{42:3} Lit. \fbib{way we should walk}} and what we should do.''

\v{4}Jeremiah the prophet told them, ``I've heard, and I'm going to pray to the \divine{Lord} your God just as you have requested. Whatever the \divine{Lord} answers, I'll tell you. I won't withhold anything from you.''

\v{5}Then they told Jeremiah, ``May the \divine{Lord} be a true and faithful witness against us if we don't do everything that the \divine{Lord} your God tells us through you.\fnote{\fbackref{42:5} Lit. \fbib{sends to you for us}} \v{6}Whether it seems good or bad, we will obey the \divine{Lord} our God to whom we send you, so it may go well for us. Indeed, we will obey the \divine{Lord} our God.''
\passage{The \divine{Lord}'s Answer through Jeremiah}

\v{7}At the end of ten days a message from the \divine{Lord} came to Jeremiah. \v{8}So he called Kareah's son Jonathan, all the military leaders who were with him, and all the people from the least to the greatest. \v{9}He told them, ``This is what the \divine{Lord} God of Israel says, to whom you sent me to take your request:

\begin{poetry}
\poeml \v{10}`If you will just remain in this land, I'll build you up and not pull you down. I'll plant you and not uproot you, for I'm sorry about the disaster I've brought on you. \v{11}Don't be afraid of the king of Babylon as you have been.\fnote{\fbackref{42:11} Lit. \fbib{whom you fear}} Don't fear him,' declares the \divine{Lord}, `because I am with you to save you and deliver you from his control. \v{12}I'll show you compassion, so he will have compassion on you and return you to your land. \v{13}But if you disobey the \divine{Lord} your God by saying, ``We won't stay in this land,'' \v{14}and you also say, ``No, but we will go to the land of Egypt where we won't see war or hear the sound of the trumpet or hunger for bread, and there we will stay,'' \v{15}then hear this message from the \divine{Lord}, remnant of Judah: `This is what the \divine{Lord} of the Heavenly Armies, the God of Israel, says: ``If you are really determined\fnote{\fbackref{42:15} Lit. \fbib{really set your faces}} to go into Egypt, and you go there to settle, \v{16}the sword that you fear will overtake you there in the land of Egypt. The famine that you dread will pursue you into Egypt, and there you will die. \v{17}All the people who are determined to go into Egypt to settle there will die by the sword, by famine, and by the plague. No one will survive the disaster that I'll bring on them.'' \v{18}For this is what the \divine{Lord} of the Heavenly Armies, the God of Israel, says: `Just as my anger and my wrath were poured out on the inhabitants of Jerusalem, so my wrath will be poured out on you when you enter Egypt. You will be a curse and an object of horror, ridicule, and scorn, and you will never again see this place.' \v{19}The \divine{Lord} has told you, remnant of Judah, `Don't go to Egypt!' So be fully aware that I've warned you, today, \v{20}that you have deceived yourselves. Indeed, you yourselves sent me to the \divine{Lord} your God, saying, `Pray to the \divine{Lord} your God for us, and whatever the \divine{Lord} our God tells us we will do.' \v{21}I've told you today, but you haven't obeyed the \divine{Lord} your God in all that he sent me to tell\fnote{\fbackref{42:21} The Heb. lacks \fbib{tell}} you. \v{22}Now, be fully aware that you will die by the sword, by famine, and by plague in the place where you want to settle.''\fnote{\fbackref{42:22} Lit. \fbib{to go to settle}}
\end{poetry}
\labelchapt{43}
\passage{The Refugees Reject the \divine{Lord}'s Instruction}

\chapt{43}
\v{1}When Jeremiah had finished telling all the people all the words that the \divine{Lord} their God had sent him to tell them---that is, all these words---\v{2}Hoshaiah's son Azariah, Kareah's son Johanan, and all the arrogant men told Jeremiah, ``You're lying! The \divine{Lord} our God didn't send you to say, `Don't go to Egypt to settle there.' \v{3}Indeed, Neriah's son Baruch is inciting you against us in order to give us into the hands of the Chaldeans, to kill us, or to take us into exile to Babylon.''

\v{4}So Kareah's son Johanan, all the military leaders, and all the people did not obey the instructions given by\fnote{\fbackref{43:4} The Heb. lacks \fbib{the instructions given by}} the \divine{Lord} to remain in the land of Judah. \v{5}Kareah's son Johanan and all the military leaders took the entire remnant of Judah that had returned from all the nations where they had been scattered to settle in the land of Judah---\v{6}the young men, the women, the children, the daughters of the king, and everyone whom Nebuzaradan the captain of the guard had left with Ahikam's son Gedaliah, the grandson of Shaphan, along with Jeremiah the prophet and Neriah's son Baruch. \v{7}So they went into the land of Egypt, because they did not obey the \divine{Lord}, and they travelled as far as Tahpanhes.\fnote{\fbackref{43:7} \fbib{Tahpanhes} was a city in the delta region of Egypt.}
\passage{Nebuchadnezzar's Invasion of Egypt Predicted}

\v{8}Then this message from the \divine{Lord} came to Jeremiah in Tahpanhes: \v{9}``Take large stones in your hands, and, in the sight of the men of Judah, bury them in the mortar of the brickwork at the entrance of Pharaoh's house in Tahpanhes. \v{10}Then say to them, `This is what the \divine{Lord} of the Heavenly Armies, the God of Israel, says: ``I'm going to send for my servant Nebuchadnezzar king of Babylon. I'll take him and set his throne over these stones that I've buried, and he will spread his canopy over them. \v{11}He will come and attack the land of Egypt---those meant for death will be put to death, those meant for captivity will be taken captive, and those meant for the sword will be put to the sword. \v{12}He\fnote{\fbackref{43:12} So LXX; Heb. reads \fbib{I}} will set fire to the temples\fnote{\fbackref{43:12} Lit. \fbib{houses; and so throughout the section}} of the gods of Egypt. He will burn their idols\fnote{\fbackref{43:12} Lit. \fbib{them}} and take them captive. He will wrap himself with the land of Egypt like a shepherd wraps himself with a garment, and then he will leave from there in peace. \v{13}He will shatter the pillars of Heliopolis\fnote{\fbackref{43:13} Lit. \fbib{beth-shemesh}; i.e. \fbib{house of the sun}} in the land of Egypt and will burn the temples of the gods of Egypt with fire.''\,'\,''
\labelchapt{44}
\passage{Jeremiah Warns the Refugees in Egypt}

\chapt{44}
\v{1}This is the message that came to Jeremiah for all the Judeans who were living in the land of Egypt, who were living in Migdol, Tahpanhes, Memphis, and in the land of Pathros,\fnote{\fbackref{44:1} I.e. in southern Egypt} saying, \v{2}``This is what the \divine{Lord} of the Heavenly Armies, the God of Israel, says: `You have seen the disaster that I brought on Jerusalem and all the cities of Judah. Look, they're in ruins today, with no one living in them, \v{3}because of the\fnote{\fbackref{44:3} Lit. \fbib{their}} wickedness that they did, provoking me to anger by continuing to offer sacrifices and worship other gods that neither they nor you nor your ancestors had known. \v{4}Yet I sent all my servants the prophets to you again and again,\fnote{\fbackref{44:4} Lit. \fbib{getting up early and sending}} saying, ``Don't do this repulsive thing that I hate.'' \v{5}`But they didn't listen or pay attention\fnote{\fbackref{44:5} Lit. \fbib{turn their ears}} by turning from their wickedness and not offering sacrifices to other gods. \v{6}My wrath and my anger were poured out, and they burned in the cities of Judah and the streets of Jerusalem so that they have become a ruin and a desolate place, as is the case today.'

\v{7}``Now, this is what the \divine{Lord} of the Heavenly Armies, the God of Israel, says: `Why are you doing great harm to yourselves so as to cut off from Judah\fnote{\fbackref{44:7} Lit. \fbib{from the midst of Judah}} man and woman, child and infant from you, leaving yourselves without a remnant? \v{8}And why have you provoked me to anger by the works of your hands,\fnote{\fbackref{44:8} I.e. idols} by offering sacrifices to other gods in the land of Egypt where you have come to settle so that you cut yourselves off and become an object of ridicule and scorn among all the nations of the earth? \v{9}Have you forgotten the evil deeds of your ancestors, the evil deeds of the kings of Judah, the evil deeds of their\fnote{\fbackref{44:9} Lit. \fbib{his}} wives, your evil deeds, and the evil deeds of your wives, that they did in the land of Judah and the streets of Jerusalem? \v{10}To this day they haven't humbled themselves, they haven't shown reverence for the \divine{Lord},\fnote{\fbackref{44:10} The Heb. lacks \fbib{for the \divine{Lord}}} and they haven't lived according to my Law and my statutes that I set before them and before their ancestors.'

\v{11}``Therefore, this is what the \divine{Lord} of the Heavenly Armies, the God of Israel, says: `Look, I've determined to bring disaster on you\fnote{\fbackref{44:11} Lit. \fbib{have set my face against you for disaster}} and to cut off all Judah. \v{12}I'll take the remnant of Judah that determined to go to the land of Egypt to settle there, and all of them\fnote{\fbackref{44:12} The Heb. lacks \fbib{of them}} will come to an end in the land of Egypt. They'll fall by the sword, and they'll come to an end by famine. They'll become a curse, an object of horror, ridicule, and scorn. \v{13}I'll punish those who live in the land of Egypt just as I punished Jerusalem---with the sword, with famine, and with plague. \v{14}Of the remnant of Judah that came into the land of Egypt to settle there, no one will escape or survive to return to the land of Judah where they long to return and live.\fnote{\fbackref{44:14} Lit. \fbib{live there}} Indeed, they won't return, except for some\fnote{\fbackref{44:14} The Heb. lacks \fbib{for some}} refugees.'\,''
\passage{The Refugees Refuse to Repent}

\v{15}Then all the men who knew that their wives were offering sacrifices to other gods and all the women who were standing by---a large group, including all the people who were living in the land of Egypt in Pathros---answered Jeremiah: \v{16}``As for the message that you reported to us in the name of the \divine{Lord}, we won't listen to you! \v{17}Rather, we will keep doing everything that we said we would\fnote{\fbackref{44:17} Lit. \fbib{every word that comes out of our mouths}} by offering sacrifices to the Queen of Heaven\fnote{\fbackref{44:17} I.e. the Near Eastern fertility goddess Ishtar} and by pouring out liquid offerings to her just as we, our ancestors, our kings, and our leaders did in the cities of Judah and the streets of Jerusalem. Then we had plenty of bread, things went well for us, and we didn't experience disaster. \v{18}From the time we stopped offering sacrifices to the Queen of Heaven and pouring out liquid offerings to her, we have lacked everything, and we have been consumed\fnote{\fbackref{44:18} Lit. \fbib{have come to an end}} by the sword and famine. \v{19}Indeed, we\fnote{\fbackref{44:19} I.e. the women} are going to continue offering sacrifices to the Queen of Heaven and pouring out liquid offerings to her. And do you think we have made\fnote{\fbackref{44:19} Lit. \fbib{And have we made}} cakes to represent her or poured out liquid offerings for her without our husbands' approval?''\fnote{\fbackref{44:19} Lit. \fbib{apart from our husbands}}
\passage{Final Judgment Proclaimed}

\v{20}Then Jeremiah spoke a message to all the people, to the young men, to the women, and to all the people who were answering him: \v{21}``As for the sacrifices that you, your ancestors, your kings, your officials, and the people of the land offered in the cities of Judah and the streets of Jerusalem, the \divine{Lord} remembered them, did he not? And they came to his attention, did they not? \v{22}The \divine{Lord} could no longer bear it because of your evil deeds and the repulsive things that you did. So your land has become a ruin and an object of horror and ridicule without an inhabitant, as is the case today. \v{23}Because you offered sacrifices and sinned against the \divine{Lord}, you didn't obey the \divine{Lord} and didn't live according to his Law, his statutes, or his testimonies; therefore, this disaster has happened to you, as is the case today.''

\v{24}Then Jeremiah told all the people and all the women, ``All you people of Judah who are in the land of Egypt, listen to this message from the \divine{Lord}! \v{25}This is what the \divine{Lord} of the Heavenly Armies, the God of Israel, says: `You and your wives have spoken with your mouths and acted with your hands: ``We will certainly carry through\fnote{\fbackref{44:25} Lit. \fbib{cause to stand}} on the vows that we vowed to offer sacrifices to the Queen of Heaven and pour out liquid offerings to her!'' Go ahead, carry through on your vows, and diligently do what you vowed!' \v{26}But\fnote{\fbackref{44:26} Or \fbib{Therefore}} listen to this message from the \divine{Lord}, all you people of\fnote{\fbackref{44:26} The Heb. lacks \fbib{you people of}} Judah who are living in the land of Egypt. `Look, I've sworn by my great name', says the \divine{Lord}, `my name will no longer be invoked by the mouth of any person in the entire land of Egypt, as he says, ``As surely as the Lord \divine{God}\fnote{\fbackref{44:26} MT word usually translated \fbib{\divine{Lord}}} lives{\ldots}''\fnote{\fbackref{44:26} I.e. using the \divine{Lord}'s name to confirm an oath}

\v{27}```Look, I'm watching over them to bring disaster rather than good. Every person of Judah in the land of Egypt will be brought to an end by the sword and by famine until they're completely gone. \v{28}The ones who escape the sword will return from the land of Egypt to the land of Judah, few in number. Then all the remnant of Judah who have come into the land of Egypt to settle will know whose message will stand, mine or theirs. \v{29}This will be a sign to you,' declares the \divine{Lord}, `that I'll punish you in this place so that you may know that my words concerning disaster against you will surely stand.'

\v{30}This is what the \divine{Lord} says: ``Look, I'm going to give Pharaoh Hophra, king of Egypt, into the hands of his enemies and into the hands of those seeking his life, just as I gave Zedekiah king of Judah into the hands of Nebuchadnezzar king of Babylon, his enemy who was seeking his life.''
\labelchapt{45}
\passage{God's Message to Baruch}

\chapt{45}
\v{1}This is\fnote{\fbackref{45:1} The Heb. lacks \fbib{This is}} the message that Jeremiah the prophet spoke to Neriah's son Baruch, when in the fourth year of the reign of\fnote{\fbackref{45:1} The Heb. lacks \fbib{the reign of}} Josiah's son King Jehoiakim of Judah had, at Jeremiah's dictation, written these words in a scroll: \v{2}``This is what the \divine{Lord} God of Israel says to you, Baruch: \v{3}`You have said, ``How terrible for me, for the \divine{Lord} has added sorrow to my pain. I'm weary with my groaning, and I haven't found rest.''\,' \v{4}Say this to him: `This is what the \divine{Lord} says: ``Look! What I've built I'm about to tear down, and what I've planted I'm about to pull up---and this will involve the entire land.'' \v{5}Are you seeking great things for yourself? Don't seek them. Indeed, I'm about to bring disaster on all flesh,' declares the \divine{Lord}, `but your life will be spared\fnote{\fbackref{45:5} Lit. \fbib{life as a spoil of war}} wherever you go.'\,''
\labelchapt{46}
\passage{Prophecies against the Nations}

\chapt{46}
\v{1}This is\fnote{\fbackref{46:1} The Heb. lacks \fbib{This is}} the message from the \divine{Lord} that came to Jeremiah the prophet concerning the nations.
\passage{Prophecies against Egypt: Its Defeat at Carchemish}

\v{2}To Egypt: Concerning the army of King Pharaoh Neco of Egypt, which was encamped by the Euphrates River at Carchemish and which King Nebuchadnezzar of Babylon defeated in the fourth year of the reign of\fnote{\fbackref{46:2} The Heb. lacks \fbib{the reign of}} Josiah's son Jehoiakim, king of Judah.

\begin{poetry}
\poeml \v{3}``Prepare buckler and shield, \\
\poemll    and advance into the battle! \\
\poeml \v{4}Harness the horses! \\
\poemll    Riders, mount up! \\
\poeml Take your\fnote{\fbackref{46:4} The Heb. lacks \fbib{your}} positions with your\fnote{\fbackref{46:4} The Heb. lacks \fbib{your}} helmets! \\
\poemll    Polish lances, \\
\poemll    and put on armor! \\
\poeml \v{5}Why am I seeing this?\fnote{\fbackref{46:5} The Heb. lacks \fbib{this}} \\
\poeml They're terrified, \\
\poemll    they have turned back. \\
\poeml Their warriors are crushed, \\
\poemll    and they take flight. \\
\poeml They don't look back. \\
\poemll    Terror is on every side,'' \\
\poemlll       declares the \divine{Lord}. \\
\poeml \v{6}``The swift cannot flee, \\
\poemll    nor can the strong escape. \\
\poeml In the north, beside the Euphrates River, \\
\poemll    they stumble and fall. \\
\poeml \v{7}Who is this, rising like the Nile, \\
\poemll    like rivers whose waters surge? \\
\poeml \v{8}Egypt is rising like the Nile, \\
\poemll    like rivers whose waters surge. \\
\poeml He says, `I'll rise and cover the land.\fnote{\fbackref{46:8} Or \fbib{earth}} \\
\poemll    I'll destroy the city and its inhabitants.' \\
\poeml \v{9}Horses, get up! \\
\poeml Chariots, drive furiously! \\
\poeml Let the warriors go forward, \\
\poemll    Ethiopia and Put, who carry shields, \\
\poemlll       and the Lydians who handle and bend the bow. \\
\poeml \v{10}That day belongs to the \divine{Lord} of the Heavenly Armies. \\
\poemll    It is a day of vengeance to take vengeance on his foes. \\
\poeml The sword will devour and be satisfied, \\
\poemll    and will drink its fill of their blood. \\
\poeml For the Lord \divine{God} of the Heavenly Armies \\
\poemll    will hold a sacrifice in the land of the north, \\
\poemlll       by the Euphrates river. \\
\poeml \v{11}Go up to Gilead and get balm,\fnote{\fbackref{46:11} \fbib{Balm} was a spice with medicinal uses.} \\
\poemll    virgin daughter of Egypt! \\
\poeml In vain you multiply remedies, \\
\poemll    but there is no healing for you. \\
\poeml \v{12}The nations have heard of your disgrace, \\
\poemll    and your cry of distress fills the earth. \\
\poeml Indeed, one warrior stumbles over another, \\
\poemll    and both of them fall down together.''
\end{poetry}
\passage{Nebuchadnezzar's Conquest of Egypt}

\v{13}This is the message that the \divine{Lord} spoke to Jeremiah the prophet about the coming of King Nebuchadnezzar of Babylon to conquer\fnote{\fbackref{46:13} Lit. \fbib{strike down}} the land of Egypt.

\begin{poetry}
\poeml \v{14}``Announce in Egypt, proclaim in Migdol. \\
\poemll    Proclaim also in Memphis and Tahpanhes. \\
\poeml Say, `Take your positions and be ready, \\
\poemll    for the sword will devour all around you.' \\
\poeml \v{15}Why are your warriors prostrate? \\
\poemll    They don't stand\fnote{\fbackref{46:15} Or \fbib{Why does Apis flee and your bull not stand?}} because the \divine{Lord} has brought them down. \\
\poeml \v{16}They repeatedly stumble and fall. \\
\poemll    They say to each other, `Get up! \\
\poeml Let's go back to our people \\
\poemll    and to the land of our birth, \\
\poemlll       away from the oppressor's sword.' \\
\poeml \v{17}There they'll cry out, \\
\poeml `Pharaoh, king of Egypt, is just a big noise. \\
\poemll    He has let the appointed time pass by.'\fnote{\fbackref{46:17} I.e. he has missed the opportunity} \\
\poeml \v{18}As certainly as I'm alive and living,'' declares the King, \\
\poemll    whose name is the \divine{Lord} of the Heavenly Armies, \\
\poeml ``Indeed, one will come like Tabor among the mountains \\
\poemll    and like Carmel by the sea. \\
\poeml \v{19}Prepare your baggage for exile, \\
\poemll    daughter living in Egypt, \\
\poeml for Memphis will become a desolate place. \\
\poemll    It will become a ruin without inhabitant. \\
\poeml \v{20}Egypt is a beautiful calf,\fnote{\fbackref{46:20} Or \fbib{heifer}} \\
\poemll    but a horsefly from the north is surely coming. \\
\poeml \v{21}Even the mercenary troops in her ranks \\
\poemll    are like a fattened calf. \\
\poeml They too will turn around, \\
\poemll    and will flee together. \\
\poeml They won't stand, \\
\poemll    for the day of their disaster is coming on them, \\
\poemlll       the time of their punishment. \\
\poeml \v{22}Her cry will be like that of a fleeing serpent \\
\poemll    when they come in strength. \\
\poemlll       They're coming to her with axes like woodcutters. \\
\poeml \v{23}They'll cut down her forest, though it's impenetrable,'' \\
\poemll    declares the \divine{Lord}, \\
\poeml ``for they're more numerous than locusts, \\
\poemll    and there are too many of them to count. \\
\poeml \v{24}The daughter of Egypt will be put to shame, \\
\poemll    she will be given into the hands of the people from the north.''
\end{poetry}

\v{25}The \divine{Lord} of the Heavenly Armies, the God of Israel says, ``Look, I'm going to punish Amon of Thebes, Pharaoh, Egypt, its gods and its kings, Pharaoh, and those who trust in him. \v{26}I'll give them to those who are seeking their lives and to King Nebuchadnezzar of Babylon and his officers.\fnote{\fbackref{46:26} Or \fbib{servants}} Then afterwards, Egypt will be inhabited as in times past,'' declares the \divine{Lord}.
\passage{Israel will be Delivered}

\begin{poetry}
\poeml \v{27}``As for you, my servant Jacob, don't be afraid, \\
\poemll    and Israel, don't be dismayed. \\
\poeml For I'll deliver you from a distant place, \\
\poemll    and your descendants from the land of their captivity. \\
\poeml Jacob will return. \\
\poemll    He will be undisturbed and secure, \\
\poemlll       and no one will cause him to fear. \\
\poeml \v{28}``As for you, my servant Jacob, don't be afraid, \\
\poemll    and Israel, don't be dismayed,'' \\
\poemlll       declares the \divine{Lord}, ``for I am with you. \\
\poeml Indeed, I'll make an end of all the nations \\
\poemll    where I scattered you; \\
\poemll    but I won't make an end of you! \\
\poeml I'll discipline you justly, \\
\poemll    but I'll certainly not leave you unpunished.''
\end{poetry}
\labelchapt{47}
\passage{A Prophecy against the Philistines}

\chapt{47}
\v{1}This is\fnote{\fbackref{47:1} The Heb. lacks \fbib{This is}} the message from the \divine{Lord} that came to Jeremiah the prophet concerning the Philistines, before Pharaoh conquered Gaza. \v{2}This is what the \divine{Lord} says:

\begin{poetry}
\poeml ``Look, waters are rising from the north, \\
\poemll    and they'll become an overflowing river. \\
\poeml They'll overflow the land and all that fills it\fnote{\fbackref{47:2} Lit. \fbib{its fullness}}--- \\
\poemll    the city and those that live in it. \\
\poeml People will cry out, \\
\poemll    and all those living in the land will wail. \\
\poeml \v{3}At the sound of the galloping hooves of his horses,\fnote{\fbackref{47:3} Lit. \fbib{his strong ones}} \\
\poemll    at the rumbling of his chariots, \\
\poemlll       the clatter of his wheels, \\
\poeml fathers won't turn back for their\fnote{\fbackref{47:3} The Heb. lacks \fbib{their}} children \\
\poemll    because their hands are weak, \\
\poeml \v{4}for the day is coming to destroy all the Philistines, \\
\poemll    to cut off from Tyre and Sidon \\
\poemlll       every helper who remains. \\
\poeml For the \divine{Lord} is destroying the Philistines, \\
\poemll    the remnant of the coastlands of Caphtor.\fnote{\fbackref{47:4} I.e. Crete and the Aegean islands from which the Philistines came} \\
\poeml \v{5}Baldness\fnote{\fbackref{47:5} I.e. the head was shaved as a rite of mourning} is coming to Gaza. \\
\poemll    Ashkelon is silenced. \\
\poeml Remnant of their valley, \\
\poemll    how long will you gash yourself?\fnote{\fbackref{47:5} I.e. people made cuts on their bodies as an act of mourning} \\
\poeml \v{6}Ah, sword of the \divine{Lord}, \\
\poemll    how long before you are quiet? \\
\poeml Put yourself into your scabbard, \\
\poemll    be at rest, be silent! \\
\poeml \v{7}How can it be quiet, \\
\poemll    when the \divine{Lord} has ordered disaster \\
\poeml to come to Ashkelon and the seashore? \\
\poemll    That's where he has assigned it.''
\end{poetry}
\labelchapt{48}
\passage{A Prophecy against Moab}

\chapt{48}
\v{1}To Moab: This is what the \divine{Lord} of the Heavenly Armies, the God of Israel, says:

\begin{poetry}
\poeml ``How terrible for Nebo, for it's laid waste. \\
\poemll    Kiriathaim is put to shame, it's captured. \\
\poemlll       The fortress is put to shame, it's shattered. \\
\poeml \v{2}The pride of Moab is no more. \\
\poemll    In Heshbon they plotted evil against her: \\
\poeml `Come and let's eliminate her as a nation.' \\
\poemll    Madmen\fnote{\fbackref{48:2} \fbib{Madmen} was a town in Moab. The name sounds like the Heb. word \fbib{silenced}} will also be silenced, \\
\poemlll       and the sword will pursue you. \\
\poeml \v{3}The sound of crying will come from Horonaim, \\
\poemll    devastation and great destruction. \\
\poeml \v{4}Moab will be destroyed; \\
\poemll    her little ones will cry out. \\
\poeml \v{5}Indeed, at the ascent of Luhith \\
\poemll    people will go up with bitter weeping. \\
\poeml At the descent of Horonaim \\
\poemll    the anguished cries over the destruction will be heard. \\
\poeml \v{6}Flee, save your lives, \\
\poemll    and you will be like a wild donkey\fnote{\fbackref{48:6} So LXX; MT reads \fbib{Aroer}} in the desert. \\
\poeml \v{7}But, because you trust in your deeds and your riches, \\
\poemll    you will also be captured. \\
\poeml Chemosh\fnote{\fbackref{48:7} \fbib{Chemosh} was the chief Moabite deity.} will go out into exile, \\
\poemll    along with his priests and officials. \\
\poeml \v{8}A destroyer will come to every town \\
\poemll    and no town will escape. \\
\poeml The valley will be ruined and the plateau destroyed.'' \\
\poemll    This\fnote{\fbackref{48:8} The Heb. lacks \fbib{this}.} is what the \divine{Lord} has said! \\
\poeml \v{9}``Put salt\fnote{\fbackref{48:9} Or \fbib{take wing}} on Moab\fnote{\fbackref{48:9} I.e. as a sign of destruction} \\
\poemll    for she will surely fall. \\
\poeml Her towns will become desolate places,\fnote{\fbackref{48:9} Lit. \fbib{a desolation}} \\
\poemll    without any inhabitants in them. \\
\poeml \v{10}Cursed is the one who is slack \\
\poemll    in doing the \divine{Lord}'s work. \\
\poeml Cursed is the one who holds back his sword \\
\poemll    from shedding\fnote{\fbackref{48:10} The Heb. lacks \fbib{shedding}} blood. \\
\poeml \v{11}Moab has been at ease from his youth. \\
\poemll    He has been undisturbed like wine\fnote{\fbackref{48:11} The Heb. lacks \fbib{like wine}} on its dregs \\
\poemlll       and not poured from vessel to vessel. \\
\poeml He has not gone into exile. \\
\poemll    Therefore, his flavor has remained, \\
\poemlll       and his aroma has not changed.
\end{poetry}

\v{12}``Therefore, look, days are coming,'' declares the \divine{Lord}, ``when I'll send those who tip over vessels\fnote{\fbackref{48:12} The Heb. lacks \fbib{vessels}} to him, and they'll tip him over. They'll empty his vessels and shatter his jars. \v{13}Moab will be ashamed because of Chemosh just as the house of Israel was ashamed because of Bethel, their confidence.

\begin{poetry}
\poeml \v{14}``How can you say, `We're strong warriors, \\
\poemll    and soldiers ready\fnote{\fbackref{48:14} The Heb. lacks \fbib{ready}} for battle'? \\
\poeml \v{15}Moab will be destroyed, \\
\poemll    and the enemy\fnote{\fbackref{48:15} Lit. \fbib{he}} will come up against her cities. \\
\poeml Her finest young men will go down to slaughter,'' \\
\poemll    declares the King, \\
\poemlll       whose name is the \divine{Lord} of the Heavenly Armies. \\
\poeml \v{16}``Moab's disaster is near at hand, \\
\poemll    and his calamity is coming very quickly. \\
\poeml \v{17}Mourn for him, all who live around him, \\
\poemll    and all who know his name. \\
\poeml Say, `Oh how the mighty rod is broken, \\
\poemll    the glorious staff.' \\
\poeml \v{18}``Come down from glory, and sit on parched ground, \\
\poemll    O woman who lives in Dibon, \\
\poeml for the destroyer of Moab will come up \\
\poemll    against you to destroy you. \\
\poemlll       He will destroy your strongholds. \\
\poeml \v{19}Stand by the road and keep watch, \\
\poemll    O woman who lives in Aroer. \\
\poeml Ask the man who flees and the woman who escapes. \\
\poemll    Say, `What happened'? \\
\poeml \v{20}Moab will be put to shame, \\
\poemll    for it will be destroyed. \\
\poeml Wail and cry out. \\
\poemll    Announce by the Arnon that Moab is destroyed. \\
\poeml \v{21}Judgment has come to the plateau:\fnote{\fbackref{48:21} Moab was located on a plateau overlooking the Jordan River.} \\
\poemll    to Holon and Jahzah, \\
\poeml and against Mephaath, \\
\poemll    \v{22}Dibon, Nebo, and Beth-diblathaim, \\
\poeml \v{23}against Kiriathaim, Beth-gamul, and Beth-meon, \\
\poeml \v{24}against Kerioth, Bozrah, \\
\poeml and all the towns in the land of Moab, \\
\poemll    both far and near. \\
\poeml \v{25}The strength\fnote{\fbackref{48:25} Lit. \fbib{horn}} of Moab is cut off, and his arm is broken,'' \\
\poemll    declares the \divine{Lord}. \\
\poeml \v{26}``Make him drunk for he has exalted himself \\
\poemll    against the \divine{Lord}. \\
\poeml Moab will wallow in his vomit, \\
\poemll    and he will be the object of mocking. \\
\poeml \v{27}Wasn't Israel an object of mocking for you? \\
\poemll    Wasn't he treated like a thief,\fnote{\fbackref{48:27} Lit. \fbib{found among thieves}} \\
\poeml so that whenever you spoke about him \\
\poemll    you shook your head in contempt?\fnote{\fbackref{48:27} The Heb. lacks \fbib{in contempt}} \\
\poeml \v{28}Abandon the cities, and live on the cliffs,\fnote{\fbackref{48:28} Or \fbib{among the crags}; i.e. on the face of a cliff} \\
\poemll    you inhabitants of Moab. \\
\poeml Be like a dove that builds a nest \\
\poemll    by the mouth of a cave. \\
\poeml \v{29}We have heard about Moab's pride--- \\
\poemll    he's very proud--- \\
\poeml his haughtiness, his arrogance, \\
\poemll    his insolence, and his conceit.\fnote{\fbackref{48:29} Lit. \fbib{the elevation of his heart}} \\
\poeml \v{30}I know his gall,'' \\
\poemll    declares the \divine{Lord}, \\
\poeml ``and it's futile; \\
\poemll    the boasting that they do is futile. \\
\poeml \v{31}Therefore, I'll wail for Moab, \\
\poemll    and for the whole of Moab I'll cry out, \\
\poemll    for the men of Kir-heres I'll moan. \\
\poeml \v{32}More than the weeping for Jazer,\fnote{\fbackref{48:32} \fbib{Jazer} was a town in Moab.} \\
\poemll    I'll weep for you, vine of Sibmah.\fnote{\fbackref{48:32} \fbib{Sibmah} was a town in Moab noted for its vineyards.} \\
\poeml Your branches spread out to the sea, \\
\poemll    and reached as far as the Sea of Jazer.\fnote{\fbackref{48:32} I.e. the Dead Sea} \\
\poeml On your summer fruit and grapes\fnote{\fbackref{48:32} Lit. \fbib{your vintage}} \\
\poemll    the destroyer will fall. \\
\poeml \v{33}Gladness and rejoicing will be taken away \\
\poemll    from the fruitful land.\fnote{\fbackref{48:33} Or \fbib{from Carmel}} \\
\poeml From the land of Moab I'll cause the wine \\
\poemll    in the wine presses to stop flowing.\fnote{\fbackref{48:33} The Heb. lacks \fbib{flowing}} \\
\poeml The workers won't tread\fnote{\fbackref{48:33} Lit. \fbib{He won't tread}} the grapes with a loud shout.\fnote{\fbackref{48:33} Lit. \fbib{with a shout, shout}} \\
\poemll    There will be no shout!
\end{poetry}

\v{34}``From the cry of Heshbon, to Elealeh, to Jahaz they have lifted up their voice. From Zoar to Horonaim and to Eglath-shelishiyah, even the waters of Nimrim will become a desolate place. \v{35}In Moab,'' declares the \divine{Lord}, ``I'll put an end to the one who offers a burnt offering on the high place and to the one who burns incense to his gods. \v{36}Therefore my heart wails for Moab like flutes\fnote{\fbackref{48:36} I.e. flutes were played as a part of mourning for the dead} and my heart wails for the men of Kir-heres like flutes. Therefore they'll lose the abundance they produced. \v{37}Indeed every head will be bald\fnote{\fbackref{48:37} I.e. heads were shaved as a sign of mourning} and every beard cut short.\fnote{\fbackref{48:37} I.e. beards were cut short as a sign of mourning} There will be gashes on all the hands\fnote{\fbackref{48:37} I.e. people cut themselves as a sign of mourning} and sackcloth on the loins. \v{38}On all the housetops of Moab and in the streets there will be nothing but mourning, for I'll break Moab like a vessel that no one wants,'' declares the \divine{Lord}. \v{39}``How it will be shattered! How they'll wail! How Moab will turn his back in shame! Moab will be an object of ridicule and terror to all those around him.''

\begin{poetry}
\poeml \v{40}For this is what the \divine{Lord} says: \\
\poeml ``Look, like an eagle one will fly swiftly \\
\poemll    and spread his wings against Moab. \\
\poeml \v{41}The towns\fnote{\fbackref{48:41} Or \fbib{Kerioth}} will be captured \\
\poemll    and the strongholds seized. \\
\poeml On that day the hearts of the warriors of Moab \\
\poemll    will be like the heart of a woman in labor. \\
\poeml \v{42}Moab will be destroyed as a nation\fnote{\fbackref{48:42} Lit. \fbib{from being a nation}} \\
\poemll    because he exalted himself against the \divine{Lord}. \\
\poeml \v{43}Terror, pit, and trap will be used\fnote{\fbackref{48:43} The Heb. lacks \fbib{used}} against you \\
\poemll    who live in Moab,'' \\
\poemlll       declares the \divine{Lord}. \\
\poeml \v{44}``The one who flees from the terror \\
\poemll    will fall into a pit. \\
\poeml And the one who comes up out of the pit \\
\poemll    will be caught in a trap. \\
\poeml For I'll bring upon her, that is upon Moab, \\
\poemll    the time of her\fnote{\fbackref{48:44} Lit. \fbib{their}} punishment,'' \\
\poemlll       declares the \divine{Lord}. \\
\poeml \v{45}``The fugitives will stand without strength \\
\poemll    in the shadow of Heshbon, \\
\poeml for fire will go out from Heshbon \\
\poemll    and a flame from the middle of Sihon. \\
\poeml It will devour the forehead of Moab \\
\poemll    and the heads of the rebellious people.\fnote{\fbackref{48:45} Lit. \fbib{sons of tumult}} \\
\poeml \v{46}How terrible for you, Moab! \\
\poemll    The people of Chemosh\fnote{\fbackref{48:46} \fbib{Chemosh} was the chief deity of Moab.} will perish. \\
\poeml Indeed, your sons will be taken into captivity, \\
\poemll    and your daughters as well.\fnote{\fbackref{48:46} Lit. \fbib{daughters into captivity}} \\
\poeml \v{47}But I'll restore the fortunes of Moab in the latter days,'' \\
\poemll    declares the \divine{Lord}. \\
\poeml This concludes the judgment on Moab.
\end{poetry}
\labelchapt{49}
\passage{Prophecies against Ammon}

\chapt{49}
\v{1}To the people of Ammon:

This is what the \divine{Lord} says:

\begin{poetry}
\poeml ``Does Israel have no sons? \\
\poemll    Does he have no heir? \\
\poeml Why then has Milcom\fnote{\fbackref{49:1} \fbib{Milcom} or Molech was the chief deity of the Ammonites.} taken possession of Gad, \\
\poemll    and his people settled in its towns? \\
\poeml \v{2}Therefore, look, the time is\fnote{\fbackref{49:2} Lit. \fbib{the days are}} coming,'' \\
\poemll    declares the \divine{Lord}, \\
\poeml ``when I'll cause a battle cry to be heard in \\
\poemll    Rabbah\fnote{\fbackref{49:2} \fbib{Rabbah} was the capital city of Ammon.} of the Ammonites. \\
\poeml It will become a desolate mound, \\
\poemll    and its towns will be burned with fire. \\
\poeml Israel will take possession of those who possessed him,'' \\
\poemll    says the \divine{Lord}. \\
\poeml \v{3}``Wail, Heshbon, because Ai is destroyed. \\
\poemll    Cry out, daughters of Rabbah, \\
\poemlll       put on sackcloth and lament. \\
\poeml Run back and forth inside the walls, \\
\poemll    for Milcom is going into exile \\
\poemlll       along with his priests and his princes. \\
\poeml \v{4}Why do you boast in your valleys? \\
\poeml Your valley is flowing away, faithless daughter, \\
\poemll    who trusted in her treasures, \\
\poemlll       saying, `Who will come against me?' \\
\poeml \v{5}Look, I'm bringing terror on you from all around you,'' \\
\poemll    declares the Lord GOD of the Heavenly Armies. \\
\poeml ``You will be driven out, fleeing recklessly,\fnote{\fbackref{49:5} Lit. \fbib{each one before him}} \\
\poemll    and there will be no one to gather the fugitives. \\
\poeml \v{6}But afterwards I'll restore the fortunes of \\
\poemll    the people of Ammon,''\fnote{\fbackref{49:6} Lit. \fbib{the sons of Ammon}} \\
\poemlll       declares the \divine{Lord}.
\end{poetry}
\passage{Prophecies against Edom}

\v{7}To Edom:

\begin{poetry}
\poeml This is what the \divine{Lord} of the Heavenly Armies says: \\
\poemll    ``Is there no longer wisdom in Teman? \\
\poeml Has counsel perished among the prudent? \\
\poemll    Is their wisdom gone? \\
\poeml \v{8}Flee, turn around! \\
\poemll    Go to a remote place to stay, \\
\poeml residents of Dedan! \\
\poemll    For I'll bring Esau's\fnote{\fbackref{49:8} I.e. the Edomites were descendants of Esau} disaster on him \\
\poemll    at the time when I punish him. \\
\poeml \v{9}If the grape harvesters came to you, \\
\poemll    would they not leave gleanings? \\
\poeml If thieves came at night, they would destroy \\
\poemll    only until they had enough. \\
\poeml \v{10}But I'll strip Esau bare. \\
\poemll    I'll uncover his hiding places so he cannot conceal himself. \\
\poeml His offspring, his relatives, \\
\poemll    and his neighbors will be destroyed, \\
\poemlll       and he will no longer exist. \\
\poeml \v{11}Leave your orphans. I'll keep them alive. \\
\poemll    Let your widows trust in me.''
\end{poetry}

\v{12}For this is what the \divine{Lord} says: ``Look, those who don't deserve\fnote{\fbackref{49:12} Lit. \fbib{it was not their judgment}} to drink the cup will surely drink it, and will you actually go unpunished? You won't go unpunished! You will certainly drink it!\fnote{\fbackref{49:12} The Heb. lacks \fbib{it}} \v{13}Indeed, I've sworn by myself,'' declares the \divine{Lord}, ``that Bozrah will become an object of horror and scorn, a waste, and an object of ridicule. All her towns will become perpetual ruins.''

\begin{poetry}
\poeml \v{14}I've heard a message from the \divine{Lord}, \\
\poemll    and a messenger has been sent \\
\poemlll       among the nations: \\
\poeml ``Gather together and come up against her, \\
\poemll    and rise up to fight. \\
\poeml \v{15}Indeed, I'll make you the least of the nations, \\
\poemll    despised among men. \\
\poeml \v{16}The terror you cause and the pride of your heart have deceived you. \\
\poeml You who live in hidden places in the rocks, \\
\poemll    who hold on to the heights of the hill, \\
\poeml although you make your nest high like the eagle, \\
\poemll    I'll bring you down from there,'' \\
\poemlll       declares the \divine{Lord}.
\end{poetry}

\v{17}``Edom will become an object of horror. Everyone who passes by her will be horrified and will scoff\fnote{\fbackref{49:17} Lit. \fbib{hiss}; i.e. hissing was an expression of contempt} because of all her wounds. \v{18}Just like the overthrow of Sodom and Gomorrah and their\fnote{\fbackref{49:18} Lit. \fbib{her}} neighbors,'' says the \divine{Lord}, ``no one will live there. No human being will reside in it. \v{19}Look, like a lion comes up from the thicket of the Jordan to a pasture that grows year round,\fnote{\fbackref{49:19} Lit. \fbib{a perpetual pasture}} so I'll drive them\fnote{\fbackref{49:19} So LXX; Heb. \fbib{him}; i.e. the Edomites} away from her in an instant, and I'll appoint whomever is chosen over her. Indeed, who is like me? Who gives me counsel? Who is the shepherd who will stand against me?'' \v{20}Therefore, hear the plan that the \divine{Lord} has made against Edom, and the strategy that he devised against the inhabitants of Teman. Surely he will drag the little ones of the flock away. Surely their pasture will be desolate because of them. \v{21}The earth will quake at the sound of their fall. A cry---it's her voice---is heard at the Reed\fnote{\fbackref{49:21} So MT; LXX lacks \fbib{Reed}} Sea. \v{22}Look, he will rise up and fly swiftly like an eagle. He will spread his wings against Bozrah, and on that day the hearts of the warriors of Edom will be like the heart of a woman in labor.
\passage{Prophecies against Damascus}

\v{23}To Damascus:

\begin{poetry}
\poeml ``Hamath and Arpad will be humiliated. \\
\poemll    Their courage melts because they have heard bad news. \\
\poemlll       There is anxiety like\fnote{\fbackref{49:23} Lit. \fbib{by}} the sea that cannot be calmed. \\
\poeml \v{24}Damascus will become weak. \\
\poemll    She will turn to flee, but panic will seize her. \\
\poeml Distress and anguish will take hold of her \\
\poemll    like that of\fnote{\fbackref{49:24} The Heb. lacks \fbib{that of}} a woman giving birth. \\
\poeml \v{25}Why\fnote{\fbackref{49:25} Lit. \fbib{How}} is the famous city,\fnote{\fbackref{49:25} Lit. \fbib{city of praise}} the joyful town, \\
\poemll    not abandoned? \\
\poeml \v{26}Therefore her young men will fall in her streets, \\
\poemll    and all her soldiers will be silenced on that day,'' \\
\poemlll       declares the \divine{Lord} of the Heavenly Armies. \\
\poeml \v{27}``I'll kindle a fire in the wall of Damascus, \\
\poemll    and it will devour the strongholds of Ben-hadad.''
\end{poetry}
\passage{Prophecies against Kedar and Hazor}

\v{28}To Kedar and the kingdoms of Hazor that King Nebuchadnezzar of Babylon destroyed:

\begin{poetry}
\poeml This is what the \divine{Lord} says: \\
\poeml ``Arise, go against Kedar! \\
\poemll    Plunder the people of the east! \\
\poeml \v{29}Take their tents and their flocks, \\
\poemll    their tent curtains and all their goods. \\
\poeml Take their camels away from them. \\
\poemll    Cry out against them, `Terror is all around!' \\
\poeml \v{30}Flee! Run away quickly! \\
\poemll    Go to a remote place to stay, residents of Hazor,'' \\
\poemlll       declares the \divine{Lord}. \\
\poeml ``For King Nebuchadnezzar of Babylon has formed a plan \\
\poemll    and devised a strategy against them. \\
\poeml \v{31}``Arise, go up against a nation at ease, living securely,'' \\
\poemll    declares the \divine{Lord}, \\
\poemlll       ``without gates or bars, living alone. \\
\poeml \v{32}Their camels will become booty, \\
\poemll    their many herds will become spoil. \\
\poeml I'll scatter to the winds \\
\poemll    those who shave the corners of their beards,\fnote{\fbackref{49:32} Lit. \fbib{cut off of the side}} \\
\poeml and I'll bring disaster on them from every side,'' \\
\poemll    declares the \divine{Lord}. \\
\poeml \v{33}``Hazor will become a dwelling place for jackals, \\
\poemll    a perpetual wasteland. \\
\poeml No one will live there; \\
\poemll    no human being will reside in it.''
\end{poetry}
\passage{Prophecies against Elam}

\v{34}This is what came as a message from the \divine{Lord} to Jeremiah the prophet about Elam at the beginning of the reign of King Zedekiah of Judah:

\begin{poetry}
\poeml \v{35}This is what the \divine{Lord} of the Heavenly Armies says: \\
\poeml ``Look, I'm going to break the bow of Elam, \\
\poemll    the finest of their troops. \\
\poeml \v{36}I'll bring the four winds against Elam \\
\poemll    from the four corners of the heavens, \\
\poeml and I'll scatter them to all these winds. \\
\poemll    There will be no nation to which the exiles \\
\poemlll       from Elam won't go. \\
\poeml \v{37}I'll terrify Elam before their enemies \\
\poemll    and before those who seek to kill them. \\
\poeml I'll bring on them disaster and become fiercely angry at them,'' \\
\poemll    declares the \divine{Lord}. \\
\poeml ``I'll send the sword after them, \\
\poemll    until I've made an end of them. \\
\poeml \v{38}I'll put my throne in Elam, \\
\poemll    and destroy the king and the officials there,'' \\
\poemlll       declares the \divine{Lord}. \\
\poeml \v{39}``But in the latter days I'll restore \\
\poemll    the fortunes of Elam,'' \\
\poemlll       declares the \divine{Lord}.
\end{poetry}
\labelchapt{50}
\passage{Prophecies against Babylon}

\chapt{50}
\v{1}This is\fnote{\fbackref{50:1} The Heb. lacks \fbib{This is}} the message that the \divine{Lord} spoke through the prophet Jeremiah about Babylon, the land of the Chaldeans.

\begin{poetry}
\poeml \v{2}``Declare and proclaim among the nations. \\
\poemll    Lift up a banner and proclaim. \\
\poeml Don't conceal anything.\fnote{\fbackref{50:2} The Heb. lacks \fbib{anything}} \\
\poemll    Say, `Babylon will be captured. \\
\poeml Bel\fnote{\fbackref{50:2} \fbib{Bel} was another name for \fbib{Marduk}, the sun god of Babylon} will be disgraced, \\
\poemll    and Marduk will be destroyed. \\
\poeml Her idols will be disgraced, \\
\poemll    and her filthy images will be destroyed.' \\
\poeml \v{3}For a nation from the north will go up against her. \\
\poemll    It will make her land into an object of horror, \\
\poemlll       and no one will live in it. \\
\poeml Both people and animals will wander off, \\
\poemll    and they'll leave. \\
\poeml \v{4}In those days, and at that time,'' \\
\poemll    declares the \divine{Lord}, \\
\poeml ``the people of Israel will come together \\
\poemll    with the people of Judah. \\
\poeml They'll be weeping as they travel along, \\
\poemll    and they'll be seeking the \divine{Lord} their God. \\
\poeml \v{5}They'll ask the way to Zion, \\
\poemll    turning their faces in that direction. \\
\poeml They'll come\fnote{\fbackref{50:5} So with LXX; MT reads \fbib{Come!}} and join themselves to the \divine{Lord} \\
\poemll    in an everlasting covenant that won't be forgotten. \\
\poeml \v{6}My people have become lost sheep. \\
\poemll    Their shepherds have led them astray, \\
\poemlll       turning them toward the mountains. \\
\poeml They go from mountain to hill. \\
\poemll    They have forgotten their resting place. \\
\poeml \v{7}All who find them devour them, \\
\poemll    but their enemies say, `We're not guilty, \\
\poeml because they have sinned against \\
\poemll    the \divine{Lord}, the habitation of righteousness, \\
\poemlll       the \divine{Lord}, the hope of their ancestors.' \\
\poeml \v{8}Move away from the middle of Babylon, \\
\poemll    and go out of the land of the Chaldeans. \\
\poemlll       Be like male goats at the head\fnote{\fbackref{50:8} Lit. \fbib{in front of}} of the flock. \\
\poeml \v{9}Indeed, I'm going to stir up \\
\poemll    and bring against Babylon \\
\poeml a great company of nations \\
\poemll    from the land of the north. \\
\poeml They'll deploy for battle against her, \\
\poemll    and from there she will be captured. \\
\poeml Their arrows will be like a skilled warrior; \\
\poemll    they won't miss their targets.\fnote{\fbackref{50:9} Lit. \fbib{won't return empty-handed}} \\
\poeml \v{10}The Chaldeans will become plunder, \\
\poemll    and all who plunder them will get more than enough,'' \\
\poemlll       declares the \divine{Lord}. \\
\poeml \v{11}``Though you rejoice, though you exult, \\
\poemll    you plunderers of my inheritance, \\
\poeml though you skip around like a heifer in the grass\fnote{\fbackref{50:11} So LXX; MT reads \fbib{like a threshing heifer}} \\
\poemll    and neigh like stallions, \\
\poeml \v{12}your mother will be greatly devastated, \\
\poemll    she who gave birth to you will be ashamed. \\
\poeml She will become the least of the nations, \\
\poemll    a wilderness, a dry land, and a desert. \\
\poeml \v{13}Because of the anger of the \divine{Lord} \\
\poemll    she won't be inhabited, \\
\poemll    but will be utterly devastated. \\
\poeml Everyone who passes by Babylon will be horrified \\
\poemll    and will scoff\fnote{\fbackref{50:13} Lit. \fbib{hiss}; i.e. hissing was an expression of contempt} because of all her wounds. \\
\poeml \v{14}Deploy the troops all around Babylon. \\
\poemll    All who bend the bow, shoot at her \\
\poeml and spare no arrows, \\
\poemll    for she has sinned against the \divine{Lord}. \\
\poeml \v{15}Raise a battle cry against her on every side. \\
\poemll    She has surrendered,\fnote{\fbackref{50:15} Lit. \fbib{she has given her hand}} her pillars have fallen, \\
\poemlll       her walls are thrown down. \\
\poeml For this is the vengeance of the \divine{Lord}. \\
\poemll    Take vengeance on her; \\
\poemlll       as she has done, do to her. \\
\poeml \v{16}Eliminate from Babylon the one who plants seeds \\
\poemll    and the one who uses the sickle at harvest time. \\
\poeml Because of the oppressor's sword, let each one turn \\
\poemll    toward his own people and flee to his own land.''
\end{poetry}
\passage{Hope for Israel}

\v{17}``Israel is a scattered flock, driven out by lions. The first to devour him was the king of Assyria, and then afterward\fnote{\fbackref{50:17} The Heb. lacks \fbib{afterward}} King Nebuchadnezzar of Babylon gnawed\fnote{\fbackref{50:17} The Heb. lacks \fbib{gnawed}} his bones. \v{18}Therefore this is what the \divine{Lord} of the Heavenly Armies, the God of Israel, says: `Look, I'm about to judge the king of Babylon and his land, just as I've judged the king of Assyria. \v{19}I'll bring Israel back to his pasture. He will graze on Carmel, on Bashan, on Mt. Ephraim, and on Gilead---his hunger will be satisfied. \v{20}In those days and at that time,' declares the \divine{Lord}, `the iniquity of Israel will be searched for, but there will be none; and the sin of Judah, but none will be found, because I'll pardon those I leave as a remnant.'\,''
\passage{God's Judgment on Babylon}

\begin{poetry}
\poeml \v{21}``Go up against the land of Merathaim\fnote{\fbackref{50:21} \fbib{Merathaim} was an area in southern Mesopotamia; the Heb. word means \fbib{double rebellion}} \\
\poemll    and the inhabitants of Pekod.\fnote{\fbackref{50:21} \fbib{Pekod} was a region in southern Mesopotamia; the Heb. word means \fbib{punishment}} \\
\poeml Kill them with swords, and completely destroy them,'' \\
\poemll    declares the \divine{Lord}, \\
\poemlll       ``and do everything that I've commanded you. \\
\poeml \v{22}The noise of battle is in the land, \\
\poemll    and great destruction. \\
\poeml \v{23}How the hammer of all the earth is cut off and broken! \\
\poemll    How Babylon has become a horror among the nations! \\
\poeml \v{24}I'll set a trap for you, \\
\poemll    and you will be caught, Babylon, \\
\poemlll       but you don't realize it. \\
\poeml You will be found and also seized, \\
\poemll    because you challenged the \divine{Lord}! \\
\poeml \v{25}``The \divine{Lord} will open his armory, \\
\poemll    and bring out the weapons of his anger. \\
\poeml Indeed, a work of the Lord GOD\fnote{\fbackref{50:25} Heb. \fbib{Yahweh,} usually translated \fbib{\divine{Lord}}} of the Heavenly Armies \\
\poemll    will be in the land of the Chaldeans. \\
\poeml \v{26}Come to her from afar.\fnote{\fbackref{50:26} Lit. \fbib{from the end}} \\
\poemll    Open up her barns. \\
\poeml Pile her up like heaps of grain, \\
\poemll    and completely destroy her. \\
\poemlll       Don't leave any survivors. \\
\poeml \v{27}Put all her bulls to the sword, \\
\poemll    let them go down to the slaughter. \\
\poeml How terrible for them because their day has come, \\
\poemll    the time of their judgment. \\
\poeml \v{28}``The sound of fugitives and refugees \\
\poemll    will come from the land of Babylon \\
\poeml to declare in Zion the vengeance of the \divine{Lord} our God, \\
\poemll    vengeance for his Temple. \\
\poeml \v{29}``Summon many to Babylon, \\
\poemll    all those who bend the bow. \\
\poeml Camp all around her, \\
\poemll    let no one escape. \\
\poeml Repay her according to her deeds. \\
\poemll    Do to her just as she has done. \\
\poeml For she has behaved arrogantly against the Lord, \\
\poemll    against the Holy One of Israel. \\
\poeml \v{30}Therefore, her warriors will fall in her streets, \\
\poemll    and all her soldiers will be silenced on that day,'' \\
\poemlll       declares the \divine{Lord}. \\
\poeml \v{31}``Look, I'm against you, arrogant one,'' \\
\poemll    declares the \divine{Lord} God of the Heavenly Armies. \\
\poeml ``Indeed your day is coming, \\
\poemll    the time of your judgment. \\
\poeml \v{32}The arrogant one will stumble and fall, \\
\poemll    and there will be no one to lift him up. \\
\poeml I'll set fire to his cities, \\
\poemll    and it will devour everything around him.'' \\
\poeml \v{33}This is what the \divine{Lord} of the Heavenly Armies says: \\
\poeml ``The people of\fnote{\fbackref{50:33} Lit. \fbib{sons of}} Israel are oppressed, \\
\poemll    along with the people of\fnote{\fbackref{50:33} Lit. \fbib{sons of}} Judah. \\
\poeml All their captors have held on to them \\
\poemll    and refused to let them go. \\
\poeml \v{34}Their Redeemer\fnote{\fbackref{50:34} I.e. the one who pleads their case in a court of law} is strong, \\
\poemll    the \divine{Lord} of the Heavenly Armies is his name. \\
\poeml He will vigorously plead their case \\
\poemll    in order to bring rest to the earth, \\
\poemlll       but turmoil to the inhabitants of Babylon. \\
\poeml \v{35}A sword against the Chaldeans,'' \\
\poemll    declares the \divine{Lord}, \\
\poeml ``and against the inhabitants of Babylon, \\
\poemll    against her officials and her wise men. \\
\poeml \v{36}A sword against the diviners.\fnote{\fbackref{50:36} Lit. \fbib{empty talkers}; a pun on the Babylonian word for these priests} \\
\poemll    They'll be made fools. \\
\poeml A sword against her warriors. \\
\poemll    They'll be shattered. \\
\poeml \v{37}A sword against her horses, against her chariots,\fnote{\fbackref{50:37} Lit. \fbib{against his horses, against his chariots}} \\
\poemll    and against all the foreign troops\fnote{\fbackref{50:37} Lit. \fbib{mixed peoples}} in her midst. \\
\poeml They'll become women. \\
\poemll    A sword against her treasures. \\
\poemll    They'll be plundered. \\
\poeml \v{38}A drought against her waters. \\
\poemll    They'll dry up. \\
\poeml For it's a land of idols, \\
\poemll    and they go mad over their terrifying images. \\
\poeml \v{39}Therefore the desert creatures \\
\poemll    along with hyenas will live there. \\
\poeml They'll live in it with ostriches, \\
\poemll    but people won't live in it again. \\
\poemlll       They won't inhabit it from generation to generation. \\
\poeml \v{40}Just as when God overthrew Sodom, \\
\poemll    Gomorrah, and their neighbors,'' \\
\poemlll       declares the \divine{Lord}, \\
\poeml ``so also no one will live there. \\
\poemll    No human being will reside in it. \\
\poeml \v{41}``Look, people are coming from the north. \\
\poemll    A great nation and many kings will be stirred up \\
\poemlll       from the ends of the earth. \\
\poeml \v{42}They grab bow and spear. \\
\poemll    They're cruel and show no mercy. \\
\poeml Their sound roars like the sea, \\
\poemll    as they ride on horses \\
\poeml deployed like men ready for battle \\
\poemll    against you, daughter of Babylon. \\
\poeml \v{43}The king of Babylon has heard the news about them, \\
\poemll    and his hands hang limp. \\
\poeml Distress has seized him, \\
\poemll    like a woman in labor.
\end{poetry}

\v{44}``Look, like a lion comes up from the thicket of the Jordan to a pasture that grows year round,\fnote{\fbackref{50:44} Lit. \fbib{a perpetual pasture}} so I'll drive them away from her in an instant, and I'll appoint whomever is chosen over her. Indeed, who is like me? Who gives me counsel? Who is the shepherd who will stand against me?'' \v{45}Therefore, hear the plan that the \divine{Lord} has made against Babylon, and the strategy that he devised against the land of the Chaldeans. Surely they'll drag the little ones of the flock away. Surely their pasture will be desolate because of them. \v{46}At the shout that Babylon has been seized, the earth will be shaken, and the cry will be heard among the nations.
\labelchapt{51}
\passage{Judgment against Babylon}

\chapt{51}
\v{1}This is what the \divine{Lord} says:

\begin{poetry}
\poeml ``Look, I'm going to stir up a destroying wind \\
\poemll    against Babylon and the inhabitants of Leb-kamai.\fnote{\fbackref{51:1} I.e. a cryptogram for Chaldea} \\
\poeml \v{2}I'll send foreigners to Babylon, \\
\poemll    and they'll winnow her, \\
\poemlll       and devastate\fnote{\fbackref{51:2} Lit. \fbib{empty out}} her land. \\
\poeml They'll come against her from every side \\
\poemll    on the day of her\fnote{\fbackref{51:2} The Heb. lacks \fbib{her}} disaster. \\
\poeml \v{3}Don't let the archer\fnote{\fbackref{51:3} Lit. \fbib{one who bends the bow}} bend the bow; \\
\poemll    don't let him rise up in his armor. \\
\poeml Don't spare her young men. \\
\poemll    Completely destroy her entire army. \\
\poeml \v{4}The slain will fall in the land of Chaldea, \\
\poemll    pierced through in her streets. \\
\poeml \v{5}Indeed, Israel and Judah haven't been \\
\poemll    abandoned\fnote{\fbackref{51:5} Lit. \fbib{widowed}} by their\fnote{\fbackref{51:5} Lit. \fbib{his}} God, \\
\poeml by the \divine{Lord} of the Heavenly Armies, \\
\poemll    although their land is full of guilt \\
\poemlll       against the Holy One of Israel.'' \\
\poeml \v{6}Flee from Babylon,\fnote{\fbackref{51:6} Lit. \fbib{from the midst of Babylon}} \\
\poemll    and each of you, escape with your life! \\
\poeml Don't be destroyed\fnote{\fbackref{51:6} Or \fbib{silent}} because of her guilt, \\
\poemll    for it's time for the \divine{Lord}'s vengeance. \\
\poemlll       He is paying back what is due to her. \\
\poeml \v{7}Babylon was a golden cup in the \divine{Lord}'s hand, \\
\poemll    making the whole earth drunk. \\
\poeml The nations drank her wine, \\
\poemll    therefore the nations have gone mad. \\
\poeml \v{8}Suddenly, Babylon fell down and was shattered. \\
\poemll    Wail for her! \\
\poeml Bring balm for her wound, \\
\poemll    perhaps she will be healed. \\
\poeml \v{9}We tried to heal Babylon, \\
\poemll    but she wouldn't be healed. \\
\poeml Leave her, and let each of us go to his own country. \\
\poemll    For her judgment has reached to the heavens, \\
\poemlll       and is lifted up to the sky. \\
\poeml \v{10}The \divine{Lord} will vindicate us. \\
\poemll    Come! Let us declare the work of the \divine{Lord} our God in Zion. \\
\poeml \v{11}Sharpen the arrows, fill the quivers! \\
\poeml The \divine{Lord} has stirred up the spirit \\
\poemll    of the kings of the Medes--- \\
\poemlll       he has decided to destroy Babylon. \\
\poeml Indeed, it's the \divine{Lord}'s vengeance, \\
\poemll    vengeance for his Temple. \\
\poeml \v{12}Lift up the battle standard\fnote{\fbackref{51:12} I.e. Give the signal to attack} against Babylon's walls. \\
\poemll    Strengthen the guard; \\
\poemlll       post watchmen.\fnote{\fbackref{51:12} Or \fbib{guards}} \\
\poeml Set men in position for an ambush. \\
\poemll    For the \divine{Lord} will both plan and carry out what he has \\
\poemlll       declared against the inhabitants of Babylon. \\
\poeml \v{13}You who live beside many waters, \\
\poemll    rich in treasures, \\
\poeml your end has come, \\
\poemll    your life thread is cut.\fnote{\fbackref{51:13} Or \fbib{the measure of your unjust gain}} \\
\poeml \v{14}The \divine{Lord} of the Heavenly Armies \\
\poemll    has sworn by himself: \\
\poeml ``I'll surely fill you with soldiers\fnote{\fbackref{51:14} Lit. \fbib{men}} like a swarm of locusts, \\
\poemll    and they'll sing songs of victory over you.''
\passage{Praise to the God of Jacob}
\poeml \v{15}He made the earth by his power. \\
\poemll    He established the world by his wisdom, \\
\poemlll       and by his understanding he spread out the heavens. \\
\poeml \v{16}When his voice sounds, there is thunder from \\
\poemll    the waters of heaven, \\
\poeml and he makes clouds rise up \\
\poemll    from the ends of the earth. \\
\poeml He makes lightning for the rain \\
\poemll    and brings wind out of his storehouses. \\
\poeml \v{17}Everyone is stupid\fnote{\fbackref{51:17} I.e. like a beast} and without knowledge. \\
\poemll    Every goldsmith is put to shame by his own idols, \\
\poeml for his images are false,\fnote{\fbackref{51:17} Lit. \fbib{deception}} \\
\poemll    and there is no life in them. \\
\poeml \v{18}They're worthless, a work of mockery, \\
\poemll    and when the time of punishment comes,\fnote{\fbackref{51:18} Lit. \fbib{at the time of their punishment}} \\
\poemlll       they'll perish. \\
\poeml \v{19}The Portion of Jacob\fnote{\fbackref{51:19} I.e. \fbib{Portion of Jacob} is a name for the \fbib{\divine{Lord}}} is not like these. \\
\poemll    He made everything, \\
\poeml including the tribe of his inheritance. \\
\poemll    The \divine{Lord} of the Heavenly Armies is his name.
\passage{The \divine{Lord}'s Instrument of Judgment}
\poeml \v{20}``You are my war-club and \\
\poemll    weapons of war. \\
\poeml I'll smash nations with you \\
\poemll    and destroy kingdoms with you. \\
\poeml \v{21}I'll smash the horse and its rider with you. \\
\poemll    I'll smash the chariot and its rider with you. \\
\poeml \v{22}I'll smash man and woman with you. \\
\poemll    I'll smash old man and young boy with you. \\
\poemlll       I'll smash young man and young woman\fnote{\fbackref{51:22} Or \fbib{virgin}} with you. \\
\poeml \v{23}I'll smash the shepherd and his flock with you. \\
\poemll    I'll smash the farmer and his team of oxen with you. \\
\poemlll       I'll smash governors and officials with you.
\end{poetry}

\v{24}``Before your eyes I'll repay Babylon and all the inhabitants of Chaldea for all the evil that they did in Zion,'' declares the \divine{Lord}.

\begin{poetry}
\poeml \v{25}``Look, I'm against you, destroying mountain, \\
\poemll    who destroys the whole earth,'' \\
\poemlll       declares the \divine{Lord}. \\
\poeml ``I'll stretch out my hand against you \\
\poemll    and roll you down from the crags. \\
\poemlll       And I'll make you a burned-out mountain. \\
\poeml \v{26}They won't get a cornerstone \\
\poemll    or a foundation stone from you, \\
\poeml because you will be a wasteland forever,'' \\
\poemll    declares the \divine{Lord}. \\
\poeml \v{27}Lift up a battle standard in the land. \\
\poemll    Blow a trumpet among the nations. \\
\poeml Consecrate the nations against her. \\
\poemll    Summon the kingdoms of Ararat, Minni, \\
\poemlll       and Ashkenaz against her. \\
\poeml Appoint a commander against her, \\
\poemll    bring up horses like bristling locusts. \\
\poeml \v{28}Consecrate the nations against her, \\
\poemll    the kings of the Medes, their governors, their prefects, \\
\poemlll       and every land under their domination. \\
\poeml \v{29}The land quakes and writhes \\
\poemll    because the \divine{Lord}'s purposes \\
\poeml against Babylon stand firm, \\
\poemll    to make the land of Babylon a waste without inhabitants. \\
\poeml \v{30}The warriors of Babylon have stopped fighting. \\
\poemll    They stay in their strongholds; \\
\poeml their strength is dried up; \\
\poemll    they have become like women. \\
\poeml Her buildings are set on fire; \\
\poemll    the bars of her gates are broken. \\
\poeml \v{31}One runner runs to meet another runner,\fnote{\fbackref{51:31} Lit. \fbib{to meet a runner}} \\
\poemll    and one messenger to meet another messenger,\fnote{\fbackref{51:31} Lit. \fbib{to meet a messenger}} \\
\poeml to tell the king of Babylon that his city has been seized \\
\poemll    from one end to the other.\fnote{\fbackref{51:31} Lit. \fbib{from the end}} \\
\poeml \v{32}The fords have been captured, \\
\poemll    and the marshes burned with fire. \\
\poemlll       The soldiers are terrified. \\
\poeml \v{33}For this is what the \divine{Lord} of the Heavenly Armies, \\
\poemll    the God of Israel, says: \\
\poeml ``The daughter of Babylon is like a threshing floor \\
\poemll    at the time when it's pounded down.\fnote{\fbackref{51:33} I.e. threshing floors were pounded and smoothed in preparation for an upcoming harvest} \\
\poeml In just a little while, the time of her harvest will come.''
\passage{Judah's Complaint against Babylon}
\poeml \v{34}``King Nebuchadnezzar of Babylon has devoured me \\
\poemll    and crushed me. \\
\poeml He set me down \\
\poemll    like an empty vessel. \\
\poeml He swallowed me like a monster, \\
\poemll    and filled his belly with my delicacies. \\
\poemlll       Then he washed me away. \\
\poeml \v{35}May the violence done to me \\
\poemll    and my flesh be on Babylon,'' \\
\poemlll       says the inhabitant of Zion. \\
\poeml ``May my blood be on the inhabitants of Chaldea,'' \\
\poemll    says Jerusalem. \\
\poeml \v{36}Therefore this is what the \divine{Lord} says: \\
\poeml ``Look, I'm going to argue your case \\
\poemll    and take vengeance for you. \\
\poeml I'll dry up her sea \\
\poemll    and make her fountain dry.\fnote{\fbackref{51:36} I.e. dry up the source of Babylon's waters} \\
\poeml \v{37}Babylon will become a heap of ruins, \\
\poemll    a refuge for jackals, \\
\poeml a desolate place \\
\poemll    and an object of scorn.\fnote{\fbackref{51:37} Lit. \fbib{hissing}; i.e. as a sign of mocking and contempt} \\
\poeml \v{38}They'll roar together like young lions; \\
\poemll    they'll growl like lion cubs. \\
\poeml \v{39}When they're excited\fnote{\fbackref{51:39} Lit. \fbib{hot}} I'll serve them their banquet, \\
\poemll    and make them drunk until they're merry. \\
\poeml They'll sleep forever and won't wake up,'' \\
\poemll    declares the \divine{Lord}. \\
\poeml \v{40}``I'll bring them down like lambs for the slaughter, \\
\poemll    like rams with male goats. \\
\poeml \v{41}``How Sheshak\fnote{\fbackref{51:41} \fbib{Sheshak} is a cryptogram for Babylon.} will be captured, \\
\poemll    and the prince of all the earth seized! \\
\poeml How Babylon will become an object of horror \\
\poemll    among the nations! \\
\poeml \v{42}The sea will come up against Babylon, \\
\poemll    and she will be covered by wave upon wave.\fnote{\fbackref{51:42} Lit. \fbib{its many waves}} \\
\poeml \v{43}Her cities will become an object of horror, \\
\poemll    a dry land and a desert, \\
\poeml a land in which no one lives, \\
\poemll    and through which no human being passes. \\
\poeml \v{44}I'll punish Bel\fnote{\fbackref{51:44} \fbib{Bel} was another name for \fbib{Marduk}, the sun god of Babylon.} in Babylon, \\
\poemll    and I'll make what he has swallowed \\
\poemlll       come out of his mouth. \\
\poeml The nations will no longer stream to him. \\
\poemll    Even the wall of Babylon will fall. \\
\poeml \v{45}``Come out of her, my people, \\
\poemll    flee for your lives from the \divine{Lord}'s anger! \\
\poeml \v{46}Do this\fnote{\fbackref{51:46} Lit. \fbib{And}} now, so your heart does not grow faint, \\
\poemll    and so you don't become frightened \\
\poemlll       because of the rumors\fnote{\fbackref{51:46} Lit. \fbib{rumor}} that are heard in the land--- \\
\poeml a rumor comes one year\fnote{\fbackref{51:46} Lit. \fbib{in a year}} and then after it \\
\poemll    another rumor\fnote{\fbackref{51:46} Lit. \fbib{a rumor}} comes the next year\fnote{\fbackref{51:46} Lit. \fbib{in a year}} \\
\poeml about violence in the land \\
\poemll    and one ruler against another ruler.\fnote{\fbackref{51:46} Lit. \fbib{a ruler against a ruler}} \\
\poeml \v{47}Therefore, look, days are coming \\
\poemll    when I'll punish the idols of Babylon. \\
\poeml Her entire land will be put to shame, \\
\poemll    and all her slain will fall in her midst. \\
\poeml \v{48}Then the heavens and the earth \\
\poemll    and all that are in them \\
\poemlll       will shout for joy about Babylon \\
\poeml because the destroyers will come \\
\poemll    out of the north against her,'' \\
\poemlll       declares the \divine{Lord}. \\
\poeml \v{49}``So Babylon will fall \\
\poemll    because of the slain of Israel, \\
\poeml even as the slain of all the earth \\
\poemll    have fallen because of Babylon. \\
\poeml \v{50}Go, you who escaped the sword! \\
\poemll    Don't stand around! \\
\poeml Remember the \divine{Lord} from far away, \\
\poemll    and let Jerusalem come to your mind. \\
\poeml \v{51}We have been put to shame \\
\poemll    because we have heard insults. \\
\poeml Disgrace has covered our faces because foreigners have \\
\poemll    come into the Holy Places of the \divine{Lord}'s house. \\
\poeml \v{52}``Therefore, look, days are coming,'' \\
\poemll    declares the \divine{Lord}, \\
\poeml ``when I'll punish her idols, \\
\poemll    and throughout her land the wounded will groan. \\
\poeml \v{53}Though Babylon should reach up to the heavens \\
\poemll    and fortify her high fortresses, \\
\poeml from me destroyers will come to her,'' \\
\poemll    declares the \divine{Lord}. \\
\poeml \v{54}``The sound of a cry is coming from Babylon, \\
\poemll    great destruction from the land of the Chaldeans. \\
\poeml \v{55}For the \divine{Lord} is destroying Babylon, \\
\poemll    and he will make the loud sounds from her disappear.\fnote{\fbackref{51:55} Lit. \fbib{perish}} \\
\poeml Their waves will roar like many waters, \\
\poemll    the noise of their voices will sound forth. \\
\poeml \v{56}Indeed, the destroyer is coming against her, \\
\poemll    against Babylon. \\
\poeml Her warriors are captured, \\
\poemll    and her bows are broken. \\
\poeml For the \divine{Lord} is a God of recompense, \\
\poemll    and he will repay in full. \\
\poeml \v{57}I'll make their leaders, their wise men, \\
\poemll    their governors, their deputies, \\
\poeml and their warriors drunk so that they sleep forever \\
\poemll    and don't wake up,'' \\
\poeml declares the King \\
\poemll    whose name is the \divine{Lord} of the Heavenly Armies. \\
\poeml \v{58}This is what the \divine{Lord} of the Heavenly Armies says: \\
\poeml ``The broad wall of Babylon will be completely leveled, \\
\poemll    and its high gate set on fire. \\
\poeml and so the peoples toil for nothing, \\
\poemll    and the nations weary themselves only for fire.''
\end{poetry}
\passage{Jeremiah's Symbolic Message against Babylon}

\v{59}This is\fnote{\fbackref{51:59} The Heb. lacks \fbib{This is}} the message that Jeremiah the prophet delivered\fnote{\fbackref{51:59} Lit. \fbib{commanded}} to Neriah's son Seraiah, the grandson of Mahseiah, when he went with King Zedekiah of Judah to Babylon in the fourth year of his reign. Seraiah was the quartermaster. \v{60}Jeremiah wrote on a single scroll all the disasters that would come on Babylon, all these things that were written about Babylon. \v{61}Jeremiah told Seraiah, ``When you come to Babylon, see that you read all these words, \v{62}and say, `\divine{Lord}, you have declared about this place that you would destroy it so that there wouldn't be an inhabitant in it, neither human nor animal, because it will be a wasteland forever.' \v{63}When you finish reading this scroll, tie a rock around it and throw it into the middle of the Euphrates. \v{64}Then say, `Babylon will sink like this and won't rise from the disaster that I'm bringing on her. Her people\fnote{\fbackref{51:64} Lit. \fbib{They}} will be exhausted.'\,''

\begin{poetry}
\poeml This concludes the writings of Jeremiah.
\end{poetry}
\labelchapt{52}
\passage{The Fall of Jerusalem}

\chapt{52}
\v{1}Zedekiah was 21 years old when he began to rule, and he ruled for 11 years in Jerusalem. His mother's name was Hamutal, the daughter of Jeremiah of Libnah. \v{2}Zedekiah\fnote{\fbackref{52:2} Lit. \fbib{He}} had done evil in the \divine{Lord}'s sight, just as Jehoiakim had done. \v{3}Because Jerusalem and Judah had angered the Lord, he cast them out of his presence. Zedekiah rebelled against the king of Babylon, \v{4}and in the ninth year of his reign, in the tenth month, on the tenth day, King Nebuchadnezzar of Babylon came against Jerusalem with all his army. He encamped near it and set up siege works all around it. \v{5}The city was under siege until the eleventh year of the reign of\fnote{\fbackref{52:5} The Heb. lacks \fbib{the reign of}} King Zedekiah. \v{6}By the ninth day of the fourth month the famine became so severe that there was no food for the people of the land. \v{7}The wall of\fnote{\fbackref{52:7} The Heb. lacks \fbib{The wall of}} the city was broken through, and all the soldiers fled, leaving the city at night through the gate between the two walls next to the king's garden, even though the Chaldeans were all around the city. They went in the direction of the Arabah.\fnote{\fbackref{52:7} I.e. the Jordan Valley}

\v{8}The Chaldean army went after the king, overtook Zedekiah in the plains of Jericho, and all his troops were scattered from him. \v{9}They captured the king and brought him to the king of Babylon at Riblah in the land of Hamath, where the king of Babylon\fnote{\fbackref{52:8} Lit. \fbib{he}} passed judgment on him. \v{10}The king of Babylon killed Zedekiah's sons before his eyes, and he also killed all the Judean officials\fnote{\fbackref{52:9} Or \fbib{princes}} at Riblah. \v{11}He blinded Zedekiah and bound him in bronze shackles. Then the king of Babylon took him to Babylon and put him in prison until he died.
\passage{The Destruction of the Temple}

\v{12}In the fifth month, on the tenth day of the month---it was the nineteenth year of the reign of\fnote{\fbackref{52:12} The Heb. lacks \fbib{the reign of}} King Nebuchadnezzar, king of Babylon---Nebuzaradan the captain of the guard who served\fnote{\fbackref{52:12} Lit. \fbib{who stood before}} the king of Babylon, entered Jerusalem. \v{13}He burned the \divine{Lord}'s Temple, the king's house, and all the houses in Jerusalem. He also burned every public building\fnote{\fbackref{52:13} Or \fbib{He burned every large house}} with fire. \v{14}All the Chaldean troops who were with the captain of the guard tore down all the walls around Jerusalem. \v{15}Nebuzaradan the captain of the guard carried into exile some of the poorest of the people, the rest of the people left in the city, the deserters who had defected to the king of Babylon, and the rest of the craftsmen. \v{16}But Nebuzaradan the captain of the guard left some of the poorest people of the land to be vinedressers and farmers.\fnote{\fbackref{52:16} Lit. \fbib{tillers}}

\v{17}The Chaldeans broke in pieces the bronze pillars that were in the \divine{Lord}'s Temple and the stands and the bronze sea that were in the \divine{Lord}'s Temple, and they carried all the\fnote{\fbackref{52:17} Lit. \fbib{their bronze}} bronze to Babylon. \v{18}They took away the pots, the shovels, the snuffers, the basins, the pans, and all the bronze utensils that were used in the temple service. \v{19}The captain of the guard took away the bowls, the fire pans, the basins, the pots, the lamp stands, the pans, and the bowls for libations, both those made of gold and those made of silver. \v{20}There was too much bronze to weigh in the two pillars, the one sea, the twelve bronze oxen that were under the sea,\fnote{\fbackref{52:20} The Heb. lacks \fbib{the sea}} and the stands which King Solomon had made for the \divine{Lord}'s Temple. \v{21}Each of the pillars was twelve cubits\fnote{\fbackref{52:21} I.e. about eighteen feet; a cubit was about eighteen inches} high and its circumference twelve cubits.\fnote{\fbackref{52:21} Lit. \fbib{a line of twelve cubits would surround it};i.e. about eighteen feet; a cubit was about eighteen inches} It was hollow and about a handbreadth\fnote{\fbackref{52:21} Lit. \fbib{four fingers}} thick. \v{22}On each pillar\fnote{\fbackref{52:22} Lit. \fbib{on it}} was a capital of bronze, and the height of each capital was five cubits.\fnote{\fbackref{52:22} I.e. about seven and a half feet; a cubit was about eighteen inches} Latticework and pomegranates, all of bronze, were all around the capital. And the second pillar was like this, including the pomegranates. \v{23}There were 96 pomegranates open to view.\fnote{\fbackref{52:23} Or \fbib{evenly spread}} In all, there were 100 pomegranates all around the latticework.
\passage{Executions and Deportations to Babylon}

\v{24}The captain of the guard arrested Seraiah the chief priest, Zephaniah the next ranking priest,\fnote{\fbackref{52:24} Lit. \fbib{the number two priest}} and the three guards of the gate.\fnote{\fbackref{52:24} Lit. \fbib{of the threshold}; i.e. high Temple officials} \v{25}From the city he arrested one of the officers who had been in charge of the troops, seven men from the king's personal advisors who were found in the city, the secretary of the commander of the army who mustered the people of the land, and 60 men of the people of the land who were found inside the city. \v{26}Nebuzaradan the captain of the guard arrested them and brought them to the king of Babylon at Riblah. \v{27}The king of Babylon struck them down and killed them at Riblah in the land of Hamath. So Judah went into exile from the land.

\v{28}These are the people Nebuchadnezzar took into exile: in the seventh year, 3,023 Judeans; \v{29}in Nebuchadnezzar's eighteenth year, 832 people from Jerusalem; \v{30}in Nebuchadnezzar's twenty-third year, Nebuzaradan the captain of the guard took 745 people from Judah into exile. All the people taken into exile\fnote{\fbackref{52:30} The Heb. lacks \fbib{taken into exile}} numbered 4,600.
\passage{Jehoiachin Released from Prison}

\v{31}In the first year of his reign, King Evil-merodach of Babylon, showed favor to King Jehoiachin of Judah by releasing him from prison on the twenty-fifth day of the twelfth month in the thirty-seventh year of the exile of King Jehoiachin of Judah. \v{32}He spoke kindly to him and gave him a seat above the seats of the other\fnote{\fbackref{52:32} The Heb. lacks \fbib{other}} kings who were in Babylon with him. \v{33}Jehoiachin\fnote{\fbackref{52:33} Lit. \fbib{He}} changed his prison clothes and regularly dined with the king\fnote{\fbackref{52:33} Lit. \fbib{ate food before him}} as long as he lived. \v{34}As for his living expenses, a regular allowance was given him daily by the king of Babylon as long as he lived,\fnote{\fbackref{52:34} Lit. \fbib{all the days of his life}} until the day of his death.

\bookheader{Lamentations}
\labelbook{Lam}

\bookpretitle{The Book of}
\booktitle{Lamentations}

\labelchapt{1}
\passage{The Sorrowful City\passagenote{This book is an acrostic---successive verses begin with a consecutive letter of the Heb. alphabet except in chapter 3, where every three verses begin with the same consecutive Heb. letter.}}

\chapt{1}
\v{1}How lonely she lies,

\begin{poetry}
\poemll    the city that thronged with people! \\
\poeml Like a widow she has become, \\
\poemll    this great one among nations! \\
\poeml The princess among provinces \\
\poemll    has become a vassal. \\
\poeml \v{2}Bitterly she cries in the night, \\
\poemll    as tears stream down\fnote{\fbackref{1:2} The Heb. lacks \fbib{stream down}} her cheeks. \\
\poeml No one consoles her \\
\poemll    of all her friends. \\
\poeml All her neighbors have betrayed her; \\
\poemll    they have become her enemies. \\
\poeml \v{3}Judah has gone into exile \\
\poemll    to escape affliction and servitude. \\
\poeml She that sat among the nations, \\
\poemll    has found no rest. \\
\poeml All her pursuers overtook her \\
\poemll    amid narrow passes. \\
\poeml \v{4}The roads that lead to Zion are in mourning, \\
\poemll    because no one travels to the festivals. \\
\poeml All her gates are desolate; \\
\poemll    her priests are moaning. \\
\poeml Her young women\fnote{\fbackref{1:4} Lit. \fbib{virgins}} are grieving,\fnote{\fbackref{1:4} Or \fbib{are led away}.} \\
\poemll    and she is bitter. \\
\poeml \v{5}Her adversaries dominate her, \\
\poemll    her enemies prosper. \\
\poeml For the \divine{Lord} has made her suffer \\
\poemll    because of her many transgressions. \\
\poeml Her children have gone away, \\
\poemll    taken into captivity in the presence of the enemy. \\
\poeml \v{6}Fled from cherished\fnote{\fbackref{1:6} Lit. \fbib{from the daughter of}} Zion \\
\poemll    are all that were her splendor. \\
\poeml Her princes have become like deer \\
\poemll    that cannot find their feeding grounds. \\
\poeml They flee with strength exhausted \\
\poemll    from their pursuers. \\
\poeml \v{7}Jerusalem remembers\fnote{\fbackref{1:7} Or \fbib{Remember, \divine{Lord}, Jerusalem,}} \\
\poemll    her time of affliction and misery; \\
\poeml all her valued belongings\fnote{\fbackref{1:7} Or \fbib{Perished are all her valued belongings}} \\
\poemll    of days gone by, \\
\poeml when her people fell into enemy hands, \\
\poemll    with no one to help her, \\
\poeml and her enemies stared at her, \\
\poemll    mocking her downfall. \\
\poeml \v{8}Jerusalem sinned greatly, \\
\poemll    and she became unclean.\fnote{\fbackref{1:8} Lit. \fbib{has been removed}; i.e. due to ritual uncleanness} \\
\poeml All who honored her now despise her, \\
\poemll    because they saw her naked. \\
\poeml She herself groans \\
\poemll    and turns her face away. \\
\poeml \v{9}Uncleanness has soiled her skirts, \\
\poemll    and she gave no thought to what would follow. \\
\poeml She fell in such a startling way, \\
\poemll    with no one to comfort her. \\
\poeml Look, \divine{Lord}, upon my affliction, \\
\poemll    because my enemy is boasting. \\
\poeml \v{10}The adversary seized in his hands \\
\poemll    everything she valued. \\
\poeml She watched the nations\fnote{\fbackref{1:10} Or \fbib{watched foreigners}} \\
\poemll    enter her sanctuary; \\
\poeml those you forbade to enter \\
\poemll    your place of meeting. \\
\poeml \v{11}All her people groaned \\
\poemll    as they searched for food. \\
\poeml They traded their valuables in order to eat, \\
\poemll    to keep themselves alive.\fnote{\fbackref{1:11} Lit. \fbib{to eat to refresh the soul}} \\
\poeml Look, \divine{Lord}, and see \\
\poemll    how I have become dishonored. \\
\poeml \v{12}May it not befall you,\fnote{\fbackref{1:12} Lit. \fbib{It is not for you}} \\
\poemll    all who pass along the road! \\
\poeml Look and see: \\
\poemll    Is there any grief \\
\poeml like my grief \\
\poemll    dealt out to me, \\
\poeml by which the \divine{Lord} afflicted me \\
\poemll    in the time of his fierce wrath? \\
\poeml \v{13}He sent fire from on high,\fnote{\fbackref{1:13} Lit. \fbib{high into my bones}} \\
\poemll    making it penetrate my bones.\fnote{\fbackref{1:13} Lit. \fbib{overcoming her}} \\
\poeml He stretched out a net at my feet, \\
\poemll    forcing me to turn back. \\
\poeml He made me desolate; \\
\poemll    I'm fainting all day long. \\
\poeml \v{14}The yoke of my sins was bound on,\fnote{\fbackref{1:14} Lit. \fbib{was heavy}} \\
\poemll    fastened together by his hand. \\
\poeml They settled on my neck; \\
\poemll    he caused my strength to fail. \\
\poeml The \divine{Lord} placed me in the power \\
\poemll    of those I cannot resist. \\
\poeml \v{15}He rejected all the valiant men--- \\
\poemll    the \divine{Lord}, in my midst. \\
\poeml He set a time to meet with me \\
\poemll    to crush my young warriors. \\
\poeml The \divine{Lord} has trampled, as in a winepress, \\
\poemll    the fair virgin that is\fnote{\fbackref{1:15} Lit. \fbib{the virgin daughter of}} Judah. \\
\poeml \v{16}Because of all this, I weep; \\
\poemll    my eyes\fnote{\fbackref{1:16} Lit. \fbib{my eyes my eyes}} stream with tears \\
\poeml because far from me \\
\poemll    is the comforter of my soul. \\
\poeml My children are sorrowful, \\
\poemll    because the enemy has won. \\
\poeml \v{17}Zion spreads out her hands;\fnote{\fbackref{1:17} Or \fbib{Zion rent her linen garments}} \\
\poemll    no one is there to comfort her. \\
\poeml The \divine{Lord} has issued an order against\fnote{\fbackref{1:17} Or \fbib{\divine{Lord} kept watch over}} Jacob, \\
\poemll    that all who are around him are to be his enemies; \\
\poeml Jerusalem has become \\
\poemll    unclean among them. \\
\poeml \v{18}The \divine{Lord} is in the right, \\
\poemll    but I rebelled against his commands. \\
\poeml Listen, please, all you people, \\
\poemll    and look at my pain--- \\
\poeml my young men and women\fnote{\fbackref{1:18} Lit. \fbib{virgins}} \\
\poemll    have gone into captivity. \\
\poeml \v{19}I called out to my lovers,\fnote{\fbackref{1:19} Or \fbib{friends}} \\
\poemll    but they deceived me. \\
\poeml My priests and my elders \\
\poemll    have died within the city \\
\poeml while looking for something to eat \\
\poemll    to keep themselves alive. \\
\poeml \v{20}Look, \divine{Lord}, how distressed I am; \\
\poemll    all my insides are churning. \\
\poeml My heart is troubled within me, \\
\poemll    because I vigorously rebelled. \\
\poeml Outside the sword brings loss of life, \\
\poemll    while at home death rules. \\
\poeml \v{21}People\fnote{\fbackref{1:21} Lit. \fbib{They}} heard how I groan, \\
\poemll    with no one to comfort me. \\
\poeml All my adversaries have heard about my troubles; \\
\poemll    they rejoice that you have caused them. \\
\poeml Bring on the day you have promised, \\
\poemll    so my adversaries\fnote{\fbackref{1:21} Lit. \fbib{so they}} will become like me. \\
\poeml \v{22}May all of their wickedness come to your attention, \\
\poemll    and deal with them \\
\poeml as you have done with me \\
\poemll    because of all my transgressions. \\
\poeml For I am constantly groaning, \\
\poemll    and my heart is faint.
\end{poetry}
\labelchapt{2}
\passage{The Condition of Israel}

\begin{poetry}
\poeml \chapt{2}
\v{1}How the Lord in his wrath \\
\poemll    shamed\fnote{\fbackref{2:1} Or \fbib{enveloped}} cherished\fnote{\fbackref{2:1} Lit. \fbib{the daughter of}} Zion! \\
\poeml He cast down from heaven to earth \\
\poemll    the glory of Israel, \\
\poeml He did not remember his footstool\fnote{\fbackref{2:1} I.e. the Temple} \\
\poemll    in the time of his anger. \\
\poeml \v{2}The Lord swallowed up without pity \\
\poemll    all of Jacob's habitations. \\
\poeml In his wrath he tore down \\
\poemll    the strongholds of fair Judah.\fnote{\fbackref{2:2} Lit. \fbib{of the daughter of Judah}} \\
\poeml He cast to the ground in dishonor \\
\poemll    both her kingdom and its rulers. \\
\poeml \v{3}In his fierce wrath he cut off \\
\poemll    all the strength\fnote{\fbackref{2:3} Lit. \fbib{every horn}} of Israel. \\
\poeml He withdrew his protection\fnote{\fbackref{2:3} Lit. \fbib{his right hand}} \\
\poemll    as the enemy approached.\fnote{\fbackref{2:3} Lit. \fbib{in front of the enemy}} \\
\poeml He burned Jacob like a blazing fire \\
\poemll    consumes everything around it. \\
\poeml \v{4}He bent his bow against us\fnote{\fbackref{2:4} The Heb. lacks \fbib{against us}} as would an enemy, \\
\poemll    his right hand cocked as would an adversary. \\
\poeml He has killed everyone in whom we took pride; \\
\poemll    in the tent of cherished\fnote{\fbackref{2:4} Lit. of \fbib{the daughter of}} Zion he poured out \\
\poemlll       his anger like fire. \\
\poeml \v{5}The Lord has become like an enemy--- \\
\poemll    he has devoured Israel. \\
\poeml He has devoured all of her palaces, \\
\poemll    destroying her fortresses. \\
\poeml He filled cherished Judah\fnote{\fbackref{2:5} Lit. \fbib{the daughters of Judah}} \\
\poemll    with mourning and lament. \\
\poeml \v{6}He plowed under his Temple\fnote{\fbackref{2:6} Lit. \fbib{tent}} like a garden, \\
\poemll    spoiling his tent. \\
\poeml The \divine{Lord} abolished in Zion \\
\poemll    both festivals and Sabbaths. \\
\poeml In his fierce wrath he despised \\
\poemll    both king and priest. \\
\poeml \v{7}The Lord rejected his altar, \\
\poemll    disavowing his sanctuary. \\
\poeml He gave up her palace walls \\
\poemll    to the control of the enemy. \\
\poeml They shouted in the \divine{Lord}'s Temple, \\
\poemll    as though they were attending a day of celebration. \\
\poeml \v{8}The \divine{Lord} planned to destroy \\
\poemll    the walls of cherished\fnote{\fbackref{2:8} Lit. \fbib{of the daughter of}} Zion. \\
\poeml He measured them with his line. \\
\poemll    He did not withhold his hand from destruction. \\
\poeml He made both ramparts and defensive walls mourn; \\
\poemll    they languish together. \\
\poeml \v{9}Jerusalem's\fnote{\fbackref{2:9} Lit. \fbib{Her}} gates collapsed to the ground; \\
\poemll    he destroyed and broke the bars of her gates.\fnote{\fbackref{2:9} The Heb. lacks \fbib{gates}} \\
\poeml Both king and prince have gone into captivity.\fnote{\fbackref{2:9} Lit. \fbib{into the nations}} \\
\poemll    There is no instruction,\fnote{\fbackref{2:9} Or \fbib{Law}; or \fbib{The priests do not give their guidance}} \\
\poeml and the prophets receive \\
\poemll    no vision from the \divine{Lord}. \\
\poeml \v{10}The leaders of cherished\fnote{\fbackref{2:10} Lit. \fbib{of the daughter of}} Zion \\
\poemll    sit silently on the ground; \\
\poeml they throw dust on their heads \\
\poemll    and dress in mourning clothes. \\
\poeml The young women of Jerusalem \\
\poemll    bow their heads in sorrow.\fnote{\fbackref{2:10} Lit. \fbib{heads to the ground}} \\
\poeml \v{11}My eyes are worn out from crying, \\
\poemll    my insides are churning, \\
\poeml My emotions pour out in grief\fnote{\fbackref{2:11} Lit. \fbib{my liver empties upon the ground}} \\
\poemll    because my people are destroyed--- \\
\poeml Children and infants faint \\
\poemll    in the streets of the city. \\
\poeml \v{12}They ask their mothers, \\
\poemll    ``Is there anything to eat or drink?''\fnote{\fbackref{2:12} Lit. \fbib{any grain and wine}} \\
\poeml They faint in the streets of the city \\
\poemll    like wounded men. \\
\poeml Their life ebbs away \\
\poemll    while they lie on their mother's bosom. \\
\poeml \v{13}What can be said about you? \\
\poemll    To what should you be compared, fair\fnote{\fbackref{2:13} Lit. \fbib{daughter of}} Jerusalem? \\
\poeml To what may I liken you, \\
\poemll    so I may comfort you, fair one\fnote{\fbackref{2:13} Lit. \fbib{virgin daughter}} of Zion? \\
\poeml Indeed, your wound is as deep as the sea--- \\
\poemll    who can heal you? \\
\poeml \v{14}Your prophets look on your behalf; \\
\poemll    they see false and deceptive visions. \\
\poeml They did not expose your sins \\
\poemll    in order to restore what had been captured.\fnote{\fbackref{2:14} Lit. \fbib{restore your captivity}} \\
\poeml Instead, they crafted oracles for you \\
\poemll    that are false and misleading. \\
\poeml \v{15}Everyone who passes by on the road \\
\poemll    shake their fists\fnote{\fbackref{2:15} Or \fbib{road clap their hands}} at you. \\
\poeml They hiss and shake their heads \\
\poemll    at cherished\fnote{\fbackref{2:15} Lit. \fbib{at the daughter}} Jerusalem: \\
\poeml ``Is this the city men used to call `The Perfection of Beauty,' \\
\poemll    and `The Joy of the Entire Earth'\,''? \\
\poeml \v{16}All of your enemies \\
\poemll    insult you with gaping mouths. \\
\poeml They hiss and grind their teeth while saying, \\
\poemll    ``We have devoured her completely. \\
\poeml Yes, this is the day that we anticipated! \\
\poemll    We found it at last;\fnote{\fbackref{2:16} The Heb. lacks \fbib{at last}} we have seen it!'' \\
\poeml \v{17}The \divine{Lord} did what he planned. \\
\poemll    He carried out his threat. \\
\poeml Just as he commanded long ago, \\
\poemll    he has torn down without pity; \\
\poeml He let the enemy boast about you \\
\poemll    and has exalted the power\fnote{\fbackref{2:17} Lit. \fbib{horn}} of your enemies. \\
\poeml \v{18}Cry out from your heart to the Lord, \\
\poemll    wall of fair\fnote{\fbackref{2:18} Lit. \fbib{of the daughter of}} Zion! \\
\poeml Let your tears run down like a river \\
\poemll    day and night. \\
\poeml Allow yourself no rest, \\
\poemll    and don't stop crying. \\
\poeml \v{19}Get up and cry aloud in the night, \\
\poemll    at the beginning of every hour.\fnote{\fbackref{2:19} Lit. \fbib{of the night watches}} \\
\poeml Pour out your heart like water \\
\poemll    in the presence of the Lord! \\
\poeml Lift up your hands toward him \\
\poemll    for the lives of your children, \\
\poeml who are fainting away \\
\poemll    at every street corner. \\
\poeml \v{20}Look, \divine{Lord}, and take note: \\
\poemll    To whom have you done this? \\
\poeml Should women eat their offspring, \\
\poemll    the children they have cuddled? \\
\poeml Should priests and prophets be slain \\
\poemll    in the sanctuary of the Lord? \\
\poeml \v{21}Young men and the aged \\
\poemll    lie on the ground in the streets; \\
\poeml my young women and young men \\
\poemll    have fallen by the sword. \\
\poeml You killed them in your anger, \\
\poemll    slaughtering them without pity. \\
\poeml \v{22}You have invited those who terrorize me to come around, \\
\poemll    as if today were a festival. \\
\poeml No one has escaped or survived \\
\poemll    the time of the \divine{Lord}'s anger. \\
\poeml My enemy has finished off \\
\poemll    those whom I cuddled and raised.
\end{poetry}
\labelchapt{3}
\passage{The \divine{Lord}'s Purposes for Affliction}

\begin{poetry}
\poeml \chapt{3}
\v{1}I am a man familiar with affliction--- \\
\poemll    under the rod of God's\fnote{\fbackref{3:1} Lit. \fbib{his}} anger. \\
\poeml \v{2}He has led me---brought me \\
\poemll    into darkness, not into light. \\
\poeml \v{3}He truly turned his hand against me, \\
\poemll    again and again, all day long. \\
\poeml \v{4}He made my flesh and skin prematurely old; \\
\poemll    he broke my bones. \\
\poeml \v{5}He laid siege against me, \\
\poemll    surrounding me with bitterness and suffering. \\
\poeml \v{6}He has forced me to live in darkness, \\
\poemll    like those who are long dead. \\
\poeml \v{7}He has walled me in so I cannot escape; \\
\poemll    he placed heavy chains on me. \\
\poeml \v{8}Indeed, when I cry out, calling for help, \\
\poemll    he shuts out my prayer. \\
\poeml \v{9}He impeded my way with blocks of stone, \\
\poemll    making my paths uneven. \\
\poeml \v{10}He is like\fnote{\fbackref{3:10} The Heb. lacks \fbib{like}} a bear that lies in wait for me, \\
\poemll    a lion in hiding. \\
\poeml \v{11}He forced me off my path, \\
\poemll    tearing me to pieces and making me desolate. \\
\poeml \v{12}He bent his bow, \\
\poemll    aiming at me with his arrow. \\
\poeml \v{13}He caused his war arrows\fnote{\fbackref{3:13} Lit. \fbib{caused the sons of his arrows}} \\
\poemll    to pierce my vital organs. \\
\poeml \v{14}I have become a laughingstock to all my people, \\
\poemll    the object of their taunts throughout the day. \\
\poeml \v{15}He has filled me with bitterness, \\
\poemll    making me drink wormwood. \\
\poeml \v{16}He broke my teeth on gravel, \\
\poemll    covering me with dust. \\
\poeml \v{17}You have removed peace from my life; \\
\poemll    I have forgotten what prosperity is.\fnote{\fbackref{3:17} Lit. \fbib{forgotten prosperity}} \\
\poeml \v{18}So I say, ``My strength is gone \\
\poemll    as is my hope in the \divine{Lord}.'' \\
\poeml \v{19}Remember my affliction and homelessness--- \\
\poemll    wormwood and gall! \\
\poeml \v{20}My mind keeps reflecting on it, \\
\poemll    and I become depressed.\fnote{\fbackref{3:20} Lit. \fbib{and sinks within me}} \\
\poeml \v{21}This is what comes to mind, \\
\poemll    and therefore I have hope: \\
\poeml \v{22}Because of the \divine{Lord}'s gracious love we are not consumed, \\
\poemll    since his compassions never end. \\
\poeml \v{23}They are new every morning--- \\
\poemll    great is your faithfulness! \\
\poeml \v{24}``The \divine{Lord} is all I have,''\fnote{\fbackref{3:24} Lit. \fbib{is my portion}} says my soul, \\
\poemll    ``Therefore I will trust in him.'' \\
\poeml \v{25}The \divine{Lord} is good to those who wait for him, \\
\poemll    to the person who searches for him. \\
\poeml \v{26}It is good to hope and wait patiently \\
\poemll    for the \divine{Lord}'s salvation. \\
\poeml \v{27}It is good when a young man carries the yoke \\
\poemll    of discipline\fnote{\fbackref{3:27} The Heb. lacks \fbib{of discipline}} in his youth. \\
\poeml \v{28}He is to sit apart and remain silent, \\
\poemll    because the \divine{Lord}\fnote{\fbackref{3:28} Lit. \fbib{because he}} has laid it upon him. \\
\poeml \v{29}Let him fall face down in the dust, \\
\poemll    so there may yet be hope. \\
\poeml \v{30}He will endure being slapped in the face, \\
\poemll    bringing him public disgrace. \\
\poeml \v{31}Indeed, the Lord will not always \\
\poemll    reject us\fnote{\fbackref{3:31} The Heb. lacks \fbib{us}}--- \\
\poeml \v{32}though he causes grief, \\
\poemll    his compassion abounds according to his gracious love. \\
\poeml \v{33}For he does not deliberately hurt \\
\poemll    or grieve human beings. \\
\poeml \v{34}When any of the prisoners of the earth \\
\poemll    are crushed underfoot, \\
\poeml \v{35}when a person's rights are perverted \\
\poemll    in defiance of the Most High. \\
\poeml \v{36}When a man is thwarted in his appeal, \\
\poemll    does the Lord condone\fnote{\fbackref{3:36} Lit. \fbib{see}} it? \\
\poeml \v{37}Who can command, and it happens, \\
\poemll    without the Lord having ordered it? \\
\poeml \v{38}Do not both good and evil things proceed \\
\poemll    from the mouth of the Most High? \\
\poeml \v{39}Why should anyone living complain, \\
\poemll    any mortal, about being punished for sin? \\
\poeml \v{40}Let us examine our lifestyles, \\
\poemll    putting them to the test, \\
\poemlll       and turn back to the \divine{Lord}. \\
\poeml \v{41}Let us lift up our hearts \\
\poemll    and our hands \\
\poemlll       to God in heaven. \\
\poeml \v{42}As for us, we have sinned and rebelled; \\
\poemll    but you have not pardoned us.\fnote{\fbackref{3:42} The Heb. lacks \fbib{us}} \\
\poeml \v{43}Clothing yourself with anger, you pursued us. \\
\poemll    You killed without pity, \\
\poeml \v{44}You covered yourself with a cloud \\
\poemll    that prayer cannot pierce. \\
\poeml \v{45}You have reduced us to scum and garbage \\
\poemll    among the nations. \\
\poeml \v{46}All our enemies \\
\poemll    jeer at us with gaping mouths. \\
\poeml \v{47}Panic and pitfalls beset us, \\
\poemll    along with devastation and ruin. \\
\poeml \v{48}My eyes run with rivers of tears \\
\poemll    over the destruction of my cherished\fnote{\fbackref{3:48} Lit. \fbib{of the daughter of}} people. \\
\poeml \v{49}My tears pour\fnote{\fbackref{3:49} Lit. \fbib{My eye pours}} down ceaselessly; \\
\poemll    I am far from relief \\
\poeml \v{50}until the \divine{Lord} bends down \\
\poemll    to see from heaven. \\
\poeml \v{51}What I see\fnote{\fbackref{3:51} Lit. \fbib{My eye}} grieves my soul \\
\poemll    because of all the young women\fnote{\fbackref{3:51} Lit. \fbib{the daughters}} of my city. \\
\poeml \v{52}My enemies hunted me like a bird, \\
\poemll    viciously and without justification. \\
\poeml \v{53}They dumped me alive into a pit, \\
\poemll    sealing me in with stone.\fnote{\fbackref{3:53} Lit. \fbib{pit, casting a stone at me}} \\
\poeml \v{54}Water closed over my head, \\
\poemll    and I said, ``I'm a dead man.''\fnote{\fbackref{3:54} Lit. ``\fbib{I'm cut off.''}} \\
\poeml \v{55}I called on your name, \divine{Lord}, \\
\poemll    from the depths of the Pit,\fnote{\fbackref{3:55} I.e. the place of punishment in the afterlife} \\
\poeml \v{56}You heard my voice--- \\
\poemll    don't close your ear to my sighs and cries.\fnote{\fbackref{3:56} Lit. \fbib{my relief, to my cry}} \\
\poeml \v{57}You drew near when I called out to you. \\
\poemll    You said, ``Stop being afraid'' \\
\poeml \v{58}Lord, you have defended my cause; \\
\poemll    you have redeemed my life. \\
\poeml \v{59}\divine{Lord}, you observed how I have been wronged; \\
\poemll    now make your ruling in my case. \\
\poeml \v{60}You examined their plans for vengeance, \\
\poemll    all of their plots against me. \\
\poeml \v{61}\divine{Lord}, you listened to their insults--- \\
\poemll    all their plots against me, \\
\poeml \v{62}the whisperings of my opponents, \\
\poemll    their scheming against me all day long. \\
\poeml \v{63}Watch! Whether they sit down or stand up, \\
\poemll    they mock me with their songs. \\
\poeml \v{64}Pay them back, \divine{Lord}, \\
\poemll    according to their actions. \\
\poeml \v{65}Give them an anguished heart; \\
\poemll    may your curse be upon them! \\
\poeml \v{66}Pursue them in your anger \\
\poemll    and destroy them from under the \divine{Lord}'s heaven.
\end{poetry}
\labelchapt{4}
\passage{Zion's Punishment}

\begin{poetry}
\poeml \chapt{4}
\v{1}How tarnished the gold has become, \\
\poemll    the finest gold debased! \\
\poeml Sacred stones\fnote{\fbackref{4:1} Or \fbib{gems}} have been scattered \\
\poemll    at every street corner. \\
\poeml \v{2}Though the precious people of Zion \\
\poemll    were like fine gold, \\
\poeml how they are valued like clay vessels, \\
\poemll    the handiwork of a potter! \\
\poeml \v{3}Even wild animals nurse, \\
\poemll    suckling their young; \\
\poeml but the women of my people are cruel, \\
\poemll    like ostriches in the wilderness. \\
\poeml \v{4}The nursing child's tongue \\
\poemll    cleaves to its palate from thirst. \\
\poeml Young children beg for bread, \\
\poemll    but no one gives them any. \\
\poeml \v{5}Those who enjoyed delicacies \\
\poemll    lie desolate in the streets. \\
\poeml Those who were reared wearing purple \\
\poemll    scavenge in piles of trash. \\
\poeml \v{6}The guilt of my cherished people surpasses the sin of Sodom, \\
\poemll    which was overthrown in a moment, \\
\poemlll       without a hand to help her. \\
\poeml \v{7}Her princes\fnote{\fbackref{4:7} Or \fbib{Nazirites}} were purer than snow, \\
\poemll    whiter than milk. \\
\poeml Their bodies were more ruddy\fnote{\fbackref{4:7} I.e. reddish brown skin color} than rubies, \\
\poemll    their beards like the color of precious stones. \\
\poeml \v{8}Now their faces are blacker than coal; \\
\poemll    they are unrecognized in the streets. \\
\poeml Their skin clings to their bones; \\
\poemll    it has become dry like a stick. \\
\poeml \v{9}Those who die by the sword are better off \\
\poemll    than those who die from starvation, \\
\poeml who slowly waste away like those pierced through \\
\poemll    for lack of food from the fields. \\
\poeml \v{10}With their own hands, compassionate women \\
\poemll    boil their own children--- \\
\poeml they become their food--- \\
\poemll    when my beloved people were\fnote{\fbackref{4:10} Lit. \fbib{when the daughter of my people was}} destroyed. \\
\poeml \v{11}The \divine{Lord} has exhausted his wrath, \\
\poemll    pouring out his fierce anger. \\
\poeml He kindled a fire in Zion, \\
\poemll    consuming its foundations. \\
\poeml \v{12}None of the kings of the earth would have believed, \\
\poemll    nor the world's inhabitants, \\
\poeml that the adversary and the enemy \\
\poemll    could have breached the gates of Jerusalem. \\
\poeml \v{13}Due to the sins committed by her prophets, \\
\poemll    and the iniquities of her priests \\
\poeml who shed in her midst, \\
\poemll    the blood of the righteous, \\
\poeml \v{14}people stagger around in the streets like the blind, \\
\poemll    defiled by blood \\
\poeml unclean so that no one is able \\
\poemll    to touch their clothing. \\
\poeml \v{15}``Go away! Unclean!'' \\
\poemll    they shouted at them. \\
\poemlll       ``Go away! Go away! Don't touch!'' \\
\poeml When they fled away and wandered, \\
\poemll    those among the nations decreed, \\
\poemlll       ``They cannot live here!'' \\
\poeml \v{16}The \divine{Lord} himself separated them; \\
\poemll    he will do nothing more for them. \\
\poeml They did not respect their own priests; \\
\poemll    they did not honor their elders. \\
\poeml \v{17}Our eyes failed, \\
\poemll    searching in vain for hope; \\
\poeml we kept watching and looking \\
\poemll    for a nation that would not help. \\
\poeml \v{18}Our steps were closely stalked, \\
\poemll    so we couldn't travel on our own streets. \\
\poeml Our end is near, \\
\poemll    our days are over; \\
\poemlll       indeed, our end has come. \\
\poeml \v{19}Our pursuers were swifter \\
\poemll    than soaring eagles;\fnote{\fbackref{4:19} Lit. \fbib{than eagles of the heavens}} \\
\poeml they pursued us over the mountains, \\
\poemll    lying in wait for us in the wilderness. \\
\poeml \v{20}The \divine{Lord}'s anointed, \\
\poemll    the breath of our life, \\
\poemlll       was captured in their pits. \\
\poeml About him we had said, \\
\poemll    ``Under his protection we will survive \\
\poemlll       among the nations.'' \\
\poeml \v{21}Celebrate and rejoice, you women\fnote{\fbackref{4:21} Lit. \fbib{daughter}} of Edom, \\
\poemll    who live in the land of Uz. \\
\poeml But to you the cup also will pass--- \\
\poemll    you will become drunk and stripped naked. \\
\poeml \v{22}The punishment for your sin is complete, you women\fnote{\fbackref{4:22} Lit. \fbib{daughter}} of Zion, \\
\poemll    and God\fnote{\fbackref{4:22} Lit. \fbib{he}} will no longer exile you. \\
\poeml He will punish your iniquity, you women\fnote{\fbackref{4:22} Lit. \fbib{daughter}} of Edom, \\
\poemll    and he will expose your sins.
\end{poetry}
\labelchapt{5}
\passage{A Prayer for Deliverance}

\begin{poetry}
\poeml \chapt{5}
\v{1}\divine{Lord}, remember what has happened to us. \\
\poemll    Pay attention, and look at our shame! \\
\poeml \v{2}Our inheritance has\fnote{\fbackref{5:2} Or \fbib{possessions have}} been turned over to strangers, \\
\poemll    and our homes to foreigners. \\
\poeml \v{3}We are now orphans---without fathers--- \\
\poemll    and our mothers are like widows. \\
\poeml \v{4}We pay to drink our own water, \\
\poemll    and our own wood is sold to us at high price. \\
\poeml \v{5}Our pursuers breathe down\fnote{\fbackref{5:5} Lit. \fbib{pursuers are on}} our necks; \\
\poemll    we are weary, but there is no rest for us. \\
\poeml \v{6}We made a deal with the Egyptians and the Assyrians \\
\poemll    for the price of food.\fnote{\fbackref{5:6} Lit. \fbib{Assyrians, being satisfied with bread}} \\
\poeml \v{7}Our ancestors sinned and no longer exist \\
\poemll    yet we continue to bear the consequences of their sin. \\
\poeml \v{8}Slaves rule over us, \\
\poemll    and no one delivers us from their control.\fnote{\fbackref{5:8} Lit. \fbib{hand}} \\
\poeml \v{9}We risk our lives to obtain our food, \\
\poemll    facing death\fnote{\fbackref{5:9} Lit. \fbib{facing the sword}} in the desert. \\
\poeml \v{10}Our skin blisters\fnote{\fbackref{5:10} Lit. \fbib{blackens}} as from an oven, \\
\poemll    due to ravaging blasts of the famine. \\
\poeml \v{11}They have raped women in Zion, \\
\poemll    young women\fnote{\fbackref{5:11} Lit. \fbib{the virgins}} in the towns of Judah. \\
\poeml \v{12}Princes they have hung by their hands; \\
\poemll    elders\fnote{\fbackref{5:12} Lit. \fbib{the faces of the elders}} they have disrespected. \\
\poeml \v{13}Our\fnote{\fbackref{5:13} The Heb. lacks \fbib{Our}} young men must grind grain with a millstone; \\
\poemll    our\fnote{\fbackref{5:13} The Heb. lacks \fbib{our}} youths stumble under the weight of wood. \\
\poeml \v{14}Our\fnote{\fbackref{5:14} The Heb. lacks \fbib{Our}} elders have ceased ruling\fnote{\fbackref{5:14} The Heb. lacks \fbib{ruling}} at the gate; \\
\poemll    our\fnote{\fbackref{5:14} The Heb. lacks \fbib{our}} young men have abandoned\fnote{\fbackref{5:14} The Heb. lacks \fbib{have abandoned}} their music. \\
\poeml \v{15}The joy of our hearts has ceased, \\
\poemll    and our dancing has turned into dirges. \\
\poeml \v{16}The crown has fallen from our head--- \\
\poemll    woe to us, because we have sinned! \\
\poeml \v{17}This is why our hearts faint, \\
\poemll    and why our eyes grow dim: \\
\poeml \v{18}Because Mount Zion is desolate; \\
\poemll    foxes roam around it. \\
\poeml \v{19}You, \divine{Lord}, are forever--- \\
\poemll    your throne endures from generation to generation. \\
\poeml \v{20}So why have you completely forgotten us, \\
\poemll    forsaking us for so long? \\
\poeml \v{21}Restore us to yourself, \divine{Lord}, \\
\poemll    so that we may return. \\
\poeml Renew our days as before, \\
\poeml \v{22}unless you have utterly rejected us \\
\poemlll       and are angry with us without limit.\end{poetry}

\bookheader{Ezekiel}
\labelbook{Ezek}

\bookpretitle{The Book of the Prophet}
\booktitle{Ezekiel}

\labelchapt{1}
\passage{An Introduction to Ezekiel's Visions}

\chapt{1}
\v{1}On the fifth day of the fourth month of the thirtieth year of the exile to Babylon,\fnote{\fbackref{1:1} The Heb. lacks \fbib{of the exile to Babylon}} while I was among the captives on the bank of\fnote{\fbackref{1:1} The Heb. lacks \fbib{the bank of}} the Chebar River, heaven opened up and I saw visions from God.
\passage{The Vision of the Fire Cloud}

\v{2}On the fifth day\fnote{\fbackref{1:2} The Heb. lacks \fbib{day}} of the month in the fifth year of King Jehoiachin's imprisonment in exile, \v{3}a message from\fnote{\fbackref{1:3} Lit. \fbib{the word of}} the \divine{Lord} came directly to Buzi's son Ezekiel,\fnote{\fbackref{1:3} The Heb. name \fbib{Ezekiel} means \fbib{My strengthener is God}} the priest, by the Chebar River in the land of the Chaldeans.\fnote{\fbackref{1:3} I.e. Aramaic speaking people of southern Mesopotamia; or magi-astrologers; and so throughout the book} The hand of the \divine{Lord} rested upon him there.

\v{4}I was amazed to see a wind storm blow\fnote{\fbackref{1:4} Lit. \fbib{come}} in from the north, consisting of\fnote{\fbackref{1:4} The Heb. lacks \fbib{consisting of}} a massive cloud and fire that was flashing back and forth, surrounded by bright light. From deep within the cloud,\fnote{\fbackref{1:4} Lit. \fbib{within it}} something was shining that appeared to have a color like bronze that had been placed in fire until it glowed.
\passage{The Vision of the Four Beings}

\v{5}Deep inside it, the likenesses of four living beings were visible. Their appearances were similar to human forms, \v{6}except that they each had four faces, four pairs of wings,\fnote{\fbackref{1:6} Or \fbib{four wings}} \v{7}and straight legs. Their feet resembled calves' hooves, but they gleamed like polished bronze. \v{8}From under their wings there were human hands on their four sides.

Now as to their four faces and four pairs of wings, \v{9}their pairs of wings overlapped each other. They moved in straight directions without turning their faces around as they moved. \v{10}The form of their faces was human, but each of the four also had the face of a lion to the right, the face of an ox to the left, and the face of an eagle behind them.\fnote{\fbackref{1:10} The Heb. lacks \fbib{behind them}} \v{11}That's what their faces were like. Their wings spread out above and around them, one pair overlapping another, with one pair covering themselves. \v{12}Each moved in straight directions. Wherever they decided\fnote{\fbackref{1:12} Lit. \fbib{wherever their spirit was}} to go, they went without turning themselves.

\v{13}Now, in the midst of the living beings there was something that\fnote{\fbackref{1:13} So LXX; MT reads \fbib{These living beings}} appeared to glow like coals kindled by a fire,\fnote{\fbackref{1:13} Or \fbib{appeared like glowing coals of fire}} like torches that moved back and forth between the living beings. The fire was dazzling, and lightning flashed from the fire. \v{14}The living beings moved around, in appearance resembling lightning.
\passage{The Vision of the Wheels}

\v{15}As I observed the living beings, I noticed one wheel on the earth beside each being---that is, for the four of them.\fnote{\fbackref{1:15} Lit. \fbib{of their faces}} \v{16}Their wheels and their construction details looked like gold-colored beryl.\fnote{\fbackref{1:16} Lit. \fbib{like tarshish}; i.e. a semi-precious stone similar to beryl or yellow jasper} Each wheel was identical in form to the others,\fnote{\fbackref{1:16} Lit. \fbib{the four wheels}} and they appeared to have been constructed and designed as if one wheel were within another. \v{17}Whenever the four moved, no matter which of four directions, they moved without turning around.

\v{18}Their wheel rims were ornate\fnote{\fbackref{1:18} Lit. \fbib{lofty}; or \fbib{high}} and terrifying. They were full of eyes that surrounded the four of them. \v{19}Whenever the living beings moved, the wheels moved, too. Whenever the living beings rose from the earth, the wheels rose also. \v{20}Whatever direction these spirits went, the wheels would be lifted up along with them, because the wheels were alive.\fnote{\fbackref{1:20} Lit. \fbib{the spirit of the living beings resided in the wheels}} \v{21}They moved around whenever they wanted to move around,\fnote{\fbackref{1:21} Lit. \fbib{In their moving they moved}} and they stood still whenever they wanted to stand still;\fnote{\fbackref{1:21} Lit. \fbib{and in their standing they stood}} and whenever they rose from the earth, the wheels remained close beside them, because the wheels were also alive.\fnote{\fbackref{1:21} Lit. \fbib{the spirit of the living beings resided also in the wheels}}
\passage{The Vision of the Wings}

\v{22}There was spread out over the heads of the living beings what looked like a canopy,\fnote{\fbackref{1:22} Or \fbib{expanse}} in outward appearance resembling ice, \v{23}and underneath the canopy, their wings spread out straight over their heads toward each other. They each also had two wings with which they covered themselves, one wing covering its body on one side and one wing covering itself on the other side.

\v{24}I also heard the sound of their wings, like the sound of roaring\fnote{\fbackref{1:24} Or \fbib{of an abundant amount of}} water, like the voice of the Almighty, or like a boisterous crowd within an army camp. Whenever they stopped flying, they lowered their wings. \v{25}A sound came from above the canopy that was spread out over their heads. Whenever they stood still, they lowered their wings. \v{26}From above the canopy that was spread out over their heads, there appeared to be something reminiscent of a throne, resembling sapphire\fnote{\fbackref{1:26} Or \fbib{of lapis lazuli}} in form.
\passage{The Vision of the Glory of God}

There was the likeness of the appearance of a human being seated on the likeness of the throne high above. \v{27}I noticed that from what appeared to look like his waist upward there was something that looked like metal that glowed as if it were immersed in fire. Below this there was something resembling fire, with a radiant light surrounding him. \v{28}The appearance of the radiant light resembled that of a rainbow shining in a cloud on a rainy day. This was what the appearance of the form of the glory of the \divine{Lord} resembled. When I saw all of this,\fnote{\fbackref{1:28} The Heb. lacks \fbib{all of this}} I fell flat on my face. Then I heard a voice speaking.
\labelchapt{2}
\passage{Ezekiel's Commission to Prophesy}

\chapt{2}
\v{1}``Son of Man,'' the \divine{Lord} said,\fnote{\fbackref{2:1} The Heb. lacks \fbib{the \divine{Lord} said}} ``get up on your feet. I want to talk to you.'' \v{2}Even while he was speaking to me, the Spirit entered me, set me on my feet, and I listened to the voice that had been speaking to me.

\v{3}``Son of Man, I'm sending you to that rebellious people, the Israelis, who have rebelled against me the same way their ancestors did. And they're still rebels\fnote{\fbackref{2:3} The Heb. lacks \fbib{And they're still rebels}} to this very day! \v{4}They're stubborn\fnote{\fbackref{2:4} Lit. \fbib{They're children of hard faces}} and strong willed. I'm sending you to them to tell them what the \divine{Lord} says. \v{5}Whether this rebellious group\fnote{\fbackref{2:5} Lit. \fbib{house}} listens to you or not, at least\fnote{\fbackref{2:5} The Heb. lacks \fbib{at least}} they'll realize that a prophet had appeared in their midst!

\v{6}``Now as for you, Son of Man, never be afraid of them or of anything they have to say, because being with them will be like settling down to live among briers, thorn bushes, and scorpions! Don't be afraid of anything they have to say, and don't be awed by their appearance, since they are a rebellious group.\fnote{\fbackref{2:6} Lit. \fbib{house}} \v{7}You are to tell them whatever I have to say to them, whether they listen or not, since they are rebellious.''
\passage{The Vision of the Edible Scroll}

\v{8}``Son of Man, you are to listen to what I tell you. You are never to be rebellious like they are: a rebellious group.\fnote{\fbackref{2:8} Lit. \fbib{house}} Now, open your mouth and eat what I'm giving you{\ldots}''

\v{9}As I watched, all of a sudden there was a hand being stretched out in my direction! And there was a scroll \v{10}being unrolled right in front of me! Written on both sides were lamentations, mourning, and cries of grief.\fnote{\fbackref{2:10} The Heb. lacks \fbib{of grief}}
\labelchapt{3}
\passage{Ezekiel's Commission to Prophesy}

\chapt{3}
\v{1}Then he told me, ``Son of Man, eat! Eat what you see\fnote{\fbackref{3:1} Lit. \fbib{find}}---this scroll---and then go talk to the house of Israel.'' \v{2}So I opened my mouth and he fed me\fnote{\fbackref{:2} Lit. \fbib{he caused me to eat}} the scroll.

\v{3}Then he told me, ``Son of Man, fill your stomach and digest this scroll that I'm giving you.'' So I ate it, and it was like sweet honey in my mouth.

\v{4}Then he told me, ``Son of Man, go to the house of Israel and tell them what I have to say to them, \v{5}because you're not going to a people whose speech you cannot understand or whose language is difficult to speak. Instead, you're going to the house of Israel. \v{6}This isn't a large group of people whose speech is unintelligible to you or whose language is difficult for you to comprehend. Frankly, if I had sent you to that kind of people,\fnote{\fbackref{3:6} Lit. \fbib{to them}} they would certainly have listened to you! \v{7}But the house of Israel won't listen to you, since they weren't willing to listen to me. That's because the entire house of Israel is hard-headed and hard-hearted. \v{8}So pay attention! I'm going to make you just as obstinate\fnote{\fbackref{3:8} Lit. \fbib{making your face hard against their faces}} and unyielding as they are.\fnote{\fbackref{3:8} \fbib{and your forehead as hard as their foreheads}} \v{9}I'm making you harder than flint---like diamond! So you are not to fear them or be intimidated by how they look at you,\fnote{\fbackref{3:9} The Heb. lacks \fbib{at you}} since they're a rebellious group.''
\passage{Ezekiel is Commissioned to Speak}

\v{10}Next, he told me, ``Son of Man, take to heart every word that I'm telling you. Listen carefully, \v{11}then go immediately\fnote{\fbackref{3:11} Lit. \fbib{walk and go}} to the exiles; that is, to your people's descendants, and tell them, `This is what the Lord \divine{God} says{\ldots}' whether they listen or not.''\fnote{\fbackref{3:11} Lit. \fbib{or fail}}

\v{12}Then the Spirit lifted me up and I heard a great earthquake behind me and the glory of the \divine{Lord} arose from his place, \v{13}accompanied by the sound of the wings of the living creatures gently touching each other and with the sound of the wheels emanating from the front, accompanied by a great earthquake.
\passage{Ezekiel Addresses the Israelis}

\v{14}Then the Spirit lifted me up and carried me away. I went bitterly with an angry attitude as the hand of the \divine{Lord} rested on me. \v{15}I came to the exiles at Tel-abib by the Chebar River and sat down among them for seven days, appalled. \v{16}At the end of the seven days, this message from the \divine{Lord} came to me: \v{17}``Son of Man,'' he said,\fnote{\fbackref{3:17} The Heb. lacks \fbib{he said}} ``I've appointed you to be a watchman\fnote{\fbackref{3:17} cf. 2Sam 18:24; 2King 9:17} over the house of Israel. Therefore when you hear a message that comes from me, you are to warn them for me.

\v{18}``So when I say to a wicked person, `You're about to die,' if you don't warn or instruct that wicked person that his behavior\fnote{\fbackref{3:18} Lit. \fbib{ways}} is wicked so he can live, that wicked person will die in his sin, but I'll hold you responsible for his death.\fnote{\fbackref{3:18} Lit. \fbib{but I'll seek his blood from your hand}} \v{19}If you warn the wicked person, and he doesn't repent of his wickedness or of his wicked behavior,\fnote{\fbackref{3:19} Lit. \fbib{ways}} he'll die in his sin, but you will have saved your own life.

\v{20}``When a righteous man abandons his righteousness to practice unrighteousness, I'll set a stumbling block before him. He'll die. If you don't warn him, he'll die in his sin and the righteous deeds that he had practiced won't be remembered, but you'll be held responsible for his death.\fnote{\fbackref{3:20} Lit. \fbib{but I'll seek his blood from your hand}} \v{21}If you warn the righteous person, so that he\fnote{\fbackref{3:21} Lit. \fbib{righteous person}.} doesn't commit sin, then he'll live, since he had been warned. And you will have saved your life.''
\passage{Ezekiel Sees God in the Valley}

\v{22}The hand of the \divine{Lord} was upon me, and he told me, ``Get up! Go to the valley, and I'll speak with you there.'' \v{23}So I got up, went to the valley, and there was the glory of the \divine{Lord}, standing there just like\fnote{\fbackref{3:23} Lit. \fbib{there like the glory of the \divine{Lord} that}} I had seen at the Chebar River. So I fell on my face.

\v{24}The Spirit entered me, rested on me, caused me to stand on my feet, and then he spoke to me. This is what he had to say: ``Go barricade yourself in your house. \v{25}Now pay attention! They're going to bind you with ropes, tying you up right in their midst, so you won't be able to circulate freely among them. \v{26}Meanwhile, I'll make your tongue stick to the roof of your mouth so that you'll be mute and unable to reprove them, since they're a rebellious group.\fnote{\fbackref{3:26} Lit. \fbib{house}} \v{27}But when I speak with you, I'll open your mouth so you can say to them, `This is what the Lord \divine{God} says:

\begin{poetry}
\poeml ``As for those who will listen, \\
\poemll    `Let them listen,' \\
\poeml but as for those who refuse, \\
\poemll    `Let them refuse,' \\
\poemlll       since they're a rebellious group.''\,'\,''\fnote{\fbackref{3:27} Lit. \fbib{house}}
\end{poetry}
\labelchapt{4}
\passage{The Vision of the Brick}

\chapt{4}
\v{1}``And now Son of Man, you are to take a brick,\fnote{\fbackref{4:1} Or \fbib{tile}} set it in front of you, and inscribe on it the outline of\fnote{\fbackref{4:1} The Heb. lacks \fbib{the outline of}} the city---that is, Jerusalem.\fnote{\fbackref{4:1} I.e. a symbolic map of the city} \v{2}You are to lay siege against it, build a rampart around it, set a bulwark against it, encircle it with a berm, set up camps against it, and place battering rams around it. \v{3}Then you are to take a flat, iron plate and set it up as an iron wall between you and the city.

``Next, you are to turn toward it, oppose\fnote{\fbackref{4:3} Lit. \fbib{it, set your face against}} it, and place it under siege, because you are to lay siege to it. All of this will serve as a sign to the house of Israel.

\v{4}``Now as for you, you are to sleep\fnote{\fbackref{4:4} Lit. \fbib{lay}} on your left side, symbolically\fnote{\fbackref{4:4} The Heb. lacks \fbib{symbolically}} bearing the punishment\fnote{\fbackref{4:4} Or \fbib{iniquity}} of the house of Israel while you're counting the days you'll be sleeping on your left side\fnote{\fbackref{4:4} Lit. \fbib{on it}} to bear symbolically\fnote{\fbackref{4:4} The Heb. lacks \fbib{symbolically}} the punishment for\fnote{\fbackref{4:4} Or \fbib{the iniquity of}} their sin. \v{5}I've assigned you to sleep this way for 390 days, representing the years they've been sinning,\fnote{\fbackref{4:5} I.e. one year for each day} as you bear symbolically\fnote{\fbackref{4:5} The Heb. lacks \fbib{symbolically}} the punishment of the house of Israel. \v{6}When you have completed this, you are to sleep\fnote{\fbackref{4:6} Lit. \fbib{lay}} on your right side, symbolically\fnote{\fbackref{4:6} The Heb. lacks \fbib{symbolically}} bearing the iniquity of Judah for 40 days. Each day that I've assigned to you represents one year. \v{7}After this, you are to turn toward the rampart of Jerusalem and oppose\fnote{\fbackref{4:7} Lit. \fbib{and to set your face against}} it with your bare arms, because I'm going to prophesy about it. \v{8}Look! I'll tie you up\fnote{\fbackref{4:8} Lit. \fbib{I'll set ropes on you}.} so that you're unable to turn from one side to the other until you've completed your siege.''
\passage{Ezekiel's Menu}

\v{9}``Furthermore, you are to take some wheat, barley, beans, lentils, millet, and spelt, and mix them together in one container. Then you are to make bread from these grains sufficient to supply you through the time during which you'll be sleeping on your side. You are to eat it for 390 days. \v{10}The food that you'll be eating is to consist of portions weighing 20 shekels,\fnote{\fbackref{4:10} I.e. about eight ounces; a shekel weighed about 0.4 ounces} to be consumed daily at regular intervals.\fnote{\fbackref{4:10} Lit. \fbib{it from time to time}} \v{11}You are to measure one sixth of one hin\fnote{\fbackref{4:11} I.e. about a pint and a half} of water each time you drink it. \v{12}You are to eat it as barley cakes and bake it right in front of them, using dried human dung for cooking fuel.''\fnote{\fbackref{4:12} The Heb. lacks \fbib{for cooking fuel}}

\v{13}Then the \divine{Lord} said, ``This is how the Israelis will be eating unclean food among the nations, where I'll be sending them.''

\v{14}``Now, Lord \divine{God},'' I replied, ``I've never been defiled, ever since I was young until now. I haven't eaten an animal that died on its own or was torn by beasts, and no unclean meat has ever entered my mouth!''

\v{15}``Very well,'' he responded. ``I'll allow you to substitute cow's dung for human dung. Cook your food\fnote{\fbackref{4:15} Lit. \fbib{bread}} over that.''

\v{16}He also told me, ``Son of Man, look! I'm about to disrupt the source\fnote{\fbackref{4:16} Lit. \fbib{staff}} of bread in Jerusalem. As a result, they'll ration bread by weight while their terror continues to grow and they'll ration drinking water while their horror continues to mount! \v{17}Indeed, they'll need bread and water, but everyone will be panic-stricken as they waste away in their iniquity.''
\labelchapt{5}
\passage{Ezekiel Shaves with a Sword}

\chapt{5}
\v{1}``Now as for you, Son of Man, you are to go find a sharp sword and use it like a barber's razor. You are to cut your hair and beard. Then you are to take a weighing scale and divide your shaved hair into three parts.\fnote{\fbackref{5:1} The Heb. lacks \fbib{your shaved hair into three parts}} \v{2}You are to burn a third of it in the middle of the city when you've finished your siege. Next, you are to take another third of it and beat it with your sword. Last, you are to scatter the remaining third to the wind, after which I'll unsheathe my sword and pursue them. \v{3}You are to preserve a few strands of hair and hide them in the folds\fnote{\fbackref{5:3} Lit. \fbib{wings}} of your garment. \v{4}Then you are to take a few strands, throw them in the fire, and incinerate them. A fire will proceed to the house of Israel from there.''
\passage{Jerusalem's Desolation Predicted}

\v{5}``This is what the Lord \divine{God} says, `This is Jerusalem. I placed her in the center of nations, with many\fnote{\fbackref{5:5} The Heb. lacks \fbib{many}} nations surrounding her. \v{6}But she rebelled against my ordinances and my statutes. She practiced more evil than all the nations and territories around her. They rejected my ordinances and didn't live by\fnote{\fbackref{5:6} Lit. \fbib{didn't walk in}} my statutes.'

\v{7}``Therefore this is what the Lord \divine{God} says: `Because you're more disrespectful than the nations that surround you, you didn't follow my statutes or follow my ordinances. You didn't even follow the ordinances of the surrounding nations!'

\v{8}``Therefore this is what the Lord \divine{God} says: `Watch out! I---that's right, even I---am against you. I'll carry out my sentence among you right in front of the nations. \v{9}In fact, I'm going to do what I've never done before and what I'll never again do, because of all of your loathsome behavior: \v{10}Fathers will eat their children in your midst. After this, your sons will eat their fathers as I carry out my sentence against you and scatter your survivors to the winds!'

\v{11}``Therefore, as sure as I live,'' declares the Lord \divine{God}, ``because you've defiled my sanctuary with every loathsome thing and every abomination, I'll restrain myself, and I'll show neither pity nor compassion.\fnote{\fbackref{5:11} Lit. \fbib{and my eyes won't show pity and I won't have compassion}} \v{12}A third of you will die by pestilence, starving because of the famine in your midst. Another third will die violently by the violence of war\fnote{\fbackref{5:12} Lit. \fbib{will fall by the sword}} around you. The final third I'll scatter to the wind as I unsheathe my sword to pursue them.

\v{13}``Only then will I stop being angry---my burning in anger. Then they'll know that I've spoken out in my arduous anger. Only then will my burning anger\fnote{\fbackref{5:13} The Heb. lacks \fbib{anger}} against them be exhausted. \v{14}I'm also going to turn you into a waste and an object of insult among the nations that surround you and in front of every person who passes by. \v{15}As a result, Jerusalem\fnote{\fbackref{5:15} MT reads \fbib{it}; DSS 11QEzek reads \fbib{you}} will become an insult, an object of taunt, an example of chastisement, and a useless waste to all the nations that surround you when I carry out my sentence against you in my anger, my burning rage, and my burning rebukes. I, the \divine{Lord}, have spoken it. \v{16}I'll send arrows of severe famine in their direction, meant for destruction, which I'll shoot, intending to destroy them. I'll make you have more and more famines that will attack you, and I'll disrupt your source of food.\fnote{\fbackref{5:16} Lit. \fbib{your staff of bread}}

\v{17}``I'll send famine and wild beasts against you that will rob you of your children.\fnote{\fbackref{5:17} Lit. \fbib{will make you childless}} Pestilence and bloodshed will devastate you when\fnote{\fbackref{5:17} Lit. \fbib{because}} I'll declare war on\fnote{\fbackref{5:17} Lit. \fbib{I'll bring the sword against}} you. I, the \divine{Lord}, have spoken.''
\labelchapt{6}
\passage{Prophecy against the Mountains of Israel}

\chapt{6}
\v{1}The \divine{Lord} continued with his message to me. \v{2}``Son of Man,'' he said, ``turn your face to oppose the mountains of Israel and prophesy against them. \v{3}Tell the mountains of Israel to listen as the Lord \divine{God} speaks. This is what the Lord \divine{God} has to say to the mountains, hills, streams, and the valleys: `Look! I'm about to bring my sword against you. I'm going to destroy your high places. \v{4}Your altars will become desolate and your sun pillars will be shattered. I'll throw your slain down right in front of your idols. \v{5}I'll place the corpses of the Israelis in front of their idols. I'll scatter your bones around your altar. \v{6}In all the places where you live, the cities will be desolate. The high places will also be desolate so that your altars will be laid waste, bearing the punishment appropriate to them.\fnote{\fbackref{6:6} The Heb. lacks \fbib{appropriate to them}} Your idols will be shattered, your sun pillars will be hewn down, and your works will be obliterated. \v{7}The fatally wounded among you will fall, and at that time you'll know that I am the \divine{Lord}. \v{8}I'll leave a remnant among you---those who will escape the sword when I'll have scattered you throughout the earth. \v{9}Your survivors will remember me among the nations where they'll be taken captives. I've been crushed by their unfaithful\fnote{\fbackref{6:9} Lit. \fbib{whoring}} hearts that have turned against me. \v{10}Then they'll know that I am the \divine{Lord}. I didn't declare this evil that's intended for them\fnote{\fbackref{6:10} \fbib{evil to do to them}} without a reason.'\,''

\v{11}This is what the Lord \divine{God} says: ``Clap your hands and stamp your feet! Say, `Oh, no!' Because of all the detestable evil that has come from Israel's house, they'll fall by the sword, famine, and pestilence. \v{12}The one who lives far away will die by pestilence and the one who is near will die violently.\fnote{\fbackref{6:12} Lit. \fbib{will fall by the sword}} The survivors and their surveillance details will die by famine as I exhaust my rage against them.

\v{13}``You'll learn\fnote{\fbackref{6:13} Or \fbib{know}} that I am the \divine{Lord}, when the fatally wounded will be among their idols, around their altars, on every hill, on top of the mountains, under every luxuriant tree, and under all the full-grown\fnote{\fbackref{6:13} Lit. \fbib{under every high}} foliage---every place where they've offered fragrant aromas to all their idols. \v{14}I'll stretch out my hands to strike\fnote{\fbackref{6:14} Lit. \fbib{hands against}} them and send devastation to the land, from the wilderness of Diblah, throughout all their dwelling places. Then they'll know that I am the \divine{Lord}.''
\labelchapt{7}
\passage{The End has Come}

\chapt{7}
\v{1}This message from the \divine{Lord} arrived for me: \v{2}``Son of Man, this is what Lord \divine{God} says to the land of Israel: `It's over! All four corners of the land are out of time! \v{3}Your time is up! I'm sending my anger against you to judge you according to how you live your lives,\fnote{\fbackref{7:3} Lit. \fbib{to your ways}} and I'm going to pay you back with the consequences of all your detestable practices. \v{4}I won't be showing pity on you and I won't be showing compassion. I'm going to turn your own lifestyles against you while your detestable practices remain among you. Then you'll learn\fnote{\fbackref{7:4} Or \fbib{know}} that I am the \divine{Lord}.'\,''
\passage{One Bad Thing after Another}

\v{5}``This is what the Lord \divine{God} says: `It's one evil event after another!

```Look out! It's coming!

\v{6}```The end is coming!

```The end is here!

```And it's looking in your direction!\fnote{\fbackref{7:6} Lit. \fbib{looking for you}}

```Look out! It's arrived!

\v{7}```Your doom has come to you, you who live in the land. The time has arrived, and the day of confusion is near. There will be no shouts of joy on the mountains. \v{8}Very soon now, I'll pour out my burning anger on you. I'll complete expressing my anger at you, judge you according to your behavior, and repay you for all your detestable practices. \v{9}I won't be showing pity or compassion. I'll repay you according to your behavior while your detestable practices remain among you. And you'll know that I, the \divine{Lord}, have been attacking you.'\,''\fnote{\fbackref{7:9} The Heb. lacks \fbib{you}}
\passage{The Harvest Approaches}

\v{10}``Look out! The day!

``Look out! It's coming!

``Doom has blossomed.

``Arrogance has sprouted!

\v{11}``Violence has matured into a branch that is wicked. No one will survive from that vast crowd, from their wealthy people, or from the famous among them.

\v{12}``The time has come!

``The day has arrived.\fnote{\fbackref{7:12} Lit. \fbib{reached}} Don't let the buyer rejoice, nor the seller lament, because wrath is coming to attack the entire multitude. \v{13}The seller won't regain what he has sold while the crowd remains\fnote{\fbackref{7:13} Lit. \fbib{while they're}} alive, because the vision concerning the entire multitude won't be annulled. No person will be able to survive because of the sin in his life.

\v{14}``They've sounded the alarm,\fnote{\fbackref{7:14} Lit. \fbib{They've blown the trumpet}} and everyone is prepared, but no one is marching for battle, since I'm angry at the entire multitude. \v{15}The sword lurks outside, but pestilence and famine are on the prowl inside the house. Whoever is in the field will die by violence,\fnote{\fbackref{7:15} Lit. \fbib{by the sword}} while famine and pestilence will devour those in the city. \v{16}Fugitives will escape to the mountains like doves fleeing through the valleys, all of them moaning because of their own iniquity. \v{17}Every hand will be limp. Every knee will glisten with sweat.''\fnote{\fbackref{7:17} Lit. \fbib{water}}
\passage{The Coming Terror}

\v{18}``They'll clothe themselves with sackcloth, terror will overcome them, shame will cover their faces, and baldness will spread over their entire heads. \v{19}They'll fling their silver into the streets, and their gold will be cast away as impure. Their silver and gold won't be able to deliver them during the time\fnote{\fbackref{7:19} Lit. \fbib{day}} of the \divine{Lord}'s wrath. They won't be able to satisfy their appetites or fill their stomachs, because their iniquity has tripped them up.''
\passage{The Temple Defiled}

\v{20}``As for his beautiful ornament,\fnote{\fbackref{7:20} I.e. the temple in Jerusalem} he set it up in majesty, but they made detestable images and loathsome idols. Therefore, I'll give them something loathsome--- \v{21}I'll give it as plunder into the control of strangers and as the spoils of war to the wicked who will invade the land to profane it. \v{22}I'll turn my face away from them so that they'll defile my treasured place. Robbers will enter and profane it!

\v{23}``Forge a chain, because the land is full of bloody judgment and the city is filled with violence. \v{24}Therefore, I'm bringing the worst of the nations, who will take possession of their houses. I'll cause the pride of the mighty to cease, and their sanctuaries will be profaned.

\v{25}``When destruction comes, they'll seek peace, but there will be none to be found. \v{26}Disaster upon disaster will come, followed by rumor after rumor. They'll seek an oracle from the prophet, but the Law will be gone from the priests, and counsel from the elders.

\v{27}``The king will mourn, the prince will be clothed with desolation,\fnote{\fbackref{7:27} Or \fbib{with torn garments}} and the hands of the people of the land will tremble. I'll deal with them according to their behavior and I will judge them by how they've judged. Then they'll learn\fnote{\fbackref{7:27} Or \fbib{know}} that I am the \divine{Lord}.''
\labelchapt{8}
\passage{The Vision of Jerusalem}

\chapt{8}
\v{1}In the sixth year, on the fifth day of the sixth month, I had just sat down in my house, with the elders of Judah seated in front of me. All of a sudden, the hand of the Lord \divine{God} touched me \v{2}and I saw a likeness comparable to the appearance of a man. From his thighs downward there was the appearance of fire, and from his waist upward, there was the appearance of brightness that looked like brass.

\v{3}The form of a hand reached out and took me by the hair of my head. Then the Spirit lifted me up between the earth and sky, brought me toward Jerusalem, and in visions that came from God took me through the doors of the inner gate that faced north, where an image that provoked God's jealous anger had been erected.

\v{4}All of a sudden, the glory of the God of Israel was there! It looked like what I had seen back in the valley. \v{5}Then he told me, ``Son of Man, look up toward the north.''

So I looked off toward the north. Suddenly, off toward the north, facing the gate that led to the altar, the image that provoked God's jealousy was standing near the entrance.

\v{6}Then the Spirit\fnote{\fbackref{8:6} Lit. \fbib{Then he}} told me, ``Son of Man, don't you see what they're doing? The house of Israel practices awful, detestable things here, so I'm going far away from my sanctuary. But you're about to see things even more detestable than these.''
\passage{Idol Worship in the Temple}

\v{7}Then the Spirit\fnote{\fbackref{8:7} Lit. \fbib{Then he}} brought me to the entrance of the court. As I watched, all of a sudden, there was a\fnote{\fbackref{8:7} Lit. \fbib{one}} hole in the wall! \v{8}Then he told me, ``Son of Man, dig through the wall!'' So I dug into the wall. That's when I uncovered an entrance!

\v{9}Then he told me, ``Go on through that entrance, so you may see the wicked, detestable things that they're committing here.''

\v{10}So I entered, looked around, and there was every form of crawling thing, loathsome animals, and all kinds of idols from the house of Israel carved all around the wall. \v{11}I saw 70 men from the elders of the house of Israel standing among them, including Shaphan's son Jaazaniah. Each man held a censer in his hand. As the scent of the cloud of incense ascended, \v{12}the Spirit\fnote{\fbackref{8:12} Lit. \fbib{he}} asked me, ``Do you see, Son of Man, what the elders of Israel's house are doing in secret, each in the chamber of his own carved idol? They keep saying, `God doesn't see us. The \divine{Lord} has abandoned the land.'\,''

\v{13}Then the Spirit\fnote{\fbackref{8:13} Lit. \fbib{Then he}} told me, ``You're about to see even more detestable practices that they're doing!''
\passage{Women Weeping for Tammuz}

\v{14}Then he brought me to the entrance of the gate to the \divine{Lord}'s Temple, which faced the north. That's where I saw women seated, weeping for Tammuz. \v{15}Then he asked me, ``Do you see this, Son of Man? You're about to see even more detestable practices than these.''
\passage{Sun Worship in the Temple}

\v{16}Then he brought me to the inner court of the \divine{Lord}'s Temple. There, at the entrance to the \divine{Lord}'s Temple, between the porch and the altar, were 25 men, with their backs toward the \divine{Lord}'s Temple and facing the east, prostrating themselves to the sun.

\v{17}``Do you see this, Son of Man?'' he asked me. ``Is it an insignificant thing for Judah's house to commit the detestable things that they're doing here? They've filled the land with violence and turned away from me, causing me to become angry again. Look how they're sniffing with their noses!\fnote{\fbackref{8:17} So MT; i.e. using flora to create or sustain an altered state during idolatrous worship; LXX reads \fbib{with contempt}} \v{18}I'm going to deal with them in rage and anger. I'll show neither pity nor compassion. They'll cry loudly directly in my ears, but I won't listen to them.''
\labelchapt{9}
\passage{The Vision of the Executioners}

\chapt{9}
\v{1}Then the Spirit\fnote{\fbackref{9:1} Lit. \fbib{Then he}} shouted right in my ears with a loud voice! ``Come forward,'' he said, ``you executioners of the city, and bring your weapon of destruction in your\fnote{\fbackref{9:1} Lit. \fbib{his}} hand!''

\v{2}All of a sudden, I noticed six men approaching from the direction of the upper gate, which faces north. Each of them held a destructive weapon in his hand. Among them there was one man, clothed in linen, who was equipped with a writing set\fnote{\fbackref{9:2} I.e., a case containing ink and writing implements} at his side. They went in and presented themselves beside the bronze altar. \v{3}Then the glory that is Israel's God arose from the cherubim on which he had been seated and settled on the threshold of the Temple. He called out to the man dressed in linen who wore the writing case at his side.

\v{4}The \divine{Lord} told him, ``Go throughout the city of Jerusalem and put a mark on the foreheads of everyone who sighs and moans over all of the loathsome things that are happening in it.''

\v{5}As I continued to listen, he also told the others, ``Follow him through the city and start killing. Don't spare anyone you see, and don't show pity of any kind. \v{6}You are to execute old men, young men, young women, little children, and women. But don't touch anyone who has been marked. Begin at my Holy Place!'' And so they started with the elders who were in standing in front of the Temple.

\v{7}``Desecrate my Temple,'' he told them, ``and fill its courtyard with the dead!'' So they went out and began striking down people throughout the city.
\passage{Ezekiel Intercedes for Israel}

\v{8}While they were out carrying out the executions, I was left alone. So I fell on my face and cried out, ``O Lord \divine{God}, are you going to destroy all of the survivors of Israel when you pour out your anger on Jerusalem?''

\v{9}``The house of Israel and Judah is guilty---and theirs is a stubborn guilt, at that!'' he replied to me. ``The land is filled with blood, and the city overflows with injustice, because they keep saying, `The \divine{Lord} has abandoned the land,' and `The \divine{Lord} isn't watching.' \v{10}So as for me, I'm not going to show pity, and I won't look in their direction with mercy. I'm repaying them for what they have done.''

\v{11}Then I noticed the man dressed in linen who wore the writing case by his side as he brought back this message: ``I've done as you have commanded me.''
\labelchapt{10}
\passage{The Vision of God's Throne}

\chapt{10}
\v{1}As I continued to watch, there on the expanse above the heads of the cherubim was a massive\fnote{\fbackref{10:1} The Heb. lacks \fbib{massive}} sapphire stone that resembled a throne in form and appearance. \v{2}The \divine{Lord}\fnote{\fbackref{10:2} Lit. \fbib{He}} spoke to the man who was clothed in white linen, telling him, ``Go between the whirling wheels, under the cherubim, and fill your hands with burning coals from among the cherubim. Then scatter them\fnote{\fbackref{10:2} The Heb. lacks \fbib{them}} over the city.'' So he entered as I watched.\fnote{\fbackref{10:2} Lit. \fbib{entered in my sight}}

\v{3}Now the cherubim were standing on the south\fnote{\fbackref{10:3} Lit. \fbib{right side}} side of the entrance to the Temple, when the man entered and a cloud filled the inner court. \v{4}The glory of the \divine{Lord} rose above the cherub and moved to the threshold of the Temple. A cloud filled the Temple and the court was filled with the brilliance of the \divine{Lord}'s glory. \v{5}The sound of the wings of the cherubim, reminiscent of the voice of the Sovereign God when he speaks, could be heard as far as the outer court.

\v{6}He issued this order to the man who was clothed in white linen: ``Take fire from within the whirling wheels, among the cherubim.'' So he went and stood beside the wheels.
\passage{Ezekiel's Vision of the Cherubim}

\v{7}Then a cherub stretched out his hand to the fire, which was among the cherubim, took some of the fire, and placed it in the hands of the one clothed in white linen, who took it and left. \v{8}There appeared to be human hands under the wings of the cherubim.

\v{9}As I continued to watch, I observed four wheels beside the cherubim, one wheel beside each cherub.\fnote{\fbackref{10:9} Lit. \fbib{cherub and another wheel beside another cherub.}} The wheels resembled beryl stone. \v{10}In appearance, the four wheels looked like they consisted of a wheel within a wheel. \v{11}Whenever they moved, they proceeded without turning around as they moved, but they followed in the direction where their head was facing, without looking around as they moved.

\v{12}Their entire bodies, backs, hands, and wings were filled with eyes around, including each of their four wheels. \v{13}The wheels whose sound I was hearing were called ``the whirling wheels''. \v{14}Each had four faces. The first one was the face of a cherub, the second the face of a man, the third the face of a lion, and the fourth the face of an eagle.

\v{15}The cherubim arose. These were the same beings that I had seen at the Chebar River. \v{16}When the cherubim moved, the wheels went alongside them. But when the cherubim started to ascend, beating their wings to rise above the earth, the wheels beside them didn't turn. \v{17}When they stood still, the wheels stood still. When they rose up, the wheels rose up, too, because they were alive.\fnote{\fbackref{10:17} Lit. \fbib{because the spirit of the living beings resided in the wheels}}

\v{18}Then the glory of the \divine{Lord} moved away from the threshold of the Temple and stood over the cherubim. \v{19}The cherubim lifted their wings and rose above the earth while I watched. They went out, along with their wheels, and stood at the entrance to the east gate of the \divine{Lord}'s Temple as the glory of Israel's God remained above, covering them.

\v{20}These were the living beings that I had seen under the God of Israel on the bank of the Chebar River. I knew that they were cherubim. \v{21}Each one had four faces. Each one had four wings, and the form of human hands could be seen under their wings. \v{22}As to the likeness of their faces, they were like what I had seen on the bank of the Chebar River. They each moved straight ahead.
\labelchapt{11}
\passage{The Vision of the Eastern Gate}

\chapt{11}
\v{1}The Spirit lifted me up and brought me to the east facing gate of the \divine{Lord}'s Temple. At the entrance of the gate I saw 25 men. Included among them were Azzur's son Jaazaniah and Benaiah's son Pelatiah, who were princes of the people.

\v{2}Then he told me, ``Son of Man, these men are plotting evil and are giving wicked advice in this city. \v{3}They keep saying, `The right time to build families\fnote{\fbackref{11:3} Lit. \fbib{houses}} hasn't yet arrived. The city is the pot and we are the meat.' \v{4}Therefore you are to prophesy against them. Prophesy, Son of Man!''
\passage{God Rebukes those who Plot Evil}

\v{5}Just then the Spirit of the \divine{Lord} took control of\fnote{\fbackref{11:5} Lit. \fbib{\divine{Lord} fell on}} me and told me, ``You are to say, `This is what the \divine{Lord} says: ``You've said, O house of Israel, that I know what goes through your mind.\fnote{\fbackref{11:5} Or \fbib{spirit}} \v{6}You've increased the number of fatally wounded in this city and you've filled your streets with the dead.''

\v{7}`Therefore this is what the Lord \divine{God} says, ``The corpses that you've laid out in your midst are the meat, and this city is the cooking pot. But you'll be taken out from the middle of it. \v{8}You've feared the sword,\fnote{\fbackref{11:8} I.e. execution during military invasion; and so throughout the chapter} but I'm bringing violent death in your direction,''\fnote{\fbackref{11:8} Lit. \fbib{bringing the sword against you}} declares the Lord \divine{God}. \v{9}``I'm bringing you out from the middle of it and I'm going to deliver you into the hands of strangers, because I'm going to carry out my sentence against you. \v{10}You're going to die violently,\fnote{\fbackref{11:10} Lit. \fbib{to fall by the sword}} and I'll judge you as far as the borders of Israel. Then you'll learn\fnote{\fbackref{11:10} Or \fbib{know}} that I am the \divine{Lord}. \v{11}This city won't be your cooking pot and neither will you be the meat in it, because I'm going to judge you as far as the borders of Israel. \v{12}Then you'll learn\fnote{\fbackref{11:12} Or \fbib{know}} that I am the \divine{Lord}, because you didn't live by my statues or obey my ordinances. Instead, you obeyed the ordinances of the nations around you.''\,'\,''
\passage{Ezekiel Reacts to Pelatiah's Death}

\v{13}While I was prophesying, Benaiah's son Pelatiah died, so I fell on my face and cried out with a loud voice. ``Ah, Lord \divine{God},'' I said, ``are you going to put an end to the survivors within Israel?''

\v{14}Then this message came to me from the \divine{Lord}: \v{15}``Son of Man, your brothers, your other relatives, your fellow exiles,\fnote{\fbackref{11:15} So LXX and Syr; MT reads \fbib{your redeemers}} and the entire house of Israel are the people to whom the inhabitants of Jerusalem have said, `They've abandoned the \divine{Lord}. This land was given to us for an inheritance.'\,''
\passage{The Future Hope of Israel}

\v{16}``Therefore you are to say, `This is what the Lord \divine{God} says, ``Although I've removed them far away to live among the nations, and although I've scattered them throughout the earth, yet I've continued to be their sanctuary, even for the short time that they will be living in the lands to which they've gone.''\,'

\v{17}``Therefore you are to say, `This is what the Lord \divine{God} says, ``I'm going to gather you from among the nations, assembling you from the lands among which you have been dispersed. I'll give you the land of Israel. \v{18}When they return from there and cast away all of their loathsome things and detestable practices, \v{19}then I'll give them a united heart, placing a new spirit within them.\fnote{\fbackref{11:19} So LXX, Syriac, Targums, and Vulgate; MT reads \fbib{you} (pl)} I'll remove their stubborn heart\fnote{\fbackref{11:19} Lit. \fbib{heart from their flesh}} and give them a heart that's sensitive to me.\fnote{\fbackref{11:19} Lit. \fbib{heart of flesh}} \v{20}When they live by my statutes and keep my ordinances by observing them, then they'll be my people and I will be their God. \v{21}But to those whose hearts delight in loathsome things and detestable practices, I'll bring the consequences of their behavior crashing down on their own heads,'' declares the Lord \divine{God}.'\,''
\passage{The Cherubim Leave}

\v{22}Then the cherubim arose, with their wheels alongside, and the glory of Israel's God remained above and over them. \v{23}The glory of the \divine{Lord} went up from the middle of the city and stood on the mountain, east of the city. \v{24}Then in a vision from the Spirit of God, the Spirit lifted me up and brought me to the exiles in Chaldea. At that point, the vision that I had been observing ended. \v{25}Later, I spoke to the exiles concerning everything the \divine{Lord} had spoken that I had witnessed.
\labelchapt{12}
\passage{Ezekiel Packs for Exile}

\chapt{12}
\v{1}This message came to me from the \divine{Lord}: \v{2}``Son of Man, you live in a rebellious house that has eyes to see, but they can't see, and ears to hear, but they can't hear, since they're a rebellious house.

\v{3}``So now, Son of Man, you are to prepare your luggage for a trip into exile, and then you are to leave during the daytime so they see you leaving. Leave from your place to another while they're watching. Then perhaps they'll realize that they're a rebellious house.

\v{4}``Bring out your luggage, like you're packing to go into exile, and do this during the daytime while they're watching you.\fnote{\fbackref{12:4} Lit. \fbib{before their eyes}; and so through v. 7} Later that evening, leave while they're watching you like someone heading into exile. \v{5}While they continue to watch, dig a hole for yourself in the wall and enter through it.

\v{6}``While they're watching, carry your luggage\fnote{\fbackref{12:6} Lit. \fbib{carry it}} on your shoulder and go out in total\fnote{\fbackref{12:6} Lit. \fbib{thick}} darkness. Cover your face so that you won't see the land, because I'm using you as a sign to Israel's house.''

\v{7}I did just as I was commanded. I brought out the luggage as if it were luggage for exile. I did this during the day. Then in the evening I dug a hole in the wall with my hand and brought the luggage out in total\fnote{\fbackref{12:7} Lit. \fbib{thick}} darkness and carried it out on my shoulder while they were watching.
\passage{The Meaning of the Message}

\v{8}The next morning, this message came to me from the \divine{Lord}: \v{9}``Son of Man, didn't the house of Israel, that rebellious house, ask you, `What are you doing?' \v{10}Answer them, `This is what the Lord \divine{God} says, ``This oracle concerns the prince of Jerusalem and the whole of Israel's house that is in their midst. \v{11}Tell them, `I'm a sign for you. Just as I enacted it,\fnote{\fbackref{12:11} Lit. \fbib{did}} it's going to happen to them. They'll go into exile and captivity. \v{12}Then the prince, who will be one of them, will carry his luggage\fnote{\fbackref{12:12} The Heb. lacks \fbib{his luggage}} on his shoulder in the dark and will go out. They'll dig a hole in the wall for him to go through. His face will be covered so that he won't be able to see the land with his eyes. \v{13}But I'll throw my net over him. As a result, he'll be captured with my net, and with it I'll bring him to Babel, the land of the Chaldeans. He won't see it, though he'll die there. \v{14}I'll scatter every attendant who surrounds him, along with his entire army, to every wind. When I unsheathe my sword to pursue them, \v{15}they'll learn\fnote{\fbackref{12:15} Or \fbib{know}} that I am the \divine{Lord}, when I've dispersed them among the nations and scattered them throughout the earth.''\,'\,''
\passage{The Purpose of the Surviving Remnant}

\v{16}``But I'll preserve\fnote{\fbackref{12:16} Or \fbib{retain}} a few people out of the violent death,\fnote{\fbackref{12:16} Lit. \fbib{the sword}} famine, and pestilence, so they can recount their detestable practices among the nations when they'll go there. Then they'll know that I am the \divine{Lord}.''
\passage{The Coming Devastation}

\v{17}This message came to me from the \divine{Lord}: \v{18}``Son of Man, eat your bread with trembling and drink your water with quivering and anxiety. \v{19}Then tell the people of the land, `This is what the \divine{Lord} says to the inhabitants of Jerusalem, to Israel's land: ``They'll eat their food in anxiety and drink their water in trepidation, because their land will be desolate in its entirety due to all the violence committed by all who live in it. \v{20}The towns that are inhabited will lie in ruins, because the land will be devastated. Then they'll learn\fnote{\fbackref{12:20} Or \fbib{know}} that I am the \divine{Lord}.''\,'\,''
\passage{The Coming Fulfillment of Visions}

\v{21}Later, this message came to me from the \divine{Lord}: \v{22}``Son of Man, what's this proverb you have concerning Israel's land that says, `The days pass slowly and every vision ends in nothing.'?\fnote{\fbackref{12:22} Lit. \fbib{vision is destroyed}} \v{23}Therefore you are to tell them, `This is what the Lord \divine{God} says, ``I'm about to put an end to use of this proverb in Israel. It will never be used again as a proverb in Israel. Instead, tell them that the days are drawing near when every vision will be fulfilled. \v{24}There will no longer be worthless visions and flattering divinations in the midst of Israel's house. \v{25}Because I am the \divine{Lord}, I'll speak and the message that I communicate will be accomplished without delay. While you continue to be a rebellious house, I'll speak the message and then fulfill it,'' declares the Lord \divine{God}.'\,''
\passage{The Imminent Fulfillment}

\v{26}Later, this message came to me from the \divine{Lord}: \v{27}``Son of Man, pay attention! The house of Israel keeps on saying, `The vision that he's talking about concerns the distant future. He's prophesying concerning times that are far in the future!' \v{28}Therefore tell them, `This is what the Lord \divine{God} says, ``None of my messages will be delayed any longer. Any message that I speak will be fulfilled,'' declares the Lord \divine{God}.'\,''
\labelchapt{13}
\passage{A Prophecy against Prophets}

\chapt{13}
\v{1}This message came to me from the \divine{Lord}: \v{2}``Son of Man, prophesy against the prophets of Israel, who even now are prophesying, and tell those prophets that keep on prophesying according to what they wish would happen,\fnote{\fbackref{13:2} Lit. \fbib{prophesying from their heart}} `Listen to what the \divine{Lord} says.'\,''

\v{3}``This is what the Lord \divine{God} says, `How terrible it will be for the false prophets who walk according to their own wrong inclinations\fnote{\fbackref{13:3} Lit. \fbib{spirit}} and see nothing. \v{4}Israel, your prophets have become like foxes among ruins. \v{5}You didn't go up to repair\fnote{\fbackref{13:5} The Heb. lacks \fbib{repair}} the breaches in the walls and you didn't build the walls so Israel's house would be able to endure battle on the Day of the \divine{Lord}. \v{6}Instead, they crafted\fnote{\fbackref{13:6} Lit. \fbib{they have seen}} false prophecies and divination.

```They say, ``{\ldots}declares the \divine{Lord},'' even though the \divine{Lord} didn't send them. And they hope for the fulfillment of their message. \v{7}You've crafted\fnote{\fbackref{13:7} Lit. \fbib{seen}} a false prophesy and spoken deceptive divination, haven't you? But then you say, ``{\ldots}declares the \divine{Lord},'' although I haven't spoken a single word.

\v{8}```Therefore this is what the Lord \divine{God} says, ``Because you've spoken falsehood and deceptions, I am therefore opposing\fnote{\fbackref{13:8} Lit. \fbib{against}} you,'' declares the Lord \divine{God}. \v{9}My hand will oppose the prophets who see false visions and speak deceptive divinations. They won't be included with the council of my people, nor will they be entered into the registry of Israel's house or enter Israel's land. Then you'll know that I am the Lord \divine{God}, \v{10}because they've truly caused my people to stray saying, ``Peace,'' but there's no peace.'\,''
\passage{Metaphor of the Whitewashed Wall}

``When someone builds a wall, they coat it with whitewash. \v{11}Tell those who coat it with whitewash that it will fall. It will be washed off by the rain. Great hailstones will fall and a stormy wind will strip it off.\fnote{\fbackref{13:11} Lit. \fbib{rip it open}} \v{12}Look! When the wall collapses, won't it be said of you, `Where's the coat of paint that you spread all over the wall?'

\v{13}``Therefore this is what the Lord \divine{God} says, `In my burning anger, I'll rip it open with a windstorm. In my anger, I'll rinse it off with rain, and put an end to it with a hailstorm in my destructive rage. \v{14}I'll tear down the wall that you've smeared with whitewash, level it to the ground, and tear out its foundation. Then it will collapse---and you'll perish with it! Then you'll know that I am the \divine{Lord}.

\v{15}```That's how I'll vent my anger on the wall and on the ones who coated it with whitewash. And I'll say to you, ``The wall is gone and so are those who coated it.''\fnote{\fbackref{13:15} Lit. \fbib{Those who coated it are not.}} \v{16}The prophets of Israel prophesied about Jerusalem and saw visions of peace concerning her, yet there's no peace,'\,'' declares the Lord \divine{God}.
\passage{A Rebuke to Israel's Women}

\v{17}``And now, Son of Man, turn toward and oppose\fnote{\fbackref{13:17} Lit. \fbib{Man, set your face against}} the women\fnote{\fbackref{13:17} Lit. \fbib{daughters}} of your people who prophesy according to their own wrong inclinations\fnote{\fbackref{13:17} Lit. \fbib{spirit}} and prophesy against them. \v{18}Tell them, `This is what the Lord \divine{God} says, ``How terrible it will be for those women who sew magical bracelets on all their wrists and make one-size-fits all headbands,\fnote{\fbackref{13:18} Or \fbib{veils}} in order to entrap their souls. Will you hunt for the souls of my people and remain alive? \v{19}You've profaned me among my people for a handful of barley and a morsel of bread. You're causing people to die who shouldn't have to die, and you're causing people to live who shouldn't survive, when you deceive my people who tend to listen to lies.''

\v{20}```Therefore, this is what the Lord \divine{God} says, ``Watch out! I'm opposing your amulets with which you hunt souls as one would swat at a flying insect.\fnote{\fbackref{13:20} Lit. \fbib{flying thing}} I'll tear them off your arms and then deliver those people, whom you've hunted like birds. \v{21}I'll also tear off your headbands\fnote{\fbackref{13:21} Or \fbib{veils}} and deliver my people from your grip so that they won't be under your control anymore. Then you'll know that I am the \divine{Lord}.

\v{22}`````Because you've dismayed the heart of the righteous---whom I never intended to dismay---with lies, and because you've encouraged\fnote{\fbackref{13:22} Lit. \fbib{you've strengthened the hand of}} the wicked so that he wouldn't abandon his evil behavior and by doing so live, \v{23}you'll no longer see false visions or again practice divination, because I'm going to deliver my people from your power. Then you'll know that I am the \divine{Lord}.''\,'\,''
\labelchapt{14}
\passage{A Prophecy against Idolatry}

\chapt{14}
\v{1}Later, some men from the elders of Israel came to visit me. After they had sat down in my presence, \v{2}this message came to me from the \divine{Lord}.

\v{3}``Son of Man, these men have taken idols into their hearts. They've placed the stumbling block that is their own iniquity right in front of their faces. Should I be consulted by them at all? \v{4}Therefore, speak up and tell them, `This is what the Lord \divine{God} says, ``Every person from Israel's house who follows his idols and sets the stumbling block that is his own sin in front of his face, and then consults a prophet, I the \divine{Lord} will answer him according to how many idols he embraces. \v{5}I'll do this in order to capture the hearts of Israel's house who have become alienated from me due to all of their idols.''\,'\,''
\passage{An Exhortation to Turn Away}

\v{6}``Therefore you are to tell Israel's house, `This is what the Lord \divine{God} says, ``Turn away! Turn away from your idols, and abandon your detestable practices! \v{7}For when a native Israeli or a resident alien abandons me to set up idols in his heart behind my back, and then places the stumbling block of his iniquity right in front of his own face, then approaches a prophet to inquire of me on behalf of his own self-interest, I, the \divine{Lord} will answer him myself. \v{8}I'm determined to oppose that person\fnote{\fbackref{14:8} Lit. \fbib{man}} and make him an example. Proverbs will be written about him\fnote{\fbackref{14:8} The Heb. lacks \fbib{will be written abut him}} when I eliminate him from my people. Then you'll know that I am the \divine{Lord}.''\,'\,''
\passage{On False Prophets}

\v{9}``Now as to the prophet, if through deceit he delivers a message, I the \divine{Lord} have deceived that prophet! I'll reach out in opposition to him and exterminate him from among my people Israel. \v{10}They'll bear the consequences of their guilt, and the prophet will be just as guilty as the one who seeks that prophet's guidance. \v{11}Then Israel's house won't wander away from me again, nor will they defile themselves again with all their transgressions. They'll become my people and I'll be their God,'' declares the Lord \divine{God}.
\passage{On Noah, Daniel, and Job}

\v{12}This message came to me from the \divine{Lord}: \v{13}``Son of Man, when a nation\fnote{\fbackref{14:13} Lit. \fbib{land}} sins against me by a treacherous act,\fnote{\fbackref{14:13} Lit. \fbib{nation acts treacherously by a treacherous act}} I'll reach out to oppose it, destroying its source of food,\fnote{\fbackref{14:13} Lit. will break \fbib{in pieces its staff of bread.}} by sending famine against it, and by destroying both people and beast within it. \v{14}Though these three men, Noah, Daniel,\fnote{\fbackref{14:14} cf. Eze 28:3} and Job lived in that land, they would only save their own lives on account of their righteousness,'' declares the Lord \divine{God}.

\v{15}``If I were to make wild animals pass throughout the land, so that they kill its residents\fnote{\fbackref{14:15} Lit. \fbib{children}} and it were to become desolate because no one will travel through it due to those wild animals,\fnote{\fbackref{4:15} Lit. \fbib{in the face of living beings}} \v{16}then even though these three men were in it, as I live,'' says the Lord \divine{God}, ``they wouldn't be able to deliver even their sons or daughters. They would only save themselves, but the land would become desolate.

\v{17}``Or if I were to bring war to\fnote{\fbackref{14:17} Lit. \fbib{bring a sword against}} that land and say, `Hey, sword! Pass throughout the land so I can destroy both man and beasts in it,' \v{18}though these three men lived there, as I live,'' declares the Lord \divine{God}, ``they couldn't deliver their own sons and daughters. They would only save themselves.

\v{19}``Or if I were to send a pestilence against that land and pour out my anger in it with bloodshed, destroying both man and beast in it, \v{20}even though Noah, Daniel, and Job were among them, as I live'' says the Lord \divine{God}, ``they couldn't save their own sons or daughters. They would only save their own souls due to their own righteousness.''

\v{21}This is what the Lord \divine{God} says, ``I'm sending four of my most destructive judgments---military invasion,\fnote{\fbackref{14:21} Lit. \fbib{judgments---the sword}} famine, wild animals, and pestilence---into Jerusalem to destroy both human beings and livestock in it. \v{22}But look! There will be a remnant who escapes, a few sons and daughters to be brought out. Look! They'll come out to you and you'll see how they've lived and what they've done, and you'll be comforted concerning the catastrophe that I brought on Jerusalem, including everything that I brought against her. \v{23}They'll comfort you when you see how they've lived and what they've done, because you'll know for certain that I haven't done anything that I've done against them without any reason,''\fnote{\fbackref{14:23} Or \fbib{cause}} declares the Lord \divine{God}.
\labelchapt{15}
\passage{A Message about Vines}

\chapt{15}
\v{1}This message came to me from the \divine{Lord}: \v{2}``Son of Man, how does wood from a vine compare to a branch taken from any of the trees in the forest? \v{3}Is wood ever taken from it to make anything practical? Can it even be made into a peg to hang something on? \v{4}After all, it's useful only for kindling a fire, isn't it? And once you've burnt up the ends and charred through the middle of it, is it useful for anything else? \v{5}If it was useless before it was burned, now that it's been burned and charred through, it's even more useless!

\v{6}Therefore this is what the Lord \divine{God} says: ``Just as the wood from a grape vine is removed from the forest and used for kindling fires, I'm giving the inhabitants of Jerusalem over \v{7}to punishment. They may have escaped one fire, but the coming fire will burn them up completely, and they will know that I am the \divine{Lord}, when I set myself in opposition to\fnote{\fbackref{15:7} Lit. \fbib{set my face against}} them \v{8}and dedicate the land to desolation because of their unfaithful unbelief,'' declares the Lord \divine{God}.
\labelchapt{16}
\passage{A Prophecy Confronting Jerusalem}

\chapt{16}
\v{1}This message came to me from the \divine{Lord}: \v{2}``Son of Man, make known to Israel her detestable practices. \v{3}You are to declare, `This is what the Lord \divine{God} says to Jerusalem: ``Your birth place\fnote{\fbackref{16:3} Lit. \fbib{Your origin and birth}} was the territory that belonged to the Canaanites. Your father was an Ammonite and your mother was a Hittite. \v{4}Now as to your birth, on the day you were born your umbilical cord wasn't cut. You weren't washed with water to clean you, and nobody rubbed you with salt. And it's certain that you weren't wrapped in strips of cloth. \v{5}Nobody pitied you to do any of these things for you, and nobody showed you any compassion. You were tossed outside on the ground, because you\fnote{\fbackref{16:5} Lit. \fbib{your soul}} were detested from the day you were born.

\v{6}`````When I passed by you, I saw you kicking around, covered in your own blood. That's when I told you, `Live!'---while you were wallowing in your blood. I commanded you to live, even as you lay there in your own blood. \v{7}I made you increase like sprouting grain\fnote{\fbackref{16:7} Lit. \fbib{like that which sprouts}} in the field. As a result, you multiplied greatly. Eventually, you reached the age when young women start wearing jewelry. Your breasts were formed, your hair had grown, but you were still bare and naked.''\,'\,''
\passage{God's Betrothal to Jerusalem}

\v{8}``When I passed by you again, I looked at you, and noticed that it was your proper time for love. I spread my cloak\fnote{\fbackref{16:8} Lit. \fbib{wings}} over you to cover your nakedness. I made a solemn promise to you and entered into a covenant with you,'' declares the Lord \divine{God}. ``You belong to me. \v{9}I bathed you with water, rinsed your own blood from you, and anointed you with oil. \v{10}Then I covered you with embroidered clothing, clothed your feet with leather sandals, wrapped\fnote{\fbackref{16:10} Lit. \fbib{bound}} you with fine linen, and dressed you in silk. \v{11}I adorned you with jewels, placing bracelets on your hand and necklaces on your neck. \v{12}I put a ring in your nose, earrings in your ears, and a crown encrusted with jewels on your head. \v{13}You were adorned with gold, silver, clothing of fine linen, silk, and embroidery. You ate food made from the finest flour, honey, and olive oil. You were exceedingly beautiful, attaining royal status. \v{14}Your fame\fnote{\fbackref{16:14} Lit. \fbib{name}} spread throughout the nations because of your beauty. You were perfectly beautiful due to my splendor with which I endowed you,'' declares the Lord \divine{God}.
\passage{Jerusalem's Arrogant Unfaithfulness}

\v{15}``But you trusted in your beauty. You did what whores do, as a result of your fame. You passed out your sexual favors\fnote{\fbackref{16:15} Lit. \fbib{adulteries}} to anyone who passed by, giving yourself to anyone. \v{16}You took some of your clothes and made gaily-colored high places and prostituted yourself all around them---something which had never happened before nor will ever happen again.

\v{17}``You also took your fine jewelry---including my gold and my silver that I had given you. Then you made for yourself male images and had sex with them! \v{18}You took your embroidered gowns and made clothes to cover them. Then you offered my olive oil and incense to them.

\v{19}``Not only that, you took the food I gave you---my fine flour, olive oil, and honey with which I fed you, and you offered\fnote{\fbackref{16:19} Or \fbib{set}} them to those gods\fnote{\fbackref{16:19} The Heb. lacks \fbib{to those gods}} in order to appease them.\fnote{\fbackref{16:19} Lit. \fbib{gods as a pleasing aroma}} That's exactly what happened,'' says the Lord \divine{God}. \v{20}``Then you took your sons and daughters whom you bore for me and sacrificed them for your idols to eat. As though your prostitutions were an insignificant thing, \v{21}you also slaughtered my sons and offered them to idols, incinerating them in fire.\fnote{\fbackref{16:21} The Heb. lacks \fbib{in fire}} \v{22}Throughout all of your detestable practices and immorality, you never did remember your earlier life when you were bare, naked, and wallowing in your own blood.''
\passage{The Unfaithfulness of God's People}

\v{23}``How terrible! How terrible it will be for all of your wickedness!'' declares the Lord \divine{God}. \v{24}``You built raised mounds and high places for yourself on every plaza. \v{25}At every street corner you made your beauty abhorrent when you made yourself available for sex to\fnote{\fbackref{16:25} Lit. \fbib{you parted your legs for}} anyone who was passing by. By doing this, you kept on committing more and more immorality. \v{26}Then you committed immorality with your neighbors, the Egyptians,\fnote{\fbackref{16:26} Lit. \fbib{sons of Egypt}} with perverted lust,\fnote{\fbackref{16:26} Lit. \fbib{with great flesh}} and by doing so you fornicated even more, provoking me to anger.

\v{27}``Therefore, look out! I've reached out to oppose you. I withdrew your rations and delivered you\fnote{\fbackref{16:27} Lit. \fbib{soul}} to those Philistine women who hate you. Even they were embarrassed at your wicked ways! \v{28}You committed immorality with the Assyrians,\fnote{\fbackref{16:28} Lit. \fbib{sons of Assyria}} because you still weren't satisfied. You committed immorality with them, but you still weren't satisfied. \v{29}You committed even more immorality with that land of the merchants, the Chaldeans. But you weren't satisfied even with these!

\v{30}``How weak is your heart,'' declares the Lord \divine{God}, ``when you committed all of these deeds, the acts of an imperious whore! \v{31}When you built your mound on every street corner and constructed your high place at every plaza, you weren't like a common prostitute, in that you've insulted the wages of a prostitute \v{32}who commits adultery, preferring a stranger over her husband!

\v{33}``All prostitutes receive gifts, but you give your gifts to all your lovers, then you bribe them to come to you from everywhere to get your sexual favors!\fnote{\fbackref{16:33} Lit. \fbib{your fornications}} \v{34}You're different from other women when you commit immorality---no one can match you in that!\fnote{\fbackref{16:34} Lit. \fbib{in how you commit immorality}} After all, you pay fees, but no fee is given to you. You're certainly different!''
\passage{The Coming Punishment}

\v{35}``Therefore listen to this message from the \divine{Lord}, you whore! \v{36}This is what the Lord \divine{God} says: `Because your lust has been poured out and your nakedness has been uncovered by your acts of fornication with your lovers, and because of all your detestable idols and the blood of your sons, whom you offered to them, \v{37}therefore, watch out! I'm about to gather all your lovers from whom you've received your pleasure, everyone whom you've loved, and those whom you've hated. I'll gather them together to oppose you from every side, and they'll uncover your nakedness in their presence. Then they'll see you completely naked. \v{38}I'll judge you with the same standards by which I issue verdicts against a woman who commits adultery and murder.\fnote{\fbackref{16:38} Lit. \fbib{and one who sheds blood}} I'll avenge the blood you've shed with impassioned wrath.\fnote{\fbackref{16:38} Lit. \fbib{wrath and passion}}

\v{39}``I'll also deliver you into their control, and they'll break down your mounds, tear down your high places, strip off your clothes, remove your fine jewels, and then they'll leave you stark naked! \v{40}They'll bring a mob against you to stone you to death\fnote{\fbackref{16:40} Lit. \fbib{death with stones}} and cut you into pieces with their swords. \v{41}Then they'll burn your houses and carry out my sentence\fnote{\fbackref{16:41} Lit. \fbib{and execute judgment}} against you in the sight of many women.

``That's how I'll make you stop your prostitution so you won't pay any prostitute's fees anymore. \v{42}I'll stop being angry with you, and I'll cease being jealous.\fnote{\fbackref{16:42} Lit. \fbib{I'll turn aside my jealousy from you}} I'll be calm and not be indignant anymore. \v{43}Because you didn't remember the time when you were young, but instead you provoked me to anger because of all these things, watch out! I'm going to bring your behavior back to haunt\fnote{\fbackref{16:43} The Heb. lacks \fbib{haunt}} you!'' declares the Lord \divine{God}. ``Didn't you do this wicked thing, in addition to all your other\fnote{\fbackref{16:43} The Heb. lacks \fbib{other}} detestable practices?''
\passage{Like Mother, Like Daughter}

\v{44}``Now, everyone who likes proverbs will quote this proverb about you, `Like mother, like daughter.' \v{45}You're the daughter of your mother, who loathed her husband and children. You're the sister of your sisters, who loathed their husbands and children.

``Your mother was a Hittite and your father was an Amorite. \v{46}Your elder sister was Samaria. She and her daughters lived in the north,\fnote{\fbackref{16:46} Lit. \fbib{lived on your left}} while your younger sister who lived in the south\fnote{\fbackref{16:46} Lit. \fbib{lived on your right}} with her daughters was Sodom. \v{47}It wasn't just that you lived like they did and committed their detestable practices, but in just a little while your behavior led you to become more corrupt than they were!''
\passage{Sins of Sodom}

\v{48}``As I live,'' declares the Lord \divine{God}, ``your sister Sodom and her daughters didn't do what you and your daughters have done. \v{49}Look! This was the sin of your sister Sodom and her daughters: Pride, too much food, undisturbed peace, and failure to help\fnote{\fbackref{16:49} Lit. \fbib{to strengthen the hand of}} the poor and needy. \v{50}In their arrogance, they committed detestable practices in my presence, so when I saw it, I removed them. \v{51}Samaria didn't commit half of your sins---you practiced more detestable deeds than they did! You've caused your sister to be more righteous than you, because of the detestable practices that you've committed. \v{52}So now, bear your own shame as you mediate for your sisters. The sins that you've committed are more detestable than theirs. That makes them more righteous than you. Indeed, be ashamed and bear your reproach, because you've made your sisters to be more righteous than you.''
\passage{A Change in Circumstances}

\v{53}``I'll bring them back from their captivity---that is, from the captivity of Sodom and her daughters, along with the captivity of Samaria and her daughters and the captivity of your captives among them. \v{54}But you'll continue to bear your own reproach and be humiliated for everything that you've done. You'll be a comfort to them. \v{55}Your sister Sodom and her daughters will be restored to their former status. Samaria and her daughters will be restored to their former status. Then you and your daughters will be restored to your former status.

\v{56}``When you were being so arrogant, you never once mentioned your sister Sodom \v{57}before your wickedness was revealed. Now you've become an object of derision to the inhabitants\fnote{\fbackref{16:57} Lit. \fbib{daughters}} of Aram and its neighbors, including the Philistines\fnote{\fbackref{16:57} Lit. \fbib{the daughters of the}}---all those around you who despise you. \v{58}You are to bear the punishment of your wickedness and detestable practices,'' declares the \divine{Lord}, \v{59}``since the Lord \divine{God} says, `I'll deal with you according to what you've done, when you despised your oath by breaking the covenant.

\v{60}```Meanwhile, as for me, I'll remember my covenant with you from when you were young, because I'll establish an eternal covenant with you. \v{61}Then you'll remember your behavior and be ashamed when you greet your sisters---your elder sister and your younger sister. I'll give them to you as daughters, but not on account of my covenant with you. \v{62}I'll establish my covenant with you, and then you'll know that I am the \divine{Lord}. \v{63}Then you will remember, be ashamed, and you won't open your mouth anymore due to humiliation when I will have made atonement for you for everything that you've done,' declares the Lord \divine{God}.''
\labelchapt{17}
\passage{The Parable of the Eagle}

\chapt{17}
\v{1}This message came to me from the \divine{Lord}: \v{2}``Son of Man, compose a riddle and relate a parable to Israel's house. \v{3}Tell them, `This is what the Lord \divine{God} says, ``A massive eagle with gigantic wings, long pinions, and full, multi-colored plumage came to Lebanon and took away the top of the cedar.\fnote{\fbackref{17:3} I.e. a genus of coniferous evergreen in the family \fbib{Pinaceae}; and so throughout the book} \v{4}He plucked off the top of its shoot, brought it to a land of merchants, and set it down in a city full of traders. \v{5}Then the eagle took a seed from the land and planted it in fertile ground. He planted it like a willow tree next to abundant waters. \v{6}It flourished and became a low, spreading vine. Its branches turned toward him, and its roots spread under him to become a vine that put out shoots and spread out its branches.

\v{7}`````All of a sudden, there was another eagle with gigantic wings and thick plumage. The vine stretched its roots hungrily toward him and spread its branches out to him in order to be watered on the terraces where it was planted. \v{8}It was transplanted into good soil\fnote{\fbackref{17:8} Or \fbib{ground}} near abundant water, and it produced branches and bore fruit, becoming a magnificent vine.''\,'

\v{9}``Tell them, `This is what the Lord \divine{God} says, ``Will it prosper? Won't he pull up its roots, and strip it bare so all its fresh foliage dries up? It won't be by great strength or by a great army that it will be uprooted. \v{10}Look! Because it's a transplanted vine, won't it wither when the east wind hits it? It will surely wither in the terraces where it had started to sprout.''\,'\,''
\passage{The Meaning of the Parable}

\v{11}This message came to me from the \divine{Lord}: \v{12}``Tell my\fnote{\fbackref{17:12} Lit. \fbib{the}} rebellious house, `Don't you know what these things mean? Look! The king of Babylon came to Jerusalem, captured her king and princes, and took them with him to Babylon. \v{13}Then he took one of the royal descendants, made a covenant with him, and put him under an oath of loyalty, taking the leaders of the land captive \v{14}in order to humiliate the kingdom so it wouldn't be able to return to power, but would still be able to continue as long as he keeps his covenant. \v{15}But he rebelled against the king of Babylon\fnote{\fbackref{17:15} Lit. \fbib{against him}} by sending his messengers to Egypt to obtain horses and a large army. Will he succeed? Or will the one who did this escape? Will he break the covenant, but still be delivered?'\,''
\passage{God will Punish the King}

\v{16}``As long as I live,'' declares the Lord \divine{God}, ``in Babylon, that place where the king has enthroned him, whose oath he despised so as to break his covenant, he'll die with him. \v{17}Pharaoh, with his massive army and large battalions won't protect him when mounds and siege walls are built to destroy many people.\fnote{\fbackref{17:17} Lit. \fbib{souls}} \v{18}He despised the oath he had made and broke the covenant. Look! Because he willingly submitted,\fnote{\fbackref{17:18} Lit. \fbib{He has given his hand}} yet he has done all these things, he won't escape.

\v{19}Therefore, this is what the Lord \divine{God} says, ``As long as I live, because he despised my oath and broke my covenant, he's going to suffer the consequences.\fnote{\fbackref{17:19} Lit. \fbib{covenant, I'll bring it upon his head}} \v{20}I'll spread my net over him so that he'll be caught in my snare. I'll bring him to Babylon and carry out my sentence there because of his treachery toward me. \v{21}The fugitives of his troops will die by the sword, and the survivors will be scattered to the four\fnote{\fbackref{17:21} Lit. \fbib{to all the}} winds. Then you'll know that I, the \divine{Lord}, have spoken.''
\passage{The Transplanted Vine}

\v{22}This is what the Lord \divine{God} says, ``I'm also going to take a shoot from the top of a cedar and plant it. I'll pluck off its delicate twigs and transplant it on a high and lofty mountain. \v{23}I'll transplant it on Israel's land, and it will grow branches, bear fruit, and become a majestic cedar. All sorts\fnote{\fbackref{17:23} Lit. \fbib{wing}} of birds will rest under it, and they'll settle down in the shade of its branches. \v{24}Then all the trees of the fields will know that I, the \divine{Lord}, bring down the lofty tree and exalt the lowly tree. I dry up the green\fnote{\fbackref{17:24} Or \fbib{fresh}} tree and cause the dry tree to bud. I the \divine{Lord} have spoken this, and I will fulfill it.''
\labelchapt{18}
\passage{The Outdated Proverb}

\chapt{18}
\v{1}This message came to me from the \divine{Lord}: \v{2}``Why do you cite this proverb when you talk about Israel's land: `The fathers eat sour grapes but it's their children's teeth that have become numb.' \v{3}As long as I live,'' declares the \divine{Lord}, ``you won't use this proverb about Israel anymore. \v{4}Look! Every living soul belongs to me---the father's as well as the son's.\fnote{\fbackref{18:4} Lit. \fbib{As the soul of the father, so the soul of the son belongs to me}.} So pay attention! The person who keeps on sinning is going to die.''
\passage{Standards of Righteous Behavior}

\v{5}``If a person is righteous, and practices what's lawful and right, \v{6}if he doesn't eat at mountain shrines, and doesn't look to the idols that have been erected in Israel's house, if he doesn't defile his neighbor's wife or approach a woman during her time of menstrual separation, \v{7}if he doesn't oppress anyone, but instead returns the debtor's security for his debt, if he doesn't rob anyone, but instead shares his food with the hungry and gives clothes to those who are naked, \v{8}if he doesn't lend with usury or exact interest, but instead refuses to participate in\fnote{\fbackref{18:8} Lit. \fbib{instead withdraws his hand from}} what is unjust, if he administers true justice between people,\fnote{\fbackref{18:8} Lit. \fbib{between man and man}} \v{9}if he lives his life\fnote{\fbackref{18:9} Lit. \fbib{he walks}} consistent with my statutes and keeps my ordinances by practicing what's true, then he's righteous and will certainly live,'' declares the Lord \divine{God}.
\passage{Standards of Unrighteous Behavior}

\v{10}``Now suppose that person produces a son who's violent, a murderer, and practices any of these things, \v{11}even though the father\fnote{\fbackref{18:11} Lit. \fbib{though he}} hasn't done any of these things. The son who eats at mountain shrines, defiles his neighbor's wife, \v{12}oppresses the afflicted and the poor, robs others, doesn't return security for a debt, looks to idols, does detestable things, \v{13}loans with usury, and exacts interest; will he live? He certainly will not! He has done all these detestable practices. He will certainly die, and his guilt will be his own fault.''\fnote{\fbackref{18:13} Lit. \fbib{his blood will be on him}}
\passage{Personal Accountability for Sin}

\v{14}``Now suppose that he produced a son who practiced all of his father's sins, but then that son\fnote{\fbackref{18:14} The Heb. lacks \fbib{that son}} began to fear me and stopped doing all of these things. \v{15}That is, suppose he doesn't eat at the mountain shrines, doesn't look to the idols of Israel's house, doesn't defile his neighbor's wife, \v{16}doesn't oppress anyone, doesn't take possession of a debtor's pledge, or doesn't steal, but instead shares his food with the hungry, gives clothes to those who are naked, \v{17}doesn't refuse to help the afflicted, or refuses to loan with usury or exact interest, but instead follows my ordinances and lives his life consistent with my statutes. He won't die because of his father's sin, will he? No! He'll certainly live. \v{18}As for his father, watch out! If he wrongfully oppressed or robbed his brother and did what wasn't good among his people, he'll die because of\fnote{\fbackref{18:18} Lit. \fbib{die in}} his own sin.''
\passage{The Person who Sins will Die}

\v{19}``Yet you keep asking, `Why wouldn't the son bear the punishment of his father's sin?' Because the son has done what was lawful and right, and has kept all my statutes and obeyed them, he's certainly going to live. \v{20}The soul who sins dies. The son won't bear the punishment of his father's sin and the father won't bear the punishment of his son's sin. The righteous deeds of that righteous person will be attributed to him, while the wicked deeds of the wicked person will be charged against him. \v{21}But if the wicked person turns from all his sins, which he did and keeps my statutes, then he'll live. He won't die. \v{22}None of the transgressions that he had committed will be held\fnote{\fbackref{18:22} Lit. \fbib{remembered}} against him. Because of the righteous deeds that he had done, he'll live.

\v{23}``I don't take delight in the death of the wicked, do I?'' asks the Lord \divine{God}. ``Shouldn't I rather delight\fnote{\fbackref{18:23} The Heb. lacks \fbib{shouldn't I rather delight}} when he turns from his wicked ways and lives? \v{24}But when the righteous person abandons his righteous deeds and commits evil, detestable practices, as wicked people do, he won't live, will he? None of the righteous acts that he had done will be remembered. He'll die in his treacherous unfaithfulness and sins that he had committed.''
\passage{Accusing God of Unrighteousness}

\v{25}``Yet you keep saying, `The \divine{Lord} isn't being consistent with his standards.' Pay attention, you house of Israel: Is my behavior really inconsistent with my standards? Isn't it your behavior that isn't just?

\v{26}``When a righteous person turns from his righteous deeds and does evil, he'll die because of that evil. He'll die because of his unrighteous acts that he committed. \v{27}When a wicked person quits\fnote{\fbackref{18:27} Or \fbib{abandons}} his wicked behavior\fnote{\fbackref{18:27} Lit. \fbib{ways that he had committed}} and does what's just and right, he'll be enabled to live.\fnote{\fbackref{18:27} Lit. \fbib{he makes his soul come alive}} \v{28}Because he reconsidered his transgression and turned away from everything that he had been doing, he'll certainly live and not die. \v{29}Yet Israel's house keeps saying, `The \divine{Lord} isn't being consistent with his standards.' Is it my behavior that's inconsistent with my standards?\fnote{\fbackref{18:29} The Heb. has \fbib{adjusted to the standard}} Is it not your behavior that's inconsistent with my standards?''\fnote{\fbackref{18:29} The Heb. has \fbib{adjusted to the standard}}
\passage{A Command to Repent}

\v{30}``Therefore, Israel, I'm going to judge you according to the behavior of each and every one of you,'' declares the Lord \divine{God}. ``So repent and turn from all your sins so that sin won't keep on being a stumbling block for you. \v{31}Stop your transgressing---the deeds by which you've rebelled---and then make for yourselves a new heart and a new spirit. Why should you die, you house of Israel? \v{32}I don't take pleasure in the death of anyone who dies,'' declares the \divine{Lord}. ``So repent, so you may live!''
\labelchapt{19}
\passage{A Prophecy against Israel's Nobles}

\chapt{19}
\v{1}``Now as for you, publish\fnote{\fbackref{19:1} Lit. \fbib{sound forth}} this mourning psalm about Israel's leaders. \v{2}Tell them:

\begin{poetry}
\poeml `What a lioness your mother was among lions! \\
\poemll    She reared her cubs in the midst of fierce young males. \\
\poeml \v{3}She raised one cub in particular, \\
\poemll    teaching that fierce lion to become a hunter-prowler--- \\
\poemlll       to eat human beings. \\
\poeml \v{4}The nations heard about him. \\
\poemll    He had become caught in their trap.\fnote{\fbackref{19:4} Lit. \fbib{pit}} \\
\poeml They brought him with hooks \\
\poemll    to the land of Egypt. \\
\poeml \v{5}When she learned that her plans had been frustrated \\
\poemll    and that her hopes were dashed, \\
\poeml she took another of her cubs \\
\poemll    and turned him into a fierce lion. \\
\poeml \v{6}He prowled around among the lions, \\
\poemll    became a strong, young lion, \\
\poeml and learned to become a hunter-prowler--- \\
\poemll    to eat human beings. \\
\poeml \v{7}He raped\fnote{\fbackref{19:7} Lit. \fbib{knew}} the women, \\
\poemll    devastating their towns. \\
\poeml The land was made desolate, \\
\poemll    and all the while the land was filled \\
\poemlll       with the sound of his roaring. \\
\poeml \v{8}The surrounding nations attacked. \\
\poemll    They tossed their net over him, \\
\poemlll       and he was caught in their trap.\fnote{\fbackref{19:8} Lit. \fbib{pit}} \\
\poeml \v{9}They imprisoned him in a cage with hooks \\
\poemll    and brought him to the king of Babel. \\
\poeml Then they placed him in their dungeon \\
\poemll    where his voice would no longer be heard \\
\poemlll       on the mountains of Israel. \\
\poeml \v{10}`Your mother was like a vine \\
\poemll    entwining a pomegranate,\fnote{\fbackref{19:10} So LXX; MT reads \fbib{in your blood}, misreading the Heb. \fbib{a pomegranate}} \\
\poeml planted by water, full of fruit, \\
\poemll    and full of branches \\
\poemlll       because it had been watered generously. \\
\poeml \v{11}Strong were its boughs, \\
\poemll    suitable for use in the scepter of a ruler. \\
\poeml It reached to the clouds, \\
\poemll    noticeable because of its height \\
\poemlll       and its abundant branches. \\
\poeml \v{12}Yet in anger it was uprooted \\
\poemll    and cast down to the earth. \\
\poeml An east wind desiccated its fruit; \\
\poemll    its strong branches broke off and withered, \\
\poemlll       and a fire consumed them. \\
\poeml \v{13}Now it is planted in the desert, \\
\poemll    in a dry and thirsty land! \\
\poeml \v{14}Fire had burned through its branches, \\
\poemll    consuming its shoots and fruits. \\
\poeml No strong branches remain in it, \\
\poemll    and there is no scepter to rule!'
\end{poetry}

``This is a lamentation, and it is to be used in mourning.''
\labelchapt{20}
\passage{A Prophecy against Israel's Elders}

\chapt{20}
\v{1}On the seventh year, on the tenth day\fnote{\fbackref{20:1} The Heb. lacks \fbib{day}} of the fifth month, men came from the elders of Israel to seek the \divine{Lord}. They sat down in front of me.

\v{2}``Son of Man,'' the \divine{Lord} told me, \v{3}``Tell the elders of Israel, `This is what the Lord \divine{God} asks, ``Did you come to inquire of me? As long as I live, I won't let myself be sought by you,'' declares the Lord \divine{God}.'

\v{4}``Will you judge them? Son of Man, will you indeed judge them? Teach them about the detestable things that their ancestors did. \v{5}Tell them, `This is what the Lord \divine{God} says, ``The day I chose Israel, when I made my commitment\fnote{\fbackref{20:5} Lit. \fbib{When I lifted my hand}, and so throughout.} to the descendants of Jacob's house, I revealed myself to them in the land of Egypt and I made my promise to them with the words, `I am the \divine{Lord} your God.' \v{6}That day I promised to bring them out of the land of Egypt to the land that I had explored for them---a land flowing with milk and honey. It's the most beautiful of all lands. \v{7}Then I told them, `Each of you are to abandon your detestable practices.\fnote{\fbackref{20:7} Lit. \fbib{practices before your eyes}} You are not to defile yourselves with Egypt's idols. I am the \divine{Lord} your God.'\,''\,'\,''
\passage{A Brief History of Israel's Rebellion}

\v{8}``But they rebelled against me and weren't willing to obey me. None of them abandoned their detestable practices\fnote{\fbackref{20:8} Lit. \fbib{practices before their eyes}} or their Egyptian idols. So I said, `I'll pour out my anger on them, extending my fury in the middle of the land of Egypt.' \v{9}I did this so my reputation\fnote{\fbackref{20:9} Lit. \fbib{name}} might not be tarnished among the nations where they were living, among whom I made myself known in their presence when I brought them out of the land of Egypt. \v{10}I brought them out of the land of Egypt to bring them to the wilderness \v{11}where I gave them my statutes and revealed my ordinances to them, which if a person\fnote{\fbackref{20:11} Lit. \fbib{man}} observes, he'll live by them. \v{12}Also, I instituted\fnote{\fbackref{20:12} Lit. \fbib{gave}} my Sabbath for them as a sign between me and them, so they would know that I am the \divine{Lord}, who has set them apart.''
\passage{Israel Rebels in the Wilderness}

\v{13}``But the house of Israel rebelled against me in the wilderness. They didn't live by\fnote{\fbackref{20:13} Lit. \fbib{walk in}} my statutes. They despised my ordinances, which if a person observes, he'll live by them. They greatly profaned my Sabbaths. So I said I would pour out my anger on them and bring them to an end in the wilderness. \v{14}I did this so my reputation wouldn't be tarnished among the nations in whose presence I had brought them out.

\v{15}``Moreover, I solemnly swore to them in the wilderness that I wouldn't bring them to the land that I had given them---a land flowing with milk and honey, the most beautiful of all lands---\v{16}because they kept on rejecting my ordinances. They didn't live life consistent with my statutes, they profaned my Sabbaths, and their hearts followed\fnote{\fbackref{20:16} Lit. \fbib{walked}} their idols. \v{17}Even then, I\fnote{\fbackref{20:17} Lit. \fbib{my eyes}} looked on them with compassion and didn't completely destroy them in the wilderness. \v{18}I told their children in the wilderness, `You are not to follow the statutes of your ancestors, observe their ordinances, or be defiled by their idols. \v{19}I am the \divine{Lord} your God. You are to follow my statutes, observe my ordinances, and keep them. \v{20}You are to make my Sabbaths holy, and you are to let them serve as a sign between you and me, so that you may know that I am the \divine{Lord} your God.'\,''
\passage{More of Israel's Rebellion}

\v{21}``But they rebelled against me. They didn't live according to my statutes, observe my ordinances, or practice them, by which a person will live. They also kept profaning my Sabbaths. So I said that I was going to pour out my anger on them and in my anger I'm going to bring about a complete end to them in the wilderness. \v{22}But I withdrew my decision\fnote{\fbackref{20:22} Lit. \fbib{hand}} so my reputation wouldn't be tarnished among the nations before whose eyes I brought them out.

\v{23}``Furthermore, I solemnly swore in the wilderness to disperse them among the nations and scatter them to other\fnote{\fbackref{20:23} The Heb. lacks \fbib{other}} lands \v{24}because they didn't observe my ordinances. Instead, they rejected my statutes, profaned my Sabbaths, and worshipped\fnote{\fbackref{20:24} Lit. \fbib{Their eyes went after}} their ancestors' idols. \v{25}So I gave them statutes that weren't good and ordinances by which they could not live. \v{26}I made them unclean because of their offerings, so they made all their firstborn\fnote{\fbackref{20:26} Lit. \fbib{their first to open the womb}} to pass through the fire, so that I could make them astonished. Then they'll know that I am the \divine{Lord}.''
\passage{The Blasphemy of Israel's Ancestors}

\v{27}``Therefore, Son of Man, you are to speak to the children of Israel and tell them, `This is what the Lord \divine{God} says: ``Your ancestors also blasphemed me in their treacherous behavior against me. \v{28}I brought them to the land that I had promised to give them. But whenever they saw any high hill and or any leafy tree, they slaughtered their sacrifices there and presented their offerings that provoked my anger. There they presented their pleasing aromas and poured out their drink offering. \v{29}So I asked them, `What is the high place to which you're going?' That's why the name of the place has been called Bamah\fnote{\fbackref{20:29} The Heb. name \fbib{Bamah} means \fbib{What is?}} to this day.''\,'

\v{30}``Therefore you are to say to Israel's house, `This is what the Lord \divine{God} says: ``Will you defile yourselves like your ancestors did by acting as a prostitute, consistent with their horrible deeds? \v{31}When you present your gifts and make your sons pass through the fire, you continue to defile yourselves with your idols to this day. Should I be inquired of by you, you house of Israel? As I live,'' declares the \divine{Lord}, ``I certainly won't be inquired of by you.'' \v{32}The thing that you're imagining\fnote{\fbackref{20:32} Lit. \fbib{that is coming upon your spirits}} is never going to happen, since you're thinking, ``We'll be like the nations, like the clans of other\fnote{\fbackref{20:32} The Heb. lacks \fbib{other}} lands who serve gods made from wood and stone.''\,'\,''
\passage{The Coming Discipline of Israel}

\v{33}``As I live,'' declares the Lord \divine{God}, ``with my powerful hand and outstretched arm, along with my wrath poured out, I'll reign as king over you. \v{34}I'll bring you out from the peoples and gather you from the lands where you were scattered. With a powerful hand, with an outstretched arm, and with wrath poured out, \v{35}I'll bring you into the wilderness of the nations. I'll judge you right there, face to face. \v{36}Just as I judged your ancestors in the wilderness in the land of Egypt, so I'll judge you,'' declares the \divine{Lord}. \v{37}``I'll cause you to pass under the rod until I will have brought you into the bond of the covenant. \v{38}I'll eliminate the rebels among you, along with those who are transgressing against me. I'll bring them out of the land where you've lived, but they won't be able to enter the land of Israel. Then you'll know that I am the \divine{Lord}.''
\passage{The Coming Regathering of Israel}

\v{39}And now, you house of Israel, this is what the Lord \divine{God} says, ``Go ahead and serve your idols, both now and later, but later you'll listen to me, and you won't profane my sacred name again by your offerings and idols. \v{40}For on my holy mountain, on Israel's high mountains,'' declares the Lord \divine{God}, ``the whole of Israel's house---all of it---will serve me there in the land. I'll accept them there. And there I'll demand your offerings, the first fruits of your portions of all your sacred things.

\v{41}``When I will have brought you from among the people and have gathered you from the lands where you were scattered, I'll accept you as a pleasing aroma. I'll reveal my holiness among you, and the entire world will see it. \v{42}Then you'll know that I, the \divine{Lord}, brought you to the land of Israel, to the land that I promised to give to your ancestors. \v{43}You'll remember all your practices and evil actions by which you've become defiled. You'll loathe yourselves\fnote{\fbackref{20:43} Lit. \fbib{your souls in your own sight}} because of all the evil things you've done. \v{44}Then you'll know that I am the \divine{Lord}, when I will have dealt with you for the benefit of my own reputation and not according to your evil attitudes or corrupt practices, you house of Israel,'' declares the Lord \divine{God}.
\passage{Coming Judgment on the South}

\v{45}\fnote{\fbackref{20:45} This v. is 21:1 1n MT}This message came to me from the \divine{Lord}: \v{46}``Son of Man, turn to the south and oppose it, talking toward the south. \v{47}Prophesy against the forest of the Negev,\fnote{\fbackref{20:47} I.e. southern regions of the Sinai peninsula; cf. Josh 10:40} `Listen to this message from the \divine{Lord}. This is what the Lord \divine{God} says: ``Look out! I'm about to ignite a fire and set it against you. It will devour every tree---whether green or dry---that lives in you. This powerful flame will not be extinguishable, and the entire surface from south to north will be scorched by it. \v{48}Then everyone\fnote{\fbackref{20:48} Lit. \fbib{Then all flesh}} will see that I, the \divine{Lord}, have kindled it, because it won't be extinguished.''\,'\,''
\passage{Ezekiel's Complaint to God}

\v{49}Then I said, ``O Lord \divine{God}! They're saying about me, `Isn't he one to propound parables?'\,''
\labelchapt{21}
\passage{A Prophecy against Jerusalem}

\chapt{21}
\v{1}\fnote{\fbackref{21:1} This v. is 21:6 in MT}This message came to me from the \divine{Lord}: \v{2}``Son of Man, look toward Jerusalem, preach\fnote{\fbackref{21:2} Lit. \fbib{Drop a word}} against its sanctuaries, and prophesy against Israel's land. \v{3}Declare to Israel, `This is what the \divine{Lord} says: ``Watch out! I'm against you! I'm going to unsheathe my sword to kill both the righteous and the wicked among you. \v{4}Since I'm going to kill both the righteous and the wicked among you, I'll be unsheathing my sword against everyone from south to north. \v{5}Then everyone will know that I am the \divine{Lord}, who unsheathed my sword, and who won't have to unsheathe it again.''\,'

\v{6}``And now, Son of Man, you are to start groaning until you're sick to your stomach.\fnote{\fbackref{21:6} Lit. \fbib{until your loins break}} You are to groan bitterly right in front of them.\fnote{\fbackref{21:6} Lit. \fbib{bitterly before their eyes}} \v{7}When they'll ask you, `Why are you groaning?' you are to say, `Because of the news that just arrived. Every heart will melt with fear, every hand will grow limp, every spirit will grow faint, and every knee will glisten with sweat.' Look! It has come and it will be fulfilled,'' declares the Lord \divine{God}.
\passage{God's Sword and Scepter}

\v{8}This message came to me from the \divine{Lord}: \v{9}``Son of Man, prophesy and say, `This is what the Lord \divine{God} says:

\begin{poetry}
\poeml `A sword! \\
\poemll    A sword is being sharpened. \\
\poemlll       It's also being polished. \\
\poeml \v{10}It's being sharpened for slaughter, \\
\poemll    and being polished to gleam like lightning.' \\
\poeml ``We shouldn't be rejoicing, should we, \\
\poemll    while my Son's scepter, the sword, \\
\poemlll       is despising\fnote{\fbackref{21:10} The verb \fbib{despising} requires the Heb. antecedent \fbib{the sword}, which The Heb. lacks} every tree?\fnote{\fbackref{21:10} I.e. every living human being in Israel} \\
\poeml \v{11}It's intended to be polished \\
\poemll    so it can be grasped in the hand. \\
\poeml The sword is sharpened. \\
\poemll    It's polished for placement \\
\poemlll       into the hand of the executioner.'' \\
\poeml \v{12}`Cry and wail, you Son of Man! \\
\poemll    It's headed against my people. \\
\poeml It's also against all the princes of Israel, \\
\poemll    who have been tossed to the sword, \\
\poemlll       along with my people. \\
\poeml So it's time to grieve like you mean it,\fnote{\fbackref{21:12} Lit. \fbib{So strike your thigh}} \\
\poeml \v{13}because testing is sure to come.
\end{poetry}

`In putting the sword to the test along with the scepter, it won't keep on rejecting, will it?' declares the Lord \divine{God}.''
\passage{A Double and Triple Judgment}

\begin{poetry}
\poeml \v{14}``Now, Son of Man, prophesy \\
\poemll    as you strike your hands together. \\
\poeml Let the sword that fatally wounds be doubled and tripled. \\
\poemll    That great, slaughtering sword closes in on them. \\
\poeml I've set in place a slaughtering sword \\
\poemll    at the entrance to all their gates, \\
\poeml \v{15}so that their hearts melt \\
\poemll    and the number of those who stumble increase. \\
\poeml I've set in place a slaughtering sword \\
\poemll    at the entrance to all their gates. \\
\poeml Oh, no! It's made like lightning. \\
\poemll    It's drawn to slaughter. \\
\poeml \v{16}Be sharp! \\
\poemll    Attack on the right, \\
\poeml or parry to your left, \\
\poemll    wherever you point your sword.\fnote{\fbackref{21:16} Lit. \fbib{face}} \\
\poeml \v{17}I will also clap my hands; \\
\poemll    then my anger will subside.\fnote{\fbackref{21:17} Lit. \fbib{rest}} \\
\poemlll       I, the \divine{Lord} have spoken it.''
\end{poetry}
\passage{Two Pathways to Invasion}

\v{18}This message came to me from the \divine{Lord}: \v{19}``Now, Son of Man, demarcate two pathways by which the sword of Babylon's king will arrive in the land. Both pathways will lead from a single land.

``Make a sign,\fnote{\fbackref{21:19} Lit. \fbib{hand}} carving it out and installing it at the junction on the way to the city. \v{20}Set it to point one way for bringing the sword against Rabbah, the descendants of Ammon, and the other way against Judah and fortified Jerusalem.

\v{21}``Meanwhile, Babylon's king is standing at the fork of the road,\fnote{\fbackref{21:21} Lit. \fbib{at the point of departure}} where he can head in either of two directions, and that's where he is practicing divination. Shaking his arrows, he's asking questions of his teraphim while he examines livers. \v{22}On his right hand he is divining against Jerusalem, preparing to set up battering rams, preparing\fnote{\fbackref{21:22} Lit. \fbib{rams, to open the mouth}} for the slaughter, getting ready to sound the alarm for battle,\fnote{\fbackref{21:22} Lit. \fbib{to shout}} setting the battering rams in place at the gates, building siege mounds, and erecting a siege wall. \v{23}In their view, it will seem to be a false prophecy, but because they swore allegiance, he'll make them remember their guilt as he takes them captive.''
\passage{Imminent Invasion}

\v{24}Therefore this is what the Lord \divine{God} says: ``Because you remembered your sins when your transgressions were uncovered, your sins are visibly evident in everything you've done. And since you've remembered them, you'll be taken captive.

\begin{poetry}
\poeml \v{25}``So now, you dishonored and wicked prince of Israel, \\
\poemll    whose day has come \\
\poemlll       in this time of final punishment,
\end{poetry}

\begin{poetry}
\poeml \v{26}This is what the Lord \divine{God} says: \\
\poeml `Remove your turban! \\
\poemll    Take off your crown! \\
\poeml Things aren't going to remain \\
\poemll    as they used to be. \\
\poeml What is lowly will be exalted, \\
\poemll    and what is lofty will be brought low. \\
\poeml \v{27}A ruin! A ruin! \\
\poemll    I'm bringing about ruin!' \\
\poeml But this also will not happen \\
\poemll    until he who has authority over it arrives, \\
\poemlll       because I'll give it to him.''
\end{poetry}
\passage{A Rebuke to Ammon}

\v{28}And now Son of Man, prophesy to the Ammonites that this is what the Lord \divine{God} says to the Ammonites about their approaching humiliation:

\begin{poetry}
\poeml ``A sword! A sword is being drawn for slaughter. \\
\poemll    It's polished to flash like lightning. \\
\poeml \v{29}When they see empty visions about you \\
\poemll    while they're divining lies for you, \\
\poeml to lay you on the necks of the wicked, \\
\poemll    who are fatally wounded, \\
\poeml whose days have come, \\
\poemll    their time for punishment. \\
\poeml \v{30}Return it to its scabbard.
\end{poetry}

\begin{poetry}
\poeml ``At the place where you were formed, \\
\poemll    in the land of your origin, \\
\poemlll       there is where I'll judge you. \\
\poeml \v{31}I'm going to pour out my indignation all over you. \\
\poemll    I'll blow my fierce wrath against you \\
\poeml and deliver you into the control of brutal men \\
\poemll    who are skilled at destruction. \\
\poeml \v{32}You'll be fuel\fnote{\fbackref{21:32} Lit. \fbib{food}} for the conflagration. \\
\poemll    Your blood will be spilled\fnote{\fbackref{21:32} The Heb. lacks \fbib{spilled}} throughout the land. \\
\poeml You won't be remembered anymore, \\
\poemll    now that I, the \divine{Lord}, have spoken.''
\end{poetry}
\labelchapt{22}
\passage{A Prophecy against Jerusalem}

\chapt{22}
\v{1}This message came to me from the \divine{Lord}: \v{2}``Now, Son of Man, will you truly judge that blood-stained city? Then make her aware of all of her detestable practices.

\v{3}``You are to say, `This is what the Lord \divine{God} says: ``The city keeps on shedding blood within her, hastening her time to be judged. She crafts idols that defile her.

\v{4}`````You're guilty because of the blood that you've shed. You were defiled by the idols that you've made. You've caused your judgment day to draw near and have even come to the end of\fnote{\fbackref{22:4} The Heb. lacks \fbib{the end of}} your life. Therefore, I've made you an object of derision among the nations and to other\fnote{\fbackref{22:4} The Heb. lacks \fbib{other}} lands. \v{5}Those who are both near and far away from you will scoff at you, because your reputation will be infamous and full of turmoil.

\v{6}`````Watch out! Each of the princes of Israel has misused his authority to shed blood. \v{7}They've treated mothers and fathers among you with contempt. They've oppressed the foreigner among you. They've maltreated the orphan and the widow among you.

\v{8}`````You have despised my sacred things and profaned my Sabbaths. \v{9}Slanderous men live among you, intent on shedding blood. They've eaten at the top of mountain shrines. They've crafted plans to do evil things among you. \v{10}They've revealed the nakedness of their father in your midst. They've humiliated those among you who were unclean due to their impurity. \v{11}One of you commits detestable practices with his neighbor's wife. Another sexually defiles his daughter-in-law. Another humiliates his sister, his own father's daughter. \v{12}They take bribes among you to shed blood. You've taken usury and exacted interest. You've gained control over your neighbor through extortion. And you've forgotten me,'' declares the Lord \divine{God}.

\v{13}``Watch out! I'm vehemently angry about\fnote{\fbackref{22:13} Lit. \fbib{I'm going to strike my hands}} the unjust gains that you've earned, and about the blood that has been shed among you. \v{14}Can your heart stand up to this? Can your hands remain strong when I deal with you? I, the \divine{Lord}, have spoken and will fulfill this. \v{15}I'm going to disperse you among the nations and scatter you to other lands. I'm going to put an end to your uncleanness. \v{16}When you've been defiled in the sight of the nations, then you'll know that I am the \divine{Lord}.''\,'\,''
\passage{God's Purging Fire}

\v{17}This message came to me from the \divine{Lord}: \v{18}``Son of Man, Israel has become like dross to me. All of them are like remnants of bronze, tin, iron, and lead in a furnace---the dross left over from smelting silver. \v{19}Therefore this is what the Lord \divine{God} says, `Because all of you have become dross, watch out! I'm going to gather all of you at the center of Jerusalem, \v{20}just like a smelter gathers all the silver, bronze, lead, and tin to the center of a furnace and injects fire in order to melt it, I'm going to gather you in my anger and rage, make you settle down---and then I'm going to melt you down. \v{21}Indeed, I'm going to gather you together and exhale the fire of my fury, and then you'll be melted from the inside out \v{22}like melting silver at the center of a furnace. When you've been melted from the center out, then you'll know that I am the \divine{Lord}. I'll pour out my anger on you.'\,''
\passage{God Rebukes Prophets and Priests}

\v{23}This message came to me from the \divine{Lord}: \v{24}``Son of Man, tell her,\fnote{\fbackref{22:24} I.e. Israel personified as a woman} `You're a land that hasn't been purified, one that hasn't been rained on in the day of indignation. \v{25}There's a conspiracy of prophets within her, and like a roaring lion tearing its prey, they've devoured people, and confiscated treasures, and taken precious things. They've added to the population of widows within her. \v{26}Her priests have violated my Law and profaned my sacred things. They didn't differentiate between what's sacred and what's common. They didn't instruct others to discern clean from unclean things. They refused to keep\fnote{\fbackref{22:26} Lit. \fbib{refused their eyes from}} my Sabbaths.

```I'm constantly being profaned among them. \v{27}Her princes within her are like wolves tearing their prey apart. They shed blood, destroying souls, and make unjust gain.

\v{28}```Her prophets whitewashed all of these things through false visions and lying divinations. They kept on saying, ``This is what the Lord \divine{God} says{\ldots}'', when the \divine{Lord} has not spoken. \v{29}The people of the land were vigorously oppressive and took possession of plunder by violence. They've afflicted the poor and the needy and unjustly treated the foreigner. \v{30}I sought for a man among them to build the wall and stand in the breach in my presence on behalf of the land so that it won't be destroyed, but I found no one, \v{31}so I poured my indignation over them. With my fierce anger, I've consumed them. I brought the consequences of\fnote{\fbackref{22:31} The Heb. lacks \fbib{the consequences of}} their behavior upon them,'\fnote{\fbackref{22:31} Lit. \fbib{upon their heads}} declares the Lord \divine{God}.''
\labelchapt{23}
\passage{Introducing Oholah and Oholibah}

\chapt{23}
\v{1}This message came to me from the \divine{Lord}: \v{2}``Son of Man, here are two sisters who are daughters from the same mother. \v{3}They committed sexual immorality in Egypt. They did this\fnote{\fbackref{23:3} Lit. \fbib{They committed sexual immorality}} in their youth. There, their breasts were caressed. Their virgin breasts were fondled. \v{4}The older one was named Oholah\fnote{\fbackref{23:4} The Heb. name \fbib{Oholah} means \fbib{she worships at a tent shrine}} and her sister was named Oholibah.\fnote{\fbackref{23:4} The Heb. name \fbib{Oholibah} means \fbib{she is a tent shrine}} They belonged to me and gave birth to sons and daughters. Now as to their real identities, Oholah refers to Samaria and Oholibah to Jerusalem.''
\passage{The Sins of Samaria}

\v{5}``Oholah committed sexual immorality while she belonged to me. She lusted for Assyria's warriors, \v{6}who were clothed in blue---including governors and commanders. All of them were desirable young men---horsemen mounted on horses. \v{7}She bestowed her sexual favors\fnote{\fbackref{23:7} Lit. \fbib{her sexual immorality}} on them---all of them, the best of the Assyrians---and with whomever she lusted for.

``She defiled herself with all their idols. \v{8}She never abandoned the immorality that she practiced in Egypt during her youth, where they laid down with her and fondled her virgin breasts, lavishing her with all kinds of favors. \v{9}Therefore, I turned her over to the control\fnote{\fbackref{23:9} Lit. \fbib{hands}} of her lovers, that is, into the control\fnote{\fbackref{23:9} Lit. \fbib{hands}} of the Assyrians for whom she lusted. \v{10}They stripped her naked, took away her sons and daughters, and executed her with a sword. She became an object of ridicule\fnote{\fbackref{23:10} Lit. \fbib{became a name}} among other nations\fnote{\fbackref{23:10} Lit. \fbib{among the women}} when they punished her.''
\passage{The Sins of Jerusalem}

\v{11}``Her sister Oholibah saw this, but she was more corrupt in her lust and sexual immorality than her sister had been in her own sexual immorality. \v{12}She lusted after the Assyrians---governors, commanders, warriors clothed in gorgeous attire, cavalry mounted on their horses---all of them desirable young men. \v{13}I saw that she was defiled, because the two of them both were on the same\fnote{\fbackref{23:13} Lit. \fbib{one}} path.

\v{14}``She became even more sexually immoral when she saw the images of the Chaldean men who had been carved in red on their walls. \v{15}Girded with waistbands around their loins, with flowing turbans on their heads, all of them looked like chariot officers, similar to the Babylonians from Chaldea, where they had been born.

\v{16}``She lusted after them when she saw them, so she sent messengers to summon them from Chaldea. \v{17}The Babylonians came to her love nest\fnote{\fbackref{23:17} Lit. \fbib{best}} and defiled her with their sexual immorality. As a result, she was defiled by them. Even so, she turned away from them in disgust. \v{18}She displayed her immorality publicly and stripped herself naked, so I turned away in disgust from her, just as I had turned away in disgust from her sister.

\v{19}``Nevertheless, she became even more sexually immoral, even reminiscing about when she was young, when she kept on practicing sexual immorality in the land of Egypt. \v{20}She lusted after her paramours, whose genitals are\fnote{\fbackref{23:20} Lit. \fbib{whose flesh is}} like those of donkeys, and whose emissions are like those of horses. \v{21}Think about the wickedness that you practiced when you were young, when the Egyptians fondled your breasts, the breasts of your youth.''
\passage{God's Rebuke to Jerusalem}

\v{22}``Therefore, Oholibah, this is what the Lord \divine{God} says: `Look! I'm about to stir up your lovers against you, the ones from whom you've turned away in disgust. I'm going to bring them against you from every direction---\v{23}the Babylonians, all the Chaldeans, Pekod, Shoa, Koa, and all of the Assyrians with them. They're all desirable young men, governors, commanders, chariot officers, and famous men, all of them mounted on horses.

\v{24}```They'll invade you with weapons, chariots, wagons, and a vast army. They'll set themselves in place to attack you from every direction with large shields, small shields, and helmets. I'll turn over judgment to them, and they'll punish you according to their own standards.\fnote{\fbackref{23:24} Lit. \fbib{punishment}} \v{25}I'll expend my jealousy on you so they'll deal with you in anger. They'll cut off your noses and your ears. Your survivors will die violently.\fnote{\fbackref{23:25} Lit. \fbib{die by the sword}} They'll take your sons and daughters away from you, but your survivors will be incinerated. \v{26}They'll strip off your clothes and confiscate your jewelry.\fnote{\fbackref{23:26} Lit. \fbib{your articles of beauty}} \v{27}That's how I'll put an end to your obscene conduct and sexual immorality that you kept on practicing since the day you left\fnote{\fbackref{23:27} Lit. \fbib{practicing from}} the land of Egypt so that you won't look in Egypt's direction or even remember it anymore.'

\v{28}``This is what the Lord \divine{God} says, `I'm about to turn you over to the control\fnote{\fbackref{23:28} Lit. \fbib{hands}} of those you hate, to the control of those from whom you turned away in disgust. \v{29}They'll deal with you with hatred. They'll take away your productivity, leaving you naked and defenseless, so that the nakedness of your sexual immorality will be uncovered---your licentious sexual immorality. \v{30}These things will happen to you because of your sexual immorality that was patterned after what the nations do. You've been defiled by their idols. \v{31}You took the path of your sister, so I'll place her cup in your hand.'

\v{32}``This is what the Lord \divine{God} says: `You'll drink from your sister's cup, which is both large and deep. You'll become a laughing stock and an object of derision, since the cup is so full! \v{33}You'll be filled with drunkenness and grief. The cup that belongs to your sister Samaria is filled with horror and devastation, \v{34}but you'll drink from it and drain it completely. As for the vessel, you'll break it to pieces and you'll tear at your breasts, for I've spoken,' declares the Lord \divine{God}.

\v{35}``Therefore this is what the Lord \divine{God} says: `Because you abandoned me and threw me behind your back, you will bear the consequences of your obscene conduct and sexual immorality.'\,''
\passage{What Israel and Samaria Did}

\v{36}Then the \divine{Lord} spoke to me. ``Son of Man, speak out in judgment of both Oholah and Oholibah. Make their detestable practices widely known, \v{37}because they've committed adultery, and blood covers their hands. They've also committed adultery with their idols, making their sons born to me to pass through the fire\fnote{\fbackref{23:37} The Heb. lacks \fbib{fire}} as an offering to\fnote{\fbackref{23:37} Lit. \fbib{as food for}} them.

\v{38}``They've also done this to me: They defiled my sanctuary and profaned my Sabbaths, all at the same time!\fnote{\fbackref{23:38} The Heb. has \fbib{day}} \v{39}When they killed their sons as offerings to\fnote{\fbackref{23:39} Lit. \fbib{sons for}} their idols, they brought them to my sanctuary and defiled it.\fnote{\fbackref{23:39} I.e. with their corpses} Look what they've done with my Temple!

\v{40}``In addition, they sent messengers for men to come from afar. When they arrived, you bathed yourself for them, painted your eyes, adorned yourself with jewelry, \v{41}then sat down on an elegant bed. A table was arranged in front of it, on which you set out my incense and oil. \v{42}The sound of a carefree multitude accompanied her. Men from a multitude of peoples were coming---including Sabeans\fnote{\fbackref{23:42} Or \fbib{drunkards}} from the wilderness, adorned\fnote{\fbackref{23:42} Lit. \fbib{they put}} with bracelets on their hands and beautiful crowns on their heads.

\v{43}``After she had worn herself out by her adulterous behavior, I asked her, `Will they continue with their sexual immorality and with their prostitution?' \v{44}They've gone to her, like men do, to have sex with a prostitute. They\fnote{\fbackref{23:44} Lit. \fbib{He}} had sex with Oholah and Oholibah, those licentious women. \v{45}Righteous men will judge them with punishments fit for adulterers and for those who shed blood, because they're adulterers with blood on their hands.''
\passage{The Coming Invasion}

\v{46}This is what the Lord \divine{God} says: ``Bring an army\fnote{\fbackref{23:46} Lit. \fbib{company}} against them and deliver them over to terror and plunder. \v{47}Then the army will stone them with stones and cut them to pieces with their swords. They'll kill their sons and daughters and incinerate their houses. \v{48}I'll cause obscene conduct to stop throughout the land, because all the women will be admonished not to practice their obscene conduct. \v{49}You'll receive the consequences for your obscene conduct and bear the punishment for your sins of idolatry. Then you'll know that I am the Lord \divine{God}.''
\labelchapt{24}
\passage{God Brews His Judgment}

\chapt{24}
\v{1}In the ninth year, in the tenth month, and on the tenth day of the month, this message came to me from the \divine{Lord}:

\v{2}``Son of Man, write down the name of this day, this very day. The king of Babylon has laid siege to Jerusalem on this very day. \v{3}So compose a parable for the rebellious house. Tell them, `This is what the Lord God says:

\begin{poetry}
\poeml ``Prepare your pot for boiling! \\
\poemll    Set it in place. \\
\poemlll       Fill it up with water, too. \\
\poeml \v{4}Gather together the best pieces of meat on it--- \\
\poemll    including the thighs and the shoulders--- \\
\poemlll       and fill it with the choicest bones. \\
\poeml \v{5}Take the best bones from the flock, \\
\poemll    pile wood\fnote{\fbackref{24:5} The Heb. lacks \fbib{wood}} under the pot\fnote{\fbackref{24:5} Lit. \fbib{under it}} for the bones, \\
\poeml bring it to a boil, \\
\poemll    and then cook the bones in it.''\,'\,''
\end{poetry}
\passage{Woe to Jerusalem}

\v{6}``This is what the Lord \divine{God} says:

\begin{poetry}
\poeml `How terrible it is for that blood-filled city, \\
\poemll    to the pot whose rust remains in it, \\
\poemlll       whose rust won't come off. \\
\poeml Empty it one piece at a time. \\
\poemll    Don't let a lot fall on it. \\
\poeml \v{7}Her blood was in it. \\
\poemll    She poured it out onto bare rock. \\
\poeml She didn't pour it out on the ground, \\
\poemll    intending to cover it with dirt. \\
\poeml \v{8}In order to stir up my anger \\
\poemll    and in order to take vengeance, \\
\poeml I set the blood on a bare rock \\
\poemll    so that it cannot be covered.'
\end{poetry}

\v{9}``Therefore this is what the Lord \divine{God} says:

\begin{poetry}
\poeml `How terrible it is for that blood-filled city--- \\
\poemll    I'm also going to add to my\fnote{\fbackref{24:9} The Heb. lacks \fbib{my}} pile of kindling. \\
\poeml \v{10}Pile up the wood! \\
\poemll    Make the fire burn hot. \\
\poeml Boil the meat! \\
\poemll    Mix the seasonings. \\
\poeml Burn those bones to a crisp! \\
\poeml \v{11}Make the pot stand empty on the coals \\
\poemll    until its bronze glows red,\fnote{\fbackref{24:11} Lit. \fbib{its copper burns hot}} \\
\poeml its rust can be scoured off,\fnote{\fbackref{24:11} Or \fbib{is poured out}} \\
\poemll    and its dross completely removed. \\
\poeml \v{12}The pot\fnote{\fbackref{24:12} Lit. \fbib{She}} wearies me,\fnote{\fbackref{24:12} The Heb. lacks \fbib{me}} \\
\poemll    but its thick\fnote{\fbackref{24:12} Lit. \fbib{great}} rust won't come off, \\
\poemlll       even with fire. \\
\poeml \v{13}There is wickedness in your obscene conduct. \\
\poemll    Even though I've cleansed you, \\
\poemlll       you uncleanness cannot be washed away. \\
\poeml You cannot be cleansed again \\
\poemll    until my rage against you has subsided.'
\end{poetry}

\v{14}```I, the \divine{Lord} have spoken. It will happen, because I'm going to do it. I won't hold back, have compassion, or change my mind.\fnote{\fbackref{24:14} Or \fbib{repent}} They'll judge you according to your ways and deeds,' declares the Lord \divine{God}.''
\passage{The Death of Ezekiel's Wife}

\v{15}This message came to me from the \divine{Lord}: \v{16}``Son of Man, pay attention! I'm about to take away your most precious treasure\fnote{\fbackref{24:16} Lit. \fbib{away the desire of your eyes}} with a single, fatal stroke, but you are not to mourn, weep, nor even let tears well up in your eyes.\fnote{\fbackref{24:16} Or \fbib{tears come}} \v{17}You are to weep in silence, but you are not to participate in mourning rituals.\fnote{\fbackref{24:17} Lit. \fbib{to mourn the dead}} You are to keep your turban on your head and your sandals on your feet. You are not to cover your mouth\fnote{\fbackref{24:17} Lit. \fbib{moustache}} or eat what your comforters bring to you.''\fnote{\fbackref{24:17} Lit. \fbib{eat the food of men}}

\v{18}So I spoke to the people in the morning, and my wife died that evening. The next\fnote{\fbackref{24:18} The Heb. lacks \fbib{next}} morning, I did as I had been commanded.

\v{19}Then the people told me, ``Are you going to explain what these things that you're doing should mean to us?''

\v{20}So I responded, ``This message came to me from the \divine{Lord}: \v{21}`Tell the house of Israel that this is what the Lord \divine{God} says: ``Look! I'm about to profane my sanctuary, the source of your proud strength, the desire of your eyes, and the object of your affection. Your sons and daughters, whom you've left behind, will die by the sword. \v{22}That's why you will soon be doing what I've just done. You are not to cover your mouth\fnote{\fbackref{24:22} Lit. \fbib{moustache}} or eat what your comforters bring to you.\fnote{\fbackref{24:22} Lit. \fbib{eat the food of men}} \v{23}Your turbans will be on your heads and your sandals will be on your feet. You won't mourn or weep. Instead, you'll waste away in your sins. Every one of you will groan to his relative. \v{24}That's how Ezekiel will be an example for you. You'll be doing exactly what he has done. When it happens, then you'll know that I am the Lord \divine{God}.''\,'

\v{25}``And now, Son of Man, on the day that I take their strength, joy, and glory from them, those whom they love to watch, the focus of their affection---their sons and daughters--- \v{26}at that time,\fnote{\fbackref{24:26} Lit. \fbib{day}} a fugitive will come to you and will bring you the news.\fnote{\fbackref{24:26} Lit. \fbib{will make ears hear}} \v{27}Your mouth will freely speak to the fugitive. You won't be silent any longer. You'll be a sign to them. Then they'll know that I am the \divine{Lord}.''
\labelchapt{25}
\passage{A Message Condemning Ammon}

\chapt{25}
\v{1}This message came to me from the \divine{Lord}: \v{2}``Son of Man, turn your attention\fnote{\fbackref{25:2} Lit. \fbib{face}} to the descendants of Ammon and rebuke\fnote{\fbackref{25:2} Lit. \fbib{and prophesy against}} them. \v{3}Tell the Ammonites: `Listen to a message from\fnote{\fbackref{25:3} Lit. \fbib{to the word of}} the Lord \divine{God}! This is what the Lord \divine{God} says: ``Because you have said, `Aha!'\fnote{\fbackref{25:3} I.e. an expression of delight upon hearing that disaster has befallen another} about my sanctuary when it was desecrated, about the land of Israel when it became desolate, and about the households of Judah when they went into exile, \v{4}therefore you'd better look out! I'm going to turn you over to men\fnote{\fbackref{25:4} Lit. \fbib{to children}} from the East, who will dominate you. You will become their property. They will set up military encampments and permanent places\fnote{\fbackref{25:4} Lit. \fbib{tents}} in which to live among you, and then they'll eat your fruit and drink your milk. \v{5}I will turn Rabbah\fnote{\fbackref{25:5} I.e. the capital city of the Ammonites, located east of the Jordan River} into a pasture for camels, and Ammon will become a resting place for flocks of sheep. That's how they'll learn that I am the \divine{Lord}.''\,'\,''
\passage{Why God Condemned Ammon}

\v{6}``This is what the Lord \divine{God} says: `Because you've applauded, stamped your feet, and rejoiced with all sorts of malice in your heart\fnote{\fbackref{25:6} I.e. expressions of delight upon hearing that disaster has befallen another} against the land of Israel, \v{7}therefore you'd better\fnote{\fbackref{25:7} The Heb. lacks \fbib{you'd better}} watch out! I'm raising a clenched fist\fnote{\fbackref{25:7} Lit. \fbib{I'm stretching out my hand}} in your direction! I'm about to feed you to the surrounding\fnote{\fbackref{25:7} The Heb. lacks \fbib{surrounding}} nations as war plunder. I'm going to eliminate you as a nation and kill off those of you who survive to live in other\fnote{\fbackref{25:7} The Heb. lacks \fbib{other}} countries. I'm going to destroy you, and that's how you'll learn that I am the \divine{Lord}.'\,''
\passage{A Message Rebuking Moab and Seir}

\v{8}``This is what the Lord \divine{God} says: `Because Moab and Seir are claiming, ``Judah's citizens are\fnote{\fbackref{25:8} Lit. \fbib{Judah's household is}} just like every other\fnote{\fbackref{25:8} The Heb. lacks \fbib{other}} nation,'' \v{9}therefore you'd better watch out! I'm going to tear open Moab's flanks, starting with its frontier cities---the very glory of the nation!---including Beth-jeshimoth,\fnote{\fbackref{25:9} This city was originally intended to be owned by the tribe of Reuben. Cf. Num 33:49; Josh 12:3; 13:20. The name means \fbib{House of Destruction}.} Baal-meon,\fnote{\fbackref{25:9} This city was originally intended to be owned by the tribe of Reuben. Cf. Num 33:38; 1Chron 5:8. The name means \fbib{Lord of the Habitation}.} and Kiriathaim.\fnote{\fbackref{25:9} This city was originally intended to be owned by the tribe of Reuben. Cf. Num 32:37; Josh 13:19. The name means \fbib{Twin Cities}.} \v{10}I'm going to turn these cities\fnote{\fbackref{25:10} Lit. \fbib{give it}} over to men\fnote{\fbackref{25:10} Lit. \fbib{to children}} from the East, who will dominate you. You will become their property. As a result, Ammon will be forgotten as a nation. \v{11}I'm also going to punish Moab, and that's how they'll learn that I am the \divine{Lord}.'\,''
\passage{The Coming Destruction of Edom}

\v{12}``This is what the Lord \divine{God} says: `Because Edom has made it their practice to seek extraordinary vengeance against Judah's citizens,\fnote{\fbackref{25:12} Lit. \fbib{household}} and by doing so has incurred extraordinary guilt by taking revenge against them,' \v{13}therefore this is what the Lord \divine{God} says: `I'm going to raise my clenched fist\fnote{\fbackref{25:13} Lit. \fbib{to stretch out my hand}} in Edom's direction and eliminate every single human being and animal from Edom! I'm going to turn everything into a wasteland, starting with Teman, and Dedan will fall by violence!\fnote{\fbackref{25:13} Lit. \fbib{by the sword}} \v{14}I'm going to inundate Edom with\fnote{\fbackref{25:14} Lit. \fbib{give Edom}} my retribution, using my people Israel to carry it out! They'll deliver my anger, acting as an agent of my fury. Edom will come to know my vengeance,' declares the Lord \divine{God}.'\,''
\passage{A Message Condemning Philistia}

\v{15}``This is what the Lord \divine{God} says: `Because Philistia has made it their practice to carry out retribution, accompanied by extraordinary malice in their personal vendettas---vendettas that spring from their everlasting hostility--- \v{16}this is what the Lord \divine{God} says: ``Look out! I'm raising my clenched fist\fnote{\fbackref{25:16} Lit. \fbib{to stretch out my hand}} in Philistia's direction. I'm going to execute\fnote{\fbackref{25:16} Lit. \fbib{cut off}; the Heb. verb is the root upon which the Heb. term \fbib{Cherethites} is based.} the Cherethites\fnote{\fbackref{25:16} Lit. \fbib{executioners}; i.e. Philistines who originally served as bodyguards for King David, but by the date of this writing had become rogue mercenaries who harassed the territory of ancient Israel.} and destroy what's left of the coastline of the Mediterranean\fnote{\fbackref{25:16} The Heb. lacks \fbib{Mediterranean}} Sea. \v{17}I'll take vengeance on them, punishing them severely in my anger. They'll know that I am the \divine{Lord} when I take my vengeance on them.''\,'\,''
\labelchapt{26}
\passage{A Message Condemning Tyre}

\chapt{26}
\v{1}During the eleventh year, on the first day of the month of our captivity\fnote{\fbackref{26:1} The Heb. lacks \fbib{of our captivity}}, this message came to me from the \divine{Lord}: \v{2}``Son of Man, because Tyre has been saying about Jerusalem,

\begin{poetry}
\poeml `The international gateway is broken down! \\
\poemll    It's wide open to me! \\
\poeml I will be replenished, \\
\poemll    now that it lies in ruins!'
\end{poetry}

\v{3}``Therefore this is what the Lord \divine{God} says: `Watch out! I'm coming to get\fnote{\fbackref{26:3} Lit. \fbib{coming against}} you, Tyre! I'm about to bring many nations to attack you. They'll come in wave after wave, like the advancing tide,\fnote{\fbackref{26:3} Lit. \fbib{come like the sea brings waves}} \v{4}and will destroy the city walls of Tyre. After they break down her fortified towers, I'll scrape away the city's debris, right down to the bare bedrock, \v{5}and it will become a place where nets will be spread out right in the middle of the Mediterranean\fnote{\fbackref{26:5} The Heb. lacks \fbib{Mediterranean}} Sea. Because I have declared this to happen,' declares the Lord \divine{God}, `Tyre will be treated as the spoils of war by the invading\fnote{\fbackref{26:5} The Heb. lacks \fbib{invading}} nations. \v{6}Furthermore, her citizens\fnote{\fbackref{26:6} Lit. \fbib{daughters}} who live on the mainland will be executed with swords. That's how they'll learn that I am the \divine{Lord}.'\,''
\passage{Nebuchadnezzar's Invasion}

\v{7}``This is what the Lord \divine{God} says: `Watch out! I'm about to bring from the north King Nebuchadnezzar of Babylon, that king of kings. He'll come with horses, chariots, cavalry, and a vast army. \v{8}He'll execute your citizens who live on the mainland with swords. He'll build siege engines to attack you. Then he'll construct siege ramps against you and build huge shields to protect themselves\fnote{\fbackref{26:8} The Heb. lacks \fbib{to protect themselves}} against you.

\v{9}```He'll direct the blows of his battering rams against your walls and will breach your fortified towers with axes.\fnote{\fbackref{26:9} Or \fbib{swords}} \v{10}There will be so many horses that the dust raised by them will cover you completely. The walls of your city will tremble from the noise of Nebuchadnezzar's\fnote{\fbackref{26:10} Lit. \fbib{their}} cavalry, wagons, and chariots when they enter through the gates of your city, as men enter a city that has been breached.

\v{11}```Their horses will trample all the public places as he executes your inhabitants with swords. The most fortified of your pillars will be torn to the ground. \v{12}They will plunder your riches and loot your businesses. They'll tear down your walls and demolish your luxurious homes. They'll grab the stones, wood, and rubble from the destruction and dump it all into the Mediterranean\fnote{\fbackref{26:12} The Heb. lacks \fbib{Mediterranean}} Sea.

\v{13}```I'll silence the noise of your songs and the music of your harps won't be heard anymore. \v{14}I'll turn you into bare rock, and your city will become a place to spread nets. You will never be built again, because I the \divine{Lord} have decreed this,' declares the Lord \divine{God}.''
\passage{Terror at Tyre's Destruction}

\v{15}``This is what the Lord \divine{God} says to Tyre: `When your wounded citizens groan while the slaughter takes place among you, the people who live in the coastlands will tremble in terror as they hear about your fall, will they not? \v{16}That's when all the kings of the seafaring nations will abandon their thrones, strip off their fancy clothes, and collapse trembling on the ground. They'll be so frightened as they observe what has happened to you that they'll be unable to stop trembling. They will be utterly appalled at you! \v{17}They'll sing this mourning song for you:

\begin{poetry}
\poeml ``How lost you are, \\
\poemll    you inhabited city, \\
\poemlll       that was built in the middle of the sea! \\
\poeml How famous you were! \\
\poemll    How strong on the sea! \\
\poeml She and her inhabitants \\
\poemll    inflicted\fnote{\fbackref{26:17} Lit. \fbib{inhabitants who inflicted}} terror to everyone \\
\poemlll       who lived within her.'' \\
\poeml \v{18}`Now the coastland inhabitants \\
\poemll    will tremble on the day that you fall. \\
\poeml The coastland inhabitants, \\
\poemll    who make their living from\fnote{\fbackref{26:18} Lit. \fbib{who are by}} the sea, \\
\poemlll       will be terrified when you pass away!'
\end{poetry}

\v{19}``This is what the Lord \divine{God} says: `When I turn your city into a ghost town, when I flood you with deep water that covers you completely, \v{20}I'll make sure that you go straight to the Pit,\fnote{\fbackref{26:20} I.e. the place of punishment in the afterlife} into the lowest part of the earth, where you'll be with people who lived in ancient times. You'll keep company there with the dead, who have gone into the Pit.\fnote{\fbackref{26:20} I.e. the place of punishment in the afterlife} As a result, your city\fnote{\fbackref{26:20} Lit. \fbib{result, you}} won't be inhabited. Meanwhile, I will display my glory in the land of the living. \v{21}I'm going to send terrifying calamity in your direction, and you won't exist any longer. You might be sought after, but you'll never be found again,' declares the Lord \divine{God}.''
\labelchapt{27}
\passage{A Message Condemning Tyre}

\chapt{27}
\v{1}This message came to me from the \divine{Lord}: \v{2}``Son of Man, compose a mourning song for Tyre. \v{3}Tell Tyre, who lives at the gateway to the Mediterranean\fnote{\fbackref{27:3} The Heb. lacks \fbib{Mediterranean}} Sea, who serves as the international merchant to many coastal districts: `This is what the Lord \divine{God} says:

\begin{poetry}
\poeml ``Tyre, you've been claiming, \\
\poemll    ``I am beauty perfected.' \\
\poeml \v{4}You've set your national boundary in international waters. \\
\poemll    Your builders made you downright beautiful!''\,'\,''\fnote{\fbackref{27:4} Or \fbib{you perfectly beautiful}}
\passage{Tyre's Luxurious Sailing Vessels}
\poeml \v{5}`They brought in a ship \\
\poemll    made with pine planking from Senir, \\
\poeml configured with a mast carved from a cedar from Lebanon, \\
\poeml \v{6}equipped with oars \\
\poemll    made from oaks from Bashan, \\
\poeml with ivory-inlaid cypress wood\fnote{\fbackref{27:6} Lit. \fbib{daughter of Assyria}} decking \\
\poemll    imported from the coastlands of Cypress, \\
\poeml \v{7}with sails made with embroidered Egyptian linen, \\
\poemll    festooned with blue banners, \\
\poeml and with your sun shades made \\
\poemll    with purple cloth from Cypress. \\
\poeml \v{8}Your sailors were conscripted \\
\poemll    from Sidon and Arvad, \\
\poeml and your officers served aboard \\
\poemll    as pilots. \\
\poeml \v{9}The wise men and elders from Gebal accompanied you, \\
\poemll    serving as ship's carpenters. \\
\poeml All the maritime navies and their seaman also accompanied you \\
\poemll    to assist you in doing business internationally.''
\passage{Tyre's International Makeup}
\poeml \v{10}``Soldiers from Persia,\fnote{\fbackref{27:10} I.e. modern Iran} Lud,\fnote{\fbackref{27:10} I.e., fourth son of Shem, progenitor of the Lydians. The Heb. name \fbib{Lud} means \fbib{strife.}} and Libya,\fnote{\fbackref{27:10} Lit. \fbib{Put;} i.e. the third son of Ham (cf. Gen 10:6); the Heb. name means \fbib{bow}} \\
\poemll    served in your army. \\
\poemlll       They were your mighty soldiers. \\
\poeml Their helmets and shields adorned your barracks walls, \\
\poemll    and they won battle decorations for you. \\
\poeml \v{11}Mercenaries from Arvad and Helech \\
\poemll    stood guard duty on your walls, \\
\poemlll       while brave men manned your towers. \\
\poeml They hung their shields all around your walls--- \\
\poemll    just the right touch to perfect your interior decorating!''\fnote{\fbackref{27:11} Lit. \fbib{your beauty}}
\end{poetry}
\passage{Tyre's Trading Partners}

\v{12}`Tarshish was your business partner because of your phenomenal wealth. They traded silver, iron, tin, and lead for your merchandise. \v{13}Greece, Tubal,\fnote{\fbackref{27:13} I.e. a son of Noah's son Japheth; this people resided in what is now modern eastern Turkey} and Meshech\fnote{\fbackref{27:13} I.e. a son of Noah's son Japheth; this people resided in what is now modern Armenia} bartered with you, exchanging slaves and bronze vessels for your wares. \v{14}Beth-togarmah traded horses, war horses, and mules in exchange for what you had to sell. \v{15}Men from the low country south of Edom\fnote{\fbackref{27:15} Lit. \fbib{The descendants of Dedan}; i.e., a son of Cush (cf. Gen 10:7) or a grandson of Abraham through Keturah (Gen 25:3)} and many of the coastlands were your markets for ivory tusks and ebony that they brought to trade with you.

\v{16}``Aram was one of your customers because you had so much merchandise. They paid by trading turquoise, purple yarn, embroidered goods, Egyptian linen,\fnote{\fbackref{27:16} Lit. \fbib{byssus}; i.e. a white cotton cloth} coral, and rubies. \v{17}The territories of Judah and Israel were your clients, too. They traded wheat from their distribution centers,\fnote{\fbackref{27:17} Lit. \fbib{from Minnith}; perhaps a site in Ammon, east of the Jordan River} baked goods, honey, oil, and ointments for your merchandise.

\v{18}``Because you have so much to sell and are so rich, Damascus has been your trading partner, exchanging wine from Helbon, unbleached wool, \v{19}and casks of wine from Izal\fnote{\fbackref{27:19} Or \fbib{Uzal}, i.e., a city located in a wine growing region between Haran and the Tigris River} for your wrought iron, cassia wood, and aromatic reeds.

\v{20}``Dedan traded with you, exchanging riding blankets. \v{21}Arabia, including all the princes of Kedar, came to you, shopping for lambs, rams, and goats. \v{22}Traders from Sheba and Raamah paid for the best of what you had to offer with all types of spices, precious stones, and gold. \v{23}Haran, Canneh, Eden,\fnote{\fbackref{27:23} These cities are thought to have been located in ancient Mesopotamia} merchants from Sheba, Asshur, and Chilmad did business with you, \v{24}trading garments made into the finest blue and embroidered mantels, and also multi-colored carpets, ropes, and other merchandise. \v{25}Ocean-going fleets\fnote{\fbackref{27:25} Lit. \fbib{Ships of Tarshish}; i.e., a class of vessel capable of travelling the oceans} carried your merchandise.''
\passage{Tyre's Coming Storm}

\begin{poetry}
\poeml ``How filled you were! \\
\poemll    How glorious you were, \\
\poemlll       at home in the heart of the sea! \\
\poeml \v{26}But your rowers have brought you \\
\poemll    into dangerous waters. \\
\poeml The east wind has broken you \\
\poemll    in the heart of the ocean! \\
\poeml \v{27}Your wealth, your products, your merchandise \\
\poemll    your sailors, your pilots, \\
\poeml your tailors, your salesmen, \\
\poemll    all your mercenaries with you--- \\
\poeml your entire company with you--- \\
\poemll    will fall into the midst of the sea \\
\poemlll       on the day when you're overthrown! \\
\poeml \v{28}When your ships' captains cry out, \\
\poemll    the pasturelands along the coast will cry out! \\
\poeml \v{29}Everyone who handles an oar will abandon ship, \\
\poemll    they'll head straight for dry land, \\
\poeml \v{30}and they will cry so loud \\
\poemll    you won't be able to make yourself heard! \\
\poemlll       How bitterly they'll cry! \\
\poeml They'll throw dust on their heads \\
\poemll    and wallow in ashes. \\
\poeml \v{31}They'll shave their heads bald because of you. \\
\poemll    They'll dress themselves in sackcloth \\
\poeml and weep for you with deep bitterness of heart, \\
\poemll    with the most pitiful of mourning. \\
\poeml \v{32}In the depth of their despair \\
\poemll    they'll compose a lament for you. \\
\poeml This is what they'll say:
\end{poetry}

\begin{poetry}
\poeml `Who is like Tyre? \\
\poemll    Who is so silent in the midst of the sea?' \\
\poeml \v{33}Your merchandise went out over the oceans \\
\poemll    to satisfy many nations; \\
\poeml with the abundance of your wealth \\
\poemll    you enriched the kings of the earth.
\end{poetry}

\begin{poetry}
\poeml \v{34}``But now it's your time to be wrecked \\
\poemll    at the bottom of the sea! \\
\poeml Your products and your workers have sunk, \\
\poemll    and so have you! \\
\poeml \v{35}Everyone who lives by the sea \\
\poemll    is appalled at your destruction. \\
\poeml Their leaders are terrified--- \\
\poemll    their faces reflect their fears! \\
\poeml \v{36}Traders circulate among the people, hissing at you. \\
\poemll    What a horror you've become! \\
\poeml Now you will cease to exist \\
\poemll    forever and ever!''
\end{poetry}
\labelchapt{28}
\passage{Prophecy against Tyre}

\chapt{28}
\v{1}This message came to me from the \divine{Lord}: \v{2}``Son of Man, tell Tyre's Commander-in-Chief,\fnote{\fbackref{28:2} Lit. \fbib{Nagid}; i.e. a senior officer entrusted with dual roles of operational oversight and administrative authority} `This is what the Lord \divine{God} says:

\begin{poetry}
\poeml ``Because your heart is arrogant,\fnote{\fbackref{28:2} Lit. \fbib{tall}} \\
\poemll    and because you keep saying, \\
\poeml `I have taken my seat, \\
\poemll    I am a god, \\
\poemlll       seated in God's seat right in the middle\fnote{\fbackref{28:2} Lit. \fbib{heart}} of the sea,'\fnote{\fbackref{28:2} I.e., an allusion to Tyre's location on an island off the coast of Lebanon} \\
\poeml and because you're a man, \\
\poemll    and not a god, \\
\poeml even though you pretend \\
\poemll    that you have a god-like heart{\ldots} \\
\poeml \v{3}Look! You're wiser than Daniel, aren't you? \\
\poemll    No secret is too mysterious for you! \\
\poeml \v{4}Your wisdom and understanding \\
\poemll    brought you phenomenal wealth. \\
\poeml You've brought gold and silver \\
\poemll    into your treasuries. \\
\poeml \v{5}By your great wisdom, \\
\poemll    by your skills in trading \\
\poeml you have amassed wealth for yourself \\
\poemll    and your heart has become arrogant \\
\poemlll       because of your wealth.'' \\
\poeml \v{6}Therefore this is what the Lord \divine{God} says: \\
\poeml ``Because you've made your heart \\
\poemll    like that of God \\
\poeml \v{7}Therefore, look! \\
\poemll    I'm bringing foreigners in your direction, \\
\poemlll       the most terrifying of nations! \\
\poeml They will direct their violence\fnote{\fbackref{28:7} Lit. \fbib{swords}} \\
\poemll    against the grandeur \\
\poemlll       that you've created by your wisdom. \\
\poeml \v{8}They'll send you down to the Pit,\fnote{\fbackref{28:8} I.e. the place of punishment in the afterlife} \\
\poemll    and you'll die defiled in the depths of the sea. \\
\poeml \v{9}Is that when you'll say, `I'm God' \\
\poemll    to the face of those who will be killing you? \\
\poeml After all, you're a man, \\
\poemll    and have never been a god, \\
\poeml especially when you're under the control\fnote{\fbackref{28:9} Lit. \fbib{you're in the hand}} of those \\
\poemll    who will defile you! \\
\poeml \v{10}You will die a death fit for the uncircumcised \\
\poemll    at the hand of foreigners.'' \\
\poeml `for I have said it will be so,' \\
\poemll    declares the \divine{Lord}.''
\end{poetry}
\passage{A Rebuke for Tyre's King}

\v{11}Another message came to me from the \divine{Lord}, and this is what it said: \v{12}``Son of Man, start singing this lamentation for the king of Tyre. Tell him, `This is what the Lord \divine{God} says:

\begin{poetry}
\poeml ``You served as my\fnote{\fbackref{28:12} The Heb. lacks \fbib{my}} model, \\
\poemll    my example of complete wisdom \\
\poemlll       and perfect beauty. \\
\poeml \v{13}You used to be in Eden--- \\
\poemll    God's paradise! \\
\poeml You wore precious stones for clothing: \\
\poeml ruby,\fnote{\fbackref{28:13} Lit. \fbib{red}} topaz, diamond,\fnote{\fbackref{28:13} Or \fbib{emerald}} \\
\poemll    beryl,\fnote{\fbackref{28:13} Lit. \fbib{Tarshish}; i.e. a yellow stone} onyx, jasper, \\
\poemlll       sapphire,\fnote{\fbackref{28:13} Or \fbib{lapis lazuli}; a bright blue stone} turquoise, and carbuncle. \\
\poeml Your settings were crafted in gold, \\
\poemll    along with your engravings. \\
\poeml On the day of your creation \\
\poemll    they had been prepared! \\
\poeml \v{14}``You were the anointed cherub; \\
\poemll    having been set in place \\
\poeml on the holy mountain of God, \\
\poemll    you walked in the midst of fiery stones. \\
\poeml \v{15}You were blameless in your behavior\fnote{\fbackref{28:15} Lit. \fbib{ways}} \\
\poemll    from the day you were created \\
\poemlll       until wickedness was discovered in you. \\
\poeml \v{16}Since your vast business dealings \\
\poemll    filled you with violent intent\fnote{\fbackref{28:16} Lit. \fbib{with violence}} \\
\poeml from top to bottom,\fnote{\fbackref{28:16} Lit. \fbib{in your midst}} \\
\poemll    you sinned, \\
\poeml so I cast you away as defiled \\
\poemll    from the mountain of God. \\
\poeml I destroyed you, \\
\poemll    you guardian cherub, \\
\poemlll       from the midst of the fiery stones. \\
\poeml \v{17}Your heart grew arrogant because of your beauty; \\
\poemll    you annihilated your own wisdom \\
\poemlll       because of your splendor. \\
\poeml Then I threw you to the ground \\
\poemll    in the presence of kings, \\
\poemlll       giving them a good look at you! \\
\poeml \v{18}By all of your iniquity \\
\poemll    and unrighteous businesses \\
\poeml you defiled your sanctuaries, \\
\poemll    so I'm going to bring out fire from within you \\
\poeml and burn you to ashes on the earth \\
\poemll    before the whole watching world! \\
\poeml \v{19}Everyone who knows you \\
\poemll    throughout all the nations \\
\poeml will be appalled at your calamity \\
\poemll    and you will no longer exist forever.''\,'\,''
\end{poetry}
\passage{The Judgment of Sidon}

\v{20}Another message came to me from the \divine{Lord}, who had this to say:

\v{21}``Son of Man, turn your attention\fnote{\fbackref{28:21} Lit. \fbib{face}} to Sidon and prophesy against her.\fnote{\fbackref{28:21} I.e., the city personified as a woman} \v{22}Tell her:

\begin{poetry}
\poeml `Pay attention to me, Sidon! \\
\poemll    I'm against you, \\
\poemlll       and I'm going to glorify myself right in your midst.' \\
\poeml They'll learn that I am the \divine{Lord} \\
\poemll    when I carry out these punishments \\
\poemlll       and manifest my holiness in her midst. \\
\poeml \v{23}I'm going to send disease into that city\fnote{\fbackref{28:23} Lit. \fbib{disease into her}} \\
\poemll    and blood into her streets. \\
\poeml People will drop dead in her midst \\
\poemll    from the violence done to\fnote{\fbackref{28:23} Lit. \fbib{the sword brought against}} her from every side. \\
\poemlll       Then they'll learn that I am the Lord \divine{God}.''
\end{poetry}
\passage{The Future Regathering of Israel}

\v{24}``The house of Israel will never again suffer from painful briers and sharp thorn bushes that surround them on every side, and they will learn that I am the \divine{Lord}. \v{25}This is what the Lord \divine{God} says:

`When I gather the house of Israel from the nations to which I've scattered them, I will show them my holiness before the watching world, and they will live on the land that I gave to my servant Jacob. \v{26}They will live in safety in the land,\fnote{\fbackref{28:26} Lit. \fbib{in her}} building houses and planting vineyards. They'll live in safety while I judge everyone who maligns them among those who surround them. At that time they'll learn that I am the \divine{Lord} their God.'\,''
\labelchapt{29}
\passage{Prophecy against Egypt}

\chapt{29}
\v{1}In the tenth year, on the twelfth day of the tenth month, a message came to me from the \divine{Lord}, who had this to say:

\v{2}``Son of Man, turn your attention to Pharaoh, king of Egypt, and prophesy against him and the entire nation of Egypt. \v{3}Tell him that this is what the Lord \divine{God} says:

\begin{poetry}
\poeml `Watch out! I'm coming to get\fnote{\fbackref{29:3} Lit. \fbib{I'm against}} you, \\
\poemll    Pharaoh, king of Egypt! \\
\poeml You big monster! \\
\poemll    You lay in wait in the middle of your waterways\fnote{\fbackref{29:3} Or \fbib{Nile}; i.e. the Nile River, and so throughout the chapter} and say, \\
\poeml ``My waterways belong to me! \\
\poemll    I made them for myself!'' \\
\poeml \v{4}`So I'm going to plant a hook in your jaw \\
\poemll    and make the fish in your waterways grab hold of your scales. \\
\poeml I'll bring you up out of the middle of your waterways, \\
\poemll    along with all of the fish from your waterways that cling to your scales, \\
\poeml \v{5}Then I'll fling you out into the desert, \\
\poemll    you and all those fish in your waterways. \\
\poeml You'll fall out in the open fields; \\
\poemll    you'll never be reunited. \\
\poeml I'm giving you to the wild beasts of the earth \\
\poemll    and to the birds of the sky, \\
\poemlll       and they will dine on you! \\
\poeml \v{6}`Then everyone living in Egypt \\
\poemll    will know that I am the \divine{Lord}, \\
\poeml because they have been an unreliable ally\fnote{\fbackref{29:6} Lit. \fbib{been a staff made of reeds}} \\
\poemll    to the house of Israel. \\
\poeml \v{7}When they reached out to you for support, \\
\poemll    you tore their hands \\
\poemlll       and dislocated all of their shoulders. \\
\poeml When they tried to lean on you, \\
\poemll    they couldn't control their own bowels.'\fnote{\fbackref{29:7} Lit. \fbib{all of their loins became unstable}; so LXX; i.e. an involuntary physiological response from terror in the aftermath of their military defeat}
\end{poetry}

\v{8}``Therefore this is what the Lord \divine{God} says: `Look out! I'm bringing violent death\fnote{\fbackref{29:8} Lit. \fbib{bringing a sword}} in your direction! I'm going to kill every person and animal, \v{9}and the land of Egypt will be turned into a desolate ruin. Then you will know that I am the \divine{Lord}. Because Egypt said, ``The Nile is mine. I made it!'' \v{10}therefore watch out! I'm coming to get\fnote{\fbackref{29:10} Lit. \fbib{I'm against}} you! I'm going to attack your waterways, and then I'm going to make the land of Egypt a total wasteland from the Aswan\fnote{\fbackref{29:10} Lit. \fbib{Syene}, an Egyptian frontier town near the southern border with Ethiopia} fortress to the border of Ethiopia!\fnote{\fbackref{29:10} Lit. \fbib{Cush}; the Heb. name means \fbib{black}} \v{11}Neither man nor beast will walk through that area. It won't even be inhabited for 40 years. \v{12}I'll see to it that Egypt becomes a devastated land in the midst of devastated lands. Her cities deep inside her territories will be laid waste and desolate for 40 years. I will scatter Egypt among the nations and disperse them throughout the land.'\,''
\passage{Restoration of Egypt after Judgment}

\v{13}``Because this is what the \divine{Lord} says: `At the end of 40 years I'll gather the Egyptians from the people among whom they have been scattered. \v{14}I'll restore the economy\fnote{\fbackref{29:14} Lit. \fbib{fortunes}} of Egypt and return them to the land of Pathros,\fnote{\fbackref{29:14} I.e. southern Egypt} from which they originated, and there they will remain an insignificant kingdom, \v{15}the least significant of kingdoms. It will never again dominate other nations. I will make them so small that they will never again rule any nation. \v{16}Egypt will never again be a source of confidence to the nation\fnote{\fbackref{29:16} Lit. \fbib{house}} of Israel. Instead, Egypt will serve as a reminder of when they sinned by turning to Egypt for help. Then they'll know that I am the Lord \divine{God}.'\,''
\passage{Egypt Given to Nebuchadnezzar}

\v{17}On the first day of the first month of the twenty-seventh year of our captivity,\fnote{\fbackref{29:17} The Heb. lacks \fbib{of our captivity}} a message came to me from the \divine{Lord}, who had this to say:

\v{18}``Son of Man, King Nebuchadnezzar of Babylon made his army work very hard to attack Tyre. They tore their hair out and rubbed their shoulders raw! Despite all of that work trying to capture Tyre, neither he nor his army got paid from Tyre for all that! \v{19}Therefore this is what the Lord \divine{God} says: `I'm going to give the land of Egypt to King Nebuchadnezzar of Babylon. He's going to carry off her wealth, confiscate her war implements, and use it all to pay wages for his army! \v{20}I've given him the land of Egypt as a reward for attacking Tyre for me,' declares the Lord \divine{God}. \v{21}`When that day comes about, I'll strengthen Israel's military might, and I will give you an audience in their midst. Then they will know that I am the \divine{Lord}.'\,''
\labelchapt{30}
\passage{The Day of the \divine{Lord}}

\chapt{30}
\v{1}Another message came to me from the \divine{Lord}, who had this to say:

\v{2}``Son of Man, here's what you are to prophesy and announce,

\begin{poetry}
\poeml `This is what the Lord \divine{God} says: \\
\poeml ``Wail out loud! \\
\poemll    Oh no! The day! \\
\poeml \v{3}For comes now the day--- \\
\poemll    comes now the Day of the \divine{Lord}, \\
\poeml the day of clouds! \\
\poemll    The time of the gentiles\fnote{\fbackref{30:3} Or \fbib{nations}} is fulfilled!\fnote{\fbackref{30:3} Or \fbib{gentiles shall come to pass}; or \fbib{It shall be the time of judgment for the gentiles}} \\
\poeml \v{4}War\fnote{\fbackref{30:4} Lit. \fbib{The sword}} will come to Egypt, \\
\poemll    and Ethiopia will be in anguish \\
\poeml when the slain fall in Egypt, \\
\poemll    when her wealth is carried off, \\
\poemlll       and her foundations are demolished.
\end{poetry}

\v{5}``Ethiopia,\fnote{\fbackref{30:5} Lit. \fbib{Cush}; the Heb. name means \fbib{black}} Libya,\fnote{\fbackref{30:5} Lit. \fbib{Put}; the Heb. name means \fbib{bow}} descendants of\fnote{\fbackref{30:5} The Heb. lacks \fbib{descendants of}} Lud,\fnote{\fbackref{30:5} I.e., the fourth son of Shem, progenitor of the Lydians. The Heb. name \fbib{Lud} means \fbib{strife.}} all those who have mixed themselves,\fnote{\fbackref{30:5} So LXX; cf. Dan 2:43; MT reads \fbib{all the twilight}} and Libya\fnote{\fbackref{30:5} Lit. \fbib{Chub}; the Heb. means \fbib{horde}}---along with everyone in the land of Israel who is in league\fnote{\fbackref{30:5} Or \fbib{who have joined in a covenant}} with them---will die violently.''\,'\,''\fnote{\fbackref{30:5} Lit. \fbib{will fall by the sword}}
\passage{Continued Judgment on Egypt}

\begin{poetry}
\poeml \v{6}``This is what the \divine{Lord} says: \\
\poeml `Those who are supporting Egypt will fall; \\
\poemll    her majestic strength that she brought\fnote{\fbackref{30:6} The Heb. lacks \fbib{that she brought}} from the Aswan\fnote{\fbackref{30:6} Lit. \fbib{Syene}, an Egyptian frontier town near the southern border with Ethiopia} fortress will collapse \\
\poemlll       by the sword that invades her,' \\
\poeml declares the Lord \divine{God}. \\
\poeml \v{7}They'll remain desolate among desolate lands, \\
\poemll    their cities will be named among those that are ruined. \\
\poeml \v{8}They will know that I am the \divine{Lord} \\
\poemll    when I kindle my fire in Egypt \\
\poemlll       and all who help her are crushed.
\end{poetry}

\v{9}`When that happens, couriers will go out in ships to terrify Ethiopia\fnote{\fbackref{30:9} Lit. \fbib{Cush}; the Heb. name means \fbib{black}} in its complacency. Anguish will visit them as it will visit Egypt. Watch out! It's coming!'\,''
\passage{Foreigners will Invade Egypt}

\begin{poetry}
\poeml \v{10}``This is what the \divine{Lord} says: \\
\poeml `I'm putting an end to that gang from Egypt, \\
\poemll    and I'm going to use King Nebuchadnezzar of Babylon, to do it! \\
\poeml \v{11}He and his ruthless army with him will be brought \\
\poemll    to destroy the land. \\
\poeml They'll draw their swords and attack\fnote{\fbackref{30:11} Lit. \fbib{swords against}} Egypt, \\
\poemll    filling the land with the dead! \\
\poeml \v{12}I'll dry up their waterways, \\
\poemll    and evil men will sell off the land. \\
\poeml I'm going to make that land desolate, \\
\poemll    along with everything that's in it, \\
\poeml and I'm going to use foreigners to do it. \\
\poemll    I, the \divine{Lord} have spoken!'\,''
\passage{Destruction of Egypt's Gods}
\poeml \v{13}``This is what the Lord \divine{God} says: \\
\poeml `I will destroy the idols \\
\poemll    and put an end to the images that come from Memphis. \\
\poeml There will no longer be a prince from the land of Egypt, \\
\poemll    and I will terrify the land of Egypt. \\
\poeml \v{14}I'm going to turn Pathros into a desolation, \\
\poemll    set fire to Zoan,\fnote{\fbackref{30:14} I.e. the residence city of Egypt's Pharaoh at the time of the exodus (c. 1440 BCE)} \\
\poemlll       and judge Thebes.\fnote{\fbackref{30:14} Lit. \fbib{No}; i.e. the ancient capital of Egypt} \\
\poeml \v{15}I'll pour out my anger on Sin,\fnote{\fbackref{30:15} So MT; LXX reads \fbib{Sain}; i.e. Pelusium, a fortified city on Egypt's northeastern border} \\
\poemll    Egypt's strong fortress, \\
\poemlll       and I'll eliminate the gangs in Thebes. \\
\poeml \v{16}I'll set fire to Egypt, \\
\poemll    and Aswan\fnote{\fbackref{30:16} LXX reads \fbib{Syene}, an Egyptian frontier town near the southern border with Ethiopia; MT reads \fbib{Sin}} will writhe in agony. \\
\poeml Thebes will be demolished, \\
\poemll    and Memphis will face daily distress. \\
\poeml \v{17}The young men of On and Pi-beseth will die violently,\fnote{\fbackref{30:17} Lit. \fbib{will fall by the sword}} \\
\poemll    and their cities will be taken captive. \\
\poeml \v{18}It will be a dark day for Tahpanhes \\
\poemll    when I break the yokes of Egypt. \\
\poeml That's when her arrogant power will come to an end. \\
\poemll    She'll be covered by a cloud, \\
\poemlll       and her citizens\fnote{\fbackref{30:18} Lit. \fbib{daughters}} will go into captivity. \\
\poeml \v{19}I will judge Egypt, \\
\poemll    and they will learn that I am the \divine{Lord}.'\,''
\end{poetry}
\passage{Babylon's Victory}

\v{20}On the seventh day of the first month of the eleventh year of our captivity,\fnote{\fbackref{30:20} The Heb. lacks \fbib{of our captivity}} a message came to me from the \divine{Lord}. It had this to say: \v{21}``Son of Man, I've broken the arm of Pharaoh, king of Egypt. Look! It hasn't been set in a splint for healing or wrapped with a bandage so it could be strong enough to hold a sword! \v{22}Therefore this is what the Lord \divine{God} says:

`I'm coming to attack Pharaoh, king of Egypt, and I'm going to break both of his arms, the strong one and the wounded one. That will make him drop his sword. \v{23}I'm going to scatter Egypt throughout the surrounding\fnote{\fbackref{30:23} The Heb. lacks \fbib{surrounding}} nations and disperse them throughout the world. \v{24}I'm going to strengthen the military might\fnote{\fbackref{30:24} Lit. \fbib{the arm}} of the king of Babylon, put my own sword in his hand, and break Pharaoh's strength.\fnote{\fbackref{30:24} Lit. \fbib{arms}} Then Pharaoh\fnote{\fbackref{30:24} Lit. \fbib{he}} will groan like a dying man right in front of the king of Babylon.\fnote{\fbackref{30:24} Lit. \fbib{of him}} \v{25}When I strengthen the military might of Babylon, the military might of Pharaoh will fail, and then they will learn that I am the \divine{Lord} when I place my own sword in the hand of the king of Babylon. He will attack the land of Egypt. \v{26}When I scatter the Egyptians among the nations and disperse them throughout the world, they will learn that I am the \divine{Lord}.'\,''
\labelchapt{31}
\passage{Egypt Learns from Assyria's Demise}

\chapt{31}
\v{1}On the first day of the third month of the eleventh year of our captivity,\fnote{\fbackref{31:1} The Heb. lacks \fbib{of our captivity}} this message came to me from the \divine{Lord}: \v{2}``Son of Man, tell this to Pharaoh, king of Egypt and his gangs:

\begin{poetry}
\poeml `Who do you think you are? \\
\poemll    What makes you so great? \\
\poeml \v{3}Think about Assyria,\fnote{\fbackref{31:3} Lit. \fbib{Asshur}} \\
\poemll    that cedar of Lebanon, \\
\poeml beautiful with its branches, \\
\poemll    like a shady forest, \\
\poeml with an awesome height, \\
\poemll    its summit touches the clouds. \\
\poeml \v{4}Abundant water made it great, \\
\poemll    Subterranean rivers made it grow. \\
\poeml Rivers surrounded the area where it had been planted, \\
\poemll    and water channels nourished all the trees in the fields. \\
\poeml \v{5}That's why it grew taller than any of the trees in the fields. \\
\poemll    Its boughs flourished. \\
\poeml Its branches grew luxurious \\
\poemll    because all the water made it spread out well. \\
\poeml \v{6}The birds in the sky made nests in its boughs; \\
\poemll    all the beasts of the field gave birth under its branches. \\
\poemlll       All the great nations rested in its shade. \\
\poeml \v{7}`Beautiful because it was so great, \\
\poemll    with its long branches, \\
\poemlll       it was rooted in many bodies of water.\fnote{\fbackref{31:7} Lit. \fbib{many waters}} \\
\poeml \v{8}The cedars in God's garden could not compare to it; \\
\poemll    Fir trees could not match its boughs. \\
\poeml The plane tree\fnote{\fbackref{31:8} i.e. a species of trees that could readily be stripped of their bark; cf. Gen 30:37} never grew branches like it, \\
\poemll    and no tree in God's garden compares to its beauty. \\
\poeml \v{9}I made it beautiful, \\
\poemll    including all of its branches; \\
\poeml all the trees in God's garden of Eden envied it!'\,''
\end{poetry}
\passage{Assyria's Fall Due to Arrogance}

\v{10}``Therefore this is what the Lord \divine{God} says: `Because of its towering height, with its summit reaching into the clouds, and because it was haughty in its position,\fnote{\fbackref{31:10} Lit. \fbib{heart}} \v{11}I turned it over to the leader of those\fnote{\fbackref{31:11} Lit. \fbib{to the hand of the leader of the}} nations, who dealt with it thoroughly. I have driven it away because of its wickedness. \v{12}Foreign dictators have trimmed it down to size and abandoned it. Its branches have fallen off on mountains and in all the valleys. Its boughs have broken off in all the ravines of the land. All the nations of the earth have moved out of its shade and abandoned it. \v{13}All the birds in the sky will live among its ruins, and the wild animals\fnote{\fbackref{31:13} Lit. \fbib{the animals of the field}} will forage among its branches. \v{14}As a result, none of its watered trees will grow tall, their tops will never reach to the clouds, and they'll never grow so high again, because all of them have been appointed\fnote{\fbackref{31:14} Lit. \fbib{given}} to death in the world beneath where human beings go, that is, down to the Pit.'\,''\fnote{\fbackref{31:14} I.e. the realm of eternal punishment in the afterlife}
\passage{Fear at Assyria's Fall}

\v{15}``This is what the Lord \divine{God} says: `On the day that it descended into Sheol,\fnote{\fbackref{31:15} I.e. the realm of the afterlife} I shut down its water supplies, covered over its deep water, and shut down its rivers. As a result its abundant water sources dried up, and I caused Lebanon to mourn for it. All the trees of the field wilted because of it. \v{16}I made the nations tremble when they heard that Assyria\fnote{\fbackref{31:16} Lit. \fbib{he}} was falling, descending into Sheol\fnote{\fbackref{31:16} I.e. the realm of the afterlife} to join those who go down into the Pit.\fnote{\fbackref{31:16} I.e. the realm of punishment in the afterlife} Then all of the trees of Eden in the world below were comforted, including the choicest and best of Lebanon, all of whom were well-watered. \v{17}They also went down with it into Sheol,\fnote{\fbackref{31:17} I.e. the realm of the afterlife} to those who had been killed violently\fnote{\fbackref{31:17} Lit. \fbib{killed by the sword}} and to those who had trusted in its strength by living in its shadow among the nations. \v{18}So tell me now, which of the trees of Eden compares to you in glory or greatness? Nevertheless, you'll be brought down, along with those trees of Eden, to the earth below. You'll lie in the middle of the uncircumcised, with those who have been killed in war.\fnote{\fbackref{31:18} Lit. \fbib{killed by the sword}} Pharaoh and all his gang will be just like this!' declares the Lord \divine{God}.''
\labelchapt{32}
\passage{Another Prophecy about Egypt}

\chapt{32}
\v{1}On the first day of the twelfth month of the twelfth year of our captivity,\fnote{\fbackref{32:1} The Heb. lacks \fbib{of our captivity}} a message came to me from the \divine{Lord}, who had this to say:

\v{2}``Son of Man, start singing this lamentation about Pharaoh, king of Egypt. Tell him,

\begin{poetry}
\poeml `You may have called yourself a lion among nations, \\
\poemll    but you're a monster at sea. \\
\poeml You thrash about in your rivers, \\
\poemll    muddy the water with your feet, \\
\poemlll       and relieve yourself in the rivers.' \\
\poeml \v{3}``This is what the Lord \divine{God} says: \\
\poemll    `I'm coming fishing for you! \\
\poeml Right in the sight of many nations \\
\poemll    they'll haul you up in my dragnet. \\
\poeml \v{4}I'll fling you up onto the land; \\
\poemll    I'll haul you into the field, \\
\poeml I'll make every carrion-eating bird come to dine on you, \\
\poemll    and I'll make all the scavenging animals gorge themselves on you. \\
\poeml \v{5}I'll cover the mountains with your flesh \\
\poemll    and fill their valleys with your rotting carcass.\fnote{\fbackref{32:5} So MT; the Syr. Peshittas read \fbib{your maggots}} \\
\poeml \v{6}I'll drench the land with your blood, \\
\poemll    right up to the mountains, \\
\poeml and the ravines will overflow \\
\poemll    with blood that comes from you! \\
\poeml \v{7}When I extinguish your lights, \\
\poemll    I'll cover the heavens \\
\poemlll       and darken their stars. \\
\poeml I'll cover the sun with a cloud \\
\poemll    and the moon won't reflect its light. \\
\poeml \v{8}I'll darken the bright lights in the sky above you \\
\poemll    and bring darkness to your territory,' \\
\poemlll       declares the Lord \divine{God}.
\end{poetry}

\v{9}```I'll bring distress to the hearts of many nations when I destroy you among nations whose territories you have not known. \v{10}I'll make many nations be appalled at you, and their kings will be terrified because of you when I brandish my sword right in their face. They will all tremble from fear for their own safety\fnote{\fbackref{32:10} Lit. \fbib{soul}} on the day that you fall!'

\v{11}``This is what the Lord \divine{God} says: `The army\fnote{\fbackref{32:11} Lit. \fbib{sword}} of the king of Babylon will attack you. \v{12}I'm going to make your gangs die using the weapons of valiant warriors, all of whom are ruthless people.

\begin{poetry}
\poeml `They will devastate the majesty of Egypt, \\
\poemll    destroying all of its hordes. \\
\poeml \v{13}I'm going to destroy all of its livestock \\
\poemll    along its many riverbanks. \\
\poeml Human feet won't muddy the rivers anymore, \\
\poemll    nor will the hooves of livestock stir up the water. \\
\poeml \v{14}That's when I'll make their waterways flow smoothly, \\
\poemll    and their rivers flow like olive oil,'\fnote{\fbackref{32:14} I.e. the rivers will be undisturbed by human activity} \\
\poemlll       declares the Lord \divine{God}.'' \\
\poeml \v{15}`When I turn the land of Egypt into a desolation, \\
\poemll    and the land is emptied of everything that used to fill it, \\
\poeml when I strike everyone who lives there, \\
\poemll    they will learn that I am the \divine{Lord}.'
\end{poetry}

\v{16}``This has been a lamentation. They will chant it, and the citizens\fnote{\fbackref{32:16} Lit. \fbib{daughters}} of the nations will chant it, too. They'll chant it about Egypt and about all of its hordes.''
\passage{Babylon's Invasion of Egypt}

\v{17}On the fifteenth day of the first\fnote{\fbackref{32:17} The Heb. lacks \fbib{the first}} month of the twelfth year of our captivity,\fnote{\fbackref{32:17} The Heb. lacks \fbib{of our captivity}} a message from the \divine{Lord} came to me, and this is what it said: \v{18}``Son of Man, mourn about the hordes of Egypt. Bring them down---that is, her and the citizens\fnote{\fbackref{32:18} Lit. \fbib{daughters}} of those majestic\fnote{\fbackref{32:18} Or \fbib{powerful}} nations---whose destiny is the deep part of the Pit.\fnote{\fbackref{32:18} I.e. the realm of punishment in the afterlife}

\begin{poetry}
\poeml \v{19}``So who's more beautiful than you? \\
\poemll    You'll be buried with the uncircumcised.\fnote{\fbackref{32:19} I.e. as one who does not honor God, and so throughout the chapter}
\end{poetry}

\v{20}``They'll die along with others who are killed violently.\fnote{\fbackref{32:20} Lit. \fbib{killed with the sword}, and so throughout the chapter} Egypt has been given over to violence,\fnote{\fbackref{32:20} Lit. \fbib{to the sword}, and so throughout the chapter} which will carry off both it and its hordes.''
\passage{Egypt Condemned by the Dead}

\v{21}``Mighty leaders will address them and those who assist them right out of the middle of Sheol:\fnote{\fbackref{32:21} I.e. the realm of the afterlife} `They've come down and will lie still, these uncircumcised people who have died violently.'\fnote{\fbackref{32:21} Lit. \fbib{by the sword}, and so throughout the chapter} \v{22}Assyria will be there, along with all of those who keep company with her,\fnote{\fbackref{32:22} I.e. the nation personified as a woman} all of them killed violently. \v{23}Her grave will be set in the remotest part of the Pit,\fnote{\fbackref{32:23} I.e. the realm of punishment in the afterlife} surrounded by those who accompanied her. All of them will have been killed, executed violently, who spread terror throughout the land of the living.

\v{24}``Elam will be there. Its hordes will surround Elam's\fnote{\fbackref{32:24} Lit. \fbib{its}} grave. All of them have been killed. They died\fnote{\fbackref{32:24} Lit. \fbib{fell}} violently, and they have descended uncircumcised into the world below after having spread terror throughout the land of the living. They will bear the shame of those who descend to the Pit.\fnote{\fbackref{32:24} I.e. the realm of punishment in the afterlife} \v{25}They have prepared a bed for her and for her hordes that surround her graves. All of them are uncircumcised, having been killed violently, because they had spread terror throughout the land of the living. They will bear the shame of those who descend to the Pit\fnote{\fbackref{32:25} I.e. the realm of punishment in the afterlife} and will take their place among the dead.

\v{26}``Meshech and Tubal will be there, along with all of the hordes that surround her grave. Every one of them is uncircumcised, killed violently, because they spread terror throughout the land of the living. \v{27}They won't be buried with dead warriors from ancient times, who went straight to Sheol,\fnote{\fbackref{32:27} I.e. the realm of the afterlife} buried with their war weapons, with their swords placed under their heads and their shields laid on top of their bones, since they spread terror throughout the land of the living. \v{28}You'll be broken, and you'll lie down with the uncircumcised who died violently.

\v{29}``Edom will be there, along with its kings and princes who despite all their power have been killed violently. They, too, are lying dead, along with the uncircumcised; that is, with those who descend into the Pit.\fnote{\fbackref{32:29} I.e. the realm of punishment in the afterlife}

\v{30}``All of the princes from the North are there, along with the Sidonians, who have gone down in shame to join those who have been killed because of all the terror they caused by their military might. They lie dead, uncircumcised, with those who have been killed violently. They will bear their shame, along with those who descend into the Pit.\fnote{\fbackref{32:30} I.e. the realm of punishment in the afterlife}

\v{31}``When Pharaoh sees them, he will take comfort in his hordes. Pharaoh and all his army will die violently,'' says the Lord \divine{God}, \v{32}``because he spread terror throughout the land of the living. Therefore he'll be laid to rest among the uncircumcised, who have been killed violently; that is, Pharaoh and all of his hordes,'' declares the Lord \divine{God}.
\labelchapt{33}
\passage{Warnings for Watchmen}

\chapt{33}
\v{1}This message came to me from the \divine{Lord}: \v{2}``Son of Man, speak to your nation's children and tell them: `If I bring war\fnote{\fbackref{33:2} Lit. \fbib{bring a sword}} to a land, and the people of that land appoint one of their conscripted men\fnote{\fbackref{33:2} Or \fbib{of the men from their border}} to serve as a sentinel, \v{3}and if he notices that violence\fnote{\fbackref{33:3} Lit. \fbib{that a sword}} is approaching and sounds an alarm to warn the people, \v{4}then if anyone who hears the sound of the alarm does not heed the warning, when the sword arrives and destroys him, his shed blood will remain his own responsibility.\fnote{\fbackref{33:4} Lit. \fbib{will be on him}} \v{5}After all, he heard the alarm sounding, but did not heed the warning, so his shed blood will remain his own responsibility.\fnote{\fbackref{33:5} Lit. \fbib{will be on him}} If he had heeded the warning, he would have saved himself.\fnote{\fbackref{33:5} Lit. \fbib{have rescued his soul}} \v{6}If that sentinel notices that violence is approaching, but does not sound an alarm, then because the nation does not take warning and the sword arrives and destroys their lives because of their guilt, I'll seek retribution for their shed blood from the\fnote{\fbackref{33:6} Lit. \fbib{from the hand of the}; i.e. as if the sentinel were responsible for the death} one who was acting as sentinel.'\,''
\passage{Warning for Ezekiel}

\v{7}``Now as for you, Son of Man, I've established you as a sentinel for the house of Israel. So whenever you hear a message from me,\fnote{\fbackref{33:7} Lit. \fbib{from my mouth}} you are to warn the people\fnote{\fbackref{33:7} Lit. \fbib{warn them}} from me. \v{8}If I should say to a certain wicked person, `You wicked man, you're certainly about to die,' but you don't warn him to turn from his wicked behavior,\fnote{\fbackref{33:8} Lit. \fbib{ways}} he'll die in his guilt, but I'll seek retribution for his bloodshed from you.\fnote{\fbackref{33:8} Lit. \fbib{from your hand}; i.e. as if the sentinel were responsible for the death} \v{9}However, if you warn the wicked to turn from his behavior\fnote{\fbackref{33:9} Lit. \fbib{way}} and he does not do so, he will die in his guilt, and you will have saved yourself.''\fnote{\fbackref{33:9} Lit. \fbib{have rescued your soul}}
\passage{God Hates the Death of the Wicked}

\v{10}```Now, Son of Man, tell this to the house of Israel:

`You keep saying, ``Our crimes and sins burden us so much that we're rotting away, so how can we keep on living?''\,'

\v{11}``Tell them, `As certainly as I'm alive and living,' declares the Lord \divine{God}, `I receive no pleasure in the death of the wicked. Instead, my pleasure is that the wicked repent from their behavior\fnote{\fbackref{33:11} Lit. \fbib{way}} and live. Turn back! Turn back, all of you, from your wicked behavior! Why do you have to die, you house of Israel?'\,''
\passage{Human Effort is Useless in Sustaining Righteousness}

\v{12}``And now, Son of Man, say this to your people: `The righteousness of the righteous won't save them when they keep on committing crimes against me, the wickedness of the wicked won't keep them from remaining away\fnote{\fbackref{33:12} Lit. \fbib{from stumbling}} when they're turning from their wickedness, and no righteous person will keep on living by their righteousness when they sin.'

\v{13}``If I tell the righteous person that he will certainly live, if he trusts in his own righteousness and commits evil, none of his righteousness will be remembered, and he will die because of the wrong that he commits.

\v{14}``If I tell the wicked person that he will certainly die, if he turns from his sin and acts with justice and righteousness, \v{15}returning what has been placed as collateral for a loan, paying back what he has taken, following the regulations that promote life, and committing no iniquity, he will certainly live, and not die. \v{16}None of the sins that he has committed will be remembered against him. Since he did what is just and right, he will certainly live.

\v{17}``Nevertheless, your people's children keep saying, `Living life the Lord's way\fnote{\fbackref{33:17} Lit. \fbib{saying, `The way of}} isn't right,' when all the while it is their way of living\fnote{\fbackref{33:17} The Heb. lacks \fbib{of living}} that isn't right. \v{18}When a righteous man forsakes his own righteousness and commits evil acts, he will die because of those acts, \v{19}and when the wicked turn away from their wickedness and do what is just and right, he will certainly live because of that. \v{20}``And yet you keep saying, `Living life\fnote{\fbackref{33:20} Lit. \fbib{saying, `The way of}} the Lord's way isn't right,' But I will judge every one of you according to the way you live, you house of Israel!''
\passage{False Reliance on Abraham's Heritage}

\v{21}On the fifth day of the tenth month of the twelfth year of our captivity, a fugitive who had escaped from Jerusalem came and informed me, ``The city has been destroyed.''

\v{22}Now the hand of the \divine{Lord} had been touching me the evening before that fugitive arrived, so the \divine{Lord} had given me something to say by the time the messenger\fnote{\fbackref{33:22} Lit. \fbib{time he}} arrived the next morning. He opened my mouth and I no longer had nothing to say to him.\fnote{\fbackref{33:22} The Heb. lacks \fbib{to him}} \v{23}As a result, this message came to me from the \divine{Lord}:

\v{24}``Son of Man, those who are living among these ruins of the land of Israel keep saying, `Abraham was only one man, but he was able to possess the land! As for us, we're a multitude, and the land has been given to us as an inheritance.' \v{25}So tell them, `This is what the Lord \divine{God} says: ``You keep eating flesh along with its blood, you keep looking to your idols, and you keep shedding blood, and you're going to take possession of the land? \v{26}You keep trusting in your weapons, you continue to commit loathsome deeds, men keep defiling their neighbors' wives, and you're going to take possession of the land?'

\v{27}``Tell them this: `This is what the Lord \divine{God} says: ``As certainly as I'm alive and living, those who live in the wastelands are certain to die violently,\fnote{\fbackref{33:27} Lit. \fbib{die by the sword}} I'll give those who die in the open fields to the wild animals for food, and whoever takes refuge in caves and fortified places will die of diseases. \v{28}Then I'll turn the land into a desolate ruin and her arrogant strength will come to an abrupt end. The mountains of Israel will become so desolate that no one will be able to travel over them.'' \v{29}`Then they'll learn that I am the \divine{Lord}, when I've turned their land into a desolate wasteland because of all of the loathsome deeds that they've committed.'\,''
\passage{The Disobedient Exiles of Babylon}

\v{30}``Now as for you, Son of Man, your nation's children keep gathering together to talk about you beside the walls and at the doorway to their houses. Everyone tells one another, `Please come! Let's go hear what the \divine{Lord} has to say!' \v{31}Then they come to you as a group, sit down right in front of you as if they were my people, hear your words---and then they don't do what you say---\fnote{\fbackref{33:31} The Heb. lacks \fbib{what you say}} because they're seeking only their own desires,\fnote{\fbackref{33:31} Lit. \fbib{because their lust is in their mouths}} they pursue ill-gotten profits, and they keep following their own self-interests. \v{32}As far as they are concerned, you sing romantic songs with a beautiful voice and play a musical instrument well. They'll listen to what you have to say, but they won't put it into practice! \v{33}When all of this comes about---and you can be sure that it will!---they'll learn that a prophet has been in their midst.''
\labelchapt{34}
\passage{Israel's False Shepherds}

\chapt{34}
\v{1}A message came from the \divine{Lord} for me, and it had this to say: \v{2}``Son of Man, prophesy against Israel's shepherds. Tell those shepherds, `This is what the Lord \divine{God} says:

``Woe to you shepherds of Israel who have been feeding yourselves and not the sheep. Shouldn't shepherds feed the sheep? \v{3}You're eating the best parts,\fnote{\fbackref{34:3} Lit. \fbib{the fat}} clothing yourselves with the wool, and slaughtering the home-grown sheep without having fed the sheep! \v{4}You haven't strengthened the weak, treated the sick, set broken bones, regathered the scattered, or looked for the lost. Instead, you've dominated them with brutal force and ruthlessness.

\v{5}``Since they have no shepherd, they have been scattered around and have become prey for all sorts of wild animals. How scattered they are! \v{6}My sheep have gone wandering on all of the mountains, on all of the hills, and throughout every high place in the whole world, with no one to look for them or go out in search of them.

\v{7}``Therefore listen to what the \divine{Lord} says, you shepherds: \v{8}`As certainly as I'm alive and living, my sheep have truly become victims, food for all of the wild animals because there are no shepherds. My shepherds did not go searching for my flock. Instead, the shepherds fed themselves, and my flock they would not feed!'

\v{9}``Therefore, you shepherds, listen to what the \divine{Lord} says: \v{10}`This is what the Lord \divine{God} says: ``Watch out, I'm coming after you shepherds! I'm going to demand my sheep back from them\fnote{\fbackref{34:10} Lit. \fbib{from their hand}} and fire them as shepherds. The shepherds won't be shepherds anymore when I snatch my flock right out of their mouths so they can't be eaten by them anymore.''\,'\,''
\passage{The Coming True Shepherd}

\v{11}``This is what the \divine{Lord} says: `Watch me! I'm going to search for my flock. I'll watch over them myself. \v{12}Just as a shepherd looks after his flock during the day time while he is with them, so also I'm going to watch over my sheep, delivering them from every place where they've been scattered during the times of gloom and doom.\fnote{\fbackref{34:12} Lit. \fbib{cloud}} \v{13}I'm going to bring them out from foreign\fnote{\fbackref{34:13} Lit. \fbib{the}} nations and from foreign lands. Then I'll bring them to their own land and feed them in Israel---on the mountains, in their valleys, and in all of their settlements throughout the land. \v{14}I'll feed them in excellent pastures, and even the very heights of Israel's mountains will serve as verdant pastures for them in which they'll rest and feed---yes, even on the fertile mountains of Israel! \v{15}I will feed my sheep and give them rest,' declares the Lord \divine{God}. \v{16}`I'm going to seek both the lost as well as the scattered, and bring them both back so their broken bones can be set and the sick can be healed. But in righteousness I'll exterminate the fat and the stiff-necked.'\,''
\passage{God's Message to His Sheep}

\v{17}``Now as for you, my flock, this is what the Lord \divine{God} says: `Watch out! I'm going to judge between one sheep and another, and between the rams and the goats. \v{18}Is it such an insignificant thing to you that you're feeding in good pastures but trampling down the other pastures with your feet? Or that as you're drinking from the clear streams you're muddying the rest with your feet? \v{19}My flock is grazing on what you've been treading down with your feet and they're drinking what you've been making muddy with your feet!'

\v{20}``Therefore this is what the Lord \divine{God} says to them: `Watch me! I'm going to judge between the fat sheep and the lean sheep, \v{21}since you've been bumping aside all the weaker sheep with your backsides and shoulders, butting them with your horns until they're scattered around outside. \v{22}That's how I'll save my sheep so they won't be plundered any longer. I'm going to judge between one sheep and another.'\,''
\passage{God's Shepherd: His Servant David}

\v{23}```Then I'll install one shepherd for them---my servant David---and he will feed them, will be there for them, and will serve as their shepherd. \v{24}I, the \divine{Lord}, will be their God, and my servant David will rule among them as Prince.' I, the \divine{Lord}, have spoken this.

\v{25}``I'm going to enter into a covenant with them, one of peace, and I'll eliminate wild beasts from the land so they can live securely in the wilderness and sleep in the forests. \v{26}I'm going to make\fnote{\fbackref{34:26} Lit. \fbib{give}} them and everything that surrounds my hill\fnote{\fbackref{34:26} I.e. Mount Zion} a blessing. I'll send down the rain! At the appropriate time there will be a rainstorm of blessing! \v{27}I'll bring fruit to the trees in the orchards, the land will yield its produce, they will live securely on their land, and they will learn that I am the \divine{Lord}, when I break the bar that has been their yoke and deliver them from the control of those who have enslaved them. \v{28}They will no longer be plundered by the nations, and wild animals will no longer devour them. They will settle down confidently, with nothing to frighten them. \v{29}I'm going to prepare for them the best of gardening spots. They will no longer live as victims in a land of starvation, nor will they have to bear the insults of the international community. \v{30}That's when they'll learn that I, the \divine{Lord} their God, am with them, and that they, the house of Israel, are my people,' declares the Lord \divine{God}. \v{31}`And as for you, my sheep, the flock that I'm pasturing, you are mankind, and I am your God,' declares the Lord \divine{God}.''
\labelchapt{35}
\passage{Prophecy against Mount Seir}

\chapt{35}
\v{1}A message came to me from the \divine{Lord} and it went like this: \v{2}``Son of Man, turn your attention\fnote{\fbackref{35:2} Lit. \fbib{face}} toward Mount Seir\fnote{\fbackref{35:2} This mountain, the modern \fbib{Jebel esh-sher\'{a}}, is located in the mountain range that extends south of the Dead Sea toward the Gulf of Aqaba, and is bordered by the Arabah Valley to the west.} and begin to prophesy against it. \v{3}Tell them,\fnote{\fbackref{35:3} Lit. \fbib{him}; i.e. the city personified as a single person} `This is what the Lord \divine{God} says:

\begin{poetry}
\poeml ``Watch out! I'm coming to get you, Mount Seir! \\
\poemll    I'm stretching out my hand to strike you, \\
\poemlll       and I'm going to turn you into a desolate wasteland. \\
\poeml \v{4}I'm going to turn your cities into ghost towns, \\
\poemll    and you will become a ruin. \\
\poeml Then you will learn \\
\poemll    that I am the \divine{Lord}.
\end{poetry}

\v{5}``Because of your undying hatred, you kept on making the Israelis experience abuse\fnote{\fbackref{35:5} Lit. \fbib{to the hand of the sword}} during the time of their calamity, even when they were in their final stages\fnote{\fbackref{35:5} Lit. \fbib{time}} of punishment, \v{6}therefore as I'm alive and living,'' declares the Lord \divine{God}, ``I'm turning you over to bloodshed,\fnote{\fbackref{35:6} The Heb. word \fbib{blood} sounds like \fbib{Edom}, the territory south of the Dead Sea in which Mt. Seir, the modern \fbib{Jebel esh-sher\'{a}}, is located} and bloodshed will certainly overtake you, since you never have hated shedding blood. That's why bloodshed will certainly pursue you. \v{7}I'm turning Mount Seir over to ruin and desolation. I'm going to eliminate everyone who comes and goes, \v{8}and I'll fill that\fnote{\fbackref{35:8} Lit. \fbib{his}} mountain with the dead. Those who die by violence\fnote{\fbackref{35:8} Lit. \fbib{by the sword}} will cover your hills, and fill your valleys and all your ravines! \v{9}I will turn you into an everlasting wasteland, and your cities will never be inhabited. Then you'll learn that I am the \divine{Lord}!

\v{10}``Because you have claimed, `These two nations and these two lands are going to belong to me, and we will take possession of them, even though the \divine{Lord} is there,' \v{11}therefore as I'm alive and living'' declares the Lord \divine{God}, ``I'm going to deal with you as your anger deserves. When I judge you, I'll treat you like you did the Israelis\fnote{\fbackref{35:11} Lit. \fbib{did them}}---that is, with the same kind of envy that motivated your constant hatred of them. \v{12}That's how you'll know that I, the \divine{Lord}, have heard every loathsome, reviling thing that you've had to say against the mountains of Israel, such as, `They're desolate, and we'll eat them for dinner!' \v{13}Not only that, you've arrogantly reviled me many times over, and I've heard every word!

\v{14}``So this is what the Lord \divine{God} says: `Just as the earth rejoices, I'm going to turn you into a desolate wasteland. \v{15}Just as you rejoiced when Israel's inheritance became desolate, I'm going to do the same thing to you. Mount Seir, you and Edom---all of you---will become a desolate wasteland.' Then they will learn that I am the \divine{Lord}.''
\labelchapt{36}
\passage{Prophecy to Israel's Mountains}

\chapt{36}
\v{1}``Now as for you, Son of Man, prophesy to Israel's mountains and tell them, `Listen to this message from the \divine{Lord}, you mountains of Israel: \v{2}``This is what the Lord \divine{God} says: `The enemy has been saying about you, ``Good! The ancient heights are back in our possession!''\,'\,''\,'

\v{3}``Therefore this is what you are to prophesy: `Here's what the Lord \divine{God} says, ``You've been turned into a desolate wasteland and crushed by everyone who surrounds you for a very, very good reason. You've become the property of all the other nations and you've become the object of gossip and whispering campaigns of the nations.''\,'\,''

\v{4}``Therefore listen to what the Lord \divine{God} has to say, you mountains of Israel: `This is what the Lord \divine{God} says to the mountains, to the hills, to the waterways, to the valleys, to the desolate wastelands, and to the abandoned cities that have become an object of derision to the mountains that surround you: \v{5}```Because this is what the Lord \divine{God} says: ``Motivated by my burning zealousness, I have spoken against the rest of the surrounding nations, including Edom, who confiscated my land, taking possession of it with joyful enthusiasm and with animal-like malice, in order to plunder Israel's\fnote{\fbackref{36:5} Lit. \fbib{her}} pastures.' \v{6}``Therefore prophesy concerning the land of Israel and speak to its mountains, hills, ravines, and valleys. Tell them, `This is what the Lord \divine{God} says: ``Pay attention! In my zealous anger I'm speaking because you've had to endure the mockery of the world's nations.''\,'\,''

\v{7}``Therefore this is what the Lord \divine{God} says: ``I hereby raise my hand to swear this oath: the nations that surround you will have their own mockery to endure! \v{8}But you mountains of Israel are going to sprout branches and bear fruit for my people Israel, because they'll be coming soon.'\,''
\passage{The Future of Israel's Mountains}

\v{9}``Watch me! I'm on your side! I'll be turning in your direction, and you mountains\fnote{\fbackref{36:9} The Heb. lacks \fbib{mountains}} will be plowed and planted. \v{10}I'm going to make the entire house of Israel grow---every single member\fnote{\fbackref{36:10} Lit. \fbib{every human being}}---and their cities will be inhabited with all the ruins rebuilt. \v{11}I'll make both the population and the livestock increase throughout your territories. They'll increase and be fruitful. I'll make your territories to be settled like you were in the past, and you will be treated better than you ever were before. At that time you will know that I am the \divine{Lord}.

\v{12}``I'll lead my people, my nation of Israel, across you mountains,\fnote{\fbackref{36:12} The Heb. lacks \fbib{mountains}} and they will take possession of you again, and you'll be their inheritance once more. Never again will you leave them robbed of children.

\v{13}``This is what the Lord \divine{God} says: `Because some have been saying to you, ``You mountains\fnote{\fbackref{36:13} The Heb. lacks \fbib{mountains}} have been devouring human beings and leaving people childless,'' \v{14}therefore you will no longer devour human beings or leave their nation childless,' declares the Lord \divine{God}. \v{15}`I won't let you hear other people mock you, and no nation will ever make you childless again,' declares the Lord \divine{God}.''
\passage{Israel's Past Punishments}

\v{16}This message came to me from the \divine{Lord}: \v{17}``Son of Man, when the house of Israel was living on their own land, they defiled it with their lifestyles\fnote{\fbackref{36:17} Lit. \fbib{ways}} and behavior; they were as disqualified to be with me as a menstruating woman is to you.\fnote{\fbackref{36:17} The Heb. lacks \fbib{is to you}} \v{18}So I poured out my anger on them because of all the bloodshed throughout the land and because they had defiled it with their idolatry. \v{19}Then I scattered them among the nations, dispersing them to other lands, just as their lifestyles and behavior deserved. That's how I judged them. \v{20}Nevertheless, when they arrived in those nations, they continued to profane my holy name. It was said about them, `These are the \divine{Lord}'s people, even though they've left his land.' \v{21}I've been concerned about my holy reputation,\fnote{\fbackref{36:21} Lit. \fbib{name}; and so throughout the chapter} which the house of Israel has been defiling throughout all of the nations where they've gone.''
\passage{The Coming Renewal of Israel}

\v{22}``Therefore tell the house of Israel, `This is what the Lord \divine{God} says: ``I'm not about to act for your sake, you house of Israel, but for the sake of my holy reputation, which you have been defiling throughout all of the nations where you've gone. \v{23}I'm going to affirm\fnote{\fbackref{36:23} Or \fbib{consecrate}} my great reputation that has been defiled among the nations (that is, that you have defiled in their midst), and those people will learn that I am the \divine{Lord},'' declares the Lord \divine{God}, ``when I affirm my holiness in front of their very eyes. \v{24}I'm going to remove you from the nations, gather you from all of the territories, and bring you all back to your own land. \v{25}I'll sprinkle pure water on you all, and you'll be cleansed from your impurity and from all of your idols.''

\v{26}`````I'm going to give you a new heart, and I'm going to give you a new spirit within all of your deepest parts. I'll remove that rock-hard heart of yours\fnote{\fbackref{36:26} Lit. \fbib{heart from your flesh}} and replace it with one that's sensitive to me.\fnote{\fbackref{36:26} Lit. \fbib{with a heart of flesh}} \v{27}I'll place my spirit within you, empowering you to live according to\fnote{\fbackref{36:27} Lit. \fbib{to walk in}} my regulations and to keep my just decrees. \v{28}You'll live in the land that I gave to your ancestors, you'll be my people, and I will be your God. \v{29}In addition, I'll deliver you from everything that makes you unclean. I'll call out to the grain you plant, ordering it to produce abundant yields, and I will never bring famine in your direction. \v{30}I'll increase the yields of your fruit trees and crops so that you'll never again experience the disgrace of famine that occurs in other nations. \v{31}Then you'll remember your lifestyles and practices that were not good. You'll hate yourselves as you look at your own iniquities and loathsome practices. \v{32}I won't be doing any of this for your sake,'' declares the Lord \divine{God}, ``so keep that in mind. Be ashamed and frustrated because of your behavior, you house of Israel!''\,'\,''
\passage{The Restoration of Israel's Cities}

\v{33}``This is what the Lord \divine{God} says: `At the same time\fnote{\fbackref{36:33} Lit. \fbib{On the day}} that I cleanse you from all of your guilt, I'll make your cities become inhabited again and the desolate wastelands will be rebuilt. \v{34}The desolate fields will be cultivated, replacing the former wasteland that everyone who passed by in times past\fnote{\fbackref{36:34} The Heb. lacks \fbib{in times past}} had noticed. \v{35}They will say, ``This wasteland has become like the garden of Eden, and what used to be desolate ruins are now fortified and inhabited.'' \v{36}Then the surviving people that live around you will learn that I, the \divine{Lord}, have rebuilt these ruins and have cultivated these pastures that used to be desolate wastelands. I, the \divine{Lord}, have spoken this, and I'm going to bring it about!'

\v{37}``This is what the Lord \divine{God} has to say: `I'm going to allow the house of Israel to ask anything they want from me, including this: I'm going to increase their population as a shepherd increases his flock. \v{38}The desolate cities will be filled with flocks of human beings, just like Jerusalem used to be filled with flocks of sheep during the times of the appointed festivals. Then they will know that I am the \divine{Lord}.'\,''
\labelchapt{37}
\passage{The Vision of the Valley of Bones}

\chapt{37}
\v{1}The \divine{Lord} laid his hand on me and brought me out by the Spirit of the \divine{Lord} to the middle of a valley that was filled with bones. \v{2}He led me here and there throughout\fnote{\fbackref{37:2} Lit. \fbib{me over them all around all around}} the valley, and I was amazed to see that the surface of the entire valley was covered with myriads of very dry bones! \v{3}The \divine{Lord}\fnote{\fbackref{37:3} Lit. \fbib{He}} asked me, ``Son of Man, will these bones ever live?''

``Lord \divine{God},'' I replied, ``you know the answer to that!''\fnote{\fbackref{37:3} The Heb. lacks \fbib{the answer to that}}

\v{4}Then the \divine{Lord} told me, ``Prophesy to these bones. Tell them: `You dry bones, listen to the message from the \divine{Lord}: ``\v{5}This is what the Lord \divine{God} says to you\fnote{\fbackref{37:5} Lit. \fbib{these}} dry bones! `Pay attention! I'm bringing my Spirit into you right now, and you're going to live! \v{6}I'm going to grow tendons on you, regenerate your flesh, cover you with skin, and make you breathe again so that you can come back to life and learn that I am the \divine{Lord}.'\,''\,'\,''
\passage{The Bones are Raised to Life}

\v{7}So I prophesied, just as I had been ordered to do so. Immediately there was a noise and a rattling---and then all of a sudden the bones came together by themselves! Each bone came together, all of them attached together!\fnote{\fbackref{37:7} Lit. \fbib{together, one to another}} \v{8}As I continued to watch, I saw tendons growing on the bones,\fnote{\fbackref{37:8} Lit. \fbib{on them}} and muscles growing and covering them, and then skin covered the flesh from above. But the bodies weren't breathing. \v{9}Then he ordered me, ``Prophesy to the Spirit, Son of Man. Tell the Spirit, `This is what the Lord \divine{God} says: ``Come from the four winds, you Spirit, and breathe into these people who have been killed, so they will live.''\,'\,'' \v{10}So I prophesied as I had been ordered, breath entered them, and they began to live. They stood on their own feet as a vast, united army.
\passage{The Vision is Interpreted for Ezekiel}

\v{11}``These bones represent the entire house of Israel,'' the \divine{Lord}\fnote{\fbackref{37:11} Lit. \fbib{Israel,'' he}} explained to me. ``Look how they keep saying, `Our bones are dried up, and our future is lost. We've been completely eliminated!' \v{12}``Therefore prophesy to them, and tell them, `Watch me! I'm going to open your graves, lift you out of those graves, and bring my people back into the land of Israel. \v{13}Then you'll learn that I am the \divine{Lord}, when I've opened your graves and caused you to come up out of them, my people. \v{14}I'm going to place my Spirit in you all, and you will live. I'll place you all into your land, and you'll learn that I, the \divine{Lord}, have been speaking and doing this,' declares the \divine{Lord}.'\,''
\passage{The Future Union of Israel and Judah}

\v{15}A message came to me from the \divine{Lord}, and this is what it was: \v{16}``Now as for you, Son of Man, grab a stick of wood for yourself and write on it these words:

\begin{poetry}
\poeml `\divine{For Judah and the Israelis, his companions}'
\end{poetry}

``Then grab another stick and write on it:

\begin{poetry}
\poeml `\divine{For Joseph, the stick of Ephraim, and all the house of Israel, his companions}'
\end{poetry}

\v{17}``Then join them together end-to-end so that they become a single baton in your hand. \v{18}When the descendants of your people ask you, `Would you please explain to us what you mean by this?' \v{19}you are to tell them, `This is what the \divine{Lord} says: ``Watch me! I'm taking the baton that represents Joseph, which Ephraim is holding in his hand, along with his companions the tribes of Israel, and I'm going to join them with the baton that represents Judah. I'm making them a single baton, that is, a complete baton in my hand.''

\v{20}``The batons on which you engrave your writing are to remain right in front of them in your hand. \v{21}Then tell them, `This is what the Lord \divine{God} says: ``Watch me take the Israelis out of the nations where they've gone and return them from every direction. I'm going to bring them back into their own land. \v{22}I'm going to make them a united people in the land, on the mountains of Israel, and I'll set a single king to rule over them. They'll never again be two separate people. They'll never again be divided into two kingdoms. \v{23}They will never again defile themselves with their idols, with other loathsome things, or with any of their sins. Instead, I will deliver them from all of the places where they have sinned, and then I'll cleanse them. They will be my people and I will be their God.''\,'\,''
\passage{David's Rule as King}

\v{24}`````My servant King David will be there for them, and one shepherd will be appointed for them. They will live according to my decrees, keep my regulations, and practice them. \v{25}They will live in the land that I gave to my servant Jacob and on which your ancestors lived. They will live in that land, along with their children and grandchildren, forever. David my servant will be their everlasting leader. \v{26}I'll make a secure covenant\fnote{\fbackref{37:26} Or \fbib{a covenant of peace}} with them, one that will last forever. I will establish them, make them increase in population,\fnote{\fbackref{37:26} The Heb. lacks \fbib{in population}} and will place my sanctuary in their midst forever. \v{27}I will pitch my tent among them and will be their God. They will be my people, \v{28}and the nations will learn that I, the \divine{Lord}, am the sanctifier of Israel when I place my sanctuary in their midst forever.''\,'\,''
\labelchapt{38}
\passage{The Prophecy against Gog}

\chapt{38}
\v{1}This message from the \divine{Lord} came to me: \v{2}``Son of Man, turn your attention toward Gog,\fnote{\fbackref{38:2} I.e. a mountain tribe north of Assyria, and so through chapter 39} from the land of Magog,\fnote{\fbackref{38:2} I.e. a son of Noah's son Japheth; the area includes what is now modern eastern Turkey} leader of the head\fnote{\fbackref{38:2} Or \fbib{of Rosh,}} of Meshech,\fnote{\fbackref{38:2} I.e. a son of Noah's son Japheth; this people resided in what is now modern Armenia} and of Tubal.\fnote{\fbackref{38:2} I.e. a son of Noah's son Japheth; the area includes what is now modern eastern Turkey} Prophesy this against him: \v{3}`This is what the Lord \divine{God} says: ``Watch out! I'm coming after you, Gog, leader of the head\fnote{\fbackref{38:3} Or \fbib{of Rosh,}} of Meshech,\fnote{\fbackref{38:3} I.e. a son of Noah's son Japheth; this people resided in what is now modern Armenia} and of Tubal.\fnote{\fbackref{38:3} I.e. a son of Noah's son Japheth; this people resided in what is now modern eastern Turkey} \v{4}I'm going to turn you around, put hooks into your jaws, and bring you out---you and your whole army---along with your horses and cavalry riders, all of them richly attired, a magnificent company replete with buckler and shield, and all of them wielding battle swords. \v{5}Persia,\fnote{\fbackref{38:5} I.e. the area includes what is now modern Iran} Cush,\fnote{\fbackref{38:5} I.e. this area includes what is now modern Ethiopia and Somalia} and Libya\fnote{\fbackref{38:5} Lit. \fbib{Put}; the Heb. name means \fbib{bow}} will be accompanying them, all of them equipped with shields and helmets. \v{6}Gomer\fnote{\fbackref{38:6} I.e. a son of Noah's son Jepheth; the area encompasses what is now modern Turkey, Iran, Afghanistan, and Iraq.} with all its troops, and the household of Togarmah\fnote{\fbackref{38:6} I.e. named after Gomer, the region encompasses what is now Armenia} from the remotest parts of the north with all its troops---many people will accompany you. \v{7}Be prepared. Yes, prepare yourself---you and all of your many battalions that have gathered together around you to protect you.

\v{8}`````Many days from now---in the latter years---you will be summoned to a land that has been restored from violence.\fnote{\fbackref{38:8} Lit. \fbib{from the sword}} You will be gathered from many nations to the mountains of Israel, which formerly had been a continuous waste, but which will be populated with people who have been brought back from the nations. All of them will be living there securely. \v{9}You'll arise suddenly, like a tornado, coming like a windstorm\fnote{\fbackref{38:9} Or \fbib{cloud}} to cover the land, you and all your soldiers with you, along with many nations.''\,'\,''
\passage{The Invasion Strategy}

\v{10}``This is what the Lord \divine{God} says: `This is what's going to happen on the very day that you begin your invasion: You'll be thinking,\fnote{\fbackref{38:10} Lit. \fbib{thinking in your heart}} making evil plans, \v{11}and boasting, ``I'm going to invade a land comprised of open country\fnote{\fbackref{38:11} I.e. territory characterized by a lack of military fortifications} that is at rest, its people\fnote{\fbackref{38:11} The Heb. lacks \fbib{its people}} living confidently, all of whose inhabitants will be living securely, with neither fortification nor bars on their doors. \v{12}I'm going to confiscate anything I can put my hands on. I'll attack the restored ruins and the people who have been gathered together from the nations, who are acquiring livestock and other goods, and who live at the center of the world's attention.''\fnote{\fbackref{38:12} The Heb. lacks `\fbib{s attention}} \v{13}`Businessmen based in Sheba,\fnote{\fbackref{38:13} I.e. what is now southwest Saudi Arabia} Dedan,\fnote{\fbackref{38:13} I.e. what is now southern Saudi Arabia} Tarshish,\fnote{\fbackref{38:13} I.e. a city accessible from the Red Sea to which ships based on the Elanitic Gulf could sail} and all of its growling lions will ask you, ``Are you coming for war spoils? Have you assembled your armies to carry off silver and gold, and to gather lots of war booty?''\,'\,''
\passage{God's Rebuke to Gog}

\v{14}``Therefore, Son of Man, prophesy to Gog and tell him, `This is what the Lord \divine{God} says: ``When the day comes when my people are living securely, won't you be aware of it? \v{15}You'll come in from your home\fnote{\fbackref{38:15} Lit. \fbib{place}} in the remotest parts of the north. You'll come with many nations, all of them riding along on horses. You'll be a huge, combined army. \v{16}You'll come up to invade my people Israel like a storm cloud to cover the land. In the last days, Gog, I'll bring you up to invade my land so that the world will learn to know me when I show them how holy I am before their very eyes.''\,'\,''
\passage{A Prediction for the Distant Future}

\v{17}``This is what the Lord \divine{God} says: `Surely you're the one about whom I spoke years ago in the writings\fnote{\fbackref{38:17} Lit. \fbib{ago by the hand}} of my servants, Israel's prophets, aren't you? They predicted back then that I would bring you up after many years, didn't they? \v{18}So it will be that on that day, when Gog\fnote{\fbackref{38:18} I.e. a mountain tribe north of Assyria} invades the land of Israel,' declares the Lord \divine{God}, `my zeal will ignite my anger. \v{19}Because of my zeal and burning anger, at that time\fnote{\fbackref{38:19} Lit. \fbib{day}} there will be a massive earthquake throughout the land of Israel. \v{20}I'm going to shake the fish in the sea, the birds in the sky, the wild beasts, all the creatures that crawl on the earth, and every single human being who lives on the surface of the earth. Mountains will collapse, as will their mountain passages, and every wall will fall to the ground. \v{21}Then I'll call for war against Gog\fnote{\fbackref{38:21} I.e. a mountain tribe north of Assyria} on top of every mountain,' declares the Lord \divine{God}, `and every weapon of war will be turned against their fellow soldier. \v{22}I'll judge them with disease and bloodshed. I'll shower him, his soldiers, and the vast army that accompanies him with a torrential flood, hailstones, fire, and sulfur. \v{23}I will exalt myself and demonstrate my holiness, making myself known to many people, who will learn that I am the \divine{Lord}.'\,''
\labelchapt{39}
\passage{The Destruction of Gog}

\chapt{39}
\v{1}``Now as for you, Son of Man, prophesy against Gog\fnote{\fbackref{39:1} I.e. a mountain tribe north of Assyria} and tell him, `This is what the Lord \divine{God} has to say: ``Watch out, Gog, you leader of the head\fnote{\fbackref{39:1} Or \fbib{of Rosh,}} of Meshech and of Tubal! \v{2}I'm going to turn you around, drag you along to your destruction,\fnote{\fbackref{39:2} So LXX; MT reads \fbib{around, lead you}} and bring you up from the farthest parts of the north, and carry you to the mountains of Israel. \v{3}There I will strike your bow from your left hand and your arrows from your right, causing your fall. \v{4}You will collapse on the mountains of Israel, along with all of your soldiers and the nations that have accompanied you. There the raptors, vultures,\fnote{\fbackref{39:4} Or \fbib{carrion feeders}} and wild animals will feed on you. \v{5}You will fall dead in the open fields, because I have ordered this to happen,' declares the Lord \divine{God}. \v{6}`I'm also going to incinerate Magog, along with those who are settled down and at home in the islands. That's when they'll learn that I am the \divine{Lord}. \v{7}I'll make my holiness and reputation\fnote{\fbackref{39:7} Lit. \fbib{name}} known in the midst of my people Israel, and I won't let my holiness be profaned anymore. The nations will learn that I, the \divine{Lord}, am holy in the midst of Israel. \v{8}Pay attention! It's coming and will certainly happen,' declares the Lord \divine{God}. `This will be the day about which I've spoken!'\,''\,'\,''
\passage{The Aftermath of the Battle}

\v{9}``After all this happens, the people who live in the cities of Israel will be kindling fires for seven years, using small shields, large shields, bows, arrows, clubs, personal weapons, and spears to do so. \v{10}They won't need to cut down trees from the fields nor gather firewood from the forests, because they will light fires with the weapons as they plunder the plunderers and loot the looters!'' declares the Lord \divine{God}. \v{11}``When all of this happens, I'm going to set aside a grave site for Gog in Israel's Traveler's Valley,\fnote{\fbackref{39:11} Lit. \fbib{in the Crossover Valley}; or \fbib{Israel, the valley where people cross}} near the approach\fnote{\fbackref{39:11} I.e. to the north, as one travels from Jerusalem} to the Dead Sea. She\fnote{\fbackref{39:11} MT does not identify the woman} will block off everyone who tries to bypass it. There they will bury Gog, and rename the area `Valley of Gog's Gang'.\fnote{\fbackref{39:11} Lit. \fbib{Gog's Crowd'} and so in v. 15} \v{12}The house of Israel will be burying them for seven months in order to purify the land. \v{13}Everyone in the land will be involved in the burials, and this will serve as a reminder for them that I have glorified myself,'' declares the Lord \divine{God}. \v{14}``Men will be assigned to travel continuously throughout the land, exploring for seven full months as they go about burying the bodies that remain from the battle\fnote{\fbackref{39:14} The Heb. lacks \fbib{from the battle}} on the surface of the ground, so that the land may be sterilized. \v{15}As scouts go searching throughout the land, whenever they see someone's bones, they will place a sign beside the remains until the remains have been buried in the Valley of Gog's Gang. \v{16}They'll also name the city that is there `Hamonah,'\fnote{\fbackref{39:16} The Heb. name means \fbib{The Crowds}} as they purify the land.''
\passage{An Invitation to Dine on Human Flesh}

\v{17}``Now as for you, Son of Man, this is what the Lord \divine{God} has to say: `Tell all of the birds and wild beasts, ``Come! Gather together and participate in the sacrifice that I'm going to make for you. This great sacrifice will take place on the mountains of Israel, where you'll be eating flesh and drinking blood. \v{18}You'll eat the flesh of mighty men and drink the blood of the world's princes, drinking the blood of these rams, lambs, goats, bulls, all of them fattened as if they're from Bashan, fit for slaughter! \v{19}You'll eat until you're fat and satiated. You'll drink blood until you're drunk from the sacrifice that I'm going to make for all of you. \v{20}You'll be fully satiated at my table, dining on\fnote{\fbackref{39:20} The Heb. lacks \fbib{dining on}} horse flesh, horsemen, elite soldiers, and every kind of warrior,'' declares the Lord \divine{God}. \v{21}`I'm going to display my glory among the people, and every nation will see the judgment that I administer by my own hand among them. \v{22}The house of Israel will learn that I am the \divine{Lord} their God from that day forward! \v{23}The nations will also learn that because of Israel's sin the house of Israel went into captivity, since they were unfaithful in their behavior toward me. As a result, I hid my presence from them, turned them over to the control of their enemies, and they died by violence.\fnote{\fbackref{39:23} Lit. \fbib{they fell by the sword}} \v{24}It was because of their defilement and transgression that I treated them this way by hiding my presence from them.'\,''
\passage{The Final Restoration of Israel}

\v{25}``Therefore this is what the Lord \divine{God} has to say: `I'm going to restore the resources of Jacob and show mercy to the entire house of Israel. I'll be zealous for my own reputation\fnote{\fbackref{39:25} Lit. \fbib{name}} and for my holiness. \v{26}They'll forget their shame and all of their unfaithfulness by which they behaved so unfaithfully toward me. They will live on their land in confidence, not in fear. \v{27}When I bring them back from the nations and gather them together from the lands that belong to their enemies, I will demonstrate my holiness through them right in front of the eyes of the world,\fnote{\fbackref{39:27} Lit. \fbib{of many peoples}} \v{28}and they will learn that I am the \divine{Lord} their God, who sent them into exile and who gathered them back to their land. I will not leave any of them remaining in exile. \v{29}I'll no longer hide my presence from them again when I pour out my Spirit on the house of Israel,' declares the Lord \divine{God}.''
\labelchapt{40}
\passage{The Vision of Jerusalem}

\chapt{40}
\v{1}At the beginning of year 25 of our captivity, on the tenth day of the fourteenth year after the destruction of Jerusalem\fnote{\fbackref{40:1} Lit. \fbib{of the city}}---on that very day---the \divine{Lord} grabbed me in his hand and took me there. \v{2}God brought me in a series of visions to the land of Israel and placed me on top of a very high mountain, where to the south there was something that looked like the outline of a city. \v{3}That's where he took me. All of a sudden, there was a man whose appearance resembled glowing\fnote{\fbackref{40:3} The Heb. lacks \fbib{glowing}} bronze! He had a measuring reed and line in his hand as he stood in the city gate. \v{4}This is what the man told me: ``Son of Man, watch carefully,\fnote{\fbackref{40:4} Lit. \fbib{watch with your own eyes}} listen closely,\fnote{\fbackref{40:4} Lit. \fbib{listen with your own ears}} and remember\fnote{\fbackref{40:4} Lit. \fbib{and put in your heart}} everything I'm going to be showing you, because you've been brought here to be shown what you're about to see. Be sure that you tell the house of Israel everything that you observe.''
\passage{Measuring the Temple Grounds}

\v{5}All of a sudden, we were at the exterior wall that completely surrounded the Temple. The man whom I had observed held a measuring reed that was six cubits\fnote{\fbackref{40:5} I.e. about 10.5 feet, given the designated measurement in royal cubits, about 21 inches} long as measured in cubits that were a cubit and a handbreadth\fnote{\fbackref{40:5} I.e. the royal cubit, which measured about 21 inches} long. As he measured the thickness of the wall, he measured out one reed.\fnote{\fbackref{40:5} I.e. about 10.5 feet; the reed was six royal cubits} Its height was also one reed.\fnote{\fbackref{40:5} I.e. about 10.5 feet, ; the reed was six royal cubits} \v{6}Then he went over to the gate that faced toward the east, ascended its steps, and measured its thresholds. One threshold measured one reed\fnote{\fbackref{40:6} I.e. about 10.5 feet; the reed was six royal cubits} and the other one measured one reed.\fnote{\fbackref{40:6} I.e. about 10.5 feet; the reed was six royal cubits} \v{7}Each guardhouse\fnote{\fbackref{40:7} Or \fbib{alcove}; and so throughout the chapter} measured one reed\fnote{\fbackref{40:7} I.e. about 10.5 feet; the reed was six royal cubits} long and one reed\fnote{\fbackref{40:7} I.e. about 10.5 feet; the reed was six royal cubits} wide, and the distance\fnote{\fbackref{40:7} The Heb. lacks \fbib{the distance}} between each guardhouse was five cubits.\fnote{\fbackref{40:7} I.e. about 8.75 feet, given the designated measurement in royal cubits, about 21 inches} The threshold of the gate near the vestibule facing away from the Temple entrance\fnote{\fbackref{40:7} The Heb. lacks \fbib{entrance}} measured one reed.\fnote{\fbackref{40:7} I.e. about 10.5 feet; the reed was six royal cubits}

\v{8}Next, he measured the vestibule of the gate facing away from the Temple entrance at one reed.\fnote{\fbackref{40:8} I.e. about 10.5 feet; the reed was six royal cubits} \v{9}He measured the vestibule of the gate inside at eight cubits\fnote{\fbackref{40:9} I.e. about 14 feet; the royal cubit was 21 inches} and the doorjambs at two cubits.\fnote{\fbackref{40:9} I.e. about 42 inches; the royal cubit was 21 inches} (The vestibule at the gate faced away from the Temple.) \v{10}Gate guardhouses stood facing east, numbering three on each side,\fnote{\fbackref{40:10} Lit. \fbib{three from here and there}} each of them of equal size\fnote{\fbackref{40:10} Lit. \fbib{from here one measurement}} to the door jamb; that is, having the same\fnote{\fbackref{40:10} The Heb. lacks \fbib{that is, having the same}} measurement on each side.\fnote{\fbackref{40:10} Lit. \fbib{measurement from here and from here}} \v{11}He measured the width of the gateway at ten cubits,\fnote{\fbackref{40:11} I.e. about 17.5 feet; the royal cubit was 21 inches} and the length of the gate at thirteen cubits.\fnote{\fbackref{40:11} I.e. about 22.75 feet; the royal cubit was 21 inches}

\v{12}The retaining\fnote{\fbackref{40:12} Lit. \fbib{border}; or \fbib{barrier}} wall in front of the guardhouses measured one cubit\fnote{\fbackref{40:12} I.e. about 21 inches; the royal cubit was 21 inches} wide. It stood one cubit\fnote{\fbackref{40:12} I.e. about 21 inches; the royal cubit was 21 inches} from the wall to the guardhouses, which were six cubits\fnote{\fbackref{40:12} I.e. about 10.5 feet; the royal cubit was 21 inches} square.\fnote{\fbackref{40:12} Lit. \fbib{were six cubits from here and six cubits from here.}} \v{13}He measured the gate from the roof of the guardhouses to the roof of another\fnote{\fbackref{40:13} The Heb. lacks \fbib{of another}} at 25 cubits\fnote{\fbackref{40:13} I.e. about 43.75 feet; the royal cubit was 21 inches} from doorway to opposite doorway. \v{14}Then he measured\fnote{\fbackref{40:14} Lit. \fbib{made}} the open air porch\fnote{\fbackref{40:14} So LXX; MT reads \fbib{the jamb}} at 60 cubits\fnote{\fbackref{40:14} I.e. about 106.75 feet; the royal cubit was 21 inches} from the doorjamb of the courtyard that encompassed the gate. \v{15}The distance from the front entrance gate to the vestibule of the inner gate measured 50 cubits.\fnote{\fbackref{40:15} I.e. about 87.5 feet; the royal cubit was 21 inches} \v{16}Latticed windows faced the guardhouses, their side pillars within the gate all around, and also for the porches. Windows were placed all around inside, and the side pillars were engraved with palm trees.
\passage{The Outer Court}

\v{17}Next, he brought me into the outer court, where chambers and a paved area had been constructed all around the courtyard, with 30 chambers facing the pavement. \v{18}The pavement to the side\fnote{\fbackref{40:18} Or \fbib{The lower pavement}} of the gates corresponded to the length of the gates. \v{19}He also measured the width from the front lower gate to the front of the exterior inner court at 100 cubits\fnote{\fbackref{40:19} I.e. about 175 feet; the royal cubit was 21 inches} to the east and to the north.
\passage{The North Facing Outer Court}

\v{20}Next, he measured the length and width of the outer north-facing gate to the courtyard. \v{21}It was equipped\fnote{\fbackref{40:21} The Heb. lacks \fbib{was equipped}} with three guardhouses on each side. Its side pillars and porches had measurements identical to the first gate: 50 cubits\fnote{\fbackref{40:21} I.e. about 87.5 feet; the royal cubit was 21 inches} long and 25 cubits\fnote{\fbackref{40:21} I.e. about 43.75 feet; the royal cubit was 21 inches} wide. \v{22}Its windows, porches, and palm tree ornaments had measurements identical to the east-facing gate. Reached by seven ascending\fnote{\fbackref{40:22} The Heb. lacks \fbib{ascending}} steps, its porch lay\fnote{\fbackref{40:22} The Heb. lacks \fbib{lay}} to the front of the steps. \v{23}From a gate that stood opposite the northern gate he measured 100 cubits,\fnote{\fbackref{40:23} I.e. about 175 feet; the royal cubit was 21 inches} as well as from the eastern gate.
\passage{The South Facing Gate}

\v{24}Then he led me toward the south, where there was a gate with side pillar and porch measurements identical to the others. \v{25}The gate and its porches contained windows all around, identical to the other windows. The length of the porch\fnote{\fbackref{40:25} The Heb. lacks \fbib{of the porch}} was 50 cubits\fnote{\fbackref{40:25} I.e. about 87.5 feet; the royal cubit was 21 inches} and its width was 25 cubits.\fnote{\fbackref{40:25} I.e. about 43.75 feet; the royal cubit was 21 inches} \v{26}Seven steps led up to it, with a porch in front of them. Palm tree ornaments were engraved on its side pillars, one on each side. \v{27}The inner court contained a south-facing gate measuring 100 cubits\fnote{\fbackref{40:27} I.e. about 175 feet; the royal cubit was 21 inches} from gate to gate toward the south.
\passage{The Inner Southern Court}

\v{28}Next, he brought me to the inner courtyard by way of the south-facing gate. He measured the south-facing gate as having measurements identical to the others. \v{29}The measurements of its guardhouses, its side pillars, and its porches were identical to the others. The gate and its porches contained windows all around. The length of the porch\fnote{\fbackref{40:29} The Heb. lacks \fbib{of the porch}} was 50 cubits\fnote{\fbackref{40:29} I.e. about 87.5 feet; the royal cubit was 21 inches} and its width was 25 cubits.\fnote{\fbackref{40:29} I.e. about 43.75 feet; the royal cubit was 21 inches} \v{30}Porches lay all around, measuring 25 cubits\fnote{\fbackref{40:30} I.e. about 43.75 feet; the royal cubit was 21 inches} long and five cubits\fnote{\fbackref{40:30} I.e. about 8.75 feet; the royal cubit was 21 inches} wide, \v{31}leading to the outer courtyard. Palm tree ornaments were engraved on its side pillars. The stairway leading to it contained eight steps.
\passage{The Inner Eastern Court}

\v{32}Then he brought me into the inner east-facing courtyard, where he measured the gate, identical to the others. \v{33}The measurement of its guardhouses, side pillars, and porches was identical to the others. The gate and its porches contained windows all around. The length of the porch\fnote{\fbackref{40:33} The Heb. lacks \fbib{of the porch}} was 50 cubits\fnote{\fbackref{40:33} I.e. about 87.5 feet; the royal cubit was 21 inches} and its width was 25 cubits,\fnote{\fbackref{40:33} I.e. about 43.75 feet; the royal cubit was 21 inches} \v{34}leading to the outer courtyard. Palm tree ornaments were engraved on its side pillars. The stairway leading to it contained eight steps.
\passage{The North Facing Gate}

\v{35}Next, he brought me to the north-facing gate, where he measured the gate, identical to the others. \v{36}The measurement of its guardhouses, side pillars, and porches was identical to the others. The gate and its porches contained windows all around. The length of the porch\fnote{\fbackref{40:36} The Heb. lacks \fbib{of the porch}} was 50 cubits\fnote{\fbackref{40:36} I.e. about 87.5 feet; the royal cubit was 21 inches} and its width was 25 cubits,\fnote{\fbackref{40:36} I.e. about 43.75 feet; the royal cubit was 21 inches} \v{37}leading to the outer courtyard. Palm tree ornaments were engraved on its side pillars. The stairway leading to it contained eight steps. \v{38}There was a chamber with a doorway by the side pillars next to the gate where they prepare\fnote{\fbackref{40:38} Lit. \fbib{rinse}} the burnt offerings.

\v{39}In the porch leading in front of the gate there were two tables on either side for slaughtering burnt offerings, sin offerings, and guilt offerings, \v{40}and on the outer side, approaching the northern gateway, there were two tables, as well as two tables on the opposite side of the porch in front of the gate. \v{41}In that way, there were four tables on each side in front of the gate, for a total of eight tables for use in slaughtering the offerings.\fnote{\fbackref{40:41} The Heb. lacks \fbib{the offerings}}

\v{42}There were four tables carved from stone for the burnt offering, each one and a half cubits\fnote{\fbackref{40:42} I.e. about 31.5 inches; the royal cubit was 21 inches} long, one and a half cubits\fnote{\fbackref{40:42} I.e. about 31.5 inches; the royal cubit was 21 inches} wide, and one cubit\fnote{\fbackref{40:42} I.e. about 21 inches; the royal cubit was 21 inches} high, on which the instruments are laid for slaughtering burnt offerings and sacrifices. \v{43}Double hooks, a single handbreadth\fnote{\fbackref{40:43} I.e. about 3 inches} in length, were installed all around in this portion of\fnote{\fbackref{40:43} The Heb. lacks \fbib{this portion of}} the temple area.
\passage{The Inner Gate}

\v{44}From outside leading into the inner gate there were chambers for the choir. One was beside the north gate facing the south, and another was at the south gate facing the north. \v{45}The angel\fnote{\fbackref{40:45} Lit. \fbib{He}} told me, ``This south-facing chamber is for the priests who maintain the Temple, \v{46}and the north-facing chamber is for the priests who maintain the altar. These are Zadok's descendants, who, as descendants of Levi approach the \divine{Lord} to minister directly to him.'' \v{47}He measured the court in the form of a square at 100 cubits\fnote{\fbackref{40:47} I.e. about 175 feet; the royal cubit was 21 inches} long and 100 cubits\fnote{\fbackref{40:47} I.e. about 175 feet; the royal cubit was 21 inches} wide. The altar stood in front of the Temple.
\passage{The Temple Porch}

\v{48}Next, he brought me to the Temple porch and measured the side pillars at five cubits\fnote{\fbackref{40:48} I.e. about 8.75 feet; the royal cubit was 21 inches} on each side. The width of the gate measured three cubits\fnote{\fbackref{40:48} I.e. about 5.25 feet; the royal cubit was 21 inches} on each side. \v{49}The porch was 20 cubits\fnote{\fbackref{40:49} I.e. about 35 feet; the royal cubit was 21 inches} long and eleven cubits\fnote{\fbackref{40:49} I.e. about 19.25 feet; the royal cubit was 21 inches} wide. The stairway by which it was ascended was equipped with columns attached to its side pillars, one on each side.
\labelchapt{41}
\passage{The Vision of the Temple}

\chapt{41}
\v{1}Next he brought me to the Temple and measured its door jambs at six cubits\fnote{\fbackref{41:1} I.e. about 10.5 feet; the royal cubit was 21 inches} wide on each side of the structure.\fnote{\fbackref{41:1} Lit. \fbib{tent}} \v{2}The entrance was ten cubits\fnote{\fbackref{41:2} I.e. about 17.5 feet; the royal cubit was 21 inches} wide and its door jambs were five cubits\fnote{\fbackref{41:2} I.e. about 8.75 feet; the royal cubit was 21 inches} wide on each side. He measured the length of the nave at 40 cubits\fnote{\fbackref{41:2} I.e. about 35 feet; the royal cubit was 21 inches} and its width at 20 cubits.\fnote{\fbackref{41:2} I.e. about 70 feet; the royal cubit was 21 inches}

\v{3}Then he went inside and measured the door jambs at two cubits\fnote{\fbackref{41:3} I.e. about 42 inches; the royal cubit was 21 inches} wide and the doorway at six cubits\fnote{\fbackref{41:3} I.e. about 10.5 feet; the royal cubit was 21 inches} high. The doorway was seven cubits\fnote{\fbackref{41:3} I.e. about 12.25 feet; the royal cubit was 21 inches} wide. \v{4}He measured its length at 20 cubits,\fnote{\fbackref{41:4} I.e. about 35 feet; the royal cubit was 21 inches} its width at 20 cubits\fnote{\fbackref{41:4} I.e. about 35 feet; the royal cubit was 21 inches} in front of the structure,\fnote{\fbackref{41:4} I.e. the separation between the Holy Place and the most holy area} and then he told me, ``This is the most holy area.''

\v{5}Next, he measured the Temple walls at six cubits\fnote{\fbackref{41:5} I.e. about 10.5 feet; the royal cubit was 21 inches} high and the width of the side chambers at four cubits\fnote{\fbackref{41:5} I.e. about seven feet; the royal cubit was 21 inches} around all four sides of the Temple. \v{6}The side chambers consisted of three stories, each above the other, with 30 chambers in each story. The side chambers extended out from the wall that faced the inside of the chambers where the chambers were fastened together, but the chamber walls were not fastened directly into the Temple walls themselves. \v{7}The side chambers surrounding the Temple were wider at each successive story, because the surrounding structure ascended by proportional increments as it rose, ascending to the highest story by going up successively from the lowest.

\v{8}I observed a raised platform that surrounded the Temple, and the foundations of the side chambers were a full six cubits\fnote{\fbackref{41:8} I.e. about 10.5 feet; the royal cubit was 21 inches} deep. \v{9}The outer wall of the side chambers was five cubits\fnote{\fbackref{41:9} I.e. about 8.75 feet; the royal cubit was 21 inches} thick, and there was an empty space between the Temple's side chambers \v{10}and its outer chambers 20 cubits\fnote{\fbackref{41:10} I.e. about 35 feet; the royal cubit was 21 inches} in width, surrounding the Temple on each side. \v{11}The side chamber doorway facing the free space contained a single north-facing doorway and a second south-facing doorway. The width of the free space was five cubits\fnote{\fbackref{41:11} I.e. about 8.75 feet; the royal cubit was 21 inches} all around the perimeter.\fnote{\fbackref{41:11} The Heb. lacks \fbib{the perimeter}} \v{12}The building that faced the west side of the courtyard was 70 cubits\fnote{\fbackref{41:12} I.e. about 122.5 feet; the royal cubit was 21 inches} wide, and the building's wall was five cubits\fnote{\fbackref{41:12} I.e. about 8.75 feet; the royal cubit was 21 inches} thick all around. It was 90 cubits\fnote{\fbackref{41:12} I.e. about 157.5 feet; the royal cubit was 21 inches} long.
\passage{The Temple}

\v{13}Then he measured the Temple. It was 100 cubits\fnote{\fbackref{41:13} I.e. about 175 feet; the royal cubit was 21 inches} long, and the courtyard, its building, and its walls were 100 cubits\fnote{\fbackref{41:13} I.e. about 175 feet; the royal cubit was 21 inches} long. \v{14}The front of the Temple and its east-facing courtyard were each\fnote{\fbackref{41:14} The Heb. lacks \fbib{each}} 100 cubits\fnote{\fbackref{41:14} I.e. about 175 feet; the royal cubit was 21 inches} long. \v{15}Next, he measured 100 cubits\fnote{\fbackref{41:15} I.e. about 175 feet; the royal cubit was 21 inches} as the length of the structure toward the front of the courtyard that stood behind it, where it housed a gallery on each side of it. Then he measured the Temple and the inner porticos\fnote{\fbackref{41:15} Or \fbib{porches}} of the courtyard, \v{16}the thresholds, the shielded\fnote{\fbackref{41:16} Or \fbib{latticed}} windows, and the surrounding three-storied galleries that stood opposite. From the ground to the shielded\fnote{\fbackref{41:16} Or \fbib{latticed}} windows, they were paneled with wood all around, \v{17}including up to the doorway, up to the Temple (both within and without) and all around both sides of the inner wall, according to his measurement. \v{18}There were carved cherubim and palm trees, alternating with a palm tree between a cherub, and each cherub had two faces, \v{19}with a human face looking\fnote{\fbackref{41:19} The Heb. lacks \fbib{looking}} toward the palm tree on one side and a young lion's face looking\fnote{\fbackref{41:19} The Heb. lacks \fbib{looking}} toward the palm tree on the other side. These carvings extended all the way around the Temple, \v{20}from the ground to above the doorway, as well as on the walls of the main sanctuary.

\v{21}The door posts of the main sanctuary were square. Each door post was identical in appearance to the others. \v{22}The altar was made of wood, three cubits\fnote{\fbackref{41:22} I.e. about 5.25 feet; the royal cubit was 21 inches} high and two cubits\fnote{\fbackref{41:22} I.e. about 42 inches; the royal cubit was 21 inches} long. Its corners, base, and sides were of wood. He told me, ``This table stands in the \divine{Lord}'s presence.''

\v{23}The nave and the sanctuary each were equipped with double doors. \v{24}Each door had two sections mounted on hinges,\fnote{\fbackref{41:24} Lit. \fbib{two swinging sections}} for a total of two sections for one door and two sections for the other. \v{25}The doors of the nave had carvings engraved on them, consisting of cherubim and palm trees identical to those on the walls. The front of the exterior porch was equipped with a wooden threshold. \v{26}Shielded windows and palm trees were visible\fnote{\fbackref{41:26} The Heb. lacks \fbib{visible}} on both sides; that is, on the sides of the porch, the side chambers of the Temple, and on its thresholds.
\labelchapt{42}
\passage{The Vision of the Outer Court}

\chapt{42}
\v{1}Then he brought me to the outer, north-facing courtyard into the chamber that stood opposite the structure that was facing north. \v{2}It stood 100 cubits\fnote{\fbackref{42:2} I.e. about 175 feet; the royal cubit was 21 inches} long and 50 cubits\fnote{\fbackref{42:2} I.e. about 87.5 feet; the royal cubit was 21 inches} wide, with a door in the middle.\fnote{\fbackref{42:2} Lit. \fbib{north}} \v{3}Opposite the 20 cubits\fnote{\fbackref{42:3} I.e. about 35 feet; the royal cubit was 21 inches} wide inner court, and opposite the paved area that comprised the outer court, there were three stories of galleries that faced each other. \v{4}In front of the chambers there was an inner walkway ten cubits wide and 100 cubits\fnote{\fbackref{42:4} So with LXX Syr; MT reads \fbib{and one cubit}} wide, the openings to which were on the\fnote{\fbackref{42:4} Or \fbib{which faced}} north. \v{5}The upper chambers were narrower, since the galleries required more space than did the lower and middle portions of the building. \v{6}The three part structure had no columns, unlike the courts, which is why the upper chambers were offset from the ground upward, more so than the lower and middle chambers.

\v{7}The outer wall by the side of the chambers toward the outer court and facing the chambers was 50 cubits\fnote{\fbackref{42:7} I.e. about 87.5 feet; the royal cubit was 21 inches} long. \v{8}While the chambers in the outer court were 50 cubits\fnote{\fbackref{42:8} I.e. about 87.5 feet; the royal cubit was 21 inches} in length, the chambers facing the Temple were 100 cubits\fnote{\fbackref{42:8} I.e. about 175 feet; the royal cubit was 21 inches} long. \v{9}Below these chambers, as one might enter from the outer court, was the east side entrance. \v{10}There were chambers built into the thick part of the wall of the court facing the east; that is, facing the separate area toward the front of the building, \v{11}with a passageway in front of them, similar in appearance to the chambers that were on the north, proportional to their length and width, with all of their exits according to their arrangements and doorways. \v{12}Corresponding to the chamber doorways facing the south was an opening at the beginning of the passage; that is, the passage in front of the corresponding part of the wall facing east as one might enter.
\passage{The Place for Holy Things}

\v{13}Then he told me, ``The north and south chamber, which are opposite the courtyard, are consecrated areas where the priests who approach the \divine{Lord} will eat consecrated offerings and lay the consecrated grain offerings, sin offerings, and guilt offerings, because the area is holy. \v{14}When the priest enters, they will not enter the outer court from the sanctuary without having removed their garments worn during their time of ministry, because they are holy. They will put on different clothes, and then they will approach the area reserved for the people.''

\v{15}After he had finished measuring the inner temple, he brought me out through the east-facing gate and measured it all around. \v{16}He measured the east side at 500 reeds,\fnote{\fbackref{42:16} I.e. about one mile, and so through vs. 20} according to the length of the measuring stick, \v{17}the north side at 500 reeds, according to the length of the measuring stick, \v{18}the south side at 500 reeds, according to the length of the measuring stick, \v{19}and the west side at 500 reeds, according to the length of the measuring stick. \v{20}He measured a wall that encompassed all four sides, 500 hundred long and 500 wide, dividing between the sacred and common areas.
\labelchapt{43}
\passage{The Vision of the Gates}

\chapt{43}
\v{1}Next, he brought me to the east-facing gate, \v{2}and the glory of the God of Israel was coming from the east. His voice sounded like roaring\fnote{\fbackref{43:2} Lit. \fbib{many}} water, and the land shimmered from his glory. \v{3}His appearance in the vision that I was having was similar to what I observed in the vision where he had come to destroy the city, and also like the visions that I saw by the Chebar River. I fell flat on my face \v{4}while the glory of the \divine{Lord} entered the Temple through the east-facing gate. \v{5}Just then, the Spirit lifted me up and carried me into the inner courtyard, where the glory of the \divine{Lord} was filling the Temple! \v{6}I heard someone speaking to me from the Temple, and a man appeared, standing beside me!
\passage{God to Live among His People}

\v{7}``Son of Man,'' the Lord \divine{God} told\fnote{\fbackref{43:7} Lit. \fbib{he told}} me, ``This is where my throne is, where I place the soles of my feet, and where I will live among the Israelis forever. The house of Israel will no longer defile my holy name---neither they nor their kings---by their unfaithfulness, by the lifeless idols\fnote{\fbackref{43:7} Lit. \fbib{the corpses}} of their kings on their funeral mounds,\fnote{\fbackref{43:7} Or \fbib{their high places}} \v{8}by their setting up their threshold too close to my threshold and their door post too close to my door post, with a wall between me and them. After all, they have defiled my holy name by the loathsome things that they did, so I devoured them in my anger. \v{9}But now let them send their unfaithfulness---that is, the lifeless idols\fnote{\fbackref{43:9} Lit. \fbib{the corpses}} of their kings---far away from me, and I will live among them forever.''
\passage{Ezekiel Describes the Temple}

\v{10}``And now, Son of Man, describe the Temple to the house of Israel. They will be ashamed because of their sin. They will measure its pattern. \v{11}If they are ashamed of everything that they've done, you are to reveal to them the design of the Temple, its structure, its exits and entrances, its plans, its ordinances, and all of its regulations. Write it down where they can see it and remember all of its designs and regulations, so they will implement them. \v{12}This is to be the regulation for the Temple: the entire area on top of the mountain is to be considered wholly consecrated. This is to be the law of the Temple.''
\passage{The Altar}

\v{13}``Here are the measurements of the altar in cubits that were a cubit and a handbreadth\fnote{\fbackref{43:13} I.e. the royal cubit, which measured about 21 inches} long: its base is a cubit\fnote{\fbackref{43:13} I.e. about 21 inches; the royal cubit was about 21 inches} long and a cubit\fnote{\fbackref{43:13} I.e. about 21 inches; the royal cubit was about 21 inches} wide, and its border around the edge at one handbreadth\fnote{\fbackref{43:13} I.e. about three inches} is to be the height of the altar. \v{14}From the base on the ground to its lower edge is to be two cubits,\fnote{\fbackref{43:14} I.e. about 3.5 feet; the royal cubit was about 21 inches} with its width to be one cubit.\fnote{\fbackref{43:14} I.e. about 21 inches; the royal cubit was about 21 inches} From the lesser ledge to the larger edge is to be four cubits.\fnote{\fbackref{43:14} I.e. about seven feet; the royal cubit was about 21 inches} Its width is to be one cubit.\fnote{\fbackref{43:14} I.e. about 21 inches; the royal cubit was about 21 inches} \v{15}The hearth is to be four cubits high,\fnote{\fbackref{43:15} The Heb. lacks \fbib{high}} and four horns are to extend upwards from the hearth. \v{16}The hearth is to be twelve cubits\fnote{\fbackref{43:16} The Heb. lacks \fbib{cubits}} long and twelve cubits\fnote{\fbackref{43:16} The Heb. lacks \fbib{cubits}} wide; that is, it will be a four-sided square. \v{17}It is to have a ledge fourteen cubits\fnote{\fbackref{43:17} The Heb. lacks \fbib{cubits}} long by fourteen cubits\fnote{\fbackref{43:17} The Heb. lacks \fbib{cubits}} wide around the four sides. Its border is to be half a cubit\fnote{\fbackref{43:17} I.e. about 10.5 inches; the royal cubit was about 21 inches} and its base is to be a cubit\fnote{\fbackref{43:17} I.e. about 21 inches; the royal cubit was about 21 inches} all around, with its steps facing east.''
\passage{The Offerings}

\v{18}Then he told me, ``This is what the Lord \divine{God} says: `These are the regulations for the altar, starting the day that it is constructed for presenting burnt offerings and sprinkling blood. \v{19}You are to present to the Levitical priests, Zadok's descendants, who will approach me to serve me, a young bull for a sin offering,' declares the Lord \divine{God}. \v{20}You are to take some of its blood and put it on the four horns of the altar,\fnote{\fbackref{43:20} The Heb. lacks \fbib{of the altar}} on the four corners of its ledge, and on the border that surrounds it, thus cleansing it and making an atonement for it. \v{21}You are also to present a bull for a sin offering, incinerating it in the appointed place at the Temple, outside the sanctuary.

\v{22}`The second day following commencement of offerings,\fnote{\fbackref{43:22} The Heb. lacks \fbib{following commencement of offerings}} you are to offer a male goat without defect for a sin offering to cleanse the altar the same way they cleansed it with the bull. \v{23}After you've finished the cleansing, you are to present a young bull without defect and a ram from the flock without defect. \v{24}You are to present them in the \divine{Lord}'s presence, and the priests are to throw salt on them and then present them as a burnt offering to the \divine{Lord}.

\v{25}`Every day for a week, you are to prepare a goat for a sin offering, a young bull, and a ram from the flock, each\fnote{\fbackref{43:25} The Heb. lacks \fbib{each}} without defect. \v{26}For a seven day period they are to make atonement for the altar, purifying it and consecrating it. \v{27}When they will have completed this period,\fnote{\fbackref{43:27} Lit. \fbib{completed these days}} starting the next day,\fnote{\fbackref{43:27} Lit. \fbib{period, from the eighth day following,}} the priests are to offer your burnt offerings on the altar, along with your peace offerings, and I will accept you,' declares the Lord \divine{God}.''
\labelchapt{44}
\passage{The Vision of the Outer Gates}

\chapt{44}
\v{1}Then the Lord \divine{God}\fnote{\fbackref{44:1} Lit. \fbib{Then he}} brought me back through the east-facing outer gate of the sanctuary. But it was shut. \v{2}The \divine{Lord} told me, ``This gate is to remain shut. It will not be opened. No man is to enter through it, because the \divine{Lord} God of Israel entered through it, so it is to remain shut. \v{3}The Regent\fnote{\fbackref{44:3} The Heb. lacks \fbib{Regent}; and so through chapter 48} Prince\fnote{\fbackref{44:3} I.e. a ruler who will govern with a king's authority in the name of one holding higher supremacy; and so through chapter 48} will be seated there,\fnote{\fbackref{44:3} Lit. \fbib{will sit in it}} as Regent Prince, and will dine in the \divine{Lord}'s presence, entering through the portico of the gate and exiting through it also.''
\passage{The Front of the Temple}

\v{4}Then he brought me through the north-facing gate to the front of the Temple. As I looked, the glory of the \divine{Lord} filled the \divine{Lord}'s Temple, and I fell flat on my face! \v{5}Then the \divine{Lord} told me, ``Son of Man, watch carefully,\fnote{\fbackref{44:5} Lit. \fbib{watch with your own eyes}} listen closely,\fnote{\fbackref{44:5} Lit. \fbib{listen with your own ears}} and remember\fnote{\fbackref{44:5} Lit. \fbib{and put in your heart}} everything I'm going to be telling you about all the statutes pertaining to the \divine{Lord}'s Temple and all of its laws. Pay careful attention to the entrance to the Temple, along with all of the exits to the sanctuary.''
\passage{A Rebuke to the Rebellious}

\v{6}``You are to tell the Resistance---that is, the house of Israel, `This is what the Lord \divine{God} says: ``I've had enough of all of your loathsome behavior, you house of Israel! \v{7}You kept on bringing in foreigners, those who were uncircumcised in heart and flesh, to profane my sanctuary by being inside my Temple, and by doing so you've emptied my covenant, all the while offering my food---the fat and the blood---in addition to all of the other loathsome things you've done.\fnote{\fbackref{44:7} The Heb. lacks \fbib{you've done}} \v{8}Furthermore, you haven't paid attention to the requirements for my holy things. Instead, you placed foreigners in charge of my sanctuary.''\,'

\v{9}``This is what the Lord \divine{God} says, `No foreigner who is both uncircumcised in heart and flesh, of all the foreigners who are among the Israelis is to enter my sanctuary. \v{10}But the descendants of Levi, who went far away from me when Israel abandoned me, who left me to follow their idols, are to bear the punishment of their iniquity. \v{11}Nevertheless, they are to serve in my sanctuary, overseeing the gates of the Temple, taking care of the Temple, slaughtering the burnt offerings and the sacrifices presented for the people, standing in the presence of the people, and ministering to them. \v{12}Because they kept serving them in the presence of their idols, becoming a sin-filled stumbling block to the house of Israel,' declares the Lord \divine{God}.

```I have sworn to them that they are to bear the consequences of their iniquity. \v{13}They are not to come near me to serve me as a priest, nor approach any of my holy things, including the most holy things. Instead, they are to bear the shame of the loathsome things that they have done. \v{14}Nevertheless, I will appoint them to take care of my Temple, including all of its service and everything that is to be done inside of it.'\,''
\passage{Levitical Ordinances}

\v{15}``The descendants of Zadok, Levitical priests who took care of my sanctuary when the Israelis wandered away from me, are to come near me to minister to me. They are to stand before me to offer the fat and the blood to me,'' declares the Lord \divine{God}. \v{16}``They are to enter my sanctuary, approach my table to minister to me, and carry out my requirements. \v{17}Whenever they enter at the gates of the inner court, they are to be clothed with linen garments. They are not to wear wool when they are ministering within the gates of the inner courtyard or in the Temple. \v{18}Linen turbans are to be on their heads, and they are to wear linen undergarments. Also, they are not to clothe themselves with anything that makes them perspire.

\v{19}``When they enter the outer courtyard, that is, the outer courtyard where the people are, they are to take off their garments in which they were ministering, lay them in the consecrated chambers, and put on different garments so they will not transfer\fnote{\fbackref{44:19} Or \fbib{transmit}} holiness to the people through their garments. \v{20}Also, they are not to shave their heads nor let their hair grow long. Instead, they are certainly to trim the hair on\fnote{\fbackref{44:20} The Heb. lacks \fbib{the hair on}} their heads. \v{21}None of the priests are to drink wine after entering the inner courtyard. \v{22}They are not to marry\fnote{\fbackref{44:22} Lit. \fbib{take}} a widow or a divorced woman. Instead, they are to marry\fnote{\fbackref{44:22} Lit. \fbib{take}} virgins from the descendants of the house of Israel, or a widow who is the widow of a priest.''
\passage{Duties of Ministry}

\v{23}``They are to teach my people how to discern\fnote{\fbackref{44:23} The Heb. lacks \fbib{how to discern}} what is holy in contrast to what is common, showing them how to discern between what is unclean and clean. \v{24}When disputes arise, they are to serve as a judge, adjudicating matters according to my ordinances. They are to enforce my laws, my statutes, all of my appointed festivals, and they are to sanctify my Sabbaths. \v{25}They are not to come in contact with a dead body, so they don't defile themselves, except in the case of their father, mother, son, daughter, brother, or for an unmarried sister, on whose behalf they may defile themselves. \v{26}After he is cleansed from that contact,\fnote{\fbackref{44:26} The Heb. lacks \fbib{from that contact}} he is to not to minister for seven days. \v{27}On the day that he returns to the sanctuary's inner court to minister, he is to offer his own sin offering,'' declares the Lord \divine{God}.
\passage{Ministerial Inheritances}

\v{28}``Now with respect to the priests'\fnote{\fbackref{44:28} Lit. \fbib{to their}} inheritances, I am to be their inheritance, and you are to give them no possession in Israel, since I am their possession. \v{29}They are to eat the grain offerings, sin offering, and guilt offering. Everything consecrated in Israel is to belong to them. \v{30}The first portion of all the first fruits of every kind and every offering of any kind is to be for the priests. You are to give the priest the first portion of your grain. As a result a blessing will rest on your household. \v{31}However, the priests are not to eat any bird or animal that has died a natural death or that has been torn apart.''
\labelchapt{45}
\passage{Israel's Future Temple Park}

\chapt{45}
\v{1}``When you divide the land for an inheritance, you are to present a Terumah\fnote{\fbackref{45:1} Lit. \fbib{Gift}; i.e. a special section of Israel's land to be dedicated to the \fbib{}\divine{Lord} as a national temple park; cf. Eze 48:8ff} to the \divine{Lord}, a consecrated portion of the land 25,000 cubits\fnote{\fbackref{45:1} The Heb. lacks \fbib{cubit}; if the unit of measurement is royal cubits, the length is about 8.29 miles.} long and 20,000\fnote{\fbackref{45:1} So LXX; MT reads \fbib{10,000}} cubits\fnote{\fbackref{45:1} LXX and MT lack \fbib{cubit}; if the unit of measurement is 20,000 royal cubits, the length is about 6.6 miles.} wide. Everything within this area is to be treated as holy. \v{2}A Holy Place is to be dedicated from this area in the form of a square measuring 500 by 500 cubits,\fnote{\fbackref{45:2} The Heb. lacks \fbib{cubits}} with a 50 cubit\fnote{\fbackref{45:2} I.e. about 87.5 feet; the royal cubit was about 21 inches} buffer zone\fnote{\fbackref{45:2} Lit. \fbib{cubit open space}} surrounding it. \v{3}From this area a measure is to be made 25,000 cubits\fnote{\fbackref{45:3} The Heb. lacks \fbib{cubits}; if the unit of measurement is royal cubits, the length is about 8.29 miles.} long and 10,000 cubits\fnote{\fbackref{45:3} The Heb. lacks \fbib{cubits}} wide, which is to contain the sanctuary, the holiest of holy objects. \v{4}It is to be a holy portion of the land, set aside\fnote{\fbackref{45:4} The Heb. lacks \fbib{set aside}} for the priests who serve the sanctuary, who approach the \divine{Lord} to serve him. It is to be a place for their houses, as well as the Holy Place of the sanctuary. \v{5}An area 25,000 cubits\fnote{\fbackref{45:5} The Heb. lacks \fbib{cubits}; if the unit of measurement is royal cubits, the length is about 8.29 miles.} long by 10,000 cubits wide is to be set aside\fnote{\fbackref{45:5} The Heb. lacks \fbib{to be set aside}} for use by the Levite\fnote{\fbackref{45:5} I.e. the ministry formerly held by the descendants of Levi} servants of the Temple, 20 parcels\fnote{\fbackref{45:5} Or \fbib{chambers}; so with MT; LXX reads \fbib{temple, cities}} for their residential properties. \v{6}The land allocation for the city is to be set at 5,000 cubits\fnote{\fbackref{45:6} The Heb. lacks \fbib{cubits}; if the unit of measurement is royal cubits, the length is about 1.66 miles.} wide and 25,000 cubits\fnote{\fbackref{45:6} The Heb. lacks \fbib{cubits}; if the unit of measurement is royal cubits, the length is about 8.29 miles.} long, adjacent to the sanctuary district, reserved for the entire house of Israel.''
\passage{The Portion for the Regent Prince}

\v{7}``The Regent Prince is to have a portion\fnote{\fbackref{45:7} The Heb. lacks \fbib{a portion}} on both sides of the consecrated allotment for the sanctuary and the city's land allotment, adjacent to both on the west\fnote{\fbackref{45:7} Lit. \fbib{the sea side facing the sea}} and the east sides, comparable in length to one of the portions from the west\fnote{\fbackref{45:7} Lit. \fbib{the sea side}} border to the east border. \v{8}This property in Israel is to belong to the Regent Prince,\fnote{\fbackref{45:8} Lit. \fbib{to him}} so my regent princes will no longer mistreat my nation. The remaining portion of the land is to be allotted to the house of Israel, that is, to its tribes.''
\passage{An Exhortation to Honest Business}

\v{9}``This is what the Lord \divine{God} says, `Enough of you, you regent princes of Israel! Abandon your violence and destruction. Practice what is just and right instead! Stop confiscating property from my people!' declares the Lord \divine{God}. \v{10}`You're to use an honest scale, an honest dry measure,\fnote{\fbackref{45:10} Lit. \fbib{honest ephah}} and an honest liquid measure!\fnote{\fbackref{45:10} Lit. \fbib{bath}} \v{11}The ephah and the bath are to be of equal volume; that is, the bath is to contain one tenth of an omer and the ephah one tenth of an omer. The omer is to be the standard on which their volume measurement is to be based. \v{12}The shekel\fnote{\fbackref{45:12} A shekel weighed about 0.4 ounces} is to weigh 20 gerahs. The mina\fnote{\fbackref{45:12} Or \fbib{maneh}; the Babylonian standard was equivalent to 1/60\textsuperscript{th} of a talent, with a talent weighing about 75 pounds} is to be comprised of three coins weighing\fnote{\fbackref{45:12} The Heb. lacks \fbib{comprised of three coins weighing}} 20, 25, and fifteen shekels, respectively.'\,''
\passage{Weight Standards for Offerings}

\v{13}``Here are the standards for presenting offerings: a sixth of an ephah that is based on the standard omer of wheat, and a sixth of an ephah based on the standard omer of barley. \v{14}The olive oil quota is to be based on the bath, measured at ten baths to each omer, which is equal to one kor. \v{15}The sheep quota is to be one from each flock of 200 taken from the pastures of Israel. From all of these you are to present grain offerings, burnt offerings, and peace offerings, to make atonement for them,'' declares the Lord \divine{God}.

\v{16}``The entire nation living in the land is to present this offering to the Regent Prince in Israel. \v{17}The Regent Prince is to provide the burnt offerings, grain offerings, and drink offerings at the festivals, on the New Moons and Sabbaths, and at all of the prescribed festivals of the house of Israel. He is to provide the grain offerings, burnt offerings, and peace offerings in order to make atonement for the house of Israel.''

\v{18}``This is what the Lord \divine{God} says, `On the first day of the first month, you are to present a young bull without defect in order to cleanse the sanctuary. \v{19}The priest is to place some of the blood from the sin offering on the door posts of the Temple, on the four corners of the ledge around the altar, and on the posts of the gate leading to the inner court. \v{20}You are also to do this on the seventh day of the month, to make atonement for any person who wanders away or who sins through ignorance in order to make atonement for the Temple.

\v{21}```On the fourteenth day of the first month, you are to observe the Passover as a festival for seven days. Unleavened bread is to be eaten. \v{22}On that day, the Regent Prince is to provide, both for himself and for all the people who live in the land, a bull for a sin offering. \v{23}Each day during the seven days of the festival, he is to provide a burnt offering to the \divine{Lord}, consisting of seven bulls and seven rams without defect, offered each day throughout the seven days, along with a male goat offered each day as a sin offering.

\v{24}```The Regent Prince\fnote{\fbackref{45:24} Lit. \fbib{He}} is also to present a grain offering consisting of an ephah with each bull and an ephah with each ram, along with a hin of olive oil mixed with an ephah of grain. \v{25}On the fifteenth day of the seventh month, during a seven day festival, the Regent Prince\fnote{\fbackref{45:25} Lit. \fbib{festival, he}} is to present these as daily sin offerings, burnt offerings, and grain offerings mixed with oil.'\,''
\labelchapt{46}
\passage{Regulations for the Inner Court}

\chapt{46}
\v{1}``This is what the Lord \divine{God} says: `The inner, east-facing courtyard is to remain shut during the six working days of the week,\fnote{\fbackref{46:1} The Heb. lacks \fbib{of the week}} but on the Sabbath day it is to be opened, as well as on the day of the New Moon. \v{2}The Regent Prince is to enter through the portico of the gate from outside and is to stand at the doorframe of the gate where the priests are to present the Regent Prince's\fnote{\fbackref{46:2} Lit. \fbib{his}} burnt offerings and peace\fnote{\fbackref{46:2} Or \fbib{fellowship}; and so throughout the chapter} offerings. Then the Regent Prince\fnote{\fbackref{46:2} Lit. \fbib{Then he}} is to worship at the threshold of the gate and go out. The gate is not to be closed until evening. \v{3}The people who live\fnote{\fbackref{46:3} The Heb. lacks \fbib{who live}} in the land are to worship at the doorway of the gate on the Sabbaths and New Moons in the \divine{Lord}'s presence.'\,''
\passage{Sabbath Offerings by the Regent Prince}

\v{4}```The burnt offering that the Regent Prince is to present to the \divine{Lord} on the Sabbath day is to consist of six lambs without defect, a ram without defect, \v{5}a grain offering with the ram consisting of an ephah, a grain offering with the lambs consisting of whatever amount he brings with him, and a hin of oil with each ephah of grain.\fnote{\fbackref{46:5} The Heb. lacks \fbib{of grain}} \v{6}Furthermore, each New Moon there is to be a young bull presented without defect, six male lambs, and a ram without defect. \v{7}The Regent Prince\fnote{\fbackref{46:7} Lit. \fbib{He}} is to present an ephah\fnote{\fbackref{46:7} I.e. five gallons in volume} of grain\fnote{\fbackref{46:7} The Heb. lacks \fbib{of grain}} along with the bull, an ephah\fnote{\fbackref{46:7} I.e. five gallons in volume} of grain\fnote{\fbackref{46:7} The Heb. lacks \fbib{of grain}} along with the ram, a grain offering---consisting of as much\fnote{\fbackref{46:7} The Heb. lacks \fbib{a grain offering---consisting of as much}} as he is able to give---and a hin\fnote{\fbackref{46:7} I.e. about a gallon} of olive oil with each ephah\fnote{\fbackref{46:7} I.e. five gallons in volume} of grain.\fnote{\fbackref{46:7} The Heb. lacks \fbib{of grain}}

\v{8}```The Regent Prince is to enter through the portico of the gate and is to leave the same way he came in. \v{9}When the people who live\fnote{\fbackref{46:9} The Heb. lacks \fbib{who live}} in the land come into the \divine{Lord}'s presence during the festivals, whoever enters through the northern gate is to leave through the southern gate, and whoever enters through the southern gate is to leave through the northern gate. No one is to leave by the same route that he enters, but instead is to go straight out. \v{10}The Regent Prince is to enter when they are coming in, and he is to leave when they go out.'\,''
\passage{Daily Offerings by the Regent Prince}

\v{11}```The grain offering for the festivals and appointed festivals is to include an ephah\fnote{\fbackref{46:11}I.e. five gallons in volume} with a bull, an ephah\fnote{\fbackref{46:11}I.e. five gallons in volume} with a ram, and as much grain with the lambs as the Regent Prince\fnote{\fbackref{46:11} Lit. \fbib{as he}} brings with him, along with a hin\fnote{\fbackref{46:11} I.e. about a gallon} of oil with each ephah. \v{12}Whenever the Regent Prince presents a voluntary offering, burnt offering, or peace offering, he is to present it voluntarily to the \divine{Lord}, and the east-facing gate is to be opened for him. He is to provide his burnt offering and peace offering as he does on the Sabbath. When he leaves, the gate is to be shut behind him. \v{13}He is to present a one year old lamb without defect for a burnt offering to the \divine{Lord} in the morning every day. \v{14}In addition, he is to present a grain offering with it every morning, consisting of a sixth of an ephah\fnote{\fbackref{46:14} I.e. about 5/6 of a gallon} mixed with one third of a hin\fnote{\fbackref{46:14} I.e. about 1/6 of a gallon} of oil. This grain offering is to be offered to the \divine{Lord} as a permanent ordinance. \v{15}They are to present the lamb offering, the grain offering, and the oil every morning as an ongoing\fnote{\fbackref{46:15} Or \fbib{perpetual}} burnt offering.'\,''
\passage{Gifts by the Regent Prince}

\v{16}``This is what the Lord \divine{God} says: `If the Regent Prince gives a gift to someone,\fnote{\fbackref{46:16} Lit. \fbib{to a man among his sons}} it is to remain with the man's descendants as their own inheritance. \v{17}But if he gives a gift to any of his servants, it is to belong to the servant\fnote{\fbackref{46:17} Lit. \fbib{to him}} until the Year of Release, at which time it is to be returned to the Regent Prince. His inheritance is to belong only to his sons. \v{18}The Regent Prince is not to appropriate the nation's inheritance nor take advantage of them by taking their property from them. Instead, he is to provide an inheritance for his sons from his own possessions so that my people will not be separated from their possessions.'\,''
\passage{The Place Where Offerings are Boiled}

\v{19}Then the angel\fnote{\fbackref{46:19} Lit. \fbib{Then he}} brought me in through an entrance beside the gate into the north-facing chambers dedicated to the priests. As I looked toward the rear\fnote{\fbackref{46:19} Lit. \fbib{north}} of the far western end, I saw a place \v{20}about which he said, ``This is where the priests will be boiling the guilt and sin offerings and baking the grain offerings so they don't bring them through the outer courtyard, thus diminishing the people's holiness.''\fnote{\fbackref{46:20} Or \fbib{thus transmitting holiness to the people}} \v{21}Then he brought me out to the exterior courtyard and led me across to each of the four corners of the courtyard. There in each corner was an enclosed area set aside, \v{22}all of them the same size; that is, each was 40 cubits\fnote{\fbackref{46:22} I.e. about 70 feet; a royal cubit was about 21 inches} long and 30 cubits\fnote{\fbackref{46:22} I.e. about 52.5 feet; a royal cubit was about 21 inches} wide. \v{23}A low wall\fnote{\fbackref{46:23} Lit. \fbib{A row}} built of masonry surrounded each courtyard, with boiling places set in rows in the wall. \v{24}He told me, ``This is where\fnote{\fbackref{46:24} Lit. \fbib{is the house}} the ministers of the Temple will be preparing\fnote{\fbackref{46:24} Lit. \fbib{boiling}} the sacrifices that will be presented by the people.
\labelchapt{47}
\passage{The Vision of the Temple River}

\chapt{47}
\v{1}After this, he brought me back to the doorway to the Temple. To my amazement, there was water flowing out toward the east from beneath the threshold of the Temple! (The Temple faced eastward.) The water flowed down from beneath the right side of the Temple,\fnote{\fbackref{47:1} I.e. the right side as one inside the temple faced toward the east.} that is, from the south-facing side where the altar was located. \v{2}Then he brought me out through the north gateway and around to the one outside that faces toward the east. To my amazement, water was trickling out from that part of\fnote{\fbackref{47:2} The Heb. lacks \fbib{that part of}} the south side, too!

\v{3}As the man went out toward the east, he carried a measuring line in his hand. He measured out 1,000 cubits\fnote{\fbackref{47:3} I.e. about 1,750 feet; a royal cubit was about 21 inches} as he led me through water that was ankle-deep. \v{4}Then he measured out another 1,000 cubits,\fnote{\fbackref{47:4} The Heb. lacks \fbib{cubits}} where he led me through water that was knee-deep. And then he measured out another 1,000 cubits,\fnote{\fbackref{47:4} The Heb. lacks \fbib{cubits}} where the water was waist-deep. \v{5}When he had measured out another 1,000 cubits,\fnote{\fbackref{47:5} The Heb. lacks \fbib{cubits}} the water had become deep enough that I wasn't able to ford it. Instead, I would have had to swim through it.

\v{6}Then, as he was bringing me back along the river bank, he asked me, ``Son of Man, did you see all of this?'' \v{7}As we were coming back, I was amazed to see that there were many, many trees lining both banks of the river. \v{8}He told me, ``This river flows toward the eastern territories all the way down into the Arabah,\fnote{\fbackref{47:8} I.e. Israel's southern desert areas including from the Sea of Galilee to the Red Sea through the Dead Sea} and from there its water flows toward the Dead\fnote{\fbackref{47:8} The Heb. lacks \fbib{Dead}} Sea, where the sea water turns fresh. \v{9}It will support all kinds of living creatures that will thrive abundantly wherever the river flows. There will be a great many fish, because this water will flow there and turn the salt water fresh. As a result, everything will live wherever the river flows. \v{10}A day will come when fishermen will line its banks---from En-gedi to En-eglaim\fnote{\fbackref{47:10} I.e. a city probably located on the northwest shore of the Dead Sea, not far from En-gedi, perhaps modern `Ain Feshka} there will be plenty of room to spread out nets. There will be all sorts of species of fish, as abundant as the fish that live in the Mediterranean\fnote{\fbackref{47:10} Lit. \fbib{Great}} Sea. There will be lots of them!

\v{11}``The river delta\fnote{\fbackref{47:11} The Heb. lacks \fbib{delta}} will consist of swamps and marshes that will remain a salt water wetland preserve. \v{12}Lining each side of the river banks, all sorts of species of fruit trees will be growing. Their leaves will never wither and their fruit will never fail. They will bear fruit every month, because the water that nourishes them will be flowing from the sanctuary. Their fruit will be for food and their leaves will contain substances that promote healing.''
\passage{Israel's Future Borders Delineated}

\v{13}This is what the Lord \divine{God} says: ``This is to be the territorial border by which you apportion the land for an inheritance among the twelve tribes of Israel, with Joseph double-portioned. \v{14}Apportion it for their inheritances, distributing everything equally as if you were distributing things to your own\fnote{\fbackref{47:14} The Heb. lacks \fbib{own}} brother, which is how I promised to give it to your ancestors. This way, the land will fall to you as an inheritance.

\v{15}``This is to be the\fnote{\fbackref{47:15} The Heb. lacks \fbib{is to be the}} border for the land: on the north side, from the Mediterranean\fnote{\fbackref{47:15} Lit. \fbib{Great}} Sea by the Hethlon Road to the entrance of Zedad, \v{16}Hamath, Berothah, Sibraim (which lies between the border of Damascus and the border of Hamath), and Hazer-hatticon, which is on the border of Hauran. \v{17}The border is to proceed from the Mediterranean\fnote{\fbackref{47:17} The Heb. lacks \fbib{Mediterranean}} Sea to Hazer-enan (a border of Damascus), and on the north facing north is to be the border of Hamath. This is to be the north side.

\v{18}``The eastern extremity is to proceed from between Hauran and\fnote{\fbackref{47:18} Lit. \fbib{Hauran and then between}} Damascus, then between Gilead, and then through the land of Israel---the Jordan River.\fnote{\fbackref{47:18} The Heb. lacks \fbib{River}} You are to measure from the northern border to the Dead\fnote{\fbackref{47:18} Lit. \fbib{Eastern}} Sea. This is to be the eastern perimeter.

\v{19}``You are to determine the southern extremity running from Tamar as far as the waters of Meribath-kadesh, then from there proceeding to the Wadi,\fnote{\fbackref{47:19} I.e. the Wadi of Egypt, a seasonal stream or river that channels water during rain seasons but is dry at other times; ancient Israel's southwestern-most border} and then to the Mediterranean\fnote{\fbackref{47:19} Lit. \fbib{Great}} Sea. This is to be the southern\fnote{\fbackref{47:19} Or \fbib{Negev}} perimeter.

\v{20}``The western\fnote{\fbackref{47:20} Lit. \fbib{sea}; i.e. the border running along the Mediterranean Sea} perimeter is to be the Mediterranean\fnote{\fbackref{47:20} Lit. \fbib{Great}} Sea, from the southernmost border to a location opposite the entrance to Hamath. This is to be the western\fnote{\fbackref{47:20} Lit. \fbib{sea}; i.e. the border running along the Mediterranean Sea} perimeter.

\v{21}``You are to apportion this land among yourselves according to the tribes of Israel, \v{22}dividing it by lottery among yourselves and among the foreigners who live among you and bear children among you. You are to treat them like native-born Israelis. Among you they,\fnote{\fbackref{47:22} Or \fbib{Israelis among you. They}} too, are to be allotted an inheritance with the tribes of Israel. \v{23}Furthermore, you are to provide the foreigner's inheritance there in the tribe within which he remains,'' declares the Lord \divine{God}.
\labelchapt{48}
\passage{Regulations for Israel's Northern Land Divisions}

\chapt{48}
\v{1}``These are the names of the tribes from the northernmost extremity westward\fnote{\fbackref{48:1} Lit. \fbib{to the sea}; i.e. in the direction of the Mediterranean Sea} along the road from Hethlon to the entrance of Hamath,\fnote{\fbackref{48:1} Or \fbib{to Lebo-hamath}} Hazar-enan (a border of Damascus) northward to the coast\fnote{\fbackref{48:1} Lit. \fbib{to the sea}; i.e. in the direction of the Mediterranean Sea} of Hamath. The perimeter is to run\fnote{\fbackref{48:1} The Heb. lacks \fbib{is to run}} east-to-west;\fnote{\fbackref{48:1} Lit. \fbib{from east to the Sea}; and so throughout the chapter} the tribe of\fnote{\fbackref{48:1} The Heb. lacks \fbib{The tribe of}; and so throughout the chapter} Dan with one portion;\fnote{\fbackref{48:1} The Heb. lacks \fbib{portion}; and so throughout the list} \v{2}running along the border of the tribe of Dan from the eastern perimeter to the western perimeter, the tribe of Asher with one portion; \v{3}running along the border of the tribe of Asher from the eastern perimeter to the western perimeter, the tribe of Naphtali with one portion; \v{4}running along the border of the tribe of Naphtali from the eastern perimeter to the western perimeter, the tribe of Manasseh with one portion; \v{5}running along the border of the tribe of Manasseh from the eastern perimeter to the western perimeter, the tribe of Ephraim with one portion; \v{6}running along the border of the tribe of Ephraim from the eastern perimeter to the western perimeter, the tribe of Reuben with one portion; \v{7}and running along the border of the tribe of Reuben from the eastern perimeter to the western perimeter, the tribe of Judah with one portion.''
\passage{Israel's National Temple Allotment}

\v{8}``Running along the border of the tribe of Judah from the eastern perimeter to the western perimeter you are to set apart the Terumah,\fnote{\fbackref{48:8} Lit. \fbib{Gift}; i.e. a special section of Israel's land to be dedicated to the \fbib{}\divine{Lord} as a national temple park, and so throughout the chapter; cf. Eze 45:1-7} 25,000 units\fnote{\fbackref{48:8} The Heb. lacks \fbib{units}; i.e. the measuring unit is unspecified throughout the chapter} wide, with its east-west length equal to one of the other apportionments, from the eastern perimeter to the western perimeter, with the Temple in the middle of it. \v{9}The Terumah that you are to give to the \divine{Lord} is to be 25,000 units wide.''\fnote{\fbackref{48:9} The Heb. lacks \fbib{wide}}
\passage{Allotments for the Priests}

\v{10}``The holy Terumah is to be reserved for these, the priests: Toward the north, 25,000 units in length;\fnote{\fbackref{48:10} The Heb. lacks \fbib{in length}} toward the west, 10,000 units in width; toward the east, 10,000 units in width; and toward the south, 25,000 units in length. The \divine{Lord}'s sanctuary is to be in its midst. \v{11}It is to be for use by\fnote{\fbackref{48:11} The Heb. lacks \fbib{It is to be for use by}} priests from the descendants of Zadok, who have observed the things with which I charged them and who did not wander astray when the Israelis went astray, just as the descendants of Levi wandered astray. \v{12}It is to be a Terumah for them from the allotment of the land, a Most Holy Place, adjoining the border of the descendants of Levi.''

\v{13}``Alongside the border of the priests, the descendants of Levi are to be allotted 25,000 units in length and 10,000 units in width. The entire length is to be 25,000 units and its width 10,000 units. \v{14}Furthermore, they are not to sell or exchange any part of it, nor transfer these first fruits\fnote{\fbackref{48:14} Or \fbib{this choice portion}} of the land, because it is holy to the \divine{Lord}.

\v{15}``The rest, 5,000 units wide and 25,000 units along its front, will serve as a common portion for use by the city for housing and open spaces, since the city is to be in its midst. \v{16}These are to be its dimensions: the north side, 4,500 units; the south side, 4,500 units; the east side, 4,500 units; and the west side 4,500 units. \v{17}The city is to have urban areas set aside: on the north 250 units; on the south, 250 units, on the east, 250 units; and on the west, 250 units.

\v{18}``The remainder of the length that borders the holy Terumah is to be 10,000 units long eastward and 10,000 units westward. It is to lie adjacent to the holy Terumah. Its harvest will produce food for those who work in the city. \v{19}The city workers who cultivate it are to come from all the tribes of Israel. \v{20}The entire Terumah is to be\fnote{\fbackref{48:20} The Heb. lacks \fbib{is to be}} 25,000 units by 25,000 units---you are to reserve it as a holy Terumah in the form of a square within the city limits.''
\passage{The Allotment for the Regent Prince}

\v{21}``Now the remainder of the allotment\fnote{\fbackref{48:21} The Heb. lacks \fbib{of the allotment}} on either side of the holy Terumah is to be for the Regent Prince and for city property\fnote{\fbackref{48:21} I.e. public property}---adjoining the 25,000 units along the eastern border and adjoining the 25,000 units along the western border, and parallel to the allotments. These are to be for the Regent Prince. The holy Terumah and the sanctuary of the Temple is to stand in the middle of it. \v{22}Except for what belongs to the descendants of Levi and the city property, which will stand in the middle of what belongs to the Regent Prince, whatever is between the border of Judah and the border of Benjamin is to belong to the Regent Prince.''
\passage{The Allotment for the Tribes}

\v{23}``Now as to the rest of the tribes: from the east side to the west side, Benjamin is to retain one portion.\fnote{\fbackref{48:23} Lit. \fbib{Benjamin, one}; and so through v. 27} \v{24}Adjacent to the border of Benjamin running from east to west, Simeon is to retain one portion. \v{25}Adjacent to the border of Simeon running from east to west, Issachar is to retain one portion. \v{26}Adjacent to the border of Issachar running from east to west, Zebulun is to retain one portion. \v{27}Adjacent to the border of Zebulun running from east to west, Gad is to retain one portion. \v{28}Adjacent to the border of Gad to the south and extending toward the south, the border is to proceed from Tamar to the waters of Meribath-kadesh, then to the Wadi\fnote{\fbackref{48:28} I.e. a seasonal stream or river that channels water during rain seasons but is dry at other times; ancient Israel's southwestern-most border} of Egypt,\fnote{\fbackref{48:28} I.e. ancient Israel's southwestern-most border} and from there to the Mediterranean\fnote{\fbackref{48:28} Lit. \fbib{Great}} Sea. \v{29}This is the land that you are to allocate by lottery to the tribes of Israel as their inheritance, and these are their respective divisions,'' declares the Lord \divine{God}.
\passage{The Gates of the City}

\v{30}``These are the exits to the city: On the north side, 4,500 units by measurement, \v{31}are to be the gates of the city. Named after the tribes of Israel, three gates are to serve the north site: one named the Reuben Gate, one named the Judah Gate, and one named the Levi Gate. \v{32}On the east side, 4,500 units by measurement,\fnote{\fbackref{48:32} The Heb. lacks \fbib{by measurement}} there are to be three gates: one named the Joseph Gate, one named the Benjamin Gate, and one named the Dan Gate. \v{33}On the south side, 4,500 units by measurement, there are to be three gates: one named the Simeon Gate, one named the Issachar Gate, and one named the Zebulun Gate. \v{34}On the west side, 4,500 units by measurement,\fnote{\fbackref{48:34} The Heb. lacks \fbib{by measurement}} there are to be three gates: one named the Gad Gate, one named the Asher Gate, and one named the Naphtali Gate. \v{35}A perimeter is to measure 18,000 units, and the name of the city from that time on is to be:

\begin{poetry}
\poeml `\divine{The Lord is There}.'\,''\end{poetry}

\bookheader{Daniel}
\labelbook{Dan}

\bookpretitle{The Book of the Prophet}
\booktitle{Daniel}

\labelchapt{1}
\passage{Hostages of the Babylonian Captivity}

\chapt{1}
\v{1}In the third year of the reign of King Jehoiakim of Judah, King Nebuchadnezzar of Babylon came to Jerusalem and laid siege to it. \v{2}Within a week, the Lord handed King Jehoiakim of Judah over to him, along with valuable objects from the house of God. Nebuchadnezzar\fnote{Lit. \fbib{He}} brought them to the temple of his god in the land of Shinar\fnote{I.e. Babylon} and stored them\fnote{Lit. \fbib{the valuable objects}} in its treasure house.\fnote{Lit. \fbib{in the treasure house of his god}}

\v{3}Later, the king ordered Ashpenaz, his chief officer,\fnote{Lit. \fbib{eunuch}; i.e. an overseer in the king's court; and so throughout the chapter} to bring in some Israelis of royal and noble descent. \v{4}They were to be young men without physical defect, handsome in appearance, skilled in all wisdom, quick to learn, prudent in how they used knowledge, and capable of serving in the king's palace. They were to learn the literature and language of the Chaldeans.\fnote{I.e. Aramaic speaking wise men from Mesopotamia; or magi-astrologers; and so throughout the book; cf. Matt 2:1}

\v{5}The king assigned them fine food and choice wine on a daily basis, ordering them to be trained for three years, at the end of which time they would enter the king's service.\fnote{Lit. \fbib{would stand before the king}} \v{6}Included among the people of Judah were Daniel,\fnote{The Heb. name \fbib{Daniel} means \fbib{God is my judge}} Hananiah, Mishael, and Azariah. \v{7}The chief officer assigned the name ``Belteshazzar'' to Daniel, the name ``Shadrach'' to Hananiah, the name ``Meshach'' to Mishael, and the name ``Abednego'' to Azariah.
\passage{Daniel Chooses God's Standard}

\v{8}Daniel determined within himself not to become defiled by the king's menu of rich foods or by the king's wine, so he requested permission\fnote{The Heb. lacks \fbib{permission}} from the chief officer not to defile himself. \v{9}God granted to Daniel grace and compassion on the part of the chief officer. \v{10}The chief officer told Daniel, ``I fear his majesty the king, who has determined what you eat and drink. If he notices that your faces are more pale than the other\fnote{The Heb. lacks \fbib{other}} young men in your group, I will forfeit my head to the king.''

\v{11}But Daniel told the guard whom the chief officer had appointed over Daniel, Hananiah, Mishael, and Azariah, \v{12}``Please test your servants for ten days and let us be given vegetables to eat and water to drink. \v{13}Then compare how we\fnote{Lit. \fbib{they}} look with the young men who ate the king's rich food, and treat your servants in accordance with what you observe.''

\v{14}So he listened to what Daniel said\fnote{Lit. \fbib{listened according to this word}} and tested them for ten days. \v{15}At the end of ten days their appearance was better and their faces were well-nourished\fnote{Lit. \fbib{were fatter of flesh}} compared to the young men who ate the king's rich food. \v{16}So the guard took away their rich food and wine,\fnote{Lit. \fbib{wine of their drinks}} giving them vegetables. \v{17}As for these four young men, God gave them knowledge, aptitude for learning, and wisdom. Daniel also could understand all kinds of visions and dreams.

\v{18}Then at the end of the training period\fnote{Lit. \fbib{the days}} that the king had established, the chief officer brought them in before Nebuchadnezzar. \v{19}When the king spoke to them, none of them compared to Daniel, Hananiah, Mishael, or Azariah as they stood before the king. \v{20}In every matter of wisdom or understanding that the king discussed with\fnote{Lit. \fbib{king sought from}} them, he found them ten times superior to all the astrologers and enchanters\fnote{Or \fbib{occult practitioners}} in his entire palace.

\v{21}So Daniel remained there in service\fnote{The Heb. lacks \fbib{in service}} until the first year of King Cyrus.\fnote{I.e. until the fall of Babylon as a world empire}
\labelchapt{2}
\passage{King Nebuchadnezzar's Dream}

\chapt{2}
\v{1}During the second year of Nebuchadnezzar's reign, Nebuchadnezzar had dreams that troubled him.\fnote{Lit. \fbib{troubled his spirit}} As a result, he couldn't sleep.\fnote{Lit. \fbib{result, sleep departed from him}} \v{2}So the king gave orders to summon diviners, enchanters,\fnote{Or \fbib{conjurers} and so throughout the book} sorcerers, and Chaldeans to reveal to the king what he had dreamed. When they came and stood before him,\fnote{Lit. \fbib{before the king}} \v{3}the king told them, ``I have dreamed a dream and I\fnote{Lit. \fbib{and my spirit}} will remain troubled until I can understand it.''\fnote{Lit. \fbib{understand the dream}}

\v{4}The Chaldeans responded to the king in Aramaic:\fnote{At this point the text changes from Heb. to Aram. until the end of ch. 7.} ``May the king live forever. Tell the dream to your servants, and we'll reveal its meaning.''

\v{5}In reply the king told the Chaldeans, ``Here is what I have commanded: If you don't tell me both the dream and its meaning, you'll be destroyed and your houses will be reduced to rubble. \v{6}But if you do relate the dream to me as well as its meaning, you'll receive gifts, rewards, and great honor from me. Therefore reveal the dream to me, along with its meaning.''

\v{7}They replied again, ``Let the king tell his servants the dream, and we'll disclose its meaning.''

\v{8}The king responded,\fnote{Lit. \fbib{responded and said}} ``I'm convinced that you're stalling for time because you're aware of what I've commanded. \v{9}So if you don't disclose the dream to me, there will be only one sentence for all of you. You have conspired together to present lies and corrupt interpretations until the situation changes. Now tell me the dream and I'll know that you can reveal its true\fnote{The Aram. lacks \fbib{true}} meaning.''

\v{10}The Chaldeans answered the king directly, ``There's not a single man on earth who can do\fnote{Lit. \fbib{reveal}} what the king has commanded. No king, lord, or ruler has ever asked such a thing from any diviner, enchanter,\fnote{Or \fbib{occult practitioner}} or Chaldean. \v{11}Furthermore, what the king is asking is so difficult that no one can reveal it except the gods---and they don't live with human beings.''

\v{12}At this point, the king flew into a rage\fnote{Lit. \fbib{king was furious and very angry}} and issued an order to destroy all the advisors\fnote{Lit. \fbib{the wise men}} of Babylon. \v{13}When the order went out to kill the advisors,\fnote{Lit. \fbib{the wise men}} they searched for Daniel and his friends to kill them, too.\fnote{The Aram. lacks \fbib{too}}
\passage{Daniel Requests Time to Answer the King}

\v{14}Daniel responded with wisdom and discretion to Arioch, the king's executioner, who had gone out to execute the advisors\fnote{Lit. the \fbib{wise men}} of Babylon. \v{15}He asked\fnote{Lit. \fbib{answered and said}} him,\fnote{Lit. \fbib{asked Arioch, the king's executioner}} ``Why such a harsh decree from the king?'' Then Arioch informed Daniel, \v{16}so Daniel went to ask Nebuchadnezzar\fnote{Lit. \fbib{the king}} for an appointment to see him\fnote{Lit. \fbib{for time}}, and it was granted him so that he could reveal the meaning to the king. \v{17}Then Daniel went home and told his friends Hananiah, Mishael, and Azariah about the king's\fnote{The Aram. lacks \fbib{king's}} command. \v{18}Daniel\fnote{Lit. \fbib{He}} was seeking mercy, in order to ask about this mystery in the presence of the God of heaven, so that Daniel and his friends might not be executed along with the rest of the advisors\fnote{Lit. the \fbib{wise men}} of Babylon.
\passage{The King's Dream is Revealed to Daniel}

\v{19}When the mystery was revealed to Daniel in a vision later that night, Daniel blessed the God of heaven \v{20}and said,

\begin{poetry}
\poeml ``May the name of God be blessed forever and ever; \\
\poemll    wisdom and power are his for evermore. \\
\poeml \v{21}It is God\fnote{Lit. \fbib{he}} who alters the times and seasons, \\
\poemll    and he removes kings and promotes kings. \\
\poeml He gives wisdom to the wise \\
\poemll    and knowledge to the discerning. \\
\poeml \v{22}He reveals what is profoundly mysterious \\
\poemll    and knows what is in the darkness; \\
\poemlll       with him dwells light. \\
\poeml \v{23}To you, God of my ancestors, I give thanks and praise, \\
\poemll    because you have given me wisdom and power; \\
\poeml you have now revealed to me what we asked of you \\
\poemll    by making known to us what the king commanded.''
\end{poetry}

\v{24}After this,\fnote{Lit. \fbib{All on account of that}} Daniel went to Arioch, whom the king had appointed to execute the advisors\fnote{Lit. \fbib{the wise men}} of Babylon. He told him, ``Don't destroy the advisors\fnote{Lit. \fbib{the wise men}} of Babylon. Bring me before the king and I'll explain the meaning to him.''\fnote{Lit. \fbib{the king}}

\v{25}Then Arioch quickly brought Daniel into the king's presence and informed him: ``I've found a man from the Judean captives who will make known the meaning to the king.''
\passage{Daniel Reveals the Meaning of the Dream}

\v{26}King Nebuchadnezzar\fnote{Lit. \fbib{The king}} replied by saying to Daniel (whose Babylonian\fnote{The Aram. lacks \fbib{Babylonian}} name is Belteshazzar), ``Are you able to tell me about the dream\fnote{Lit. \fbib{dream I saw}} and its meaning?''

\v{27}By way of answer, Daniel addressed the king:\fnote{Lit. \fbib{answered and said before the king}}

\begin{poetry}
\poeml ``None of the advisors,\fnote{Lit. \fbib{the wise men}} enchanters,\fnote{Or \fbib{occult practitioners}} diviners, or astrologers\fnote{Or \fbib{those who gaze at entrails}} can explain the secret that the king has requested to be made known.\fnote{The Aram. lacks \fbib{to be made known}} \v{28}But there is a God in heaven who reveals secrets, and he is making known to King Nebuchadnezzar what will happen in the latter days. \\
\poeml ``While you were in bed, the dream and the visions that came to your head were as follows: \v{29}Your majesty,\fnote{Lit. \fbib{O King}, and so throughout the book} when you were in bed, thoughts came to your mind\fnote{The Aram. lacks \fbib{to your mind}} about what would happen in the future, and the Revealer of Secrets has made known to you what will take place. \v{30}As for me, this secret was made known to me, not because my own wisdom is greater than anyone else alive, but in order that the meaning may be made known to the king, and that you might understand the thoughts of your heart. \\
\poeml \v{31}``Your majesty, while you were watching, you observed an enormous statue. This magnificent statue stood before you with extraordinary brilliance. Its appearance was terrifying. \v{32}That statue had a head made\fnote{The Aram. lacks \fbib{made}} of pure gold, with its chest and arms made\fnote{The Aram. lacks \fbib{made}} of silver, its abdomen and thighs made\fnote{The Aram. lacks \fbib{made}} of bronze, \v{33}its legs made\fnote{The Aram. lacks \fbib{made}} of iron, and its feet made\fnote{The Aram. lacks \fbib{made}} partly of iron and partly of clay. \\
\poeml \v{34}``As you were watching, a rock was quarried---but not with human hands---and it struck the iron and clay feet of the statue, breaking them to pieces. \v{35}Then the iron, the clay, the bronze, the silver, and the gold were broken in pieces together and became like chaff from a summer threshing floor that the breeze carries away without leaving a trace.\fnote{Lit. \fbib{trace of them}} Then the rock that struck the statue grew into\fnote{Lit. \fbib{became}} a huge mountain and filled the entire earth. \\
\poeml \v{36}``This was the dream, and we'll now relate its meaning to the king. \v{37}You, your majesty, king of kings---to whom the God of heaven has given the kingdom, the power, the strength, and the glory, \v{38}so that wherever people,\fnote{Lit. \fbib{sons of mankind}} wild animals, or birds of the sky live, he has placed them under your control, giving you dominion over them all---you're that head of gold. \\
\poeml \v{39}``After you, another kingdom will arise that is inferior to\fnote{Or \fbib{lower than}} yours, and then a third kingdom of bronze will arise to rule all the earth. \v{40}Then there will be a fourth kingdom, as strong as iron. Just as all things are broken to pieces and shattered by iron, so it will shatter and crush everything. \\
\poeml \v{41}``The feet and toes that you saw, made partly of potter's clay and partly of iron, represent\fnote{The Aram. lacks \fbib{represent}} a divided kingdom. It will still have the strength of iron, in that you saw iron mixed with clay. \v{42}Just as their toes and feet are part iron and part clay, so will the kingdom be both strong and brittle. \v{43}Just as you saw iron mixed with clay, so they will mix themselves with human offspring.\fnote{Lit. \fbib{seed}} Furthermore,\fnote{Or \fbib{will develop alliances through intermarriages, and}} they won't remain together, just as iron doesn't mix with clay. \\
\poeml \v{44}``During the reigns of those kings, the God of heaven will set up a kingdom that will never be destroyed, nor its sovereignty\fnote{Or \fbib{kingdom}} left in the hands of another people. It will shatter and crush all of these kingdoms, and it will stand forever. \v{45}Now, just as you saw that the stone was cut out of the mountain without human hands---and that it crushed the iron, bronze, clay, silver, and gold to pieces---so also the great God has revealed to the king what will take place after this. Your dream will come true, and its meaning will prove trustworthy.''
\end{poetry}
\passage{Nebuchadnezzar Promotes Daniel and His Friends}

\v{46}Then King Nebuchadnezzar fell on his face before Daniel, paid honor to him, and commanded that an offering and incense be presented on his behalf. \v{47}The king told Daniel, ``Truly your God is the God of gods, the Lord of kings, and the Revealer of Secrets, because you were able to reveal this mystery.'' \v{48}Then the king promoted Daniel to a high position and lavished many great gifts on him, including making him ruler over the entire province of Babylon and chief administrator over the advisors\fnote{Lit. \fbib{the wise men}} of Babylon. \v{49}Moreover, Daniel requested that the king appoint Shadrach, Meshach, and Abednego administrators over the province of Babylon, while Daniel himself remained in the royal court.
\labelchapt{3}
\passage{Dedicating the Image to Nebuchadnezzar}

\chapt{3}
\v{1}Some time later, king Nebuchadnezzar built a golden statue, making it 60 cubits\fnote{I.e. about 90 feet; a cubit was about eighteen inches} high and six cubits\fnote{I.e. about nine feet; a cubit was about eighteen inches} wide. He set it up in the Dura Valley\fnote{Or \fbib{Plain}} within the province of Babylon. \v{2}Then King Nebuchadnezzar summoned the regional authorities,\fnote{Or \fbib{satraps}} governors, deputy governors, advisors, treasurers, judges, magistrates, and all of the other\fnote{The Aram. lacks \fbib{other}} administrators of the provinces, ordering them to come to the dedication of the statue that he\fnote{Lit. \fbib{Nebuchadnezzar}} had erected.

\v{3}So the regional authorities,\fnote{Or \fbib{satraps}} governors, deputy governors, advisors, treasurers, judges, magistrates, and all of the other\fnote{The Aram. lacks \fbib{other}} administrators of the provinces assembled to dedicate the statue that King Nebuchadnezzar had erected. They took their places in front of the statue that he\fnote{Lit. \fbib{Nebuchadnezzar}} had erected. \v{4}Then a herald proclaimed aloud:

\begin{poetry}
\poeml ``People of all\fnote{The Aram. lacks \fbib{of all}} nations, and languages are commanded: \v{5}Whenever you hear the sound of the trumpet, the flute, the lyre, the four-stringed lyre, and the harp, playing together along with various instruments, you are to fall down and worship the golden statue that was set up by King Nebuchadnezzar. \v{6}Anyone who does not fall down and worship is immediately to be thrown into the blazing fire furnace.''
\end{poetry}

\v{7}Therefore, when all of the people ``heard the sound of the trumpet, the flute, the lyre, the four-stringed lyre, and the harp, playing together along with various other\fnote{The Aram. lacks \fbib{other}} instruments,'' all the ``people, nations, and languages'' began to fall down and worship the golden statue that King Nebuchadnezzar had set up.
\passage{Daniel's Friends are Accused}

\v{8}Just then, certain influential Chaldeans took this opportunity to come forward and denounce the Jews. \v{9}They told King Nebuchadnezzar, ``Your majesty, live forever. \v{10}You, your majesty, issued this decree:

\begin{poetry}
\poeml `Every man who hears the sound of the trumpet, the flute, the lyre, the four-stringed lyre, and the harp, playing together along with various other\fnote{The Aram. lacks \fbib{other}} instruments is to fall down and worship the golden statue. \v{11}Whoever does not fall down and worship is to be thrown into a blazing fire furnace.'
\end{poetry}

\v{12}``Certain influential Jewish men whom you appointed to manage the city of Babylon---Shadrach, Meshach, and Abednego---have neither paid attention to you, your majesty, nor served your gods. And they won't worship the golden statue that you set up.''
\passage{The Threat of the Fire Furnace}

\v{13}Nebuchadnezzar flew into a rage and furiously ordered that Shadrach, Meshach, and Abednego be brought before him.\fnote{Lit. \fbib{before the king}} \v{14}Nebuchadnezzar asked them, ``Is it true, Shadrach, Meshach, and Abednego, that you don't worship my gods and that you don't worship the golden statue that has been set up? \v{15}Now, if you are ready at this very moment to obey `the sound of the trumpet, the flute, the lyre, the four-stringed lyre, and the harp,' and worship the image that I have made{\ldots} If you do not so worship, you will immediately have cast yourselves into the middle of the blazing fire, and what god is there who can deliver you from my power?''\fnote{Lit. \fbib{hands}}
\passage{Daniel's Friends Answer King Nebuchadnezzar}

\v{16}Shadrach, Meshach, and Abednego answered King Nebuchadnezzar, ``It's not necessary for us to respond in this matter. \v{17}Your majesty, if it be his will,\fnote{The Aram. lacks \fbib{his will}} our God whom we serve can deliver us from the blazing fire furnace, and he will deliver us from you.\fnote{Lit. \fbib{from your hand}} \v{18}But if not, rest assured, your majesty, that we won't serve your gods, and we won't worship the golden statue that you have set up.''
\passage{The King Orders an Execution}

\v{19}Out of control with rage, Nebuchadnezzar's facial expression changed toward Shadrach, Meshach, and Abednego, and he ordered\fnote{Lit. \fbib{answered and ordered}} that the furnace be heated seven times hotter than usual. \v{20}Then he issued orders to his elite guard to bind Shadrach, Meshach, and Abednego with ropes\fnote{The Aram. lacks \fbib{with ropes}; and so through vs. 24} and throw them into the blazing fire furnace. \v{21}So the elite guard tied them up fully clothed, still wearing their robes, tunics, and turbans, and threw them into the blazing fire furnace, \v{22}because the king's command was so drastic. Since the furnace was blazing hot, its flames killed those who threw Shadrach, Meshach, and Abednego into the blazing fire. \v{23}Bound firmly with ropes, these three men Shadrach, Meshach, and Abednego fell into the blazing fire furnace.
\passage{The Fourth Man in the Furnace}

\v{24}Astonished, King Nebuchadnezzar stood up in terror, and asked his advisers, ``Didn't we throw three men into the fire, bound firmly with ropes?''

In reply they told the king, ``Yes, your majesty.''

\v{25}``Look!'' he told them,\fnote{Lit. \fbib{answered and said}} ``I see four men walking untied and unharmed in the middle of the fire, and the appearance of the fourth resembles a divine being.''\fnote{Lit. \fbib{a son of the gods}}

\v{26}Then Nebuchadnezzar approached the opening of the blazing fire furnace. He shouted out, ``Shadrach, Meshach, and Abednego, servants of the Most High God, come out and come here!'' So Shadrach, Meshach, and Abednego came out of the fire. \v{27}The regional authorities,\fnote{Or \fbib{satraps}} viceroys, governors, and royal advisors gazed at those men and saw that the fire had no effect on their bodies---not a hair on their head was singed, their clothes were not burned, and they did not smell of fire.

\v{28}Nebuchadnezzar spoke up and announced:

\begin{poetry}
\poeml ``Blessed be the God of Shadrach, Meshach and Abednego! He sent his angel to deliver his servants who trusted in him. They disobeyed the king's command and were willing to risk their lives in order not to serve or worship any god except their own God. \v{29}So I decree that people from any nation or language who say anything against the God of Shadrach, Meshach and Abednego will be destroyed and their house reduced to rubble, because there is no other god who can save like this.''
\end{poetry}

\v{30}Then the king promoted Shadrach, Meshach, and Abednego within the province of Babylon.
\labelchapt{4}
\passage{Nebuchadnezzar's Testimonial}

\chapt{4}
\v{1}\fnote{This v. is 3:31 in MT, and so through v. 3.}\divine{An Official Statement}\fnote{The Aram. lacks \fbib{An Official Statement}}

\divine{from Nebuchadnezzar}

\divine{the King}

To the people of all nations and languages who live on earth.

Peace and prosperity to you!

\begin{poetry}
\poeml \v{2}It gives me great pleasure to tell about the signs and wonders that the Most High God has done for me. \\
\poeml \v{3}How great are his signs! \\
\poeml How powerful are his wonders! \\
\poeml His kingdom is an eternal kingdom, and his dominion lasts from generation to generation.
\passage{Nebuchadnezzar's Dream}
\poeml \v{4}\fnote{This v. is 4:1 in MT, and so throughout the chapter.}I, Nebuchadnezzar, was resting in my home and prospering in my palace. \v{5}I had a dream that made me afraid. The thoughts that went through my mind while in bed and the visions in my head terrified me. \v{6}So I gave an order to bring in all of the advisors\fnote{Lit. \fbib{the wise men}} of Babylon so they would tell me the interpretation of the dream. \\
\poeml \v{7}Then the diviners, enchanters,\fnote{Or \fbib{occult practitioners}} Chaldeans, and astrologers\fnote{Or \fbib{those who gaze at entrails}} came in, and I told them the dream. But they could not reveal its interpretation to me. \v{8}Eventually, Daniel appeared before me. (He is called Belteshazzar, in accordance with the name of my god, and the spirit of the holy gods is within him.) I told him my dream: \\
\poeml \v{9}``Belteshazzar, chief of the diviners, since I know that the spirit of the holy gods is within you, and no mystery too difficult for you, explain to me the vision of my dream that I saw, along with its interpretation. \v{10}This is what I saw in the visions of my head while I was in bed: I was looking and---listen carefully!---I saw a tree in the middle of the earth, the height of which was very great. \v{11}The tree grew large, became strong, and its top reached the sky. It could be seen to the ends of the earth. \v{12}Its foliage was beautiful, its fruit bountiful, and its food sufficient for everyone. The animals of the field found shade under it, the birds of the sky lived in its branches, and every creature was fed from it. \\
\poeml \v{13}``Then I saw in the visions of my head while I was in bed---and take careful notice!---I saw a holy observer descend from heaven. \v{14}He called out aloud: \\
\poeml `Cut down the tree and cut off its branches. Strip off its foliage and scatter its fruit. Let the animals get out from under it, and let the birds leave\fnote{The Aram. lacks \fbib{leave}} its branches. \v{15}Nevertheless, leave the stump and its roots in the ground, but bind it with iron and bronze in the field grass. Let him be drenched with dew from the sky, and let him graze with the animals in the grass of the earth. \v{16}Let his mind be changed from that of a man, and let him be given the mind of an animal until seven seasons of time pass by for him. \v{17}This order is announced by the observers, and the holy ones declare the verdict, so that the living may know that the Most High is sovereign over human kingdoms and grants them to whomever he desires, and he places the least important of men over them.' \\
\poeml \v{18}``This is the dream that I, King Nebuchadnezzar, saw. Belteshazzar, tell me its meaning, since none of the advisors\fnote{Lit. \fbib{the wise men}} in my kingdom can tell me its interpretation. But you are able to do so\fnote{The Aram. lacks \fbib{to do so}} because the spirit of the holy gods is in you.''
\end{poetry}
\passage{Daniel's Interpretation}

\v{19}Then Daniel (also known as Belteshazzar) was greatly troubled for a while and was terrified by his thoughts. The king said, ``Belteshazzar, don't let the dream or its meaning terrify you.''

Belteshazzar responded, ``Your majesty, if only\fnote{The Aram. lacks \fbib{if only}} the dream were about your enemies and its meaning about those who oppose you! \v{20}The tree that you saw, which grew large and strong until its top reached the sky and became visible to the whole earth \v{21}with beautiful leaves and abundant fruit---enough food for everyone---and under which wild animals of the field found shelter and the birds of the air had nests in its branches--- \v{22}it's you, your majesty! You've become great and strong, your greatness has grown to the heavens, and your dominion reaches to the distant parts of the earth.

\v{23}``Your majesty saw a holy observer descending from heaven and saying, `Cut down the tree and destroy it, but leave the stump in the ground, along with its roots, bound with iron and bronze in the field grass. Let him be soaked with the dew of the sky and live with the wild animals of the field until seven seasons pass over him.'

\v{24}``This is the meaning, your majesty, and this is the decree that the Most High has issued against his majesty, the king: \v{25}You'll be driven from people, and you'll live among wild animals of the field. You'll eat grass like cattle and be soaked with the dew of the sky while seven years pass you by\fnote{Lit. \fbib{seven seasons pass over you}}---until you realize that the Most High is sovereign over human kingdoms and grants them to whomever he desires. \v{26}Just as it was ordered to leave the stump of the tree in the ground\fnote{The Aram. lacks \fbib{in the ground}} along with its roots, so your kingdom will be restored to you when you realize that Heaven rules over everything.\fnote{The Aram. lacks \fbib{everything}} \v{27}Therefore, your majesty, may my advice be acceptable to you: Stop your sinning, do what's right, and put a stop to your wickedness by showing kindness to the oppressed. Perhaps your tranquility will continue.''
\passage{The Dream Comes True}

\v{28}All of this happened to King Nebuchadnezzar. \v{29}About a year later,\fnote{Lit. \fbib{At the end of 12 months}} as the king was walking on the roof of the royal palace of Babylon, \v{30}he\fnote{Lit. \fbib{the king}} commented to himself,\fnote{Lit. \fbib{commented and said}} ``Isn't Babylon great? I've built a royal palace in it by my own might and power, for the sake\fnote{Lit. \fbib{glory}} of my majesty.''

\v{31}As these words were being spoken by the king, a voice came out of heaven: ``King Nebuchadnezzar, this is declared to you:

\begin{poetry}
\poeml `The kingdom has been taken\fnote{Or \fbib{has departed}} from you! \v{32}You're to be driven away from people. You're to live with the wild animals of the field. You are to be made to eat grass like cattle, and seven years will pass you by\fnote{Lit. \fbib{seven seasons will pass over you}} until you realize that the Most High is sovereign over human kingdoms and grants them to whomever he desires.'\,''
\end{poetry}

\v{33}The decree was fulfilled against Nebuchadnezzar immediately. He was driven away from people to eat grass like cattle, and his body was drenched with dew from the sky, until his hair grew like eagles' feathers and his nails like birds' claws.''
\passage{The King's Sanity Returns}

\v{34}``When that period of time was over, I, Nebuchadnezzar, lifted my eyes to heaven and my sanity returned to me. I blessed the Most High, praising and honoring the one who lives forever:

\begin{poetry}
\poeml For his sovereignty is eternal, \\
\poemll    and his kingdom continues from generation to generation. \\
\poeml \v{35}All who live on the earth \\
\poemll    are nothing compared to him. \\
\poeml He does what he wishes \\
\poemll    with the heavenly armies \\
\poemlll       and with those who live on earth. \\
\poeml No one can hold back his power \\
\poemll    or say to him, `What did you do?'
\end{poetry}

\v{36}At that moment I recovered my sanity, and my honor and majesty returned to me, for the sake\fnote{Lit. \fbib{glory}} of my kingdom. My advisors and officials sought me out, my throne was restored, and even more greatness than I had before was added to me. \v{37}In conclusion, I, Nebuchadnezzar, praise, exalt, and give glory to the King of heaven:

\begin{poetry}
\poeml For everything he does is true, \\
\poemll    his ways are just, \\
\poemlll       and he is able to humble those who walk in pride.''
\end{poetry}
\labelchapt{5}
\passage{Belshazzar's Festival}

\chapt{5}
\v{1}King Belshazzar put on a great festival for a thousand of his officials. He joined all\fnote{The Aram. lacks \fbib{all}} one thousand of them in getting drunk. \v{2}Under the influence of wine, Belshazzar ordered that the gold and silver vessels his grandfather Nebuchadnezzar had taken from the Temple in Jerusalem be brought in so the king, his officials, his wives, and his mistresses\fnote{Or \fbib{concubines}; i.e. secondary wives} could drink from them. \v{3}As ordered, they brought in the gold vessels that had been taken from the sanctuary of God's Temple in Jerusalem, and the king, his officials, his wives, and mistresses\fnote{Or \fbib{concubines}; i.e. secondary wives} drank from them. \v{4}As they drank the wine, they praised gods of gold, silver, bronze, iron, wood, and stone.
\passage{The Handwriting on the Wall}

\v{5}At that moment, humanlike fingers of a hand appeared near the lamp stand of the royal palace and wrote on the plaster of the wall. \v{6}While the king watched the back of the hand as it was writing, his facial expression changed. Utterly frightened, he lost control of his own bowels\fnote{Lit. \fbib{frightened, the joints of his loins were loosened}; i.e. an involuntary physiological response from terror} and his knees knocked together.

\v{7}The king cried out to bring in enchanters,\fnote{Or \fbib{occult practitioners}} Chaldeans, and astrologers. He announced to the advisors\fnote{Or \fbib{the wise men}} of Babylon, ``Whoever can read this writing and tell me its meaning will be clothed in purple, have a gold chain placed around his neck, and will become the third highest ruler in the kingdom.''

\v{8}Then all the king's advisors came in, but they were unable to read the writing or tell the king what it meant. \v{9}So King Belshazzar became even more frightened, and his facial expression showed it. His officials also were thrown into confusion.

\v{10}Hearing\fnote{The Aram. lacks \fbib{Hearing}} the voices of the king and his officials, the queen entered the banquet hall. ``Your majesty, live forever,'' the queen said. ``Don't be frightened by your thoughts or allow your facial expression to show it. \v{11}There's a man in your kingdom in whom dwells\fnote{The Aram. lacks \fbib{dwells}} the spirit of the holy gods. During your grandfather's reign, he was found to have insight, intelligence, and wisdom, like that\fnote{Lit. \fbib{like the wisdom}} of the gods. Your grandfather, King Nebuchadnezzar---your kingly predecessor---appointed him to be chief administrator over the magicians, enchanters,\fnote{Or \fbib{occult practitioners}} Chaldeans, and astrologers, \v{12}because he was found to have an extraordinary spirit, knowledge, and understanding, along with an ability to interpret dreams, explain riddles, and solve difficult problems. His name is Daniel, whom the king renamed Belteshazzar. Call for Daniel, and he will reveal the meaning of the writing.''\fnote{The Aram. lacks \fbib{of the writing}}
\passage{Daniel Interprets the Handwriting}

\v{13}Then Daniel was brought before the king. The king spoke up and told Daniel, ``So you are Daniel, one of the Judean exiles whom my grandfather the king brought from Judah! \v{14}I've heard about you, that a spirit of the gods is in you and that you have insight, discernment, and extraordinary wisdom. \v{15}Take note that the advisors\fnote{Lit. the \fbib{wise men}} and enchanters\fnote{Or \fbib{occult practitioners}} were brought before me to read the writing and explain its meaning, but they were unable to do so.\fnote{Lit. \fbib{to tell the matter}} \v{16}However, I've heard that you can provide meaning and interpretation, and that you can solve difficult problems. If you are able to read the writing and report its meaning, you will be clothed in purple, have a gold chain placed around your neck, and you will become the third highest ruler in the kingdom.''

\v{17}At this, Daniel answered, speaking directly to\fnote{Lit. \fbib{speaking before}} the king, ``Let your gifts and rewards be given to someone else. However, I'll read the writing for the king and tell him its meaning. \v{18}Your majesty, the Most High God gave your grandfather Nebuchadnezzar sovereignty, as well as greatness, glory, and splendor. \v{19}And because of the greatness that he gave him, all peoples, nations, and languages revered and feared him. He executed those whom he desired to execute, he spared those whom he wished to spare, he promoted those whom he desired to promote, and he humbled those whom he wished to humble. \v{20}But when he\fnote{Lit. \fbib{his heart}} became arrogant and his spirit hardened, he was removed from his royal throne and his glory was taken away from him. \v{21}He was driven away from human society\fnote{Lit. \fbib{sons of the people}} and given the mind of an animal. He lived with wild donkeys, ate grass like cattle, and his body was soaked with dew from the sky until he realized that the Most High God is sovereign over human kingdoms and places over them whomever he desires.

\v{22}``But you, Belshazzar, his grandson, haven't humbled yourself, even though you knew all of this.

\v{23}``You've exalted yourself against the Lord of heaven.

``You've had the vessels from his Temple brought into your presence.

``And you, your officials, and your wives and mistresses drank wine from them.

``You praised gods of silver, gold, bronze, iron, wood, and stone, which can't see, hear, or demonstrate knowledge.

``But you didn't honor God, who holds in his power your very life and all your ways.

\v{24}``Therefore, the hand\fnote{Lit. \fbib{the palm of the hand}} that wrote this inscription was sent from his presence. \v{25}This is the written inscription:

\divine{MENE, MENE, TEKEL and PARSIN}

\v{26}These are the meanings of the words:

\begin{poetry}
\poeml MENE: God has audited\fnote{Lit. \fbib{numbered}} your kingdom---and has ended it. \\
\poeml \v{27}TEKEL: You've been weighed on the scales---and you don't measure up.\fnote{Lit. \fbib{and found lacking}} \\
\poeml \v{28}PERES: Your kingdom has been divided---and will be given to the Medes and Persians.''
\end{poetry}

\v{29}Then Belshazzar gave orders to clothe Daniel in purple, to place a chain of gold around his neck, and to proclaim him the third highest ruler of the kingdom.

\v{30}That night Belshazzar, king of the Chaldeans, was killed, \v{31}\fnote{This v. is 6:1 in MT}and Darius the Mede took over the kingdom at the age of 62.
\labelchapt{6}
\passage{Daniel's Service to Darius}

\chapt{6}
\v{1}\fnote{This v. is 6:2 in MT, and so throughout the chapter.}It pleased Darius to appoint 120 regional authorities\fnote{Or \fbib{satraps}} over the kingdom throughout the realm, \v{2}along with three chief administrators from them, one of which was Daniel. The regional authorities\fnote{Or \fbib{satraps}} reported to these three administrators,\fnote{The Aram. lacks \fbib{three administrators}} so that the king would experience no losses. \v{3}Daniel distinguished himself among all the administrators and regional authorities,\fnote{Or \fbib{satraps}} because he was of an extraordinary spirit. Therefore the king planned to appoint him over the whole kingdom.
\passage{A Plot to Destroy Daniel}

\v{4}Because of this, the administrators and regional authorities\fnote{Or \fbib{satraps}} tried to bring allegations of dereliction of duty in government affairs against Daniel, but they were unable to find any charges of corruption. Daniel\fnote{Lit. \fbib{he}} was trustworthy, and no evidence of\fnote{The Aram. lacks \fbib{of evidence}} negligence or corruption could be found against him. \v{5}So these men said, ``We'll never find any basis for complaint against Daniel unless we build it on the requirements of his God.''

\v{6}Then these administrators and regional authorities\fnote{Or \fbib{satraps}} went as a group to the king and said this, ``Your majesty, live forever! \v{7}All of the royal administrators, prefects, regional authorities,\fnote{Or \fbib{satraps}} scribes, and governors have concluded that the king should establish and enforce an edict that anyone who prays to any god or man for the next 30 days (except to you, your majesty) is to be thrown into the lions' pit. \v{8}Therefore, your majesty, establish the decree and sign the written document so it can't be changed, in accordance with the laws of the Medes and Persians that can't be repealed.'' \v{9}So King Darius signed the edict contained in the written document.
\passage{Daniel is Accused}

\v{10}When Daniel learned that the written document had been signed, he went to an upstairs room in his house that had windows opened facing Jerusalem. Three times a day he would kneel down, pray, and give thanks to his God, just as he had previously done.

\v{11}The conspirators\fnote{Lit. \fbib{These men}} then went as a group and found Daniel praying and seeking help before his God. \v{12}So they approached the king and asked, ``Didn't you sign an edict that for the next 30 days if anyone prays to any god or man, except to you, your majesty, he would be thrown into the lions' pit?''

The king responded, ``The decree has been established, in accordance with the laws of the Medes and Persians that can't be repealed.''

\v{13}Then they told the king, ``Daniel, who is one of the Judean exiles, pays no attention to you, your majesty, or to the written decree, since he is still praying three times a day.''

\v{14}When the king heard this, he was greatly upset, because he was determined to make every effort to save Daniel before the sun set. \v{15}But the men who had gone as a group to the king told him,\fnote{Lit. \fbib{the king}} ``Remember, your majesty, that according to the laws of the Medes and Persians, any decree or edict that the king establishes cannot be repealed.''
\passage{Daniel in the Lions' Pit}

\v{16}At this point, the king ordered Daniel brought in and thrown into the lions' pit. The king spoke to Daniel, ``Your God, whom you serve constantly, will deliver you himself.'' \v{17}A stone was brought and placed over the opening to the pit, and the king affixed a seal to it with his personal signet ring and with the signet rings of his officials so that no one would interfere with Daniel's situation. \v{18}Then the king retired to his palace to spend the night fasting. He enjoyed no entertainment, and he couldn't sleep.

\v{19}The king got up at dawn and went quickly to the lions' pit. \v{20}As he approached where Daniel was in the pit, he cried out to him\fnote{Lit. \fbib{Daniel}} in a voice filled with anguish, ``Daniel, servant of the living God, has your God, whom you serve constantly, been able to deliver you from the lions?''

\v{21}Daniel replied to the king, ``May your majesty live forever! \v{22}My God sent his angel and sealed the mouths of the lions. They have not harmed me, proving that I'm innocent before him. Also against you, your majesty, I've committed no offense.''

\v{23}The king was ecstatic, so he gave orders for Daniel to be released from the pit. Daniel was taken up from the pit, and no injury was found to have been inflicted on him, because he had believed in his God. \v{24}Then the king gave orders to bring those men who had tried to have Daniel devoured, and they threw them, their children, and their wives into the lions' pit. They had not reached the floor of the pit before the lions had overtaken them and crushed all their bones.
\passage{Darius Exonerates Daniel}

\v{25}Afterward, King Darius wrote to all peoples, nations, and languages who lived throughout his realm:

``May great prosperity be yours!

\v{26}``I hereby decree that in every area of my kingdom men\fnote{Lit. \fbib{they}} are to fear and tremble before the God of Daniel.

\begin{poetry}
\poeml For he is the living God, \\
\poemll    who endures forever. \\
\poeml His kingdom is one that will not be destroyed, \\
\poemll    and his dominion continues forever. \\
\poeml \v{27}He delivers and rescues \\
\poemll    and performs signs and wonders \\
\poemlll       in heaven and on earth. \\
\poeml He has delivered Daniel \\
\poemll    from the power of the lions.''
\end{poetry}

\v{28}Daniel achieved success during the reigns of Darius and Cyrus the Persian.
\labelchapt{7}
\passage{The Vision of the Four Beasts}

\chapt{7}
\v{1}In the first year of the reign of King Belshazzar of Babylon, Daniel dreamed a dream, receiving visions in his mind while in bed, after which he recorded the dream, relating this summary of events.

\v{2}Daniel said, ``I observed the vision during the night. Look! The four winds of the skies were stirring up the Mediterranean\fnote{Lit. \fbib{Great}} Sea. \v{3}Four magnificent animals were rising from the sea, each different from the other. \v{4}The first resembled a lion, but it had eagles' wings. I continued to watch until its wings were plucked off, it was lifted up off the ground, and it was forced to stand on two feet like a man. A human soul\fnote{Lit. \fbib{heart}} was imparted to it.

\v{5}``Then look!---a second animal resembling a bear followed it.\fnote{The Aram. lacks \fbib{it}} It was raised up on one side, with three ribs held between the teeth in its mouth. Therefore people kept telling it, `Get up and devour lots of meat!'

\v{6}``After this, I continued to watch---and look!---there was another one, resembling a leopard with four birds' wings on its back. The animal also had four heads, and authority was imparted to it.

\v{7}``After this, I continued to observe the night visions. And look!---there was a fourth awe-inspiring, terrifying, and viciously strong animal! It had large, iron teeth. It devoured and crushed things,\fnote{The Aram. lacks \fbib{things}} and trampled under its feet whatever remained. Different from all of the other previous animals, it had ten horns.

\v{8}``While I was thinking about the horns---look---another horn, this time\fnote{The Aram. lacks \fbib{this time}} a little one, grew up among them. Three of the first horns were yanked up by their roots right in front of it. Look! It had eyes like those of a human being and a mouth that boasted with audacious claims.''
\passage{The Vision of the Ancient of Days}

\v{9}``I kept on watching until the Ancient of Days was seated. His clothes were white, like snow, and the hair on his head was like pure wool. His throne burned with flaming fire, and its wheels burned with fire. \v{10}A river of fire flowed out from before him. Thousands upon thousands were serving him, with millions upon millions waiting before him. The court sat in judgment,\fnote{The Aram. lacks \fbib{in judgment}} and record books were unsealed.

\v{11}``I continued watching because of the audacious words that the horn was speaking. I kept observing until the animal was killed and its body destroyed and given over to burning fire. \v{12}Now as to the other animals, their authority was removed, but they were granted a reprieve from execution\fnote{Lit. \fbib{a prolonging of life}} for an appointed period of time.''
\passage{The Vision of the Son of Man}

\v{13}``I continued to observe the night vision---and look!---someone like the Son of Man was coming, accompanied by heavenly clouds. He approached the Ancient of Days and was presented before him. \v{14}To him dominion was bestowed, along with glory and a kingdom, so that all peoples, nations, and languages are to serve him. His dominion is an everlasting dominion---it will never pass away---and his kingdom is one that will never be destroyed.''
\passage{The Vision Interpreted}

\v{15}``Now as for me, Daniel, I was emotionally troubled, and what I had seen in the visions kept alarming me. \v{16}So I approached one of those who was standing nearby and began to ask the meaning of all of this. He spoke to me and caused me to understand the interpretation of these things. \v{17}He said, `These four great animals are four kings who will rise to power from the earth. \v{18}But the saints of the Highest will receive the kingdom forever, inheriting it\fnote{The Aram. lacks \fbib{inheriting it}} forever and ever.'

\v{19}``I wanted to learn the precise significance of the fourth animal that was different from all the others, extremely awe-inspiring, with iron teeth and bronze claws, and that had devoured and crushed things,\fnote{The Aram. lacks \fbib{things}} trampling under its feet whatever remained. \v{20}Also, I wanted to learn the significance of\fnote{The Aram. lacks \fbib{I wanted to learn the significance of}} the ten horns on its head and the other horn that had arisen, before which three of them had fallen---that is, the horn with eyes and a mouth that uttered magnificent things and which was greater in appearance than its fellows.

\v{21}``As I continued to watch, that same horn waged war against the saints, and was prevailing against them \v{22}until the Ancient of Days arrived to pass judgment in favor of the saints of the Highest One and the time came for the saints to take possession of the kingdom. \v{23}So he said:

\begin{poetry}
\poeml `The fourth animal will be a fourth kingdom on the earth, different from all the kingdoms. It will devour the entire earth, trampling it down and crushing it. \v{24}Now as to the ten horns, ten kings will rise to power from this kingdom, and another king\fnote{The Aram. lacks \fbib{king}} will rise to power after them. He will be different from the previous kings,\fnote{The Aram. lacks \fbib{kings}} and will defeat three kings. \v{25}He'll speak out against the Most High and wear down the saints of the Highest One. He'll attempt to alter times and laws, and they'll be given into his control for a time, times, and half a time. \v{26}Nevertheless, the court will convene, and his authority will be removed, annulled, and destroyed forever. \v{27}Then the kingdom, authority, and magnificence of all nations of the earth\fnote{Lit. \fbib{of the kingdoms under the whole heaven}} will be given to the people who are the saints of the Highest One. His kingdom will endure forever, and all authorities will serve him and obey him.'
\end{poetry}

\v{28}``At this point the vision ended. As for me, Daniel, my thoughts continued to alarm me, and I lost my natural color, but I kept quiet about the matter.''\fnote{Lit. \fbib{kept the matter in my heart}}
\labelchapt{8}
\passage{The Vision of the Ram and Goat}

\chapt{8}
\v{1}\fnote{At this point the text reverts to Heb. for the rest of the book.}``During the third year of King Belshazzar's reign, I, Daniel, saw a vision after the earlier vision that had appeared to me. \v{2}As I observed the vision, I looked around the citadel of Susa in Elam Province. While I watched, I found myself beside the Ulai Canal. \v{3}``Then I turned my head\fnote{Lit. \fbib{eyes}} to look, and to my surprise, a two-horned ram was standing beside the canal. The two horns grew long,\fnote{Or \fbib{higher}; Lit. \fbib{horns were exalted}} the first one growing longer than\fnote{Lit. \fbib{one exalted from}} the second, with the longer one springing up last. \v{4}I watched the ram charging westward, northward, and southward. No animal could stand before him, nor was there anyone who could deliver from his control.\fnote{Lit. \fbib{hand}} He did as he pleased and exalted himself.

\v{5}``As I watched and wondered, a male goat was coming from the west over the surface of the entire earth without touching the ground. The goat had a distinctive horn between its eyes. \v{6}It approached the ram with the two horns that I had observed while standing beside the canal, and charged at him, out of control with rage.\fnote{Lit. \fbib{him in his mighty wrath}} \v{7}I saw it approach the ram, overflowing with fury at him, and run into him with the full force of its strength. The goat\fnote{Lit. \fbib{It}} shattered the ram's\fnote{Lit. \fbib{shattered his}} two horns, and the ram could not oppose it. So the goat\fnote{Lit. \fbib{it}} threw him to the ground and trampled him. No one could rescue the ram from its control.\fnote{Lit. \fbib{hand}} \v{8}Then the goat grew extremely great, but when it was strong, its great horn was shattered. In its place, four distinctive horns grew out in all directions.''\fnote{Lit. \fbib{out to the four winds of heaven}}
\passage{The Insignificant Horn}

\v{9}``A somewhat insignificant horn emerged from one of them. It moved\fnote{Or \fbib{expanded} and so throughout the chapter} rapidly\fnote{Or \fbib{remarkably}} against the south, against the east, and against the Glory.\fnote{Or \fbib{Beauty}; i.e. God} \v{10}Then it moved against the Heavenly Army. It persuaded some of the Heavenly Army to fall to the earth, along with some of the stars, and it trampled them. \v{11}Then it set itself in arrogant opposition to the Prince of the Heavenly Army, from whom the regular burnt offering was taken away, in order to overthrow his sanctuary. \v{12}Because of the transgression, the Heavenly Army will be given over, along with the regular burnt offering, and in that rebellion truth will be cast to the ground, while he continues to prosper and to act.''
\passage{The Duration of the Desolation}

\v{13}``Then I heard one holy person speaking, and another holy person addressed the one who was speaking: `In the vision about the regular burnt offering, how much time elapses while the desecration terrifies and both the Holy Place and the Heavenly Army are trampled?'

\v{14}``He told me, `For 2,300 days.\fnote{Lit. \fbib{2,300 twilights and dawnings}} Then the Holy Place will be restored.'\,''
\passage{Gabriel Interprets the Vision}

\v{15}``After I, Daniel, had seen the vision, I tried to understand it. All of a sudden, there was standing in front of me one who appeared to be valiant. \v{16}I heard the voice of a man calling out from the Ulai Canal,\fnote{The Heb. lacks \fbib{Canal}} `Gabriel, interpret what that fellow has been seeing.'

\v{17}``As he approached where I was standing, I became terrified and fell on my face. But he told me, `Son of man, understand that the vision pertains to the time of the end.'

\v{18}``While he had been speaking with me, I had fainted\fnote{Lit. \fbib{had fallen into a deep sleep}} on my face, but he touched me and enabled me to stand upright on my feet. \v{19}Then he said,

\begin{poetry}
\poeml `Pay attention! I'm going to brief you about what will happen at the end of the period of wrath, because its end is appointed. \v{20}The ram that you saw with a pair of horns are the kings of Media and Persia. \v{21}The demonic\fnote{Lit. \fbib{shaggy}} goat is the king of Greece,\fnote{Lit. \fbib{Javan}} and the great horn between its eyes is its first king. \v{22}The shattered horn\fnote{The Heb. lacks \fbib{horn}} and the four that took its place are four kingdoms that will come from his nation, but they will not have his strength. \\
\poeml \v{23}``Toward the end of their rule, as the desecrations proceed, an insolent king will arise, proficient at deception. \v{24}Mighty will be his skills, but not from his own abilities. He'll be remarkably destructive, will succeed, and will do whatever he wants, destroying mighty men and the holy people. \v{25}Through his skill he'll cause deceit to prosper under his leadership. He'll promote himself and will destroy many while they are secure. He'll take a stand against the Prince of Princes, yet he'll be crushed without human help.\fnote{Lit. \fbib{without a hand}} \v{26}The vision about the twilights and dawnings that has been related is trustworthy, but keep its vision secret, because it pertains to the distant future.'
\end{poetry}

\v{27}Then I, Daniel, was exhausted and ill for days, but afterward I got up and went about the king's business. Nevertheless, I was astonished by the vision, and could not understand it.''
\labelchapt{9}
\passage{Daniel's Prayer}

\chapt{9}
\v{1}``In the first year of the reign of Darius son of Ahasuerus, a descendant of the Medes, who was made king over the kingdom of the Chaldeans\fnote{Or \fbib{Babylonians}}--- \v{2}in the first year of his reign I, Daniel, noted in the Scripture the total years that were assigned\fnote{The Heb. lacks \fbib{assigned}} by the message from the \divine{Lord} to Jeremiah the prophet for the completion of the desolations of Jerusalem: 70 years.

\v{3}``So I turned my attention to the Lord God, seeking him in prayer and supplication, accompanied with fasting, sackcloth, and ashes. \v{4}I prayed to the \divine{Lord} my God, confessing and saying:

\begin{poetry}
\poeml `Lord! Great and awesome God, who keeps his\fnote{The Heb. lacks \fbib{his}} covenant and gracious love for those who love him and obey his commandments, \v{5}we've sinned, we've practiced evil, we've acted wickedly, and we've rebelled, turning away from your commands and from your regulations. \v{6}Furthermore, we haven't listened to your servants, the prophets, who spoke in your name to our kings, to our officials, to our ancestors, and to all of the people of the land. \\
\poeml \v{7}`To you, Lord, belongs righteousness, but to us, open humiliation---even to this day, to the men of Judah, the residents of Jerusalem, and to all Israel, both those who are nearby and those who are far away in all the lands to which you drove them because of their unfaithful acts that they committed against you. \\
\poeml \v{8}`Open humiliation belongs to us, \divine{Lord}, to our kings, our officials, and our ancestors, because we've sinned against you. \v{9}But to the Lord our God belong mercy and forgiveness, though we've rebelled against him \v{10}and have not obeyed the voice of the \divine{Lord} our God by walking in his laws that he gave us through his servants the prophets. \v{11}And all Israel flouted your Law, turning aside from it and not obeying your voice. Because we've sinned against him, the curse has been poured upon us, along with the oath written in the Law of Moses the servant of God. \\
\poeml \v{12}`He has confirmed his accusation\fnote{Lit. \fbib{word}} that he spoke against us and against our rulers who governed us by bringing upon us great calamity, because nowhere in the universe\fnote{Lit. \fbib{because under all of the heavens}} has anything been done like what has been done to Jerusalem. \v{13}As it's written in the Law of Moses,\fnote{Cf. Lev. 26:14-15; Deut 28:15-68} all this calamity has befallen us, but we still haven't sought the \divine{Lord} our God by turning from our lawlessness to pay attention to your truth. \v{14}So the \divine{Lord} watched for the right time to bring the calamity upon us, because the \divine{Lord} our God is righteous regarding everything he does, but we have not obeyed his voice. \\
\poeml \v{15}`And now, Lord our God, who brought your people from the land of Egypt with a mighty hand and who made a name for yourself that remains to this day---we've sinned. We've acted wickedly. \v{16}Lord, in view of all your righteous acts, please turn your anger and wrath away from your city Jerusalem, your holy mountain. Because of our sins and the iniquities of our ancestors, Jerusalem and your people have become an embarrassment to all of those around us. \\
\poeml \v{17}`So now, O\fnote{Lit. \fbib{our}} God, listen to the prayer of your servant and to his requests, and look with favor on your desolate sanctuary, for the sake of the Lord. \v{18}Turn your ear and listen, O God. Open your eyes and look at our desolation and at the city that is called by your name. We're not presenting our requests before you because of our righteousness, but because of your great compassion. \\
\poeml \v{19}`Lord, listen! \\
\poeml `Lord, forgive! \\
\poeml `Lord, take note and take action! \\
\poeml `For your own sake, don't delay, my God, because your city and your people are called by your name.'\,''
\end{poetry}
\passage{Gabriel's Answer: The Seventy Weeks}

\v{20}``While I was still speaking in prayer, confessing my sin and the sin of my people Israel and placing my request in the presence of the \divine{Lord} my God on behalf of the holy mountain of God--- \v{21}while I was still speaking, Gabriel the man of God whom I had seen in the previous vision, appeared to me about the time of the evening offering. \v{22}He gave instructions, and this is what he spoke to me:

\begin{poetry}
\poeml `Daniel, I've now come to give you insight and understanding. \v{23}Because you're highly regarded, the answer was issued when you began your prayer, and I've come to tell you. Pay attention to my message and you'll understand the vision. \v{24}Seventy weeks\fnote{Lit. \fbib{sevens}; i.e. seven time periods of unspecified duration, and so through v. 27} have been decreed concerning your people and your holy city: to restrain transgression, to put an end to sin, to make atonement for lawlessness, to establish everlasting righteousness, to conclude vision and prophecy, and to anoint the Most Holy Place. \v{25}So be informed and discern that seven weeks and 62 weeks will elapse\fnote{The Heb. lacks \fbib{will elapse}} from the issuance of the command to restore and rebuild Jerusalem until the Anointed Commander.\fnote{Lit. \fbib{until Messiah Nagid}; i.e. a senior officer entrusted with dual roles of operational oversight and management authority} The plaza and moat will be rebuilt, though in troubled times. \v{26}Then after the 62 weeks, the anointed one\fnote{Or \fbib{the Messiah}} will be cut down (but not for himself).\fnote{Or \fbib{cut off, and will have no successor}; the Heb. lacks \fbib{successor}} Then the people of the Coming Commander\fnote{Lit. \fbib{Nagid}; i.e. a senior officer entrusted with dual roles of operational oversight and management authority} will destroy both the city and the Sanctuary. Its ending will come like a flood, and until the end there will be war, with desolations having been decreed. \v{27}He will make a binding covenant with many for one week, and for half of the week he will suspend both the sacrifice and grain offerings. Destructive people will cause desolation on the pinnacle until it is complete and what has been decreed is poured out on the desolator.'\,''
\end{poetry}
\labelchapt{10}
\passage{Daniel's Vision}

\chapt{10}
\v{1}In the third year of Cyrus, king of Persia, a message was revealed to Daniel (also known as Belteshazzar). The message was trustworthy and concerned a great conflict. He understood it and had insight concerning the vision.

\v{2}``At that time I, Daniel, had been mourning for three straight weeks.\fnote{Lit. \fbib{for three weeks of days}} \v{3}I ate no fancy foods---neither meat nor wine entered my mouth. Furthermore, I didn't use any ointment until the end of the entire three weeks.\fnote{Lit. \fbib{the three weeks of days}} \v{4}On the twenty-fourth day of the first month, while I was beside the bank of the great Tigris\fnote{Lit. \fbib{Hiddekel}} River, \v{5}I lifted up my eyes to look, and to my surprise, there was a certain man dressed in linen, whose waist was encircled with gold from Uphaz! \v{6}His body was like beryl,\fnote{Lit. \fbib{Tarshish}; a yellow semi-precious stone named after its region of origin} his face flashed like lightning, his eyes were like flaming torches, his arms and legs were like polished bronze, and his speech roared\fnote{The Heb. lacks \fbib{roared}} like that of a crowd.

\v{7}``Now I, Daniel, was the only one to receive the vision---the men who were with me didn't see it.\fnote{Lit. \fbib{see the vision}} However, an enormous fear overwhelmed them, so they ran away to hide, \v{8}and I was left alone to observe this magnificent vision. Nevertheless, no strength remained in me---my face lost its color, and I became weak. \v{9}As I listened to the sound of his words, I fell down on my face unconscious, with my face to the ground.''
\passage{Daniel is Given Understanding}

\v{10}``All of a sudden, a hand touched me and lifted me upon my hands and knees. \v{11}He told me, `Daniel, man highly regarded, understand the message that I'm about to relate to you. Stand up, because I've been sent to you.' When he spoke this statement to me, I stood there trembling.

\v{12}```Don't be afraid, Daniel,' he told me, `because from the first day that you committed yourself to understand and to humble yourself before your God, your words were heard. I've come in answer to\fnote{The Heb. lacks \fbib{answer to}} your prayers. \v{13}However, the prince of the kingdom of Persia opposed me for 21 days. Then all of a sudden, Michael, one of the chief angels,\fnote{Lit. \fbib{princes}} came to assist me! I had been detained there near the kings of Persia. \v{14}Now I've come to help you understand what will happen to your people in the days to come, because the vision pertains to those days.'

\v{15}``After he had spoken to me like this, I bowed my face to the ground, unable to speak. \v{16}But suddenly someone who resembled a human being touched my lips, so addressing the one who was standing in front of me, I opened my mouth and said, `Sir,\fnote{Lit. \fbib{My lord}} I'm overwhelmed with anguish by this vision. I have no strength left.\fnote{The Heb. lacks \fbib{left}} \v{17}So how can a servant of my lord talk with someone like you, sir?\fnote{Lit. \fbib{like my lord}} And as for me, there's no strength left in me, and I can hardly breathe.'

\v{18}``Then this person who looked like a man touched me again and strengthened me \v{19}and said, `Don't be afraid, man highly regarded. Be at peace, and be strong.'

``As soon as he spoke to me, I gained strength and replied, `Sir, please\fnote{Lit. \fbib{May my lord}} speak, now that you've strengthened me.'

\v{20}``Then he said, `Do you understand why I came to you? Soon I'll return to fight the prince of Persia. I'm going forth to war---and take note---the prince of Greece\fnote{Lit. \fbib{Javan}} is coming! \v{21}I'll inform you about what has been recorded in the Book of Truth. No one stands firmly with me against these opponents,\fnote{The Heb. lacks \fbib{opponents}} except Michael your prince.\chapt{11}
\v{1}In year one of King Darius the Mede, I arose to fortify and strengthen him.'\,''
\labelchapt{11}
\passage{International Conflicts to Come}

\v{2}```Now I'll tell you the truth: Pay attention! Three more kings will arise in Persia. Then a fourth will gain more than them all. As soon as he gains power by means of his wealth, he'll stir up everyone against the Grecian kingdoms.

\v{3}```A mighty king will come to power, and he'll rule with awesome energy, doing whatever he pleases. \v{4}However, after he has come to power, his kingdom will be broken and parceled out in all directions.\fnote{Lit. \fbib{out to the four winds of heaven}} It won't go to his succeeding descendants, nor will its power match how he ruled, because his sovereignty will be uprooted and given to successors besides them.

\v{5}```The southern king will become strong, along with one of his officials, who will become stronger than he and rule over his own realm with great power. \v{6}After a number of years, they'll become allies and the daughter of the southern king will go to the northern king in order to craft alliances. But she won't remain in power, nor will he retain his power. Instead, she'll be surrendered, along with her entourage, the one who fathered her, and the one who supported her at that time.

\v{7}```One of her family line will replace him. He'll come against the army and enter the fortress of the northern king, conquering them and becoming victorious. \v{8}He'll also take their gods, their molten images, and their valuable vessels of silver and gold into Egypt as hostages. He'll avoid the northern king for a number of years. \v{9}Then he'll come against the realm of the southern king and then return to his own territory. \v{10}His sons will prepare for war, assembling an army of considerable force. One of them will come on forcefully, overflowing, passing through, and waging war up to his own fortress.

\v{11}```The southern king will fly into a rage and march out to fight the northern king. He'll gather a large army, but that army will be handed over to him. \v{12}When that army has been defeated, he'll become overconfident and slaughter many thousands, but he won't succeed. \v{13}The northern king will return and raise a greater army than before. After a few years, he'll advance with a great force and with a vast amount of armaments.'\,''
\passage{Rebellion against the Southern King}

\v{14}```During those years, many will rebel against the southern king. The more violent ones among your people will rebel in order to fulfill this vision, but they will fail. \v{15}Then the northern king will come, erect a siege ramp, and capture a fortified city. The southern forces won't prevail---not even with their best troops---and they'll have no strength to take a stand.

\v{16}```However, the one who invades him will do whatever he wants to do. No one will oppose him. He'll establish himself in the Beautiful Land, wielding devastating power. \v{17}He'll decide to come with the full power of his kingdom, bringing with him an alliance that he'll implement. He'll give him a daughter in marriage to overthrow it, but it won't succeed or work out for him. \v{18}Then he'll turn his attention to the coastal lands\fnote{Or \fbib{islands}} and will capture many. But a commander will put an end to his insolence, repaying him for his scorn. \v{19}He'll turn his attention toward the fortresses in his own territory, but he'll stumble and fall, and won't endure. \v{20}His successor will send out a tax collector for royal splendor, but in a short period of time he'll be shattered, though neither in anger nor in battle.'\,''
\passage{The Despicable King}

\v{21}```In his place there will arise a despicable person, upon whom no royal authority has been conferred, but he'll invade in a time of tranquility, taking over the kingdom through deception. \v{22}Overwhelming forces will be carried away before him, along with the Commander-in-Chief\fnote{Lit. \fbib{Nagid}; i.e. a senior officer entrusted with dual roles of operational oversight and administrative authority} of the covenant. \v{23}From the time that an alliance is made with him, he'll act deceitfully, and he will go up and take power with only a small group of nations. \v{24}He'll invade the most prosperous areas of the province during a time of tranquility, accomplishing what neither his predecessors nor his ancestors ever could. He'll distribute war spoils, booty, and wealth to them, and he'll plot the overthrow of fortresses, though only for a time. \v{25}He'll encourage himself against the southern king by raising\fnote{The Heb. lacks \fbib{raising}} a large army. As a result, the southern king will mobilize for war with a large and powerful army, but he won't succeed because they will devise elaborate schemes against him. \v{26}His own security detail\fnote{Lit. \fbib{Those who eat his delicacies}} will undermine him, his army will be swept away, and many will fall and be killed in battle.\fnote{The Heb. lacks \fbib{in battle}} \v{27}Now as for the two kings, their intentions will be evil, and they'll promote deception at their dinner table, but none of this will succeed, because the end won't have come yet. \v{28}Then he'll return to his homeland with great wealth, will focus his attention against the holy covenant, and will take action as he returns to his land.'\,''
\passage{Desecration of the Sanctuary}

\v{29}```At the scheduled time he'll return, moving southward, but the end result won't be as before, \v{30}because ships will come against him from the Mediterranean islands.\fnote{Lit. \fbib{from Kittim}} Disheartened, he'll return, incited to vehemence against the holy covenant, and he'll take action. As he returns, he'll show deference to those who abandon the holy covenant. \v{31}Armed forces will arise from his midst, and they'll desecrate the fortified Sanctuary, abolish the daily sacrifice, and establish the destructive desecration. \v{32}Through flattery he'll corrupt those who act wickedly toward the covenant, but people who know their God will be strong and take action. \v{33}Insightful people\fnote{I.e. believers; cf. v. 35} will impart understanding to many, though they'll fall by sword, by fire, by captivity, and as war booty for a while.\fnote{Lit. \fbib{for days}} \v{34}When they fall, they'll be given some relief, but many will join them by pretending to be sympathetic to their cause. \v{35}Some of the insightful will fall so they may be refined, purged, and purified until the time of the end, since it will surely come about.'\,''
\passage{The King Who Calls Himself God}

\v{36}```The king will do as he pleases. He'll exalt and magnify himself above every god, speaking amazing things against the God of Gods. He'll succeed until the indignation is completed, because what has been determined must be carried out. \v{37}He'll recognize neither the gods of his ancestors nor those desired by women---he won't recognize any god, because he'll exalt himself above everything. \v{38}He'll glorify the god of fortresses,\fnote{Or \fbib{forces}} a god whom his ancestors never knew, honoring him with gold, silver, valuable jewels, and treasures. \v{39}He'll take action against the strongest fortresses. With the help of a foreign god, he'll recognize those who honor him, making them rule over many, and he'll parcel out the land for a profit.

\v{40}```At the time of the end, the southern king will oppose him, and the northern king will overrun him with chariots, cavalry, and many ships. He'll invade countries, moving swiftly and sweeping through. \v{41}He'll enter the Beautiful Land, and many will fall, even though these will escape his control: Edom, Moab, and certain Ammonite officials.

\v{42}He'll extend his power over other countries, and even the land of Egypt won't escape. \v{43}He'll capture treasures of gold, silver, and all the treasures of Egypt, with the Libyans and Cushites\fnote{Or \fbib{Nubians}} at his feet. \v{44}However, reports from the east and the north will alarm him, and he'll march out in great anger, intending to destroy and to desolate many. \v{45}When he pitches his royal pavilions between the seas\fnote{I.e. between the Mediterranean Sea and the Dead Sea} facing the mountain of holy Glory, he'll come to his end, and no one will help him.'\,''
\labelchapt{12}
\passage{The End Times}

\chapt{12}
\v{1}```At that time, Michael\fnote{Or \fbib{time, the One who is like God}; i.e. the Messiah} will arise, the great prince who will stand up on behalf of your people, and a time of trouble will come like there has never been since nations began until that time. Also at that time, your people will be delivered---everyone who will have been written in the book. \v{2}Many of those who are sleeping in the dust of the earth will awaken---some to life everlasting, and some to disgrace and everlasting contempt. \v{3}Those who manifest wisdom will shine like the brightness of the expanse of heaven, and those who turn many to righteousness will shine\fnote{The Heb. lacks \fbib{will shine}} like the stars for ever and ever. \v{4}Now as for you, Daniel, roll up your scroll and seal your words until the time of the end. Many will rush around, while knowledge increases.'\,''
\passage{The Vision of the Two Speakers}

\v{5}``Then while I, Daniel, continued watching, suddenly two others stood there, one on this side of the river bank and one on the other side. \v{6}One asked the man dressed in linen clothes, who was standing\fnote{The Heb. lacks \fbib{standing}} above the waters of the river, `How long until the fulfillment of the wonders?'

\v{7}``I heard the man dressed in linen clothes, who was standing\fnote{The Heb. lacks \fbib{standing}} above the waters of the river as he lifted his right and left hands to heaven and swore by the one who lives forever that it would be for a time, times, and a half. When the shattering of the power of the holy people has occurred, all these things will conclude.''
\passage{Daniel's Unanswered Question}

\v{8}``I heard, but I didn't understand. So I asked, `Sir,\fnote{Lit. \fbib{My lord}} what happens next?'

\v{9}``He answered, `Go on your way, Daniel, because these matters\fnote{Or \fbib{words}} are wrapped up and sealed until the time of the end. \v{10}Many will be purified, cleansed, and refined, though the wicked will continue to act wickedly, and none of the wicked will understand. Nevertheless, the insightful\fnote{Or \fbib{wise}} will understand. \v{11}There will be\fnote{The Heb. lacks \fbib{There will be}} 1,290 days from the time the regular burnt offering\fnote{Or \fbib{sacrifice}} is rescinded and the destructive desolation established. \v{12}Blessed is the one who perseveres and attains to the 1,335 days. \v{13}Now as for you, keep on going until the end---you'll rest and then rise to receive your reward at the end of the age.'\,''\fnote{Lit. \fbib{days}}

\addcontentsline{toc}{chapter}{Shorter Prophetic Writings Before the Exile}
\bookheader{Hosea}
\labelbook{Hos}

\bookpretitle{The Book of the Prophet}
\booktitle{Hosea}

\labelchapt{1}
\passage{The Word of the \divine{Lord} to Hosea}

\chapt{1}
\v{1}A message from the \divine{Lord} came\fnote{\fbackref{1:1} Lit. \fbib{that came}} to Beeri's son Hosea\fnote{\fbackref{1:1} The Heb. name \fbib{Hosea} means \fbib{salvation}} during the reigns of\fnote{\fbackref{1:1} Lit. \fbib{in the days of}} Uzziah, Jotham, Ahaz, and Hezekiah, kings of Judah, and during the reign of\fnote{\fbackref{1:1} Lit. \fbib{in the days of}} Joash's son Jeroboam, who was king of Israel.
\passage{Hosea's Wife and Family}

\v{2}When a message from the \divine{Lord} came to Hosea, the \divine{Lord} told him,\fnote{\fbackref{1:2} Lit. \fbib{Hosea}} ``Go marry a prostitute and have children with her,\fnote{\fbackref{1:2} Lit. \fbib{children of prostitution}} because the land is prostituting itself by departing from the \divine{Lord}.'' \v{3}So he went out and married Diblaim's daughter Gomer. She conceived with him and gave birth to a son.
\passage{Naming the Children}

\v{4}The \divine{Lord} told Hosea,\fnote{\fbackref{1:4} Lit. \fbib{him}} ``Name the child\fnote{\fbackref{1:4} Lit. \fbib{Name him}} `Jezreel,'\fnote{\fbackref{1:4} The Heb. name \fbib{Jezreel} means ``God sows''} because in a little while I'll avenge the blood that was shed by Jehu's dynasty at Jezreel. I'll put an end to the kingdom of the house of Israel. \v{5}At that time I'll shatter the military strength\fnote{\fbackref{1:5} Lit. \fbib{the bow}} of Israel in the Valley of Jezreel.''

\v{6}Gomer\fnote{\fbackref{1:6} Lit. \fbib{She}} conceived again and gave birth to a daughter, so the \divine{Lord}\fnote{\fbackref{1:6} Lit. \fbib{he}} told Hosea,\fnote{\fbackref{1:6} Lit. \fbib{him}} ``Name her `Lo-ruhamah,'\fnote{\fbackref{1:6} The Heb. name \fbib{Lo-ruhamah} means \fbib{No mercy}} because I will no longer be showing mercy to the house of Israel, nor will I forgive them. \v{7}But I'll have mercy on the house of Judah, and I'll save them by the \divine{Lord} their God---I will not save them by the bow, by the sword, by battle, by horses, or by cavalry.''

\v{8}After Gomer\fnote{\fbackref{1:8} Lit. \fbib{she}} had weaned Lo-ruhamah, she conceived again and gave birth to a son, \v{9}so the \divine{Lord}\fnote{\fbackref{1:9} Lit. \fbib{he}} told Hosea,\fnote{\fbackref{1:9} Lit. \fbib{him}} ``Name him `Lo-ammi,'\fnote{\fbackref{1:9} The Heb. name \fbib{Lo-ammi} means \fbib{Not my people}} because you are not my people, and I will not be your God.\fnote{\fbackref{1:9} Lit. \fbib{be for you}} \v{10}\fnote{\fbackref{1:10} This vs. is 2:1 in MT}Despite this, the number of the people of Israel will be like ocean sand, which can neither be measured nor counted. And the time will come when instead of it being said,\fnote{\fbackref{1:10} Lit. \fbib{told them}} `You are not my people,' it will be said,\fnote{\fbackref{1:10} Lit. \fbib{told them}} `You are children of the living God.' \v{11}\fnote{\fbackref{1:11} This vs. is 2:2 in MT}And the people of Judah and the people of Israel will be united as one. They will appoint for themselves a single leader and will take dominion over\fnote{\fbackref{1:11} Lit. \fbib{will go up from}} the land, for great will be the day of Jezreel.\chapt{2}
\v{1}\fnote{\fbackref{2:1} This vs. is 2:3 in MT, and so throughout the chapter.}So call your brothers `Ammi,'\fnote{\fbackref{2:1} Lit. \fbib{My People}} and your sisters `Ruhamah.'\,''\fnote{\fbackref{2:1} Lit. \fbib{Beloved}; i.e. in contrast to the names \fbib{Lo-Ruhamah} and \fbib{Lo-Ammi} in vv. 2:6-9}
\labelchapt{2}
\passage{Gomer is Rebuked}

\begin{poetry}
\poeml \v{2}``Call your mother to account, call her--- \\
\poemll    for she is not my wife, \\
\poemlll       and I'm not her husband. \\
\poeml Let her do away with her seductive looks \\
\poemll    and remove her adultery from between her breasts. \\
\poeml \v{3}Otherwise, I'll strip her naked--- \\
\poemll    as she was on the day she was born--- \\
\poeml make her like a wilderness, \\
\poemll    turn her into a parched land, \\
\poemlll       and cause her to die of thirst. \\
\poeml \v{4}Furthermore, I'll not show pity on her children, \\
\poemll    since they are children of prostitution. \\
\poeml \v{5}Indeed, their mother has committed prostitution--- \\
\poemll    the one who has been conceiving them has acted disgracefully--- \\
\poeml when she said, \\
\poemll    `I'm going after my lovers, \\
\poemlll       who provide me food and water, \\
\poemll    as well as my wool, my flax, \\
\poemlll       my oil, and my wine.' \\
\poeml \v{6}``Look how I'm blocking her\fnote{\fbackref{2:6} So with LXX. MT reads \fbib{your}} path with thorns \\
\poemll    and building a wall to hinder\fnote{\fbackref{2:6} Lit. \fbib{wall against}} her, \\
\poemlll       so she can't find her way. \\
\poeml \v{7}She will pursue her lovers, \\
\poemll    but she won't catch up with them. \\
\poeml She will seek them, \\
\poemll    but she won't find them. \\
\poeml Then she will say, \\
\poemll    `I'll go back and return to my first husband, \\
\poemlll       because it was better for me then than now.' \\
\poeml \v{8}She didn't recognize \\
\poemll    that it was I who provided her grain, wine, and oil, \\
\poeml and it was I who gave her silver, \\
\poemll    while they crafted gold for Baal. \\
\poeml \v{9}``Therefore I'll return \\
\poemll    and take back my grain at harvest time \\
\poemlll       and my new wine in its season. \\
\poeml I'll take back my wool and my flax \\
\poemll    that was to have covered her nakedness. \\
\poeml \v{10}So now I'll reveal her lewdness to the eyes of her lovers, \\
\poemll    and no man will rescue her from my control.\fnote{\fbackref{2:10} Lit. \fbib{hand}} \\
\poeml \v{11}I'll put a stop to her mirth, \\
\poemll    along with her celebrations, her New Moons, her Sabbaths,\fnote{\fbackref{2:11} So with LXX. MT reads \fbib{Sabbath}} and all of her festive assemblies. \\
\poeml \v{12}I'll destroy her vines and her fig trees, \\
\poemll    about which she said, \\
\poeml `These are the earnings that my lovers paid me. \\
\poemll    I'll make them grow into a forest, \\
\poemlll       and the wild animals will eat from them.' \\
\poeml \v{13}I'll punish her for the time she has devoted to the Baals,\fnote{\fbackref{2:13} Lit. \fbib{for the days of the Baalim}; i.e. Canaanite deities} \\
\poemll    to whom she burned incense, \\
\poeml and for whom she put on her earrings and jewels \\
\poemll    so she could go after her lovers and forget me,'' \\
\poemlll       declares the \divine{Lord}.
\passage{Alluring a Wayward Wife}
\poeml \v{14}``Therefore, look! I will now allure her. \\
\poemll    I will make her go out to the wilderness, \\
\poemlll       and will speak to her heart. \\
\poeml \v{15}There I will restore her vineyards to her, \\
\poemll    and the Valley of Achor will become a doorway to hope. \\
\poeml There she will respond as she did in her youth, \\
\poemll    when she came up from Egypt.''
\passage{The Restoration of Israel}
\poeml \v{16}``It will come about at that time,'' \\
\poemll    declares the \divine{Lord}, \\
\poeml ``that you will address me as `My husband,' \\
\poemll    and you will no longer call me `My master'.\fnote{\fbackref{2:16} Heb. \fbib{baali}, a word play alluding to the Canaanite deity of a similar name} \\
\poeml \v{17}I will remove the names of the Baals\fnote{\fbackref{2:17} I.e. Canaanite deities} from her vocabulary\fnote{\fbackref{2:17} Lit. \fbib{mouth}}--- \\
\poemll    they will not be remembered by their names anymore. \\
\poeml \v{18}I will make a covenant with them at that time, \\
\poemll    a covenant\fnote{\fbackref{2:18} The Heb. lacks \fbib{a covenant}} with the wild animals of the field, \\
\poeml with the birds of the air, \\
\poemll    and with the creatures of the ground. \\
\poeml I will banish\fnote{\fbackref{2:18} Lit. \fbib{break}} the battle bow, the sword, and war from the earth. \\
\poemll    I will cause my people\fnote{\fbackref{2:18} Lit. \fbib{cause them}} to lie down where it is safe. \\
\poeml \v{19}I will make you my wife forever--- \\
\poeml I will make you my wife in a way that is righteous, \\
\poemll    in a manner that is just, \\
\poeml by a love that is gracious, \\
\poemll    and by a motive that is mercy. \\
\poeml \v{20}I will make you my wife because of my\fnote{\fbackref{2:20} Lit. \fbib{wife in}} faithfulness, \\
\poemll    and you will know the \divine{Lord}. \\
\poeml \v{21}``It will come about at that time that I will respond,'' \\
\poemll    declares the \divine{Lord}, \\
\poeml ``I will respond to the heavens, \\
\poemll    and they will respond to the earth. \\
\poeml \v{22}The earth will respond with grain, new wine, and oil, \\
\poemll    and they will respond to Jezreel.\fnote{\fbackref{2:22} The Heb. name \fbib{Jezreel} means \fbib{God sows}} \\
\poeml \v{23}I will plant my people\fnote{\fbackref{2:23} Lit. \fbib{planted her}} in the land for myself. \\
\poemll    I will show mercy on her who has received no mercy\fnote{\fbackref{2:23} Heb. \fbib{on Lo-Ruhamah}} \\
\poeml I will say to those who are not my people,\fnote{\fbackref{2:23} Heb. \fbib{to Lo-Ammi}} `You are my people!' \\
\poemll    and they will say, `You are\fnote{\fbackref{2:23} The Heb. lacks \fbib{You are}} my God.'\,''
\end{poetry}
\labelchapt{3}
\passage{Hosea Reconciles with His Wife}

\chapt{3}
\v{1}Then the \divine{Lord} told me: ``Go love your wife\fnote{\fbackref{3:1} Or \fbib{love a woman}} again, even though she is being loved by another and is committing adultery. Love her the same way\fnote{\fbackref{3:1} Lit. \fbib{adultery, as}} the \divine{Lord} loves the people of Israel, even though they look to other gods and love raisin cakes.''\fnote{\fbackref{3:1} I.e. cakes used for offerings; cf. 2Sam 6:19} \v{2}So I bought her back for myself for fifteen pieces of silver and one and a half omers\fnote{\fbackref{3:2} I.e. about ten bushels} of barley.

\v{3}I told her, ``You will remain with me a long time,\fnote{\fbackref{3:3} Lit. \fbib{me many days}} you won't be promiscuous, you won't be involved with any man, and I'll do the same.''\fnote{\fbackref{3:3} Lit. \fbib{same with you}}

\v{4}Likewise, the people of Israel will dwell a long time\fnote{\fbackref{3:4} Lit. \fbib{will live many days}} without a king, without a prince, without sacrifice, without sacred\fnote{\fbackref{3:4} The Heb. lacks \fbib{sacred}} pillars, and with neither ephod nor teraphim.\fnote{\fbackref{3:4} I.e. images used for divination} \v{5}Afterward, the people of Israel will return and seek the \divine{Lord} their God and David their king. They will come in awe to the \divine{Lord} and to his goodness in the last days.
\labelchapt{4}
\passage{God Accuses Israel}

\begin{poetry}
\poeml \chapt{4}
\v{1}``Hear this message from the \divine{Lord}, people of Israel. \\
\poemll    Indeed, the \divine{Lord} brings a charge \\
\poemll    against the people who live in the land--- \\
\poeml for there is no truth and no gracious love \\
\poemll    or knowledge of God in the land. \\
\poeml \v{2}Swearing, lying, murder, theft, and adultery are rampant, \\
\poemll    and blood mingles with blood. \\
\poeml \v{3}Therefore the land will mourn, \\
\poemll    and all who live there will languish, \\
\poeml along with the wild animals of the field and the birds of the air. \\
\poemll    Even the fish in the sea will disappear. \\
\poeml \v{4}``Let no one fight or bring charges against another, \\
\poemll    for my dispute is with you, priest. \\
\poeml \v{5}So you will stumble during the day, \\
\poemll    the prophet also will stumble with you at night, \\
\poemlll       and I will destroy your mother.\fnote{\fbackref{4:5} I.e. \fbib{Israel}} \\
\poeml \v{6}My people are destroyed because they lack knowledge of me.\fnote{\fbackref{4:6} The Heb. lacks \fbib{of me}} \\
\poemll    Because you rejected that knowledge, \\
\poemlll       I will reject you as a priest for me. \\
\poeml Since you forget the Law of your God, \\
\poemll    I will also forget your children. \\
\poeml \v{7}``The more they increased in number,\fnote{\fbackref{4:7} The Heb. lacks \fbib{in number}} \\
\poemll    the more they sinned against me, \\
\poemlll       so I will change their glory into shame. \\
\poeml \v{8}They feed on the sin of my people; \\
\poemll    they purpose in their heart to transgress. \\
\poeml \v{9}So it will be: like people, like priest. \\
\poemll    I will punish them for their lifestyles, \\
\poemlll       rewarding them according to their behavior. \\
\poeml \v{10}They will eat, \\
\poemll    but will not be satisfied. \\
\poeml They will engage in prostitution, \\
\poemll    but they won't increase, \\
\poemlll       because they have stopped listening to the \divine{Lord}. \\
\poeml \v{11}``Sexual immorality, wine, and fresh wine seduce the heart of my people.\fnote{\fbackref{4:11} So LXX. The Heb. lacks \fbib{of my people}} \\
\poeml \v{12}My people seek counsel from their piece of wood, \\
\poemll    and their diviner's rod\fnote{\fbackref{4:12} Lit. \fbib{their carved wood}} speaks to them. \\
\poeml For a spirit of prostitution causes them to go astray; \\
\poemll    in their immorality they desert their God. \\
\poeml \v{13}They offer sacrifices on the mountain tops, \\
\poemll    burning offerings on the hills, \\
\poeml under oaks, poplars, and terebinth\fnote{\fbackref{4:13} Or \fbib{great}} trees, \\
\poemll    since their shade is very good. \\
\poeml Therefore your daughters are prostitutes \\
\poemll    and your daughters-in-law commit adultery. \\
\poeml \v{14}However, I'm not going to punish your daughters \\
\poemll    when they commit prostitution, \\
\poeml nor your daughters-in-law \\
\poemll    when they commit adultery, \\
\poeml because their men are themselves immoral--- \\
\poemll    they offer sacrifices with prostitutes. \\
\poeml These people who aren't discerning will stumble. \\
\poeml \v{15}``Even though you prostitute yourself, Israel--- \\
\poemll    let not Judah incur guilt--- \\
\poeml don't go to Gilgal, \\
\poemll    or visit Beth-aven, \\
\poemlll       or swear an oath using the \divine{Lord}'s name.\fnote{\fbackref{4:15} Lit. \fbib{oath, `{\ldots}as the \divine{Lord} lives'}} \\
\poeml \v{16}For Israel is as obstinate as a stubborn mule!\fnote{\fbackref{4:16} Or \fbib{cow}} \\
\poemll    Nevertheless, will not the \divine{Lord} feed them like a lamb in a broad pasture? \\
\poeml \v{17}Ephraim has become entwined with idols; \\
\poemll    leave him alone! \\
\poeml \v{18}While drinking to excess, they prostitute themselves. \\
\poemll    They're in love with dishonor. \\
\poeml \v{19}A wind storm will carry them away in its embrace, \\
\poemll    and their sacrifices will bring them shame.''
\end{poetry}
\labelchapt{5}
\passage{Judgment against Israel}

\begin{poetry}
\poeml \chapt{5}
\v{1}``Hear this, priests, \\
\poemll    pay attention, house of Israel, \\
\poemlll       listen, royal family! \\
\poeml For judgment is coming your way,\fnote{\fbackref{5:1} Lit. \fbib{is toward you}} \\
\poemll    because you have been a trap to Mizpah, \\
\poemlll       a snare spread out on Mount\fnote{\fbackref{5:1} The Heb. lacks \fbib{Mount}} Tabor. \\
\poeml \v{2}The rebels are deep into their slaughter; \\
\poemll    I am punishing them all. \\
\poeml \v{3}I know Ephraim, \\
\poemll    and Israel cannot hide from me, \\
\poeml since you, Ephraim, have been acting like a prostitute, \\
\poemll    defiling Israel. \\
\poeml \v{4}``Their actions hinder them from turning to their God, \\
\poemll    because a spirit of fornication is in their midst, \\
\poemlll       and the \divine{Lord} they do not know. \\
\poeml \v{5}The arrogance of Israel testifies against him; \\
\poemll    therefore Israel and Ephraim will stumble in their iniquity, \\
\poemlll       and Judah with them. \\
\poeml \v{6}They will go with their flocks and herds \\
\poemll    to seek the \divine{Lord}, \\
\poeml but they will not find him; \\
\poemll    he has withdrawn from them. \\
\poeml \v{7}They have been unfaithful to the \divine{Lord}, \\
\poemll    having raised unbelieving children. \\
\poeml In the coming month they will be devoured, \\
\poemll    along with their fields.\fnote{\fbackref{5:7} Or \fbib{inheritance}} \\
\poeml \v{8}``Sound the trumpet in Gibeah, \\
\poemll    and the alarm in Ramah. \\
\poeml Cry out at Beth-aven \\
\poemll    Go out,\fnote{\fbackref{5:8} So LXX.; MT \fbib{After you}} Benjamin! \\
\poeml \v{9}Ephraim will be desolate \\
\poemll    when it is rebuked. \\
\poeml I have made known among the tribes of Israel \\
\poemll    what will surely come about. \\
\poeml \v{10}The princes of Judah have become \\
\poemll    like those who move boundary markers: \\
\poemlll       I will pour out my anger on them like water. \\
\poeml \v{11}Ephraim is crushed, \\
\poemll    broken by judgment, \\
\poemlll       because he\fnote{\fbackref{5:11} I.e. Ephraim as an individual personifying the northern kingdom of Israel.} willingly pursued idols.\fnote{\fbackref{5:11} So LXX. MT \fbib{willingly went away from the command}} \\
\poeml \v{12}Therefore I will consume\fnote{\fbackref{5:12} The Heb. lacks \fbib{will consume}} Ephraim like a moth, \\
\poemll    and the house of Judah as rottenness consumes. \\
\poeml \v{13}When Ephraim examined his illness \\
\poemll    and Judah his injury, \\
\poeml then Ephraim went to Assyria, \\
\poemll    and inquired of the great king; \\
\poeml but he could not cure you \\
\poemll    nor heal your injury. \\
\poeml \v{14}Therefore I will be like a lion to Ephraim, \\
\poemll    and like a young lion to the house of Judah. \\
\poeml I---even I---will tear them\fnote{\fbackref{5:14} The Heb. lacks \fbib{them}} to pieces, \\
\poemll    and then I will leave. \\
\poeml I will take them\fnote{\fbackref{5:14} The Heb. lacks \fbib{them}} away, \\
\poemll    and there will be no rescue. \\
\poeml \v{15}``I will leave and go back to my place \\
\poemll    until they admit their offense \\
\poemlll       and seek my face. \\
\poeml When affliction comes to them, \\
\poemll    they will eagerly seek me.''
\end{poetry}
\labelchapt{6}
\passage{A Call for Israel to Repent}

\begin{poetry}
\poeml \chapt{6}
\v{1}``Come, let us return to the \divine{Lord}; \\
\poemll    even though he has torn us,\fnote{\fbackref{6:1} The Heb. lacks \fbib{us}} \\
\poemlll       he will heal us. \\
\poeml Even though he has wounded us, \\
\poemll    he will bind our wounds.\fnote{\fbackref{6:1} The Heb. lacks \fbib{our wounds}} \\
\poeml \v{2}After two days, he will restore us to life, \\
\poemll    on the third day he will raise us up, \\
\poemlll       and we will live in his presence. \\
\poeml \v{3}Let us know, \\
\poemll    let us pursue knowledge of the \divine{Lord}; \\
\poemlll       his coming is as certain as the dawn. \\
\poeml He will come to us like the rain, \\
\poemll    like the autumn and spring rains come on the earth. \\
\poeml \v{4}``What am I to do with you, Ephraim? \\
\poemll    What am I to do with you, Judah? \\
\poeml Your love is like a morning rain cloud--- \\
\poemll    it passes away like the morning dew. \\
\poeml \v{5}Therefore I cut them\fnote{\fbackref{6:5} The Heb. lacks \fbib{them}} to pieces by the prophets, \\
\poemll    killing them by the words from my mouth. \\
\poemlll       The verdict against you shines like a beacon. \\
\poeml \v{6}For it is love that I seek, \\
\poemll    and not sacrifice; \\
\poemlll       knowledge of God more than burnt offerings. \\
\poeml \v{7}``But like Adam,\fnote{\fbackref{6:7} Or \fbib{men}} they broke the covenant; \\
\poemll    in this they have acted deceitfully against me. \\
\poeml \v{8}Gilead is a lawless town; \\
\poemll    it is polluted by bloodshed. \\
\poeml \v{9}Like a gang of thieves that stalk a man, \\
\poemll    priests commit murder along the road to Shechem, \\
\poemlll       committing shameful crimes. \\
\poeml \v{10}I have seen a horrible evil in the house of Israel--- \\
\poemll    Ephraim's promiscuity. \\
\poemlll       Israel is defiled. \\
\poeml \v{11}``So, Judah, a harvest has been appointed for you \\
\poemll    when I restore my people from captivity.''
\end{poetry}
\labelchapt{7}
\passage{God Accuses Israel}

\begin{poetry}
\poeml \chapt{7}
\v{1}``When I was healing Israel, \\
\poeml Ephraim's sin was uncovered, \\
\poemlll       along with Samaria's wickedness. \\
\poeml While they craft lying schemes, \\
\poemll    the thief invades, \\
\poemlll       and the gang of thieves plunders outside. \\
\poeml \v{2}It never occurs to them that I remember all their sin. \\
\poemll    Now their actions have caught up with them, \\
\poemlll       and they have my attention.\fnote{\fbackref{7:2} Lit. \fbib{they are before my face}} \\
\poeml \v{3}They please the king with their evil, \\
\poemll    and the princes with their dishonesty. \\
\poeml \v{4}All of them are adulterers--- \\
\poemll    they burn like an oven prepared by the baker, \\
\poeml who has ceased stoking it \\
\poemll    until the dough is leavened. \\
\poeml \v{5}``On the king's festival day \\
\poemll    the princes got drunk from wine, \\
\poemlll       so the king\fnote{\fbackref{7:5} Lit. \fbib{so he}} joined the mockers. \\
\poeml \v{6}For they have stirred up themselves\fnote{\fbackref{7:6} Lit. \fbib{up their heart}} like an oven \\
\poemll    as they lie in ambush. \\
\poeml Their baker sleeps through the night; \\
\poemll    in the morning, the oven\fnote{\fbackref{7:6} Lit. \fbib{morning, it}} will be blazing like a fire. \\
\poeml \v{7}They all burn like an oven; \\
\poemll    they have consumed their judges; \\
\poeml all their kings have fallen--- \\
\poemll    not even one of them calls on me. \\
\poeml \v{8}``Ephraim compromises with\fnote{\fbackref{7:8} Or \fbib{dilutes himself among}} the nations; \\
\poemll    he's a half-baked cake.\fnote{\fbackref{7:8} Lit. \fbib{a cake not turned}} \\
\poeml \v{9}Foreigners have consumed his strength, \\
\poemll    and he hasn't noticed. \\
\poeml Furthermore, his head is sprinkled with gray hair, \\
\poemll    but he doesn't realize it. \\
\poeml \v{10}Israel's arrogance testifies against him;\fnote{\fbackref{7:10} Lit. \fbib{testifies in his face}} \\
\poemll    but they do not return to the \divine{Lord} their God, \\
\poemlll       nor seek him in all of this. \\
\poeml \v{11}``Ephraim is also like a silly dove, \\
\poemll    lacking sense:\fnote{\fbackref{7:11} Lit. \fbib{heart}} \\
\poeml They call out to Egypt, \\
\poemll    and turn toward Assyria. \\
\poeml \v{12}When they go, \\
\poemll    I'll cast my net over them. \\
\poeml I'll bring them down, as one shoots\fnote{\fbackref{7:12} The Heb. lacks \fbib{one shoots}} birds in the sky. \\
\poemll    I'll chasten them, \\
\poemlll       as the assembly has already heard. \\
\poeml \v{13}Woe to them--- \\
\poemll    because they have run away from me. \\
\poeml Ruin to them--- \\
\poemll    because they have sinned against me. \\
\poeml Even though I redeemed them, \\
\poemll    they spread lies against me. \\
\poeml \v{14}They will not cry to me from their heart--- \\
\poemll    instead, they wail on their beds. \\
\poeml They gather together to eat and drink,\fnote{\fbackref{7:14} Lit. \fbib{together for grain and fresh wine}} \\
\poemll    turning away from me. \\
\poeml \v{15}``Though I have taught them \\
\poemll    and strengthened their arms, \\
\poemlll       nevertheless they plot evil against me. \\
\poeml \v{16}They return---but not to the Most High. \\
\poemll    They are like a defective weapon.\fnote{\fbackref{7:16} Lit. \fbib{bow}} \\
\poeml Their princes will fall by the sword \\
\poemll    because of their raging tongue, \\
\poemlll       and they will be a laughingstock in the land of Egypt.''
\end{poetry}
\labelchapt{8}
\passage{Reaping the Wind Storm}

\begin{poetry}
\poeml \chapt{8}
\v{1}``Sound the ram's horn! \\
\poeml Like a vulture\fnote{\fbackref{8:1} Or \fbib{eagle}} the enemy\fnote{\fbackref{8:1} Lit. \fbib{he}} will come against the Temple of the \divine{Lord}, \\
\poeml because Israel\fnote{\fbackref{8:1} Lit. \fbib{they}} violated my covenant, \\
\poemll    transgressing my Law. \\
\poeml \v{2}They cry out to me, \\
\poemll    `God, we of Israel acknowledge you.' \\
\poeml \v{3}``Israel has discarded what is good. \\
\poemll    The enemy will pursue them.\fnote{\fbackref{8:3} Lit. \fbib{him}} \\
\poeml \v{4}They set kings in place, \\
\poemll    but not by me. \\
\poeml They established princes, \\
\poemll    whom I did not recognize. \\
\poeml They crafted idols for themselves from their silver and gold; \\
\poemll    as a result, they will be destroyed. \\
\poeml \v{5}Your calf,\fnote{\fbackref{8:5} I.e. the idol they crafted (cf. v. 4)} Samaria, has been thrown away. \\
\poemll    My anger is burning against them. \\
\poemlll       How long until they become pure again? \\
\poeml \v{6}Because from Israel it was fashioned by craftsmen, \\
\poemll    it is not God; \\
\poemlll       therefore Samaria's calf will be broken in pieces. \\
\poeml \v{7}``Because they sow the wind, \\
\poemll    they will reap the wind storm. \\
\poeml The plant has no stalk \\
\poemll    and its bud yields no grain. \\
\poeml Even if there's a harvest, \\
\poemll    foreigners will gobble it up. \\
\poeml \v{8}Israel has been devoured; \\
\poemll    now they will live among the nations \\
\poemlll       like a worthless container. \\
\poeml \v{9}``Because they went over to Assyria, \\
\poemll    they are like a wild donkey alone by itself. \\
\poemlll       Ephraim has hired some lovers. \\
\poeml \v{10}Even though they sold themselves to the nations, \\
\poemll    I will gather them. \\
\poeml They will mourn for a while \\
\poemll    for the burden they were to the king and princes.\fnote{\fbackref{8:10} So LXX. MT \fbib{king of princes}} \\
\poeml \v{11}``The more altars Ephraim builds for sin, \\
\poemll    the more altars there will be for sin. \\
\poeml \v{12}I prescribed great things from my Law for them,\fnote{\fbackref{8:12} Lit. \fbib{him}} \\
\poemll    but they considered them profane. \\
\poeml \v{13}They offer me meat from the sacrifices of my offerings, \\
\poemll    and they eat from it, \\
\poemlll       but the \divine{Lord} does not accept them. \\
\poeml He will now remember their transgression \\
\poemll    and pay them back for their sins; \\
\poemlll       to Egypt they will return. \\
\poeml \v{14}``Israel has neglected its maker in building palaces. \\
\poemll    Judah has multiplied its fortified cities, \\
\poeml but I will send fire to their cities, \\
\poemll    and it will consume their fortresses.''
\end{poetry}
\labelchapt{9}
\passage{Punishment for Israel}

\begin{poetry}
\poeml \chapt{9}
\v{1}``Don't celebrate, Israel, like other nations would rejoice, \\
\poeml because you left your God by committing fornication, \\
\poemlll       loving the profit you gained on all of the threshing floors. \\
\poeml \v{2}Neither threshing floor nor winepress will sustain them, \\
\poemll    and the new wine will disappoint her. \\
\poeml \v{3}They will not live in the \divine{Lord}'s land--- \\
\poemll    Ephraim will return to Egypt, \\
\poemlll       and they will eat unclean food in Assyria. \\
\poeml \v{4}They won't present wine offerings to the \divine{Lord}, \\
\poemll    nor will they please him. \\
\poeml Their sacrifices will seem like food for mourners--- \\
\poemll    everyone who eats them will become unclean; \\
\poemlll       none of them will enter the Temple of the \divine{Lord}. \\
\poeml \v{5}``What will you do on the designated holiday, \\
\poemll    when the \divine{Lord}'s festival comes? \\
\poeml \v{6}Look! They have gone away because of the destruction--- \\
\poemll    Egypt will gather them up, \\
\poemlll       and Memphis\fnote{\fbackref{9:6} So LXX. Heb. \fbib{Moph}; i.e. the capital of Lower Egypt} will bury them. \\
\poeml Weeds will overgrow their inheritance,\fnote{\fbackref{9:6} Lit. \fbib{their precious silver}} \\
\poemll    and thorns will grow\fnote{\fbackref{9:6} Lit. \fbib{be}} in their tents. \\
\poeml \v{7}The time for your judgment has now come; \\
\poemll    payday is here--- \\
\poemlll       and Israel knows it. \\
\poeml The prophet is a fool, \\
\poemll    and the spiritual man is insane. \\
\poeml Because of your great sin, \\
\poemll    the hatred against you\fnote{\fbackref{9:7} The Heb. lacks \fbib{against you}} is great. \\
\poeml \v{8}While Ephraim stands watch with my God, \\
\poemll    the prophet has snares set that will trap his ways, \\
\poemlll       and hostility lodges in the Temple of his God. \\
\poeml \v{9}They have corrupted themselves deeply, \\
\poemll    as did Gibeah\fnote{\fbackref{9:9} Cf. Judg 19:1} in its day. \\
\poeml Therefore God\fnote{\fbackref{9:9} Lit. \fbib{he}} will remember their lawlessness, \\
\poemll    and he will pay them back for their sins. \\
\poeml \v{10}``I found Israel, \\
\poemll    as one finds\fnote{\fbackref{9:10} The Heb. lacks \fbib{finds}} grapes in the wilderness; \\
\poeml Your ancestors seemed to me like the fruit \\
\poemll    gleaned from a fig tree's first harvest. \\
\poeml When they went to Baal-peor,\fnote{\fbackref{9:10} Cf. Num 25:1-3} \\
\poemll    they devoted themselves to that filth, \\
\poeml and they became loathsome, \\
\poemll    like what they loved. \\
\poeml \v{11}The glory of Ephraim will fly away like a bird--- \\
\poemll    no birth, no pregnancy, not even a conception. \\
\poeml \v{12}Even if they rear their children, \\
\poemll    I will, in turn, make them childless--- \\
\poeml in fact, woe to them \\
\poemll    when I turn away from them! \\
\poeml \v{13}Ephraim, as I see it, is like Tyre, \\
\poemll    planted in a comfortable place; \\
\poeml Ephraim will bear children \\
\poemll    but they will be executed.'' \\
\poeml \v{14}Give them, \divine{Lord}--- \\
\poemll    What will you give? \\
\poemlll       You will give them a womb that miscarries and dry breasts. \\
\poeml \v{15}``All of their wickedness started\fnote{\fbackref{9:15} The Heb. lacks \fbib{started}} in Gilgal, \\
\poemll    because I began to hate them there. \\
\poeml Because of the wickedness of their behavior, \\
\poemll    I will drive them from my Temple. \\
\poeml I will not love them anymore; \\
\poemll    all their leaders are rebels. \\
\poeml \v{16}Ephraim is blighted;\fnote{\fbackref{9:16} Or \fbib{stricken}} \\
\poemll    its roots shriveled. \\
\poemlll       It can bear no fruit. \\
\poeml Even if they bear children, \\
\poemll    I will kill their cherished offspring. \\
\poeml \v{17}``My God will reject them, \\
\poemll    because they did not obey him, \\
\poemlll       and they will become wanderers among the nations.''
\end{poetry}
\labelchapt{10}
\passage{The Coming Destruction}

\begin{poetry}
\poeml \chapt{10}
\v{1}``Israel, the overgrown\fnote{\fbackref{10:1} Or \fbib{empty}} vine, bears fruit like itself; \\
\poeml the more fruitful they become, \\
\poemlll       the more altars they build. \\
\poeml The better the land, \\
\poemll    the more ornate the stone idols.\fnote{\fbackref{10:1} Or \fbib{pillars}} \\
\poeml \v{2}Their hearts are divided; \\
\poemll    from now on they are to be found guilty. \\
\poeml God\fnote{\fbackref{10:2} Lit. \fbib{He}} will tear down their altars, \\
\poemll    he will destroy their stone idols.\fnote{\fbackref{10:2} Or \fbib{pillars}} \\
\poeml \v{3}From now on they will say, \\
\poemll    `We have no king, \\
\poemlll       because we did not fear the \divine{Lord}--- \\
\poemll    and what would a king do for us?' \\
\poeml \v{4}Their word is falsely given \\
\poemll    as they make their agreements;\fnote{\fbackref{10:4} Lit. \fbib{make a covenant}} \\
\poeml so judgment springs up \\
\poemll    like poisonous weeds in the furrows of a field.\fnote{\fbackref{10:4} Cf. Deut 29:18} \\
\poeml \v{5}``The residents of Samaria will be terrified \\
\poemll    because of the cows\fnote{\fbackref{10:5} I.e. Canaanite heifer deities} of Beth-aven. \\
\poeml Its people will mourn over Beth-aven,\fnote{\fbackref{10:5} Lit. \fbib{it}} \\
\poemll    along with the priests who will mourn its glory, \\
\poemlll       because that glory has departed.\fnote{\fbackref{10:5} The Heb. verb \fbib{depart} is similar to the Heb. verb \fbib{mourn}} \\
\poeml \v{6}Indeed, that glory\fnote{\fbackref{10:6} Lit. \fbib{Indeed, it}} will be carried to Assyria--- \\
\poemll    it will become a present for an avenging king.\fnote{\fbackref{10:6} Cf. 2Kings 15-16; 2Chr 28:19-20} \\
\poeml Ephraim will be disgraced, \\
\poemll    and Israel will become ashamed of its decision. \\
\poeml \v{7}Samaria's king will float away \\
\poemll    like driftwood on the surface of water. \\
\poeml \v{8}Destroyed will be the high places of Aven, \\
\poemll    that are the sin of Israel. \\
\poemlll       Both thorn and thistle will grow up over their altars. \\
\poeml They will call out to the mountains, `Cover us!' \\
\poemll    and to the hills, `Fall on us!' \\
\poeml \v{9}``From the time of Gibeah, \\
\poemll    you have sinned, Israel; \\
\poeml There they took their stand; \\
\poemll    the battle at Gibeah could not subdue the lawless. \\
\poeml \v{10}When I'm ready, I will chasten them; \\
\poemll    and the people will gather against them, \\
\poemlll       to imprison them for their two unrighteous acts.''\fnote{\fbackref{10:10} So LXX. MT reads \fbib{two eyes}}
\passage{Israel Urged to Sow in Righteousness}
\poeml \v{11}``Ephraim, the well-trained heifer, \\
\poemll    loves to thresh grain, \\
\poeml so I will spare her neck. \\
\poemll    I will turn Ephraim into a pack animal. \\
\poeml Judah will pull the plow, \\
\poemll    and Jacob will turn up the fallow ground. \\
\poeml \v{12}Sow in righteousness in your own interest, \\
\poemll    reap in gracious love, \\
\poemlll       break up your own unprepared ground; \\
\poeml It is now time to inquire of\fnote{\fbackref{10:12} Or \fbib{to worship}} the \divine{Lord}, \\
\poemll    until he comes to pour out righteousness for you. \\
\poeml \v{13}You have plowed\fnote{\fbackref{10:13} Or \fbib{fashioned}} evil; \\
\poemll    you have reaped unrighteousness; \\
\poemlll       you have eaten the fruit of hypocrisy; \\
\poeml because you trusted in your own direction, \\
\poemll    and in the number of your mighty forces. \\
\poeml \v{14}Therefore a disaster will come upon your people, \\
\poemll    and all of your fortresses will be ruined. \\
\poeml As Shalman\fnote{\fbackref{10:14} Possibly Shalmaneser, king of Assyria; cf. 2Kings 17:3} destroyed Beth-arbel in wartime, \\
\poemll    mothers were\fnote{\fbackref{10:14} Lit. \fbib{a mother was}} dashed to pieces \\
\poemlll       along with their children. \\
\poeml \v{15}The same will happen to you, Bethel, \\
\poemll    because of your great evil--- \\
\poemlll       early one morning the king of Israel will be totally silenced.''
\end{poetry}
\labelchapt{11}
\passage{God Loves Israel}

\begin{poetry}
\poeml \chapt{11}
\v{1}``When Israel was a young child I loved him, \\
\poemll    and from Egypt I called my son. \\
\poeml \v{2}The more I called out to them, \\
\poemll    the farther they fled from me;\fnote{\fbackref{11:2} So LXX. MT reads \fbib{They called to them, so they went away from them.}} \\
\poeml they sacrificed to Baals,\fnote{\fbackref{11:2} I.e. Canaanite deities} \\
\poemll    burning incense to carved images. \\
\poeml \v{3}Yet it was I who taught Ephraim to walk, \\
\poemll    supporting them by their arms, \\
\poemlll       but they never knew that I was healing them. \\
\poeml \v{4}I guided them with human kindness, \\
\poemll    with loving reins. \\
\poeml I acted toward them \\
\poemll    like one who removes a yoke from their neck; \\
\poemlll       I bent down and fed them. \\
\poeml \v{5}``They will not return to the land of Egypt; \\
\poemll    instead, the Assyrian will be their king, \\
\poemlll       because they kept refusing to repent. \\
\poeml \v{6}The sword will fall on their cities, \\
\poemll    consuming and devouring their fortified gates,\fnote{\fbackref{11:6} Lit. \fbib{devouring the bars}} \\
\poemlll       despite their planning. \\
\poeml \v{7}My people are determined to turn away from me; \\
\poemll    though they call to the Most High, \\
\poemlll       no one is worshiping. \\
\poeml \v{8}``How can I give up on you, Ephraim? \\
\poemll    I will deliver you, will I not, Israel? \\
\poeml How can I treat you like Admah? \\
\poemll    I can't make you like Zeboim,\fnote{\fbackref{11:8} cf. Deut 29:23} can I? \\
\poeml My heart stirs within me; \\
\poemll    my compassion also fans into flame! \\
\poeml \v{9}I will not act in my anger; \\
\poemll    I will not return to destroy Ephraim, \\
\poeml For I am God, \\
\poemll    and not a human--- \\
\poeml the Holy One among you--- \\
\poemll    so I will not enter the city in anger.\fnote{\fbackref{11:9} The Heb. lacks \fbib{in anger}} \\
\poeml \v{10}They will go after the \divine{Lord}, \\
\poemll    who will roar like a lion; \\
\poeml and when he roars, \\
\poemll    the children will come trembling from the west. \\
\poeml \v{11}Trembling like a bird, they will come out of Egypt, \\
\poemll    and as a dove from the land of Assyria; \\
\poeml and I will settle them in their houses,'' \\
\poemll    declares the \divine{Lord}. \\
\poeml \v{12}\fnote{\fbackref{11:12} This v. is 12:1 in MT}``Ephraim surrounds me with lies, \\
\poemll    and the house of Israel surrounds me\fnote{\fbackref{11:12} The Heb. lacks \fbib{surrounds me}} with deceit, \\
\poeml But Judah still rules with God, \\
\poemll    and remains faithful, along with the godly ones.''
\end{poetry}
\labelchapt{12}
\passage{Israel's Sin}

\begin{poetry}
\poeml \chapt{12}
\v{1}\fnote{\fbackref{12:1} This v. is 12:2 in MT, and so throughout the chapter.}``Ephraim feeds on the wind, \\
\poemll    chasing after the eastern winds, \\
\poemlll       storing up lies and desolation day after day. \\
\poeml They are making a contract with the Assyrians, \\
\poemll    and sending oil to Egypt. \\
\poeml \v{2}The \divine{Lord} accuses Judah, \\
\poemll    and will punish Jacob according to his ways; \\
\poemlll       he will repay him for what he does. \\
\poeml \v{3}He circumvented his brother\fnote{\fbackref{12:3} Lit. \fbib{He grabbed his brother by the heal}} in the womb, \\
\poemll    and as an adult he fought with God. \\
\poeml \v{4}He even fought the angel and won; \\
\poemll    he cried and prayed to him. \\
\poeml Then at Bethel he found him, \\
\poemll    and there he spoke with us--- \\
\poeml \v{5}the \divine{Lord} God of the Heavenly Armies--- \\
\poemll    the \divine{Lord} is his name.\fnote{\fbackref{12:5} Or \fbib{his traditional name}} \\
\poeml \v{6}So you, return to your God; \\
\poemll    guard grace and justice, \\
\poemlll       and look to your God always. \\
\poeml \v{7}``Now as for the merchant,\fnote{\fbackref{12:7} MT word for \fbib{merchant} sounds like \fbib{Canaan}} \\
\poemll    deceitful balances remain in his hand, \\
\poemlll       and he loves to defraud. \\
\poeml \v{8}Ephraim claims, \\
\poemll    `I have become rich, \\
\poeml I have made a fortune! \\
\poemll    Because of all my wealth, \\
\poemlll       no one will find any iniquity or sin in me.' \\
\poeml \v{9}``Yet I remain the \divine{Lord} your God, \\
\poemll    who brought you out of the land of Egypt. \\
\poeml I will make you live in tents again, \\
\poemll    as in the festival of that name.\fnote{\fbackref{12:9} I.e. the Festival of Tents} \\
\poeml \v{10}I spoke to the prophets, \\
\poemll    giving revelation after revelation, \\
\poemlll       and employing parables in the prophetic writings.\fnote{\fbackref{12:10} Lit. \fbib{parables by the hand of the prophets}} \\
\poeml \v{11}``There's iniquity in Gilead, isn't there? \\
\poemll    They have become truly vain. \\
\poeml They sacrifice bulls in Gilgal; \\
\poemll    their altars are like piles of stone in furrowed fields. \\
\poeml \v{12}Jacob fled into the land of Aram;\fnote{\fbackref{12:12} I.e. Syria} \\
\poemll    Israel served there to obtain his wife, \\
\poemlll       tending sheep to gain\fnote{\fbackref{12:12} The Heb. lacks \fbib{to gain}} his wife. \\
\poeml \v{13}``By a prophet the \divine{Lord} brought Israel out of Egypt, \\
\poemll    and by a prophet he\fnote{\fbackref{12:13} I.e. Israel} was rescued. \\
\poeml \v{14}Ephraim has stirred up violent anger; \\
\poemll    therefore the guilt of his blood will remain on him, \\
\poemlll       and his Lord will repay him for his contempt.''
\end{poetry}
\labelchapt{13}
\passage{The \divine{Lord}'s Anger against Israel}

\begin{poetry}
\poeml \chapt{13}
\v{1}``When the tribe of\fnote{\fbackref{13:1} The Heb. lacks \fbib{the tribe of}} Ephraim spoke, there was trembling; \\
\poemll    and it was exalted within Israel. \\
\poeml But when they offended God by Baal, \\
\poemll    they died, \\
\poeml \v{2}but now they are sinning more and more, \\
\poemll    crafting idols from melted silver. \\
\poeml Their idols are made with the most exacting skill, \\
\poemll    all of it the work of craftsmen. \\
\poeml People\fnote{\fbackref{13:2} Lit. \fbib{They}} say about them, \\
\poemll    `They offer human sacrifice, \\
\poemlll       and kiss calf-shaped idols.'\fnote{\fbackref{13:2} Lit. \fbib{kissing calves}} \\
\poeml \v{3}Therefore they will be like morning clouds, \\
\poemll    like early morning dew that evaporates, \\
\poeml like chaff blown away from the threshing floor, \\
\poemll    or like smoke from a chimney.''\fnote{\fbackref{13:3} Or \fbib{window}}
\passage{The \divine{Lord} is Israel's God}
\poeml \v{4}``I am the \divine{Lord} your God \\
\poemll    from the land of Egypt, \\
\poeml and you have known no god except for me, \\
\poemll    because except for me there is no savior. \\
\poeml \v{5}I took care of\fnote{\fbackref{13:5} So LXX. MT reads \fbib{I knew}} you in the wilderness, \\
\poemll    in a waterless land. \\
\poeml \v{6}As their pastures flourished, \\
\poemll    all their desires were met.\fnote{\fbackref{13:6} Lit. \fbib{flourished, they were satiated}} \\
\poeml As they were satiated, \\
\poemll    they became arrogant \\
\poemlll       and therefore ignored me. \\
\poeml \v{7}``So I will be like a lion to them. \\
\poemll    Like a leopard I will stalk them along the road. \\
\poeml \v{8}I will confront them like a bear deprived of her cubs;\fnote{\fbackref{13:8} The Heb. lacks \fbib{of her cubs}} \\
\poemll    I will tear open their ribs. \\
\poeml I will devour them like a lion--- \\
\poemll    the wild beasts will rip them apart. \\
\poeml \v{9}``You have destroyed yourself, Israel, \\
\poemll    although I remain your help. \\
\poeml \v{10}Now where is your king? \\
\poemll    Will he save you in all your cities? \\
\poeml And where are\fnote{\fbackref{13:10} The Heb. lacks \fbib{where are}} your judges, \\
\poemll    about whom you demanded, \\
\poemlll       `Give me a king and officials!'? \\
\poeml \v{11}I gave you a king in my anger, \\
\poemll    and I took him\fnote{\fbackref{13:11} The Heb. lacks \fbib{him}} away in my fury.'' \\
\poeml \v{12}``Ephraim's guilt is on record; \\
\poemll    his sin is stored away. \\
\poeml \v{13}When the time of childbirth comes, \\
\poemll    he will be so foolish \\
\poemlll       that he will refuse to be born.'' \\
\poeml \v{14}``From the power of Sheol I will rescue them, \\
\poemll    from death I will redeem them. \\
\poeml Death, where are\fnote{\fbackref{13:14} Or \fbib{death, I will be}} your plagues? \\
\poemll    Sheol, where is\fnote{\fbackref{13:14} Or \fbib{Sheol, I will be}} your destruction? \\
\poemlll       My eyes will remain closed to your pleas for\fnote{\fbackref{13:14} The Heb. lacks \fbib{your pleas for}} compassion. \\
\poeml \v{15}Even though he is fruitful compared to his relatives, \\
\poemll    an east wind will come, \\
\poeml the \divine{Lord}'s wind storm from the wilderness, \\
\poemll    and his spring will evaporate. \\
\poeml His fountain will dry up, \\
\poemll    and the \divine{Lord}'s\fnote{\fbackref{13:15} Lit. \fbib{and it}} wind storm will plunder \\
\poemlll       all the expensive vessels of the treasury. \\
\poeml \v{16}\fnote{\fbackref{13:16} This v. is 14:1 in MT}Samaria will be held guilty, \\
\poemll    because she has rebelled against her God. \\
\poeml By the sword they will fall--- \\
\poemll    with their infants dashed to pieces, \\
\poemlll       and their pregnant women torn open.''
\end{poetry}
\labelchapt{14}
\passage{A Call to Repentance}

\begin{poetry}
\poeml \chapt{14}
\v{1}\fnote{\fbackref{14:1} This v. is 14:2 in MT, and so throughout the chapter.}``Return, Israel, to the \divine{Lord} your God, \\
\poemll    for you have fallen due to your own iniquity. \\
\poeml \v{2}Bring a prepared speech with you \\
\poemll    as you return to the \divine{Lord}. Say to him: \\
\poeml `Take away all our\fnote{\fbackref{14:2} The Heb. lacks \fbib{our}} iniquity, \\
\poemll    and accept what is good. \\
\poeml Then we will present the fruit\fnote{\fbackref{14:2} So LXX and DSS. MT reads \fbib{bulls}} of our lips. \\
\poeml \v{3}Assyria won't save us; \\
\poemll    we won't be riding on horses, \\
\poeml Nor will we be saying anymore to the work of our hands, \\
\poemll    ``You are\fnote{\fbackref{14:3} The Heb. lacks \fbib{You are}} our God.'' \\
\poemlll       Indeed, in you the orphan finds mercy.' \\
\poeml \v{4}``I will correct their apostasy, \\
\poemll    loving them freely, \\
\poemlll       since my anger will have turned away from them.\fnote{\fbackref{14:4} Lit. \fbib{him}} \\
\poeml \v{5}I will be like the dew to Israel; \\
\poemll    Israel\fnote{\fbackref{14:5} Lit. \fbib{he}} will blossom like a lily, \\
\poemlll       growing roots like the cedars of\fnote{\fbackref{14:5} The Heb. lacks \fbib{the cedars of}} Lebanon. \\
\poeml \v{6}Israel's\fnote{\fbackref{14:6} Lit. \fbib{His}} branches will spread out, \\
\poemll    and its beauty will be like an olive tree, \\
\poemlll       with its scent like that of Lebanon. \\
\poeml \v{7}Those who live under its protection\fnote{\fbackref{14:7} Lit. \fbib{shadow}} will surely return. \\
\poemll    Their grain will flourish; \\
\poeml they will blossom like a vine, \\
\poemll    and Israel's\fnote{\fbackref{14:7} Lit. \fbib{His}} scent will be like wine from Lebanon. \\
\poeml \v{8}``Ephraim, what have I in common with idols? \\
\poemll    I have listened and will pay attention to him. \\
\poeml I am like a flourishing cypress; \\
\poemll    in me will your fruit be found.''
\passage{Concluding Counsel}
\poeml \v{9}Whoever is wise, let him understand these things. \\
\poemll    Whoever is discerning, let him know them. \\
\poeml For the ways of the \divine{Lord} are right: \\
\poemll    the righteous follow his example, \\
\poemlll       but the rebellious stumble in them.\end{poetry}

\bookheader{Joel}
\labelbook{Joel}

\bookpretitle{The Book of the Prophet}
\booktitle{Joel}

\labelchapt{1}
\passage{The Coming Invasion}

\chapt{1}
\v{1}This message from the \divine{Lord} came to Pethuel's son Joel.\fnote{The Heb. name \fbib{Joel} means \fbib{The \divine{Lord} is God}}

\begin{poetry}
\poeml \v{2}``Hear this, you elders! \\
\poeml Listen, all of you residents of the land! \\
\poeml Has there ever been anything like this during your lifetime,\fnote{Lit. \fbib{this in your days}} \\
\poemll    or even when your ancestors were alive?\fnote{Lit. \fbib{even in the days of your ancestors}} \\
\poeml \v{3}Pass it on to your children, \\
\poemll    and from\fnote{The Heb. lacks \fbib{from}} your children to their children, \\
\poemlll       and from\fnote{The Heb. lacks \fbib{from}} their children to the following generation. \\
\poeml \v{4}Whatever the devouring locust left behind \\
\poemll    the locust swarm has consumed! \\
\poeml Whatever the locust swarm has left behind, \\
\poemll    the young locust\fnote{Or \fbib{caterpillar}} has consumed! \\
\poeml Whatever the young locust\fnote{Or \fbib{caterpillar}} has left behind, \\
\poemll    the ravaging locust has consumed!''
\passage{A Call to Mourning}
\poeml \v{5}``Wake up, you drunkards! \\
\poemll    Cry aloud and howl, you wine drinkers, \\
\poemlll       because your supply of new wine has been snatched from you.\fnote{Lit. \fbib{from your lips}} \\
\poeml \v{6}Indeed, a nation has invaded my land--- \\
\poemll    it is strong and its population is too large to count\fnote{Lit. \fbib{and innumerable}}--- \\
\poeml with teeth like a lion \\
\poemll    and fangs\fnote{Or \fbib{jaws}} like a lioness. \\
\poeml \v{7}That nation\fnote{Lit. \fbib{It}} laid waste my vines, \\
\poemll    and stripped bare my fig tree, \\
\poemlll       discarding it. \\
\poeml It stripped off\fnote{Lit. \fbib{made white}} its bark. \\
\poeml \v{8}``Grieve like a virgin, \\
\poemll    who, dressed in her mourner's clothes,\fnote{Or \fbib{in sackcloth}} \\
\poemlll       cries out in memory\fnote{The Heb. lacks \fbib{in memory}} of the man she was going to marry.\fnote{Lit. \fbib{the husband of her youth}} \\
\poeml \v{9}Both grain offering and wine offering have been removed from the \divine{Lord}'s Temple;\fnote{Or \fbib{house}; and so throughout the book} \\
\poemll    the priests and ministering servants of the \divine{Lord} are mourning.''
\passage{The Coming Famine}
\poeml \v{10}``The fields lie in ruins \\
\poemll    and the ground is dried up.\fnote{Or \fbib{ground mourns}} \\
\poeml Indeed, the grain is ruined, \\
\poemll    the new wine has evaporated, \\
\poemlll       and the olive oil has run out. \\
\poeml \v{11}Be dismayed, you farmers! \\
\poemll    Cry aloud, you vintners, \\
\poemlll       for the wheat and barley, \\
\poemll    because the harvest in your fields has been lost. \\
\poeml \v{12}The grapevine is shriveled \\
\poemll    and the fig tree is withered, \\
\poeml along with the pomegranate tree, the palm tree, the apple tree \\
\poemll    and all of the cultivated trees.\fnote{Lit. \fbib{the trees of the field}} \\
\poeml Truly, joy has evaporated from Adam's children.''\fnote{Lit. \fbib{from sons of mankind}}
\passage{A Call to Mourn and Repent}
\poeml \v{13}``Put on your mourning clothes, you priests; \\
\poemll    and cry aloud, you ministering servants at the altar! \\
\poeml Come! Stay the night in mourner's clothes,\fnote{Or \fbib{in sackcloth}} you ministers of my God, \\
\poemll    because the grain offering and the wine offering is held back from the Temple of your God. \\
\poeml \v{14}Set apart time for a fast! \\
\poemll    Call a solemn assembly! \\
\poeml Gather the elders and everyone living in the land to the Temple of the \divine{Lord} your God, \\
\poemll    and cry out to the \divine{Lord}!''
\passage{A Lament about the Day of the \divine{Lord}}
\poeml \v{15}Oh, no! For the Day of the \divine{Lord} approaches, \\
\poemll    and like destruction from the Almighty, it will come! \\
\poeml \v{16}Isn't our food supply cut off right in front of us,\fnote{Lit. \fbib{cut off before our eyes}} \\
\poemll    along with joy and gladness from the Temple of our God? \\
\poeml \v{17}Seeds shrivel within their furrows, \\
\poemll    the storehouses lie empty, \\
\poeml and granaries stand in ruins \\
\poemll    because the grain has withered. \\
\poeml \v{18}Oh, how the livestock groan! \\
\poemll    The herds of cattle\fnote{Or \fbib{oxen}} wander about \\
\poemlll       because they have no pasture. \\
\poeml Even flocks of sheep suffer! \\
\poeml \v{19}To you, \divine{Lord}, I cry out, \\
\poemll    because fire has devoured the open pastures, \\
\poemlll       and has set all the cultivated trees\fnote{Lit. \fbib{the trees of the field}} ablaze. \\
\poeml \v{20}The livestock also cries out to you, \\
\poemll    because their water sources have evaporated \\
\poemlll       and because fire has consumed the open pastures.
\end{poetry}
\labelchapt{2}
\passage{The Warning of God}

\begin{poetry}
\poeml \chapt{2}
\v{1}``Sound the ram's horn in Zion! \\
\poeml Sound an alarm on my holy mountain! \\
\poeml Tremble, all of you\fnote{The Heb. lacks \fbib{of you}} inhabitants of the land, \\
\poeml because the Day of the \divine{Lord} is coming. \\
\poemlll       Oh, how near it is! \\
\poeml \v{2}A day of doom and gloom, \\
\poemll    a day of clouds and shadows\fnote{Cf. Zeph 1:15b} \\
\poeml like the dawn spreads out to cover the mountains--- \\
\poemll    a people strong and robust. \\
\poeml Never has there been anything like it, \\
\poemll    neither will anything follow to compare with\fnote{The Heb. lacks \fbib{to compare with}} it, \\
\poemlll       even through the lifetime of generation upon generation.''\fnote{Lit. \fbib{the years of generation and generation}}
\passage{Joel's Description of the Approaching Army}
\poeml \v{3}``A fire blazes in their presence, \\
\poemll    and behind them a conflagration rages. \\
\poeml Before they come, the land is like the garden in Eden; \\
\poemll    after they leave, there is only a barren wasteland. \\
\poemlll       Indeed, nothing escapes them. \\
\poeml \v{4}As to their form, they're like horses; \\
\poemll    and like chariot horses, how they can\fnote{The Heb. lacks \fbib{can}} run! \\
\poeml \v{5}They leap like the rumbling of chariots echoing from mountain tops, \\
\poemll    like the roar of wild fire that devours the chaff, \\
\poemlll       as an army\fnote{Lit. \fbib{people}} firmly established in battle array. \\
\poeml \v{6}The people are terrified in their presence; \\
\poemll    every face grows pale.\fnote{Lit. \fbib{gathers blackness}; cf. Nah 2:10b} \\
\poeml \v{7}They run like elite soldiers, \\
\poemll    climbing ramparts like men trained for war. \\
\poeml Each man advances in proper order, \\
\poemll    never breaking rank. \\
\poeml \v{8}Neither does a man crowd his fellow soldier;\fnote{Lit. \fbib{his brother}} \\
\poemll    each one marches in his own path. \\
\poeml When they fall by the sword \\
\poemll    they are not injured. \\
\poeml \v{9}They swarm through the city, \\
\poemll    running upon its ramparts. \\
\poeml Climbing atop the houses, \\
\poemll    they enter through windows like a thief.''
\passage{Great is the Day of the \divine{Lord}}
\poeml \v{10}``The land quivers in their presence; \\
\poemll    even the heavens shake. \\
\poeml The sun and moon will grow dark, \\
\poemll    and the stars will stop shining. \\
\poeml \v{11}The \divine{Lord} will shout in the presence of his forces, \\
\poemll    because his encampment is very great; \\
\poeml for powerful is he who carries out his message. \\
\poemll    Truly the Day of the \divine{Lord} is great, and very terrifying. \\
\poemlll       Who will be able to survive\fnote{Or \fbib{comprehend}} it?''
\passage{Repentance and Restoration}
\poeml \v{12}``Yet even now,'' declares the \divine{Lord}, \\
\poemll    ``Turn back to me with your whole heart, \\
\poemlll       with fasting, tears, and mourning. \\
\poeml \v{13}Tear your hearts, not your garments;\fnote{An allusion to Heb. custom of tearing the outer clothing in response to mourning} \\
\poemll    and turn back to the \divine{Lord} your God. \\
\poeml For he is gracious and compassionate, \\
\poemll    slow to become angry, \\
\poeml overflowing in gracious love, \\
\poemll    and grieves about this evil. \\
\poeml \v{14}Who knows? He will turn back and relent, will he not, \\
\poemll    leaving behind a blessing, \\
\poemlll       even a grain offering and drink offering for the \divine{Lord} your God?''
\passage{A Public Call to a Solemn Assembly}
\poeml \v{15}``Sound the ram's horn in Zion! \\
\poemll    Dedicate a fast and call for a solemn assembly! \\
\poeml \v{16}Gather the people! \\
\poemll    Dedicate the congregation! \\
\poeml Bring in the elders. \\
\poemll    Gather the youngsters \\
\poemlll       and even the nursing infants. \\
\poeml Call the bridegroom from his wedding preparations,\fnote{Lit. \fbib{Bring out the bridegroom from his wedding chamber}} \\
\poemll    and the bride from her dressing room. \\
\poeml \v{17}As they serve\fnote{The Heb. lacks \fbib{As they serve}} between the porch and the altar, \\
\poemll    let the priests and ministers of the \divine{Lord} weep and pray: \\
\poeml `Spare your people, \divine{Lord}, \\
\poemll    and do not make your heritage a disgrace \\
\poemlll       so that nations ridicule them. \\
\poeml Why should they say among the people, \\
\poemll    ``Where is their God?''\,'\,''
\passage{Response to the People's Repentance}
\poeml \v{18}Then the \divine{Lord} will show great concern for his land, \\
\poemll    and will have compassion on his people. \\
\poeml \v{19}The \divine{Lord} will say to his people, \\
\poemll    ``Look! I will send you grain, new wine, and oil, \\
\poemlll       and you will be content with them. \\
\poeml I will no longer cause you to be a disgrace among the nations.''
\passage{Destruction of the Invaders}
\poeml \v{20}``I will remove the northerners\fnote{Lit. \fbib{the North}; i.e. the army that comes from the North} from you, \\
\poemll    driving them\fnote{Lit. \fbib{him}; i.e. the northern army symbolized as an individual} to a barren and desolate land--- \\
\poeml the front toward the Dead Sea\fnote{Lit. \fbib{the eastern sea}} \\
\poemll    and the back toward the Mediterranean.\fnote{Lit. \fbib{the western sea}} \\
\poeml Their stench will rise, \\
\poemll    and their stinking odor will ascend, \\
\poemlll       because they have done great things.''
\passage{The \divine{Lord}'s Restoration of the Land}
\poeml \v{21}``Stop being afraid, land! \\
\poemll    Rejoice and be glad, \\
\poemlll       because the \divine{Lord} will do great things. \\
\poeml \v{22}Stop being afraid, beasts of the field, \\
\poemll    because the desert pastures will bloom, \\
\poeml the trees will bear their fruit, \\
\poemll    and the fig tree and vine will deliver their wealth. \\
\poeml \v{23}And so be glad, children of Zion, \\
\poemll    and rejoice in the \divine{Lord} your God, \\
\poeml because he has given you the right amount of early rain, \\
\poemll    and he will cause the rain to fall for you, \\
\poemlll       both the early rain and the later rain as before. \\
\poeml \v{24}The threshing floors will be smothered in grain, \\
\poemll    and the vats will overflow with wine and oil. \\
\poeml \v{25}``Then I will restore to you the years that the locust swarm devoured, \\
\poemll    as did the young locust, the other locusts, and the ravaging locust, \\
\poemlll       that great army of mine that I sent among you. \\
\poeml \v{26}You will have plenty to eat, and will be fully satisfied. \\
\poemll    You will praise the name of the \divine{Lord} your God, \\
\poeml who has performed wonders specifically for you. \\
\poemll    And my people will never be ashamed. \\
\poeml \v{27}As a result, you will know that I am in the midst of Israel; \\
\poemll    that I myself am the \divine{Lord} your God--- \\
\poemlll       and there is none other! \\
\poeml And my people will never be ashamed.''
\passage{The Day of the \divine{Lord}}
\poeml \v{28}\fnote{This verse is 3:1 in MT, v. 29 is 3:2, and so through the end of the chapter.}``Then it will come about at a later time \\
\poemll    that I will pour out my Spirit on every person. \\
\poeml Your sons and your daughters will prophesy. \\
\poemll    Your elderly people will dream dreams, \\
\poemlll       and your young people will see visions. \\
\poeml \v{29}Also at that time I will pour out my Spirit \\
\poemll    upon men and women servants. \\
\poeml \v{30}I will display warnings in the heavens, \\
\poemll    and on the earth blood, fire, and columns of smoke. \\
\poeml \v{31}The sun will be given over to darkness, \\
\poemll    and the moon to blood, \\
\poemlll       before the coming of the great and terrifying Day of the \divine{Lord}. \\
\poeml \v{32}And everyone who calls upon the name of the \divine{Lord} will be delivered. \\
\poemll    For as the \divine{Lord} has said, \\
\poemlll       `In Mount Zion and in Jerusalem there will be those who escape, \\
\poemlll       the survivors whom the \divine{Lord} is calling.'\,''
\end{poetry}
\labelchapt{3}
\passage{The Coming Judgment of Nations}

\begin{poetry}
\poeml \chapt{3}
\v{1}\fnote{This verse is 4:1 in MT, and so through the end of the chapter.}``Look, now! In those very days and at that time, \\
\poeml when I restore prosperity to\fnote{Or \fbib{bring back the captivity of}} Judah and Jerusalem, \\
\poeml \v{2}I will gather all nations, \\
\poemll    bringing them down to the Valley of Jehoshaphat. \\
\poeml I will set out my case against\fnote{Or \fbib{will judge}} them there, \\
\poemll    on behalf of my people, my heritage Israel, \\
\poeml whom they scattered among the nations, \\
\poemll    apportioning my land among themselves.\fnote{The Heb. lacks \fbib{among themselves}} \\
\poeml \v{3}They cast lots for my people--- \\
\poemll    they sold a young boy in exchange for a prostitute, \\
\poeml and a girl for wine, \\
\poemll    so they could drink.''
\passage{The \divine{Lord}'s Judgment upon Philistia}
\poeml \v{4}``Furthermore, what have you to do with me, \\
\poemll    Tyre, Sidon, and all the sea coasts of Philistia? \\
\poemlll       Are you taking revenge on me? \\
\poeml If you are taking revenge on me, \\
\poemll    I'll send it back on you\fnote{Lit. \fbib{your head}} swiftly and promptly, \\
\poeml \v{5}since you took my silver and gold, \\
\poemll    carried my precious treasures into your temples, \\
\poeml \v{6}and sold Judah's and Jerusalem's descendants to the Greeks,\fnote{Lit. \fbib{Jevanites}; i.e. descendants of Javan} \\
\poemll    so you can remove them far from their homeland! \\
\poeml \v{7}``Look, I will bring them up from where you sold them, \\
\poemll    I will turn your revenge back upon you,\fnote{Lit. \fbib{turn back your reward}} \\
\poeml \v{8}and I will sell your sons and daughters into the control of the people of Judah. \\
\poemll    And they will sell them to the people of Sheba, a country far away.'' \\
\poemlll       Indeed, the \divine{Lord} has spoken.''
\passage{The \divine{Lord}'s Call to Judgment}
\poeml \v{9}``Declare this among the nations: \\
\poemll    `Prepare for war! \\
\poeml Wake up your elite forces! \\
\poemll    Let all the soldiers draw near! \\
\poemlll       Call them up! \\
\poeml \v{10}Beat your plow blades into swords, \\
\poemll    and your pruning knives into spears! \\
\poemlll       Let the frail say, ``I am strong!'' \\
\poeml \v{11}Hurry and come, all you gentiles! \\
\poemll    Gather yourselves together!'\,''
\end{poetry}

\begin{poetry}
\poemlll       ``\divine{Lord}, cause your mighty army\fnote{Lit. \fbib{mighty ones}} to come down. \\
\poeml \v{12}``Let the nations be awakened \\
\poemll    and come to the Valley of Jehoshaphat; \\
\poemlll       because I will sit to judge all the surrounding nations. \\
\poeml \v{13}Put in the sickle, \\
\poemll    because the harvest is ripe. \\
\poeml Come and go down, \\
\poemll    because the winepress is full. \\
\poeml The wine vats are overflowing, \\
\poemll    because their evil is great! \\
\poeml \v{14}``Multitudes, multitudes \\
\poemll    in the Valley of Judgment! \\
\poeml For the Day of the \divine{Lord} is near \\
\poemll    in the Valley of Judgment! \\
\poeml \v{15}The sun and moon will grow dark, \\
\poemll    and the stars will stop shining. \\
\poeml \v{16}``The \divine{Lord} will roar from Zion, \\
\poemll    and shout from Jerusalem. \\
\poeml The heavens and the earth will shake, \\
\poemll    but the \divine{Lord} will be the refuge of his people, \\
\poemlll       and the strength of the people of Israel.''
\passage{God's Blessings on His People}
\poeml \v{17}``And truly you will know that I am the \divine{Lord} your God, \\
\poemll    dwelling in Zion, my holy mountain. \\
\poeml Then Jerusalem will be holy, \\
\poemll    and no foreigners will invade her again. \\
\poeml \v{18}It will come about at that time \\
\poemll    that the mountains will drip with newly pressed wine, \\
\poeml the hills will flow with milk, \\
\poemll    and the streams of Judah will flow abundantly. \\
\poeml A fountain will spring from the Temple of the \divine{Lord}, \\
\poemll    to water the Valley of the Acacias. \\
\poeml \v{19}Egypt will be desolate, \\
\poemll    and Edom will be a desert, \\
\poeml because of violence against the people of Judah \\
\poemll    since they shed innocent blood in their land. \\
\poeml \v{20}But Judah will live forever, \\
\poemll    and Jerusalem from generation to generation. \\
\poeml \v{21}I will acquit their bloodguilt that has not yet been acquitted. \\
\poemll    For the \divine{Lord} lives in Zion!''\end{poetry}

\bookheader{Amos}
\labelbook{Amos}

\bookpretitle{The Book of the Prophet}
\booktitle{Amos}

\labelchapt{1}
\passage{Amos is Called to Prophesy}

\chapt{1}
\v{1}The words of Amos,\fnote{The Heb. name \fbib{Amos} means \fbib{burden}} who was among the sheep breeders of Tekoa, which he spoke\fnote{Lit. \fbib{saw}} concerning Israel during the reign of\fnote{Or \fbib{in the days of}} Uzziah, king of Judah and during the reign of\fnote{Or \fbib{in the days of}} Joash's son Jeroboam, king of Israel, two years before the earthquake.

\begin{poetry}
\poeml \v{2}He said, ``From Zion the \divine{Lord} roars, \\
\poemll    and from Jerusalem he shouts aloud. \\
\poeml The shepherds' pastures will languish, \\
\poemll    and Carmel's summit will wither.''
\passage{A Warning to Damascus}
\poeml \v{3}This is what the \divine{Lord} says: \\
\poeml ``For three transgressions of Damascus \\
\poemll    ---and now for a fourth--- \\
\poemlll       I will not turn away; \\
\poeml because they have trampled down\fnote{Or \fbib{have threshed}} Gilead \\
\poemll    with ironclad threshing sleds. \\
\poeml \v{4}So I will send down fire upon the house of Hazael, \\
\poemll    and it will devour the palaces of Ben-hadad. \\
\poeml \v{5}I will shatter the gate bars of Damascus, \\
\poemll    and I will cut off the residents of the Aven Valley, \\
\poeml along with the one who holds the scepter from Beth-eden; \\
\poemll    and the people of Aram will be exiled to Kir,'' \\
\poemlll       says the \divine{Lord}.
\end{poetry}
\passage{A Warning to Gaza}

\v{6}This is what the \divine{Lord} says:

\begin{poetry}
\poeml ``For three transgressions of Gaza \\
\poemll    ---and now for a fourth--- \\
\poemlll       I will not turn away; \\
\poeml because they exiled the entire population, \\
\poemll    delivering them to Edom. \\
\poeml \v{7}So I will send down fire upon the wall of Gaza, \\
\poemll    and it will devour their fortified citadels; \\
\poeml \v{8}and I will cut off the inhabitants of Ashdod, \\
\poemll    along with Ashkelon's ruler.\fnote{Lit. \fbib{with the one who holds the scepter of Ashkelon}} \\
\poeml I will turn to attack\fnote{Lit. \fbib{turn my hand against}} Ekron, \\
\poemll    and the rest of the Philistines will die,'' \\
\poemlll       says the Lord \divine{God}.
\end{poetry}
\passage{A Warning to Tyre}

\v{9}This is what the \divine{Lord} says:

\begin{poetry}
\poeml ``For three transgressions of Tyre \\
\poemll    ---and now for a fourth--- \\
\poemlll       I will not turn away; \\
\poeml because they delivered the entire population to Edom, \\
\poemll    and did not remember their covenant with their relatives.\fnote{Or \fbib{brothers}} \\
\poeml \v{10}So I will send down fire upon the wall of Tyre, \\
\poemll    and it will devour their fortified citadels.''
\end{poetry}
\passage{A Warning to Edom}

\v{11}This is what the \divine{Lord} says:

\begin{poetry}
\poeml ``For three transgressions of Edom \\
\poemll    ---and now for a fourth--- \\
\poemlll       I will not turn away; \\
\poeml because he\fnote{I.e. the nation personified as an individual} pursued his brother with a sword, \\
\poemll    refusing to be compassionate.\fnote{Lit. \fbib{sword, abandoning his compassion}} \\
\poeml His anger was raging\fnote{Lit. \fbib{anger was tearing away}} continuously; \\
\poemll    he kept up his unending wrath. \\
\poeml \v{12}So I will send down fire upon Teman, \\
\poemll    and it will devour the fortified citadels of Bozrah.''
\end{poetry}
\passage{A Warning to Ammon}

\v{13}This is what the \divine{Lord} says:

\begin{poetry}
\poeml ``For three transgressions of the Ammonites \\
\poemll    ---and now for a fourth--- \\
\poemlll       I will not turn away; \\
\poeml because they ripped open the pregnant women of Gilead \\
\poemll    in order to enlarge their national borders.\fnote{Or \fbib{their boundary}} \\
\poeml \v{14}So I will send down fire upon the wall of Rabbah, \\
\poemll    and it will devour their fortified citadels \\
\poemlll       with an alarm sounding in the time of battle, \\
\poemll    and with a whirlwind in the time of storm. \\
\poeml \v{15}Their king will go into captivity--- \\
\poemll    he and his princes together,'' \\
\poemlll       says the \divine{Lord}.
\end{poetry}
\labelchapt{2}
\passage{A Warning to Moab}

\chapt{2}
\v{1}This is what the \divine{Lord} says:

\begin{poetry}
\poeml ``For three transgressions of Moab \\
\poeml because they\fnote{Lit. \fbib{he}; i.e. the nation personified as an individual} cremated the bones of the king of Edom, \\
\poemll    burning them\fnote{The Heb. lacks \fbib{burning them}} to lime. \\
\poeml \v{2}So I will send down fire upon Moab, \\
\poemll    and it will devour the fortified citadels of Kerioth. \\
\poeml Moab will die in the uproar of battle,\fnote{The Heb. lacks \fbib{of battle}} \\
\poemll    with a war cry \\
\poemlll       and with the trumpeting of the ram's horn. \\
\poeml \v{3}I will execute their rulers among them, \\
\poemll    killing all of their officials as well,'' \\
\poemlll       says the \divine{Lord}.
\end{poetry}
\passage{A Warning to Judah}

\v{4}This is what the \divine{Lord} says:

\begin{poetry}
\poeml ``For three transgressions of Judah \\
\poemll    ---and now for a fourth--- \\
\poemlll       I will not turn away; \\
\poeml because they\fnote{Lit. \fbib{he}; i.e. the nation personified as an individual} rejected the Law of the \divine{Lord} \\
\poemll    and did not keep his statutes. \\
\poeml Their own lies made them wander off, \\
\poemll    following along the same path their ancestors walked. \\
\poeml \v{5}So I will send down fire upon Judah, \\
\poemll    and it will devour the fortified citadels of Jerusalem.''
\end{poetry}
\passage{A Warning to Israel}

\v{6}This is what the \divine{Lord} says:

\begin{poetry}
\poeml ``For three transgressions of Israel \\
\poemll    ---and now for a fourth--- \\
\poemlll       I will not turn away; \\
\poeml because they sold the righteous for money, \\
\poemll    and the poor for sandals, \\
\poeml \v{7}moving quickly\fnote{Lit. \fbib{They chase}} to rub the face\fnote{Lit. \fbib{head}} of the needy in the dirt. \\
\poeml Corrupting\fnote{Lit. \fbib{turning aside}} the ways of the humble, \\
\poemll    a man and his father go to the same woman, \\
\poemlll       deliberately defiling my holy name. \\
\poeml \v{8}They lay down beside every altar, \\
\poemll    on garments pledged as collateral,\fnote{I.e. violating Deut 24:10-13 in addition to idolatry} \\
\poeml drinking wine paid for through fines \\
\poemll    imposed by the temple of their gods. \\
\poeml \v{9}Yet it was I who destroyed the Amorites in front of them, \\
\poemll    though their height seemed like a cedar,\fnote{I.e. a genus of coniferous evergreen in the family \fbib{Pinaceae}} \\
\poemll    though their strength seemed like an oak, \\
\poeml but whose fruit I destroyed from above \\
\poemll    and the roots from beneath. \\
\poeml \v{10}Furthermore, I brought you up from the land of Egypt, \\
\poemll    leading you in the wilderness for 40 years, \\
\poemlll       to take possession of the land of the Amorites. \\
\poeml \v{11}I also raised up your sons to be prophets, \\
\poemll    and from your young men I raised up Nazirites.\fnote{I.e. men who make a special vow with God (cf. Num 6:1-21)} \\
\poeml Is this not true, people of Israel?'' \\
\poemll    declares the \divine{Lord}. \\
\poeml \v{12}``But you forced the Nazirites to drink wine, \\
\poemll    and commanded the prophets, \\
\poemlll       `You are not to prophesy!' \\
\poeml \v{13}``Oh, how I am burdened down with you, \\
\poemll    as a wagon is overloaded with harvested grain! \\
\poeml \v{14}So the swift runner will not escape,\fnote{Lit. \fbib{So flight will escape the swift runner}} \\
\poemll    the valiant will not fortify his strength, \\
\poemlll       and the mighty warrior will not save his life. \\
\poeml \v{15}The skilled archer will not be able to stand, \\
\poemll    the swift runner will not survive, \\
\poemlll       and the mounted rider will not preserve his own life. \\
\poeml \v{16}Even the bravest of elite troops will run away naked at that time,'' \\
\poemll    declares the \divine{Lord}.
\end{poetry}
\labelchapt{3}
\passage{A Higher Standard of Accountability}

\chapt{3}
\v{1}``Listen to this message that the \divine{Lord} has spoken about you, people of Israel. It concerns the entire family that I brought from the land of Egypt:

\begin{poetry}
\poeml \v{2}`You alone have I known from among all of the families of mankind; \\
\poemll    therefore I will hold you accountable for all your iniquities.'\,''
\passage{Seven Questions to Ponder}
\poeml \v{3}``Will a couple walk in unity \\
\poemll    without having met? \\
\poeml \v{4}Will a lion roar in the forest \\
\poemll    without having found its prey? \\
\poeml Will a young lion cry from its den \\
\poemll    without having caught anything? \\
\poeml \v{5}Does a bird fall into a snare on the ground \\
\poemll    without any bait in the trap? \\
\poeml Will a trap snap shut \\
\poemll    when there is nothing to catch? \\
\poeml \v{6}And when an alarm\fnote{Lit. \fbib{trumpet}} sounds in the city, \\
\poemll    the people will tremble, won't they? \\
\poeml If there is trouble in a city, \\
\poemll    the \divine{Lord} has brought it about, has he not?''
\passage{The \divine{Lord}'s Purposes}
\poeml \v{7}``Truly the Lord \divine{God} will do nothing he has mentioned \\
\poemll    without revealing his purposes to his servants the prophets. \\
\poeml \v{8}A lion has roared! \\
\poemll    Who will not fear? \\
\poeml The Lord \divine{God} has spoken! \\
\poemll    Who will not prophesy? \\
\poeml \v{9}Announce this\fnote{The Heb. lacks \fbib{this}} in the fortified citadels of Ashdod, \\
\poemll    and in the fortified citadels of the land of Egypt. \\
\poeml Tell them, `Gather together on the mountains of Samaria; \\
\poemll    look at the great misery among the citadels,\fnote{Lit. \fbib{them}} \\
\poemlll       along with the oppression within Egypt.'\fnote{Lit. \fbib{her}; i.e. Egypt personified as a woman} \\
\poeml \v{10}Because they do not know how to act right,'' \\
\poemll    declares the \divine{Lord}, \\
\poeml ``they are filling their strongholds with treasures \\
\poemll    that they took from others by violence into their fortified citadels.'' \\
\poeml \v{11}Therefore this is what the Lord \divine{God} says: \\
\poeml ``An enemy will surround the land. \\
\poemll    He\fnote{I.e. the invading forces personified as an individual} will pull down your defenses, \\
\poemlll       and plunder your fortified citadels.'' \\
\poeml \v{12}This is what the \divine{Lord} says: \\
\poeml ``Just as a shepherd might save from the lion's mouth \\
\poemll    only two leg bones or a scrap of an ear, \\
\poeml the Israelis will be saved in a similar manner--- \\
\poemll    those in Samaria who sit on the remains of their broken beds,\fnote{Lit. \fbib{on the corner of a bed}} \\
\poemlll       and those in Damascus who lie on the edge of their couches.'' \\
\poeml \v{13}``Listen and testify against the house of Jacob,'' \\
\poemll    declares the Lord \divine{God}, the God of the Heavenly Armies, \\
\poeml \v{14}``because on that day I will lay out the charges against Israel. \\
\poemll    I will also bring judgment upon the altars of Bethel; \\
\poeml the horns of the altar will be cut off \\
\poemll    and will fall to the ground. \\
\poeml \v{15}I will wreck both the winter house and the summer house, \\
\poemll    and the ivory houses will fall.\fnote{Lit. \fbib{perish}} \\
\poeml These palaces will surely fall,'' \\
\poemll    declares the \divine{Lord}.
\end{poetry}
\labelchapt{4}
\passage{Judgment on the Women of Israel}

\begin{poetry}
\poeml \chapt{4}
\v{1}``Listen to this message, you fat cows from Bashan, \\
\poeml who live on the Samaritan mountains, \\
\poemll    who oppress the poor, \\
\poeml who rob the needy, \\
\poemll    and who constantly ask your husbands for one more drink!'' \\
\poeml \v{2}The Lord \divine{God} has taken a sacred oath:\fnote{Lit. \fbib{oath on his holiness}} \\
\poeml ``The day is coming when they\fnote{Lit. \fbib{coming upon you, and he}; i.e. the invading forces personified as an individual} will take you away on fishhooks, \\
\poemll    every last one of you on fishhooks. \\
\poeml \v{3}Each of you will go out through the breaches of the walls\fnote{The Heb. lacks \fbib{of the walls}} \\
\poemll    straight to Mt. Hermon,''\fnote{Heb. \fbib{Harmonah}} \\
\poemlll       declares the \divine{Lord}.
\passage{The \divine{Lord}'s Rebuke to Israel}
\poeml \v{4}``Come to Bethel and sin, \\
\poemll    to Gilgal and sin even more! \\
\poeml Bring along your morning sacrifices, \\
\poemll    and pay your tithes every other day.\fnote{Lit. \fbib{tithes for the three days}} \\
\poeml \v{5}While you're at it,\fnote{Lit. \fbib{And}} present a thank offering with leaven, \\
\poemll    and publicize your freewill offerings, \\
\poeml letting everyone hear about it, \\
\poemll    because this is what you really love to do, you Israelis,'' \\
\poemlll       declares the Lord \divine{God}.
\passage{Israel's Refusal to Return to God}
\poeml \v{6}``I also have scheduled\fnote{Lit. \fbib{appointed}} food shortages\fnote{Lit. \fbib{appointed clean teeth}} for you in all of your cities, \\
\poemll    and lack of bread in all of your settlements, \\
\poeml but you haven't returned to me,'' \\
\poemll    declares the \divine{Lord}. \\
\poeml \v{7}``I therefore have withheld the rain from you \\
\poemll    three months before the harvest, \\
\poeml causing rain to come upon one city, \\
\poemll    but not upon another, \\
\poeml and upon one field \\
\poemll    but not upon another, \\
\poemlll       so that it would wither. \\
\poeml \v{8}So the people of\fnote{The Heb. lacks \fbib{the people of}} two or three cities staggered away to another\fnote{Lit. \fbib{to a single}} city \\
\poemll    in order to obtain drinking water, \\
\poeml but you have not returned to me,'' \\
\poemll    declares the \divine{Lord}. \\
\poeml \v{9}``I afflicted you with blight and fungus; \\
\poemll    and the locust swarm devoured the harvest \\
\poemlll       of your gardens, your vineyards, your fig trees, and your olive trees, \\
\poemll    but you have not returned to me,'' \\
\poemlll       declares the \divine{Lord}. \\
\poeml \v{10}``I sent plagues among you as I did with Egypt. \\
\poemll    I killed your choicest young men with the sword. \\
\poeml I took your horses away from you. \\
\poemll    I filled your noses with the stench of your encampments, \\
\poeml but you have not returned to me,'' \\
\poemll    declares the \divine{Lord}. \\
\poeml \v{11}``I overthrew your cities,\fnote{The Heb. lacks \fbib{cities}} \\
\poemll    as God overthrew Sodom and Gomorrah. \\
\poeml You've become like a burning ember, snatched from the fire, \\
\poemll    but you have not returned to me,'' \\
\poemlll       declares the \divine{Lord}. \\
\poeml \v{12}``Therefore this is what I will do to you, Israel. \\
\poemll    Because I am about to do this, \\
\poemlll       prepare to be summoned to your God, Israel!'' \\
\poeml \v{13}Look! The one who crafts mountains, \\
\poemll    who creates the wind, \\
\poeml who reveals what he is thinking to mankind, \\
\poemll    who darkens the morning light, \\
\poeml who tramples down the high places of the land--- \\
\poemll    the \divine{Lord}, the God of the Heavenly Armies is his name.
\end{poetry}
\labelchapt{5}
\passage{A Lament for Israel}

\begin{poetry}
\poeml \chapt{5}
\v{1}``Hear this accusation\fnote{Lit. \fbib{word}} that I am bringing against you:
\end{poetry}

\begin{poetry}
\poeml `A dirge, house of Israel: \\
\poeml \v{2}Fallen is Israel the virgin---never to rise again! \\
\poemll    She is abandoned on her own land, \\
\poemlll       with no one to raise her up.' \\
\poeml \v{3}``For this is what the Lord \divine{God} says: \\
\poeml `The city that is sending out a thousand \\
\poemll    will have a hundred left; \\
\poeml The city\fnote{The Heb. lacks \fbib{The city}} that is sending out a hundred \\
\poemll    will have ten left of the house of Israel.'\,''
\passage{Seek God, and Live}
\poeml \v{4}``For this is what the \divine{Lord} says to the house of Israel: \\
\poeml `Seek me and live, \\
\poeml \v{5}but don't seek Bethel. \\
\poeml Don't go to Gilgal, \\
\poemlll       and don't pass over to Beer-sheba. \\
\poeml Because Gilgal will surely go into captivity,\fnote{The root Heb. for \fbib{Gilgal} is a pun on the Heb. \fbib{go into captivity}} \\
\poemll    and Bethel will come to nothing. \\
\poeml \v{6}`Seek the \divine{Lord} and live! \\
\poemll    Otherwise, he may break out like a fire in the house of Joseph \\
\poemlll       and devour Bethel,\fnote{So MT; LXX reads \fbib{devour the House of Israel}} \\
\poemll    and there will be no one to extinguish it. \\
\poeml \v{7}Those of you who are making justice taste bitter,\fnote{Lit. \fbib{are turning justice into wormwood}} \\
\poemll    and who have thrown righteousness to the ground: \\
\poeml \v{8}Seek\fnote{The Heb. lacks \fbib{Seek}} the one who fashions the Pleiades and Orion, \\
\poemll    who turns the deep darkness\fnote{Or \fbib{the shadow of death}} into morning, \\
\poemll    who darkens day into night, \\
\poemll    who calls out to the waters of the sea, \\
\poemlll       pouring them out onto the surface of the earth--- \\
\poemll    the \divine{Lord} is his name. \\
\poeml \v{9}It is he who is raining sudden destruction \\
\poemll    upon the strong like lightning,\fnote{The Heb. lacks \fbib{like lightning}} \\
\poemlll       so that ruin comes upon the fortress. \\
\poeml \v{10}They have hated those who are presenting their cases in court,\fnote{Lit. \fbib{in the gate}} \\
\poemll    detesting the one who speaks truthfully. \\
\poeml \v{11}`Therefore, since you trample the poor continuously, \\
\poemll    taxing his grain, \\
\poemll    building houses of stone in which you won't live \\
\poemll    and planting fine vineyards from which you won't drink--- \\
\poeml \v{12}and because I know that your transgressions are many, \\
\poemll    and your sins are numerous \\
\poeml as you oppose the righteous, \\
\poemll    taking bribes as a ransom, \\
\poemlll       and turning away the poor in court\fnote{Lit. \fbib{in the gate}}--- \\
\poeml \v{13}therefore the prudent person remains silent at such a time, \\
\poemll    for the time is evil. \\
\poeml \v{14}`Pursue good and not evil, \\
\poemll    so that you may live, \\
\poeml and this is what will happen:\fnote{Lit. \fbib{And so it was}} \\
\poemll    The \divine{Lord} God of the Heavenly Armies will be with you, \\
\poemlll       as you have been claiming. \\
\poeml \v{15}Hate evil and love good, \\
\poemll    and establish justice in court---\fnote{Lit. \fbib{in the gates}} \\
\poeml perhaps the \divine{Lord}, the God of the Heavenly Armies, \\
\poeml will be gracious to the survivors of Joseph.'\,'' \\
\poeml \v{16}Therefore this is what the \divine{Lord}, the God of the Heavenly Armies, the Lord, says: \\
\poeml `There will be dirges in all of the streets; \\
\poemlll       and in all of the highways they will cry out in anguish.\fnote{Lit. \fbib{will say, ``Alas! Alas!''}} \\
\poeml They will call the farmer to mourning \\
\poemll    and those who lament\fnote{I.e. professional mourners} to grieve. \\
\poeml \v{17}And in all of the vineyards there will be mourning \\
\poemll    when I pass through your midst,' \\
\poemlll       says the \divine{Lord}.''
\passage{The Fearful Day of the \divine{Lord}}
\poeml \v{18}``Woe to those who are craving the Day of the \divine{Lord}! \\
\poemll    How is it to your benefit, this Day of the \divine{Lord}? \\
\poemlll       It's a day of\fnote{The Heb. lacks \fbib{a day of}} darkness to you, and not light. \\
\poeml \v{19}It will be like a man who runs from a lion, \\
\poemll    only to encounter a bear; \\
\poeml or who comes home, leans his hand against a wall, \\
\poemll    and a serpent bites him! \\
\poeml \v{20}Will not the Day of the \divine{Lord} be darkness, and not light--- \\
\poemll    pitch black at that, without a ray of sunshine?''
\passage{Let Justice Roll On}
\poeml \v{21}``I hate---I despise---your festival days, \\
\poemll    and your solemn convocations stink.\fnote{Lit. \fbib{and I smell no pleasant scent in your solemn assemblies}} \\
\poeml \v{22}And\fnote{Lit. \fbib{Because}} if you send up burnt offerings to me \\
\poemll    as well as your grain offerings, \\
\poeml I will not accept them, \\
\poemll    nor will I consider your peace offerings of fattened cattle. \\
\poeml \v{23}Spare me your noisy singing--- \\
\poemll    I will not listen to your musical instruments.\fnote{I.e. a stringed instrument such as a harp or lyre} \\
\poeml \v{24}``But let justice roll on like many\fnote{The Heb. lacks \fbib{many}} waters, \\
\poemll    and righteousness like an ever-flowing river. \\
\poeml \v{25}``Was it to me that you brought offerings and gifts \\
\poemll    in the desert for 40 years, house of Israel? \\
\poeml \v{26}And you carried the tent of your king\fnote{LXX reads \fbib{of Moloch}; MT reads \fbib{carried Sikkuth your king}}--- \\
\poemll    and Saturn,\fnote{Heb. \fbib{Kiyyun}} your star god idols\fnote{So MT; LXX reads \fbib{and the star of your God Raiphan, the images}} that you crafted for yourselves. \\
\poeml \v{27}So I will cause you to be taken captive beyond Damascus,'' \\
\poemll    says the \divine{Lord}, \\
\poemlll       whose name is God of the Heavenly Armies.
\end{poetry}
\labelchapt{6}
\passage{Mourning for the House of Israel}

\begin{poetry}
\poeml \chapt{6}
\v{1}``Woe to those who are at ease in Zion, \\
\poemll    to those who rest on the mountain of Samaria--- \\
\poeml the famous men of the nations \\
\poemll    to whom the house of Israel came! \\
\poeml \v{2}Cross over to Calneh\fnote{I.e. a Mesopotamian city} and look around, \\
\poemll    then go on to that great city of\fnote{The Heb. lacks \fbib{city of}} Hamath, \\
\poemlll       and from there go down to Gath of the Philistines. \\
\poeml Are you better than these kingdoms? \\
\poemll    Or is their territory more extensive than yours? \\
\poeml \v{3}``Disbelieving that a day of evil will come,\fnote{The Heb. lacks \fbib{will come}} \\
\poemll    embracing opportunities to commit violence,\fnote{Lit. \fbib{yet are pressing hard the seat of violence}} \\
\poeml \v{4}lying on ivory beds, \\
\poemll    stretching out on your couches, \\
\poeml eating lambs from the flock, \\
\poemll    and fattened calves from the stall, \\
\poeml \v{5}chanting to the sound of stringed instruments as if they were David, \\
\poemll    composing songs to themselves as if they were musicians, \\
\poeml \v{6}drinking wine from bowls, \\
\poemll    anointing themselves with the choicest of oils, \\
\poemlll       but not grieving on the occasion of Joseph's ruin--- \\
\poeml \v{7}therefore you will be the first to go into exile, \\
\poemll    and the celebrations of those who are lounging will end.''
\passage{The \divine{Lord} Swears an Oath}
\poeml \v{8}``The Lord \divine{God} has sworn by himself,'' \\
\poemll    declares the \divine{Lord}, the God of the Heavenly Armies, \\
\poeml ``I utterly detest the arrogance of Jacob; \\
\poemll    I hate his fortresses; \\
\poeml and I will deliver up the city, \\
\poemll    along with everyone in it. \\
\poeml \v{9}``And if there are ten men remaining in one house, \\
\poemll    they will die. \\
\poeml \v{10}One's relative will pick up the corpse\fnote{Lit. \fbib{bones}} \\
\poemll    to carry them from the house for burning,\fnote{Or \fbib{house to burn incense}} \\
\poeml saying to whomever remains inside the house, \\
\poemll    `Is there anyone still with you?' \\
\poeml And he will say, `No.' \\
\poemll    He will respond, `Be quiet, \\
\poemlll       because we do not mention the name ``\divine{Lord}''.' \\
\poeml \v{11}For indeed, the \divine{Lord} is giving the command--- \\
\poemll    and he will smash the large house to rubble \\
\poemlll       and the small house into bits. \\
\poeml \v{12}``Horses don't run over bare rock, do they? \\
\poemll    One doesn't plow rock\fnote{The Heb. lacks \fbib{rock}} with oxen, does he? \\
\poeml But you have turned justice to gall, \\
\poemll    and the fruit of righteousness into bitterness.\fnote{Lit. \fbib{wormwood}} \\
\poeml \v{13}You rejoice in nothing worth mentioning--- \\
\poemll    that is, you keep on saying, \\
\poeml `We captured Karnaim by our own strength of will \\
\poemll    and by our own effort, didn't we?' \\
\poeml \v{14}``So look, house of Israel! I will raise up a nation against you,'' \\
\poemll    declares the \divine{Lord}, the God of the Heavenly Armies, \\
\poeml ``and they will harass you from the entrance of Hamath \\
\poemll    to the wadi\fnote{I.e. perhaps the Wadi of Egypt, a seasonal stream or river that channels water during rain seasons but is dry at other times; ancient Israel's southwestern-most border} of the wilderness.''
\end{poetry}
\labelchapt{7}
\passage{The Vision of Locusts}

\chapt{7}
\v{1}This is what the Lord \divine{God} showed me: Look! He was forming locust swarms as the latter plantings were just beginning to sprout. Indeed, the king had just taken his first fruit tax.\fnote{So MT; LXX reads \fbib{the locusts have one king, Gog.}} \v{2}And so it came about that when the swarm\fnote{The Heb. lacks \fbib{the swarm}} had finished eating the grass of the land, I was saying,

\begin{poetry}
\poeml ``Lord \divine{God}, forgive---please! \\
\poemll    How will Jacob stand, since he is small?''
\end{poetry}

\v{3}So the \divine{Lord} relented from this. ``This will not happen,'' said the \divine{Lord}.
\passage{The Vision of Fire}

\v{4}This is what the Lord \divine{God} showed me: Look! The Lord \divine{God} was calling for judgment by fire, and it was drying up the great depths of the ocean\fnote{The Heb. lacks \fbib{of the ocean}} and consuming the land. \v{5}So I kept on saying,

\begin{poetry}
\poeml ``Lord \divine{God}, forgive---please! \\
\poemll    How will Jacob stand, since he is so small?''
\end{poetry}

\v{6}So the \divine{Lord} relented from this. ``This will not happen, either,'' said the Lord \divine{God}.
\passage{The Vision of the Plumb Line}

\v{7}This is what he showed me: Look! The Lord was standing upon a wall that stood straight and true, with a plumb line in his hand.\fnote{Lit. \fbib{wall by a plumb line}} \v{8}And the \divine{Lord} was asking me, ``What do you see, Amos?''

I replied, ``A plumb line.''

So the Lord said,

\begin{poetry}
\poeml ``Look, I have set a plumb line \\
\poemll    in the midst of my people Israel. \\
\poemlll       I will no longer spare them. \\
\poeml \v{9}Isaac's high places will be destroyed, \\
\poemll    and the sanctuaries of Israel will be ruined. \\
\poemlll       I will rise in opposition to the house of Jeroboam with my\fnote{The Heb. lacks \fbib{my}} sword.''
\end{poetry}
\passage{A Rebuke for Amaziah}

\v{10}So Amaziah priest of Bethel sent a message\fnote{The Heb. lacks \fbib{a message}} to Jeroboam king of Israel. It said, ``Amos has been conspiring against you in the very heart of the house of Israel! The land cannot bear everything he has to say, \v{11}because Amos is saying this:

\begin{poetry}
\poeml `By the sword will Jeroboam die, \\
\poemll    and Israel will surely go into exile \\
\poemlll       far from her homeland.'\,''
\end{poetry}

\v{12}So Amaziah kept saying to Amos, ``Get out of here, you seer! Go back to the land of Judah. Live\fnote{Lit. \fbib{Eat}} there and prophesy there. \v{13}Don't prophesy anymore at Bethel, because it's the king's sanctuary and a temple of the kingdom.''

\v{14}Amos replied in answer to Amaziah,

\begin{poetry}
\poeml ``I am no prophet, \\
\poemll    nor am I a prophet's son, \\
\poeml for I have been shepherding \\
\poemll    and picking the fruit of\fnote{The Heb. lacks \fbib{the fruit of}} sycamore\fnote{The sycamore fruit tree native to Israel bears figs} trees.
\end{poetry}

\v{15}But the \divine{Lord} took me from tending the flock and the \divine{Lord} kept saying to me, `Go, prophesy to my people Israel.'

\v{16}``Very well then, hear this message from the \divine{Lord}:

\begin{poetry}
\poeml `You are saying, \\
\poemll    ``Don't prophesy against Israel, \\
\poemlll       and don't preach against the house of Isaac.'' \\
\poeml \v{17}`Therefore this is what the \divine{Lord} says: \\
\poemll    ``Your wife will become a whore in the city, \\
\poemlll       and your sons and daughters will die by the sword. \\
\poemll    Your land will be divided and apportioned, \\
\poemlll       and you will die in a foreign\fnote{Lit. \fbib{in an unclean}} land. \\
\poemll    Israel will surely go into exile, \\
\poemlll       far from its homeland.''\,'\,''
\end{poetry}
\labelchapt{8}
\passage{The Vision of a Fruit Basket}

\chapt{8}
\v{1}This is what the Lord \divine{God} showed me: Look! A basket of summer fruit! \v{2}And he was asking, ``What do you see, Amos?''

I answered, ``A basket of summer fruit.''

Then the \divine{Lord} told me,

\begin{poetry}
\poeml ``The end\fnote{The Heb. \fbib{end} sounds like Heb. word \fbib{summer fruit}} approaches for my people Israel. \\
\poemll    I will no longer spare them.\fnote{Lit. \fbib{him}} \\
\poeml \v{3}At that time,'' \\
\poemll    declares the Lord \divine{God}, \\
\poeml ``the temple songs will be wailing. \\
\poemll    Many bodies will accumulate everywhere. \\
\poeml \v{4}``Hear this, you who are swallowing up the needy, \\
\poemll    who intend to make the poor of the land fail, \\
\poeml \v{5}and who are saying, \\
\poemll    `When will the New Moon fade \\
\poemlll       so we may sell grain, \\
\poemll    and the Sabbath conclude\fnote{The Heb. lacks \fbib{conclude}} \\
\poemlll       so we may market winnowed wheat?--- \\
\poeml shortchanging the measure,\fnote{The Heb. \fbib{ephah}} \\
\poemll    raising the price, \\
\poemlll       falsifying the scales by treachery, \\
\poeml \v{6}buying the poor for cash,\fnote{Lit. \fbib{silver}} \\
\poemll    and the needy for a pair of sandals, \\
\poemlll       selling chaff mixed in with the wheat.' \\
\poeml \v{7}``The \divine{Lord} has sworn by the pride of Jacob: \\
\poemll    I will never forget anything they have done. \\
\poeml \v{8}Surely the land will tremble because of this, won't it? \\
\poemll    And all who live in it will mourn, won't they? \\
\poeml The entire land will swell up like a flooded\fnote{The Heb. lacks \fbib{flooded}} river. \\
\poemll    It will be stirred up and then will sink \\
\poemlll       like the river of Egypt. \\
\poeml \v{9}It will come about at that time,'' declares the Lord \divine{God}, \\
\poemll    ``I will cause the sun to set at noon \\
\poemlll       and the earth to darken in the daylight. \\
\poeml \v{10}I will turn your festivals into mourning, \\
\poemll    and all of your songs to dirges. \\
\poeml I will cause all of you to put on sackcloth \\
\poemll    and to shave all of your heads. \\
\poeml I will make that time like mourning for an only son, \\
\poemll    and its conclusion will be like the end of\fnote{The Heb. lacks \fbib{the end of}} a bitter day.''
\passage{A Famine of the Word of God}
\poeml \v{11}``Look! The days are coming,'' \\
\poemll    declares the Lord \divine{God}, \\
\poeml ``when I will send a famine throughout the land--- \\
\poemll    not a famine of food or a thirst for water--- \\
\poemlll       but rather a famine of hearing the words of the \divine{Lord}. \\
\poeml \v{12}People\fnote{Lit. \fbib{They}} will stagger from sea to sea, \\
\poemll    from north to east. \\
\poeml They will run back and forth, \\
\poemll    searching for a message from the \divine{Lord}, \\
\poemlll       but they won't find it. \\
\poeml \v{13}At that time, \\
\poemll    the beautiful virgins will faint, \\
\poemlll       as will the strong young men---from thirst. \\
\poeml \v{14}Those who have been swearing oaths by the sin of Samaria, \\
\poemll    or who say, `As your god lives, Dan{\ldots}' \\
\poeml or who say, `As the way of Beer-sheba lives{\ldots}'--- \\
\poemll    will fall, and will never rise again.''
\end{poetry}
\labelchapt{9}
\passage{Israel to be Destroyed}

\chapt{9}
\v{1}I saw the Lord standing beside the altar as he was saying,

\begin{poetry}
\poeml ``Strike the doorposts \\
\poemll    so that the thresholds tremble, \\
\poemlll       bringing them down on the heads of all of them. \\
\poeml Those who survive I will kill with the sword. \\
\poemll    Those who flee will not escape. \\
\poemlll       There will be no deliverance for the fugitives. \\
\poeml \v{2}``Even if they burrow into Sheol,\fnote{I.e. the realm of the dead} \\
\poemll    from there my hand will find them. \\
\poeml Even if they ascend to the heavens, \\
\poemll    from there I will bring them down. \\
\poeml \v{3}Even if they hide at the top of Mount\fnote{The Heb. lacks \fbib{Mount}} Carmel, \\
\poemll    from there I will search and seize them. \\
\poeml Even if they hide from my sight in the depths of the sea, \\
\poemll    from there I will order the serpent to strike them. \\
\poeml \v{4}Even if they go into exile among their enemies, \\
\poemll    from there I will order the sword to kill them. \\
\poeml I will fix my gaze on them to inflict disaster, \\
\poemll    and not to do good.\fnote{Lit. \fbib{them for evil and not for good.}} \\
\poeml \v{5}``The Lord \divine{God} of the Heavenly Armies \\
\poeml who is touching the earth so that it melts \\
\poemll    and all of its inhabitants mourn there--- \\
\poemlll       the land rises like the Nile\fnote{The Heb. lacks \fbib{Nile}} River, \\
\poemlll       but sinks like the river of Egypt--- \\
\poeml \v{6}who is building his stairway to heaven \\
\poemll    and setting its foundation on earth; \\
\poeml who is calling for the waters of the sea \\
\poemll    and pouring them out over the surface of the land--- \\
\poemlll       the \divine{Lord} is his name! \\
\poeml \v{7}``Aren't you people of Israel like the people of Cush to me?'' \\
\poemll    declares the \divine{Lord}. \\
\poeml ``I brought Israel up from the land of Egypt, did I not, \\
\poemll    as well as the Philistines from Caphtor\fnote{I.e. possibly Crete} \\
\poemlll       and the Arameans from Kir? \\
\poeml \v{8}Look! The eyes of the Lord \divine{God} are on the sinful kingdom. \\
\poemll    I will destroy it from the face of the earth; \\
\poeml but I will not totally destroy the house of Jacob,'' \\
\poemll    declares the \divine{Lord}. \\
\poeml \v{9}``Look! I'm giving the order: \\
\poemll    I will sift the house of Israel throughout all the nations, \\
\poemlll       as one sifts with a sieve, \\
\poemll    yet not a single kernel will reach the ground! \\
\poeml \v{10}All sinners among my people will die by the sword, \\
\poemll    especially all who are saying, \\
\poemlll       `Disaster will not come upon or conquer us!'\,''
\passage{Israel to be Restored}
\poeml \v{11}``At that time I will restore David's fallen tent, \\
\poemll    restoring its torn places. \\
\poeml I will restore its ruins, \\
\poemll    rebuilding it as it was long ago, \\
\poeml \v{12}so my people\fnote{Lit. \fbib{so they}} may inherit the remnant of Edom \\
\poemll    and all of the nations that bear my name,'' \\
\poemlll       declares the \divine{Lord} who is bringing this about. \\
\poeml \v{13}``Look! The days are coming,'' \\
\poemll    declares the \divine{Lord}, \\
\poeml ``when the one who sows will overtake the harvester \\
\poemll    and the treader of grapes will overtake\fnote{The Heb. lacks \fbib{will overtake}} the planter. \\
\poeml Fresh wine will drip down from the mountains, \\
\poemll    cascading down from the hills. \\
\poeml \v{14}I will surely restore my people Israel; \\
\poemll    they will rebuild the ruined cities \\
\poemlll       and inhabit them.\fnote{The Heb. lacks \fbib{them}} \\
\poeml They will plant vineyards \\
\poemll    and drink the wine from them. \\
\poeml They will plant gardens \\
\poemll    and eat the fruit from them. \\
\poeml \v{15}I will plant the people of Israel\fnote{Lit. \fbib{plant them}} in their own land, \\
\poemll    never again to be torn out of their land \\
\poemlll       that I gave them,'' \\
\poemll    says the \divine{Lord} your God.\end{poetry}

\bookheader{Obadiah}
\labelbook{Obad}

\bookpretitle{The Book of the Prophet}
\booktitle{Obadiah}

\passage{Coming Judgment against Edom}

\begin{poetry}
\poeml \v{1}Obadiah's\fnote{\fbackref{1} The Heb. name \fbib{Obadiah} means \fbib{Servant of the \divine{Lord}}} vision: \\
\poeml This is what the Lord \divine{God} has to say about Edom. \\
\poeml We have heard a report from the \divine{Lord}, \\
\poemll    and a messenger has been dispatched among the nations to say\fnote{\fbackref{1} The Heb. lacks \fbib{to say}} \\
\poeml ``Get up! Let us rise up against her to fight!''
\passage{God's Announcement to Edom}
\poeml \v{2}``Look! I will make you insignificant among the nations; \\
\poemll    you will be utterly despised. \\
\poeml \v{3}The arrogance in your heart has deceived you, \\
\poemll    who inhabit hidden places on rocky cliffs, \\
\poemll    whose dwelling is in the heights, \\
\poemll    who say continuously to yourself,\fnote{\fbackref{3} Lit. \fbib{continually in your heart}} \\
\poemlll       `Who will bring me down to the ground?' \\
\poeml \v{4}Though you soar high like the eagle \\
\poemll    and make your nest among the stars, \\
\poeml I will bring you down even from there,'' \\
\poemlll       declares the \divine{Lord}.\fnote{\fbackref{4} Cf. Jer 49:14-16}
\passage{The Harvest from Edom's Arrogance}
\poeml \v{5}``If thieves came against you, \\
\poemll    if marauding gangs by night \\
\poemlll       ---Oh, how you will be destroyed!--- \\
\poeml Would they not steal only until they had enough? \\
\poeml If grape pickers came to you, \\
\poemll    would they not leave some\fnote{\fbackref{5} The Heb. lacks \fbib{some}} grapes to be gleaned? \\
\poeml \v{6}``Oh, how Esau is ransacked, \\
\poemll    how his hidden treasures are thoroughly\fnote{\fbackref{6} The Heb. lacks \fbib{thoroughly}} searched out! \\
\poeml \v{7}All your allies will force you out of the land,\fnote{\fbackref{7} Lit. \fbib{out to the border}} \\
\poemll    your associates will deceive you and prevail against you. \\
\poeml Your friends\fnote{\fbackref{7} Lit. \fbib{Your bread}; i.e. those who eat your bread} will lay out a trap for you, \\
\poemll    and you will\fnote{\fbackref{7} The Heb. lacks \fbib{you will}} never understand it! \\
\poeml \v{8}``In that day,'' declares the \divine{Lord}, \\
\poemll    ``will I not destroy the wise from Edom, \\
\poemll    and those with understanding from Esau's Mountain? \\
\poeml \v{9}Teman, our mighty soldiers will be dismayed, \\
\poemll    so that every man from Esau's Mountain will be slaughtered.''\fnote{\fbackref{9} Lit. \fbib{will be cut off by slaughter}}
\passage{Judgment for Edom's Cruelty to Jacob}
\poeml \v{10}``Shame will overwhelm you \\
\poemll    because of the violence you inflicted on your brother Jacob, \\
\poeml and you will be excluded\fnote{\fbackref{10} Lit. \fbib{cut off}} forever. \\
\poeml \v{11}``On the day you just stood by,\fnote{\fbackref{11} Or \fbib{stood in opposition}} \\
\poeml when\fnote{\fbackref{11} Lit. \fbib{in the day}} strangers carried away Jacob's\fnote{\fbackref{11} Lit. \fbib{his}} wealth \\
\poeml and foreigners entered his gates, \\
\poemll    casting lots for Jerusalem, \\
\poeml you were just like one of them. \\
\poeml \v{12}``You should not have gloated over your brother,\fnote{\fbackref{12} Lit. \fbib{in the day of your brother}} \\
\poemll    on the day of his calamity. \\
\poeml You should not have rejoiced \\
\poemll    when\fnote{\fbackref{12} Lit. \fbib{in the day}} the descendants of Judah were perishing. \\
\poeml You should not have boasted\fnote{\fbackref{12} Lit. \fbib{have let your mouth boast}} \\
\poemll    when\fnote{\fbackref{12} Lit. \fbib{in the day}} they were in distress. \\
\poeml \v{13}``You should not have entered the gate of my people \\
\poemll    on the day of their disaster.\fnote{\fbackref{13} The Heb. words \fbib{their disaster} may be a word play on the Heb. word \fbib{Edom}} \\
\poeml Also, you should not have gloated over Judah's\fnote{\fbackref{13} Lit. \fbib{his}} misfortune \\
\poemll    on the day of his disaster,\fnote{\fbackref{13} The Heb. words \fbib{his disaster} may be a word play on the Heb. word \fbib{Edom}} \\
\poeml nor should you have plundered his wealth \\
\poemll    on the day of his disaster.\fnote{\fbackref{13} The Heb. words \fbib{his disaster} may be a word play on the Heb. word \fbib{Edom}} \\
\poeml \v{14}And you should not have taken your stand at the crossroads \\
\poemll    to cut down his fleeing refugees, \\
\poeml nor should you have handed over his survivors \\
\poemll    on the day of his distress.''
\passage{The \divine{Lord}'s Judgment and Israel's Final Victory}
\poeml \v{15}``Indeed, the Day of the \divine{Lord} approaches all nations. \\
\poemll    As you have done it will be done to you--- \\
\poemlll       your deeds will return to haunt you!\fnote{\fbackref{15} Lit. \fbib{return on your own head}} \\
\poeml \v{16}Just as you have drunk from the cup of my wrath\fnote{\fbackref{16} The Heb. lacks \fbib{from the cup of my wrath}} upon my holy mountain, \\
\poemll    so will all nations drink from the cup of my wrath\fnote{\fbackref{16} The Heb. lacks \fbib{from the cup of my wrath}} perpetually. \\
\poeml They will drink, they will gulp it down, \\
\poemll    and they will be as if they had never existed! \\
\poeml \v{17}``But there will be a delivered remnant on Mount Zion. \\
\poemll    There will be holiness, \\
\poemlll       and the house of Jacob will take back their possessions. \\
\poeml \v{18}``The house of Jacob will be a fire, \\
\poeml and the house of Joseph a flame, \\
\poemll    but the house of Esau will be kindling. \\
\poeml Then Jacob and Joseph\fnote{\fbackref{18} Lit. \fbib{They}} will burn and consume Esau,\fnote{\fbackref{18} Lit. \fbib{them}} \\
\poemll    and no survivor will remain from the house of Esau.'' \\
\poeml Indeed, the \divine{Lord} has spoken it. \\
\poeml \v{19}``Those in the Negev\fnote{\fbackref{19} I.e. the southern regions of the Sinai peninsula; cf. Josh 10:40} will possess Esau's Mountain, \\
\poemll    and those in the Shephelah\fnote{\fbackref{19} I.e. the verdant central lowlands of Israel; cf. Josh 10:40} the Philistines. \\
\poeml They will possess the fields of Ephraim \\
\poemll    and the fields of Samaria, \\
\poeml while Benjamin will possess the territory of\fnote{\fbackref{19} The Heb. lacks \fbib{will possess the territory of}} Gilead. \\
\poeml \v{20}The exiles, the Israeli host, \\
\poemll    will possess the territory of the\fnote{\fbackref{20} The Heb. lacks \fbib{the territory of}} Canaanites all the way to Zarephath. \\
\poeml The exiles of Jerusalem who are in Sepharad\fnote{\fbackref{20} I.e. perhaps Sardis, capital of Lydia, Saparda in eastern Assyria, Sparta in Greece, or a location in Spain (so \fbib{Targ of Jonathan})} \\
\poemll    will possess the cities of the Negev.\fnote{\fbackref{20} I.e. the southern regions of the Sinai peninsula; cf. Josh 10:40} \\
\poeml \v{21}Deliverers will assemble on Mount Zion to judge Esau's Mountain, \\
\poemll    and to the \divine{Lord} will the kingdom belong!''\end{poetry}

\bookheader{Jonah}
\labelbook{Jonah}

\bookpretitle{The Book of the Prophet}
\booktitle{Jonah}

\labelchapt{1}
\passage{Jonah is Called to Go to Nineveh}

\chapt{1}
\v{1}Now this message from the \divine{Lord} came to Amittai's son Jonah:\fnote{\fbackref{1:1} The Heb. name \fbib{Jonah} means \fbib{dove}} \v{2}``Get up and go to Nineveh, that great city! Then cry out in protest\fnote{\fbackref{1:2} The Heb. lacks \fbib{in protest}} against it, because their evil has come to my attention.''\fnote{\fbackref{1:2} Lit. \fbib{has come up before me}}
\passage{Jonah Runs from God's Call}

\v{3}But Jonah got up and fled from the \divine{Lord} to Tarshish.\fnote{\fbackref{1:3} I.e. a city far to the West} He went down to Joppa, secured passage on a ship bound for Tarshish, paid the fare, and boarded, intending to go with the mariners\fnote{\fbackref{1:3} Lit. \fbib{with them}} to Tarshish to escape from the \divine{Lord}. \v{4}Then the \divine{Lord} sent\fnote{\fbackref{1:4} Lit. \fbib{threw}} a great wind over the sea, and a severe storm broke out. It seemed as if the ship were\fnote{\fbackref{1:4} Or \fbib{out so that the ship seemed that it was}} about to break up. \v{5}At this point the mariners became terrified, and each man cried out to his gods. They began to throw the cargo into the sea in order to lighten the vessel. But Jonah had gone down into the vessel's hold, had lain down, and was fast asleep. \v{6}So the captain approached him, and told him, ``What are you doing asleep? Get up! Call on your gods! Maybe your\fnote{\fbackref{1:6} The Heb. lacks \fbib{your}} god will think about us so we won't die!''

\v{7}Meanwhile, each crewman told another, ``Come on! Let's cast lots to find out whose fault it is that we're in this trouble.'' So they cast lots, and the lot indicated Jonah! \v{8}So they interrogated him: ``Tell us, why has this trouble come upon us? What's your occupation? Where'd you come from? What's your home country? What's your nationality?''

\v{9}``I'm a Hebrew,'' he replied, ``and I'm afraid of the \divine{Lord} God of heaven, who made the sea---along with the dry land!''

\v{10}In mounting terror, the men asked him, ``What have you done?'' The men were aware that he was fleeing from the \divine{Lord}, because he had admitted this to them.
\passage{Jonah is Thrown Overboard}

\v{11}Because the sea was growing more and more stormy, they asked him, ``What do we have to do to you so the sea will calm down for us?''

\v{12}Jonah\fnote{\fbackref{1:12} Lit. \fbib{He}} told them, ``Pick me up and toss me into the sea. Then the sea will calm down for you, because I know that it's my fault that this mighty storm has come\fnote{\fbackref{1:12} The Heb. lacks \fbib{has come}} upon you.'' \v{13}Even so, the crewmen rowed hard to bring the ship toward dry land, but they were unsuccessful, because the sea was growing more and more stormy.

\v{14}At last they cried out to the \divine{Lord}, ``Please, \divine{Lord}, do not let us perish because of this man's life, and do not hold us responsible for innocent blood, because you, \divine{Lord}, have done what pleased you.'' \v{15}So they picked up Jonah and tossed him into the sea, and the sea stopped raging. \v{16}Then the men feared the \divine{Lord} greatly, offered a sacrifice to the \divine{Lord}, and made vows.

\v{17}\fnote{\fbackref{1:17} This vs. is 2:1 in MT}Now the \divine{Lord} had prepared a large sea creature\fnote{\fbackref{1:17} Lit. \fbib{fish}, and so throughout the book} to swallow Jonah, and Jonah was inside the sea creature for three days and three nights.
\labelchapt{2}
\passage{Jonah's Prayer for Deliverance}

\chapt{2}
\v{1}\fnote{\fbackref{2:1} 2:1 is 2:2 in MT, 2:2 is 2:3 in MT, and so through vs. 10}Then Jonah prayed to the \divine{Lord} his God from inside the sea creature. \v{2}He said:

\begin{poetry}
\poeml ``I called out to the \divine{Lord} from the midst of affliction directed at me,\fnote{\fbackref{2:2} Lit. \fbib{affliction to me}} \\
\poemll    and he answered me. \\
\poeml From the depths\fnote{\fbackref{2:2} Lit. \fbib{belly}} of death\fnote{\fbackref{2:2} Heb. \fbib{Sheol}; i.e. the realm of the dead} I cried out for help; \\
\poemll    and you heard my cry.\fnote{\fbackref{2:2} Or \fbib{voice}} \\
\poeml \v{3}You cast me into the deep--- \\
\poemll    into the heart of the sea. \\
\poeml Flood waters engulfed me. \\
\poemll    All your breakers and your waves swirled over me. \\
\poeml \v{4}So I told myself,\fnote{\fbackref{2:4} Or \fbib{I thought}} `I have been driven away from you.\fnote{\fbackref{2:4} Lit. \fbib{from your attention}} \\
\poemll    How\fnote{\fbackref{2:4} Lit. \fbib{Indeed, surely}} will I again gaze on your holy Temple?' \\
\poeml \v{5}Flood waters encompassed me, \\
\poemll    the deep surrounded me \\
\poemlll       while seaweed wrapped around my head. \\
\poeml \v{6}I sank to the roots of the mountains; \\
\poemll    the earth's prison\fnote{\fbackref{2:6} The Heb. lacks \fbib{prison}} bars closed\fnote{\fbackref{2:6} The Heb. lacks \fbib{closed}} around me forever. \\
\poemlll       Yet you resurrect the dead\fnote{\fbackref{2:6} Lit. \fbib{you bring life up}} from the Pit,\fnote{\fbackref{2:6} I.e. the place of punishment in the afterlife} \divine{Lord} my God! \\
\poeml \v{7}``As my life was fading away, \\
\poemll    I remembered the \divine{Lord}; \\
\poemlll       and my prayer came to you in your holy Temple. \\
\poeml \v{8}Those who cling to vain idols \\
\poemll    leave behind the gracious love that could have been theirs.\fnote{\fbackref{2:8} Or \fbib{leave behind their gracious love}} \\
\poeml \v{9}But as for me, with a voice of thanksgiving I will sacrifice to you; \\
\poemll    what I have vowed I will pay. \\
\poeml Deliverance\fnote{\fbackref{2:9} Or \fbib{Salvation}} is the \divine{Lord}'s!''
\end{poetry}

\v{10}Then the \divine{Lord} spoke to the sea creature, and it spewed Jonah onto the dry land.
\labelchapt{3}
\passage{The \divine{Lord} Again Calls Jonah to Go to Nineveh}

\chapt{3}
\v{1}This message from the \divine{Lord} came to Jonah a second time: \v{2}``Get up and go to Nineveh, that great city, and proclaim to it the message that I tell you.'' \v{3}So Jonah got up and went to Nineveh to do what the \divine{Lord} had ordered.

Now Nineveh was a very large city,\fnote{\fbackref{3:3} Lit. \fbib{a great city of God}; i.e. a city of enormous size} requiring\fnote{\fbackref{3:3} The Heb. lacks \fbib{requiring}} a three-day journey to cross through it.\fnote{\fbackref{3:3} The Heb. lacks \fbib{to cross through it}} \v{4}As Jonah started into the city on the first day's journey, he proclaimed the message, ``40 days more and Nineveh will be overthrown!''
\passage{The City of Nineveh Repents}

\v{5}The people of Nineveh believed God. They called for a fast and put on sackcloth, from the greatest of them to the least important. \v{6}When the message reached the king of Nineveh, he got up from his throne, removed his royal garments, covered himself with sackcloth, and sat down in ashes. \v{7}Then he had this proclamation published throughout Nineveh:

\begin{poetry}
\poeml ``By decree of the king and his nobles: \\
\poeml No man or animal, herd or flock, is to taste anything, graze, or drink water. \v{8}Instead, let both man and animal clothe themselves with sackcloth and cry out to God forcefully. Let every person turn from his evil ways and from his tendency to do violence.\fnote{\fbackref{3:8} Lit. \fbib{from the violence that is in their palms}} \v{9}Who knows but that God may relent, have compassion, and turn from his fierce anger, so that we are not exterminated?''
\end{poetry}

\v{10}God took note of what they did---that they turned from their evil ways. Because God relented concerning the trouble about which he had warned them, he did not carry it out.
\labelchapt{4}
\passage{Jonah's Anger at God's Kindness}

\chapt{4}
\v{1}Greatly displeased, Jonah flew into a rage. \v{2}So he prayed to the \divine{Lord}, ``\divine{Lord}, isn't this what I said while I was still in my home country? That's why I fled previously to Tarshish, because I knew you're a compassionate God, slow to anger, overflowing with gracious love, and reluctant\fnote{\fbackref{4:2} Or \fbib{sorrowful}} to send trouble. \v{3}Therefore, \divine{Lord}, please kill me, because it's better for me to die than to live!''

\v{4}The \divine{Lord} replied, ``Does being angry make you right?''
\passage{Jonah's Discouragement}

\v{5}Then Jonah left the city and sat down on the eastern side.\fnote{\fbackref{4:5} Lit. \fbib{down east of the city}} There he made a shelter for himself and sat down under its shade to see what would happen to the city. \v{6}The \divine{Lord} God prepared a vine plant,\fnote{\fbackref{4:6} Or \fbib{castor bean plant}; or \fbib{gourd}; and so throughout the chapter} and it grew over Jonah to shade his head and provide relief from his misery. Jonah was happy---indeed, he was ecstatic---about the vine plant. \v{7}But at dawn the next day, God provided a worm that attacked the vine plant so that it withered away. \v{8}When the sun rose, God prepared a harsh east wind. The sun beat down on Jonah's head, he became faint, and he begged to die. ``It is better for me to die than to live!'' he said.

\v{9}Then God asked Jonah, ``Is your anger about the vine plant justified?''

And he answered, ``Absolutely! I'm so angry I could die!''

\v{10}But the \divine{Lord} asked, ``You cared about a vine plant that you neither worked on nor cultivated? A vine plant that grew up overnight and died overnight? \v{11}So why shouldn't I be concerned about Nineveh, that great city, in which there are more than 120,000 human beings who do not know their right hand from their left,\fnote{\fbackref{4:11} I.e. young children or infants} as well as a lot of livestock?

\bookheader{Micah}
\labelbook{Mic}

\bookpretitle{The Book of the Prophet}
\booktitle{Micah}

\labelchapt{1}
\passage{God's Coming Judgment}

\chapt{1}
\v{1}This message from the \divine{Lord} came to Micah\fnote{\fbackref{1:1} The Heb. name \fbib{Micah} means \fbib{Who is like the \divine{Lord}?}} of Moresheth during the reigns of\fnote{\fbackref{1:1} Lit. \fbib{days}} the Judean kings Jotham, Ahaz, and Hezekiah concerning the vision he saw about Samaria and Jerusalem:

\begin{poetry}
\poeml \v{2}``Listen, people! All of you! \\
\poemll    Earth! Pay attention, and all you inhabitants of it! \\
\poeml May the Lord \divine{God} be a witness against you--- \\
\poemll    the Lord from his holy Temple. \\
\poeml \v{3}Look here! The \divine{Lord} is coming from his place! \\
\poemll    He will come down \\
\poemlll       and will trample down the high places\fnote{\fbackref{1:3} I.e. the sites of idol worship} throughout the land. \\
\poeml \v{4}The mountains will melt under him \\
\poemll    and the valleys will split apart, \\
\poeml like wax in the presence of fire \\
\poemll    and like water gushing down a steep incline. \\
\poeml \v{5}All this comes about due to the transgression of Jacob, \\
\poemll    and due to the sins of the house of Israel. \\
\poeml What is Jacob's sin? It's Samaria, isn't it? \\
\poemll    And what's Judah's high place?\fnote{\fbackref{1:5} I.e. the sites of idol worship} It's Jerusalem, isn't it? \\
\poeml \v{6}``So I will turn Samaria into a mound of dirt in a field, \\
\poemll    a place to plant vineyards. \\
\poeml And I will dump her building stones into the valley, \\
\poemll    uncovering her foundation. \\
\poeml \v{7}All of her carved images will be crushed to pieces, \\
\poemll    all the earnings of her prostitution will be burned up, \\
\poemlll       and I will destroy all her idols; \\
\poeml because she collected the wages of prostitution, \\
\poemll    and to the wages of prostitution they will return.''
\passage{The Coming Destruction}
\poeml \v{8}``Therefore I will cry out and grieve loudly; \\
\poemll    I will walk around stripped and naked. \\
\poeml I will cry out like a jackal \\
\poemll    and mourn like a company of ostriches. \\
\poeml \v{9}For Samaria's\fnote{\fbackref{1:9} Lit. \fbib{her}} injury is fatal, \\
\poemll    reaching all the way to Judah, \\
\poemlll       extending even to the gate of my people---to Jerusalem.'' \\
\poeml \v{10}``Don't discuss it in Gath!\fnote{\fbackref{1:10} A city in Philistia} \\
\poemll    Don't cry bitterly in Akim!\fnote{\fbackref{1:10} So LXX; MT reads \fbib{Don't cry at all}} \\
\poemlll       Roll in the ashes, Beth-leaphrah! \\
\poeml \v{11}Run away, you residents of Shaphir, \\
\poemll    displaying your shameful nakedness. \\
\poeml Don't come out, you residents of Zaanan!\fnote{\fbackref{1:11} A city of Judah} \\
\poemll    Your firm standing will disappear as Beth-ezel mourns. \\
\poeml \v{12}Even though the inhabitants of Maroth long for success, \\
\poemll    nevertheless evil descended from the \divine{Lord} to the gate of Jerusalem. \\
\poeml \v{13}``You inhabitants of Lachish, harness your chariot to your swiftest steed--- \\
\poemll    the daughter of Zion has begun to sin--- \\
\poeml because within you the transgressions of Israel were revealed. \\
\poeml \v{14}Therefore give your gifts to Moresheth-gath; \\
\poemll    that is, the houses of Achzib as a deceitful symbol\fnote{\fbackref{1:14} Or \fbib{a lie}} to the kings of Israel. \\
\poeml \v{15}Nevertheless, I will deliver an heir to you, inhabitants of Mareshah--- \\
\poemll    to Adullam the glory of Israel will come. \\
\poeml \v{16}``Shave your head \\
\poemll    and cut off your locks as you mourn your beloved children. \\
\poeml Make yourself bald as an eagle, \\
\poemll    because they will go from you into exile!''
\end{poetry}
\labelchapt{2}
\passage{God's Warning to His People}

\begin{poetry}
\poeml \chapt{2}
\v{1}``Woe to those who are crafting iniquity, \\
\poeml planning evil well into the night!\fnote{\fbackref{2:1} Lit. \fbib{evil upon their beds}} \\
\poeml When morning's light comes, \\
\poemll    they carry out their plans\fnote{\fbackref{2:1} The Heb. lacks \fbib{their plans}} because they have the power to do so. \\
\poeml \v{2}They covet fields and seize them; \\
\poemll    they covet\fnote{\fbackref{2:2} The Heb. lacks \fbib{they covet}} houses, and grab them, too. \\
\poeml They harass the valiant man, along with his household, \\
\poemll    an individual and his estate. \\
\poeml \v{3}``Therefore this is what the \divine{Lord} says, \\
\poeml `I'm crafting evil against this family, \\
\poemll    from which you can't escape.\fnote{\fbackref{2:3} Lit. \fbib{can't remove your necks}} \\
\poeml You won't strut around arrogantly, \\
\poemll    because the times are evil.' \\
\poeml \v{4}``When this happens,\fnote{\fbackref{2:4} Lit. \fbib{In that day}} someone will compose a proverb about you, lamenting sorrowfully, \\
\poemll    `We are completely ruined! \\
\poemlll       He has given my people's heritage to others.\fnote{\fbackref{2:4} The Heb. lacks \fbib{to others}} \\
\poemll    How he has removed it from me, \\
\poemlll       dividing up our fields!' \\
\poeml \v{5}``This is why there will not be left even a single person \\
\poemll    to settle boundary disputes\fnote{\fbackref{2:5} Lit. \fbib{to stretch out a measuring line}} in the \divine{Lord}'s community. \\
\poeml \v{6}To those who speak out, they will declare, \\
\poemll    `Don't prophesy to anyone!' \\
\poemlll       so their shame won't go away. \\
\poeml \v{7}``It is said, house of Jacob, \\
\poemll    `The Spirit of the \divine{Lord} is limited, \\
\poemlll       if he acts this way, is he not?' \\
\poeml ``But my words benefit those who live righteously, do they not? \\
\poeml \v{8}Lately my people have acted like an enemy--- \\
\poemll    you strip travelers who thought they were\fnote{\fbackref{2:8} The Heb. lacks \fbib{thought they were}} as secure \\
\poemlll       as those who return from war. \\
\poeml \v{9}You have evicted the wives of my people from their dream homes; \\
\poemll    you have removed my majesty from their children permanently. \\
\poeml \v{10}``Get up and go, \\
\poemll    because there's no rest for you here! \\
\poeml Since everything\fnote{\fbackref{2:10} Lit. \fbib{it}} is polluted, it can only cause destruction, \\
\poemll    even heavy destruction. \\
\poeml \v{11}Suppose a man who keeps company with a deceiving spirit prophesies like this: \\
\poemll    `Drink wine and strong drink!' \\
\poemlll       Won't the people accept him as a prophet?''
\passage{The Coming Judgment}
\poeml \v{12}``Jacob, how I will surely gather all of you! \\
\poemll    How I will gather the survivors of Israel! \\
\poeml I will gather them together like sheep in a pen,\fnote{\fbackref{2:12} Or \fbib{sheep of Bozrah}} \\
\poemll    like the flock in the midst of the sheepfold. \\
\poemlll       There will be a great commotion because of all\fnote{\fbackref{2:12} The Heb. lacks \fbib{all}} the people. \\
\poeml \v{13}God will stand up and break through\fnote{\fbackref{2:13} Lit. \fbib{The one who breaks through will arise}} in their presence. \\
\poemll    Then they will pass through the gate, going out by it. \\
\poeml Their king will pass in front of them \\
\poemll    with the \divine{Lord} at their head.''
\labelchapt{3}
\poeml \chapt{3}
\v{1}``He will say, `Listen, you leaders of Jacob, \\
\poeml you officials of the house of Israel! \\
\poeml You should know justice, should you not?--- \\
\poeml \v{2}you who despise good and love evil, \\
\poemll    who tear off the skin of my people,\fnote{\fbackref{3:2} The Heb. lacks \fbib{of my people}} \\
\poemlll       along with the flesh from their bones. \\
\poeml \v{3}You eat the flesh of my people, \\
\poemll    flaying their skin from them. \\
\poeml You break their bones, \\
\poemll    chopping them in pieces like meat\fnote{\fbackref{3:3} The Heb. lacks \fbib{meat}} for a pot, \\
\poemlll       like meat destined for a soup kettle.' \\
\poeml \v{4}``Then they will cry to the \divine{Lord}, \\
\poemll    but he will not listen to them. \\
\poeml In fact, he will hide his face from them at that time, \\
\poemll    because they were so wicked in what they were doing.''
\passage{God's Judgment against False Prophets}
\poeml \v{5}``This is what the \divine{Lord} says about the prophets \\
\poemll    who are causing my people to go astray, \\
\poeml who are calling out `Peace' when they're being fed,\fnote{\fbackref{3:5} Lit. \fbib{who bite with their teeth, crying out ``Peace''}} \\
\poemll    but who declare war against those who won't feed them:\fnote{\fbackref{3:5} Lit. \fbib{against him who puts nothing in their mouths}} \\
\poeml \v{6}`You will have nights without visions, \\
\poemll    and darkness without prophecy. \\
\poeml The sun will set on the prophets, \\
\poemll    and the day will darken for them. \\
\poeml \v{7}Those who see visions will be put to shame, \\
\poemll    and the diviners will be disgraced---every one of them--- \\
\poeml they will cover their faces,\fnote{\fbackref{3:7} Lit. \fbib{beard}} \\
\poemll    because there will be no answer from God.'\,''
\passage{The Message of God's Prophet}
\poeml \v{8}``As for me, I am truly filled with power by the Spirit of the \divine{Lord}, \\
\poemll    filled\fnote{\fbackref{3:8} The Heb. lacks \fbib{filled}} with judgment and power \\
\poeml to announce to Jacob his transgression, \\
\poemll    and to Israel his sin. \\
\poeml \v{9}Please listen to this, you leaders of the house of Jacob, \\
\poemll    you officials of the house of Israel, \\
\poeml you who hate administering justice, \\
\poemll    who pervert the very meaning of\fnote{\fbackref{3:9} Lit. \fbib{pervert all}} equity, \\
\poeml \v{10}who are building up Zion by means of bloodshed, \\
\poemll    and Jerusalem by means of iniquity. \\
\poeml \v{11}Her leaders judge for the money, \\
\poemll    her priests teach only when they're paid , \\
\poemlll       and her prophets prophesy for cash. \\
\poeml Even so, don't they all rely on the \divine{Lord} as they ask, \\
\poemll    `The \divine{Lord} is among us, is he not? \\
\poemlll       Nothing bad can possibly happen to us!' \\
\poeml \v{12}``Therefore, because of you, Zion will be plowed up like a field, \\
\poemll    and Jerusalem will become heaps of rubble, \\
\poemlll       and the Temple Mount like a forest high place.''
\end{poetry}
\labelchapt{4}
\passage{The Future Reign of God}

\begin{poetry}
\poeml \chapt{4}
\v{1}``But in the last days it will come about \\
\poemll    that the Temple Mount of the \divine{Lord} will be firmly set \\
\poemlll       as the leading mountain. \\
\poeml It will be exalted above its surrounding\fnote{\fbackref{4:1} The Heb. lacks \fbib{surrounding}} hills, \\
\poemll    and people will stream toward it. \\
\poeml \v{2}Many nations will approach and say, \\
\poemll    `Come, let's go up to the mountain of the \divine{Lord}, \\
\poemlll       and to the Temple of the God of Jacob. \\
\poemll    He will teach us about his ways, \\
\poemlll       and we will walk according to his directions.' \\
\poeml ``Indeed, the Law will proceed from Zion, \\
\poemll    and the message of the \divine{Lord} from Jerusalem. \\
\poeml \v{3}And he will judge among many people, \\
\poemll    rebuking strong nations far away; \\
\poeml and they will reshape their swords as plowshares \\
\poemll    and their spears as pruning hooks. \\
\poeml No nation will threaten another,\fnote{\fbackref{4:3} Lit. \fbib{A nation will not lift up a sword against a nation}} \\
\poemll    nor will they train for war anymore. \\
\poeml \v{4}Instead, each man will sit in the shade of\fnote{\fbackref{4:4} Lit. \fbib{sit under}} his grape vines \\
\poemll    and beneath the shade of\fnote{\fbackref{4:4} The Heb. lacks \fbib{the shade of}} his fig tree,'' \\
\poemlll       since the\fnote{\fbackref{4:4} Lit. \fbib{the mouth of the}} \divine{Lord} of the Heavenly Armies has spoken. \\
\poeml \v{5}``Because all of the people will walk, \\
\poemll    each person in the name of his God, \\
\poemlll       and we will walk in the name of the \divine{Lord} our God forever and ever. \\
\poeml \v{6}``At that time,'' declares the \divine{Lord}, \\
\poemll    ``I will assemble the lame; \\
\poeml and I will gather those whom I have scattered, \\
\poemll    along with those whom I have afflicted. \\
\poeml \v{7}I will transform the lame into survivors, \\
\poemll    and those who were scattered into a strong nation; \\
\poeml and the \divine{Lord} will reign over them in Mount Zion, \\
\poemll    now and forever.''
\passage{Zion's Captivity}
\poeml \v{8}``And you, watchtower of the flock, \\
\poemll    you stronghold of the daughter of Zion, \\
\poeml it will happen even to you: \\
\poemlll       The former dominion, even the kingdom of the daughter of Jerusalem, will come. \\
\poeml \v{9}Why are you crying so loud now? \\
\poemll    There's no king among you, is there? \\
\poeml Perhaps your advisor has died? \\
\poemll    For pain has overtaken you like a woman in labor. \\
\poeml \v{10}Be in pain! Be in labor, you daughter of Zion, \\
\poemll    like a woman about to give birth, \\
\poeml because now you will depart from the city, \\
\poemll    living in the open fields. \\
\poeml To Babylon you will go. \\
\poemll    There you will be delivered, \\
\poemlll       there the \divine{Lord} will rescue you from the power\fnote{\fbackref{4:10} Lit. \fbib{hand}} of your enemies.''
\passage{The Nations Despise Zion}
\poeml \v{11}``Now many nations have gathered against you, saying, \\
\poemll    `Let her be defiled,' and \\
\poemlll       `Let's look down on Zion.' \\
\poeml \v{12}But they don't know the thoughts of the \divine{Lord}, \\
\poemll    and they don't understand his tactics, \\
\poemlll       for he will gather them like harvested grain to his threshing floor. \\
\poeml \v{13}Get up and smash\fnote{\fbackref{4:13} Lit. \fbib{thresh}} them to pieces, daughter of Zion, \\
\poemll    because I will make your horn like iron \\
\poemlll       and your hooves like bronze! \\
\poeml And you will beat many people to pieces, \\
\poemll    and I will consecrate their dishonest\fnote{\fbackref{4:13} The Heb. lacks \fbib{dishonest}} gain to the \divine{Lord} \\
\poemlll       and their assets to the Lord of the entire earth.''
\labelchapt{5}
\poeml \chapt{5}
\v{1}\fnote{\fbackref{5:1} 5:1 is 4:14 in MT}``Now marshal yourselves as troops.\fnote{\fbackref{5:1} Lit. \fbib{yourselves, you daughter of mobs}} \\
\poeml He has laid siege to us. \\
\poemll    They will strike the judge\fnote{\fbackref{5:1} Or \fbib{ruler}} of Israel on the cheek with a rod.''
\passage{The Ruler from Bethlehem}
\poeml \v{2}\fnote{\fbackref{5:2} 5:2 is 5:1 in MT, 5:3 is 5:2 in MT, and so throughout the chapter}``As for you, Bethlehem of Ephrathah, \\
\poemll    even though you remain least among the clans\fnote{\fbackref{5:2} Or \fbib{thousands}} of Judah, \\
\poeml nevertheless, the one who rules in Israel for me \\
\poemll    will emerge from you. \\
\poeml His existence has been\fnote{\fbackref{5:2} The Heb. lacks \fbib{has been}} from antiquity, \\
\poemll    even from eternity. \\
\poeml \v{3}Therefore that ruler\fnote{\fbackref{5:3} Lit. \fbib{Therefore he}} will abandon them \\
\poemll    until the woman in labor gives birth. \\
\poeml Then the rest of his countrymen will return to the Israelis.'' \\
\poeml \v{4}``Then he will take his stand, \\
\poemll    shepherding by means of the strength of the \divine{Lord}, \\
\poemlll       by the power\fnote{\fbackref{5:4} Or \fbib{majesty}} of the name of the \divine{Lord} his God. \\
\poeml And they will be firmly\fnote{\fbackref{5:4} The Heb. lacks \fbib{firmly}} established; \\
\poemll    indeed, from then on he will become great--- \\
\poemlll       to the ends of the earth. \\
\poeml \v{5}And he will be our peace.''
\passage{God's Judgment on Assyria}
\poeml ``When the Assyrian invades our land, \\
\poemll    trampling through our palaces, \\
\poeml we will raise up seven shepherds against him, \\
\poemll    even eight significant men. \\
\poeml \v{6}The shepherds\fnote{\fbackref{5:6} Lit. \fbib{And they}} will devastate the land of Assyria with the sword, \\
\poemll    along with the entrances to the land of Nimrod. \\
\poeml ``This is how he will vanquish\fnote{\fbackref{5:6} Lit. \fbib{will deliver from}} Assyria \\
\poemll    when he invades our land, \\
\poemlll       trampling within our borders: \\
\poeml \v{7}The survivors of Jacob will live among many nations, \\
\poemll    as dew from the \divine{Lord}, \\
\poemlll       as showers upon the grass. \\
\poeml They will look to no one, \\
\poemll    and will place no hope in human beings. \\
\poeml \v{8}The survivors of Jacob will live among the nations; \\
\poemll    they will live among many nations, \\
\poemlll       like a lion among flocks of sheep, \\
\poeml who, if he passes through, \\
\poemll    will trample and tear down \\
\poemlll       with no one to deliver. \\
\poeml \v{9}You will turn your power\fnote{\fbackref{5:9} Lit. \fbib{hand}} against your adversaries, \\
\poemll    and all of your enemies will be cut down.''
\passage{God Removes Idolatry}
\poeml \v{10}``It will come about at that time,'' declares the \divine{Lord}, \\
\poemll    ``I will tear away your horses from you, \\
\poemlll       and I will destroy your chariots. \\
\poeml \v{11}I will cut off the cities of your land, \\
\poemll    and I will tear down all of your fortifications. \\
\poeml \v{12}I will render your witchcraft powerless,\fnote{\fbackref{5:12} Lit. \fbib{will cut witchcraft out of your hands}} \\
\poemll    and mediums will no longer exist among you. \\
\poeml \v{13}I will separate you from your carved images and sacred pillars, \\
\poemll    and you no longer will worship \\
\poemlll       what you've made with your hands. \\
\poeml \v{14}I will uproot your cultic gods\fnote{\fbackref{5:14} Heb. \fbib{Asherim}} from you, \\
\poemll    and I will tear down your cities. \\
\poeml \v{15}I will execute vengeance, anger, and fury \\
\poemll    on the nations who do not obey.''
\end{poetry}
\labelchapt{6}
\passage{The \divine{Lord}'s Indictment against Israel}

\chapt{6}
\v{1}Please hear what the \divine{Lord} says:

\begin{poetry}
\poemll    ``Get up and make your case before the mountains, \\
\poemlll       and let the hills listen to your voice. \\
\poeml \v{2}Listen, you mountains, to the \divine{Lord}'s argument! \\
\poemll    Listen, you\fnote{\fbackref{6:2} The Heb. lacks \fbib{Listen, you}} strong foundations of the earth, \\
\poeml because the \divine{Lord} has a dispute with his people, \\
\poemll    and he will set out his case before Israel. \\
\poeml \v{3}``My people, what have I done to you, \\
\poemll    and how have I offended you? \\
\poemlll       Answer me! \\
\poeml \v{4}For I brought you up from the land of Egypt, \\
\poemll    and ransomed you from the house of slavery, \\
\poemlll       sending Moses, Aaron, and Miriam into your presence. \\
\poeml \v{5}``My people, recall how king Balak of Moab deliberated, \\
\poemll    and how Beor's son Balaam counseled him from Shittim to Gilgal, \\
\poemlll       so that you may know the righteousness of the \divine{Lord}.''
\passage{The Nature of True Righteousness}
\poeml \v{6}How am I to present myself in the \divine{Lord}'s presence \\
\poemll    and bow in the presence of the High God? \\
\poeml Should I present myself with burnt offerings, \\
\poemll    with year-old calves? \\
\poeml \v{7}Will the \divine{Lord} be pleased with thousands of rams, \\
\poemll    or with endless\fnote{\fbackref{6:7} Lit. \fbib{with ten thousand}} rivers of oil? \\
\poeml Am I to give my firstborn to pay for\fnote{\fbackref{6:7} The Heb. lacks \fbib{to pay for}} my rebellion, \\
\poemll    the fruit of my body in exchange for\fnote{\fbackref{6:7} The Heb. lacks \fbib{in exchange for}} my soul? \\
\poeml \v{8}He has made it clear to you, mortal man, what is good \\
\poemll    and what the \divine{Lord} is requiring from you--- \\
\poeml to act with justice, \\
\poemll    to treasure the \divine{Lord}'s\fnote{\fbackref{6:8} The Heb. lacks \fbib{the \divine{Lord}'s}} gracious love, \\
\poemlll       and to walk humbly in the company of your God.
\passage{A Call to Honest Business Practices}
\poeml \v{9}The voice of the \divine{Lord} cries out to the city--- \\
\poemll    wisdom fears your name: \\
\poemlll       ``Heed the rod, and the one who prepared it! \\
\poeml \v{10}Are there still wicked treasures in the house of the wicked, \\
\poemll    along with deceitful and abominable measuring standards?\fnote{\fbackref{6:10} Lit. \fbib{with short, detestable ephahs}} \\
\poeml \v{11}Will I tolerate those who maintain deceptive standards\fnote{\fbackref{6:11} Lit. \fbib{maintain evil balances}} \\
\poemll    and who use deceitful weights in their business?\fnote{\fbackref{6:11} The Heb. lacks \fbib{in their business}} \\
\poeml \v{12}Her rich people are filled with violence, \\
\poemll    and her inhabitants tell lies--- \\
\poemlll       their tongues speak deceitfully! \\
\poeml \v{13}``Therefore I will make you ill when I attack you; \\
\poemll    I will bring you to ruin because of your offenses. \\
\poeml \v{14}You'll eat, \\
\poemll    but you won't have enough; \\
\poemlll       and hunger will be common among you. \\
\poeml You'll horde things, \\
\poemll    but you won't save them, \\
\poeml and what you preserve \\
\poemll    I'll give over to the sword. \\
\poeml \v{15}You'll plant, \\
\poemll    but you won't reap. \\
\poeml You'll crush the olive harvest, \\
\poemll    but you'll have no oil to anoint yourself. \\
\poeml You'll tread out the grapes, \\
\poemll    but you'll never drink wine. \\
\poeml \v{16}You keep Omri's\fnote{\fbackref{6:16} I.e. King Omri of Israel, father of Ahab (cf. 2Kings 8:26; 2Chr 22:2)} statutes \\
\poemll    and observe the customs of the house of Ahab. \\
\poeml Because you live according to their advice, \\
\poemll    I'll make you desolate \\
\poemlll       and turn your\fnote{\fbackref{6:16} Lit. \fbib{its}} inhabitants into an object of scorn. \\
\poeml Therefore you will bear the shame of my people.''
\end{poetry}
\labelchapt{7}
\passage{The Evil Behavior of the People}

\begin{poetry}
\poeml \chapt{7}
\v{1}Poor me! \\
\poeml I feel like those who harvest summer fruit, \\
\poemlll       or like those who pick grapes--- \\
\poeml there are no clusters to eat \\
\poemll    or any fresh fruit that I want. \\
\poeml \v{2}The faithful have died off, \\
\poemll    and there is not one upright human being in the land. \\
\poeml They all stalk one another\fnote{\fbackref{7:2} The Heb. lacks \fbib{one another}} with lethal intent, \\
\poemll    a man will even hunt his own brother with a net. \\
\poeml \v{3}And speaking of evil, \\
\poemll    they practice it eagerly---with both hands! \\
\poeml Both leader and judge demand a bribe, \\
\poemll    the famous confess their perverted desires, \\
\poemlll       and they scheme together. \\
\poeml \v{4}The best of them is like a thorn, \\
\poemll    and their most upright like a hedge of thorns. \\
\poeml The day announced by\fnote{\fbackref{7:4} Lit. \fbib{day of}} your watchmen--- \\
\poemll    and by your own calculations---approaches. \\
\poemlll       Now it's your time to be\fnote{\fbackref{7:4} The Heb. lacks \fbib{time to be}} confused! \\
\poeml \v{5}Don't trust your friends, \\
\poemll    don't confide in a companion, \\
\poemlll       watch what you say to your wife.\fnote{\fbackref{7:5} Lit. \fbib{her who lies near your bosom}} \\
\poeml \v{6}The son disrespects his father, \\
\poemll    the daughter rebels against her mother, \\
\poeml the daughter-in-law against her mother-in-law, \\
\poemll    and a man's enemies are the people of his own house.\fnote{\fbackref{7:6} Cf. Matt 10:34-36}
\passage{Micah Looks to God}
\poeml \v{7}But as for me, I will look to the \divine{Lord}; \\
\poemll    I will wait for the God who will deliver me. \\
\poemlll       My God will hear me. \\
\poeml \v{8}Don't be glad on my account, my enemy. \\
\poemll    When I fall, I'll get up. \\
\poeml Though I sit in darkness, \\
\poemll    the \divine{Lord} is a light for me. \\
\poeml \v{9}I will endure the \divine{Lord}'s anger--- \\
\poemll    since I have sinned against him--- \\
\poeml until he takes over my defense, \\
\poemll    administers justice on my behalf, \\
\poeml and brings me out to the light, \\
\poemll    where I will gaze on his righteousness. \\
\poeml \v{10}Then my enemy will observe it, \\
\poemll    and shame will engulf the ones\fnote{\fbackref{7:10} Lit. \fbib{will bury her}} who asked me, \\
\poemlll       `Where is the \divine{Lord} your God?' \\
\poeml My own eyes will see them,\fnote{\fbackref{7:10} Lit. \fbib{her}} \\
\poemll    they\fnote{\fbackref{7:10} Lit. \fbib{she}} will be trampled on like mud in the streets.
\passage{A Word of Restoration}
\poeml \v{11}When the time comes\fnote{\fbackref{7:11} Lit. \fbib{In the day}} for rebuilding your walls, \\
\poemll    that time\fnote{\fbackref{7:11} Lit. \fbib{in the day}} will surely be extended.\fnote{\fbackref{7:11} Or \fbib{that day the decree will be removed far away}} \\
\poeml \v{12}At that time\fnote{\fbackref{7:12} Lit. \fbib{At that day}} armies\fnote{\fbackref{7:12} Lit. \fbib{they}} will invade you from Assyria, \\
\poemll    from Egyptian cities to the Euphrates\fnote{\fbackref{7:12} The Heb. lacks \fbib{Euphrates}} River, \\
\poeml from sea to sea \\
\poemll    and from mountain to mountain. \\
\poeml \v{13}The land will become desolate \\
\poemll    because of its inhabitants, \\
\poemlll       and as a result of their behavior. \\
\poeml \v{14}Use your rod to shepherd your people, \\
\poemll    the flock that belongs to you, \\
\poemlll       that lives alone in the forest of Carmel. \\
\poeml Let them find pasture in Bashan and Gilead, \\
\poemll    as they did long ago. \\
\poeml \v{15}As I did when\fnote{\fbackref{7:15} Lit. \fbib{As in the days}} you came out of the land of Egypt, \\
\poemll    I will show you\fnote{\fbackref{7:15} Lit. \fbib{him}} awesome things. \\
\poeml \v{16}The nations will look on \\
\poemll    and will be ashamed in spite of all their power; \\
\poeml they will cup their hands over their mouths, \\
\poemll    and their ears will be deaf. \\
\poeml \v{17}They will lick the dust like a serpent; \\
\poemll    they will crawl from their strongholds like snakes. \\
\poeml They will fear the \divine{Lord} our God. \\
\poemll    They will be terrified because of you.
\passage{Who is like God?}
\poeml \v{18}Is there any God like you, \\
\poemll    forgiving iniquity, \\
\poemlll       passing over transgressions by the survivors who are your\fnote{\fbackref{7:18} Lit. \fbib{his}} heritage?\fnote{\fbackref{7:18} I.e. God's people} \\
\poeml He is not angry forever, \\
\poemll    because he delights in gracious love. \\
\poeml \v{19}He will again show us compassion; \\
\poemll    he will subdue our iniquities. \\
\poeml You will hurl all their sins into the deepest sea. \\
\poeml \v{20}You will remain true to Jacob, \\
\poemll    and merciful to Abraham, \\
\poeml as you promised our ancestors long ago.\end{poetry}

\bookheader{Nahum}
\labelbook{Nah}

\bookpretitle{The Book of the Prophet}
\booktitle{Nahum}

\labelchapt{1}
\passage{Nahum's Vision}

\chapt{1}
\v{1}A pronouncement\fnote{\fbackref{1:1} Or \fbib{revelation}} about Nineveh: The record of the vision of Nahum\fnote{\fbackref{1:1} The Heb. name \fbib{Nahum} means \fbib{comfort}} from Elkosh.
\passage{The \divine{Lord}'s Anger against Assyria}

\begin{poetry}
\poeml \v{2}A jealous God, the \divine{Lord} avenges. \\
\poemll    The \divine{Lord} avenges; \\
\poeml The Lord is an angry husband. \\
\poeml The \divine{Lord} takes vengeance on his enemies, \\
\poemll    reserving anger for his adversaries. \\
\poeml \v{3}The \divine{Lord} is slow to anger and powerful, \\
\poemll    and he will never let the guilty\fnote{\fbackref{1:3} The Heb. lacks \fbib{the guilty}} go unpunished. \\
\poeml The \divine{Lord}'s path is in the windstorm and hurricane; \\
\poemll    thunderclouds are dust beneath his feet. \\
\poeml \v{4}He rebukes the sea, and it evaporates; \\
\poemll    he dries up all the rivers. \\
\poeml Bashan and Carmel wither, \\
\poemll    while the flowers of Lebanon languish. \\
\poeml \v{5}Mountains shake because of him, \\
\poemll    and the hills melt. \\
\poeml The earth goes into upheaval at his presence, \\
\poemll    as does the world with all of its inhabitants. \\
\poeml \v{6}Who can stand before his fury? \\
\poemll    And who can endure his fierce anger? \\
\poeml His displeasure pours out like fire, \\
\poemll    and rocks are broken to pieces because of him.
\passage{The \divine{Lord}'s Goodness in the Midst of Judgment}
\poeml \v{7}The \divine{Lord} is good--- \\
\poemll    a refuge in troubled times.\fnote{\fbackref{1:7} Or \fbib{in a day of trouble}} \\
\poeml He knows those who are confiding in him. \\
\poeml \v{8}But with an overwhelming deluge he will bring utter desolation to Nineveh,\fnote{\fbackref{1:8} Lit. \fbib{to its place}} \\
\poemll    and his enemies he will pursue with darkness. \\
\poeml \v{9}What are you scheming against the \divine{Lord}? \\
\poemll    He will bring about utter desolation--- \\
\poemlll       adversity will not strike twice! \\
\poeml \v{10}Indeed, while tangled as by a thorn bush, \\
\poemll    while drunken as by a strong drink, \\
\poemlll       the Ninevites\fnote{\fbackref{1:10} Lit. \fbib{they}} will be burned like dry straw. \\
\poeml \v{11}Someone has left you who plans evil against the \divine{Lord}. \\
\poemll    He is a demonic counselor.\fnote{\fbackref{1:11} Or \fbib{a worthless counselor}; Lit. \fbib{a counselor of Belial}}
\passage{The \divine{Lord}'s Rebuke to Assyria}
\poeml \v{12}This is what the \divine{Lord} says: \\
\poeml ``No matter how strong they are,\fnote{\fbackref{1:12} The Heb. lacks \fbib{they are}} \\
\poemll    and no matter how numerous, \\
\poemlll       they will surely be annihilated\fnote{\fbackref{1:12} Lit. \fbib{be cut down}} and pass away. \\
\poeml Though I have afflicted you,\fnote{\fbackref{1:12} The Heb. lacks \fbib{you}} \\
\poemll    I will do so no more. \\
\poeml \v{13}Now I will break off Assyria's\fnote{\fbackref{1:13} Lit. \fbib{his}} yoke from you, \\
\poemll    and tear apart your shackles.'' \\
\poeml \v{14}Now this is what the Lord has decreed about you, Nineveh:\fnote{\fbackref{1:14} The Heb. lacks \fbib{Nineveh}} \\
\poemll    ``There will be no more children born\fnote{\fbackref{1:14} Lit. \fbib{sown}} to carry on your name. \\
\poeml I will cut out the graven and molten images from the temples of your gods. \\
\poemll    I myself will dig your grave, \\
\poemlll       because you are vile.''
\passage{The Sure and Certain Deliverance of Judah}
\poeml \v{15}\fnote{\fbackref{1:15} This verse is 2:1 in MT}Look! There on the mountains! \\
\poemll    The feet of the one who brings good news, \\
\poemlll       who broadcasts a message of peace. \\
\poeml Judah, celebrate your solemn festivals \\
\poemll    and keep your vows, \\
\poeml because the wicked will never again invade you. \\
\poemll    Nineveh\fnote{\fbackref{1:15} Lit. \fbib{It}} will be\fnote{\fbackref{1:15} Or \fbib{has been}} completely eliminated!
\end{poetry}
\labelchapt{2}
\passage{The Coming Invasion of Nineveh}

\chapt{2}
\v{1}\fnote{\fbackref{2:1} This verse is 2:2 in MT, and so throughout the chapter}You are being attacked by advancing forces!

\begin{poetry}
\poeml Guard your rampart! \\
\poemlll       Watch your roads!\fnote{\fbackref{2:1} Lit. \fbib{the road}} \\
\poeml Prepare yourselves!\fnote{\fbackref{2:1} Lit. \fbib{Strengthen the loins!}} \\
\poemll    Marshall your defenses!\fnote{\fbackref{2:1} Lit. \fbib{Marshall power}} \\
\poeml \v{2}For the \divine{Lord} will restore the glory of Jacob, \\
\poemll    just as he will restore\fnote{\fbackref{2:2} The Heb. lacks \fbib{will restore}} the glory of Israel, \\
\poeml although plunderers have devastated them, \\
\poemll    vandalizing their vine branches. \\
\poeml \v{3}The shields deployed by\fnote{\fbackref{2:3} Lit. \fbib{shield of}} Israel's\fnote{\fbackref{2:3} Lit. \fbib{its}} elite forces are scarlet, \\
\poemll    their valiant men are clothed in crimson. \\
\poeml When they are prepared, \\
\poemll    the polished armament on their chariots will shine, \\
\poemlll       and lances will be brandished about ferociously.\fnote{\fbackref{2:3} The Heb. lacks \fbib{ferociously}} \\
\poeml \v{4}Their chariots storm through the streets, \\
\poemll    jostling each other along broad avenues. \\
\poeml They look like torches, \\
\poemll    as they dart around like lightning. \\
\poeml \v{5}He will remember to summon\fnote{\fbackref{2:5} The Heb. lacks \fbib{to summon}} his finest troops. \\
\poemll    They will stumble on their way, \\
\poemlll       hurrying over to Nineveh's\fnote{\fbackref{2:5} The Heb. lacks \fbib{Nineveh's}} wall. \\
\poeml Their defensive shield is in place. \\
\poeml \v{6}The river gates will be opened, \\
\poemll    and the palace will collapse. \\
\poeml \v{7}It has been determined: \\
\poemll    The woman\fnote{\fbackref{2:7} I.e. Nineveh personified as a woman} is unveiled and sent away, \\
\poemlll       her servant girls mourn. \\
\poeml Beating their breasts, \\
\poemll    they whimper like doves. \\
\poeml \v{8}Nineveh is a reservoir whose water is draining away. \\
\poemll    ``Wait! Wait!'' they cry,\fnote{\fbackref{2:8} The Heb. lacks \fbib{they cry}} \\
\poemlll       yet not even one person\fnote{\fbackref{2:8} The Heb. lacks \fbib{person}} looks back. \\
\poeml \v{9}Take the silver! Take the gold! \\
\poemll    There is no end to the treasure--- \\
\poemlll       fabulous riches of every imagination. \\
\poeml \v{10}Nineveh\fnote{\fbackref{2:10} Lit. \fbib{She}; i.e. Nineveh personified as a woman} is devastated, deserted, and desolate. \\
\poemll    Her heart melts, her knees knock. \\
\poeml Every stomach is upset, \\
\poemll    every face grows pale.\fnote{\fbackref{2:10} Lit. \fbib{gathers blackness}; cf. Joel 2:6b}
\passage{Nineveh: the Lion's Den Destroyed}
\poeml \v{11}Where is this lion's den? \\
\poemll    Where is the place where the young lions fed, \\
\poeml where the lion and its mate walked with their young, \\
\poemll    the place where they feared nothing? \\
\poeml \v{12}This lion renders its prey to pieces to feed its whelps, \\
\poemll    and strangles enough prey\fnote{\fbackref{2:12} The Heb. lacks \fbib{prey}} for its mate, \\
\poeml filling its lairs with prey \\
\poemll    and its dens with rendered flesh. \\
\poeml \v{13}``I am against you,'' declares the \divine{Lord} of the Heavenly Armies, \\
\poemll    ``and I will send your chariots up in smoke. \\
\poeml A sword will devour your young lions, \\
\poemll    I will eliminate your prey from the earth, \\
\poemlll       and the voice of your messengers will no longer be heard.''
\end{poetry}
\labelchapt{3}
\passage{The Coming Judgment of Nineveh}

\chapt{3}
\v{1}Woe to this city, contaminated with shed blood,

\begin{poetry}
\poeml all full of lies and robberies--- \\
\poemlll       it is\fnote{\fbackref{3:1} The Heb. lacks \fbib{it is}} never without victims! \\
\poeml \v{2}The crack of whips \\
\poemll    and the clamor of wheels! \\
\poeml The galloping horses \\
\poemll    and the bounding chariots! \\
\poeml \v{3}The cavalry attacks--- \\
\poemll    the flashing sword \\
\poemlll       and the glittering spear! \\
\poeml Many are the slain--- \\
\poemll    so many casualties!--- \\
\poeml No end to bodies, \\
\poemll    and the soldiers\fnote{\fbackref{3:3} Lit. \fbib{They}} trip over the corpses. \\
\poeml \v{4}Innumerable are the harlotries of this well-favored whore, \\
\poemll    this mistress of witchcraft, \\
\poeml who enslaves nations through her fornication \\
\poemll    and families through her sorcery.
\passage{God's Decree against Nineveh}
\poeml \v{5}``Look, I am against you,'' declares the \divine{Lord} of the Heavenly Armies, \\
\poemll    ``so I will pull up your dress over your face. \\
\poeml I will show your nakedness to the nations, \\
\poemll    and your disgrace to the kingdoms. \\
\poeml \v{6}I will hurl abominable filth upon you, \\
\poemll    making you look foolish, \\
\poemlll       and making an example of you. \\
\poeml \v{7}It will be that everyone who looks at you will run away, saying, \\
\poemll    `Nineveh has been violently overthrown! \\
\poemlll       Who will mourn for her? \\
\poemll    Where will I find anyone to comfort you?'\,''
\passage{Thebes: an Example of God's Justice}
\poeml \v{8}``Are you any better than Thebes,\fnote{\fbackref{3:8} Lit. \fbib{than No-Amon}; i.e. Thebes, capital of southern Egypt (cf. Jer 46.25)} \\
\poemll    which sits by the upper Nile, surrounded by water? \\
\poeml The sea was her defense, \\
\poemll    the waters her wall of protection.\fnote{\fbackref{3:8} The Heb. lacks \fbib{of protection}} \\
\poeml \v{9}Sudan\fnote{\fbackref{3:9} Lit. \fbib{Cush}} was her source of strength, along with Egypt--- \\
\poemll    there were no limits. \\
\poeml Put and the Libyans were her allies. \\
\poeml \v{10}But she, too, was exiled--- \\
\poemll    she went into captivity! \\
\poeml Her young children were dashed to pieces \\
\poemll    at every crossroad of their streets, \\
\poeml and her famous citizens were sold by lottery, \\
\poemll    while all of her aristocrats were put in chains. \\
\poeml \v{11}You will also become drunk. \\
\poemll    You will disappear, \\
\poemlll       trying to hide from your enemies. \\
\poeml \v{12}All your defenses are like fig trees with ripe early fruit--- \\
\poemll    when shaken, it falls right into the devourer's mouth. \\
\poeml \v{13}Look at your people---\fnote{\fbackref{3:13} I.e. Nineveh's army} \\
\poemll    they are women! \\
\poeml Your borders stand wide open to your enemies, \\
\poemll    while fire consumes the bars of your gates.''
\passage{The Futility of Avoiding God's Judgment}
\poeml \v{14}``Draw water, because a siege is coming!\fnote{\fbackref{3:14} The Heb. lacks \fbib{coming}} \\
\poemll    Strengthen your fortresses! \\
\poeml Make the clay good and strong! \\
\poemll    Mix the mortar! \\
\poemlll       Go get your brick molds!\fnote{\fbackref{3:14} I.e. this verse appears to be affirming the uselessness of constructing a defense against God's coming judgment.} \\
\poeml \v{15}In that place fire will consume you, \\
\poemll    the sword will cut you down, \\
\poemlll       consuming you as locusts do. \\
\poeml Multiply yourself like locusts, \\
\poemll    increase like swarming grasshoppers. \\
\poeml \v{16}You added to your inventory of businessmen--- \\
\poemll    they number more than the stars of heaven. \\
\poeml The creeping locust sheds its skin and flies away. \\
\poeml \v{17}Your imperial guards are like the swarming grasshopper; \\
\poemll    your marshals are like hordes of grasshoppers, \\
\poemlll       settling in the stone walls on a chilly day. \\
\poeml The sun rises, and they flee away; \\
\poemll    no one knows where they went. \\
\poeml \v{18}Hey king of Assyria! Your shepherds are asleep \\
\poemll    and your nobles are lying down! \\
\poeml Your people lie scattered on the mountains, \\
\poemll    and there is no one to gather them together. \\
\poeml \v{19}There is no healing for your injury--- \\
\poemll    your wound is fatal. \\
\poeml Everyone who hears about you will applaud, \\
\poemll    because who hasn't escaped your endless evil?''\end{poetry}

\bookheader{Habakkuk}
\labelbook{Hab}

\bookpretitle{The Book of the Prophet}
\booktitle{Habakkuk}

\labelchapt{1}
\passage{Habakkuk's Oracle}

\chapt{1}
\v{1}The pronouncement\fnote{\fbackref{1:1} Or \fbib{revelation}} that the prophet Habakkuk\fnote{\fbackref{1:1} The Heb. name \fbib{Habakkuk} means \fbib{embrace}} perceived.
\passage{The Prophet's First Complaint}

\begin{poetry}
\poeml \v{2}``How long, \divine{Lord}, must I cry out for help, \\
\poeml but you won't listen? \\
\poeml I'm crying out to you, `Violence!' \\
\poemll    but you aren't providing deliverance. \\
\poeml \v{3}Why are you forcing me to look at iniquity \\
\poemll    and to stare at wickedness? \\
\poeml Social havoc and oppression are all around me; \\
\poemll    there are legal conflicts, and disputes abound. \\
\poeml \v{4}Therefore, the Law has become paralyzed, \\
\poemll    and justice never comes about. \\
\poeml Because criminals outnumber\fnote{\fbackref{1:4} Lit. \fbib{are surrounding}} the righteous, \\
\poemll    whenever judgments are issued, they come out crooked.''
\passage{God's Response: The Coming Chaldean Invasion}
\poeml \v{5}``Look out at the nations and pay attention! \\
\poemll    Be astounded! Be really astounded! \\
\poeml Because something is happening in your lifetime \\
\poemll    that you won't believe, even if it were described down to the smallest detail.\fnote{\fbackref{1:5} The Heb. lacks \fbib{down to the smallest detail}} \\
\poeml \v{6}Watch out! For I am bringing in the Chaldeans,\fnote{\fbackref{1:6} I.e. Babylonian invaders} \\
\poemll    that cruel and impetuous\fnote{\fbackref{1:6} Or \fbib{rash}} people, \\
\poeml who sweep across the earth \\
\poemll    dispossessing people\fnote{\fbackref{1:6} The Heb. lacks \fbib{people}} from homes not their own. \\
\poeml \v{7}They are terrible and fearsome; \\
\poemll    their brand of justice and sense of honor derive only from themselves! \\
\poeml \v{8}Their horses are swifter than leopards, \\
\poemll    and more cunning than wolves that attack at night. \\
\poeml Their horsemen are galloping \\
\poemll    as they approach from far away. \\
\poeml They swoop in like ravenous vultures.\fnote{\fbackref{1:8} Or \fbib{eagles}} \\
\poeml \v{9}``They all come to oppress--- \\
\poemll    hordes of them, their faces pressing onward--- \\
\poeml they take prisoners as numerous as\fnote{\fbackref{1:9} The Heb. lacks \fbib{as numerous as}} the desert sand! \\
\poeml \v{10}They make fun of kings, \\
\poemll    deriding those who rule. \\
\poeml They laugh at all of the fortified places, \\
\poemll    constructing ramps to seize them. \\
\poeml \v{11}Then like\fnote{\fbackref{1:11} The Heb. lacks \fbib{like}} the wind sweeping by \\
\poemll    they will pass through--- \\
\poeml they're guilty because they say\fnote{\fbackref{1:11} The Heb. lacks \fbib{they say}} their power is their god.''
\passage{The Prophet's Second Complaint}
\poeml \v{12}``Haven't you existed forever, \\
\poemll    \divine{Lord} my God, my Holy One? \\
\poemlll       We won't die! \\
\poeml \divine{Lord}, you've prepared them\fnote{\fbackref{1:12} I.e. the Babylonian invaders} for judgment; \\
\poemll    Rock, you've sentenced them\fnote{\fbackref{1:12} I.e. the Babylonian invaders} to correction. \\
\poeml \v{13}Your eyes are too pure to gaze upon evil; \\
\poemll    and you cannot tolerate wickedness. \\
\poeml So why do you tolerate the treacherous? \\
\poemll    And why do you stay silent \\
\poemlll       while the wicked devour those who are more righteous than they are? \\
\poeml \v{14}``You have fashioned mankind like fish in the ocean, \\
\poemll    like creeping things that have no ruler. \\
\poeml \v{15}The adversary\fnote{\fbackref{1:15} I.e. the Babylonian invaders} captures them with a hook, \\
\poemll    gathering them up in a fishing net. \\
\poeml He collects them with a dragnet, \\
\poemll    rejoicing and gloating over his catch.\fnote{\fbackref{1:15} The Heb. lacks \fbib{over his catch}} \\
\poeml \v{16}Therefore he sacrifices to his fishing net, \\
\poemll    and burns incense in the presence of his dragnet, \\
\poeml because by them his assets increase \\
\poemll    and he gets plenty of food. \\
\poeml \v{17}Is he to continue to empty his fishing net? \\
\poemll    Will he ever stop killing entire\fnote{\fbackref{1:17} The Heb. lacks \fbib{entire}} nations without mercy?''
\end{poetry}
\labelchapt{2}
\passage{Habakkuk Waits for God's Answer}

\begin{poetry}
\poeml \chapt{2}
\v{1}``I will stand at my guard post \\
\poeml and station myself on a tower. \\
\poeml I will wait and see what the \divine{Lord}\fnote{\fbackref{2:1} The Heb. lacks \fbib{the \divine{Lord}}} will say about me \\
\poemll    and what I\fnote{\fbackref{2:1} Syr \fbib{he}} will answer when he reprimands me.\fnote{\fbackref{2:1} Lit. \fbib{answer at my reprimand}}''
\passage{God's Reply to the Prophet's Complaint}
\poeml \v{2}When he answered, the \divine{Lord} told me: \\
\poeml ``Write out the revelation, \\
\poemll    engraving it clearly on the tablets, \\
\poemlll       so that a courier may run with it.\fnote{\fbackref{2:2} Or \fbib{that whoever reads it may run}} \\
\poeml \v{3}For the revelation pertains to an appointed time--- \\
\poemll    it speaks truthfully\fnote{\fbackref{2:3} Lit. \fbib{speaks without deception}} about the end. \\
\poeml Though it delays, wait for it, \\
\poemll    because it will surely come about--- \\
\poemlll       it will not be late! \\
\poeml \v{4}``Notice their\fnote{\fbackref{2:4} I.e. the Babylonian invaders} arrogance--- \\
\poemll    they have no inward uprightness\fnote{\fbackref{2:4} Lit. \fbib{no uprightness of soul}}--- \\
\poemlll       but the righteous will live by their faith. \\
\poeml \v{5}Moreover, just as wine leads astray the proud and powerful man, \\
\poemll    he\fnote{\fbackref{2:5} I.e. Babylonian invaders personified in their king} remains restless; \\
\poeml he\fnote{\fbackref{2:5} I.e. Babylonian invaders personified in their king} has expanded his appetite--- \\
\poemll    like the afterlife\fnote{\fbackref{2:5} Lit. \fbib{Sheol}; i.e. the realm of the dead} or death itself, he\fnote{\fbackref{2:5} I.e. Babylonian invaders personified in their king} is never satisfied. \\
\poeml He\fnote{\fbackref{2:5} I.e. Babylonian invaders personified in their king} gathers to himself all of the nations, \\
\poemll    taking captive all of the people for himself.''
\passage{Judgment on the Plunderer of Nations}
\poeml \v{6}``Will not all of these ridicule him \\
\poemll    with mocking scorn? They will say, \\
\poeml `Woe to the one who hordes for himself what isn't his. \\
\poemll    How long will you enrich yourself by extortion?'\fnote{\fbackref{2:6} Lit. \fbib{by loans}} \\
\poeml \v{7}Won't your creditors revolt unexpectedly? \\
\poemll    Won't those who make you tremble wake up? \\
\poemlll       As a result, you'll become their prey! \\
\poeml \v{8}Because you plundered many nations, \\
\poemll    all of their remnants will plunder you. \\
\poeml Human blood has been shed,\fnote{\fbackref{2:8} The Heb. lacks \fbib{has been shed}} \\
\poemll    and violence has been done to\fnote{\fbackref{2:8} Lit. \fbib{violence of}} the land, \\
\poemlll       to the city, and to all who live in it.''
\passage{Judgment on Those who Think They are Safe}
\poeml \v{9}``Woe to the one who amasses profit upon unjust profit \\
\poemll    in order to establish his household, \\
\poeml so he can establish a secure place\fnote{\fbackref{2:9} Lit. \fbib{establish his nest}} on the heights \\
\poemll    and escape from the power of evil. \\
\poeml \v{10}You have brought shame to yourself\fnote{\fbackref{2:10} Lit. \fbib{to your house}} by killing many people--- \\
\poemll    you are forfeiting your own life. \\
\poeml \v{11}Indeed, the stone will cry out from the wall \\
\poemll    and the rafter will respond from the woodwork.''
\passage{Judgment on the Lawless}
\poeml \v{12}``Woe to the one who founds a city upon bloodshed, \\
\poemll    and constructs a city by lawlessness. \\
\poeml \v{13}Is it not because of the \divine{Lord} of the Heavenly Armies \\
\poemll    that people grow tired putting out fires,\fnote{\fbackref{2:13} Lit. \fbib{tired for the sufficiency of fire}} \\
\poemlll       and nations weary themselves over nothing? \\
\poeml \v{14}Indeed, the earth will be filled \\
\poemll    with knowledge of the glory of the \divine{Lord}, \\
\poemlll       as water fills\fnote{\fbackref{2:14} Lit. \fbib{covers}} the sea.''
\passage{Judgment on the Violent}
\poeml \v{15}``Woe to the one who supplies his neighbor with a drink! \\
\poemll    You are forcing your bottle\fnote{\fbackref{2:15} Lit. \fbib{wine skin}} on him,\fnote{\fbackref{2:15} The Heb. lacks \fbib{on him}} \\
\poemlll       making him drunk so you can see them naked. \\
\poeml \v{16}You are filled with dishonor instead of glory. \\
\poemll    So go ahead,\fnote{\fbackref{2:16} The Heb. lacks \fbib{go ahead}} drink and be naked! \\
\poemll    The \divine{Lord}\fnote{\fbackref{2:16} Lit. \fbib{The power of the \divine{Lord}'s right hand}} will turn against you, \\
\poemll    and utter disgrace will debase your reputation.\fnote{\fbackref{2:16} Lit. \fbib{will come upon your glory}} \\
\poeml \v{17}Indeed, the violence done to Lebanon will overtake you, \\
\poemll    and the destruction of the beasts will terrorize you---\fnote{\fbackref{2:17} The Heb. lacks \fbib{you}} \\
\poeml because you shed human blood \\
\poemll    and did violence to\fnote{\fbackref{2:17} Lit. \fbib{violence of}} the land, to the city, and to all who live in it.''
\passage{Judgment on the Idol Maker}
\poeml \v{18}``Where is the benefit in owning\fnote{\fbackref{2:18} The Heb. lacks \fbib{owning}} a carved image, \\
\poemll    that motivates its maker to carve\fnote{\fbackref{2:18} Lit. \fbib{because its maker carved}} it? \\
\poeml It is only a cast image--- \\
\poemll    a teacher that lies--- \\
\poeml because the engraver entrusts himself to his carving, \\
\poemll    crafting speechless idols. \\
\poeml \v{19}``Woe to the one who says to a tree, `Wake up!' \\
\poemll    or `Arise!' to a speechless stone. \\
\poeml Idols\fnote{\fbackref{2:19} Lit. \fbib{Things}} like this can't teach, can they? \\
\poemll    Look, even though it is overlaid with gold and silver, \\
\poemlll       there's no breath in it at all.''
\passage{The \divine{Lord}'s Final Counsel to Habakkuk}
\poeml \v{20}``The \divine{Lord} is in his holy Temple. \\
\poemll    All the earth---be quiet in his presence.''
\end{poetry}
\labelchapt{3}
\passage{Habakkuk's Prayer of Faith}

\chapt{3}
\v{1}A prayer by the prophet Habakkuk, set to music.\fnote{\fbackref{3:1} Lit. \fbib{prayer upon Shigionoth}}

\begin{poetry}
\poeml \v{2}\divine{Lord}, as I listen to what has been said about you, \\
\poemll    I am afraid. \\
\poeml \divine{Lord}, revive your work throughout all of our lives--- \\
\poemll    reveal yourself\fnote{\fbackref{3:2} The Heb. lacks \fbib{yourself}} throughout all of our lives--- \\
\poeml when you\fnote{\fbackref{3:2} The Heb. lacks \fbib{you}} are angry, \\
\poemll    remember compassion. \\
\poeml \v{3}God comes from Teman\fnote{\fbackref{3:3} I.e. an Edomite desert town}--- \\
\poemll    the Holy One from Mount Paran.\fnote{\fbackref{3:3} I.e. in the Sinai desert}
\end{poetry}
\interlude{Interlude}

\begin{poetry}
\poeml His glory spreads throughout the heavens, \\
\poemll    and praises about him fill the earth. \\
\poeml \v{4}His radiance is like sunlight; \\
\poemll    beams of light shine\fnote{\fbackref{3:4} The Heb. lacks \fbib{shine}} from his hand, \\
\poemlll       where his strength lays hidden. \\
\poeml \v{5}Before him pestilence walks, \\
\poemll    and disease follows behind him.\fnote{\fbackref{3:5} Lit. \fbib{follows at his feet}} \\
\poeml \v{6}He stood up and shook the land; \\
\poemll    with his stare he startled the nations. \\
\poeml The age-old mountains were shattered, \\
\poemll    and the ancient hilltops bowed down. \\
\poeml His ways are eternal. \\
\poeml \v{7}I saw the tents of Cushan in distress, \\
\poemll    and the tent curtains of the land of Midian in anguish. \\
\poeml \v{8}Was the \divine{Lord} displeased with the rivers? \\
\poemll    Was your anger directed\fnote{\fbackref{3:8} The Heb. lacks \fbib{directed}} against the watercourses \\
\poemlll       or your wrath against the sea? \\
\poeml Indeed, you rode upon your horses, \\
\poemll    upon your chariots of deliverance. \\
\poeml \v{9}Your bow was exposed, \\
\poemll    and your\fnote{\fbackref{3:9} The Heb. lacks \fbib{your}} arrows targeted by command.
\end{poetry}
\interlude{Interlude}

\begin{poetry}
\poemlll       You split the earth with rivers. \\
\poeml \v{10}When the mountains looked upon you, they trembled; \\
\poemll    the overflowing water passed by, \\
\poeml the ocean shouted, \\
\poemll    and its waves\fnote{\fbackref{3:10} Lit. \fbib{hands}} surged upward. \\
\poeml \v{11}The sun and moon stand still in their orbits; \\
\poemll    at the glint of your arrows they speed along, \\
\poemlll       even at the gleam of your flashing spear. \\
\poeml \v{12}You march through the land in righteous\fnote{\fbackref{3:12} The Heb. lacks \fbib{righteous}} indignation; \\
\poemll    you tread down the nations in anger. \\
\poeml \v{13}You marched out to deliver your people, \\
\poemll    to deliver with your anointed. \\
\poeml You struck the head of the house of the wicked; \\
\poemll    you stripped him naked from head to foot.
\end{poetry}
\interlude{Interlude}

\begin{poetry}
\poeml \v{14}With his own lances you pierced the heads of his warriors, \\
\poemll    who came out like a windstorm to scatter us\fnote{\fbackref{3:14} Lit. \fbib{me}}--- \\
\poemlll       their joy is to devour the afflicted who are in hiding. \\
\poeml \v{15}You rode on the sea with your horses, \\
\poemll    even riding\fnote{\fbackref{3:15} The Heb. lacks \fbib{even riding}} the crested waves of mighty waters.
\passage{Habakkuk's Response}
\poeml \v{16}I heard and I trembled within. \\
\poemll    My lips quivered at the noise. \\
\poeml My legs gave way beneath me,\fnote{\fbackref{3:16} Lit. \fbib{Rottenness enters my bones}} \\
\poemll    and I trembled. \\
\poeml Nevertheless, I await the day of distress \\
\poemll    that will dawn on our invaders. \\
\poeml \v{17}Even though the fig tree does not blossom, \\
\poemll    and there are no grapes on the vines; \\
\poeml even if the olive harvest fails, \\
\poemll    and the fields produce nothing edible; \\
\poeml even if the flock is snatched from the sheepfold, \\
\poemll    and there is no herd in the stalls--- \\
\poeml \v{18}as for me, I will rejoice in the \divine{Lord}. \\
\poemll    I will find my joy in the God who delivers me. \\
\poeml \v{19}The \divine{Lord} God is my strength--- \\
\poemll    he will make my feet like those of a deer, \\
\poemlll       equipping me to tread on my mountain heights.
\end{poetry}
\psalminfo{For the choir director:}
\psalminfo{On my stringed instruments.}

\bookheader{Zephaniah}
\labelbook{Zeph}

\bookpretitle{The Book of the Prophet}
\booktitle{Zephaniah}

\labelchapt{1}
\passage{The Imminent Destruction of Judah}

\chapt{1}
\v{1}This message from the \divine{Lord} came to Cushi's son Zephaniah,\fnote{The Heb. name \fbib{Zephaniah} means \fbib{The \divine{Lord} has treasured}} the grandson of Gedaliah and great-grandson of Hezekiah's son Amariah, during the reign\fnote{Lit. \fbib{in the days of}} of Amon's son Josiah, king of Judah:

\begin{poetry}
\poeml \v{2}``I'll utterly sweep away everything from\fnote{Lit. \fbib{from the face of}} the land,'' \\
\poemll    declares the \divine{Lord}. \\
\poeml \v{3}``I'll consume both human beings and animals--- \\
\poemll    I'll consume the birds of the sky, \\
\poeml the fish in the sea, \\
\poemll    and the wicked along with their sin,\fnote{Lit. \fbib{with the heap of rubble}} \\
\poeml when I eliminate human beings from the land,'' \\
\poemlll       declares the \divine{Lord}. \\
\poeml \v{4}``I will also stretch out my hand against Judah, \\
\poemll    and upon all inhabitants of Jerusalem. \\
\poeml I'll wipe out every trace of Baal from this place, \\
\poemll    and the name\fnote{Or \fbib{authority}} of the pagan priests\fnote{Lit. \fbib{the Chemarim}; i.e. idol worshiping priests}, \\
\poemlll       along with my own\fnote{The Heb. lacks \fbib{my own}} priests. \\
\poeml \v{5}I'll wipe out\fnote{The Heb. lacks \fbib{I'll wipe out}} those who worship the stars \\
\poemll    that they view\fnote{The Heb. lacks \fbib{that they view}} from their housetops, \\
\poeml those who bow down and swear to the \divine{Lord} \\
\poemll    and who also swear by Milcom,\fnote{I.e. the national idol of the Ammonites} \\
\poeml \v{6}those who turn away from the \divine{Lord}, \\
\poemll    don't seek the \divine{Lord}, \\
\poemlll       and never ask for his help.\fnote{Lit. \fbib{and don't inquire of him}}''
\passage{The Approaching Day of the \divine{Lord}}
\poeml \v{7}Remain silent in the presence of the Lord \divine{God}, \\
\poemll    because the Day of the \divine{Lord} approaches, \\
\poeml and because the \divine{Lord} has prepared a sacrifice \\
\poemll    for those whom he has invited to be consecrated. \\
\poeml \v{8}``It will come about during\fnote{Lit. \fbib{in the day of}} the \divine{Lord}'s sacrifice \\
\poemll    that I'll punish the officials,\fnote{Lit. \fbib{princes}} the royal descendants, \\
\poemlll       and all who wear foreign clothing.\fnote{Or \fbib{wear cultic vestments}} \\
\poeml \v{9}At the same time, I'll punish every idol worshipper,\fnote{Lit. \fbib{everyone who leaps over the threshold}; cf. 1Sam 5:5} \\
\poemll    especially those who are filling their master's temple with violence and deceit. \\
\poeml \v{10}When all of this happens,''\fnote{Lit. \fbib{It will come about in that day}} declares the \divine{Lord}, \\
\poemll    ``a loud shriek will come\fnote{The Heb. lacks \fbib{will come}} from the Fish Gate, \\
\poeml and howling from the Mishneh\fnote{Or \fbib{from The Second}; i.e. a section of Jerusalem} Quarter, \\
\poemll    along with great destruction from the hills.''
\passage{Divine Judgment on the Business Community}
\poeml \v{11}``Wail, you who live in the market district, \\
\poemll    because all of the merchants will be crushed \\
\poemlll       and all of their customers\fnote{Lit. \fbib{all who carry silver}} will be eliminated.\fnote{Lit. \fbib{be cut off}} \\
\poeml \v{12}And it will come about that I will search Jerusalem with candles,\fnote{Or \fbib{with oil lamps}} \\
\poemll    punishing the self-satisfied and complacent, \\
\poeml who say to themselves, \\
\poemll    `The \divine{Lord} will do neither good nor evil.' \\
\poeml \v{13}Therefore their possessions will be seized as plunder \\
\poemll    and their homes left in ruins. \\
\poeml They may build houses, \\
\poemll    but they won't live in them. \\
\poeml They may plant vineyards, \\
\poemll    but they won't drink their wine.''
\passage{Zephaniah's Description of the Day of the \divine{Lord}}
\poeml \v{14}``The great Day of the \divine{Lord} approaches--- \\
\poemll    How it comes, hurrying faster and faster! \\
\poeml The sound of the Day of the \divine{Lord} there \\
\poemll    includes the bitter cry of the mighty soldier. \\
\poeml \v{15}That day will be filled with wrath, \\
\poemll    a day of trouble and tribulation, \\
\poeml a day of desolation and devastation, \\
\poemll    a day of doom and gloom, \\
\poeml a day of clouds and shadows,\fnote{Cf. Joel 2:2a} \\
\poeml \v{16}a day of trumpet and battle cry \\
\poemlll       against fortified cites and watch\fnote{Lit. \fbib{corner}} towers. \\
\poeml \v{17}``And I'll bring so much distress to people \\
\poemll    that they will walk around like the blind. \\
\poeml Because they have sinned against the \divine{Lord}, \\
\poemll    their blood will be poured out like dust \\
\poemlll       and their intestines will spill out\fnote{The Heb. lacks \fbib{will spill out}} like manure. \\
\poeml \v{18}Neither their silver nor their gold will deliver them \\
\poemll    in the Day of the \divine{Lord}'s wrath; \\
\poeml but the entire land will be consumed \\
\poemll    by the fire of his jealousy, \\
\poeml for he will bring the inhabitants of the land to a sudden end.''
\end{poetry}
\labelchapt{2}
\passage{A Plea for Repentance to the People}

\begin{poetry}
\poeml \chapt{2}
\v{1}``Gather together! \\
\poeml Yes, indeed, gather together, \\
\poemlll       you shameless nation! \\
\poeml \v{2}Before the decree is carried out, \\
\poemll    before the day flies away like chaff, \\
\poeml before the fierce anger of the \divine{Lord} visits you, \\
\poemll    before the Day of the \divine{Lord}'s wrath surprises\fnote{Lit. \fbib{wrath comes upon}} you, \\
\poeml \v{3}seek the \divine{Lord}, all you humble people of the land, \\
\poemll    who do what he commands. \\
\poeml Seek righteousness! \\
\poemll    Seek humility! \\
\poemlll       Maybe you will be protected in the Day of the \divine{Lord}'s anger.''
\passage{The Coming Destruction of Philistine Cities}
\poeml \v{4}``For Gaza will be forsaken,\fnote{MT word \fbib{forsaken} sounds like The Heb. name \fbib{Gaza.}} \\
\poemll    and Ashkelon deserted--- \\
\poeml Ashdod will be emptied at high noon; \\
\poemll    even Ekron will be uprooted.\fnote{MT word \fbib{uprooted} sounds like The Heb. name \fbib{Ekron}.} \\
\poeml \v{5}Woe to those who live along the coast, \\
\poemll    the people of Philistia!\fnote{Lit. \fbib{of the Cherethites}} \\
\poeml This message from the \divine{Lord} is being spoken against you, \\
\poemll    Canaan, land of the Philistines: \\
\poemlll       `I'll destroy you until no one lives there!' \\
\poeml \v{6}The Philistine\fnote{Lit. \fbib{Cherethite}} coast will become meadows \\
\poemll    for shepherds and sheep pens. \\
\poeml \v{7}The survivors of Judah will find pasture on it; \\
\poemll    at twilight they will lie down in the houses of Ashkelon, \\
\poeml for the \divine{Lord} their God will visit them, \\
\poemll    restoring their prosperity.''\fnote{Or \fbib{them, turning away their captivity}}
\passage{The Lord's Rebuke to Moab and Ammon}
\poeml \v{8}``I've heard Moab's insults \\
\poemll    and the curses from those Ammonites\fnote{Lit. \fbib{the descendants of Ammon}; so also in v. 9.} \\
\poeml by which they defame my people \\
\poemll    and boast about their territory. \\
\poeml \v{9}Therefore as I'm alive and living,'' declares the \divine{Lord} of the Heavenly Armies, the God of Israel, \\
\poemll    ``Moab will surely become like Sodom, \\
\poemlll       and the Ammonites\fnote{Lit. \fbib{the descendants of Ammon}; so also in v. 9.} like Gomorrah, \\
\poeml a place overrun by weeds and salty marshes, \\
\poemll    unpopulated forever. \\
\poeml The survivors of my people will confiscate their property, \\
\poemll    and those who remain of my nation will inherit what was theirs.\fnote{Cf. Obad 19-21} \\
\poeml \v{10}This they\fnote{I.e. the people of Edom and Moab} will have in lieu of their pride, \\
\poemll    because they have insulted and mocked the people of the \divine{Lord} of the Heavenly Armies. \\
\poeml \v{11}The \divine{Lord} will incite them to terror, \\
\poemll    because he will cause all the gods of the earth to waste away. \\
\poeml They will worship him, \\
\poemll    every person in his own home, \\
\poemlll       including even the coastlands of the nations.''
\passage{The \divine{Lord}'s Rebuke to Cush and Nineveh}
\poeml \v{12}``Now as for you, Cush,\fnote{I.e. Sudan, Ethiopia, or southern Iraq} \\
\poemll    you\fnote{Lit. \fbib{they}} will surely be slain by my sword! \\
\poeml \v{13}``And the \divine{Lord}\fnote{Lit. \fbib{And he}} will attack\fnote{Lit. \fbib{stretch out his hand toward}} the north, destroying Assyria. \\
\poemll    He will turn Nineveh into a desolate ruin, \\
\poemlll       as dry as a desert wilderness. \\
\poeml \v{14}Flocks will lie down in her midst, \\
\poemll    along with animals of every kind. \\
\poeml Desert owls and screeching owls will nest at the top of the pillars, \\
\poemll    hooting through the vacant\fnote{The Heb. lacks \fbib{vacant}} windows, \\
\poemlll       `Ruin sits at these doorsills,' \\
\poeml for he will expose even the cedar framework. \\
\poeml \v{15}This is that carefree city that lived irresponsibly, \\
\poemll    that told herself, `Me first!'\fnote{Lit. \fbib{said in her heart, `I'}} \\
\poemlll       and, `There will be no one else!' \\
\poeml How ruined she has become--- \\
\poemll    a habitat for wild animals! \\
\poeml Everyone who passes by her will sneer at her \\
\poemll    and make obscene gestures!''\fnote{Lit. \fbib{and wag his hand}}
\end{poetry}
\labelchapt{3}
\passage{The \divine{Lord}'s Rebuke to Jerusalem}

\begin{poetry}
\poeml \chapt{3}
\v{1}Woe to this filthy, polluted, and oppressive city! \\
\poeml \v{2}It won't obey anyone.\fnote{Lit. \fbib{obey the voice}} \\
\poeml It won't accept discipline. \\
\poemll    It does not trust in the \divine{Lord}. \\
\poemlll       It does not approach God. \\
\poeml \v{3}Its national officials\fnote{Lit. \fbib{Her princes}} are roaring lions; \\
\poemll    its judges are like wolves of the night \\
\poemlll       that don't leave the bones for the morning. \\
\poeml \v{4}Its prophets are arrogant and treacherous. \\
\poemll    Its priests have contaminated the sanctuary. \\
\poemlll       They give perverse interpretations of the Law.\fnote{Lit. \fbib{They do violence to the Law}} \\
\poeml \v{5}The righteous \divine{Lord} who lives\fnote{The Heb. lacks \fbib{within her}} within her will do no wrong; \\
\poemll    he will bring justice to light morning by morning. \\
\poeml He never fails, \\
\poemll    but the unjust are shameless.
\passage{The Integrity of God's Justice}
\poeml \v{6}``I have destroyed\fnote{Lit. \fbib{have cut off}} nations--- \\
\poemll    their fortifications are deserted. \\
\poeml I have turned their main thoroughfares into wastelands \\
\poemll    where no one will travel. \\
\poeml Their cities are desolate; \\
\poemll    as a result, not one man remains--- \\
\poemlll       no, not even a single resident. \\
\poeml \v{7}I have said, `If only you would fear me, \\
\poemll    if only you would take my instructions to heart.' \\
\poemlll       Then their houses would not have been torn down. \\
\poeml I have chastened them, \\
\poemll    but they were eager to corrupt everything they were doing.''
\passage{The Future Deliverance of Israel}
\poeml \v{8}``Just you wait!'' declares the \divine{Lord}. \\
\poemll    ``The day will come when I stand up as a prosecutor,\fnote{Or \fbib{witness}} \\
\poeml for I am determined to assemble the nations \\
\poemll    and to gather the kingdoms, \\
\poeml in order to pour out my indignation upon them--- \\
\poemll    all of my fierce anger. \\
\poeml All the earth will be consumed by the fire of my jealousy. \\
\poeml \v{9}Indeed, then I will return my people to a pure language\fnote{Or \fbib{to pure lips}} \\
\poemll    so that they all may call upon the name of the \divine{Lord}, \\
\poemlll       serving him with a united will.\fnote{Lit. \fbib{shoulder}} \\
\poeml \v{10}``From beyond the rivers of Sudan\fnote{Or Ethiopia; Lit. \fbib{Cush}} my worshipers\fnote{Lit. \fbib{incense burners}}--- \\
\poemll    including my dispersed people--- \\
\poemlll       will present offerings to me. \\
\poeml \v{11}When this happens,\fnote{Lit. \fbib{On that day}} you will not be ashamed of your actions \\
\poemll    by which you sinned against me, \\
\poeml because I will remove from among you those who revel in pride. \\
\poemll    Arrogance will have no place in my holy mountain. \\
\poeml \v{12}I will keep a humble and gentle people among you, \\
\poemll    and they will trust in the name of the \divine{Lord}. \\
\poeml \v{13}The survivors of Israel will not practice lawlessness \\
\poemll    nor tell lies, \\
\poeml nor will a deceitful message be found in their mouths, \\
\poemll    because they will eat and rest, \\
\poemlll       with no one to cause fear.''
\passage{Israel's Future Joys}
\poeml \v{14}``Sing aloud, daughter of Zion! \\
\poemll    Shout out, Israel! \\
\poemlll       Rejoice with all of your heart, daughter of Jerusalem! \\
\poeml \v{15}The \divine{Lord} has acquitted you;\fnote{Lit. \fbib{has withdrawn your judgments}} \\
\poemll    turning back your adversaries. \\
\poeml Israel's king, the \divine{Lord}, is among you; \\
\poemll    you will not fear disaster anymore. \\
\poeml \v{16}``When all of this happens,\fnote{Lit. \fbib{In that day}} it will be told Jerusalem, \\
\poemll    `Don't be afraid!'' \\
\poeml and to Zion, \\
\poemll    `Don't lose courage!''\fnote{Lit. \fbib{Don't let your hands drop}} \\
\poeml \v{17}The \divine{Lord} your God among you is powerful--- \\
\poemll    he will save \\
\poeml and he will take joyful delight in you. \\
\poemll    In his love he will renew you\fnote{The Heb. lacks \fbib{you}} with his love; \\
\poemlll       he will celebrate with singing because of you. \\
\poeml \v{18}I will gather the afflicted from the solemn assembly; \\
\poemll    those who were with you, \\
\poemlll       who were bearing a burden of disgrace. \\
\poeml \v{19}``Watch how I deal with everyone who oppresses you! \\
\poemll    At that time I will rescue the one who is lame, \\
\poemlll       and I will draw to me\fnote{The Heb. lacks \fbib{to me}} the one who has been driven away. \\
\poeml I will honor\fnote{Lit. \fbib{give}} them with praise \\
\poemll    and with a good reputation in every land \\
\poemlll       where they have been put to shame. \\
\poeml \v{20}At that time I will gather you; \\
\poemll    at that time I will bring you home.\fnote{The Heb. lacks \fbib{home}} \\
\poeml Indeed, I will give you a good reputation, \\
\poemll    making you praiseworthy among all of the people of the world, \\
\poemlll       when I restore your prosperity before your eyes,'' says the \divine{Lord}.\end{poetry}

\addcontentsline{toc}{chapter}{Shorter Prophetic Writings After the Exile}
\bookheader{Haggai}
\labelbook{Hag}

\bookpretitle{The Book of the Prophet}
\booktitle{Haggai}

\labelchapt{1}
\passage{Call to Rebuild the Temple}

\chapt{1}
\v{1}On the first day of the sixth month of the second year of the reign of\fnote{The Heb. lacks \fbib{the reign of}} King Darius, this message from the \divine{Lord} came by\fnote{Lit. \fbib{by the hand of}} Haggai\fnote{The Heb. name \fbib{Haggai} means \fbib{festive}} the prophet to Shealtiel's son Zerubbabel, governor of Judah, and to Jehozadak's son Joshua, the high priest:

\v{2}``This is what the \divine{Lord} of the Heavenly Armies says: `These people keep saying, ``No, the right\fnote{The Heb. lacks \fbib{right}} time for rebuilding the \divine{Lord}'s Temple has not yet come.''\,'\,''

\v{3}Then this message from the \divine{Lord} came by\fnote{Lit. \fbib{by the hand of}} Haggai the prophet: \v{4}``Is it the right\fnote{The Heb. lacks \fbib{right}} time for all of you to live in your own paneled houses while this house remains in ruins?''
\passage{Consequences of Not Rebuilding}

\v{5}``Now this is what the \divine{Lord} of the Heavenly Armies, says: `Carefully consider your ways:

\begin{poetry}
\poeml \v{6}You have sowed much \\
\poemll    but have reaped little. \\
\poeml You have eaten \\
\poemll    but don't have enough to become satisfied. \\
\poeml You have drunk \\
\poemll    but don't have enough to become intoxicated. \\
\poeml You have clothed yourself \\
\poemll    but don't have enough to keep warm. \\
\poeml And the hired laborer deposits his salary \\
\poemll    in a bag full of holes!'\,''
\end{poetry}
\passage{Command to Rebuild the Temple}

\v{7}``This is what the \divine{Lord} of the Heavenly Armies says: `Carefully consider your ways: \v{8}Go up into the mountains, bring timber, and reconstruct my house. Then I will be pleased with it and I will be honored,' says the \divine{Lord}. \v{9}`You turned away in pursuit of abundance, but look at how little you found!\fnote{The Heb. lacks \fbib{you found}} What you did manage to bring home, I blew away! And why?' declares the \divine{Lord} of the Heavenly Armies. `It's because of my house! It lies in ruins while each of you runs off to his own house! \v{10}That is why the heavens keep withholding dew from you, and the earth withholds her fruit. \v{11}I sent a drought on the land, on the mountains, on your grain, on your new wines, on your oil---on everything the ground produces---on men, on livestock, and on everything you do!\fnote{Lit. \fbib{every product of your labor}}'\,''
\passage{The People Obey}

\v{12}Then Shealtiel's son Zerubbabel, Jehozadak's son Joshua the high priest, and all the rest\fnote{Or \fbib{the whole remnant}} of the people obeyed the \divine{Lord} their God and the words of Haggai the prophet, because the \divine{Lord} their God had sent him. And the people feared the \divine{Lord}. \v{13}Haggai, the messenger of the \divine{Lord}, spoke to the people with a special commission\fnote{Or \fbib{message}} from the \divine{Lord}: ```I am with you,' declares the \divine{Lord}.''

\v{14}Then the \divine{Lord} revitalized the spirit of Shealtiel's son Zerubbabel, governor of Judah, the spirit of Jehozadak's son Joshua the high priest, and the spirit of all the rest\fnote{Or \fbib{the whole remnant}} of the people, so they came and began to work on the house of their God, the \divine{Lord} of the Heavenly Armies. \v{15}This took place on the twenty-fourth day of the sixth month of the second year of the reign of\fnote{The Heb. lacks \fbib{the reign of}} King Darius.
\labelchapt{2}
\passage{The Future Glory of the \divine{Lord}'s House}

\chapt{2}
\v{1}On the twenty-first day of the seventh month, this message from the \divine{Lord} came by\fnote{Lit. \fbib{by the hand of}} Haggai the prophet: \v{2}``Speak to Shealtiel's son Zerubbabel, governor of Judah, to Jehozadak's son Joshua, the high priest, and to the rest\fnote{Or \fbib{the whole remnant}} of the people, asking, \v{3}`Who is left among you who saw this house in its former glory? And what does it look like now? From what you can see, it seems like nothing, doesn't it? \v{4}Now be strong, Zerubbabel,' declares the \divine{Lord}, `and be strong, Joshua son of Jehozadak, the high priest, and be strong, all you people of the land,' declares the \divine{Lord}. `Go to work, because I am with you,' declares the \divine{Lord} of the Heavenly Armies. \v{5}`My Spirit continues to dwell among you, according to\fnote{The Heb. lacks \fbib{according to}} the covenant I established when you came out from Egypt. Don't be afraid!'

\v{6}``For this is what the \divine{Lord} of the Heavenly Armies says: `Once more, in a little while, I will make the heavens, the earth, the sea, and the dry land to shake. \v{7}I will shake all nations, and the One desired by all nations will come. Then I will fill this house with glory,' says the \divine{Lord} of the Heavenly Armies.

\v{8}``The silver belongs to me, as does the gold,''\fnote{Lit. \fbib{me and the gold belongs to me}} declares the \divine{Lord} of the Heavenly Armies. \v{9}``The glory of this present house will be greater than was the former,'' declares the \divine{Lord} of the Heavenly Armies. ``And in this place I will grant peace,'' declares the \divine{Lord} of the Heavenly Armies.''
\passage{God's Promise to His Sinful People}

\v{10}On the twenty-fourth day\fnote{The Heb. lacks \fbib{day}} of the ninth month\fnote{The Heb. lacks \fbib{month}} of the second year of the reign of\fnote{The Heb. lacks \fbib{the reign of}} King Darius, this message from the \divine{Lord} came to Haggai the prophet: \v{11}``This is what the \divine{Lord} of the Heavenly Armies says: `Ask the priests about what the Law says: \v{12}``If a man carries consecrated meat in the folds of his garment, and if his garment touches bread, stew, wine, oil, or any other edible thing,\fnote{Lit. \fbib{or all food}} will these things become consecrated?''\,'\,''\fnote{Or \fbib{holy}}

The priests answered, ``No.''

\v{13}So Haggai responded, ``If someone who is unclean\fnote{I.e. ritually defiled} because of contact with a dead body were to touch any of these things, would they become unclean?''\fnote{I.e. ritually defiled}

The priests responded, ``They would be unclean.''\fnote{I.e. ritually defiled}

\v{14}Then Haggai answered, ```That's how I look at this people and this nation,' declares the \divine{Lord}. `And that's how it is with everything they undertake\fnote{Lit. \fbib{with every work of their hands}} and with what they offer there---it is unclean.\fnote{I.e. ritually defiled} \v{15}Pay attention\fnote{The Heb. lacks \fbib{Pay attention}} from now on to how things used to be before one stone had been laid upon another in the Temple of the \divine{Lord}. \v{16}When someone came to a pile of grain to get 20 measures,\fnote{Lit. \fbib{a pile of 20}} there were only ten. Or when someone approached the wine press to siphon out 50 measures, there were only 20. \v{17}I punished you and everything that you undertook\fnote{Lit. \fbib{with every work of your hand}}---with scorching wind, with mildew, and with hail, and you still did not return\fnote{The Heb. lacks \fbib{return}} to me,' declares the \divine{Lord}. \v{18}`Pay attention from now on, from this twenty-fourth day of the ninth month,\fnote{The Heb. lacks \fbib{month}} when the foundation of the \divine{Lord}'s Temple was laid. Pay attention! \v{19}Is there seed left in the granary? Up until now, neither the vine, the fig tree, the pomegranate, nor the olive tree has borne fruit, but from this very day I will bless you.'\,''\fnote{The Heb. lacks \fbib{you}}
\passage{God's Promise to Zerubbabel}

\v{20}This message from the \divine{Lord} came a second time to Haggai on the twenty-fourth day of the month: \v{21}``Speak to Zerubbabel, governor of Judah. Tell him, `I'm going to shake the heavens and the earth. \v{22}I will overthrow royal thrones. I will annihilate the strength of national governments. I will overthrow chariots along with those who drive them. Both\fnote{The Heb. lacks \fbib{Both}} horses and their riders will fall, each one by means of his comrade's weapon.\fnote{Lit. \fbib{sword}} \v{23}On that day,' declares the \divine{Lord} of the Heavenly Armies, `I will take you, my servant Zerubbabel son of Shealtiel,' declares the \divine{Lord}, `and I will set you in place like a signet ring. For I have chosen you,' declares the \divine{Lord} of the Heavenly Armies.''

\bookheader{Zechariah}
\labelbook{Zech}

\bookpretitle{The Book of the Prophet}
\booktitle{Zechariah}

\labelchapt{1}
\passage{A Call to Return}

\chapt{1}
\v{1}In the eighth month of the second year\fnote{\fbackref{1:1} I.e. sometime from mid-October to mid-November 520 BC} of the reign of\fnote{\fbackref{1:1} The Heb. lacks \fbib{the reign of}} Darius, this message from the \divine{Lord} came to Berechiah's son Zechariah,\fnote{\fbackref{1:1} The Heb. name \fbib{Zechariah} means \fbib{the \divine{Lord} remembers}} the grandson of Iddo the prophet: \v{2}``The \divine{Lord} was very angry with your ancestors. \v{3}So tell them, `This is what the \divine{Lord} of the Heavenly Armies says: ``Return to me,'' declares the \divine{Lord} of the Heavenly Armies, ``and I will return to you.\fnote{\fbackref{1:3} Lit. \fbib{you,'' declares the \divine{Lord} of the Heavenly Armies}} \v{4}Don't be like your ancestors, to whom the former prophets proclaimed: `This is what the \divine{Lord} of the Heavenly Armies says: ``It's time to turn from your evil lifestyles\fnote{\fbackref{1:4} Lit. \fbib{paths}} and from your evil actions,'' `but they would neither listen nor pay attention to me,'\,'' declares the \divine{Lord}.' \v{5}``Your ancestors---where are they? And the prophets---do they live forever? \v{6}But my words and my statutes that I gave as commands to my servants the prophets---did they not overwhelm your ancestors? And they returned to me:\fnote{\fbackref{1:6} The Heb. lacks \fbib{to me}} `The \divine{Lord} of the Heavenly Armies acted toward us just as he planned to do---in keeping with our lifestyles\fnote{\fbackref{1:6} Lit. \fbib{paths}} and in keeping with our actions.'\,''
\passage{The Vision of Horses}

\v{7}On the twenty-fourth day of the eleventh month (the month Shebat) in the second year of the reign of\fnote{\fbackref{1:7} The Heb. lacks \fbib{the reign of}} Darius, this message from the \divine{Lord} came to Berechiah's son Zechariah, the grandson of Iddo the prophet: \v{8}``I stared into the night, and there was a man mounted on a red horse! The horse\fnote{\fbackref{1:8} Lit. \fbib{He}} was standing among the myrtle trees in a ravine. Behind him there were red, brown,\fnote{\fbackref{1:8} Or \fbib{sorrel}} and white horses.''

\v{9}Then I asked, ``Who are these, sir?''\fnote{\fbackref{1:9} Lit. \fbib{My lord}}

The messenger who was talking to me answered, ``I will tell you who these are.''

\v{10}The man who stood among the myrtle trees answered, ``These are the ones whom the \divine{Lord} sent out to wander throughout the earth.''

\v{11}Then they reported to the angel of the \divine{Lord} who stood among the myrtle trees, ``We have wandered throughout the earth---and look!---the entire earth is at rest. Everything is quiet and peaceful.''\fnote{\fbackref{1:11} Lit. \fbib{is rest and remains}}

\v{12}And the angel of the \divine{Lord} replied, ``\divine{Lord} of the Heavenly Armies, how long will it be until you show mercy to Jerusalem and to the cities of Judah, with whom you have been angry for these past seventy years?''

\v{13}So the \divine{Lord} answered the angel who was speaking to me with kind and comforting words.
\passage{The \divine{Lord}'s Concern for Zion}

\v{14}Then the angel who was speaking to me told me, ``Announce this: `This is what the \divine{Lord} of the Heavenly Armies says: ``I have a deep concern for Jerusalem, a great concern for Zion. \v{15}I am deeply angry with the nations who are complacent, with whom I was only a little displeased---but they made things worse!'' \v{16}`Therefore this is what the \divine{Lord} says: ``I have returned to Jerusalem with compassionate intentions. My Temple will be rebuilt there,'' declares the \divine{Lord} of the Heavenly Armies, ``and the measuring line will be stretched out over Jerusalem.''\,'\,''
\passage{The Future Prosperity of Zion}

\v{17}``Also announce the following: `This is what the \divine{Lord} of the Heavenly Armies says: ``My cities will again overflow with prosperity. The \divine{Lord} will comfort Zion once more and will choose Jerusalem again.''\,'\,''
\passage{The Vision of Four Horns}

\v{18}\fnote{\fbackref{1:18} This v. is 2:1 in MT, 1:19 is 2:2 in MT, and so through v. 21}Then I looked up and saw four horns. \v{19}I asked the angel who was talking to me, ``What are those?''

So he answered me, ``Those are the forces\fnote{\fbackref{1:19} Lit. \fbib{horns}} that have dispersed Judah, Israel, and Jerusalem.''
\passage{The Vision of Four Artisans}

\v{20}Then the \divine{Lord} showed me four artisans.

\v{21}Then I asked, ``What have they come to do?''

He answered, ``Those horns that dispersed Judah---so that no one could lift up his head---those artisans\fnote{\fbackref{1:21} The Heb. lacks \fbib{artisans}} are coming to disrupt the power\fnote{\fbackref{1:21} Lit. \fbib{to terrify the horns}} of nations, tearing them down now that they've come to power and dispersed the land of Judah.''
\labelchapt{2}
\passage{The Vision of the Measuring Line}

\chapt{2}
\v{1}\fnote{\fbackref{2:1} This v. is 2:5 in MT, and so throughout the chapter}Then I looked up and saw a man with a measuring line in his hand. \v{2}I asked, ``Where are you going?''

He responded, ``To measure Jerusalem in order to determine its width and length.''

\v{3}Look! That angel who was talking to me left, and another angel came forward to meet him. \v{4}That other angel told him, ``Run and tell that young man: `Jerusalem will be an inhabited city without walls due to the number of people and livestock within it. \v{5}I myself will be an encircling rampart of fire,' declares the \divine{Lord}, `and I will be the glory in her midst.'

\v{6}```Come now! Come now! Flee from the land of the north,' declares the \divine{Lord}, `for I have dispersed you like the four winds of heaven,' declares the \divine{Lord}.

\v{7}```Come now, Zion! Escape, you who are living with the residents of Babylon. \v{8}For this is what the \divine{Lord} of the Heavenly Armies says: ``In pursuit of glory I was sent to the nations who plundered you, because whoever injures you injures the pupil of my eye. \v{9}And look, I will shake my fist over them, and they will become plunder for their former\fnote{\fbackref{2:9} The Heb. lacks \fbib{former}} slaves. And you will know that the \divine{Lord} of the Heavenly Armies sent me.''\,'\,''
\passage{The \divine{Lord} will Live in Zion}

\v{10}``Sing and rejoice, daughter of Zion! Take note! I am coming to live in your midst,'' declares the \divine{Lord}. \v{11}``Many nations will cling to the \divine{Lord} at that time\fnote{\fbackref{2:11} Lit. \fbib{day}} and will become my people. I will live in your midst, and you will know that the \divine{Lord} of the Heavenly Armies has sent me to you. \v{12}The \divine{Lord} will take possession of Judah as his own property in the Holy Land---and he will choose Jerusalem again. \v{13}Be silent, every living thing, in the presence of the \divine{Lord}, because he is emerging\fnote{\fbackref{2:13} Lit. \fbib{awakened}} from his Holy Place.''
\labelchapt{3}
\passage{The Vision of the High Priest}

\chapt{3}
\v{1}Then I saw Joshua the High Priest standing in the presence of the angel of the \divine{Lord}, with Satan\fnote{\fbackref{3:1} The Heb. name \fbib{Satan} means \fbib{The Opponent} or \fbib{The Accuser}} standing at his right to oppose him.

\v{2}The \divine{Lord} told Satan, ``The \divine{Lord} rebuke you, Satan---in fact, may the \divine{Lord} who has chosen Jerusalem rebuke you! This man is a burning brand plucked from the fire, is he not?''

\v{3}Now Joshua was wearing filthy clothes as he stood in the presence of the angel.

\v{4}So the angel\fnote{\fbackref{3:4} Lit. \fbib{So he}} continued to tell those who were standing in his presence, ``Remove his filthy clothes.''

And he told Joshua,\fnote{\fbackref{3:4} Lit. \fbib{him}} ``Look how I've removed your iniquity. Now I'm clothing you with fine garments.''

\v{5}Then I said, ``Let them place a pure diadem\fnote{\fbackref{3:5} Or \fbib{clean turban}} on his head.''

So they placed the pure diadem\fnote{\fbackref{3:5} Or \fbib{clean turban}} on his head and clothed him with fine garments while the angel of the \divine{Lord} was standing beside them.\fnote{\fbackref{3:5} The Heb. lacks \fbib{beside them}}
\passage{The \divine{Lord}'s Charge to Joshua}

\v{6}Then the angel of the \divine{Lord} gave this charge to Joshua: \v{7}``This is what the \divine{Lord} of the Heavenly Armies says: `If you will live according to my ways, and if you will keep what I have entrusted to you,\fnote{\fbackref{3:7} The Heb. lacks \fbib{to you}} then you will also administer my Temple, and you will also guard my courtyards. And I will give you access to these who serve\fnote{\fbackref{3:7} Lit. \fbib{stand}} here.

\v{8}```Listen, High Priest Joshua, you and those companions of yours who sit with you,\fnote{\fbackref{3:8} Lit. \fbib{sit in your presence}} for these men are a sign that I am presenting my servant, the Branch.\fnote{\fbackref{3:8} Cf. Ps 132:17; Jer 23:5; 33:15} \v{9}Look, the stone that I put in place in Joshua's presence---on that one stone are seven eyes.\fnote{\fbackref{3:9} Or \fbib{facets}} And look, I will do the engraving myself,' declares the \divine{Lord} of the Heavenly Armies, `and I will remove the perversity of that land in a single day. \v{10}At that time,'\fnote{\fbackref{3:10} Lit. \fbib{day}} declares the \divine{Lord} of the Heavenly Armies, `you will invite each of your neighbors to join you\fnote{\fbackref{3:10} The Heb. lacks \fbib{to join you}} under the vine and fig tree.'\,''
\labelchapt{4}
\passage{The Vision of the Golden Menorah}

\chapt{4}
\v{1}Then the angel who had been speaking with me returned and woke me up as if I had been asleep. \v{2}Then he asked me, ``What do you see?''

So I said, ``I have been watching---and look!---there is a menorah made completely of gold with a bowl on top of it. And there are seven lights on it, along with seven feeder channels to the lamps, which are also on top of it. \v{3}Two olive trees are near it, one on the right side of the bowl and one on the left.''

\v{4}Then I asked the angel who had been speaking with me, ``Sir,\fnote{\fbackref{4:4} Lit. \fbib{My lord}} what are these?''

\v{5}The angel who had been speaking with me answered by asking, ``You don't know what these mean, do you?''

So I responded, ``No, sir.''
\passage{The \divine{Lord}'s First Charge to Zerubbabel}

\v{6}Then he replied to me, ``This is this message from the \divine{Lord} to Zerubbabel: `Not by valor nor by strength, but only by my Spirit,' says the \divine{Lord} of the Heavenly Armies. \v{7}`Who are you, great mountain? You will become a plain in Zerubbabel's presence, and he will position the capstone, exulting over it, ``How beautiful! How beautiful!''\,'\,''
\passage{The \divine{Lord}'s Second Charge to Zerubbabel}

\v{8}Then this message from the \divine{Lord} came to me again: \v{9}``Zerubbabel's hands have laid the foundation of this Temple, and his hands will finish it, so that you will know that the \divine{Lord} of the Heavenly Armies has sent me to all of you. \v{10}For who has despised the time\fnote{\fbackref{4:10} Lit. \fbib{day}} of insignificant things? They will rejoice to see the plumb line in the hand of Zerubbabel.\fnote{\fbackref{4:10} I.e. building the temple} These seven lights\fnote{\fbackref{4:10} The Heb. lacks \fbib{lights}} represent the eyes of the \divine{Lord}, which are looking throughout all of the earth.''
\passage{The Two Anointed Ones}

\v{11}Then I asked the angel,\fnote{\fbackref{4:11} Lit. \fbib{him}} ``What are these two olive trees, one\fnote{\fbackref{4:11} The Heb. lacks \fbib{one}} on the right of the menorah and one\fnote{\fbackref{4:11} The Heb. lacks \fbib{one}} on the left?'' \v{12}I also asked him a second question: ``What are these two olive branches on either side of\fnote{\fbackref{4:12} Lit. \fbib{that are at the hand of}} the two golden feeder channels that carry the golden oil to the seven lights?''\fnote{\fbackref{4:12} The Heb. lacks \fbib{oil to the seven lights}}

\v{13}The angel\fnote{\fbackref{4:13} Lit. \fbib{He}} replied, ``You don't know what these are, do you?''

I responded to him, ``No, sir.''

\v{14}He said, ``These are the two anointed ones,\fnote{\fbackref{4:14} Lit. \fbib{sons of fresh oil}} who stand continuously beside the Lord of the whole earth.''
\labelchapt{5}
\passage{The Vision of the Flying Scroll}

\chapt{5}
\v{1}Then I looked up and saw a flying scroll! \v{2}And the angel\fnote{\fbackref{5:2} Lit. \fbib{he}} asked me, ``What do you see?''

I answered him, ``I'm looking at a flying scroll. It's 20 cubits\fnote{\fbackref{5:2} I.e. about 30 feet; a cubit was about eighteen inches} long and ten cubits\fnote{\fbackref{5:2} I.e. about fifteen feet; a cubit was about eighteen inches} wide.''

\v{3}He responded to me, ``This is the curse that is going out over the surface of the whole earth, because, according to this side of the scroll,\fnote{\fbackref{5:3} The Heb. lacks \fbib{side of the scroll}} all thieves will be ejected, and according to the other side of the scroll,\fnote{\fbackref{5:3} The Heb. lacks \fbib{side of the scroll}} all liars will be ejected.''

\v{4}``I am bringing this about,'' declares the \divine{Lord} of the Heavenly Armies. ``The curse\fnote{\fbackref{5:4} Lit. \fbib{it}} will enter the house of the thief and the house of the one who lies in my name. The curse\fnote{\fbackref{5:4} Lit. \fbib{It}} will remain in his house until that house\fnote{\fbackref{5:4} Lit. \fbib{it}} has been completely destroyed, right down to its timber and stones.''
\passage{The Vision of the Basket}

\v{5}Then the angel who had been talking with me stepped forward and told me, ``Please look up and see what's going out.''

\v{6}So I asked, ``What is it?''\fnote{\fbackref{5:6} Lit. \fbib{is she}}

He replied, ``This is a basket\fnote{\fbackref{5:6} Lit. \fbib{ephah}; i.e. a measure of dry grain about 20 dry quarts in capacity; and so throughout the chapter} making its appearance.'' He also said, ``This is what it appears to be in\fnote{\fbackref{5:6} Lit. \fbib{is the eyes in}} the entire land.''
\passage{The Vision of the Woman in the Basket}

\v{7}Look, a round lead cover was being lifted, and there was one woman seated inside the basket! \v{8}And the angel\fnote{\fbackref{5:8} Lit. \fbib{he}} said, ``This is evil!'' So he shoved her back into the basket and snapped the round, lead cover over the opening.
\passage{The Vision of the Two Winged Women}

\v{9}Then I looked up to see two women coming forward with the wind filling their wings. (They had wings like those of a stork.) They took up the basket, holding it between the earth and sky.

\v{10}So I asked the angel who was talking to me, ``Where are they taking the basket?''

\v{11}He answered me, ``To the land of Shinar,\fnote{\fbackref{5:11} I.e. the Babylon area} so they can build a temple to the woman in the basket.\fnote{\fbackref{5:11} Lit. \fbib{to her}} Then when its preparations are complete, the basket\fnote{\fbackref{5:11} Lit. \fbib{complete, it}} will be set there in its place.''
\labelchapt{6}
\passage{The Vision of the Four Chariots}

\chapt{6}
\v{1}Then I looked up and saw four chariots coming out from between two mountains! And the mountains were made of brass! \v{2}Attached to the first chariot were red horses. Attached to the second chariot were black horses. \v{3}Attached to the third chariot were white horses. Attached to the fourth chariot were speckled horses and gray\fnote{\fbackref{6:3} Or strong} horses. \v{4}In response, I asked the angel who had been talking with me, ``Sir, what are these?''

\v{5}The angel told me, ``These are four heavenly spirits that are going out on patrol after having presented themselves to the Lord of the whole earth. \v{6}The black horses are headed into the north country, and the white ones are headed out after them. The speckled horses are headed toward the south country.''

\v{7}Then the gray horses went out. They were eager to go out on patrol throughout the earth, so the angel\fnote{\fbackref{6:7} Lit. \fbib{he}} said, ``Go patrol the earth.'' So they went out on patrol throughout the earth.

\v{8}Then he called to me, ``Look! The horses that went north have caused my spirit to rest in the north country.''
\passage{The Future Prosperity of the Branch}

\v{9}Then this message from the \divine{Lord} came to me: \v{10}``Go take up an offering\fnote{\fbackref{6:10} The Heb. lacks \fbib{an offering}} from those who came out of\fnote{\fbackref{6:10} The Heb. lacks \fbib{those who came out of}} the Babylonian\fnote{\fbackref{6:10} The Heb. lacks \fbib{Babylonian}} exile, that is, from Heldai, from Tobijah, and from Jedaiah. Go along with them today into the house of Zephaniah's son Josiah, who returned from Babylon. \v{11}Take silver and gold and fashion crowns to set upon the head of Joshua son of Johozadak, the High Priest. \v{12}Then tell him, `This is what the \divine{Lord} of the Heavenly Armies says: ``Here is the man whose name is The Branch.\fnote{\fbackref{6:12} i.e. Zerubbabel; cf. Zech 3:8} He will branch out from where he is and will rebuild the Temple of the \divine{Lord}. \v{13}Yes, he will indeed rebuild the Temple of the \divine{Lord}, and he will exalt its majesty by sitting and ruling on his throne. He will serve as priest on his throne, and no contention\fnote{\fbackref{6:13} Lit. \fbib{and peaceful counsel}} will exist between them. \v{14}The crowns will go to Helem, to Tobijah, to Jedaiah, and to Zephaniah's son Hen, as a memorial in the Temple of the \divine{Lord}. \v{15}Those who are now far away will come and do reconstruction work in the Temple of the \divine{Lord}. Then you will know that the \divine{Lord} of the Heavenly Armies has sent me to you. This will come about if you diligently obey the voice of the \divine{Lord} your God.''\,'\,''
\labelchapt{7}
\passage{A Rebuke about Selfish Fasts}

\chapt{7}
\v{1}During the fourth year of the reign of\fnote{\fbackref{7:1} The Heb. lacks \fbib{the reign of}} King Darius, a message from the \divine{Lord} came to Zechariah on the fourth day of the ninth month Kislev.\fnote{\fbackref{7:1} I.e. c. 7 December 518 BC} \v{2}The people of\fnote{\fbackref{7:2} The Heb. lacks \fbib{The people of}} Bethel were sending\fnote{\fbackref{7:2} Lit. \fbib{was sending}} Sharezer, Regem-melech, and their men to pray in the \divine{Lord}'s presence \v{3}and to speak to the priests assigned\fnote{\fbackref{7:3} The Heb. lacks \fbib{assigned}} to the Temple of the \divine{Lord} of the Heavenly Armies along with the prophets, asking, ``Am I to go about mourning, denying myself throughout the fifth month,\fnote{\fbackref{7:3} I.e. the anniversary month of the Jerusalem temple's destruction} as I have these many years?''

\v{4}Then this message from the \divine{Lord} of the Heavenly Armies came to me: \v{5}``Talk to everyone in the land, as well as to the priests. Ask them, `When you were fasting and mourning during the fifth and seventh months\fnote{\fbackref{7:5} The Heb. lacks \fbib{months}} for the past seventy years, were you really fasting for me? \v{6}And when you eat and drink, you're eating and drinking for your own benefit, aren't you? \v{7}Isn't this what the \divine{Lord} proclaimed through the former prophets, when a prosperous Jerusalem was inhabited, as were its surrounding cities, the Negev,\fnote{\fbackref{7:7} I.e. the southern regions of the Sinai peninsula; cf. Josh 10:40} and the Shephelah?'\,''\fnote{\fbackref{7:7} I.e. the verdant central lowlands of Israel; cf. Josh 10:40}
\passage{The Consequence of Turning from God}

\v{8}This message from the \divine{Lord} came to Zechariah again: \v{9}``This is what the \divine{Lord} of the Heavenly Armies says: `Administer true justice, and show gracious love and mercy toward each other.\fnote{\fbackref{7:9} Lit. \fbib{toward his brother}} \v{10}You are not to wrong the widow, orphans, the foreigner, or the poor, and you are not to plan evil against each other.\fnote{\fbackref{7:10} Lit. \fbib{toward his brother}} \v{11}But they refused to pay attention, turned their backs, and stopped listening. \v{12}They made their hearts hard like a diamond, to keep from obeying the Law and the messages that the \divine{Lord} of the Heavenly Armies sent by his Spirit through the former prophets. \v{13}Therefore, just as when I\fnote{\fbackref{7:13} Lit. \fbib{he}} cried out and they would not listen, so also they will cry out, and I will not listen,' says the \divine{Lord} of the Heavenly Armies. \v{14}`I will scatter them to all of the nations, which they have not known.'\,''

Now the earth was left desolate after them. As a result, no one came or went because they had turned a pleasant land into a desert.
\labelchapt{8}
\passage{What the \divine{Lord} will Do for Zion}

\chapt{8}
\v{1}This is this message from the \divine{Lord} of the Heavenly Armies:

\v{2}``This is what the \divine{Lord} of the Heavenly Armies says: `I'm greatly jealous about Zion, and that makes me furious about her.'

\v{3}``This is what the \divine{Lord} says: `I will return to Zion and I will live in the midst of Jerusalem. And Jerusalem will be called, ``The City of Truth'' and the mountain of the \divine{Lord} of the Heavenly Armies will be called,\fnote{\fbackref{8:3} The Heb. lacks \fbib{will be called}} ``The Holy Mountain''.'

\v{4}``This is what the \divine{Lord} of the Heavenly Armies says: `There will yet be old men and old women sitting in the parks\fnote{\fbackref{8:4} Or \fbib{streets}} of Jerusalem, each one of them holding canes in their hands due to their old age! \v{5}The city parks\fnote{\fbackref{8:5} Or \fbib{streets}} will be filled with boys and girls. They will play in the city's\fnote{\fbackref{8:5} Lit. \fbib{in its}} open parks.'

\v{6}``This is what the \divine{Lord} of the Heavenly Armies says: `It may seem impossible to the survivors of this people, but is it impossible for me?' declares the \divine{Lord} of the Heavenly Armies.

\v{7}``This is what the \divine{Lord} of the Heavenly Armies says: `Look! After having saved my people from the land of the east and from the land of the west, \v{8}I will also bring them back to live in the midst of Jerusalem. They will be my people and I will truly be their righteous God.'\,''
\passage{An Exhortation to Rebuild the Temple}

\v{9}``This is what the \divine{Lord} of the Heavenly Armies says: `Be strong so the Temple can be built, you who are now listening to this message spoken by the prophets when the foundation was laid to the Temple of the \divine{Lord} of the Heavenly Armies. \v{10}Before then, everyone was unemployed---even the draft animals---and no one was safe coming or going because of the enemy, because I caused everyone to oppose each other.\fnote{\fbackref{8:10} Lit. \fbib{I set everyone against his neighbor}}

\v{11}```But now I will not treat the survivors of this people as I did formerly,' declares the \divine{Lord} of the Heavenly Armies. \v{12}`For there will be a sowing of peace: the vine will produce its fruit, the earth will produce its full\fnote{\fbackref{8:12} The Heb. lacks \fbib{full}} yield, the sky will produce its dew, and I will make the survivors of this people inherit all these things. \v{13}Furthermore, house of Judah and house of Israel, even though you used to be a curse among the nations, now I will surely save you, and you will be a blessing. Stop being afraid. Instead, grow stronger.'\,''\fnote{\fbackref{8:13} Lit. \fbib{your hands become strong.}}
\passage{The \divine{Lord} will Do Good to Jerusalem and Judah}

\v{14}``This is what the \divine{Lord} of the Heavenly Armies says: `Just as I intended to bring harm to you when your ancestors angered me,' says the \divine{Lord} of the Heavenly Armies, `and I did not relent, \v{15}so I have decided at this time\fnote{\fbackref{8:15} Lit. \fbib{at these days}} to do good things for Jerusalem and for the house of Judah. So stop being afraid. \v{16}Here's what you must do: tell the truth to your neighbors, administer true and peaceful justice in your courtrooms,\fnote{\fbackref{8:16} Lit. \fbib{gates}} \v{17}don't plot evil in your heart against a neighbor, and don't tolerate\fnote{\fbackref{8:17} Lit. \fbib{and love no}} false testimonies,\fnote{\fbackref{8:17} Or \fbib{oaths}} because I hate all these things,' declares the \divine{Lord}.''

\v{18}Once again this message from the \divine{Lord} come to me: \v{19}``This is what the \divine{Lord} of the Heavenly Armies says: `The fasts that occur in the fourth, fifth, seventh, and tenth months will be joyful and glad times for the house of Judah, replete with\fnote{\fbackref{8:19} The Heb. lacks \fbib{replete with}} cheerful festivals. Therefore, love truth and peace.'\,''
\passage{Future Visits by Nations to Jerusalem}

\v{20}``This is what the \divine{Lord} of the Heavenly Armies says: `In the future, people will come, including residents of many cities, \v{21}and they will travel from one place to another place and say, ``Let's go quickly to pray in the presence of the \divine{Lord} and to inquire of the \divine{Lord} of the Heavenly Armies.' And I will go, too. \v{22}Many people and powerful nations will come to inquire of the \divine{Lord} of the Heavenly Armies in Jerusalem, and to pray in the presence of the \divine{Lord}.''\,'\,''

\v{23}``This is what the \divine{Lord} of the Heavenly Armies says: `In the future, ten men speaking\fnote{\fbackref{8:23} The Heb. lacks \fbib{speaking}} all the languages of the nations will grab hold of one Jewish person by the hem of his garment and say, ``Let us go up to Jerusalem\fnote{\fbackref{8:23} The Heb. lacks \fbib{to Jerusalem}} with you, because we heard that God is with you.''\,'\,''
\labelchapt{9}
\passage{Judgment on Israel's Enemies}

\begin{poetry}
\poeml \chapt{9}
\v{1}A declaration: this message from the \divine{Lord} in the land of Hadrach,\fnote{\fbackref{9:1} I.e. a district near Damascus} \\
\poemll    with Damascus its confederate,\fnote{\fbackref{9:1} Cf. Isa 7:2; or \fbib{resting place}} \\
\poeml because the eyes of humanity\fnote{\fbackref{9:1} Or \fbib{Aram}} will look to\fnote{\fbackref{9:1} Lit. \fbib{will be on}} the \divine{Lord}, \\
\poemll    along with those of\fnote{\fbackref{9:1} The Heb. lacks \fbib{those of}} all the tribes of Israel. \\
\poeml \v{2}Also Hamath, which borders on it--- \\
\poemll    along with Tyre and Sidon--- \\
\poemlll       indeed they are very wise. \\
\poeml \v{3}``Tyre built itself a fortification, \\
\poemll    hoarding silver like dust \\
\poemlll       and pure gold like mud in a street. \\
\poeml \v{4}Look! The Lord will evict her, \\
\poemll    striking at her power in the sea, \\
\poemlll       and she will be consumed by fire. \\
\poeml \v{5}Ashkelon will see it happen and will be terrified; \\
\poemll    Gaza will tremble greatly. \\
\poeml Ekron will be ashamed of her expectations, \\
\poemll    Gaza's king will perish, \\
\poemlll       and Ashkelon will become uninhabited. \\
\poeml \v{6}A strange people\fnote{\fbackref{9:6} Lit. \fbib{A bastard}} will inhabit Ashdod, \\
\poemll    and I will eliminate the arrogance of Philistia. \\
\poeml \v{7}I will remove the blood from its mouth, \\
\poemll    along with its abominations from between its teeth. \\
\poeml Its survivors will be dedicated to our God; \\
\poemll    It will be like a clan of Judah, \\
\poemlll       and Ekron will be as a Jebusite.\fnote{\fbackref{9:7} I.e. a Jerusalemite} \\
\poeml \v{8}I will set a garrison around my Temple, \\
\poemll    to hinder those who might come and go, \\
\poeml and to guard against oppressors who intend to invade; \\
\poemll    for I have taken note of this with my eyes.''
\passage{The Coming of Zion's King}
\poeml \v{9}``Rejoice greatly, daughter of Zion; \\
\poemll    cry out, daughter of Jerusalem! \\
\poeml Look! Your king is coming to you. \\
\poemll    He is righteous, \\
\poemlll       and he is able to save. \\
\poeml He is humble, \\
\poemll    and is riding on a colt, \\
\poemlll       the foal of a donkey. \\
\poeml \v{10}I will banish\fnote{\fbackref{9:10} Lit. \fbib{will cut off}} chariots from Ephraim \\
\poemll    and horses from Jerusalem. \\
\poeml War weapons\fnote{\fbackref{9:10} Lit. \fbib{The war bow}} will be banished, \\
\poemll    and your king\fnote{\fbackref{9:10} Lit. \fbib{and he}} will speak peace to the nations. \\
\poeml His dominion will extend from sea to sea, \\
\poemll    and from the River to the farthest portion of the earth. \\
\poeml \v{11}Now concerning you and my blood covenant with you, \\
\poemll    I have liberated your prisoners \\
\poemlll       from a waterless pit. \\
\poeml \v{12}Return to your fortress, you prisoners who have hope. \\
\poemll    Even today I am telling you: \\
\poemlll       In return I will repay you double. \\
\poeml \v{13}For I have bent Judah as if it were my bow, \\
\poemll    loading it with Ephraim. \\
\poeml I raised up your sons, Zion, \\
\poemll    against your sons, Greece, \\
\poemlll       wielding you like a mighty warrior's sword.''
\passage{Israel's \divine{Lord} Appears}
\poeml \v{14}The \divine{Lord} will appear over them--- \\
\poemll    his arrow will shoot like lightning. \\
\poeml The Lord \divine{God} will blow the trumpet, \\
\poemll    and go out with the southern windstorm. \\
\poeml \v{15}The \divine{Lord} of the Heavenly Armies will defend them; \\
\poemll    they will devour and conquer those who sling stones. \\
\poeml They will drink and be boisterous like those who are drunk. \\
\poemll    They will be filled to the brim with blood,\fnote{\fbackref{9:15} The Heb. lacks \fbib{with blood}} \\
\poemlll       like the corners of the altar. \\
\poeml \v{16}The \divine{Lord} their God will save them at that time\fnote{\fbackref{9:16} Lit. \fbib{day}} \\
\poemll    as his very own flock of people, \\
\poeml because they are his crown jewels, \\
\poemll    exalted throughout his land. \\
\poeml \v{17}For how great is his goodness, \\
\poemll    and how great is his beauty! \\
\poeml Grain will make the young men thrive, \\
\poemll    and new wine the virgins.
\end{poetry}
\labelchapt{10}
\passage{The \divine{Lord}'s Care for Judah}

\begin{poetry}
\poeml \chapt{10}
\v{1}``Ask the \divine{Lord} for rain in the spring\fnote{\fbackref{10:1} Lit. \fbib{the season of the latter rain}; i.e. the March and April rains}--- \\
\poemll    the \divine{Lord} who fashions lightning thunderstorms, \\
\poeml giving rain showers to mankind\fnote{\fbackref{10:1} Lit. \fbib{them}} \\
\poemll    along with grain in the fields.\fnote{\fbackref{10:1} Lit. \fbib{fields to a man}} \\
\poeml \v{2}Truly the family idols\fnote{\fbackref{10:2} Lit. \fbib{the teraphim;} i.e. household images apparently used for divination} talk nonsense \\
\poemll    and the diviners discern lies, \\
\poemlll       describing delusional dreams. \\
\poeml Since their comfort is vacuous, \\
\poemll    they wander off on their own like sheep, \\
\poemlll       because there is no shepherd. \\
\poeml \v{3}``Against the shepherds my anger rises--- \\
\poemll    I am punishing the leaders\fnote{\fbackref{10:3} Lit. \fbib{rams}} also, \\
\poeml because the \divine{Lord} of the Heavenly Armies has visited his flock, the house of Judah, \\
\poemll    appointing them as his royal war horse for battle. \\
\poeml \v{4}From them arises\fnote{\fbackref{10:4} The Heb. lacks \fbib{arises}} the cornerstone and tent peg, \\
\poemll    from them the battle bow, \\
\poemlll       from them arise all sorts of oppressive rulers. \\
\poeml \v{5}They will be like mighty soldiers \\
\poemll    who trample mud in the streets during battle. \\
\poeml They will fight because the \divine{Lord} is with them, \\
\poemll    and the opposing\fnote{\fbackref{10:5} The Heb. lacks \fbib{opposing}} horsemen will be confused. \\
\poeml \v{6}``I will fortify the house of Judah, \\
\poemll    and the house of Joseph I will save. \\
\poeml I will surely bring them back, \\
\poemll    because I care about them. \\
\poeml They will be as if I had never cast them away. \\
\poemll    Since I am the \divine{Lord} their God, \\
\poemlll       I will answer them. \\
\poeml \v{7}``The people of Ephraim will become like mighty soldiers; \\
\poemll    they will be glad, like those who have wine. \\
\poeml Their children will see this and rejoice; \\
\poemll    their hearts will find joy in the Lord. \\
\poeml \v{8}I will whistle for them, gathering them together, \\
\poemll    because I have redeemed them, \\
\poemlll       and they will multiply as they were before.\fnote{\fbackref{10:8} Lit. \fbib{will increase as they increased}} \\
\poeml \v{9}I will scatter them among the nations, \\
\poemll    and so they will remember me in distant countries. \\
\poeml They will rear their children, \\
\poemll    and they will return. \\
\poeml \v{10}I will bring them once again out of the land of Egypt, \\
\poemll    gathering them from Assyria. \\
\poeml I will bring them into the land of Gilead and Lebanon, \\
\poemll    but there will not be enough space for them. \\
\poeml \v{11}They\fnote{\fbackref{10:11} So LXX. MT reads \fbib{He will}} will pass through the sea of affliction, \\
\poemll    and they\fnote{\fbackref{10:11} So LXX. MT reads \fbib{He will}} will strike the waves\fnote{\fbackref{10:11} I.e. perhaps reminiscent of Elijah striking the Jordan River; cf. 2Kings 2:8} in that sea. \\
\poeml All of the depths of the Nile will evaporate, \\
\poemll    Assyria's arrogance will be brought down low, \\
\poemlll       and the ruling power\fnote{\fbackref{10:11} Lit. \fbib{the scepter}} of Egypt will disappear. \\
\poeml \v{12}``I will strengthen them in the \divine{Lord}, \\
\poemll    and they will march here and there in his name,'' \\
\poemlll       declares the \divine{Lord}.
\end{poetry}
\labelchapt{11}
\passage{Destruction of Lebanon and Bashan}

\begin{poetry}
\poeml \chapt{11}
\v{1}Open your doors, Lebanon, \\
\poemll    and fire will consume your cedars.\fnote{\fbackref{11:1} I.e. a genus of coniferous evergreen in the family \fbib{Pinaceae}; and so throughout the chapter} \\
\poeml \v{2}Wail, cypress tree, \\
\poemll    for the cedar has fallen \\
\poemlll       while the stately trees are destroyed. \\
\poeml Wail, oak trees of Bashan, \\
\poemll    for the old growth forest has been cut down. \\
\poeml \v{3}Hear\fnote{\fbackref{11:3} The Heb. lacks \fbib{Hear}} the wailing of the shepherds, \\
\poemll    for the magnificence of the forest\fnote{\fbackref{11:3} The Heb. lacks \fbib{of the forest}} is ruined! \\
\poeml Hear\fnote{\fbackref{11:3} The Heb. lacks \fbib{Hear}} the roar of the lions, \\
\poemll    for the Jordan's arrogance is ruined!
\end{poetry}
\passage{The Shepherd}

\v{4}This is what the \divine{Lord} my God says: ``Shepherd the flock marked for\fnote{\fbackref{11:4} Lit. \fbib{flock of the}} slaughter. \v{5}Their buyers slaughter them without being punished, continuing to sell them as they say, `Bless the \divine{Lord}!' and, `I'm rich!' Meanwhile, their shepherds show them no compassion. \v{6}Therefore I will no longer show compassion upon those who live in the land,'' declares the \divine{Lord}. ``Look! I will deliver every single person into the control\fnote{\fbackref{11:6} Lit. \fbib{hand}} of his neighbor and into the control\fnote{\fbackref{11:6} Lit. \fbib{hand}} of the king. Even though they assault the land, I will not deliver it from their control.''\fnote{\fbackref{11:6} Lit. \fbib{hand}}

\v{7}So I became shepherd of the flock marked for\fnote{\fbackref{11:7} Lit. \fbib{flock of the}} slaughter, paying attention to the oppressed of the flock. I took two staffs---naming one ``Pleasant'' and the other one ``Union''---and then I pastured the flock. \v{8}In a single month I got rid of three shepherds because I grew tired of them, and they despised me. \v{9}So I said, ``I will no longer be your shepherd. Let those who are about to die perish, and let what is about to be destroyed be destroyed. As for the survivors, let them devour each other.''

\v{10}Then I took the staff that I had named ``Pleasant'' and broke it, showing I was breaking my covenant that I had made with all of the people. \v{11}It was broken at that time\fnote{\fbackref{11:11} Lit. \fbib{day}} so the oppressed of the flock who were observing me would know that it had been a message from the \divine{Lord}.

\v{12}I told them, ``If it's alright with you, pay me what I've earned. But if it isn't, don't.''

So they paid out what I had earned---30 pieces of silver.\fnote{\fbackref{11:12} The Heb. lacks a specific unit of measurement}

\v{13}Then the \divine{Lord} told me, ``Throw the money\fnote{\fbackref{11:13} The Syr reads \fbib{Throw it}} into the treasury\fnote{\fbackref{11:13} According to Syr}---that magnificent value they placed on me!''

So I took the 30 shekels of\fnote{\fbackref{11:13} The Heb. lacks a specific unit of measurement} silver and threw them into the treasury\fnote{\fbackref{11:13} According to Syr} of the Temple of the \divine{Lord}. \v{14}Then I broke my second staff---the one I had named ``Union''---breaking the union between the house of Judah and the house of Israel.
\passage{God's Curse on the Worthless Shepherd}

\v{15}The \divine{Lord} told me, ``Pick up the tools of a worthless shepherd again, \v{16}for I am now raising up a shepherd in the land who will neither search for the lost, nor care for the young, nor fix the broken, nor sustain the healthy. Instead, he will devour the meat of the best of the sheep, tearing off their hoofs.''

\begin{poetry}
\poeml \v{17}``Woe to the worthless shepherd, \\
\poemll    who deserts the flock! \\
\poeml May the sword strike his arm \\
\poemll    and his right eye. \\
\poeml May his arm wither \\
\poemll    and his right eye be completely blind.''
\end{poetry}
\labelchapt{12}
\passage{The \divine{Lord} will Deliver Jerusalem}

\chapt{12}
\v{1}A declaration: a message from the \divine{Lord} to Israel. ``The \divine{Lord}, who stretches out the heavens, who lays the foundation of the earth, and who frames the spirit of man within himself, declares, \v{2}`Look, I am making Jerusalem an unstable cup\fnote{\fbackref{12:2} I.e. one that cannot be set down without spilling its contents} toward all of its surrounding armies when they lay siege against Judah and Jerusalem. \v{3}It will come about at that time\fnote{\fbackref{12:3} Lit. \fbib{day}} that I will make Jerusalem a heavy weight; so everyone who burdens themselves with it will be crushed,\fnote{\fbackref{12:3} Lit. \fbib{sliced}} even though all of the nations of the earth gather themselves against it. \v{4}At that time,'\fnote{\fbackref{12:4} Lit. \fbib{day}} declares the \divine{Lord}, `I will strike every horse with panic and every rider with insanity. I will keep my eyes on the house of Judah, but I will blind every horse of the invading\fnote{\fbackref{12:4} The Heb. lacks \fbib{invading}} armies. \v{5}The leaders of Judah will say to themselves, ``Those who live in Jerusalem are my strength through the \divine{Lord} of the Heavenly Armies, their God.'' \v{6}`At that time,\fnote{\fbackref{12:6} Lit. \fbib{day}} I will make the leaders of Judah like a brazier filled with blazing wood, or like a torch setting fire to harvested grain. They will devour all the invading\fnote{\fbackref{12:6} Lit. \fbib{surrounding}} armies, both on the right hand and on the left. As a result, Jerusalem will again be inhabited in its rightful place---as the real\fnote{\fbackref{12:6} The Heb. lacks \fbib{the real}} Jerusalem.'\,''

\v{7}The \divine{Lord} will deliver the tents of Judah first, so that neither the glory of the house\fnote{\fbackref{12:7} Or \fbib{family}} of David nor the glory of the inhabitants of Jerusalem overshadows Judah. \v{8}At that time,\fnote{\fbackref{12:8} Lit. \fbib{day}} the \divine{Lord} will defend those who live in Jerusalem, and the one who is feeble among them at that time will be like David. The entire house of David will be like God---indeed, like the angel of the \divine{Lord} in their midst!

\v{9}```At that time,\fnote{\fbackref{12:9} Lit. \fbib{day}} I will search out and destroy all of the nations who have come against Jerusalem. \v{10}I will pour out on the house of David and on the residents of Jerusalem a spirit of grace and of supplications, and they will look to me---the one whom they pierced.'\,''\fnote{\fbackref{12:10} The quotation possibly continues through 13:1}
\passage{Mourning in Jerusalem}

Then they will mourn for him, as for an only son. They will grieve bitterly for him, as for a firstborn son. \v{11}At that time,\fnote{\fbackref{12:11} Lit. \fbib{day}} Jerusalem will mourn deeply, like the mourning about Hadad-rimmon\fnote{\fbackref{12:11} Cf. 2Chr 35:20-25} that took place in the plain of Megiddo. \v{12}And so the land will mourn, families by families, alone by themselves---the family of the house of David by itself with their wives by themselves, the family of the house of Nathan by itself with their wives by themselves, \v{13}the family of the house of Levi by itself with their wives by themselves, the family of Shimei by itself with their wives by themselves---\v{14}all of the surviving families by themselves, along with their wives by themselves.\chapt{13}
\v{1}At that time,\fnote{\fbackref{13:1} Lit. \fbib{day}} a fountain will be opened for the house of David and for those who live in Jerusalem so they can be cleansed from\fnote{\fbackref{13:1} The Heb. lacks \fbib{so they can be cleansed from}} sin and ceremonial impurity.
\labelchapt{13}
\passage{Cessation of Prophecy}

\v{2}``At that time,''\fnote{\fbackref{13:2} Lit. \fbib{day}} declares the \divine{Lord} of the Heavenly Armies, ``I will eliminate the names of the idols from the land, and they will not be remembered anymore. I will also force both prophet and demon\fnote{\fbackref{13:2} Lit. \fbib{and unclean spirit}} to leave the land. \v{3}It will also come about that if any man would dare to\fnote{\fbackref{13:3} Lit. \fbib{would still}} prophesy, then his father and his mother who bore him will respond to him, `You will not live, because you are speaking lies in the name of the \divine{Lord}.' Then his father and mother who bore him will stab him for prophesying. \v{4}Furthermore, it will come about at that time\fnote{\fbackref{13:4} Lit. \fbib{day}} that every prophet will become ashamed of his vision as he prophesies. They will wear no rough garments intended to deceive others.''\fnote{\fbackref{13:4} The Heb. lacks \fbib{others}}
\passage{The Injured Servant of Mankind}

\v{5}``He\fnote{\fbackref{13:5} MT subject is third sing.; i.e. the \fbib{\divine{Lord}.}} will say, `I am no mere\fnote{\fbackref{13:5} The Heb. lacks \fbib{mere}} prophet. A servant of mankind am I, because a man dedicated to this\fnote{\fbackref{13:5} The Heb. lacks \fbib{to this}} have I been from my youth.'

\v{6}``Someone will say to him, `What are these injuries to your hands?'

``He will reply, `{\ldots}what I received at my friend's house.'

\begin{poetry}
\poeml \v{7}``Arise, sword, against my shepherd, \\
\poemll    against the mighty one who is related to me,'' \\
\poemlll       declares the \divine{Lord} of the Heavenly Armies. \\
\poeml ``Strike the shepherd, \\
\poemll    the sheep will be scattered, \\
\poemlll       and I will turn against the insignificant ones. \\
\poeml \v{8}It will come about in all of the land,'' \\
\poemll    declares the \divine{Lord}, \\
\poeml ``that two thirds of the people living there will die, \\
\poemll    but a third will survive who live there. \\
\poeml \v{9}And I will bring that surviving third through, \\
\poemll    testing them as if through fire, \\
\poeml purifying them like silver, \\
\poemll    assaying them like gold. \\
\poeml They will call on my name, \\
\poemll    and I will answer them. \\
\poeml I will say, `This is my people,' \\
\poemll    and they will say, `The \divine{Lord} is my God.'\,''
\end{poetry}
\labelchapt{14}
\passage{The \divine{Lord} Comes to the Mount of Olives}

\chapt{14}
\v{1}``Look! A day is coming for the \divine{Lord}, when your plunder will be divided among you. \v{2}I will gather all the nations against Jerusalem, to lay siege against it. The city will be captured, the houses will be ransacked, the women raped, and half of the city will go into exile, but the remaining people will not be cut off from the city. \v{3}Then the \divine{Lord} will go out to battle against those nations, waging war as in a day of battle. \v{4}His feet will stand in that day on the Mount of Olives, east of Jerusalem. Then the Mount of Olives will be split in two from east to west, forming\fnote{\fbackref{14:4} The Heb. lacks \fbib{forming}} a very large valley, with half of the mountain moving toward the north and half toward the south. \v{5}You will run away through my mountain valley, because the valley of the mountains will extend as far as Azal. You will flee, as you fled from the earthquake during the reign of King Uzziah of Judah. And so the \divine{Lord} my God will come, and all his holy ones will be accompanying you.''
\passage{A Unique Day}

\v{6}``At that time,\fnote{\fbackref{14:6} Lit. \fbib{day}} the daylight will be neither bright nor overcast. \v{7}It will be a unique day, known only to the \divine{Lord}---neither daytime nor nighttime---and it will come about at twilight there will be light! \v{8}At that time,\fnote{\fbackref{14:8} Lit. \fbib{day}} flowing waters will run perennially\fnote{\fbackref{14:8} Lit. \fbib{waters in summer and in winter will be}} from Jerusalem, half toward the Dead\fnote{\fbackref{14:8} Lit. \fbib{eastern}} Sea and half to the Mediterranean\fnote{\fbackref{14:8} Lit. \fbib{western}} Sea. \v{9}The \divine{Lord} will be king over all the earth at that time.\fnote{\fbackref{14:9} Lit. \fbib{day}} There\fnote{\fbackref{14:9} Or \fbib{earth. At that time, there}} will be one \divine{Lord}, and his name the only one. \v{10}The entire land will become like the Arabah plain from Geba\fnote{\fbackref{14:10} I.e. an ancient city about 6 miles northeast of Jerusalem} to Rimmon, south of Jerusalem. It will be raised up and inhabited where it is, from the Gate of Benjamin to the First Gate, then to the Corner Gate, to the Hananel Tower, and to the king's winepresses. \v{11}People\fnote{\fbackref{14:11} Lit. \fbib{They}} will live there, there will be no more destruction, and Jerusalem will be safely inhabited.''
\passage{God's Judgment on Jerusalem's Attackers}

\v{12}``This will be the plague with which the \divine{Lord} inflicts all of the people who have attacked Jerusalem: he will cause their flesh to rot away, even while they're standing on their feet. He will cause their eyes to rot away in their sockets, and their tongues to rot away in their mouths. \v{13}At that time,\fnote{\fbackref{14:13} Lit. \fbib{day}} they will be stricken with a terrible panic from the \divine{Lord}. Everyone will attack each other. \v{14}Judah, too, will fight at Jerusalem. Then the wealth of the surrounding nations will be gathered up: gold, silver, and clothing in great abundance. \v{15}A similar plague will also strike horses, mules, camels, donkeys, and all of the animals in those camps.''
\passage{Discipline of the Nations}

\v{16}``It will come about that all of the survivors of the nations who came against Jerusalem will come there from year to year to worship the King, the \divine{Lord} of the Heavenly Armies, and to observe the Festival of Tents. \v{17}If anyone from the families of the earth will not come to Jerusalem to worship the King, the \divine{Lord} of the Heavenly Armies, there will be no rain for them. \v{18}If the people of Egypt do not come to Jerusalem\fnote{\fbackref{14:18} The Heb. lacks \fbib{to Jerusalem}} to take part, they will have no annual Nile overflow.\fnote{\fbackref{14:18} The Heb. lacks \fbib{annual Nile overflow}} A plague will come from the \divine{Lord} to strike the nations who do not come to observe the Festival of Tents. \v{19}This will be the punishment for Egypt and all nations who do not come to observe the Festival of Tents.''
\passage{Holiness to the \divine{Lord}}

\v{20}``At that time,\fnote{\fbackref{14:20} Lit. \fbib{day}} there will be written on the bells of the horses:

\divine{Holiness to the Lord}

and the pots in the Temple of the \divine{Lord} will be like the bowls in front of the altar---\v{21}every pot in Jerusalem and in Judah will be consecrated to the \divine{Lord} of the Heavenly Armies. Everyone who offers sacrifices will come, will take them, and will cook in them. Furthermore, at that time,\fnote{\fbackref{14:21} Lit. \fbib{day}} there will no longer be a Canaanite in the Temple of the \divine{Lord} of the Heavenly Armies.''

\bookheader{Malachi}
\labelbook{Mal}

\bookpretitle{The Book of the Prophet}
\booktitle{Malachi}

\labelchapt{1}
\passage{God's First Complaint: Against His People---A Despised Love}

\chapt{1}
\v{1}A declaration: a message from the \divine{Lord} to Israel by Malachi.\fnote{The Heb. name \fbib{Malachi} means \fbib{My messenger}}

\v{2}``I've loved you,'' says the \divine{Lord}. ``But you ask, `How have you loved us?'

``Was not Esau Jacob's brother?'' declares the \divine{Lord}, ``yet I loved Jacob, \v{3}rejected\fnote{Lit. \fbib{hated}} Esau, turned his mountains into a wasteland, and gave\fnote{The Heb. lacks \fbib{gave}} his inheritance to desert jackals. \v{4}Even though Edom may claim, `We were crushed, but we will return and rebuild the ruins,' this is what the \divine{Lord} of the Heavenly Armies says: ``They may rebuild, but I'll demolish. People\fnote{Lit. \fbib{they}} will call them, `The Wicked Land,' and, `The People With Whom the \divine{Lord} is Forever Angry.' \v{5}Your own eyes will see this, and you will say, `Great is the \divine{Lord} even beyond the borders of Israel!'\,''
\passage{God's Second Complaint: Against His Priests---A Despised Offering}

\v{6}``A son honors his father and a servant his master. So if I'm a father, where is my honor? And if I'm a master, where is my respect?'' says the \divine{Lord} of the Heavenly Armies to you priests who are despising my name. ``But you ask, `How have we despised your name?' \v{7}By presenting defiled food on my altar. And you ask, `How have we defiled you?' By saying, `The Table of the \divine{Lord} is contemptible.' \v{8}When you bring blind animals for sacrifice, is that not wrong? And when you sacrifice crippled or diseased animals,\fnote{The Heb. lacks \fbib{animals}} is that not wrong? Offer that to your governor---would he be pleased with you or receive you favorably?'' asks the \divine{Lord} of the Heavenly Armies. \v{9}``And now, go ahead and implore God by saying, `Be gracious to us.' Will he receive you favorably and accept offerings like that from your hand?'' asks the \divine{Lord} of the Heavenly Armies.
\passage{Useless Offerings and Useless Altar Fires}

\v{10}``Oh, that one of you would shut the Temple\fnote{The Heb. lacks \fbib{temple}} doors and not light useless fires on the altar! I'm not pleased with you,'' says the \divine{Lord} of the Heavenly Armies, ``and I'll accept no offerings from you.\fnote{Lit. \fbib{from your hands}} \v{11}Even so, from where the sun rises to where it sets my name will be great among the Gentiles. Incense will be brought to me\fnote{Lit. \fbib{to my name}} everywhere, along with pure offerings, because my name will be great among the Gentiles,'' says the \divine{Lord} of the Heavenly Armies. \v{12}``But you are profaning my name\fnote{Lit. \fbib{it}} by saying that the Table of the \divine{Lord} is defiled and that its fruit and its food are contemptible.

\v{13}``And you say, `What a burden!' and sniff contemptuously at it,'' says the \divine{Lord} of the Heavenly Armies, ``when you present maimed, crippled, and diseased animals,\fnote{The Heb. lacks \fbib{animals}} and when you bring the offering. Should I accept this from your hand?'' asks the \divine{Lord}. \v{14}``Cursed is the deceiver who has an acceptable\fnote{The Heb. lacks \fbib{acceptable}} male in his flock, and vows to give it,\fnote{The Heb. lacks \fbib{to give it}} but sacrifices a mutilated one to the \divine{Lord}. Indeed, I am a great king,'' says the \divine{Lord} of the Heavenly Armies, ``and my name is feared among the Gentiles.''
\labelchapt{2}
\passage{God's Third Complaint: Against His Priests---Failing to Honor Him}

\chapt{2}
\v{1}Now this commandment is for you priests: \v{2}``If you don't listen, and if you don't choose\fnote{Lit. \fbib{don't set your hearts}} to give honor to my name,'' says the \divine{Lord} of the Heavenly Armies, ``then I'll curse both you and your blessings.\fnote{Lit. \fbib{I'll send on you the curse and I'll curse your blessings}} I've even cursed them\fnote{Lit. \fbib{it}} already, because none of you are taking it to heart. \v{3}Look! I'm rebuking your descendants because of you, and I'll spread waste\fnote{I.e. the parts of the sacrificial animal that were discarded after the offering had been presented.} on your faces, the waste of your festival sacrifices, and you will be carried off with it.

\v{4}``You will know that I sent this commandment to you in order to continue my covenant with Levi,'' says the \divine{Lord} of the Heavenly Armies. \v{5}``My covenant with him was for life and peace, and I gave the commandments\fnote{Lit. \fbib{them}} to him so he would fear me. He did fear me as he stood in my presence.\fnote{Lit. \fbib{in the presence of my name}} \v{6}True teachings were in his mouth, and falsehood was not found on his lips. He walked with me peacefully and righteously, and he turned many from sin. \v{7}For the lips of a priest should preserve knowledge, and people should seek instruction from his mouth, because he's the messenger of the \divine{Lord} of the Heavenly Armies.

\v{8}``But you priests\fnote{The Heb. lacks \fbib{priests}} turned aside from the way, and by your teaching you caused many to stumble. You have violated the covenant of Levi,'' says the \divine{Lord} of the Heavenly Armies. \v{9}``So I also made you despised, humiliating you before all of the people, because you aren't following my ways and are showing partiality when you teach.''
\passage{A Plea and a Prayer from Malachi}

\v{10}Do we not have one father?\fnote{The reference is either to Abraham as ancestor of Israel or to God as heavenly Father.} Has not one God created us? Why does each of us act deceitfully, each man against his own brother, to profane the covenant of our ancestors? \v{11}Judah has become unfaithful, and a detestable thing was committed in Israel and Jerusalem. Indeed, Judah profaned the Holy Place of the \divine{Lord}, which he loves, and married a daughter of a foreign god. \v{12}May the \divine{Lord} exclude from the community\fnote{Lit. \fbib{tents}} of Jacob any man who does this, whoever he may be,\fnote{Lit. \fbib{he who awakens and he who answers}} even though he brings offerings to the \divine{Lord} of the Heavenly Armies.
\passage{God's Fourth Complaint: Against His Priests---Marital Abuses}

\v{13}``This is another thing you do: you flood the altar of the \divine{Lord} with tears, weeping and wailing because he no longer pays attention to your offering nor takes pleasure in it from your hand. \v{14}Yet you ask, `For what reason?' Because the \divine{Lord} acts as a witness between you and the wife of your youth, because you were unfaithful to her, your partner, the wife of your covenant. \v{15}Did he not make them\fnote{The Heb. lacks \fbib{them}} one? And the vestige of the spirit remains in him. And why did he make them one? He was seeking godly offspring. So guard yourselves in your spirit, and don't be unfaithful to the wife of your youth.

\v{16}``Indeed, the \divine{Lord} God of Israel says that he hates divorce, along with the one who conceals his violence by outward appearances,''\fnote{Lit. \fbib{by his garments}} says the \divine{Lord} of the Heavenly Armies. ``So guard yourselves carefully,\fnote{Lit. \fbib{yourselves in your spirit}} and don't be unfaithful.''
\passage{God's Fifth Complaint: Against His People---Complaining about God.}

\v{17}``You have wearied the \divine{Lord} with your words. You ask, `How have we wearied you?' By your saying, `All who do evil are good in the eyes of the \divine{Lord} and he's pleased with them,' or `Where is the God of justice?'\,''
\labelchapt{3}
\passage{The Coming of the Messenger}

\chapt{3}
\v{1}``Watch out! I'm sending my messenger, and he will prepare the way before me. Then suddenly, the \divine{Lord} you are looking for will come to his Temple. He is the messenger of the covenant whom you desire. Watch out! He is coming!'' says the \divine{Lord} of the Heavenly Armies.

\v{2}But who will survive the day when he comes? Or who can stand when he appears? Because he's like a refiner's fire and a launderer's soap, \v{3}he will sit refining and purifying silver, purifying the descendants of Levi, refining them like gold and silver. Then they'll bring a righteous offering to the \divine{Lord}. \v{4}Then the offering to the \divine{Lord} by Judah and Jerusalem will be acceptable as it was in the past, even as in former years.
\passage{The Judgment of God}

\v{5}``I'll come near to you for judgment. I'll be a witness, quick to speak against sorcerers, against adulterers, against those who swear falsely, against those who defraud the laborer of his wage, against those who defraud\fnote{The Heb. lacks \fbib{against those who defraud}} the widow and the orphan, against those who deprive the alien of justice, and against those who don't fear me,'' says the \divine{Lord} of the Heavenly Armies. \v{6}``Because I the \divine{Lord} don't change; therefore you children of Jacob aren't destroyed.''
\passage{God's Sixth Complaint: Against His People---Gifts and Offerings}

\v{7}``Ever since the time of your ancestors, you have turned away from my decrees and haven't kept them. Return to me and I'll return to you,'' says the \divine{Lord} of the Heavenly Armies. ``But you ask, `How will we return?' \v{8}``Will a person rob God? Yet you are robbing me! But you ask, `How are we robbing you?' ``By the tithe and the offering. \v{9}You are cursed under the curse---the entire nation---because you are robbing me!

\v{10}``Bring the entire tithe into the storehouse that there may be food in my house. So put me to the test in this right now,'' says the \divine{Lord} of the Heavenly Armies, ``and see if I won't throw open the windows\fnote{Or \fbib{floodgates}} of heaven for you and pour out on you blessing without measure. \v{11}And I'll prevent the devourer from harming you,\fnote{Lit. \fbib{I'll rebuke the devourer from you}} so that he does not destroy the crops of your land. Nor will the vines in your fields drop their fruit,'' says the \divine{Lord} of the Heavenly Armies.

\v{12}``Then all the nations will call you blessed, for you will be a land of delight,'' says the \divine{Lord} of the Heavenly Armies.
\passage{God's Seventh Complaint: Against His People---Slandering God}

\v{13}``You have spoken arrogant words against me,'' says the \divine{Lord}. ``Yet you ask, `What did we say against you?' \v{14}You said, `It is futile to serve God,' and, `What did we get out of it\fnote{Lit. \fbib{What gain}} when we carried out his requirements and went about like mourners in the presence of the \divine{Lord} of the Heavenly Armies?' \v{15}and, `Now we call the arrogant one blessed. Those who do evil prosper and those who challenge God escape the consequences.'\,''\fnote{The Heb. lacks \fbib{the consequences}}
\passage{The Repentance of the Righteous}

\v{16}Then those who feared the \divine{Lord} talked with each other. The \divine{Lord} listened and took note,\fnote{Lit. \fbib{heard}} and a scroll of remembrance was written in his presence about those who fear the \divine{Lord} and honor His name. \v{17}``They'll be mine,'' says the \divine{Lord} of the Heavenly Armies, ``in the day when I prepare my treasured possession. I'll spare them, just as a man spares his own son who serves him. \v{18}When you return, you will see the difference between the righteous and the wicked, and between the one who serves God and the one who does not.''
\labelchapt{4}
\passage{The Great Day of the \divine{Lord}\passagenote{Chap 4:1-6 is chapter 3:19-24 in MT}}

\chapt{4}
\v{1}``The coming day\fnote{I.e. the Day of the \fbib{\divine{Lord}}} is certainly going to burn like a furnace. All the arrogant and all who practice evil will be stubble---the coming day will set them on fire,'' says the \divine{Lord} of the Heavenly Armies, ``so that it will leave them neither root nor branch. \v{2}But the Sun of Righteousness will arise with healing in its light\fnote{Lit. \fbib{wings}; i.e. its sun beams} for those who fear my name. You will go out and leap like calves released\fnote{The Heb. lacks \fbib{released}} from their stalls \v{3}and trample down the wicked. Indeed, they will become ashes under the soles of your feet on the day I do this,'' says the \divine{Lord} of the Heavenly Armies.
\passage{The Coming of Elijah the Prophet}

\v{4}``Remember the Law of Moses my servant that I gave him at Horeb for all Israel---both the decrees and laws.

\v{5}``Pay attention! I'm sending Elijah the prophet to you before the great and dreadful Day of the \divine{Lord} comes, \v{6}and he will turn the hearts of fathers to children, and the hearts of children to their fathers. Otherwise, I'll come, strike the land, and utterly destroy it.''

\newpage
\def\thetestament{New Testament}
\addcontentsline{toc}{part}{New Testament}
\addcontentsline{toc}{chapter}{The Manifestation of the Good News}
\bookheader{Matthew}
\labelbook{Matt}

\bookpretitle{The Gospel According to}
\booktitle{Matthew}

\labelchapt{1}
\passage{An Introduction to Jesus the Messiah}

\chapt{1}
\v{1}This is\fnote{\fbackref{1:1} The Gk. lacks \fbib{This is}} a record of the life\fnote{\fbackref{1:1} Or \fbib{birth}} of Jesus the Messiah, the son of David, the son of Abraham.
\passage{The Ancestry of Jesus}
\passageinfo{(Luke 3:23-28)}

\v{2}Abraham fathered Isaac, Isaac fathered Jacob, and Jacob fathered Judah and his brothers. \v{3}Judah fathered Perez and Zerah by Tamar, Perez fathered Hezron, Hezron fathered Aram, \v{4}Aram fathered Amminadab, Amminadab fathered Nahshon, and Nahshon fathered Salmon. \v{5}Salmon fathered Boaz by Rahab, Boaz fathered Obed by Ruth, Obed fathered Jesse, \v{6}and Jesse fathered King David.

David fathered Solomon by the wife of Uriah, \v{7}Solomon fathered Rehoboam, Rehoboam fathered Abijah, Abijah fathered Asaph,\fnote{\fbackref{1:7} Other mss. read \fbib{Asa}; cf. 1Chr 3:10} \v{8}Asaph\fnote{\fbackref{1:8} Other mss. read \fbib{Asa}; cf. 1Chr 3:10} fathered Jehoshaphat, Jehoshaphat fathered Joram, Joram fathered Uzziah, \v{9}Uzziah fathered Jotham, Jotham fathered Ahaz, Ahaz fathered Hezekiah, \v{10}Hezekiah fathered Manasseh, Manasseh fathered Amos,\fnote{\fbackref{1:10} Other mss. read \fbib{Amon}; cf. 1Chr 3:13} and Amos\fnote{\fbackref{1:10} Other mss. read \fbib{Amon}; cf. 1Chr 3:13} fathered Josiah. \v{11}Josiah fathered Jechoniah\fnote{\fbackref{1:11} Or \fbib{Jeconiah}; cf. Jer 22:28-30} and his brothers at the time of the deportation to Babylon.

\v{12}After the deportation to Babylon, Jechoniah\fnote{\fbackref{1:12} Or \fbib{Jeconiah}; cf. Jer 22:28-30} fathered Salathiel,\fnote{\fbackref{1:12} Or \fbib{Shealtiel}; cf. 1Chr 3:17; Ezra 3:2,8; 5:2; Neh 12:1; Luke 3:27} Salathiel fathered Zerubbabel, \v{13}Zerubbabel fathered Abiud, Abiud fathered Eliakim, Eliakim fathered Azor, \v{14}Azor fathered Zadok, Zadok fathered Achim, Achim fathered Eliud, \v{15}Eliud fathered Eleazar, Eleazar fathered Matthan, and Matthan fathered Jacob. \v{16}Jacob fathered Joseph, the husband of Mary, who was the mother of Jesus,\fnote{\fbackref{1:16} Lit. \fbib{of whom Jesus was born}} who is called the Messiah.\fnote{\fbackref{1:16} Or \fbib{Christ}}

\v{17}So all the generations from Abraham to David totaled fourteen\fnote{\fbackref{1:17} I.e. the numerical value of the Heb. name \fbib{David}} generations, and from David to the deportation to Babylon totaled fourteen generations, and from the deportation to Babylon to the Messiah\fnote{\fbackref{1:17} Or \fbib{Christ}} there were fourteen generations.
\passage{The Birth of Jesus}
\passageinfo{(Luke 2:1-7)}

\v{18}Now the birth of Jesus the Messiah\fnote{\fbackref{1:18} Or \fbib{Christ}} happened in this way. When his mother Mary was engaged\fnote{\fbackref{1:18} Engagement involved a legally binding promise of marriage.} to Joseph, before they lived together she was discovered to be pregnant by the Holy Spirit. \v{19}Her husband Joseph, being a righteous man and unwilling to disgrace her, decided to divorce her secretly.

\v{20}After he had thought about it, an angel of the Lord appeared to him in a dream. ``Joseph, son of David,'' he said, ``don't be afraid to take Mary as your wife, because what has been conceived in her is from the Holy Spirit. \v{21}She will give birth to a son, and you are to name him Jesus,\fnote{\fbackref{1:21} The name \fbib{Jesus} means \fbib{The Lord saves}} because he is the one who will save his people from their sins.''

\v{22}Now all this happened to fulfill what was declared by the Lord through the prophet when he said,

\begin{poetry}
\poeml \v{23}``See, a virgin will become pregnant \\
\poemll    and give birth to a son, \\
\poemlll       and they will name him Immanuel,''\fnote{\fbackref{1:23} Cf. Isa 7:14, citing a statement by the \fbib{}\divine{Lord}}
\end{poetry}

which means, ``God with us.'' \v{24}When Joseph got up from his sleep, he did as the angel of the Lord had commanded him and took Mary as\fnote{\fbackref{1:24} The Gk. lacks \fbib{Mary as}} his wife. \v{25}He did not have marital relations with\fnote{\fbackref{1:25} Lit. \fbib{not know}} her until she had given birth to a son;\fnote{\fbackref{1:25} Other mss. read \fbib{to her firstborn son}} and he named him Jesus.
\labelchapt{2}
\passage{The Visit of the Wise Men}

\chapt{2}
\v{1}After Jesus had been born in Bethlehem of Judea during the reign\fnote{\fbackref{2:1} Lit. \fbib{days}} of King Herod, wise men\fnote{\fbackref{2:1} Lit. \fbib{magoi} (Magi); i.e. Aramaic speaking wise men from Mesopotamia; or magi-astrologers; and so throughout the chapter; cf. Dan 1:4} arrived in Jerusalem from the east \v{2}and asked, ``Where is the one who was born king of the Jews? We saw his star in the east\fnote{\fbackref{2:2} Or \fbib{at its rising}} and have come to worship him.''

\v{3}When King Herod heard this, he was disturbed, as was all of Jerusalem. \v{4}He called together all the high priests and scribes of the people and asked them where the Messiah\fnote{\fbackref{2:4} Or \fbib{Christ}} was to be born. \v{5}They told him, ``In Bethlehem of Judea, because that is what was written by the prophet:

\begin{poetry}
\poeml \v{6}`O Bethlehem in the land of Judah, \\
\poemll    you are by no means least among the rulers of Judah, \\
\poeml because from you will come a ruler \\
\poemll    who will shepherd\fnote{\fbackref{2:6} Or \fbib{govern}} my people Israel.'\,''\fnote{\fbackref{2:6} Cf. Mic 5:2; 2 Sam 5:2}
\end{poetry}

\v{7}Then Herod secretly called together the wise men, found out from them the time the star had appeared, \v{8}and sent them to Bethlehem. He told them,\fnote{\fbackref{2:8} The Gk. lacks \fbib{them}} ``As you go, search carefully for the child. When you find him, tell me so that I, too, may go and worship him.''

\v{9}After listening to the king, they set out, and the star they had seen in the east\fnote{\fbackref{2:9} Or \fbib{at its rising}} went ahead of them until it came and stopped over the place where the child was. \v{10}When they saw the star, they were ecstatic with joy. \v{11}After they went into the house and saw the child with his mother Mary, they fell down and worshipped him. Then they opened their treasure sacks and offered him gifts of gold, frankincense, and myrrh. \v{12}Having been warned in a dream not to go back to Herod, they left for their own country by a different road.
\passage{The Escape to Egypt}

\v{13}After they had gone, an angel of the Lord appeared to Joseph in a dream. ``Get up, take the child and his mother, and flee to Egypt,'' he said. ``Stay there until I tell you, because Herod intends to search for the child and kill him.'' \v{14}So Joseph\fnote{\fbackref{2:14} Lit. \fbib{he}} got up, took the child and his mother, and left at night for Egypt. \v{15}He stayed there until Herod's death in order to fulfill what was declared by the Lord\fnote{\fbackref{2:15} MT source citation reads \fbib{}\divine{Lord}} through the prophet when he said, ``Out of Egypt I called my Son.''\fnote{\fbackref{2:15} Cf. Hos 11:1}
\passage{The Massacre of the Infants}

\v{16}Herod flew into a rage when he learned that he had been tricked by the wise men, so he ordered the execution of all the male children in Bethlehem and all its neighboring regions, who were two years old and younger, according to the time that he had determined from the wise men. \v{17}Then what was declared by the prophet Jeremiah was fulfilled when he said,

\begin{poetry}
\poeml \v{18}``A voice was heard in Ramah: \\
\poemll    wailing and great mourning. \\
\poeml Rachel was crying for her children. \\
\poemll    She refused to be comforted, \\
\poemlll       because they no longer existed.''\fnote{\fbackref{2:18} Cf. Jer 31:15}
\end{poetry}
\passage{The Return to Nazareth}

\v{19}But after Herod died, an angel of the Lord appeared in a dream to Joseph in Egypt. \v{20}``Get up,'' he said. ``Take the child and his mother, and go to the land of Israel, because those who were trying to kill\fnote{\fbackref{2:20} Lit. \fbib{were seeking the life of}} the child are dead.''

\v{21}So Joseph\fnote{\fbackref{2:21} Lit. \fbib{he}} got up, took the child and his mother, and went into the land of Israel. \v{22}But when he heard that Archelaus was ruling over Judea in place of his father Herod, he was afraid to go there, after having been warned in a dream. So he left for the region of Galilee \v{23}and settled in a town called Nazareth in order to fulfill what was said by the prophets: ``He will be called a Nazarene.''\fnote{\fbackref{2:23} The Gk. \fbib{Nazoraios} may be a word play between Heb. \fbib{netser,} meaning \fbib{branch} (cf. Isa 11:1), and the name \fbib{Nazareth.}}
\labelchapt{3}
\passage{John the Baptist Prepares the Way}
\passageinfo{(Mark 1:1-8; Luke 3:1-9, 15-17; John 1:19-28)}

\chapt{3}
\v{1}About this time,\fnote{\fbackref{3:1} Lit. \fbib{In those days}} John the Baptist appeared, preaching in the Judean wilderness, \v{2}``Repent, because the kingdom from\fnote{\fbackref{3:2} Lit. \fbib{of}} heaven is near!'' \v{3}He was the one the prophet Isaiah was referring to when he said,

\begin{poetry}
\poeml ``He is a voice calling out in the wilderness: \\
\poemll    `Prepare the way for the Lord!\fnote{\fbackref{3:3} MT and DSS source citations read \fbib{}\divine{Lord}} \\
\poemlll       Make his paths straight!'\,''\fnote{\fbackref{3:3} Cf. Isa 40:3}
\end{poetry}

\v{4}John had clothing made of camel's hair and wore\fnote{\fbackref{3:4} The Gk. lacks \fbib{wore}} a leather belt around his waist. His diet consisted of grasshoppers\fnote{\fbackref{3:4} Or \fbib{locust-shaped carob seed pods}} and wild honey.

\v{5}Then the people of\fnote{\fbackref{3:5} The Gk. lacks \fbib{the people of}} Jerusalem, all Judea, and the entire region along the Jordan began flocking to him, \v{6}being baptized by him in the Jordan River while they confessed their sins.

\v{7}But when John\fnote{\fbackref{3:7} Lit. \fbib{he}} saw many Pharisees and Sadducees coming to where he was baptizing,\fnote{\fbackref{3:7} Lit. \fbib{to his baptism}} he told them, ``You children of serpents! Who warned you to flee from the coming wrath? \v{8}Produce fruit that is consistent with repentance! \v{9}Don't think you can say to yourselves, `We have father Abraham!' because I tell you that God can raise up descendants for Abraham from these stones! \v{10}The ax already lies against the roots of the trees. So every tree that isn't producing good fruit will be cut down and thrown into the fire. \v{11}I am baptizing you with\fnote{\fbackref{3:11} Or \fbib{in}} water as evidence of repentance,\fnote{\fbackref{3:11} Lit. \fbib{for repentance}} but the one who is coming after me is stronger than I am, and I am not worthy to carry his sandals. It is he who will baptize you with\fnote{\fbackref{3:11} Or \fbib{in}} the Holy Spirit and fire. \v{12}His winnowing fork is in his hand. He will clean up his threshing floor and gather his grain into the barn, but he will burn the chaff with inextinguishable fire.''
\passage{Jesus is Baptized}
\passageinfo{(Mark 1:9-11; Luke 3:21-22)}

\v{13}Then Jesus came from Galilee to the Jordan to be baptized by John. \v{14}But John tried to stop him, saying, ``I need to be baptized by you, and are you coming to me?''

\v{15}But Jesus answered him, \red{``Let it be this way for now, because this is the proper way for us to fulfill all righteousness.''}\fnote{\fbackref{3:15} Cf. Dan 9:24}

At this, he permitted him to be baptized.\fnote{\fbackref{3:15} The Gk. lacks \fbib{to be baptized}} \v{16}When Jesus had been baptized, he immediately came up out of the water. Suddenly, the heavens opened up for him, and he saw the Spirit of God descending like a dove and coming to rest on him. \v{17}Then a voice from heaven said, ``This is my Son, whom I love. I am pleased with him!''
\labelchapt{4}
\passage{Jesus is Tempted by Satan}
\passageinfo{(Mark 1:12-13; Luke 4:1-13)}

\chapt{4}
\v{1}After this, Jesus was led by the Spirit into the wilderness to be tempted by the devil. \v{2}After fasting for 40 days and 40 nights, he finally became hungry.

\v{3}Then the tempter came. ``Since\fnote{\fbackref{4:3} Or ``\fbib{If, as is the case,}} you are the Son of God,'' he said, ``tell these stones to become loaves of bread.''

\v{4}But he answered, \red{``It is written,}

\begin{poetry}
\poeml \red{`One must not live on bread alone,} \\
\poemll    \red{but on every word coming} \\
\poemlll       \red{out of the mouth of God.'\,''}\fnote{\fbackref{4:4} Deut 8:3; MT source citation reads \fbib{}\divine{Lord}}
\end{poetry}

\v{5}Then the devil took him to the Holy City\fnote{\fbackref{4:5} I.e. Jerusalem} and had him stand on the highest point of the Temple. \v{6}He told Jesus,\fnote{\fbackref{4:6} Lit. \fbib{him}} ``Since\fnote{\fbackref{4:6} Or ``\fbib{If, as is the case,}} you are the Son of God, throw yourself down, because it is written,

\begin{poetry}
\poeml `God\fnote{\fbackref{4:6} Lit. \fbib{He}} will put his angels in charge of you,'
\end{poetry}

and,

\begin{poetry}
\poeml `With their hands they will hold you up, \\
\poemll    so that you will never hit your foot against a rock.'\,''\fnote{\fbackref{4:6} Cf. Ps 91:11-12}
\end{poetry}

\v{7}Jesus responded to him, \red{``It is also written, `You must not tempt the Lord\fnote{\fbackref{4:7} MT source citation reads \fbib{}\divine{Lord}} your God.'\,''}\fnote{\fbackref{4:7} Cf. Deut 6:16}

\v{8}Once more the devil took him to a very high mountain and showed him all the kingdoms of the world, along with their splendor. \v{9}He told Jesus,\fnote{\fbackref{4:9} Lit. \fbib{him}} ``I will give you all these things if you will bow down and worship me!''

\v{10}Then Jesus told him, \red{``Go away,\fnote{\fbackref{4:10} Other mss. read \fbib{Get behind me}} Satan! Because it is written,}

\begin{poetry}
\poeml \red{`You must worship the Lord your God} \\
\poemll    \red{and serve only him.'\,''}\fnote{\fbackref{4:10} Deut 6:13}
\end{poetry}

\v{11}So the devil left him, and angels came and began ministering to him.
\passage{Jesus Begins His Ministry in Galilee}
\passageinfo{(Mark 1:14-15; Luke 4:14-15)}

\v{12}Now when Jesus\fnote{\fbackref{4:12} Lit. \fbib{he}} heard that John had been arrested, he went back to Galilee. \v{13}He left Nazareth and settled in Capernaum by the sea, in the regions of Zebulun and Naphtali, \v{14}in order to fulfill what was declared by the prophet Isaiah when he said,

\begin{poetry}
\poeml \v{15}``O Land of Zebulun and Land of Naphtali, \\
\poemll    on the road to the sea, across the Jordan, \\
\poemlll       Galilee of the unbelievers!\fnote{\fbackref{4:15} Lit. \fbib{gentiles} ; i.e. unbelieving non-Jews} \\
\poeml \v{16}The people living\fnote{\fbackref{4:16} Lit. \fbib{sitting}} in darkness have seen a great light, \\
\poemll    and for those living\fnote{\fbackref{4:16} Lit. \fbib{sitting}} in the land and shadow of death, \\
\poemlll       a light has risen.''\fnote{\fbackref{4:16} Cf. Isa 9:1-2}
\end{poetry}

\v{17}From then on, Jesus began to announce, \red{``Repent, because the kingdom from\fnote{\fbackref{4:17} Lit. \fbib{of}} heaven is near!''}
\passage{Jesus Calls Four Fishermen}
\passageinfo{(Mark 1:16-20; Luke 5:1-11)}

\v{18}While Jesus\fnote{\fbackref{4:18} Lit. \fbib{he}} was walking beside the Sea of Galilee, he saw two brothers---Simon (also\fnote{\fbackref{4:18} The Gk. lacks \fbib{also}} called Peter) and his brother Andrew. They were casting a net into the sea, because they were fishermen. \v{19}\red{``Follow me,''} he told them,\red{ ``and I will make you fishers of people!''} \v{20}So at once they left their nets and followed him. \v{21}Going on from there he saw two other brothers---James, son of Zebedee, and his brother John. They were in a boat with their father Zebedee repairing their nets. When he called them, \v{22}they immediately left the boat and their father and followed him.
\passage{Jesus Ministers to Many People}
\passageinfo{(Luke 6:17-19)}

\v{23}Then he went throughout Galilee, teaching in their synagogues, proclaiming the gospel of the kingdom, and healing every disease and every illness among the people. \v{24}His fame spread throughout Syria, and people\fnote{\fbackref{4:24} Lit. \fbib{they}} brought to him everyone who was sick---those afflicted with various diseases and pains, the demon-possessed, the epileptics, and the paralyzed---and he healed them. \v{25}Large crowds from Galilee, the Decapolis,\fnote{\fbackref{4:25} Lit. \fbib{the Ten Cities,} a loose federation of ten cities strongly influenced by Greek culture} Jerusalem, Judea, and from across the Jordan followed him.
\labelchapt{5}
\passage{Jesus Teaches about the Kingdom}

\chapt{5}
\v{1}When Jesus\fnote{\fbackref{5:1} Lit. \fbib{he}} saw the crowds, he went up on the hill. After taking his seat, his disciples came to him, \v{2}and he began\fnote{\fbackref{5:2} Lit. \fbib{he opened his mouth and began}} to teach them:
\passage{The Blessed Attitudes}
\passageinfo{(Luke 6:20-23)}

\begin{poetry}
\poeml \v{3}\red{``How blessed are those who are destitute in spirit,} \\
\poemll    \red{because the kingdom from\fnote{\fbackref{5:3} Lit. \fbib{of}} heaven belongs to them!} \\
\poeml \v{4}\red{``How blessed are those who mourn,} \\
\poemll    \red{because it is they who will be comforted!} \\
\poeml \v{5}\red{``How blessed are those who are humble,}\fnote{\fbackref{5:5} Or \fbib{gentle}} \\
\poemll    \red{because it is they who will inherit the earth!} \\
\poeml \v{6}\red{``How blessed are those who are hungry and thirsty for righteousness,}\fnote{\fbackref{5:6} Or \fbib{justice}} \\
\poemll    \red{because it is they who will be satisfied!} \\
\poeml \v{7}\red{``How blessed are those who are merciful,} \\
\poemll    \red{because it is they who will receive mercy!} \\
\poeml \v{8}\red{``How blessed are those who are pure in heart,} \\
\poemll    \red{because it is they who will see God!} \\
\poeml \v{9}\red{``How blessed are those who make peace,} \\
\poemll    \red{because it is they who will be called God's children!} \\
\poeml \v{10}\red{``How blessed are those who are persecuted for righteousness' sake,} \\
\poemll    \red{because the kingdom from\fnote{\fbackref{5:10} Lit. \fbib{of}} heaven belongs to them!}
\end{poetry}

\v{11}\red{``How blessed are you whenever people\fnote{\fbackref{5:11} Lit. \fbib{they}} insult you, persecute you, and say all sorts of evil things against you falsely\fnote{\fbackref{5:11} Other mss. lack \fbib{falsely}} because of me!} \v{12}\red{Rejoice and be extremely glad, because your reward in heaven is great! That's how they persecuted the prophets who came before you.''}
\passage{Salt and Light in the World}
\passageinfo{(Mark 9:50; Luke 14:34-35)}

\v{13}\red{``You are the salt of the world. But if the salt should lose its taste, how can it be made salty again? It's good for nothing but to be thrown out and trampled on by people.}

\v{14}\red{``You are the light of the world. A city located on a hill can't be hidden.} \v{15}\red{People\fnote{\fbackref{5:15} Lit. \fbib{They}} don't light a lamp and put it under a basket but on a lamp stand, and it gives light to everyone in the house.} \v{16}\red{In the same way, let your light shine before people in such a way that they will see your good actions and glorify your Father in heaven.''}
\passage{Jesus Fulfills the Law and the Prophets}

\v{17}\red{``Do not think that I came to destroy the Law or the Prophets. I didn't come to destroy them, but to fulfill them,} \v{18}\red{because I tell all of you\fnote{\fbackref{5:18} The Gk. pronoun \fbib{you} is pl.} with certainty that until heaven and earth disappear, not one letter\fnote{\fbackref{5:18} Lit. \fbib{one iota}} or one stroke of a letter will disappear from the Law until everything has been accomplished.} \v{19}\red{So whoever sets aside\fnote{\fbackref{5:19} Or \fbib{breaks}} one of the least of these commandments and teaches others to do the same will be called least in the kingdom from\fnote{\fbackref{5:19} Lit. \fbib{of}} heaven. But whoever does them and teaches them will be called great in the kingdom from\fnote{\fbackref{5:19} Lit. \fbib{of}} heaven} \v{20}\red{because I tell you, unless your righteousness greatly exceeds that of the scribes and Pharisees, you will never enter the kingdom from\fnote{\fbackref{5:20} Lit. \fbib{of}} heaven!''}
\passage{Teaching about Anger}

\v{21}\red{``You have heard that it was told those who lived long ago, `You are not to commit murder,'\fnote{\fbackref{5:21} Cf. Exod 20:13; Deut 5:17} and, `Whoever murders will be subject to punishment.'}\fnote{\fbackref{5:21} Cf. Exod 21:12; Lev 24:17} \v{22}\red{But I say to you, anyone who is angry with his brother without a cause\fnote{\fbackref{5:22} Other mss. lack \fbib{without a cause}} will be subject to punishment. And whoever says to his brother `Raka!'\fnote{\fbackref{5:22} \fbib{Raka} is Aram. for \fbib{You worthless one}} will be subject to the Council.\fnote{\fbackref{5:22} Or \fbib{Sanhedrin}} And whoever says, `You fool!' will be subject to hell}\fnote{\fbackref{5:22} Lit. \fbib{Gehenna}; a Gk. transliteration of the Heb. for \fbib{Valley of Hinnom} fire.}

\v{23}\red{``So if you are presenting your gift at the altar and remember there that your brother has something against you,} \v{24}\red{leave your gift there before the altar and first go and be reconciled to your brother. Then come and offer your gift.} \v{25}\red{Come to terms quickly with your opponent while you are on the way to court,\fnote{\fbackref{5:25} Lit. \fbib{while you are with him on the way}} or your opponent may hand you over to the judge, and the judge to the guard, and you will be thrown into prison.} \v{26}\red{I tell you\fnote{\fbackref{5:26} The Gk. pronoun \fbib{you} is sing.} with certainty, you will not get out of there until you pay back the last dollar!''}\fnote{\fbackref{5:26} Lit. \fbib{quadran}; i.e. about 1/64\textsuperscript{th} of a daily wage for a common worker}
\passage{Teaching about Adultery}

\v{27}\red{``You have heard that it was said, `You are not to commit adultery.'}\fnote{\fbackref{5:27} Cf. Exod 20:14; Deut 5:18} \v{28}\red{But I say to you, anyone who stares at a woman with lust for her has already committed adultery with her in his heart.} \v{29}\red{So if your right eye causes you to sin, tear it out and throw it away. It is better for you to lose one of your body parts than to have your whole body thrown into hell.}\fnote{\fbackref{5:29} Lit. \fbib{Gehenna}; a Gk. transliteration of the Heb. for \fbib{Valley of Hinnom}} \v{30}\red{And if your right hand causes you to sin, cut it off and throw it away from you. It is better for you to lose one of your body parts than to have your whole body go into hell.''}\fnote{\fbackref{5:30} Lit. \fbib{Gehenna}; a Gk. transliteration of the Heb. for \fbib{Valley of Hinnom}}
\passage{Teaching about Divorce}
\passageinfo{(Matthew 19:1-12; Mark 10:1-12; Luke 16:18)}

\v{31}\red{``It was also said, `Whoever divorces his wife must give her a written notice of divorce.'}\fnote{\fbackref{5:31} Cf. Deut 24:1, 3} \v{32}\red{But I say to you, any man who divorces his wife, except for sexual immorality, causes her to commit adultery, and whoever marries a divorced woman commits adultery.''}
\passage{Teaching about Oaths}

\v{33}\red{``Again, you have heard that it was told those who lived long ago, `You must not swear an oath falsely,' but, `You must fulfill your oaths to the Lord.'}\fnote{\fbackref{5:33} Cf. Lev 19:12; Num 30:2; Deut 23:21-23} \v{34}\red{But I tell you not to swear at all, neither by heaven, because it is God's throne,} \v{35}\red{nor by the earth, because it is his footstool, nor by Jerusalem, because it is the city of the Great King.} \v{36}\red{Nor should you swear by your head, because you cannot make one hair white or black.} \v{37}\red{Instead, let your message be `Yes' for `Yes' and `No' for `No.' Anything more than that comes from the evil one.''}
\passage{Teaching about Retaliation}
\passageinfo{(Luke 6:29-30)}

\v{38}\red{``You have heard that it was said, `An eye for an eye and a tooth for a tooth.'}\fnote{\fbackref{5:38} Cf. Exod 21:24; Lev 24:20; Deut 19:21} \v{39}\red{But I tell you not to resist an evildoer. On the contrary, whoever slaps you on the right cheek, turn the other to him as well.} \v{40}\red{If anyone wants to sue you and take your shirt, let him have your coat as well.} \v{41}\red{And if anyone forces you to go one mile,\fnote{\fbackref{5:41} A Roman \fbib{milion} (mile) consisted of 1,000 paces, or about 1,611 yards} go two with him.} \v{42}\red{Give to the person who asks you for something, and do not turn away from the person who wants to borrow something from you.''}
\passage{Teaching about Love for Enemies}
\passageinfo{(Luke 6:27-28, 32-36)}

\v{43}\red{``You have heard that it was said, `You must love your neighbor'\fnote{\fbackref{5:43} Cf. Lev 19:18} and hate your enemy.} \v{44}\red{But I say to you, love your enemies, and pray for those who persecute you,} \v{45}\red{so that you will become children of your Father in heaven, because he makes his sun rise on both evil and good people, and he lets rain fall on the righteous and the unrighteous.} \v{46}\red{If you love those who love you, what reward will you have? Even the tax collectors do the same, don't they?} \v{47}\red{And if you greet only your relatives, that's no great thing you're doing, is it? Even the unbelievers\fnote{\fbackref{5:47} Lit. \fbib{to the gentiles} ; i.e. unbelieving non-Jews; other mss. read \fbib{the tax collectors}} do the same, don't they?} \v{48}\red{So be perfect,\fnote{\fbackref{5:48} Or \fbib{mature}} as your heavenly Father is perfect.''}\fnote{\fbackref{5:48} Or \fbib{mature}}
\labelchapt{6}
\passage{Teaching about Giving to the Poor}

\chapt{6}
\v{1}\red{``Be careful not to practice your righteousness in front of people in order to be noticed by them. If you do, you will have no reward from your Father in heaven}. \v{2}\red{So whenever you give to the poor, don't blow a trumpet before you like the hypocrites do in the synagogues and in the streets so that they will be praised by people. I tell all of you\fnote{\fbackref{6:2} The Gk. pronoun \fbib{you} is pl.} with certainty, they have their full reward!} \v{3}\red{But when you give to the poor, don't let your left hand know what your right hand is doing,} \v{4}\red{so that your giving may be done in secret. And your Father who sees in secret will reward you.''}\fnote{\fbackref{6:4} Other mss. read \fbib{reward you openly}}
\passage{Teaching about Prayer}
\passageinfo{(Luke 11:2-4)}

\v{5}\red{``And whenever you pray, don't be like the hypocrites who love to stand in the synagogues and on the street corners so that they will be seen by people. I tell all of you\fnote{\fbackref{6:5} The Gk. pronoun \fbib{you} is pl.} with certainty, they have their full reward!} \v{6}\red{But whenever you pray, go into your room, close the door, and pray to your Father who is hidden. And your Father who sees from the hidden place will reward you.}\fnote{\fbackref{6:6} Other mss. read \fbib{reward you openly}}

\v{7}\red{``When you are praying, don't say meaningless things\fnote{\fbackref{6:7} Or \fbib{words}} like the unbelievers\fnote{\fbackref{6:7} Lit. \fbib{gentiles} ; i.e. unbelieving non-Jews} do, because they think they will be heard by being so wordy.} \v{8}\red{Don't be like them, because your Father knows what you need before you ask him.} \v{9}\red{Therefore, this is how you should pray:}

\begin{poetry}
\poeml \red{`Our Father in heaven,} \\
\poemll    \red{may your name be kept holy.} \\
\poeml \v{10}\red{May your kingdom come.} \\
\poemll    \red{May your will be done,} \\
\poemlll       \red{on earth as it is in heaven.} \\
\poeml \v{11}\red{Give us today our daily bread,} \\
\poeml \v{12}\red{and forgive us our sins,}\fnote{\fbackref{6:12} Or \fbib{debts}} \\
\poemll    \red{as we have forgiven those who have sinned against us.}\fnote{\fbackref{6:12} Or \fbib{forgiven our debtors}} \\
\poeml \v{13}\red{And never bring us into temptation,} \\
\poemll    \red{but deliver us from the evil one.'}\fnote{\fbackref{6:13} Other mss. read \fbib{evil one. For yours is the kingdom and the power and the glory forever. Amen.}}
\end{poetry}

\v{14}\red{Because if you forgive people their offenses, your heavenly Father will also forgive you.} \v{15}\red{But if you do not forgive people their offenses,\fnote{\fbackref{6:15} Other mss. lack \fbib{their offenses}} your Father will not forgive your offenses.''}
\passage{Teaching about Fasting}

\v{16}\red{``Whenever you fast, don't be gloomy like the hypocrites, because they put on sad faces to show others they are fasting. I tell all of you\fnote{\fbackref{6:16} The Gk. pronoun \fbib{you} is pl.} with certainty, they have their full reward!} \v{17}\red{But when you fast, put oil on\fnote{\fbackref{6:17} Or \fbib{anoint}} your head and wash your face,} \v{18}\red{so that your fasting will not be noticed by others but by your Father who is in the hidden place. And your Father who watches from the hidden place will reward you.''}\fnote{\fbackref{6:18} Other mss. read \fbib{reward you openly}}
\passage{Teaching about Treasures}
\passageinfo{(Luke 12:33-34)}

\v{19}\red{``Stop storing up treasures for yourselves on earth, where moths and rust destroy and where thieves break in and steal.} \v{20}\red{But keep on storing up treasures for yourselves in heaven, where moths and rust do not destroy and where thieves do not break in and steal,} \v{21}\red{because where your treasure is, there your heart will be also.''}
\passage{The Lamp of the Body}
\passageinfo{(Luke 11:34-36)}

\v{22}\red{``The eye is the lamp of the body. So if your eye is healthy, your whole body will be full of light.} \v{23}\red{But if your eye is evil, your whole body will be full of darkness. Therefore, if the light within you has turned into darkness, how great is that darkness!''}
\passage{God and Riches}
\passageinfo{(Luke 16:13)}

\v{24}\red{``No one can serve two masters, because either he will hate one and love the other, or be loyal to one and despise the other. You cannot serve God and riches!''}\fnote{\fbackref{6:24} Lit. \fbib{mammon}; i.e. \fbib{wealth}}
\passage{Stop Worrying}
\passageinfo{(Luke 12:22-34)}

\v{25}\red{``That's why I'm telling you to stop worrying about your life---what you will eat or what you will drink\fnote{\fbackref{6:25} Other mss. lack \fbib{or what you will drink}}---or about your body---what you will wear. Life is more than food, isn't it, and the body more than clothing?} \v{26}\red{Look at the birds in the sky. They don't plant or harvest or gather food into barns, and yet your heavenly Father feeds them. You are more valuable than they are, aren't you?} \v{27}\red{Can any of you add a single hour to the length of your life\fnote{\fbackref{6:27} Or \fbib{add one cubit to your height}} by worrying?} \v{28}\red{And why do you worry about clothes? Consider the lilies in the field and how they grow. They don't work or spin yarn,} \v{29}\red{but I tell you that not even Solomon in all of his splendor was clothed like one of them.} \v{30}\red{Now if that is the way God clothes the grass in the field, which is alive today and thrown into an oven tomorrow, won't he clothe you much better---you who have little faith?}

\v{31}\red{``So don't ever worry by saying, `What are we going to eat?' or `What are we going to drink?' or `What are we going to wear?'} \v{32}\red{because it is the unbelievers\fnote{\fbackref{6:32} Lit. \fbib{gentiles} ; i.e. unbelieving non-Jews} who are eager for all those things. Surely your heavenly Father knows that you need all of them!} \v{33}\red{But first be concerned about God's kingdom and his righteousness,\fnote{\fbackref{6:33} Other mss. read \fbib{his kingdom and righteousness}} and all of these things will be provided for you as well.} \v{34}\red{So never worry about tomorrow, because tomorrow will worry about itself. Each day has enough trouble of its own.''}
\labelchapt{7}
\passage{Judging Others}
\passageinfo{(Luke 6:37-38, 41-42)}

\chapt{7}
\v{1}\red{``Stop judging, so that you won't be judged,} \v{2}\red{because the way that you judge\fnote{\fbackref{7:2} Lit. \fbib{measure}} others will be the way that you will be judged, and you will be evaluated by the standard with which you evaluate others.}

\v{3}\red{``Why do you see the speck in your brother's eye but fail to notice the beam in your own eye?} \v{4}\red{Or how can you say to your brother, `Let me take the speck out of your eye,' when the beam is in your own eye?} \v{5}\red{You hypocrite! First remove the beam from your own eye, and then you will see clearly enough to remove the speck from your brother's eye.''}
\passage{Despising the Holy}

\v{6}\red{``Never give what is holy to dogs or throw your pearls before pigs. Otherwise, they will trample them with their feet and then turn around and attack you.''}
\passage{Ask, Search, Knock}
\passageinfo{(Luke 11:9-13)}

\v{7}\red{``Keep asking, and it will be given to you. Keep searching, and you will find. Keep knocking, and the door\fnote{\fbackref{7:7} Lit. \fbib{and it}} will be opened for you.} \v{8}\red{Because everyone who keeps asking will receive, and the person who keeps searching will find, and the person who keeps knocking will have the door\fnote{\fbackref{7:8} Lit. \fbib{have it}} opened.}

\v{9}\red{``There isn't a person among you who would give his son a stone if he asked for bread, is there?} \v{10}\red{Or if he asks for a fish, he wouldn't give him a snake, would he?} \v{11}\red{So if you who are evil know how to give good gifts to your children, how much more will your Father in heaven give good things to those who keep on asking him!} \v{12}\red{Therefore, whatever you want people to do for you, do the same for them, because this summarizes the Law and the Prophets.''}
\passage{The Narrow Gate}
\passageinfo{(Luke 13:24)}

\v{13}\red{``Go in through the narrow gate, because the gate is wide and the road is spacious that leads to destruction, and many people are entering by it.} \v{14}\red{How narrow is the gate and how constricted is the road that leads to life, and there aren't many people who find it!''}
\passage{A Tree is Known by Its Fruit}
\passageinfo{(Luke 6:43-44)}

\v{15}\red{``Beware of false prophets who come to you in sheep's clothing but inwardly are savage wolves.} \v{16}\red{You will know them by their fruit. Grapes aren't gathered from thorns, or figs from thistles, are they?} \v{17}\red{In the same way, every good tree produces good fruit, but a rotten tree produces bad fruit.} \v{18}\red{A good tree cannot produce bad fruit, and a rotten tree cannot produce good fruit.} \v{19}\red{Every tree that doesn't produce good fruit will be cut down and thrown into a fire.} \v{20}\red{So by their fruit you will know them.''}
\passage{I Never Knew You}
\passageinfo{(Luke 6:46; 13:25-27)}

\v{21}\red{``Not everyone who keeps saying to me, `Lord, Lord,' will get into the kingdom from\fnote{\fbackref{7:21} Lit. \fbib{of}} heaven, but only the person who keeps doing the will of my Father in heaven.} \v{22}\red{Many will say to me on that day, `Lord, Lord, we prophesied in your name, drove out demons in your name, and performed many miracles in your name, didn't we?'} \v{23}\red{Then I will tell them plainly, `I never knew you. Get away from me, you who practice evil!'\,''}\fnote{\fbackref{7:23} Cf. Ps 6:8}
\passage{The Two Foundations}
\passageinfo{(Luke 6:47-49)}

\v{24}\red{``Therefore, everyone who listens to these messages\fnote{\fbackref{7:24} Or \fbib{words}} of mine and puts them into practice is like a wise man who built his house on a rock.} \v{25}\red{The rain fell, the floods came, and the winds blew and beat against that house, but it did not collapse because its foundation was on the rock.}

\v{26}\red{``Everyone who keeps on hearing these messages\fnote{\fbackref{7:26} Or \fbib{words}} of mine and never puts them into practice is like a foolish man who built his house on sand.} \v{27}\red{The rain fell, the floods came, the winds blew and battered that house, and it collapsed---and its collapse was total.''}

\v{28}When Jesus had finished saying all these things,\fnote{\fbackref{7:28} Lit. \fbib{finished all these sayings}} the crowds were utterly amazed at his teaching, \v{29}because he was teaching them like a person who had authority, and not like their scribes.
\labelchapt{8}
\passage{Jesus Cleanses a Leper}
\passageinfo{(Mark 1:40-45; Luke 5:12-16)}

\chapt{8}
\v{1}When Jesus\fnote{\fbackref{8:1} Lit. \fbib{he}} came down from the hillside, large crowds followed him. \v{2}Suddenly, a leper\fnote{\fbackref{8:2} I.e. a man with a serious skin disease} came up to him, fell down before him, and said, ``Sir,\fnote{\fbackref{8:2} Or \fbib{Lord}} if you want to, you can make me clean.''\fnote{\fbackref{8:2} I.e. restored to health and qualified to participate in worship}

\v{3}So Jesus\fnote{\fbackref{8:3} Lit. \fbib{He}} reached out his hand, touched him, and said, \red{``I do want to. Be clean!''} And instantly his leprosy was made clean. \v{4}Then Jesus told him, \red{``See to it that you don't speak to anyone. Instead, go and show yourself to the priest, and then offer the sacrifice that Moses commanded\fnote{\fbackref{8:4} Cf. Lev 14:3-32} as proof to the authorities.''}\fnote{\fbackref{8:4} Lit. \fbib{to them}}
\passage{Jesus Heals a Centurion's Servant}
\passageinfo{(Luke 7:1-10; John 4:43-54)}

\v{5}When Jesus\fnote{\fbackref{8:5} Lit. \fbib{he}} returned to Capernaum, a centurion\fnote{\fbackref{8:5} I.e. a commander of about 100 soldiers} came up to him and begged him repeatedly, \v{6}``Sir,\fnote{\fbackref{8:6} Or \fbib{Lord}} my servant is lying at home paralyzed and in terrible pain.''

\v{7}Jesus\fnote{\fbackref{8:7} Lit. \fbib{He}} told him, \red{``I will come and heal him.''}

\v{8}The centurion replied, ``Sir,\fnote{\fbackref{8:8} Or \fbib{Lord}} I am not worthy to have you come under my roof. But just say the word, and my servant will be healed, \v{9}because I, too, am a man under authority and I have soldiers under me. I say to one of them\fnote{\fbackref{8:9} The Gk. lacks \fbib{of them}} `Go' and he goes, to another `Come' and he comes, and to my servant `Do this' and he does it.''

\v{10}When Jesus heard this, he was amazed and told those who were following him, \red{``I tell all of you\fnote{\fbackref{8:10} The Gk. pronoun \fbib{you} is pl.} with certainty, not even\fnote{\fbackref{8:10} Other mss. read \fbib{in no one}} in Israel have I found this kind of faith!} \v{11}\red{I tell all of you,\fnote{\fbackref{8:11} The Gk. pronoun \fbib{you} is pl.} many will come from east and west and will feast with Abraham, Isaac, and Jacob in the kingdom from\fnote{\fbackref{8:11} Lit. \fbib{of}} heaven.} \v{12}\red{But the unfaithful heirs\fnote{\fbackref{8:12} Lit. \fbib{the sons}} of that kingdom will be thrown into the darkness outside. In that place there will be wailing and gnashing of teeth.''}\fnote{\fbackref{8:12} I.e. extreme pain}

\v{13}\red{``Go,''} Jesus told the centurion, \red{``and it will be done for you, just as you have believed.''} And his servant was healed that very hour.
\passage{Jesus Heals Many People}
\passageinfo{(Mark 1:29-34; Luke 4:38-41)}

\v{14}When Jesus went into Peter's house, he saw Peter's\fnote{\fbackref{8:14} Lit. \fbib{his}} mother-in-law lying in bed, sick with a fever. \v{15}He touched her hand, and the fever left her. Then she got up and began serving him.

\v{16}When evening came, people\fnote{\fbackref{8:16} Lit. \fbib{they}} brought to him many who were possessed by demons. He drove out the spirits by speaking a command\fnote{\fbackref{8:16} Lit. \fbib{spirits with a word}} and healed everyone who was sick. \v{17}This was to fulfill what was declared by the prophet Isaiah when he said,

\begin{poetry}
\poeml ``It was he who took our illnesses away \\
\poemll    and removed our diseases.''\fnote{\fbackref{8:17} Cf. Isa 53:4}
\end{poetry}
\passage{The Would-be Followers of Jesus}
\passageinfo{(Luke 9:57-62)}

\v{18}When Jesus saw the large crowds around him, he gave orders to cross to the other side of the Sea of Galilee.\fnote{\fbackref{8:18} The Gk. lacks \fbib{of the Sea of Galilee}} \v{19}Just then, a scribe came up and told him, ``Teacher, I will follow you wherever you go.''

\v{20}Jesus told him,

\begin{poetry}
\poeml \red{``Foxes have holes and birds\fnote{\fbackref{8:20} Lit. \fbib{birds in the sky}} have nests,} \\
\poemll    \red{but the Son of Man has no place to rest.''}\fnote{\fbackref{8:20} Lit. \fbib{no place to lay his head}}
\end{poetry}

\v{21}Then another of his disciples told him, ``Lord,\fnote{\fbackref{8:21} Or \fbib{Sir}} first let me go and bury my father.''

\v{22}But Jesus told him, \red{``Follow me, and let the dead bury their own dead.''}
\passage{Jesus Calms the Sea}
\passageinfo{(Mark 4:35-41; Luke 8:22-25)}

\v{23}When Jesus\fnote{\fbackref{8:23} Lit. \fbib{he}} got into the boat, his disciples went with him. \v{24}Suddenly, a violent storm came up on the sea, so that the boat began to be swamped by the waves. Yet Jesus\fnote{\fbackref{8:24} Lit. \fbib{he}} kept sleeping. \v{25}They\fnote{\fbackref{8:25} Other mss. read \fbib{The disciples}} went to him and woke him up. ``Lord!'' they cried, ``Save us! We're going to die!''

\v{26}He asked them, \red{``Why are you afraid, you who have little faith?''} Then he got up and rebuked the winds and the sea, and there was a great calm.

\v{27}The men were amazed. ``What kind of man is this?'' they asked. ``Even the winds and the sea obey him!''
\passage{Jesus Heals Two Demon-Possessed Men}
\passageinfo{(Mark 5:1-20; Luke 8:26-39)}

\v{28}When Jesus\fnote{\fbackref{8:28} Lit. \fbib{he}} arrived on the other side in the region of the Gerasenes,\fnote{\fbackref{8:28} Other mss. read \fbib{Gergesenes}; still other mss. read \fbib{Gadarenes}} two demon-possessed men met him as they were coming out of the tombs. They were so violent that no one could travel on that road. \v{29}Suddenly, they screamed, ``What do you want with us, Son of God? Did you come here to torture us before the proper time?''

\v{30}Now a large herd of pigs was grazing some distance away from them. \v{31}So the demons began to plead with Jesus,\fnote{\fbackref{8:31} Lit. \fbib{him}} saying, ``If you drive us out, send us into that herd of pigs.''

\v{32}He told them, \red{``Go,''} and they came out and went into the pigs. Suddenly, the whole herd rushed down the cliff into the sea and drowned in the water.

\v{33}Now when those who had been taking care of the pigs ran away, they came into the city and reported everything, especially what had happened to the demon-possessed men. \v{34}Then the whole city went out to meet Jesus, and as soon as they saw him, they begged him to leave their region.
\labelchapt{9}
\passage{Jesus Heals a Paralyzed Man}
\passageinfo{(Mark 2:1-12; Luke 5:17-26)}

\chapt{9}
\v{1}After getting into a boat, Jesus\fnote{\fbackref{9:1} Lit. \fbib{he}} crossed to the other side and came to his own city. \v{2}All at once some people\fnote{\fbackref{9:2} Lit. \fbib{they}} brought him a paralyzed man lying on a stretcher. When Jesus saw their faith, he told the paralyzed man, \red{``Be courageous, son! Your sins are forgiven.''}

\v{3}Then some of the scribes told themselves, ``This fellow is blaspheming!''

\v{4}But Jesus, knowing\fnote{\fbackref{9:4} Lit. \fbib{saw}} what they were thinking, replied, \red{``Why do you have such evil thoughts in your hearts?} \v{5}\red{Which is easier: to say, `Your sins are forgiven,' or to say, `Get up and walk'?} \v{6}\red{But so you will know that the Son of Man has authority on earth to forgive sins{\ldots}''} he told the paralyzed man, \red{``Get up, pick up your stretcher, and go home!''}

\v{7}So the man\fnote{\fbackref{9:7} Lit. \fbib{he}} got up and went home. \v{8}When the crowds saw this, they became frightened\fnote{\fbackref{9:8} Other mss. read \fbib{they were amazed}} and glorified God for giving such authority to humans.
\passage{Jesus Calls Matthew}
\passageinfo{(Mark 2:13-17; Luke 5:27-32)}

\v{9}As Jesus went on from there, he saw a man named Matthew sitting at the tax collector's desk and told him, \red{``Follow me.''} So he got up and followed him.

\v{10}While he was having dinner at Matthew's\fnote{\fbackref{9:10} Lit. \fbib{the}} home, many tax collectors and sinners arrived and began eating with Jesus and his disciples. \v{11}The Pharisees saw this and asked his disciples, ``Why does your teacher eat with tax collectors and sinners?''

\v{12}When Jesus\fnote{\fbackref{9:12} Lit. \fbib{he}} heard that, he said, \red{``Healthy people don't need a physician, but sick people do.} \v{13}\red{Go and learn what this means: `I want mercy and not sacrifice,'\fnote{\fbackref{9:13} Cf. Hos 6:6} because I did not come to call righteous people, but sinners.''}
\passage{A Question about Fasting}
\passageinfo{(Mark 2:18-22; Luke 5:33-39)}

\v{14}Then John's disciples came to Jesus\fnote{\fbackref{9:14} Lit. \fbib{him}} and asked, ``Why do we and the Pharisees fast often,\fnote{\fbackref{9:14} Other mss. lack \fbib{often}} but your disciples don't fast?''

\v{15}Jesus asked them, \red{``The wedding guests\fnote{\fbackref{9:15} Lit. \fbib{The children of the wedding hall}; or \fbib{The children of the groom}} can't mourn as long as the groom is with them, can they? But the time will come when the groom will be taken away from them, and then they will fast.''}
\passage{The Unshrunk Cloth}
\passageinfo{(Mark 2:21; Luke 5:36)}

\v{16}\red{``No one patches an old garment with a piece of unshrunk cloth, because the patch pulls away from the garment, and a worse tear results.} \v{17}\red{Nor do people\fnote{\fbackref{9:17} Lit. \fbib{they}} pour new wine into old wineskins. If they do, the skins will burst, the wine will spill out, and the skins will be ruined. Instead, they pour new wine into fresh wineskins, and both are preserved.''}
\passage{Jesus Heals a Woman and Resurrects a Girl}
\passageinfo{(Mark 5:21-43; Luke 8:40-56)}

\v{18}While Jesus\fnote{\fbackref{9:18} Lit. \fbib{he}} was telling them these things, an official came up and fell down before him. ``My daughter has just died,'' he said. ``But come and lay your hand on her, and she will live.'' \v{19}So Jesus got up and followed him, along with his disciples.

\v{20}Just then a woman who had been suffering from chronic bleeding for twelve years came up behind him and touched the tassel of his garment, \v{21}because she had been saying to herself, ``If I just touch his robe, I will get well.''

\v{22}When Jesus turned and saw her, he said, \red{``Be courageous, daughter! Your faith has made you well.''} And from that very hour the woman was well.

\v{23}When Jesus came to the official's house and saw the flute players and the crowd making a commotion, \v{24}he said, \red{``Go away! The young lady hasn't died, but is sleeping.''} But they ridiculed him with laughter. \v{25}When the crowd had been driven outside, he went in, took her by the hand, and the young lady got up. \v{26}The news of this spread throughout that land.
\passage{Jesus Heals Two Blind Men}

\v{27}As Jesus was traveling on from there, two blind men followed him, shouting, ``Have mercy on us, Son of David!'' \v{28}When he had gone into the house, the blind men came to him.

Jesus asked them, \red{``Do you believe I can do this?''}

They told him, ``Yes, Lord!''\fnote{\fbackref{9:28} Or \fbib{Sir}}

\v{29}Then he touched their eyes and said, \red{``According to your faith, let it be done for you!''} \v{30}And their eyes were opened. Then Jesus sternly told them, \red{``See to it that nobody knows about this.''} \v{31}But they went out and spread the news about him throughout that region.
\passage{Jesus Heals a Man who Couldn't Talk}

\v{32}As the men\fnote{\fbackref{9:32} Lit. \fbib{As they}} were going out, a man who couldn't talk because he was demon-possessed was brought to him. \v{33}As soon as the demon had been driven out, the man\fnote{\fbackref{9:33} Lit. \fbib{the man who couldn't talk}} began to speak. The crowds were amazed and said, ``Nothing like this has ever been seen in Israel!''

\v{34}But the Pharisees kept saying, ``He drives out demons by the ruler of demons.''\fnote{\fbackref{9:34} Other mss. lack this verse}
\passage{The Compassion of Jesus}

\v{35}Then Jesus began traveling throughout all the cities and villages, teaching in their synagogues, proclaiming the gospel of the kingdom, and healing every disease and every illness. \v{36}When he saw the crowds, he was deeply moved with compassion for them, because they were troubled and helpless, like sheep without a shepherd.

\v{37}Then he told his disciples, \red{``The harvest is vast, but the workers are few.} \v{38}\red{So ask the Lord of the harvest to send out workers into his harvest.''}
\labelchapt{10}
\passage{Jesus Appoints Twelve Apostles}
\passageinfo{(Mark 3:13-19; Luke 6:12-16)}

\chapt{10}
\v{1}Then Jesus\fnote{\fbackref{10:1} Lit. \fbib{he}} called his twelve disciples to him and gave them authority over unclean spirits, so that they could drive them out and heal every disease and every illness. \v{2}These are the names of the twelve apostles: first, Simon (who is called Peter) and his brother Andrew; James, the son of Zebedee, and his brother John; \v{3}Philip and Bartholomew; Thomas and Matthew the tax collector; James, the son of Alphaeus, and Thaddaeus;\fnote{\fbackref{10:3} Other mss. read \fbib{Lebbaeus called Thaddaeus}} \v{4}Simon the Cananaean\fnote{\fbackref{10:4} \fbib{Cananaean} is Aram. for \fbib{Zealot}.} and Judas Iscariot, who later\fnote{\fbackref{10:4} Lit. \fbib{also}} betrayed Jesus.\fnote{\fbackref{10:4} Lit. \fbib{him}}
\passage{Jesus Sends Out the Twelve Disciples}
\passageinfo{(Mark 6:7-13; Luke 9:1-6)}

\v{5}These were the Twelve whom Jesus sent out after he had given them these\fnote{\fbackref{10:5} The Gk. lacks \fbib{these}} instructions: \red{``Don't turn on to the road that leads to the unbelievers,\fnote{\fbackref{10:5} Lit. \fbib{gentiles} ; i.e. unbelieving non-Jews} and don't enter Samaritan towns.} \v{6}\red{Instead, go to the lost sheep of the nation\fnote{\fbackref{10:6} Lit. \fbib{house}} of Israel.} \v{7}\red{As you go, make this announcement: `The kingdom from\fnote{\fbackref{10:7} Lit. \fbib{of}} heaven is near!'} \v{8}\red{Heal the sick, raise the dead, cleanse lepers, drive out demons.}

\begin{poetry}
\poeml \red{You have received without payment,} \\
\poemll    \red{so give without payment.}
\end{poetry}

\v{9}\red{Don't take any gold, silver, or copper in your moneybags,} \v{10}\red{or a traveling bag for the trip, or an extra shirt,\fnote{\fbackref{10:10} Lit. \fbib{two shirts}} or sandals, or a walking stick, because a worker deserves his food.}

\v{11}\red{``Whatever town or village you enter, find out who is receptive\fnote{\fbackref{10:11} Lit. \fbib{worthy}} in it and stay there until you leave.} \v{12}\red{As you enter the house, greet its occupants.}\fnote{\fbackref{10:12} The Gk. lacks \fbib{its occupants}} \v{13}\red{If the household is receptive,\fnote{\fbackref{10:13} Lit. \fbib{worthy}} let your blessing of peace come on it. But if it isn't receptive,\fnote{\fbackref{10:13} Lit. \fbib{worthy}} let your blessing of peace return to you.} \v{14}\red{If no one welcomes you or listens to your words, as you leave that house or town, shake its dust off your feet.} \v{15}\red{I tell all of you\fnote{\fbackref{10:15} The Gk. pronoun \fbib{you} is pl.} with certainty, it will be more bearable for the region of Sodom and Gomorrah on the day of judgment than for that town!''}
\passage{Future Persecutions}
\passageinfo{(Matthew 24:9-14; Mark 13:9-13; Luke 21:12-19)}

\v{16}\red{``Pay attention, now! I am sending you out like sheep among wolves. So be as cunning as serpents and as innocent as doves.} \v{17}\red{Watch out for people who will hand you over to the local councils and whip you in their synagogues.} \v{18}\red{You will be brought before governors and kings because of me, to testify to them and to unbelievers.}\fnote{\fbackref{10:18} Lit. \fbib{gentiles} ; i.e. unbelieving non-Jews} \v{19}\red{When they hand you over, don't worry about how you are to speak\fnote{\fbackref{10:19} The Gk. lacks \fbib{you are to speak}} or what you are to say, because in that hour what you are to say will be given to you.} \v{20}\red{It won't be you speaking, but the Spirit of your Father speaking through\fnote{\fbackref{10:20} Or \fbib{in}} you.}

\v{21}\red{``Brother will hand brother over for execution, and a father his child. Children will rebel against parents and have them put to death.} \v{22}\red{You will be hated by everyone because of my name. But the person who endures to the end will be saved.} \v{23}\red{So when they persecute you in one town, flee to the next, because I tell all of you\fnote{\fbackref{10:23} The Gk. pronoun \fbib{you} is pl.} with certainty that you will not have gone through the towns of Israel before the Son of Man comes.}

\v{24}\red{``A disciple is not above his teacher, and a slave is not above his master.} \v{25}\red{It is enough for a disciple to be like his teacher and a slave to be like his master. If they have called the head of the house Beelzebul,\fnote{\fbackref{10:25} The name means \fbib{Lord of the Flies}; i.e. Satan} how much more will they do the same to\fnote{\fbackref{10:25} The Gk. lacks \fbib{will they do the same to}} those of his household!''}
\passage{Fear God}
\passageinfo{(Luke 12:2-7)}

\v{26}\red{``So never be afraid of them, because there is nothing hidden that will not be revealed, and nothing secret that will not be made known.} \v{27}\red{What I tell you in darkness you must speak in the daylight, and what is whispered\fnote{\fbackref{10:27} Lit. \fbib{what you hear}} in your ear you must shout from the housetops.} \v{28}\red{Stop being\fnote{\fbackref{10:28} Or \fbib{Don't be}} afraid of those who kill the body but can't kill the soul. Instead, be afraid of the one who can destroy both body and soul in hell.}\fnote{\fbackref{10:28} \fbib{Gehenna}; a Gk. transliteration of the Heb. for \fbib{Valley of Hinnom}}

\v{29}\red{``Two sparrows are sold for a penny, aren't they? Yet not one of them will fall to the ground without your Father's permission.}\fnote{\fbackref{10:29} Lit. \fbib{apart from your Father}} \v{30}\red{Indeed, even the hairs on your head have all been counted!} \v{31}\red{So stop being\fnote{\fbackref{10:31} Or \fbib{don't be}} afraid. You are worth more than a bunch of sparrows.''}
\passage{Acknowledging the Messiah}
\passageinfo{(Luke 12:8-9)}

\v{32}\red{``Therefore, everyone who acknowledges me before people I, too, will acknowledge before my Father in heaven.} \v{33}\red{But whoever denies me before people I, too, will deny before my Father in heaven.''}
\passage{Not Peace, but Division}
\passageinfo{(Luke 12:49-53)}

\v{34}\red{``Do not think that I came to bring peace on earth. I did not come to bring peace but a sword!}\fnote{\fbackref{10:34} I.e. conflict} \v{35}\red{I came to turn}

\begin{poetry}
\poeml \red{`a man against his father,} \\
\poemll    \red{a daughter against her mother,} \\
\poemll    \red{and a daughter-in-law against her mother-in-law.} \\
\poeml \v{36}\red{A person's enemies will include members of his own family.'}\fnote{\fbackref{10:36} Cf. Mic 7:6}
\end{poetry}
\passage{The Cost of Discipleship}
\passageinfo{(Luke 14:25-27)}

\v{37}\red{``The one who loves his father or mother more than me isn't worthy of me, and the one who loves a son or daughter more than me isn't worthy of me.} \v{38}\red{The one who doesn't take up his cross and follow me isn't worthy of me.} \v{39}\red{The one who finds his life will lose it, and the one who loses his life because of me will find it.''}
\passage{Rewards}
\passageinfo{(Mark 9:41)}

\v{40}\red{``The one who receives you receives me, and the one who receives me receives the one who sent me.} \v{41}\red{The one who receives a prophet as\fnote{\fbackref{10:41} Lit. \fbib{in the name of}} a prophet will receive a prophet's reward, and the one who receives a righteous person as\fnote{\fbackref{10:41} Lit. \fbib{in the name of}} a righteous person will receive a righteous person's reward.} \v{42}\red{I tell all of you\fnote{\fbackref{10:42} The Gk. pronoun \fbib{you} is pl.} with certainty, whoever gives even a cup of cold water to one of these little ones because he is\fnote{\fbackref{10:42} Lit. \fbib{in the name of}} a disciple will never lose his reward.''}
\labelchapt{11}

\chapt{11}
\v{1}When Jesus had finished instructing his twelve disciples, he left there to teach and preach in their home towns.
\passage{John the Baptist Sends Messengers to Jesus}
\passageinfo{(Luke 7:18-35)}

\v{2}Now when John heard in prison about the activities of the Messiah,\fnote{\fbackref{11:2} Or \fbib{Christ}} he sent a message\fnote{\fbackref{11:2} The Gk. lacks \fbib{a message}} by his disciples \v{3}and asked him, ``Are you the Coming One, or should we wait for someone else?''

\v{4}Jesus answered them, \red{``Go and tell John what you hear and observe:} \v{5}\red{the blind see, the lame walk, lepers are cleansed, the deaf hear, the dead are raised, and the destitute hear the good news.} \v{6}\red{How blessed is anyone who is not offended by me!''}

\v{7}As they were leaving, Jesus began to speak to the crowds about John. \red{``What did you go out into the wilderness to see? A reed shaken by the wind?} \v{8}\red{Really, what did you go out to see? A man dressed in fancy clothes? See, those who wear fancy clothes live in kings' houses.} \v{9}\red{Really, what did you go out to see? A prophet? Yes, I tell you, and even more than a prophet!} \v{10}\red{This is the man about whom it is written,}

\begin{poetry}
\poeml \red{`See, I am sending my messenger ahead of you,} \\
\poemll    \red{who will prepare your way before you.'}\fnote{\fbackref{11:10} Cf. Mal 3:1}
\end{poetry}

\v{11}\red{I tell all of you\fnote{\fbackref{11:11} The Gk. pronoun \fbib{you} is pl.} with certainty, among those born of women no one has appeared who is greater than John the Baptist. Yet even the least important person in the kingdom from\fnote{\fbackref{11:11} Lit. \fbib{of}} heaven is greater than he.}

\v{12}\red{``From the days of John the Baptist until the present, the kingdom from\fnote{\fbackref{11:12} Lit. \fbib{of}} heaven has been forcefully advancing, and violent people have been attacking it,} \v{13}\red{because the Law and all the Prophets prophesied up to the time of John.} \v{14}\red{If you are willing to accept it, he is Elijah who is to come.}\fnote{\fbackref{11:14} Or \fbib{is about to come}} \v{15}\red{Let the person who has ears\fnote{\fbackref{11:15} Other mss. read \fbib{ears to hear}} listen!}

\v{16}\red{``To what can I compare the people living today? They're\fnote{\fbackref{11:16} Lit. \fbib{compare this generation? It's}} like little children who sit in the marketplaces and shout to each other,}

\begin{poetry}
\poeml \v{17}\red{`A wedding song we played for you,} \\
\poemll    \red{the dance you all did scorn.} \\
\poeml \red{A woeful dirge we chanted, too,} \\
\poemll    \red{but then you would not mourn.'}
\end{poetry}

\v{18}\red{Because John didn't come eating or drinking, yet people\fnote{\fbackref{11:18} Lit. \fbib{they}} say, `He has a demon!'} \v{19}\red{The Son of Man came eating and drinking, and they say, `Look, a glutton and a drunk, a friend of tax collectors and sinners!'}

\begin{poetry}
\poeml \red{Absolved from every act of sin,} \\
\poemll    \red{is wisdom by her kith and kin.''}\fnote{\fbackref{11:19} Lit. by all her children; other mss. read by her actions}
\end{poetry}
\passage{Jesus Denounces Unrepentant Cities}
\passageinfo{(Luke 10:13-15)}

\v{20}Then Jesus\fnote{\fbackref{11:20} Lit. \fbib{he}} began to denounce the cities in which most of his miracles had taken place, because they didn't repent. \v{21}\red{``How terrible it will be for you, Chorazin! How terrible it will be for you, Bethsaida! Because if the miracles that happened in you had taken place in Tyre and Sidon, they would have repented long ago in sackcloth and ashes.} \v{22}\red{Indeed I tell you, it will be more bearable for Tyre and Sidon on Judgment Day than for you!}

\v{23}\red{``And you, Capernaum! You won't be lifted up to heaven, will you? You'll go down to Hell!\fnote{\fbackref{11:23} Lit. \fbib{Hades}; i.e. the realm of the dead} Because if the miracles that happened in you had taken place in Sodom, it would have remained to this day.} \v{24}\red{Indeed I tell you, it will be more bearable for the land of Sodom on Judgment Day than for you!''}
\passage{Jesus Praises the Father and Invites the Disciples to Come to Him}
\passageinfo{(Luke 10:21-22)}

\v{25}At that time Jesus said, \red{``I praise you, Father, Lord of heaven and earth, because you have hidden these things from wise and intelligent people and have revealed them to infants.} \v{26}\red{Yes, Father, because this is what was pleasing to you.} \v{27}\red{All things have been entrusted to me by my Father. No one fully knows the Son except the Father, and no one fully knows the Father except the Son and the person to whom the Son chooses to reveal him.}

\v{28}\red{``Come to me, all of you who are weary and loaded down with burdens, and I will give you rest.} \v{29}\red{Place my yoke on you and learn from me, because I am gentle and humble,\fnote{\fbackref{11:29} Lit. \fbib{humble in heart}} and you will find rest for your souls,}\fnote{\fbackref{11:29} Cf. Jer 6:16} \v{30}\red{because my yoke is pleasant,\fnote{\fbackref{11:30} Or \fbib{kind}} and my burden is light.''}
\labelchapt{12}
\passage{Jesus is Lord of the Sabbath}
\passageinfo{(Mark 2:23-28; Luke 6:1-5)}

\chapt{12}
\v{1}At that time, Jesus walked through the grain fields on a Sabbath.\fnote{\fbackref{12:1} Lit. \fbib{on the Sabbaths}} His disciples became hungry and began picking heads of grain to eat. \v{2}When the Pharisees saw this, they told him, ``Look! Your disciples are doing what is not lawful to do on the Sabbath!''

\v{3}But he told them, \red{``Haven't you read what David did when he and his companions were hungry?} \v{4}\red{How is it that he went into the house of God and ate the Bread of the Presence, which was not lawful for him and his companions to eat but was reserved\fnote{\fbackref{12:4} Lit. \fbib{but only}} for the priests?} \v{5}\red{Or haven't you read in the Law that on every Sabbath\fnote{\fbackref{12:5} Lit. \fbib{on the Sabbaths}} the priests in the Temple violate the Sabbath\fnote{\fbackref{12:5} I.e. by carrying out priestly duties} and yet are innocent?}\fnote{\fbackref{12:5} Cf. Exod 25:30; 29:32-33; Lev 8:31} \v{6}\red{But I tell you, something greater than the Temple is here!} \v{7}\red{If you had known what `I want mercy and not sacrifice'\fnote{\fbackref{12:7} Cf. Hos 6:6} means, you would not have condemned the innocent,} \v{8}\red{for the Son of Man is Lord of the Sabbath.''}
\passage{Jesus Heals a Man with a Paralyzed Hand}
\passageinfo{(Mark 3:1-6; Luke 6:6-11)}

\v{9}Moving on from there, Jesus\fnote{\fbackref{12:9} Lit. \fbib{he}} went into their synagogue. \v{10}Suddenly, a man with a paralyzed hand appeared. The people\fnote{\fbackref{12:10} Lit. \fbib{They}} asked Jesus\fnote{\fbackref{12:10} Lit. \fbib{him}} if it was lawful to heal on Sabbath days,\fnote{\fbackref{12:10} Lit. \fbib{on the Sabbaths}} intending to accuse him of doing something wrong.

\v{11}But he asked them, \red{``Is there a man among you who, if he had one sheep and it fell into a ditch on the Sabbath, wouldn't take hold of it and pull it out?} \v{12}\red{How much more is a human being worth than a sheep! So it is lawful to do good on Sabbath days.''}\fnote{\fbackref{12:12} Lit. \fbib{on the Sabbaths}}

\v{13}Then he told the man, \red{``Hold out your hand.''} He held it out and it became normal, as healthy as his other hand. \v{14}The Pharisees, however, went out and plotted against Jesus\fnote{\fbackref{12:14} Lit. \fbib{him}} to kill him.
\passage{Jesus, God's Chosen Servant}

\v{15}When Jesus became aware of this, he left that place. Many crowds\fnote{\fbackref{12:15} Other mss. lack \fbib{crowds}} followed him, and he healed all of them, \v{16}ordering them not to make him known. \v{17}This was to fulfill what was declared by the prophet Isaiah when he said,

\begin{poetry}
\poeml \v{18}``Here is my Servant whom I have chosen, \\
\poemll    whom I love, and with whom I am pleased! \\
\poeml I will put my Spirit on him, \\
\poemll    and he will proclaim justice\fnote{\fbackref{12:18} Or \fbib{judgment}} to \red{unbelievers.}\fnote{\fbackref{12:18} Lit. \fbib{gentiles} ; i.e. unbelieving non-Jews} \\
\poeml \v{19}He will not quarrel or shout, \\
\poemll    and no one will hear him shouting\fnote{\fbackref{12:19} Lit. \fbib{hear his voice}} in the streets. \\
\poeml \v{20}He will not snap off a broken reed \\
\poemll    or snuff out a smoldering wick \\
\poemlll       until he has brought justice\fnote{\fbackref{12:20} Or \fbib{judgment}} through to victory. \\
\poeml \v{21}And in his name unbelievers\fnote{\fbackref{12:21} Lit. \fbib{gentiles} ; i.e. unbelieving non-Jews} will hope.''\fnote{\fbackref{12:21} Cf. Isa 42:1-4}
\end{poetry}
\passage{Jesus is Accused of Working with Beelzebul}
\passageinfo{(Mark 3:20-30; Luke 11:14-23; 12:10)}

\v{22}Then a demon-possessed man who was blind and unable to talk was brought to him. Jesus\fnote{\fbackref{12:22} Lit. \fbib{He}} healed him so that the man\fnote{\fbackref{12:22} Lit. \fbib{the man who was unable to talk}} could speak and see. \v{23}All the crowds were amazed and kept saying, ``This man isn't the Son of David, is he?''

\v{24}But when the Pharisees heard this, they said, ``This man drives out demons only by Beelzebul, the ruler of the demons.''

\v{25}He knew what they were thinking and told them, \red{``Every kingdom divided against itself is destroyed, and every city or household divided against itself will not stand.} \v{26}\red{So if Satan drives out Satan, he is divided against himself. How, then, can his kingdom stand?} \v{27}\red{If I drive out demons by Beelzebul, by whom do your own followers\fnote{\fbackref{12:27} Lit. \fbib{sons}} drive them out? That is why they will be your judges!} \v{28}\red{But if I drive out demons by the Spirit of God, then the kingdom of God has come to you.} \v{29}\red{How can someone go into a strong man's house and carry off his possessions without first tying up the strong man? Then he can ransack his house.} \v{30}\red{The person who isn't with me is against me, and the person who isn't gathering with me is scattering.} \v{31}\red{So I tell you, every sin and blasphemy will be forgiven,\fnote{\fbackref{12:31} Lit. \fbib{will be forgiven to people}} but blasphemy against the Spirit will not be forgiven.} \v{32}\red{Whoever speaks a word against the Son of Man will be forgiven, but whoever speaks against the Holy Spirit will not be forgiven, either in this age or in the one to come.''}
\passage{A Tree is Known by Its Fruit}
\passageinfo{(Luke 6:43-45)}

\v{33}\red{``Either make the tree good and its fruit good, or make the tree rotten and its fruit rotten, because a tree is known by its fruit.} \v{34}\red{You children of serpents! How can you say anything good when you are evil? The mouth speaks about what overflows from the heart.} \v{35}\red{A good person brings good things out of a good treasure house, and an evil person brings evil things out of an evil treasure house.} \v{36}\red{I tell you, on Judgment Day people will give an account for every thoughtless\fnote{\fbackref{12:36} Or \fbib{worthless}} word they have uttered,} \v{37}\red{because by your words you will be acquitted, and by your words you will be condemned.''}
\passage{The Sign of Jonah}
\passageinfo{(Mark 8:11-12; Luke 11:29-32)}

\v{38}Then some of the scribes and Pharisees told Jesus,\fnote{\fbackref{12:38} Lit. \fbib{him}} ``Teacher, we want to see a sign from you.''

\v{39}But he replied to them, \red{``An evil and adulterous generation craves a sign. Yet no sign will be given to it except the sign of the prophet Jonah,} \v{40}\red{because just as Jonah was in the stomach of the sea creature for three days and three nights,\fnote{\fbackref{12:40} Cf. Jonah 1:17} so the Son of Man will be in the heart of the earth for three days and three nights.} \v{41}\red{The men of Nineveh will stand up at the judgment and condemn the people living today,\fnote{\fbackref{12:41} Lit. \fbib{judgment with this generation and condemn it}} because they repented at the preaching of Jonah. But look---something greater than Jonah is here!} \v{42}\red{The queen of the south will stand up and condemn the people living today,\fnote{\fbackref{12:42} Lit. \fbib{judgment with this generation and condemn it}} because she came from so far away\fnote{\fbackref{12:42} Lit. \fbib{from the ends of the earth}} to hear the wisdom of Solomon. But look! Something greater than Solomon is here!''}
\passage{The Return of the Unclean Spirit}
\passageinfo{(Luke 11:24-26)}

\v{43}\red{``Whenever an unclean spirit goes out of a person, it wanders through waterless places looking for a place to rest, but finds none.} \v{44}\red{Then it says, `I will go back to my home that I left.' When it arrives, it finds it empty, swept clean, and put in order.} \v{45}\red{Then it goes and brings with it seven other spirits more evil than itself, and they go in and settle there. And so the final condition of that person becomes worse than the first. That's just what will happen to this evil generation!''}
\passage{The True Family of Jesus}
\passageinfo{(Mark 3:31-35; Luke 8:19-21)}

\v{46}While Jesus\fnote{\fbackref{12:46} Lit. \fbib{he}} was still speaking to the crowds, his mother and brothers stood outside, wanting to speak to him. \v{47}Someone told him, ``Look! Your mother and your brothers are standing outside, asking to speak to you.''\fnote{\fbackref{12:47} Other mss. lack this verse}

\v{48}He asked the man who told him, \red{``Who is my mother, and who are my brothers?''} \v{49}Then pointing with his hand at his disciples, he said, \red{``Here are my mother and my brothers,} \v{50}\red{because whoever does the will of my Father in heaven is my brother and sister and mother.''}
\labelchapt{13}
\passage{The Parable about a Farmer}
\passageinfo{(Mark 4:1-9; Luke 8:4-18)}

\chapt{13}
\v{1}That day Jesus left the house and sat down beside the sea. \v{2}Such large crowds gathered around him that he got into a boat and sat down, while the entire crowd stood on the shore. \v{3}Then he began to tell them many things in parables. He said, \red{``Listen! A farmer went out to sow.} \v{4}\red{As he was sowing, some seeds fell along the path, and birds came and ate them up.} \v{5}\red{Other seeds fell on stony ground, where they did not have a lot of soil. They sprouted at once because the soil wasn't deep.} \v{6}\red{But when the sun came up, they were scorched. Since they did not have any roots, they dried up.} \v{7}\red{Other seeds fell among thorn bushes, and the thorn bushes grew higher and choked them out.} \v{8}\red{But other seeds fell on good soil and produced a crop, some 100, some 60, and some 30 times what was sown.}\fnote{\fbackref{13:8} The Gk. lacks \fbib{what was sown}} \v{9}\red{Let the person who has ears\fnote{\fbackref{13:9} Other mss. read \fbib{ears to hear}} listen!''}
\passage{The Purpose of the Parables}
\passageinfo{(Mark 4:10-12; Luke 8:9-10)}

\v{10}Then the disciples came and asked Jesus,\fnote{\fbackref{13:10} Lit. \fbib{him}} ``Why do you speak to people\fnote{\fbackref{13:10} Lit. \fbib{to them}} in parables?''

\v{11}He answered them, \red{``You have been given knowledge about the secrets of the kingdom from\fnote{\fbackref{13:11} Lit. \fbib{of}} heaven, but it hasn't been given to them,} \v{12}\red{because to anyone who has something, more will be given, and he will have more than enough. But from the one who doesn't have anything, even what he has will be taken away from him.} \v{13}\red{That's why I speak to them in parables, because}

\begin{poetry}
\poeml \red{`they look but don't see,} \\
\poemll    \red{and they listen but don't hear or understand.'}
\end{poetry}

\v{14}\red{``With them the prophecy of Isaiah is being fulfilled, which says:}

\begin{poetry}
\poeml \red{`You will listen and listen but never understand.} \\
\poemll    \red{You will look and look but never comprehend,} \\
\poeml \v{15}\red{for this people's heart has become dull,} \\
\poemll    \red{and their ears are hard of hearing.}\fnote{\fbackref{13:15} Lit. \fbib{they hear with ears of heaviness}} \\
\poeml \red{They have shut their eyes} \\
\poemll    \red{so that they might not see with their eyes,} \\
\poemlll       \red{and hear with their ears,} \\
\poeml \red{and understand with their heart and turn,} \\
\poemll    \red{and I would heal them.'}\fnote{\fbackref{13:15} Cf. Isa 6:9-10}
\end{poetry}

\v{16}\red{``How blessed are your eyes because they see, and your ears because they hear!} \v{17}\red{I tell all of you\fnote{\fbackref{13:17} The Gk. pronoun \fbib{you} is pl.} with certainty, many prophets and righteous people longed to see the things you see but did not see them, and to hear the things you hear but did not hear them.''}
\passage{Jesus Explains the Parable about the Farmer}
\passageinfo{(Mark 4:13-20; Luke 8:11-15)}

\v{18}\red{``Listen, then, to the parable about the farmer.} \v{19}\red{When anyone hears the word about the kingdom yet doesn't understand it, the evil one comes and snatches away what was sown in his heart. This is what was sown along the path.} \v{20}\red{As for what was sown on the stony ground, this is the person who hears the word and accepts it joyfully at once,} \v{21}\red{but since he doesn't have any root in himself, he lasts for only a short time. When trouble or persecution comes along because of the word, he immediately falls away.} \v{22}\red{As for what was sown among the thorn bushes, this is the person who hears the word, but the worries of life and the deceitful pleasures of wealth choke the word so that it can't produce a crop.} \v{23}\red{But as for what was sown on good soil, this is the person who hears the word, understands it, and produces a crop that yields 100, 60, or 30 times what was sown.''}\fnote{\fbackref{13:23} The Gk. lacks \fbib{what was sown}}
\passage{The Parable about the Weeds among the Wheat}

\v{24}He presented another parable to them: \red{``The kingdom from\fnote{\fbackref{13:24} Lit. \fbib{of}} heaven may be compared to a man who sowed good seed in his field.} \v{25}\red{While people were sleeping, his enemy came and sowed weeds among the wheat and went away.} \v{26}\red{When the crop came up and bore grain, the weeds appeared, too.}

\v{27}\red{``The owner's servants came and asked him, `Master, you sowed good seed in your field, didn't you? Then where did these weeds come from?'}

\v{28}\red{``He told them, `An enemy did this!'}

\red{``The servants asked him, `Do you want us to go and pull them out?'}

\v{29}\red{``He said, `No! If you pull out the weeds, you might pull out the wheat with them.} \v{30}\red{Let both grow together until the harvest, and at harvest time I will tell the reapers, ``Gather the weeds first and tie them in bundles for burning, but bring the wheat into my barn.''\,'\,''}
\passage{The Parables about a Mustard Seed and Yeast}
\passageinfo{(Mark 4:30-32; Luke 13:18-21)}

\v{31}He presented another parable to them, saying, \red{``The kingdom from\fnote{\fbackref{13:31} Lit. \fbib{of}} heaven is like a mustard seed that a man took and planted in his field.} \v{32}\red{Although it is the smallest of\fnote{\fbackref{13:32} Or \fbib{it is smaller than}} all seeds, when it is fully grown it is larger than the garden plants and becomes a tree, and the birds in the sky come and nest in its branches.''}

\v{33}He told them another parable: \red{``The kingdom from\fnote{\fbackref{13:33} Lit. \fbib{of}} heaven is like yeast that a woman took and mixed with\fnote{\fbackref{13:33} Lit. \fbib{hid in}} three measures of flour until all of it was leavened.''}
\passage{Why Jesus Used Parables}
\passageinfo{(Mark 4:33-34)}

\v{34}Jesus told the crowds all these things in parables. He did not tell them anything without using\fnote{\fbackref{13:34} The Gk. lacks \fbib{using}} a parable. \v{35}This was to fulfill what was declared by the prophet\fnote{\fbackref{13:35} Other mss. read \fbib{Isaiah the prophet}} when he said,

\begin{poetry}
\poeml ``I will open my mouth to speak\fnote{\fbackref{13:35} The Gk. lacks \fbib{to speak}} in parables. \\
\poemll    I will declare what has been hidden \\
\poemlll       since the creation of the world.''\fnote{\fbackref{13:35} Cf. Ps 78:2}
\end{poetry}
\passage{Jesus Explains the Parable about the Weeds}

\v{36}Then Jesus\fnote{\fbackref{13:36} Lit. \fbib{he}} left the crowds and went into the house. His disciples came to him and asked, ``Explain to us the parable about the weeds in the field.''

\v{37}He answered, \red{``The person who sowed good seed is the Son of Man,} \v{38}\red{while the field is the world. The good seed are those who belong to\fnote{\fbackref{13:38} Lit. \fbib{the sons of}} the kingdom, while the weeds are those who belong to\fnote{\fbackref{13:38} Lit. \fbib{the sons of}} the evil one.} \v{39}\red{The enemy who sowed them is the devil, the harvest is the end of the age, and the reapers are the angels.} \v{40}\red{Just as weeds are gathered and burned with fire, so it will be at end of the\fnote{\fbackref{13:40} Other mss. read \fbib{this}} age.} \v{41}\red{The Son of Man will send his angels, and they will gather from his kingdom everything that causes others to sin and those who practice lawlessness} \v{42}\red{and they will throw them into a blazing furnace. In that place there will be wailing and gnashing of teeth.}\fnote{\fbackref{13:42} I.e. extreme pain} \v{43}\red{Then the righteous will shine like the sun in their Father's kingdom. Let the person who has ears\fnote{\fbackref{13:43} Other mss. read \fbib{ears to hear}} listen!''}
\passage{The Parable about a Hidden Treasure}

\v{44}\red{``The kingdom from\fnote{\fbackref{13:44} Lit. \fbib{of}} heaven is like treasure hidden in a field that a man found and hid. In his excitement he went and sold everything he had and bought that field.''}
\passage{The Parable about a Valuable Pearl}

\v{45}\red{``Again, the kingdom from\fnote{\fbackref{13:45} Lit. \fbib{of}} heaven is like a merchant searching for fine pearls.} \v{46}\red{When he found a very valuable pearl, he went and sold everything he had and bought it.''}
\passage{The Parable about a Net}

\v{47}\red{``Again, the kingdom from\fnote{\fbackref{13:47} Lit. \fbib{of}} heaven is like a large net thrown into the sea that gathered all kinds of fish.} \v{48}\red{When it was full, the fishermen\fnote{\fbackref{13:48} Lit. \fbib{they}} hauled it ashore. Then they sat down, sorted the good fish into containers, and threw the bad ones away.} \v{49}\red{That is how it will be at the end of the age. The angels will go out, cull out the evil people from among the righteous ones,} \v{50}\red{and will throw them into a blazing furnace. In that place there will be wailing and gnashing of teeth.''}\fnote{\fbackref{13:50} I.e. extreme pain}
\passage{New and Old Treasures}

\v{51}\red{``Do you understand all these things?''}

They told him, ``Yes.''

\v{52}Then he told them, \red{``That is why every scribe who has been trained for the kingdom from\fnote{\fbackref{13:52} Lit. \fbib{of}} heaven is like the master of a household who brings both new and old things out of his treasure chest.''}
\passage{Jesus is Rejected at Nazareth}
\passageinfo{(Mark 6:1-6; Luke 4:16-30)}

\v{53}When Jesus had finished these parables, he left that place. \v{54}He went to his hometown and began teaching the people\fnote{\fbackref{13:54} Lit. \fbib{them}} in their synagogue in such a way that they were amazed and asked, ``Where did this man get this wisdom and these miracles? \v{55}This is the builder's\fnote{\fbackref{13:55} Or \fbib{carpenter's}} son, isn't it? His mother is named Mary, isn't she? His brothers are James, Joseph, Simon, and Judas, aren't they? \v{56}And his sisters are all with us, aren't they? So where did this man get all these things?'' \v{57}And they were offended by him.

But Jesus told them, \red{``A prophet is without honor only in his hometown and in his own home.''} \v{58}He did not perform many miracles there because of their unbelief.
\labelchapt{14}
\passage{The Death of John the Baptist}
\passageinfo{(Mark 6:14-29; Luke 9:7-9)}

\chapt{14}
\v{1}At that time Herod the tetrarch,\fnote{\fbackref{14:1} I.e. Herod Antipas, a son of King Herod and ruler over one of four districts in and around the territory of Israel.} hearing about the fame of Jesus, \v{2}told his servants, ``This is John the Baptist! He has been raised from the dead, and that's why these miracles are being done by him.'' \v{3}Herod had arrested John, bound him with chains, and put him in prison on account of Herodias, his brother Philip's\fnote{\fbackref{14:3} Other mss. read \fbib{his brother's}} wife.

\v{4}John had been telling him, ``It is not lawful for you to have her.'' \v{5}Although Herod\fnote{\fbackref{14:5} Lit. \fbib{he}} wanted to kill him, he was afraid of the crowd, since they regarded John\fnote{\fbackref{14:5} Lit. \fbib{him}} as a prophet.

\v{6}But when Herod's birthday celebration was held, the daughter of Herodias danced before the guests.\fnote{\fbackref{14:6} Lit. \fbib{in the middle}} She pleased Herod \v{7}so much that he promised with an oath to give her whatever she asked for. \v{8}Prompted by her mother, she said, ``Give me, right here on a platter, the head of John the Baptist.'' \v{9}Under pressure because of his promises and his assembled guests, the king ordered that it be done. \v{10}So he sent word\fnote{\fbackref{14:10} The Gk. lacks \fbib{word}} and had John beheaded in prison. \v{11}His head was brought on a platter and given to the girl, and she took it to her mother. \v{12}When John's\fnote{\fbackref{14:12} Lit. \fbib{his}} disciples came, they carried off the body and buried it. Then they went and told Jesus.
\passage{Jesus Feeds More than Five Thousand People}
\passageinfo{(Mark 6:30-44; Luke 9:10-17; John 6:1-14)}

\v{13}When Jesus heard this, he left that place and went\fnote{\fbackref{14:13} The Gk. lacks \fbib{and went}} by boat to a deserted place by himself. The crowds heard of it and followed him on foot from the neighboring\fnote{\fbackref{14:13} The Gk. lacks \fbib{neighboring}} towns. \v{14}When he got out of the boat,\fnote{\fbackref{14:14} The Gk. lacks \fbib{of the boat}} he saw a large crowd. He had compassion for them and healed their sick. \v{15}When evening had come, the disciples went to him and said, ``This is a deserted place, and it's already late. Send the crowds away so that they can go into the villages and buy food for themselves.''

\v{16}But Jesus told them, \red{``They don't need to go away. You give them something to eat.''}

\v{17}They told him, ``We don't have anything here except five loaves of bread and two fish.''

\v{18}He said, \red{``Bring them to me.''} \v{19}Then he ordered the crowds to sit down on the grass. Taking the five loaves and the two fish, he looked up to heaven and blessed them. Then he broke the loaves in pieces and gave them to his disciples, and the disciples gave them\fnote{\fbackref{14:19} The Gk. lacks \fbib{gave them}} to the crowds. \v{20}All of them ate and were filled. Then the disciples\fnote{\fbackref{14:20} Lit. \fbib{they}} picked up what was left of the broken pieces, twelve baskets full. \v{21}Now those who had eaten were about 5,000 men, besides women and children.
\passage{Jesus Walks on the Sea}
\passageinfo{(Mark 6:45-52; John 6:16-21)}

\v{22}Jesus\fnote{\fbackref{14:22} Lit. \fbib{He}} immediately had the disciples get into a boat and cross to the other side ahead of him, while he sent the crowds away. \v{23}After dismissing the crowds, he went up on a hillside by himself to pray. When evening came, he was there alone. \v{24}By this time the boat was in the middle of the sea\fnote{\fbackref{14:24} Other mss. read \fbib{many furlongs from the land}} and was being battered by the waves, because the wind was against them. \v{25}Shortly before dawn,\fnote{\fbackref{14:25} Lit. \fbib{In the fourth watch of the night}} Jesus\fnote{\fbackref{14:25} Lit. \fbib{he}} came to them, walking on the sea. \v{26}When the disciples saw him walking on the sea, they were terrified and cried out, ``It's a ghost!'' And they screamed in terror.

\v{27}\red{``Have courage!''} Jesus immediately told them. \red{``It's me. Stop being afraid!''}

\v{28}Peter answered him, ``Lord, if it's you, order me to come to you on the water.''

\v{29}Jesus\fnote{\fbackref{14:29} Lit. \fbib{He}} said, \red{``Come on!''} So Peter got down out of the boat, started walking on the water, and came\fnote{\fbackref{14:29} Other mss. read \fbib{to go}} to Jesus.

\v{30}But when he noticed the strong\fnote{\fbackref{14:30} Other mss. lack \fbib{strong}} wind, he was frightened. As he began to sink, he shouted, ``Lord, save me!''

\v{31}At once Jesus reached out his hand, caught him, and asked him, \red{``You who have so little faith, why did you doubt?''} \v{32}As they got into the boat, the wind stopped blowing.

\v{33}Then the men in the boat began to worship Jesus,\fnote{\fbackref{14:33} Lit. \fbib{him}} saying, ``You certainly are the Son of God!''
\passage{Jesus Heals the Sick in Gennesaret}
\passageinfo{(Mark 6:53-56)}

\v{34}They crossed over and came ashore at Gennesaret. \v{35}When the men of that place recognized Jesus,\fnote{\fbackref{14:35} Lit. \fbib{him}} they sent word\fnote{\fbackref{14:35} The Gk. lacks \fbib{word}} throughout that region and brought him everyone who was sick. \v{36}They kept begging him to let them touch just the tassel of his garment, and everyone who touched it was completely healed.
\labelchapt{15}
\passage{Jesus Challenges the Tradition of the Elders}
\passageinfo{(Mark 7:1-23)}

\chapt{15}
\v{1}Then some Pharisees and scribes came from Jerusalem to Jesus and asked, \v{2}``Why do your disciples disregard the tradition of the elders? They don't wash their hands when they eat.''\fnote{\fbackref{15:2} Lit. \fbib{eat bread}}

\v{3}But he answered them, \red{``Why do you also disregard the commandment of God because of your tradition?} \v{4}\red{Because God said,\fnote{\fbackref{15:4} Other mss. read \fbib{commanded, saying}} `You are to honor your father and your mother,'\fnote{\fbackref{15:4} Cf. Exod 20:12; Deut 5:16} and, `Whoever curses father or mother must certainly be put to death.'}\fnote{\fbackref{15:4} Cf. Exod 21:17; Lev 20:9} \v{5}\red{But you say, `Whoever tells his father or his mother, ``Whatever support you might have received from me has been given to God,''}\fnote{\fbackref{15:5} Lit. \fbib{is a gift}} \v{6}\red{does not have to honor his father.'\fnote{\fbackref{15:6} Other mss. read \fbib{his father or his mother}} Because of your tradition, then, you have disregarded\fnote{\fbackref{15:6} Or \fbib{invalidated}} the authority of God's word.}\fnote{\fbackref{15:6} Other mss. read \fbib{law}; still other mss. read \fbib{commandment}} \v{7}\red{You hypocrites! How well did Isaiah prophesy of you when he said,}

\begin{poetry}
\poeml \v{8}\red{`These people honor me with their lips,} \\
\poemll    \red{but their hearts are far from me.} \\
\poeml \v{9}\red{Their worship of me is empty,} \\
\poemll    \red{because they teach human rules as doctrines.'\,''}\fnote{\fbackref{15:9} Cf. Isa 29:13}
\end{poetry}

\v{10}Then calling out to the crowd, he addressed them, \red{``Listen and understand!} \v{11}\red{It is not what goes into the mouth that makes a person unclean. It is what comes out of the mouth that makes a person unclean.''}

\v{12}Then the disciples came and asked him, ``Do you realize that the Pharisees were offended when they heard this statement?''

\v{13}He replied, \red{``Every plant that my heavenly Father did not plant will be pulled up by the roots.} \v{14}\red{Leave them alone. They are blind guides of the blind.\fnote{\fbackref{15:14} Other mss. lack \fbib{of the blind}} If one blind person leads another blind person, both will fall into a ditch.''}

\v{15}Then Peter told him, ``Explain to us this\fnote{\fbackref{15:15} Other mss. read \fbib{the}} parable.''

\v{16}Jesus\fnote{\fbackref{15:16} Lit. \fbib{He}} said, \red{``Are you still so ignorant?} \v{17}\red{Don't you know that everything that goes into the mouth passes into the stomach and then is expelled as waste?} \v{18}\red{But the things that come out of the mouth come from the heart, and it is those things that make a person unclean.} \v{19}\red{It is out of the heart that evil thoughts come, as well as murder, adultery, sexual immorality, stealing, false testimony, and slander.}\fnote{\fbackref{15:19} Or \fbib{blasphemy}} \v{20}\red{These are the things that make a person unclean. But eating with unwashed hands doesn't make a person unclean.''}
\passage{A Canaanite Woman's Faith}
\passageinfo{(Mark 7:24-30)}

\v{21}Then Jesus left that place and went to the region of Tyre and Sidon. \v{22}Suddenly, a Canaanite woman from that territory came near and began to shout, ``Have mercy on me, Lord, Son of David! My daughter is severely demon-possessed!'' \v{23}But he didn't answer her at all.\fnote{\fbackref{15:23} Lit. \fbib{a word}}

Then his disciples came up and kept urging him, ``Send her away, because she keeps on screaming as she follows\fnote{\fbackref{15:23} Lit. \fbib{screaming behind}} us.''

\v{24}But he replied, \red{``I was sent only to the lost sheep of the nation\fnote{\fbackref{15:24} Lit. \fbib{house}} of Israel.''}

\v{25}Then she came and fell down before him, saying, ``Lord, help me!''

\v{26}He replied, \red{``It's not right\fnote{\fbackref{15:26} Other mss. read \fbib{lawful}} to take the children's bread and throw it to the puppies.''}

\v{27}She said, ``Yes, Lord. But even the puppies eat the crumbs that fall from their masters' tables.''

\v{28}Then Jesus answered her, \red{``Lady,\fnote{\fbackref{15:28} Lit. \fbib{Woman}} your faith is great! What you want is granted.''} That very hour her daughter was healed.
\passage{Jesus Heals Many People}

\v{29}Jesus left there and went along the Sea of Galilee. Then he went up on a hillside and sat down. \v{30}Large crowds came to him, bringing with them the lame, the blind, the crippled, those unable to talk, and many others. They placed them at his feet, and he healed them. \v{31}As a result, the crowd was amazed to see those who were unable to talk speaking, the crippled healed, the lame walking, and the blind seeing. So they praised the God of Israel.
\passage{Jesus Feeds More than Four Thousand People}
\passageinfo{(Mark 8:1-10)}

\v{32}Then Jesus called his disciples and said, \red{``I have compassion for the crowd because they have already been with me for three days and have nothing to eat. I don't want to send them away without food, or they may faint on the road.''}

\v{33}The disciples asked him, ``Where in the wilderness are we to get enough bread to feed such a crowd?''

\v{34}Jesus asked them, \red{``How many loaves of bread do you have?''}

They said, ``Seven, and a few small fish.''

\v{35}Ordering the crowd to sit down on the ground, \v{36}he took the seven loaves and the fish and gave thanks. Then he broke them in pieces and kept giving them to his disciples, and the disciples gave them\fnote{\fbackref{15:36} The Gk. lacks \fbib{gave them}} to the crowds. \v{37}All of them ate until they were filled, then the disciples\fnote{\fbackref{15:37} Lit. \fbib{they}} picked up what was left of the broken pieces---seven baskets full. \v{38}Now those who had eaten were four thousand men, besides women and children. \v{39}After he sent the crowds away, he got into a boat and went to the region of Magadan.\fnote{\fbackref{15:39} Other mss. read \fbib{Magdala}}
\labelchapt{16}
\passage{Interpreting the Time}
\passageinfo{(Mark 8:11-13; Luke 12:54-56)}

\chapt{16}
\v{1}When the Pharisees and Sadducees arrived, in order to test Jesus\fnote{\fbackref{16:1} Lit. \fbib{him}} they asked him to show them a sign from heaven. \v{2}He replied to them, \red{``You say,}

\begin{poetry}
\poeml \red{`Red sky at night,} \\
\poemll    \red{what a delight!} \\
\poeml \v{3}\red{Red sky in the morning,} \\
\poemll    \red{cloudy and storming.'}
\end{poetry}

\red{You know how to interpret the appearance of the sky, yet you can't interpret the signs of the times?}\fnote{\fbackref{16:3} Other mss. lack \fbib{You say. . . the signs of the times?}} \v{4}\red{An evil and adulterous generation craves a sign, but no sign will be given to it except the sign of Jonah.''} Then he left them and went away.
\passage{The Yeast of the Pharisees and Sadducees}
\passageinfo{(Mark 8:14-21)}

\v{5}When his disciples reached the other side, they had forgotten to take any bread along. \v{6}Jesus told them, \red{``Watch out! Beware of the yeast of the Pharisees and Sadducees!}''

\v{7}As they began to discuss this among themselves, they kept saying, ``We didn't bring along any bread.''

\v{8}Knowing this, Jesus asked them, \red{``You who have little faith, why are you discussing among yourselves the fact that you don't have any bread?} \v{9}\red{Don't you understand yet? Don't you remember the five loaves for the 5,000 and how many baskets you collected,} \v{10}\red{or the seven loaves for the 4,000 and how many baskets you collected?} \v{11}\red{How can you fail to understand that I wasn't talking to you about bread? Beware of the yeast of the Pharisees and Sadducees!''}

\v{12}Then they understood that he did not say to beware of the yeast used in bread,\fnote{\fbackref{16:12} Other mss. read \fbib{the loaf of bread}} but of the teaching of the Pharisees and Sadducees.
\passage{Peter Declares His Faith in Jesus}
\passageinfo{(Mark 8:27-30; Luke 9:18-21)}

\v{13}When Jesus had come to the region of Caesarea Philippi, he asked his disciples, \red{``Who do people say the Son of Man is?''}

\v{14}They said, ``Some say\fnote{\fbackref{16:14} The Gk. lacks \fbib{say}} John the Baptist, others Elijah, and still others Jeremiah or one of the prophets.''

\v{15}He asked them, \red{``But who do you say I am?''}

\v{16}Simon Peter answered, ``You are the Messiah,\fnote{\fbackref{16:16} Or \fbib{Christ}} the Son of the living God!''

\v{17}Then Jesus told him, \red{``How blessed you are, Simon son of Jonah,\fnote{\fbackref{16:17} Or \fbib{Simon son of John}; Lit. \fbib{Simon bar Jonah}; cf. John 1:42} since flesh and blood has not revealed this to you, though my Father in heaven has.} \v{18}\red{I tell you that you are Peter,\fnote{\fbackref{16:18} Lit. \fbib{Petros}} and it is on this rock\fnote{\fbackref{16:18} Lit. \fbib{petra}} that I will build my congregation,\fnote{\fbackref{16:18} Or \fbib{church}} and the powers of hell\fnote{\fbackref{16:18} Lit. \fbib{the gates of Hades,} a reference to the realm of the dead} will not conquer it.} \v{19}\red{I will give you the keys to the kingdom from\fnote{\fbackref{16:19} Lit. \fbib{of}} heaven. Whatever you prohibit on earth will have been prohibited\fnote{\fbackref{16:19} Or \fbib{will be prohibited}} in heaven, and whatever you permit on earth will have been permitted\fnote{\fbackref{16:19} Or \fbib{will be permitted}} in heaven.''}

\v{20}Then he strictly ordered the disciples not to tell anyone that he was the Messiah.\fnote{\fbackref{16:20} Or \fbib{Christ}}
\passage{Jesus Predicts His Death and Resurrection}
\passageinfo{(Mark 8:31-9:1; Luke 9:21-27)}

\v{21}From that time on, Jesus began to show his disciples that he would have to go to Jerusalem and suffer a great deal because of the elders, the high priests, and the scribes. Then he would be killed, but on the third day he would be raised. \v{22}Peter took him aside and began to rebuke him, saying, ``God be merciful to you, Lord! This must never happen to you!''

\v{23}But Jesus\fnote{\fbackref{16:23} Lit. \fbib{he}} turned and told Peter, \red{``Get behind me, Satan! You are an offense\fnote{\fbackref{16:23} Or \fbib{a hindrance}} to me, because you are not thinking God's thoughts but human thoughts!''}

\v{24}Then Jesus told his disciples, \red{``If anyone wants to follow me, he must deny himself, pick up his cross, and follow me continuously.} \v{25}\red{Whoever wants to save his life will lose it, but whoever loses his life for my sake will find it,} \v{26}\red{because what profit will a person have if he gains the whole world and forfeits his life? Or what can a person give in exchange for his life?} \v{27}\red{The Son of Man is going to come with his angels in his Father's glory, and then he will repay everyone according to what he has done.} \v{28}\red{I tell all of you\fnote{\fbackref{16:28} The Gk. pronoun \fbib{you} is pl.} with certainty, some people standing here will not experience\fnote{\fbackref{16:28} Lit. \fbib{taste}} death before they see the Son of Man coming in his kingdom.''}
\labelchapt{17}
\passage{Jesus' Appearance is Changed}
\passageinfo{(Mark 9:2-13; Luke 9:28-36)}

\chapt{17}
\v{1}Six days later, Jesus took Peter, James, and his brother John and led them up a high mountain by themselves. \v{2}His appearance was changed in front of them, his face shone like the sun, and his clothes became as white as light. \v{3}Suddenly, Moses and Elijah appeared to them, talking with Jesus.\fnote{\fbackref{17:3} Lit. \fbib{him}}

\v{4}Then Peter told Jesus, ``Lord, it's good that we're here! If you want, I'll set up three shelters\fnote{\fbackref{17:4} Or \fbib{tents}}---one for you, one for Moses, and one for Elijah.'' \v{5}He was still speaking when a bright cloud suddenly overshadowed them.

A voice from the cloud said, ``This is my Son, whom I love. I am pleased with him. Keep on listening to him!''

\v{6}When the disciples heard this, they fell on their faces and were terrified.

\v{7}But Jesus came up to them and touched them, saying, \red{``Get up, and stop being afraid.''} \v{8}When they raised their eyes, they saw no one but Jesus all by himself.

\v{9}On their way down the mountain, Jesus ordered them, \red{``Don't tell anyone about this vision until the Son of Man has been raised from the dead.''}

\v{10}So the disciples asked him, ``Why, then, do the scribes say that Elijah must come first?''

\v{11}He answered them, \red{``Elijah is indeed coming and will restore all things.} \v{12}\red{But I tell you that Elijah has already come, yet people\fnote{\fbackref{17:12} Lit. \fbib{they}} did not recognize him and treated him just as they pleased. In the same way, the Son of Man is going to suffer at their hands.''} \v{13}Then the disciples understood that he had been speaking to them about John the Baptist.
\passage{Jesus Heals a Boy with a Demon}
\passageinfo{(Mark 9:14-29; Luke 9:37-42)}

\v{14}As they approached the crowd, a man came up to Jesus,\fnote{\fbackref{17:14} Lit. \fbib{him}} knelt down in front of him, \v{15}and said, ``Sir,\fnote{\fbackref{17:15} Or \fbib{Lord}} have mercy on my son, because he is an epileptic and suffers terribly. Often he falls into fire and often into water. \v{16}I brought him to your disciples, but they couldn't heal him.''

\v{17}Jesus replied, \red{``You unbelieving and perverted generation! How long must I be with you? How long must I put up with you? Bring him here to me!''} \v{18}Then Jesus rebuked the demon and it came out of him, and the boy was healed that very hour.

\v{19}Then the disciples came to Jesus privately and asked, ``Why couldn't we drive it out?''

\v{20}He told them, \red{``Because of your lack of faith.\fnote{\fbackref{17:20} Other mss. read \fbib{your little faith}} I tell all of you\fnote{\fbackref{17:20} The Gk. pronoun \fbib{you} is pl.} with certainty, if you have faith like a grain of mustard seed, you can say to this mountain, `Move from here to there,' and it will move, and nothing will be impossible for you.} \v{21}\red{But this kind does not come out except by prayer and fasting.''}\fnote{\fbackref{17:21} Other mss. lack this verse}
\passage{Jesus Again Predicts His Death and Resurrection}
\passageinfo{(Mark 9:30-32; Luke 9:43-45)}

\v{22}While they were gathering together\fnote{\fbackref{17:22} Other mss. read \fbib{were staying}} in Galilee, Jesus told them, \red{``The Son of Man is going to be betrayed into human hands.} \v{23}\red{They will kill him, but he will be raised on the third day.''} Then they were filled with grief.
\passage{Questions about the Temple Tax}

\v{24}When they came to Capernaum, the collectors of the temple tax\fnote{\fbackref{17:24} Lit. \fbib{didrachma}} came up to Peter and asked, ``Your teacher pays the temple tax,\fnote{\fbackref{17:24} Lit. \fbib{didrachma}} doesn't he?''

\v{25}He answered, ``Yes.''

When Peter\fnote{\fbackref{17:25} Lit. \fbib{he}} went home,\fnote{\fbackref{17:25} Or \fbib{went into the house}} Jesus spoke to him first and asked him, \red{``What do you think, Simon? From whom do kings on the earth collect tolls or tributes? From their own subjects,\fnote{\fbackref{17:25} Lit. \fbib{sons}} or from foreigners?''}

\v{26}``From foreigners,'' he replied.

So Jesus told him, \red{``In that case, the subjects\fnote{\fbackref{17:26} Lit. \fbib{sons}} are exempt.} \v{27}\red{However, so that we don't offend them, go to the sea and throw in a hook. Take the first fish that comes up, open its mouth, and you will find a coin.\fnote{\fbackref{17:27} Lit. \fbib{stater,} a coin worth two didrachmas} Take it and give it to them for me and you.''}
\labelchapt{18}
\passage{True Greatness}
\passageinfo{(Mark 9:33-37; Luke 9:46-48)}

\chapt{18}
\v{1}At that time the disciples came to Jesus and asked, ``Who, then, is the greatest in the kingdom from\fnote{\fbackref{18:1} Lit. \fbib{of}} heaven?''

\v{2}Calling a little child forward, he had him stand among them. \v{3}Then he said, \red{``I tell all of you\fnote{\fbackref{18:3} The Gk. pronoun \fbib{you} is pl.} with certainty, unless you change\fnote{\fbackref{18:3} Lit. \fbib{turn}} and become like little children, you will never get into the kingdom from\fnote{\fbackref{18:3} Lit. \fbib{of}} heaven.} \v{4}\red{Therefore, whoever humbles himself as this little child is the greatest in the kingdom from\fnote{\fbackref{18:4} Lit. \fbib{of}} heaven,} \v{5}\red{and whoever receives a little child like this in my name receives me.''}
\passage{Causing Others to Sin}
\passageinfo{(Mark 9:42-48; Luke 17:1-2)}

\v{6}\red{``If anyone causes one of these little ones who believe in me to sin, it would be better for him if a large millstone were hung around his neck and he were drowned at the bottom of the sea.} \v{7}\red{How terrible it will be for the world due to its temptations to sin! Temptations to sin are bound to happen, but how terrible it will be for that person who causes someone to sin!}

\v{8}\red{``So if your hand or your foot causes you to sin, cut it off and throw it away. It is better for you to enter life injured or crippled than to have two hands or two feet and be thrown into eternal fire.} \v{9}\red{And if your eye causes you to sin, tear it out and throw it away. It is better for you to enter life with one eye than to have two eyes and be thrown into hell\fnote{\fbackref{18:9} Lit. \fbib{Gehenna}; a Gk. transliteration of the Heb. for \fbib{Valley of Hinnom}} fire.}

\v{10}\red{``See to it that you do not despise one of these little ones, because I tell you, their angels in heaven always have access to my Father in heaven.} \v{11}\red{For the Son of Man came to save the lost.''}\fnote{\fbackref{18:11} Other mss. lack this verse.}
\passage{The Parable about the Faithful Shepherd}
\passageinfo{(Luke 15:1-7)}

\v{12}\red{``What do you think? If a man has 100 sheep and one of them strays, he leaves the 99 in the hills and goes to look for the one that has strayed, doesn't he?} \v{13}\red{If he finds it, I tell all of you\fnote{\fbackref{18:13} The Gk. pronoun \fbib{you} is pl.} with certainty that he rejoices over it more than over the 99 that haven't strayed.} \v{14}\red{In the same way, it is not the will of your\fnote{\fbackref{18:14} Other mss. read \fbib{our}; still other mss. read \fbib{my}} Father in heaven that one of these little ones should be lost.''}
\passage{Dealing with a Brother who Sins}
\passageinfo{(Luke 17:3)}

\v{15}\red{``If your brother sins against you,\fnote{\fbackref{18:15} Other mss. lack \fbib{against you}} go and confront him while the two of you are alone. If he listens to you, you have won back your brother.} \v{16}\red{But if he doesn't listen, take one or two others with you so that `every word may be confirmed by the testimony\fnote{\fbackref{18:16} Lit. \fbib{mouth}} of two or three witnesses.'}\fnote{\fbackref{18:16} Cf. Deut 19:15} \v{17}\red{If, however, he ignores them, tell it to the congregation.\fnote{\fbackref{18:17} Or \fbib{church}} If he also ignores the congregation,\fnote{\fbackref{18:17} Or \fbib{church}} regard him as an unbeliever\fnote{\fbackref{18:17} Lit. \fbib{gentile} ; i.e. an unbelieving non-Jew} and a tax collector.}

\v{18}\red{``I tell all of you\fnote{\fbackref{18:18} The Gk. pronoun \fbib{you} is pl.} with certainty, whatever you prohibit on earth will have been prohibited\fnote{\fbackref{18:18} Or \fbib{will be prohibited}} in heaven, and whatever you permit on earth will have been permitted\fnote{\fbackref{18:18} Or \fbib{will be permitted}} in heaven.} \v{19}\red{Furthermore, I tell all of you\fnote{\fbackref{18:19} The Gk. pronoun \fbib{you} is pl.} with certainty that if two of you agree on earth about anything you request, it will be done for you by my Father in heaven,} \v{20}\red{because where two or three have come together in my name, I am there among them.''}
\passage{The Parable about an Unforgiving Servant}

\v{21}Then Peter came up and asked him, ``Lord, how many times may my brother sin against me and I have to forgive him? Seven times?''

\v{22}Jesus told him, \red{``I tell you, not just seven times, but 77 times!}\fnote{\fbackref{18:22} Or \fbib{seventy times seven}} \v{23}\red{``That is why the kingdom from\fnote{\fbackref{18:23} Lit. \fbib{of}} heaven may be compared to a king who wanted to settle accounts with his servants.} \v{24}\red{When he had begun to settle the accounts, a person who owed him 10,000 talents\fnote{\fbackref{18:24} 10,000 talents is the price paid in silver by Haman to King Ahasuerus as a bribe to annihilate the Jews; Cf. Esther 3:9; a talent was worth a lifetime of wages for an average laborer} was brought to him.} \v{25}\red{Because he couldn't pay, his master ordered him, his wife, his children, and everything that he owned to be sold so that payment could be made.} \v{26}\red{Then the servant fell down and bowed low before him, saying, `Be patient\fnote{\fbackref{18:26} Other mss. read \fbib{Master, be patient}} with me, and I will repay you everything!'} \v{27}\red{The master of that servant had compassion and released him, canceling his debt.}

\v{28}\red{``But when that servant went away, he found one of his fellow servants who owed him a hundred denarii.\fnote{\fbackref{18:28} The denarius was the usual day's wage for a laborer.} He grabbed him, seized him by the throat, and said, `Pay what you owe!'} \v{29}\red{Then his fellow servant fell down and began begging him, `Be patient with me and I will repay you!'} \v{30}\red{But he refused and had him thrown into prison until he could repay the debt.}

\v{31}\red{``When his fellow servants saw what had happened, they were very disturbed and went and reported to their master everything that had occurred.} \v{32}\red{Then his master sent for him and told him, `You evil servant! I canceled that entire debt for you because you begged me.} \v{33}\red{Shouldn't you have had mercy on your fellow servant, just as I had mercy on you?'} \v{34}\red{In anger his master handed him over to the jailers until he could repay the entire debt.} \v{35}\red{This is how my heavenly Father will treat each one of you unless you forgive your brother from your hearts.''}
\labelchapt{19}
\passage{Teaching about Divorce}
\passageinfo{(Mark 10:1-12)}

\chapt{19}
\v{1}When Jesus had finished saying these things,\fnote{\fbackref{19:1} Lit. \fbib{finished these sayings}} he left Galilee and went to the territory of Judea on the other side\fnote{\fbackref{19:1} I.e. the east side} of the Jordan. \v{2}Large crowds followed him, and he healed them there.

\v{3}Some Pharisees came to him in order to test him. They asked, ``Is it lawful for a man\fnote{\fbackref{19:3} Other mss. lack \fbib{for a man}} to divorce his wife for any reason?''

\v{4}He answered them, \red{``Haven't you read that the one who made\fnote{\fbackref{19:4} Other mss. read \fbib{created}} them at the beginning `made them male and female'}\fnote{\fbackref{19:4} Cf. Gen 1:27; 5:2} \v{5}\red{and said, `That is why a man will leave his father and mother and be united with his wife, and the two will become one flesh'?}\fnote{\fbackref{19:5} Cf. Gen 2:24} \v{6}\red{So they are no longer two, but one flesh. Therefore, what God has joined together, man must never separate.''}

\v{7}They asked him, ``Why, then, did Moses order us `to give a certificate of divorce and divorce her'?''\fnote{\fbackref{19:7} Cf. Deut 24:1, 3}

\v{8}He told them, \red{``It was because of your hardness of heart that Moses allowed you to divorce your wives. But from the beginning it was not this way.} \v{9}\red{I tell you that whoever divorces his wife, except for sexual immorality, and marries another woman commits adultery.''}\fnote{\fbackref{19:9} Other mss. read \fbib{adultery, and the man who marries a divorced woman commits adultery}}

\v{10}His disciples asked him, ``If that is the relationship of a man with his wife, it's not worth getting married!''

\v{11}\red{``Not everyone can accept this saying,''} he replied, \red{``except those to whom celibacy\fnote{\fbackref{19:11} Lit. \fbib{it}} has been granted,} \v{12}\red{because some men are celibate from birth,\fnote{\fbackref{19:12} Lit. \fbib{from the mother's womb}} while some are celibate because they have been made that way by others. Still others are celibate because they have made themselves that way for the sake of the kingdom from\fnote{\fbackref{19:12} Lit. \fbib{of}} heaven. Let anyone accept this who can.''}
\passage{Jesus Blesses the Little Children}
\passageinfo{(Mark 10:13-16; Luke 18:15-17)}

\v{13}Then some little children were brought to him so that he might lay his hands on them and pray. But the disciples rebuked those who brought\fnote{\fbackref{19:13} The Gk. lacks \fbib{those who brought}} them. \v{14}Jesus, however, said, \red{``Let the little children come to me, and stop keeping them away, because the kingdom from\fnote{\fbackref{19:14} Lit. \fbib{of}} heaven belongs to people like these.''} \v{15}When he had laid his hands on them, he went on from there.
\passage{A Rich Man Comes to Jesus}
\passageinfo{(Mark 10:17-22; Luke 18:18-23)}

\v{16}Just then a man came up to Jesus.\fnote{\fbackref{19:16} Lit. \fbib{him}} ``Teacher,''\fnote{\fbackref{19:16} Other mss. read \fbib{Good Teacher}} he asked, ``what good deed should I do to have eternal life?''

\v{17}Jesus\fnote{\fbackref{19:17} Lit. \fbib{He}} asked him, \red{``Why ask me about what is good? There is only one who is good.\fnote{\fbackref{19:17} Other mss. read \fbib{Why do you call me good? No one is good except for one---God}} If you want to get into that life, you must keep the commandments.''}

\v{18}The young man\fnote{\fbackref{19:18} Lit. \fbib{He}} asked him, ``Which ones?''

Jesus said, \red{```You must not murder,\fnote{\fbackref{19:18} Cf. Exod 20:13; Deut 5:17} you must not commit adultery,\fnote{\fbackref{19:18} Cf. Exod 20:14; Deut 5:18} you must not steal,\fnote{\fbackref{19:18} Cf. Exod 20:15; Deut 5:19} you must not give false testimony,}\fnote{\fbackref{19:18} Cf. Exod 20:16; Deut 5:20} \v{19}\red{honor your father and mother,'\fnote{\fbackref{19:19} Cf. Exod 20:12; Deut 5:16} and, `you must love your neighbor as yourself.'\,''}\fnote{\fbackref{19:19} Cf. Lev 19:18}

\v{20}The young man told him, ``I have kept all of these.\fnote{\fbackref{19:20} Other mss. read \fbib{kept all of these since I was a young man}} What do I still lack?''

\v{21}Jesus told him, \red{``If you want to be perfect, go and sell what you own and give the money\fnote{\fbackref{19:21} The Gk. lacks \fbib{the money}} to the destitute, and you will have treasure in heaven. Then come back and follow me.''} \v{22}But when the young man heard this statement he went away sad, because he had many possessions.
\passage{Salvation and Reward}
\passageinfo{(Mark 10:23-31; Luke 18:24-30)}

\v{23}Then Jesus told his disciples, \red{``I tell all of you\fnote{\fbackref{19:23} The Gk. pronoun \fbib{you} is pl.} with certainty, it will be hard for a rich person to get into the kingdom from\fnote{\fbackref{19:23} Lit. \fbib{of}} heaven.} \v{24}\red{Again I tell you, it is easier for a camel to squeeze through the eye of a needle than for a rich person to get into the kingdom of God.''}

\v{25}When the disciples heard this, they were completely astonished. ``Who, then, can be saved?'' they asked.

\v{26}Jesus looked at them intently and said, \red{``For humans this is impossible, but for God all things are possible.''}

\v{27}``Look!'' Peter replied. ``We have left everything and followed you. So what will we get?''

\v{28}Jesus told them, \red{``I tell all of you\fnote{\fbackref{19:28} The Gk. pronoun \fbib{you} is pl.} with certainty, when the Son of Man sits on his glorious throne in the renewed creation, you who have followed me will also sit on twelve thrones, governing the twelve tribes of Israel.} \v{29}\red{In fact, everyone who has left his homes, brothers, sisters, father, mother, children, or fields because of my name will receive a hundred times as much\fnote{\fbackref{19:29} Other mss. read \fbib{many times as much}} and will inherit eternal life.} \v{30}\red{But many who are first will be last, and the last will be first.''}
\labelchapt{20}
\passage{The Workers in the Vineyard}

\chapt{20}
\v{1}\red{``The kingdom from\fnote{\fbackref{20:1} Lit. \fbib{of}} heaven is like a landowner who went out early in the morning to hire workers for his vineyard.} \v{2}\red{After agreeing to pay the workers one denarius\fnote{\fbackref{20:2} The denarius was the usual day's wage for a laborer.} a day, he sent them into his vineyard.} \v{3}\red{When he went out about nine o'clock,\fnote{\fbackref{20:3} Lit. \fbib{the third hour}} he saw others standing in the marketplace without work.} \v{4}\red{He told them, `You go into the vineyard, too, and I will pay you whatever is right.'} \v{5}\red{So off they went. He went out again about noon\fnote{\fbackref{20:5} Lit. \fbib{the sixth hour}} and about three o'clock\fnote{\fbackref{20:5} Lit. \fbib{the ninth hour}} and did the same thing.} \v{6}\red{About five o'clock\fnote{\fbackref{20:6} Lit. \fbib{the eleventh hour}} he went out and found some others standing around. He asked them, `Why are you standing here all day long without work?'} \v{7}\red{They told him, `Because no one has hired us.' He told them, `You go into the vineyard as well.'}

\v{8}\red{``When evening came, the owner of the vineyard told his manager, `Call the workers and give them their wages, beginning with the last and ending with\fnote{\fbackref{20:8} Lit. \fbib{and up to}} the first.'} \v{9}\red{Those who were hired at five o'clock\fnote{\fbackref{20:9} Lit. \fbib{the eleventh hour}} came, and each received a denarius.}

\v{10}\red{``When the first came, they thought they would receive more, but each received a denarius as well.} \v{11}\red{When they received it, they began to complain to the landowner,} \v{12}\red{`These last fellows worked only one hour, but you paid them the same as us, and we've been working all day,\fnote{\fbackref{20:12} Lit. \fbib{we've endured the burden of the day}} enduring the scorching heat!'}

\v{13}\red{``But he told one of them, `Friend, I'm not treating you unfairly. You did agree with me for a denarius, didn't you?} \v{14}\red{Take what is yours and go. I want to give this last man as much as I gave you.}\fnote{\fbackref{20:14} Lit. \fbib{to this last man as also to you}} \v{15}\red{I am allowed to do what I want with my own money,\fnote{\fbackref{20:15} Lit. \fbib{things}} am I not? Or are you envious\fnote{\fbackref{20:15} Lit. \fbib{Or is your eye evil}} because I'm generous?'}

\v{16}\red{``In the same way, the last will be first, and the first will be last, because many are called, but few are chosen.''}\fnote{\fbackref{20:16} Other mss. lack \fbib{For many are called, but few are chosen}}
\passage{Jesus Predicts His Death and Resurrection a Third Time}
\passageinfo{(Mark 10:32-34; Luke 18:31-34)}

\v{17}When Jesus was going up to Jerusalem, he took the twelve disciples\fnote{\fbackref{20:17} Other mss. lack \fbib{disciples}} aside and told them as they were walking along, \v{18}\red{``See, we are going up to Jerusalem, and the Son of Man will be handed over to the high priests and scribes, and they will condemn him to death.} \v{19}\red{Then they will hand him over to unbelievers\fnote{\fbackref{20:19} Lit. \fbib{to the gentiles} ; i.e. unbelieving non-Jews} to be mocked, whipped, and crucified, but on the third day he will be raised.''}
\passage{The Request of James and John}
\passageinfo{(Mark 10:35-45)}

\v{20}Then the mother of Zebedee's sons came to Jesus\fnote{\fbackref{20:20} Lit. \fbib{him}} with her sons. She bowed down in front of him to ask him for a favor. \v{21}He asked her, \red{``What do you want?''}

She told him, ``Promise\fnote{\fbackref{20:21} Lit. \fbib{Say}} that in your kingdom these two sons of mine will sit on your right and on your left.''

\v{22}Jesus replied, \red{``You don't realize what you're asking. Can you drink from the cup that I'm going to drink from?''}\fnote{\fbackref{20:22} Other mss. read \fbib{to drink from, or be baptized with the baptism with which I'm going to be baptized?}}

They told him, ``We can.''

\v{23}He told them, \red{``You will indeed drink from my cup. But it's not up to me to grant you a seat at my right hand or at my left. These positions have already been prepared for others by my Father.''}

\v{24}When the ten heard this, they became furious with the two brothers. \v{25}But Jesus called the disciples\fnote{\fbackref{20:25} Lit. \fbib{them}} and said, \red{``You know that the rulers of the unbelievers\fnote{\fbackref{20:25} Lit. \fbib{gentiles} ; i.e. unbelieving non-Jews} lord it over them and their superiors act like tyrants over them.} \v{26}\red{That's not the way it should be among you. Instead, whoever wants to be great among you must be your servant,} \v{27}\red{and whoever wants to be first among you must be your slave.} \v{28}\red{That's the way it is with the Son of Man. He did not come to be served, but to serve and to give his life as a ransom for many people.''}
\passage{Jesus Heals Two Blind Men}
\passageinfo{(Mark 10:46-52; Luke 18:35-43)}

\v{29}As they were leaving Jericho, a large crowd followed Jesus.\fnote{\fbackref{20:29} Lit. \fbib{him}} \v{30}When two blind men who were sitting by the roadside heard that Jesus was passing by, they shouted, ``Have mercy on us, Lord,\fnote{\fbackref{20:30} Other mss. read \fbib{Jesus}} Son of David!'' \v{31}When the crowd told them harshly to be silent, they shouted even louder, ``Have mercy on us, Lord, Son of David!''

\v{32}Jesus stopped and called them, saying, \red{``What do you want me to do for you?''}

\v{33}They told him, ``Lord, we want to be able to see!''\fnote{\fbackref{20:33} Lit. \fbib{Lord, that our eyes be opened}} \v{34}Then Jesus, deeply moved with compassion, touched their eyes, and at once they could see again. So they followed him.
\labelchapt{21}
\passage{The King Enters Jerusalem}
\passageinfo{(Mark 11:1-11; Luke 19:28-38; John 12:12-19)}

\chapt{21}
\v{1}When they came near Jerusalem and had reached Bethphage on the Mount of Olives, Jesus sent two disciples on ahead and \v{2}told them, \red{``Go into the village ahead of you. At once you will find a donkey tied up and a colt with it. Untie them, and bring them to me.} \v{3}\red{If anyone says anything to you, tell him, `The Lord needs them,' and that person will send them at once.''}

\v{4}Now this happened to fulfill what had been spoken through the prophet when he said,

\begin{poetry}
\poeml \v{5}``Tell the daughter\fnote{\fbackref{21:5} I.e. people} of Zion, \\
\poemll    `Look, your king is coming to you!\fnote{\fbackref{21:5} Cf. Isa 62:11} \\
\poeml He is humble and mounted on a donkey, \\
\poemll    even on a colt of a donkey.'\,''\fnote{\fbackref{21:5} Cf. Zech 9:9}
\end{poetry}

\v{6}So the disciples went and did as Jesus had directed them. \v{7}They brought the donkey and the colt and put their coats on them, and he sat upon them. \v{8}Many people in the crowd spread their own coats on the road, while others began cutting down branches from the trees and spreading them on the road. \v{9}Both the crowds that went ahead of him and those that followed him kept shouting,

\begin{poetry}
\poeml ``Hosanna\fnote{\fbackref{21:9} Hosanna is Heb. for \fbib{Please save} or \fbib{Praise.}} to the Son of David! \\
\poeml How blessed is the one who comes \\
\poemll    in the name of the Lord!\fnote{\fbackref{21:9} MT source citation reads \fbib{\divine{Lord}}} \\
\poeml Hosanna in the highest heaven!''\fnote{\fbackref{21:9} Cf. Ps 118:25-26; Ps 148:1}
\end{poetry}

\v{10}When he came into Jerusalem, the whole city was trembling with excitement. The people\fnote{\fbackref{21:10} Lit. \fbib{They}} were asking, ``Who is this?''

\v{11}The crowds kept saying, ``This is the prophet Jesus, the man from Nazareth in Galilee.''
\passage{Confrontation in the Temple over Money}
\passageinfo{(Mark 11:15-19; Luke 19:45-48; John 2:13-22)}

\v{12}Then Jesus went into the Temple,\fnote{\fbackref{21:12} Other mss. read \fbib{temple of God}} threw out everyone who was selling and buying in the Temple, and overturned the moneychangers' tables and the chairs of those who sold doves. \v{13}He told them, \red{``It is written, `My house is to be called a house of prayer,'\fnote{\fbackref{21:13} Cf. Isa 56:7; Jer 7:11} but you are turning it into a hideout\fnote{\fbackref{21:13} Lit. \fbib{cave}} for bandits!''}

\v{14}Blind and lame people came to him in the Temple, and he healed them. \v{15}But when the high priests and the scribes saw the amazing things that he had done and the children shouting in the Temple, ``Hosanna\fnote{\fbackref{21:15} Hosanna is Heb. for \fbib{Please save} or \fbib{Praise.}} to the Son of David,'' they became furious \v{16}and asked him, ``Do you hear what these people are saying?''

Jesus told them, \red{``Yes! Haven't you ever read, `From the mouths of infants and nursing babies you have created praise'?''}\fnote{\fbackref{21:16} Cf. Ps 8:2} \v{17}Then he left them and went out of the city to Bethany and spent the night there.
\passage{Jesus Curses a Fig Tree}
\passageinfo{(Mark 11:12-14, 20-24)}

\v{18}In the morning, as Jesus\fnote{\fbackref{21:18} Lit. \fbib{he}} was returning to the city, he became hungry. \v{19}Seeing a fig tree by the roadside, he went up to it but found nothing on it except leaves. He told it, \red{``May fruit never come from you again!''} And immediately the fig tree dried up.

\v{20}When the disciples saw this, they were amazed. ``How did the fig tree dry up so quickly?'' they asked.

\v{21}Jesus answered them, \red{``I tell all of you\fnote{\fbackref{21:21} The Gk. pronoun \fbib{you} is pl.} with certainty, if you have faith and do not doubt, not only will you be able to do what has been done to the fig tree, but you will also say to this mountain, `Be removed and thrown into the sea,' and it will happen.} \v{22}\red{You will receive whatever you ask for in prayer, if you believe.''}
\passage{Jesus' Authority is Challenged}
\passageinfo{(Mark 11:27-33; Luke 20:1-8)}

\v{23}Then Jesus\fnote{\fbackref{21:23} Lit. \fbib{he}} went into the Temple. While he was teaching, the high priests and the elders of the people came to him and asked, ``By what authority are you doing these things, and who gave you this authority?''

\v{24}Jesus answered them, \red{``I, too, will ask you one question.\fnote{\fbackref{21:24} Lit. \fbib{one word}} If you answer it for me, I will also tell you by what authority I am doing these things.} \v{25}\red{Where did John's authority to baptize\fnote{\fbackref{21:25} Lit. \fbib{John's baptism}} come from? From heaven or from humans?''}

They began discussing this among themselves: ``If we say, `From heaven,' he will ask us, `Then why didn't you believe him?' \v{26}But if we say, `From humans,' we are afraid of the crowd, because everyone regards John as a prophet.'' \v{27}So they told Jesus, ``We don't know.''

He in turn told them, \red{``Then I won't tell you by what authority I am doing these things.''}
\passage{The Parable about Two Sons}

\v{28}\red{``But what do you think? A man had two sons. He went to the first and said, `Son, go and work in the vineyard today.'} \v{29}\red{His son\fnote{\fbackref{21:29} Lit. \fbib{He}} replied, `I don't want to,' but later he changed his mind and went.} \v{30}\red{Then the father\fnote{\fbackref{21:30} Lit. \fbib{he}} went to the other son\fnote{\fbackref{21:30} The Gk. lacks \fbib{son}} and told him the same thing. He replied, `I will,\fnote{\fbackref{21:30} The Gk. lacks \fbib{will}} sir,' but he didn't go.} \v{31}\red{Which of the two did the father's will?''}

They answered, ``The first one.''

Jesus told them,\red{ ``I tell all of you\fnote{\fbackref{21:31} The Gk. pronoun \fbib{you} is pl.} with certainty, tax collectors and prostitutes will get into God's kingdom ahead of you!} \v{32}\red{John came to you living a righteous life,\fnote{\fbackref{21:32} Lit. \fbib{you in the way of righteousness}} and you didn't believe him, but the tax collectors and prostitutes did. But even when you saw that, you didn't change your minds\fnote{\fbackref{21:32} Or \fbib{didn't repent}} at last and believe him.''}
\passage{The Parable about the Tenant Farmers}
\passageinfo{(Mark 12:1-12; Luke 20:9-19)}

\v{33}\red{``Listen to another parable. There was a landowner who planted a vineyard, put a wall around it, dug a wine press in it, and built a watchtower. Then he leased it to tenant farmers and went abroad.} \v{34}\red{When harvest time approached, he sent his servants to the tenant farmers to collect his produce.} \v{35}\red{But the farmers took his servants and beat one, killed another, and attacked another with stones.} \v{36}\red{Again, he sent other servants to them, a greater number than the first, but the tenant farmers\fnote{\fbackref{21:36} Lit. \fbib{they}} treated them the same way.} \v{37}\red{Finally, he sent his son to them, thinking, `They will respect my son.'} \v{38}\red{But when the tenant farmers saw his son, they told one another, `This is the heir. Come on, let's kill him and get his inheritance!'} \v{39}\red{So they grabbed him, threw him out of the vineyard, and killed him.} \v{40}\red{Now when the owner of the vineyard returns, what will he do to those farmers?''}

\v{41}They told him, ``He will put those horrible men to a horrible death. Then he will lease the vineyard to other farmers who will give him his produce at harvest time.''

\v{42}Jesus asked them, \red{``Have you never read in the Scriptures,}

\begin{poetry}
\poeml \red{`The stone that the builders rejected} \\
\poemll    \red{has become the cornerstone.}\fnote{\fbackref{21:42} Or \fbib{capstone}} \\
\poeml \red{This was the Lord's\fnote{\fbackref{21:42} MT source citation reads \fbib{\divine{Lord}'s}} doing,} \\
\poemll    \red{and it is amazing in our eyes.'?}\fnote{\fbackref{21:42} Cf. Ps 118:22-23}
\end{poetry}

\v{43}\red{That is why I tell you that the kingdom of God will be taken away from you and given to a people who will produce fruit for it.} \v{44}\red{The person who falls over this stone will be broken to pieces, but it will crush anyone on whom it falls.''}\fnote{\fbackref{21:44} Other mss. lack this verse}

\v{45}When the high priests and the Pharisees heard his parables, they knew that he was talking about them. \v{46}Although they wanted to arrest him, they were afraid of the crowds, who considered Jesus\fnote{\fbackref{21:46} Lit. \fbib{him}} to be a prophet.
\labelchapt{22}
\passage{The Parable about a Banquet}
\passageinfo{(Luke 14:15-24)}

\chapt{22}
\v{1}Again Jesus spoke to them in parables. He said, \v{2}\red{``The kingdom from\fnote{\fbackref{22:2} Lit. \fbib{of}} heaven may be compared to a king who gave a wedding banquet for his son.} \v{3}\red{He sent his servants to call those who had been invited to the wedding, but they refused to come.} \v{4}\red{So\fnote{\fbackref{22:4} Lit. \fbib{Again}} he sent other servants after saying, `Tell those who have been invited, ``Look! I've prepared my dinner. My oxen and fattened calves have been slaughtered. Everything is ready. Come to the wedding!''\,'} \v{5}\red{But they paid no attention to this and went away, one to his farm, another to his business.} \v{6}\red{The rest grabbed the king's\fnote{\fbackref{22:6} Lit. \fbib{his}} servants, treated them brutally, and then killed them.} \v{7}\red{Then the king became outraged. He sent his troops, and they destroyed those murderers and burned their city.}

\v{8}\red{``Then he told his servants, `The wedding is ready, but those who were invited were not worthy.} \v{9}\red{So go into the roads leading out of town and invite as many people as you can find to the wedding.'} \v{10}\red{Those servants went out into the streets and brought in everyone they found, evil and good alike, and the wedding hall was packed with guests.}

\v{11}\red{``When the king came in to see the guests, he noticed a man there who was not wearing wedding clothes.} \v{12}\red{He asked him, `Friend, how did you get in here without wedding clothes?' But the man\fnote{\fbackref{22:12} Lit. \fbib{he}} was speechless.} \v{13}\red{Then the king told his servants, `Tie his hands and feet, and throw him into the darkness outside!' In that place there will be weeping and gnashing of teeth,}\fnote{\fbackref{22:13} I.e. extreme pain} \v{14}\red{because many are invited, but few are chosen.''}
\passage{A Question about Paying Taxes}
\passageinfo{(Mark 12:13-17; Luke 20:20-26)}

\v{15}Then the Pharisees went and planned how to trap Jesus\fnote{\fbackref{22:15} Lit. \fbib{him}} in conversation. \v{16}They sent their disciples to him along with the Herodians.\fnote{\fbackref{22:16} I.e. Royal party sympathizers} They said, ``Teacher, we know that you are sincere and that you teach the way of God truthfully. You don't favor any individual, because you pay no attention to external appearance. \v{17}So tell us what you think. Is it lawful to pay taxes to Caesar or not?''

\v{18}Recognizing their wickedness, Jesus asked, \red{``Why are you testing me, you hypocrites?} \v{19}\red{Show me the coin used for the tax.''}

They brought him a denarius.\fnote{\fbackref{22:19} The denarius was the usual day's wage for a laborer.} \v{20}Then he asked them, \red{``Whose face and name is this?''}

\v{21}They told him, ``Caesar's.''

So he told them, \red{``Then give back to Caesar the things that are Caesar's, and to God the things that are God's.''}

\v{22}When they heard this, they were amazed. Then they left him and went away.
\passage{A Question about the Resurrection}
\passageinfo{(Mark 12:18-27; Luke 20:27-40)}

\v{23}That same day some Sadducees, who claim there is no resurrection, came to Jesus\fnote{\fbackref{22:23} Lit. \fbib{him}} and asked him, \v{24}``Teacher, Moses said, `If a man dies having no children, his brother must marry the widow and have children for his brother.'\fnote{\fbackref{22:24} Cf. Deut 25:5-6} \v{25}Now there were seven brothers among us. The first one married and died, and since he had no children, he left his widow to his brother. \v{26}The same thing happened with the second brother, and then the third, and finally with the rest of the brothers.\fnote{\fbackref{22:26} Lit. \fbib{with the seven}} \v{27}Finally, the woman died, too. \v{28}Now in the resurrection, whose wife of the seven will she be, since all of them had married\fnote{\fbackref{22:28} The Gk. lacks \fbib{married}} her?''

\v{29}Jesus answered them, \red{``You are mistaken because you don't know the Scriptures or God's power,} \v{30}\red{because in the resurrection, people\fnote{\fbackref{22:30} Lit. \fbib{they}} neither marry nor are given in marriage but are like the angels\fnote{\fbackref{22:30} Other mss. read \fbib{God's angels}} in heaven.} \v{31}\red{As for the resurrection from the dead, haven't you read what was spoken to you by God when he said,} \v{32}\red{`I am the God of Abraham, the God of Isaac, and the God of Jacob'?\fnote{\fbackref{22:32} Cf. Exod 3:6} He\fnote{\fbackref{22:32} Other mss. read \fbib{God}} is not the God of the dead, but of the living.''}

\v{33}When the crowds heard this, they were amazed at his teaching.
\passage{The Greatest Commandment}
\passageinfo{(Mark 12:28-34; Luke 10:25-28)}

\v{34}When the Pharisees heard that Jesus\fnote{\fbackref{22:34} Lit. \fbib{he}} had silenced the Sadducees, they met together in the same place. \v{35}One of them, an expert in the Law, tested him by asking, \v{36}``Teacher, which is the greatest commandment in the Law?''

\v{37}Jesus\fnote{\fbackref{22:37} Lit. \fbib{He}} told him, \red{```You must love the Lord your God with all your heart, with all your soul, and with all your mind.'}\fnote{\fbackref{22:37} Cf. Deut 6:5} \v{38}\red{This is the greatest and most important\fnote{\fbackref{22:38} Or \fbib{first}} commandment.} \v{39}\red{The second is exactly like it: `You must love your neighbor as yourself.'}\fnote{\fbackref{22:39} Cf. Lev 19:18} \v{40}\red{All the Law and the Prophets depend on these two commandments.''}
\passage{A Question about David's Son}
\passageinfo{(Mark 12:35-37; Luke 20:41-44)}

\v{41}While the Pharisees were still\fnote{\fbackref{22:41} The Gk. lacks \fbib{still}} gathered, Jesus asked them, \v{42}\red{``What do you think about the Messiah?\fnote{\fbackref{22:42} Or \fbib{Christ}} Whose son is he?''}

They told him, ``David's.''

\v{43}He asked them, \red{``Then how can David by the Spirit call him `Lord' when he says,}

\begin{poetry}
\poeml \v{44}\red{`The Lord\fnote{\fbackref{22:44} MT source citation reads \fbib{}\divine{Lord}} told my Lord,} \\
\poemll    \red{``Sit at my right hand,} \\
\poemlll       \red{until I put your enemies under your feet.''\,'?}\fnote{\fbackref{22:44} Cf. Ps 110:1; MT source citation reads \fbib{}\divine{Lord}}
\end{poetry}

\v{45}\red{If David calls him `Lord', how can he be his son?''}

\v{46}No one could answer him at all,\fnote{\fbackref{22:46} Lit. \fbib{a word}} and from that day on no one dared to ask him another question.
\labelchapt{23}
\passage{Jesus Denounces the Scribes and the Pharisees}
\passageinfo{(Mark 12:38-40; Luke 20:45-47)}

\chapt{23}
\v{1}Then Jesus told the crowds and his disciples, \v{2}\red{``The scribes and the Pharisees administer the authority of Moses,}\fnote{\fbackref{23:2} Lit. \fbib{Pharisees sit in Moses' seat}} \v{3}\red{so do whatever they tell you and follow it, but stop doing what they do, because they don't do what they say.} \v{4}\red{They tie up burdens that are heavy and unbearable and lay them on people's shoulders, but they refuse to lift a finger to remove them.}

\v{5}\red{``They do everything to be seen by people. They increase the size of their phylacteries\fnote{\fbackref{23:5} I.e. leather cases containing Scripture texts} and lengthen the tassels of their garments.} \v{6}\red{They love to have the places of honor at festivals, the best seats in the synagogues,} \v{7}\red{to be greeted in the marketplaces, and to be called `Rabbi'\fnote{\fbackref{23:7} \fbib{Rabbi} is Heb. for \fbib{Master} and/or \fbib{Teacher}.} by people.}

\v{8}\red{``But you are not to be called `Rabbi,'\fnote{\fbackref{23:8} \fbib{Rabbi} is Heb. for Master and/or Teacher} because you have only one teacher, and all of you are brothers.} \v{9}\red{And don't call anyone on earth `Father,' because you have only one Father, the one in heaven.} \v{10}\red{Nor are you to be called `Teachers,' because you have only one teacher, the Messiah!}\fnote{\fbackref{23:10} Or \fbib{Christ}} \v{11}\red{The person who is greatest among you must be your servant.} \v{12}\red{Whoever exalts himself will be humbled, and whoever humbles himself will be exalted.}

\v{13}\red{``How terrible it will be for you, scribes and Pharisees, you hypocrites! You shut the door to the kingdom from\fnote{\fbackref{23:13} Lit. \fbib{of}} heaven in people's faces. You don't go in yourselves, and you don't allow those who are trying to enter to go in.}

\v{14}\red{``How terrible it will be for you, scribes and Pharisees, you hypocrites! You devour widows' houses and say long prayers to cover it up. Therefore, you will receive greater condemnation!}\fnote{\fbackref{23:14} Other mss. omit vs. 14.}

\v{15}\red{``How terrible it will be for you, scribes and Pharisees, you hypocrites! You travel over land and sea to make a single convert, and when this happens you make him twice as fit for\fnote{\fbackref{23:15} Lit. \fbib{twice as much a son of}} hell\fnote{\fbackref{23:15} Lit. \fbib{Gehenna}; a Gk. transliteration of the Heb. for \fbib{Valley of Hinnom}} as you are.}

\v{16}\red{``How terrible it will be for you, blind guides! You say, `Whoever swears an oath by the sanctuary is excused,\fnote{\fbackref{23:16} Lit. \fbib{is nothing}} but whoever swears an oath by the gold of the sanctuary must keep his oath.'}\fnote{\fbackref{23:16} Lit. \fbib{owes a debt}} \v{17}\red{You blind fools! What is more important, the gold or the sanctuary that made the gold holy?} \v{18}\red{Again you say,\fnote{\fbackref{23:18} Lit. \fbib{And}} `Whoever swears an oath by the altar is excused,\fnote{\fbackref{23:18} Lit. \fbib{is nothing}} but whoever swears by the gift that is on it must keep his oath.'}\fnote{\fbackref{23:18} Lit. \fbib{owes a debt}} \v{19}\red{You blind men!\fnote{\fbackref{23:19} Other mss. read \fbib{blind and foolish men}} Which is more important, the gift or the altar that makes the gift holy?} \v{20}\red{Therefore, the one who swears an oath by the altar swears by it and by everything on it.} \v{21}\red{The one who swears an oath by the sanctuary swears by it and by the one who lives there.} \v{22}\red{And the one who swears an oath by heaven swears by God's throne and by the one who sits on it.}

\v{23}\red{``How terrible it will be for you, scribes and Pharisees, you hypocrites! You give a tenth of your mint, dill, and cummin, but have neglected the more important matters of the Law: justice, mercy, and faithfulness.\fnote{\fbackref{23:23} Or \fbib{faith}} These are the things you should have practiced, without neglecting the others.} \v{24}\red{You blind guides! You filter out a gnat, yet swallow a camel!}

\v{25}\red{``How terrible it will be for you, scribes and Pharisees, you hypocrites! You clean the outside of the cup and the plate, but on the inside they are full of greed and self-indulgence.} \v{26}\red{You blind Pharisee! First clean the inside of the cup,\fnote{\fbackref{23:26} Other mss. read \fbib{of the cup and the plate}} so that its outside may also be clean.}

\v{27}\red{``How terrible it will be for you, scribes and Pharisees, you hypocrites! You are like whitewashed tombs that look beautiful on the outside but inside are full of dead people's bones and every kind of impurity.} \v{28}\red{In the same way, on the outside you look righteous to people, but inside you are full of hypocrisy and lawlessness.}

\v{29}\red{``How terrible it will be for you, scribes and Pharisees, you hypocrites! You build tombs for the prophets and decorate the monuments of the righteous.} \v{30}\red{Then you say, `If we had been living in the days of our ancestors, we would have had no part with them in shedding\fnote{\fbackref{23:30} The Gk. lacks \fbib{shedding}} the blood of the prophets.'} \v{31}\red{Therefore, you testify against yourselves that you are descendants of those who murdered the prophets.} \v{32}\red{Then finish what your ancestors started!}\fnote{\fbackref{23:32} Lit. \fbib{Fill up the measure of your ancestors}} \v{33}\red{You snakes, you children of serpents! How can you escape being condemned to hell?}\fnote{\fbackref{23:33} Lit. \fbib{Gehenna}; a Gk. transliteration of the Heb. for \fbib{Valley of Hinnom}}

\v{34}\red{``That is why I am sending you prophets, wise men, and scribes. Some of them you will kill and crucify, and some of them you will whip in your synagogues and persecute from town to town.} \v{35}\red{As a result, you will be held accountable for\fnote{\fbackref{23:35} Lit. \fbib{on you will come}} all the righteous blood shed on earth, from the blood of the righteous Abel\fnote{\fbackref{23:35} Cf. Gen 4:8} to the blood of Berechiah's son Zechariah,\fnote{\fbackref{23:35} Cf. 2Chr 24:20-21} whom you murdered between the sanctuary and the altar.} \v{36}\red{I tell all of you\fnote{\fbackref{23:36} The Gk. pronoun \fbib{you} is pl.} with certainty, all these things will happen to those living today.''}\fnote{\fbackref{23:36} Lit. \fbib{to this generation}}
\passage{Jesus Rebukes Jerusalem}
\passageinfo{(Luke 13:34-35)}

\v{37}\red{``O Jerusalem, Jerusalem, who kills the prophets and stones to death those who have been sent to her! How often I wanted to gather your children together as a hen gathers her chicks under her wings, but you were unwilling!} \v{38}\red{Look! Your house is left abandoned!} \v{39}\red{I tell you, you will not see me again until you say, `How blessed is the one who comes in the name of the Lord!'\,''}\fnote{\fbackref{23:39} Cf. Ps 118:26; MT source citation reads \fbib{Lord}}
\labelchapt{24}
\passage{Jesus Predicts the Destruction of the Temple}
\passageinfo{(Mark 13:1-2; Luke 21:5-6)}

\chapt{24}
\v{1}As Jesus left the Temple and was walking away, his disciples came up to him to point out to him the Temple buildings. \v{2}But he told them, \red{``You see all these things, don't you? I tell all of you\fnote{\fbackref{24:2} The Gk. pronoun \fbib{you} is pl.} with certainty, there isn't a single stone here that will be left standing on top of another. They will all be torn down.''}
\passage{Cults, Revolutions, Famines, and Earthquakes}
\passageinfo{(Mark 13:3-13; Luke 21:7-19)}

\v{3}While Jesus\fnote{\fbackref{24:3} Lit. \fbib{he}} was sitting on the Mount of Olives, the disciples came to him privately and said, ``Tell us, when will these things take place, and what will be the sign of your coming and of the end of the age?''

\v{4}Jesus answered them, \red{``See to it that no one deceives you,} \v{5}\red{because many will come in my name and say, `I'm the Messiah,'\fnote{\fbackref{24:5} Or \fbib{Christ}} and they will deceive many people.} \v{6}\red{You'll hear of wars and rumors of wars. See to it that you aren't alarmed. These things must take place, but the end hasn't come yet,} \v{7}\red{because nation will rise up in arms against nation, and kingdom against kingdom. There will be famines and earthquakes in various places.} \v{8}\red{But all these things are only the beginning of the birth pains.''}
\passage{Future Persecution, Apostasy, and Evangelism}
\passageinfo{(Mark 13:9-13; Luke 21:12-19)}

\v{9}\red{``Then they'll hand you over to suffer\fnote{\fbackref{24:9} Or \fbib{tribulation}} and will kill you, and you'll be hated by all the nations\fnote{\fbackref{24:9} Or \fbib{gentiles}} because of my name.} \v{10}\red{Then many people will fall away, will betray one another, and will hate one another.} \v{11}\red{Many false prophets will appear and deceive many people,} \v{12}\red{and because lawlessness will increase, the love of many people will grow cold.} \v{13}\red{But the person who endures to the end will be saved.} \v{14}\red{And this gospel of the kingdom will be proclaimed throughout the world as a testimony to all nations,\fnote{\fbackref{24:14} Or \fbib{gentiles}} and then the end will come.''}
\passage{Signs of the End}
\passageinfo{(Mark 13:14-23; Luke 21:20-24)}

\v{15}\red{``So when you see the destructive desecration, mentioned by the prophet Daniel, standing in the Holy Place} (let the reader take note),\fnote{\fbackref{24:15} Cf. Dan 9:27; 11:31; 12:11; this parenthetical statement may have also been uttered by Jesus} \v{16}\red{then those who are in Judea must flee to the mountains.} \v{17}\red{Anyone who's on the housetop must not come down to get what is in his house,} \v{18}\red{and anyone who's in the field must not turn back to get his coat.}

\v{19}\red{``How terrible it will be for women who are pregnant or who are nursing babies in those days!} \v{20}\red{Pray that it may not be in winter or on a Sabbath when you flee,} \v{21}\red{because at that time there will be great suffering,\fnote{\fbackref{24:21} Or \fbib{tribulation}} the kind that hasn't happened from the beginning of the world until now and certainly won't ever happen again.} \v{22}\red{If those days hadn't been limited, no one\fnote{\fbackref{24:22} Lit. \fbib{flesh}} would survive. But for the sake of the elect, those days will be limited.}

\v{23}\red{``At that time, if anyone says to you, `Look! Here's the Messiah!'\fnote{\fbackref{24:23} Or \fbib{Christ}} or `There he is!', don't believe it,} \v{24}\red{because false messiahs\fnote{\fbackref{24:24} Or \fbib{christs}} and false prophets will appear and display great signs and wonders to deceive, if possible, even the elect.} \v{25}\red{Remember, I've told you this beforehand.} \v{26}\red{So if they say to you, `Look! He's in the wilderness,' don't go out looking for him.\fnote{\fbackref{24:26} The Gk. lacks \fbib{looking for him}} And if they say, `Look! He's in the storeroom,' don't believe it,} \v{27}\red{because just as the lightning comes from the east and flashes as far as the west, so will be the coming of the Son of Man.} \v{28}\red{Wherever there's a corpse, there the vultures\fnote{\fbackref{24:28} Or \fbib{eagles}} will gather.}
\passage{The Coming of the Son of Man}
\passageinfo{(Mark 13:24-27; Luke 21:25-28)}

\v{29}\red{``Immediately after the troubles\fnote{\fbackref{24:29} Or \fbib{tribulation}} of those days,}

\begin{poetry}
\poeml \red{`The sun will be darkened,} \\
\poemll    \red{the moon will not give its light,} \\
\poeml \red{the stars will fall from the sky,} \\
\poemll    \red{and the powers of heaven will be shaken loose.'}\fnote{\fbackref{24:29} Cf. Isa 13:10; 34:4; Joel 2:10}
\end{poetry}

\v{30}\red{Then the sign of the Son of Man will appear in the sky, and all `the tribes of the land\fnote{\fbackref{24:30} Or \fbib{earth}} will mourn'\fnote{\fbackref{24:30} Cf. Zech 12:12} when they see `the Son of Man coming on the clouds of heaven'\fnote{\fbackref{24:30} Cf. Dan 7:13} with power and great glory.} \v{31}\red{He'll send out his angels with a loud trumpet blast, and they'll gather his elect from the four winds, from one end of heaven to another.''}
\passage{The Lesson from the Fig Tree}
\passageinfo{(Mark 13:28-31; Luke 21:29-33)}

\v{32}\red{``Now learn a lesson\fnote{\fbackref{24:32} Or \fbib{parable}} from the fig tree. When its branches become tender and it produces leaves, you know that summer is near.} \v{33}\red{In the same way, when you see all these things, you'll know that the Son of Man\fnote{\fbackref{24:33} Lit. \fbib{that he}} is near, right at the door.} \v{34}\red{I tell all of you\fnote{\fbackref{24:34} The Gk. pronoun \fbib{you} is pl.} with certainty, this generation won't disappear until these things happen.} \v{35}\red{Heaven and earth will disappear, but my words will never disappear.''}
\passage{The Unknown Day and Hour of Messiah's Return}
\passageinfo{(Mark 13:32-37; Luke 17:20-36)}

\v{36}\red{``No one knows when that day or hour will come\fnote{\fbackref{24:36} Lit. \fbib{about that day and hour}}---not the angels in heaven, nor the Son,\fnote{\fbackref{24:36} Other mss. lack \fbib{nor the Son}} but only the Father,} \v{37}\red{because just as it was in the days of Noah, so it will be when the Son of Man comes.} \v{38}\red{In those days before the flood, people\fnote{\fbackref{24:38} Lit. \fbib{they}} were eating and drinking, marrying and giving in marriage right up to the day when Noah went into the ark.} \v{39}\red{They were unaware of what was happening\fnote{\fbackref{24:39} The Gk. lacks \fbib{of what was happening}} until the flood came and swept all of them away. That's how it will be when the Son of Man comes.} \v{40}\red{At that time, two people will be in the field. One will be taken, and one will be left behind.} \v{41}\red{Two women will be grinding grain\fnote{\fbackref{24:41} The Gk. lacks \fbib{grain}} at the mill. One will be taken, and one will be left behind.}

\v{42}\red{``So keep on watching, because you don't know on what day your Lord is coming.} \v{43}\red{But be sure of this: if the owner of the house had known when during\fnote{\fbackref{24:43} Lit. \fbib{known at what watch of}} the night the thief would be coming, he would have stayed awake and not allowed his house to be broken into.} \v{44}\red{So you, too, must be ready, because the Son of Man will come at an hour you are not expecting.''}
\passage{The Faithful or the Wicked Servant}
\passageinfo{(Luke 12:42-48)}

\v{45}\red{``Who, then, is the faithful and wise servant whom his master has put in charge of his household to give the others\fnote{\fbackref{24:45} Lit. \fbib{them}} their food at the right time?} \v{46}\red{How blessed is that servant whom his master finds doing this when he comes!} \v{47}\red{I tell all of you\fnote{\fbackref{24:47} The Gk. pronoun \fbib{you} is pl.} with certainty, he will put him in charge of all his property.}

\v{48}\red{``But if that wicked servant says to himself,\fnote{\fbackref{24:48} Lit. \fbib{in his heart}} `My master has been delayed,'} \v{49}\red{and begins to beat his fellow servants and eat and drink with the drunks,} \v{50}\red{that servant's master will come on a day when he doesn't expect him and at an hour that he doesn't know.} \v{51}\red{Then his master\fnote{\fbackref{24:51} Lit. \fbib{he}} will punish him severely\fnote{\fbackref{24:51} Lit. \fbib{will cut him in pieces}} and assign him a place with the hypocrites. In that place there will be weeping and gnashing of teeth.''}\fnote{\fbackref{24:51} I.e. extreme pain and anger}
\labelchapt{25}
\passage{The Parable about the Ten Bridesmaids}

\chapt{25}
\v{1}\red{``At that time, the kingdom from\fnote{\fbackref{25:1} Lit. \fbib{of}} heaven will be comparable to ten bridesmaids\fnote{\fbackref{25:1} Lit. \fbib{virgins}} who took their oil lamps and went out to meet the groom.}\fnote{\fbackref{25:1} Other mss. read \fbib{the groom and the bride}} \v{2}\red{Now five of them were foolish, and five were wise,} \v{3}\red{because when the foolish ones took their lamps, they didn't take any oil with them.} \v{4}\red{But the wise ones took flasks of oil with their lamps.} \v{5}\red{Since the groom was late, all of them became sleepy and lay down.} \v{6}\red{But at midnight there came a shout: `The groom is here! Come out to meet him!'} \v{7}\red{Then all the bridesmaids\fnote{\fbackref{25:7} Lit. \fbib{virgins}} woke up and got their lamps ready.}

\v{8}\red{``But the foolish ones told the wise, `Give us some of your oil, because our lamps are going out!'}

\v{9}\red{``But the wise ones replied, `No! There will never be enough for us and for you. You'd better go to the dealers and buy some for yourselves.'}

\v{10}\red{``While they were away buying it, the groom arrived. Those who were ready went with him into the wedding banquet, and the door was closed.} \v{11}\red{Later, the other bridesmaids\fnote{\fbackref{25:11} Lit. \fbib{virgins}} arrived and said, `Lord, lord, open up for us!'}

\v{12}\red{``But he replied, `I tell all of you}\fnote{\fbackref{25:12} The Gk. pronoun \fbib{you} is pl. with certainty, I don't know you!'} \v{13}\red{So keep on watching, because you don't know the day or the hour.''}\fnote{\fbackref{25:13} Other mss. read \fbib{the hour when the Son of Man will come}}
\passage{The Parable about the Talents}

\v{14}\red{``Similarly, it is like a man going on a trip, who called his servants and turned his money over to them.} \v{15}\red{To one man he gave five talents,\fnote{\fbackref{25:15} A talent was worth a lifetime of wages for an average laborer.} to another two, and to another one, based on their ability. Then he went on his trip.}

\v{16}\red{``The one who received five talents\fnote{\fbackref{25:16} A talent was worth a lifetime of wages for an average laborer.} went out at once and invested them and earned five more.} \v{17}\red{In the same way, the one who had two talents\fnote{\fbackref{25:17} A talent was worth a lifetime of wages for an average laborer.} earned two more.} \v{18}\red{But the one who received one talent\fnote{\fbackref{25:18} A talent was worth a lifetime of wages for an average laborer.} went off, dug a hole in the ground, and buried his master's money.}

\v{19}\red{``After a long time, the master of those servants returned and settled accounts with them.} \v{20}\red{The one who had received five talents came up and brought five more talents. `Master,' he said, `you gave me five talents. See, I've earned five more talents.'}

\v{21}\red{``His master told him, `Well done, good and trustworthy servant! Since you've been trustworthy with a small amount, I'll put you in charge of a large amount. Come and share your master's joy!'}

\v{22}\red{``The one with two talents also came forward and said, `Master, you gave me two talents. See, I've earned two more talents.}

\v{23}\red{``His master told him, `Well done, good and trustworthy servant! Since you've been trustworthy with a small amount, I'll put you in charge of a large amount. Come and share your master's joy!'}

\v{24}\red{``Then the one who had received one talent came forward and said, `Master, I knew that you were a hard man, harvesting where you haven't planted and gathering where you haven't scattered any seed.} \v{25}\red{Since I was afraid, I went off and hid your talent in the ground. Here, take what's yours!'}

\v{26}\red{``His master answered him, `You evil and lazy servant! So you knew that I harvested where I haven't planted and gathered where I haven't scattered any seed?} \v{27}\red{Then you should've invested my money with the bankers. When I returned, I would've received my money back with interest.'} \v{28}\red{Then the master said,\fnote{\fbackref{25:28} The Gk. lacks `\fbib{Then the master said'}} `Take the talent from him and give it to the man who has the ten talents,} \v{29}\red{because to everyone who has something, more will be given, and he'll have more than enough. But from the person who has nothing, even what he has will be taken away from him.} \v{30}\red{Throw this useless servant into the darkness outside! In that place there will be weeping and gnashing of teeth.'\,''}\fnote{\fbackref{25:30} I.e. extreme pain}
\passage{The Judgment of the Nations}

\v{31}\red{``When the Son of Man comes in his glory and all the angels are with him, he will sit on his glorious throne.} \v{32}\red{All the nations will be assembled in front of him, and he will cull them out, one from another, like a shepherd separates sheep from goats.} \v{33}\red{He will put the sheep on his right but the goats on his left.}

\v{34}\red{``Then the king will say to those on his right, `Come, you who have been blessed by my Father! Inherit the kingdom prepared for you from the foundation of the world,} \v{35}\red{because I was hungry, and you gave me something to eat. I was thirsty, and you gave me something to drink. I was a stranger, and you welcomed me.} \v{36}\red{I was naked, and you clothed me. I was sick, and you took care of me. I was in prison, and you visited me.'}

\v{37}\red{``Then the righteous will say to him, `Lord, when did we see you hungry and give you something to eat, or thirsty and give you something to drink?} \v{38}\red{When did we see you as a stranger and welcome you, or see you naked and clothe you?} \v{39}\red{When did we see you sick or in prison, and visit you?'}

\v{40}\red{The king will answer them, `I tell all of you\fnote{\fbackref{25:40} The Gk. pronoun \fbib{you} is pl.} with certainty, since you did it for one of the least important of these brothers of mine, you did it for me.'}

\v{41}\red{``Then he will say to those on his left, `Get away from me, you who are accursed, into the eternal fire that has been prepared for the devil and his angels!} \v{42}\red{Here's why: I was hungry, and you gave me nothing to eat. I was thirsty, and you gave me nothing to drink.} \v{43}\red{I was a stranger, and you didn't welcome me. I was naked, and you didn't clothe me. I was sick and in prison, and you didn't visit me.'}

\v{44}\red{``Then they will reply, `Lord, when did we see you hungry or thirsty or as a stranger or naked or sick or in prison and didn't help you?'}

\v{45}\red{Then he will say to them, `I tell all of you\fnote{\fbackref{25:45} The Gk. pronoun \fbib{you} is pl.} with certainty, since you didn't do it for one of the least important of these, you didn't do it for me.'} \v{46}\red{These people will go away into eternal punishment, but the righteous will go\fnote{\fbackref{25:46} The Gk. lacks \fbib{will go}} into eternal life.''}
\labelchapt{26}
\passage{The Plot to Kill Jesus}
\passageinfo{(Mark 14:1-2; Luke 22:1-2; John 11:45-53)}

\chapt{26}
\v{1}When Jesus had finished saying all these things,\fnote{\fbackref{26:1} Lit. \fbib{finished all these sayings}} he told his disciples, \v{2}\red{``You know that the Passover will take place in two days, and the Son of Man will be handed over to be crucified.''}

\v{3}Then the high priests and the elders of the people assembled in the courtyard of the high priest, who was named Caiaphas. \v{4}They conspired to arrest Jesus by treachery and to kill him. \v{5}But they kept saying, ``This must not happen during the festival. Otherwise, there'll be a riot among the people.''
\passage{A Woman Anoints Jesus}
\passageinfo{(Mark 14:3-9; John 12:1-8)}

\v{6}While Jesus was in Bethany at the home of Simon the leper, \v{7}a woman came to him with an alabaster jar of very expensive perfume and poured it on his head while he sat at the table. \v{8}But when the disciples saw this, they became irritated and said, ``Why this waste? \v{9}Surely this perfume could've been sold for a high price and the money\fnote{\fbackref{26:9} The Gk. lacks \fbib{the money}} given to the destitute.''

\v{10}But knowing this,\fnote{\fbackref{26:10} The Gk. lacks \fbib{this}} Jesus asked them, \red{``Why are you bothering the woman? She has done a beautiful thing for me.} \v{11}\red{You'll always have the destitute with you, but you'll not always have me.} \v{12}\red{When she poured this perfume on my body, she was preparing me for burial.} \v{13}\red{I tell all of you\fnote{\fbackref{26:13} The Gk. pronoun \fbib{you} is pl.} with certainty, wherever this gospel is proclaimed throughout the whole world, what she has done will also be told as a memorial to her.''}
\passage{Judas Agrees to Betray Jesus}
\passageinfo{(Mark 14:10-11; Luke 22:3-6)}

\v{14}Then one of the Twelve, who was called Judas Iscariot, went to the high priests \v{15}and inquired, ``What are you willing to give me if I betray Jesus\fnote{\fbackref{26:15} Lit. \fbib{him}} to you?'' They offered him 30 pieces of silver, \v{16}and from then on he began to look for an opportunity to betray Jesus.\fnote{\fbackref{26:16} Lit. \fbib{him}}
\passage{The Passover with the Disciples}
\passageinfo{(Mark 14:12-21; Luke 22:7-14, 21-23; John 12:21-30)}

\v{17}On the first day of the Festival\fnote{\fbackref{26:17} The Gk. lacks \fbib{day of the Festival}} of Unleavened Bread, the disciples approached Jesus and asked, ``Where do you want us to make preparations for you to eat the Passover meal?''

\v{18}He said, \red{``Go to a certain man in the city and say to him, `The Teacher says, ``My time is near. I will celebrate the Passover with my disciples at your house.''\,'\,''}\fnote{\fbackref{26:18} Lit. \fbib{with you}} \v{19}So the disciples did as Jesus had directed them, and they prepared the Passover meal.

\v{20}When evening came, Jesus\fnote{\fbackref{26:20} Lit. \fbib{he}} was sitting at the table with the Twelve.\fnote{\fbackref{26:20} Other mss. read \fbib{twelve disciples}} \v{21}While they were eating, he said, \red{``I tell all of you\fnote{\fbackref{26:21} The Gk. pronoun \fbib{you} is pl.} with certainty, one of you is going to betray me.''}

\v{22}Feeling deeply distressed, each one began to ask him, ``Surely I am not the one, Lord?''

\v{23}He replied, \red{``The man who has dipped his hand into the bowl with me will betray me.} \v{24}\red{The Son of Man is going away, just as it has been written about him. How terrible it will be for that man by whom the Son of Man is betrayed! It would have been better for him if he had never been born.''}

\v{25}Then Judas, who was going to betray him, asked, ``Rabbi,\fnote{\fbackref{26:25} \fbib{Rabbi} is Heb. for \fbib{Master} and/or \fbib{Teacher}.} I'm not the one, am I?''

Jesus\fnote{\fbackref{26:25} Lit. \fbib{He}} told him, \red{``You have said so.''}
\passage{The Lord's Supper}
\passageinfo{(Mark 14:22-26; Luke 22:15-20)}

\v{26}While they were eating, Jesus took a loaf of bread and blessed it. Then he broke it in pieces and handed it to the disciples, saying, \red{``Take this and eat it. This is my body.''}

\v{27}Then he took a cup, gave thanks, and gave it to them, saying, \red{``Drink from it, all of you,} \v{28}\red{because this is my blood of the new\fnote{\fbackref{26:28} Other mss. lack \fbib{new}} covenant that is being poured out for many people for the forgiveness of sins.} \v{29}\red{I tell all of you\fnote{\fbackref{26:29} The Gk. pronoun \fbib{you} is pl.} I will never again drink the product of the vine until that day when I drink it with you once again in my Father's kingdom.''}

\v{30}After singing a hymn, they went out to the Mount of Olives.
\passage{Jesus Predicts Peter's Denial}
\passageinfo{(Mark 14:27-31; Luke 22:31-34; John 13:36-38)}

\v{31}Then Jesus told them, \red{``All of you will turn against me this very night, because it is written,}

\begin{poetry}
\poeml \red{`I will strike the shepherd,} \\
\poemll    \red{and the sheep of the flock will be scattered.'}\fnote{\fbackref{26:31} Cf. Zech 13:7}
\end{poetry}

\v{32}\red{However, after I have been raised, I will go to Galilee ahead of you.''}

\v{33}But Peter told him, ``Even if everyone else turns against you, I certainly won't!''

\v{34}Jesus told him, \red{``I tell you\fnote{\fbackref{26:34} The Gk. pronoun \fbib{you} is sing.} with certainty, before a rooster crows this very night, you will deny me three times.''}

\v{35}Peter told him, ``Even if I have to die with you, I will never deny you!'' And all the disciples said the same thing.
\passage{Jesus Prays in the Garden of Gethsemane}
\passageinfo{(Mark 14:32-42; Luke 22:39-46)}

\v{36}Then Jesus went with them to a place called Gethsemane. He told the disciples, \red{``Sit down here while I go over there and pray.''} \v{37}Taking Peter and the two sons of Zebedee with him, he began to be grieved and troubled. \v{38}Then he told them, \red{``I'm so deeply grieved that I feel I'm about to die. Wait here and stay awake with me.''}

\v{39}Going on a little farther, he fell on his face and prayed, \red{``O my Father, if it is possible, let this cup pass from me. Yet not what I want but what you want.''}

\v{40}When he went back to the disciples, he found them asleep. He told Peter, \red{``So, you men couldn't stay awake with me for one hour, could you?} \v{41}\red{All of you must stay awake and pray that you won't be tempted. The spirit is indeed willing, but the body\fnote{\fbackref{26:41} Lit. \fbib{flesh}} is weak.''}

\v{42}He went away a second time and prayed, \red{``My Father, if this cup cannot go away unless I drink it, let your will be done.''} \v{43}Then he came back and found them asleep, because they could not keep their eyes open. \v{44}After leaving them again, he went away and prayed again for the third time, saying the same thing.

\v{45}Then he came back to the disciples and told them, \red{``You might as well keep on sleeping and resting.\fnote{\fbackref{26:45} Or \fbib{Are you still sleeping and resting?}} Look! The time is near for the Son of Man to be betrayed into the hands of sinners.} \v{46}\red{Get up! Let's go! See, the one who is betraying me is near!''}
\passage{Jesus is Arrested}
\passageinfo{(Mark 14:43-50; Luke 22:47-53; John 18:3-12)}

\v{47}Just then, while Jesus\fnote{\fbackref{26:47} Lit. \fbib{he}} was still speaking, Judas, one of the Twelve, arrived. A large crowd armed with swords and clubs was with him. They were from the high priests and elders of the people. \v{48}Now the betrayer personally had given them a signal, saying, ``The one I kiss\fnote{\fbackref{26:48} People customarily greeted their friends with a kiss.} is the man. Arrest him.''

\v{49}So Judas\fnote{\fbackref{26:49} Lit. \fbib{he}} immediately went up to Jesus and said, ``Hello, Rabbi!''\fnote{\fbackref{26:49} \fbib{Rabbi} is Heb. for \fbib{Master} and/or \fbib{Teacher}.} and kissed him tenderly.

\v{50}Jesus asked him, \red{``Friend, why are you here?''}\fnote{\fbackref{26:50} Or \fbib{do what you came for}} Then the other men\fnote{\fbackref{26:50} Lit. \fbib{they}} surged forward, took hold of Jesus, and arrested him.

\v{51}Suddenly, one of the men with Jesus reached out his hand, drew his sword, and struck the high priest's servant, cutting off his ear. \v{52}Jesus told him, \red{``Put your sword back in its place! Everyone who uses a sword will be killed by a sword.} \v{53}\red{Don't you think that I could call on my Father, and he would send me more than twelve legions of angels now?} \v{54}\red{How, then, would the Scriptures be fulfilled that say this must happen?''}

\v{55}At this point,\fnote{\fbackref{26:55} Lit. \fbib{In that hour}} Jesus asked the crowds, \red{``Have you come out with swords and clubs to arrest me as if I were a bandit?\fnote{\fbackref{26:55} Or \fbib{revolutionary}} Day after day I sat teaching in the Temple, yet you didn't arrest me.} \v{56}\red{But all of this has happened so that the writings of the prophets might be fulfilled.''}

Then all the disciples deserted Jesus\fnote{\fbackref{26:56} Lit. \fbib{him}} and ran away.
\passage{Jesus is Tried before the High Priest}
\passageinfo{(Mark 14:53-65; Luke 22:54-55, 63-71; John 18:13-14, 19-24)}

\v{57}Those who had arrested Jesus took him to Caiaphas, the high priest, where the scribes and the elders had assembled. \v{58}Peter, however, followed him at a distance as far as the high priest's courtyard. He went inside and sat down with the servants to see how this would end.

\v{59}Meanwhile, the high priests and the whole Council\fnote{\fbackref{26:59} Or \fbib{Sanhedrin}} were looking for false testimony against Jesus in order to have him put to death. \v{60}But they couldn't find any, even though many false witnesses had come forward. At last two men came forward \v{61}and stated, ``This man said, `I can destroy the sanctuary of God and rebuild it in three days.'\,''

\v{62}At this, the high priest stood up and asked Jesus,\fnote{\fbackref{26:62} Lit. \fbib{him}} ``Don't you have any answer to what these men are testifying against you?'' \v{63}But Jesus was silent. Then the high priest told him, ``I command you by the living God to tell us if you are the Messiah,\fnote{\fbackref{26:63} Or \fbib{Christ}} the Son of God!''

\v{64}Jesus told him, \red{``You have said so. Nevertheless I tell you, from now on you will see `the Son of Man seated at the right hand of Power'\fnote{\fbackref{26:64} Cf. Ps 110:1} and `coming on the clouds of heaven.'\,''}\fnote{\fbackref{26:64} Cf. Dan 7:13}

\v{65}Then the high priest tore his robes and said, ``He has blasphemed! Why do we still need witnesses? Listen! You yourselves have just heard the blasphemy! \v{66}What is your verdict?''

They replied, ``He deserves to die!''

\v{67}Then they spit in his face and hit him. Some slapped him, \v{68}saying, ``Prophesy to us, you Messiah!\fnote{\fbackref{26:68} Or \fbib{Christ}} Who hit\fnote{\fbackref{26:68} Lit. \fbib{Who is the one who hit}} you?''
\passage{Peter Denies Jesus}
\passageinfo{(Mark 14:66-72; Luke 22:56-62; John 18:15-18, 25-27)}

\v{69}Now Peter was sitting outside in the courtyard when a servant girl came up to him and said, ``You, too, were with Jesus the Galilean.''

\v{70}But he denied it in front of them all. ``I don't know what you're talking about!'' he exclaimed.

\v{71}As he went out to the gateway, another woman saw him and told those who were there, ``This man was with Jesus from Nazareth.''\fnote{\fbackref{26:71} Or \fbib{Jesus the Nazarene}; the Gk. \fbib{Nazoraios} may be a word play between Heb. \fbib{netser,} meaning \fbib{branch} (see Isa 11:1), and the name \fbib{Nazareth.}}

\v{72}Again he denied it and swore with an oath, ``I don't know the man!''

\v{73}After a little while, the people who were standing there came up and told Peter, ``Obviously you're also one of them, because your accent gives you away.''

\v{74}Then he began to curse violently. ``I don't know the man!'' he swore solemnly. Just then a rooster crowed. \v{75}Peter remembered the words of Jesus when he said, \red{``Before a rooster crows, you'll deny me three times.''} Then he went outside and cried bitterly.
\labelchapt{27}
\passage{Jesus is Taken to Pilate}
\passageinfo{(Mark 15:1; Luke 23:1-2; John 18:28-32)}

\chapt{27}
\v{1}When morning came, all the high priests and elders of the people conspired against Jesus to put him to death. \v{2}They bound him with chains, led him away, and handed him over to Pontius\fnote{\fbackref{27:2} Other mss. lack \fbib{Pontius}} Pilate, the governor.
\passage{The Death of Judas}
\passageinfo{(Acts 1:18-19)}

\v{3}Then Judas, who had betrayed him, regretted what had happened when he saw that Jesus\fnote{\fbackref{27:3} Lit. \fbib{he}} was condemned. He brought the 30 pieces of silver back to the high priests and elders, \v{4}saying, ``I have sinned by betraying innocent\fnote{\fbackref{27:4} Other mss. read \fbib{righteous}} blood.''

But they replied, ``What do we care? Attend to that yourself.'' \v{5}Then he flung the pieces of silver into the sanctuary, went outside, ran away, and hanged himself.

\v{6}The high priests picked up the pieces of silver and said, ``It is not lawful to put this into the Temple treasury, because it is blood money.'' \v{7}So they decided to use the money to buy the Potter's Field as a burial ground for foreigners. \v{8}That is why that field has been called the Field of Blood to this day. \v{9}Then what had been declared through the prophet Jeremiah was fulfilled when he said,

\begin{poetry}
\poeml ``They\fnote{\fbackref{27:9} Or \fbib{I}} took the 30 pieces of silver, \\
\poemll    the value of the man on whom a price had been set by the Israelis, \\
\poeml \v{10}and they\fnote{\fbackref{27:10} Other mss. read \fbib{I}} gave them for the potter's field, \\
\poemll    as the Lord commanded me.''\fnote{\fbackref{27:10} Cf. Zech 11:12-13; Jer 32:6-9}
\end{poetry}
\passage{Pilate Questions Jesus}
\passageinfo{(Mark 15:2-5; Luke 23:3-5; John 18:33-38)}

\v{11}Meanwhile, Jesus was made to stand in front of the governor. The governor asked him, ``Are you the king of the Jews?''

Jesus said, \red{``You say so.''}

\v{12}While Jesus\fnote{\fbackref{27:12} Lit. \fbib{he}} was being accused by the high priests and elders, he made no reply. \v{13}Then Pilate asked him, ``Don't you hear how many charges they're bringing against you?'' \v{14}But Jesus\fnote{\fbackref{27:14} Lit. \fbib{he}} did not reply at all, so that the governor was very surprised.
\passage{Jesus is Sentenced to Death}
\passageinfo{(Mark 15:6-15; Luke 23:13-25; John 18:39-19:16)}

\v{15}At every festival\fnote{\fbackref{27:15} I.e. Passover Festival} the governor had a custom of releasing to the crowd any prisoner whom they wanted. \v{16}At that time they were holding a notorious prisoner named Barabbas.\fnote{\fbackref{27:16} Other mss. read \fbib{Jesus Barabbas}} \v{17}So when the people\fnote{\fbackref{27:17} Lit. \fbib{they}} had gathered, Pilate asked them, ``Which man do you want me to release for you---Barabbas,\fnote{\fbackref{27:17} Other mss. read \fbib{Jesus Barabbas}} or Jesus who is called `the Messiah'?''\fnote{\fbackref{27:17} Or \fbib{Christ}} \v{18}He did this\fnote{\fbackref{27:18} The Gk. lacks \fbib{He did this}} because he knew that they had handed him over out of jealousy.

\v{19}While he was sitting on the judge's seat, his wife sent him a message\fnote{\fbackref{27:19} The Gk. lacks \fbib{a message}} that said, ``Have nothing to do with that righteous man, because today I have suffered terribly due to a dream I had about him.''

\v{20}But the high priests and elders persuaded the crowds to ask for Barabbas and to demand that Jesus be put to death. \v{21}So the governor asked them, ``Which of the two men do you want me to release for you?''

``Barabbas!'' they replied.

\v{22}Pilate asked them, ``Then what should I do with Jesus, who is called the Messiah?''\fnote{\fbackref{27:22} Or \fbib{Christ}}

They all said, ``Let him be crucified!''

\v{23}He asked, ``What has he done wrong?''

But they kept shouting louder and louder, ``Let him be crucified!''

\v{24}Pilate saw that he was getting nowhere, but that a riot was about to break out instead. So he took some water and washed his hands in front of the crowd, saying, ``I am innocent of this man's\fnote{\fbackref{27:24} Other mss. read \fbib{this righteous man's}} blood. Attend to that yourselves.''

\v{25}All the people answered, ``Let his blood be on us and our children!'' \v{26}Then he released Barabbas for them, but he had Jesus whipped and handed over to be crucified.
\passage{The Soldiers Make Fun of Jesus}
\passageinfo{(Mark 15:16-20; John 19:2-3)}

\v{27}Then the governor's soldiers took Jesus into the imperial headquarters\fnote{\fbackref{27:27} Lit. \fbib{praetorium}} and gathered the whole company of soldiers around him. \v{28}They stripped\fnote{\fbackref{27:28} Other mss. read \fbib{clothed}} him and put a scarlet robe on him. \v{29}Twisting some thorns into a victor's crown, they placed it on his head and put\fnote{\fbackref{27:29} The Gk. lacks \fbib{put}} a stick in his right hand. They knelt down in front of him and began making fun of him, saying, ``Long live the king of the Jews!'' \v{30}Then they spit on him and took the stick and hit him repeatedly on his head. \v{31}When they had finished making fun of him, they stripped him of the robe, put his own clothes back on him, and led him away to crucify him.
\passage{Jesus is Crucified}
\passageinfo{(Mark 15:21-32; Luke 23:26-43; John 19:17-27)}

\v{32}As they were leaving, they found a man from Cyrene named Simon, whom they forced to carry Jesus'\fnote{\fbackref{27:32} Lit. \fbib{his}} cross. \v{33}When they came to a place called Golgotha (which means ``Skull Place''), \v{34}they offered him a drink of wine mixed with gall. But when he tasted it, he refused to drink it. \v{35}After they had crucified him, they determined who would get his clothes by throwing dice for them.\fnote{\fbackref{27:35} The Gk. lacks \fbib{for them}} \v{36}Then they sat down there and continued guarding him. \v{37}Above his head they placed the charge against him. It read, ``This is Jesus, the king of the Jews.''

\v{38}At that time two bandits\fnote{\fbackref{27:38} Or \fbib{revolutionaries}} were crucified with him, one on his right and the other on his left. \v{39}Those who passed by kept insulting\fnote{\fbackref{27:39} Or \fbib{blaspheming}} him, shaking their heads, \v{40}and saying, ``You who were going to destroy the sanctuary and rebuild it in three days---save yourself! If you're the Son of God, come down from the cross!''

\v{41}In the same way the high priests, along with the scribes and elders, were also making fun of him. They kept saying, \v{42}``He saved others but can't save himself! He is the king of Israel. Let him\fnote{\fbackref{27:42} Other mss. read \fbib{If he is the king of Israel, let him}} come down from the cross now, and we will believe in him. \v{43}He trusts in God. Let God\fnote{\fbackref{27:43} Lit. \fbib{him}} rescue him, if he wants to do so now. After all, he said, \red{`I am the Son of God.'}''

\v{44}In a similar way, the bandits\fnote{\fbackref{27:44} Or \fbib{revolutionaries}} who were being crucified with him kept insulting him.
\passage{Jesus Dies on the Cross}
\passageinfo{(Mark 15:33-41; Luke 23:44-49; John 19:28-30)}

\v{45}From noon\fnote{\fbackref{27:45} Lit. \fbib{the sixth hour}} on, darkness came over the whole land\fnote{\fbackref{27:45} Or \fbib{earth}} until three in the afternoon.\fnote{\fbackref{27:45} Lit. \fbib{the ninth hour}} \v{46}About three o'clock,\fnote{\fbackref{27:46} Lit. \fbib{the ninth hour}} Jesus cried out with a loud voice, \red{``Eli, eli,\fnote{\fbackref{27:46} \fbib{Eli, eli} are Gk. transliterations for the Heb. \fbib{My God, my God} in Ps 22:1} lema sabachthani?''},\fnote{\fbackref{27:46} \fbib{lema sabachthani} is a Gk. transliteration for the Aram. rendering of the Heb. \fbib{in Ps 22:1, which means Why have you forsaken me?}} which means, \red{``My God, my God, why have you forsaken me?''}\fnote{\fbackref{27:46} Cf. Ps 22:1}

\v{47}When some of the people standing there heard this, they said, ``He's calling for Elijah.''\fnote{\fbackref{27:47} Elijah in Heb. sounds like \fbib{Eli}} \v{48}So one of the men ran off at once, took a sponge, and soaked it in some sour wine. Then he put it on a stick and offered Jesus\fnote{\fbackref{27:48} Lit. \fbib{him}} a drink.

\v{49}But the others kept saying, ``Wait! Let's see if Elijah will come and save him.''\fnote{\fbackref{27:49} Other mss. read \fbib{save him.'' And another took a spear and pierced his side, and water and blood came out.}}

\v{50}Then Jesus cried out with a loud voice again and died.\fnote{\fbackref{27:50} Or \fbib{and gave up his spirit}} \v{51}Suddenly, the curtain\fnote{\fbackref{27:51} This curtain separated the Holy Place from the Most Holy Place.} in the sanctuary was torn in two from top to bottom, the earth shook, rocks were split open, \v{52}tombs were opened, and many\fnote{\fbackref{27:52} Lit. \fbib{and the corpses of many}} saints who had died\fnote{\fbackref{27:52} Lit. \fbib{fallen asleep}} were brought back to life. \v{53}After his resurrection, they came out of their tombs, went into the Holy City,\fnote{\fbackref{27:53} I.e. Jerusalem} and appeared to many people.

\v{54}When the centurion\fnote{\fbackref{27:54} A Roman centurion commanded about 100 men.} and those guarding Jesus with him saw the earthquake and the other things that were taking place, they were terrified and said, ``This man certainly was the Son of God!''

\v{55}Now many women were also there, watching from a distance. They had accompanied Jesus from Galilee and had ministered to\fnote{\fbackref{27:55} Or \fbib{provided for}} him. \v{56}Among them were Mary Magdalene,\fnote{\fbackref{27:56} Or \fbib{Mary of Magdala}} Mary the mother of James and Joseph, and the mother of Zebedee's sons.
\passage{Jesus is Buried}
\passageinfo{(Mark 15:42-47; Luke 23:50-56; John 19:38-42)}

\v{57}Later that evening, a rich man arrived from Arimathea. His name was Joseph, and he had become a disciple of Jesus. \v{58}He went to Pilate and asked for the body of Jesus, and Pilate ordered it to be done. \v{59}So Joseph took the body and wrapped it in a clean linen cloth. \v{60}Then he placed it in his own new tomb, which he had cut out of the rock. After rolling a large stone across the door of the tomb, he left, \v{61}but Mary Magdalene\fnote{\fbackref{27:61} Or \fbib{Mary of Magdala}} and the other Mary remained there, sitting in front of the tomb.
\passage{The Tomb is Secured}

\v{62}The following day (that is, after the Day of Preparation), the high priests and Pharisees gathered before Pilate \v{63}and said, ``Sir, we remember how that impostor said while he was still alive, \red{`I will be raised after three days.'} \v{64}Therefore, order the tomb to be secured until the third day, or his disciples may go and steal him and then tell the people, `He has been raised from the dead.' Then the last deception would be worse than the first one.''

\v{65}Pilate told them, ``You have\fnote{\fbackref{27:65} Or \fbib{Take}} a military guard. Go and make the tomb\fnote{\fbackref{27:65} Lit. \fbib{it}} as secure as you know how.'' \v{66}So they went and secured the tomb by putting a seal on the stone in the presence of the guards.
\labelchapt{28}
\passage{Jesus is Raised from the Dead}
\passageinfo{(Mark 16:1-8; Luke 24:1-12; John 20:1-10)}

\chapt{28}
\v{1}After the Sabbaths,\fnote{\fbackref{28:1} I.e., the Passover and Saturday} around dawn on the first day of the week,\fnote{\fbackref{28:1} Lit. \fbib{first of the Sabbaths}} Mary Magdalene\fnote{\fbackref{28:1} Or \fbib{Mary of Magdala}} and the other Mary went to take a look at the burial site.\fnote{\fbackref{28:1} Or \fbib{the tomb}} \v{2}Suddenly, there was a powerful earthquake, because an angel of the Lord had come down from heaven, approached the stone, rolled it away, and was sitting on top of it. \v{3}His appearance was bright as lightning, and his clothes were white as snow. \v{4}Trembling from fear, even the guards themselves became catatonic.\fnote{\fbackref{28:4} Lit. \fbib{became like the dead}}

\v{5}Responding to the women, the angel said, ``Stop being frightened! I know you're looking for Jesus, who was crucified. \v{6}He is not here. He has been raised, just as he said. Come and see the place where he\fnote{\fbackref{28:6} Other mss. read \fbib{the Lord}} was lying. \v{7}Then go quickly and tell his disciples that he is risen from the dead. He is going ahead of you into Galilee, and you will see him there. Remember, I have told you!''

\v{8}So they quickly left the tomb, terrified but also ecstatic, and ran to tell Jesus'\fnote{\fbackref{28:8} Lit. \fbib{his}} disciples. \v{9}Suddenly, Jesus met them and said, \red{``Greetings!''} They went up to him, took hold of his feet, and worshipped him. \v{10}Then Jesus told them, \red{``Stop being frightened! Go and tell my brothers to leave for Galilee, and they will see me there.''}
\passage{The Guards Report to the High Priests}

\v{11}While the women were on their way, some of the guards went into the city and told the high priests everything that had happened. \v{12}So they met with the elders and agreed on a plan to give the soldiers a large\fnote{\fbackref{28:12} Or \fbib{a sufficient}} amount of money. \v{13}They said, ``Say that Jesus'\fnote{\fbackref{28:13} Lit. \fbib{his}} disciples came at night and stole him while you were sleeping. \v{14}If this is reported to the governor, we'll personally persuade him to keep you out of trouble.''\fnote{\fbackref{28:14} Lit. \fbib{from worry}} \v{15}So the soldiers\fnote{\fbackref{28:15} Lit. \fbib{they}} took the money, did as they were instructed, and this story has been spread among the Jews to this day.
\passage{Jesus Commissions His Disciples}
\passageinfo{(Mark 16:14-18; Luke 24:36-49; John 20:19-23; Acts 1:6-8)}

\v{16}The eleven disciples went into Galilee to the hillside to which Jesus had directed them. \v{17}When they saw him, they worshipped him, though some had doubts. \v{18}Then Jesus approached them and told them, \red{``All authority in heaven and on earth has been given to me.} \v{19}\red{Therefore, as you go, disciple people in all nations, baptizing them in the name of the Father, and the Son, and the Holy Spirit,} \v{20}\red{teaching them to obey everything that I've commanded you. And remember, I am with you each and every day\fnote{\fbackref{28:20} Lit. \fbib{you all the days}} until the end of the age.''}\fnote{\fbackref{28:20} Other mss. read \fbib{age. Amen}}

\bookheader{Mark}
\labelbook{Mark}

\bookpretitle{The Gospel According to}
\booktitle{Mark}

\labelchapt{1}
\passage{John the Baptist Prepares the Way for Jesus}
\passageinfo{(Matthew 3:1-12; Luke 3:1-9, 15-17; John 1:19-28)}

\chapt{1}
\v{1}This is\fnote{\fbackref{1:1} The Gk. lacks \fbib{This is}} the beginning of the gospel of Jesus the Messiah,\fnote{\fbackref{1:1} Or \fbib{Christ}} the Son of God.\fnote{\fbackref{1:1} Other mss. lack \fbib{the Son of God}} \v{2}As it is written in the prophet Isaiah,

\begin{poetry}
\poeml ``See! I am sending my messenger ahead of you, \\
\poemll    who will prepare your way.\fnote{\fbackref{1:2} Cf. Mal 3:1} \\
\poeml \v{3}He is a voice calling out in the wilderness: \\
\poemll    `Prepare the way for the Lord!\fnote{\fbackref{1:3} MT source citation reads \fbib{}\divine{Lord}} \\
\poemlll       Make his paths straight!'\,''\fnote{\fbackref{1:3} Cf. Isa 40:3}
\end{poetry}

\v{4}John was baptizing in the wilderness, proclaiming a baptism about repentance for the forgiveness of sins. \v{5}People from\fnote{\fbackref{1:5} The Gk. lacks \fbib{People from}} the whole Judean countryside and all the people of Jerusalem were flocking to him, being baptized by him while they confessed their sins. \v{6}Now John was dressed in camel's hair with\fnote{\fbackref{1:6} Lit. \fbib{and}} a leather belt around his waist. He ate grasshoppers\fnote{\fbackref{1:6} Or \fbib{locust-shaped carob seed pods}} and wild honey. \v{7}He kept proclaiming, ``The one who is coming after me is stronger than I am, and I am not worthy to bend down and untie his sandal straps. \v{8}I baptized you with\fnote{\fbackref{1:8} Or \fbib{in}} water, but it is he who will baptize you with\fnote{\fbackref{1:8} Or \fbib{in}} the Holy Spirit.''
\passage{Jesus is Baptized}
\passageinfo{(Matthew 3:13-17; Luke 3:21-22)}

\v{9}In those days Jesus came from Nazareth in Galilee and was baptized by John in the Jordan. \v{10}Just as he was coming up out of the water, he saw the heavens split open and the Spirit descending on him like a dove. \v{11}Then a voice came from heaven: ``You are my Son, whom I love. I am pleased with you!''
\passage{Jesus is Tempted by Satan}
\passageinfo{(Matthew 4:1-11; Luke 4:1-13)}

\v{12}At once the Spirit drove him into the wilderness. \v{13}He was in the wilderness for 40 days being tempted by Satan. He was among wild animals, and angels were ministering to him.
\passage{Jesus Begins His Ministry in Galilee}
\passageinfo{(Matthew 4:12-17; Luke 4:14-15)}

\v{14}Now after John had been arrested, Jesus went to Galilee and proclaimed the gospel about the kingdom\fnote{\fbackref{1:14} Other mss. lack \fbib{about the kingdom}} of God. \v{15}He said, \red{``The time is now! The }\red{kingdom}\red{ of }\red{God}\red{ is near! Repent, and keep believing the gospel!''}
\passage{Jesus Calls His First Disciples}
\passageinfo{(Matthew 4:18-22; Luke 5:1-11)}

\v{16}While Jesus\fnote{\fbackref{1:16} Lit. \fbib{he}} was walking beside the Sea of Galilee, he saw Simon and his brother Andrew. They were throwing a net into the sea because they were fishermen. \v{17}Jesus told them, \red{``Follow me, and I}\red{'}\red{ll make you fishers of people!''} \v{18}So immediately they left their nets and followed him. \v{19}Going on a little farther, he saw James, the son of Zebedee, and his brother John. They were in a boat repairing their nets. \v{20}He immediately called them, and they left their father Zebedee in the boat with the hired men and followed him.
\passage{Jesus Heals a Man with an Unclean Spirit}
\passageinfo{(Luke 4:31-37)}

\v{21}Then they went to Capernaum. As soon as it was the Sabbaths, Jesus\fnote{\fbackref{1:21} Lit. \fbib{he}} went into the synagogue and began to teach. \v{22}The people\fnote{\fbackref{1:22} Lit. \fbib{They}} were utterly amazed at his teaching, because he was teaching them like one with authority, and not like their scribes. \v{23}All of a sudden,\fnote{\fbackref{1:23} Lit. \fbib{Immediately}} there was a man in their synagogue who had an unclean spirit! He screamed, \v{24}``What do you want with us, Jesus of Nazareth? Have you come to destroy us? I know who you are---the Holy One of God!''

\v{25}But Jesus rebuked him. \red{``Be quiet,}\red{''} he ordered, \red{``a}\red{nd come out of him!''} \v{26}At this, the unclean spirit shook the man,\fnote{\fbackref{1:26} Lit. \fbib{him}} cried out with a loud voice, and came out of him.

\v{27}All the people were so stunned that they kept saying to each other, ``What is this? A new teaching with authority! He tells even the unclean spirits what to do, and they obey him!'' \v{28}At once his fame began to spread throughout the surrounding region of Galilee.
\passage{Jesus Heals Many People}
\passageinfo{(Matthew 8:14-17; Luke 4:38-41)}

\v{29}After they left the synagogue, they went directly to the house of Simon and Andrew, along with James and John. \v{30}Now Simon's mother-in-law was lying in bed, sick with a fever, so they promptly told Jesus\fnote{\fbackref{1:30} Lit. \fbib{him}} about her. \v{31}He went up to her, took her by the hand, and helped her up. The fever left her, and she began serving them. \v{32}When evening came, after the sun had set, people\fnote{\fbackref{1:32} Lit. \fbib{they}} started bringing to him everyone who was sick or possessed by demons. \v{33}In fact, the whole city gathered at the door. \v{34}He healed many who were sick with various diseases and drove out many demons. However, he wouldn't allow the demons to speak because they knew who he was.
\passage{Jesus Goes on a Preaching Tour}
\passageinfo{(Luke 4:42-44)}

\v{35}In the morning, while it was still very dark, Jesus\fnote{\fbackref{1:35} Lit. \fbib{he}} got up and went to a deserted place and prayed there. \v{36}Simon and his companions searched frantically for him. \v{37}When they found him, they told him, ``Everyone's looking for you.''

\v{38}\red{``Let's go to the neighboring town,'' }he replied, \red{``so I can preach there, too, because that's why I came.''} \v{39}So he went throughout Galilee, preaching in their synagogues and driving out demons.
\passage{Jesus Cleanses a Leper}
\passageinfo{(Matthew 8:1-4; Luke 5:12-16)}

\v{40}Then a leper\fnote{\fbackref{1:40} I.e. a man with a serious skin disease} came to Jesus\fnote{\fbackref{1:40} Lit. \fbib{him}} and began pleading with him. He fell on his knees and told him, ``If you want to, you can make me clean.''

\v{41}Moved with compassion, Jesus\fnote{\fbackref{1:41} Lit. \fbib{he}} reached out his hand, touched him, and told him, \red{``I do want to. Be made clean!''} \v{42}Instantly the leprosy left him, and he was clean.

\v{43}Then Jesus\fnote{\fbackref{1:43} Lit. \fbib{he}} sternly warned him and encouraged him to go at once. \v{44}He told the man, \red{``Be sure that you don't tell anyone. Instead, go and show yourself to the priest, and then offer for your cleansing what Moses commanded as proof to the authorities.''}\fnote{\fbackref{1:44} Lit. \fbib{to them}} \v{45}But when the man\fnote{\fbackref{1:45} Lit. \fbib{he}} left, he began to proclaim it freely. He spread the news so widely that Jesus\fnote{\fbackref{1:45} Lit. \fbib{he}} could no longer enter a town openly, but had to stay out in deserted places. Still, people\fnote{\fbackref{1:45} Lit. \fbib{they}} kept coming to him from everywhere.
\labelchapt{2}
\passage{Jesus Heals a Paralyzed Man}
\passageinfo{(Matthew 9:1-8; Luke 5:17-26)}

\chapt{2}
\v{1}Several days later, Jesus\fnote{\fbackref{2:1} Lit. \fbib{he}} returned to Capernaum and it was reported that he was at home. \v{2}Such a large crowd gathered that there wasn't room for them, even in front of the door. Jesus\fnote{\fbackref{2:2} Lit. \fbib{He}} was speaking his message to them \v{3}when some people\fnote{\fbackref{2:3} Lit. \fbib{they}} came and brought him a paralyzed man being carried by four men. \v{4}Since they couldn't bring him to Jesus because of the crowd, they made an opening in the roof over the place where he was. They dug through it and let down the mat on which the paralyzed man was lying.

\v{5}When Jesus saw their faith, he told the paralyzed man, \red{``Son, your sins are forgiven.''}

\v{6}Now some scribes were sitting there, arguing among themselves,\fnote{\fbackref{2:6} Lit. \fbib{in their hearts}} \v{7}``Why does this man talk this way? He is blaspheming! Who can forgive sins but God alone?''

\v{8}At once, Jesus knew in his spirit what they were saying to themselves. \red{``Why are you arguing about such things among yourselves?''}\fnote{\fbackref{2:8} Lit. \fbib{in your hearts}} he asked them. \v{9}\red{``Which is easier: to say to the paralyzed man, `Your sins are forgiven,' or `Get up, pick up your mat, and walk'?} \v{10}\red{But I want you to know}\fnote{\fbackref{2:10} Lit. \fbib{So that you will know}} \red{that the Son of Man has authority on earth to forgive sins}\red{{\ldots}'' }Then he told the paralyzed man, \v{11}\red{``I say to you, get up, pick up your mat, and go home!'' }\v{12}So the man\fnote{\fbackref{2:12} Lit. \fbib{he}} got up, immediately picked up his mat, and went out in front of all of them.

As a result, all of the people were amazed and began to glorify God as they kept on saying, ``We have never seen anything like this!''
\passage{Jesus Calls Matthew}
\passageinfo{(Matthew 9:9-13; Luke 5:27-32)}

\v{13}Jesus\fnote{\fbackref{2:13} Lit. \fbib{He}} went out again beside the sea. The whole crowd kept coming to him, and he kept teaching them. \v{14}As he was walking along, he saw Levi, the son of Alphaeus, sitting at the tax collector's desk. Jesus\fnote{\fbackref{2:14} Lit. \fbib{He}} told him, \red{``Follow me!''} So Levi\fnote{\fbackref{2:14} Lit. \fbib{he}} got up and followed him.

\v{15}Later, he was having dinner at Levi's\fnote{\fbackref{2:15} Lit. \fbib{his}} house. Many tax collectors and sinners were also eating with Jesus and his disciples, because there were many who were following him. \v{16}When the scribes and the Pharisees saw him eating with sinners and tax collectors, they asked his disciples, ``Why does he eat and drink\fnote{\fbackref{2:16} Other mss. lack \fbib{and drink}} with tax collectors and sinners?''

\v{17}When Jesus heard that, he told them, \red{``Healthy people don't need a physician, but sick ones do. I did not come to call righteous people, but sinners.''}
\passage{A Question about Fasting}
\passageinfo{(Matthew 9:14-17; Luke 5:33-39)}

\v{18}Now John's disciples and the Pharisees would fast regularly. Some people\fnote{\fbackref{2:18} Lit. \fbib{They}} came and asked Jesus,\fnote{\fbackref{2:18} Lit. \fbib{him}} ``Why do John's disciples and the Pharisees' disciples fast, but your disciples don't fast?''

\v{19}Jesus replied, \red{``The wedding guests}\fnote{\fbackref{2:19} Lit. \fbib{The children of the wedding hall}; or \fbib{The children of the groom}}\red{ can't fast while the groom is with them, can they? As long as they have the groom with them, they can't fast.} \v{20}\red{But the time will come when the groom will be taken away from them, and then they will fast on that day.''}
\passage{The Unshrunk Cloth}
\passageinfo{(Matthew 9:16; Luke 5:36)}

\v{21}\red{``No one patches an old garment with a piece of unshrunk cloth. If he does, the patch pulls away from it---the new from the old---and a worse tear is made.} \v{22}\red{And no one pours new wine into old wineskins. If he does, the wine will make the skins burst, and both the wine and the skins will be ruined. Instead, new wine is poured}\fnote{\fbackref{2:22} The Gk. lacks \fbib{is poured}} \red{into fresh wineskins.''}
\passage{Jesus is Lord of the Sabbath}
\passageinfo{(Matthew 12:1-8; Luke 6:1-5)}

\v{23}Jesus\fnote{\fbackref{2:23} Lit. \fbib{He}} happened to be going through the grain fields on a Sabbath.\fnote{\fbackref{2:23} Lit. \fbib{on the Sabbaths}} As they made their way, his disciples began picking the heads of grain. \v{24}The Pharisees asked him, ``Look! Why are they doing what is not lawful on Sabbath days?''\fnote{\fbackref{2:24} Lit. \fbib{on the Sabbaths}}

\v{25}He asked them, \red{``Haven't you read what David did when he and his companions were hungry and in need?} \v{26}\red{How was it that he went into the House of God during the lifetime}\fnote{\fbackref{2:26} The Gk. lacks \fbib{the lifetime}}\red{ of Abiathar the high priest and ate the Bread of the }\red{Presence, which was not lawful for anyone but the priests to eat, and gave some of it to his companions?''}

\v{27}Then he told them, \red{``The Sabbath was made for people, not people for the Sabbath.} \v{28}\red{Therefore, the Son of Man is Lord even of the Sabbath.''}
\labelchapt{3}
\passage{Jesus Heals a Man with a Paralyzed Hand}
\passageinfo{(Matthew 12:9-14; Luke 6:6-11)}

\chapt{3}
\v{1}Jesus\fnote{\fbackref{3:1} Lit. \fbib{He}} went into the synagogue again, and a man with a paralyzed hand was there. \v{2}The people\fnote{\fbackref{3:2} Lit. \fbib{They}} watched Jesus\fnote{\fbackref{3:2} Lit. \fbib{him}} closely to see whether he would heal him on the Sabbath,\fnote{\fbackref{3:2} Lit. \fbib{Sabbaths}} intending to accuse him of doing something wrong. \v{3}He told the man with the paralyzed hand, \red{``Come forward.''}\fnote{\fbackref{3:3} Lit. \fbib{into the middle}} \v{4}Then he asked them, \red{``Is it lawful to do good or to do evil on Sabbath days,}\fnote{\fbackref{3:4} Lit. \fbib{on the Sabbaths}}\red{ to save a life or to destroy it?''} But they remained silent.

\v{5}Jesus\fnote{\fbackref{3:5} Lit. \fbib{He}} looked around at them in anger, deeply hurt because of their hard hearts. Then he told the man, \red{``Hold out your hand.''} The man\fnote{\fbackref{3:5} Lit. \fbib{He}} held it out, and his hand was restored to health. \v{6}Immediately the Pharisees and Herodians\fnote{\fbackref{3:6} I.e. Royal party sympathizers} went out and began to plot how to kill him.
\passage{Jesus Encounters a Large Crowd}

\v{7}So Jesus withdrew with his disciples to the sea. A large crowd from Galilee, Judea, \v{8}Jerusalem, Idumea, from across the Jordan, and from the region around Tyre and Sidon followed him. They came to him because they kept hearing about everything he was doing. \v{9}Jesus\fnote{\fbackref{3:9} Lit. \fbib{He}} told his disciples to have a boat ready for him so that the crowd wouldn't crush him, \v{10}because he had healed so many people that everyone who had diseases kept crowding up against him in order to touch him. \v{11}Whenever the unclean spirits saw him, they would fall down in front of him and scream, ``You are the Son of God!'' \v{12}But he sternly ordered them again and again not to tell people who he was.
\passage{Jesus Appoints Twelve Apostles}
\passageinfo{(Matthew 10:1-4; Luke 6:12-16)}

\v{13}Then Jesus\fnote{\fbackref{3:13} Lit. \fbib{he}} went up on a hillside and called to himself those whom he had decided on, and they approached him. \v{14}He appointed the Twelve,\fnote{\fbackref{3:14} Or \fbib{appointed twelve}} whom he called apostles, to accompany him, to be sent out to preach, \v{15}and to have the authority to drive out demons. \v{16}He appointed the Twelve:\fnote{\fbackref{3:16} Other mss. lack \fbib{He appointed the Twelve}} Simon (whom he named Peter), \v{17}Zebedee's sons James and his brother John (whom he named Boanerges, that is, Sons of Thunder), \v{18}Andrew, Philip, Bartholomew, Matthew, Thomas, James son of Alphaeus, Thaddeus,\fnote{\fbackref{3:18} Other mss. read \fbib{Lebbaeus}} Simon the Cananaean,\fnote{\fbackref{3:18} \fbib{Cananaean} is Aram. for \fbib{Zealot.}} \v{19}and Judas Iscariot, who also betrayed him.
\passage{Jesus is Accused of Working with Beelzebul}
\passageinfo{(Matthew 12:22-32; Luke 11:14-23; 12:10)}

\v{20}Then he went home. Such a large crowd gathered again that Jesus and his disciples\fnote{\fbackref{3:20} Lit. \fbib{so that they}} couldn't even eat. \v{21}When his family heard about it, they went to restrain him, because they kept saying, ``He's out of his mind!''

\v{22}The scribes who had come down from Jerusalem kept repeating, ``He has Beelzebul,'' and, ``He drives out demons by the ruler of demons.''

\v{23}So Jesus\fnote{\fbackref{3:23} Lit. \fbib{he}} called them together and began to speak to them in parables. \red{``How can Satan drive out Satan?} \v{24}\red{If} \red{a kingdom is divided against itself, that kingdom cannot stand.} \v{25}\red{And if a household is divided against itself, that household w}\red{o}\red{n}\red{'}\red{t stand.} \v{26}\red{So if Satan rebels against himself and is divided, he cannot stand. Indeed, his end has come.} \v{27}\red{No one can go into a strong man's house and carry off his possessions without first tying up the strong man. Then he can ransack his house.} \v{28}\red{I tell all of you}\fnote{\fbackref{3:28} The Gk. pronoun \fbib{you} is pl.}\red{ with certainty, people will be forgiven their sins and whatever blasphemies they utter.}\fnote{\fbackref{3:28} Lit. \fbib{they blaspheme}} \v{29}\red{But whoever blasphemes against the Holy Spirit can never be forgiven, but is guilty of eternal sin.''} \v{30}{\ldots}because they had been saying, ``He has an unclean spirit.''
\passage{The True Family of Jesus}
\passageinfo{(Matthew 12:46-50; Luke 8:19-21)}

\v{31}Then his mother and his brothers arrived. Milling around outside, they sent for him, continuously summoning him. \v{32}A crowd was sitting around him. They told him, ``Look! Your mother and your brothers\fnote{\fbackref{3:32} Other mss. read \fbib{your brothers and sisters}} are outside asking for you.''

\v{33}He answered them, \red{``Who are my mother and my brothers?''} \v{34}Then looking at the people sitting around him, he said, \red{``Here are my mother and my brothers!} \v{35}\red{Whoever does the will of God is my brother and sister and mother.''}
\labelchapt{4}
\passage{The Parable about a Farmer}
\passageinfo{(Matthew 13:1-9; Luke 8:4-18)}

\chapt{4}
\v{1}Then Jesus\fnote{\fbackref{4:1} Lit. \fbib{he}} began to teach again beside the sea. Such a large crowd gathered around him that he got into a boat and sat in it,\fnote{\fbackref{4:1} Lit. \fbib{on the sea}} while the entire crowd remained beside the sea on the shore. \v{2}He began teaching them many things in parables. While he was teaching them he said, \v{3}\red{``Listen! A farmer went out to sow. }\v{4}\red{As he was sowing, some seeds fell along the path, and birds came and ate them up.} \v{5}\red{Others fell on stony ground, where they didn}\red{'}\red{t have a lot of soil. They sprouted at once}\red{,}\red{ because the soil wasn't deep.} \v{6}\red{But when the sun came up, they were scorched. Since they didn}\red{'}\red{t have any roots, they dried up.} \v{7}\red{Others fell among thorn bushes, and the thorn bushes came up and choked them out, and they didn}\red{'}\red{t produce anything. }\v{8}\red{But others fell on good soil and produced a crop. They grew up, increased in size, and produced 30, 60, or 100 times what was sown.''}\fnote{\fbackref{4:8} The Gk. lacks \fbib{what was sown}} \v{9}He added, \red{``Let the person who has ears to hear, listen!''}
\passage{The Purpose of the Parables}
\passageinfo{(Matthew 13:10-17; Luke 8:9-10)}

\v{10}When he was alone with the Twelve and those around him, they began to ask him about the parables. \v{11}He told them, \red{``The secret about the }\red{kingdom}\red{ of }\red{God}\red{ has been given to you. But to those on the outside, everything comes in parables} \v{12}\red{so} \red{that}

\begin{poetry}
\poeml \red{`they may see clearly but not perceive,} \\
\poemll    \red{and they may hear clearly but not understand,} \\
\poemlll       \red{otherwise they might turn around and be forgiven.'\,''}\fnote{\fbackref{4:12} Cf. Isa 6:9-10}
\end{poetry}
\passage{Jesus Explains the Parable about the Farmer}
\passageinfo{(Matthew 13:18-23; Luke 8:11-15)}

\v{13}Then he told them, \red{``You don't understand this parable, so how can you understand any of the parables?} \v{14}\red{The farmer sows the word.} \v{15}\red{Some people are like the seeds}\fnote{\fbackref{4:15} Lit. \fbib{These are the ones}} \red{along the path, where the word is sown. When they hear it, Satan immediately comes and takes away the word that was sown in them.} \v{16}\red{Others are like the seeds}\fnote{\fbackref{4:16} Lit. \fbib{These are the ones}} \red{sown on the stony ground. When they hear the word, at once they joyfully accept it,} \v{17}\red{but since they don't have any roots, they last for only a short time. When trouble or persecution comes along because of the word, they immediately fall away.} \v{18}\red{Still others are like the seeds}\fnote{\fbackref{4:18} Lit. \fbib{are those}} \red{sown among the thorn bushes. These are the people who hear the word,} \v{19}\red{but the worries of life, the deceitful pleasures of wealth, and the desires for other things come in and choke }\red{the word so that it can't produce a crop.} \v{20}\red{Others are like the seeds}\fnote{\fbackref{4:20} Lit. \fbib{are those}} \red{sown on good soil. They hear the word, accept it, and produce crops---30, 60, or 100 times what was sown.''}\fnote{\fbackref{4:20} The Gk. lacks \fbib{what was sown}}
\passage{A Light under a Basket}
\passageinfo{(Luke 8:16-18)}

\v{21}Then Jesus\fnote{\fbackref{4:21} Lit. \fbib{he}} told them, \red{``A lamp isn't brought indoors to be put under a basket or under a bed, is it? It's to be put on a lamp stand, isn't it?} \v{22}\red{Nothing is hidden except for the purpose of having it revealed, and nothing is secret except for the purpose of having it come to light.} \v{23}\red{If anyone has ears to hear, let him listen!}

\v{24}He went on to say to them, \red{``Pay attention to what you're hearing! You will be evaluated by the same standard with which you do your evaluating, and still more will be given to you,} \v{25}\red{because whoever has something, will have more given to him. But whoever has nothing, even what he has will be taken away.''}
\passage{The Parable about a Growing Seed}

\v{26}He was also saying, \red{``The }\red{kingdom}\red{ of }\red{God}\red{ is like a man who scatters seeds on the ground.} \v{27}\red{He sleeps and gets up night and day while the seeds sprout and grow, although he doesn't know how }\v{28}\red{t}\red{he ground produces grain by itself}\red{---}\red{first the stalk, then the head, then the full grain in the head. }\v{29}\red{But when the grain is ripe, he immediately starts cutting with his sickle because the harvest time has come.''}
\passage{The Parable about a Mustard Seed}
\passageinfo{(Matthew 13:31-32; Luke 13:18-19)}

\v{30}He was also saying, \red{``How can we show what the }\red{kingdom}\red{ of }\red{God}\red{ is like, or what parable can we use to describe it?} \v{31}\red{It}\red{'}\red{s like a mustard seed planted in the ground. Although it}\red{'}\red{s the smallest of}\fnote{\fbackref{4:31} Or \fbib{smaller than}} \red{all the seeds on earth,} \v{32}\red{when it}\red{'}\red{s planted it comes up and becomes larger than all the garden plants. It grows such large branches that the birds in the sky can nest in its shade.''}
\passage{Why Jesus Used Parables}
\passageinfo{(Matthew 13:34-35)}

\v{33}With many other parables like these, Jesus\fnote{\fbackref{4:33} Lit. \fbib{he}} kept speaking his message to them according to their ability to understand. \v{34}He did not tell them anything without using\fnote{\fbackref{4:34} The Gk. lacks \fbib{using}} a parable, though he explained everything to his disciples in private.
\passage{Jesus Calms the Sea}
\passageinfo{(Matthew 8:23-27; Luke 8:22-25)}

\v{35}That day, when evening had come, he told them, \red{``Let's cross to the other side.''} \v{36}So they left the crowd and took him away in a boat without making any special preparations.\fnote{\fbackref{4:36} Lit. \fbib{boat just as he was}} Other boats were with him. \v{37}A violent windstorm came up, and the waves began breaking into the boat, so that the boat was rapidly\fnote{\fbackref{4:37} Lit. \fbib{already}} becoming swamped.

\v{38}But Jesus\fnote{\fbackref{4:38} Lit. \fbib{he}} was in the back of the boat, asleep on a cushion. So they woke him up and asked him, ``Teacher, don't you care that we're going to die?''

\v{39}Then he got up, rebuked the wind, and told the sea, \red{``Calm down! Be still!''} Then the wind stopped blowing, and there was a great calm. \v{40}He asked them, \red{``Why are you such cowards? Don't you have any faith yet?''}

\v{41}Overcome with fear, they kept saying to one another, ``Who is this man? Even the wind and the sea obey him!''
\labelchapt{5}
\passage{Jesus Heals a Demon-Possessed Man}
\passageinfo{(Matthew 8:28-34; Luke 8:26-39)}

\chapt{5}
\v{1}They arrived at the other side of the sea in the territory of the Gerasenes.\fnote{\fbackref{5:1} Other mss. read \fbib{Gergesenes}; still other mss. read \fbib{Gadarenes}} \v{2}Just as Jesus\fnote{\fbackref{5:2} Lit. \fbib{he}} stepped out of the boat, a man with an unclean spirit came out of the tombs and met him. \v{3}He lived among the tombs, and no one could restrain him any longer, not even with a chain. \v{4}He had often been restrained with shackles and chains, but had snapped the chains apart and broken the shackles in pieces. No one could tame him. \v{5}He kept screaming night and day among the tombs and on the mountainsides, and kept cutting himself with stones.

\v{6}When he saw Jesus from a distance, he ran and fell down in front of him, \v{7}screaming in a loud voice, ``What do you want with me, Jesus, Son of the Most High God? I beg you in the name of\fnote{\fbackref{5:7} The Gk. lacks \fbib{the name of}} God never to torment me!''

\v{8}Jesus\fnote{\fbackref{5:8} Lit. \fbib{Because he}} had been saying to him, \red{``Come out of the man, you unclean spirit!''} \v{9}Then Jesus\fnote{\fbackref{5:9} Lit. \fbib{he}} asked him, \red{``What's your name?''}

He told him, ``My name is Legion,\fnote{\fbackref{5:9} A Roman legion consisted of about 6,000 men.} because there are many of us.'' \v{10}He kept pleading with Jesus\fnote{\fbackref{5:10} Lit. \fbib{him}} not to send them out of that region.

\v{11}Now a large herd of pigs was grazing on a hillside nearby. \v{12}So the demons\fnote{\fbackref{5:12} Lit. \fbib{they}} begged him, ``Send us among the pigs, so that we can go into them!'' \v{13}So he let them do this. The unclean spirits came out of the man\fnote{\fbackref{5:13} The Gk. lacks \fbib{of the man}} and went into the pigs, and the herd of about 2,000 rushed down the cliff into the sea and drowned there.\fnote{\fbackref{5:13} Lit. \fbib{drowned in the sea}}

\v{14}Now when those who had been taking care of the pigs ran away, they reported what had happened\fnote{\fbackref{5:14} Lit. \fbib{they reported it}} in the city and countryside. So the people\fnote{\fbackref{5:14} Lit. \fbib{they}} went to see what had happened. \v{15}When they came to Jesus and saw the man who had been possessed by the legion of demons, sitting there dressed and in his right mind, they were frightened. \v{16}The people who had seen it told them what had happened to the demon-possessed man and the pigs. \v{17}So they began to beg Jesus\fnote{\fbackref{5:17} Lit. \fbib{him}} to leave their territory.

\v{18}As Jesus\fnote{\fbackref{5:18} Lit. \fbib{he}} was getting into the boat, the man who had been demon-possessed kept begging him to let him go with him. \v{19}But Jesus\fnote{\fbackref{5:19} Lit. \fbib{he}} wouldn't let him. Instead, he told him, \red{``Go home to your family, and tell them how much the Lord has done for you and how merciful he has been to you.''} \v{20}So the man\fnote{\fbackref{5:20} Lit. \fbib{he}} left and began proclaiming in the Decapolis\fnote{\fbackref{5:20} Lit. \fbib{the Ten Cities,} a loose federation of ten cities strongly influenced by Greek culture.} how much Jesus had done for him. And everyone was utterly amazed.
\passage{Jesus Heals a Woman and Resurrects a Girl}
\passageinfo{(Matthew 9:18-26; Luke 8:40-56)}

\v{21}When Jesus again had crossed to the other side in a boat,\fnote{\fbackref{5:21} Other mss. lack \fbib{in a boat}} a large crowd gathered around him by the seashore. \v{22}Then a synagogue leader named Jairus arrived. When he saw Jesus,\fnote{\fbackref{5:22} Lit. \fbib{him}} he fell at his feet \v{23}and begged him urgently, saying, ``My little daughter is dying. Come and lay your hands on her so that she may get well and live.'' \v{24}So Jesus\fnote{\fbackref{5:24} Lit. \fbib{he}} went with him. A huge crowd kept following him and jostling him.

\v{25}Now there was a woman who had been suffering from chronic bleeding for twelve years. \v{26}Although she had endured a great deal under the care of many doctors and had spent all of her money, she had not been helped at all, but rather grew worse. \v{27}Since she had heard about Jesus, she came up behind him in the crowd and touched his robe, \v{28}because she had been saying, ``If I can just touch his robe, I will get well.'' \v{29}Her bleeding stopped at once, and she felt in her body that she was healed from her illness.

\v{30}Immediately Jesus became aware that power had gone out of him. So he turned around in the crowd and asked, \red{``Who touched my clothes?''}

\v{31}His disciples asked him, ``You see the crowd jostling you, and yet you ask, \red{`Who touched me?'}'' \v{32}But he kept looking around to look at the woman who had done this. \v{33}So the woman, knowing what had happened to her, came forward fearfully, fell down trembling in front of him, and told him the whole truth.

\v{34}He told her, \red{``Daughter, your faith has made you well. Go in peace and be healed from your illness.''}

\v{35}While he was still speaking, some people\fnote{\fbackref{5:35} Lit. \fbib{they}} came from the synagogue leader's home\fnote{\fbackref{5:35} Lit. \fbib{from the synagogue leader}} and said, ``Your daughter is dead. Why bother the Teacher anymore?''

\v{36}But when Jesus heard\fnote{\fbackref{5:36} Other mss. read \fbib{overheard}} what they said, he told the synagogue leader, \red{``Stop being afraid! Just keep on believing.''} \v{37}Jesus\fnote{\fbackref{5:37} Lit. \fbib{He}} allowed no one to go further with him except Peter, James, and John, the brother of James.

\v{38}When they came to the home of the synagogue leader, Jesus\fnote{\fbackref{5:38} Lit. \fbib{he}} saw mass confusion. People\fnote{\fbackref{5:38} Lit. \fbib{They}} were crying and sobbing loudly. \v{39}He entered the house\fnote{\fbackref{5:39} The Gk. lacks \fbib{the house}} and asked them, \red{``Why all this confusion and crying? The child isn't dead. She's sleeping.''} \v{40}They laughed and laughed at him. But when he had driven all of them outside, he took the child's father and mother, along with the men who were with him, and went into the room\fnote{\fbackref{5:40} The Gk. lacks \fbib{the room}} where the child was.

\v{41}He took her by the hand and told her, \red{``Talitha koum,''}\fnote{\fbackref{5:41} \fbib{Talitha Koum} is Heb./Aram. for \fbib{Little girl, get up!}} which means, \red{``Young lady, I tell you, get up!''} \v{42}The young lady got up at once and started to walk. She was twelve years old. Instantly they were overcome with astonishment. \v{43}But Jesus\fnote{\fbackref{5:43} Lit. \fbib{he}} strictly ordered them not to let anyone know about this. He also told them to give her something to eat.
\labelchapt{6}
\passage{Jesus is Rejected at Nazareth}
\passageinfo{(Matthew 13:53-58; Luke 4:16-30)}

\chapt{6}
\v{1}Jesus\fnote{\fbackref{6:1} Lit. \fbib{He}} left that place and went back to his hometown,\fnote{\fbackref{6:1} I.e. Nazareth} and his disciples followed him. \v{2}When the Sabbath came, he began to teach in the synagogue, and many who heard him were utterly amazed. ``Where did this man get all these things?'' they asked. ``What is this wisdom that has been given to him? What great miracles are being done by his hands! \v{3}This is the builder,\fnote{\fbackref{6:3} Or \fbib{carpenter}} the son of Mary, and the brother of James, Joseph, Judas, and Simon, isn't it? His sisters are here with us, aren't they?'' And they were offended by him.

\v{4}Jesus had been telling them, \red{``A prophet is without honor only in his hometown, among his relatives, and in his own home.''} \v{5}He couldn't perform a miracle there except to lay his hands on a few sick people and heal them. \v{6}He was amazed at their unbelief. Then he went around to the villages and continued teaching.
\passage{Jesus Sends Out the Twelve}
\passageinfo{(Matthew 10:1, 5-15; Luke 9:1-6)}

\v{7}He called the Twelve and began to send them out two by two, giving them authority over unclean spirits. \v{8}He instructed them to take nothing along on the trip except a walking stick---no bread, no traveling bag, nothing in their moneybag. \v{9}They could wear sandals but not take along an extra shirt.\fnote{\fbackref{6:9} Lit. \fbib{along two shirts}} \v{10}He told them repeatedly, \red{``Whenever you go into a home, stay there until you leave that place.} \v{11}\red{If any place will not welcome you and the} \red{people}\fnote{\fbackref{6:11} Lit. \fbib{they}} \red{refuse to listen to you, when you leave, shake its dust off your feet as a testimony against them.''} \v{12}So they went and preached that people\fnote{\fbackref{6:12} Lit. \fbib{they}} should repent. \v{13}They also kept driving out many demons and anointing with oil many who were sick, and healing them.
\passage{The Death of John the Baptist}
\passageinfo{(Matthew 14:1-12; Luke 9:7-9)}

\v{14}King Herod heard about this, because Jesus'\fnote{\fbackref{6:14} Lit. \fbib{his}} name had become well-known. He was\fnote{\fbackref{6:14} Other mss. read \fbib{They were}} saying, ``John the Baptist has been raised from the dead! That's why he is able to do these miracles.''

\v{15}Others were saying, ``He is Elijah.''

Still others were saying, ``He is a prophet like one of the other\fnote{\fbackref{6:15} The Gk. lacks \fbib{other}} prophets.''

\v{16}But when Herod heard about it, he said, ``John, whom I beheaded, has been raised,'' \v{17}because Herod himself had sent men who arrested\fnote{\fbackref{6:17} Lit. \fbib{sent and arrested}} John, bound him with chains, and put him in prison on account of Herodias, his brother Philip's wife, whom Herod\fnote{\fbackref{6:17} Lit. \fbib{he}} had married.

\v{18}John had been telling Herod, ``It's not lawful for you to have your brother's wife.'' \v{19}So Herodias bore a grudge against John\fnote{\fbackref{6:19} Lit. \fbib{him}} and wanted to kill him. But she couldn't do it \v{20}because Herod was afraid of John. He knew that John\fnote{\fbackref{6:20} Lit. \fbib{he}} was a righteous and holy man, and so he protected him. Whenever he listened to John,\fnote{\fbackref{6:20} Lit. \fbib{him}} he did much of what he said.\fnote{\fbackref{6:20} Lit. \fbib{did many things}; other mss. read \fbib{he became very disturbed}} In fact, he liked listening to him.

\v{21}An opportunity came during Herod's birthday celebration, when he gave a banquet for his top officials, military officers, and the most important people of Galilee. \v{22}When the daughter of Herodias\fnote{\fbackref{6:22} Other mss. read \fbib{his daughter by Herodias}} came in and danced, she pleased Herod and his guests. So the king told the girl, ``Ask me for anything you want, and I'll give it to you.'' \v{23}He swore with an oath to her, ``I'll give you anything you ask for, up to half of my kingdom.''

\v{24}So she went out and asked her mother, ``What should I ask for?''

Her mother\fnote{\fbackref{6:24} Lit. \fbib{She}} replied, ``The head of John the Baptist.''

\v{25}Immediately the girl\fnote{\fbackref{6:25} Lit. \fbib{she}} hurried back to the king with her request, ``I want you to give me right now the head of John the Baptist on a platter.''

\v{26}The king was deeply saddened, yet because of his oaths and his guests he was reluctant to refuse her. \v{27}So without delay the king sent a soldier and ordered him to bring John's\fnote{\fbackref{6:27} Lit. \fbib{his}} head. The soldier\fnote{\fbackref{6:27} Lit. \fbib{He}} went and beheaded him in prison. \v{28}Then he brought John's\fnote{\fbackref{6:28} Lit. \fbib{his}} head on a platter and gave it to the girl, and the girl gave it to her mother. \v{29}When John's\fnote{\fbackref{6:29} Lit. \fbib{his}} disciples heard about this, they came and carried off his body and laid it in a tomb.
\passage{Jesus Feeds More than Five Thousand People}
\passageinfo{(Matthew 14:13-21; Luke 9:10-17; John 6:1-14)}

\v{30}The apostles gathered around Jesus and told him everything they had done and taught. \v{31}He told them, \red{``Come away to a deserted place all by yourselves and rest for a while,''} because so many people were coming and going\fnote{\fbackref{6:31} The Gk. lacks \fbib{and going}} that they didn't even have time to eat. \v{32}So they went away in a boat to a deserted place by themselves. \v{33}But when many people saw them leave and recognized them, they hurried on foot from all the towns and arrived ahead of them. \v{34}When he got out of the boat,\fnote{\fbackref{6:34} The Gk. lacks \fbib{of the boat}} he saw a large crowd. He had compassion for them, because they were like sheep without a shepherd, and he began to teach them many things.

\v{35}When it was quite late, his disciples came to him and said, ``This is a deserted place, and it's already late. \v{36}Send the crowds\fnote{\fbackref{6:36} Lit. \fbib{them}} away so that they can go to the neighboring farms and villages and buy themselves something to eat.''

\v{37}But he answered them, \red{``You give them something to eat.''}

They asked him, ``Should we go and buy 200 denarii\fnote{\fbackref{6:37} The denarius was the usual day's wage for a laborer.} worth of bread and give it to them to eat?''

\v{38}He asked them, \red{``How many loaves of bread do you have? Go and see.''}

They found out and told him, ``Five loaves\fnote{\fbackref{6:38} The Gk. lacks \fbib{loaves}} and two fish.''

\v{39}Then he ordered them to have all the people sit down in groups on the green grass. \v{40}So they sat down in groups of hundreds and fifties. \v{41}Taking the five loaves and the two fish, he looked up to heaven and blessed them. Then he broke the loaves in pieces and kept giving them to his disciples to set before the people.\fnote{\fbackref{6:41} Lit. \fbib{before them}} He also divided the two fish among them all. \v{42}All of them ate and were filled. \v{43}Then the disciples\fnote{\fbackref{6:43} Lit. \fbib{they}} picked up twelve baskets full of leftover bread and fish. \v{44}There were 5,000 men who had eaten the loaves.
\passage{Jesus Walks on the Sea}
\passageinfo{(Matthew 14:22-33; John 6:16-21)}

\v{45}Jesus\fnote{\fbackref{6:45} Lit. \fbib{He}} immediately had his disciples get into a boat and cross to Bethsaida ahead of him, while he sent the crowd away. \v{46}After saying goodbye to them, he went up on a hillside to pray. \v{47}When evening had come, the boat was in the middle of the sea, while he was alone on the land. \v{48}He saw that his disciples\fnote{\fbackref{6:48} Lit. \fbib{that they}} were straining at the oars, because the wind was against them. Shortly before dawn\fnote{\fbackref{6:48} Lit. \fbib{In the fourth watch of the night}} he came to them, walking on the sea. He intended to go up right beside them, \v{49}but when they saw him walking on the sea, they thought it was a ghost and began to scream. \v{50}All of them saw him and were terrified. Immediately he told them, \red{``Have courage! It}\red{'}\red{s }\red{me}\red{. Stop being afraid!''}

\v{51}Then he got into the boat with them, and the wind stopped blowing. The disciples\fnote{\fbackref{6:51} Lit. \fbib{They}} were utterly astounded, \v{52}because they didn't understand the significance of the loaves. Instead, their hearts were hardened.
\passage{Jesus Heals the Sick in Gennesaret}
\passageinfo{(Matthew 14:34-36)}

\v{53}When they had crossed over, they came ashore at Gennesaret and anchored the boat. \v{54}As soon as they got out of the boat, the people recognized Jesus.\fnote{\fbackref{6:54} Lit. \fbib{him}} \v{55}They ran all over the countryside and began carrying the sick on their mats to any place where they heard he was. \v{56}Wherever he went, whether into villages, towns, or farms, people\fnote{\fbackref{6:56} Lit. \fbib{they}} would place their sick in the marketplaces and beg him to let them touch even the tassel of his garment, and everyone who touched it was healed.
\labelchapt{7}
\passage{Jesus Challenges the Tradition of the Elders}
\passageinfo{(Matthew 15:1-20)}

\chapt{7}
\v{1}The Pharisees and some of the scribes who had come from Jerusalem gathered around Jesus.\fnote{\fbackref{7:1} Lit. \fbib{him}} \v{2}They noticed that some of his disciples were eating\fnote{\fbackref{7:2} Lit. \fbib{eating bread}} with unclean hands, that is, without washing them. \v{3}(The Pharisees---and indeed all the Jewish people---don't eat unless they wash their hands properly,\fnote{\fbackref{7:3} Lit. \fbib{with a fist}} following the tradition of their elders. \v{4}They don't eat anything from the marketplace unless they dip it in water. They also observe many other traditions, such as the proper washing of washing cups, jars, brass pots, and dinner tables.)\fnote{\fbackref{7:4} Other mss. lack \fbib{and dinner tables}} \v{5}So the Pharisees and the scribes asked Jesus,\fnote{\fbackref{7:5} Lit. \fbib{him}} ``Why don't your disciples live according to the tradition of the elders? Instead, they eat\fnote{\fbackref{7:5} Lit. \fbib{eat bread}} with unclean hands.''

\v{6}He told them, \red{``Isaiah was right when he prophesied about you hypocrites. As it is written,}

\begin{poetry}
\poeml \red{`These people honor me with their lips,} \\
\poemll    \red{but their hearts are far from me.} \\
\poeml \v{7}\red{Their worship of me is worthless,} \\
\poemll    \red{because they teach human rules as doctrines.'}\fnote{\fbackref{7:7} Cf. Isa 29:13}
\end{poetry}

\v{8}\red{You abandon the commandment of God and hold to human tradition.''}

\v{9}Then he told them,\red{ ``You have such a fine way of rejecting the commandment of God in order to keep your own tradition! }\v{10}\red{Because Moses said, `Honor your father and your mother,'}\fnote{\fbackref{7:10} Cf. Exod 20:12; Deut 5:16}\red{ and}\red{,}\red{ `Whoever curses his father or mother must certainly be put to death.'}\fnote{\fbackref{7:10} Cf. Exod 21:17; Lev 20:9} \v{11}\red{But you say, `If anyone tells his father or mother, ``Whatever support you might have received from me is Corban,''\,' }(that is, an offering to God) \v{12}\red{`you no longer let him do anything for }\red{his father or mother.' }\v{13}\red{You are destroying the word of God through your tradition that you have handed down. And you do many other things like that.''}

\v{14}Then he called to the crowd again and told them, \red{``Listen to me, all of you, and understand!} \v{15}\red{Nothing that goes into a person from the outside can make him unclean. It's what comes out of a person that makes a person unclean.} \v{16}\red{If anyone has ears to hear, let him listen!''}\fnote{\fbackref{7:16} Other mss. lack this verse.}

\v{17}When he had left the crowd and gone home, his disciples began asking him about the parable. \v{18}He asked them, \red{``Are you so ignorant? Don't you know that nothing that goes into a person from the outside can make him unclean?} \v{19}\red{Because it doesn't go into his heart but into his stomach, and }\red{then into the sewer,}\fnote{\fbackref{7:19} Or \fbib{drain}}\red{ thereby expelling}\fnote{\fbackref{7:19}Or \fbib{cleansing from}; the Gk. lacks \fbib{thereby}}\red{ all foods.''} \v{20}Then he continued, \red{``It's what comes out of a person that makes a person unclean,} \v{21}\red{because it's from within, from the human heart, that evil thoughts come, as well as sexual immorality, stealing, murder,} \v{22}\red{adultery, greed, wickedness, cheating, shameless lust, envy, slander,}\fnote{\fbackref{7:22} Or \fbib{blasphemy}} \red{arrogance, and foolishness. }\v{23}\red{All these things come from inside and make a person unclean.''}
\passage{A Canaanite Woman's Faith}
\passageinfo{(Matthew 15:21-28)}

\v{24}Jesus\fnote{\fbackref{7:24} Lit. \fbib{He}} left that place and went to the territory of Tyre and Sidon.\fnote{\fbackref{7:24} Other mss. lack \fbib{and Sidon}} He went into a house, not wanting anyone to know he was there. However, it couldn't be kept a secret. \v{25}In fact, a woman whose little daughter had an unclean spirit immediately heard about him and came and fell down at his feet. \v{26}Now the woman happened to be a Greek, born in Phoenicia in Syria. She kept asking him to drive the demon out of her daughter. \v{27}But he kept telling her, \red{``First let the children be filled. It is not right to take the children's bread and throw it to the puppies.''}

\v{28}But she answered him, ``Yes,\fnote{\fbackref{7:28} Other mss. lack \fbib{Yes}} Lord. Yet even the puppies under the table eat some of the children's crumbs.''

\v{29}Then he told her, \red{``Because you have said this, go! The demon has left your daughter.''} \v{30}So she went home and found her child lying in bed, and the demon was gone.
\passage{Jesus Heals a Deaf Man with a Speech Impediment}

\v{31}Then Jesus\fnote{\fbackref{7:31} Lit. \fbib{he}} left the territory of Tyre and passed through Sidon towards the Sea of Galilee, in the territory of the Decapolis.\fnote{\fbackref{7:31} Lit. \fbib{the Ten Cities,} a loose federation of ten cities strongly influenced by Greek culture} \v{32}Some people\fnote{\fbackref{7:32} Lit. \fbib{They}} brought him a deaf man who also had a speech impediment. They begged him to lay his hand on him. \v{33}Jesus\fnote{\fbackref{7:33} Lit. \fbib{He}} took him away from the crowd to be alone with him. Putting his fingers into the man's\fnote{\fbackref{7:33} Lit. \fbib{his}} ears, he touched the man's\fnote{\fbackref{7:33} Lit. \fbib{his}} tongue with saliva.

\v{34}Then he looked up to heaven, sighed, and told him, \red{``Ephphatha,''}\fnote{\fbackref{7:34} \fbib{Ephphatha} is Heb./Aram. for \fbib{Be opened!}} that is, \red{``Be opened!''} \v{35}The man's\fnote{\fbackref{7:35} Lit. \fbib{his}} hearing and speech were restored at once, and he began to talk normally. \v{36}Jesus\fnote{\fbackref{7:36} Lit. \fbib{He}} ordered the people\fnote{\fbackref{7:36} Lit. \fbib{them}} not to tell anyone, but the more he kept ordering them, the more they kept spreading the news.

\v{37}Amazed beyond measure, they kept on saying, ``He does everything well! He even makes deaf people hear and mute people talk!''
\labelchapt{8}
\passage{Jesus Feeds More than Four Thousand People}
\passageinfo{(Matthew 15:32-39)}

\chapt{8}
\v{1}At that time, after a large crowd again had gathered together with nothing to eat, Jesus\fnote{\fbackref{8:1} Lit. \fbib{he}} called his disciples and told them, \v{2}\red{``I have compassion for the crowd}\red{,}\red{ because they}\red{'}\red{ve already been with me for three days and have nothing to eat. }\v{3}\red{If I send them away to their homes hungry, they}\red{'}\red{ll faint on the road. Some of them have come a long distance.''}

\v{4}His disciples answered him, ``Where could anyone get enough bread to feed these people out here in the wilderness?''

\v{5}He asked them, \red{``How many loaves of bread do you have?''}

``Seven,'' they said.

\v{6}So he ordered the crowd to sit down on the ground. Then he took the seven loaves and gave thanks. He broke them in pieces and kept giving them to his disciples to distribute. So they served them to the crowd. \v{7}They also had a few small fish. He blessed them and said that the fish\fnote{\fbackref{8:7} Lit. \fbib{these}} should also be distributed. \v{8}The people\fnote{\fbackref{8:8} Lit. \fbib{They}} ate and were filled. Then the disciples\fnote{\fbackref{8:8} Lit. \fbib{they}} picked up the leftover pieces---seven large baskets full. \v{9}Now about 4,000 men were there. Then he sent them on their way. \v{10}Immediately he got into a boat with his disciples and went to the region of Dalmanutha.\fnote{\fbackref{8:10} Other mss. read \fbib{Mageda}; still other mss. read \fbib{Magdala}}
\passage{Interpreting the Time}
\passageinfo{(Matthew 16:1-4; Luke 12:54-56)}

\v{11}The Pharisees arrived and began arguing with Jesus.\fnote{\fbackref{8:11} Lit. \fbib{him}} They tested him by demanding from him a sign from heaven. \v{12}He sighed deeply in his spirit and remarked, \red{``Why do those living today}\fnote{\fbackref{8:12} Lit. \fbib{Why does this generation}}\red{ demand a sign? I tell all of you}\fnote{\fbackref{8:12} The Gk. pronoun \fbib{you} is pl.}\red{ with certainty, no sign will be given to this generation.''} \v{13}Leaving them, he got into a boat again and crossed to the other side.
\passage{The Yeast of the Pharisees and Sadducees}
\passageinfo{(Matthew 16:5-12)}

\v{14}Now the disciples\fnote{\fbackref{8:14} Lit. \fbib{they}} had forgotten to take any bread along, but they had one loaf with them in the boat. \v{15}Jesus\fnote{\fbackref{8:15} Lit. \fbib{He}} had been warning them, \red{``Watch out! Beware of the yeast of the Pharisees and the yeast of Herod!''}\fnote{\fbackref{8:15} Other mss. read \fbib{of the Herodians}}

\v{16}So they were discussing with one another the fact that they didn't have any bread. \v{17}Knowing this, Jesus\fnote{\fbackref{8:17} Lit. \fbib{he}} asked them, \red{``Why are you discussing the fact that you don't have any bread? Don't you understand or perceive yet? Are your hearts hardened?} \v{18}\red{Do} \red{you have eyes}\red{,}\red{ but fail to see? Do you have ears}\red{, }\red{but fail to hear?}\fnote{\fbackref{8:18} Cf. Jer 5:21} \red{Don't you remember?} \v{19}\red{When} \red{I broke the five loaves for the 5,000, how many baskets did you fill with leftover pieces?''}

They told him, ``Twelve.''

\v{20}``\red{When I broke}\fnote{\fbackref{8:20} The Gk. lacks \fbib{I broke}} \red{the seven loaves}\fnote{\fbackref{8:20} The Gk. lacks \fbib{loaves}} \red{for the 4,000, how many large baskets did you fill with the leftover pieces?''}

They told him, ``Seven.''

\v{21}Then he asked them, \red{``Don't you understand yet?''}
\passage{Jesus Heals a Blind Man in Bethsaida}

\v{22}As they came to Bethsaida, some people\fnote{\fbackref{8:22} Lit. \fbib{they}} brought a blind man to Jesus\fnote{\fbackref{8:22} Lit. \fbib{him}} and begged him to touch him.\fnote{\fbackref{8:22} The Gk. lacks \fbib{him}} \v{23}Jesus\fnote{\fbackref{8:23} Lit. \fbib{Then he}} took the blind man by the hand and led him out of the village. He spit into his eyes, placed his hands on him, and asked him, \red{``Do you see anything?''}

\v{24}The man\fnote{\fbackref{8:24} Lit. \fbib{He}} looked up and said, ``I see people, but they look like trees walking around.'' \v{25}Then Jesus\fnote{\fbackref{8:25} Lit. \fbib{he}} placed his hands on the man's\fnote{\fbackref{8:25} Lit. \fbib{his}} eyes again, and he saw clearly. His sight was restored, and he saw everything perfectly, even from a distance.

\v{26}Then Jesus\fnote{\fbackref{8:26} Lit. \fbib{he}} sent him home, saying, \red{``Don't go into the village or tell anyone in the village.''}\fnote{\fbackref{8:26} Other mss. lack \fbib{or tell anyone in the village}}
\passage{Peter Declares His Faith in Jesus}
\passageinfo{(Matthew 16:13-20; Luke 9:18-21)}

\v{27}Then Jesus and his disciples set out for the villages around Caesarea Philippi. On the way he was asking his disciples, \red{``Who do people say I am?''}

\v{28}They answered him, ``Some say\fnote{\fbackref{8:28} The Gk. lacks \fbib{Some say}} John the Baptist, others Elijah, and still others one of the prophets.''

\v{29}Then he began to ask them, \red{``But who do you say I am?''}

Peter answered him, ``You are the Messiah!''\fnote{\fbackref{8:29} Or \fbib{Christ}} \v{30}Jesus\fnote{\fbackref{8:30} Lit. \fbib{He}} sternly ordered them not to tell anyone about him.
\passage{Jesus Predicts His Death and Resurrection}
\passageinfo{(Matthew 16:21-28; Luke 9:21-27)}

\v{31}Then he began to teach them that the Son of Man would have to suffer a great deal and be rejected by the elders, the high priests, and the scribes. Then he would be killed, but after three days he would rise again. \v{32}He was speaking about this matter quite openly.

Peter took him aside and began to rebuke him. \v{33}But turning and looking at his disciples, Jesus\fnote{\fbackref{8:33} Lit. \fbib{he}} rebuked Peter, saying, \red{``Get behind me, Satan, because you're not thinking God's thoughts}\red{,}\red{ but human thoughts!''}

\v{34}Then Jesus\fnote{\fbackref{8:34} Lit. \fbib{he}} called the crowd to himself along with his disciples and told them, \red{``If anyone wants to follow me, he must deny himself, pick up his cross, and follow me }\red{continuously}\red{,} \v{35}\red{because whoever wants to save his life will lose it, but whoever loses his life for my sake and for the gospel will save it.} \v{36}\red{What profit will a person have if he gains the whole world and forfeits his life? }\v{37}\red{Indeed, what can a person give in exchange for his life?} \v{38}\red{If anyone is ashamed of me and my words in this adulterous and sinful generation, the Son of Man will be ashamed of him when he comes with the holy angels in his Father's glory.''}

\chapt{9}
\v{1}Then he told them, \red{``I tell all of you}\fnote{\fbackref{9:1} The Gk. pronoun \fbib{you} is pl.}\red{ with certainty, some people standing here will not experience}\fnote{\fbackref{9:1} Lit. \fbib{taste}} \red{death until they see the }\red{kingdom}\red{ of }\red{God}\red{ arrive with power.''}
\labelchapt{9}
\passage{Jesus' Appearance is Changed}
\passageinfo{(Matthew 17:1-13; Luke 9:28-36)}

\v{2}Six days later, Jesus took Peter, James, and John and led them up a high mountain to be alone with him. His appearance was changed in front of them, \v{3}and his clothes became dazzling white, whiter than anyone\fnote{\fbackref{9:3} Lit. \fbib{anyone who cleans}} on earth could bleach them. \v{4}Then Elijah appeared to them, accompanied by Moses, and they were talking with Jesus.

\v{5}Then Peter told Jesus, ``Rabbi,\fnote{\fbackref{9:5} \fbib{Rabbi} is Heb. for \fbib{Master} and/or \fbib{Teacher}} it's good that we're here! Let's set up three shelters\fnote{\fbackref{9:5} Or \fbib{tents}}---one for you, one for Moses, and one for Elijah.'' \v{6}(Peter\fnote{\fbackref{9:6} Lit. \fbib{He}} didn't know how to respond, because they were terrified.)

\v{7}Then a cloud appeared and overshadowed them. A voice came out of the cloud and said,\fnote{\fbackref{9:7} The Gk. lacks \fbib{and said}} ``This is my Son, whom I love. Keep on listening to him!'' \v{8}Suddenly, as they looked around, they saw no one with them but Jesus alone.

\v{9}On their way down the mountain, Jesus\fnote{\fbackref{9:9} Lit. \fbib{he}} ordered them not to tell anyone what they had seen until the Son of Man had risen from the dead. \v{10}They kept the matter to themselves but argued about what ``rising from the dead'' meant. \v{11}So they asked him, ``Don't the scribes say that Elijah must come first?''

\v{12}He told them, \red{``Elijah is indeed coming first and will restore all things. Why, then, is it written that the Son of Man must suffer a great deal and be treated shamefully?} \v{13}\red{But I tell you that Elijah has come, yet people}\fnote{\fbackref{9:13} Lit. \fbib{they}} \red{treated him just as they pleased, as it is written about him.''}
\passage{Jesus Heals a Boy with a Demon}
\passageinfo{(Matthew 17:14-20)}

\v{14}As they approached the other\fnote{\fbackref{9:14} The Gk. lacks \fbib{other}} disciples, they saw a large crowd around them and some scribes arguing with them. \v{15}The whole crowd was very surprised to see Jesus\fnote{\fbackref{9:15} Lit. \fbib{him}} and ran to welcome him.

\v{16}He asked the scribes,\fnote{\fbackref{9:16} Lit. \fbib{them}} \red{``What are you arguing about with them?''}

\v{17}A man in the crowd answered him, ``Teacher, I brought my son to you. He has a spirit that won't let him talk. \v{18}Whenever it brings on a seizure, it throws him to the ground. Then he foams at the mouth, grinds his teeth, and becomes stiff. So I asked your disciples to drive the spirit\fnote{\fbackref{9:18} Lit. \fbib{it}} out, but they didn't have the power.''

\v{19}Jesus\fnote{\fbackref{9:19} Lit. \fbib{He}} told them, \red{``You unbelieving generation! How long must I be with you? How long must I put up with you? Bring him to me!''}

\v{20}So they brought the boy\fnote{\fbackref{9:20} Lit. \fbib{him}} to him. When the spirit saw Jesus,\fnote{\fbackref{9:20} Lit. \fbib{him}} it immediately threw the boy\fnote{\fbackref{9:20} Lit. \fbib{him}} into convulsions. He fell on the ground and kept rolling around and foaming at the mouth. \v{21}Then Jesus\fnote{\fbackref{9:21} Lit. \fbib{he}} asked his father, \red{``How long has this been happening to him?''} He said, ``Since he was a child. \v{22}The spirit\fnote{\fbackref{9:22} Lit. \fbib{It}} has often thrown him into fire and into water to destroy him. But if you are able to do anything, have pity on us and help us!''

\v{23}Jesus told him, \red{```If you are able?' Everything is possible for the person who believes!''}

\v{24}With tears flowing,\fnote{\fbackref{9:24} Other mss. lack \fbib{With tears flowing}} the child's father at once cried out, ``I do believe! Help my unbelief!''

\v{25}When Jesus saw that a crowd was running to the scene, he rebuked the unclean spirit, saying to it, \red{``You spirit that won't let him talk or hear---I command you to come out of him and never enter him again!''} \v{26}The spirit\fnote{\fbackref{9:26} Lit. \fbib{It}} screamed, shook the child\fnote{\fbackref{9:26} The Gk. lacks \fbib{the child}} violently, and came out. The boy was like a corpse, and many said that he was dead. \v{27}But Jesus took his hand and helped him up, and he stood up.

\v{28}When Jesus\fnote{\fbackref{9:28} Lit. \fbib{he}} came home, his disciples asked him privately, ``Why couldn't we drive the spirit\fnote{\fbackref{9:28} Lit. \fbib{drive it}} out?''

\v{29}He told them, \red{``This kind can come out only by prayer and fasting.''}\fnote{\fbackref{9:29} Other mss. lack \fbib{and fasting}}
\passage{Jesus Again Predicts His Death and Resurrection}
\passageinfo{(Matthew 17:22-23; Luke 9:43-45)}

\v{30}Then they left that place and passed through Galilee. Jesus\fnote{\fbackref{9:30} Lit. \fbib{He}} didn't want anyone to find out about it, \v{31}because he was teaching his disciples, \red{``The Son of }\red{Man will be betrayed into human hands. They will kill him, but after being dead for three days he will be raised.''} \v{32}They didn't understand what this statement meant, and they were afraid to ask him.
\passage{True Greatness}
\passageinfo{(Matthew 18:1-5; Luke 9:46-48)}

\v{33}Then they came to Capernaum. While Jesus\fnote{\fbackref{9:33} Lit. \fbib{he}} was at home, he asked the disciples,\fnote{\fbackref{9:33} Lit. \fbib{them}} \red{``What were you arguing about on the road?''} \v{34}But they kept silent, because they had argued on the road with one another about who was the greatest.

\v{35}So he sat down, called the Twelve, and told them, \red{``If anyone wants to be first}\red{,}\red{ he must be last of all and servant of all.''} \v{36}Then he took a little child and had him stand among them. He took him in his arms and told them, \v{37}\red{``Whoever welcomes a child like this in my name welcomes me, and whoever welcomes me welcomes not me but the one who sent me.''}
\passage{The True Follower of Jesus}
\passageinfo{(Luke 9:49-50)}

\v{38}John told Jesus,\fnote{\fbackref{9:38} Lit. \fbib{him}} ``Teacher, we saw someone driving out demons in your name. We tried to stop him, because he wasn't a follower like us.''

\v{39}But Jesus said, \red{``Don't stop him, because no one who works a miracle in my name can slander me soon afterwards.} \v{40}\red{Whoever is not against us is for us.} \v{41}\red{I tell all of you}\fnote{\fbackref{9:41} The Gk. pronoun \fbib{you} is pl.} \red{with certainty, whoever gives you a cup of water to drink because you belong to the Messiah}\fnote{\fbackref{9:41} Or \fbib{Christ}} \red{will never lose his reward.''}
\passage{Causing Others to Sin}
\passageinfo{(Matthew 18:6-9; Luke 17:1-2)}

\v{42}\red{``If anyone causes one of these little ones who believe in me to sin, it would be better for him if a large millstone were hung around his neck and he were thrown into the sea.} \v{43}\red{So if your hand causes you to sin, cut it off. It}\red{'}\red{s better for you to enter life injured than to have two hands and go to hell,}\fnote{\fbackref{9:43} Lit. \fbib{Gehenna}; a Gk. transliteration of the Heb. for \fbib{Valley of Hinnom}} \red{to the fire that cannot be put out.} \v{44}\red{In that place, worms never die, and the fire is never put out.}\fnote{\fbackref{9:44} Other mss. lack this verse}

\v{45}\red{And if your foot causes you to sin, cut it off. It}\red{'}\red{s better for you to enter life crippled than to have two feet and be thrown into hell.}\fnote{\fbackref{9:45} Lit. \fbib{Gehenna}; a Gk. transliteration of the Heb. for \fbib{Valley of Hinnom}} \v{46}\red{In that place, worms never die, and the fire is never put out.}\fnote{\fbackref{9:46} Other mss. lack this verse}

\v{47}\red{And if your eye causes you to sin, tear it out. It}\red{'}\red{s better for you to enter the kingdom of God with one eye than to have two eyes and be thrown into hell.}\fnote{\fbackref{9:47} Lit. \fbib{Gehenna}; a Gk. transliteration of the Heb. for \fbib{Valley of Hinnom}} \v{48}\red{In that place, worms never die, and the fire is never put out.}

\v{49}\red{Because everyone will be salted with fire, and every sacrifice will be salted with salt.}\fnote{\fbackref{9:49} Other mss. lack \fbib{and every sacrifice will be salted with salt}} \v{50}\red{Salt is good. But if salt loses its taste, how can you restore its flavor? Keep on having salt among yourselves, and live in peace with one another.''}
\labelchapt{10}
\passage{Teaching about Divorce}
\passageinfo{(Matthew 19:1-12)}

\chapt{10}
\v{1}Then Jesus\fnote{\fbackref{10:1} Lit. \fbib{he}} left that place and went into the territory of Judea on the other side\fnote{\fbackref{10:1} I.e. the east side} of the Jordan. Crowds gathered around him as usual, and he began to teach them again as was his custom. \v{2}Some Pharisees came to test him. They asked, ``Is it lawful for a man to divorce his wife?''

\v{3}\red{``Wha}\red{t}\red{ did Moses command you?''} he responded.

\v{4}They said, ``Moses allowed a man to write a certificate of divorce and to divorce her.''\fnote{\fbackref{10:4} Cf. Deut 24:1, 3}

\v{5}But Jesus told them, \red{``It was because of your hardness of heart that he wrote this command for you.} \v{6}\red{But from the beginning of creation, `God}\fnote{\fbackref{10:6} Other mss. read \fbib{He}} \red{made them male and female.'}\fnote{\fbackref{10:6} Cf. Gen 1:27; 5:2} \v{7}\red{T}\red{hat}\red{'}\red{s why }\red{`}\red{a man will leave his father and mother and be united with his wife,} \v{8}\red{and the two will become one flesh.'}\fnote{\fbackref{10:8} Cf. Gen 2:24} \red{So they}\red{'}\red{re no longer two, but one flesh. }\v{9}\red{Therefore, what God has joined together, man must never separate.''}

\v{10}Back in the house, the disciples asked him about this again. \v{11}So he told them, \red{``Whoever divorces his wife and marries another woman commits adultery against her.} \v{12}\red{And} \red{if a woman}\fnote{\fbackref{10:12} Lit. \fbib{she}} \red{divorces her husband and marries another man, she commits adultery.''}
\passage{Jesus Blesses the Little Children}
\passageinfo{(Matthew 19:13-15; Luke 18:15-17)}

\v{13}Some people\fnote{\fbackref{10:13} Lit. \fbib{They}} were bringing little children to Jesus\fnote{\fbackref{10:13} Lit. \fbib{him}} to have him touch them. But the disciples rebuked those who brought\fnote{\fbackref{10:13} The Gk. lacks \fbib{those who brought}} them. \v{14}When Jesus saw this, he became furious and told them, \red{``Let the little children come to me, and stop keeping them away, because the kingdom of God belongs to people like these.} \v{15}\red{I tell all of you}\fnote{\fbackref{10:15} The Gk. pronoun \fbib{you} is pl.} \red{with certainty, whoever doesn't receive the }\red{kingdom}\red{ of }\red{God}\red{ as a little child will never enter it.''} \v{16}Then after he had hugged the children,\fnote{\fbackref{10:16} Lit. \fbib{hugged them}} he tenderly blessed them as he laid his hands on them.
\passage{A Rich Man Comes to Jesus}
\passageinfo{(Matthew 19:16-30; Luke 18:18-30)}

\v{17}As Jesus\fnote{\fbackref{10:17} Lit. \fbib{he}} was setting out again,\fnote{\fbackref{10:17} Lit. \fbib{out on a journey}} a man ran up to him, knelt down in front of him, and asked him, ``Good Teacher, what must I do to inherit eternal life?''

\v{18}\red{``Why do you call me good?'' }Jesus asked him. \red{``Nobody is good except for one---God.} \v{19}\red{You know the commandments: `Never murder.'}\fnote{\fbackref{10:19} Cf. Exod 20:13; Deut 5:17} \red{`Never commit adultery.'}\fnote{\fbackref{10:19} Cf. Exod 20:14; Deut 5:18} \red{`Never steal.'}\fnote{\fbackref{10:19} Cf. Exod 20:15; Deut 5:19} \red{`Never give false testimony.'}\fnote{\fbackref{10:19} Cf. Exod 20:16; Deut 5:20} \red{`Never cheat.' `Honor your father and mother.'\,''}\fnote{\fbackref{10:19} Cf. Exod 20:12; Deut 5:16}

\v{20}The man\fnote{\fbackref{10:20} Lit. \fbib{He}} replied to him, ``Teacher, I have obeyed all of these since I was a young man.''

\v{21}Jesus looked at him and loved him. Then he told him, \red{``You're missing one thing. Go and sell everything you own, give the money}\fnote{\fbackref{10:21} The Gk. lacks \fbib{the money}} \red{to the destitute, and you will have treasure in heaven. Then come back and follow me.''} \v{22}Shocked at this statement, the man\fnote{\fbackref{10:22} Lit. \fbib{he}} went away sad, because he had many possessions.
\passage{Salvation and Reward}
\passageinfo{(Matthew 19:23-26; Luke 18:24-30)}

\v{23}Then Jesus looked around and told his disciples, \red{``How hard it will be for those who are wealthy to enter the }\red{kingdom}\red{ of }\red{God}\red{!''} \v{24}The disciples were startled by these words, but Jesus told them again, \red{``Children, how hard it is for those who trust in their wealth}\fnote{\fbackref{10:24} Other mss. lack \fbib{for those who trust in their wealth}} \red{to get into the }\red{kingdom}\red{ of }\red{God}\red{!} \v{25}\red{It}\red{'}\red{s easier for a camel to squeeze through the eye of a needle than for a rich person to get into the kingdom of God.''}

\v{26}The disciples\fnote{\fbackref{10:26} Lit. \fbib{They}} were utterly amazed and asked one another,\fnote{\fbackref{10:26} Other mss. read \fbib{to him}} ``Then who can be saved?''

\v{27}Jesus looked at them intently and said, \red{``For humans it}\red{'}\red{s impossible, but not for God. All things are possible for God.''}

\v{28}Then Peter began to say to him, ``See, we have left everything and followed you.'' \v{29}Jesus said, \red{``I tell all of you}\fnote{\fbackref{10:29} The Gk. pronoun \fbib{you} is pl.} \red{with certainty, there is no one who has left his home, brothers, sisters, mother, father, children, or fields because of me and the gospel} \v{30}\red{who will not receive a hundred times as much here in this world---homes, brothers, sisters, mothers, children, and fields, along with persecution---as well as eternal life in the age to come.} \v{31}\red{But many who are first will be last, and the last will be first.''}

\v{32}Now Jesus and his disciples\fnote{\fbackref{10:32} Lit. \fbib{They}} had been on the road going up to Jerusalem, with Jesus walking ahead of them. They were astonished, and the others who followed were afraid.
\passage{Jesus Predicts His Death and Resurrection a Third Time}
\passageinfo{(Matthew 20:17-19; Luke 18:31-34)}

Once again, Jesus\fnote{\fbackref{10:32} Lit. \fbib{he}} took the Twelve aside and began to tell them what was going to happen to him. \v{33}\red{``}\red{Pay attention! W}\red{e}\red{'}\red{re going up to Jerusalem. The Son of Man will be handed over to the high priests and the scribes, and they}\red{'}\red{ll condemn him to death. Then they}\red{'}\red{ll hand him over to the }\red{unbeliever}\red{s}\red{,}\fnote{\fbackref{10:33} Lit. \fbib{gentiles} ; i.e. unbelieving non-Jews} \v{34}\red{and} \red{they}\red{'}\red{ll make fun of him, spit on him, whip him, and kill him. But after three days he}\red{'}\red{ll be raised.''}
\passage{The Request of James and John}
\passageinfo{(Matthew 20:20-28)}

\v{35}James and John, the sons of Zebedee, went to Jesus\fnote{\fbackref{10:35} Lit. \fbib{him}} and told him, ``Teacher, we want you to do for us whatever we ask you.''

\v{36}He asked them, \red{``What do you want me to do for you?''}

\v{37}They asked him, ``Let us sit in your glory, one on your right and one on your left.''

\v{38}But Jesus told them, \red{``You don't realize what you're asking. Can you drink from the cup that I'm going to drink from or be baptized with the baptism with which I'm going to be baptized?''}

\v{39}They told him, ``We can.''

Jesus told them,\red{ ``You will drink from the cup that I'm going to drink and be baptized with the baptism with which I'm going to be baptized.} \v{40}\red{But it's not up to me to grant you a seat at my right or my left. Those positions have already been prepared for others.''}

\v{41}When the ten other disciples\fnote{\fbackref{10:41} The Gk. lacks \fbib{other disciples}} heard this, they began to be furious with James and John. \v{42}Jesus called his disciples\fnote{\fbackref{10:42} Lit. \fbib{called them}} and told them, \red{``You know that those who are recognized as rulers among the }\red{unbeliever}\red{s}\fnote{\fbackref{10:42} Lit. \fbib{gentiles} ; i.e. unbelieving non-Jews}\red{ lord it over them, and their superiors act like tyrants over them.} \v{43}\red{That's not the way it should be among you. Instead, whoever wants to become great among you must be your servant,} \v{44}\red{and whoever wants to be first among you must be a slave to everyone,} \v{45}\red{because even the Son of Man did not come to be served, but to serve and to give his life as a ransom for many people.''}
\passage{Jesus Heals Blind Bartimaeus}
\passageinfo{(Matthew 20:29-34; Luke 18:35-43)}

\v{46}Then they came to Jericho. As Jesus,\fnote{\fbackref{10:46} Lit. \fbib{he}} his disciples, and a large crowd were leaving Jericho, a blind beggar named Bartimaeus (that is, the son of Timaeus) was sitting by the road. \v{47}When he heard that Jesus of Nazareth was there, he began to shout, ``Jesus, Son of David, have mercy on me!'' \v{48}Many people sternly told him to be quiet, but he started shouting even louder, ``Son of David, have mercy on me!''

\v{49}So Jesus stopped and said, \red{``Call him!''}

So they called the blind man and told him, ``Have courage! Get up. He's calling you.'' \v{50}He threw off his coat, jumped up, and went to Jesus.

\v{51}Then Jesus asked him, \red{``What do you want me to do for you?''}

The blind man told him, ``Rabbouni,\fnote{\fbackref{10:51} \fbib{Rabbouni} is Heb. for \fbib{My Master} and/or \fbib{Teacher}} I want to see again.''

\v{52}Jesus told him, \red{``Go. Your faith has made you well.''} At once the man\fnote{\fbackref{10:52} Lit. \fbib{he}} could see again, and he began to follow Jesus\fnote{\fbackref{10:52} Lit. \fbib{him}} down the road.
\labelchapt{11}
\passage{The King Enters Jerusalem}
\passageinfo{(Matthew 21:1-11; Luke 19:28-40; John 12:12-19)}

\chapt{11}
\v{1}When they came near Jerusalem, at Bethphage and Bethany, near the Mount of Olives, Jesus\fnote{\fbackref{11:1} Lit. \fbib{he}} sent two of his disciples on ahead \v{2}and told them, \red{``Go into the village ahead of you. As soon as you go into it, you will find a colt tied up that no one has ever ridden.} \red{Untie it, and bring it along.} \v{3}\red{If anyone asks you, `Why are you doing this?,' say, `The Lord needs it,' and he will send it back here at once.''}

\v{4}So they went and found the colt outside in the street tied up next to a doorway. While they were untying it, \v{5}some men standing there asked them, ``What are you doing untying that colt?'' \v{6}The disciples\fnote{\fbackref{11:6} Lit. \fbib{They}} told them what Jesus had said, and the men\fnote{\fbackref{11:6} Lit. \fbib{they}} let them go.

\v{7}They brought the colt to Jesus, threw their coats on it, and he sat on it. \v{8}Many people spread their coats on the road, while others spread leafy branches that they had cut in the fields. \v{9}Those who went ahead and those who followed him were shouting,

\begin{poetry}
\poeml ``Hosanna!\fnote{\fbackref{11:9} \fbib{Hosanna} is Heb. for \fbib{Please save} or \fbib{Praise.}} \\
\poeml How blessed is the one who comes \\
\poemll    in the name of the Lord!\fnote{\fbackref{11:9} MT source citation reads \fbib{}\divine{Lord}} \\
\poeml \v{10}How blessed is the coming kingdom\fnote{\fbackref{11:10} Cf. Ps 148:1} \\
\poemll    of our ancestor David! \\
\poeml Hosanna in the highest heaven!''\fnote{\fbackref{11:10} Cf. Ps 118:25-26; Ps 148:1}
\end{poetry}

\v{11}Then Jesus went into Jerusalem and into the Temple and looked around at everything. Since it was already late, he went out with the Twelve to Bethany.
\passage{Jesus Curses a Fig Tree}
\passageinfo{(Matthew 21:18-19)}

\v{12}The next day, as they were leaving Bethany, Jesus\fnote{\fbackref{11:12} Lit. \fbib{he}} became hungry. \v{13}Seeing in the distance a fig tree covered with leaves, he went to see if he could find anything on it. When he came to it, he found nothing except leaves because it wasn't the season for figs. \v{14}So he told it, \red{``May no one ever eat fruit from you again!''} Now his disciples were listening to this.
\passage{Confrontation in the Temple over Money}
\passageinfo{(Matthew 21:12-17; Luke 19:45-48; John 2:13-22)}

\v{15}When they came to Jerusalem, he went into the Temple and began to throw out those who were selling and those who were buying in the Temple. He overturned the moneychangers' tables and the chairs of those who sold doves. \v{16}He wouldn't even let anyone carry a vessel through the Temple. \v{17}Then he began to teach them: \red{``It is written, is it not, `My house is to be called a house of prayer for all nations'?}\fnote{\fbackref{11:17} Cf. Isa 56:7; Jer 7:11} \red{But you}\red{'}\red{ve turned it into a hideout}\fnote{\fbackref{11:17} Lit. \fbib{cave}} \red{for bandits!''} \v{18}When the high priests and elders heard this, they began to look for a way to kill him, because they were afraid of him, since the whole crowd was amazed at his teaching. \v{19}When evening came, Jesus and his disciples\fnote{\fbackref{11:19} Lit. \fbib{came, they}} would leave the city.
\passage{The Lesson from the Dried Fig Tree}
\passageinfo{(Matthew 21:20-22)}

\v{20}While they were walking along early the next morning, they saw the fig tree dried up to its roots. \v{21}Remembering what Jesus had said,\fnote{\fbackref{11:21} The Gk. lacks \fbib{what Jesus had said}} Peter pointed out to him, ``Rabbi,\fnote{\fbackref{11:21} \fbib{Rabbi} is Heb. for \fbib{Master} and/or \fbib{Teacher}} look! The fig tree you cursed has dried up!''

\v{22}Jesus told his disciples,\fnote{\fbackref{11:22} Lit. \fbib{to them}} \red{``Have faith in God!} \v{23}\red{I tell all of you}\fnote{\fbackref{11:23} The Gk. pronoun \fbib{you} is pl.} \red{with certainty, if anyone says to this mountain, `Be lifted up and thrown into the sea,' if he doesn't doubt in his heart but believes that what he says will happen, it will be done for him.} \v{24}\red{That} \red{is why I tell you, whatever you ask for in prayer, believe that you have received}\fnote{\fbackref{11:24} Other mss. read \fbib{are receiving}; still other mss. read \fbib{will receive}} \red{it and it will be yours.}

\v{25}\red{``Whenever you stand up to pray, forgive whatever you have against anyone, so that your Father in heaven will forgive your sins.} \v{26}\red{But if you do not forgive, your Father in heaven will not forgive your sins.''}\fnote{\fbackref{11:26} Other mss. lack this verse}
\passage{Jesus' Authority is Challenged}
\passageinfo{(Matthew 21:23-27; Luke 20:1-8)}

\v{27}Then they went into Jerusalem again. While Jesus\fnote{\fbackref{11:27} Lit. \fbib{he}} was walking in the Temple, the high priests, the scribes, and the elders came to him \v{28}and asked him, ``By what authority are you doing these things, and who gave you this authority to do them?''

\v{29}Jesus told them, \red{``I}\red{'}\red{ll ask you one question.}\fnote{\fbackref{11:29} Lit. \fbib{one word}} \red{Answer me, and then I}\red{'}\red{ll tell you by what authority I}\red{'}\red{m doing these things.} \v{30}\red{Was} \red{John's authority to baptize}\fnote{\fbackref{11:30} Lit. \fbib{John's baptism}} \red{from heaven or from humans? Answer me.''}

\v{31}They began discussing this among themselves. ``If we say, `From heaven,' he'll say, `Then why didn't you believe him?' \v{32}But if we say, `From humans'{\ldots}?'' They were afraid of the crowd, because everyone really thought John was a prophet.

\v{33}So they answered Jesus, ``We don't know.''

Then Jesus told them, \red{``Then I won't tell you by what authority I}\red{'}\red{m doing these things.''}
\labelchapt{12}
\passage{The Parable about the Tenant Farmers}
\passageinfo{(Matthew 21:33-46; Luke 20:9-19)}

\chapt{12}
\v{1}Then Jesus\fnote{\fbackref{12:1} Lit. \fbib{he}} began to speak to them in parables. \red{``A man planted a vineyard. He put a wall around it, dug a pit for the wine press, and built a }\red{watchtower. Then he leased it to tenant farmers and went abroad.} \v{2}\red{At the right time}\red{,}\red{ he sent a servant to the farmers to collect from them a share of the produce from the vineyard. }\v{3}\red{But the farmers}\fnote{\fbackref{12:3} Lit. \fbib{they}} \red{grabbed the servant,}\fnote{\fbackref{12:3} Lit. \fbib{him}} \red{beat him, and sent him back empty-handed.} \v{4}\red{Again, the man}\fnote{\fbackref{12:4} Lit. \fbib{he}} \red{sent another servant to them. They beat the servant}\fnote{\fbackref{12:4} Lit. \fbib{him}} \red{over the head and treated him shamefully. }\v{5}\red{Then the man}\fnote{\fbackref{12:5} Lit. \fbib{he}} \red{sent another, and that one they killed. So it was with many other servants.}\fnote{\fbackref{12:5} Lit. \fbib{with many others}} \red{Some of these they beat, and others they killed.} \v{6}\red{He} \red{still had one more person to send,}\fnote{\fbackref{12:6} The Gk. lacks \fbib{more person to send}} \red{a son whom he loved. Finally, he sent him to them, saying, `They will respect my son.'} \v{7}\red{But those farmers told one another, `This is the heir. Come on, let's kill him, and the inheritance will be ours!' }\v{8}\red{So they grabbed him, killed him, and threw him out of the vineyard.}

\v{9}\red{``Now what will the owner of the vineyard do? He will come, execute the farmers, and give the vineyard to others. }\v{10}\red{Haven't you ever read this Scripture:}

\begin{poetry}
\poeml \red{`The stone that the builders rejected} \\
\poemll    \red{has become the cornerstone.}\fnote{\fbackref{12:10} Or \fbib{capstone}} \\
\poeml \v{11}\red{This was the Lord's}\fnote{\fbackref{12:11} MT source citation reads \fbib{\divine{Lord}'s}}\red{ doing,} \\
\poemll    \red{and it is amazing in our eyes'?''}\fnote{\fbackref{12:11} Cf. Ps 118:22-23}
\end{poetry}

\v{12}They were trying to arrest him but were afraid of the crowd. Realizing that he had spoken this parable against them, they left him alone and went away.
\passage{A Question about Paying Taxes}
\passageinfo{(Matthew 22:15-22; Luke 20:20-26)}

\v{13}Then they sent some Pharisees and some Herodians\fnote{\fbackref{12:13} I.e. Royal party sympathizers} to him, intending to trap him in what he said. \v{14}They came and told him, ``Teacher, we know that you are sincere. You don't favor any individual, because you pay no attention to external appearance. Instead, you teach the way of God truthfully. Is it lawful to pay taxes to Caesar or not? Should we pay them or shouldn't we?''

\v{15}Seeing through their hypocrisy, Jesus\fnote{\fbackref{12:15} Lit. \fbib{he}} replied to them, \red{``Why are you testing me? Bring me a denarius and let me look at it.''}

\v{16}So they brought one. Then he asked them, \red{``Whose face and name are on this?''}

They told him, ``Caesar's.''

\v{17}So Jesus told them, \red{``Give back to Caesar the things that are Caesar's, and to God the things that are God's.''} And they were utterly amazed at him.
\passage{A Question about the Resurrection}
\passageinfo{(Matthew 22:23-33; Luke 20:27-40)}

\v{18}Then some Sadducees, who claim there is no resurrection, came to Jesus\fnote{\fbackref{12:18} Lit. \fbib{him}} and asked him, \v{19}``Teacher, Moses wrote for us that if a man's brother dies and leaves a wife but no child, he should marry the widow and have children for his brother.\fnote{\fbackref{12:19} Cf. Deut 25:5-6} \v{20}There were seven brothers. The first one married and died without having children. \v{21}Then the second married her and died without having children, and so did the third. \v{22}None of the seven left any children. Last of all, the woman died, too. \v{23}In the resurrection, whose wife will she be, since all seven had married her?''\fnote{\fbackref{12:23} Lit. \fbib{had her as wife}}

\v{24}Jesus answered them, \red{``Aren't you mistaken because you don't know the Scriptures or God's power?} \v{25}\red{When people}\fnote{\fbackref{12:25} Lit. \fbib{they}} \red{rise from the dead, they neither marry nor are given in marriage but are like the angels in heaven.} \v{26}\red{As for the dead being raised, haven't you read in the book of Moses, in the story about the bush, how God said, `I am the God of Abraham, the God of Isaac, and the God of Jacob'?}\fnote{\fbackref{12:26} Cf. Exod 3:6} \v{27}\red{He} \red{is not the God of the dead, but of the living. You are badly mistaken!''}
\passage{The Greatest Commandment}
\passageinfo{(Matthew 22:34-40; Luke 10:25-28)}

\v{28}Then one of the scribes came near and heard the Sadducees\fnote{\fbackref{12:28} Lit. \fbib{heard them}} arguing with one another. He saw how well Jesus\fnote{\fbackref{12:28} Lit. \fbib{he}} answered them, so he asked him, ``Which commandment is the most important of them all?''

\v{29}Jesus answered, \red{``The most important is, `Hear, O Israel, the Lord}\fnote{\fbackref{12:29} MT source citation reads \fbib{}\divine{Lord}}\red{ our God is one Lord,}\fnote{\fbackref{12:30} MT source citation reads \fbib{}\divine{Lord}} \v{30}\red{and you must love the Lord}\fnote{\fbackref{12:30} MT source citation reads \fbib{}\divine{Lord}}\red{ your God with all your heart, with all your soul, with all your mind, and with all your strength.'}\fnote{\fbackref{12:30} Cf. Deut 6:4-5} \v{31}\red{The }\red{second is this: `You must love your neighbor as yourself.'}\fnote{\fbackref{12:31} Cf. Lev 19:18} \red{No other commandment is greater than these.''}

\v{32}Then the scribe told him, ``Well said,\fnote{\fbackref{12:32} The Gk. lacks \fbib{said}} Teacher! You have told the truth that `God\fnote{\fbackref{12:32} Lit. \fbib{he}} is one, and there is no other besides him.'\fnote{\fbackref{12:32} Cf. Deut 6:4} \v{33}To love him with all your heart, with all your understanding, and with all your strength, and to love your neighbor as yourself is more important than all the burnt offerings and sacrifices.''

\v{34}When Jesus saw how wisely the man\fnote{\fbackref{12:34} Lit. \fbib{he}} answered, he told him, \red{``You are not far from the }\red{kingdom}\red{ of }\red{God}\red{.''} After that, no one dared to ask him another question.
\passage{A Question about David's Son}
\passageinfo{(Matthew 22:41-46; Luke 20:41-44)}

\v{35}While Jesus was teaching in the Temple, he asked, \red{``How can the scribes say that the Messiah}\fnote{\fbackref{12:35} Or \fbib{Christ}} \red{is David's son?} \v{36}\red{David himself said by the Holy Spirit,}

\begin{poetry}
\poeml \red{`The Lord}\fnote{\fbackref{12:36} MT source citation reads \fbib{}\divine{Lord}}\red{ told my Lord,} \\
\poemll    \red{``Sit at my right hand,} \\
\poemlll       \red{until I put your enemies under your feet.''\,'}\fnote{\fbackref{12:36} Cf. Ps 110:1; other mss read \fbib{until I make your enemies a footstool for your feet.''\,'}}
\end{poetry}

\v{37}\red{David himself calls him `Lord,' so how can he be his son?''} And the large crowd kept listening to him with delight.
\passage{Jesus Denounces the Scribes}
\passageinfo{(Matthew 23:1-36; Luke 20:45-47)}

\v{38}As he taught, he said, \red{``Beware of the scribes! They like to walk around in long robes, to be greeted in the marketplaces,} \v{39}\red{and to have the best seats in the synagogues and the places of honor at banquets.} \v{40}\red{They} \red{devour widows' houses}\fnote{\fbackref{12:40} I.e. rob widows by taking their houses} \red{and say long prayers to cover it up. They will receive greater condemnation!''}
\passage{The Widow's Offering}
\passageinfo{(Luke 21:1-4)}

\v{41}As Jesus\fnote{\fbackref{12:41} Lit. \fbib{he}} sat facing the offering box, he watched how the crowd was dropping their money into it.\fnote{\fbackref{12:41} Lit. \fbib{into the offering box}} Many rich people were dropping in large amounts. \v{42}Then a destitute widow came and dropped in two small copper coins,\fnote{\fbackref{12:42} Lit. \fbib{two lepta}, the smallest coin denominated in their economy} worth about a cent.\fnote{\fbackref{12:42} Lit. \fbib{quadrans}, worth {\textonequarter} of the Roman assarion coin, or about 1/10 of a day's wage} \v{43}He called his disciples and told them, \red{``I tell all of you}\fnote{\fbackref{12:43} The Gk. pronoun \fbib{you} is pl.} \red{with certainty, this destitute widow has dropped in more than everyone who is contributing to the offering box,} \v{44}\red{because all of them contributed out of their surplus, but out of her poverty she has given everything she had to live on.''}
\labelchapt{13}
\passage{Jesus Predicts the Destruction of the Temple}
\passageinfo{(Matthew 24:1-2; Luke 21:5-6)}

\chapt{13}
\v{1}As Jesus\fnote{\fbackref{13:1} Lit. \fbib{he}} was leaving the Temple, one of his disciples told him, ``Look, Teacher, what large stones and what beautiful buildings!''

\v{2}\red{``Do you see these large buildings?'' }Jesus responded. \red{``Not one stone here will be left on another that will not be torn down.''}
\passage{Cults, Revolutions, Famines, and Earthquakes}
\passageinfo{(Matthew 24:3-14; Luke 21:7-19)}

\v{3}As Jesus\fnote{\fbackref{13:3} Lit. \fbib{he}} was sitting on the Mount of Olives facing the Temple, Peter, James, John, and Andrew were asking him privately, \v{4}``Tell us, when will these things happen, and what will be the sign when these things will be put into effect?''

\v{5}Jesus began to say to them, \red{``See to it that no one deceives you.} \v{6}\red{Many will come in my name and say, `I AM,' and they will deceive many people.}\fnote{\fbackref{13:6} The Gk. lacks \fbib{people}} \v{7}\red{But when you hear of wars and rumors of wars, stop being alarmed. These things must take place, but the end hasn't come yet,} \v{8}\red{because nation will rise up in arms against nation, and kingdom against kingdom. There will be earthquakes and famines in various places. These things are only the beginning of the birth pains.''}
\passage{Future Persecution and Evangelism}
\passageinfo{(Matthew 10:16-25; Matthew 24:9-14; Luke 21:12-19)}

\v{9}\red{``As for yourselves, be on your guard! People}\fnote{\fbackref{13:9} Lit. \fbib{They}} \red{will hand you over to local councils, and you will be beaten in their synagogues. You will stand before governors and kings to testify to them because of me.} \v{10}\red{But first, the gospel must be proclaimed to all nations.}\fnote{\fbackref{13:10} Or \fbib{gentiles}} \v{11}\red{When they take you away and hand you over for trial, don't worry ahead of time about what you will say. Instead, say whatever is given to you at that time, because it won't be you speaking, but the Holy Spirit.} \v{12}\red{Brother will betray brother to death, and a father his child. }\red{Children will rebel against their parents and have them put to death.} \v{13}\red{You will be hated continuously by everyone because of my name. But the person who endures to the end will be saved.''}
\passage{Signs of the End}
\passageinfo{(Matthew 24:15-28; Luke 21:20-24)}

\v{14}\red{``So when you see the destructive desecration standing where it should not be} (let the reader take note)\red{,}\fnote{\fbackref{13:14} Cf. Dan 9:27; 11:31; 12:11} \red{then those who are in }\red{Judea}\red{ must flee to the mountains.} \v{15}\red{The one who}\red{'}\red{s on his housetop must not come down and go into his house to take anything out of it,} \v{16}\red{and the one who}\red{'}\red{s in the field must not turn back to get his coat.}

\v{17}\red{``How terrible it will be for women who are pregnant or who are nursing babies in those days!} \v{18}\red{Pray that it may not be in winter,} \v{19}\red{because those days will be a time of suffering,}\fnote{\fbackref{13:19} Or \fbib{tribulation}} \red{a kind that has not happened from the beginning of God's creation until now, and certainly will never happen again.} \v{20}\red{If the Lord had not shortened those days, no one would be saved. But for the sake of the elect whom he has chosen, he has shortened those days.}

\v{21}\red{``At that time, if anyone says to you, `Look! Here is the Messiah!',}\fnote{\fbackref{13:21} Or \fbib{Christ}} \red{or, `Look! There he is!' don't believe it,} \v{22}\red{because false messiahs}\fnote{\fbackref{13:22} Or \fbib{christs}}\red{ and false prophets will appear and produce signs and omens to deceive, if possible, the elect.} \v{23}\red{So be on your guard! I've told you everything before it happens.''}
\passage{The Coming of the Son of Man}
\passageinfo{(Matthew 24:29-31; Luke 21:25-28)}

\v{24}\red{``But after the suffering of those days,}

\begin{poetry}
\poeml \red{`The sun will be darkened,} \\
\poemll    \red{the moon will not give its light,} \\
\poeml \v{25}\red{the stars will fall from the sky,} \\
\poemll    \red{and the powers of heaven will be shaken loose.'}\fnote{\fbackref{13:25} Cf. Isa 13:10; 34:4; Joel 2:10}
\end{poetry}

\v{26}\red{Then people}\fnote{\fbackref{13:26} Lit. \fbib{they}}\red{ will see `the Son of Man coming in clouds'}\fnote{\fbackref{13:26} Cf. Dan 7:13}\red{ with great power and glory.} \v{27}\red{He}\red{'}\red{ll send out his angels and gather his elect from the four winds, from the ends of the earth to the ends of heaven.''}
\passage{The Lesson from the Fig Tree}
\passageinfo{(Matthew 24:32-35; Luke 21:29-33)}

\v{28}\red{``Now learn a lesson}\fnote{\fbackref{13:28} Or \fbib{parable}} \red{from the fig tree. When its branches become tender and it produces leaves, you know that summer is near.} \v{29}\red{In the same way, }\red{when you see these things taking place, you will know that the Son of Man}\fnote{\fbackref{13:29} Lit. \fbib{that he}}\red{ is near, right at the door.} \v{30}\red{I tell all of you}\fnote{\fbackref{13:30} The Gk. pronoun \fbib{you} is pl.}\red{ with certainty, this generation will not disappear until all these things take place.} \v{31}\red{Heaven and earth will disappear, but my words will never disappear.''}
\passage{The Unknown Day and Hour of the Messiah's Return}
\passageinfo{(Matthew 24:36-44)}

\v{32}\red{``No one knows when that day or hour will come}\fnote{\fbackref{13:32} Lit. \fbib{about that day and hour}}\red{---not the angels in heaven, not the Son, but only the Father.} \v{33}\red{Be careful! Watch out! Because you don't know when the time will come. }\v{34}\red{It's like a man who went on a trip. As he left home, he put his servants in charge, each with his own work, and he ordered the doorkeeper to be alert.} \v{35}\red{So keep on watching, because you don't know when the master of the house is coming---whether in the evening, at three o'clock in the morning,}\fnote{\fbackref{13:35} Lit. \fbib{at cock crow}} \red{or at dawn.} \v{36}\red{Otherwise, he may come suddenly and find you asleep.} \v{37}\red{I'm telling you what I'm telling everyone: Be alert!''}
\labelchapt{14}
\passage{The Plot to Kill Jesus}
\passageinfo{(Matthew 26:1-5; Luke 22:1-2; John 11:45-53)}

\chapt{14}
\v{1}Now it was two days before the Passover and the Festival of Unleavened Bread. The high priests and the scribes were looking for a way to arrest Jesus secretly and to have him put to death, \v{2}because they kept saying, ``This must not happen during the festival. Otherwise, there'll be a riot among the people.''
\passage{A Woman Anoints Jesus}
\passageinfo{(Matthew 26:6-13; John 12:1-8)}

\v{3}While Jesus\fnote{\fbackref{14:3} Lit. \fbib{he}} was in Bethany sitting at the table in the home of Simon the leper, a woman arrived with an alabaster jar of very expensive perfume made from pure nard. She broke open the jar and poured the perfume on his head. \v{4}Irritated, some who were there asked one another, ``Why was the perfume wasted like this? \v{5}This perfume could have been sold for more than 300 denarii\fnote{\fbackref{14:5} A denarius was the average day's wage for a laborer.} and the money\fnote{\fbackref{14:5} The Gk. lacks \fbib{the money}} given to the destitute.'' So they got extremely angry with her.

\v{6}But Jesus said, \red{``Leave her alone. Why are you bothering her? She has done a beautiful thing for me,} \v{7}\red{because you}\red{'}\red{ll always have the destitute with you and can help them whenever you want, but you w}\red{on't }\red{always have me.} \v{8}\red{She has done what she could. She poured perfume on my body in preparation for my }\red{burial.} \v{9}\red{I tell all of you}\fnote{\fbackref{14:9} The Gk. pronoun \fbib{you} is pl.}\red{ with certainty, wherever the gospel is proclaimed in the whole world, what she has done will also be told as a memorial to her.''}
\passage{Judas Agrees to Betray Jesus}
\passageinfo{(Matthew 26:14-16; Luke 22:3-6)}

\v{10}Then Judas Iscariot, one of the Twelve, went to the high priests to betray Jesus\fnote{\fbackref{14:10} Lit. \fbib{him}} to them. \v{11}After they had listened to him,\fnote{\fbackref{14:11} The Gk. lacks \fbib{to him}} they were delighted and promised to give him money. So he began to look for a good opportunity to betray him.
\passage{The Passover with the Disciples}
\passageinfo{(Matthew 26:17-25; Luke 22:7-14, 21-23; John 13:21-30)}

\v{12}On the first day of the Festival\fnote{\fbackref{14:12} The Gk. lacks \fbib{of the Festival}} of Unleavened Bread, when the Passover lamb is sacrificed, Jesus'\fnote{\fbackref{14:12} Lit. \fbib{his}} disciples asked him, ``Where do you want us to go and make preparations for you to eat the Passover meal?''

\v{13}He sent two of his disciples, telling them, \red{``Go into the city, and you will meet a man carrying a jug of water. Follow him.} \v{14}\red{When he goes into a house,}\fnote{\fbackref{14:14} Lit. \fbib{Wherever he enters}} \red{say to its owner that the Teacher asks, `Where is my room where I can eat the Passover meal with my disciples?'} \v{15}\red{Then he will show you a large upstairs room that is furnished and ready. Get everything ready for us there.'' }\v{16}So the disciples left and went into the city. They found everything just as Jesus\fnote{\fbackref{14:16} Lit. \fbib{he}} had told them, and they prepared the Passover meal.

\v{17}When evening came, Jesus\fnote{\fbackref{14:17} Lit. \fbib{he}} arrived with the Twelve. \v{18}While they were at the table eating, Jesus said, \red{``I tell all of you}\fnote{\fbackref{14:18} The Gk. pronoun \fbib{you} is pl.}\red{ with certainty, one of you is going to betray me, one who is eating with me.''}

\v{19}They began to be very sad and asked him, one after the other, ``Surely I am not the one, am I?''

\v{20}He told them, \red{``It's one of you Twelve, the one who is dipping his bread into the bowl with me.} \v{21}\red{For the Son of Man is going away, just as it has been written about him, but how terrible it will be for that man by whom the Son of Man is betrayed! It would have been better for him if he had never been born.''}
\passage{The Lord's Supper}
\passageinfo{(Matthew 26:26-30; Luke 22:15-20)}

\v{22}While they were eating, Jesus\fnote{\fbackref{14:22} Lit. \fbib{he}} took a loaf of bread and blessed it. Then he broke it in pieces and handed it to them, saying, \red{``Take some. This is my }\red{body.''} \v{23}Then he took a cup, gave thanks, and gave it to them, and they all drank from it. \v{24}He told them, \red{``This is my blood of the covenant that is being poured out for many people.} \v{25}\red{I tell all of you}\fnote{\fbackref{14:25} The Gk. pronoun \fbib{you} is pl.}\red{ with certainty, I}\red{'}\red{ll never again drink the product of the vine until that day when I drink it new in the kingdom of God.''}

\v{26}After singing a hymn, they went out to the Mount of Olives.
\passage{Jesus Predicts Peter's Denial}
\passageinfo{(Matthew 26:31-35; Luke 22:31-34; John 13:36-38)}

\v{27}Then Jesus told them, \red{``All of you will turn against me, because it is written,}

\begin{poetry}
\poeml \red{`I will strike the shepherd,} \\
\poemll    \red{and the sheep will be scattered.'}\fnote{\fbackref{14:27} Cf. Zech 13:7}
\end{poetry}

\v{28}\red{However, after I}\red{'}\red{ve been raised, I}\red{'}\red{ll go to Galilee ahead of you.''}

\v{29}But Peter told him, ``Even if everyone else turns against you, I certainly won't.''

\v{30}Jesus told him, \red{``I tell you}\fnote{\fbackref{14:30} The Gk. pronoun \fbib{you} is sing.}\red{ with certainty, today, this very night, before a rooster crows twice, you}\red{'}\red{ll deny me three times.''}

\v{31}But Peter\fnote{\fbackref{14:31} Lit. \fbib{he}} kept saying emphatically, ``Even if I have to die with you, I'll never deny you!'' And all the others kept saying the same thing.
\passage{Jesus Prays in the Garden of Gethsemane}
\passageinfo{(Matthew 26:36-46; Luke 22:39-46)}

\v{32}Then they came to a place called Gethsemane, and he told his disciples, \red{``Sit down here while I pray.''} \v{33}He took Peter, James, and John along with him, and he began to feel distressed and troubled. \v{34}So he told them, \red{``I}\red{'}\red{m deeply grieved, even to the point of death. Wait here and stay awake.''}

\v{35}Going on a little farther, he fell to the ground and kept praying that if it were possible the hour might pass from him. \v{36}He kept repeating, \red{``Abba!}\fnote{\fbackref{14:36} \fbib{Abba} is Heb./Aram. for \fbib{Father}} \red{Father! All things are possible for you. Take this cup away from me. Yet not what I want but what you want.''}

\v{37}When he went back, he found his disciples\fnote{\fbackref{14:37} Lit. \fbib{found them}} asleep. \red{``Simon, are you asleep?'' }he asked Peter. \red{``You couldn't stay awake for one hour, could you?} \v{38}\red{All of you} \red{must stay awake and pray that you won't be tempted. The spirit is indeed willing, but the body}\fnote{\fbackref{14:38} Lit. \fbib{flesh}}\red{ is weak.''}

\v{39}He went away again and prayed the same prayer as before.\fnote{\fbackref{14:39} Lit. \fbib{the same word}} \v{40}Again he came back and found them asleep, because they could not keep their eyes open. They didn't even know what they should say to him.

\v{41}He came back a third time. \red{``Are you still sleeping and resting?''}\fnote{\fbackref{14:41} Or \fbib{You might as well keep on sleeping and resting.}} he asked. \red{``Enough of that! The time has come. Look! The Son of Man is being betrayed into the hands of sinners.} \v{42}\red{Get up! Let's go! See, the one who is betraying me is near!''}
\passage{Jesus is Arrested}
\passageinfo{(Matthew 26:47-56; Luke 22:47-53; John 18:3-12)}

\v{43}Just then, while Jesus\fnote{\fbackref{14:43} Lit. \fbib{he}} was still speaking, Judas, one of the Twelve, arrived. A crowd armed with swords and clubs was with him. They were from the high priests, the scribes, and the elders. \v{44}Now the betrayer personally had given them a signal, saying, ``The one I kiss\fnote{\fbackref{14:44} People customarily greeted their friends with a kiss.} is the man. Arrest him, and lead him safely away.'' \v{45}So Judas\fnote{\fbackref{14:45} Lit. \fbib{he}} immediately went up to Jesus\fnote{\fbackref{14:45} Lit. \fbib{him}} and said, ``Rabbi,''\fnote{\fbackref{14:45} Other mss. read \fbib{Rabbi, Rabbi.} The word is Heb. for \fbib{Master} and/or \fbib{Teacher}} and kissed him tenderly.

\v{46}Then the men\fnote{\fbackref{14:46} Lit. \fbib{they}} took hold of Jesus\fnote{\fbackref{14:46} Lit. \fbib{him}} and arrested him. \v{47}But one of those standing there drew his sword and struck the high priest's servant, cutting off his ear. \v{48}Jesus asked them, \red{``Have you come out with swords and clubs to arrest me as if I were a bandit?}\fnote{\fbackref{14:48} Or \fbib{revolutionary}} \v{49}\red{Day after day I was with you in the }\red{Temple}\red{ teaching, yet you didn't arrest me. But the Scriptures must be fulfilled.''} \v{50}Then all the disciples\fnote{\fbackref{14:50} Lit. \fbib{all of them}} deserted Jesus\fnote{\fbackref{14:50} Lit. \fbib{him}} and ran away.
\passage{The Young Man who Ran Away}

\v{51}A certain young man, who was wearing nothing but a linen sheet, was following Jesus.\fnote{\fbackref{14:51} Lit. \fbib{him}} When the men\fnote{\fbackref{14:51} Lit. \fbib{they}} grabbed him, \v{52}he left the linen sheet behind and ran away naked.
\passage{Jesus is Tried before the High Priest}
\passageinfo{(Matthew 26:57-68; Luke 22:54-55; John 18:13-14, 19-24)}

\v{53}Then they took Jesus to the high priest. All the high priests, elders, and scribes had gathered together. \v{54}Peter followed Jesus\fnote{\fbackref{14:54} Lit. \fbib{him}} at a distance as far as the high priest's courtyard. He was sitting with the servants and warming himself at the fire. \v{55}Meanwhile, the high priests and the whole Council\fnote{\fbackref{4:55} Or \fbib{Sanhedrin}} were looking for some testimony against Jesus in order to have him put to death, but they couldn't find any. \v{56}Although many people gave false testimony against him, their testimony didn't agree.

\v{57}Then some men stood up and gave false testimony against him, saying, \v{58}``We ourselves heard him say, \red{`I will destroy this sanctuary made by human}\fnote{\fbackref{4:58} The Gk. lacks \fbib{human}} \red{hands, and in three days I will build another one not made by human}\fnote{\fbackref{14:58} The Gk. lacks \fbib{human}} \red{hands.'}'' \v{59}But even on this point their testimony didn't agree.

\v{60}Then the high priest stood up before them\fnote{\fbackref{14:60} Lit. \fbib{in the middle}} and asked Jesus, ``Don't you have any answer to what these men are testifying against you?'' \v{61}But he kept silent and didn't answer at all. The high priest asked him again, ``Are you the Messiah,\fnote{\fbackref{14:61} Or \fbib{Christ}} the Son of the Blessed One?''

\v{62}Jesus said, \red{``I AM, and}

\begin{poetry}
\poeml \red{`you will see the Son of Man} \\
\poemll    \red{seated at the right hand of the Power'}\fnote{\fbackref{14:62} Cf. Ps 110:1} \\
\poemlll       \red{and `coming with the clouds of heaven.'\,''}\fnote{\fbackref{14:62} Cf. Dan 7:13}
\end{poetry}

\v{63}Then the high priest tore his clothes. ``Why do we still need witnesses?'' he asked. \v{64}``You have heard his blasphemy! What is your verdict?'' All of them condemned him as deserving death.

\v{65}Some of them began to spit on him. They blindfolded him and kept hitting him with their fists and telling him, ``Prophesy!'' Even the servants took him and slapped him around.
\passage{Peter Denies Jesus}
\passageinfo{(Matthew 26:69-75; Luke 22:56-62; John 18:15-18, 25-27)}

\v{66}While Peter was down in the courtyard, one of the high priest's servant girls came by. \v{67}When she saw Peter warming himself, she glared at him and said, ``You, too, were with Jesus from Nazareth.''

\v{68}But he denied it, saying, ``I don't know---or even understand---what you're talking about!'' Then he went out into the entryway. Just then a rooster crowed.\fnote{\fbackref{14:68} Other mss. lack \fbib{Just then a rooster crowed}}

\v{69}The servant girl saw him and again told those who were standing around, ``This man is one of them!'' \v{70}Again he denied it.

After a little while, the people who were standing there began to say to Peter again, ``Obviously you're one of them, because you are a Galilean!''

\v{71}Then he began to invoke a divine curse and to swear with an oath, ``I don't know this man you're talking about!'' \v{72}Just then a rooster crowed a second time.

Peter remembered that Jesus told him, \red{``Before a rooster crows twice, you will deny me three times.''} Then he broke down and cried.
\labelchapt{15}
\passage{Jesus is Taken to Pilate}
\passageinfo{(Matthew 27:1-2, 11-14; Luke 23:1-5; John 18:28-38)}

\chapt{15}
\v{1}As soon as it was morning, the high priests convened a meeting with the elders and scribes and the whole Council.\fnote{\fbackref{15:1} Or \fbib{Sanhedrin}} They bound Jesus with chains, led him away, and handed him over to Pilate. \v{2}Pilate asked him, ``Are you the king of the Jews?''

Jesus\fnote{\fbackref{15:2} Lit. \fbib{He}} answered him, \red{``You say so.''}

\v{3}The high priests kept accusing him of many things. \v{4}So Pilate asked him again, ``Don't you have any answer? Look how many accusations they're bringing against you!'' \v{5}But since Jesus no longer answered, Pilate was astonished.
\passage{Jesus is Sentenced to Death}
\passageinfo{(Matthew 27:15-26; Luke 23:13-25; John 18:39-19:16)}

\v{6}At every festival,\fnote{\fbackref{15:6} I.e. Passover} Pilate\fnote{\fbackref{15:6} Lit. \fbib{he}} would release any one prisoner whom the people\fnote{\fbackref{15:6} Lit. \fbib{they}} requested. \v{7}Now there was a man in prison named Barabbas. He had been with the insurgents who had committed murder during the rebellion. \v{8}So the crowd came and began to request that Pilate\fnote{\fbackref{15:8} The Gk. lacks \fbib{Pilate}} do for them what he always did.\fnote{\fbackref{15:8} I.e. release one condemned prisoner during Passover} \v{9}Pilate answered them, ``Do you want me to release the king of the Jews for you?'' \v{10}because he knew that the high priests had handed him over due to jealousy.

\v{11}But the high priests stirred up the crowd to get him to release Barabbas for them instead. \v{12}So Pilate asked them again, ``Then what should I do with the man you call\fnote{\fbackref{15:12} Other mss. lack \fbib{the man you call}} `the King of the Jews'?''

\v{13}``Crucify him!'' they shouted back.

\v{14}``Why?'' Pilate asked them. ``What has he done wrong?''

But they shouted even louder, ``Crucify him!''

\v{15}So Pilate, wanting to satisfy the crowd, released Barabbas for them, but he had Jesus whipped and handed over to be crucified.
\passage{The Soldiers Make Fun of Jesus}
\passageinfo{(Matthew 27:27-31; John 19:2-3)}

\v{16}The soldiers led Jesus\fnote{\fbackref{15:16} Lit. \fbib{him}} into the courtyard of the palace (that is, the governor's headquarters)\fnote{\fbackref{15:16} Lit. \fbib{praetorium}} and called out the whole guard. \v{17}They dressed him in a purple robe, twisted some thorns into a victor's crown, and placed it on his head.\fnote{\fbackref{15:17} Lit. \fbib{on him}} \v{18}They began to greet him, ``Long live the king of the Jews!'' \v{19}They kept hitting him on the head with a stick, spitting on him, kneeling in front of him, and worshiping him. \v{20}When they had finished making fun of him, they stripped him of the purple robe, put his own clothes back on him, and led him away to crucify him.
\passage{Jesus is Crucified}
\passageinfo{(Matthew 27:32-44; Luke 23:26-43; John 19:17-27)}

\v{21}They forced a certain passer-by named\fnote{\fbackref{15:21} The Gk. lacks \fbib{named}} Simon of Cyrene, the father of Alexander and Rufus, who happened to be coming in from the country, to carry Jesus'\fnote{\fbackref{15:21} Lit. \fbib{his}} cross. \v{22}They took Jesus\fnote{\fbackref{15:22} Lit. \fbib{him}} to a place called Golgotha, which means Skull Place. \v{23}They tried to give him wine mixed with myrrh, but he wouldn't accept it. \v{24}Then they crucified him. They divided his clothes among themselves by throwing dice to see what each one would get. \v{25}It was nine in the morning\fnote{\fbackref{15:25} Lit. \fbib{the third hour}} when they crucified him. \v{26}The written notice of the charge against him read, ``The king of the Jews.'' \v{27}They crucified two bandits\fnote{\fbackref{15:27} Or \fbib{revolutionaries}} with him, one on his right and the other on his left.\fnote{\fbackref{15:27} Other mss. read \fbib{on his left.} \fbib{\v{28}Then the Scripture was fulfilled that says, ``He was counted with criminals.''}} \v{29}Those who passed by kept insulting\fnote{\fbackref{15:29} Or \fbib{blaspheming}} him, shaking their heads, and saying, ``Ha! You who were going to destroy the sanctuary and rebuild it in three days--- \v{30}save yourself and come down from the cross!''

\v{31}In the same way, the high priests, along with the scribes, were also making fun of him among themselves. They kept saying, ``He saved others, but can't save himself! \v{32}Let the Messiah,\fnote{\fbackref{15:32} Or \fbib{Christ}} the king of Israel, come down from the cross now, since seeing is believing!'' Even the men who were crucified with him kept insulting him.
\passage{Jesus Dies on the Cross}
\passageinfo{(Matthew 27:45-56; Luke 23:44-49; John 19:28-30)}

\v{33}At noon,\fnote{\fbackref{15:33} Lit. \fbib{At the sixth hour}} darkness came over the whole land\fnote{\fbackref{15:33} Or \fbib{earth}} until three in the afternoon.\fnote{\fbackref{15:33} Lit. \fbib{the ninth hour}} \v{34}At three o'clock,\fnote{\fbackref{15:34} Lit. \fbib{the ninth hour}} Jesus cried out with a loud voice, \red{``Eloi, eloi,}\fnote{\fbackref{15:34} \fbib{Eloi, eloi} are Gk. transliterations for the Heb. \fbib{My God, my God} in Ps 22:1}\red{ lema sabachthani?''}\fnote{\fbackref{15:34} \fbib{lema sabachthani} is a Gk. transliteration for the Aram. rendering of the Heb. \fbib{in Ps 22:1, which means Why have you forsaken me?}} (which means, \red{``My God, my God, why have you forsaken me?''})\fnote{\fbackref{15:34} Cf. Ps 22:1}

\v{35}When some of the people standing there heard this, they said, ``Listen! He's calling for Elijah!''\fnote{\fbackref{15:35} \fbib{Elijah} in Heb. sounds like \fbib{Eloi}.}

\v{36}So someone ran and soaked a sponge in some sour wine. Then he put it on a stick and offered Jesus\fnote{\fbackref{15:36} Lit. \fbib{him}} a drink, saying, ``Wait! Let's see if Elijah comes to take him down!''

\v{37}Then Jesus gave a loud cry and breathed his last. \v{38}The curtain\fnote{\fbackref{15:38} This curtain separated the Holy Place from the Most Holy Place.} in the sanctuary was torn in two from top to bottom. \v{39}When the centurion\fnote{\fbackref{15:39} A Roman centurion commanded about 100 men.} who stood facing Jesus\fnote{\fbackref{15:39} Lit. \fbib{him}} saw how he had cried out and\fnote{\fbackref{15:39} Other mss. lack \fbib{cried out and}} breathed his last, he said, ``This man certainly was the Son of God!''

\v{40}Now there were women watching from a distance. Among them were Mary Magdalene,\fnote{\fbackref{15:40} Or \fbib{Mary of Magdala}} Mary the mother of young James and Joseph, and Salome. \v{41}They used to accompany him and care for him while he was in Galilee. Many other women who had come up to Jerusalem with him were there, too.
\passage{Jesus is Buried}
\passageinfo{(Matthew 27:57-61; Luke 23:50-56; John 19:38-42)}

\v{42}It was the Day of Preparation, that is, the day before the Sabbath. Since it was already evening, \v{43}Joseph of Arimathea, a highly respected member of the Council,\fnote{\fbackref{15:43} Or \fbib{Sanhedrin}} who was waiting for the kingdom of God, went boldly to Pilate and asked for the body of Jesus. \v{44}Pilate was amazed to hear\fnote{\fbackref{15:44} The Gk. lacks \fbib{to hear}} that Jesus\fnote{\fbackref{15:44} Lit. \fbib{he}} had already died, so he summoned the centurion to ask him if he was in fact dead. \v{45}When he learned from the centurion that he was dead, he let Joseph have the corpse. \v{46}Joseph\fnote{\fbackref{15:46} Lit. \fbib{He}} bought some linen cloth, took the body\fnote{\fbackref{15:46} Lit. \fbib{it}} down, wrapped it in the cloth, laid it in a tomb that had been cut out of the rock, and rolled a stone against the door of the tomb. \v{47}Now Mary Magdalene\fnote{\fbackref{15:47} Or \fbib{Mary of Magdala}} and Mary the mother of Joseph observed where the body\fnote{\fbackref{15:47} Lit. \fbib{where it}} had been laid.
\labelchapt{16}
\passage{Jesus is Raised from the Dead}
\passageinfo{(Matthew 28:1-8; Luke 24:1-12; John 20:1-10)}

\chapt{16}
\v{1}When the Sabbath was over, Mary Magdalene,\fnote{\fbackref{16:1} Or \fbib{Mary of Magdala}} Mary the mother of James, and Salome bought spices to go and anoint Jesus.\fnote{\fbackref{16:1} Lit. \fbib{him}} \v{2}Very early on the first day of the week,\fnote{\fbackref{16:2} Lit. \fbib{first of the Sabbaths}} when the sun had just come up, they were going to the tomb. \v{3}They kept saying to one another, ``Who will roll away the stone for us from the entrance to the tomb?'' \v{4}Then they looked up and saw that the stone had been rolled away. (It was a very large stone.)

\v{5}As they went into the tomb, they saw a young man dressed in a white robe sitting on the right side, and they were utterly astonished. \v{6}But he told them, ``Stop being astonished! You are looking for Jesus of Nazareth, who was crucified. He has been raised. He is not here. Look at the place where they laid him. \v{7}But go and tell his disciples---especially Peter---that Jesus\fnote{\fbackref{16:7} Lit. \fbib{he}} is going ahead of you to Galilee. There you will see him, just as he told you.''

\v{8}So they left the tomb and ran away, overwhelmed by shock and astonishment. They didn't say a thing to anyone, because they were afraid.\fnote{\fbackref{16:8} Some mss. end Mark here; others include verses 9-20. Some mss. conclude the book with the following shorter ending (others include the shorter ending and then continue with verses 9-20): \fbib{They reported to those who were with Peter everything they had been commanded. After this, Jesus sent out through them, from east to west, the sacred and indestructible message of eternal salvation. Amen.}}
\passage{Jesus Appears to Mary Magdalene}
\passageinfo{(Matthew 28:9-10; John 20:11-18)}

\v{9}After Jesus\fnote{\fbackref{16:9} Lit. \fbib{he}} had risen early on the first day of that week,\fnote{\fbackref{16:9} Lit. \fbib{first Sabbath}} he appeared first to Mary Magdalene,\fnote{\fbackref{16:9} Or \fbib{Mary of Magdala}} from whom he had driven out seven demons. \v{10}She went and told those who had been with Jesus\fnote{\fbackref{16:10} Lit. \fbib{him}} and who now were grieving and crying. \v{11}When they heard that he was alive and that he had been seen by her, they refused to believe Mary.\fnote{\fbackref{16:11} Lit. \fbib{her}}
\passage{Jesus Appears to Two Disciples}
\passageinfo{(Luke 24:13-35)}

\v{12}After this, Jesus\fnote{\fbackref{16:12} Lit. \fbib{he}} appeared in a different form to two disciples\fnote{\fbackref{16:12} Lit. \fbib{two of them}} as they were walking into the country. \v{13}They went back and told the others, who didn't believe them, either.
\passage{Jesus Commissions His Disciples}
\passageinfo{(Matthew 28:16-20; Luke 24:36-49; John 20:19-23; Acts 1:6-8)}

\v{14}Finally he appeared to his eleven disciples\fnote{\fbackref{16:14} The Gk. lacks \fbib{disciples}} while they were eating. He rebuked them for their unbelief and stubbornness, because they had not believed those who had seen him after he had risen. \v{15}Then he told them, \red{``As you go into the }\red{entire }\red{world, proclaim the gospel to everyone.}\fnote{\fbackref{16:15} Lit. \fbib{to the whole creation}} \v{16}\red{Whoever believes and is baptized will be saved, but whoever doesn't believe will be condemned.} \v{17}\red{These are the signs that will accompany those who believe: In my name they}\red{'}\red{ll drive out demons}\red{.} \red{they}\red{'}\red{ll }\red{speak in new }\red{languages}\red{,}\fnote{\fbackref{16:17} Other mss. lack \fbib{with their hands}} \v{18}\red{and }\red{they}\red{'}\red{ll }\red{pick up snakes with their hands}\red{.}\fnote{\fbackref{16:18} Or \fbib{new tongues}}\red{E}\red{ven if they drink any deadly poison}\red{,}\red{ it w}\red{on'}\red{t hurt them}\red{, }\red{and the}\red{y}\red{'}\red{ll place their hands on the sick, and they}\red{'}\red{ll recover.''}
\passage{Jesus is Taken Up to Heaven}
\passageinfo{(Luke 24:50-53; Acts 1:9-11)}

\v{19}So the Lord Jesus,\fnote{\fbackref{16:19} Other mss. lack \fbib{Jesus}} after talking with his disciples,\fnote{\fbackref{16:19} Lit. \fbib{with them}} was taken up to heaven and sat down at the right hand of God. \v{20}Then his disciples\fnote{\fbackref{16:20} Lit. \fbib{Then they}} went out and preached everywhere, while the Lord kept working with them and confirming the message by the signs that accompanied it.

\bookheader{Luke}
\labelbook{Luke}

\bookpretitle{The Gospel According to}
\booktitle{Luke}

\labelchapt{1}
\passage{Luke's Dedication to Theophilus}

\chapt{1}
\v{1}Since many people have attempted to write an orderly account of the events that have transpired among us, \v{2}just as they were passed down to us by those who had been eyewitnesses and servants of the word from the beginning, \v{3}I, too, have carefully investigated everything from the beginning and have decided to write an orderly account for you, most excellent Theophilus, \v{4}so that you may know the certainty of the things you have been taught.
\passage{The Birth of John the Baptist is Foretold}

\v{5}During the reign\fnote{\fbackref{1:5} Lit. \fbib{In the days}} of King Herod of Judea, there was a priest named Zechariah, who belonged to the priestly order of Abijah. His wife was a descendant of Aaron, and her name was Elizabeth. \v{6}Both of them were righteous before God, having lived blamelessly according to all of the commandments and regulations of the Lord. \v{7}They had no children because Elizabeth was barren and because both of them were getting old.\fnote{\fbackref{1:7} Lit. \fbib{were advancing in their days}}

\v{8}When Zechariah\fnote{\fbackref{1:8} Lit. \fbib{he}} was serving with his division of priests in God's presence, \v{9}he was chosen by lot to go into the sanctuary of the Lord and burn incense, according to the custom of the priests. \v{10}And the entire congregation of people was praying outside at the time when the incense was burned.

\v{11}An angel of the Lord appeared to him, standing at the right side of the incense altar. \v{12}When Zechariah saw him, he was startled, and fear overwhelmed him. \v{13}But the angel told him, ``Stop being afraid, Zechariah, because your prayer has been heard. Your wife Elizabeth will bear you a son, and you are to name him John. \v{14}You will have great joy,\fnote{\fbackref{1:14} Lit. \fbib{have joy and gladness}} and many people will rejoice at his birth, \v{15}because he will be great in the Lord's presence. He will never drink wine or any strong drink, and he will be filled with the Holy Spirit even before he is born. \v{16}He will bring many of Israel's descendants back to the Lord their God. \v{17}He is the one who will go before the Lord\fnote{\fbackref{1:17} Lit. \fbib{before him}} with the spirit and power of Elijah to turn the hearts of parents to their children and the disobedient to the wisdom of the righteous, and to prepare the people to be ready for the Lord.''

\v{18}Then Zechariah asked the angel, ``How can I be sure of this, since I am an old man, and my wife is getting older?''\fnote{\fbackref{1:18} Lit. \fbib{is advancing in her days}}

\v{19}The angel answered him, ``I am Gabriel! I stand in the very presence of God. I have been sent to speak to you and to announce this good news to you. \v{20}But because you did not believe my announcement, which will be fulfilled at its proper time,\fnote{\fbackref{1:20} Lit. \fbib{in their times}} you will become silent and unable to speak until the day this happens.''

\v{21}Meanwhile, the people kept waiting for Zechariah and wondering why he stayed in the sanctuary so long. \v{22}But when he did come out, he was unable to speak to them. Then they realized that he had seen a vision in the sanctuary. He kept motioning to them but remained unable to speak. \v{23}When the days of his service were over, he went home.

\v{24}After this,\fnote{\fbackref{1:24} Lit. \fbib{After those days}} his wife Elizabeth became pregnant and remained in seclusion for five months. She said, \v{25}``This is what the Lord did for me when he looked favorably on me and took away my public disgrace.''
\passage{The Birth of Jesus is Foretold}

\v{26}Now in the sixth month of her pregnancy,\fnote{\fbackref{1:26} The Gk. lacks \fbib{of her pregnancy}} the angel Gabriel was sent by God to a city in Galilee called Nazareth, \v{27}to a virgin engaged to a man named Joseph, a descendant\fnote{\fbackref{1:27} Lit. \fbib{of the house}} of David. The virgin's name was Mary. \v{28}The angel\fnote{\fbackref{1:28} Lit. \fbib{He}} came to her and said, ``Greetings, you who are highly favored! The Lord is with you!''\fnote{\fbackref{1:28} Other mss. read \fbib{is with you! How blessed are you among women!}} \v{29}Startled by his statement, she tried to figure out what his greeting meant.

\v{30}Then the angel told her, ``Stop being afraid, Mary, because you have found favor with God. \v{31}Listen! You will become pregnant and give birth to a son, and you are to name him Jesus. \v{32}He will be great and will be called the Son of the Most High, and the Lord God will give him the throne of his ancestor David. \v{33}He will rule over the house of Jacob forever, and his kingdom will never end.''

\v{34}Mary asked the angel, ``How can this happen, since I have not had relations with\fnote{\fbackref{1:34} Lit. \fbib{I have not known}} a man?''

\v{35}The angel answered her, ``The Holy Spirit will come over you, and the power of the Most High will surround you. Therefore, the child will be holy and will be called the Son of God. \v{36}And listen! Elizabeth, your relative, has herself conceived a son in her old age, this woman who was rumored to be barren is in her sixth month. \v{37}Nothing is impossible with respect to any of God's promises.''

\v{38}Then Mary said, ``Truly I am the Lord's servant. Let everything you have said happen to me.'' Then the angel left her.
\passage{Mary Visits Elizabeth}

\v{39}Later on,\fnote{\fbackref{1:39} Lit. \fbib{In those days}} Mary set out for a Judean city in the hill country. \v{40}She went into Zechariah's home and greeted Elizabeth. \v{41}When Elizabeth heard Mary's greeting, the baby jumped in her womb. Elizabeth was filled with the Holy Spirit \v{42}and she loudly exclaimed, ``How blessed are you among women, and how blessed is the infant in\fnote{\fbackref{1:42} Lit. \fbib{the fruit of}} your womb! \v{43}Why should this happen to me, to have the mother of my Lord visit me? \v{44}As soon as the sound of your greeting reached my ears, the baby in my womb jumped for joy. \v{45}How blessed is this woman for believing that what was spoken to her by the Lord would be fulfilled!''
\passage{Mary's Song of Praise}

\v{46}Then Mary said,

\begin{poetry}
\poeml ``My soul praises the greatness of the Lord! \\
\poeml \v{47}My spirit exults in God, my Savior, \\
\poeml \v{48}because he has looked favorably on his humble servant. \\
\poeml From now on, all generations will call me blessed, \\
\poeml \v{49}because the Almighty has done great things for me. \\
\poemlll       His name is holy. \\
\poeml \v{50}His mercy lasts from generation to generation \\
\poemll    for those who fear him. \\
\poeml \v{51}He displayed his mighty power with his arm. \\
\poemll    He scattered people who were proud in mind and heart.\fnote{\fbackref{1:51} Lit. \fbib{in the mind of their heart}} \\
\poeml \v{52}He pulled powerful rulers from their thrones \\
\poemll    and lifted up humble people. \\
\poeml \v{53}He filled hungry people with good things \\
\poemll    and sent rich people away with nothing. \\
\poeml \v{54}He helped his servant Israel, \\
\poemll    remembering to be merciful, \\
\poeml \v{55}according to the promise he made\fnote{\fbackref{1:55} Lit. \fbib{just as he spoke}} to our ancestors--- \\
\poemll    to Abraham and his descendants forever.''
\end{poetry}

\v{56}Now Mary stayed with Elizabeth\fnote{\fbackref{1:56} Lit. \fbib{with her}} about three months and then went back home.
\passage{The Birth of John the Baptist}

\v{57}When the time came for Elizabeth to have her child, she gave birth to a son. \v{58}Her neighbors and relatives heard that the Lord had shown his great mercy to her, and they rejoiced with her.

\v{59}On the eighth day they went to circumcise the child. They were going to name him Zechariah after his father, \v{60}but his mother said, ``Absolutely not! He must be named John.''

\v{61}Their friends\fnote{\fbackref{1:61} Lit. \fbib{They}} told her, ``None of your relatives has that name.''

\v{62}So they motioned to the baby's\fnote{\fbackref{1:62} Lit. \fbib{to his}} father to see what he wanted to name him. \v{63}He asked for a writing tablet and wrote, ``His name is John.'' And everyone was amazed.

\v{64}Suddenly, Zechariah could open his mouth,\fnote{\fbackref{1:64} Lit. \fbib{his mouth was opened}} his tongue was set free, and he began to speak and to praise God. \v{65}Fear came over all their neighbors, and throughout the hill country of Judea all these things were being discussed. \v{66}Everyone who heard about it kept thinking what had happened and asked, ``What will this child become?'' because it was obvious that the hand of the Lord was with him.
\passage{The Prophecy of Zechariah}

\v{67}Then his father Zechariah was filled with the Holy Spirit and prophesied:

\begin{poetry}
\poeml \v{68}``Blessed be the Lord God of Israel! \\
\poemll    He has taken care of his people and has set them free. \\
\poeml \v{69}He has raised up a mighty Savior\fnote{\fbackref{1:69} Lit. \fbib{a horn of salvation}} for us \\
\poemll    from the family of his servant David, \\
\poeml \v{70}just as he promised long ago \\
\poeml through the mouth of his holy prophets \\
\poeml \v{71}that he would save us from our enemies \\
\poemll    and from the grip of all who hate us. \\
\poeml \v{72}He has shown mercy to our ancestors \\
\poemll    and remembered his holy covenant, \\
\poeml \v{73}the oath that he swore to our ancestor Abraham. \\
\poeml He granted us \v{74}deliverance from our enemies' grip \\
\poemll    so that we could serve him without fear \\
\poeml \v{75}and be holy and righteous before him all of our days. \\
\poeml \v{76}And you, child, will be called a prophet of the Most High, \\
\poemll    because you will go ahead of the Lord to prepare his way \\
\poeml \v{77}and to give his people knowledge of salvation \\
\poemll    through forgiveness of their sins. \\
\poeml \v{78}Because of the tender mercy of our God, \\
\poemll    his light\fnote{\fbackref{1:78} Or \fbib{dawn}} from on high has visited us, \\
\poeml \v{79}to shine on those who sit in darkness and in death's shadow, \\
\poemll    and to guide our feet into the way of peace.''
\end{poetry}

\v{80}Now the child continued to grow and to become spiritually strong.\fnote{\fbackref{1:80} Or \fbib{become strong in the Spirit}} He lived in the wilderness until the day he appeared in Israel.
\labelchapt{2}
\passage{The Birth of Jesus}
\passageinfo{(Matthew 1:18-25)}

\chapt{2}
\v{1}Now in those days an order was published by Caesar Augustus that the whole world should be registered. \v{2}This was the first registration taken while Quirinius was governor of Syria. \v{3}So all the people went to their hometowns to be registered.

\v{4}Joseph, too, went up from the city of Nazareth in Galilee to Judea, to the City of David (called Bethlehem), because he was a descendant\fnote{\fbackref{2:4} The Gk. lacks \fbib{a descendant}} of the household and family of David. \v{5}He went there\fnote{\fbackref{2:5} The Gk. lacks \fbib{He went there}} to be registered with Mary, who had been promised to him in marriage and was pregnant.

\v{6}While they were there, the time came for her to have her baby, \v{7}and she gave birth to her first child, a son. She wrapped him in strips of cloth and laid him in a feeding trough, because there was no place for them in the guest quarters.
\passage{The Shepherds Visit Jesus}

\v{8}In that region there were shepherds living in the fields, watching their flock during the night. \v{9}An angel of the Lord appeared to them, and the glory of the Lord shone around them, and they were terrified. \v{10}Then the angel told them, ``Stop being afraid! Listen! I am bringing you good news of great joy for all the people. \v{11}Today your Savior, the Lord Messiah,\fnote{\fbackref{2:11} Or \fbib{Christ}} was born in the City of David. \v{12}And this will be a sign for you: You will find a baby wrapped in strips of cloth and lying in a feeding trough.''

\v{13}Suddenly, a multitude of the Heavenly Army appeared with the angel, praising God by saying, \v{14}``Glory to God in the highest, and peace on earth to people who enjoy his favor!''\fnote{\fbackref{2:14} Other mss. read \fbib{peace on earth, and favor to people}}

\v{15}When the angels had left them and gone back to heaven, the shepherds told one another, ``Let's go to Bethlehem and see what has taken place that the Lord has told us about.'' \v{16}So they went quickly and found Mary and Joseph with the baby, who was lying in the feeding trough. \v{17}When they saw this, they repeated what they had been told about this child. \v{18}All who heard it were amazed at what the shepherds told them. \v{19}However, Mary continued to treasure all these things in her heart and to ponder them. \v{20}Then the shepherds returned to their flock,\fnote{\fbackref{2:20} The Gk. lacks \fbib{to their flock}} glorifying and praising God for everything they had heard and seen, just as it had been told to them.
\passage{Jesus is Circumcised}

\v{21}After eight days had passed, the infant\fnote{\fbackref{2:21} Lit. \fbib{he}} was circumcised and named Jesus, the name given him by the angel before he was conceived in the womb.
\passage{Jesus is Presented in the Temple}

\v{22}When the time came for their purification according to the Law of Moses, Joseph and Mary\fnote{\fbackref{2:22} Lit. \fbib{they}} took Jesus\fnote{\fbackref{2:22} Lit. \fbib{him}} up to Jerusalem to present him to the Lord, \v{23}as it is written in the Law of the Lord, ``Every firstborn son is to be designated as holy to the Lord.''\fnote{\fbackref{2:23} Cf. Exod 13:2, 12, 15; MT source citation reads \fbib{}\divine{Lord}} \v{24}They also offered a sacrifice according to what is specified in the Law of the Lord: ``a pair of turtledoves or two young pigeons.''\fnote{\fbackref{2:24} Lev 12:8}

\v{25}Now a man named Simeon was in Jerusalem. This man was righteous and devout. He was waiting for the one who would comfort Israel,\fnote{\fbackref{2:25} Lit. \fbib{for the comfort of Israel}} and the Holy Spirit was upon him. \v{26}It had been revealed to him by the Holy Spirit that he would not die\fnote{\fbackref{2:26} Lit. \fbib{see death}} until he had seen the Lord's Messiah.\fnote{\fbackref{2:26} Or \fbib{Christ}}

\v{27}Led\fnote{\fbackref{2:27} The Gk. lacks \fbib{Led}} by the Spirit, he went into the Temple. When the parents brought the child Jesus to do for him what was customary under the Law, \v{28}Simeon\fnote{\fbackref{2:28} Lit. \fbib{he}} took the infant\fnote{\fbackref{2:28} Lit. \fbib{him}} in his arms and praised God, saying,

\begin{poetry}
\poeml \v{29}``Master, now you are dismissing your servant in peace \\
\poemll    according to your promise, \\
\poeml \v{30}because my eyes have seen your salvation, \\
\poeml \v{31}which you prepared for all people to see--- \\
\poeml \v{32}a light that will reveal salvation\fnote{\fbackref{2:32} Lit. \fbib{a light for revelation}} to \red{unbelievers}\fnote{\fbackref{2:32} Lit. \fbib{gentiles} ; i.e. unbelieving non-Jews} \\
\poemll    and bring glory to your people Israel.''
\end{poetry}

\v{33}Jesus'\fnote{\fbackref{2:33} Lit. \fbib{His}} father and mother kept wondering at the things being said about him. \v{34}Then Simeon\fnote{\fbackref{2:34} Lit. \fbib{he}} blessed them and told Mary, his mother, ``This infant is destined to cause many in Israel to fall and rise. Also, he will be a sign that will be opposed. \v{35}Indeed, a sword will pierce your own soul, too, so that the inner thoughts of many people might be revealed.''

\v{36}Now Anna, a prophetess, was also there. She was a descendant of Phanuel from the tribe of Asher. She was very old, having lived with her husband for seven years after her marriage, \v{37}and then as a widow for 84 years. She never left the Temple, but continued to worship there night and day with times of fasting and prayer. \v{38}Just then she came forward and began to thank God and to speak about Jesus\fnote{\fbackref{2:38} Lit. \fbib{him}} to everyone who was waiting for the redemption of Jerusalem.
\passage{The Return to Nazareth}

\v{39}After doing everything required by the Law of the Lord, Joseph and Mary\fnote{\fbackref{2:39} Lit. \fbib{they}} returned to their hometown of Nazareth in Galilee. \v{40}Meanwhile, the child continued to grow and to become strong. He was filled with wisdom, and God's favor rested upon him.
\passage{Jesus Visits the Temple}

\v{41}Every year Jesus'\fnote{\fbackref{2:41} Lit. \fbib{his}} parents would go to Jerusalem for the Passover Festival. \v{42}When Jesus\fnote{\fbackref{2:42} Lit. \fbib{he}} was twelve years old, they went up to the festival as usual. \v{43}When the days of the festival\fnote{\fbackref{2:43} The Gk. lacks \fbib{of the festival}} were over, they left for home. The young man Jesus stayed behind in Jerusalem, but his parents did not know it. \v{44}They thought that he was in their group of travelers. After traveling for a day, they started looking for him among their relatives and friends. \v{45}When they did not find him, they returned to Jerusalem, searching desperately for him. \v{46}Three days later, they found him in the Temple sitting among the teachers, listening to them, and posing questions to them. \v{47}All who heard him were amazed at his intelligence and his answers. \v{48}When Jesus' parents\fnote{\fbackref{2:48} Lit. \fbib{they}} saw him, they were shocked. His mother asked him, ``Son, why have you treated us like this? Your father and I have been worried sick looking for you!''

\v{49}He asked them, \red{``Why were you looking for me? Didn't you know that I had to be in my Father's house?''}\fnote{\fbackref{2:49} Or \fbib{about my Father's work}} \v{50}But they did not understand what he told them. \v{51}Then he went back with them, returning to Nazareth and remaining in submission to them. His mother continued to treasure all these things in her heart. \v{52}Meanwhile, Jesus kept on growing wiser and more mature, and in favor with God and his fellow man.
\labelchapt{3}
\passage{John the Baptist Prepares the Way for Jesus}
\passageinfo{(Matthew 3:1-12; Mark 1:1-8; John 1:19-28)}

\chapt{3}
\v{1}Now in the fifteenth year of the reign of Caesar Tiberius, when Pontius Pilate was governor of Judea, Herod tetrarch of Galilee, his brother Philip tetrarch of the region of Ituraea and Trachonitis, Lysanias tetrarch of Abilene, \v{2}and Annas and Caiaphas high priests, a message from God came to John, the son of Zechariah, in the wilderness. \v{3}John\fnote{\fbackref{3:3} Lit. \fbib{He}} went throughout the entire Jordan region, proclaiming a baptism about repentance for the forgiveness of sins, \v{4}as it is written in the book of the words of the prophet Isaiah,

\begin{poetry}
\poeml ``He is a voice calling out in the wilderness: \\
\poemll    `Prepare the way for the Lord!\fnote{\fbackref{3:4} Cf. Isa 40:3; MT source citation reads \fbib{}\divine{Lord}} Make his paths straight! \\
\poeml \v{5}Every valley will be filled, \\
\poemll    and every mountain and hill will be leveled. \\
\poeml The crooked ways will be made straight, \\
\poemll    and the rough roads will be made smooth. \\
\poeml \v{6}Everyone\fnote{\fbackref{3:6} Lit. \fbib{All flesh}} will see the salvation \\
\poemll    that God has provided.'\,''\fnote{\fbackref{3:6} Isa 40:3-5}
\end{poetry}

\v{7}John would say to the crowds that were coming out to be baptized by him, ``You children of serpents! Who warned you to flee from the coming wrath? \v{8}Produce fruit that is consistent with repentance! Don't begin to say to yourselves, `We have father Abraham!' because I tell you that God can raise up descendants for Abraham from these stones! \v{9}The ax already lies against the roots of the trees. So every tree not producing good fruit will be cut down and thrown into a fire.''

\v{10}The crowds kept asking him, ``What, then, should we do?''

\v{11}He answered them, ``The person who has two coats must share with the one who doesn't have any, and the person who has food must do the same.''

\v{12}Even some tax collectors came to be baptized. They asked him, ``Teacher, what should we do?''

\v{13}He told them, ``Stop collecting more money than the amount you are told to collect.''

\v{14}Even some soldiers were asking him, ``And what should we do?''

He told them, ``Never extort money from anyone by threats or blackmail, and be satisfied with your pay.''

\v{15}Now the people were filled with expectation, and all of them were wondering if John was perhaps the Messiah.\fnote{\fbackref{3:15} Or \fbib{Christ}} \v{16}John replied to all of them, ``I'm baptizing you with\fnote{\fbackref{3:16} Or \fbib{in}} water, but one is coming who is more powerful than I, and I'm not worthy to untie his sandal straps. It is he who will baptize you with\fnote{\fbackref{3:16} Or \fbib{in}} the Holy Spirit and fire. \v{17}His winnowing fork is in his hand to clean up his threshing floor. He'll gather the grain into his barn, but he'll burn the chaff with inextinguishable fire.''

\v{18}With many other exhortations John\fnote{\fbackref{3:18} Lit. \fbib{he}} continued to proclaim the good news to the people. \v{19}Now Herod the tetrarch had been rebuked by John\fnote{\fbackref{3:19} Lit. \fbib{him}} because he had married\fnote{\fbackref{3:19} Lit. \fbib{because of}} his brother's wife Herodias and because of all of the other evil things Herod had done. \v{20}Added to all this, Herod\fnote{\fbackref{3:20} Lit. \fbib{he}} locked John up in prison.
\passage{Jesus is Baptized}
\passageinfo{(Matthew 3:13-17; Mark 1:9-11)}

\v{21}When all the people had been baptized, Jesus, too, was baptized. While he was praying, heaven opened, \v{22}and the Holy Spirit descended on him, appearing in the form of a dove. Then a voice came from heaven, saying,\fnote{\fbackref{3:22} The Gk. lacks \fbib{saying}} ``You are my Son, whom I love. I am pleased with you!''\fnote{\fbackref{3:22} Other mss. read \fbib{You are my Son. Today I have become your Father.}}
\passage{The Ancestry of Jesus}
\passageinfo{(Matthew 1:1-17)}

\v{23}Jesus himself was about 30 years old when he began his ministry.\fnote{\fbackref{3:23} The Gk. lacks \fbib{his ministry}} He was (as legally calculated)\fnote{\fbackref{3:23} I.e. in conformity to genealogy reckonings then in effect; or (\fbib{so it was thought)}} the son of Joseph, the son of Heli, \v{24}the son of Matthat, the son of Levi, the son of Melchi, the son of Jannai, the son of Joseph, \v{25}the son of Mattathias, the son of Amos, the son of Nahum, the son of Esli, the son of Naggai, \v{26}the son of Maath, the son of Mattathias, the son of Semein, the son of Josech, the son of Joda, \v{27}the son of Joanan, the son of Rhesa, the son of Zerubbabel, the son of Shealtiel,\fnote{\fbackref{3:27} Or \fbib{Shealtiel}; cf. 1Chr 3:17; Ezra 3:2,8; 5:2; Neh 12:1; Matt 1:12} the son of Neri, \v{28}the son of Melchi, the son of Addi, the son of Cosam, the son of Elmadam, the son of Er, \v{29}the son of Joshua, the son of Eliezer, the son of Jorim, the son of Matthat, the son of Levi, \v{30}the son of Simeon, the son of Judah, the son of Joseph, the son of Jonam, the son of Eliakim, \v{31}the son of Melea, the son of Menna, the son of Mattatha, the son of Nathan, the son of David, \v{32}the son of Jesse, the son of Obed, the son of Boaz, the son of Salmon,\fnote{\fbackref{3:32} Other mss. read \fbib{Sala}} the son of Nahshon, \v{33}the son of Amminadab, the son of Admin, the son of Arni, the son of Hezron, the son of Perez, the son of Judah, \v{34}the son of Jacob, the son of Isaac, the son of Abraham, the son of Terah, the son of Nahor, \v{35}the son of Serug, the son of Reu, the son of Peleg, the son of Eber, the son of Shelah, \v{36}the son of Cainan,\fnote{\fbackref{3:36} The inclusion of Cainan here follows the LXX; cf. Gen 10:4,24; 11:12-13; 1Chron 1:7} the son of Arphaxad, the son of Shem, the son of Noah, the son of Lamech, \v{37}the son of Methuselah, the son of Enoch, the son of Jared, the son of Mahalaleel, the son of Cainan, \v{38}the son of Enos, the son of Seth, the son of Adam, the son of God.
\labelchapt{4}
\passage{Jesus is Tempted by Satan}
\passageinfo{(Matthew 4:1-11; Mark 1:12-13)}

\chapt{4}
\v{1}Then Jesus, filled with the Holy Spirit, returned from the Jordan. He was led by the Spirit into the wilderness, \v{2}where he was being tempted by the devil for 40 days. During that time he ate nothing at all, and when they were over he became hungry.

\v{3}The devil told him, ``Since\fnote{\fbackref{4:3} Or ``\fbib{If, as is the case,}} you are the Son of God, tell this stone to become a loaf of bread.''

\v{4}Jesus answered him, \red{``It is written,}

\begin{poetry}
\poeml \red{`One must not live on bread alone,} \\
\poemll    \red{but on every word of God.'\,''}\fnote{\fbackref{4:4} Cf. Deut 8:3; Other mss. lack \fbib{but on every word of God}}
\end{poetry}

\v{5}The devil\fnote{\fbackref{4:5} Lit. \fbib{He}} also took him to a high place\fnote{\fbackref{4:5} Lit. \fbib{took him up}} and showed him all the kingdoms of the world in an instant. \v{6}He told Jesus,\fnote{\fbackref{4:6} Lit. \fbib{him}} ``I will give you all this authority, along with their glory, because it has been given to me, and I give it to anyone I please. \v{7}So if you will worship me, all this will be yours.''

\v{8}But Jesus answered him, \red{``It is written,}

\begin{poetry}
\poeml \red{ `You must worship the Lord\fnote{\fbackref{4:8}; MT source citation reads \fbib{}\divine{Lord}} your God and serve only him.'\,''}\fnote{\fbackref{4:8} Cf. Deut 6:13}
\end{poetry}

\v{9}The devil\fnote{\fbackref{4:9} Lit. \fbib{He}} also took him into Jerusalem and had him stand on the highest point of the Temple. He told Jesus,\fnote{\fbackref{4:9} Lit. \fbib{him}} ``Since\fnote{\fbackref{4:9} Or ``\fbib{If, as is the case,}} you are the Son of God, throw yourself down from here, \v{10}because it is written,

\begin{poetry}
\poeml `God\fnote{\fbackref{4:10} Lit. \fbib{He}} will put his angels in charge of you \\
\poemll    to watch over you carefully. \\
\poeml \v{11}With their hands they will hold you up, \\
\poemll    so that you will never hit your foot against a rock.'\,''\fnote{\fbackref{4:11} Cf. Ps 91:11-12}
\end{poetry}

\v{12}Jesus answered him, \red{``It has been said, `You must not tempt the Lord\fnote{\fbackref{4:12} MT source citation reads \fbib{}\divine{Lord}} your God.'\,''}\fnote{\fbackref{4:12} Cf. Deut 6:16}

\v{13}After the devil had finished tempting Jesus in every possible way, he left him until another time.
\passage{Jesus Begins His Ministry in Galilee}
\passageinfo{(Matthew 4:12-17; Mark 1:14-15)}

\v{14}Then Jesus returned to Galilee by the power of the Spirit. Meanwhile, the news about him spread throughout the surrounding country. \v{15}He began to teach in their synagogues and was continuously receiving praise from everyone.
\passage{Jesus is Rejected at Nazareth}
\passageinfo{(Matthew 13:53-58; Mark 6:1-6)}

\v{16}Then Jesus\fnote{\fbackref{4:16} Lit. \fbib{he}} came to Nazareth, where he had been raised. As was his custom, he went into the synagogue on the Sabbath day. When he stood up to read, \v{17}the scroll of the prophet Isaiah was handed to him. Unrolling the scroll, he found the place where it was written,

\begin{poetry}
\poeml \v{18}\red{``The Spirit of the Lord\fnote{\fbackref{4:18} The MT source citation reads \fbib{}\divine{Lord}} is upon me;} \\
\poemll    \red{he has anointed me to tell} \\
\poemlll       \red{the good news to the poor.} \\
\poeml \red{He has sent me to announce release to the prisoners} \\
\poemll    \red{and recovery of sight to the blind,} \\
\poemlll       \red{to set oppressed people free,} \\
\poeml \v{19}\red{and to announce the year of the Lord's\fnote{\fbackref{4:19} The MT source citation reads \fbib{}\divine{Lord}} favor.''}\fnote{\fbackref{4:19} Cf. Isa 61:1-2; 58:6}
\end{poetry}

\v{20}Then he rolled up the scroll, gave it back to the attendant, and sat down. While the eyes of everyone in the synagogue were fixed on him, \v{21}he began to say to them, \red{``Today this Scripture has been fulfilled, as you've heard it read aloud.''}\fnote{\fbackref{4:21} Lit. \fbib{fulfilled in your ears}}

\v{22}All the people began to speak well of him and to wonder at the gracious words that flowed from his mouth. They said, ``This is Joseph's son, isn't it?''

\v{23}So he told them, \red{``You will probably quote this proverb to me, `Doctor, heal yourself! Do everything here in your hometown that we hear you did in Capernaum.'\,''}

\v{24}He added, \red{``I tell all of you\fnote{\fbackref{4:24} The Gk. pronoun \fbib{you} is pl.} with certainty, a prophet is not accepted in his hometown.} \v{25}\red{I'm telling you the truth---there were many widows in Israel in Elijah's time, when it didn't rain\fnote{\fbackref{4:25} Lit. \fbib{when the heavens were closed}} for three years and six months and there was a severe famine everywhere in the land.} \v{26}\red{Yet Elijah wasn't sent to a single one of those widows except to one at Zarephath in Sidon.} \v{27}\red{There were also many lepers in Israel in the prophet Elisha's time, yet not one of them was cleansed except Naaman the Syrian.''}

\v{28}All the people in the synagogue became furious when they heard this. \v{29}They got up, forced Jesus\fnote{\fbackref{4:29} Lit. \fbib{him}} out of the city, and led him to the edge of the hill on which their city was built, intending to throw him off. \v{30}But he walked right through the middle of them and went away.
\passage{Jesus Heals a Man with an Unclean Spirit}
\passageinfo{(Mark 1:21-28)}

\v{31}Then Jesus\fnote{\fbackref{4:31} Lit. \fbib{he}} went down to Capernaum, a city in Galilee, and began teaching the people\fnote{\fbackref{4:31} Lit. \fbib{them}} on Sabbath days.\fnote{\fbackref{4:31} Lit. \fbib{Sabbaths}} \v{32}They were utterly amazed at what he taught, because his message was spoken\fnote{\fbackref{4:32} The Gk. lacks \fbib{spoken}} with authority.

\v{33}In the synagogue was a man who had a demon.\fnote{\fbackref{4:33} Lit. \fbib{a spirit of a demon of uncleanness}} He screamed with a loud voice, \v{34}``Oh, no! What do you want with us, Jesus of Nazareth? Have you come to destroy us? I know who you are---the Holy One of God!''

\v{35}But Jesus rebuked him. \red{``Be quiet,''} he said,\red{ ``and come out of him!''} At this, the demon threw the man\fnote{\fbackref{4:35} Lit. \fbib{him}} down in the middle of the synagogue\fnote{\fbackref{4:35} The Gk. lacks \fbib{of the synagogue}} and came out of him without hurting him.

\v{36}Overwhelmed with amazement, they all kept saying to one another, ``What kind of statement is this?---because with authority and power he gives orders to unclean spirits, and they come out!'' \v{37}So news about him spread to every place in the surrounding region.
\passage{Jesus Heals Many People}
\passageinfo{(Matthew 8:14-17; Mark 1:29-34)}

\v{38}Then Jesus\fnote{\fbackref{4:38} Lit. \fbib{he}} got up to leave the synagogue and went into Simon's house. Now Simon's mother-in-law was sick with a high fever, so they asked Jesus\fnote{\fbackref{4:38} Lit. \fbib{him}} about her. \v{39}He bent over her, rebuked the fever, and it left her. She got up at once and began serving them. \v{40}When the sun was setting, everyone who had any friends\fnote{\fbackref{4:40} Lit. \fbib{people}} suffering from various diseases brought them to him. He placed his hands on each of them and began healing them. \v{41}Even demons came out of many people, screaming, ``You are the Son of God!'' But Jesus\fnote{\fbackref{4:41} Lit. \fbib{he}} rebuked them and ordered them not to speak, because they knew he was the Messiah.\fnote{\fbackref{4:41} Or \fbib{Christ}}
\passage{Jesus Goes on a Preaching Tour}
\passageinfo{(Mark 1:35-39)}

\v{42}At daybreak he left and went to a deserted place, while the crowds kept looking for him. When they came to him, they tried to keep him from leaving them. \v{43}But he told them, \red{``I have to proclaim the good news about the kingdom of God in other cities, because I was sent to do that also.''} \v{44}So he continued to preach in the synagogues of Galilee.\fnote{\fbackref{4:44} Other mss. read \fbib{of Judea}}
\labelchapt{5}
\passage{Jesus Calls His First Disciples}
\passageinfo{(Matthew 4:18-22; Mark 1:16-20)}

\chapt{5}
\v{1}One day, as the crowd was pressing in on him to listen to God's word, Jesus\fnote{\fbackref{5:1} Lit. \fbib{he}} was standing by the lake of Gennesaret. \v{2}He saw two boats lying on the shore, but the fishermen had stepped out of them and were washing their nets. \v{3}So Jesus\fnote{\fbackref{5:3} Lit. \fbib{he}} got into one of the boats (the one that belonged to Simon) and asked him to push out a little from the shore. Then he sat down and began to teach the crowds from the boat.

\v{4}When he had finished speaking, he told Simon, \red{``Push out into deep water, and lower your nets for a catch.''}

\v{5}Simon answered, ``Master, we have worked hard all night and caught nothing. But if you say so, I'll lower the nets.'' \v{6}After the men\fnote{\fbackref{5:6} Lit. \fbib{they}} had done this, they caught so many fish that the nets began to tear. \v{7}So they signaled to their partners in the other boat to come and help them. They came and filled both boats until the boats\fnote{\fbackref{5:7} Lit. \fbib{they}} began to sink. \v{8}When Simon Peter saw this, he fell down at Jesus' knees and said, ``Leave me, Lord! I am a sinful man!''--- \v{9}because Simon\fnote{\fbackref{5:9} Lit. \fbib{he}} and all the people who were with him were amazed at the number of fish they had caught, \v{10}and so were James and John, Zebedee's sons and Simon's partners.

Then Jesus told Simon, \red{``Stop being afraid. From now on you will be catching people.''} \v{11}So when they brought the boats to shore, they left everything and followed Jesus.\fnote{\fbackref{5:11} Lit. \fbib{him}}
\passage{Jesus Cleanses a Leper}
\passageinfo{(Matthew 8:1-4; Mark 1:40-45)}

\v{12}One day, while Jesus\fnote{\fbackref{5:12} Lit. \fbib{he}} was in one of the cities, a man covered with leprosy saw Jesus and fell on his face, begging him, ``Lord, if you want to, you can make me clean.''

\v{13}So Jesus\fnote{\fbackref{5:13} Lit. \fbib{he}} reached out his hand and touched him, saying, \red{``I do want to. Be clean!''} Instantly the leprosy left him. \v{14}Then Jesus\fnote{\fbackref{5:14} Lit. \fbib{he}} ordered him, \red{``Don't tell anyone. Instead, go and show yourself to the priest and make an offering for your cleansing, just as Moses commanded, as proof to the authorities.''}\fnote{\fbackref{5:14} Lit. \fbib{to them}} \v{15}But the news about Jesus\fnote{\fbackref{5:15} Lit. \fbib{him}} spread even more, and many crowds began gathering to hear him and to be healed of their diseases. \v{16}However, he continued his habit of retiring to deserted places and praying.
\passage{Jesus Heals a Paralyzed Man}
\passageinfo{(Matthew 9:1-8; Mark 2:1-12)}

\v{17}One day, as Jesus\fnote{\fbackref{5:17} Lit. \fbib{he}} was teaching, some Pharisees and teachers of the Law happened to be sitting nearby. The people\fnote{\fbackref{5:17} Lit. \fbib{They}} had come from every village in Galilee and Judea and from Jerusalem. The power of the Lord was present to heal them.\fnote{\fbackref{5:17} Other mss. read \fbib{was present with him to heal}} \v{18}Some men were bringing a paralyzed man on a stretcher. They were trying to take him into the house\fnote{\fbackref{5:18} The Gk. lacks \fbib{the house}} and place him in front of Jesus.\fnote{\fbackref{5:18} Lit. \fbib{him}} \v{19}When they couldn't find a way to get him in because of the crowd, they went up on the roof and let him down on his stretcher through the tiles into the middle of the room,\fnote{\fbackref{5:19} The Gk. lacks \fbib{of the room}} right in front of Jesus. \v{20}When Jesus\fnote{\fbackref{5:20} Lit. \fbib{he}} saw their faith, he said, \red{``Mister,\fnote{\fbackref{5:20} Lit. \fbib{Man}} your sins are forgiven.''}

\v{21}The scribes and the Pharisees began to argue among themselves, saying, ``Who is this man who is uttering blasphemies? Who can forgive sins but God alone?''

\v{22}Because Jesus knew that they were arguing, he asked them, \red{``Why are you arguing about this among yourselves?}\fnote{\fbackref{5:22} Lit. \fbib{in your hearts}} \v{23}\red{Which is easier: to say, `Your sins are forgiven,' or to say, `Get up and walk'?} \v{24}\red{But so you'll know that the Son of Man has authority on earth to forgive sins{\ldots}.''} he told the paralyzed man, \red{``I say to you: Get up, pick up your stretcher, and go home!''} \v{25}So the man\fnote{\fbackref{5:25} Lit. \fbib{he}} immediately stood up in front of them and picked up what he had been lying on. Then he went home, praising God.

\v{26}Amazement seized all the people, and they began to praise God. They were filled with fear\fnote{\fbackref{5:26} Or \fbib{awe}} and declared, ``We have seen wonderful things today!''
\passage{Jesus Calls Levi}
\passageinfo{(Matthew 9:9-13; Mark 2:13-17)}

\v{27}After that, Jesus\fnote{\fbackref{5:27} Lit. \fbib{he}} went out and saw a tax collector named Levi sitting at the tax collector's desk. He told him, \red{``Follow me!''} \v{28}So Levi\fnote{\fbackref{5:28} Lit. \fbib{he}} left everything behind, got up, and followed him.

\v{29}Later, Levi gave a large banquet at his home for Jesus.\fnote{\fbackref{5:29} Lit. \fbib{him}} A large crowd of tax collectors and others were eating with them. \v{30}The Pharisees and their scribes started complaining to Jesus'\fnote{\fbackref{5:30} Lit. \fbib{his}} disciples, ``Why do you eat and drink with tax collectors and sinners?''

\v{31}But Jesus answered them, \red{``Healthy people don't need a physician, but sick people do.} \v{32}\red{I have not come to call righteous people, but sinners, to repentance.''}
\passage{A Question about Fasting}
\passageinfo{(Matthew 9:14-17; Mark 2:18-22)}

\v{33}Then they told him, ``John's disciples frequently fast and pray, and so do those of the Pharisees. But your disciples\fnote{\fbackref{5:33} Lit. \fbib{yours}} keep right on eating and drinking.''

\v{34}But Jesus told them, \red{``You can't force the wedding guests\fnote{\fbackref{5:34} Lit. \fbib{The children of the wedding hall}; or \fbib{The children of the groom}} to fast while the groom is still with them, can you?} \v{35}\red{But the time will come when the groom will be taken away from them, and at that time they will fast.''}
\passage{The Unshrunk Cloth}
\passageinfo{(Matthew 9:16; Mark 2:21)}

\v{36}Then he told them a parable: \red{``No one tears a piece of cloth from a new garment and sews it on an old garment. If he does, the new cloth will tear, and the piece from the new won't match the old.} \v{37}\red{And no one pours new wine into old wineskins. If he does, the new wine will make the skins burst, the wine\fnote{\fbackref{5:37} Lit. \fbib{it}} will be spilled, and the skins will be ruined.} \v{38}\red{Instead, new wine is to be poured into fresh wineskins.} \v{39}\red{No one who has been drinking old wine wants new wine, because he says, `The old wine is good enough!'\,''}\fnote{\fbackref{5:39} Other mss. lack this verse}
\labelchapt{6}
\passage{Jesus is Lord of the Sabbath}
\passageinfo{(Matthew 12:1-8; Mark 2:23-28)}

\chapt{6}
\v{1}One time Jesus\fnote{\fbackref{6:1} Lit. \fbib{he}} was walking through some grain fields on a Sabbath.\fnote{\fbackref{6:1} Other mss. read \fbib{on the second Sabbath after the first}} His disciples were picking the heads of grain, rubbing them in their hands, and eating them. \v{2}Some of the Pharisees asked, ``Why are you doing what isn't lawful on Sabbath days?''\fnote{\fbackref{6:2} Lit. \fbib{on the Sabbaths}}

\v{3}Jesus answered them, \red{``Haven't you read what David did when he and his companions became hungry?} \v{4}\red{How was it that he went into the house of God, took the Bread of the Presence and ate it, which was not lawful for anyone but the priests to eat, and then gave some of it to his companions?''}

\v{5}Then he told them, \red{``The Son of Man is Lord of the Sabbath.''}
\passage{Jesus Heals a Man with a Paralyzed Hand}
\passageinfo{(Matthew 12:9-14; Mark 3:1-6)}

\v{6}Once, on another Sabbath, Jesus\fnote{\fbackref{6:6} Lit. \fbib{he}} went into a synagogue and began teaching. A man whose right hand was paralyzed was there. \v{7}The scribes and the Pharisees were watching Jesus\fnote{\fbackref{6:7} Lit. \fbib{him}} closely to see\fnote{\fbackref{6:7} The Gk. lacks \fbib{to see}} whether he would heal on the Sabbath, in order to find a way of accusing him of doing something wrong. \v{8}But Jesus\fnote{\fbackref{6:8} Lit. \fbib{he}} knew what they were thinking. So he told the man with the paralyzed hand, \red{``Get up, and stand in the middle of the synagogue.''}\fnote{\fbackref{6:8} The Gk. lacks \fbib{of the synagogue}} So he got up and stood there.

\v{9}Then Jesus asked them, \red{``I ask you, is it lawful to do good or to do evil on the Sabbath, to save a life or to destroy it?''}

\v{10}He looked around at all of them and then told the man,\fnote{\fbackref{6:10} Lit. \fbib{him}} \red{``Hold out your hand.''} The man\fnote{\fbackref{6:10} Lit. \fbib{He}} did so, and his hand was restored to health. \v{11}The others were furious\fnote{\fbackref{6:11} Or \fbib{were stupefied}} and began to discuss with each other what they could do to Jesus.
\passage{Jesus Appoints Twelve Apostles}
\passageinfo{(Matthew 10:1-4; Mark 3:13-19)}

\v{12}Now it was in those days that Jesus\fnote{\fbackref{6:12} Lit. \fbib{he}} went to a mountain to pray, and he spent the whole night in prayer to God. \v{13}When daylight came, he called his disciples and chose twelve of them, whom he also called apostles: \v{14}Simon (whom he named Peter), his brother Andrew, James, John, Philip, Bartholomew, \v{15}Matthew, Thomas, James (the son of Alphaeus), Simon (who was called the Zealot), \v{16}Judas (the son of James), and Judas Iscariot (who became a traitor).
\passage{Jesus Ministers to Many People}
\passageinfo{(Matthew 4:23-25)}

\v{17}Then Jesus\fnote{\fbackref{6:17} Lit. \fbib{he}} came down with them and stood on a level place, along with many of his disciples and a large gathering of people from all over Judea, Jerusalem, and the seacoast of Tyre and Sidon. \v{18}They had come to hear him and to be healed of their diseases. Even those who were being tormented by unclean spirits were being healed. \v{19}The entire crowd was trying to touch him, because power was coming out from him and healing all of them.
\passage{Jesus Pronounces Blessings and Judgment}
\passageinfo{(Matthew 5:1-12)}

\v{20}Then Jesus\fnote{\fbackref{6:20} Lit. \fbib{he}} looked at his disciples and said,

\begin{poetry}
\poeml \red{``How blessed are you who are destitute,} \\
\poemll    \red{because the kingdom of God is yours!} \\
\poeml \v{21}\red{How blessed are you who are hungry now,} \\
\poemll    \red{because you will be satisfied!} \\
\poeml \red{How blessed are you who are crying now,} \\
\poemll    \red{because you will laugh!}
\end{poetry}

\v{22}\red{``How blessed are you whenever people hate you, avoid you, insult you, and slander you because of the Son of Man!} \v{23}\red{Rejoice in that day and leap for joy, because your reward in heaven is great! That's the way their ancestors used to treat the prophets.}

\begin{poetry}
\poeml \v{24}\red{``But how terrible it will be for you who are rich,} \\
\poemll    \red{because you have had your comfort!} \\
\poeml \v{25}\red{How terrible it will be for you who are full now,} \\
\poemll    \red{because you will be hungry!} \\
\poeml \red{How terrible it will be for you who are laughing now,} \\
\poemll    \red{because you will mourn and cry!}
\end{poetry}

\v{26}\red{``How terrible it will be for you when everyone says nice things about you, because that's the way their ancestors used to treat the false prophets!''}
\passage{Teaching about Love for Enemies}
\passageinfo{(Matthew 5:38-48)}

\v{27}\red{``But I say to you who are listening: Love your enemies. Do good to those who hate you.} \v{28}\red{Bless those who curse you, and pray for those who insult you.} \v{29}\red{If someone strikes you on the cheek, offer him the other one as well, and if someone takes your coat, don't keep back your shirt, either.} \v{30}\red{Keep on giving to everyone who asks you for something, and if anyone takes what is yours, do not insist on getting it back.} \v{31}\red{Whatever you want people to do for you, do the same for them.}

\v{32}\red{``If you love those who love you, what thanks do you deserve? Why, even sinners love those who love them.} \v{33}\red{If you do good to those who do good to you, what thanks do you deserve? Even sinners do that.} \v{34}\red{If you lend to those from whom you expect to get something back, what thanks do you deserve? Even sinners lend to sinners to get back what they lend.} \v{35}\red{Instead, love your enemies, do good to them, and lend to them, expecting nothing in return. Then your reward will be great, and you will be children of the Most High, because he is kind even to ungrateful and evil people.} \v{36}\red{Be merciful, just as your Father is merciful.''}
\passage{Judging Others}
\passageinfo{(Matthew 7:1-5)}

\v{37}\red{``Stop judging, and you'll never be judged. Stop condemning, and you'll never be condemned. Forgive, and you'll be forgiven.} \v{38}\red{Give, and it will be given to you. A large quantity, pressed together, shaken down, and running over will be put into your lap, because you'll be evaluated by the same standard with which you evaluate others.''}

\v{39}He also told them a parable: \red{``One blind person can't lead another blind person, can he? Both will fall into a ditch, won't they?} \v{40}\red{A disciple is not better than his teacher. But everyone who is fully-trained will be like his teacher.}

\v{41}\red{``Why do you see the speck in your brother's eye but fail to notice the beam in your own eye?} \v{42}\red{How can you say to your brother, `Brother, let me take the speck out of your eye,' when you don't see the beam in your own eye? You hypocrite! First remove the beam from your own eye, and then you'll see clearly enough to remove the speck from your brother's eye.''}
\passage{A Tree is Known by Its Fruit}
\passageinfo{(Matthew 7:17-20)}

\v{43}\red{``A good tree doesn't produce rotten fruit, and a rotten tree doesn't produce good fruit,} \v{44}\red{because every tree is known by its own fruit. People\fnote{\fbackref{6:44} Lit. \fbib{They}} don't gather figs from thorny plants or pick grapes from a thorn bush.} \v{45}\red{A good person produces good from the good treasure of his heart, and an evil person produces evil from an evil treasure, because the mouth speaks from the overflow of the heart.''}
\passage{The Two Foundations}
\passageinfo{(Matthew 7:24-27)}

\v{46}\red{``Why do you keep calling me `Lord, Lord,' but don't do what I tell you?} \v{47}\red{I will show you what everyone is like who comes to me, hears my words, and acts on them.} \v{48}\red{They are like a person building a house, who dug a deep hole to lay the foundation on rock. When a flood came, the floodwaters pushed against that house but couldn't shake it, because it had been founded on the rock.}\fnote{\fbackref{6:48} Other mss. read \fbib{had been built well}} \v{49}\red{But the person who hears what I say\fnote{\fbackref{6:49} The Gk. lacks \fbib{what I say}} but doesn't act on it is like someone who built a house on the ground without any foundation. When the floodwaters pushed against it, that house\fnote{\fbackref{6:49} Lit. \fbib{it}} quickly collapsed, and the resulting destruction of that house was extensive.''}
\labelchapt{7}
\passage{Jesus Heals a Centurion's Servant}
\passageinfo{(Matthew 8:5-13; John 4:43-54)}

\chapt{7}
\v{1}After Jesus\fnote{\fbackref{7:1} Lit. \fbib{he}} had finished saying all these things\fnote{\fbackref{7:1} Lit. \fbib{finished all his sayings}} to the people who were there listening, he went to Capernaum. \v{2}There a centurion's servant, whom he valued highly, was sick and about to die. \v{3}When the centurion\fnote{\fbackref{7:3} Lit. \fbib{he}} heard about Jesus, he sent some Jewish elders to him to ask him to come and save his servant's life. \v{4}So they went to Jesus and begged him repeatedly, ``He deserves to have this done for him, \v{5}because he loves our people and built our synagogue for us.''

\v{6}So Jesus went with them. He was not far from the house when the centurion sent friends to tell Jesus,\fnote{\fbackref{7:6} Lit. \fbib{him}} ``Sir,\fnote{\fbackref{7:6} Or \fbib{Lord}} stop troubling yourself, because I'm not worthy to have you come under my roof. \v{7}That's why I didn't presume to come to you. But just say the word, and let my servant be healed, \v{8}because I, too, am a man under authority and have soldiers under me. I say to one `Go' and he goes, to another `Come' and he comes, and to my servant `Do this' and he does it.''

\v{9}When Jesus heard this, he was amazed at him. Turning to the crowd that was following him, he said, \red{``I tell you, not even in Israel have I found this kind of faith!''} \v{10}Then the men who had been sent returned to the house and found the servant in perfect health.
\passage{Jesus Raises a Widow's Son}

\v{11}Soon afterwards, Jesus\fnote{\fbackref{7:11} Lit. \fbib{he}} went to a city called Nain. His disciples and a large crowd were going along with him. \v{12}As he approached the entrance to the city, a man who had died was being carried out. He was his mother's only son, and she was a widow. A large crowd from the city was with her.

\v{13}When the Lord saw her, he felt compassion for her. He told her, \red{``You can stop crying.''} \v{14}Then he went up and touched the bier, and the men who were carrying it stopped. He said, \red{``Young man, I say to you, get up!''} \v{15}The man who had been dead sat up and began to speak, and Jesus\fnote{\fbackref{7:15} Lit. \fbib{he}} gave him back to his mother.

\v{16}Fear gripped everyone, and they began to praise God. ``A great prophet has appeared among us,'' they said, and ``God has helped his people.'' \v{17}This news about Jesus\fnote{\fbackref{7:17} Lit. \fbib{him}} spread throughout Judea and all the surrounding countryside.
\passage{John the Baptist Sends Messengers to Jesus}
\passageinfo{(Matthew 11:2-19)}

\v{18}John's disciples told him about all these things. So John called two of his disciples \v{19}and sent them to the Lord to ask, ``Are you the Coming One, or should we wait for someone else?''

\v{20}When the men had come to Jesus,\fnote{\fbackref{7:20} Lit. \fbib{him}} they said, ``John the Baptist has sent us to you to ask, `Are you the Coming One, or should we wait for someone else?'\,''

\v{21}At that time Jesus\fnote{\fbackref{7:21} Lit. \fbib{he}} had healed many people of diseases, plagues, and evil spirits, and had given sight to many who were blind. \v{22}So he answered them, \red{``Go and tell John what you have observed and heard: the blind see, the lame walk, lepers are cleansed, the deaf hear again, the dead are raised, and the destitute hear the good news.} \v{23}\red{How blessed is anyone who is not offended by me!''}

\v{24}When John's messengers had gone, Jesus\fnote{\fbackref{7:24} Lit. \fbib{he}} began to ask the crowds about John. \red{``What did you go out into the wilderness to see? A reed shaken by the wind?} \v{25}\red{Really, what did you go out to see? A man dressed in fancy clothes? Look! Those who wear fine clothes and live in luxury are in royal palaces.} \v{26}\red{Really, what did you go out to see? A prophet? Yes, I tell you, and even more than a prophet!} \v{27}\red{This is the man about whom it is written,}

\begin{poetry}
\poeml \red{`See, I am sending my messenger ahead of you,} \\
\poemll    \red{who will prepare your way before you.'}\fnote{\fbackref{7:27} Cf. Mal 3:1; Exod 23:20}
\end{poetry}

\v{28}\red{I tell you, no one has ever been born who is\fnote{\fbackref{7:28} Lit. \fbib{one among those born of women have been}} greater than John. Yet even the least important person in the kingdom of God is greater than he.''}

\v{29}By having been baptized with John's baptism, all the people who listened, including the tax collectors, acknowledged God's justice.\fnote{\fbackref{7:29} Or \fbib{judgment}} \v{30}But the Pharisees and the experts in the Law rejected God's plan for themselves\fnote{\fbackref{7:30} Or \fbib{God's decision in their case}} by refusing to be baptized by John.\fnote{\fbackref{7:30} Lit. \fbib{him}}

\v{31}Jesus continued,\fnote{\fbackref{7:31} The Gk. lacks \fbib{Jesus continued}} \red{``To what may I compare the people living today?}\fnote{\fbackref{7:31} Lit. \fbib{people of this generation}} \v{32}\red{They're like little children who sit in the marketplace and shout to each other,}

\begin{poetry}
\poeml \red{`A wedding song we played for you,} \\
\poemll    \red{the dance you all did scorn.} \\
\poeml \red{A woeful dirge we chanted, too,} \\
\poemll    \red{but then you would not mourn.'}
\end{poetry}

\v{33}\red{Because John the Baptist has come neither eating bread nor drinking wine, yet you say, `He has a demon!'} \v{34}\red{The Son of Man has come eating and drinking, and you say, `Look! He's\fnote{\fbackref{7:34} The Gk. lacks \fbib{He's}} a glutton and a drunk, a friend of tax collectors and sinners!'} \v{35}\red{Wisdom is vindicated by all\fnote{\fbackref{7:35} Other mss. lack \fbib{all}} her children.''}
\passage{Jesus Forgives a Sinful Woman}

\v{36}Now one of the Pharisees invited Jesus\fnote{\fbackref{7:36} Lit. \fbib{him}} to eat with him. So he went to the Pharisee's home and took his place at the table. \v{37}There was a woman who was a notorious\fnote{\fbackref{7:37} The Gk. lacks \fbib{notorious}} sinner in that city. When she learned that Jesus\fnote{\fbackref{7:37} Lit. \fbib{he}} was eating at the Pharisee's home, she took an alabaster jar of perfume \v{38}and knelt at his feet behind him. She was crying and began to wash his feet with her tears and dry them with her hair.\fnote{\fbackref{7:38} Lit. \fbib{the hair of her head}} Then she kissed his feet over and over again, anointing them constantly with the perfume.

\v{39}Now the Pharisee who had invited Jesus\fnote{\fbackref{7:39} Lit. \fbib{him}} saw this and told himself, ``If this man were a prophet, he would have known who is touching him and what kind of woman she is. She's a sinner!''

\v{40}Jesus told him, \red{``Simon, I have something to ask you.''}

``Teacher,'' he replied, ``ask it.''

\v{41}\red{``Two men were in debt to a moneylender. One owed him 500 denarii,\fnote{\fbackref{7:41} A denarius was the usual day's wage for a laborer.} and the other 50.} \v{42}\red{When they couldn't pay it back, he generously canceled the debts for both of them. Now which of them will love him more?''}

\v{43}Simon answered, ``I suppose the one who had the larger debt canceled.''

Jesus\fnote{\fbackref{7:43} Lit. \fbib{He}} told him, \red{``You have answered correctly.''}

\v{44}Then, turning to the woman, he told Simon, \red{``Do you see this woman? I came into your house. You didn't give me any water for my feet, but this woman has washed my feet with her tears and dried them with her hair.} \v{45}\red{You didn't give me a kiss,\fnote{\fbackref{7:45} People customarily greeted their friends with a kiss.} but this woman, from the moment I came in, has not stopped kissing my feet.} \v{46}\red{You didn't anoint my head with oil, but this woman has anointed my feet with perfume.} \v{47}\red{So I'm telling you that her sins, as many as they are, have been forgiven, and that's why she has shown such great love. But the one to whom little is forgiven loves little.''}

\v{48}Then Jesus\fnote{\fbackref{7:48} Lit. \fbib{he}} told her, \red{``Your sins are forgiven!''}

\v{49}Those who were at the table with them began to say among themselves, ``Who is this man who even forgives sins?''

\v{50}But Jesus\fnote{\fbackref{7:50} Lit. \fbib{he}} told the woman, \red{``Your faith has saved you. Go in peace.''}
\labelchapt{8}
\passage{Some Women Accompany Jesus}

\chapt{8}
\v{1}After this, Jesus\fnote{\fbackref{8:1} Lit. \fbib{he}} traveled from one city and village to another, preaching and spreading the good news about God's kingdom. The Twelve were with him, \v{2}as well as some women who had been healed of evil spirits and illnesses: Mary, also called Magdalene,\fnote{\fbackref{8:2} I.e. \fbib{Mary of Magdala}} from whom seven demons had gone out; \v{3}Joanna, the wife of Herod's household manager Chuza; Susanna; and many others. These women\fnote{\fbackref{8:3} Lit. \fbib{They}} continued to support them\fnote{\fbackref{8:3} Other mss. read \fbib{him}} out of their personal resources.
\passage{The Parable about a Farmer}
\passageinfo{(Matthew 13:1-9; Mark 4:1-9)}

\v{4}Now while a large crowd was gathering and people were coming to Jesus\fnote{\fbackref{8:4} Lit. \fbib{him}} from every city, he said in a parable: \v{5}\red{``A farmer went out to sow his seed. As he was sowing, some seeds fell along the path, were trampled on, and birds from the sky ate them up.} \v{6}\red{Others fell on stony ground, and as soon as they came up, they dried up because they had no moisture.} \v{7}\red{Others fell among thorn bushes, and the thorn bushes grew with them and choked them.} \v{8}\red{But others fell on good soil, and when they came up, they produced 100 times as much as was planted.''} As he said this, he called out, \red{``Let the person who has ears to hear, listen!''}
\passage{The Purpose of the Parables}
\passageinfo{(Matthew 13:10-17; Mark 4:10-12)}

\v{9}Then his disciples began to ask him what this parable meant. \v{10}So he said, \red{``You have been given knowledge about the secrets of the kingdom of God. But to others they are given\fnote{\fbackref{8:10} The Gk. lacks \fbib{they are given}} in parables, so that}

\begin{poetry}
\poeml \red{`they might look but not see,} \\
\poemll    \red{and they might listen but not understand.'\,''}\fnote{\fbackref{8:10} Cf. Isa 6:9-10}
\end{poetry}
\passage{Jesus Explains the Parable about the Farmer}
\passageinfo{(Matthew 13:18-23; Mark 4:13-20)}

\v{11}\red{``Now this is what the parable means. The seed is God's word.} \v{12}\red{The ones on the path are the people who listen, but then the devil comes and takes the word away from their hearts, so that they may not believe and be saved.} \v{13}\red{The ones on the stony ground are the people who joyfully welcome the word when they hear it. But since they don't have any roots, they believe for a while, but in a time of testing they fall away.} \v{14}\red{The ones that fell among the thorn bushes are the people who listen, but as they go on their way they are choked by the worries, wealth, and pleasures of life, and their fruit doesn't mature.} \v{15}\red{But the ones on the good soil are the people who hear the word but also hold on to it with good and honest hearts, producing a crop through endurance.''}
\passage{A Light under a Bowl}
\passageinfo{(Mark 4:21-25)}

\v{16}\red{``No one lights a lamp and hides it under a bowl or puts it under a bed. Instead, he puts it on a lamp stand so that those who come in will see the light.} \v{17}\red{There is nothing hidden that won't be revealed, and there is nothing secret that won't become known and come to light.} \v{18}\red{So pay attention to how you listen, because to the one who has something, more will be given. However, from the one who doesn't have, even what he thinks he has will be taken away from him.''}
\passage{The True Family of Jesus}
\passageinfo{(Matthew 12:46-50; Mark 3:31-35)}

\v{19}His mother and his brothers came to him, but they couldn't get near him because of the crowd. \v{20}Jesus\fnote{\fbackref{8:20} Lit. \fbib{He}} was told, ``Your mother and your brothers are standing outside and want to see you.''

\v{21}But he answered those people,\fnote{\fbackref{8:21} Lit. \fbib{them}} \red{``My mother and my brothers are those who hear a message from God and heed it.''}
\passage{Jesus Calms the Sea}
\passageinfo{(Matthew 8:23-27; Mark 4:35-41)}

\v{22}One day, Jesus\fnote{\fbackref{8:22} Lit. \fbib{he}} and his disciples got into a boat. He told them, \red{``Let's cross to the other side of the lake.''} So they started out.

\v{23}Now as they were sailing, Jesus\fnote{\fbackref{8:23} Lit. \fbib{he}} fell asleep. A violent storm swept over the lake, and they were taking on water and were in great danger. \v{24}So his disciples\fnote{\fbackref{8:24} Lit. \fbib{So they}} went to him, woke him up, and kept telling him, ``Master! Master! We're going to die!'' He got up and rebuked the wind and the raging waves. They stopped, and there was calm.

\v{25}Then he asked the disciples,\fnote{\fbackref{8:25} Lit. \fbib{them}} \red{``Where's your faith?''}

Frightened and amazed, they asked one another, ``Who is this man? He commands even the winds and the water, and they obey him!''
\passage{Jesus Heals a Demon-Possessed Man}
\passageinfo{(Matthew 8:28-34; Mark 5:1-20)}

\v{26}They landed in the region of the Gerasenes,\fnote{\fbackref{8:26} Other mss. read \fbib{Gadarenes}; still other mss. read \fbib{Gergesenes}} which is just across the lake from Galilee. \v{27}When Jesus\fnote{\fbackref{8:27} Lit. \fbib{he}} stepped out on the shore, a man from the city met him. This man was controlled by\fnote{\fbackref{8:27} Lit. \fbib{He had}} demons and had not worn clothes for a long time. He did not live in a house but in the tombs. \v{28}When he saw Jesus, he screamed, fell down in front of him, and cried out in a loud voice, ``What do you want from me, Jesus, Son of the Most High God? I beg you not to torture me!'' \v{29}because Jesus\fnote{\fbackref{8:29} Lit. \fbib{he}} was in the process of ordering the unclean spirit to come out of the man. On many occasions the unclean spirit\fnote{\fbackref{8:29} Lit. \fbib{it}} had seized the man,\fnote{\fbackref{8:29} Lit. \fbib{him}} and though he was kept under guard and bound with chains and shackles, he would break the chains and be driven by the demon into deserted places.

\v{30}Jesus asked the man,\fnote{\fbackref{8:30} Lit. \fbib{him}} \red{``What's your name?''}

He answered, ``Legion,''\fnote{\fbackref{8:30} A Roman legion consisted of about 6,000 men.} because many demons had gone into him. \v{31}Then the demons\fnote{\fbackref{8:31} Lit. \fbib{they}} began begging Jesus\fnote{\fbackref{8:31} Lit. \fbib{him}} not to order them to go into the bottomless pit.\fnote{\fbackref{8:31} I.e. the realm of punishment in the afterlife}

\v{32}Now a large herd of pigs was grazing there on the hillside. So the demons\fnote{\fbackref{8:32} Lit. \fbib{they}} begged Jesus\fnote{\fbackref{8:32} Lit. \fbib{him}} to let them go into those pigs, and he consented to that. \v{33}Then the demons came out of the man and went into the pigs, and the herd rushed down the cliff into the lake and drowned.

\v{34}Now when those who had been taking care of the pigs saw what had happened, they ran away and reported it in the city and in the countryside. \v{35}So the people\fnote{\fbackref{8:35} Lit. \fbib{they}} went out to see what had happened. When they came to Jesus and found the man from whom the demons had gone out sitting at Jesus' feet, dressed and in his right mind, they were frightened. \v{36}The people who had seen it told them how the demon-possessed man had been healed. \v{37}Then all the people from the region surrounding the Gerasenes\fnote{\fbackref{8:37} Other mss. read \fbib{Gadarenes}; still other mss. read \fbib{Gergesenes}} asked Jesus\fnote{\fbackref{8:37} Lit. \fbib{him}} to leave them, because they were terrified. So he got into a boat and started back.

\v{38}Now the man from whom the demons had gone out kept begging Jesus\fnote{\fbackref{8:38} Lit. \fbib{him}} to let him go with him. But Jesus\fnote{\fbackref{8:38} Lit. \fbib{he}} sent him away, saying, \v{39}\red{``Go home and tell what God has done for you.''} So the man\fnote{\fbackref{8:39} Lit. \fbib{he}} left and kept proclaiming throughout the whole city how much Jesus had done for him.
\passage{Jesus Heals a Woman and Resurrects a Girl}
\passageinfo{(Matthew 9:18-26; Mark 5:21-43)}

\v{40}When Jesus came back, the crowd welcomed him, because everyone was expecting him. \v{41}Just then a synagogue leader by the name of Jairus arrived. He fell at Jesus' feet and kept begging him to come to his home, \v{42}because his only daughter, who was about twelve years old, was dying. While Jesus\fnote{\fbackref{8:42} Lit. \fbib{he}} was on his way, the crowds continued to press in on him.

\v{43}A woman was there\fnote{\fbackref{8:43} The Gk. lacks \fbib{was there}} who had been suffering from chronic bleeding for twelve years. Although she had spent all she had on doctors,\fnote{\fbackref{8:43} Other mss. lack \fbib{Though she had spent all she had on doctors}} no one could heal her. \v{44}She came up behind Jesus\fnote{\fbackref{8:44} Lit. \fbib{him}} and touched the tassel of his garment, and her bleeding stopped at once.

\v{45}Jesus asked, \red{``Who touched me?''}

While everyone was denying it, Peter and those who were with him\fnote{\fbackref{8:45} Other mss. lack \fbib{and those who were with him}} said, ``Master, the crowds are surrounding you and pressing in on you.''

\v{46}Still Jesus said, \red{``Somebody touched me, because I know that power has gone out of me.''}

\v{47}When the woman saw that she couldn't hide, she came forward trembling. Bowing down in front of him, she explained in the presence of all the people why she had touched Jesus\fnote{\fbackref{8:47} Lit. \fbib{him}} and how she had been instantly healed. \v{48}Then he told her, \red{``Daughter, your faith has made you well. Go in peace.''}

\v{49}While he was still speaking, someone came from the synagogue leader's home\fnote{\fbackref{8:49} Lit. \fbib{from the synagogue leader}} and told him,\fnote{\fbackref{8:49} The Gk. lacks \fbib{him}} ``Your daughter is dead. Stop bothering the teacher anymore.''

\v{50}But when Jesus heard this, he told the synagogue leader,\fnote{\fbackref{8:50} Lit. \fbib{him}} \red{``Stop being afraid! Just believe, and she will get well.''}

\v{51}When he arrived at the man's\fnote{\fbackref{8:51} The Gk. lacks \fbib{man's}} house, he allowed no one to go in with him except Peter, John, James, and the young girl's father and mother. \v{52}Now everyone was crying and wailing for her. But Jesus\fnote{\fbackref{8:52} Lit. \fbib{he}} said, \red{``Stop crying! She's not dead. She's sleeping.''} \v{53}They laughed and laughed at him, because they knew she was dead. \v{54}But he took her hand and called out, \red{``Young lady, get up!''} \v{55}So her spirit returned, and she got up at once. Then Jesus\fnote{\fbackref{8:55} Lit. \fbib{he}} directed that she be given something to eat. \v{56}Her parents were amazed, but he ordered them not to tell anyone what had happened.
\labelchapt{9}
\passage{Jesus Sends Out the Twelve}
\passageinfo{(Matthew 10:5-15; Mark 6:7-13)}

\chapt{9}
\v{1}Jesus\fnote{\fbackref{9:1} Lit. \fbib{He}} called the Twelve together and gave them power and authority over all the demons and to heal diseases. \v{2}Then he sent them to proclaim the kingdom of God and to heal the sick. \v{3}He told them, \red{``Don't take anything along on your trip---no walking stick, traveling bag, bread, money, or even an extra shirt.}\fnote{\fbackref{9:3} Lit. \fbib{two shirts}} \v{4}\red{When you visit a home and stay there, and go out from there,} \v{5}\red{if people don't welcome you, when you leave that city, shake its dust off your feet as a testimony against them.''} \v{6}So they left and went from village to village, spreading the good news and healing diseases everywhere.
\passage{Herod Tries to See Jesus}
\passageinfo{(Matthew 14:1-12; Mark 6:14-29)}

\v{7}Now Herod the tetrarch heard about everything that was happening. He was puzzled because it was said by some that John had been raised from the dead, \v{8}by others that Elijah had appeared, and by still others that one of the ancient prophets had come back to life. \v{9}Herod said, ``I beheaded John. But who is this man I'm hearing so much about?'' So Herod\fnote{\fbackref{9:9} Lit. \fbib{he}} kept trying to see Jesus.\fnote{\fbackref{9:9} Lit. \fbib{him}}
\passage{Jesus Feeds More than Five Thousand People}
\passageinfo{(Matthew 14:13-21; Mark 6:30-44; John 6:1-14)}

\v{10}The apostles came back and told Jesus\fnote{\fbackref{9:10} Lit. \fbib{him}} everything they had done. Then he took them away with him privately to a city called Bethsaida. \v{11}But the crowds found out about this and followed him. He welcomed them and began to speak to them about the kingdom of God and to heal those who needed healing.

\v{12}As the day was drawing to a close, the Twelve came to him and said, ``Send the crowd away to the neighboring villages and farms so they can rest and get some food, because we are here in a deserted place.''

\v{13}But he told them, \red{``You give them something to eat.''}

They replied, ``We have nothing more than five loaves of bread and two fish---unless we go and buy food for all these people.''

\v{14}Now there were about 5,000 men. So he told his disciples, \red{``Have them sit down in groups of about 50.''} \v{15}They did this and got all of them seated. \v{16}Taking the five loaves and the two fish, he looked up to heaven and blessed them. Then he broke the loaves in pieces and kept giving them to the disciples to pass on to the crowd. \v{17}All of them ate and were filled. When they collected the leftover pieces, there were twelve baskets.
\passage{Peter Declares His Faith in Jesus}
\passageinfo{(Matthew 16:13-19; Mark 8:27-29)}

\v{18}One day, while Jesus\fnote{\fbackref{9:18} Lit. \fbib{he}} was praying privately and the disciples were with him, he asked them, \red{``Who do the crowds say I am?''}

\v{19}They answered, ``Some say\fnote{\fbackref{9:19} The Gk. lacks \fbib{Some say}} John the Baptist, others Elijah, and still others one of the ancient prophets who has come back to life.''

\v{20}He asked them, \red{``But who do you say I am?''}

``God's Messiah,''\fnote{\fbackref{9:20} Or \fbib{Christ}} Peter replied.
\passage{Jesus Predicts His Death and Resurrection}
\passageinfo{(Matthew 16:21-28; Mark 8:31-9:1)}

\v{21}He gave them strict orders, commanding them not to tell this to anyone. \v{22}He said, \red{``The Son of Man must suffer a great deal and be rejected by the elders, the high priests, and the scribes. Then he must be killed, but on the third day he will be raised.''}

\v{23}Then he told all of them, \red{``If anyone wants to come with me, he must deny himself, pick up his cross every day, and follow me continuously,} \v{24}\red{because whoever wants to save his life will lose it, but whoever loses his life for my sake will save it.} \v{25}\red{What profit will a person have if he gains the whole world, but destroys himself or is lost?} \v{26}\red{If anyone is ashamed of me and my words, the Son of Man will be ashamed of him when he comes in his glory and the glory of\fnote{\fbackref{9:26} Lit. \fbib{and that of}} the Father and the holy angels.} \v{27}\red{I tell you with certainty, some people who are standing here wom't experience\fnote{\fbackref{9:27} Lit. \fbib{taste}} death until they see the kingdom of God.''}
\passage{Jesus' Appearance is Changed}
\passageinfo{(Matthew 17:1-8; Mark 9:2-8)}

\v{28}Now about eight days after Jesus said this,\fnote{\fbackref{9:28} Lit. \fbib{after these sayings}} he took Peter, John, and James with him and went up on a mountain to pray. \v{29}While he was praying, the appearance of his face changed, and his clothes turned dazzling white. \v{30}Suddenly, two men were talking with him. They were Moses and Elijah. \v{31}They had a glorified appearance, and were discussing Jesus'\fnote{\fbackref{9:31} Lit. \fbib{his}} departure that he would shortly bring about in Jerusalem.

\v{32}Now Peter and the men with him had been overcome by sleep. When they woke up, they saw Jesus'\fnote{\fbackref{9:32} Lit. \fbib{his}} glory and the two men standing with him. \v{33}Just as Moses and Elijah\fnote{\fbackref{9:33} Lit. \fbib{Just as they}} were leaving,\fnote{\fbackref{9:33} Lit. \fbib{leaving him}} Peter told Jesus, ``Master, it's good that we're here! Let's set up three shelters\fnote{\fbackref{9:33} Or \fbib{tents}}---one for you, one for Moses, and one for Elijah.'' (Peter\fnote{\fbackref{9:33} Lit. \fbib{He}} didn't know what he was saying.) \v{34}But while he was saying this, a cloud appeared and surrounded them, and they became terrified as they were being overshadowed by the cloud.

\v{35}Then a voice came out of the cloud and said, ``This is my Son, whom I have chosen.\fnote{\fbackref{9:35} Other mss. read \fbib{whom I love}} Keep listening to him!'' \v{36}After the voice had spoken, Jesus was\fnote{\fbackref{9:36} Lit. \fbib{was found to be}} alone. The disciples\fnote{\fbackref{9:36} Lit. \fbib{They}} kept silent and at that time\fnote{\fbackref{9:36} Lit. \fbib{in those days}} told no one about what they had seen.
\passage{Jesus Heals a Boy with a Demon}
\passageinfo{(Matthew 17:14-18; Mark 9:14-27)}

\v{37}The next day, when they had come down from the mountain, a large crowd met Jesus.\fnote{\fbackref{9:37} Lit. \fbib{him}} \v{38}Suddenly, a man in the crowd shouted, ``Teacher, I beg you to look at my son, because he is my only child.\fnote{\fbackref{9:38} Lit. \fbib{only one}} \v{39}Without warning a spirit takes control of him, and he suddenly screams, goes into convulsions, and foams at the mouth. The spirit\fnote{\fbackref{9:39} Lit. \fbib{It}} mauls him and refuses to leave him. \v{40}I begged your disciples to drive it out, but they couldn't.''

\v{41}Jesus answered, \red{``You unbelieving and perverted generation! How much longer must I be with you\fnote{\fbackref{9:41} The Gk. \fbib{you} is pl.} and put up with you?\fnote{\fbackref{9:41} The Gk. \fbib{you} is pl.} Bring your\fnote{\fbackref{9:41} The Gk. \fbib{your} is sing.} son here!''} \v{42}Even while the boy\fnote{\fbackref{9:42} Lit. \fbib{he}} was coming, the demon knocked him to the ground and threw him into convulsions. But Jesus rebuked the unclean spirit, healed the boy, and gave him back to his father.
\passage{Jesus Again Predicts His Death and Resurrection}
\passageinfo{(Matthew 17:22-23; Mark 9:30-32)}

\v{43}All the people continued to be amazed at the greatness of God. Indeed, everyone was astonished at all the things Jesus\fnote{\fbackref{9:43} Lit. \fbib{he}} was doing. So he told his disciples, \v{44}\red{``Listen carefully to these words.\fnote{\fbackref{9:44} Lit. \fbib{Put these words into your ears}} The Son of Man is going to be betrayed into human hands.''} \v{45}But they didn't know what this meant. Indeed, the meaning was hidden from them so that they didn't understand it; and they were afraid to ask him about this statement.
\passage{True Greatness}
\passageinfo{(Matthew 18:1-5; Mark 9:33-37)}

\v{46}Later, an argument started among the disciples\fnote{\fbackref{9:46} Lit. \fbib{among them}} as to which of them might be the greatest. \v{47}But Jesus, knowing their inner thoughts, took a little child and had him stand beside him. \v{48}Then he told them, \red{``Whoever welcomes this little child in my name welcomes me, and whoever welcomes me welcomes the one who sent me, because the one who is least among all of you is the one who is greatest.''}
\passage{The Person who is Not against You is for You}
\passageinfo{(Mark 9:38-40)}

\v{49}John said, ``Master, we saw someone driving out demons in your name. We tried to stop him, because he wasn't a follower like us.''

\v{50}Jesus told him, \red{``Don't stop him! Because whoever is not against you is for you.''}
\passage{A Samaritan Village Refuses to Welcome Jesus}

\v{51}When the days grew closer for Jesus\fnote{\fbackref{9:51} Lit. \fbib{for him}} to be taken up to heaven,\fnote{\fbackref{9:51} The Gk. lacks \fbib{to heaven}} he was determined to continue his journey to Jerusalem. \v{52}So he sent messengers on ahead of him. On their way they went into a Samaritan village to get things ready for him. \v{53}But the people\fnote{\fbackref{9:53} Lit. \fbib{they}} would not welcome him, because he was determined to go to Jerusalem. \v{54}When his disciples James and John observed this rejection,\fnote{\fbackref{9:54} The Gk. lacks \fbib{rejection}} they asked, ``Lord, do you want us to call fire down from heaven to destroy them?''\fnote{\fbackref{9:54} Other mss. read \fbib{them, as Elijah did?''}} \v{55}But he turned and rebuked them,\fnote{\fbackref{9:55} Other mss. read \fbib{them, saying, ``You don't know what kind of spirit you are!} \fbib{\v{56}For the son of man did not come to destroy the souls of men, but to save them.''}} \v{56}and they all\fnote{\fbackref{9:56} The Gk. lacks \fbib{all}} went on to another village.
\passage{The Would-be Followers of Jesus}
\passageinfo{(Matthew 8:19-22)}

\v{57}While they were walking along the road, a man told him, ``I will follow you wherever you go.''

\v{58}Jesus told him,

\begin{poetry}
\poeml \red{``Foxes have holes and birds\fnote{\fbackref{9:58} Lit. \fbib{birds in the sky}} have nests,} \\
\poemll    \red{but the Son of Man has no place to rest.''}\fnote{\fbackref{9:58} Lit. \fbib{no place to lay his head}}
\end{poetry}

\v{59}He told another man, \red{``Follow me.''}

But he said, ``Lord,\fnote{\fbackref{9:59} Other mss. lack \fbib{Lord}} first let me go and bury my father.''

\v{60}But he told him, \red{``Let the dead bury their own dead. But you go and proclaim the kingdom of God.''}

\v{61}Still another man said, ``I will follow you, Lord, but first let me say goodbye to those at home.''

\v{62}Jesus told him, \red{``No one who puts his hand to the plow and looks back is fit for the kingdom of God.''}
\labelchapt{10}
\passage{The Mission of the Seventy}

\chapt{10}
\v{1}After this, the Lord appointed 70\fnote{\fbackref{10:1} Other mss. read \fbib{72}} other disciples\fnote{\fbackref{10:1} Lit. \fbib{others}} and was about to send them ahead of him in pairs to every town and place that he intended to go. \v{2}So he instructed them, \red{``The harvest is vast, but the workers are few. So ask the Lord of the harvest to send workers out into his harvest.} \v{3}\red{Get going! See, I am sending you out like lambs among wolves.} \v{4}\red{Don't carry a wallet, a traveling bag, or sandals, and don't greet anyone on the way.}

\v{5}\red{``Whatever house you go into, first say, `May there be peace in this house.'} \v{6}\red{If a peaceful person lives there, your greeting of peace will remain with him. But if that's not the case, your greeting\fnote{\fbackref{10:6} Lit. \fbib{it}} will come back to you.} \v{7}\red{Stay with the same family, eating and drinking whatever they provide, because the worker deserves his pay. Don't move from house to house.}

\v{8}\red{``Whenever you go into a town and the people\fnote{\fbackref{10:8} Lit. \fbib{they}} welcome you, eat whatever they serve you,} \v{9}\red{heal the sick that are there, and tell them, `The kingdom of God is near you!'} \v{10}\red{But whenever you go into a town and people\fnote{\fbackref{10:10} Lit. \fbib{they}} don't welcome you, go out into its streets and say,} \v{11}\red{`We're wiping off your town's dust that clings to our feet in protest against you! But realize this: the kingdom of God is near!'} \v{12}\red{I tell you, on the last\fnote{\fbackref{10:12} The Gk. lacks \fbib{last}} day it will be easier for Sodom than for that town!''}
\passage{Jesus Denounces Unrepentant Cities}
\passageinfo{(Matthew 11:20-24)}

\v{13}\red{``How terrible it will be for you, Chorazin! How terrible it will be for you, Bethsaida! If the miracles that happened in you had taken place in Tyre and Sidon, they would have repented long ago, sitting in sackcloth and ashes.} \v{14}\red{It will be easier for Tyre and Sidon at the judgment than for you!}

\v{15}\red{And you, Capernaum! You won't be lifted up to heaven, will you? You'll go down to Hell!}\fnote{\fbackref{10:15} Lit. \fbib{Hades}; i.e. the realm of the dead} \v{16}\red{The person who listens to you listens to me, and the person who rejects you rejects me. The person who rejects me rejects the one who sent me.''}
\passage{The Return of the Seventy}

\v{17}The 70\fnote{\fbackref{10:17} Other mss. read \fbib{72}} disciples\fnote{\fbackref{10:17} The Gk. lacks \fbib{disciples}} came back and joyously reported, ``Lord, even the demons are submitting to us in your name!''

\v{18}He told them, \red{``I watched Satan falling from heaven like lightning.} \v{19}\red{Look! I have given you the authority to trample on snakes and scorpions and to destroy\fnote{\fbackref{10:19} Lit. \fbib{and over}} all the enemy's power, and nothing will ever hurt you.} \v{20}\red{However, stop rejoicing because the spirits are submitting to you. Instead, rejoice because your names are written in heaven.''}
\passage{Jesus Praises the Father}
\passageinfo{(Matthew 11:25-27; 13:16-17)}

\v{21}At that moment, the Holy Spirit\fnote{\fbackref{10:21} Other mss. read \fbib{in the spirit}} made Jesus\fnote{\fbackref{10:21} Lit. \fbib{him}} extremely joyful, so Jesus said, \red{``I praise you, Father, Lord of heaven and earth, because you have hidden these things from wise and intelligent people and have revealed them to infants. Yes, Father, because this is what was pleasing to you.} \v{22}\red{All things have been entrusted to me by my Father. No one knows who the Son is except the Father, and no one knows\fnote{\fbackref{10:22} The Gk. lacks \fbib{no one knows}} who the Father is except the Son and the person to whom the Son chooses to reveal him.''}

\v{23}Then turning to his disciples in private, he told them, \red{``How blessed are the eyes that see what you see!} \v{24}\red{Because I tell you, many prophets and kings wanted to see the things you see but didn't see them, and to hear the things you hear but didn't hear them.''}
\passage{The Good Samaritan}

\v{25}Just then an expert in the Law stood up to test Jesus.\fnote{\fbackref{10:25} Lit. \fbib{him}} He asked, ``Teacher, what must I do to inherit eternal life?''

\v{26}Jesus\fnote{\fbackref{10:26} Lit. \fbib{He}} answered him, \red{``What is written in the Law? What do you read there?''}

\v{27}He answered, ``You must love the Lord your God with all your heart, with all your soul, with all your strength, and with all your mind.\fnote{\fbackref{10:27} Cf. Deut 6:5} And you must love\fnote{\fbackref{10:27} The Gk. lacks \fbib{you must love}} your neighbor as yourself.''\fnote{\fbackref{10:27} Cf. Lev 19:18}

\v{28}Jesus\fnote{\fbackref{10:28} Lit. \fbib{He}} told him, \red{``You have answered correctly. `Do this, and you will live.'\,''}\fnote{\fbackref{10:28} Cf. Gen 42:18}

\v{29}But the man wanted to justify himself, so he asked Jesus, ``And who is my neighbor?''

\v{30}After careful consideration, Jesus replied, \red{``A man was going down from Jerusalem to Jericho when he fell into the hands of bandits. They stripped him, beat him, and went away, leaving him half dead.} \v{31}\red{By chance, a priest was traveling along that road. When he saw the man,\fnote{\fbackref{10:31} Lit. \fbib{him}} he went by on the other side.} \v{32}\red{Similarly, a descendant of Levi came to that place. When he saw the man,\fnote{\fbackref{10:32} Lit. \fbib{him}} he also went by on the other side.} \v{33}\red{But as he was traveling along, a Samaritan came across the man.\fnote{\fbackref{10:33} Lit. \fbib{him}} When the Samaritan\fnote{\fbackref{10:33} Lit. \fbib{he}} saw him, he was moved with compassion.} \v{34}\red{He went to him and bandaged his wounds, pouring oil and wine on them. Then he put him on his own animal, brought him to an inn, and took care of him.} \v{35}\red{The next day, he took out two denarii\fnote{\fbackref{10:35} A denarius was the usual day's wage for a laborer.} and gave them to the innkeeper, saying, `Take good care of him. If you spend more than that, I'll repay you when I come back.'}

\v{36}\red{``Of these three men, who do you think was a neighbor to the man who fell into the hands of the bandits?''}

\v{37}He said, ``The one who showed mercy to him.''

Jesus told him, \red{``Go and do what he did.''}
\passage{Jesus Visits Mary and Martha}

\v{38}Now as they were traveling along, Jesus\fnote{\fbackref{10:38} Lit. \fbib{he}} went into a village. A woman named Martha welcomed him into her home. \v{39}She had a sister named Mary, who sat down at the Lord's feet and kept listening to what he was saying. \v{40}But Martha was worrying about all the things she had to do, so she came to him and asked, ``Lord, you do care that my sister has left me to do the work all by myself, don't you? Then tell her to help me.''

\v{41}The Lord answered her, \red{``Martha, Martha! You worry and fuss about a lot of things.} \v{42}\red{But there's only\fnote{\fbackref{10:42} Other mss. read \fbib{out of a few things, there's only}} one thing you need. Mary has chosen what is better,\fnote{\fbackref{10:42} Lit. \fbib{the better part}} and it is not to be taken away from her.''}
\labelchapt{11}
\passage{Teaching about Prayer}
\passageinfo{(Matthew 6:9-15; 7:7-11)}

\chapt{11}
\v{1}Once Jesus\fnote{\fbackref{11:1} Lit. \fbib{he}} was praying in a certain place. After he had finished, one of his disciples told him, ``Lord, teach us to pray, as John taught his disciples.''

\v{2}So he told them, \red{``Whenever you pray you are to say,}

\begin{poetry}
\poeml \red{`Father,\fnote{\fbackref{11:2} Other mss. read \fbib{Our Father in heaven}} may your name be kept holy.} \\
\poemll    \red{May your kingdom come.}\fnote{\fbackref{11:2} Other mss. read \fbib{kingdom come. May your will be done, on earth as it is in heaven.}} \\
\poeml \v{3}\red{Keep giving us every day our daily bread,}\fnote{\fbackref{11:3} Or \fbib{our bread from above}} \\
\poeml \v{4}\red{and forgive us our sins,} \\
\poeml \red{as we forgive everyone who sins against us.}\fnote{\fbackref{11:4} Lit. \fbib{is indebted to us}} \\
\poeml \red{And never bring us into temptation.'\,''}\fnote{\fbackref{11:4} Other mss. read \fbib{into temptation, but deliver us from the evil one}}
\end{poetry}

\v{5}Then he told them, \red{``Suppose one of you has a friend, and you go to him at midnight and say to him, `Friend, let me borrow three loaves of bread.} \v{6}\red{A friend of mine on a trip has dropped in on me, and I don't have anything to serve him.'} \v{7}\red{Suppose he answers from inside, `Stop bothering me! The door is already locked, and my children are here with us in the bedroom.\fnote{\fbackref{11:7} Lit. \fbib{with me in the bed}} I can't get up and give you anything!'} \v{8}\red{I tell you, even though that man\fnote{\fbackref{11:8} Lit. \fbib{though he}} doesn't want to get up and give him anything because he is his friend, he will get up and give him whatever he needs because of his persistence.}
\passage{Ask, Search, Knock}
\passageinfo{(Matthew 7:7-12)}

\v{9}\red{So I say to you: Keep asking, and it will be given you. Keep searching, and you will find. Keep knocking, and the door\fnote{\fbackref{11:9} Lit. \fbib{it}} will be opened for you,} \v{10}\red{because everyone who keeps asking will receive, and the person who keeps searching will find, and the person who keeps knocking will have the door\fnote{\fbackref{11:10} Lit. \fbib{it}} opened.} \v{11}\red{What father among you, if his son asks for bread, would give him a stone, or if he asks for a fish,\fnote{\fbackref{11:11} Other mss. read \fbib{What father among you, if his son asks for a fish}} would give him a snake instead of the fish?} \v{12}\red{Or if he asks for an egg, would he give him a scorpion?} \v{13}\red{So if you who are evil know how to give good gifts to your children, how much more will the Father in heaven give the Holy Spirit to those who keep asking him!''}
\passage{Jesus is Accused of Working with Beelzebul}
\passageinfo{(Matthew 12:22-30; Mark 3:20-27)}

\v{14}Jesus\fnote{\fbackref{11:14} Lit. \fbib{He}} was driving a demon out of a man who was\fnote{\fbackref{11:14} Lit. \fbib{driving out a demon that was}} unable to talk. When the demon had gone out, the man\fnote{\fbackref{11:14} Lit. \fbib{the man who was unable to talk}} began to speak, and the crowds were amazed. \v{15}But some of them said, ``He drives out demons by Beelzebul, the ruler of the demons.'' \v{16}Others, wanting to test Jesus,\fnote{\fbackref{11:16} Lit. \fbib{him}} kept asking him for a sign from heaven.

\v{17}Since he knew what they were thinking, he told them, \red{``Every kingdom divided against itself is devastated, and a divided household collapses.}\fnote{\fbackref{11:17} Lit. \fbib{and house falls on house}} \v{18}\red{Now, if Satan is divided against himself, how can his kingdom last? After all, you say that I drive out demons by Beelzebul.} \v{19}\red{If I drive out demons by Beelzebul, by whom do your own followers\fnote{\fbackref{11:19} Lit. \fbib{sons}} drive them out? That is why they will be your judges!} \v{20}\red{But if I drive out demons by the power\fnote{\fbackref{11:20} Lit. \fbib{finger}} of God, then the kingdom of God has come to you.}

\v{21}\red{``When a strong man, fully armed, guards his own mansion, his property is safe.} \v{22}\red{But when a stronger man than he attacks and defeats him, the stronger man\fnote{\fbackref{11:22} Lit. \fbib{him, he}} strips off that man's armor in which he trusted and then divides his plunder.}

\v{23}\red{``The person who isn't with me is against me, and the person who doesn't gather with me scatters.''}
\passage{The Return of the Unclean Spirit}
\passageinfo{(Matthew 12:43-45)}

\v{24}\red{``Whenever an unclean spirit goes out of a person, it wanders through dry places looking for a place to rest but doesn't find any. So it says, `I will go back to my home that I left.'} \v{25}\red{When it gets back home, it finds it swept clean and put in order.} \v{26}\red{Then it goes and brings with it seven other spirits more evil than itself, and they all go in and settle there. And so the final condition of that person is worse than the first.''}
\passage{True Blessedness}

\v{27}As Jesus\fnote{\fbackref{11:27} Lit. \fbib{he}} was saying this, a woman in the crowd raised her voice and told him, ``How blessed is the womb that gave birth to you and the breasts that nursed you!''

\v{28}But he said, \red{``Instead, how blessed are those who hear God's word and obey it!''}
\passage{The Sign of Jonah}
\passageinfo{(Matthew 12:38-42; Mark 8:12)}

\v{29}Now as the crowds continued to throng around Jesus,\fnote{\fbackref{11:29} Lit. \fbib{him}} he went on to say, \red{``This people living today are\fnote{\fbackref{11:29} Lit. \fbib{This generation is}} an evil generation. It craves a sign, but no sign will be given to it except the sign of Jonah,} \v{30}\red{because just as Jonah became a sign\fnote{\fbackref{11:30} The Gk. lacks \fbib{a sign}} to the people of Nineveh, so the Son of Man will be a sign to this generation.} \v{31}\red{The queen of the south will stand up at the judgment and condemn the people living today,\fnote{\fbackref{11:31} Lit. \fbib{condemn this generation}} because she came from the ends of the earth to hear the wisdom of Solomon. But look, something greater than Solomon is here!} \v{32}\red{The men of Nineveh will stand up at the judgment and condemn the people living today,\fnote{\fbackref{11:32} Lit. \fbib{condemn this generation}} because they repented at the preaching of Jonah. But look, something greater than Jonah is here!''}
\passage{The Lamp of the Body}
\passageinfo{(Matthew 5:15; 6:22-23)}

\v{33}\red{``No one lights a lamp and puts it in a hiding place\fnote{\fbackref{11:33} Or \fbib{cellar}} or under a basket,\fnote{\fbackref{11:33} Other mss. lack \fbib{or under a basket}} but on a lamp stand, so that those who enter may see its light.} \v{34}\red{Your eye is the lamp of your body. When your eye is healthy, your whole body is full of light. But when it's evil, your body is full of darkness.} \v{35}\red{Therefore, be careful that the light in you isn't darkness.} \v{36}\red{Now if your whole body is full of light, with no part of it in darkness, it will be as full of light as when a lamp gives you light with its rays.''}
\passage{Jesus Denounces the Pharisees and the Experts in the Law}
\passageinfo{(Matthew 23:1-36; Mark 12:38-40; Luke 20:45-47)}

\v{37}After Jesus\fnote{\fbackref{11:37} Lit. \fbib{he}} had said this, a Pharisee invited him to have a meal with him. So Jesus\fnote{\fbackref{11:37} Lit. \fbib{he}} went and took his place at the table. \v{38}The Pharisee was surprised to see that he didn't first wash before the meal. \v{39}But the Lord told him, \red{``Now you Pharisees clean the outside of the cup and the dish, but on the inside you are full of greed and evil.} \v{40}\red{You fools! The one who made the outside made the inside, too, didn't he?} \v{41}\red{So give what is inside to the poor, and then everything will be clean for you.}

\v{42}\red{``How terrible it will be for you Pharisees! You give a tenth of your mint, spices, and every kind of herb, but you neglect justice and the love of God. These are the things you should have practiced, without neglecting the others.}

\v{43}\red{How terrible it will be for you Pharisees! You love to have the places of honor in the synagogues and to be greeted in the marketplaces.}

\v{44}\red{How terrible it will be for you! You are like unmarked graves---people walk on them without realizing it.''}

\v{45}Then one of the experts in the Law told him, ``Teacher, when you say these things, you insult us, too.''

\v{46}Jesus\fnote{\fbackref{11:46} Lit. \fbib{He}} said, \red{``How terrible it will be for you experts in the Law, too! You load people with burdens that are hard to carry, yet you don't even lift a finger to ease those burdens.}

\v{47}\red{How terrible it will be for you! You build monuments for the prophets, and it was your ancestors who killed them!} \v{48}\red{So you are witnesses and approve of the deeds of your ancestors, because they killed those for whom you are building monuments.} \v{49}\red{That is why the Wisdom of God said, `I will send them prophets and apostles. They will kill some of them and persecute others,'}\fnote{\fbackref{11:49} The source of this quotation is unknown.} \v{50}\red{so those living today\fnote{\fbackref{11:50} Lit. \fbib{\v{50}that this generation}} will be charged with the blood of all the prophets that was shed since the foundation of the world,} \v{51}\red{from the blood of Abel\fnote{\fbackref{11:51} Cf. Gen 4:8} to the blood of Zechariah,\fnote{\fbackref{11:51} Cf. 2Chr 24:20-21} who died between the altar and the sanctuary. Yes, I tell you, it will be charged against this generation!}

\v{52}\red{How terrible it will be for you experts in the Law! You have taken away the key to knowledge. You didn't go in yourselves, and you kept out those who were trying to go in.''}

\v{53}As Jesus\fnote{\fbackref{11:53} Lit. \fbib{he}} was leaving, the scribes and the Pharisees began to oppose him fiercely, interrogating him about many things. \v{54}They watched him closely in an effort to trap him in something he might say.
\labelchapt{12}
\passage{A Warning against Hypocrisy}

\chapt{12}
\v{1}Meanwhile, the people\fnote{\fbackref{12:1} Lit. \fbib{crowd}} had gathered by the thousands and were trampling on one another. Jesus\fnote{\fbackref{12:1} Lit. \fbib{He}} began to speak first to his disciples. \red{``Watch out for the yeast---that is, the hypocrisy---of the Pharisees!} \v{2}\red{There is nothing covered up that won't be exposed and nothing secret that won't be made known.} \v{3}\red{Therefore, what you have said in darkness will be heard in the daylight, and what you have whispered\fnote{\fbackref{12:3} Lit. \fbib{spoken in the ear}} in private rooms will be shouted from the housetops.''}
\passage{Fear God}
\passageinfo{(Matthew 10:28-31)}

\v{4}\red{``But I tell you, my friends, never be afraid of those who kill the body and after that can't do anything more.} \v{5}\red{I'll show you the one you should be afraid of. Be afraid of the one who has the authority to throw you into hell\fnote{\fbackref{12:5} Lit. \fbib{Gehenna}; a transliteration of the Heb. for \fbib{Valley of Hinnom}} after killing you. Yes, I tell you, be afraid of him!}

\v{6}\red{``Five sparrows are sold for two pennies, aren't they? Yet not one of them is forgotten by God.} \v{7}\red{Why, even all the hairs on your head have been counted! Stop being afraid. You are worth more than a bunch of sparrows.''}
\passage{Acknowledging the Messiah}
\passageinfo{(Matthew 10:32-33; 12:32; 10:19-20)}

\v{8}\red{``But I tell you, the Son of Man will acknowledge before God's angels everyone who acknowledges me before people.} \v{9}\red{But whoever denies me before people will be denied before God's angels.} \v{10}\red{Everyone who speaks a word against the Son of Man will be forgiven, but the person who blasphemes against the Holy Spirit won't be forgiven.} \v{11}\red{When people\fnote{\fbackref{12:11} Lit. \fbib{they}} bring you before synagogue leaders,\fnote{\fbackref{12:11} Lit. \fbib{synagogues}} rulers, or authorities, don't worry about how you will defend yourselves or what\fnote{\fbackref{12:11} Lit. \fbib{how or what}} you will say,} \v{12}\red{because at that time the Holy Spirit will teach you what you are to say.''}
\passage{The Parable of the Rich Fool}

\v{13}Then someone in the crowd told him, ``Teacher, tell my brother to divide the family inheritance with me.''

\v{14}But Jesus\fnote{\fbackref{12:14} Lit. \fbib{he}} asked him, \red{``Mister,\fnote{\fbackref{12:14} Lit. \fbib{Man}} who appointed me to be a judge or arbitrator over you people?''} \v{15}Then he told them, \red{``Be careful to guard yourselves against every kind of greed, because a person's life doesn't consist of the amount of possessions he has.''}

\v{16}Then he told them a parable. He said, \red{``The land of a certain rich man produced good crops.} \v{17}\red{So he began to think to himself, `What should I do, since I have no place to store my crops?'} \v{18}\red{Then he said, `This is what I'll do. I'll tear down my barns and build bigger ones, and I'll store all my grain and goods in them.} \v{19}\red{Then I'll say to myself, ``You've stored up plenty of good things for many years. Take it easy, eat, drink, and enjoy yourself.''\,'} \v{20}\red{But God told him, `You fool! This very night your life will be demanded back from you. Now who will get the things you've accumulated?'} \v{21}\red{That's how it is with the person who stores up treasures for himself rather than with God.''}
\passage{Stop Worrying}
\passageinfo{(Matthew 6:25-34, 19-21)}

\v{22}Then Jesus\fnote{\fbackref{12:22} Lit. \fbib{he}} told his disciples, \red{``That's why I'm telling you to stop worrying about your life---what you will eat---or about your body---what you will wear,} \v{23}\red{because life is more than food, and the body more than clothing.} \v{24}\red{Consider the crows.\fnote{\fbackref{12:24} Or \fbib{ravens}} They don't plant or harvest, they don't even have a storeroom or barn, yet God feeds them. How much more valuable are you than birds!} \v{25}\red{Can any of you add an hour to the length of your life\fnote{\fbackref{12:25} Or \fbib{add one cubit to your height}; a cubit was about eighteen inches} by worrying?} \v{26}\red{So if you can't do a small thing like that, why worry about other things?} \v{27}\red{Consider how the lilies grow. They don't work or spin yarn, but I tell you that not even Solomon in all his splendor was clothed like one of them.} \v{28}\red{Now if that's the way God clothes the grass in the field, which is alive today and thrown into an oven tomorrow, how much more will he clothe you---you who have little faith?}

\v{29}\red{``So stop concerning yourselves about what you will eat or what you will drink, and stop being distressed,} \v{30}\red{because it is the unbelievers\fnote{\fbackref{12:30} Lit. \fbib{gentiles} ; i.e. unbelieving non-Jews} who are concerned about all these things. Surely your Father knows that you need them!} \v{31}\red{Instead, be concerned about his\fnote{\fbackref{12:31} Other mss. read \fbib{God's}} kingdom, and these things will be provided for you as well.} \v{32}\red{Stop being afraid, little flock, because your Father is pleased to give you the kingdom.}

\v{33}\red{``Sell your possessions, and give the money to the poor. Make yourselves wallets that don't wear out---a dependable treasure in heaven, where no thief can get close and no moth can destroy anything.} \v{34}\red{Because where your treasure is, there your heart will be also.''}
\passage{The Watchful Servants}
\passageinfo{(Matthew 24:45-51)}

\v{35}\red{``You must keep your belts fastened and your lamps burning.} \v{36}\red{Be like people who are waiting for their master to return from a wedding. As soon as he arrives and knocks, they will open the door for him.} \v{37}\red{How blessed are those servants whom the master finds watching for him when he comes! I tell all of you\fnote{\fbackref{12:37} The Gk. pronoun \fbib{you} is pl.} with certainty, he himself will put on an apron, make them sit down at the table, and go around and serve them.} \v{38}\red{How blessed they will be if their master\fnote{\fbackref{12:38} Lit. \fbib{if he}} comes in the middle of the night or near dawn\fnote{\fbackref{12:38} Lit. \fbib{in the second or the third watch}} and finds them awake!}\fnote{\fbackref{12:38} Lit. \fbib{finds them so}} \v{39}\red{But be sure of this: if the homeowner had known at what time the thief were coming, he would have watched and\fnote{\fbackref{12:39} Other mss. lack \fbib{would have watched and}} would not have let his house be broken into.} \v{40}\red{So be ready, because the Son of Man is coming at a time when you don't expect him.''}

\v{41}Peter asked, ``Lord, are you telling this parable just for us or for everyone?''

\v{42}The Lord said, \red{``Who, then, is the faithful and careful servant manager whom his master will put in charge of giving all his other servants their share of food at the right time?} \v{43}\red{How blessed is that servant whom his master finds doing this when he comes!} \v{44}\red{I tell you with certainty, he will put him in charge of all his property.} \v{45}\red{But if that servant says to himself,\fnote{\fbackref{12:45} Lit. \fbib{in his heart}} `My master is taking a long time to come back,' and begins to beat the other servants and to eat, drink, and get drunk,} \v{46}\red{the master of that servant will come on a day when he doesn't expect him and at an hour that he doesn't know. Then his master\fnote{\fbackref{12:46} Lit. \fbib{he}} will punish him severely\fnote{\fbackref{12:46} Lit. \fbib{cut him in pieces}} and assign him a place with unfaithful people.} \v{47}\red{That servant who knew what his master wanted but didn't prepare himself or do what was wanted will receive a severe beating.} \v{48}\red{But the servant\fnote{\fbackref{12:48} Lit. \fbib{the one}} who did things that deserved a beating without knowing it will receive a light beating. Much will be required from everyone to whom much has been given. But even more will be demanded from the one to whom much has been entrusted.''}
\passage{Not Peace, but Division}
\passageinfo{(Matthew 10:34-36)}

\v{49}\red{``I've come to set the earth on fire, and how I wish it were already ablaze!} \v{50}\red{I have a baptism to be baptized with, and what stress I am under until it's completed!} \v{51}\red{Do you think that I came to bring peace on earth? Not at all, I tell you, but rather division!} \v{52}\red{From now on, five people in one household will be divided, three against two and two against three.} \v{53}\red{They will be divided father against son, son against father, mother against daughter, daughter against mother, mother-in-law against daughter-in-law, and daughter-in-law against mother-in-law.''}
\passage{Interpreting the Time}
\passageinfo{(Matthew 16:2-3; Mark 8:11-13)}

\v{54}Then Jesus\fnote{\fbackref{12:54} Lit. \fbib{he}} told the crowds, \red{``When you see a cloud coming in from the west, you immediately say, `There's going to be a storm,' and that's what happens.} \v{55}\red{When you see a south wind blowing, you say, `It's going to be hot,' and so it is.} \v{56}\red{You hypocrites! You know how to interpret the appearance of the earth and the sky, yet you don't know how to interpret the present time?''}
\passage{Settling with Your Opponent}
\passageinfo{(Matthew 5:25-26)}

\v{57}\red{``Why don't you judge for yourselves what is right?} \v{58}\red{For example, when you go with your opponent in front of a ruler, do your best to settle with him on the way there. Otherwise, you will be dragged in front of the judge, and the judge will hand you over to an officer, and the officer will throw you into prison.} \v{59}\red{I tell you, you will never get out of there until you pay back the last penny!''}
\labelchapt{13}
\passage{Repent or Die}

\chapt{13}
\v{1}At that time, some people who were there told Jesus\fnote{\fbackref{13:1} Lit. \fbib{him}} about the Galileans whose blood Pilate had mixed with their sacrifices.\fnote{\fbackref{13:1} I.e. whom Pilate had executed while they were sacrificing animals} \v{2}He asked them, \red{``Do you think that these Galileans were more sinful than all the other Galileans because they suffered like this?} \v{3}\red{Absolutely not, I tell you! But if you don't repent, then you, too, will all die.} \v{4}\red{What about those eighteen people who were killed when the tower at Siloam fell on them? Do you think they were worse offenders than all the other people living in Jerusalem?} \v{5}\red{Absolutely not, I tell you! But if you don't repent, then you, too, will all die.''}
\passage{The Parable about an Unfruitful Fig Tree}

\v{6}Then Jesus\fnote{\fbackref{13:6} Lit. \fbib{he}} told them this parable: \red{``A man had a fig tree that had been planted in his vineyard. He went to look for fruit on it but didn't find any.} \v{7}\red{So he told the gardener, `Look here! For three years I've been coming to look for fruit on this tree but I haven't found any. Cut it down! Why should it waste the soil?'} \v{8}\red{But the gardener\fnote{\fbackref{13:8} Lit. \fbib{he}} replied, `Sir, leave it alone for one more year, until I dig around it and fertilize it.} \v{9}\red{Maybe next year it will bear fruit. If not, then cut it down.'\,''}
\passage{Jesus Heals a Woman on the Sabbath}

\v{10}Jesus\fnote{\fbackref{13:10} Lit. \fbib{He}} was teaching in one of the synagogues on the Sabbath. \v{11}A woman was there who had a spirit that had disabled her for eighteen years. She was hunched over and completely unable to stand up straight. \v{12}When Jesus saw her, he called to her and said, \red{``Woman, you are free from your illness.''} \v{13}Then he placed his hands on her, and immediately she stood up straight and began praising God.

\v{14}But the synagogue leader, indignant because Jesus had healed on the Sabbath, told the crowd, ``There are six days when work is to be done. So come on those days to be healed, and not on the Sabbath day.''

\v{15}The Lord replied to him, \red{``You hypocrites! Doesn't each of you on the Sabbath untie his ox or donkey and lead it out of its stall to give it some water?} \v{16}\red{Shouldn't this woman, a descendant of Abraham whom Satan has kept bound for eighteen long years, be set free from this bondage on the Sabbath day?''} \v{17}Even as he was saying this, all of his opponents were blushing with shame. But the rest of the crowd was rejoicing at all the wonderful things he was doing.
\passage{The Parables about a Mustard Seed and Yeast}
\passageinfo{(Matthew 13:31-33; Mark 4:30-32)}

\v{18}So Jesus\fnote{\fbackref{13:18} Lit. \fbib{he}} went on to say, \red{``What is the kingdom of God like? What can I compare it to?} \v{19}\red{It is like a mustard seed that someone took and planted in his garden. It grew and became a tree, and the birds of the sky nested in its branches.''}

\v{20}Again he said, \red{``To what can I compare the kingdom of God?} \v{21}\red{It's like yeast that a woman took and mixed with\fnote{\fbackref{13:21} Lit. \fbib{hid in}} three measures of flour until all of it was leavened.''}
\passage{The Narrow Door}
\passageinfo{(Matthew 7:13-14, 21-23)}

\v{22}Then Jesus\fnote{\fbackref{13:22} Lit. \fbib{he}} taught in one town and village after another as he made his way to Jerusalem. \v{23}Someone asked him, ``Lord,\fnote{\fbackref{13:23} Or \fbib{Sir}} are only a few people going to be saved?''

He told them, \v{24}\red{``Keep on struggling to enter through the narrow door, because I tell you that many people will try to enter, but won't be able to do so.} \v{25}\red{After the homeowner gets up and closes the door, you can stand\fnote{\fbackref{13:25} Lit. \fbib{begin to stand}} outside, knock on the door, and say again and again, `Lord, open the door for us!' But he will answer you, `I don't know where you come from.'} \v{26}\red{Then you will say,\fnote{\fbackref{13:26} Lit. \fbib{begin to say}} `We ate and drank with you, and you taught in our streets.'} \v{27}\red{But he will tell you, `I don't know where you come from. Get away from me, all you who practice evil!'} \v{28}\red{In that place there will be crying and gnashing of teeth\fnote{\fbackref{13:28} I.e. extreme pain} when you see Abraham, Isaac, Jacob, and all the prophets in the kingdom of God, and you yourselves being driven away on the outside.} \v{29}\red{People will come from east and west, and from north and south, and will eat in the kingdom of God.} \v{30}\red{You see, some who are last will be first, and some who are first will be last.}
\passage{Jesus Rebukes Jerusalem}
\passageinfo{(Matthew 23:37-39)}

\v{31}At that hour some Pharisees came and told Jesus,\fnote{\fbackref{13:31} Lit. \fbib{him}} ``Leave and get away from here, because Herod wants to kill you!''

\v{32}He told them, \red{``Go and tell that fox, `Listen! I am driving out demons and healing today and tomorrow, and on the third day I will finish my work.} \v{33}\red{But I must be on my way today, tomorrow, and the next day, because it's not possible for a prophet to be killed outside of Jerusalem.'} \v{34}\red{O Jerusalem, Jerusalem, who kills the prophets and stones to death those who have been sent to her! How often I wanted to gather your children together as a hen gathers her chicks under her wings, but you people were unwilling!} \v{35}\red{Look! Your house is left vacant to you. I tell you, you will not see me again until you say, `How blessed is the one who comes in the name of the Lord!'\,''}\fnote{\fbackref{13:35} Cf. Ps 118:26; MT source citation reads \fbib{\divine{Lord}}}
\labelchapt{14}
\passage{Jesus Heals a Man on the Sabbath}

\chapt{14}
\v{1}One Sabbath, Jesus\fnote{\fbackref{14:1} Lit. \fbib{he}} went to the house of a leader of the Pharisees to eat a meal. The guests\fnote{\fbackref{14:1} Lit. \fbib{They}} were watching Jesus\fnote{\fbackref{14:1} Lit. \fbib{him}} closely. \v{2}A man whose body was swollen with fluid suddenly appeared in front of him. \v{3}So Jesus asked the Pharisees and experts in the Law, \red{``Is it lawful to heal on the Sabbath or not?''} \v{4}But they kept silent. So he took hold of the man,\fnote{\fbackref{14:4} Lit. \fbib{him}} healed him, and sent him away. \v{5}Then he asked them, \red{``If your son\fnote{\fbackref{14:5} Other mss. read \fbib{donkey}; still other mss. read \fbib{sheep}} or ox falls into a well on the Sabbath day, you would pull him out immediately, wouldn't you?''} \v{6}And they couldn't argue with him about this.
\passage{A Lesson about Guests}

\v{7}When Jesus\fnote{\fbackref{14:7} Lit. \fbib{he}} noticed how the guests were choosing the places of honor, he told them a parable. \v{8}\red{``When you are invited by someone to a wedding banquet, don't sit down at the place of honor in case someone more important than you was invited by the host.}\fnote{\fbackref{14:8} Lit. \fbib{by him}} \v{9}\red{Then the host who invited both of you would come to you and say, `Give this person your place.' In disgrace, you would have to take the place of least honor.} \v{10}\red{But when you are invited, go and sit down at the place of least honor. Then, when your host comes, he will tell you, `Friend, move up higher,' and you will be honored in the presence of everyone who eats with you.} \v{11}\red{Because everyone who exalts himself will be humbled, but the person who humbles himself will be exalted.''}

\v{12}Then he told the man who had invited him, \red{``When you give a luncheon or a dinner, stop inviting only\fnote{\fbackref{14:12} The Gk. lacks \fbib{only}} your friends, brothers, relatives, or rich neighbors. Otherwise, they may invite you in return and you would be repaid.} \v{13}\red{Instead, when you give a banquet, make it your habit to invite the poor, the crippled, the lame, and the blind.} \v{14}\red{Then you will be blessed because they can't repay you. And you will be repaid when the righteous are resurrected.''}
\passage{The Parable about a Banquet}
\passageinfo{(Matthew 22:1-10)}

\v{15}Now one of those eating with him heard this and told him, ``How blessed is the person who will eat\fnote{\fbackref{14:15} Lit. \fbib{eat bread}} in the kingdom of God!''

\v{16}Jesus\fnote{\fbackref{14:16} Lit. \fbib{He}} told him, \red{``A man gave a large banquet and invited many people.} \v{17}\red{When it was time for the banquet, he sent his servant to tell those who were invited, `Come! Everything is now ready.'} \v{18}\red{Every single one of them began asking to be excused. The first told him, `I bought a field, and I need to go out and inspect it. Please excuse me.'} \v{19}\red{Another said, `I bought five pairs of oxen, and I'm on my way to try them out. Please excuse me.'} \v{20}\red{Still another said, `I recently got married, so I can't come.'}

\v{21}\red{``So the servant went back and reported all this to his master. Then the master of the house became angry and told his servant, `Go quickly into the streets and alleys of the town and bring back the poor, the crippled, the blind, and the lame.'} \v{22}\red{The servant said, `Sir, what you ordered has been done, and there is still room.'} \v{23}\red{Then the master told the servant, `Go out into the streets and the lanes and make the people come in, so that my house may be full.} \v{24}\red{Because I tell all of you, none of those men who were invited will taste anything at my banquet.'\,''}
\passage{The Cost of Discipleship}
\passageinfo{(Matthew 10:37-39)}

\v{25}Now large crowds were traveling with Jesus.\fnote{\fbackref{14:25} Lit. \fbib{him}} He turned and told them, \v{26}\red{``If anyone comes to me and does not hate his father, mother, wife, children, brothers, and sisters, as well as his own life, he can't be my disciple.} \v{27}\red{Whoever doesn't carry his cross and follow me can't be my disciple.}

\v{28}\red{``Suppose one of you wants to build a tower. He will first sit down and estimate the cost to see whether he has enough money to finish it, won't he?} \v{29}\red{Otherwise, if he lays a foundation and can't finish the building,\fnote{\fbackref{14:29} The Gk. lacks \fbib{the building}} everyone who watches will begin to ridicule him} \v{30}\red{and say, `This person started a building but couldn't finish it.'}

\v{31}\red{``Or suppose a king is going to war against another king. He will first sit down and consider whether with 10,000 men he can fight the one coming against him with 20,000 men, won't he?} \v{32}\red{If he can't, he will send a delegation to ask for terms of peace while the other king\fnote{\fbackref{14:32} Lit. \fbib{while he}} is still far away.} \v{33}\red{In the same way, none of you can be my disciple unless he gives up all his possessions.''}
\passage{Tasteless Salt}
\passageinfo{(Matthew 5:13; Mark 9:50)}

\v{34}\red{``Now, salt is good. But if the salt should lose its taste, how can its flavor be restored?} \v{35}\red{It's suitable neither for the soil nor for the manure pile. People\fnote{\fbackref{14:35} Lit. \fbib{They}} throw it away. Let the person who has ears to hear, listen!''}
\labelchapt{15}
\passage{The Parable about the Faithful Shepherd}
\passageinfo{(Matthew 18:12-14)}

\chapt{15}
\v{1}Now all the tax collectors and sinners kept coming to listen to Jesus.\fnote{\fbackref{15:1} Lit. \fbib{him}} \v{2}But the Pharisees and the scribes kept complaining, ``This man welcomes sinners and eats with them.'' \v{3}So he told them this parable:

\v{4}\red{``Suppose one of you has 100 sheep and loses one of them. He leaves the 99 in the wilderness and looks for the one that is lost until he finds it, doesn't he?} \v{5}\red{When he finds it, he puts it on his shoulders and rejoices.} \v{6}\red{Then he goes home, calls his friends and neighbors together, and says to them, `Rejoice with me, because I've found my lost sheep!'} \v{7}\red{In the same way, I tell you that there will be more joy in heaven over one sinner who repents than over 99 righteous people who don't need to repent.''}
\passage{The Story of the Diligent Housewife}

\v{8}\red{``Or suppose a woman has ten coins and loses one of them.\fnote{\fbackref{15:8} Lit. \fbib{one coin}} She lights a lamp, sweeps the house, and searches carefully until she finds it, doesn't she?} \v{9}\red{When she finds it, she calls her friends and neighbors together and says, `Rejoice with me, because I have found the coin that I lost!'} \v{10}\red{In the same way, I tell you that there is joy in the presence of God's angels over one sinner who repents.''}
\passage{The Story of the Loving Father}

\v{11}Then Jesus\fnote{\fbackref{15:11} Lit. \fbib{he}} said, \red{``A man had two sons.} \v{12}\red{The younger one told his father, `Father, give me my share of the estate.' So the father\fnote{\fbackref{15:12} Lit. \fbib{he}} divided his property between them.} \v{13}\red{A few days later, the younger son gathered everything he owned and traveled to a distant country. There he wasted it all\fnote{\fbackref{15:13} Lit. \fbib{wasted his possessions}} on wild living.} \v{14}\red{After he had spent everything, a severe famine took place throughout that country, and he began to be in need.} \v{15}\red{So he went out to work for one of the citizens of that country, who sent him into his fields to feed pigs.} \v{16}\red{No one would give him anything, even though he would gladly have filled himself with the husks the pigs were eating.}

\v{17}\red{``Then he came to his senses and said, `How many of my father's hired men have more food than they can eat, and here I am starving to death!} \v{18}\red{I will get up, go to my father, and say to him, ``Father, I have sinned against heaven\fnote{\fbackref{15:18} I.e. God} and you.} \v{19}\red{I don't deserve to be called your son anymore. Treat me like one of your hired men.''\,'}

\v{20}\red{``So he got up and went to his father. While he was still far away, his father saw him and was filled with compassion. He ran to his son,\fnote{\fbackref{15:20} The Gk. lacks \fbib{to his son}} threw his arms around him, and kissed him affectionately.} \v{21}\red{Then his son told him, `Father, I have sinned against heaven\fnote{\fbackref{15:21} I.e. God} and you. I don't deserve to be called your son anymore.'}\fnote{\fbackref{15:21} Other mss. read \fbib{anymore. Treat me like one of your hired men.}} \v{22}\red{But the father told his servants, `Hurry! Bring out the best robe and put it on him, and put a ring on his finger and sandals on his feet.} \v{23}\red{Bring the fattened calf and kill it, and let's eat and celebrate!} \v{24}\red{Because my son was dead and has come back to life. He was lost and has been found.' And they began to celebrate.}

\v{25}\red{``Now the father's\fnote{\fbackref{15:25} Lit. \fbib{Now his}} older son was in the field. As he was coming back to the house, he heard music and dancing.} \v{26}\red{So he called to one of the servants and asked what was happening.} \v{27}\red{The servant\fnote{\fbackref{15:27} Lit. \fbib{He}} told him, `Your brother has come home, and your father has killed the fattened calf because he got him back safely.'}

\v{28}\red{``Then the older son\fnote{\fbackref{15:28} Lit. \fbib{he}} became angry and wouldn't go into the house.\fnote{\fbackref{15:28} Lit. \fbib{wouldn't go in}} So his father came out and began to plead with him.} \v{29}\red{But he answered his father, `Listen! All these years I've worked like a slave for you. I've never disobeyed a command of yours. Yet you've never given me so much as a young goat for a festival\fnote{\fbackref{15:29} The Gk. lacks \fbib{for a festival}} so I could celebrate with my friends.} \v{30}\red{But this son of yours spent your money on prostitutes, and when he came back, you killed the fattened calf for him!'}

\v{31}\red{``His father\fnote{\fbackref{15:31} Lit. \fbib{He}} told him, `My child, you are always with me, and everything I have is yours.} \v{32}\red{But we had to celebrate and rejoice, because this brother of yours was dead and has come back to life. He was lost and has been found.'\,''}
\labelchapt{16}
\passage{The Parable about a Dishonest Manager}

\chapt{16}
\v{1}Now Jesus\fnote{\fbackref{16:1} Lit. \fbib{he}} was saying to the disciples, \red{``A rich man had a servant manager who was accused of wasting his assets.} \v{2}\red{So he called for him and asked him, `What's this I hear about you? You can't be my manager any longer. Now give me a report about your management!'}

\v{3}\red{``Then the servant manager told himself, `What should I do? My master is taking my position away from me. I'm not strong enough to plow, and I'm ashamed to beg.} \v{4}\red{I know what I'll do so that people\fnote{\fbackref{16:4} Lit. \fbib{they}} will welcome me into their homes when I'm dismissed from my job.'}

\v{5}\red{``So he called for each of his master's debtors. He asked the first, `How much do you owe my master?'} \v{6}\red{The man replied, `A hundred jars of olive oil.' The manager\fnote{\fbackref{16:6} Lit. \fbib{He}} told him, `Get your bill. Sit down quickly and write ``50.''\,'} \v{7}\red{Then he asked another debtor,\fnote{\fbackref{16:7} The Gk. lacks \fbib{debtor}} `How much do you owe?' The man replied, `A hundred containers of wheat.' The manager\fnote{\fbackref{16:7} Lit. \fbib{He}} told him, `Get your bill and write ``80.''\,'} \v{8}\red{The master praised the dishonest servant manager for being so clever, because worldly people\fnote{\fbackref{16:8} Lit. \fbib{the sons of this age}} are more clever than enlightened people\fnote{\fbackref{16:8} Lit. \fbib{the sons of light}} in dealing with their own.}\fnote{\fbackref{16:8} Lit. \fbib{own generation}}

\v{9}\red{``I'm telling you, make friends for yourselves by means of unrighteous wealth, so that when it fails, they will welcome you into eternal homes.}\fnote{\fbackref{16:9} Lit. \fbib{tents}} \v{10}\red{Whoever is faithful with very little is also faithful with a lot, and whoever is dishonest with very little is also dishonest with a lot.} \v{11}\red{So if you haven't been faithful with unrighteous wealth, who will trust you with true wealth?} \v{12}\red{And if you haven't been faithful with what belongs to foreigners, who will give you what is your own?}

\v{13}\red{``No servant can serve two masters, because either he will hate one and love the other, or be loyal to one and despise the other. You cannot serve both God and wealth!''}
\passage{The Law and the Kingdom of God}
\passageinfo{(Matthew 11:12-13)}

\v{14}Now the Pharisees, who love money, had been listening to all this and began to ridicule Jesus.\fnote{\fbackref{16:14} Lit. \fbib{him}} \v{15}So he told them, \red{``You try to justify yourselves in front of people, but God knows your hearts, because what is highly valued by people is detestable to God.}

\v{16}\red{``The Law and the Prophets remained\fnote{\fbackref{16:16} The Gk. lacks \fbib{remained}} until John. Since then, the good news about the kingdom of God has been proclaimed, and everyone entering it is under attack.} \v{17}\red{However, it is easier for heaven and earth to disappear than for one stroke of a letter in the Law to be dropped.} \v{18}\red{Any man who divorces his wife and marries another woman commits adultery, and the man who marries a woman divorced from her husband commits adultery.''}
\passage{The Rich Man and Lazarus}

\v{19}\red{``Once there was a rich man who used to dress in purple and fine linen and live in great luxury every day.} \v{20}\red{A beggar named Lazarus, who was covered with sores, was brought to his gate.} \v{21}\red{He was always trying to satisfy his hunger with what fell\fnote{\fbackref{16:21} Other mss. read \fbib{the scraps that fell}} from the rich man's table. Even the dogs used to come and lick his sores.}

\v{22}\red{``One day, the beggar died and was carried away by the angels to Abraham's side. The rich man also died and was buried.} \v{23}\red{In the afterlife,\fnote{\fbackref{16:23} Lit. \fbib{Hades}, i.e. the realm of the dead} where he was in constant torment, he looked up and saw Abraham far away and Lazarus by his side.} \v{24}\red{So he shouted, `Father Abraham, have mercy on me! Send Lazarus to dip the tip of his finger in water to cool off my tongue, because I am suffering in this fire.'}

\v{25}\red{``But Abraham said, `My child, remember that during your lifetime you received blessings,\fnote{\fbackref{16:25} Lit. \fbib{good things}} while Lazarus received hardships.\fnote{\fbackref{16:25} Lit. \fbib{and Lazarus in like manner evil things}} But now he is being comforted here, while you suffer.} \v{26}\red{Besides all this, a wide chasm has been fixed between us, so that those who want to cross from this side to you cannot do so, nor can they cross from your side to us.'}

\v{27}\red{``The rich man\fnote{\fbackref{16:27} Lit. \fbib{He}} said, `Then I beg you, father, send Lazarus\fnote{\fbackref{16:27} Lit. \fbib{him}} to my father's house---} \v{28}\red{because I have five brothers---to warn them, so that they won't end up in this place of torture, too.'}

\v{29}\red{``Abraham said, `They have Moses and the Prophets. They should listen to them!'}

\v{30}\red{``But the rich man\fnote{\fbackref{16:30} Lit. \fbib{he}} replied, `No, father Abraham! But if someone from the dead went to them, they would repent.'}

\v{31}\red{``Then Abraham\fnote{\fbackref{16:31} Lit. \fbib{he}} told him, `If your brothers\fnote{\fbackref{16:31} Lit. \fbib{If they}} do not listen to Moses and the Prophets, they will not be persuaded, even if someone were to rise from the dead.'\,''}
\labelchapt{17}
\passage{Causing Others to Sin}
\passageinfo{(Matthew 18:6-7, 21-22; Mark 9:42)}

\chapt{17}
\v{1}Jesus\fnote{\fbackref{17:1} Lit. \fbib{He}} told his disciples, \red{``It is inevitable that temptations to sin will come, but how terrible it will be for the person through whom they come!} \v{2}\red{It would be better for him if a millstone were hung around his neck and he were thrown into the sea than for him to cause one of these little ones to sin.}

\v{3}\red{``Watch yourselves! If your brother sins, rebuke him, and if he repents, forgive him.} \v{4}\red{Even if he sins against you seven times in a day and comes back to you seven times and says, `I repent,' you must forgive him.''}
\passage{Faith and Obedience}

\v{5}Then the apostles told the Lord, ``Give us more faith!''

\v{6}The Lord replied, \red{``If you have faith the size of\fnote{\fbackref{17:6} Lit. \fbib{faith of}} a mustard seed, you could say to this mulberry tree, `Be uprooted and planted in the sea,' and it would obey you!}

\v{7}\red{``Suppose a man among you has a servant plowing or watching sheep. Would he say to him when he comes in from the field, `Come at once and have something to eat'?} \v{8}\red{Of course not. Instead, he would say to him, `Get dinner ready for me, and put on your apron and wait on me until I eat and drink. Then you can eat and drink.'} \v{9}\red{He doesn't praise the servant for doing what was commanded, does he?} \v{10}\red{That's the way it is with you. When you have done everything you were ordered to do, say, `We are worthless servants. We have done only what we ought to have done.'\,''}
\passage{Jesus Cleanses Ten Lepers}

\v{11}One day, Jesus\fnote{\fbackref{17:11} Lit. \fbib{he}} was traveling along the border between Samaria and Galilee on the way to Jerusalem. \v{12}As he was going into a village, ten lepers met him. They stood at a distance \v{13}and shouted, ``Jesus, Master, have mercy on us!''

\v{14}When Jesus\fnote{\fbackref{17:14} Lit. \fbib{he}} saw them, he told them, \red{``Go and show yourselves to the priests.''} While they were going, they were made clean. \v{15}But one of them, when he saw that he had been healed, came back and praised God with a loud voice. \v{16}He fell on his face at Jesus'\fnote{\fbackref{17:16} Lit. \fbib{his}} feet and thanked him. Now that man\fnote{\fbackref{17:16} Lit. \fbib{Now he}} was a Samaritan.

\v{17}Jesus asked, \red{``Ten men were made clean, weren't they? Where are the other nine?} \v{18}\red{Except for this foreigner, were any of them found to return and give praise to God?''} \v{19}Then he told the man,\fnote{\fbackref{17:19} Lit. \fbib{him}} \red{``Get up, and go home! Your faith has made you well.''}
\passage{The Coming of the Kingdom}
\passageinfo{(Matthew 24:23-28, 36-41)}

\v{20}Once Jesus\fnote{\fbackref{17:20} Lit. \fbib{he}} was asked by the Pharisees when the kingdom of God would come. He answered them, \red{``The kingdom of God is not coming with a visible display.} \v{21}\red{People\fnote{\fbackref{17:21} Lit. \fbib{They}} won't be saying, `Look! Here it is!' or `There it is!' because now the kingdom of God is among\fnote{\fbackref{17:21} Or \fbib{within}} you.''}

\v{22}Then Jesus\fnote{\fbackref{17:22} Lit. \fbib{he}} told the disciples, \red{``The time will come during which you will long to see one of these days when the Son of Man is with you,\fnote{\fbackref{17:22} The Gk. lacks \fbib{is with you}} but you will not see it.} \v{23}\red{People\fnote{\fbackref{17:23} Lit. \fbib{They}} will say to you, `Look! There he is!' or `Look! Here he is!' But don't go and chase after him.} \v{24}\red{Because just as lightning flashes and shines from one end of the sky to the other, so will the Son of Man be in his time.}\fnote{\fbackref{17:24} Lit. \fbib{day}; other mss. lack \fbib{in his day}} \v{25}\red{But first he must suffer a great deal and be rejected by those living today.}\fnote{\fbackref{17:25} Lit. \fbib{by this generation}}

\v{26}\red{``Just as it was in Noah's time, so it will be in the days of the Son of Man.} \v{27}\red{People\fnote{\fbackref{17:27} Lit. \fbib{They}} were eating, drinking, marrying, and being given in marriage right up to the day when Noah went into the ark. Then the flood came and destroyed all of them.} \v{28}\red{So it was in the days of Lot. People\fnote{\fbackref{17:28} Lit. \fbib{They}} were eating and drinking, buying and selling, planting and building.} \v{29}\red{But on the day when Lot left Sodom, fire and sulfur rained down from heaven and destroyed all of them.} \v{30}\red{The day when the Son of Man is revealed will be like that.}

\v{31}\red{``The person who is on the housetop that day must not come down to get his belongings out of his house. The person in the field, too, must not turn back to what's left behind.} \v{32}\red{Remember Lot's wife!} \v{33}\red{Whoever tries to save his life\fnote{\fbackref{17:33} Other mss. read \fbib{to make his life secure}} will lose it, but whoever loses his life will preserve it.} \v{34}\red{I tell you, two will be seated on the same couch\fnote{\fbackref{17:34} I.e. an armless divan upon which guests reclined to partake of a meal} that night. The one will be taken, and the other will be left behind.} \v{35}\red{Two women will be grinding grain\fnote{\fbackref{17:35} The Gk. lacks \fbib{grain}} together. The one will be taken, and the other will be left behind.''}\fnote{\fbackref{17:35} Other mss. read \fbib{left behind.} \fbib{\v{36}Two people will be in a field. One will be taken, and the other will be left behind}}

\v{37}Then they asked him, ``Where, Lord, will this take place?''\fnote{\fbackref{17:37} The Gk. lacks \fbib{will this take place}}

He told them, \red{``Wherever there's a corpse, there the vultures\fnote{\fbackref{17:37} Or \fbib{eagles}} will gather.''}
\labelchapt{18}
\passage{The Parable about the Judge and the Widow}

\chapt{18}
\v{1}Jesus\fnote{\fbackref{18:1} Lit. \fbib{He}} told his disciples\fnote{\fbackref{18:1} Lit. \fbib{them}} a parable about their need to pray all the time and never give up. \v{2}He said, \red{``In a city there was a judge who didn't fear God or respect people.} \v{3}\red{In that city there was also a widow who kept coming to him and saying, `Grant me justice against my adversary.'} \v{4}\red{For a while the judge\fnote{\fbackref{18:4} Lit. \fbib{he}} refused. But later, he told himself, `I don't fear God or respect people,} \v{5}\red{yet because this widow keeps bothering me, I will grant her justice. Otherwise, she will keep coming and wear me out.'\,''}

\v{6}Then the Lord added, \red{``Listen to what the unrighteous judge says.} \v{7}\red{Won't God grant his chosen people justice when they cry out to him day and night? Is he slow to help them?} \v{8}\red{I tell you, he will give them justice quickly. But when the Son of Man comes, will he find faith on earth?''}
\passage{The Parable about the Pharisee and the Tax Collector}

\v{9}Jesus\fnote{\fbackref{18:9} Lit. \fbib{He}} also told this parable to some people who trusted in themselves, thinking they were righteous, but who looked down on everyone else: \v{10}\red{``Two men went up to the Temple to pray. One was a Pharisee, and the other was a tax collector.} \v{11}\red{The Pharisee stood by himself and prayed, `O God, I thank you that I'm not like other people---thieves, dishonest people, adulterers, or even this tax collector.} \v{12}\red{I fast twice a week, and I give a tenth of my entire income.'}

\v{13}\red{``But the tax collector stood at a distance and would not even look up to heaven. Instead, he continued to beat his chest and said, `O God, be merciful to me, the sinner that I am!'}\fnote{\fbackref{18:13} The Gk. lacks \fbib{that I am}} \v{14}\red{I tell you, this man, rather than the other one, went down to his home justified, because everyone who exalts himself will be humbled, but the person who humbles himself will be exalted.''}
\passage{Jesus Blesses the Little Children}
\passageinfo{(Matthew 19:13-15; Mark 10:13-16)}

\v{15}Now some people\fnote{\fbackref{18:15} Lit. \fbib{they}} were even bringing their infants to Jesus\fnote{\fbackref{18:15} Lit. \fbib{him}} to have him touch them. But when the disciples saw this, they sternly told the people\fnote{\fbackref{18:15} Lit. \fbib{them}} not to do that. \v{16}Jesus, however, called for the children\fnote{\fbackref{18:16} Lit. \fbib{them}} and said, \red{``Let the little children come to me, and stop keeping them away, because the kingdom of God belongs to people like these.} \v{17}\red{I tell all of you\fnote{\fbackref{18:17} The Gk. pronoun \fbib{you} is pl.} with certainty, whoever doesn't receive the kingdom of God as a little child will never get into it at all.''}
\passage{A Rich Man Comes to Jesus}
\passageinfo{(Matthew 19:16-30; Mark 10:17-31)}

\v{18}Then an official asked Jesus,\fnote{\fbackref{18:18} Lit. \fbib{him}} ``Good Teacher, what must I do to inherit eternal life?''

\v{19}\red{``Why do you call me good?''} Jesus asked him. \red{``Nobody is good except for one---God.} \v{20}\red{You know the commandments: `Never commit adultery.\fnote{\fbackref{18:20} Cf. Exod 20:14; Deut 5:18} Never murder.\fnote{\fbackref{18:20} Cf. Exod 20:13; Deut 5:17} Never steal.\fnote{\fbackref{18:20} Cf. Exod 20:15; Deut 5:19} Never give false testimony.\fnote{\fbackref{18:20} Cf. Exod 20:16; Deut 5:20} Honor your father and mother.'\,''}\fnote{\fbackref{18:20} Cf. Exod 20:12; Deut 5:16}

\v{21}The official\fnote{\fbackref{18:21} Lit. \fbib{He}} replied, ``I have kept all of these since I was a young man.''

\v{22}When Jesus heard this, he told him, \red{``You still need to do one thing. Sell everything you have and give the money\fnote{\fbackref{18:22} The Gk. lacks \fbib{the money}} to the destitute, and you will have treasure in heaven. Then come back and follow me.''} \v{23}But when the official\fnote{\fbackref{18:23} Lit. \fbib{he}} heard this he became sad, because he was very rich.
\passage{Salvation and Reward}
\passageinfo{(Matthew 19:23-26; Mark 10:23-31)}

\v{24}So when Jesus saw how sad he was, he\fnote{\fbackref{18:24} Other mss. read \fbib{So Jesus looked at him and}} said, \red{``How hard it is for rich people to get into the kingdom of God!} \v{25}\red{Indeed, it's easier for a camel to squeeze through the eye of a needle than for a rich person to get into the kingdom of God.''}

\v{26}Those who were listening to Jesus\fnote{\fbackref{18:26} Lit. \fbib{him}} asked, ``Then who can be saved?''

\v{27}Jesus\fnote{\fbackref{18:27} Lit. \fbib{He}} replied, \red{``The things that are impossible for people are possible for God.''}

\v{28}Then Peter said, ``See, we have left everything we have and followed you.''

\v{29}Jesus\fnote{\fbackref{18:29} Lit. \fbib{He}} told them, \red{``I tell all of you\fnote{\fbackref{18:29} The Gk. pronoun \fbib{you} is pl.} with certainty, there is no one who has left his home, wife, brothers, parents, or children because of the kingdom of God} \v{30}\red{who will not receive many times as much in this world, as well as eternal life in the age to come.''}
\passage{Jesus Predicts His Death and Resurrection a Third Time}
\passageinfo{(Matthew 20:17-19; Mark 10:32-34)}

\v{31}Jesus\fnote{\fbackref{18:31} Lit. \fbib{He}} took the Twelve aside and told them, \red{``Pay attention! We're going up to Jerusalem. Everything written by the prophets about the Son of Man will be fulfilled,} \v{32}\red{because he'll be handed over to the unbelievers,\fnote{\fbackref{18:32} Lit. \fbib{gentiles} ; i.e. unbelieving non-Jews} and will be mocked, insulted, and spit on.} \v{33}\red{After they have whipped him, they'll kill him, but on the third day he'll rise again.''} \v{34}But they didn't understand any of this. What he said was hidden from them, and they didn't know what he meant.
\passage{Jesus Heals a Blind Man}
\passageinfo{(Matthew 20:29-34; Mark 10:46-50)}

\v{35}As Jesus\fnote{\fbackref{18:35} Lit. \fbib{he}} was approaching Jericho, there was a blind man sitting by the road begging. \v{36}When he heard the crowd going by, he asked what was happening. \v{37}They told him that Jesus from Nazareth\fnote{\fbackref{18:37} Or \fbib{Jesus the Nazarene}; the Gk. \fbib{Nazoraios} may be a word play between Heb. \fbib{netser,} meaning \fbib{branch} (see Isa 11:1), and the name \fbib{Nazareth.}} was coming by. \v{38}So he shouted, ``Jesus, Son of David, have mercy on me!'' \v{39}The people at the front of the crowd\fnote{\fbackref{18:39} The Gk. lacks \fbib{of the crowd}} sternly told him to be quiet, but he started shouting even louder, ``Son of David, have mercy on me!''

\v{40}Then Jesus stopped and ordered the man to be brought to him. When he came near, Jesus\fnote{\fbackref{18:40} Lit. \fbib{he}} asked him, \v{41}\red{``What do you want me to do for you?''}

He said, ``Lord, I want to see again!''

\v{42}So Jesus told him, \red{``See again! Your faith has made you well.''} \v{43}Immediately the man\fnote{\fbackref{18:43} Lit. \fbib{he}} could see again and began to follow Jesus,\fnote{\fbackref{18:43} Lit. \fbib{him}} glorifying God. All the people saw this and gave praise to God.
\labelchapt{19}
\passage{Jesus and Zacchaeus}

\chapt{19}
\v{1}As Jesus\fnote{\fbackref{19:1} Lit. \fbib{he}} entered Jericho and was passing through it, \v{2}a man named Zacchaeus appeared. He was a leading tax collector, and a rich one at that! \v{3}He was trying to see who Jesus was, but he couldn't do so due to the crowd, since he was a short man. \v{4}So he ran ahead and climbed a sycamore tree to see Jesus,\fnote{\fbackref{19:4} Lit. \fbib{him}} who was going to pass that way.

\v{5}When Jesus came to the tree,\fnote{\fbackref{19:5} Lit. \fbib{to the place}} he looked up and said, \red{``Zacchaeus, hurry and come down! I must stay at your house today.''} \v{6}Zacchaeus\fnote{\fbackref{19:6} Lit. \fbib{He}} came down quickly and was glad to welcome him into his home.\fnote{\fbackref{19:6} The Gk. lacks \fbib{into his home}}

\v{7}But all the people who saw this began to complain: ``Jesus\fnote{\fbackref{19:7} Lit. \fbib{He}} is going to be the guest of a notorious\fnote{\fbackref{19:7} The Gk. lacks \fbib{notorious}} sinner!''

\v{8}Later, Zacchaeus stood up and announced to the Lord, ``Look! I'm giving half of my possessions to the destitute, and if I have accused anyone falsely, I'm repaying four times as much as I owe.''\fnote{\fbackref{19:8} The Gk. lacks \fbib{as much as I owe}; cf. Exod 22:1, 2 Sam 12:6}

\v{9}Then Jesus told him, \red{``Today salvation has come to this home, because this man\fnote{\fbackref{19:9} Lit. \fbib{because he}} is also a descendant of Abraham,} \v{10}\red{and the Son of Man has come to seek and to save the lost.''}
\passage{The Parable about the Coins}

\v{11}As they were listening to this, Jesus\fnote{\fbackref{19:11} Lit. \fbib{he}} went on to tell a parable because he was near Jerusalem and because the people\fnote{\fbackref{19:11} Lit. \fbib{they}} thought that the kingdom of God would appear immediately. \v{12}So he said, \red{``A prince went to a distant country to be appointed king and then to return.} \v{13}\red{He called ten of his servants and gave them ten coins.\fnote{\fbackref{19:13} Lit. \fbib{minas.} A mina was equivalent to about eight months of wages for a laborer.} He told them, `Invest this money until I come back.'} \v{14}\red{But the citizens of his country hated him and sent a delegation to follow him and to announce, `We don't want this man to rule over us!'}

\v{15}\red{``After he was appointed king, the prince\fnote{\fbackref{19:15} Lit. \fbib{he}} came back. He ordered the servants to whom he had given the money to be called so he could find out what they had earned by investing.} \v{16}\red{The first servant\fnote{\fbackref{19:16} The Gk. lacks \fbib{servant}} came and said, `Sir, your coin has earned ten more coins.'} \v{17}\red{The king\fnote{\fbackref{19:17} Lit. \fbib{He}} told him, `Well done, good servant! Because you have been trustworthy in a very small thing, take charge of ten cities.'}

\v{18}\red{``The second servant\fnote{\fbackref{19:18} The Gk. lacks \fbib{servant}} came and said, `Your coin, sir, has earned five coins.'} \v{19}\red{The king\fnote{\fbackref{19:19} Lit. \fbib{He}} told him, `You take charge of five cities.'}

\v{20}\red{``Then the other servant\fnote{\fbackref{19:20} The Gk. lacks \fbib{servant}} came and said, `Sir, look! Here's your coin. I've kept it in a cloth for safekeeping} \v{21}\red{because I was afraid of you. You are a hard man. You withdraw what you didn't deposit and harvest what you didn't plant.'} \v{22}\red{The king\fnote{\fbackref{19:22} Lit. \fbib{He}} told him, `I will judge you by your own words, you evil servant! You knew, did you, that I was a hard man, and that I withdraw what I didn't deposit and harvest what I didn't plant?} \v{23}\red{Then why didn't you put my money in the bank? When I returned, I could have collected it with interest.'}

\v{24}\red{``So the king\fnote{\fbackref{19:24} Lit. \fbib{he}} told those standing nearby, `Take the coin away from him and give it to the man who has the ten coins.'} \v{25}\red{They answered him, `Sir, he already\fnote{\fbackref{19:25} The Gk. lacks \fbib{already}} has ten coins!'} \v{26}\red{`I tell you, to everyone who has something, more will be given, but from the person who has nothing, even what he has will be taken away.} \v{27}\red{But as for these enemies of mine who didn't want me to be their king---bring them here and slaughter them in my presence!'\,''}
\passage{The King Enters Jerusalem}
\passageinfo{(Matthew 21:1-11; Mark 11:1-11; John 12:12-19)}

\v{28}After Jesus\fnote{\fbackref{19:28} Lit. \fbib{he}} had said this, he traveled on and went up to Jerusalem. \v{29}As he approached Bethphage and Bethany at the Mount of Olives, he sent two of his disciples on ahead. \v{30}\red{``Go into the village ahead of you,''} he said. \red{``As you enter, you will find a colt tied up that no one has ever ridden.\fnote{\fbackref{19:30} Lit. \fbib{ever sat on}} Untie it and bring it along.} \v{31}\red{If anyone asks you why you are untying it, say this: `The Lord needs it.'\,''}

\v{32}So those who were sent went off and found it as Jesus\fnote{\fbackref{19:32} Lit. \fbib{he}} had told them. \v{33}While they were untying the colt, its owners asked them, ``Why are you untying the colt?''

\v{34}The disciples\fnote{\fbackref{19:34} Lit. \fbib{They}} answered, ``The Lord needs it.'' \v{35}Then they brought the colt to Jesus and put their coats on it, and Jesus sat upon it.

\v{36}As he was riding along, people\fnote{\fbackref{19:36} Lit. \fbib{they}} kept spreading their coats on the road. \v{37}He was now approaching the descent from the Mount of Olives. The whole crowd of disciples began to rejoice and to praise God with a loud voice because of all the miracles they had seen. \v{38}They said,

\begin{poetry}
\poeml ``How blessed is the king \\
\poemll    who comes in the name of the Lord!\fnote{\fbackref{19:38} Cf. Ps 118:26; MT source citation reads \fbib{\divine{Lord}}} \\
\poeml Peace in heaven, \\
\poemll    and glory in the highest heaven!''
\end{poetry}

\v{39}Some of the Pharisees in the crowd told Jesus,\fnote{\fbackref{19:39} Lit. \fbib{him}} ``Teacher, tell your disciples to be quiet.''

\v{40}He replied, \red{``I tell you, if they were quiet, the stones would cry out!''}

\v{41}When he came closer and saw the city, he began to grieve over it: \v{42}\red{``If you\fnote{\fbackref{19:42} I.e. the city of Jerusalem personified; and so throughout the paragraph} had only known today what could have brought you peace! But now it is hidden from your sight,} \v{43}\red{because the days will come\fnote{\fbackref{19:43} Lit. \fbib{come on you}} when your enemies will build walls around you, surround you, and close you in on every side.} \v{44}\red{They will level you to the ground---you and those who live\fnote{\fbackref{19:44} Lit. \fbib{and your children}} within your city limits.\fnote{\fbackref{19:44} The Gk. lacks \fbib{city limits}} They will not leave one stone on another within your walls,\fnote{\fbackref{19:44} Lit. \fbib{within you}} because you didn't recognize the time when you were visited.''}\fnote{\fbackref{19:44} Lit. \fbib{the time of your visitation}}
\passage{Confrontation in the Temple over Money}
\passageinfo{(Matthew 21:12-17; Mark 11:15-19; John 2:13-22)}

\v{45}Then Jesus\fnote{\fbackref{19:45} Lit. \fbib{he}} went into the Temple and began to throw out those who were selling things. \v{46}He told them, \red{``It is written, `My house is to be called a house of prayer,'\fnote{\fbackref{19:46} Cf. Isa 56:7} but you have turned it into a hideout\fnote{\fbackref{19:46} Lit. \fbib{cave}} for bandits!''}

\v{47}Then he began teaching in the Temple every day. The high priests, the scribes, and the leaders of the people kept looking for a way to kill him, \v{48}but they couldn't find a way to do it, because all the people were eager to hear him.
\labelchapt{20}
\passage{Jesus' Authority is Challenged}
\passageinfo{(Matthew 21:23-27; Mark 11:27-33)}

\chapt{20}
\v{1}One day, while Jesus\fnote{\fbackref{20:1} Lit. \fbib{he}} was teaching the people in the Temple and telling them the good news, the high priests and the scribes came with the elders \v{2}and asked him, ``Tell us: By what authority are you doing these things, and who gave you this authority?''

\v{3}He answered them, \red{``I, too, will ask you a question.\fnote{\fbackref{20:3} Lit. \fbib{word}} Tell me:} \v{4}\red{Was John's authority to baptize\fnote{\fbackref{20:4} Lit. \fbib{John's baptism}} from heaven or from humans?''}

\v{5}They discussed this among themselves: ``If we say, `From heaven,' he will ask, `Then why didn't you believe him?' \v{6}But if we say, `From humans,' all the people will stone us to death, because they are convinced that John was a prophet.'' \v{7}So they answered that they didn't know where it was from.

\v{8}Then Jesus told them, \red{``Then I won't tell you by what authority I'm doing these things.''}
\passage{The Parable about the Tenant Farmers}
\passageinfo{(Matthew 21:33-46; Mark 12:1-12)}

\v{9}Then he began to tell the people this parable: \red{``A man planted a vineyard, leased it to tenant farmers, and went abroad for a long time.} \v{10}\red{At the right time he sent a servant to the farmers in order to get his share of the produce of the vineyard. But the farmers beat him and sent him back empty-handed.} \v{11}\red{He sent another servant, and they beat him, too, treated him shamefully, and sent him back empty-handed.} \v{12}\red{Then he sent a third, and they wounded him and threw him out, too.}

\v{13}\red{``Then the owner of the vineyard said, `What should I do? I'll send my son whom I love. Maybe they'll respect him.'} \v{14}\red{But when the farmers saw him, they talked it over among themselves and said, `This is the heir. Let's kill him so that the inheritance will be ours!'} \v{15}\red{So they threw him out of the vineyard and killed him. Now what will the owner of the vineyard do to them?} \v{16}\red{He will come and destroy those farmers and give the vineyard to others.''}

Those who heard him said, ``That must never happen!''

\v{17}But Jesus\fnote{\fbackref{20:17} Lit. \fbib{he}} looked at them and asked, \red{``What does this text mean:}

\begin{poetry}
\poeml \red{`The stone that the builders rejected} \\
\poemll    \red{has become the cornerstone'?}\fnote{\fbackref{20:17} Or \fbib{capstone}; cf. Ps 118:22}
\end{poetry}

\v{18}\red{Everyone who falls on that stone will be broken to pieces, and it will crush anyone on whom it falls.''}

\v{19}When the scribes and the high priests realized that Jesus\fnote{\fbackref{20:19} Lit. \fbib{he}} had told this parable about them, they wanted to arrest him right then, but they were afraid of the crowd.
\passage{A Question about Paying Taxes}
\passageinfo{(Matthew 22:15-22; Mark 12:13-17)}

\v{20}So they watched him closely and sent spies who pretended to be honest men in order to trap him in what he would say. They wanted to hand him over to the jurisdiction\fnote{\fbackref{20:20} Lit. \fbib{the power and authority}} of the governor, \v{21}so they asked him, ``Teacher, we know that you're right in what you say and teach, and that you don't favor any individual, but teach the way of God truthfully. \v{22}Is it lawful for us to pay taxes to Caesar or not?''

\v{23}But he discerned their craftiness and responded to them, \v{24}\red{``Show me a denarius. Whose face and name does it have?''}

``Caesar's,'' they replied.

\v{25}So he told them, \red{``Then give back to Caesar the things that are Caesar's, and to God the things that are God's.''}

\v{26}So they couldn't catch him before the people in what he said. Amazed at his answer, they became silent.
\passage{A Question about the Resurrection}
\passageinfo{(Matthew 22:23-33; Mark 12:18-27)}

\v{27}Now some Sadducees, who claim there is no resurrection, came to Jesus\fnote{\fbackref{20:27} Lit. \fbib{him}} \v{28}and asked him, ``Teacher, Moses wrote for us that if a man's brother dies and leaves a wife but no child, the man\fnote{\fbackref{20:28} Lit. \fbib{the brother}} should marry the widow and have children for his brother. \v{29}Now there were seven brothers. The first one married and died childless. \v{30}Then the second \v{31}and the third married her. In the same way, all seven died and left no children. \v{32}Finally, the woman died, too. \v{33}Now in the resurrection, whose wife will the woman be, since the seven had married her?''

\v{34}Jesus told them, \red{``Those who belong to this age marry and are married,} \v{35}\red{but those who are considered worthy of a place in that age and in the resurrection from the dead neither marry nor are given in marriage.} \v{36}\red{Nor can they die anymore, because they are like the angels and, since they share in the resurrection, are God's children.} \v{37}\red{Even Moses demonstrated in the story about the bush that the dead are raised, when he calls the Lord, `the God of Abraham, the God of Isaac, and the God of Jacob.'}\fnote{\fbackref{20:37} Cf. Exod 3:6, 15, 16} \v{38}\red{He is not the God of the dead, but of the living, because he considers all people to be alive to him.''}

\v{39}Then some of the scribes replied, ``Teacher, you have given a fine answer.'' \v{40}Then they no longer dared to ask him another question.
\passage{A Question about David's Son}
\passageinfo{(Matthew 22:41-46; Mark 12:35-37)}

\v{41}Then he asked them, \red{``How can people\fnote{\fbackref{20:41} Lit. \fbib{they}} say that the Messiah\fnote{\fbackref{20:41} Or \fbib{Christ}} is David's son?} \v{42}\red{Because David himself in the book of Psalms says,}

\begin{poetry}
\poeml \red{`The Lord\fnote{\fbackref{20:42} MT source citation reads \fbib{\divine{Lord}}} told my Lord,} \\
\poemll    \red{``Sit at my right hand,} \\
\poeml \v{43}\red{until I make your enemies a footstool for your feet.''\,'}\fnote{\fbackref{20:43} Cf. Ps 110:1}
\end{poetry}

\v{44}\red{So David calls him `Lord.' Then how can he be his son?''}
\passage{Jesus Denounces the Scribes}
\passageinfo{(Matthew 23:1-36; Mark 12:38-40; Luke 11:37-54)}

\v{45}While all the people were listening, he told his disciples, \v{46}\red{``Beware of the scribes! They like to walk around in long robes and love to be greeted in the marketplaces and to have the best seats in the synagogues and the places of honor at banquets.} \v{47}\red{They devour widows' houses\fnote{\fbackref{20:47} I.e. rob widows by taking their houses} and say long prayers to cover it up. They will receive greater condemnation!''}
\labelchapt{21}
\passage{The Widow's Offering}
\passageinfo{(Mark 12:41-44)}

\chapt{21}
\v{1}Now Jesus\fnote{\fbackref{21:1} Lit. \fbib{he}} looked up and saw rich people dropping their gifts into the offering box.\fnote{\fbackref{21:1} Or \fbib{treasury}} \v{2}Then he saw a destitute widow drop in two small copper coins.\fnote{\fbackref{21:2} Lit. \fbib{lepta}, the smallest coin denominated in the 1\textsuperscript{st} century Jewish economy} \v{3}He said, \red{``I tell you with certainty, this destitute widow has dropped in more than all of them,} \v{4}\red{because all the others contributed to the offering\fnote{\fbackref{21:4} Other mss. read \fbib{to the offering of God}} out of their surplus, but she, in her poverty, dropped in everything she had to live on.''}
\passage{Jesus Predicts the Destruction of the Temple}
\passageinfo{(Matthew 24:1-2; Mark 13:1-2)}

\v{5}Now while some people were talking about the Temple---how it was decorated with beautiful stones and gifts dedicated to God---he said, \v{6}\red{``As for these things that you see, the time will come when not one stone will be left on another that won't be knocked down.''}
\passage{The Coming Wars and Revolutions}
\passageinfo{(Matthew 24:3-8; Mark 13:3-8)}

\v{7}Then they asked him, ``Teacher, when will these things take place, and what will be the sign that these things are about to take place?''

\v{8}He said, \red{``Be careful that you are not deceived, because many will come in my name and say, `I AM' and, `The time has come.' Don't follow them.} \v{9}\red{When you hear of wars and revolutions, never be alarmed, because these sort of things must take place first, but the end won't come right away.''}

\v{10}Then he went on to say to them, \red{``Nation will rise up in arms against nation, and kingdom against kingdom.} \v{11}\red{There will be great earthquakes, famines, and plagues in various places, and there will be fearful events and awful signs from heaven.''}
\passage{Future Persecutions}
\passageinfo{(Matthew 10:16-25; Matthew 24:9-14; Mark 13:9-13)}

\v{12}\red{``But before all these things take place, people\fnote{\fbackref{21:12} Lit. \fbib{they}} will arrest you and persecute you. They will hand you over to synagogues and prisons, and you will be brought before kings and governors for my name's sake.} \v{13}\red{It will give you an opportunity to testify.} \v{14}\red{So purpose in your hearts not to prepare your defense ahead of time,} \v{15}\red{because I will give you the ability to speak, along with wisdom, that none of your opponents will be able to resist or refute.} \v{16}\red{You will be betrayed even by parents, brothers, relatives, and friends, and they will put some of you to death.} \v{17}\red{You will be hated continuously by everyone because of my name.} \v{18}\red{And yet not a hair on your head will be lost.} \v{19}\red{By your endurance you will protect your lives.''}
\passage{Signs of the End}
\passageinfo{(Matthew 24:15-21; Mark 13:14-19)}

\v{20}\red{``When you see Jerusalem surrounded by armies, then understand that its devastation is approaching.} \v{21}\red{Then those in Judea must flee to the mountains, those inside the city must leave it, and those in the countryside must not go into it,} \v{22}\red{because these are the days of vengeance when all that is written will be fulfilled.}

\v{23}\red{``How terrible it will be for those women who are pregnant or who are nursing babies in those days!---because there will be great distress in the land\fnote{\fbackref{21:23} Or \fbib{on earth}} and wrath against this people.} \v{24}\red{They will fall by the edge of the sword and be carried off as captives among all the nations, and Jerusalem will be trampled on by the unbelievers\fnote{\fbackref{21:24} Lit. \fbib{gentiles} ; i.e. unbelieving non-Jews} until the times of the unbelievers\fnote{\fbackref{21:24} Lit. \fbib{gentiles} ; i.e. unbelieving non-Jews} are fulfilled.''}
\passage{The Coming of the Son of Man}
\passageinfo{(Matthew 24:29-31; Mark 13:24-27)}

\v{25}``\red{There will be signs in the sun, the moon, and the stars, and there will be distress on earth among the nations that are confused by the roaring of the sea and its waves.} \v{26}\red{People will faint from fear and apprehension because of the things that are to come on the inhabited world, because the powers of heaven will be shaken.} \v{27}\red{Then they will see `the Son of Man coming in a cloud'\fnote{\fbackref{21:27} Cf. Dan 7:13} with power and great glory.}

\v{28}\red{``Now when these things begin to take place, stand up and lift up your heads, because your deliverance is approaching.''}
\passage{The Lesson from the Fig Tree}
\passageinfo{(Matthew 24:32-35; Mark 13:28-31)}

\v{29}Then he told them a parable: \red{``Look at the fig tree and all the trees.} \v{30}\red{As soon as they produce leaves, you can see for yourselves and know that summer is already near.} \v{31}\red{In the same way, when you see these things taking place, you will know that the kingdom of God is near.} \v{32}\red{I tell all of you\fnote{\fbackref{21:32} The Gk. pronoun \fbib{you} is pl.} with certainty, this generation won't disappear until all these things take place.} \v{33}\red{Heaven and earth will disappear, but my words won't ever disappear.''}
\passage{Be Alert}

\v{34}\red{``Constantly be on your guard so that your hearts will not be loaded down with self-indulgence, drunkenness, and the worries of this life, or that day will take you by surprise} \v{35}\red{like a trap, because it will come on everyone who lives on the face of the earth.} \v{36}\red{So be alert at all times, praying that you may have strength to escape all these things that are going to take place and to take your stand in the presence of the Son of Man.''}

\v{37}Now during the day Jesus\fnote{\fbackref{21:37} Lit. \fbib{he}} would teach in the Temple, but when evening came he would go out and spend the night on what is called the Mount of Olives. \v{38}And all the people would get up early in the morning to listen to him in the Temple.
\labelchapt{22}
\passage{The Plot to Kill Jesus}
\passageinfo{(Matthew 26:1-5, 14-16; Mark 14:1-2, 10-11; John 11:45-53)}

\chapt{22}
\v{1}Now the Festival of Unleavened Bread, which is called the Passover, was near. \v{2}So the high priests and the scribes were looking for a way to put him to death, because they were afraid of the crowd.

\v{3}But Satan went into Judas called Iscariot, who belonged to the circle\fnote{\fbackref{22:3} Lit. \fbib{number}} of the Twelve. \v{4}So he went off and discussed with the high priests and the Temple police how he could betray Jesus\fnote{\fbackref{22:4} Lit. \fbib{him}} to them. \v{5}They were delighted, and agreed to give him money. \v{6}Judas\fnote{\fbackref{22:6} Lit. \fbib{He}} accepted their offer and began to look for a good opportunity to betray Jesus\fnote{\fbackref{22:6} Lit. \fbib{him}} to them when no crowd was present.
\passage{The Passover with the Disciples}
\passageinfo{(Matthew 26:17-25; Mark 14:12-21; John 13:21-30)}

\v{7}Then the day of the Festival\fnote{\fbackref{22:7} The Gk. lacks \fbib{of the Festival}} of Unleavened Bread came, on which the Passover lamb was to be sacrificed. \v{8}So Jesus\fnote{\fbackref{22:8} Lit. \fbib{he}} sent Peter and John, saying, \red{``Go and make preparations for us to eat the Passover meal.''}

\v{9}They asked him, ``Where do you want us to prepare it?''

\v{10}He told them, \red{``Just after you go into the city, a man carrying a jug of water will meet you. Follow him into the house he enters} \v{11}\red{and say to the owner of the house, `The Teacher asks you, ``Where is the room where I can eat the Passover meal with my disciples?''\,'} \v{12}\red{Then he will show you a large upstairs room that is furnished. Get things ready for us there.''} \v{13}So they went and found everything just as Jesus\fnote{\fbackref{22:13} Lit. \fbib{he}} had told them, and they prepared the Passover meal.
\passage{The Lord's Supper}
\passageinfo{(Matthew 26:26-30; Mark 14:22-26)}

\v{14}Now when the hour came, Jesus\fnote{\fbackref{22:14} Lit. \fbib{he}} took his place at the table, along with his apostles. \v{15}He told them, \red{``I have eagerly desired to eat this Passover meal with you before I suffer,} \v{16}\red{because I tell all of you,\fnote{\fbackref{22:16} The Gk. pronoun \fbib{you} is pl.} I will never eat it again until it finds its fulfillment in the kingdom of God.''}

\v{17}Then he took a cup, gave thanks, and said, \red{``Take this and share it among yourselves,} \v{18}\red{because I tell you, from now on I will never drink the product of the vine until the kingdom of God comes.''}

\v{19}Then he took a loaf of bread, gave thanks, broke it in pieces, and handed it to them, saying, \red{``This is my body, which is given for you. Keep on doing this in memory of me.''}

\v{20}He did the same with the cup after supper, saying, \red{``This cup is the new covenant sealed\fnote{\fbackref{22:20} The Gk. lacks \fbib{sealed}} by my blood, which is being poured out for you.} \v{21}\red{Yet look! The hand of the man who is betraying me is with me on the table!} \v{22}\red{The Son of Man is going away, just as it has been determined, but how terrible it will be for that man by whom he is betrayed!''} \v{23}Then they began to discuss among themselves which one of them was going to do this.
\passage{An Argument about Greatness}

\v{24}Now an argument sprang up among them as to which one of them was to be regarded as the greatest. \v{25}But he told them, \red{``The kings of the unbelievers\fnote{\fbackref{22:25} Lit. \fbib{gentiles} ; i.e. unbelieving non-Jews} lord it over them, and those who exercise authority over them are called benefactors.} \v{26}\red{But you are not to do so. On the contrary, the greatest among you should become like the youngest, and the one who leads should become like the one who serves.} \v{27}\red{Because who is greater, the one who sits at the table, or the one who serves? It is the one at the table, isn't it? But I'm among you as one who serves.}

\v{28}\red{``You are the ones who have always stood by me in my trials.} \v{29}\red{And I confer a kingdom on you, just as my Father has conferred a kingdom\fnote{\fbackref{22:29} The Gk. lacks \fbib{a kingdom}} on me,} \v{30}\red{so that you may eat and drink at my table in my kingdom and sit down on thrones to govern the twelve tribes of Israel.''}
\passage{Jesus Predicts Peter's Denial}
\passageinfo{(Matthew 26:31-35; Mark 14:27-31; John 13:36-38)}

\v{31}\red{``Simon, Simon, listen! Satan has asked permission to sift all of you like wheat,} \v{32}\red{but I have prayed for you that your own faith may not fail. When you have come back, you must strengthen your brothers.''}

\v{33}Peter\fnote{\fbackref{22:33} Lit. \fbib{He}} told him, ``Lord, I am ready to go even to prison and to die with you!''

\v{34}But Jesus\fnote{\fbackref{22:34} Lit. \fbib{he}} said, \red{``I tell you, Peter, the rooster will not crow today until you deny three times that you know me.''}
\passage{Be Prepared for Trouble}

\v{35}Then Jesus\fnote{\fbackref{22:35} Lit. \fbib{he}} asked his disciples,\fnote{\fbackref{22:35} Lit. \fbib{to them}} \red{``When I sent you out without a wallet, traveling bag, or sandals, you didn't lack anything, did you?''}

They replied, ``Nothing at all.''

\v{36}Then he told them, \red{``But now whoever has a wallet must take it along, and his traveling bag, too. And the one who has no sword must sell his coat and buy one.} \v{37}\red{Because I tell you, what has been written about me must be fulfilled: `He was counted among the criminals.'\fnote{\fbackref{22:37} Cf. Isa 53:12} Indeed, what is written about me must be fulfilled.''}

\v{38}So they said, ``Lord, look! Here are two swords.''

He answered them, \red{``Enough of that!''}\fnote{\fbackref{22:38} Or \fbib{That is enough}}
\passage{Jesus Prays on the Mount of Olives}
\passageinfo{(Matthew 26:36-46; Mark 14:32-42)}

\v{39}Then he left and went to the Mount of Olives, as usual. The disciples went with him. \v{40}When he arrived, he told them, \red{``Keep on praying that you may not be tempted.''} \v{41}Then he withdrew from them about a stone's throw, knelt down, and began to pray, \v{42}\red{``Father, if you are willing, take this cup away from me. Yet not my will but yours be done.''}

\v{43}Then an angel from heaven appeared to him and strengthened him. \v{44}In his anguish he prayed more earnestly, and his sweat became like large drops of blood falling on the ground.\fnote{\fbackref{22:44} Other mss. lack verses 43 and 44}

\v{45}When he got up from prayer, he went to the disciples and found them asleep from sorrow. \v{46}He asked them, \red{``Why are you sleeping? Get up and keep on praying that you may not be tempted.''}
\passage{Jesus is Arrested}
\passageinfo{(Matthew 26:47-56; Mark 14:43-50; John 18:3-11)}

\v{47}While Jesus\fnote{\fbackref{22:47} Lit. \fbib{he}} was still speaking, a crowd arrived. The man called Judas, one of the Twelve, was leading them, and he came close to Jesus to kiss\fnote{\fbackref{22:47} People customarily greeted their friends with a kiss.} him. \v{48}But Jesus asked him, \red{``Judas, are you betraying the Son of Man with a kiss?''}

\v{49}When those who were around Jesus\fnote{\fbackref{22:49} Lit. \fbib{him}} saw what was about to take place, they asked, ``Lord, should we attack with our swords?'' \v{50}Then one of them struck the high priest's servant, cutting off his right ear.

\v{51}But Jesus said, \red{``No more of this!''} So he touched the wounded man's\fnote{\fbackref{22:51} The Gk. lacks \fbib{wounded man's}} ear and healed him.

\v{52}Then Jesus told the high priests, the Temple police, and the elders, who had come for him, \red{``Have you come out with swords and clubs as if I were a bandit?}\fnote{\fbackref{22:52} Or \fbib{revolutionary}} \v{53}\red{While I was with you day after day in the Temple, you didn't lay a hand on me. But this is your hour, when darkness reigns!''}\fnote{\fbackref{22:53} Lit. \fbib{your hour and the power of darkness}}
\passage{Peter Denies Jesus}
\passageinfo{(Matthew 26:57-58, 69-75; Mark 14:53-54, 66-72; John 18:12-18, 25-27)}

\v{54}Then they arrested him, led him away, and brought him to the high priest's house. But Peter was following at a distance. \v{55}When they had kindled a fire in the middle of the courtyard and had taken their seats, Peter, too, sat down among them. \v{56}A servant girl saw him sitting by the fire, stared at him, and said, ``This man was with him, too.''

\v{57}But he denied it, ``I don't know him, woman!'' he responded.

\v{58}A little later, a man looked at him and said, ``You are one of them, too.''

But Peter said, ``Mister,\fnote{\fbackref{22:58} Lit. \fbib{Man}} I am not!''

\v{59}About an hour later, another man emphatically asserted, ``This man was certainly with him, because he is a Galilean!''

\v{60}But Peter said, ``Mister,\fnote{\fbackref{22:60} Lit. \fbib{Man}} I don't know what you're talking about!'' Just then, while he was still speaking, a rooster crowed.

\v{61}Then the Lord turned and looked straight at Peter. And Peter remembered the word from the Lord, and how he had told him, \red{``Before a rooster crows today, you will deny me three times.''} \v{62}So he went outside and cried bitterly.
\passage{Jesus is Insulted and Beaten}
\passageinfo{(Matthew 26:67-68; Mark 14:65)}

\v{63}Then the men who were holding Jesus in custody began to make fun of him while they beat him. \v{64}They blindfolded him and asked him over and over again, ``Prophesy! Who is the one who hit you?'' \v{65}And they kept insulting\fnote{\fbackref{22:65} Or \fbib{blaspheming}} him in many other ways.

\v{66}As soon as day came, the elders of the people, the high priests, and the scribes assembled and brought him before their Council.\fnote{\fbackref{22:66} Or \fbib{Sanhedrin}} \v{67}They said, ``If you are the Messiah,\fnote{\fbackref{22:67} Or \fbib{Christ}} tell us.''

But he told them, \red{``If I tell you, you won't believe me,} \v{68}\red{and if I ask you a question, you won't answer me.} \v{69}\red{But from now on the Son of Man will be seated at the right hand of the mighty God.''}\fnote{\fbackref{22:69} Or \fbib{the power of God}}

\v{70}Then they all asked, ``Are you, then, the Son of God?''

He answered them, \red{``You said it---I AM.''}

\v{71}``Why do we need any more testimony?'' they asked. ``We have heard it ourselves from his own mouth!''
\labelchapt{23}
\passage{Jesus is Taken to Pilate}
\passageinfo{(Matthew 27:1-2, 11-14; Mark 15:1-5; John 18:28-38)}

\chapt{23}
\v{1}Then the whole crowd got up and took him to Pilate. \v{2}They began to accuse him, ``We found this man corrupting our nation, forbidding us to pay taxes to Caesar, and saying that he is the Messiah,\fnote{\fbackref{23:2} Or \fbib{Christ}} a king.''

\v{3}Then Pilate asked him, ``Are you the king of the Jews?''

He answered, \red{``You say so.''}

\v{4}Then Pilate told the high priests and crowds, ``I do not find anything chargeable in this man.''

\v{5}But they kept insisting, ``He is stirring up the people with what he teaches all over Judea, from where he started in Galilee to this place.''
\passage{Jesus is Sent to Herod}

\v{6}When Pilate heard this, he asked whether the man was a Galilean. \v{7}When he learned with certainty that Jesus\fnote{\fbackref{23:7} Lit. \fbib{he}} came from Herod's jurisdiction, he sent him off to Herod, who was in Jerusalem at that time. \v{8}Now Herod was very glad to see Jesus, because he had been wanting to see him for a long time on account of what he had heard about him. He was also hoping to see some sign done by him. \v{9}So he continued to question him for a long time, but Jesus\fnote{\fbackref{23:9} Lit. \fbib{he}} gave him no answer at all. \v{10}Meanwhile, the high priests and the scribes stood nearby and continued to accuse him vehemently. \v{11}Even\fnote{\fbackref{23:11} Other mss. lack \fbib{Even}} Herod and his soldiers treated him with contempt and made fun of him. He put a magnificent robe on Jesus\fnote{\fbackref{23:11} Lit. \fbib{him}} and sent him back to Pilate. \v{12}So Herod and Pilate became friends with each other that very day. Before this they had been enemies.
\passage{Jesus is Sentenced to Death}
\passageinfo{(Matthew 27:15-26; Mark 15:6-15; John 18:39-19:16)}

\v{13}Then Pilate called the high priests, the other\fnote{\fbackref{23:13} The Gk. lacks \fbib{other}} leaders, and the people together \v{14}and told them, ``You brought this man to me as one who turns the people against the government. And here in your presence I have examined him and have found him `Not Guilty' of the charges you make against him. \v{15}Neither does Herod, because he sent him back to us! Indeed, this man\fnote{\fbackref{23:15} Lit. \fbib{Indeed, he}} has done nothing to deserve death. \v{16}So I will punish him and let him go.''

\v{17}Now he was obligated to release someone for them at the festival.\fnote{\fbackref{23:17} Other mss. lack verse 17} \v{18}But they all shouted out together, ``Away with this man! Release Barabbas for us!'' \v{19}(This man had been put in prison for murder and for a revolt that had taken place in the city.) \v{20}But Pilate wanted to let Jesus go, so he appealed to them again, \v{21}but they continued to shout, ``Crucify him! Crucify him!''

\v{22}Then he spoke to them a third time: ``What has he done wrong? I have found nothing in him worthy of death. So I will punish him and let him go.'' \v{23}But they kept pressing him with loud shouts, demanding that Jesus\fnote{\fbackref{23:23} Lit. \fbib{he}} be crucified, and their shouts began to prevail.

\v{24}Then Pilate pronounced his sentence that their demand should be carried out. \v{25}So he released the man who had been put in prison for revolt and murder---the man whose release\fnote{\fbackref{23:25} The Gk. lacks \fbib{whose release}} they continued to demand---but he let them have their way with Jesus.\fnote{\fbackref{23:25} Lit. \fbib{he turned Jesus over to their will}}
\passage{Jesus is Crucified}
\passageinfo{(Matthew 27:32-44; Mark 15:21-32; John 19:17-27)}

\v{26}As they led Jesus\fnote{\fbackref{23:26} Lit. \fbib{him}} away, they grabbed Simon, a man from Cyrene, as he was coming in from the country, and they put the cross on him and made him carry it behind Jesus. \v{27}A large crowd of people followed him, including some women who kept mourning and wailing for him.

\v{28}But Jesus turned to them and said, \red{``Women\fnote{\fbackref{23:28} Lit. \fbib{Daughters}} of Jerusalem, stop crying for me. Instead, cry for yourselves and for your children,} \v{29}\red{because the time is surely coming when people\fnote{\fbackref{23:29} Lit. \fbib{they}} will say, `How blessed are the women who couldn't bear children and the wombs that never bore and the breasts that never nursed!'} \v{30}\red{Then people\fnote{\fbackref{23:30} Lit. \fbib{they}} will begin to say to the mountains, `Fall on us!', and to the hills, `Cover us up!'}\fnote{\fbackref{23:30} Cf. Hos 10:8} \v{31}\red{And if they do this when the wood is green, what will happen when it is dry?''}

\v{32}Two others, who were criminals, were also led away to be executed with Jesus.\fnote{\fbackref{23:32} Lit. \fbib{him}} \v{33}When they reached the place called The Skull, they crucified him there with the criminals, one on his right and one on his left. \v{34}Jesus kept saying, \red{``Father, forgive them, because they don't know what they're doing.''}\fnote{\fbackref{23:34} Some mss. lack \fbib{Jesus kept saying, ``Father, forgive them, because they don't know what they're doing.''}} Then they divided his clothes among them by throwing dice.

\v{35}Meanwhile, the people stood looking on. The leaders were mocking him by saying, ``He saved others. Let him save himself, if he is the Messiah\fnote{\fbackref{23:35} Or \fbib{Christ}} of God, the chosen one!''

\v{36}The soldiers also made fun of Jesus\fnote{\fbackref{23:36} Lit. \fbib{him}} by coming up and offering him sour wine, \v{37}saying, ``If you are the king of the Jews, save yourself!'' \v{38}There was also an inscription over him written in Greek, Latin, and Hebrew:\fnote{\fbackref{23:38} Other mss. lack \fbib{written in Greek, Latin, and Hebrew}} ``This is the King of the Jews.''

\v{39}Now one of the criminals hanging there kept insulting\fnote{\fbackref{23:39} Or \fbib{blaspheming}} him, ``You are the Messiah,\fnote{\fbackref{23:39} Or \fbib{Christ}} aren't you? Save yourself{\ldots}and us!''

\v{40}But the other criminal rebuked him, ``Aren't you afraid of God, since you are suffering the same penalty? \v{41}We have been condemned justly, because we are getting what we deserve for what we have done, but this man has done nothing wrong.'' \v{42}Then he went on to plead, ``Jesus, remember me when you come into\fnote{\fbackref{23:42} Other mss. read \fbib{in}} your kingdom!''

\v{43}Jesus\fnote{\fbackref{23:43} Lit. \fbib{He}} told him, \red{``I tell you\fnote{\fbackref{23:43} The Gk. pronoun \fbib{you} is sing.} with certainty, today you will be with me in Paradise.''}
\passage{Jesus Dies on the Cross}
\passageinfo{(Matthew 27:45-56; Mark 15:33-41; John 19:28-30)}

\v{44}It was already about noon,\fnote{\fbackref{23:44} Lit. \fbib{the sixth hour}} and the whole land\fnote{\fbackref{23:44} Or \fbib{earth}} became dark until three in the afternoon\fnote{\fbackref{23:44} Lit. \fbib{the ninth hour}} \v{45}because the sun had stopped shining, and the curtain\fnote{\fbackref{23:45} This curtain separated the Holy Place from the Most Holy Place.} in the sanctuary was torn in two. \v{46}Then Jesus cried out with a loud voice and said, \red{``Father, into your hands I entrust my spirit.''}\fnote{\fbackref{23:46} Cf. Ps 31:5} After he said this, he breathed his last.

\v{47}When the centurion\fnote{\fbackref{23:47} A Roman centurion commanded about 100 men.} saw what had taken place, he praised God and said, ``This man certainly was righteous!'' \v{48}When all the crowds who had come together for this spectacle saw what had taken place, they beat their chests and left. \v{49}But all his acquaintances, including the women who had followed him from Galilee, were standing at a distance watching these things.
\passage{Jesus is Buried}
\passageinfo{(Matthew 27:57-61; Mark 15:42-47; John 19:38-42)}

\v{50}Now there was a man named Joseph, a member of the Council,\fnote{\fbackref{23:50} Or \fbib{Sanhedrin}} a good and righteous man--- \v{51}he had not voted for their plan and action---from the Jewish town of Arimathea; and he was waiting for the kingdom of God. \v{52}He went to Pilate and asked for the body of Jesus. \v{53}Then he took it down, wrapped it in a linen cloth, and laid it in a tomb cut in the rock, in which no one had yet been laid.

\v{54}It was the Preparation Day, and the Sabbath was just beginning. \v{55}So the women who had come with Jesus\fnote{\fbackref{23:55} Lit. \fbib{him}} from Galilee, following close behind, saw the tomb and how his body was laid. \v{56}Then they went back and prepared spices and perfumes, and on the Sabbath they rested according to the commandment.
\labelchapt{24}
\passage{Jesus is Raised from the Dead}
\passageinfo{(Matthew 28:1-10; Mark 16:1-8; John 20:1-10)}

\chapt{24}
\v{1}But at early dawn on the first day of the week,\fnote{\fbackref{24:1} Lit. \fbib{first of the Sabbaths}} they went to the tomb, taking the spices they had prepared. \v{2}They found the stone rolled away from the tomb, \v{3}but when they went in, they didn't find the body of the Lord Jesus.\fnote{\fbackref{24:3} Other mss. lack \fbib{of the Lord Jesus}} \v{4}While they were puzzling over this, two men in dazzling robes suddenly stood beside them. \v{5}While the women remained terrified, bowing their faces to the ground, the men\fnote{\fbackref{24:5} Lit. \fbib{they}} asked them, ``Why are you looking among the dead for someone who is living? \v{6}He is not here, but has been raised.\fnote{\fbackref{24:6} Other mss. lack \fbib{He is not here, but has been raised}} Remember what he told you while he was still in Galilee: \v{7}\red{`The Son of Man must be handed over to sinful men, be crucified, and rise on the third day.'}''

\v{8}Then the women\fnote{\fbackref{24:8} Lit. \fbib{Then they}} remembered Jesus'\fnote{\fbackref{24:8} Lit. \fbib{his}} words. \v{9}They returned from the tomb and reported all these things to the eleven disciples\fnote{\fbackref{24:9} The Gk. lacks \fbib{disciples}} and all the others. \v{10}The women who told the apostles about it were Mary Magdalene,\fnote{\fbackref{24:10} Or \fbib{Mary of Magdala}} Joanna, Mary the mother of James, and some\fnote{\fbackref{24:10} Lit. \fbib{the}} others. \v{11}But what they said seemed nonsense to them, so they did not believe them. \v{12}Peter, however, got up and ran to the tomb. He stooped down and saw only the linen cloths. Then he went home, wondering about what had happened.\fnote{\fbackref{24:12} Other mss. lack verse 12.}
\passage{Jesus Meets Two Disciples}
\passageinfo{(Mark 16:12-13)}

\v{13}On the same day, two of Jesus' followers\fnote{\fbackref{24:13} Lit. \fbib{of them}} were walking to a village called Emmaus, about 60 stadia\fnote{\fbackref{24:13} I.e. about 7.2 miles; one Roman stadion was about 640 feet} from Jerusalem. \v{14}They were talking with each other about all these things that had taken place. \v{15}While they were discussing and analyzing what had happened,\fnote{\fbackref{24:15} The Gk. lacks \fbib{what had happened}} Jesus himself approached and began to walk with them, \v{16}but their eyes were prevented from recognizing him.

\v{17}He asked them, \red{``What are you discussing with each other as you're walking along?''} They stood still and looked gloomy.

\v{18}The one whose name was Cleopas answered him, ``Are you the only visitor to Jerusalem who doesn't know what happened there in the past few days?''

\v{19}He asked them, \red{``What things?''}

They answered him, ``The events involving Jesus of Nazareth,\fnote{\fbackref{24:19} Other mss. read \fbib{the Nazorean}} who was a prophet, mighty in what he said and did before God and all the people, \v{20}and how our high priests and leaders handed him over to be condemned to death and had him crucified. \v{21}But we kept hoping that he would be the one to redeem\fnote{\fbackref{24:21} Or \fbib{to free}} Israel. What is more, this is now the third day since these things occurred. \v{22}Even some of our women have startled us by what they told us.\fnote{\fbackref{24:22} The Gk. lacks \fbib{by what they told us}} They were at the tomb early this morning \v{23}and didn't find his body there, so they came back and told us that they had seen a vision of angels, who were saying that he was alive. \v{24}Then some of those who were with us went to the tomb and found it just as the women had said. However, they didn't see him.''

\v{25}Then Jesus\fnote{\fbackref{24:25} Lit. \fbib{he}} told them, \red{``O, how foolish you are! How slow you are to believe everything the prophets said!} \v{26}\red{The Messiah\fnote{\fbackref{24:26} Or \fbib{Christ}} had to suffer these things and then enter his glory, didn't he?''} \v{27}Then, beginning with Moses and all the Prophets, he explained to them all the passages of Scripture about himself.

\v{28}As they came near the village where the two men\fnote{\fbackref{24:28} Lit. \fbib{where they}} were headed, Jesus\fnote{\fbackref{24:28} Lit. \fbib{he}} acted as though he were going farther. \v{29}But they strongly urged him, ``Stay with us, because it is almost evening and the daylight is nearly gone.'' So he went in to stay with them.

\v{30}While he was at the table with them, he took the bread, blessed it, broke it in pieces, and gave it to them. \v{31}Then their eyes were opened, and they knew who he was. And he vanished from them.

\v{32}Then they asked each other, ``Our hearts kept burning within us\fnote{\fbackref{24:32} Other mss. lack \fbib{within us}} as he was talking to us on the road and explaining the Scriptures to us, didn't they?''

\v{33}They got up right away, went back to Jerusalem, and found the eleven disciples\fnote{\fbackref{24:33} The Gk. lacks \fbib{disciples}} and their companions all together. \v{34}They kept saying, ``The Lord has really risen and has appeared to Simon!'' \v{35}Then the two men\fnote{\fbackref{24:35} Lit. \fbib{Then they}} began to tell what had happened on the road and how they had recognized him when he broke the bread in pieces.
\passage{Jesus Appears to the Disciples}
\passageinfo{(Matthew 28:16-20; Mark 16:14-18; John 20:19-23; Acts 1:6-8)}

\v{36}While they were all talking about this, Jesus\fnote{\fbackref{24:36} Lit. \fbib{he}} himself stood among them and told them, \red{``Peace be with you.''}\fnote{\fbackref{24:36} Other mss. lack \fbib{and told them, ``Peace be with you.''}}

\v{37}They were startled and terrified, thinking they were seeing a ghost. \v{38}But Jesus\fnote{\fbackref{24:38} Lit. \fbib{he}} told them, \red{``What's frightening you? And why are you doubting?} \v{39}\red{Look at my hands and my feet, because it's really me. Touch me and look at me, because a ghost doesn't have flesh and bones as you see that I have.''} \v{40}After he had said this, he showed them his hands and his feet.\fnote{\fbackref{24:40} Other mss. lack verse 40} \v{41}Even though they were still skeptical due to their joy and astonishment, Jesus\fnote{\fbackref{24:41} Lit. \fbib{he}} asked them, \red{``Do you have anything here to eat?''}

\v{42}They gave him a piece of broiled fish, \v{43}and he took it and ate it in their presence. \v{44}Then he told them, \red{``These are the words that I spoke to you while I was still with you---that everything written about me in the Law of Moses, the Prophets, and the Psalms had to be fulfilled.''}

\v{45}Then he opened their minds so that they might understand the Scriptures. \v{46}He told them, \red{``This is how it is written: the Messiah\fnote{\fbackref{24:46} Or \fbib{Christ}} was to suffer and rise from the dead on the third day,} \v{47}\red{and then repentance and forgiveness of sins is to be proclaimed in his name to all the nations, beginning at Jerusalem.} \v{48}\red{You are witnesses of these things.} \v{49}\red{I am sending to you what my Father promised, so stay here in the city until you have been clothed with power from on high.''}
\passage{Jesus is Taken up to Heaven}
\passageinfo{(Mark 16:19-20; Acts 1:9-11)}

\v{50}Later, he led them out as far as Bethany, lifted up his hands, and blessed them. \v{51}While he was blessing them, he left them and was taken up to heaven.\fnote{\fbackref{24:51} Other mss. lack \fbib{and was taken up to heaven}} \v{52}They worshipped him and\fnote{\fbackref{24:52} Other mss. lack \fbib{worshipped him and}} returned to Jerusalem filled with great joy. \v{53}They were continually in the Temple, blessing\fnote{\fbackref{24:53} Other mss. read \fbib{praising}; still other mss. read \fbib{praising and blessing}} God.\fnote{\fbackref{24:53} Other mss. read \fbib{God. Amen.}}

\bookheader{John}
\labelbook{John}

\bookpretitle{The Gospel According to}
\booktitle{John}

\labelchapt{1}
\passage{The Word and Creation}

\chapt{1}
\v{1}In the beginning, the Word existed. The Word was with God, and the Word was God. \v{2}He existed in the beginning with God. \v{3}Through him all things were made, and apart from him nothing was made that has been made. \v{4}In him was life, and that life brought light to humanity.\fnote{\fbackref{1:4} Lit. \fbib{was the light of people}} \v{5}And the light shines on in the darkness, and the darkness has never put it out.\fnote{\fbackref{1:5} Or \fbib{understood it}}
\passage{John's Witness to the Word}

\v{6}There was a man sent from God, whose name was John. \v{7}He came as a witness to testify about the light, so that all might believe because of him. \v{8}John\fnote{\fbackref{1:8} Lit. \fbib{He}} was not the light, but he came\fnote{\fbackref{1:8} The Gk. lacks \fbib{he came}} to testify about the light. \v{9}This\fnote{\fbackref{1:9} Lit. \fbib{He}} was the true light that enlightens every person by his coming into the world.\fnote{\fbackref{1:9} Or \fbib{every person who is coming into the world}} \v{10}He was in the world, and the world was made through him. Yet the world did not recognize him.
\passage{Responses to the Word}

\v{11}He came to his own creation,\fnote{\fbackref{1:11} Or \fbib{possessions}} yet his own people did not receive him. \v{12}However, to all who received him, those believing in his name, he gave authority to become God's children, \v{13}who were born, not merely in a genetic sense,\fnote{\fbackref{1:13} Lit. \fbib{not of bloods}} nor from lust,\fnote{\fbackref{1:13} Lit. \fbib{from desire of the flesh}} nor from man's desire, but from the will of\fnote{\fbackref{1:13} The Gk. lacks \fbib{the will of}} God.
\passage{The Word Becomes Human}

\v{14}The Word became flesh and lived\fnote{\fbackref{1:14} Lit. \fbib{pitched his tent}} among us. We gazed on his glory, the kind of glory that belongs to the Father's unique Son,\fnote{\fbackref{1:14} The Gk. lacks \fbib{Son}} who is full of grace and truth. \v{15}John told the truth about him when he cried out, ``This is the person about whom I said, `The one who comes after me ranks higher than me, because he existed before me.'\,'' \v{16}We have all received one gracious gift after another from his abundance,\fnote{\fbackref{1:16} Lit. \fbib{received grace for grace}} \v{17}because while the Law was given through Moses, grace and truth came through Jesus the Messiah.\fnote{\fbackref{1:17} Or \fbib{Christ}} \v{18}No one has ever seen God. The unique God,\fnote{\fbackref{1:18} Other mss. read \fbib{Son}} who is close to the Father's side, has revealed him.
\passage{The Testimony of John the Baptist}
\passageinfo{(Matthew 3:1-12; Mark 1:2-8; Luke 3:15-17)}

\v{19}This was John's testimony when the Jewish leaders\fnote{\fbackref{1:19} I.e. Judean leaders; lit. \fbib{the Jews}} sent priests and descendants of Levi to him from Jerusalem to ask him, ``Who are you?''

\v{20}He spoke openly and, remaining true to himself,\fnote{\fbackref{1:20} Lit. \fbib{and did not deny}} admitted, ``I am not the Messiah.''\fnote{\fbackref{1:20} Or \fbib{Christ}}

\v{21}So they asked him, ``Well then, are you Elijah?''

John\fnote{\fbackref{1:21} Lit. \fbib{He}} said, ``I am not.''

``Are you the Prophet?''

He answered, ``No.''

\v{22}``Who are you?'' they asked him. ``We must give an answer to those who sent us. What do you say about yourself?''

\v{23}He replied, ``I am

\begin{poetry}
\poeml `{\ldots}a voice crying out in the wilderness, \\
\poemll    ``Prepare the Lord's\fnote{\fbackref{1:23} MT source citation reads \fbib{\divine{Lord}'s}} highway,''\,'\fnote{\fbackref{1:23} Cf. Isa 40:3}
\end{poetry}

as the prophet Isaiah said.''

\v{24}Now those men\fnote{\fbackref{1:24} Lit. \fbib{Now they}} had been sent from the Pharisees. \v{25}They asked him, ``Why, then, are you baptizing if you are not the Messiah,\fnote{\fbackref{1:25} Or \fbib{Christ}} or Elijah, or the Prophet?''

\v{26}John answered them, ``I am baptizing with\fnote{\fbackref{1:26} Or \fbib{in}} water, but among you stands a man whom you do not know, \v{27}the one who is coming after me, whose sandal straps I am not worthy to untie.'' \v{28}This happened in Bethany\fnote{\fbackref{1:28} Other mss. read \fbib{Bethabara}} on the other side\fnote{\fbackref{1:28} I.e. the east side} of the Jordan, where John was baptizing.

\v{29}The next day, John\fnote{\fbackref{1:29} Lit. \fbib{he}} saw Jesus coming toward him and said, ``Look, the Lamb of God who takes away the sin of the world! \v{30}This is the one about whom I said, `After me comes a man who ranks above me, because he existed before me.' \v{31}I didn't recognize him, but I came baptizing with\fnote{\fbackref{1:31} Or \fbib{in}} water so that he might be revealed to Israel.''

\v{32}John also testified, ``I saw the Spirit coming down from heaven like a dove, and it remained on him. \v{33}I didn't recognize him, but the one who sent me to baptize with\fnote{\fbackref{1:33} Or \fbib{in}} water told me, `The person on whom you see the Spirit descending and remaining is the one who baptizes with\fnote{\fbackref{1:33} Or \fbib{in}} the Holy Spirit.' \v{34}I have seen this and have testified that this is the Son\fnote{\fbackref{1:34} Other mss. read \fbib{Chosen One}} of God.''
\passage{The First Disciples}

\v{35}The next day, John was standing there again with two of his disciples. \v{36}As he watched Jesus walk by, he said, ``Look, the Lamb of God!'' \v{37}When the two disciples heard him say this, they followed Jesus.

\v{38}But when Jesus turned around and saw them following, he asked them, \red{``What are you looking for?''}

They asked him, ``Rabbi,'' (which is translated ``Teacher''), ``where are you staying?''

\v{39}He told them, \red{``Come and see!''} So they went and saw where he was staying, and they remained with him that day. It was about four o'clock in the afternoon.\fnote{\fbackref{1:39} Lit. \fbib{the tenth hour}}

\v{40}Andrew, Simon Peter's brother, was one of the two who heard John and followed Jesus.\fnote{\fbackref{1:40} Lit. \fbib{him}} \v{41}The first thing Andrew\fnote{\fbackref{1:41} Lit. \fbib{He}} did was to find his brother Simon and say to him, ``We have found the Anointed One!''\fnote{\fbackref{1:41} The Gk. word \fbib{messias} is a transliteration of the Heb. word for \fbib{Messiah}} (which is translated ``Messiah'').\fnote{\fbackref{1:41} Or \fbib{Christ}}

\v{42}He led Simon\fnote{\fbackref{1:42} Lit. \fbib{him}} to Jesus. Jesus looked at him intently and said, \red{``You are Simon, John's son.\fnote{\fbackref{1:42} Cf. Matthew 16:17} You will be called Cephas!''}\fnote{\fbackref{1:42} \fbib{Cephas} means \fbib{rock} in Aram.} (which is translated ``Peter'').\fnote{\fbackref{1:42} \fbib{Peter} means \fbib{rock} in Gk.}
\passage{Jesus Calls Philip and Nathaniel}

\v{43}The next day, Jesus decided to go away to Galilee, where he found Philip and told him, \red{``Follow me.''} \v{44}Now Philip was from Bethsaida, the hometown of Andrew and Peter.

\v{45}Philip found Nathaniel and told him, ``We have found the man about whom Moses in the Law and the Prophets wrote---Jesus, the son of Joseph, from Nazareth.''

\v{46}Nathaniel asked him, ``From Nazareth? Can anything good come from there?''

Philip told him, ``Come and see!''

\v{47}Jesus saw Nathaniel coming toward him and said about him, \red{``Look, a genuine Israeli, in whom there is no deceit!''}

\v{48}Nathaniel asked him, ``How do you know me?''

Jesus answered him, \red{``Before Philip called you, while you were under the fig tree, I saw you.''}

\v{49}Nathaniel replied to him, ``Rabbi,\fnote{\fbackref{1:49} \fbib{Rabbi} is Heb. for \fbib{Master} and/or \fbib{Teacher}} you are the Son of God! You are the King of Israel!''

\v{50}Jesus told him, \red{``Do you believe because I told you that I saw you under the fig tree? You will see greater things than that.''} \v{51}Then he told him, \red{``Truly, I tell all of you\fnote{\fbackref{1:51} The Gk. pronoun \fbib{you} is pl.} emphatically, you will see heaven standing open and the angels of God going up and coming down to the Son of Man.''}
\labelchapt{2}
\passage{Jesus Changes Water into Wine}

\chapt{2}
\v{1}On the third day of that week\fnote{\fbackref{2:1} The Gk. lacks \fbib{of that week}} there was a wedding in Cana of Galilee. Jesus' mother was there, \v{2}and Jesus and his disciples had also been invited to the wedding. \v{3}When the wine ran out, Jesus' mother told him, ``They don't have any more wine.''

\v{4}\red{``How does that concern us, dear lady?''}\fnote{\fbackref{2:4} Or \fbib{us, woman}} Jesus asked her.\red{ ``My time hasn't come yet.''}

\v{5}His mother told the servants, ``Do whatever he tells you.''

\v{6}Now standing there were six stone water jars used for the Jewish rites of purification, each one holding from two to three measures.\fnote{\fbackref{2:6} I.e. about 25 gallons each; the Gk. \fbib{metron} contained about 8.4 gallons} \v{7}Jesus told the servants,\fnote{\fbackref{2:7} Lit. \fbib{them}} \red{``Fill the jars with water.''} So they filled them up to the brim. \v{8}Then he told them, \red{``Now draw some out and take it to the man in charge of the banquet.''} So they did.

\v{9}When the man in charge of the banquet tasted the water that had become wine (without knowing where it had come from, though the servants who had drawn the water knew), he\fnote{\fbackref{2:9} Lit. \fbib{the man in charge of the banquet}} called for the bridegroom \v{10}and told him, ``Everyone serves the best wine first, and the cheap kind when people\fnote{\fbackref{2:10} Lit. \fbib{they}} are drunk. But you have kept the best wine until now!'' \v{11}Jesus did this, the first\fnote{\fbackref{2:11} Or \fbib{beginning}} of his signs, in Cana of Galilee. He revealed his glory, and his disciples believed in him.

\v{12}After this, Jesus\fnote{\fbackref{2:12} Lit. \fbib{he}} went down to Capernaum---he, his mother, his brothers, and his disciples---and they remained there for a few days.
\passage{Confrontation in the Temple over Money}
\passageinfo{(Matthew 21:12-13; Mark 11:15-17; Luke 19:45-46)}

\v{13}The Jewish Passover was near, and Jesus went up to Jerusalem. \v{14}In the Temple he found people selling cattle, sheep, and doves, as well as moneychangers sitting at their tables. \v{15}After making a whip out of cords, he drove all of them out of the Temple, including the sheep and the cattle. He scattered the coins of the moneychangers and knocked over their tables.

\v{16}Then he told those who were selling the doves, \red{``Take these things out of here! Stop making my Father's house a marketplace!''} \v{17}His disciples remembered that it was written, ``Zeal for your house will consume me.''\fnote{\fbackref{2:17} Cf. Ps 69:9}

\v{18}Then the Jewish leaders\fnote{\fbackref{2:18} I.e. Judean leaders; lit. \fbib{the Jews}} asked him, ``What sign can you show us as authority for doing these things?''

\v{19}Jesus answered them, \red{``Destroy this sanctuary, and in three days I will rebuild it.''}

\v{20}The Jewish leaders\fnote{\fbackref{2:20} I.e. Judean leaders; lit. \fbib{The Jews}} said, ``This sanctuary has been under construction for 46 years, and you're going to rebuild it in three days?'' \v{21}But the sanctuary he was speaking about was his own body. \v{22}After he had been raised from the dead, his disciples remembered that he had said this. So they believed the Scripture and the statement that Jesus had made.\fnote{\fbackref{2:22} Lit. \fbib{spoken}}
\passage{Jesus Knows All People}

\v{23}While Jesus\fnote{\fbackref{2:23} Lit. \fbib{he}} was in Jerusalem for the Passover Festival, many people believed in him\fnote{\fbackref{2:23} Lit. \fbib{in his name}} because they saw the signs that he was doing. \v{24}Jesus, however, did not entrust himself to them, because he knew all people \v{25}and didn't need anyone to tell him what people were like, because he himself knew what was in every person.\fnote{\fbackref{2:25} Lit. \fbib{in a person}}
\labelchapt{3}
\passage{Jesus Talks with Nicodemus}

\chapt{3}
\v{1}Now there was a man from the Pharisees, a leader of the Jews, whose name was Nicodemus. \v{2}He came to Jesus\fnote{\fbackref{3:2} Lit. \fbib{him}} at night and told him, ``Rabbi,\fnote{\fbackref{3:2} \fbib{Rabbi} is Heb. for \fbib{Master} and/or \fbib{Teacher}} we know that you have come from God as a teacher, because no one can perform these signs that you are doing unless God is with him.''

\v{3}Jesus replied to him, \red{``Truly, I tell you\fnote{\fbackref{3:3} The Gk. pronoun \fbib{you} is sing.} emphatically, unless a person is born from above\fnote{\fbackref{3:3} Or \fbib{born again}} he cannot see the kingdom of God.''}

\v{4}Nicodemus asked him, ``How can a person be born when he is old? He can't go back into his mother's womb a second time and be born, can he?''

\v{5}Jesus answered, \red{``Truly, I tell you\fnote{\fbackref{3:5} The Gk. pronoun \fbib{you} is sing.} emphatically, unless a person is born of water and Spirit he cannot enter the kingdom of God.}\fnote{\fbackref{3:5} Other mss. read \fbib{of heaven}} \v{6}\red{What is born of the flesh is flesh, and what is born of the Spirit is spirit.} \v{7}\red{Don't be astonished that I told you, `All of you must be born from above.'}\fnote{\fbackref{3:7} Or \fbib{born again}} \v{8}\red{The wind\fnote{\fbackref{3:8} The Gk. word can be translated both \fbib{wind} and \fbib{spirit.}} blows where it wants to. You hear its sound, but you don't know where it comes from or where it is going. That's the way it is with everyone who is born of the Spirit.''}

\v{9}Nicodemus asked him, ``How can that be?''

\v{10}Jesus answered him, \red{``You're the teacher of Israel, and you can't understand this?} \v{11}\red{Truly, I tell you\fnote{\fbackref{3:11} The Gk. pronoun \fbib{you} is sing.} emphatically, we know what we're talking about, and we testify about what we've seen. Yet you people\fnote{\fbackref{3:11} The Gk. lacks \fbib{people}} do not accept our testimony.} \v{12}\red{If I have told you people\fnote{\fbackref{3:12} The Gk. lacks \fbib{people}} about earthly things and you do not believe, how will you believe if I tell you about heavenly things?}

\v{13}\red{``No one has gone up to heaven except the one who came down from heaven, the Son of Man who is in heaven.}\fnote{\fbackref{3:13} Other mss. lack \fbib{who is in heaven}} \v{14}\red{Just as Moses lifted up the serpent in the wilderness, so must the Son of Man be lifted up,} \v{15}\red{so that everyone who believes in him would have eternal life.}\fnote{\fbackref{3:15} The quotation possibly concludes with this verse instead of with verse 21.}

\v{16}\red{``For this is how God loved the world: He gave his unique Son so that everyone who believes in him would not be lost but have eternal life.} \v{17}\red{Because God sent the Son into the world, not to condemn the world, but that the world would be saved through him.} \v{18}\red{Whoever believes in him is not condemned, but whoever does not believe has already been condemned, because he has not believed in the name of God's unique Son.} \v{19}\red{And this is the basis for judgment: The light has come into the world, but people loved the darkness more than the light because their actions were evil.} \v{20}\red{Everyone who practices wickedness hates the light and does not come to the light, so that his actions may not be exposed.}\fnote{\fbackref{3:20} Other mss. read \fbib{as being evil}} \v{21}\red{But whoever does what is true comes to the light, so that it may become evident that his actions have God's approval.''}\fnote{\fbackref{3:21} Lit. \fbib{actions are in God}}
\passage{John the Baptist Talks about Jesus}

\v{22}After this, Jesus and his disciples went into the Judean countryside. He spent some time there with them and began baptizing. \v{23}John was also baptizing in Aenon, near Salim, because there was plenty of water there. People\fnote{\fbackref{3:23} Lit. \fbib{They}} kept coming and were being baptized, \v{24}since John had not yet been thrown into prison.

\v{25}Then a controversy about ritual purification sprang up between a certain Jew\fnote{\fbackref{3:25} Other mss. read \fbib{between the Jews}; i.e. between certain Jewish leaders} and John's disciples, \v{26}so they went to John and told him, ``Rabbi,\fnote{\fbackref{3:26} \fbib{Rabbi} is Heb. for \fbib{Master} and/or \fbib{Teacher}} the man who was with you on the other side\fnote{\fbackref{3:26} I.e. the east side} of the Jordan, the one about whom you testified---look, he's baptizing, and everyone is going to him!''

\v{27}John replied, ``No one can receive anything unless it has been given to them from heaven. \v{28}You yourselves are my\fnote{\fbackref{3:28} Other mss. lack \fbib{my}} witnesses that I said, `I am not the Messiah,\fnote{\fbackref{3:28} Or \fbib{Christ}} but I have been sent ahead of him.' \v{29}It is the bridegroom who gets the bride, yet the bridegroom's friend, who merely\fnote{\fbackref{3:29} The Gk. lacks \fbib{merely}} stands by and listens for him, is overjoyed to hear the bridegroom's voice. That's why this joy of mine is now complete. \v{30}He must become more important, but I must become less important.''
\passage{The One who Comes from Above}

\v{31}The one who comes from above is superior to everything. The one who is of the earth belongs to the earth and speaks about earthly things.\fnote{\fbackref{3:31} Lit. \fbib{of the earth}} The one who comes from heaven is superior to everything. \v{32}He testifies about what he has seen and heard, yet no one accepts his testimony. \v{33}The person who has accepted his testimony has acknowledged that God is truthful.\fnote{\fbackref{3:33} Or \fbib{true}} \v{34}The one whom God sent speaks the words of God, because God\fnote{\fbackref{3:34} Lit. \fbib{he}} does not give the Spirit in limited measure to him.\fnote{\fbackref{3:34} The Gk. lacks \fbib{to him}} \v{35}The Father loves the Son and has put everything in his hands. \v{36}The one who believes in the Son has eternal life, but the one who disobeys the Son will not see life. Instead, the wrath of God remains on him.
\labelchapt{4}
\passage{Jesus Meets a Samaritan Woman}

\chapt{4}
\v{1}Now when Jesus\fnote{\fbackref{4:1} Other mss. read \fbib{the Lord}} realized that the Pharisees had heard he was making and baptizing more disciples than John--- \v{2}although it was not Jesus who did the baptizing but his disciples--- \v{3}he left Judea and went back to Galilee. \v{4}Now it was necessary for him to go through Samaria. \v{5}So he came to a town in Samaria called Sychar, near the piece of land that Jacob had given to his son Joseph. \v{6}Jacob's Well was also there, and Jesus, tired out by the journey, sat down by the well. It was about noon.\fnote{\fbackref{4:6} Lit. \fbib{the sixth hour}}

\v{7}A Samaritan woman came to draw water, and Jesus told her, \red{``Please give me a drink,''} \v{8}since his disciples had gone off into town to buy food.

\v{9}The Samaritan woman asked him, ``How can you, a Jew, ask for a drink from me, a Samaritan woman?'' Because Jews do not have anything to do with Samaritans.\fnote{\fbackref{4:9} Other mss. lack \fbib{For Jews do not have anything to do with Samaritans.}}

\v{10}Jesus answered her, \red{``If you knew the gift of God, and who it is who is saying to you, `Please give me a drink,' you would have been the one to ask him, and he would have given you living water.''}

\v{11}The woman\fnote{\fbackref{4:11} Other mss. read \fbib{She}} told him, ``Sir, you don't have a bucket, and the well is deep. Where are you going to get this living water? \v{12}You're not greater than our ancestor Jacob, who gave us the well and drank from it, along with his sons and his flocks, are you?''

\v{13}Jesus answered her, \red{``Everyone who drinks this water will become thirsty again.} \v{14}\red{But whoever drinks the water that I will give him will never become thirsty again. The water that I will give him will become a well of water for him, springing up to eternal life.''}

\v{15}The woman told him, ``Sir, give me this water, so that I won't get thirsty or have to keep coming here to draw water.''

\v{16}He told her, \red{``Go and call your husband, and come back here.''}

\v{17}The woman answered him, ``I don't have a husband.''

Jesus told her, \red{``You are quite right in saying, `I don't have a husband,'} \v{18}\red{because you have had five husbands, and the man you have now is not your husband. What you have said is true.''}

\v{19}The woman told him, ``Sir, I see that you are a prophet! \v{20}Our ancestors worshipped on this mountain. But you Jews\fnote{\fbackref{4:20} The Gk. lacks \fbib{Jews}} say that the place where people should worship is in Jerusalem.''

\v{21}Jesus told her, \red{``Believe me, dear lady,\fnote{\fbackref{4:21} Or \fbib{me, woman}} the hour is coming when you Samaritans\fnote{\fbackref{4:21} The Gk. lacks \fbib{Samaritans}} will worship the Father neither on this mountain nor in Jerusalem.} \v{22}\red{You don't know what you're worshiping. We Jews\fnote{\fbackref{4:22} The Gk. lacks \fbib{Jews}} know what we're worshiping, because salvation comes from the Jews.} \v{23}\red{Yet the time is coming, and is now here, when true worshipers will worship the Father in spirit\fnote{\fbackref{4:23} Or \fbib{in the Spirit}} and truth. Indeed, the Father is looking for people like that to worship him.} \v{24}\red{God is spirit,\fnote{\fbackref{4:24} Or \fbib{Spirit}} and those who worship him must worship in spirit\fnote{\fbackref{4:24} Or \fbib{in the Spirit}} and truth.''}

\v{25}The woman told him, ``I know that the Anointed One\fnote{\fbackref{4:25} The Gk. word \fbib{messias} is a transliteration of the Heb. word for \fbib{Messiah}} is coming, who is being called `the Messiah'.\fnote{\fbackref{4:25} Or \fbib{Christ}} When that person comes, he will explain everything.''

\v{26}\red{``I am he,''} Jesus replied, \red{``the one who is speaking to you.''}

\v{27}At this point his disciples arrived, and they were astonished that he was talking to a woman. Yet no one said, ``What do you want from her?''\fnote{\fbackref{4:27} The Gk. lacks \fbib{from her}} or, ``Why are you talking to her?'' \v{28}Then the woman left her water jar and went back to town. She told people, \v{29}``Come, see a man who told me everything I've ever done! Could he possibly be the Messiah?''\fnote{\fbackref{4:29} Or \fbib{Christ}} \v{30}The people\fnote{\fbackref{4:30} Lit. \fbib{They}} left the town and started on their way to him.

\v{31}Meanwhile, the disciples were urging him, ``Rabbi,\fnote{\fbackref{4:31} \fbib{Rabbi} is Heb. for \fbib{Master} and/or \fbib{Teacher}} have something to eat.''

\v{32}But he told them, \red{``I have food to eat that you know nothing about.''}

\v{33}So the disciples began to say to one another, ``No one has brought him anything to eat, have they?''

\v{34}Jesus told them, \red{``My food is doing the will of the one who sent me and completing his work.} \v{35}\red{You say, don't you, `In four more months the harvest will begin?' Look, I tell you, open your eyes and observe that the fields are ready\fnote{\fbackref{4:35} Lit. \fbib{white}} for harvesting now!} \v{36}\red{The one who harvests is already receiving his wages and gathering a crop for eternal life, so that the one who sows and the one who harvests may rejoice together.} \v{37}\red{In this respect the saying is true: `One person sows, and another person harvests.'}\fnote{\fbackref{4:37} Cf. Mic 6:15} \v{38}\red{I have sent you to harvest what you have not worked for. Others have worked, and you have adopted their work as your own.''}

\v{39}Now many of the Samaritans of that town believed in Jesus\fnote{\fbackref{4:39} Lit. \fbib{in him}} because the woman had testified, ``He told me everything I've ever done.''

\v{40}So when the Samaritans came to Jesus,\fnote{\fbackref{4:40} Lit. \fbib{him}} they asked him to stay with them, and he stayed there for two days. \v{41}And many more believed because of what he said. \v{42}They kept telling the woman, ``It is no longer because of what you said that we believe, because now we have heard him ourselves, and we know that he really is the Savior of the world.''
\passage{Jesus Heals an Official's Son}
\passageinfo{(Matthew 8:5-13; Luke 7:1-10)}

\v{43}Two days later, Jesus\fnote{\fbackref{4:43} Lit. \fbib{he}} left for Galilee from there, \v{44}since Jesus himself had testified that a prophet has no honor in his own country. \v{45}When he arrived in Galilee, the Galileans welcomed him because they had seen everything that he had done in Jerusalem during the festival and because they, too, had gone to the festival. \v{46}So Jesus\fnote{\fbackref{4:46} Lit. \fbib{he}} returned to Cana in Galilee, where he had turned the water into wine. Meanwhile, in Capernaum there was a government official whose son was ill. \v{47}When this man heard that Jesus had come from Judea to Galilee, he went to him and asked him repeatedly to come down and heal his son, because he was about to die.

\v{48}Jesus told him, \red{``Unless you people\fnote{\fbackref{4:48} The Gk. lacks \fbib{people}} see signs and wonders, you will never believe.''}

\v{49}The official told him, ``Sir,\fnote{\fbackref{4:49} Or \fbib{Lord}} please come down before my little boy dies.''

\v{50}Jesus told him, \red{``Go home. Your son will live.''} The man believed what Jesus told him and started back home.

\v{51}While he was on his way, his servants met him and told him that his child\fnote{\fbackref{4:51} Other mss. read \fbib{son}} was alive. \v{52}So he asked them at what hour he had begun to recover, and they told him, ``The fever left him yesterday at one o'clock in the afternoon.''\fnote{\fbackref{4:52} Lit. \fbib{the seventh hour}}

\v{53}Then the father realized that this was the very hour when Jesus had told him, \red{``Your son will live.''} So he himself believed, along with his whole family.

\v{54}Now this was the second sign that Jesus did after coming from Judea to Galilee.
\labelchapt{5}
\passage{The Healing at the Pool}

\chapt{5}
\v{1}Later on, there was another\fnote{\fbackref{5:1} Other mss. read \fbib{the}} festival of the Jews, and Jesus went up to Jerusalem. \v{2}Near the Sheep Gate in Jerusalem is a pool called Bethesda\fnote{\fbackref{5:2} Other mss. read \fbib{Bethzatha}; still other mss. read \fbib{Bethsaida}} in Hebrew. It has five colonnades, \v{3}and under these a large number of sick people were lying---blind, lame, or paralyzed---waiting for the movement of the water.\fnote{\fbackref{5:3} Other mss. lack \fbib{waiting for the movement of the water}} \v{4}At certain times an angel of the Lord would go down into the pool and stir up the water, and whoever stepped in first after the stirring of the water was healed of whatever disease he had.\fnote{\fbackref{5:4} Other mss. lack v. 4}

\v{5}One particular man was there who had been ill for 38 years. \v{6}When Jesus saw him lying there and knew that he had already been there a long time, he asked him, \red{``Do you want to get well?''}

\v{7}The sick man answered him, ``Sir, I don't have anyone to put me into the pool when the water is stirred up. While I'm trying to get there, someone else steps down ahead of me.''

\v{8}Jesus told him, \red{``Stand up, pick up your mat, and walk!''} \v{9}The man immediately became well, and he picked up his mat and started walking. Now that day was a Sabbath.

\v{10}So the Jewish leaders\fnote{\fbackref{5:10} I.e. Judean leaders; lit. \fbib{the Jews}} told the man who had been healed, ``It is the Sabbath, and it is not lawful for you to carry your mat.

\v{11}But he answered them, ``The man who made me well told me, \red{`Pick up your mat and walk.'}''

\v{12}They asked him, ``Who is the man who told you, \red{`Pick it up and walk'}?''

\v{13}But the one who had been healed did not know who it was, because Jesus had slipped away from the crowd in that place. \v{14}Later on, Jesus found him in the Temple and told him, \red{``Look! You have become well. Stop sinning or something worse may happen to you.''} \v{15}The man went off and told the Jewish leaders\fnote{\fbackref{5:15} I.e. Judean leaders; lit. \fbib{the Jews}} that it was Jesus who had made him well. \v{16}So the Jewish leaders\fnote{\fbackref{5:16} I.e. Judean leaders; lit. \fbib{the Jews}} began persecuting Jesus,\fnote{\fbackref{5:16} Other MSS add \fbib{and were seeking to kill him}} because he kept doing such things on the Sabbath.

\v{17}But Jesus\fnote{\fbackref{5:17} Other mss. read \fbib{he}} answered them, \red{``My Father has been working until now, and I, too, am working.''} \v{18}So the Jewish leaders\fnote{\fbackref{5:18} I.e. Judean leaders; lit. \fbib{the Jews}} were trying all the harder to kill him, because he was not only breaking the Sabbath but was also calling God his own Father, thereby making himself equal to God.
\passage{The Authority of the Son}

\v{19}Jesus told them, \red{``Truly, I tell all of you\fnote{\fbackref{5:19} The Gk. pronoun \fbib{you} is pl.} emphatically, the Son can do nothing on his own accord, but only what he sees the Father doing, What the Father does, the Son does likewise.} \v{20}\red{The Father loves the Son and shows him everything he is doing, and he will show him even greater actions than these, so that you may be amazed.} \v{21}\red{Just as the Father raises the dead and gives them life, so also the Son gives life to those he chooses.} \v{22}\red{The Father judges no one, but has given all authority to judge to the Son,} \v{23}\red{so that everyone may honor the Son as they honor the Father. Whoever does not honor the Son does not honor the Father who sent him.}

\v{24}\red{``Truly, I tell all of you\fnote{\fbackref{5:24} The Gk. pronoun \fbib{you} is pl.} emphatically, whoever hears what I say and believes in the one who sent me has eternal life and will not be judged, but has passed from death to life.} \v{25}\red{Truly, I tell all of you\fnote{\fbackref{5:25} The Gk. pronoun \fbib{you} is pl.} emphatically, the time approaches, and is now here, when the dead will hear the voice of the Son of God, and those who hear it will live.} \v{26}\red{Just as the Father has life in himself, so also he has granted the Son to have life in himself,} \v{27}\red{and he has given him authority to judge, because he is the Son of Man.} \v{28}\red{Don't be amazed at this, because the time is approaching when everyone in their graves will hear the Son of Man's\fnote{\fbackref{5:28} Lit. \fbib{hear his}} voice} \v{29}\red{and will come out---those who have done what is good to the resurrection that leads to\fnote{\fbackref{5:29} Lit. \fbib{resurrection of}} life, and those who have practiced what is evil to the resurrection that ends in\fnote{\fbackref{5:29} Lit. \fbib{resurrection of}} condemnation.}\fnote{\fbackref{5:29} Or \fbib{judgment}} \v{30}\red{I can do nothing on my own accord. I judge according to what I hear, and my judgment is just, because I do not seek my own will but the will of the one who sent me.''}
\passage{Jesus' Greater Testimony}

\v{31}\red{``If I testify on my own behalf, my testimony is not trustworthy.} \v{32}\red{There is another who testifies about me, and I know\fnote{\fbackref{5:32} Other mss. read \fbib{you know}} that the testimony he gives about me is true.} \v{33}\red{You have sent messengers\fnote{\fbackref{5:33} The Gk. lacks \fbib{messengers}} to John, and he has testified to the truth.} \v{34}\red{I myself do not accept human testimony, but I am saying these things so that you may be saved.} \v{35}\red{That man John\fnote{\fbackref{5:35} The Gk. lacks \fbib{John}} was a lamp that burns and brightly shines, and for a while you were willing to rejoice in his light.}

\v{36}\red{``But I have a greater testimony than John's, because the actions that the Father has given me to complete---the very actions that I am doing---testify on my behalf that the Father has sent me.} \v{37}\red{Moreover, the Father who sent me has himself testified on my behalf. You have never heard his voice or seen what he looks like,} \v{38}\red{nor do you have his word at work\fnote{\fbackref{5:38} Lit. \fbib{word abiding}} in you, because you do not believe in the one whom he sent.} \v{39}\red{You examine the Scriptures carefully because you suppose that in them you have eternal life. Yet they testify about me.} \v{40}\red{But you are not willing to come to me to have life.}

\v{41}\red{``I do not accept human praise.} \v{42}\red{I know that you do not have the love of God in you.} \v{43}\red{I have come in my Father's name, and you do not accept me. Yet if another man comes in his own name, you will accept him.} \v{44}\red{How can you believe when you accept each other's praise and do not look for the praise that comes from the only God?}\fnote{\fbackref{5:44} Other mss. read \fbib{the only One}} \v{45}\red{Do not suppose that I will be the one to accuse you before the Father. Your accuser is Moses, on whom you have set your hope,} \v{46}\red{because if you believed Moses, you would believe me, since he wrote about me.} \v{47}\red{But if you do not believe what he wrote, how will you believe my words?''}
\labelchapt{6}
\passage{Jesus Feeds More than Five Thousand}
\passageinfo{(Matthew 14:13-21; Mark 6:30-44; Luke 9:10-17)}

\chapt{6}
\v{1}After this, Jesus went away to the other side of the Sea of Galilee (that is, to Tiberias). \v{2}A large crowd kept following him because they had seen the signs that he was performing by healing the sick. \v{3}But Jesus went up on a hillside and sat down there with his disciples.

\v{4}Now the Passover, the festival of the Jews, was near. \v{5}When Jesus looked up and saw that a large crowd was coming toward him, he asked Philip, \red{``Where can we buy bread for these people to eat?''} \v{6}Jesus\fnote{\fbackref{6:6} Lit. \fbib{He}} said this to test him, because he himself knew what he was going to do.

\v{7}Philip answered him, ``Two hundred denarii\fnote{\fbackref{6:7} The denarius was the usual day's wage for a laborer.} worth of bread isn't enough for each of them to have a little.''

\v{8}One of his disciples, Andrew, who was Simon Peter's brother, told him, \v{9}``There's a little boy here who has five barley loaves and two small fish. But what are these among so many people?''

\v{10}Jesus said, \red{``Have the people sit down.''} Now there was plenty of grass in that area, so they sat down, numbering about 5,000 men.

\v{11}Then Jesus took the loaves, gave thanks, and distributed them to those who were seated. He also distributed\fnote{\fbackref{6:11} Lit. \fbib{Likewise also}} as much fish as they wanted. \v{12}When they were completely satisfied, Jesus\fnote{\fbackref{6:12} Lit. \fbib{he}} told his disciples, \red{``Collect the pieces that are left over so that nothing is wasted.''} \v{13}So they collected and filled twelve baskets full of pieces of the five barley loaves left over by those who had eaten.

\v{14}When the people saw the sign\fnote{\fbackref{6:14} Other mss. read \fbib{signs}} that he had done, they kept saying, ``Truly this is the Prophet who was to come into the world!'' \v{15}Then Jesus, realizing that they were about to come and take him by force to make him king, withdrew\fnote{\fbackref{6:15} Other mss. read \fbib{fled}} again to the hillside by himself.
\passage{Jesus Walks on the Sea}
\passageinfo{(Matthew 14:22-27; Mark 6:45-52)}

\v{16}When evening came, his disciples went down to the sea, \v{17}got into a boat, and started across the sea to Capernaum. Darkness had already fallen, and Jesus had not yet come to them. \v{18}A strong wind was blowing, and the sea was getting rough. \v{19}After they had rowed about 25 or 30 stadia,\fnote{\fbackref{6:19} I.e. three or four miles; the Roman mile contained eight stadia, one stadion was about 604.5 feet long} they saw Jesus walking on the sea toward their boat. They became terrified. \v{20}But he told them, \red{``It is I. Stop being afraid!''} \v{21}So they were glad to take him on board, and immediately the boat reached the land toward which they were going.
\passage{Jesus the Bread of Life}

\v{22}The next day, the crowd that had remained on the other side of the sea noticed that only one boat had been there, and no other, and that Jesus had not gotten into that boat with his disciples. Instead, his disciples had gone away by themselves. \v{23}Other small boats from Tiberias arrived near the place where they had eaten the bread after the Lord had given thanks.\fnote{\fbackref{6:23} Other mss. lack \fbib{after the Lord had given thanks}} \v{24}When the crowd saw that neither Jesus nor his disciples were there, they got into these boats and went to Capernaum to look for Jesus.

\v{25}When they had found him on the other side of the sea, they asked him, ``Rabbi,\fnote{\fbackref{6:25} \fbib{Rabbi} is Heb. for \fbib{Master} and/or \fbib{Teacher}} when did you get here?''

\v{26}Jesus replied to them, \red{``Truly, I tell all of you\fnote{\fbackref{6:26} The Gk. pronoun \fbib{you} is pl.} emphatically, you are looking for me, not because you saw signs, but because you ate the loaves and were completely satisfied.} \v{27}\red{Do not work for food that perishes but for food that lasts for eternal life, which the Son of Man will give you, because God the Father has set his seal on him.''}

\v{28}Then they asked him, ``What must we do to perform God's works?''

\v{29}Jesus answered them, \red{``This is God's work: to believe in the one whom he has sent.''}

\v{30}So they asked him, ``What sign are you going to do so that we may see it and believe in you? What actions are you performing? \v{31}Our ancestors ate the manna in the wilderness, just as it is written, `He gave them bread from heaven to eat.'\,''\fnote{\fbackref{6:31} Cf. Ps 78:24; Exod 16:15; Num 11:7-9}

\v{32}Jesus told them, \red{``Truly, I tell all of you\fnote{\fbackref{6:32} The Gk. pronoun \fbib{you} is pl.} emphatically, it was not Moses who gave you the bread from heaven, but it is my Father who gives you the true bread from heaven.} \v{33}\red{The bread of God is the one who comes down from heaven and gives life to the world.''}

\v{34}Then they told him, ``Sir, give us this bread all the time.''

\v{35}Jesus told them, \red{``I am the bread of life. Whoever comes to me will never become hungry, and whoever believes in me will never become thirsty.} \v{36}\red{I told you that you have seen me,\fnote{\fbackref{6:36} Other mss. lack \fbib{me}} yet you don't believe.} \v{37}\red{Everything the Father gives me will come to me, and I'll never turn away the one who comes to me.} \v{38}\red{I have come down from heaven, not to do my own will, but the will of the one who sent me.} \v{39}\red{And this is the will of the one who sent me, that I should not lose anything that he has given me, but should raise it to life on the last day.} \v{40}\red{This is my Father's will: That everyone who sees the Son and believes in him should have eternal life, and I will raise him to life on the last day.''}

\v{41}Then the Jewish leaders\fnote{\fbackref{6:41} I.e. Judean leaders; lit. \fbib{the Jews}} began grumbling about him because he said, \red{``I am the bread that came down from heaven.''}

\v{42}They kept saying, ``This is Jesus, the son of Joseph, isn't it, whose father and mother we know? So how can he say, \red{`I have come down from heaven'}?''

\v{43}Jesus answered them, \red{``Stop grumbling among yourselves.} \v{44}\red{No one can come to me unless the Father who sent me draws him, and I will raise him to life on the last day.} \v{45}\red{It is written in the Prophets, `And all of them will be taught by God.'\fnote{\fbackref{6:45} Cf. Isa 54:13} Everyone who has listened to the Father and has learned anything comes to me.} \v{46}\red{Not that anyone has seen the Father except the one who comes from God. This one has seen the Father.} \v{47}\red{Truly, I tell all of you\fnote{\fbackref{6:47} The Gk. pronoun \fbib{you} is pl.} emphatically, the one who believes in me\fnote{\fbackref{6:47} Other mss. lack \fbib{in me}} has eternal life.} \v{48}\red{I'm the bread of life.} \v{49}\red{Your ancestors ate the manna in the wilderness and died.} \v{50}\red{This is the bread that comes down from heaven, so that a person may eat it and not die.} \v{51}\red{I'm the living bread that came down from heaven. If anyone eats this bread, he'll live forever. And the bread I will give for the life of the world is my flesh.''}

\v{52}Then the Jewish leaders\fnote{\fbackref{6:52} I.e. Judean leaders; lit. \fbib{the Jews}} debated angrily with each other, asking, ``How can this man give us his flesh to eat?''

\v{53}So Jesus told them, \red{``Truly, I tell all of you\fnote{\fbackref{6:53} The Gk. pronoun \fbib{you} is pl.} emphatically, unless you eat the flesh of the Son of Man and drink his blood, you don't have life in yourselves.} \v{54}\red{Whoever eats my flesh and drinks my blood has eternal life, and I'll raise him to life on the last day,} \v{55}\red{because my flesh is real\fnote{\fbackref{6:55} Or \fbib{true}} food, and my blood is real\fnote{\fbackref{6:55} Or \fbib{true}} drink.} \v{56}\red{The person who eats my flesh and drinks my blood remains in me, and I in him.} \v{57}\red{Just as the living Father sent me and I live because of the Father, so the one who feeds on me will also live because of me.} \v{58}\red{This is the bread that came down from heaven, not the kind that your ancestors ate. They died, but the one who eats this bread will live forever.''} \v{59}He said this while teaching in the synagogue at Capernaum.
\passage{The Words of Eternal Life}

\v{60}When many of his disciples heard this, they said, ``This is a difficult statement. Who can accept\fnote{\fbackref{6:60} Lit. \fbib{listen to}} it?''

\v{61}But Jesus, knowing within himself that his disciples were grumbling about this, asked them, \red{``Does this offend you?} \v{62}\red{What if you saw the Son of Man going up to the place where he was before?} \v{63}\red{It's the Spirit who gives life; the flesh accomplishes nothing. The words that I've spoken to you are spirit and life.} \v{64}\red{But there are some among you who don't believe...''}---because Jesus knew from the beginning those who weren't believing, as well as the one who would betray him. \v{65}So he said, \red{``That's why I told you that no one can come to me unless it be granted him by the Father.''} \v{66}As a result,\fnote{\fbackref{6:66} Or \fbib{From this time}} many of his disciples turned back and no longer associated\fnote{\fbackref{6:66} Lit. \fbib{walked}} with him.

\v{67}So Jesus asked the Twelve, \red{``You don't want to leave, too, do you?''}

\v{68}Simon Peter answered him, ``Lord, to whom would we go? You have the words of eternal life. \v{69}Besides, we have believed and remain convinced that you are the Holy One of God.''\fnote{\fbackref{6:69} Other mss. read \fbib{the Messiah, the Son of the living God}}

\v{70}Jesus answered them, \red{``I chose you, the Twelve, didn't I? Yet one of you is a devil.''} \v{71}Now he was speaking about Judas, the son of Simon Iscariot,\fnote{\fbackref{6:71} Other mss. read \fbib{Judas Iscariot, the son of Simon}} because this man was going to betray him, even though he was one of the Twelve.
\labelchapt{7}
\passage{The Unbelief of Jesus' Brothers}

\chapt{7}
\v{1}After this, Jesus traveled\fnote{\fbackref{7:1} Lit. \fbib{walked}} throughout Galilee, because he didn't want to travel\fnote{\fbackref{7:1} Lit. \fbib{to walk}} in Judea, since the Jewish leaders\fnote{\fbackref{7:1} I.e. Judean leaders; lit. \fbib{the Jews}} there were trying to kill him. \v{2}Now the Jewish Festival of Tents\fnote{\fbackref{7:2} Or \fbib{Booths}} was approaching. \v{3}So his brothers told him, ``You should leave this place and go to Judea, so that your disciples can see the actions that you're doing, \v{4}since no one acts in secret if he wants to be known publicly. If you're going to do these things, you should reveal yourself to the world!'' \v{5}Not even his brothers believed in him.

\v{6}Jesus told them, \red{``My time has not yet come, but your time is always here.}\fnote{\fbackref{7:6} Lit. \fbib{ready}} \v{7}\red{The world cannot hate you, but it hates me because I testify against it that its actions are evil.} \v{8}\red{Go up to the festival yourselves. I am not going to this festival yet,\fnote{\fbackref{7:8} Other mss. lack \fbib{yet}} because my time hasn't fully come yet.''} \v{9}After saying this, he remained in Galilee.
\passage{Jesus Arrives in Jerusalem}

\v{10}But after his brothers had gone up to the festival, he went up himself, not openly but, as it were,\fnote{\fbackref{7:10} Other mss. lack \fbib{as it were}} in secret. \v{11}The Jewish leaders\fnote{\fbackref{7:11} I.e. Judean leaders; lit. \fbib{The Jews}} kept looking for him at the festival, asking, ``Where is that man?'' \v{12}And there was a great deal of discussion about him among the crowds.\fnote{\fbackref{7:12} Other mss. read \fbib{crowd}} Some were saying, ``He is a good man,'' while others were saying, ``No, he is deceiving the crowds!'' \v{13}Nevertheless, no one would speak openly about him because they were afraid of the Jewish leaders.\fnote{\fbackref{7:13} I.e. Judean leaders; lit. \fbib{the Jews}}
\passage{Jesus Openly Declares His Authority}

\v{14}Halfway through the festival, Jesus went up to the Temple and began teaching. \v{15}The Jewish leaders\fnote{\fbackref{7:15} I.e. Judean leaders; lit. \fbib{the Jews}} were astonished and remarked, ``How can this man be so educated when he has never gone to school?''

\v{16}Jesus replied to them, \red{``My teaching is not mine but comes from the one who sent me.} \v{17}\red{If anyone wants to do his will, he'll know whether this teaching is from God or whether I'm speaking on my own.} \v{18}\red{The one who speaks on his own seeks his own praise. But the one who seeks the praise of him who sent him is genuine, and there's nothing false in him.} \v{19}\red{Moses gave you the Law, didn't he? Yet none of you is keeping the Law. Why are you trying to kill me?''}

\v{20}The crowd answered, ``You have a demon! Who is trying to kill you?''

\v{21}Jesus answered them, \red{``I performed one action, and all of you are astonished.} \v{22}\red{Moses gave you circumcision---not that it is from Moses, but from the Patriarchs---and so you circumcise a man on the Sabbath.} \v{23}\red{If a man receives circumcision on the Sabbath so that the Law of Moses may not be broken, are you angry with me because I made a man perfectly well on the Sabbath?} \v{24}\red{Stop judging by appearances, but judge with righteous judgment!''}
\passage{Is This the Messiah?}

\v{25}Then some of the people of Jerusalem began saying, ``This is the man they are trying to kill, isn't it? \v{26}And look, he is speaking in public, and they are not saying anything to him! Can it be that the authorities really know that this is the Messiah?\fnote{\fbackref{7:26} Or \fbib{Christ}} \v{27}We know where this man comes from. But when the Messiah\fnote{\fbackref{7:27} Or \fbib{Christ}} comes, no one will know where he comes from.''

\v{28}At this point Jesus, still teaching in the Temple, shouted, \red{``So you know me and know where I've come from? I haven't come on my own accord. But the one who sent me is true, and he's the one you don't know.} \v{29}\red{I know him because I've come from him, and he sent me.''}

\v{30}Then the Jewish leaders\fnote{\fbackref{7:30} I.e. Judean leaders; lit. \fbib{the Jews}} tried to seize him, but no one laid a hand on him because his hour had not yet come. \v{31}However, many in the crowd believed in him. They kept saying, ``When the Messiah\fnote{\fbackref{7:31} Or \fbib{Christ}} comes, he won't do more signs than this man has done, will he?''
\passage{Officers are Sent to Arrest Jesus}

\v{32}The Pharisees heard the crowd debating these things about him, so the high priests and the Pharisees sent officers to arrest Jesus.\fnote{\fbackref{7:32} Lit. \fbib{him}}

\v{33}Then Jesus said, \red{``I'll be with you only a little while longer, and then I'm going back to the one who sent me.} \v{34}\red{You'll look for me but won't find me.\fnote{\fbackref{7:34} Other mss. lack \fbib{me}} And where I am, you cannot come.''}

\v{35}Then the Jewish leaders\fnote{\fbackref{7:35} I.e. Judean leaders; lit. \fbib{the Jews}} asked one another, ``Where does this man intend to go that we won't be able to find him? Surely he's not going to the Dispersion\fnote{\fbackref{7:35} I.e. the Jewish communities outside the land of Israel} among the Greeks and teach the Greeks, is he? \v{36}What does this statement mean that he said, \red{`You'll look for me but won't find me,'} and, \red{`Where I am, you cannot come'}?''
\passage{Rivers of Living Water}

\v{37}On the last and most important day of the festival, Jesus stood up and shouted, \red{``If anyone is thirsty, let him come to me\fnote{\fbackref{7:37} Other mss. lack \fbib{to me}} and drink!} \v{38}\red{The one who believes in me, as the Scripture has said, will have rivers of living water flowing from his heart.''} \v{39}Now he said this about the Spirit, whom those who were believing in him were to receive, because the Spirit\fnote{\fbackref{7:39} Other mss. read \fbib{Holy Spirit}} was not yet present\fnote{\fbackref{7:39} Other mss. read \fbib{given}} and Jesus had not yet been glorified.
\passage{Division among the People}

\v{40}When they heard these words, some in the crowd were saying, ``This really is the Prophet,'' \v{41}while others were saying, ``This is the Messiah!''\fnote{\fbackref{7:41} Or \fbib{Christ}}

But some were saying, ``The Messiah\fnote{\fbackref{7:41} Or \fbib{Christ}} doesn't come from Galilee, does he? \v{42}Doesn't the Scripture say that the Messiah\fnote{\fbackref{7:42} Or \fbib{Christ}} is from David's family and from Bethlehem, the village where David lived?'' \v{43}So there was a division in the crowd because of him. \v{44}Some of them were wanting to seize him, but no one laid hands on him.
\passage{The Unbelief of the Authorities}

\v{45}Then the officers returned to the high priests and Pharisees, who asked them, ``Why didn't you bring him?''

\v{46}The officers answered, ``No man ever spoke like that!''

\v{47}Then the Pharisees replied to them, ``You haven't been deceived, too, have you? \v{48}None of the authorities or Pharisees has believed in him, have they? \v{49}But this mob that does not know the Law---they're under a curse!''

\v{50}One of their own, Nicodemus (the man who had previously met with Jesus),\fnote{\fbackref{7:50} Lit. \fbib{him}} asked them, \v{51}``Surely our Law does not condemn\fnote{\fbackref{7:51} Or \fbib{judge}} a person without first hearing from him and finding out what he is doing, does it?''

\v{52}They answered him, ``You aren't from Galilee, too, are you? Search and see that no prophet comes from Galilee.'' \v{53}Then all of them went to their own homes.
\labelchapt{8}
\passage{The Woman Caught in Adultery}

\chapt{8}
\v{1}Jesus, however, went to the Mount of Olives. \v{2}At daybreak he appeared again in the Temple, and all the people came to him. So he sat down and began to teach them. \v{3}But the scribes and the Pharisees brought a woman who had been caught in adultery.\fnote{\fbackref{8:3} Other mss. read \fbib{in sin}} After setting her before them,\fnote{\fbackref{8:3} Lit. \fbib{in the middle}} \v{4}they told him, ``Teacher, this woman has been caught in the very act of adultery. \v{5}Now in the Law, Moses commanded us to stone such women to death. What do you say?'' \v{6}They said this to test him, so that they might have a charge against him. But Jesus bent down and began to write on the ground with his finger.

\v{7}When they persisted in questioning him, he straightened up and told them, \red{``Let the person among you who is without sin be the first to throw a stone at her.''} \v{8}Then he bent down again and continued writing on the ground.\fnote{\fbackref{8:8} Other mss read \fbib{ground the sins of each one of them}} \v{9}When they heard this, they went away one by one,\fnote{\fbackref{8:9} Other mss. read \fbib{one by one, being convicted by their conscience}} beginning with the oldest,\fnote{\fbackref{8:9} Other mss. read \fbib{from the oldest to the youngest}} and he was left alone with the woman standing there.\fnote{\fbackref{8:9} Lit. \fbib{in the middle}} \v{10}Then Jesus stood up and asked her, \red{``Dear lady,\fnote{\fbackref{8:10} Or \fbib{Woman}} where are your accusers?\fnote{\fbackref{8:10} Other mss. read \fbib{where are they?}} Hasn't anyone condemned you?''}

\v{11}``No one, sir,''\fnote{\fbackref{8:11} Or \fbib{Lord}} she replied.

Then Jesus said, \red{``I don't condemn you, either. Go home, and from now on don't sin anymore.''}\fnote{\fbackref{8:11} Some mss. lack 7:53-8:11; one ms. locates the text after Luke 21:38; some mss. locate it after John 7:36, some after John 7:52, and some after John 21:25.}
\passage{Jesus the Light of the World}

\v{12}Later on, Jesus spoke to them again, saying, \red{``I am the light of the world. The one who follows me will never walk in darkness, but will have the light of life.''}

\v{13}The Pharisees told him, ``You're testifying about yourself. Your testimony isn't valid.''\fnote{\fbackref{8:13} Or \fbib{true}}

\v{14}Jesus answered them, \red{``Even though I'm testifying about myself, my testimony is valid\fnote{\fbackref{8:14} Or \fbib{true}} because I know where I've come from and where I'm going. But you don't know where I come from or where I'm going.} \v{15}\red{You're judging by human standards,\fnote{\fbackref{8:15} Lit. \fbib{according to the flesh}} but I'm not judging anyone.} \v{16}\red{Yet even if I should judge, my judgment would be valid,\fnote{\fbackref{8:16} Or \fbib{true}} because it is not I alone who judges, but I and the one who sent me.} \v{17}\red{In your own Law it is written that the testimony of two people is valid.}\fnote{\fbackref{8:17} Or \fbib{true}} \v{18}\red{I'm testifying about myself, and the Father who sent me is testifying about me.''}

\v{19}Then they asked him, ``Where is this Father of yours?''

Jesus replied, \red{``You don't know me or my Father. If you had known me, you would've known my Father, too.''} \v{20}He spoke these words in the treasury, while he was teaching in the Temple. Yet no one arrested him, because his hour had not yet come.
\passage{The One from Above}

\v{21}Later on, he told them again, \red{``I'm going away, and you'll look for me, but you will die in your sin.\fnote{\fbackref{8:21} I.e. their sinful nature} You cannot come where I'm going.''}

\v{22}So the Jewish leaders\fnote{\fbackref{8:22} I.e. Judean leaders; lit. \fbib{the Jews}} were asking, ``He isn't going to kill himself, is he? Is that why he said,\fnote{\fbackref{8:22} Lit. \fbib{Because he said}} \red{`You cannot come where I'm going'}?''

\v{23}He told them, \red{``You are from below, I'm from above. You are of this world, but I'm not of this world.} \v{24}\red{That is why I told you that you will die in your sins, for unless you believe that I AM, you'll die in your sins.''}

\v{25}Then they asked him, ``Who are you?''

Jesus told them, \red{``What have I been telling you all along?}\fnote{\fbackref{8:25} Or \fbib{from the beginning}} \v{26}\red{I have much to say about you and to condemn you for.\fnote{\fbackref{8:26} The Gk. lacks \fbib{you for}} But the one who sent me is truthful,\fnote{\fbackref{8:26} Or \fbib{true}} and what I've heard from him I declare to the world.''}

\v{27}They didn't realize that he was talking to them about the Father. \v{28}So Jesus told them, \red{``When you have lifted up the Son of Man, then you will know that I AM, and that I do nothing on my own authority. Instead, I speak only what the Father has taught me.} \v{29}\red{Moreover, the one who sent me is with me. He has never left me alone, because I always do what pleases him.''} \v{30}While he was saying these things, many believed in him.
\passage{Freedom and Slavery}

\v{31}So Jesus told those Jews who had believed in him, \red{``If you continue in my word, you are really my disciples.} \v{32}\red{And you will know the truth, and the truth will set you free.''}

\v{33}They replied to him, ``We are Abraham's descendants and have never been slaves to anybody. So how can you say, \red{`You will be set free'}?''

\v{34}Jesus answered them, \red{``Truly, I tell all of you\fnote{\fbackref{8:34} The Gk. pronoun \fbib{you} is pl.} emphatically, that everyone who commits sin is a slave of sin.}\fnote{\fbackref{8:34} Other mss. lack \fbib{of sin}} \v{35}\red{The slave does not remain in the household forever, but the son does remain forever.} \v{36}\red{So if the Son sets you free, you will be free indeed!''}
\passage{The Real Children of Abraham}

\v{37}\red{``I know that you are Abraham's descendants. Yet you are trying to kill me because you've not received what I've told you.} \v{38}\red{I declare what I've seen in my\fnote{\fbackref{8:38} Other mss. read \fbib{the}} Father's presence, and you're doing what you've heard from your father.''}

\v{39}They replied to him, ``Our father is Abraham!''

Jesus told them, \red{``If you were Abraham's children, you would be doing what Abraham did.}\fnote{\fbackref{8:39} Lit. \fbib{the works of Abraham}} \v{40}\red{But now you're trying to kill me, a man who has told you the truth that I heard from God. Abraham would'nt have done that.} \v{41}\red{You are doing your father's actions.''}

They told him, ``We're not illegitimate children. We have one Father, God himself.''

\v{42}Jesus told them, \red{``If God were your Father, you would've loved me, because I came from God and am here. I haven't come on my own accord, but he sent me.} \v{43}\red{Why don't you understand what I've said? It's because you can't listen to my words.} \v{44}\red{You belong to your father the devil, and you want to carry out the desires of your father. He was a murderer from the beginning and has never stood for truth, since there is no truth in him. Whenever he tells a lie, he speaks in character, because he is a liar and the father of lies.} \v{45}\red{But it is because I speak the truth that you don't believe me.} \v{46}\red{Can any of you prove me guilty of sin? If I'm telling the truth, why don't you believe me?} \v{47}\red{The one who belongs to God listens to the words of God. The reason you don't listen is because you don't belong to God.''}
\passage{Jesus is Superior to Abraham}

\v{48}The Jewish leaders\fnote{\fbackref{8:48} I.e. Judean leaders; lit. \fbib{the Jews}} replied to him, ``Surely we're right in saying that you are a Samaritan and have a demon, aren't we?''

\v{49}Jesus answered, \red{``I don't have a demon. On the contrary, I honor my Father, and you dishonor me.} \v{50}\red{I don't seek my own glory. There is one who seeks it, and he is the Judge.} \v{51}\red{Truly, I tell all of you\fnote{\fbackref{8:51} The Gk. pronoun \fbib{you} is pl.} emphatically, if anyone keeps my word, he will never see death.''}

\v{52}Then the Jewish leaders\fnote{\fbackref{8:52} I.e. Judean leaders; lit. \fbib{The Jews}} told him, ``Now we really know that you have a demon. Abraham died, and so did the prophets, but you say, \red{`If anyone keeps my word, he will never taste death.'} \v{53}You aren't greater than our father Abraham, who died, are you? The prophets also died. Who are you making yourself out to be?''

\v{54}Jesus answered, \red{``If I were trying to glorify myself, my glory would mean nothing. It is my Father who glorifies me, of whom you say, `He is our God.'} \v{55}\red{You don't know him, but I know him. If I were to say that I don't know him, I would be a liar like you. But I do know him and keep his word.} \v{56}\red{Your father Abraham rejoiced that he would see my day, and he saw it and was glad.''}

\v{57}Then the Jewish leaders\fnote{\fbackref{8:57} I.e. Judean leaders; lit. \fbib{the Jews}} asked him, ``You are not even 50 years old, yet you have seen Abraham?''\fnote{\fbackref{8:57} Other mss. read \fbib{Abraham has seen you?}}

\v{58}Jesus told them, \red{``Truly, I tell all of you\fnote{\fbackref{8:58} The Gk. pronoun \fbib{you} is pl.} emphatically, before there was an Abraham, I AM!''} \v{59}At this, they picked up stones to throw at him, but Jesus hid himself and went out of the Temple.
\labelchapt{9}
\passage{Jesus Heals a Blind Man}

\chapt{9}
\v{1}As he was walking along, he observed a man who had been blind from birth. \v{2}His disciples asked him, ``Rabbi,\fnote{\fbackref{9:2} \fbib{Rabbi} is Heb. for \fbib{Master} and/or \fbib{Teacher}} who sinned, this man or his parents, that caused him to be born blind?''

\v{3}Jesus answered, \red{``Neither this man nor his parents sinned. This happened so that\fnote{\fbackref{9:3} Lit. \fbib{But so that}} God's work might be revealed in him.} \v{4}\red{I\fnote{\fbackref{9:4} Other mss. read \fbib{We}} must do the work of the one who sent me\fnote{\fbackref{9:4} Other mss. read \fbib{us}} while it is day. Night is approaching, when no one can work.} \v{5}\red{As long as I'm in the world, I'm the light of the world.''} \v{6}After saying this, he spit on the ground and made mud with the saliva. Then he spread the mud on the man's eyes \v{7}and told him, \red{``Go and wash in the pool of Siloam''} (which is translated ``Sent One''). So he went off, washed, and came back seeing.

\v{8}Then the neighbors and those who had previously seen him as a beggar said, ``This is the man who used to sit and beg, isn't it?''

\v{9}Some were saying, ``It's him,'' while others were saying, ``No, but it's someone like him.''

But he himself kept saying, ``It's me!''

\v{10}So they asked him, ``How, then, did you gain your eyesight?''

\v{11}He said, ``The man named Jesus made some mud, spread it on my eyes, and told me, \red{`Go to Siloam and wash.'} So off I went and washed, and I received my sight.''

\v{12}They asked him, ``Where is that man?''

He said, ``I don't know!''
\passage{The Pharisees Investigate the Healing}

\v{13}So they brought to the Pharisees the man who had once been blind. \v{14}Now it was a Sabbath day when Jesus made the mud and healed\fnote{\fbackref{9:14} Lit. \fbib{opened}} his eyes. \v{15}So the Pharisees also began to ask him how he had gained his sight. He told them, ``He put mud on my eyes, then I washed, and now I can see.''

\v{16}Some of the Pharisees began to remark, ``This man is not from God because he does not keep the Sabbath.''

But others were saying, ``How can a sinful man perform such signs?'' And there was a division among them.

\v{17}So they asked the formerly\fnote{\fbackref{9:17} The Gk. lacks \fbib{formerly}} blind man again, ``What do you say about him, since it was your eyes he healed?''\fnote{\fbackref{9:17} Lit. \fbib{opened}}

He said, ``He is a prophet.''

\v{18}The Jewish leaders\fnote{\fbackref{9:18} I.e. Judean leaders; lit. \fbib{The Jews}} did not believe that the man\fnote{\fbackref{9:18} Lit. \fbib{believe about him that he}} had been blind and had gained sight until they summoned his parents\fnote{\fbackref{9:18} Lit. \fbib{the parents of the man who had been given sight}} \v{19}and asked them, ``Is this your son, the one you say was born blind? How does he now see?''

\v{20}His parents replied, ``We know that this is our son and that he was born blind. \v{21}But we don't know how it is that he now sees, and we don't know who opened his eyes. Ask him. He is of age and can speak for himself.'' \v{22}His parents said this because they were afraid of the Jewish leaders,\fnote{\fbackref{9:22} I.e. Judean leaders; lit. \fbib{the Jews}} since the Jewish leaders\fnote{\fbackref{9:22} I.e. Judean leaders; lit. \fbib{the Jews}} had already agreed that anyone who acknowledged that Jesus\fnote{\fbackref{9:22} Lit. \fbib{he}} was the Messiah\fnote{\fbackref{9:22} Or \fbib{Christ}} would be thrown out of the synagogue. \v{23}That's why his parents said, ``He is of age. Ask him.''

\v{24}The Jewish leaders\fnote{\fbackref{9:24} Lit. \fbib{They}} summoned the man who had been blind a second time and told him, ``Give glory to God! We know that this man is a sinner.''

\v{25}But he responded, ``I don't know whether he is a sinner or not. The one thing I do know is that I used to be blind and now I can see!''

\v{26}Then they asked him, ``What did he do to you? How did he heal\fnote{\fbackref{9:26} Lit. \fbib{open}} your eyes?''

\v{27}He answered them, ``I've already told you, but you didn't listen. Why do you want to hear it again? You don't want to become his disciples, too, do you?''

\v{28}At this, they turned on him in fury and said, ``You are his disciple, but we are disciples of Moses! \v{29}We know that God has spoken to Moses, but we do not know where this fellow comes from.''

\v{30}The man answered them, ``This is an amazing thing! You don't know where he comes from, yet he healed\fnote{\fbackref{9:30} Lit. \fbib{opened}} my eyes. \v{31}We know that God doesn't listen to sinners, but he does listen to anyone who worships him and does his will. \v{32}Ever since creation it has never been heard that anyone healed\fnote{\fbackref{9:32} Lit. \fbib{opened}} the eyes of a man who was born blind. \v{33}If this man were not from God, he couldn't do anything like that.''

\v{34}They asked him, ``You were born a sinner\fnote{\fbackref{9:34} Lit. \fbib{born entirely in sins}} and you are trying to instruct us?'' And they threw him out.
\passage{Spiritual Blindness}

\v{35}Jesus heard that they had thrown him out. So when he found him, he asked him, \red{``Do you believe in the Son of Man?''}\fnote{\fbackref{9:35} Other mss. read \fbib{Son of God}}

\v{36}He answered, ``And who is he, sir?\fnote{\fbackref{9:36} Or \fbib{Lord}} Tell me,\fnote{\fbackref{9:36} The Gk. lacks \fbib{Tell me}} so that I may believe in him.''

\v{37}Jesus told him, \red{``You have seen him. He is the person who is talking with you.''}

\v{38}He said, ``Lord, I do believe,'' and worshipped him.

\v{39}Then Jesus said, \red{``I have come into this world to judge it, so that those who are blind may see and so that those who see may become blind.''}

\v{40}Some of the Pharisees who were near him overheard this and asked him, ``We aren't blind, too, are we?''

\v{41}Jesus told them, \red{``If you were blind, you would not have any sin. But now that you insist, `We see,' your sin still exists.''}
\labelchapt{10}
\passage{The Illustration of the Sheepfold}

\chapt{10}
\v{1}\red{``Truly, I tell all of you\fnote{\fbackref{10:1} The Gk. pronoun \fbib{you} is pl.} emphatically, the person who doesn't enter the sheepfold through the gate, but climbs in by some other way, is a thief and a bandit.} \v{2}\red{The one who enters through the gate is the shepherd of the sheep.} \v{3}\red{It's to him the gatekeeper opens the gate, and it's his voice the sheep hear. He calls his own sheep by name and leads them out.} \v{4}\red{When he has driven out all his own, he goes ahead of them, and the sheep follow him because they recognize his voice.} \v{5}\red{They'll never follow a stranger, but will run away from him because they don't recognize the voice of strangers.''} \v{6}Jesus used this illustration with them, but they didn't understand what he was saying to them.
\passage{Jesus the Good Shepherd}

\v{7}So again Jesus said, \red{``Truly, I tell all of you\fnote{\fbackref{10:7} The Gk. pronoun \fbib{you} is pl.} emphatically, I'm the gate for the sheep.} \v{8}\red{All who came before me\fnote{\fbackref{10:8} Other mss. lack \fbib{before me}} are thieves and bandits, but the sheep didn't listen to them.} \v{9}\red{I'm the gate. If anyone enters through me, he will be saved. He'll come in and go out and find pasture.} \v{10}\red{The thief comes only to steal, slaughter, and destroy. I've come that they may have life, and have it abundantly.}

\v{11}\red{``I'm the good shepherd. The good shepherd lays down\fnote{\fbackref{10:11} Other mss. read \fbib{gives}} his life for the sheep.} \v{12}\red{The hired worker, who isn't the shepherd and doesn't own the sheep, sees the wolf coming, deserts the sheep, and runs away. So the wolf snatches them and scatters them,} \v{13}\red{because he's a hired worker, and the sheep don't matter to him.}

\v{14}\red{I'm the good shepherd. I know my own and my own know me,} \v{15}\red{just as the Father knows me and I know the Father. And I lay down\fnote{\fbackref{10:15} Other mss. read \fbib{give}} my life for the sheep.} \v{16}\red{I have other sheep that don't belong to this fold. I must lead these also, and they'll listen to my voice. So there will be one flock and one shepherd.} \v{17}\red{This is why the Father loves me, because I lay down my life in order to take it back again.} \v{18}\red{No one is taking it from me; I lay it down of my own free will. I have the authority to lay it down, and I have the authority to take it back again. This is what my Father has commanded me.''}

\v{19}Once again there was a division among the Jews\fnote{\fbackref{10:19} I.e. Judean leaders} because of what Jesus had been saying. \v{20}Many of them were saying, ``He has a demon and is insane. Why bother listening to him?''

\v{21}Others were saying, ``These are not the words of a man who is demon-possessed. A demon cannot open the eyes of the blind, can it?''
\passage{Jesus is Rejected by the Jews}

\v{22}Now\fnote{\fbackref{10:22} Other mss. read \fbib{Then}} Hanukkah\fnote{\fbackref{10:22} Or \fbib{the Festival of Dedication}} was taking place in Jerusalem. It was winter, \v{23}and Jesus was walking around in the Temple inside the open porch of Solomon. \v{24}So the Jewish leaders\fnote{\fbackref{10:24} I.e. Judean leaders; lit. \fbib{the Jews}} surrounded him and quizzed him, ``How long are you going to keep us in suspense? If you're the Messiah,\fnote{\fbackref{10:24} Or \fbib{Christ}} tell us so plainly.''

\v{25}Jesus answered them, \red{``I have told you, but you don't believe it. The actions that I do in my Father's name testify on my behalf,} \v{26}\red{but you don't believe, because you don't belong to my sheep.}\fnote{\fbackref{10:26} Other mss. read \fbib{my sheep, just as I told you}} \v{27}\red{My sheep hear my voice. I know them, and they follow me.} \v{28}\red{I give them eternal life, they'll never be lost, and no one will snatch them out of my hand.} \v{29}\red{What my Father has given me\fnote{\fbackref{10:29} Some MSS read \fbib{My father, who gave them to me,}} is more important than anything,\fnote{\fbackref{10:29} Or \fbib{is greater than everything else}} and no one can snatch it from the Father's hand.} \v{30}\red{I and the Father are one.''}

\v{31}Again the Jewish leaders\fnote{\fbackref{10:31} I.e. Judean leaders; lit. \fbib{the Jews}} picked up stones to stone him to death.

\v{32}Jesus replied to them, \red{``I've shown you many good actions from my\fnote{\fbackref{10:32} Other mss. read \fbib{the}} Father. For which of them are you going to stone me?''}

\v{33}The Jewish leaders\fnote{\fbackref{10:33} I.e. Judean leaders; lit. \fbib{The Jews}} answered him, ``We are not going to stone you for a good action, but for blasphemy, because you, a mere man, are making yourself God!''

\v{34}Jesus replied to them, \red{``Is it not written in your\fnote{\fbackref{10:34} Other mss. read \fbib{the}} Law, `I said, ``You are gods''\,'?}\fnote{\fbackref{10:34} Cf. Ps 82:6} \v{35}\red{If he called those to whom a message from God came `gods' (and the Scripture cannot be disregarded),} \v{36}\red{how can you say to the one whom the Father has consecrated and sent into the world, `You're blaspheming,' because I said, `I'm the Son of God'?} \v{37}\red{If I'm not doing my Father's actions, don't believe me.} \v{38}\red{But if I'm doing them, even though you don't believe me, believe the actions, so that you may know and understand\fnote{\fbackref{10:38} Other mss. read \fbib{believe}} that the Father is in me and I am in the Father.''}

\v{39}Again they tried to seize him, but he slipped away from them.\fnote{\fbackref{10:39} Lit. \fbib{away out of their hands}} \v{40}Then he went away again across the Jordan to the place where John had been baptizing at first, and he remained there. \v{41}Many people came to him and kept saying, ``John never performed a sign, but everything that John said about this man is true!'' \v{42}And many believed in Jesus\fnote{\fbackref{10:42} Lit. \fbib{him}} there.
\labelchapt{11}
\passage{The Death of Lazarus}

\chapt{11}
\v{1}Now a certain man was ill, Lazarus from Bethany, the village of Mary and her sister Martha. \v{2}Mary was the woman who anointed the Lord with perfume and wiped his feet with her hair. Her brother Lazarus was the one who was ill. \v{3}So the sisters sent word to Jesus\fnote{\fbackref{11:3} Lit. \fbib{sent to him}} and told him, ``Lord, the one whom you love is ill.''

\v{4}But when Jesus heard it, he said, \red{``This illness isn't meant to end in death. It's for God's glory, so that the Son of God may be glorified through it.''} \v{5}Now Jesus loved Martha and her sister and Lazarus. \v{6}Yet, when he heard that Lazarus\fnote{\fbackref{11:6} Lit. \fbib{he}} was ill, he stayed where he was for two more days.

\v{7}After this, he told the disciples, \red{``Let's go back to Judea.''}

\v{8}The disciples told him, ``Rabbi,\fnote{\fbackref{11:8} \fbib{Rabbi} is Heb. for \fbib{Master} and/or \fbib{Teacher}} the Jewish leaders\fnote{\fbackref{11:8} I.e. Judean leaders; lit. \fbib{the Jews}} were just now trying to stone you to death, and you are going back there again?''

\v{9}Jesus replied, \red{``There are twelve hours in the day, aren't there? If anyone walks during the day he does not stumble, because he sees the light of this world.} \v{10}\red{But if anyone walks at night he stumbles, because the light is not in him.''} \v{11}These were the things he said.

Then after this, he told them, \red{``Our friend Lazarus has fallen asleep, but I'm leaving to wake him up.''}

\v{12}So the disciples told him, ``Lord, if he has fallen asleep, he will get well.'' \v{13}Jesus, however, had been speaking about his death, but they thought that he was speaking about resting or sleeping.

\v{14}Then Jesus told them plainly, \red{``Lazarus has died.} \v{15}\red{For your sake I'm glad that I wasn't there, so that you may believe. But let's go to him.''}

\v{16}Then Thomas, who was called the Twin,\fnote{\fbackref{11:16} Lit. \fbib{Didymus}} told his fellow disciples, ``Let's go, too, so that we may die with him!''
\passage{Jesus the Resurrection and the Life}

\v{17}When Jesus arrived, he found that Lazarus\fnote{\fbackref{11:17} Lit. \fbib{he}} had already been in the tomb for four days. \v{18}Now Bethany was near Jerusalem, about fifteen stadia\fnote{\fbackref{11:18} I.e. about two miles; the Roman mile contained eight stadia, one stadion was about 604.5 feet long.} away, \v{19}and many of the Jews had come to Martha and Mary to console them about their brother. \v{20}As soon as Martha heard that Jesus was coming, she went and met him, while Mary stayed at home.

\v{21}Martha told Jesus, ``Lord, if you had been here, my brother would not have died. \v{22}But even now I know that whatever you ask of God, he\fnote{\fbackref{11:22} Lit. \fbib{God}} will give it to you.''

\v{23}Jesus told her, \red{``Your brother will rise again.''}

\v{24}Martha told him, ``I know that he will rise again in the resurrection on the last day.''

\v{25}Jesus told her, \red{``I am the resurrection and the life.\fnote{\fbackref{11:25} Other mss. lack \fbib{and the life}} The person who believes in me, even though he dies, will live.} \v{26}\red{Indeed, everyone who lives and believes in me will never die. Do you believe that?''}

\v{27}``Yes, Lord,'' she told him. ``I believe that you are the Messiah,\fnote{\fbackref{11:27} Or \fbib{Christ}} the Son of God, the one who was to come into the world.''

\v{28}When she had said this, she went away and called her sister Mary and told her privately, ``The Teacher is here and is calling for you!''

\v{29}As soon as Mary\fnote{\fbackref{11:29} Lit. \fbib{she}} heard this, she got up quickly and went to him. \v{30}Now Jesus had not yet arrived at the village but was still at the place where Martha had met him. \v{31}When the Jewish leaders\fnote{\fbackref{11:31} I.e. Judean leaders; lit. \fbib{the Jews}} who had been with her, consoling her in the house, saw Mary get up quickly and go out, they followed her, thinking that she had gone to the tomb to cry there.

\v{32}As soon as Mary came to where Jesus was and saw him, she fell down at his feet and told him, ``Lord, if you had been here, my brother wouldn't have died.''

\v{33}When Jesus saw her crying, and the Jews who had come with her crying, he was greatly troubled in spirit and deeply moved. \v{34}He asked, \red{``Where have you put him?''}

They told him, ``Lord, come and see.''

\v{35}Jesus burst into tears.

\v{36}So the Jewish leaders\fnote{\fbackref{11:36} I.e. Judean leaders; lit. \fbib{the Jews}} said, ``See how much he loved him!''

\v{37}But some of them said, ``Surely the one who opened the eyes of the blind man could have kept this man from dying, couldn't he?''
\passage{Jesus Brings Lazarus Back to Life}

\v{38}Groaning deeply again, Jesus came to the tomb. It was a cave, and a stone was lying in front of it. \v{39}Jesus said, \red{``Remove the stone.''}

Martha, the dead man's sister, told him, ``Lord, there must be a stench by now, because he's been dead for four days.''

\v{40}Jesus told her, \red{``I told you that if you believed you would see God's glory, didn't I?''} \v{41}So they removed the stone.

Then Jesus looked upward and said, \red{``Father, I thank you for hearing me.} \v{42}\red{I know that you always hear me, but I have said this for the sake of the crowd standing here, so that they may believe that you sent me.''}

\v{43}After saying this, he shouted with a loud voice, \red{``Lazarus, come out!''} \v{44}The man who had died came out, his hands and feet tied with strips of cloth, and his face wrapped in a handkerchief. Jesus told them, \red{``Untie him, and let him go.''}
\passage{The Jewish Council Plans to Kill Jesus}
\passageinfo{(Matthew 26:1; Mark 14:1-2; Luke 22:1-2)}

\v{45}Many of the Jews who had come with Mary and who had observed what Jesus did believed in him. \v{46}Some of them, however, went to the Pharisees and told them what Jesus had done. \v{47}So the high priests and the Pharisees assembled the Council\fnote{\fbackref{11:47} Or \fbib{Sanhedrin}} and said, ``What are we going to do? This man is performing many signs. \v{48}If we let him go on like this, everyone will believe in him, and the Romans will come and destroy both our Temple\fnote{\fbackref{11:48} Lit. \fbib{place}} and our nation.''

\v{49}But one of them, Caiaphas, who was high priest that year, told them, ``You don't know anything! \v{50}You don't realize that it is better for you\fnote{\fbackref{11:50} Other mss. read \fbib{for us}} to have one man die for the people than to have the whole nation destroyed.'' \v{51}Now he did not say this on his own initiative. As high priest that year, he prophesied that Jesus would die for the nation, \v{52}and not only for the nation, but that he would also gather into one the children of God who were scattered abroad. \v{53}So from that day on they resolved to put him to death. \v{54}As a result, Jesus no longer walked openly among the Jews.\fnote{\fbackref{11:54} I.e. Judean leaders} Instead, he went from there\fnote{\fbackref{11:54} I.e. from Bethany} to a town called Ephraim in the region near the wilderness. There he remained with his disciples.

\v{55}Now the Jewish Passover was approaching, and before the Passover many people from the countryside went up to Jerusalem to purify themselves. \v{56}They kept looking for Jesus and saying to one another as they stood in the Temple, ``What do you think? Surely he won't come to the festival, will he?'' \v{57}Now the high priests and the Pharisees had given orders that whoever knew where he was should tell them so that they could arrest him.
\labelchapt{12}
\passage{Mary Anoints Jesus}
\passageinfo{(Matthew 26:6-13; Mark 14:3-9)}

\chapt{12}
\v{1}Six days before the Passover, Jesus arrived in Bethany, where Lazarus lived,\fnote{\fbackref{12:1} Lit. \fbib{was}} the man whom Jesus had raised from the dead. \v{2}There they gave a dinner for him. Martha served, and Lazarus was one of those at the table with him. \v{3}Mary took a litron\fnote{\fbackref{12:3} I.e. about twelve ounces; the Gk. \fbib{litron} weighed about 12 ounces} of very expensive perfume made of pure nard and anointed Jesus' feet. She wiped his feet with her hair, and the house became filled with the fragrance of the perfume.

\v{4}But Judas Iscariot, one of his disciples, who was going to betray him, asked, \v{5}``Why wasn't this perfume sold for 300 denarii\fnote{\fbackref{12:5} Three hundred denarii was about a year's wages for a laborer.} and the money\fnote{\fbackref{12:5} The Gk. lacks \fbib{the money}} given to the destitute?'' \v{6}He said this, not because he cared about the destitute, but because he was a thief. He was in charge of the moneybag and would steal what was put into it.

\v{7}Then Jesus said, \red{``Leave her alone so she can observe the day of my burial,} \v{8}\red{because you will always have the destitute with you, but you won't always have me.''}
\passage{The Plot against Lazarus}

\v{9}When the large crowd of Jews realized that he was there, they came not only because of Jesus but also to see Lazarus, whom he had raised from the dead. \v{10}So the high priests planned to kill Lazarus, too, \v{11}since he was the reason why so many of the Jews were leaving to believe in Jesus.
\passage{The King Enters Jerusalem}
\passageinfo{(Matthew 21:1-11; Mark 11:1-11; Luke 19:28-40)}

\v{12}The next day, the large crowd that had come to the festival heard that Jesus was coming into Jerusalem. \v{13}So they took branches of palm trees and went out to meet him, shouting,

\begin{poetry}
\poeml ``Hosanna!\fnote{\fbackref{12:13} \fbib{Hosanna} is Heb. for \fbib{Please save} or \fbib{Praise}} \\
\poeml How blessed is the one who comes \\
\poemll    in the name of the Lord,\fnote{\fbackref{12:13} Cf. Ps 118:25-26; MT source citation reads \fbib{\divine{Lord}}} the King of Israel!''
\end{poetry}

\v{14}Then Jesus found a young donkey and sat upon it, as it is written:

\begin{poetry}
\poeml \v{15}``Stop being afraid, people\fnote{\fbackref{12:15} Lit. \fbib{daughter}} of Zion. \\
\poeml Look, your king is coming, \\
\poemll    sitting upon a donkey's colt!''\fnote{\fbackref{12:15} Cf. Zech 9:9}
\end{poetry}

\v{16}At first, his disciples didn't understand these things. However, when Jesus had been glorified, they remembered that these things had been written about him and that people\fnote{\fbackref{12:16} Lit. \fbib{they}} had done these things to him. \v{17}So the crowd that had been with him when he called Lazarus out of the tomb and raised him from the dead continued to testify to what they had seen.\fnote{\fbackref{12:17} The Gk. lacks \fbib{to what they had seen}} \v{18}The crowd was going out to meet Jesus\fnote{\fbackref{12:18} Lit. \fbib{him}} because they had heard that he had performed this sign. \v{19}Then the Pharisees told one another, ``You see, there is nothing you can do. Look, the world has gone after him!''
\passage{Some Greeks Ask to See Jesus}

\v{20}Now some Greeks were among those who had come up to worship at the festival. \v{21}They went to Philip (who was from Bethsaida in Galilee) and told him, ``Sir, we would like to see Jesus.''

\v{22}Philip went and told Andrew, and Andrew and Philip went and told Jesus. \v{23}Jesus told them, \red{``The hour has come for the Son of Man to be glorified.} \v{24}\red{Truly, I tell all of you\fnote{\fbackref{12:24} The Gk. pronoun \fbib{you} is pl.} emphatically, unless a grain of wheat falls into the ground and dies, it remains alone. But if it dies, it produces a lot of grain.} \v{25}\red{The one who loves his life will destroy it, and the one who hates his life in this world will preserve it for eternal life.} \v{26}\red{If anyone serves me, he must follow me. And where I am, there my servant will also be. If anyone serves me, the Father will honor him.''}
\passage{Jesus Speaks about His Death}

\v{27}\red{``Now my soul is in turmoil, and what should I say---`Father, save me from this hour'? No! It was for this very reason that I came to this hour.} \v{28}\red{Father, glorify your name.''}

Then a voice came from heaven, ``I have glorified it, and I will glorify it again!'' \v{29}The crowd standing there heard this and said that it was thunder.

Others were saying, ``An angel has spoken to him.''

\v{30}Jesus replied, \red{``This voice is for your benefit, not for mine.} \v{31}\red{Now is the time for the judgment of this world to begin.\fnote{\fbackref{12:31} Lit. \fbib{Now is the judgment of this world}} Now the ruler of this world will be thrown out.} \v{32}\red{As for me, if I am lifted up from the earth, I will draw all people to myself.''} \v{33}He said this to indicate the kind of death he was about to die.

\v{34}Then the crowd answered him, ``We have learned\fnote{\fbackref{12:34} Lit. \fbib{heard}} from the Law that the Messiah\fnote{\fbackref{12:34} Or \fbib{Christ}} remains forever. So how can you say that the Son of Man must be lifted up? Who is this Son of Man?''

\v{35}Jesus replied to the crowd,\fnote{\fbackref{12:35} Lit. \fbib{them}} \red{``The light is among you only for a short time. Walk while you have the light, so that the darkness may not overtake you. The person who walks in the darkness is in the darkness and does not know where he is going.} \v{36}\red{As long as you have the light, believe in the light, so that you may become children of light.''} After Jesus had said this, he went away and hid from them.
\passage{The Unbelief of the Jews}

\v{37}Although he had performed numerous signs in their presence, they did not believe in him, \v{38}so that what the prophet Isaiah spoke might be fulfilled when he said:

\begin{poetry}
\poeml ``Lord,\fnote{\fbackref{12:38} MT source citation reads \fbib{\divine{Lord}}} who has believed our message, \\
\poemll    and to whom has the Lord's\fnote{\fbackref{12:38} MT source citation reads \fbib{\divine{Lord}'s}} power\fnote{\fbackref{12:38} Lit. \fbib{arm}} been revealed?''\fnote{\fbackref{12:38} Isa 53:1}
\end{poetry}

\v{39}This is why they could not believe: Isaiah also said,

\begin{poetry}
\poeml \v{40}``He has blinded their eyes \\
\poemll    and hardened their heart, \\
\poeml so that they might not perceive with their eyes, \\
\poemll    and understand with their mind\fnote{\fbackref{12:40} Lit. \fbib{heart}} and turn, \\
\poemlll       and I would heal them.''\fnote{\fbackref{12:40} Cf. Isa 6:9-10}
\end{poetry}

\v{41}Isaiah said this when\fnote{\fbackref{12:41} Other mss. read \fbib{because}} he saw his glory and spoke about him. \v{42}Yet many people, even some of the authorities, believed in him, but because of the Pharisees they did not admit it so they would not be thrown out of the synagogue. \v{43}For they loved the praise of human beings more than the praise of God.
\passage{Judgment by Jesus' Word}

\v{44}Then Jesus said loudly, \red{``The one who believes in me does not believe in me only, but also\fnote{\fbackref{12:44} Lit. \fbib{me but}} in the one who sent me.} \v{45}\red{The one who sees me sees the one who sent me.} \v{46}\red{I've come into the world as light, so that everyone who believes in me won't remain in the darkness.} \v{47}\red{If anyone hears my words and doesn't keep them, I don't condemn him, because I didn't come to condemn the world, but to save it.}\fnote{\fbackref{12:47} Lit. \fbib{save the world}} \v{48}\red{The one who rejects me and doesn't receive my words has something to judge him: The word that I've spoken will judge him on the last day,} \v{49}\red{because I haven't spoken on my own authority. Instead, the Father who sent me has himself commanded me what to say and how to speak.} \v{50}\red{And I know that what he commands brings eternal life. What I speak, therefore, I speak just as the Father has told me.''}
\labelchapt{13}
\passage{Jesus Washes the Disciples' Feet}

\chapt{13}
\v{1}Now before the Passover Festival, Jesus realized that his hour had come to leave this world and return to the Father. Having loved his own who were in the world, he loved them to the end.\fnote{\fbackref{13:1} Or \fbib{loved them completely}} \v{2}By supper time, the devil had already put it into the heart of Judas, the son of Simon Iscariot, to betray him. \v{3}Because Jesus knew that the Father had given everything into his control,\fnote{\fbackref{13:3} Lit. \fbib{hands}} that he had come from God, and that he was returning to God, \v{4}therefore he got up from the table, removed his outer robe, and took a towel and fastened it around his waist. \v{5}Then he poured some water into a basin and began to wash the disciples' feet and to dry them with the towel that was tied around his waist.

\v{6}Then he came to Simon Peter, who asked him, ``Lord, are you going to wash my feet?''

\v{7}Jesus answered him, \red{``You don't realize now what I'm doing, but later on you'll understand.''}

\v{8}Peter told him, ``You must never wash my feet!''

Jesus answered him, \red{``Unless I wash you, you cannot be involved with me.''}

\v{9}Simon Peter told him, ``Lord, not just my feet, but my hands and my head as well!''

\v{10}Jesus told him, \red{``Whoever has bathed is entirely clean. He doesn't need to wash himself further, except for his feet. And you men\fnote{\fbackref{13:10} The Gk. lacks \fbib{men}} are clean, though not all of you.''} \v{11}For he knew who was going to betray him. That's why he said, \red{``Not all of you are clean.''}

\v{12}When Jesus\fnote{\fbackref{13:12} Lit. \fbib{he}} had washed their feet and put on his outer robe, he sat down again and told them, \red{``Do you realize what I've done to you?} \v{13}\red{You call me Teacher and Lord, and you are right\fnote{\fbackref{13:13} Lit. \fbib{you speak well}} because that is what I am.} \v{14}\red{So if I, your Lord and Teacher, have washed your feet, you must also wash one another's feet.} \v{15}\red{I've set an example for you, so that you may do as I have done to you.} \v{16}\red{Truly, I tell all of you\fnote{\fbackref{13:16} The Gk. pronoun \fbib{you} is pl.} emphatically, a servant isn't greater than his master, and a messenger isn't greater than the one who sent him.} \v{17}\red{If you understand these things, how blessed you are if you put them into practice!} \v{18}\red{I'm not talking about all of you. I know the ones I have chosen. But the Scripture must be fulfilled: `The one who ate bread with me\fnote{\fbackref{13:18} Other mss. read \fbib{ate my bread}} has turned against me.'}\fnote{\fbackref{13:18} Lit. \fbib{has lifted up his heel against me}; cf. Ps 41:9} \v{19}\red{I'm telling you this now, before it happens, so that when it does happen, you may believe that I AM.} \v{20}\red{Truly, I tell all of you\fnote{\fbackref{13:20} The Gk. pronoun \fbib{you} is pl.} emphatically, the one who receives whomever I send receives me, and the one who receives me receives the one who sent me.''}
\passage{Jesus Predicts His Betrayal}
\passageinfo{(Matthew 26:20-25; Mark 14:17-21; Luke 22:21-23)}

\v{21}After saying this, Jesus was deeply troubled in spirit and declared solemnly, \red{``Truly, I tell all of you\fnote{\fbackref{13:21} The Gk. pronoun \fbib{you} is pl.} emphatically, one of you is going to betray me!''} \v{22}The disciples began looking at one another, completely mystified about whom he was speaking. \v{23}One of his disciples, the one whom Jesus kept loving, had been sitting very close to him.

\v{24}So Simon Peter motioned to this man to ask Jesus about whom he was speaking. \v{25}Leaning forward on Jesus' chest, he asked him, ``Lord, who is it?''

\v{26}Jesus answered, \red{``He is the one to whom I will give this piece of bread after I have dipped it in the dish.''}\fnote{\fbackref{13:26} The Gk. lacks \fbib{in the dish}}

Then he took a piece of bread, dipped it, and gave it to Judas, the son of Simon Iscariot.\fnote{\fbackref{13:26} Other mss. read \fbib{Judas Iscariot, the son of Simon}} \v{27}After he had taken the piece of bread, Satan entered him. Then Jesus told him, \red{``Do quickly what you are going to do!''} \v{28}Now no one at the table knew why Jesus\fnote{\fbackref{13:28} Lit. \fbib{he}} said this to him. \v{29}Some thought that, since Judas had the moneybag, Jesus was telling him to buy what they needed for the festival or to give something to the destitute. \v{30}So Judas\fnote{\fbackref{13:30} Lit. \fbib{he}} took the piece of bread, immediately went outside{\ldots}and it was night.
\passage{The New Commandment}

\v{31}After Judas\fnote{\fbackref{13:31} Lit. \fbib{he}} had gone out, Jesus said, \red{``The Son of Man is now glorified, and God has been glorified by him.} \v{32}\red{If God has been glorified by him,\fnote{\fbackref{13:32} Other mss. lack \fbib{If God has been glorified by him}} God himself also will glorify the Son of Man,\fnote{\fbackref{13:32} Lit. \fbib{him}} and he will do so\fnote{\fbackref{13:32} Lit. \fbib{will glorify him}} quickly.} \v{33}\red{Little children, I'm with you only a little longer. You will look for me, but what I told the Jewish leaders\fnote{\fbackref{13:33} I.e. Judean leaders} I now tell you, `Where I'm going, you cannot come.'} \v{34}\red{I'm giving you a new commandment{\ldots}to love one another. Just as I have loved you, you also should love one another.} \v{35}\red{This is how everyone will know that you are my disciples, if you have love for one another.''}
\passage{Jesus Predicts Peter's Denial}
\passageinfo{(Matthew 26:31-34; Mark 14:27-31; Luke 22:31-34)}

\v{36}Simon Peter asked him, ``Lord, where are you going?''

Jesus answered him, \red{``I'm going where you cannot follow me now, though you will follow me later on.''}

\v{37}``Lord,\fnote{\fbackref{13:37} Other mss. lack \fbib{Lord}} why can't I follow you now?'' Peter asked him. ``I would lay down my life for you!''

\v{38}Jesus answered him, \red{``Would you lay down your life for me? I tell you\fnote{\fbackref{13:38} The Gk. pronoun \fbib{you} is sing.} emphatically, a rooster will not crow until you have denied me three times.''}
\labelchapt{14}
\passage{Jesus the Way to the Father}

\chapt{14}
\v{1}\red{``Don't let your hearts be troubled. Believe\fnote{\fbackref{14:1} Or \fbib{You believe}} in God, believe also in me.} \v{2}\red{There are many rooms in my Father's house. If there weren't, I wouldn't have told you that I am going away to prepare a place for you, would I?} \v{3}\red{And since\fnote{\fbackref{14:3} Or, \fbib{And if, as is the case,}} I'm going away to prepare a place for you, I'll come back again and welcome you into my presence, so that you may be where I am.} \v{4}\red{You know where I am going, and you know the way.''}

\v{5}Thomas asked him, ``Lord, we don't know where you are going, so how can we know the way?''

\v{6}Jesus told him, \red{``I am the way, the truth, and the life. No one comes to the Father except through me.} \v{7}\red{If you have known me, you will also know my Father. From now on you know him and have seen him.''}

\v{8}Philip told him, ``Lord, show us the Father, and that will satisfy us.''

\v{9}\red{``Have I been with you all this time, Philip, and you still do not know me?''} Jesus asked him. \red{``The person who has seen me has seen the Father. So how can you say, `Show us the Father'?} \v{10}\red{You believe, don't you, that I am in the Father and the Father is in me? The words that I say to you I don't speak on my own. It is the Father who dwells in me and who carries out his work.} \v{11}\red{Believe me, I am in the Father and the Father is in me. Otherwise, believe me\fnote{\fbackref{14:11} Other mss. lack \fbib{me}} because of what I've been doing.}\fnote{\fbackref{14:11} Lit. \fbib{of the works themselves}} \v{12}\red{Truly, I tell all of you\fnote{\fbackref{14:12} The Gk. pronoun \fbib{you} is pl.} emphatically, the one who believes in me will also do what I'm doing. He will do even greater things than these, because I'm going to the Father.} \v{13}\red{I'll do whatever you ask in my name, so that the Father may be glorified in the Son.} \v{14}\red{If you ask me\fnote{\fbackref{14:14} Other mss. lack \fbib{me}} for anything in my name, I will do it.''}
\passage{The Promise of the Helper}

\v{15}\red{``If you love me, keep\fnote{\fbackref{14:15} Other mss. read \fbib{you will keep}} my commandments.} \v{16}\red{I will ask the Father to give\fnote{\fbackref{14:16} Lit. \fbib{and he will give}} you another Helper, to be with you always.} \v{17}\red{He is the Spirit of truth, whom the world cannot receive, because it neither sees him nor recognizes him. But you recognize him, because he lives with you and will be in\fnote{\fbackref{14:17} Or \fbib{among}} you.} \v{18}\red{I'm not going to forsake you like orphans. I will come back to you.}

\v{19}\red{``In a little while the world will no longer see me, but you will see me. Because I live, you will live also.} \v{20}\red{At that time, you'll know that I am in my Father, that you are in me, and that I am in you.} \v{21}\red{The person who has my commandments and keeps them is the one who loves me. The one who loves me will be loved by my Father, and I, too, will love him and reveal myself to him.''}

\v{22}Judas (not Iscariot) asked him, ``Lord, how is it that you are going to reveal yourself to us and not to the world?''

\v{23}Jesus answered him, \red{``If anyone loves me, he will keep my word. Then my Father will love him, and we will go to him and make our home within him.} \v{24}\red{The one who doesn't love me doesn't keep my words. The words that you're hearing me say are not mine, but come from the Father who sent me.}

\v{25}\red{``I have told you this while I am still with you.} \v{26}\red{But the Helper, the Holy Spirit, whom the Father will send in my name, will teach you all things and remind you of everything that I have told you.} \v{27}\red{I'm leaving you at peace. I'm giving you my own peace. I'm not giving it to you as the world gives. So don't let your hearts be troubled, and don't be afraid.} \v{28}\red{You have heard me tell you, `I'm going away, but I'm coming back to you.' If you loved me, you would rejoice that I'm going to the Father, because the Father is greater than I am.} \v{29}\red{I've told you this now, before I leave, so that when I do leave, you will believe.} \v{30}\red{I won't talk with you much longer, because the ruler of this world is coming. He has no power over me.}\fnote{\fbackref{14:30} Lit. \fbib{has nothing in me}} \v{31}\red{But I'm doing what the Father has commanded me, to let the world know that I love the Father. Get up! Let us leave this place.''}
\labelchapt{15}
\passage{Jesus the True Vine}

\chapt{15}
\v{1}\red{``I am the true vine, and my Father is the vintner.} \v{2}\red{He cuts off every branch that does not produce fruit in me, and he cuts back every branch that does produce fruit, so that it might produce more fruit.} \v{3}\red{You are already clean because of what I've spoken to you.}

\v{4}\red{``Abide in me, and I will abide in you. Just as the branch cannot produce fruit by itself unless it abides in the vine, neither can you unless you abide in me.} \v{5}\red{I am the vine, you are the branches. The one who abides in me while I abide in him\fnote{\fbackref{15:5} Lit. \fbib{and I in him}} produces much fruit, because apart from me you can do nothing.} \v{6}\red{Unless a person abides in me, he is thrown away like a pruned\fnote{\fbackref{15:6} The Gk. lacks \fbib{pruned}} branch and dries up. People gather such branches,\fnote{\fbackref{15:6} Lit. \fbib{They gather them}} throw them into a fire, and they are burned up.}

\v{7}\red{``If you abide in me and my words abide in you, you can ask for anything you want, and you'll receive it.} \v{8}\red{This is how my Father is glorified, when you produce a lot of fruit and so prove to be my disciples.} \v{9}\red{Just as the Father has loved me, so I have loved you. So abide in my love.} \v{10}\red{If you keep my commandments, you'll abide in my love, just as I have kept my Father's commandments and abide in his love.} \v{11}\red{I've told you this, so that my joy may be in you, and that your joy may be complete.}

\v{12}\red{``This is my commandment: that you love one another as I have loved you.} \v{13}\red{No one shows\fnote{\fbackref{15:13} Lit. \fbib{has}} greater love than when he lays down his life for his friends.} \v{14}\red{You are my friends, if you do what I command you.} \v{15}\red{I don't call you servants anymore, because a servant doesn't know what his master is doing. But I've called you friends, because I've made known to you everything that I've heard from my Father.}

\v{16}\red{``You have not chosen me, but I have chosen you. I have appointed you to go and produce fruit that will last,\fnote{\fbackref{15:16} Lit. \fbib{produce fruit, and your fruit is to last}} so that whatever you ask the Father in my name, he will give it to you.} \v{17}\red{I am giving you these commandments so that you may love one another.''}
\passage{The World's Hatred}

\v{18}\red{``If the world hates you, you should realize that it hated me before you.} \v{19}\red{If you belonged to the world, the world would love you as one of its own. But because you do not belong to the world and I have chosen you out of it,\fnote{\fbackref{15:19} Lit. \fbib{out of the world}} the world hates you.} \v{20}\red{Remember what I told you: `A servant is not greater than his master.' If they persecuted me, they will also persecute you. If they kept my word, they will also keep yours.} \v{21}\red{They will do all these things to you on account of my name, because they do not know the one who sent me.}

\v{22}\red{``If I had not come and spoken to them, they would not have any sin. But now they have no excuse for their sin.} \v{23}\red{The person who hates me also hates my Father.} \v{24}\red{If I hadn't done among them the actions that no one else did, they wouldn't have any sin. But now they have seen and hated both me and my Father.} \v{25}\red{But this happened so that\fnote{\fbackref{15:25} Lit. \fbib{But so that}} what has been written in their Law might be fulfilled: `They hated me for no reason.'}\fnote{\fbackref{15:25} Cf. Ps 35:19; 69:4}

\v{26}\red{``When the Helper comes, whom I will send to you from the Father--- the Spirit of Truth, who comes from the Father---he will testify on my behalf.} \v{27}\red{You will testify also, because you have been with me from the beginning.}
\labelchapt{16}

\chapt{16}
\v{1}\red{``I have told you this to keep you from falling away.}\fnote{\fbackref{16:1} Or \fbib{from stumbling}} \v{2}\red{You'll be thrown out of the synagogues. Yes, a time is coming when the one who kills you will think he's serving God.} \v{3}\red{They'll do this because they haven't known the Father or me.} \v{4}\red{But I've told you this, so that when the time comes you'll remember that I told you about them. I didn't tell you this in the beginning, because I was still with you.''}
\passage{The Work of the Spirit}

\v{5}\red{``But now I am going to the one who sent me. Yet none of you asks me, `Where are you going?'} \v{6}\red{But because I have told you this, sorrow has filled your hearts.} \v{7}\red{However, I'm telling you the truth. It's for your advantage that I'm going away, because if I don't go away, the Helper won't come to you. But if I go, I will send him to you.} \v{8}\red{When he comes, he will convict the world of sin, righteousness, and judgment---} \v{9}\red{of sin, because they don't believe in me;} \v{10}\red{of righteousness, because I'm going to the Father and you will no longer see me;} \v{11}\red{and of judgment, because the ruler of this world has been judged.}\fnote{\fbackref{16:11} Or \fbib{condemned}}

\v{12}\red{``I still have a lot to say to you, but you cannot bear it now.} \v{13}\red{Yet when the Spirit of Truth comes, he'll guide you into all truth. He won't speak on his own accord, but he'll speak whatever he hears and will declare to you the things that are to come.} \v{14}\red{He will glorify me, because he will take what is mine and declare it to you.} \v{15}\red{All that the Father has is mine. That is why I said, `He will take what is mine and declare it to you.'} \v{16}\red{In a little while you will no longer see me, then in a little while you will see me again.''}

\v{17}At this point, some of his disciples asked each other, ``What does he mean by telling us, \red{`In a little while you will no longer see me, then in a little while you will see me again,'} and, \red{`because I am going to the Father'}?'' \v{18}They kept saying, ``What is this \red{`in a little while'} that he keeps talking about? We don't know what he means.''
\passage{Sorrow will Turn to Joy}

\v{19}Jesus knew that they wanted to ask him a question, so he asked them, \red{``Are you discussing among yourselves what I meant when I said, `In a little while you will no longer see me, then in a little while you will see me again'?} \v{20}\red{Truly, I tell all of you\fnote{\fbackref{16:20} The Gk. pronoun \fbib{you} is pl.} emphatically, you'll cry and mourn, but the world will rejoice. You'll be deeply distressed, but your pain will turn into joy.} \v{21}\red{When a woman is in labor she has pain, because her time has come. Yet when she has given birth to her child, she doesn't remember the agony anymore because of the joy of having brought a human being into the world.} \v{22}\red{Now you are having pain. But I'll see you again, and your hearts will rejoice, and no one will take your joy away from you.} \v{23}\red{On that day, you won't ask me for anything. Truly, I tell all of you\fnote{\fbackref{16:23} The Gk. pronoun \fbib{you} is pl.} emphatically, whatever you ask the Father for in my name, he will give it to you.}\fnote{\fbackref{16:23} Other mss. read \fbib{ask the Father for, he will give it to you in my name}} \v{24}\red{So far you haven't asked for anything in my name. Keep asking and you will receive, so that your joy may be complete.''}
\passage{Victory over the World}

\v{25}\red{``I have said these things to you in figurative language. The time is coming when I will no longer speak to you in figurative language, but will tell you plainly about the Father.} \v{26}\red{At that time, you will make your requests in my name, so that I will have no need to ask the Father on your behalf,} \v{27}\red{because the Father himself loves you, and because you have loved me and believed that I came from God.}\fnote{\fbackref{16:27} Other mss. read \fbib{from the Father}} \v{28}\red{I left the Father and came into the world. Now\fnote{\fbackref{16:28} Lit. \fbib{Again}} I'm leaving the world and going back to the Father.''}

\v{29}Jesus'\fnote{\fbackref{16:29} Lit. \fbib{His}} disciples said, ``Well, now you're speaking plainly and not using figurative language. \v{30}Now we know that you know everything and don't need to have anyone ask you any questions. Because of this, we believe that you have come from God.''

\v{31}Jesus answered them, \red{``Do you now believe?} \v{32}\red{Listen, the time is coming, indeed it has already come, when you will be scattered, each of you to his own home, and you will leave me all by myself. Yet I'm not alone, because the Father is with me.} \v{33}\red{I have told you this so that through me you may have peace. In the world you'll have trouble, but be courageous---I've overcome the world!''}
\labelchapt{17}
\passage{Jesus Prays for Himself, His Disciples, and His Future Followers}

\chapt{17}
\v{1}After Jesus had said this, he looked up to heaven and said, \red{``Father, the hour has come. Glorify your Son, so that the Son may glorify you.} \v{2}\red{For\fnote{\fbackref{17:2} Lit. \fbib{Just as}} you have given him authority over all humanity\fnote{\fbackref{17:2} Lit. \fbib{flesh}} so that he might give eternal life to all those you gave him.} \v{3}\red{And this is eternal life: to know you, the only true God, and the one whom you sent---Jesus the Messiah.}\fnote{\fbackref{17:3} Or \fbib{Christ}} \v{4}\red{I glorified you on earth by completing the task you gave me to do.}

\v{5}\red{``So now, Father, glorify me in your presence with the glory I had with you before the world existed.} \v{6}\red{I have made your name known to these men whom you gave me from the world. They were yours, and you gave them to me, and they have kept your word.} \v{7}\red{Now they realize that everything you gave me comes from you,} \v{8}\red{because the words that you gave me I passed on to them. They have received them and know for sure that I came from you. They have believed that you sent me.}

\v{9}\red{``I am asking on their behalf. I am not asking on behalf of the world, but on behalf of those you gave me, because they are yours.} \v{10}\red{All that is mine is yours, and what is yours is mine, and I have been glorified through them.} \v{11}\red{I am no longer in the world, but they are in the world, and I am coming to you. Holy Father, protect them by your Name, the Name\fnote{\fbackref{17:11} Lit. \fbib{the one}} that you gave me, so that they may be one, as we are one.} \v{12}\red{While I was with them, I protected them by the authority\fnote{\fbackref{17:12} Lit. \fbib{by your name}} that you gave me. I guarded them, and not one of them became lost except the one who was destined for\fnote{\fbackref{17:12} Lit. \fbib{the son of}} destruction, so that the Scripture might be fulfilled.}

\v{13}\red{``And now I am coming to you, and I say these things in the world so that they may have my joy made complete in themselves.} \v{14}\red{I have given them your word, and the world has hated them, because they do not belong to the world, just as I don't belong to the world.} \v{15}\red{I'm not asking you to take them out of the world but to protect them from the evil one.} \v{16}\red{They don't belong to the world, just as I don't belong to the world.}

\v{17}\red{``Sanctify them by the truth. Your word is truth.} \v{18}\red{Just as you sent me into the world, so I have sent them into the world.} \v{19}\red{It is for their sakes that I sanctify myself, so that they, too, may be sanctified by the truth.} \v{20}\red{I ask not only on behalf of these men,\fnote{\fbackref{17:20} The Gk. lacks \fbib{men}} but also on behalf of those who will believe in me through their message,} \v{21}\red{so that they may all be one. Just as you, Father, are in me and I am in you, may they also be one\fnote{\fbackref{17:21} Other mss. lack \fbib{one}} in us, so that the world may believe that you sent me.}

\v{22}\red{``I have given them the glory that you gave me, so that they may be one, just as we are one.} \v{23}\red{I am in them, and you are in me. May they be completely one, so that the world may know that you sent me and that you have loved them as you loved me.} \v{24}\red{Father, I want those you have given me to be with me where I am and to see my glory, which you gave me because you loved me before the creation of the world.}

\v{25}\red{``Righteous Father, the world has never known you. Yet I have known you, and these men have known that you sent me.} \v{26}\red{I made your name known to them, and will continue to make it known, so that the love you have for me\fnote{\fbackref{17:26} Lit. \fbib{the love with which you loved me}} may be in them and I myself may be in them.''}
\labelchapt{18}
\passage{Jesus is Betrayed and Arrested}
\passageinfo{(Matthew 26:47-56; Mark 14:43-40; Luke 22:47-53)}

\chapt{18}
\v{1}After Jesus had said all of this, he went with his disciples across the Kidron valley to a place where there was a garden, which he and his disciples entered. \v{2}Now Judas, who betrayed him, also knew the place, because Jesus often met there with his disciples. \v{3}So Judas took a detachment of soldiers and some officers from the high priests and the Pharisees and went there with lanterns, torches, and weapons.

\v{4}Then Jesus, knowing everything that was going to happen, went forward and asked them, \red{``Who are you looking for?''}

\v{5}They answered him, ``Jesus from Nazareth.''\fnote{\fbackref{18:5} Or \fbib{Jesus the Nazarene}; the Gk. \fbib{Nazoraios} may be a word play between Heb. \fbib{netser,} meaning \fbib{branch} (cf. Isa 11:1), and the name \fbib{Nazareth.}}

Jesus told them, \red{``I AM.''} Judas, the man who betrayed him, was standing with them.

\v{6}When Jesus\fnote{\fbackref{18:6} Lit. \fbib{he}} told them, \red{``I AM,''} they backed away and fell to the ground.

\v{7}So he asked them again, \red{``Who are you looking for?''}

They said, ``Jesus from Nazareth.''\fnote{\fbackref{18:7} Or \fbib{Jesus the Nazarene}; the Gk. \fbib{Nazoraios} may be a word play between Heb. \fbib{netser,} meaning \fbib{branch} (cf. Isa 11:1), and the name \fbib{Nazareth.}}

\v{8}Jesus replied, \red{``I told you that I am the one,\fnote{\fbackref{18:8} The Gk. lacks \fbib{the one}} so if you are looking for me, let these men go.''} \v{9}This was to fulfill what he had said, \red{``I did not lose a single one of those you gave me.''}\fnote{\fbackref{18:9} Cf. John 17:12}

\v{10}Then Simon Peter, who had a sword, drew it and struck the high priest's servant, cutting off his right ear. The servant's name was Malchus. \v{11}Jesus told Peter, \red{``Put your sword back into its sheath. Shouldn't I drink the cup that the Father has given me?''}
\passage{Jesus before the High Priest}
\passageinfo{(Matthew 26:57-58; Mark 14:53-54; Luke 22:54)}

\v{12}Then the soldiers, along with their commander and the Jewish officers, arrested Jesus and tied him up. \v{13}First they brought him to Annas, because he was the father-in-law of Caiaphas, the high priest that year. \v{14}Caiaphas was the person who had advised the Jews that it was better to have one man die for the people.
\passage{Peter Denies Jesus}
\passageinfo{(Matthew 26:69-70; Mark 14:66-68; Luke 22:55-57)}

\v{15}Simon Peter and another disciple were following Jesus. Since the other disciple was known to the high priest, he accompanied Jesus into the courtyard of the high priest. \v{16}Peter, however, stood outside the gate. So this other disciple who was known to the high priest went out and spoke to the gatekeeper and brought Peter inside. \v{17}The young woman at the gate asked Peter, ``You aren't one of this man's disciples, too, are you?''

``I am not,'' he replied.

\v{18}Meanwhile, the servants and officers were standing around a charcoal fire they had built and were warming themselves because it was cold. Peter was also standing with them, keeping himself warm.
\passage{The High Priest Questions Jesus}
\passageinfo{(Matthew 26:59-66; Mark 14:55-64; Luke 22:66-71)}

\v{19}Then the high priest questioned Jesus about his disciples and about his own teaching. \v{20}Jesus answered him, \red{``I have spoken publicly to the world. I have always taught in the synagogue or in the Temple, where all Jews meet together, and I have said nothing in secret.} \v{21}\red{Why do you question me? Question those who heard what I said. These are the people who know what I said.''}

\v{22}When he said this, one of the officers standing nearby slapped Jesus on the face and demanded, ``Is that any way to answer the high priest?''

\v{23}Jesus answered him, \red{``If I have said anything wrong, tell me what it was.\fnote{\fbackref{18:23} Lit. \fbib{about the wrong}} But if I have told the truth, why do you hit me?''} \v{24}Then Annas sent him, with his hands tied, to Caiaphas the high priest.
\passage{Peter Denies Jesus Again}
\passageinfo{(Matthew 26:71-75; Mark 14:69-72; Luke 22:58-62)}

\v{25}Meanwhile, Simon Peter was standing and warming himself. Some people\fnote{\fbackref{18:25} Lit. \fbib{They}} asked him, ``You aren't one of his disciples, too, are you?''

He denied it by saying, ``I am not!''

\v{26}Then one of the high priest's servants, a relative of the man whose ear Peter had cut off, said, ``I saw you in the garden with Jesus,\fnote{\fbackref{18:26} Lit. \fbib{him}} didn't I?'' \v{27}Peter again denied it, and immediately a rooster crowed.
\passage{Pilate Questions Jesus}
\passageinfo{(Matthew 27:1-2; Mark 15:1-5; Luke 23:1-5)}

\v{28}Then Jesus was led from Caiaphas to the governor's headquarters\fnote{\fbackref{18:28} Lit. \fbib{to the praetorium}} early in the morning. The Jews\fnote{\fbackref{18:28} Lit. \fbib{they}} did not go into the headquarters, to avoid becoming unclean\fnote{\fbackref{18:28} I.e. ceremonially unqualified} and unable to eat the Passover meal. \v{29}So Pilate came out to them and asked, ``What accusation are you bringing against this man?''

\v{30}They answered him, ``If he weren't a criminal, we wouldn't have handed him over to you.''

\v{31}Pilate told them, ``You take him and try him according to your Law.''

The Jewish leaders\fnote{\fbackref{18:31} I.e. Judean leaders; lit. \fbib{The Jews}} told him, ``It is not legal for us to put anyone to death.'' \v{32}This was to fulfill what Jesus had said\fnote{\fbackref{18:32} Lit. \fbib{the word of Jesus that he said}} when he indicated the kind of death he was to die.

\v{33}So Pilate went back into the governor's headquarters,\fnote{\fbackref{18:33} Lit. \fbib{into the praetorium}} summoned Jesus, and asked him, ``Are you the king of the Jews?''

\v{34}Jesus replied, \red{``Are you asking this on your own initiative, or did others tell you about me?''}

\v{35}Pilate replied, ``I am not a Jew, am I? It is your own nation and high priests who have handed you over to me. What have you done?''

\v{36}Jesus answered, \red{``My kingdom does not belong to this world. If my kingdom belonged to this world, my servants would fight to keep me from being handed over to the Jewish leaders.\fnote{\fbackref{18:36} I.e. Judean leaders; lit. \fbib{the Jews}} But for now my kingdom is not from here.''}

\v{37}Pilate asked him, ``So you are a king?''

Jesus answered, \red{``You say that I am a king. I was born for this, and I came into the world for this: to testify to the truth. Everyone who is committed to the truth listens to my voice.''}

\v{38}Pilate asked him, ``What is `truth'?'' and then he went out to the Jewish leaders\fnote{\fbackref{18:38} I.e. Judean leaders; lit. \fbib{the Jews}} again and told them, ``I find no basis for a charge against him. \v{39}But you have a custom that I release one person for you at Passover. Do you want me to release for you the king of the Jews?''

\v{40}At this, they shouted out again, ``Not this fellow, but Barabbas!'' Now Barabbas was a revolutionary.\fnote{\fbackref{18:40} Or \fbib{bandit}}
\labelchapt{19}
\passage{Jesus is Sentenced to Death}
\passageinfo{(Matthew 27:15-31; Mark 15:6-20; Luke 23:13-25)}

\chapt{19}
\v{1}Then Pilate had Jesus taken away and whipped. \v{2}The soldiers twisted some thorns into a victor's crown, put it on his head, and threw a purple robe on him. \v{3}They kept coming up to him and saying, ``Long live the king of the Jews!'' Then they began to slap him on the face.

\v{4}Pilate went outside again and told the Jews,\fnote{\fbackref{19:4} Lit. \fbib{them}} ``Look, I am bringing him out to you to let you know that I find no basis for a charge against him.'' \v{5}Then Jesus came outside, wearing the victor's crown of thorns and the purple robe.

Pilate told them, ``Here is the man!''

\v{6}When the high priests and the officials saw him, they shouted, ``Crucify him! Crucify him!''

Pilate told them, ``You take him and crucify him. I find no basis for a charge against him.''

\v{7}The Jewish leaders\fnote{\fbackref{19:7} I.e. Judean leaders; lit. \fbib{The Jews}} answered Pilate,\fnote{\fbackref{19:7} Lit. \fbib{him}} ``We have a law, and according to that Law he must die because he made himself out to be the Son of God.''

\v{8}When Pilate heard this, he became even more afraid. \v{9}Returning to his headquarters,\fnote{\fbackref{19:9} Lit. \fbib{to the praetorium}} he asked Jesus, ``Where are you from?''

But Jesus did not answer him.

\v{10}So Pilate asked him, ``Aren't you going to speak to me? You realize, don't you, that I have the authority to release you and the authority to crucify you?''

\v{11}Jesus answered him, \red{``You have no authority over me at all, except what was given to you from above. That's why the one who handed me over to you is guilty of a greater sin.''}

\v{12}From then on, Pilate tried to release him, but the Jewish leaders\fnote{\fbackref{19:12} I.e. Judean leaders; lit. \fbib{the Jews}} kept shouting, ``If you release this fellow, you're not a friend of Caesar! Anyone who claims to be a king is defying Caesar!''

\v{13}When Pilate heard these words, he brought Jesus outside and sat down on the judgment seat in a place called The Pavement, which in Hebrew is called Gabbatha. \v{14}Now it was the Preparation Day for the Passover, about noon.\fnote{\fbackref{19:14} Lit. \fbib{about the sixth hour}} He told the Jewish leaders,\fnote{\fbackref{19:14} I.e. Judean leaders; lit. \fbib{the Jews}} ``Here is your king!''

\v{15}Then they shouted, ``Take him away! Take him away! Crucify him!''

Pilate asked them, ``Should I crucify your king?''

The high priests responded, ``We have no king but Caesar!'' \v{16}Then Pilate\fnote{\fbackref{19:16} Lit. \fbib{he}} handed him over to be crucified, and they took Jesus away.
\passage{Jesus is Crucified}
\passageinfo{(Matthew 27:32-44; Mark 15:21-32; Luke 23:26-43)}

\v{17}Carrying the cross all by himself, he went out to what is called The Place of a Skull, which in Hebrew is called Golgotha. \v{18}There they crucified him, along with two others, one on each side of him with Jesus in the middle. \v{19}Pilate wrote an inscription and put it on the cross. It read, ``Jesus from Nazareth,\fnote{\fbackref{19:19} Or \fbib{Jesus the Nazarene}; the Gk. \fbib{Nazoraios} may be a word play between Heb. \fbib{netser,} meaning \fbib{branch} (see Isa 11:1), and the name \fbib{Nazareth.}} the King of the Jews.'' \v{20}Many Jews read this inscription, because the place where Jesus was crucified was near the city. It was written in Hebrew, Latin, and Greek.

\v{21}Then the Jewish high priests told Pilate, ``Don't write, `The King of the Jews,' but that this fellow said, \red{`I am the King of the Jews.'}''

\v{22}Pilate replied, ``What I have written I have written.''

\v{23}When the soldiers had crucified Jesus, they took his clothes and divided them into four parts, one for each soldier, and took his cloak\fnote{\fbackref{19:23} Lit. \fbib{and his tunic}} as well. The cloak was seamless, woven in one piece from the top down. \v{24}So they told each other, ``Let's not tear it. Instead, let's throw dice to see who gets it.'' This was to fulfill the Scripture that says,

\begin{poetry}
\poeml ``They divided my clothes among themselves, \\
\poemll    and for my clothing they threw dice.''\fnote{\fbackref{19:24} Cf. Ps 22:18}
\end{poetry}

So that is what the soldiers did.

\v{25}Meanwhile, standing near Jesus' cross were his mother, his mother's sister, Mary the wife of Clopas, and Mary Magdalene.\fnote{\fbackref{19:25} Or \fbib{Mary of Magdala}} \v{26}When Jesus saw his mother and the disciple whom he kept loving standing there, he told his mother, \red{``Dear lady,\fnote{\fbackref{19:26} Or \fbib{Woman}} here is your son.''} \v{27}Then he told the disciple, \red{``Here is your mother.''} And from that hour the disciple took her into his own home.
\passage{Jesus Dies on the Cross}
\passageinfo{(Matthew 27:45-56; Mark 15:33-41; Luke 23:44-49)}

\v{28}After this, when Jesus realized that everything was now completed, he said (in order to fulfill the Scripture), \red{``I'm thirsty.''} \v{29}A jar of sour wine was standing there, so they put a sponge full of the wine on a branch of hyssop and held it to his mouth. \v{30}After Jesus had taken the wine, he said, \red{``It is finished.''} Then he bowed his head and released his spirit.
\passage{Jesus' Side is Pierced}

\v{31}Since it was the Preparation Day, the Jewish leaders\fnote{\fbackref{19:31} I.e. Judean leaders; lit. \fbib{the Jews}} did not want to leave the bodies on the crosses during the Sabbath, because that was a particularly important Sabbath. So they asked Pilate to have the men's legs broken and the bodies\fnote{\fbackref{19:31} The Gk. lacks \fbib{the bodies}} removed. \v{32}So the soldiers went and broke the legs of the first man and then of the other man who had been crucified with him. \v{33}But when they came to Jesus and saw that he was already dead, they did not break his legs. \v{34}Instead, one of the soldiers pierced his side with a spear, and blood and water immediately came out. \v{35}The one who saw this has testified, and his testimony is true. He knows he is telling the truth so that you, too, may believe, \v{36}because these things happened so that the Scripture might be fulfilled:

\begin{poetry}
\poeml ``None of his bones will be broken.''\fnote{\fbackref{19:36} Cf. Exod 12:46; Num 9:12; Ps 34:20}
\end{poetry}

\v{37}In addition, another passage of Scripture says,

\begin{poetry}
\poeml ``They will look on the one whom they pierced.''\fnote{\fbackref{19:37} Cf. Zech 12:10}
\end{poetry}
\passage{Jesus is Buried}
\passageinfo{(Matthew 27:57-61; Mark 15:42-47; Luke 23:50-56)}

\v{38}Later on, Joseph of Arimathea, who was a disciple of Jesus (though a secret one because he was afraid of the Jewish leaders),\fnote{\fbackref{19:38} I.e. Judean leaders; lit. \fbib{the Jews}} asked Pilate to let him remove the body of Jesus. Pilate gave him permission, and he came and removed his body. \v{39}Nicodemus, the man who had first come to Jesus at night, also arrived, bringing a mixture of myrrh and aloes weighing about 100 litra.\fnote{\fbackref{19:39} I.e. about 75 pounds; the Gk. \fbib{litron} weighed about 12 ounces} \v{40}They took the body of Jesus and wrapped it in linen cloths along with spices, according to the burial custom of the Jews. \v{41}A garden was located in the place where he was crucified, and in that garden was a new tomb in which no one had yet been placed. \v{42}Because it was the Jewish Preparation Day, and because the tomb was nearby, they put Jesus there.
\labelchapt{20}
\passage{Jesus is Raised from the Dead}
\passageinfo{(Matthew 28:1-10; Mark 16:1-8; Luke 24:1-12)}

\chapt{20}
\v{1}On the first day of the week,\fnote{\fbackref{20:1} Lit. \fbib{first of the Sabbaths}} early in the morning and while it was still dark, Mary Magdalene\fnote{\fbackref{20:1} Or \fbib{Mary of Magdala}} went to the tomb and noticed that the stone had been removed from the tomb. \v{2}So she ran off and went to Simon Peter and the other disciple, whom Jesus kept loving. She told them, ``They have taken the Lord out of the tomb, and we don't know where they have put him!'' \v{3}So Peter and the other disciple took off for the tomb. \v{4}The two of them were running together, but the other disciple ran faster than Peter and came to the tomb first. \v{5}Bending over to look inside, he noticed the linen cloths lying there, but didn't go in. \v{6}At this point Simon Peter arrived, following him, and went straight into the tomb. He observed that the linen cloths were lying there, \v{7}and that the handkerchief that had been on Jesus' head was not lying with the linen cloths but was rolled up in a separate place. \v{8}Then the other disciple, who arrived at the tomb first, went inside, looked, and believed. \v{9}For they did not yet understand the Scripture that said\fnote{\fbackref{20:9} The Gk. lacks \fbib{that said}} that Jesus\fnote{\fbackref{20:9} Lit. \fbib{he}} had to rise from the dead. \v{10}So the disciples went back to their homes.
\passage{Jesus Appears to Mary Magdalene}
\passageinfo{(Mark 16:9-11)}

\v{11}Meanwhile, Mary\fnote{\fbackref{20:11} I.e. \fbib{Mary of Magdala}} stood crying outside the tomb. As she cried, she bent over and looked\fnote{\fbackref{20:11} The Gk. lacks \fbib{and looked}} into the tomb. \v{12}She saw two angels in white clothes who were sitting down, one at the head and the other at the foot of the place where Jesus' body had been lying. \v{13}They asked her, ``Lady,\fnote{\fbackref{20:13} Or \fbib{Woman}} why are you crying?''

She told them, ``Because they have taken away my Lord, and I don't know where they have put him.''

\v{14}After she had said this, she turned around and noticed Jesus standing there, without realizing that it was Jesus.

\v{15}Jesus asked her, \red{``Dear lady,\fnote{\fbackref{20:15} Or \fbib{Woman}} why are you crying? Who are you looking for?''}

Thinking he was the gardener, she told him, ``Sir, if you have carried him away, tell me where you have put him, and I will take him away.''

\v{16}Jesus told her, \red{``Mary!''}

She turned around and told him in Hebrew, ``Rabbouni!'' (which means ``Teacher'').

\v{17}Jesus told her, \red{``Don't hold on to me, because I haven't yet ascended to the Father. But go to my brothers and tell them, `I'm ascending to my Father and your Father, to my God and your God.'\,''}

\v{18}So Mary Magdalene\fnote{\fbackref{20:18} Or \fbib{Mary of Magdala}} went and announced to the disciples, ``I've seen the Lord!'' She also told them what he had told her.
\passage{Jesus Appears to the Disciples}
\passageinfo{(Matthew 28:16-20; Mark 16:14-18; Luke 24:36-49)}

\v{19}It was the evening of the first day of the week,\fnote{\fbackref{20:19} Or \fbib{evening of that day, the first of the Sabbaths}} and the doors of the house where the disciples had met were locked because they were afraid of the Jewish leaders.\fnote{\fbackref{20:19} I.e. Judean leaders} Jesus came and stood among them. He told them, \red{``Peace be with you.''} \v{20}After saying this, he showed them his hands and his side, and when they saw the Lord, the disciples were overjoyed. \v{21}Jesus told them again, \red{``Peace be with you. Just as the Father has sent me, so I am sending you.''} \v{22}When he had said this, he breathed on them and told them, \red{``Receive the Holy Spirit.} \v{23}\red{If you forgive people's sins, they are forgiven. If you retain people's sins, they are retained.''}
\passage{Jesus Appears to Thomas}

\v{24}Thomas, one of the Twelve (called the Twin),\fnote{\fbackref{20:24} Lit. \fbib{called Didymus}} wasn't with them when Jesus came. \v{25}So the other disciples kept telling him, ``We've seen the Lord!'' But he told them, ``Unless I see the nail marks in his hands, put my finger into them,\fnote{\fbackref{20:25} Lit. \fbib{into the nail marks}} and put my hand into his side, I'll never believe!''

\v{26}A week later, his disciples were again inside, and Thomas was with them. Even though the doors were shut, Jesus came, stood among them, and said, \red{``Peace be with you.''} \v{27}Then he told Thomas, \red{``Put your finger here, and look at my hands. Take your hand, and put it into my side. Stop doubting, but believe.''}

\v{28}Thomas answered him, ``My Lord and my God!''

\v{29}Jesus told him, \red{``Is it because you've seen me that you have believed? How blessed are those who have never seen me and yet have believed!''}
\passage{The Purpose of the Book}

\v{30}Jesus performed many other signs in the presence of his\fnote{\fbackref{20:30} Other mss. read \fbib{the}} disciples that are not recorded in this book. \v{31}But these have been recorded so that you may believe that Jesus is the Messiah,\fnote{\fbackref{20:31} Or \fbib{Christ}} the Son of God, and so that through believing you may have life in his name.
\labelchapt{21}
\passage{Jesus Appears to Seven of His Disciples}

\chapt{21}
\v{1}Later on, Jesus revealed himself again to the disciples at the Sea of Tiberias. This is what happened: \v{2}Simon Peter, Thomas (called the Twin),\fnote{\fbackref{21:2} Lit. \fbib{Didymus}} Nathaniel from Cana in Galilee, the sons of Zebedee, and two of his other disciples were together. \v{3}Simon Peter told them, ``I'm going fishing.''

They all told him, ``We'll go with you, too.'' So they went out and got into the boat but didn't catch a thing that night.

\v{4}Just as dawn was breaking, Jesus stood on the shore. The disciples didn't realize it was Jesus. \v{5}Jesus asked them, \red{``Children, you don't have any fish, do you?''}

They answered him, ``No.''

\v{6}He told them, \red{``Throw the net on the right hand side of the boat, and you'll catch\fnote{\fbackref{21:6} Lit. \fbib{find}} some.''} So they threw it out and were unable to haul it in because it was so full of fish.

\v{7}That disciple whom Jesus kept loving told Peter, ``It's the Lord!'' When Simon Peter heard that it was the Lord, he put his clothes back on, because he was practically naked, and jumped into the sea. \v{8}But the other disciples came in the boat, dragging the net full of fish. They were only about 200 cubits\fnote{\fbackref{21:8} I.e. about 100 yards; a cubit was about eighteen inches} away from the shore.

\v{9}When they arrived at the shore, they saw a charcoal fire with fish lying on it, and some bread. \v{10}Jesus told them, \red{``Bring me some of the fish you've just caught.''} \v{11}So Simon Peter went aboard and dragged the net ashore. It was full of large fish---153 of them. And although there were so many of them, the net was not torn.

\v{12}Then Jesus told them, \red{``Come, have breakfast.''} Now none of the disciples dared to ask him, ``Who are you?'', because they knew it was the Lord. \v{13}Jesus took the bread, gave it to them, and did\fnote{\fbackref{21:13} The Gk. lacks \fbib{did}} the same with the fish. \v{14}This was now the third time that Jesus revealed himself to the disciples after he had been raised from the dead.
\passage{Jesus Speaks with Peter}

\v{15}When they had finished breakfast, Jesus asked Simon Peter, \red{``Simon, son of John, do you love me more than these?''}

Peter\fnote{\fbackref{21:15} Lit. \fbib{He}} told him, ``Yes, Lord, you know that I love you.''

Jesus\fnote{\fbackref{21:15} Lit. \fbib{He}} told him, \red{``Feed my lambs.''}

\v{16}Then he asked him a second time, \red{``Simon, son of John, do you love me?''}

Peter\fnote{\fbackref{21:16} Lit. \fbib{He}} told him, ``Yes, Lord, you know that I love you.''

Jesus\fnote{\fbackref{21:16} Lit. \fbib{He}} told him, \red{``Take care of my sheep.''} \v{17}He asked him a third time, \red{``Simon, son of John, do you love me?''}

Peter was deeply hurt that he had asked him a third time, \red{``Do you love me?''} So he told him, ``Lord, you know everything. You know that I love you!''

Jesus told him, \red{``Feed my sheep.} \v{18}\red{``Truly, I tell you\fnote{\fbackref{21:18} The Gk. pronoun \fbib{you} is pl.} emphatically, when you were young, you would fasten your belt and go wherever you liked. But when you get old, you will stretch out your hands, and someone else will fasten your belt and take you where you don't want to go.''} \v{19}Now he said this to show by what kind of death he would glorify God.

After saying this, Jesus\fnote{\fbackref{21:19} Lit. \fbib{he}} told him, \red{``Keep following me.''}
\passage{Jesus and the Beloved Disciple}

\v{20}Peter turned around and noticed the disciple whom Jesus kept loving following them. He was the one who had put his head on Jesus' chest at the supper and had asked, ``Lord, who is the one who is going to betray you?''

\v{21}When Peter saw him, he said, ``Lord, what about him?''

\v{22}Jesus told him, \red{``If it's my will for him to remain until I come back, how does that concern you? You must keep following me!''} \v{23}So the rumor spread among the brothers\fnote{\fbackref{21:23} I.e. Jesus' followers} that this disciple wasn't going to die. Yet Jesus didn't say to Peter\fnote{\fbackref{21:23} Lit. \fbib{him}} that he wasn't going to die, but, \red{``If it's my will for him to remain until I come back, how does that concern you?''}

\v{24}This is the disciple who is testifying to these things and has written them down. We know that his testimony is true.

\v{25}Of course, Jesus also did many other things, and I suppose that if every one of them were written down, the world couldn't contain the books that would be written.

\addcontentsline{toc}{chapter}{The Proclaiming of the Good News}
\bookheader{Acts}
\labelbook{Acts}

\bookpretitle{The Book of}
\booktitle{Acts}

\labelchapt{1}
\passage{Introduction}

\chapt{1}
\v{1}In my first book, Theophilus, I wrote about everything Jesus did and taught from the beginning, \v{2}up to the day when he was taken up to heaven\fnote{\fbackref{1:2} The Gk. lacks \fbib{to heaven}} after giving orders by the Holy Spirit to the apostles he had chosen. \v{3}After he had suffered, he had shown himself alive to them by many convincing proofs, appearing to them during a period of 40 days and telling them about the kingdom of God.
\passage{The Promise of the Holy Spirit}

\v{4}While he was meeting with them, he ordered them, \red{``Don't leave Jerusalem. Instead, wait for what the Father has promised, about which you heard me speak,} \v{5}\red{because John baptized with\fnote{\fbackref{1:5} Or \fbib{in}} water, but you will be baptized with\fnote{\fbackref{1:5} Or \fbib{in}} the Holy Spirit a few days from now.''}

\v{6}Now those who had gathered together began to ask Jesus,\fnote{\fbackref{1:6} Lit. \fbib{him}} ``Lord, is this the time when you will restore the kingdom to Israel?'' \v{7}He answered them, \red{``It isn't for you to know what times or periods the Father has fixed by his own authority.} \v{8}\red{But you'll receive power when the Holy Spirit comes on you, and you'll be my witnesses in Jerusalem, in all Judea and Samaria, and to the ends of the earth.''}
\passage{Jesus Goes Up to Heaven}

\v{9}After saying this, Jesus\fnote{\fbackref{1:9} Lit. \fbib{he}} was taken up while those who had gathered together\fnote{\fbackref{1:9} Lit. \fbib{while they}} were watching, and a cloud took him out of their sight. \v{10}While he was going and they were gazing up toward heaven, two men in white robes stood right beside them. \v{11}They asked, ``Men of Galilee, why do you stand looking up toward heaven? This same Jesus, who has been taken up from you into heaven, will come back in the same way you saw him go up into heaven.''
\passage{A New Apostle Takes the Place of Judas}

\v{12}Then they returned to Jerusalem from the Mount of Olives,\fnote{\fbackref{1:12} Lit. \fbib{from the mountain called Olives}} which is near Jerusalem, a Sabbath day's journey away.\fnote{\fbackref{1:12} I.e. about a half mile away} \v{13}When they came into the city, these men\fnote{\fbackref{1:13} The Gk. lacks \fbib{men}} went to the upstairs room where they had been staying: Peter and John, James and Andrew, Philip and Thomas, Bartholomew and Matthew, James the son of Alphaeus and Simon the Zealot, and Judas the son\fnote{\fbackref{1:13} Or \fbib{brother}} of James. \v{14}With one mind, all of them kept devoting themselves to prayer, along with the women (including Mary the mother of Jesus) and his brothers.

\v{15}At that time,\fnote{\fbackref{1:15} Lit. \fbib{In those days}} Peter got up among the brothers (there were about 120 people present) and said, \v{16}``Brothers, the Scripture had to be fulfilled, which the Holy Spirit spoke long ago through the voice of David about Judas, who was the guide for those who arrested Jesus, \v{17}because he was one of our number and was appointed\fnote{\fbackref{1:17} Lit. \fbib{was chosen by lot}; i.e. by an ostensibly random lottery, the outcome of which was entrusted to God's providence; cf. v. 26} to share in this ministry.'' \v{18}(Now this man bought a field with the money he got for his crime. Falling on his face, he burst open in the middle, and all his intestines gushed out. \v{19}This became known to all the residents of Jerusalem, so that this field is called in their language Hakeldama, that is, ``The Field of Blood''.) \v{20}``For in the Book of Psalms it is written, `Let his estate be desolate, and let no one live on it,'\fnote{\fbackref{1:20} Cf. Ps 69:25} and, `Let someone else take over his office,'\fnote{\fbackref{1:20} Cf. Ps 109:8} \v{21}who was one of the men associated with us all the time the Lord Jesus came and went among us, \v{22}beginning when he was baptized by John until the day he was taken up from us. Therefore, someone like this\fnote{\fbackref{1:22} The Gk. lacks \fbib{like this}} must become a witness with us to his resurrection.''

\v{23}So they nominated two men---Joseph called Barsabbas, who also was called Justus, and Matthias. \v{24}Then they prayed, ``Lord, you know the hearts of all people. Show us which one of these two men you have chosen \v{25}to serve in this office of apostle,\fnote{\fbackref{1:25} Lit. \fbib{to receive the place of this service and apostleship}} from which Judas left to go to his own place.''

\v{26}So they drew lots for them, and when the lot fell on Matthias, he was enrolled with the eleven apostles.
\labelchapt{2}
\passage{The Coming of the Holy Spirit}

\chapt{2}
\v{1}When the day of Pentecost was being celebrated,\fnote{\fbackref{2:1} Lit. \fbib{Pentecost had fully arrived}} all of them were together in one place. \v{2}Suddenly, a sound like the roar of a mighty windstorm came from heaven and filled the whole house where they were sitting. \v{3}They saw tongues like flames\fnote{\fbackref{2:3} Or \fbib{tongues}} of fire that separated, and one rested on each of them. \v{4}All of them were filled with the Holy Spirit and began to speak in foreign\fnote{\fbackref{2:4} Or \fbib{different}} languages as the Spirit gave them that ability.

\v{5}Now devout Jews from every nation on earth\fnote{\fbackref{2:5} Lit. \fbib{nation under heaven}} were living in Jerusalem. \v{6}When that sound came, a crowd quickly gathered, startled because each one heard the disciples\fnote{\fbackref{2:6} Lit. \fbib{them}} speaking in his own language. \v{7}Stunned and amazed, they asked, ``All of these people who are speaking are Galileans, aren't they? \v{8}So how is it that each one of us hears them speaking in his own native language:\fnote{\fbackref{2:8} Lit. \fbib{in our language in which we were born}} \v{9}Parthians, Medes, Elamites, residents of Mesopotamia, Judea, Cappadocia, Pontus, Asia, \v{10}Phrygia, Pamphylia, Egypt, the district of Libya near Cyrene, Jewish and proselyte visitors from Rome, \v{11}Cretans, and Arabs, listening to them talk in our own languages about the great deeds of God?''

\v{12}All of them continued to be stunned and puzzled, and they kept asking one another, ``What can this mean?''

\v{13}But others kept saying in derision, ``They're drunk on sweet wine!''
\passage{Peter Addresses the Crowd}

\v{14}Then Peter stood up among the eleven apostles\fnote{\fbackref{2:14} The Gk. lacks \fbib{apostles}} and raised his voice to address them:

``Men of Judea and everyone living in Jerusalem! You must understand something, so pay close attention to my words. \v{15}These men are not drunk as you suppose, for it's only nine o'clock in the morning.\fnote{\fbackref{2:15} Lit. \fbib{the third hour of the day}} \v{16}Rather, this is what was spoken through the prophet Joel:

\begin{poetry}
\poeml \v{17}`In the last days, God says, \\
\poemll    I will pour out my Spirit on everyone.\fnote{\fbackref{2:17} Lit. \fbib{on all flesh}} \\
\poeml Your sons and your daughters will prophesy, \\
\poemll    your young men will see visions, \\
\poemlll       and your old men will dream dreams. \\
\poeml \v{18}In those days I will even pour out my Spirit \\
\poemll    on my slaves, men and women alike, \\
\poemlll       and they will prophesy. \\
\poeml \v{19}I will display wonders in the sky above \\
\poemll    and signs on the earth below: \\
\poemlll       blood, fire, and clouds of smoke. \\
\poeml \v{20}The sun will become dark, \\
\poemll    and the moon turn to blood, \\
\poemlll       before the coming of the great and glorious Day of the Lord.\fnote{\fbackref{2:20} MT source citation reads \fbib{}\divine{Lord}} \\
\poeml \v{21}Then whoever calls on the name of the Lord\fnote{\fbackref{2:21} MT source citation reads \fbib{}\divine{Lord}} will be saved.'\fnote{\fbackref{2:21} Cf. Joel 2:28-32}
\end{poetry}

\v{22}``Fellow Israelis, listen to these words: Jesus from Nazareth\fnote{\fbackref{2:22} Or \fbib{Jesus the Nazarene}; the Gk. \fbib{Nazoraios} may be a word play between Heb. \fbib{netser,} meaning \fbib{branch} (cf. Isa 11:1), and the name \fbib{Nazareth}.} was a man authenticated to you by God through miracles, wonders, and signs that God performed through him among you, as you yourselves know. \v{23}After he was arrested according to the predetermined plan and foreknowledge of God, you crucified this very man and killed him using the hands of lawless men. \v{24}But God raised him up, putting an end to the suffering of death,\fnote{\fbackref{2:24} Other mss. read \fbib{Hades} (the realm of the dead)} since it was impossible for him to be held by it. \v{25}For David says about him,

\begin{poetry}
\poeml `I always keep my eyes on the Lord,\fnote{\fbackref{2:25} Lit. \fbib{always see the Lord in front of me}; MT source citation reads \fbib{\divine{Lord}}} \\
\poemll    for he is at my right hand \\
\poemlll       so that I cannot be shaken. \\
\poeml \v{26}That is why my heart is glad \\
\poemll    and my tongue rejoices, \\
\poemlll       yes, even my body still rests securely in hope. \\
\poeml \v{27}For you will not abandon my soul to Hades\fnote{\fbackref{2:27} I.e. the realm of the dead} \\
\poemll    or allow your Holy One to experience decay. \\
\poeml \v{28}You have made the ways of life known to me, \\
\poemll    and you will fill me with gladness in your presence.'\fnote{\fbackref{2:28} Cf. Ps 16:8-11}
\end{poetry}

\v{29}``Brothers, I can tell you confidently that the patriarch David died and was buried, and that his tomb is among us to this day. \v{30}Therefore, since he was a prophet and knew that God had promised him with an oath to put one of his descendants on his throne, \v{31}he looked ahead and spoke about the resurrection of the Messiah:\fnote{\fbackref{2:31} Or \fbib{Christ}}

\begin{poetry}
\poeml `He was not abandoned to Hades,\fnote{\fbackref{2:31} I.e. the realm of the dead} \\
\poemll    and his flesh did not experience decay.'\fnote{\fbackref{2:31} Cf. Ps 16:10}
\end{poetry}

\v{32}``It was this very Jesus whom God raised---and we're all witnesses of that. \v{33}He has been exalted to the right hand of God, has received from the Father the promised Holy Spirit,\fnote{\fbackref{2:33} Or \fbib{the promise of the Holy Spirit}} and has caused you to experience what you are seeing and hearing. \v{34}After all, David did not go up to heaven, but he said,

\begin{poetry}
\poeml `The Lord\fnote{\fbackref{2:34} MT source citation reads \fbib{\divine{Lord}}} told my Lord, \\
\poemll    ``Sit at my right hand, \\
\poeml \v{35}until I make your enemies your footstool.''\,'\fnote{\fbackref{2:35} Cf. Ps 110:1}
\end{poetry}

\v{36}``Therefore, let all the people\fnote{\fbackref{2:36} Lit. \fbib{house}} of Israel understand beyond a doubt that God made this Jesus, whom you crucified, both Lord and Messiah!''\fnote{\fbackref{2:36} Or \fbib{Christ}}

\v{37}When the crowd that had gathered\fnote{\fbackref{2:37} Lit. \fbib{When they}} heard this, they were pierced to the heart. They asked Peter and the other apostles, ``Brothers, what should we do?''

\v{38}Peter answered them, ``Every one of you must repent and be baptized in the name of Jesus the Messiah\fnote{\fbackref{2:38} Or \fbib{Christ}} for the forgiveness of your sins. Then you will receive the Holy Spirit as a gift.\fnote{\fbackref{2:38} Or \fbib{the gift of the Holy Spirit}} \v{39}For this promise belongs to you and your children, as well as to all those who are distant, whom the Lord our God may call to himself.''

\v{40}Using many different expressions, Peter\fnote{\fbackref{2:40} Lit. \fbib{he}} continued to testify and to plead: ``Be saved,'' he urged them, ``from this corrupt generation!'' \v{41}So those who welcomed his message were baptized. That day about 3,000 people were added to their number.
\passage{Life among the Believers}

\v{42}The believers\fnote{\fbackref{2:42} Lit. \fbib{They}} continued to devote themselves to what the apostles were teaching, to fellowship, to the breaking of bread, and to times of prayer.\fnote{\fbackref{2:42} Lit. \fbib{to the prayers}} \v{43}A sense of fear\fnote{\fbackref{2:43} Or \fbib{awe}} came over everyone, and many wonders and signs were being done by the apostles. \v{44}All the believers were united and shared everything with one another.\fnote{\fbackref{2:44} Lit. \fbib{and had all things in common}} \v{45}They made it their practice to sell their possessions and goods and to distribute the proceeds\fnote{\fbackref{2:45} Lit. \fbib{to distribute them}} to anyone who was in need. \v{46}United in purpose, they went to the Temple every day, ate at each other's homes, and shared their food with glad and humble hearts. \v{47}They were praising God and enjoying the good will of all the people. Every day the Lord was adding to their number those who were being saved.
\labelchapt{3}
\passage{A Crippled Man is Healed}

\chapt{3}
\v{1}One afternoon, Peter and John were on their way to the Temple for the three o'clock prayer time.\fnote{\fbackref{3:1} Lit. \fbib{temple at the ninth hour}} \v{2}Now a man who had been crippled from birth was being carried in. Every day people\fnote{\fbackref{3:2} Lit. \fbib{they}} would lay him at what was called the Beautiful Gate so that he could beg from those who were going into the Temple. \v{3}When he saw that Peter and John were about to go into the Temple, he asked them to give him something.

\v{4}Peter, along with John, looked him straight in the eye and said, ``Look at us!'' \v{5}So the man\fnote{\fbackref{3:5} Lit. \fbib{he}} watched them closely, expecting to get something from them. \v{6}However, Peter said, ``I don't have any silver or gold, but I'll give you what I do have. In the name of Jesus the Messiah\fnote{\fbackref{3:6} Or \fbib{Christ}} from Nazareth, walk!''\fnote{\fbackref{3:6} Other mss. read \fbib{stand up and walk}} \v{7}Then Peter\fnote{\fbackref{3:7} Lit. \fbib{he}} took hold of his right hand and began to help him up. Immediately his feet and ankles became strong, \v{8}and he sprang to his feet, stood up, and began to walk. Then he went with them into the Temple, walking, jumping, and praising God.

\v{9}When all the people saw him walking and praising God, \v{10}they knew that he was the man who used to sit and beg at the Beautiful Gate of the Temple, and they were filled with wonder and amazement at what had happened to him.
\passage{Peter Speaks to the Onlookers}

\v{11}While he was holding on to Peter and John, all the people came running to them in what was called ``Solomon's Colonnade''. They were dumbfounded. \v{12}When Peter saw this, he told the people: ``Fellow Israelis, why are you wondering about this, and why are you staring at us as if by our own power or godliness we made him walk? \v{13}The God of Abraham, Isaac, and Jacob---the God of our ancestors---has glorified his servant Jesus, whom you betrayed and rejected in the presence of Pilate, even though he had decided to let him go. \v{14}You rejected the Holy and Righteous One and asked to have a murderer released to you, \v{15}and you killed the source of life, whom God raised from the dead. We are witnesses to that. \v{16}It is his name---that is, by faith in his name---that has healed this man whom you see and know. Yes, the faith that comes through Jesus\fnote{\fbackref{3:16} Lit. \fbib{him}} has given him this perfect health in the presence of all of you.

\v{17}``And now, brothers, I know that you acted in ignorance like your leaders. \v{18}This is how God fulfilled what he had predicted through the voice of all the prophets---that his Messiah\fnote{\fbackref{3:18} Or \fbib{Christ}} would suffer. \v{19}Therefore, repent and turn to him to have your sins blotted out, \v{20}so that times of refreshing may come from the presence of the Lord and so that he may send you Jesus, whom he appointed long ago to be the Messiah.\fnote{\fbackref{3:20} Or \fbib{Christ}} \v{21}He must remain in heaven until the time of universal restitution, which God announced long ago through the voice of his holy prophets. \v{22}In fact, Moses said,

`The Lord\fnote{\fbackref{3:22} MT source citation reads \fbib{\divine{Lord}}} your God will raise up for you a prophet like me from among your brothers. You must listen to everything he tells you.\fnote{\fbackref{3:22} Cf. Deut 18:15-16} \v{23}Any person who will not listen to that prophet will be utterly destroyed from among the people.'\fnote{\fbackref{3:23} Cf. Deut 18:19; Lev 23:29}

\v{24}``Indeed, all the prophets who have spoken, from Samuel and those who followed him, also announced these days. \v{25}You are the descendants of the prophets and the heirs\fnote{\fbackref{3:25} The Gk. lacks \fbib{the heirs}} of the covenant that God made with your\fnote{\fbackref{3:25} Other mss. read \fbib{our}} ancestors when he told Abraham, `Through your descendant\fnote{\fbackref{3:25} Lit. \fbib{seed}} all the families of the earth will be blessed.'\fnote{\fbackref{3:25} Cf. Gen 22:18; 26:4} \v{26}When God raised up his servant, he sent him first to you to bless you by turning every one of you from your evil ways.''
\labelchapt{4}
\passage{Peter and John are Tried before the Jewish Council}

\chapt{4}
\v{1}While they were speaking to the people, the priests, the commander of the Temple guards, and the Sadducees came to them. \v{2}They were greatly disturbed that Peter and John\fnote{\fbackref{4:2} Lit. \fbib{they}} were teaching the people and announcing that Jesus had been resurrected\fnote{\fbackref{4:2} Lit. \fbib{that in the case of Jesus there had been a resurrection}} from the dead. \v{3}So they arrested them and placed them in custody until the next day, since it was already evening. \v{4}But many of those who heard their message believed, and the men grew to number about 5,000.

\v{5}The next day, their rulers, elders, and scribes met in Jerusalem \v{6}with Annas the high priest, Caiaphas, John,\fnote{\fbackref{4:6} Other mss. read \fbib{Jonathan}} Alexander, and the rest of the high priest's family. \v{7}They made Peter and John\fnote{\fbackref{4:7} Lit. \fbib{them}} stand in front of them and began asking, ``By what power or by what name did you do this?''

\v{8}Peter, filled with the Holy Spirit, told them, ``Rulers and elders of the people! \v{9}If we are being questioned today for a good deed done for someone who was sick or to learn how this man was healed, \v{10}you and all the people of Israel must understand that this man stands healthy before you because of the name of Jesus from Nazareth, whom you crucified, but whom God raised from the dead. \v{11}He is

\begin{poetry}
\poeml `the stone that was rejected by you builders, \\
\poemll    which has become the cornerstone.'\fnote{\fbackref{4:11} Or \fbib{capstone}; cf. Ps 118:22}
\end{poetry}

\v{12}There is no salvation by anyone else, for there is no other name under heaven given among people by which we must be saved.''

\v{13}Now when the Jewish leaders\fnote{\fbackref{4:13} Lit. \fbib{when they}} saw the boldness of Peter and John and found out that they were uneducated and ordinary men, they were amazed and realized that they had been with Jesus. \v{14}And seeing the man who was healed standing with them, they could not say anything against them. \v{15}So they ordered them to leave the Council\fnote{\fbackref{4:15} Or \fbib{Sanhedrin}} and began to discuss the matter among themselves. \v{16}They said, ``What should we do with these men? For it's obvious to everybody living in Jerusalem that an unmistakable sign has been done by them, and we cannot deny it. \v{17}But to keep it from spreading any further among the people, let us warn them never again to speak to anyone in this name.''

\v{18}So they called Peter and John\fnote{\fbackref{4:18} Lit. \fbib{called them}} back in and ordered them not to speak or teach at all in the name of Jesus. \v{19}But Peter and John answered them, ``You must decide whether it is right in the sight of God to listen to you rather than God, \v{20}for we cannot stop talking about what we've seen and heard.''

\v{21}So they threatened Peter and John\fnote{\fbackref{4:21} Lit. \fbib{threatened them}} even more and then let them go. They couldn't find any way to punish them, because all the people were praising God for what had happened, \v{22}since the man on whom this sign of healing had been performed was more than 40 years old.
\passage{The Believers Pray for Boldness}

\v{23}After they were released, Peter and John\fnote{\fbackref{4:23} Lit. \fbib{released, they}} went to their fellow believers\fnote{\fbackref{4:23} Lit. \fbib{their own}} and told them everything the high priests and the elders had said. \v{24}When they heard this, they all raised their voices to God and said, ``Master, you made heaven and earth, the sea, and everything in them. \v{25}You said by the Holy Spirit through the voice of our ancestor, your servant David,

`Why do the \red{unbelievers}\fnote{\fbackref{4:25} Lit. \fbib{gentiles} ; i.e. unbelieving non-Jews} rage,

\begin{poetry}
\poemll    and the people devise useless plots? \\
\poeml \v{26}The kings of the earth take their stand, \\
\poemll    and rulers meet together against the Lord\fnote{\fbackref{4:26} MT source citation reads \fbib{\divine{Lord}}} \\
\poemlll       and against his Messiah.'\fnote{\fbackref{4:26} Or \fbib{Christ}; cf. Ps 2:1-2}
\end{poetry}

\v{27}For in this city both Herod and Pontius Pilate actually met together with \red{unbelievers}\fnote{\fbackref{4:27} Lit. \fbib{gentiles} ; i.e. unbelieving non-Jews} and the people of Israel to oppose your holy servant Jesus, whom you anointed, \v{28}to carry out everything that your hand and will had predetermined to take place. \v{29}Lord, pay attention to their threats now, and allow your servants to speak your word boldly \v{30}as you stretch out your hand to heal and to perform signs and wonders through the name of your holy servant Jesus.''

\v{31}When they had prayed, the place where they were meeting was shaken, and all of them were filled with the Holy Spirit and continued to speak messages from\fnote{\fbackref{4:31} Lit. \fbib{speak the word of}} God boldly.
\passage{The Believers Share Their Possessions}

\v{32}Now all the believers were one in heart and soul, and nobody called any of his possessions his own. Instead, they shared everything they owned. \v{33}With great power, the apostles continued to testify to the resurrection of the Lord Jesus, and abundant grace was on them all, \v{34}since none of them needed anything, because everyone who had land or houses would sell them and bring the money received for the things sold \v{35}and lay it at the apostles' feet. Then it was distributed to anyone who needed it. \v{36}One man,\fnote{\fbackref{4:36} Lit. \fbib{Now}} Joseph, a descendant of Levi and a native of Cyprus, who was named Barnabas by the apostles (the name\fnote{\fbackref{4:36} Lit. \fbib{which}} means ``a son of encouragement''), \v{37}sold a field that belonged to him, brought the money, and laid it at the apostles' feet.
\labelchapt{5}
\passage{Ananias and Sapphira are Punished}

\chapt{5}
\v{1}But then a man named Ananias, with the consent of his wife Sapphira, sold some property. \v{2}With his wife's full knowledge, he kept back some of the money for himself and brought the remainder and laid it at the apostles' feet.

\v{3}Peter asked, ``Ananias, why has Satan filled your heart so that you should lie to the Holy Spirit and keep back some of the money you got for the land? \v{4}As long as it remained unsold, wasn't it your own? And after it was sold, wasn't the money at your disposal? So how could you have thought of doing what you did? You didn't lie only\fnote{\fbackref{5:4} The Gk. lacks \fbib{only}} to men, but also\fnote{\fbackref{5:4} The Gk. lacks \fbib{also}} to God!''

\v{5}When Ananias heard these words, he fell down and died. And great fear seized everyone who heard about it. \v{6}The young men got up, wrapped him up, carried him outside, and buried him. \v{7}After an interval of about three hours, Ananias'\fnote{\fbackref{5:7} Lit. \fbib{his}} wife came in, not knowing what had happened. \v{8}So Peter asked her, ``Tell me, did you sell the land for that price?''

She answered, ``Yes, that was the price.''

\v{9}``How could you have agreed together to test the Spirit of the Lord?'' Peter asked her. ``Listen! The feet of the men who buried your husband are at the door, and these men\fnote{\fbackref{5:9} Lit. \fbib{and they}} will carry you outside as well.'' \v{10}She instantly fell down at Peter's\fnote{\fbackref{5:10} Lit. \fbib{his}} feet and died. When the young men came in, they found her dead. So they carried her out and buried her next to her husband. \v{11}And great fear seized the whole church and everyone else who heard about this.
\passage{The Apostles Perform Many Miracles}

\v{12}Now many signs and wonders were being performed by the apostles among the people, who were gathered together in Solomon's Colonnade. \v{13}None of the others dared join them, although the rest of the people continued to hold them in high regard. \v{14}Nevertheless, believers were being added to the Lord in increasing numbers---both men and women. \v{15}As a result, people\fnote{\fbackref{5:15} Lit. \fbib{they}} kept carrying their sick into the streets and placing them on stretchers and mats so that at least Peter's shadow might fall on some of them as he went by. \v{16}Crowds continued coming in---even from the towns around Jerusalem---bringing their sick and those who were troubled by unclean spirits, and all of them were healed.
\passage{The Apostles are Tried before the Jewish Council}

\v{17}Then the high priest and all those from the sect of the Sadducees who were with him were filled with jealousy. So they went out, \v{18}arrested the apostles, and put them in the city jail. \v{19}But at night the angel of the Lord opened the prison doors and led them out. The angel\fnote{\fbackref{5:19} Lit. \fbib{He}} told them, \v{20}``Go, stand in the Temple, and keep on telling the people the whole message about this life they can have.''\fnote{\fbackref{5:20} The Gk. lacks \fbib{they can have}}

\v{21}After the apostles\fnote{\fbackref{5:21} Lit. \fbib{this, they}} heard this, they went into the Temple at daybreak and began to teach. The high priest and those who were with him arrived, called the Council\fnote{\fbackref{5:21} Or \fbib{Sanhedrin}} and all the elders of Israel together, and sent word\fnote{\fbackref{5:21} The Gk. lacks \fbib{word}} to the prison to have the men brought in. \v{22}When the Temple police got there, they did not find them in the prison. They came back and reported, \v{23}``We found the prison securely locked and the guards standing at the doors, but when we opened them, we found no one inside.'' \v{24}When the commander of the Temple guards and the high priests heard these words, they were utterly at a loss as to what could have happened to them.

\v{25}Then someone came and told them, ``Look! The men you put in prison are standing in the Temple and teaching the people!'' \v{26}So the commander of the Temple guards went with his men to bring them back without force, because they were afraid of being stoned to death by the people. \v{27}When they brought them back, they made them stand before the Council,\fnote{\fbackref{5:27} Or \fbib{Sanhedrin}} and the high priest began to question them. \v{28}He said, ``We gave you strict orders not to teach in his name, didn't we? Yet you have filled Jerusalem with your teaching and are determined to bring this man's blood on us!''

\v{29}But Peter and the apostles answered, ``We must obey God rather than men! \v{30}The God of our ancestors raised Jesus to life after you killed him by hanging him on a tree. \v{31}God has exalted to his right hand this very man as our Leader and Savior in order to extend repentance and forgiveness of sins to Israel. \v{32}We are witnesses of these things, and so is the Holy Spirit, whom God has given to those who keep on obeying him.''

\v{33}When the Council\fnote{\fbackref{5:33} Lit. \fbib{When they}} heard this, they became furious and wanted to kill them. \v{34}But a Pharisee named Gamaliel, a teacher of the Law who was respected by all the people, stood up in the Council\fnote{\fbackref{5:34} Or \fbib{Sanhedrin}} and ordered the men to be taken outside for a little while. \v{35}Then he told them, ``Fellow Israelis, consider carefully what you propose to do to these men. \v{36}For in the recent past Theudas appeared, claiming that he was important, and about 400 men joined him. He was killed, and all his followers were dispersed and disappeared. \v{37}After that man, at the time of the census, Judas the Galilean appeared and got people to follow him. He, too, died, and all his followers were scattered.

\v{38}``I'm telling you to keep away from these men for now. Leave them alone, because if this plan or movement is of human origin, it will fail. \v{39}However, if it's from God, you won't be able to stop them, and you may even discover that you are fighting against God!''

So they were convinced by him. \v{40}After calling in the apostles and beating them, they again\fnote{\fbackref{5:40} The Gk. lacks \fbib{again}} ordered them to stop speaking in the name of Jesus and let them go. \v{41}They left the Council,\fnote{\fbackref{5:41} Or \fbib{Sanhedrin}} rejoicing to have been considered worthy to suffer dishonor for the sake of the Name. \v{42}Every day in the Temple and from house to house they kept teaching and proclaiming that Jesus is the Messiah.\fnote{\fbackref{5:42} Or \fbib{Christ}}
\labelchapt{6}
\passage{Seven Men are Chosen to Help the Apostles}

\chapt{6}
\v{1}In those days, as the number of the disciples was growing larger and larger, a complaint was made by the Hellenistic Jews against the Hebraic Jews that their widows were being neglected in the daily distribution of food. \v{2}So the Twelve called the whole group of disciples together and said, ``It is not desirable for us to neglect messages from\fnote{\fbackref{6:2} Lit. \fbib{neglect the word of}} God in order to wait on tables. \v{3}Therefore, brothers, appoint seven men among you who have a good reputation, who are full of the Spirit and wisdom, and we'll put them in charge of this work. \v{4}Then we'll devote ourselves to prayer and to the ministry of the word.''

\v{5}This suggestion pleased the whole group. So they chose Stephen, a man full of faith and the Holy Spirit, Philip, Prochorus, Nicanor, Timon, Parmenas, and Nicolaus, a gentile convert to Judaism from Antioch. \v{6}They had these men stand before the apostles, who prayed and laid their hands on them.

\v{7}So the word of God\fnote{\fbackref{6:7} Other mss. read \fbib{of the Lord}} continued to spread, and the number of disciples in Jerusalem continued to grow rapidly. Even a large number of priests became obedient to the faith.
\passage{Stephen is Arrested}

\v{8}Now Stephen, full of grace and power, was performing great wonders and signs among the people. \v{9}But some men who belonged to the Synagogue of the Freedmen (as it was called), as well as some Cyrenians, Alexandrians, and men from Cilicia and Asia, stood up and began to debate with Stephen. \v{10}But they could neither refute the wisdom nor withstand the Spirit by which he kept speaking. \v{11}So they secretly got some men to say, ``We have heard him speaking blasphemous words against Moses and God.'' \v{12}They stirred up the people, the elders, and the scribes. Then they rushed at Stephen,\fnote{\fbackref{6:12} Lit. \fbib{him}} grabbed him, and brought him before the Council.\fnote{\fbackref{6:12} Or \fbib{Sanhedrin}}

\v{13}They had false witnesses stand up and say, ``This man never stops saying things against this Holy Place and against the Law. \v{14}For we have heard him say that this Jesus from Nazareth\fnote{\fbackref{6:14} Or \fbib{Jesus the Nazarene}; the Gk. \fbib{Nazoraios} may be a word play between Heb. \fbib{netser,} meaning \fbib{branch} (cf. Isa 11:1), and the name \fbib{Nazareth.}} will destroy this place and change the customs that Moses handed down to us.'' \v{15}Then everyone who was seated in the Council\fnote{\fbackref{6:15} Or \fbib{Sanhedrin}} glared at him and saw that his face was like the face of an angel.
\labelchapt{7}
\passage{Stephen Defends Himself}

\chapt{7}
\v{1}Then the high priest asked, ``Is this true?''

\v{2}Stephen replied:

``Listen, brothers and fathers!

``The glorious God appeared to our ancestor Abraham while he was in Mesopotamia before he settled in Haran. \v{3}God\fnote{\fbackref{7:3} Lit. \fbib{He}} told him, `Leave your country and your relatives and go to the land I'll show you.'\fnote{\fbackref{7:3} Cf. Gen 12:1} \v{4}So he left the country of the Chaldeans and settled in Haran. Then after the death of his father, God had him move to this country where you now live. \v{5}God\fnote{\fbackref{7:5} Lit. \fbib{He}} gave him no property here,\fnote{\fbackref{7:5} Lit. \fbib{in it}} not even a foot of land,\fnote{\fbackref{7:5} I.e. about half of a single stride, or about 1.5 feet} yet he promised to give it to him and to his descendants\fnote{\fbackref{7:5} Lit. \fbib{seed}} after him as a permanent possession, even though he had no child.

\v{6}``This is what God promised: His descendants would be strangers in a foreign country, and its people\fnote{\fbackref{7:6} Lit. \fbib{they}} would enslave them and oppress them for 400 years. \v{7}`But I will punish the nation they serve,' said God, `and afterwards they will leave and worship me in this place.'\fnote{\fbackref{7:7} Cf. Gen 15:13-14; Exod 3:12}

\v{8}Later, God\fnote{\fbackref{7:8} Lit. \fbib{he}} gave Abraham\fnote{\fbackref{7:8} Lit. \fbib{him}} the covenant of circumcision. Later, he fathered Isaac and circumcised him on the eighth day. Then Isaac fathered Jacob, and Jacob fathered\fnote{\fbackref{7:8} The Gk. lacks \fbib{fathered}} the twelve patriarchs.

\v{9}``Joseph's brothers\fnote{\fbackref{7:9} Lit. \fbib{The patriarchs}} became jealous of him and sold Joseph as a slave\fnote{\fbackref{7:9} The Gk. lacks \fbib{as a slave}} in Egypt. However, God was with him \v{10}and rescued him from all his troubles. He granted him favor and wisdom before Pharaoh, king of Egypt, who appointed him ruler of Egypt and of his whole household.

\v{11}``But a famine spread throughout Egypt and Canaan, and with it great suffering, and our ancestors couldn't find any food. \v{12}But when Jacob heard that there was grain in Egypt, he sent our ancestors on their first trip. \v{13}On their second trip, Joseph made himself known to his brothers, and Joseph introduced his family\fnote{\fbackref{7:13} Lit. \fbib{Joseph's family became known}} to Pharaoh. \v{14}Then Joseph invited his father Jacob and all his relatives to come to him in Egypt\fnote{\fbackref{7:14} The Gk. lacks \fbib{in Egypt}}---75 persons in all. \v{15}So Jacob went down to Egypt. Then he and our ancestors died. \v{16}They were brought back to Shechem and laid in the tomb that Abraham had bought at a high price\fnote{\fbackref{7:16} Lit. \fbib{Abraham for a sum of money}} from Hamor's descendants in Shechem.

\v{17}``Now as the time approached for the fulfillment of the promise that God had made to Abraham, the people's population increased a great deal in Egypt. \v{18}Eventually, a different king who had not known Joseph became ruler of Egypt.\fnote{\fbackref{7:18} Other mss. lack \fbib{of Egypt}} \v{19}By shrewdly scheming against our people, he oppressed our ancestors and forced them to abandon their infants to the elements, so that they wouldn't live.

\v{20}``At this time Moses was born. He was beautiful in the sight of God, and for three months he was cared for in his father's house. \v{21}When he was placed outside, Pharaoh's daughter adopted him and brought him up as her own son. \v{22}So Moses learned all the wisdom of the Egyptians and became a great man, both in words and in deeds.

\v{23}``When he was 40 years old, he decided\fnote{\fbackref{7:23} Lit. \fbib{old, his heart was moved}} to visit his brothers, the descendants of Israel. \v{24}When he saw one of them being mistreated, he defended him\fnote{\fbackref{7:24} The Gk. lacks \fbib{him}} and avenged the man who was being mistreated by killing the Egyptian. \v{25}He supposed that his brothers would understand that God was using him to rescue them, but they didn't understand. \v{26}The next day, he presented himself to some of them while they were fighting and tried to reconcile them. He said, `Men, you are brothers. Why should you be hurting another?'

\v{27}``But the man who was harming his neighbor pushed Moses\fnote{\fbackref{7:27} Lit. \fbib{him}} away and said, `Who made you ruler and judge over us? \v{28}You don't want to kill me like you killed the Egyptian yesterday, do you?'\fnote{\fbackref{7:28} Cf. Exod 2:14} \v{29}Because of this, Moses fled and lived as a foreigner in the land of Midian. There he had two sons.

\v{30}``After 40 years had passed, an angel appeared to him in the flames of a burning bush in the desert near Mount Sinai. \v{31}When Moses saw it, he was amazed at the sight, and when he approached the bush\fnote{\fbackref{7:31} The Gk. lacks \fbib{the bush}} to look at it, the voice of the Lord said,\fnote{\fbackref{7:31} Lit. \fbib{came}} \v{32}`I am the God of your ancestors---the God of Abraham, Isaac, and Jacob.'\fnote{\fbackref{7:32} Cf. Exod 3:6} Moses became terrified and didn't dare to look. \v{33}Then the Lord told him, `Remove your sandals from your feet, because the place where you are standing is holy ground. \v{34}I have surely seen the oppression of my people in Egypt, I've heard their groans, and I've come down to rescue them. Now come, I'll send you to Egypt.'\fnote{\fbackref{7:34} Cf. Exod 3:4-10}

\v{35}``This same Moses---whom they rejected by saying, `Who made you ruler and judge?'\fnote{\fbackref{7:35} Cf. Exod 2:14}---was the man whom God sent to be both their ruler and deliverer with the help of the angel who had appeared to him in the bush. \v{36}It was he who led them out, performing wonders and signs in Egypt, at the Red Sea, and in the wilderness for 40 years. \v{37}It was this Moses who told the Israelis, `God will raise up a prophet for you from among your own brothers, just as he did\fnote{\fbackref{7:37} The Gk. lacks \fbib{he did}} me.'\fnote{\fbackref{7:37} Cf. Deut 18:15} \v{38}This Moses\fnote{\fbackref{7:38} Lit. \fbib{one}} is the one who was in the assembly in the wilderness with the angel who spoke to him on Mount Sinai and to our ancestors. He received living truths to give to us,\fnote{\fbackref{7:38} Other mss. read \fbib{to you}} \v{39}but our ancestors refused to obey him. Instead, they rejected him and wished to return to Egypt. \v{40}They told Aaron, `Make gods for us who will lead us. This Moses who led us out of the land of Egypt---we don't know what happened to him!'\fnote{\fbackref{7:40} Cf. Exod 32:1, 23}

\v{41}``At that time they even made a calf to be their idol, offered a sacrifice to it, and delighted in what they had made with their hands. \v{42}So God turned away from them and gave them over to worship the heavenly bodies. As it is written in the book of the Prophets:

\begin{poetry}
\poeml `O house of Israel, \\
\poemll    you didn't offer me slaughtered animals and \\
\poemlll       sacrifices those 40 years in the wilderness, did you? \\
\poeml \v{43}You even took along the tent of Moloch, \\
\poemll    the star of your god Rephan, \\
\poeml and the images you made in order to worship them. \\
\poemll    So I will take you into exile as far as Babylon.'\fnote{\fbackref{7:43} Cf. Amos 5:25-27}
\end{poetry}

\v{44}``Our ancestors had the Tent of Testimony\fnote{\fbackref{7:44} I.e. the tent containing the Ark of the Covenant} in the wilderness constructed,\fnote{\fbackref{7:44} The Gk. lacks \fbib{constructed}} just as the one who spoke to Moses directed him to make it according to the pattern he had seen. \v{45}Our ancestors brought it here with Joshua when they replaced the nations that God drove out in front of our ancestors, and it was here until the time of David. \v{46}He found favor with God and asked to design a dwelling for the house\fnote{\fbackref{7:46} Other mss. read \fbib{God}} of Jacob, \v{47}but it was Solomon who built a house for him. \v{48}However, the Most High does not live in buildings made by human\fnote{\fbackref{7:48} The Gk. lacks \fbib{human}} hands. As the prophet says,

\begin{poetry}
\poeml \v{49}```Heaven is my throne, \\
\poemll    and the earth is my footstool. \\
\poeml What kind of house can you build for me,' declares the Lord,\fnote{\fbackref{7:49} MT source citation reads \fbib{\divine{Lord}}} \\
\poemll    ``or what place is there in which I can rest? \\
\poeml \v{50}It was my hand that made all these things, wasn't it?'\,''\fnote{\fbackref{7:50} Cf. Isa 66:1-2}
\end{poetry}

\v{51}``You stubborn people with uncircumcised hearts and ears! You are always opposing the Holy Spirit, just as your ancestors used to do. \v{52}Which of the prophets did your ancestors fail to persecute? They killed those who predicted the coming of the Righteous One, and now you have become his betrayers and murderers. \v{53}You received the Law as ordained by angels, and yet you haven't obeyed it!''
\passage{Stephen is Stoned to Death}

\v{54}While they were listening to these things, they became more and more furious and began to grind their teeth at him. \v{55}But Stephen,\fnote{\fbackref{7:55} Lit. \fbib{he}} filled with the Holy Spirit, looked straight into heaven and saw the glory of God and Jesus standing at the right hand of God. \v{56}He said, ``Look! I see the heavens opened and the Son of Man standing at the right hand of God!''

\v{57}But they shouted out loud, stopped listening, and together they all rushed at him, \v{58}ran him outside of the city, and began to stone him to death. Meanwhile, the witnesses laid their coats at the feet of a young man named Saul. \v{59}As they continued to stone Stephen, he kept praying, ``Lord Jesus, receive my spirit!'' \v{60}Then he knelt down and cried out with a loud voice, ``Lord, don't hold this sin against them!'' After he had said this, he died.\fnote{\fbackref{7:60} Lit. \fbib{fell asleep}}
\labelchapt{8}
\passage{The Church is Scattered}

\chapt{8}
\v{1}Now Saul heartily approved of putting Stephen\fnote{\fbackref{8:1} Lit. \fbib{him}} to death. That day a severe persecution broke out against the church in Jerusalem, and everyone except for the apostles was scattered throughout the countryside of Judea and Samaria. \v{2}Devout men buried Stephen as they mourned loudly for him. \v{3}But Saul kept trying to destroy the church. Going into one house after another, he began dragging off men and women and throwing them in prison.
\passage{Some Samaritans Become Believers}

\v{4}Now those who were scattered went from place to place preaching the word. \v{5}Philip went down to the\fnote{\fbackref{8:5} Other mss. read \fbib{a}} city of Samaria and began to preach the Messiah\fnote{\fbackref{8:5} Or \fbib{Christ}} to the people.\fnote{\fbackref{8:5} Lit. \fbib{to them}} \v{6}The crowds, hearing his message\fnote{\fbackref{8:6} The Gk. lacks \fbib{his message}} and seeing the signs that he was doing, paid close attention to what was said by Philip. \v{7}Unclean spirits screamed with a loud voice as they came out of the many people they had possessed, and many paralyzed and lame people were healed. \v{8}As a result, there was great rejoicing in that city.

\v{9}Now in that city there was a man named Simon. He was practicing occult arts and thrilling the people of Samaria, claiming to be someone great. \v{10}Everyone from the least to the greatest paid close attention to him, saying, ``This is what we call\fnote{\fbackref{8:10} Lit. \fbib{This man is called}} the great power of God!'' \v{11}They paid careful attention to him because he had thrilled them for a long time with his occult performances. \v{12}But when Philip proclaimed the good news about the kingdom of God and about the name of Jesus the Messiah,\fnote{\fbackref{8:12} Or \fbib{Christ}} men and women believed and were baptized. \v{13}Even Simon believed, and after he was baptized he became devoted to Philip. He was amazed to see the signs and great miracles that were happening.

\v{14}Now when the apostles in Jerusalem heard that Samaritans had accepted the word of God, they sent Peter and John to them. \v{15}They went down and prayed for them to receive the Holy Spirit. \v{16}Before this, he had not come on any of them. They had only been baptized in the name of the Lord Jesus. \v{17}Then Peter and John\fnote{\fbackref{8:17} Lit. \fbib{Then they}} laid their hands on them, and they received the Holy Spirit.

\v{18}Now when Simon saw that the Spirit was given through the laying on of the apostles' hands, he offered them money \v{19}and said, ``Give me this power too, so that when I lay my hands on someone, he will receive the Holy Spirit.''

\v{20}But Peter told him, ``May your money perish with you, because you thought you could obtain God's free gift with money! \v{21}You have no part or share in what we're saying, because your heart isn't right with God. \v{22}So repent of this wickedness of yours, and pray to the Lord that, if possible, your heart's intent may be forgiven you. \v{23}For I see that you're being poisoned by bitterness and you're a prisoner of wickedness!''

\v{24}Simon answered, ``Both of you pray\fnote{\fbackref{8:24} Lit. \fbib{Pray} (pl.)} to the Lord for me that none of the things you have said will happen to me.''

\v{25}After they had given their testimony and spoken the word of the Lord, they started back to Jerusalem, continuing to proclaim the good news in many Samaritan villages.
\passage{Philip Tells an Ethiopian about Jesus}

\v{26}Now an angel of the Lord told Philip, ``Get up and go south on the road that leads from Jerusalem to Gaza. This is a deserted road.'' \v{27}So he got up and went. Now there was an Ethiopian eunuch, who was a member of the court of Candace, queen of the Ethiopians. He was in charge of all her treasures and had come up to Jerusalem to worship. \v{28}Now he was returning home, seated in his chariot, and reading from the prophet Isaiah.

\v{29}The Spirit told Philip, ``Approach that chariot and stay near it.'' \v{30}So Philip ran up to it and heard him reading the prophet Isaiah out loud.

Philip\fnote{\fbackref{8:30} Lit. \fbib{He}} asked, ``Do you understand what you're reading?''

\v{31}The man\fnote{\fbackref{8:31} Lit. \fbib{He}} replied, ``How can I unless someone guides me?'' So he invited Philip to get in and sit with him. \v{32}This was the passage of Scripture he was reading:

\begin{poetry}
\poeml ``Like a sheep he was led away to be slaughtered, \\
\poemll    and like a lamb is silent before its shearer, \\
\poemlll       so he does not open his mouth. \\
\poeml \v{33}In his humiliation, justice was denied him. \\
\poemll    Who can describe his descendants?\fnote{\fbackref{8:33} Or \fbib{generation}} \\
\poemlll       For his life is taken away from the earth.''\fnote{\fbackref{8:33} Cf. Isa 53:7-8 (LXX)}
\end{poetry}

\v{34}The eunuch asked Philip, ``I ask you, who is the prophet talking about? Himself? Or someone else?'' \v{35}Then Philip began to speak, and, starting from this Scripture, he told him the good news about Jesus.

\v{36}As they were going along the road, they came to some water. The eunuch said, ``Look, there's some water. What keeps me from being baptized?''\fnote{\fbackref{8:36} Other mss. read \fbib{\v{37}Philip said, ``If you believe with all your heart, you may.'' He replied, ``I believe that Jesus the Messiah is the Son of God.''}} \v{38}So he ordered the chariot to stop, and Philip and the eunuch both went down into the water, and Philip\fnote{\fbackref{8:38} Lit. \fbib{he}} baptized him. \v{39}When they came up out of the water, the Spirit of the Lord snatched Philip away. The eunuch went on his way rejoicing and did not see Philip\fnote{\fbackref{8:39} Lit. \fbib{him}} again. \v{40}But Philip found himself at Azotus. As he was passing through that region,\fnote{\fbackref{8:40} The Gk. lacks \fbib{the region}} he kept proclaiming the good news in all the towns until he came to Caesarea.
\labelchapt{9}
\passage{Saul Becomes a Believer}

\chapt{9}
\v{1}Meanwhile, still spewing death threats against the Lord's disciples, Saul went to the high priest. \v{2}He asked him for letters to take with him\fnote{\fbackref{9:2} The Gk. lacks \fbib{to take with him}} to the synagogues in Damascus, so that if he found any men or women belonging to the Way, he might bring them in chains to Jerusalem. \v{3}As Saul\fnote{\fbackref{9:3} Lit. \fbib{he}} traveled along and was approaching Damascus, a light from heaven suddenly flashed around him. \v{4}He dropped to the ground and heard a voice saying to him, \red{``Saul, Saul! Why are you persecuting me?''}

\v{5}He asked, ``Who are you, Lord?''\fnote{\fbackref{9:5} Or \fbib{Sir}}

The voice\fnote{\fbackref{9:5} Lit. \fbib{He}} said, \red{``I'm Jesus, whom you are persecuting.} \v{6}\red{Now get up, go into the city, and you will be told what you are to do.''}

\v{7}Meanwhile, the men who were traveling with Saul\fnote{\fbackref{9:7} Lit. \fbib{him}} were standing speechless, for they heard the voice but didn't see anyone. \v{8}When Saul got up off the ground, he couldn't see anything, even though his eyes were open. So his companions\fnote{\fbackref{9:8} Lit. \fbib{So they}} took him by the hand and led him into Damascus. \v{9}For three days he couldn't see, and he didn't eat or drink anything.

\v{10}Now in Damascus there was a disciple named Ananias. The Lord called out to him in a vision, \red{``Ananias!''}

He answered, ``Here I am, Lord.''

\v{11}The Lord told him, \red{``Get up, go to the street called Straight, and in the home of Judas look for a man from Tarsus named Saul. At this very moment he's praying.} \v{12}\red{He has seen in a vision\fnote{\fbackref{9:12} Other mss. lack \fbib{in a vision}} a man named Ananias come in and lay his hands on him so he would see again.''}

\v{13}But Ananias answered, ``Lord, I have heard many people tell how much evil this man has done to your saints in Jerusalem. \v{14}He is here with authority from the high priests to put in chains all who call on your name.''

\v{15}But the Lord told him, \red{``Go, because he's my chosen instrument to carry my name to unbelievers,\fnote{\fbackref{9:15} Lit. \fbib{gentiles} ; i.e. unbelieving non-Jews} to their kings, and to the descendants of Israel.} \v{16}\red{since I'm going to show him how much he must suffer for my name's sake.''}
\passage{Saul's Sight is Restored}

\v{17}So Ananias left and went to that house. He laid his hands on Saul\fnote{\fbackref{9:17} Lit. \fbib{on him}} and said, ``Brother Saul, the Lord Jesus, who appeared to you on the road as you were traveling, has sent me so that you may see again and be filled with the Holy Spirit.'' \v{18}All at once something like scales fell from Saul's\fnote{\fbackref{9:18} Lit. \fbib{his}} eyes, and he could see again.

He got up and was baptized, \v{19}and after eating some food, he felt strong again. For several days he stayed with the disciples in Damascus. \v{20}He immediately started to preach about Jesus in the synagogues, saying, ``This is the Son of God.''

\v{21}Everyone who heard him was astonished and said, ``This is the man who harassed those who were calling on Jesus'\fnote{\fbackref{9:21} Lit. \fbib{his}} name in Jerusalem, isn't it? Didn't he come here to bring them in chains to the high priests?'' \v{22}But Saul grew more and more persuasive, and continued to confound the Jews who lived in Damascus by proving that this man was the Messiah.\fnote{\fbackref{9:22} Or \fbib{Christ}}

\v{23}After several days had gone by, the Jewish leaders\fnote{\fbackref{9:23} I.e. Judean leaders; lit. \fbib{the Jews}} plotted to murder Saul,\fnote{\fbackref{9:23} Lit. \fbib{him}} \v{24}but their plot became known to him.\fnote{\fbackref{9:24} Lit. \fbib{Saul}} They were even watching the gates day and night to murder him, \v{25}but his disciples took him one night and let him down through the city wall by lowering him in a basket.

\v{26}When Saul\fnote{\fbackref{9:26} Lit. \fbib{he}} arrived in Jerusalem, he tried to join the disciples, but they all were afraid of him because they wouldn't believe he was a disciple. \v{27}Barnabas, however, introduced Saul\fnote{\fbackref{9:27} Lit. \fbib{him}} to the apostles, telling them how on the road Saul\fnote{\fbackref{9:27} Lit. \fbib{he}} had seen the Lord, who had spoken to him, and how courageously he had spoken in the name of Jesus in Damascus. \v{28}So he freely circulated\fnote{\fbackref{9:28} Lit. \fbib{he went in and out}} among them in Jerusalem, speaking courageously in the name of the Lord. \v{29}He kept talking and arguing with the Hellenistic Jews, but they were bent on murdering him. \v{30}When the brothers found out about the plot,\fnote{\fbackref{9:30} Lit. \fbib{about it}} they took him down to Caesarea and sent him away to Tarsus.

\v{31}So the church throughout Judea, Galilee, and Samaria enjoyed peace. As it continued to be built up and to live in the fear of the Lord, it kept increasing in numbers through the encouragement of the Holy Spirit.
\passage{Aeneas is Healed}

\v{32}Now when Peter was going around among all of the disciples,\fnote{\fbackref{9:32} Lit. \fbib{all of them}} he also visited the saints living in Lydda. \v{33}There he found a man named Aeneas who was paralyzed and had been bedridden for eight years. \v{34}Peter told him, ``Aeneas, Jesus the Messiah\fnote{\fbackref{9:34} Or \fbib{Christ}} is healing you. Get up and put away your mat!'' At once he got up, \v{35}and all the people who lived in Lydda and Sharon saw him and turned to the Lord.
\passage{Tabitha is Healed}

\v{36}In Joppa there was a disciple named Tabitha,\fnote{\fbackref{9:36} \fbib{Tabitha} is Aram. for \fbib{gazelle}.} which in Greek is Dorcas.\fnote{\fbackref{9:36} \fbib{Dorcas} is Gk. for \fbib{gazelle}.} She was known for her good actions and acts of charity that she was always doing. \v{37}At that time, she got sick and died. After they had washed her, they laid her in an upstairs room. \v{38}Since Lydda was near Joppa, the disciples heard that Peter was there and sent two men to him and begged him, ``Come here quickly!'' \v{39}So Peter got up and went with them. When he arrived, they took him upstairs. All the widows gathered around Peter,\fnote{\fbackref{9:39} Lit. \fbib{him}} crying and showing him all the shirts and coats Dorcas had made while she was still with them.

\v{40}Peter made them all go outside. After kneeling down, he prayed, turned to the body, and said, ``Tabitha, get up!'' She opened her eyes, and when she saw Peter, she sat up. \v{41}He extended his hand and helped her get up. Then he called the saints, including the widows, and gave her back to them alive. \v{42}What happened became known throughout Joppa, and many believed in the Lord. \v{43}Meanwhile, Peter\fnote{\fbackref{9:43} Lit. \fbib{he}} stayed in Joppa for several days with Simon, a leatherworker.
\labelchapt{10}
\passage{Cornelius Has a Vision}

\chapt{10}
\v{1}Now in Caesarea there was a man named Cornelius, a centurion\fnote{\fbackref{10:1} A Roman centurion commanded about 100 men.} in what was known as the Italian Regiment. \v{2}He was a devout man who feared God, as did everyone in his home. He gave many gifts to the poor among the people and always prayed to God.

\v{3}One day, about three in the afternoon,\fnote{\fbackref{10:3} Lit. \fbib{About the ninth hour of the day}} he had a vision and clearly saw an angel of God coming to him and saying to him, ``Cornelius!''

\v{4}He stared at the angel\fnote{\fbackref{10:4} Lit. \fbib{at him}} in terror and asked, ``What is it, Lord?''

The angel\fnote{\fbackref{10:4} Lit. \fbib{He}} answered him, ``Your prayers and your gifts to the poor have arisen as a reminder\fnote{\fbackref{10:4} Or \fbib{memorial}} to God. \v{5}Send men now to Joppa and summon Simon, who is called Peter. \v{6}He is a guest of Simon, a leatherworker, whose house is by the sea.''

\v{7}When the angel who had spoken to him had gone, Cornelius\fnote{\fbackref{10:7} Lit. \fbib{he}} summoned two of his household servants and a devout soldier, one of those who served him regularly. \v{8}He explained everything to them and sent them to Joppa.
\passage{Peter Has a Vision}

\v{9}Around noon\fnote{\fbackref{10:9} Lit. \fbib{About the sixth hour}} the next day, while they were on their way and coming close to the town, Peter went up on the roof to pray. \v{10}He became very hungry and wanted to eat, and while the food\fnote{\fbackref{10:10} Lit. \fbib{it}} was being prepared, he fell into a trance \v{11}and saw heaven open and something like a large linen sheet coming down, being lowered by its four corners to the ground. \v{12}In it were all kinds of four-footed animals, reptiles, and birds of the air.

\v{13}Then a voice told him,\fnote{\fbackref{10:13} Lit. \fbib{came to him}} \red{``Get up, Peter! Kill something and eat it.''}

\v{14}But Peter said, ``Absolutely not, Lord, for I have never eaten anything that is common or unclean!''

\v{15}Again the voice came to him a second time, \red{``You must stop calling unclean what God has made clean.''} \v{16}This happened three times. Then the sheet\fnote{\fbackref{10:16} Lit. \fbib{the vessel}} was quickly taken back into heaven.

\v{17}While Peter was still at a loss to know what the vision he had seen could mean, the men sent by Cornelius asked for Simon's house and went to the gate. \v{18}They called out and asked if Simon who was called Peter was staying there. \v{19}Peter was still thinking about the vision when the Spirit told him, ``Look! Three men are looking for you. \v{20}Get up, go downstairs, and don't hesitate to go with them, for I have sent them.''

\v{21}So Peter went to the men and said, ``I'm the man you're looking for. Why are you here?''

\v{22}The men replied, ``Cornelius, a centurion and an upright and God-fearing man who is respected by the whole Jewish nation, was instructed by a holy angel to send for you to come to his home to hear what you have to say.''

\v{23}So Peter\fnote{\fbackref{10:23} Lit. \fbib{he}} welcomed them as his guests. The next day, he got up and went with them, and some of the brothers from Joppa went along with him.
\passage{Peter Speaks with Cornelius}

\v{24}The next day, they arrived in Caesarea. Cornelius was expecting them and had called his relatives and close friends together. \v{25}When Peter was about to enter, Cornelius met him, bowed down at his feet, and began to worship him. \v{26}But Peter made him get up, saying, ``Stand up! I, too, am only a man.''

\v{27}As Peter\fnote{\fbackref{10:27} Lit. \fbib{he}} talked with him, he went in and found that many people had gathered. \v{28}He told them, ``You understand how wrong it is for a Jew to associate or visit with \red{unbelievers.}\fnote{\fbackref{10:28} Lit. \fbib{gentiles} ; i.e. unbelieving non-Jews} But God has shown me that I should stop calling anyone common or unclean, \v{29}and that is why I didn't hesitate when I was sent for. Now may I ask why you sent for me?''

\v{30}Cornelius replied, ``Four days ago at this very hour, three o'clock in the afternoon,\fnote{\fbackref{10:30} Lit. \fbib{the ninth hour}} I was praying in my home. All at once a man in radiant clothes stood in front of me \v{31}and said, `Cornelius, your prayer has been heard. God has remembered your gifts to the poor, \v{32}so send messengers\fnote{\fbackref{10:32} The Gk. lacks \fbib{messengers}} to Joppa and summon Simon, who is called Peter, to come to you. He is a guest in the home of Simon, a leatherworker, by the sea.' \v{33}So I sent for you immediately, and it was good of you to come. All of us are here now in the presence of God to listen to everything the Lord has ordered you to say.''

\v{34}Then Peter began to speak: ``Now I understand that God shows no partiality. \v{35}Indeed, whoever fears him and does what is right is acceptable to him in any nation. \v{36}He has sent his word to the descendants of Israel and brought them the good news of peace through Jesus the Messiah.\fnote{\fbackref{10:36} Or \fbib{Christ}} This man is the Lord of everyone. \v{37}You know what happened throughout Judea, beginning in Galilee after the baptism that John preached. \v{38}God anointed Jesus of Nazareth with the Holy Spirit and with power, and because God was with him, he went around doing good and healing everyone who was oppressed by the devil. \v{39}We are witnesses of everything Jesus\fnote{\fbackref{10:39} Lit. \fbib{he}} did in the land of the Jews, including Jerusalem.

``They hung him on a tree and killed him, \v{40}but God raised him on the third day and allowed him to appear--- \v{41}not to all the people, but to us who were chosen by God to be witnesses and who ate and drank with him after he rose from the dead. \v{42}He also ordered us to preach to the people and to testify solemnly that this is the one appointed by God to be the judge of the living and the dead. \v{43}All the prophets testify to this: everyone who believes in Jesus\fnote{\fbackref{10:43} Lit. \fbib{him}} receives forgiveness of sins through his name.''
\passage{Gentiles Receive the Holy Spirit}

\v{44}While Peter was still making this statement, the Holy Spirit fell on all the people who were listening to his message. \v{45}Then the circumcised believers who had come with Peter were amazed that the gift of the Holy Spirit had been poured out on the gentiles, too, \v{46}because they heard them speaking in foreign languages\fnote{\fbackref{10:46} Or \fbib{in tongues}; the Gk. lacks \fbib{foreign}} and praising God. Then Peter said, \v{47}``No one can stop us from using water to baptize these people who have received the Holy Spirit in the same way that we did, can they?''\fnote{\fbackref{10:47} Lit. \fbib{he}} \v{48}So Peter\fnote{\fbackref{10:48} Lit. \fbib{he}} ordered them to be baptized in the name of Jesus the Messiah.\fnote{\fbackref{0:48} Or \fbib{Christ}} Then they asked him to stay there for several days.
\labelchapt{11}
\passage{Peter Reports to the Church in Jerusalem}

\chapt{11}
\v{1}Now the apostles and the brothers who were in Judea heard that the gentiles had also accepted the word of God. \v{2}But when Peter went up to Jerusalem, those who emphasized circumcision\fnote{\fbackref{11:2} Lit. \fbib{those of the circumcision}} disagreed with him. \v{3}They said, ``You went to uncircumcised men and ate with them!''

\v{4}Then Peter began to explain to them point by point what had happened. He said, \v{5}``I was in the town of Joppa praying when in a trance I saw a vision: Something like a large linen sheet descended down from heaven, lowered by its four corners, and it came right down to me. \v{6}When I examined it closely, I saw four-footed animals of the earth, wild animals, reptiles, and birds of the air. \v{7}I also heard a voice telling me, \red{`Get up, Peter! Kill something and eat it.'} \v{8}But I replied, `Absolutely not, Lord, for nothing common or unclean has ever entered my mouth!' \v{9}Then the voice from heaven answered a second time, \red{`You must stop calling common what God has made clean!'} \v{10}This happened three times. Then everything was pulled back up to heaven.

\v{11}``At that very moment three men arrived at the house where we were staying. They had been sent to me from Caesarea. \v{12}The Spirit told me to go with them without hesitating. These six brothers went with me, too, and we entered the house of the man from Caesarea.\fnote{\fbackref{11:12} The Gk. lacks \fbib{from Caesarea}} \v{13}Then he told us how he had seen an angel standing in his home and saying, `Send messengers\fnote{\fbackref{11:13} The Gk. lacks \fbib{messengers}} to Joppa and summon Simon, who is called Peter. \v{14}He will discuss with you how you and your entire household will be saved.'

\v{15}``When I began to speak, the Holy Spirit fell on them just as he was first given to us. \v{16}Then I remembered what the Lord had said: \red{`John baptized with\fnote{\fbackref{11:16} Or \fbib{in}} water, but you will be baptized with\fnote{\fbackref{11:16} Or \fbib{in}} the Holy Spirit.'} \v{17}Now if God gave them the same gift that he gave us when we believed in the Lord Jesus, the Messiah,\fnote{\fbackref{11:17} Or \fbib{Christ}} who was I to try to stop God?''

\v{18}When they heard this, they calmed down, and praised God by saying, ``So God has given repentance that leads to life even to gentiles.''
\passage{The New Church in Antioch}

\v{19}Now the people who were scattered by the persecution that started because of Stephen went as far as Phoenicia, Cyprus, and Antioch, speaking the word to no one except Jews. \v{20}But among them were some men from Cyprus and Cyrene, who came to Antioch and began proclaiming the Lord Jesus even to the Hellenistic Jews.\fnote{\fbackref{11:20} Other mss. read \fbib{to the Greeks}} \v{21}The hand of the Lord was with them, and a large number of people believed and turned to the Lord.

\v{22}When the church in Jerusalem heard this news, they sent Barnabas all the way to Antioch. \v{23}When he arrived, he rejoiced to see what the grace of God had done,\fnote{\fbackref{11:23} Lit. \fbib{to see the grace of God}} and with hearty determination he kept encouraging all of them to remain faithful to the Lord, \v{24}because he was a good man, full of the Holy Spirit and faith. And so a large number of people was brought to the Lord.

\v{25}Then Barnabas left for Tarsus to look for Saul. \v{26}When he found him, he brought him to Antioch, and for a whole year they were guests of the church and taught many people. It was in Antioch that the disciples were first called Christians.

\v{27}At that time some prophets from Jerusalem came down to Antioch. \v{28}One of them named Agabus got up and predicted by the Spirit that there would be a severe famine all over the world. This happened during the reign of Claudius. \v{29}So all of the disciples decided they would send a contribution to the brothers living in Judea, as they were able, \v{30}by sending it through\fnote{\fbackref{11:30} Lit. \fbib{sending by the hand of}} Barnabas and Saul to the elders.
\labelchapt{12}
\passage{An Angel Frees Peter from Prison}

\chapt{12}
\v{1}About that time, Herod arrested some people who belonged to the church and mistreated them. \v{2}He even had James, the brother of John, killed with a sword. \v{3}When he saw how this was agreeable to the Jews, he proceeded to arrest Peter, too. This happened during the Festival of Unleavened Bread. \v{4}When he arrested Peter, Herod\fnote{\fbackref{12:4} Lit. \fbib{arrested him, he}} put him in prison and turned him over to four squads of soldiers to guard him, planning to bring him out to the people after Passover season.\fnote{\fbackref{12:4} The Gk. lacks \fbib{season}} \v{5}So Peter was kept in prison, but earnest prayer to God for him was being offered by the assembly.\fnote{\fbackref{12:5} Or \fbib{church}}

\v{6}That very night, before Herod was going to bring him out, Peter, bound with two chains, was sleeping between two soldiers, and guards in front of the door were watching the prisoners. \v{7}Suddenly, an angel of the Lord appeared and a light shone in the cell. He tapped Peter on his side, woke him up, and said, ``Get up quickly!'' His chains fell from his wrists. \v{8}Then the angel told him, ``Tuck in your shirt and put on your sandals!'' He did this. Then the angel\fnote{\fbackref{12:8} Lit. \fbib{he}} told him, ``Put on your coat and follow me!'' \v{9}So Peter\fnote{\fbackref{12:9} Lit. \fbib{he}} went out and began to follow him, not realizing that what was being done by the angel was real; he thought he was seeing a vision. \v{10}They passed the first guard, then the second, and came to the iron gate that led into the city. It opened by itself for them, and they went outside and proceeded one block when the angel suddenly left him.

\v{11}Then Peter came to himself and said, ``Now I'm sure that the Lord has sent his angel and rescued me from\fnote{\fbackref{12:11} Lit. \fbib{from the hands of}} Herod and from everything the Jewish people were expecting!''

\v{12}When Peter\fnote{\fbackref{12:12} Lit. \fbib{he}} realized what had happened, he went to the house of Mary, the mother of John who was also called Mark, where a large number of people had gathered and were praying. \v{13}When he knocked at the outer gate, a servant girl named Rhoda came to answer it. \v{14}On recognizing Peter's voice, she was so overjoyed that she didn't open the gate but ran back inside and announced that Peter was standing at the gate. \v{15}The other people\fnote{\fbackref{12:15} Lit. \fbib{They}} told her, ``You're out of your mind!'' But she kept insisting that it was so. Then they said, ``It's his angel.''

\v{16}Meanwhile, Peter kept on knocking and knocking. When they opened the gate, they saw him and were amazed. \v{17}He motioned to them with his hand to be quiet, and then he told them how the Lord had brought him out of the prison. He added, ``Tell this to James and the brothers.'' Then he left and went somewhere else. \v{18}When morning came, there was a great commotion among the soldiers as to what had become of Peter. \v{19}Herod searched for him but didn't find him, so he questioned the guards and ordered them to be executed. Then he left Judea, went down to Caesarea, and stayed there for a while.
\passage{The Death of Herod}

\v{20}Now Herod had been in a violent quarrel with the people of Tyre and Sidon. So they came to him as a group. After they had won over Blastus, who oversaw security\fnote{\fbackref{12:20} The Gk. lacks \fbib{security}} for the king's sleeping quarters, they asked for a peace agreement because their country depended on the king's country for food. \v{21}Therefore, at a set time Herod put on his royal robes, sat down on the royal seat, and made a speech to them. \v{22}The people kept shouting, ``This is the voice of a god, not of a man!'' \v{23}Immediately the angel of the Lord struck him down because he did not give glory to God, and he was eaten by worms and died. \v{24}But the word of God continued to grow and spread.

\v{25}When Barnabas and Saul had fulfilled their mission, they returned from\fnote{\fbackref{12:25} Other mss. read \fbib{to}} Jerusalem, bringing with them John who was also called Mark.
\labelchapt{13}
\passage{Barnabas and Saul Travel to Cyprus}

\chapt{13}
\v{1}Now Barnabas, Simeon called Niger, Lucius from Cyrene, Manaen, who grew up with Herod the tetrarch, and Saul were prophets and teachers in the church at Antioch. \v{2}While they were worshiping the Lord and fasting, the Holy Spirit said, ``Set Barnabas and Saul apart for me to do the work for which I called them.'' \v{3}Then they fasted and prayed, laid their hands on them, and let them go. \v{4}After they had been sent out by the Holy Spirit, they went to Seleucia and from there sailed to Cyprus. \v{5}Arriving in Salamis, they began to preach God's word in the Jewish synagogues. They also had John to help them.

\v{6}They went through the whole island as far as Paphos, where they found a Jewish occult practitioner and false prophet named Bar-jesus. \v{7}He was associated with the proconsul Sergius Paulus, who was an intelligent man. He sent for Barnabas and Saul because he wanted to hear the word of God. \v{8}But Elymas the occult practitioner (that is the meaning of his name) continued to oppose them and tried to turn the proconsul away from the faith. \v{9}But Saul, also known as\fnote{\fbackref{13:9} Lit. \fbib{who was also}} Paul, filled with the Holy Spirit, looked him straight in the eye \v{10}and said, ``You're full of every form of deception and trickery, you son of the devil, you enemy of all that is right! You'll never stop perverting the straight ways of the Lord, will you? \v{11}The\fnote{\fbackref{13:11} Lit. \fbib{The hand of the}} Lord is against you now, and you'll be blind and unable to see the sun for a while!'' At that moment a dark mist came over him, and he went around looking for someone to lead him by the hand. \v{12}When the proconsul saw what had happened, he believed, because he was astonished at the Lord's teaching.
\passage{Paul and Barnabas Go to Antioch in Pisidia}

\v{13}Then Paul and his men set sail from Paphos and arrived in Perga in Pamphylia, where John left them and went back to Jerusalem. \v{14}They left Perga and arrived in Antioch in Pisidia. On the Sabbath day, they went into the synagogue and sat down. \v{15}After the reading of the Law and the Prophets, the synagogue leaders asked them,\fnote{\fbackref{13:15} Lit. \fbib{sent to them}} ``Brothers, if you have any message of encouragement\fnote{\fbackref{13:15} Or \fbib{word of exhortation}} for the people, you may speak.''

\v{16}Then Paul stood up, motioned with his hand, and said:

``Men of Israel and you who fear God, listen! \v{17}The God of this people Israel chose our ancestors and made them a great people during their stay in the land of Egypt, and with a public display of power\fnote{\fbackref{13:17} Lit. \fbib{with an uplifted arm}} he led them out of there. \v{18}After he had put up with\fnote{\fbackref{13:18} Other mss. read \fbib{nourished}} them for 40 years in the wilderness, \v{19}he destroyed seven nations in the land of Canaan. Then God gave their land to the Israelis\fnote{\fbackref{13:19} Lit. \fbib{to them}} as an inheritance \v{20}for about 450 years.

``After that, he gave them judges until the time of the prophet Samuel. \v{21}When they demanded a king, God gave them Kish's son Saul, from the tribe of Benjamin, for 40 years. \v{22}Then God\fnote{\fbackref{13:22} Lit. \fbib{But he}} removed Saul\fnote{\fbackref{13:22} Lit. \fbib{him}} and made David their king, about whom he testified, `I have found that David, the son of Jesse, is a man after my own heart, who will carry out all my wishes.'\fnote{\fbackref{13:22} Cf. Ps 89:20; 1 Sam 13:14} \v{23}It was from this man's descendants that God, as he promised, brought to Israel a Savior, who is Jesus. \v{24}Before Jesus' appearance, John had already preached a baptism of repentance to all the people in Israel. \v{25}When John was finishing his work, he said, `Who do you think I am? I'm not the Messiah.\fnote{\fbackref{13:25} Lit. \fbib{one}} No, but he is coming after me, and I'm not worthy to untie the sandals on his feet.'

\v{26}``My brothers, descendants of Abraham's family, and those among you who fear God, it is to us\fnote{\fbackref{13:26} Other mss. read \fbib{to you}} that the message of this salvation has been sent. \v{27}For the people who live in Jerusalem and their leaders, not knowing who Jesus\fnote{\fbackref{13:27} Lit. \fbib{he}} was, condemned him and so fulfilled the words of the prophets that are read every Sabbath. \v{28}Although they found no reason to sentence him to death, they asked Pilate to have him executed. \v{29}When they had finished doing everything that was written about him, they took him down from the tree and placed him in a tomb. \v{30}But God raised him from the dead, \v{31}and for many days he appeared to those who had come with him to Jerusalem from Galilee. These are now his witnesses to the people. \v{32}We're telling you the good news: What God promised our ancestors \v{33}he has fulfilled for us, their descendants, by raising Jesus. As it is written in the second Psalm, `You are my Son. Today I have become your Father.'\fnote{\fbackref{13:33} Cf. Ps 2:7} \v{34}God\fnote{\fbackref{13:34} Lit. \fbib{He}} raised him from the dead, never to experience decay, as he said, `I'll give you the holy promises made to David.'\fnote{\fbackref{13:34} Cf. Isa 55:3 (LXX)} \v{35}In another Psalm\fnote{\fbackref{13:35} The Gk. lacks \fbib{Psalm}} he says, `You will not let your Holy One experience decay.'\fnote{\fbackref{13:35} Cf. Ps 16:10 (LXX)} \v{36}Now David, after he had served God's purpose in his own generation, died\fnote{\fbackref{13:36} Lit. \fbib{fell asleep}} and was buried with his ancestors, and so he experienced decay. \v{37}However, the man whom God raised did not experience decay.

\v{38}``Therefore, brothers, you must understand that through him the forgiveness of sins is proclaimed to you, \v{39}and that everyone who believes in him is justified and freed from everything that kept you from being justified by the Law of Moses. \v{40}So be careful that what the prophets said doesn't happen to you:

\begin{poetry}
\poeml \v{41}`Look, you mockers! \\
\poemll    Be amazed and die! \\
\poeml Since I am performing an action in your days, \\
\poemll    one that you would not believe \\
\poemlll       even if someone told you!'\,''\fnote{\fbackref{13:41} Cf. Hab 1:5 (LXX)}
\end{poetry}

\v{42}As Paul and Barnabas\fnote{\fbackref{13:42} Lit. \fbib{As they}} were leaving, the people kept urging them to tell them the same things the next Sabbath. \v{43}When the meeting of the synagogue broke up, many Jews and devout converts to Judaism followed Paul and Barnabas, who kept talking to them and urging them to continue in the grace of God.

\v{44}The next Sabbath almost the whole town gathered to hear the word of the Lord.\fnote{\fbackref{13:44} Other mss. read \fbib{of God}} \v{45}But when the Jewish leaders\fnote{\fbackref{13:45} I.e. Judean leaders; lit. \fbib{the Jews}} saw the crowds, they were filled with jealousy and began to object to the statements made by Paul and even to abuse him.

\v{46}Then Paul and Barnabas boldly declared, ``We had to speak God's word to you first, but since you reject it and consider yourselves unworthy of eternal life, we are now going to turn to the gentiles. \v{47}For that is what the Lord ordered us to do: `I have made you a light to the gentiles to be the means of salvation to the very ends of the earth.'\,''\fnote{\fbackref{13:47} Cf. Isa 49:6}

\v{48}When the gentiles heard this, they began rejoicing and glorifying the word of the Lord. Meanwhile, all who had been destined to eternal life believed, \v{49}and the word of the Lord began to spread throughout the whole region. \v{50}But the Jewish leaders\fnote{\fbackref{13:50} I.e. Judean leaders; lit. \fbib{the Jews}} stirred up devout women of high social standing and the officials in the city, started a persecution against Paul and Barnabas, and drove them out of their territory. \v{51}So Paul and Barnabas\fnote{\fbackref{13:51} Lit. \fbib{So they}} shook the dust off their feet in protest against them and went to Iconium. \v{52}Meanwhile, the disciples continued to be full of joy and the Holy Spirit.
\labelchapt{14}
\passage{Paul and Barnabas in Iconium}

\chapt{14}
\v{1}In Iconium, Paul and Barnabas\fnote{\fbackref{14:1} Lit. \fbib{Iconium, they}} went into the Jewish synagogue and spoke in such a way that a great number of both Jews and Greeks believed. \v{2}But the Jews who refused to believe stirred up the gentiles and poisoned their minds against the brothers. \v{3}They stayed there a considerable time and continued to speak boldly for the Lord, who kept affirming his word of grace and granting signs and wonders to be done by them. \v{4}But the people of the city were divided. Some were with the Jews, while others were with the apostles.

\v{5}Now when an attempt was made by both gentiles and Jews, along with their authorities, to mistreat and stone them, \v{6}Paul and Barnabas\fnote{\fbackref{14:6} Lit. \fbib{them}, \fbib{\v{6}they}} found out about it and fled to the Lycaonian towns of Lystra and Derbe and to the surrounding territory. \v{7}There they kept talking about the good news.
\passage{Paul and Barnabas in Lystra}

\v{8}Now in Lystra there was a man sitting down who couldn't use his feet. He had been crippled from birth and had never walked. \v{9}He was listening to Paul as he spoke. Paul\fnote{\fbackref{14:9} Lit. \fbib{He}} watched him closely, and when he saw that he had faith to be healed, \v{10}he said in a loud voice, ``Stand up straight on your feet!'' Then the man\fnote{\fbackref{14:10} Lit. \fbib{he}} jumped up and began to walk.

\v{11}When the crowds saw what Paul had done, they shouted in the Lycaonian language, ``The gods have become like men and have come down to us!'' \v{12}They began to call Barnabas Zeus, and Paul Hermes, because he was the main speaker. \v{13}The priest of the temple of Zeus, which was just outside the city, brought bulls and garlands to the gates. He and the crowds wanted to offer sacrifices.

\v{14}But when the apostles Barnabas and Paul heard of it, they tore their clothes and rushed out into the crowd, shouting, \v{15}``Men, why are you doing this? We are merely human beings with natures like yours. We are telling you the good news so you'll turn from these worthless things to the living God, who made heaven and earth, the sea, and everything in them.\fnote{\fbackref{14:15} Cf. Ex. 20:11; Psa 146:6} \v{16}In past generations he allowed all the nations to go their own ways, \v{17}yet he has not abandoned his witness: he continues to do good, to give you rain from heaven, to give you\fnote{\fbackref{14:17} The Gk. lacks \fbib{to give you}} fruitful seasons, and to fill you with food and your hearts with joy.'' \v{18}Even by saying this, it was all Paul and Barnabas\fnote{\fbackref{14:18} Lit. \fbib{all they}} could do to keep the crowds from offering sacrifices to them.
\passage{Paul and Barnabas Return to Antioch in Syria}

\v{19}But some Jews came from Antioch and Iconium and won over the crowds by persuasion. They stoned Paul and dragged him out of the town, thinking he was dead. \v{20}But the disciples formed a circle around him, and he got up and went back to town. The next day, he went on with Barnabas to Derbe.

\v{21}As they were proclaiming the good news in that city, they discipled a large number of people. Then they went back to Lystra, Iconium, and Antioch, \v{22}strengthening the disciples and encouraging them to continue in the faith. ``We must endure many hardships,'' they said, ``to get into the kingdom of God.'' \v{23}Paul and Barnabas\fnote{\fbackref{14:23} Lit. \fbib{They}} appointed elders for them in each church, and with prayer and fasting they entrusted them to the Lord in whom they had believed. \v{24}Then they passed through Pisidia and came to Pamphylia. \v{25}They spoke the word\fnote{\fbackref{14:25} Other mss. read \fbib{the word of the Lord}; still other mss. read \fbib{the word of God}} in Perga and went down to Attalia. \v{26}From there they sailed back to Antioch, where they had been entrusted to the grace of God for the work they had completed. \v{27}When they arrived, they called the church together and told them everything that God had done with them and how he had opened a door so that gentiles would believe. \v{28}Then they spent a long time with the disciples.
\labelchapt{15}
\passage{Controversy about the Law}

\chapt{15}
\v{1}Then some men came down from Judea and started to teach the brothers, ``Unless you are circumcised according to the Law of Moses, you can't be saved.'' \v{2}Paul and Barnabas had quite a dispute and argument with them. So Paul and Barnabas and some of the others were appointed to go up to Jerusalem to confer with the apostles and elders about this question. \v{3}They were sent on their way by the church, and as they were going through Phoenicia and Samaria they told of the conversion of the gentiles and brought great joy to all the brothers. \v{4}When they arrived in Jerusalem, they were welcomed by the church, the apostles, and the elders, and they reported everything that God had done through them. \v{5}But some believers from the party of the Pharisees stood up and said, ``The gentiles\fnote{\fbackref{15:5} Lit. \fbib{They}} must be circumcised and ordered to keep the Law of Moses.''

\v{6}So the apostles and the elders met to look into this claim. \v{7}After a lengthy debate, Peter stood up and told them, ``Brothers, you know that in the early days, God chose me to be the one among you through whom the gentiles would hear the message of the gospel and believe. \v{8}God, who knows everyone's heart, showed them he approved by giving them the Holy Spirit, just as he did to us. \v{9}He made no distinction between them and us, because of their faith-cleansed hearts. \v{10}So why do you test God by putting on the disciples' neck a yoke that neither our ancestors nor we could carry? \v{11}We certainly believe that it is through the grace of the Lord Jesus, the Messiah,\fnote{\fbackref{15:11} Or \fbib{Christ}} that we are saved, just as they are.''

\v{12}The whole crowd was silent as they listened to Barnabas and Paul tell about all the signs and wonders that God had done through them among the gentiles. \v{13}After Paul and Barnabas\fnote{\fbackref{15:13} Lit. \fbib{After they}} had finished speaking, James responded, ``Brothers, listen to me: \v{14}Simeon\fnote{\fbackref{15:14} I.e. Simon Peter} has explained how God first showed his concern for the gentiles by taking from among them a people for his name. \v{15}This agrees with the words of the prophets. As it is written,

\begin{poetry}
\poeml \v{16}```After this, I will come back \\
\poemll    and set up David's fallen tent again. \\
\poeml I will restore its ruined places \\
\poemll    and set it up again \\
\poeml \v{17}so that the rest of the people may search for the Lord, \\
\poemll    including all the gentiles who are called by my name,' \\
\poemlll       declares the Lord.\fnote{\fbackref{15:17} MT citation source reads \fbib{\divine{Lord}}} \\
\poeml `He is the one who has been doing these things \\
\poemll    \v{18}that have been known from long ago.'\fnote{\fbackref{15:18} Cf. Amos 9:11-12; Isa 45:21}
\end{poetry}

\v{19}``Therefore, I have decided that we should not trouble these gentiles who are turning to God. \v{20}Instead, we should write to them to keep away from things polluted by idols, from sexual immorality, from anything strangled,\fnote{\fbackref{15:20} Other mss. lack \fbib{from anything strangled}} and from blood.\fnote{\fbackref{15:20} I.e. uncooked meat} \v{21}After all, Moses has had people to proclaim him in every city for generations, and on every Sabbath his books are\fnote{\fbackref{15:21} Lit. \fbib{Sabbath he is}} read aloud in the synagogues.''
\passage{The Reply of the Church}

\v{22}Then the apostles, the elders, and the whole church decided to choose some of their men to send with Paul and Barnabas to Antioch. These were Judas, who was called Barsabbas, and Silas, who were leaders among the brothers. \v{23}They wrote this letter for them to deliver:\fnote{\fbackref{15:23} Lit. \fbib{They wrote through their hand}}

``From:\fnote{\fbackref{15:23} The Gk. lacks \fbib{From}} The apostles and the elders, your brothers

To: Their gentile brothers in Antioch, Syria, and Cilicia.

Greetings. \v{24}We have heard that some men, coming from us without instructions from us, have said things to trouble you and have unsettled you.\fnote{\fbackref{15:24} Other mss. read \fbib{you by saying, `You must be circumcised and keep the law.'}} \v{25}So we have unanimously decided to choose men and send them to you with our dear Barnabas and Paul, \v{26}who have risked their lives for the sake of our Lord Jesus, the Messiah.\fnote{\fbackref{15:26} Or \fbib{Christ}} \v{27}We have therefore sent Judas and Silas to tell you the same things by word of mouth. \v{28}For it seemed good to the Holy Spirit and to us not to place on you any burden but these essential requirements: \v{29}to keep away from food sacrificed to idols, from blood,\fnote{\fbackref{15:29} I.e. uncooked meat} from anything strangled, and from sexual immorality. If you avoid these things, you will do well. Goodbye.''

\v{30}So the men were sent on their way and arrived in Antioch. They gathered the congregation together and delivered the letter. \v{31}When the people\fnote{\fbackref{15:31} Lit. \fbib{they}} read it, they were pleased with how the letter encouraged them. \v{32}Then Judas and Silas, who were also prophets, said a lot to encourage and strengthen the brothers. \v{33}After staying there for some time, they were sent back with a greeting\fnote{\fbackref{15:33} Lit. \fbib{sent back with peace}} from the brothers to those who had sent them.\fnote{\fbackref{15:33} Other mss. read \fbib{sent them.} \fbib{\v{34}But it seemed good to Silas to remain there, and Judas went back alone.}} \v{35}Both Paul and Barnabas remained in Antioch to teach and proclaim the word of the Lord, as did many others.
\passage{Paul and Barnabas Disagree}

\v{36}A few days later, Paul told Barnabas, ``Let's go back and visit the brothers in every town where we proclaimed the word of the Lord and see how they're doing.'' \v{37}Barnabas wanted to take along John, who was called Mark, \v{38}but Paul did not think it was right to take along the man who had deserted them in Pamphylia and who had not gone with them into the work. \v{39}The disagreement was so sharp that they parted ways. Barnabas took Mark and sailed to Cyprus, \v{40}while Paul chose Silas and left after the brothers had entrusted him to the grace of the Lord.\fnote{\fbackref{15:40} Other mss. read \fbib{of God}} \v{41}He went through Syria and Cilicia and strengthened the churches.
\labelchapt{16}
\passage{Timothy Joins Paul in Lystra}

\chapt{16}
\v{1}Paul\fnote{\fbackref{16:1} Lit. \fbib{He}} also went to Derbe and Lystra, where there was a disciple named Timothy, the son of a believing Jewish wife whose husband was a Greek. \v{2}Timothy\fnote{\fbackref{16:2} Lit. \fbib{He}} was highly regarded by the brothers in Lystra and Iconium. \v{3}Paul wanted this man to go with him, so he took him and had him circumcised because of the Jews who lived in that region, since everyone knew that Timothy's\fnote{\fbackref{16:3} Lit. \fbib{that his}} father was a Greek. \v{4}As they went from town to town, they delivered the decisions reached by the apostles and elders in Jerusalem for them to obey. \v{5}So the churches continued to be strengthened in the faith and to increase in numbers every day.
\passage{Paul Has a Vision}

\v{6}Because they had been prevented by the Holy Spirit from speaking the word in Asia, Paul and Timothy\fnote{\fbackref{16:6} Lit. \fbib{Then they}} went through the region of Phrygia and Galatia. \v{7}They went as far as Mysia and tried to enter Bithynia, but the Spirit of Jesus did not permit them, \v{8}so they bypassed Mysia and went down to Troas. \v{9}During the night Paul had a vision. A man from Macedonia was standing there and pleading with him, ``Come over to Macedonia and help us!'' \v{10}As soon as he had seen the vision, we immediately looked for a way to go to Macedonia, because we were convinced that God had called us to tell the people there\fnote{\fbackref{16:10} Lit. \fbib{tell them}} the good news.
\passage{Paul and Silas in Philippi}

\v{11}Sailing from Troas, we went straight to Samothrace, the next day to Neapolis, \v{12}and from there to Philippi, an important city of the district\fnote{\fbackref{16:12} Other mss. read \fbib{a city of the first district}} of Macedonia and a Roman\fnote{\fbackref{16:12} The Gk. lacks \fbib{Roman}} colony. We were in this city for several days. \v{13}On the Sabbath day, we went out the city gate and walked\fnote{\fbackref{16:13} the Gk. lacks \fbib{walked}} along the river, where we thought there was a place of prayer. We sat down and began talking to the women who had gathered there. \v{14}A woman named Lydia, from the city of Thyatira, a dealer in purple goods, was listening to us. She was a worshiper of God, and the Lord opened her heart to listen carefully to what was being said by Paul. \v{15}When she and her family were baptized, she urged us, ``If you are convinced that I am a believer in the Lord, come and stay at my home.'' And she continued to insist that we do so.
\passage{The Fortune Teller}

\v{16}Once, as we were going to the place of prayer, we met a slave girl who had a spirit of fortune-telling and who had brought her owners a great deal of money by predicting the future. \v{17}She would follow Paul and us and shout, ``These men are servants of the Most High God and are proclaiming to you\fnote{\fbackref{16:17} Other mss. read \fbib{us}} a way of salvation!''

\v{18}She kept doing this for many days until Paul became annoyed, turned to her\fnote{\fbackref{16:18} The Gk. lacks \fbib{to her}} and told the spirit, ``I command you in the name of Jesus the Messiah\fnote{\fbackref{16:18} Or \fbib{Christ}} to come out of her!'' And it came out that very moment.\fnote{\fbackref{16:18} Lit. \fbib{that hour}}

\v{19}When her owners realized that their hope of making money was gone, they grabbed Paul and Silas and dragged them before the authorities who met together in the public square.\fnote{\fbackref{16:19} Or \fbib{in the marketplace}} \v{20}They brought them before the magistrates and said, ``These men are stirring up a lot of trouble in our city. They are Jews \v{21}and are advocating customs that we're not allowed to accept or practice as Romans.''

\v{22}The crowd joined in the attack against them. Then the magistrates had Paul and Silas\fnote{\fbackref{16:22} Lit. \fbib{had them}} stripped of their clothes and ordered them beaten with rods. \v{23}After giving them a severe beating, they threw them in jail and ordered the jailer to keep them under tight security. \v{24}Having received these orders, he put them into the inner cell and fastened their feet in leg irons.

\v{25}Around midnight, Paul and Silas were praying and singing hymns to God, and the other prisoners were listening to them. \v{26}Suddenly, there was an earthquake so violent that the foundations of the prison were shaken. All the doors immediately flew open, and everyone's chains were unfastened.

\v{27}When the jailer woke up and saw the prison doors wide open, he drew his sword and was about to kill himself, since he thought the prisoners had escaped. \v{28}But Paul shouted in a loud voice, ``Don't hurt yourself, because we are all here!''

\v{29}The jailer\fnote{\fbackref{16:29} Lit. \fbib{He}} asked for torches and rushed inside. Trembling as he knelt in front of Paul and Silas, \v{30}he took them outside and asked, ``Sirs, what must I do to be saved?''

\v{31}They answered, ``Believe on the Lord Jesus, and you and your family will be saved.'' \v{32}Then they spoke the word of the Lord\fnote{\fbackref{16:32} Other mss. read \fbib{of God}} to him and everyone in his home.

\v{33}At that hour of the night, he took them and washed their wounds. Then he and his entire family were baptized immediately. \v{34}He brought Paul and Silas\fnote{\fbackref{16:34} Lit. \fbib{brought them}} upstairs into his house and set food before them. He was thrilled, as was his household, to believe in God.

\v{35}When day came, the magistrates sent guards, who commanded, ``Release those men.''

\v{36}The jailer reported these words to Paul, and added, ``The magistrates have sent word to release you. So come out now and go in peace.''

\v{37}But Paul told the guards,\fnote{\fbackref{16:37} Lit. \fbib{told them}} ``The magistrates\fnote{\fbackref{16:37} Lit. \fbib{They}} have had us beaten publicly without a trial and have thrown us into jail, even though we are Roman citizens. Now are they going to throw us out secretly? Certainly not! Have them come and escort us out.''

\v{38}The guards reported these words to the magistrates, and they became afraid when they heard that Paul and Silas\fnote{\fbackref{16:38} Lit. \fbib{that they}} were Roman citizens. \v{39}So the magistrates\fnote{\fbackref{16:39} Lit. \fbib{So they}} came, apologized to them, and escorted them out. Then they asked them to leave the city. \v{40}Leaving the jail, Paul and Silas\fnote{\fbackref{16:40} Lit. \fbib{jail, they}} went to Lydia's house. They saw the brothers, encouraged them, and then left.
\labelchapt{17}
\passage{Paul and Silas in Thessalonica}

\chapt{17}
\v{1}Paul and Silas\fnote{\fbackref{17:1} Lit. \fbib{They}} traveled through Amphipolis and Apollonia and came to Thessalonica, where there was a Jewish synagogue. \v{2}As usual, Paul entered there and on three Sabbaths discussed the Scriptures with them. \v{3}He explained and showed them that the Messiah\fnote{\fbackref{17:3} Or \fbib{Christ}} had to suffer and rise from the dead: ``This very Jesus whom I proclaim to you is the Messiah.''\fnote{\fbackref{17:3} Or \fbib{Christ}}

\v{4}Some of them were persuaded and began to be associated with Paul and Silas, especially a large crowd of devout Greeks and the wives of many prominent men. \v{5}But the Jewish leaders\fnote{\fbackref{17:5} I.e. Judean leaders; lit. \fbib{the Jews}} became jealous, and they took some contemptible characters who used to hang out in the public square,\fnote{\fbackref{17:5} Or \fbib{in the marketplace}} formed a mob, and started a riot in the city. They attacked Jason's home and searched it for Paul and Silas in order to bring them out to the people. \v{6}When they didn't find them, they dragged Jason and some other brothers before the city officials and shouted, ``These fellows who have turned the world upside down have come here, too, \v{7}and Jason has welcomed them as his guests. All of them oppose the emperor's decrees by saying that there is another king---Jesus!''

\v{8}The crowd and the city officials were upset when they heard this, \v{9}but after they had gotten a bond from Jason and the others, they let them go.
\passage{Paul and Silas in Berea}

\v{10}That night the brothers immediately sent Paul and Silas away to Berea. When they arrived, they went into the Jewish synagogue. \v{11}These people were more receptive than those in Thessalonica. They were very willing to receive the message, and every day they carefully examined the Scriptures to see if those things were so. \v{12}Many of them believed, including a large number of prominent Greek women and men.

\v{13}But when the Jewish leaders\fnote{\fbackref{17:13} I.e. Judean leaders; lit. \fbib{the Jews}} in Thessalonica found out that the word of God had been proclaimed by Paul also in Berea, they went there to upset and incite the crowds. \v{14}Then the brothers immediately sent Paul away to the coast, but Silas and Timothy stayed there.
\passage{Paul in Athens}

\v{15}The men who escorted Paul took him all the way to Athens and, after receiving instructions to have Silas and Timothy join him as soon as possible, they left. \v{16}While Paul was waiting for them in Athens, his spirit was deeply disturbed to see the city full of idols. \v{17}So he began holding discussions in the synagogue with the Jews and other worshipers, as well as every day in the public square\fnote{\fbackref{17:17} Or \fbib{in the marketplace}} with anyone who happened to be there. \v{18}Some Epicurean and Stoic philosophers also debated with him. Some asked, ``What is this blabbermouth trying to say?'' while others said, ``He seems to be preaching about foreign gods.'' This was because Paul\fnote{\fbackref{17:18} Lit. \fbib{because he}} was telling the good news about Jesus and the resurrection.

\v{19}Then they took him, brought him before the Areopagus,\fnote{\fbackref{17:19} I.e. the city council} and asked, ``May we know what this new teaching of yours is? \v{20}It sounds rather strange to our ears, and we would like to know what it means.'' \v{21}Now all the Athenians and the foreigners living there used to spend their time doing nothing else other than listening to the latest ideas or repeating them.

\v{22}So Paul stood up in front of the Areopagus\fnote{\fbackref{17:22} I.e. the city council} and said, ``Men of Athens, I see that you are very religious in every way. \v{23}For as I was walking around and looking closely at the objects you worship, I even found an altar with this written on it: `To an unknown god.' So I am telling you about the unknown object you worship. \v{24}The God who made the world and everything in it is the Lord of heaven and earth. He doesn't live in shrines made by human hands, \v{25}and he isn't served by people\fnote{\fbackref{17:25} Lit. \fbib{hands}} as if he needed anything. He himself gives everyone life, breath, and everything else. \v{26}From one man\fnote{\fbackref{17:26} Other mss. read \fbib{From one blood}} he made every nation of humanity to live all over the earth, fixing the seasons of the year and the national boundaries within which they live, \v{27}so that they might look for God,\fnote{\fbackref{17:27} Other mss. read \fbib{for the Lord}} somehow reach for him, and find him. Of course, he is never far from any one of us. \v{28}For we live, move, and exist because of him, as some of your own poets have said: `{\ldots}Since we are his children, too.'\fnote{\fbackref{17:28} \fbib{Phainomena} (5) by Aratus, a poet of Sicilian origin (3\textsuperscript{rd} century BC). Cleanthes the Stoic (3\textsuperscript{rd} century BC) used almost identical language.} \v{29}So if we are God's children, we shouldn't think that the divine being is like gold, silver, or stone, or is an image carved by humans using their own imagination and skill. \v{30}Though God has overlooked those times of ignorance, he now commands everyone everywhere to repent, \v{31}because he has set a day when he is going to judge the world with justice\fnote{\fbackref{17:31} Or \fbib{in righteousness}} through a man whom he has appointed. He has given proof of this to everyone by raising him from the dead.''

\v{32}When they heard about a resurrection of the dead, some began joking about it, while others said, ``We will hear you again about this.'' \v{33}And so Paul left the meeting.\fnote{\fbackref{17:33} Lit. \fbib{went out from the middle of them}} \v{34}Some men joined him and became believers. With them were Dionysius, who was a member of the Areopagus,\fnote{\fbackref{17:34} I.e. the city council} a woman named Damaris, and some others along with them.
\labelchapt{18}
\passage{Paul in Corinth}

\chapt{18}
\v{1}After this, Paul\fnote{\fbackref{18:1} Lit. \fbib{he}} left Athens and went to Corinth. \v{2}There he found a Jew named Aquila, a native of Pontus, who had recently come from Italy with his wife Priscilla because Claudius had ordered all the Jews to leave Rome. Paul\fnote{\fbackref{18:2} Lit. \fbib{He}} went to visit them, \v{3}and because they had the same trade he stayed with them. They worked together because they were tentmakers by trade. \v{4}Every Sabbath, he would speak in the synagogue, trying to persuade both Jews and Greeks. \v{5}But when Silas and Timothy arrived from Macedonia, Paul devoted himself entirely to the word\fnote{\fbackref{18:5} Other mss. read \fbib{Spirit}} as he emphatically assured the Jews that Jesus is the Messiah.\fnote{\fbackref{18:5} Or \fbib{Christ}} \v{6}But when they began to oppose him and insult him, he shook out his clothes in protest and told them, ``Your blood be on your own heads! I am innocent. From now on I will go to the gentiles.''

\v{7}Then he left that place and went to the home of a man named Titius\fnote{\fbackref{18:7} Other mss. read \fbib{Titus}} Justus, who worshipped God and whose house was next door to the synagogue. \v{8}Now Crispus, the leader of the synagogue, believed in the Lord, along with his whole family. Many Corinthians who heard Paul also believed and were baptized.

\v{9}One night, the Lord told Paul in a vision, \red{``Stop being afraid to speak out! Don't remain silent!} \v{10}\red{For I am with you, and no one will lay a hand on you or harm you, because I have many people in this city.''} \v{11}So Paul\fnote{\fbackref{18:11} Lit. \fbib{he}} lived there for a year and a half and continued to teach the word of God among the people there.\fnote{\fbackref{18:11} Lit. \fbib{among them}}

\v{12}While Gallio was proconsul of Achaia, the Jewish leaders\fnote{\fbackref{18:12} I.e. Judean leaders; lit. \fbib{the Jews}} gathered together, attacked Paul, and brought him before the judge's seat. \v{13}They said, ``This man is persuading people to worship God in ways that are contrary to the Law.''

\v{14}Paul was about to speak when Gallio admonished the Jewish leaders,\fnote{\fbackref{18:14} I.e. Judean leaders; lit. \fbib{the Jews}} ``If there were some misdemeanor or crime involved, it would be reasonable to put up with you Jews. \v{15}But since it is a question about words, names, and your own Law, you will have to take care of that yourselves. I refuse to be a judge in these matters.'' \v{16}So he drove them away from the judge's seat. \v{17}Then all of them\fnote{\fbackref{18:17} Other mss. read \fbib{of the Greeks}} took Sosthenes, the synagogue leader, and began beating him in front of the judge's seat. But Gallio paid no attention to any of this.
\passage{Paul's Return Trip to Antioch}

\v{18}After staying there for quite a while longer, Paul said goodbye to the brothers and sailed for Syria, accompanied by Priscilla and Aquila. He had his hair cut in Cenchrea, since he was under a vow. \v{19}When they arrived in Ephesus, he left Priscilla and Aquila\fnote{\fbackref{18:19} Lit. \fbib{left them}} there. Then he went into the synagogue and had a discussion with the Jews. \v{20}They asked him to stay longer, but he refused. \v{21}As he told them goodbye, he said, ``I will come back\fnote{\fbackref{18:21} Other mss. read \fbib{I must at all costs keep the approaching festival in Jerusalem, but I will come back}} to you again if it is God's will.'' Then he set sail from Ephesus. \v{22}When he arrived in Caesarea, he went up to Jerusalem,\fnote{\fbackref{18:22} The Gk. lacks \fbib{to Jerusalem}} greeted the church there, and then returned to Antioch. \v{23}After spending some time there, he departed and went from place to place through the region of Galatia and Phrygia, strengthening all the disciples.
\passage{Apollos Preaches in Ephesus}

\v{24}Meanwhile, a Jew named Apollos arrived in Ephesus. He was a native of Alexandria, an eloquent man, and well versed in the Scriptures. \v{25}He had been instructed in the Lord's way, and with spiritual fervor he kept speaking and teaching accurately about Jesus, although he knew only about John's baptism. \v{26}He began to speak boldly in the synagogue, but when Priscilla and Aquila heard him, they took him home and explained God's way to him more accurately. \v{27}When Apollos\fnote{\fbackref{18:27} Lit. \fbib{he}} wanted to cross over to Achaia, the brothers wrote to the disciples there, urging them to welcome him. On his arrival he greatly helped those who, through God's\fnote{\fbackref{18:27} The Gk. lacks \fbib{God's}} grace, had believed. \v{28}He successfully refuted the Jews in public and proved by the Scriptures that Jesus is the Messiah.\fnote{\fbackref{18:28} Or \fbib{Christ}}
\labelchapt{19}
\passage{Paul in Ephesus}

\chapt{19}
\v{1}It was while Apollos was in Corinth that Paul passed through the inland districts and came to Ephesus. He found a few disciples there \v{2}and asked them, ``Did you receive the Holy Spirit when you believed?''

They answered him, ``No, we haven't even heard that there is a Holy Spirit.''

\v{3}He then asked, ``Then into what were you baptized?''

They answered, ``Into John's baptism.''

\v{4}Then Paul said, ``John baptized when they repented, telling the people to believe in the one who was to come after him, that is, in Jesus.'' \v{5}On hearing this, they were baptized in the name of the Lord Jesus. \v{6}When Paul laid his hands on them, the Holy Spirit came on them, and they began to speak in foreign languages\fnote{\fbackref{19:6} Or \fbib{tongues}; the Gk. lacks \fbib{foreign}} and to prophesy. \v{7}There were about twelve men in all.

\v{8}He went into the synagogue and spoke there boldly for three months, holding discussions and persuading those who heard him\fnote{\fbackref{19:8} Lit. \fbib{persuading them}} about the kingdom of God. \v{9}But when some people became stubborn, refused to believe, and slandered the Way in front of the people, Paul\fnote{\fbackref{19:9} Lit. \fbib{he}} left them, taking his disciples away with him, and held daily discussions in the lecture hall of Tyrannus.\fnote{\fbackref{19:9} Other mss. read \fbib{of a certain Tyrannus from the fifth hour to the tenth}} \v{10}This went on for two years, so that all who lived in Asia, Jews and Greeks alike, heard the word of the Lord. \v{11}God continued to do extraordinary miracles through Paul.\fnote{\fbackref{19:11} Lit. \fbib{through Paul's hands}} \v{12}When handkerchiefs and aprons that had touched his skin were taken to the sick, their diseases left them and evil spirits went out of them.

\v{13}Then some Jews who went around trying to drive out demons attempted to use the name of the Lord Jesus on those who had evil spirits, saying, ``I command you by that Jesus whom Paul preaches!'' \v{14}Seven sons of a Jewish high priest named Sceva were doing this.

\v{15}But the evil spirit told them, ``Jesus I know, and I am getting acquainted with Paul, but who are you?''

\v{16}Then the man with the evil spirit jumped on them, got the better of them, and so violently overpowered all of them that they fled out of the house naked and bruised. \v{17}When this became known to everyone living in Ephesus, Jews and Greeks alike, they all became terrified, and the name of the Lord Jesus began to be held in high honor. \v{18}Many who became believers kept coming to confess and talk about what they had been doing. \v{19}Moreover, many people who had practiced occult arts gathered their books and burned them in front of everybody. They estimated their value and found them to have been worth 50,000 silver coins.\fnote{\fbackref{19:19} The denomination of coin is unspecified} \v{20}In that way the word of the Lord kept spreading and triumphing.

\v{21}After these things had happened, Paul decided\fnote{\fbackref{19:21} Or \fbib{Paul resolved in the Spirit}} to go through Macedonia and Achaia and then to go on to Jerusalem. ``After I have gone there,'' he told them, ``I must also see Rome.'' \v{22}Then he sent two of his helpers, Timothy and Erastus, to Macedonia, while he himself stayed in Asia a while longer.
\passage{A Riot in Ephesus}

\v{23}Now about that time a great commotion broke out concerning the Way. \v{24}By making silver shrines of Artemis, a silversmith named Demetrius provided a large income for skilled workers. \v{25}He called a meeting of these men and others who were engaged in similar trades and said, ``Men, you well know that we get a good income from this business. \v{26}You also see and hear that, not only in Ephesus, but almost all over Asia, this man Paul has won over and taken away a large crowd by telling them that gods made by human\fnote{\fbackref{19:26} The Gk. lacks \fbib{human}} hands are not gods at all. \v{27}There is a danger not only that our business will lose its reputation but also that the temple of the great goddess Artemis will be brought into disrepute and that she will be robbed of her majesty that brought all Asia and the world to worship her.''

\v{28}When they heard this, they became furious and began to shout, ``Great is Artemis of the Ephesians!'' \v{29}The city was filled with confusion, and the people\fnote{\fbackref{19:29} Lit. \fbib{they}} rushed into the theater together, dragging with them Gaius and Aristarchus, Paul's fellow travelers from Macedonia. \v{30}Paul wanted to go into the crowd, but the disciples wouldn't let him. \v{31}Even some officials of the province of Asia who were his friends sent him a message urging him not to risk his life in the theater.

\v{32}Meanwhile, some were shouting one thing and some another, since the crowd was confused, and most of them didn't know why they were meeting. \v{33}Some of the crowd concluded it was because of Alexander, since the Jews had pushed him to the front. So Alexander motioned for silence and tried to make a defense before the people. \v{34}But when they found out that he was a Jew, they all started to shout in unison for about two hours, ``Great is Artemis of the Ephesians!''

\v{35}When the city recorder had quieted the crowd, he said, ``Men of Ephesus, who in the world\fnote{\fbackref{19:35} Lit. \fbib{who among people}} doesn't know that this city of Ephesus is the keeper of the temple of the great Artemis and of the statue that fell down from heaven?\fnote{\fbackref{19:35} Or \fbib{from Zeus}} \v{36}Since these things cannot be denied, you must be quiet and not do anything reckless. \v{37}For you have brought these men here, although they neither rob temples nor blaspheme our\fnote{\fbackref{19:37} Other mss. read \fbib{your}} goddess. \v{38}So if Demetrius and his workers have a charge against anyone, the courts are open and there are proconsuls. They should accuse one another there. \v{39}But if you want anything else, it must be settled in the regular assembly, \v{40}because we are in danger of being charged with rioting today, and there is no good reason we can give to justify this commotion.'' \v{41}After saying this, he dismissed the assembly.
\labelchapt{20}
\passage{Paul's Trip to Macedonia and Greece}

\chapt{20}
\v{1}When the uproar was over, Paul sent for the disciples and encouraged them. Then he said goodbye to them and left to go to Macedonia. \v{2}He went through those regions and encouraged the people\fnote{\fbackref{20:2} Lit. \fbib{them}} with everything he had to say. Then he went to Greece \v{3}and stayed there for three months. When he was about to sail for Syria, a plot was initiated against him by the Jews, so he decided to go back through Macedonia. \v{4}He was accompanied by Sopater (the son of Pyrrhus) from Berea, Aristarchus and Secundus from Thessalonica, Gaius from Derbe, Timothy, and Tychicus and Trophimus from Asia. \v{5}These men went on ahead and were waiting for us in Troas. \v{6}After the Festival\fnote{\fbackref{20:6} Lit. \fbib{days}} of Unleavened Bread, we sailed from Philippi, and days later we joined them in Troas and stayed there for seven days.
\passage{Paul's Farewell Visit to Troas}

\v{7}On the first day of the week, when we had met to break bread, Paul began to address the people.\fnote{\fbackref{20:7} Lit. \fbib{them}} Since he intended to leave the next day, he went on speaking until midnight. \v{8}Now there were many lamps in the upstairs room where we were meeting. \v{9}A young man named Eutychus, who was sitting in a window, began to sink off into a deep sleep as Paul kept speaking longer and longer. Overcome by sleep, he fell down from the third floor and was picked up dead. \v{10}But Paul went down, bent over\fnote{\fbackref{20:10} Lit. \fbib{fell on}} him, took him into his arms, and said, ``Stop being alarmed, because he's still alive.'' \v{11}Then he went back upstairs, broke bread, and ate. He talked with them for a long time, until dawn, and then left. \v{12}They took the young man away alive and were greatly relieved.
\passage{Paul's Trip to Miletus}

\v{13}We proceeded to the ship and sailed for Assos, where we intended to pick up Paul. He had arranged it this way, since he had planned to travel there on foot. \v{14}When he met us in Assos, we took him on board and went to Mitylene. \v{15}We sailed from there and on the following day arrived off Chios. The next day, we crossed over to Samos and stayed at Trogyllium.\fnote{\fbackref{20:15} Other mss. lack \fbib{and stayed at Trogyllium}} The day after that, we came to Miletus. \v{16}Paul had decided to sail past Ephesus to avoid spending time in Asia, as he was in a hurry to get to Jerusalem for the day of Pentecost, if that was possible.
\passage{Paul Meets with the Ephesian Elders}

\v{17}From Miletus he sent messengers\fnote{\fbackref{20:17} The Gk. lacks \fbib{messengers}} to Ephesus to ask the elders of the church to meet with him. \v{18}When they came to him, he told them, ``You know how I lived among you the entire time from the first day I set foot in Asia. \v{19}I served the Lord with all humility, with tears, and with trials that came to me through the plots of the Jews. \v{20}I never shrank from telling you anything that would help you nor from teaching you publicly and from house to house. \v{21}I testified to both Jews and Greeks about repentance to God and faith in our Lord Jesus.\fnote{\fbackref{20:21} Other mss. read \fbib{Lord Jesus, the Messiah}} \v{22}And now, compelled by the Spirit, I am on my way to Jerusalem, not knowing what will happen to me there, \v{23}except that in town after town the Holy Spirit assures me that imprisonment and suffering are waiting for me. \v{24}But I don't place any value on my life, if only I can finish my race and the ministry that I received from the Lord Jesus of testifying to the gospel of God's grace.

\v{25}``Now I know that none of you among whom I traveled preaching about the kingdom will ever see my face again. \v{26}I therefore declare to you today that I'm not responsible for the blood of any of you, \v{27}because I never shrank from telling you the whole plan of God. \v{28}Pay attention to yourselves and to the entire flock over which the Holy Spirit has made you overseers to be shepherds of God's\fnote{\fbackref{20:28} Other mss. read \fbib{the Lord's}} church, which he acquired with his own blood. \v{29}I know that when I'm gone, savage wolves will come among you and not spare the flock. \v{30}Indeed, some of your own men will arise and distort the truth in order to lure the disciples into following them. \v{31}So be alert! Remember that for three years, night and day, I never stopped tearfully warning each of you.

\v{32}``I'm now entrusting you to God and to the message of his grace, which is able to build you up and secure for you an inheritance among all who are sanctified. \v{33}I never desired anyone's silver, gold, or clothes. \v{34}You yourselves know that I worked with my own hands to support myself and those who were with me. \v{35}In every way I showed you that by working hard like this we should help the weak and remember the words that the Lord Jesus himself said, \red{`It is more blessed to give than to receive.'}''\fnote{\fbackref{20:35} This saying is not recorded in the Gospels.}

\v{36}When Paul\fnote{\fbackref{20:36} Lit. \fbib{he}} had said this, he knelt down and prayed with all of them. \v{37}All of them cried and cried\fnote{\fbackref{20:37} Lit. \fbib{Great crying came to all}} as they put their arms around Paul and kissed\fnote{\fbackref{20:37} People customarily greeted their friends with a kiss.} him affectionately. \v{38}They were especially sorrowful because of what he had said---that they would never see his face again. Then they took him to the ship.
\labelchapt{21}
\passage{Paul in Tyre}

\chapt{21}
\v{1}When we had torn ourselves away from those brothers,\fnote{\fbackref{21:1} Lit. \fbib{from them}} we sailed straight to Cos, and the next day to Rhodes, and from there to Patara.\fnote{\fbackref{21:1} Other mss. read \fbib{Patara and Myra}} \v{2}There we found a ship going across to Phoenicia, so we went aboard and sailed on. \v{3}We came in sight of Cyprus, and leaving it on our left, we sailed on to Syria and landed at Tyre because the ship was to unload its cargo there. \v{4}So we located some disciples and stayed there for seven days. Through the Spirit, they kept telling Paul not to go to Jerusalem, \v{5}but when our time there came to an end, we left and proceeded on our journey. All of them accompanied us with their wives and children out of the city. We knelt on the beach, prayed, \v{6}and said goodbye to each other. Then we reboarded the ship, and they went back home.
\passage{Paul in Caesarea}

\v{7}When we completed our voyage from Tyre, we arrived at Ptolemais, greeted the brothers there, and stayed with them for one day. \v{8}The next day, we left and came to Caesarea. We went to the home of Philip the evangelist, one of the Seven, and stayed with him. \v{9}He had four unmarried daughters who could prophesy. \v{10}After we had been there for a number of days, a prophet named Agabus arrived from Judea. \v{11}He came to us, took Paul's belt, and tied his own feet and hands with it. Then he said, ``The Holy Spirit says, `This is how the Jewish leaders\fnote{\fbackref{21:11} I.e. Judean leaders; lit. \fbib{the Jews}} in Jerusalem will tie up the man who owns this belt. Then they will hand him over to the gentiles.'\,'' \v{12}When we heard this, we and the people who lived there begged Paul\fnote{\fbackref{21:12} Lit. \fbib{him}} not to go up to Jerusalem.

\v{13}At this Paul replied, ``What do you mean by crying and breaking my heart? I'm ready not only to be tied up in Jerusalem but even to die for the name of the Lord Jesus!''

\v{14}When he could not be persuaded otherwise, we remained silent except to say, ``May the Lord's will be done.''
\passage{Paul in Jerusalem}

\v{15}When our time there ended,\fnote{\fbackref{21:15} Lit. \fbib{After those days}} we got ready to go up to Jerusalem. \v{16}Some of the disciples from Caesarea went with us. They took us to the home of Mnason to be his guests. He was from Cyprus and had been\fnote{\fbackref{21:16} The Gk. lacks \fbib{had been}} an early disciple. \v{17}When we arrived in Jerusalem, the brothers there welcomed us warmly.

\v{18}The next day, Paul went with us to visit James, and all the elders were present. \v{19}After greeting them, Paul\fnote{\fbackref{21:19} Lit. \fbib{he}} related one by one the things that God had done among the gentiles through his ministry. \v{20}When they heard about it, they praised God and told him, ``You see, brother, how many tens of thousands of believers there are among the Jews, and all of them are zealous for the Law. \v{21}But they have been told about you---that you teach all the Jews living among the gentiles to forsake the Law of Moses, and that you tell them not to circumcise their children or observe the customs. \v{22}What is to be done? They will certainly hear that you have come. \v{23}So do what we tell you. We have four men who are under a vow. \v{24}Take these men, go through the purification ceremony with them, and pay their expenses to shave their heads. Then everyone will know that there is nothing in what they have been told about you, but that you are carefully observing and keeping the Law. \v{25}As for the gentiles who have become believers, we have sent a letter with our decision that they should keep away from food that has been sacrificed to idols, from blood,\fnote{\fbackref{21:25} I.e. uncooked meat} from anything strangled,\fnote{\fbackref{21:25} Other mss. lack \fbib{from anything strangled}} and from sexual immorality.''
\passage{Paul is Arrested in the Temple}

\v{26}Then Paul took those men and the next day purified himself with them. Then he went into the Temple to announce the time when their days of purification would end and when the sacrifice would be offered for each of them. \v{27}When the seven days were almost over, the Jews from Asia, seeing Paul\fnote{\fbackref{21:27} Lit. \fbib{him}} in the Temple, stirred up a large crowd. They grabbed Paul,\fnote{\fbackref{21:27} Lit. \fbib{him}} \v{28}yelling, ``Men of Israel, help! This is the man who teaches everyone everywhere to turn against our people, the Law, and this place. More than that, he has even brought Greeks into the Temple and desecrated this Holy Place.'' \v{29}For they had earlier seen Trophimus the Ephesian in the city with him and assumed that Paul had taken him into the Temple. \v{30}The whole city was in chaos. The people rushed together, grabbed Paul, dragged him out of the Temple, and at once the doors were sealed shut.

\v{31}The crowd\fnote{\fbackref{21:31} Lit. \fbib{They}} was trying to kill Paul\fnote{\fbackref{21:31} Lit. \fbib{him}} when a report reached the tribune of the cohort\fnote{\fbackref{21:31} I.e. perhaps as many as 600 soldiers} that all Jerusalem was in an uproar. \v{32}Immediately the tribune\fnote{\fbackref{21:32} Lit. \fbib{Immediately he}} took some soldiers and officers and ran down to the crowd.\fnote{\fbackref{21:32} Lit. \fbib{them}} When the people\fnote{\fbackref{21:32} Lit. \fbib{When they}} saw the tribune and the soldiers, they stopped beating Paul. \v{33}Then the tribune came up, grabbed Paul,\fnote{\fbackref{21:33} Lit. \fbib{him}} and ordered him to be tied up with two chains. He then asked who Paul\fnote{\fbackref{21:33} Lit. \fbib{he}} was and what he had done. \v{34}Some of the crowd shouted this and some that. Since the tribune\fnote{\fbackref{21:34} Lit. \fbib{Since he}} couldn't learn the facts due to the confusion, he ordered Paul\fnote{\fbackref{21:34} Lit. \fbib{him}} to be taken into the barracks. \v{35}When Paul\fnote{\fbackref{21:35} Lit. \fbib{he}} got to the steps, he had to be carried by the soldiers because the mob had become so violent. \v{36}The crowd of people kept following him and shouting, ``Kill him!''
\passage{Paul Speaks in His Own Defense}

\v{37}Just as Paul was about to be taken into the barracks, he asked the tribune, ``May I say something to you?''

The tribune\fnote{\fbackref{21:37} Lit. \fbib{He}} asked, ``Oh, do you speak Greek? \v{38}You're not the Egyptian who started a revolt some time ago and led 4,000 assassins into the desert, are you?''

\v{39}Paul replied, ``I'm a Jew from Tarsus in Cilicia, a citizen of no insignificant city. Please let me speak to the people.'' \v{40}The tribune\fnote{\fbackref{21:40} Lit. \fbib{He}} gave him permission, and Paul, standing on the steps, motioned for the people to be silent. When everyone had quieted down, he spoke to them in the Hebrew language:
\labelchapt{22}

\chapt{22}
\v{1}``Brothers and fathers, listen to the defense that I am now making before you.'' \v{2}When they heard him speaking to them in Hebrew, they became even more quiet, and he continued:

\v{3}``I am a Jew, born in Tarsus in Cilicia but raised in this city and educated at the feet of Gamaliel in the strict ways of our ancestral Law. I am as zealous for God as all of you are today. \v{4}I persecuted this Way, even executing people,\fnote{\fbackref{22:4} The Gk. lacks \fbib{people}} and kept tying up both men and women and putting them in prison, \v{5}as the high priest and the whole council of elders can testify about me. I also received letters from them to the brothers in Damascus, and I was going there to tie up those who were there and bring them back to Jerusalem to be punished.

\v{6}``But while I was on my way and approaching Damascus about noon, a bright light from heaven suddenly flashed around me. \v{7}I fell to the ground and heard a voice saying to me, \red{`Saul! Saul! Why are you persecuting me?'}

\v{8}``I answered, `Who are you, Lord?'\fnote{\fbackref{22:8} Or \fbib{Sir}}

``He told me, \red{`I'm Jesus from Nazareth,\fnote{\fbackref{22:8} Or \fbib{Jesus the Nazarene}; the Gk. \fbib{Nazoraios} may be a word play between Heb. \fbib{netser,} meaning \fbib{branch} (cf. Isa 11:1), and the name \fbib{Nazareth.}} whom you are persecuting.'} \v{9}The men who were with me saw the light but didn't understand the voice of the one who was speaking to me.

\v{10}``Then I asked, `What am I to do, Lord?'

``The Lord told me, \red{`Get up and go into Damascus, and there you will be told everything you are destined to do.'} \v{11}Since I could not see because of the brightness of the light, the men who were with me took me by the hand and led me into Damascus.

\v{12}``A certain Ananias, who was a devout man with respect to the Law and who was highly regarded by all the Jews living there, \v{13}came to me. He stood beside me and said, `Brother Saul, receive your sight!' At that moment I could see him.

\v{14}``Then he said, `The God of our ancestors has chosen you to know his will, to see the Righteous One, and to hear his own voice,\fnote{\fbackref{22:14} Lit. \fbib{the voice of his mouth}} \v{15}because you will be his witness to all people of what you have seen and heard. \v{16}So now, what are you waiting for? Get up, be baptized, and have your sins washed away as you call on his name.'

\v{17}``Then I returned to Jerusalem. While I was praying in the Temple, I fell into a trance \v{18}and saw the Lord\fnote{\fbackref{22:18} Lit. \fbib{him}} saying to me, \red{`Hurry up and get out of Jerusalem at once, because the people\fnote{\fbackref{22:18} Lit. \fbib{they}} won't accept your testimony about me.'}

\v{19}``I said, `Lord, they themselves know that in every synagogue I kept imprisoning and beating those who believe in you. \v{20}Even when the blood of your witness Stephen was being shed, I was standing there, approving it and guarding the coats of those who were killing him.'

\v{21}``Then he told me, \red{`Go, because I'll send you far away to the gentiles.'}''

\v{22}Up to this point they listened to him, but then they began to shout, ``Away with such a fellow from the earth! He's not fit to go on living!'' \v{23}While they were yelling, tossing their coats around, and throwing dirt into the air, \v{24}the tribune ordered Paul\fnote{\fbackref{22:24} Lit. \fbib{him}} to be taken into the barracks and told the soldiers\fnote{\fbackref{22:24} Lit. \fbib{them}} to beat and question him in order to find out why the people\fnote{\fbackref{22:24} Lit. \fbib{why they}} were yelling at him like this.

\v{25}But when they had tied him up with the straps, Paul asked the centurion\fnote{\fbackref{22:25} A Roman centurion commanded about 100 men.} who was standing there, ``Is it legal for you to whip a Roman citizen who hasn't been condemned?''

\v{26}When the centurion heard this, he went to the tribune and told him, ``What are you doing? This man is a Roman citizen!''

\v{27}So the tribune went and asked Paul,\fnote{\fbackref{22:27} Lit. \fbib{him}} ``Tell me, are you a Roman citizen?''

``Yes,'' he said.

\v{28}Then the tribune replied, ``I paid a lot of money for this citizenship of mine.''

Paul said, ``But I was born a citizen.'' \v{29}Immediately those who were about to examine him stepped back, and the tribune was afraid when he found out that Paul\fnote{\fbackref{22:29} Lit. \fbib{he}} was a Roman citizen and that he had tied him up.
\passage{Paul is Brought before the Jewish Council}

\v{30}The next day, since the tribune\fnote{\fbackref{22:30} Lit. \fbib{he}} wanted to find out exactly what Paul\fnote{\fbackref{22:30} Lit. \fbib{he}} was being accused of by the Jews, he released him and ordered the high priests and the entire Council\fnote{\fbackref{22:30} Or \fbib{Sanhedrin}} to meet. Then he brought Paul down and had him stand before them.
\labelchapt{23}
\passage{Paul Defends Himself}

\chapt{23}
\v{1}Paul looked straight at the Council\fnote{\fbackref{23:1} Or \fbib{Sanhedrin}} and said, ``Brothers, with a clear conscience I have done my duty before God up to this very day.''

\v{2}Then the high priest Ananias ordered the men standing near him to strike him on the mouth. \v{3}At this Paul told him, ``God will strike you, you whitewashed wall!\fnote{\fbackref{23:3} I.e. hypocrite} How can you sit there and judge me according to the Law, and yet in violation of the Law order me to be struck?''

\v{4}The men standing near him asked, ``Do you mean to insult God's high priest?''

\v{5}Paul answered, ``I didn't realize, brothers, that he is the high priest. After all, it is written, `You must not speak evil about a ruler of your people.'\,''\fnote{\fbackref{23:5} Cf. Exod 22:28}

\v{6}When Paul saw that some of them were Sadducees and others were Pharisees, he shouted in the Council,\fnote{\fbackref{23:6} Or \fbib{Sanhedrin}} ``Brothers, I'm a Pharisee and a descendant\fnote{\fbackref{23:6} Or \fbib{son}} of Pharisees. I'm on trial concerning the hope that the dead will be resurrected.''

\v{7}After he said that, an angry quarrel broke out between the Pharisees and the Sadducees, and the assembly was divided, \v{8}because the Sadducees say that there is no resurrection and that there is no such thing as an angel or spirit, but the Pharisees believe in all those things.

\v{9}There was a great deal of shouting until some of the scribes who belonged to the party of the Pharisees stood up and argued forcefully, ``We find nothing wrong with this man. What if a spirit or an angel has spoken to him?''

\v{10}The quarrel was becoming violent, and the tribune was afraid that they would tear Paul to pieces. So he ordered the soldiers to go down, take him away from them by force, and bring him into the barracks. \v{11}That night the Lord stood near Paul\fnote{\fbackref{23:11} Lit. \fbib{him}} and said, \red{``Have courage! For just as you have testified about me in Jerusalem, you must testify in Rome, too.''}
\passage{Some Jews Plot to Kill Paul}

\v{12}In the morning, the Jewish leaders\fnote{\fbackref{23:12} I.e. Judean leaders; lit. \fbib{the Jews}} formed a conspiracy and took an oath not to eat or drink anything before they had killed Paul. \v{13}More than 40 men formed this conspiracy. \v{14}They went to the high priests and elders and said, ``We have taken a solemn oath not to taste any food before we have killed Paul. \v{15}Now then, you and the Council\fnote{\fbackref{23:15} Or \fbib{Sanhedrin}} must notify the tribune to bring him down to you on the pretext that you want to look into his case more carefully, but before he arrives we'll be ready to kill him.''

\v{16}But the son of Paul's sister heard about the ambush, so he came and got into the barracks and told Paul. \v{17}Then Paul called one of the centurions and said, ``Take this young man to the tribune, because he has something to tell him.''

\v{18}So the centurion\fnote{\fbackref{23:18} Lit. \fbib{So he}} took him, brought him to the tribune, and said, ``The prisoner Paul called me and asked me to bring this young man to you. He has something to tell you.''

\v{19}The tribune took him by the hand, stepped aside to be alone with him, and asked, ``What have you got to tell me?''

\v{20}He answered, ``The Jewish leaders\fnote{\fbackref{23:20} I.e. Judean leaders; lit. \fbib{the Jews}} have agreed to ask you to bring Paul down to the Council\fnote{\fbackref{23:20} Or \fbib{Sanhedrin}} tomorrow as though they were going to examine his case more carefully. \v{21}Don't believe them, because more than 40 of them are planning to ambush him. They've taken an oath not to eat or drink before they've killed him. They are ready now, just waiting for your consent.''

\v{22}The tribune dismissed the young man and ordered him not to tell anyone that he had notified him. \v{23}Then he summoned two centurions and ordered, ``Get 200 soldiers ready to leave for Caesarea at nine o'clock tonight,\fnote{\fbackref{23:23} Lit. \fbib{from the third hour of the night}} along with 70 mounted soldiers and 200 soldiers with spears. \v{24}Provide a mount for Paul to ride, and take him safely to Governor Felix.'' \v{25}He wrote a letter with this message:

\v{26}``From:\fnote{\fbackref{23:26} The Gk. lacks \fbib{From}} Claudius Lysias

To: Governor Felix

Greetings, Your Excellency:

\v{27}This man had been seized by the Jews and was about to be killed by them when I went with the guard and rescued him, having learned that he was a Roman citizen. \v{28}I wanted to know the exact charge they were making against him, so I had him brought before their Council.\fnote{\fbackref{23:28} Or \fbib{Sanhedrin}} \v{29}I found that, although he was charged with questions about their Law, there was no charge against him deserving death or imprisonment. \v{30}Since a plot against the man has been reported to me, I'm sending him to you at once, and I've also ordered his accusers to present their charges against him before you.''

\v{31}So the soldiers, in keeping with their orders, took Paul and brought him by night to Antipatris. \v{32}The next day, they let the horsemen ride with Paul\fnote{\fbackref{23:32} Lit. \fbib{him}} while they returned to their barracks. \v{33}When these men\fnote{\fbackref{23:33} The Gk. lacks \fbib{men}} came to Caesarea, they delivered the letter to the governor and handed Paul over to him. \v{34}After reading the letter, the governor\fnote{\fbackref{23:34} Lit. \fbib{letter, he}} asked which province Paul\fnote{\fbackref{23:34} Lit. \fbib{Paul}} was from. On learning that he was from Cilicia, \v{35}he said, ``I will hear your case when your accusers arrive.'' Then he ordered Paul\fnote{\fbackref{23:35} Lit. \fbib{Paul}} to be kept in custody in Herod's palace.\fnote{\fbackref{23:35} Lit. \fbib{praetorium}}
\labelchapt{24}
\passage{Paul Presents His Case to Felix}

\chapt{24}
\v{1}Five days later, the high priest Ananias arrived with certain elders and Tertullus, an attorney, and they summarized their case against Paul before the governor. \v{2}When Paul\fnote{\fbackref{24:2} Lit. \fbib{he}} had been summoned, Tertullus opened the prosecution by saying:

``Your Excellency Felix, since we are enjoying lasting peace because of you, and since reforms for this nation are being brought about through your foresight, \v{3}we always and everywhere acknowledge it with profound gratitude. \v{4}But so as not to detain you any further, I beg you to hear us briefly with your customary graciousness. \v{5}For we have found this man a perfect pest and an agitator among all Jews throughout the world. He is a ringleader in the sect of the Nazarenes\fnote{\fbackref{24:5} The Gk. \fbib{Nazoraios} may be a word play between Heb. \fbib{netser,} meaning \fbib{branch} (cf. Isa 11:1), and the name \fbib{Nazareth.}} \v{6}and even tried to profane the Temple, but we arrested him.\fnote{\fbackref{24:6} Other mss. read \fbib{arrested him, and we wanted to try him under our law.} \fbib{\v{7}But Tribune Lysias came along and took him out of our hands with much force,} \fbib{\v{8}ordering his accusers to come before you.}} \v{8}By examining him for yourself, you will be able to find out from him everything of which we accuse him.''

\v{9}The Jewish leaders\fnote{\fbackref{24:9} I.e. Judean leaders; lit. \fbib{The Jews}} supported his accusations by asserting that these things were true. \v{10}When the governor motioned for Paul to speak, he replied:

``Since I know that you have been a judge over this nation for many years, I am pleased to present my defense. \v{11}You can verify for yourself that I went up to worship in Jerusalem no more than twelve days ago. \v{12}They never found me debating with anyone in the Temple or stirring up a crowd in the synagogues or throughout the city, \v{13}and they cannot prove to you the charges they are now bringing against me. \v{14}However, I admit to you that in accordance with the Way, which they call a heresy,\fnote{\fbackref{24:14} Or \fbib{sect}} I worship the God of our ancestors and believe in everything written in the Law and the Prophets. \v{15}I have the same hope in God that they themselves cherish---that there is to be a resurrection of the righteous and the wicked. \v{16}Therefore, I always do my best to have a clear conscience before God and people. \v{17}After many years, I have come back to my people to bring gifts for the poor and to offer sacrifices. \v{18}They found me in the Temple doing these things just as I had completed the purification ceremony. No crowd or noisy mob was present. \v{19}But some Jews from Asia were there, and they should be here before you to accuse me if they have anything against me. \v{20}Otherwise, these men themselves should tell what wrong they found when I stood before the Council\fnote{\fbackref{24:20} Or \fbib{Sanhedrin}}--- \v{21}unless it is for the one thing I shouted as I stood among them: `It is for the resurrection of the dead that I am on trial before you today.'\,''

\v{22}Felix was rather well informed about the Way, and so he adjourned the trial with the comment, ``When Tribune Lysias arrives, I'll decide your case.'' \v{23}He ordered the centurion to guard Paul\fnote{\fbackref{24:23} Lit. \fbib{him}} but to let him have some freedom and not to keep any of his friends from caring for his needs.

\v{24}Some days later, Felix arrived with his wife Drusilla, who was Jewish. He sent for Paul and listened to him talk about faith in Jesus\fnote{\fbackref{24:24} Other mss. lack \fbib{Jesus}} the Messiah.\fnote{\fbackref{24:24} Or \fbib{Christ}} \v{25}As Paul\fnote{\fbackref{24:25} Lit. \fbib{he}} talked about righteousness, self-control, and the coming judgment, Felix became afraid and said, ``For the present you may go. When I get a chance, I will send for you again.'' \v{26}At the same time he was hoping to receive a bribe from Paul, and so he would send for him frequently to talk with him.

\v{27}After two years had passed, Felix was succeeded by Porcius Festus. Since Felix wanted to do the Jews a favor, he left Paul in prison.
\labelchapt{25}
\passage{Paul Appeals to the Emperor}

\chapt{25}
\v{1}Three days after Festus had arrived in the province, he went up from Caesarea to Jerusalem. \v{2}The high priests and Jewish leaders informed him of their charges against Paul, urging \v{3}and asking Festus\fnote{\fbackref{25:3} Lit. \fbib{him}} to have Paul\fnote{\fbackref{25:3} Lit. \fbib{as a favor against him to have him}} brought to Jerusalem as a favor. They were laying an ambush to kill him on the road.

\v{4}Festus replied that Paul was being kept in custody at Caesarea and that he himself would be going there soon. \v{5}``Therefore,'' he said, ``have your authorities come down with me and present their charges against him there, if there is anything wrong with the man.''

\v{6}Festus\fnote{\fbackref{25:6} Lit. \fbib{He}} stayed with them no more than eight or ten days and then went down to Caesarea. The next day, he sat on the judge's seat and ordered Paul brought in. \v{7}When Paul\fnote{\fbackref{25:7} Lit. \fbib{he}} arrived, the Jewish leaders\fnote{\fbackref{25:7} I.e. Judean leaders; lit. \fbib{the Jews}} who had come down from Jerusalem surrounded him and began bringing a number of serious charges against him that they couldn't prove. \v{8}Paul said in his defense, ``I have done nothing wrong against the Law of the Jews, or of the Temple, or of the emperor.''

\v{9}Then Festus, wanting to do the Jewish leaders\fnote{\fbackref{25:9} I.e. Judean leaders; lit. \fbib{the Jews}} a favor, asked Paul, ``Are you willing to go up to Jerusalem to be tried there before me on these charges?''

\v{10}But Paul said, ``I am standing before the emperor's judgment seat where I ought to be tried. I haven't done anything wrong to the Jewish leaders,\fnote{\fbackref{25:10} I.e. Judean leaders; lit. \fbib{the Jews}} as you know very well. \v{11}If I'm guilty and have done something that deserves death, I'm willing to die. But if there is nothing to their charges against me, no one can hand me over to them as a favor. I appeal to the emperor!''

\v{12}Festus talked it over with the council and then answered, ``To the emperor you have appealed; to the emperor you will go!''
\passage{King Agrippa Meets Paul}

\v{13}After several days had passed, King Agrippa and Bernice came to Caesarea to welcome Festus. \v{14}Since they were staying there for several days, Festus laid Paul's case before the king. He said, ``There is a man here who was left in prison by Felix. \v{15}When I went to Jerusalem, the high priests and the Jewish elders informed me about him and asked me to condemn him. \v{16}I answered them that it was not the Roman custom to sentence a man to be punished until the accused met his accusers face to face and had an opportunity to defend himself against the charge. \v{17}So they came here with me, and the next day without any delay I sat down in the judge's seat and ordered the man to be brought in. \v{18}When his accusers stood up, they didn't accuse him of any of the crimes\fnote{\fbackref{25:18} Other mss. read \fbib{of anything}} I was expecting. \v{19}Instead, they had several arguments with him about their own religion and about a certain Jesus who had died---but Paul kept asserting he was alive. \v{20}I was puzzled how I should investigate such matters, so I asked if he would like to go to Jerusalem and be tried there for these things. \v{21}But Paul appealed his case and asked to be held in prison until the decision of his Majesty. So I ordered him to be held in custody until I could send him to the emperor.''

\v{22}Agrippa told Festus, ``I would like to hear the man.''

``Tomorrow,'' he said, ``you will hear him.''

\v{23}The next day, Agrippa and Bernice arrived with much fanfare and went into the auditorium along with the tribunes and the leading men of the city. At the command of Festus, Paul was brought in. \v{24}Then Festus said, ``King Agrippa and all you men who are present with us! You see this man about whom the whole Jewish nation petitioned me, both in Jerusalem and here, shouting that he ought not to live any longer. \v{25}I find that he has not done anything deserving of death. But since he has appealed to his Majesty, I have decided to send him. \v{26}I have nothing reliable to write our Sovereign about him, so I have brought him to all of you, and especially to you, King Agrippa, so that I will have something to write after he is cross-examined. \v{27}For it seems to me absurd to send a prisoner without specifying the charges against him.''
\labelchapt{26}
\passage{Paul Presents His Case to Agrippa}

\chapt{26}
\v{1}Then Agrippa told Paul, ``You have permission to speak for yourself.'' So Paul stretched out his hand and began his defense.

\v{2}``I consider myself fortunate that it is before you, King Agrippa, that I can defend myself today against all the accusations of the Jewish leaders,\fnote{\fbackref{26:2} I.e. Judean leaders; lit. \fbib{the Jews}} \v{3}since you are especially familiar with all the Jewish customs and controversies. I beg you, therefore, to listen patiently to me. \v{4}All the Jews know how I lived from the earliest days of my youth with my own people and in Jerusalem. \v{5}They have known for a long time, if they would but testify to it, that I lived as a Pharisee, adhering to the standards of our strictest religious party.

\v{6}``And now I stand here on trial for the hope of the promise made by God to our ancestors. \v{7}Our twelve tribes, worshiping day and night with intense devotion, hope to attain it. It is because of this hope, O King, that I am accused by the Jews. \v{8}Why is it thought incredible by all of you that God should raise the dead? \v{9}Indeed, I myself thought it my duty to take extreme measures against the name of Jesus from Nazareth.\fnote{\fbackref{26:9} Or \fbib{Jesus the Nazarene}; the Gk. \fbib{Nazoraios} may be a word play between Heb. \fbib{netser,} meaning \fbib{branch} (cf. Isa 11:1), and the name \fbib{Nazareth.}} \v{10}That is what I did in Jerusalem. I received authority from the high priests and locked many of the saints in prison. And when I cast my vote against them, they were put to death. \v{11}I would even punish them frequently in every synagogue and try to make them blaspheme. Raging furiously against them, I would hunt them down even in distant cities.

\v{12}``That is how I happened to be traveling to Damascus with authority based on a commission from the high priests. \v{13}On the road at noon, O King, I saw a light from heaven that was brighter than the sun. It flashed around me and those who were traveling with me.

\v{14}``All of us fell to the ground, and I heard a voice asking me in the Hebrew language, \red{`Saul! Saul! Why are you persecuting me? It is hurting you to keep on kicking against the cattle prods.'}\fnote{\fbackref{26:14} Quoted in \fbib{The Bacchae} by Euripides, although Aeschylus in \fbib{Prometheus Bound} used almost identical language.}

\v{15}``I asked, `Who are you, Lord?'\fnote{\fbackref{26:15} Or \fbib{Sir}}

``The Lord answered, \red{`I'm Jesus, whom you are persecuting.} \v{16}\red{But get up and stand on your feet, because I've appeared to you for the very purpose of appointing you to be my servant and witness of what you've seen and of what I'll show you.} \v{17}\red{I'll continue to rescue you from your people and from the gentiles to whom I'm sending you.} \v{18}\red{You will help them understand\fnote{\fbackref{26:18} Lit. \fbib{will open their eyes}} and turn them from darkness to light and from Satan's control to God, so that their sins will be forgiven and they will receive a share among those who are sanctified by faith in me.'}

\v{19}``And so, King Agrippa, I was not disobedient to the heavenly vision. \v{20}Instead, I first told the people in Damascus and Jerusalem, then all the people in Judea---and after that the gentiles---to repent, turn to God, and perform deeds that are consistent with such repentance. \v{21}For this reason the Jewish leaders\fnote{\fbackref{26:21} I.e. Judean leaders; lit. \fbib{the Jews}} grabbed me in the Temple and kept trying to kill me. \v{22}I've had help from God to this day, and so I stand here to testify to both the powerful and the lowly alike, stating only what the prophets and Moses said would happen--- \v{23}that the Messiah\fnote{\fbackref{26:23} Or \fbib{Christ}} would suffer and be the first to rise from the dead and would bring light both to our people and to the gentiles.''

\v{24}As he continued his defense, Festus shouted, ``You're out of your mind, Paul! Too much education is driving you crazy!''

\v{25}But Paul said, ``I'm not out of my mind, Your Excellency Festus. I'm reporting what is absolutely true. \v{26}Indeed, the king knows about these things, and I can speak to him freely. For I'm certain that none of these things has escaped his notice, since this wasn't done in a corner. \v{27}King Agrippa, do you believe the prophets? I know you believe them!''

\v{28}Agrippa asked Paul, ``Can you so quickly persuade me to become a Christian?''

\v{29}Paul replied, ``Whether quickly or not, I wish to God that not only you but everyone listening to me today would become what I am---except for these chains!''

\v{30}Then the king, the governor, Bernice, and those who were sitting with him got up. \v{31}As they were leaving, they began to say to each other, ``This man hasn't been doing anything to deserve death or imprisonment.''

\v{32}Agrippa told Festus, ``This man could have been set free if he hadn't appealed to the emperor.''
\labelchapt{27}
\passage{Paul Sails for Rome}

\chapt{27}
\v{1}When it was decided that we should sail to Italy, Paul and some other prisoners were transferred to a centurion named Julius, who belonged to the emperor's division. \v{2}After boarding a ship from Adramyttium that was about to sail to the ports on the coast of Asia, we put out to sea. Aristarchus, a Macedonian from Thessalonica, went with us.

\v{3}The next day, we arrived at Sidon, and Julius treated Paul kindly, allowing him to visit his friends there and to receive any care he needed. \v{4}After putting out from there, we sailed on the sheltered side of Cyprus because the winds were against us. \v{5}We sailed along the sea off Cilicia and Pamphylia and reached Myra in Lycia. \v{6}There the centurion found an Alexandrian ship bound for Italy and put us on it. \v{7}We sailed slowly for a number of days and with difficulty arrived off Cnidus. Then, because the wind was against us, we sailed on the sheltered side of Crete off Cape Salome. \v{8}Sailing past it with difficulty, we came to a place called Fair Havens, near the town of Lasea. \v{9}Much time had been lost, and because navigation had become dangerous and the day of fasting had already past, Paul began to warn those on the ship,\fnote{\fbackref{27:9} Lit. \fbib{warn them}} \v{10}``Men, I see that during this voyage there will be hardship and a heavy loss not only of the cargo and ship, but also of our lives.''

\v{11}But the centurion was persuaded by the pilot and the owner of the ship and not by what Paul said. \v{12}Since the harbor was not a good place to spend the winter, most of the men favored putting out to sea from there on the chance that somehow they could reach Phoenix and spend the winter there. It is a Cretian harbor that faces southwest and northwest. \v{13}When a gentle breeze began to blow from the south, they thought they could make it to Phoenix,\fnote{\fbackref{27:13} Lit. \fbib{could achieve their purpose}} so they hoisted anchor and began sailing along the shore of Crete.

\v{14}But it was not long before a violent wind (called a northeaster) swept down from the island.\fnote{\fbackref{27:14} Lit. \fbib{from it}} \v{15}The ship was caught so that it couldn't face the wind, and we gave up and were swept along. \v{16}As we drifted to the sheltered side of a small island called Cauda,\fnote{\fbackref{27:16} Other mss. read \fbib{Clauda}} we barely managed to secure the ship's lifeboat. \v{17}The ship's crew\fnote{\fbackref{27:17} Lit. \fbib{They}} pulled it up on deck and used ropes to brace the ship. Fearing that they would hit the large sandbank near Libya,\fnote{\fbackref{27:17} Lit. \fbib{the Syrtis}} they lowered the sail and drifted along. \v{18}The next day, because we were being tossed so violently by the storm, they began to throw the cargo overboard. \v{19}On the third day they threw the ship's equipment overboard with their own hands. \v{20}For a number of days neither the sun nor the stars were to be seen, and the storm continued to rage until at last all hope of our being saved vanished.

\v{21}After they had gone a long time without food, Paul stood among his shipmates\fnote{\fbackref{27:21} Lit. \fbib{among them}} and said, ``Men, you should have listened to me and not have sailed from Crete. You would have avoided this hardship and damage. \v{22}But now I urge you to have courage, because there will be no loss of life among you, but only loss\fnote{\fbackref{27:22} The Gk. lacks \fbib{loss}} of the ship. \v{23}For just last night an angel of God, to whom I belong and whom I serve, stood by me \v{24}and said, `Stop being afraid, Paul! You must stand before the emperor. Indeed, God has given to you the lives of\fnote{\fbackref{27:24} The Gk. lacks \fbib{the lives of}} everyone who is sailing with you.' \v{25}So take courage, men, because I trust God that it will turn out just as he told me. \v{26}However, we will have to run aground on some island.''
\passage{The Shipwreck}

\v{27}It was the fourteenth night, and we were drifting through the Adriatic Sea, when about midnight the sailors suspected that land was near. \v{28}After taking soundings, they found the depth to be twenty fathoms. A little later, they took soundings again and found it was fifteen fathoms. \v{29}Fearing that we might run aground on the rocks, they dropped four anchors from the stern and began praying for daylight to come. \v{30}Meanwhile, the sailors had begun trying to escape from the ship. They lowered the lifeboat into the sea and pretended that they were going to lay out the anchors from the bow. \v{31}Paul told the centurion and the soldiers, ``Unless these men remain onboard, you cannot be saved.'' \v{32}Then the soldiers cut the ropes that held the lifeboat and set it adrift.

\v{33}Right up to daybreak Paul kept urging all of them to eat something. He said, ``Today is the fourteenth day that you have been waiting and going without food, not eating anything. \v{34}So I urge you to eat something, for it will help you survive, since none of you will lose so much as\fnote{\fbackref{27:34} The Gk. lacks \fbib{so much as}} a hair from his head.'' \v{35}After he said this, he took some bread, thanked God in front of everyone, broke it, and began to eat. \v{36}Everyone was encouraged and had something to eat. \v{37}There were 276\fnote{\fbackref{27:37} Other mss. read \fbib{76}} of us on the ship. \v{38}After they had eaten all they wanted, they began to lighten the ship by dumping its cargo of\fnote{\fbackref{27:38} Lit. \fbib{dumping the}} wheat into the sea.

\v{39}When day came, they didn't recognize the land, but they could see a bay with a beach on which they planned to run the ship ashore, if possible. \v{40}So they cut the anchors free and left them in the sea. At the same time they untied the ropes that held the steering oars, raised the foresail to the wind, and headed for the beach. \v{41}But they struck a sandbar and ran the ship aground. The bow stuck and couldn't be moved, while the stern was broken to pieces by the force of the waves. \v{42}The soldiers' plan was to kill the prisoners to keep them from swimming ashore and escaping, \v{43}but the centurion wanted to save Paul, so he prevented them from carrying out their plan. He ordered those who could swim to jump overboard first and get to land. \v{44}The rest were to follow, some on planks and others on various pieces of the ship. In this way everyone got to shore safely.
\labelchapt{28}
\passage{Paul on the Island of Malta}

\chapt{28}
\v{1}When we were safely on shore, we learned that the island was called Malta. \v{2}The people who lived there were unusually kind to us. It had started to rain and was cold, so they started a bonfire and invited us to join them\fnote{\fbackref{28:2} The Gk. lacks \fbib{to join them}} around it. \v{3}Paul gathered a bundle of sticks and put it on the fire. A poisonous snake was forced out by the heat and attached itself to Paul's\fnote{\fbackref{28:3} Lit. \fbib{his}} hand. \v{4}When the people who lived there saw the snake hanging from his hand, they told one another, ``This man must be a murderer! He may have escaped from the sea, but Justice\fnote{\fbackref{28:4} I.e. a Roman god whom they supposed punished wrongdoers} won't let him live.'' \v{5}But he shook the snake into the fire and wasn't harmed. \v{6}They were expecting him to swell up or suddenly drop dead, but after waiting a long time and seeing nothing unusual happen to him, they changed their minds and said he was a god.

\v{7}The governor of the island, whose name was Publius, owned estates in that part of the island. He welcomed us and entertained us with great hospitality for three days. \v{8}The father of Publius happened to be sick in bed with fever and dysentery. Paul went to him, prayed, and healed him by placing his hands on him. \v{9}After that had happened, the rest of the sick people on the island went to him and were healed. \v{10}The islanders\fnote{\fbackref{28:10} Lit. \fbib{They}} honored us in many ways, and when we were about to sail again,\fnote{\fbackref{28:10} The Gk. lacks \fbib{again}} they supplied us with everything we needed.
\passage{Paul Sails from Malta to Rome}

\v{11}Three months later, we continued our sailing onboard an Alexandrian ship that had spent the winter at the island. It had the Twin Brothers\fnote{\fbackref{28:11} Lit. \fbib{the Dioscuri}; i.e. Castor and Pollux, twin sons of Zeus} as its figurehead. \v{12}We stopped at Syracuse and stayed there for three days. \v{13}Then we weighed anchor and came to Rhegium. A day later, a south wind began to blow, and on the second day we came to Puteoli. \v{14}There we found some brothers and were invited to stay with them for seven days. After this, we arrived in Rome. \v{15}The brothers there heard about us and came as far as the Forum of Appius and the Three Taverns to meet us. When Paul saw them, he thanked God and felt encouraged. \v{16}When we came into Rome, Paul was allowed to live by himself with the soldier who was guarding him.
\passage{Paul in Rome}

\v{17}Three days later, Paul\fnote{\fbackref{28:17} Lit. \fbib{he}} called the leaders of the Jews together. When they assembled, he told them, ``Brothers, although I haven't done anything against our people or the customs of our ancestors, I was arrested in Jerusalem and handed over to the Romans. \v{18}They examined me and wanted to let me go because there was no reason for me to receive\fnote{\fbackref{28:18} The Gk. lacks \fbib{me to receive}} the death penalty in my case. \v{19}But the Jews objected and forced me to appeal to the emperor, even though I have no countercharge to bring against my own people. \v{20}That's why I asked to see you and speak with you, since it is for the hope of Israel that I'm wearing this chain.''

\v{21}The Jewish leaders\fnote{\fbackref{28:21} Lit. \fbib{They}} told him, ``We haven't received any letters from Judea about you, and none of the brothers coming here has reported or mentioned anything bad about you. \v{22}However, we'd like to hear from you what you believe, because people are talking against this sect everywhere.'' \v{23}So they set a day to meet with Paul\fnote{\fbackref{28:23} Lit. \fbib{him}} and came out in large numbers to see him where he was staying.

From morning until evening, he continued to explain the kingdom of God to them, trying to convince them about Jesus from the Law of Moses and the Prophets. \v{24}Some of them were convinced by what he said, but others wouldn't believe. \v{25}They disagreed with one another as they were leaving, so Paul added this statement: ``The Holy Spirit was so right when he spoke to your ancestors through the prophet Isaiah! \v{26}He said,

\begin{poetry}
\poeml `Go to this people and say, \\
\poemll    ``You will listen and listen \\
\poemlll       but never understand, \\
\poemll    and you will look and look \\
\poemlll       but never see! \\
\poeml \v{27}For this people's minds\fnote{\fbackref{28:27} Lit. \fbib{heart}} have become stupid, \\
\poemll    and their ears can barely hear, \\
\poeml and they have shut their eyes \\
\poemll    so that they may never see with their eyes, \\
\poeml and listen with their ears, \\
\poemll    and understand with their heart \\
\poeml and turn and let me heal them.''\,'\fnote{\fbackref{28:27} Cf. Isa 6:9-10}
\end{poetry}

\v{28}You must understand that this message about\fnote{\fbackref{28:28} The Gk. lacks \fbib{message about}} God's salvation has been sent to the gentiles, and they will listen.''\fnote{\fbackref{28:28} Other mss. read \fbib{will listen.} \fbib{\v{29}When he had said these words, the Jews left, arguing intensely among themselves.}}

\v{30}For two whole years Paul\fnote{\fbackref{28:30} Lit. \fbib{he}} lived in his own rented place and welcomed everyone who came to him. \v{31}He continued to preach about the kingdom of God and to teach boldly and freely about the Lord Jesus, the Messiah.\fnote{\fbackref{28:31} Or \fbib{Christ}}

\addcontentsline{toc}{chapter}{The Explanation of the Good News}
\bookheader{Romans}
\labelbook{Rom}

\bookpretitle{The Letter from Paul to the}
\booktitle{Romans}

\labelchapt{1}
\passage{Greetings from Paul}

\chapt{1}
\v{1}From:\fnote{\fbackref{1:1} The Gk. lacks \fbib{From}} Paul, a servant of Jesus the Messiah,\fnote{\fbackref{1:1} Or \fbib{Christ}; \fbib{o} ther mss. read \fbib{of the Messiah Jesus}} called to be an apostle and set apart for God's gospel, \v{2}which he promised beforehand through his prophets in the Holy Scriptures \v{3}regarding\fnote{\fbackref{1:3} Lit. \fbib{About}} his Son. He was a descendant of David with respect to his humanity \v{4}and was declared by the resurrection from the dead to be the powerful Son of God according to the spirit\fnote{\fbackref{1:4} Or \fbib{Spirit}} of holiness---Jesus the Messiah,\fnote{\fbackref{1:4} Or \fbib{Christ}} our Lord. \v{5}Through him we received grace and a commission as an apostle to bring about faithful obedience among all the gentiles for the sake of his name. \v{6}You, too, are among those who have been called to belong to Jesus the Messiah.\fnote{\fbackref{1:6} Or \fbib{Christ}}

\v{7}To: Everyone in Rome,\fnote{\fbackref{1:7} Other mss. lack \fbib{in Rome}} loved by God and called to be holy.\fnote{\fbackref{1:7} Or \fbib{saints}}

May grace and peace from God our Father and the Lord Jesus, the Messiah,\fnote{\fbackref{1:7} Or \fbib{Christ}} be yours!
\passage{Paul's Prayer and Desire to Visit Rome}

\v{8}First of all, I thank my God through Jesus the Messiah\fnote{\fbackref{1:8} Or \fbib{Christ}} for all of you, because the news about your faith is being reported throughout the world. \v{9}For God, whom I serve with my spirit by preaching the gospel about his Son, is my witness how constantly I mention you \v{10}in my prayers at all times, asking that somehow by God's will I may at last succeed in coming to you. \v{11}For I am longing to see you so that I may impart to you some spiritual gift to make you strong, \v{12}that is, that we may be mutually encouraged by each other's faith, both yours and mine.

\v{13}I do not want you to be unaware, brothers, that I often planned to come to you (but have been prevented from doing so until now), so that I might reap a harvest among you, just as I have among the rest of the gentiles. \v{14}Both to Greeks and to barbarians,\fnote{\fbackref{1:14} I.e. uncultured people} both to wise and to foolish people, I am a debtor. \v{15}That is why I am so eager to proclaim the gospel to you who live in Rome,\fnote{\fbackref{1:15} Other mss. lack \fbib{who live in Rome}} too.

\v{16}For I am not ashamed of the gospel,\fnote{\fbackref{1:16} Other mss. read \fbib{gospel of the Messiah}} because it is God's power for the salvation of everyone who believes, of the Jew first and of the Greek as well. \v{17}For in the gospel\fnote{\fbackref{1:17} Lit. \fbib{in it}} God's righteousness is being revealed from faith to faith, as it is written, ``The righteous will live by faith.''\fnote{\fbackref{1:17} Cf. Hab 2:4}
\passage{God's Wrath against Sinful Humanity}

\v{18}For God's wrath is being revealed from heaven against all the ungodliness and wickedness of those who in their wickedness suppress the truth. \v{19}For what can be known about God is plain to them, because God himself has made it plain to them. \v{20}For since the creation of the world God's\fnote{\fbackref{1:20} Lit. \fbib{his}} invisible attributes---his eternal power and divine nature---have been understood and observed by what he made, so that people\fnote{\fbackref{1:20} Lit. \fbib{they}} are without excuse. \v{21}For although they knew God, they neither glorified him as God nor gave thanks to him. Instead, their thoughts turned to worthless things,\fnote{\fbackref{1:21} Lit. \fbib{they became worthless in their thoughts}} and their senseless hearts were darkened. \v{22}Though claiming to be wise, they became fools \v{23}and exchanged the glory of the immortal God for images that looked like mortal human beings, birds, four-footed animals, and reptiles.

\v{24}For this reason, God delivered them to sexual impurity as they followed the lusts\fnote{\fbackref{1:24} Lit. \fbib{to impurity in the lusts}} of their hearts and dishonored their bodies with one another. \v{25}They exchanged God's truth for a lie and worshipped and served the creation rather than the Creator, who is blessed forever. Amen.

\v{26}For this reason, God delivered them to degrading passions as their females exchanged their natural sexual function for one that is unnatural. \v{27}In the same way, their males also abandoned their natural sexual function toward females and burned with lust toward one another. Males committed indecent acts with males, and received within themselves the appropriate penalty for their perversion.\fnote{\fbackref{1:27} Or \fbib{deviation}}

\v{28}Furthermore, because they did not think it worthwhile to keep knowing God fully, God delivered them to degraded minds to perform acts that should not be done. \v{29}They have become filled with every kind of wickedness, evil, greed, and depravity. They are full of envy, murder, quarreling, deceit, and viciousness. They are gossips, \v{30}slanderers, God-haters, haughty, arrogant, boastful, inventors of evil, disobedient to their parents, \v{31}foolish, faithless, heartless, and ruthless.\v{32}Although they know God's just requirement---that those who practice such things deserve to die---they not only do these things but even applaud others who practice them.
\labelchapt{2}
\passage{God will Judge Everyone}

\chapt{2}
\v{1}Therefore, you have no excuse---every one of you who judges. For when you pass judgment on another person, you condemn yourself, since you, the judge, practice the very same things. \v{2}Now we know that God's judgment against those who act like this is based on\fnote{\fbackref{2:2} Lit. \fbib{is according to the}} truth. \v{3}So when you, a mere man, pass judgment on those who practice these things and then do them yourself, do you think you will escape God's judgment? \v{4}Or are you unaware of his rich kindness, forbearance, and patience, that it is God's kindness that is leading you to repent?

\v{5}But because of your stubborn and unrepentant heart you are reserving wrath for yourself on the day of wrath, when God's righteous judgment will be revealed. \v{6}For he will repay everyone according to what that person has done: \v{7}eternal life to those who strive for glory, honor, and immortality by patiently doing good; \v{8}but wrath and fury for those who in their selfish pride refuse to believe the truth and practice wickedness instead. \v{9}There will be suffering and anguish for every human being who practices doing evil, for Jews first and for Greeks as well. \v{10}But there will be glory, honor, and peace for everyone who practices doing good, initially for Jews but also for Greeks as well, \v{11}because God does not show partiality.

\v{12}For all who have sinned apart from the Law will also perish apart from the Law, and all who have sinned under the Law will be judged by the Law. \v{13}For it is not merely those who hear the Law who are righteous in God's sight. No, it is those who follow the Law, who will be justified. \v{14}For whenever gentiles, who do not possess the Law, do instinctively what the Law requires, they are a law to themselves, even though they do not have the Law. \v{15}They show that what the Law requires is written in their hearts, a fact to which their own consciences testify, and their thoughts will either accuse or excuse them \v{16}on that day when God, through Jesus the Messiah,\fnote{\fbackref{2:16} Or \fbib{Christ}} will judge people's secrets according to my gospel.
\passage{Who is a Jew?}

\v{17}Now if you call yourself a Jew, and rely on the Law, and boast about God, \v{18}and know his will, and approve of what is best because you have been instructed in the Law; \v{19}and if you are convinced that you are a guide for the blind, a light to those in darkness, \v{20}an instructor of ignorant people, and a teacher of infants because you have the full content of knowledge and truth in the Law--- \v{21}as you teach others, do you fail to teach yourself? As you preach against stealing, do you steal? \v{22}As you forbid adultery, do you commit adultery? As you abhor idols, do you rob temples? \v{23}As you boast about the Law, do you dishonor God by breaking the Law? \v{24}As it is written, ``God's name is being blasphemed among the gentiles because of you.''\fnote{\fbackref{2:24} Cf. sa 52:5}

\v{25}For circumcision is valuable if you observe the Law, but if you break the Law, your having been circumcised has no more value than if you were uncircumcised. \v{26}So if a man who is uncircumcised keeps the requirements of the Law, his uncircumcision will be regarded as circumcision, won't it? \v{27}The man who is uncircumcised physically but who keeps the Law will condemn you who break the Law, even though you have the written Law\fnote{\fbackref{2:27} Lit. \fbib{what is written}} and circumcision. \v{28}For a person is not a Jew because of his appearance, nor is circumcision something just external and physical. \v{29}No, a person is a Jew inwardly, and circumcision is a matter of the heart, brought about by the Spirit, not by a written law.\fnote{\fbackref{2:29} Lit. \fbib{what is written}} That person's praise will come from God, not from people.
\labelchapt{3}
\passage{Everyone is a Sinner}

\chapt{3}
\v{1}What advantage, then, does the Jew have, or what value is there in circumcision? \v{2}There are all kinds of advantages! First of all, the Jews\fnote{\fbackref{3:2} Lit. \fbib{they}} have been entrusted with the utterances of God. \v{3}What if some of the Jews\fnote{\fbackref{3:3} Lit. \fbib{of them}} were unfaithful? Their unfaithfulness cannot cancel God's faithfulness, can it? \v{4}Of course not! God is true, even if everyone else is a liar. As it is written,

\begin{poetry}
\poeml ``You are right when you speak,\fnote{\fbackref{3:4} Lit. \fbib{are justified in your words}} \\
\poemll    and win your case when you go into court.''\fnote{\fbackref{3:4} Cf. Ps 51:4}
\end{poetry}

\v{5}But if our unrighteousness serves to confirm God's righteousness, what can we say? God is not unrighteous when he vents his wrath on us, is he? (I am talking in human terms.) \v{6}Of course not! Otherwise, how could God judge the world? \v{7}For\fnote{\fbackref{3:7} Other mss. read \fbib{But}} if through my falsehood God's truthfulness glorifies him even more, why am I still being condemned as a sinner? \v{8}Or can we say---as some people slander us by claiming that we say---``Let's do evil that good may result''? They deserve to be condemned!

\v{9}What, then, does this mean?\fnote{\fbackref{3:9} The Gk. lacks \fbib{does this mean}} Are we Jews\fnote{\fbackref{3:9} The Gk. lacks \fbib{Jews}} any better off? Not at all! For we have already accused everyone, both Jews and Greeks, of being under the power of\fnote{\fbackref{3:9} The Gk. lacks \fbib{the power of}} sin. \v{10}As it is written,

\begin{poetry}
\poeml ``Not even one person is righteous. \\
\poeml \v{11}No one understands. \\
\poemll    No one searches for God. \\
\poeml \v{12}All have turned away. \\
\poemll    They have become completely worthless. \\
\poemlll       No one shows kindness, not even one person!\fnote{\fbackref{3:12} Cf. Ps 14:1-3; 53:1-3; Eccl 7:20} \\
\poeml \v{13}Their throats are open graves. \\
\poemll    With their tongues they deceive.\fnote{\fbackref{3:13} Cf. Ps 5:9} \\
\poemlll       The venom of poisonous snakes is under their lips.\fnote{\fbackref{3:13} Cf. Ps 140:3} \\
\poeml \v{14}Their mouths are full of cursing and bitterness.\fnote{\fbackref{3:14} Cf. Ps 10:7} \\
\poeml \v{15}They run swiftly\fnote{\fbackref{3:15} Lit. \fbib{Their feet are swift}} to shed blood. \\
\poeml \v{16}Ruin and misery characterize their lives. \\
\poeml \v{17}They have not learned the path to peace.\fnote{\fbackref{3:17} Cf. Isa 59:7-8; Prov 1:16} \\
\poeml \v{18}They don't fear God.\fnote{\fbackref{3:18} Lit. \fbib{God before their eyes}}
\end{poetry}

\v{19}Now we know that whatever the Law says applies to those who are under the Law, so that every mouth may be silenced and the whole world held accountable to God. \v{20}Therefore, God\fnote{\fbackref{3:20} The Gk. lacks \fbib{God}} will not justify any human being by means of the actions prescribed by the Law, for through the Law comes the full knowledge of sin.
\passage{God Gives Us Righteousness through Faith}

\v{21}But now, apart from the Law, God's righteousness is revealed and is attested by the Law and the Prophets--- \v{22}God's righteousness through the faithfulness of Jesus\fnote{\fbackref{3:22} Or \fbib{through faith in Jesus}} the Messiah\fnote{\fbackref{3:22} Or \fbib{Christ}}--- for all who believe. For there is no distinction among people,\fnote{\fbackref{3:22} The Gk. lacks \fbib{among people}} \v{23}since all have sinned and continue to fall short of God's glory. \v{24}By his grace they are justified freely through the redemption that is in the Messiah\fnote{\fbackref{3:24} Or \fbib{Christ}} Jesus, \v{25}whom God offered as a place where atonement by the Messiah's\fnote{\fbackref{3:25} Lit. \fbib{by his}} blood would occur through faith. He did this\fnote{\fbackref{3:25} The Gk. lacks \fbib{He did this}} to demonstrate his righteousness, because he had waited patiently to deal with sins committed in the past. \v{26}He wanted\fnote{\fbackref{3:26} The Gk. lacks \fbib{He wanted}} to demonstrate at the present time that he himself is righteous and that he justifies anyone who has the faithfulness of Jesus.\fnote{\fbackref{3:26} Or \fbib{faith in Jesus}}

\v{27}What, then, is there to boast about? That has been eliminated. On what principle? On that of actions? No, but on the principle of faith. \v{28}For\fnote{\fbackref{3:28} Other mss. read \fbib{Therefore}} we maintain that a person is justified by faith apart from the actions prescribed by the Law. \v{29}Is God the God of the Jews only? Is he not the God of the gentiles, too? Yes, of the gentiles, too, \v{30}since there is only one God who will justify the circumcised on the basis of faith and the uncircumcised by that same faith. \v{31}Do we, then, abolish the Law by this faith? Of course not! Instead, we uphold the Law.
\labelchapt{4}
\passage{The Example of Abraham}

\chapt{4}
\v{1}What, then, are we to say about Abraham, our human ancestor? \v{2}For if Abraham was justified by actions, he would have had something to boast about---though not before God. \v{3}For what does the Scripture say? ``Abraham believed God, and it was credited to him as righteousness.''\fnote{\fbackref{4:3} Cf. Gen 15:6}

\v{4}Now to someone who works, wages are not considered a gift but an obligation. \v{5}However, to someone who does not work, but simply believes in the one who justifies the ungodly, his faith is credited as righteousness. \v{6}Likewise, David also speaks of the blessedness of the person whom God regards as righteous apart from actions:

\begin{poetry}
\poeml \v{7}``How blessed are those whose iniquities are forgiven \\
\poemll    and whose sins are covered! \\
\poeml \v{8}How blessed is the person whose sins \\
\poemll    the Lord\fnote{\fbackref{4:8} MT source citation reads \fbib{\divine{Lord}}} will never charge against him!''\fnote{\fbackref{4:8} Ps Cf. 32:1-2}
\end{poetry}

\v{9}Now does this blessedness come to the circumcised alone, or also to the uncircumcised? For we say, ``Abraham's faith was credited to him as righteousness.''\fnote{\fbackref{4:9} Gen Cf. 15:6} \v{10}Under what circumstances was it credited? Was he circumcised or uncircumcised? He had not yet been circumcised, but was uncircumcised. \v{11}Afterward he received the mark of circumcision as a seal of the righteousness that he had by faith while he was still uncircumcised. Therefore, he is the ancestor of all who believe while uncircumcised, in order that righteousness may be credited to them. \v{12}He is also the ancestor of the circumcised---those who are not only circumcised, but who also walk in the footsteps of the faith that our father Abraham had before he was circumcised.
\passage{The Promise Comes through Faith}

\v{13}For the promise that he would inherit the world did not come to Abraham or to his descendants through the Law, but through the righteousness produced by faith. \v{14}For if those who were given the Law\fnote{\fbackref{4:14} Lit. \fbib{those of the law}} are the heirs, then faith is useless and the promise is worthless, \v{15}for the Law produces wrath. Now where there is no Law, neither can there be any violation of it.

\v{16}Therefore, the promise\fnote{\fbackref{4:16} Lit. \fbib{it}} is based on faith, so that it may be a matter of grace and may be guaranteed for all of Abraham's\fnote{\fbackref{4:16} Lit. \fbib{his}} descendants---not only for those who were given the Law,\fnote{\fbackref{4:16} Lit. \fbib{those of the law}} but also for those who share the faith of Abraham, who is the father of us all. \v{17}As it is written, ``I have made you the father of many nations.''\fnote{\fbackref{4:17} Cf. Gen 17:5} Abraham\fnote{\fbackref{4:17} Lit. \fbib{He}} acted in faith when he stood in the presence of God, who gives life to the dead and calls into existence things that don't yet exist. \v{18}Hoping in spite of hopeless circumstances, he believed that he would become ``the father of many nations,''\fnote{\fbackref{4:18} Cf. Gen 17:5} just as he had been told:\fnote{\fbackref{4:18} Lit. \fbib{according to what was said}} ``This is how many descendants you will have.''\fnote{\fbackref{4:18} Gen 15:5} \v{19}His faith did not weaken when he thought about his own body (which was already\fnote{\fbackref{4:19} Other mss. lack \fbib{already}} as good as dead now that he was about a hundred years old) or about Sarah's inability to have children, \v{20}nor did he doubt God's promise out of a lack of faith. Instead, his faith became stronger and he gave glory to God, \v{21}being absolutely convinced that God would do what he had promised. \v{22}This is why ``it was credited to him as righteousness.''\fnote{\fbackref{4:22} Gen Cf. 15:6}

\v{23}Now the words ``it was credited to him'' were written not only for him \v{24}but also for us. Our faith will be regarded in the same way,\fnote{\fbackref{4:24} Lit. \fbib{It will be regarded}} if we believe in the one who raised Jesus our Lord from the dead. \v{25}He was sentenced to death because of our sins and raised to life to justify us.
\labelchapt{5}
\passage{We Enjoy Peace with God through Jesus}

\chapt{5}
\v{1}Therefore, since we have been justified by faith, we have\fnote{\fbackref{5:1} Other mss. read \fbib{let's have}} peace with God through our Lord Jesus the Messiah.\fnote{\fbackref{5:1} Or \fbib{Christ}} \v{2}Through him we have also obtained\fnote{\fbackref{5:2} Or \fbib{let's also obtain}} access by faith\fnote{\fbackref{5:2} Other mss. lack \fbib{by faith}} into this grace by which we have been established, and we boast\fnote{\fbackref{5:2} Or \fbib{let's boast}} because of our hope in God's glory. \v{3}Not only that, but we also boast\fnote{\fbackref{5:3} Or \fbib{let's also boast}} in our sufferings, knowing that suffering produces endurance, \v{4}endurance produces character, and character produces hope. \v{5}Now this hope does not disappoint us, because God's love has been poured out into our hearts by the Holy Spirit, who has been given to us.

\v{6}For at just the right time, while we were still powerless,\fnote{\fbackref{5:6} Or \fbib{weak}} the Messiah\fnote{\fbackref{5:6} Or \fbib{Christ}} died for the ungodly. \v{7}For it is rare for anyone to die for a righteous person, though somebody might be brave enough to die for a good person. \v{8}But God demonstrates his love for us by the fact that the Messiah\fnote{\fbackref{5:8} Or \fbib{Christ}} died for us while we were still sinners.

\v{9}Now that we have been justified by his blood, how much more will we be saved from wrath through him! \v{10}For if, while we were enemies, we were reconciled to God through the death of his Son, how much more, having been reconciled, will we be saved by his life! \v{11}Not only that, but we also continue to boast about God through our Lord Jesus the Messiah,\fnote{\fbackref{5:11} Or \fbib{Christ}} through whom we have now been reconciled.
\passage{Death in Adam, Life in the Messiah}

\v{12}Just as sin entered the world through one man, and death resulted from sin, therefore everyone dies, because everyone has sinned. \v{13}Certainly sin was in the world before the Law was given,\fnote{\fbackref{5:13} The Gk. lacks \fbib{was given}} but no record of sin is kept when there is no Law. \v{14}Nevertheless, death ruled from the time of\fnote{\fbackref{5:14} The Gk. lacks \fbib{the time of}} Adam to Moses, even over those who did not sin in the same way Adam did when he disobeyed.\fnote{\fbackref{5:14} Lit. \fbib{in the likeness of Adam's disobedience}} He is a foreshadowing of the one who would come.

\v{15}But God's free gift\fnote{\fbackref{5:15} Lit. \fbib{But the free gift}} is not like Adam's offense.\fnote{\fbackref{5:15} Lit. \fbib{like the offense}} For if many people died as the result of one man's offense, how much more have God's grace and the free gift given through the kindness of one man, Jesus the Messiah,\fnote{\fbackref{5:15} Or \fbib{Christ}} been showered on many people! \v{16}Nor can the free gift be compared to what came through the man who sinned.\fnote{\fbackref{5:16} Lit. \fbib{nor is the gift like the man who sinned}} For the sentence that followed one man's offense resulted in condemnation, but the free gift brought justification, even after many offenses. \v{17}For if, through one man, death ruled because of that man's offense, how much more will those who receive such overflowing grace and the gift of righteousness rule in life because of one man, Jesus the Messiah!\fnote{\fbackref{5:17} Or \fbib{Christ}}

\v{18}Consequently, just as one offense resulted in condemnation for everyone, so one act of righteousness results in justification and life for everyone. \v{19}For just as through one man's disobedience many people were made sinners, so also through one man's obedience many people will be made righteous. \v{20}Now the Law crept in so that the offense would increase. But where sin increased, grace increased even more, \v{21}so that, just as sin ruled by bringing death,\fnote{\fbackref{5:21} Lit. \fbib{ruled in death}} so also grace might rule by bringing justification\fnote{\fbackref{5:21} Lit. \fbib{through justification}} that results in eternal life through Jesus the Messiah,\fnote{\fbackref{5:21} Or \fbib{Christ}} our Lord.
\labelchapt{6}
\passage{No Longer Sin's Slaves, but God's Slaves}

\chapt{6}
\v{1}What should we say, then? Should we go on sinning so that grace may increase? \v{2}Of course not! How can we who died as far as sin is concerned go on living in it?

\v{3}Or don't you know that all of us who were baptized into union with the Messiah\fnote{\fbackref{6:3} Or \fbib{Christ}} Jesus were baptized into his death? \v{4}Therefore, through baptism we were buried with him into his death so that, just as the Messiah\fnote{\fbackref{6:4} Or \fbib{Christ}} was raised from the dead by the Father's glory, we too may live an entirely new life. \v{5}For if we have become united with him in a death like his, we will certainly also be united with him in a resurrection like his. \v{6}We know that our old natures were crucified with him so that our sin-laden bodies might be rendered powerless and we might no longer be slaves to sin. \v{7}For the person who has died has been freed from sin.

\v{8}Now if we have died with the Messiah,\fnote{\fbackref{6:8} Or \fbib{Christ}} we believe that we will also live with him, \v{9}for we know that the Messiah,\fnote{\fbackref{6:9} Or \fbib{Christ}} who was raised from the dead, will never die again; death no longer has mastery over him. \v{10}For when he died, he died once and for all as far as sin is concerned. But now that he is alive, he lives for God. \v{11}In the same way, you too must continuously consider yourselves dead as far as sin is concerned, but living for God through the Messiah\fnote{\fbackref{6:11} Or \fbib{Christ}} Jesus.\fnote{\fbackref{6:11} Other mss. read \fbib{the Messiah Jesus our Lord}}

\v{12}Therefore, do not let sin rule your mortal bodies so that you obey their desires. \v{13}Stop offering\fnote{\fbackref{6:13} Or \fbib{Don't offer}} the parts of your body\fnote{\fbackref{6:13} Lit. \fbib{your members}} to sin as instruments of unrighteousness. Instead, offer yourselves to God as people who have been brought from death to life and the parts of your body\fnote{\fbackref{6:13} Lit. \fbib{your members}} as instruments of righteousness to God. \v{14}For sin will not have mastery over you, because you are not under Law but under grace.

\v{15}What, then, does this mean?\fnote{\fbackref{6:15} The Gk. lacks \fbib{does this mean}} Should we go on sinning because we are not under Law but under grace? Of course not! \v{16}Don't you know that when you offer yourselves to someone as obedient slaves, you are slaves of the one you obey---either of sin, which leads to death, or of obedience, which leads to righteousness? \v{17}But thank God that, though you were once slaves of sin, you became obedient from your hearts to that form of teaching with which you were entrusted! \v{18}And since you have been freed from sin, you have become slaves of righteousness.

\v{19}I am speaking in simple\fnote{\fbackref{6:19} Lit. \fbib{human}} terms because of the frailty of your human nature.\fnote{\fbackref{6:19} Lit. \fbib{your flesh}} Just as you once offered the parts of your body\fnote{\fbackref{6:19} Lit. \fbib{your members}} as slaves to impurity and to greater and greater disobedience, so now, in the same way, you must offer the parts of your body\fnote{\fbackref{6:19} Lit. \fbib{your members}} as slaves to righteousness that leads to sanctification. \v{20}For when you were slaves of sin, you were ``free'' as far as righteousness was concerned. \v{21}What benefit did you get from doing those things you are now ashamed of? For those things resulted in death. \v{22}But now that you have been freed from sin and have become God's slaves, the benefit you reap is sanctification, and the result is eternal life. \v{23}For the wages of sin is death, but the free gift of God is eternal life in union with the Messiah\fnote{\fbackref{6:23} Or \fbib{Christ}} Jesus our Lord.
\labelchapt{7}
\passage{Now We are Released from the Law}

\chapt{7}
\v{1}Don't you realize, brothers---for I am speaking to people who know the Law---that the Law can press its claims over a person only as long as he is alive? \v{2}For a married woman is bound by the Law to her husband while he is living, but if her husband dies, she is released from the Law concerning her husband. \v{3}So while her husband is living, she will be called an adulterer if she lives with another man. But if her husband dies, she is free from this Law, so that she is not an adulterer if she marries another man.

\v{4}In the same way, my brothers, through the Messiah's\fnote{\fbackref{7:4} Or \fbib{Christ's}} body you also died as far as the Law is concerned, so that you may belong to another person, the one who was raised from the dead, and may bear fruit for God. \v{5}For while we were living according to our human nature,\fnote{\fbackref{7:5} Lit. \fbib{our flesh}} sinful passions were at work in our bodies\fnote{\fbackref{7:5} Lit. \fbib{members}} by means of the Law, to bear fruit resulting in death. \v{6}But now we have been released from the Law by dying to what enslaved us, so that we may serve in the new life of the Spirit, not under the old writings.
\passage{The Law Shows Us What Sin Is}

\v{7}What should we say, then? Is the Law sinful? Of course not! In fact, I wouldn't have become aware of sin if it had not been for the Law. I wouldn't have known what it means to covet if the Law had not said, ``You must not covet.''\fnote{\fbackref{7:7} Cf. Exod 20:17} \v{8}But sin seized the opportunity provided by this commandment and produced in me all kinds of sinful desires, since apart from the Law, sin is dead. \v{9}At one time I was alive without any connection to\fnote{\fbackref{7:9} The Gk. lacks \fbib{any connection to}} the Law.\fnote{\fbackref{7:9} Or \fbib{instruction}} But when the rule was revealed, sin sprang to life, \v{10}and I died. I found that the very rule that was intended to bring life actually brought death. \v{11}For sin, seizing the opportunity provided by the rule, deceived me and used it to kill me. \v{12}So then, the Law\fnote{\fbackref{7:12} Or \fbib{instruction}} itself is holy, and the rule is holy, just, and good.
\passage{The Problem of the Sin that Lives in Us}

\v{13}Now, did something good bring me death? Of course not! But in order that sin might be recognized as being sin, it used something good to cause my death, so that through the rule, sin might become more exposed as being\fnote{\fbackref{7:13} The Gk. lacks \fbib{exposed as being}} sinful than ever before. \v{14}For we know that the Law is spiritual, but I am merely human,\fnote{\fbackref{7:14} Lit. \fbib{am flesh}} sold as a slave to sin.\fnote{\fbackref{7:14} Lit. \fbib{sold under sin}} \v{15}I don't understand what I am doing. For I don't practice what I want to do, but instead do what I hate. \v{16}Now if I practice what I don't want to do, I am admitting that the Law is good. \v{17}As it is, I am no longer the one who is doing it, but it is the sin that is living in me.

\v{18}For I know that nothing good lives in me, that is, in my flesh. For I have the desire to do what is right, but I cannot carry it out. \v{19}For I don't do the good I want to do, but instead do the evil that I don't want to do. \v{20}But if I do what I don't want to do, I am no longer the one who is doing it, but it is the sin that is living in me.

\v{21}So I find this to be a principle:\fnote{\fbackref{7:21} Lit. \fbib{law}} when I want to do what is good, evil is right there with me. \v{22}For I delight in the Law of God in my inner being, \v{23}but I see in my body\fnote{\fbackref{7:23} Lit. \fbib{in my members}} a different principle\fnote{\fbackref{7:23} Lit. \fbib{law}} waging war with the Law in my mind and making me a prisoner of the law of sin that exists in my body.\fnote{\fbackref{7:23} Lit. \fbib{in my members}} \v{24}What a wretched man I am! Who will rescue me from this body that is infected by\fnote{\fbackref{7:24} Lit. \fbib{body of death}} death? \v{25}Thank God through Jesus the Messiah,\fnote{\fbackref{7:25} Or \fbib{Christ}} our Lord, because with my mind I myself can serve the Law of God, even while with my human nature\fnote{\fbackref{7:25} Lit. \fbib{my flesh}} I serve the law of sin.
\labelchapt{8}
\passage{The Spirit Gives Life}

\chapt{8}
\v{1}Therefore, there is now no condemnation for those who are in union with the Messiah\fnote{\fbackref{8:1} Or \fbib{Christ}} Jesus.\fnote{\fbackref{8:1} Other mss. read \fbib{Jesus, who do not live according to the flesh but according to the Spirit}} \v{2}For the Spirit's law of life in the Messiah\fnote{\fbackref{8:2} Or \fbib{Christ}} Jesus has set me\fnote{\fbackref{8:2} Other mss. read \fbib{you}} free from the Law of sin and death. \v{3}For what the Law was powerless to do in that it was weakened by the flesh, God did. By sending his own Son in the form of humanity,\fnote{\fbackref{8:3} Lit. \fbib{of the flesh}} he condemned sin by being incarnate, \v{4}so that the righteous requirement of the Law might be fulfilled in us, who do not live according to human nature but according to the Spirit.

\v{5}For those who live according to the flesh set their minds on the things of the flesh, but those who live according to the Spirit set their minds on the things of the Spirit. \v{6}To focus our minds on the human nature leads to death, but to focus our minds on the Spirit leads to life and peace. \v{7}That is why the mind that focuses on human nature is hostile toward God. It refuses to submit to the authority of God's Law because it is powerless to do so. \v{8}Indeed, those who are under the control of human nature cannot please God.

\v{9}You, however, are not under the control of the human nature but under the control of the Spirit, since God's Spirit lives in you. And if anyone does not have the Spirit of the Messiah,\fnote{\fbackref{8:9} Or \fbib{Christ}} he does not belong to him. \v{10}But if the Messiah\fnote{\fbackref{8:10} Or \fbib{Christ}} is in you, your bodies are dead due to sin, but the spirit\fnote{\fbackref{8:10} Or \fbib{Spirit}} is alive due to righteousness. \v{11}And if the Spirit of the one who raised Jesus from the dead is living in you, then the one who raised the Messiah\fnote{\fbackref{8:11} Or \fbib{Christ}} from the dead will also make your mortal bodies alive by his Spirit who lives in you.

\v{12}Consequently, brothers, we are not---with respect to human nature, that is---under an obligation to live according to human nature. \v{13}For if you live according to human nature, you are going to die, but if by the Spirit you continuously put to death the activities of the body, you will live. \v{14}For all who are led by God's Spirit are God's children. \v{15}For you have not received a spirit of slavery that leads you into fear again. Instead, you have received the Spirit of adoption by whom we cry out, ``Abba!\fnote{\fbackref{8:15} \fbib{Abba} is Aram. for \fbib{Father.}} Father!'' \v{16}The Spirit himself testifies with our spirit that we are God's children. \v{17}Now if we are children, we are heirs---heirs of God and co-heirs with the Messiah\fnote{\fbackref{8:17} Or \fbib{Christ}}---if, in fact, we share in his sufferings in order that we may also share in his glory.
\passage{God's Spirit Helps Us}

\v{18}For I consider that the sufferings of this present time are not worth comparing with the glory that will be revealed to us. \v{19}For the creation is eagerly awaiting the revelation of God's children, \v{20}because the creation has become subject to futility, though not by anything it did.\fnote{\fbackref{8:20} Lit. \fbib{by its subjecting}} The one who subjected it did so in the certainty\fnote{\fbackref{8:20} Lit. \fbib{hope}} \v{21}that the creation itself would also be set free from corrupting bondage in order to share the glorious freedom of God's children. \v{22}For we know that all the rest of creation has been groaning with the pains of childbirth up to the present time. \v{23}However, not only the creation, but we who have the first fruits of the Spirit also groan inwardly as we eagerly await our adoption, the redemption of our bodies. \v{24}For we were saved with this hope in mind.\fnote{\fbackref{8:24} The Gk. lacks \fbib{in mind}} Now a hope that can be observed is not really hope, for who hopes for what can be seen? \v{25}But if we hope for what we do not yet observe, we eagerly wait for it with patience.

\v{26}In the same way, the Spirit also helps us in our weakness, since we do not know how to pray as we should. But the Spirit himself intercedes for us\fnote{\fbackref{8:26} Other mss. lack \fbib{for us}} with groans too deep for words, \v{27}and the one who searches our hearts knows the mind of the Spirit, for the Spirit\fnote{\fbackref{8:27} Lit. \fbib{he}} intercedes for the saints according to God's will.\fnote{\fbackref{8:27} Lit. \fbib{according to God}} \v{28}And we know that for those who love God, that is, for those who are called according to his purpose, all things are working together\fnote{\fbackref{8:28} Other mss. read \fbib{that God works all things together for good for those who love God and who are called according to his purpose}} for good.

\v{29}For those whom he foreknew he also predestined to be conformed to the image of his Son, in order that the Son\fnote{\fbackref{8:29} Lit. \fbib{that he}} might be the firstborn among many brothers. \v{30}And those whom he predestined, he also called; and those whom he called, he also justified; and those whom he justified he also glorified.
\passage{Nothing Can Separate Us from God's Love}

\v{31}What, then, can we say about all of this? If God is for us, who can be against us? \v{32}The one who did not spare his own Son, but offered him as a sacrifice\fnote{\fbackref{8:32} The Gk. lacks \fbib{as a sacrifice}} for all of us, surely will give us all things, along with his Son,\fnote{\fbackref{8:32} Lit. \fbib{with him}} won't he? \v{33}Who will accuse God's elect? It is God who justifies! \v{34}Who is the one to condemn? It is the Messiah\fnote{\fbackref{8:34} Or \fbib{Christ}} Jesus who is interceding on our behalf. He died, and more importantly, has been raised and is seated at the right hand of God.

\v{35}Who will separate us from the Messiah's\fnote{\fbackref{8:35} Or \fbib{Christ's}} love? Can trouble, distress, persecution, hunger, nakedness, danger, or a violent death\fnote{\fbackref{8:35} Lit. \fbib{a sword}} do this?\fnote{\fbackref{8:35} The Gk. lacks \fbib{do this}} \v{36}As it is written,

\begin{poetry}
\poeml ``For your sake we are being put to death all day long. \\
\poemll    We are thought of as sheep headed for slaughter.''\fnote{\fbackref{8:36} Cf. Ps 44:22}
\end{poetry}

\v{37}In all these things we are triumphantly victorious due to the one who loved us. \v{38}For I am convinced that neither death, nor life, nor angels, nor rulers, nor things present, nor things to come, nor powers, \v{39}nor anything above, nor anything below, nor anything else in all creation can separate us from the love of God that is ours\fnote{\fbackref{8:39} The Gk. lacks \fbib{ours}} in union with the Messiah\fnote{\fbackref{8:39} Or \fbib{Christ}} Jesus, our Lord.
\labelchapt{9}
\passage{Paul's Concern for the Jewish People}

\chapt{9}
\v{1}I am telling the truth because I belong to\fnote{\fbackref{9:1} Lit. \fbib{truth in}} the Messiah\fnote{\fbackref{9:1} Or \fbib{Christ}}---I am not lying, and my conscience confirms it by means of the Holy Spirit. \v{2}I have deep sorrow and unceasing anguish in my heart, \v{3}for I could wish that I myself were condemned\fnote{\fbackref{9:3} Or \fbib{accursed}} and cut off from the Messiah\fnote{\fbackref{9:3} Or \fbib{Christ}} for the sake of my brothers, my own people,\fnote{\fbackref{9:3} Lit. \fbib{own relatives according to the flesh}} \v{4}who are Israelis. To them belong the adoption, the glory, the covenants,\fnote{\fbackref{9:4} Other mss. read \fbib{the covenant}} the giving of the Law, the worship, and the promises. \v{5}To the Israelis\fnote{\fbackref{9:5} Lit. \fbib{To them}} belong the patriarchs, and from them, the Messiah\fnote{\fbackref{9:5} Or \fbib{Christ}} descended,\fnote{\fbackref{9:5} Lit. \fbib{Messiah according to the flesh}} who is God over all, the one who is forever blessed. Amen.

\v{6}Now it is not as though the word of God has failed. For not all Israelis truly belong to Israel, \v{7}and not all of Abraham's descendants are his true descendants. On the contrary, ``It is through Isaac that descendants will be named for you.''\fnote{\fbackref{9:7} Cf. Gen 21:12} \v{8}That is, it is not merely the children born through natural descent who were regarded as God's children, but it is the children born through the promise who were regarded as descendants. \v{9}For this is the language of the promise: ``At this time I will return, and Sarah will have a son.''\fnote{\fbackref{9:9} Cf. Gen 18:10, 14} \v{10}Not only that, but Rebecca became pregnant by our ancestor Isaac. \v{11}Yet before their children\fnote{\fbackref{9:11} Lit. \fbib{they}} had been born or had done anything good or bad (so that God's plan of election might continue to operate \v{12}according to his calling and not by actions), Rebecca\fnote{\fbackref{9:12} Lit. \fbib{she}} was told, ``The older child will serve the younger one.''\fnote{\fbackref{9:12} Cf. Gen 25:23} \v{13}So it is written, ``Jacob I loved, but Esau I hated.''\fnote{\fbackref{9:13} Cf. Mal 1:2-3}

\v{14}What can we say, then? God is not unrighteous, is he? Of course not! \v{15}For he says to Moses, ``I will be merciful to the person I want to be merciful to, and I will be kind to the person I want to be kind to.''\fnote{\fbackref{9:15} Cf. Exod 33:19} \v{16}Therefore, God's choice\fnote{\fbackref{9:16} Lit. \fbib{it}} does not depend on a person's will or effort, but on God himself, who shows mercy. \v{17}For the Scripture says about Pharaoh,

\begin{poetry}
\poeml ``I have raised you up for this very purpose, \\
\poemll    to demonstrate my power through you \\
\poeml and that my name might be proclaimed \\
\poemll    in all the earth.''\fnote{\fbackref{9:17} Cf. Exod 9:16}
\end{poetry}

\v{18}Therefore, God\fnote{\fbackref{9:18} Lit. \fbib{he}} has mercy on whomever he chooses, and he hardens the heart of whomever he chooses.
\passage{God Chose People who are Not Jewish}

\v{19}You may ask me, ``Then why does God\fnote{\fbackref{9:19} Lit. \fbib{he}} still find fault with anybody?\fnote{\fbackref{9:19} The Gk. lacks \fbib{with anybody}} For who can resist his will?'' \v{20}On the contrary, who are you---mere man that you are---to talk back to God? Can an object that was molded say to the one who molded it, ``Why did you make me like this?'' \v{21}A potter has the right to do what he wants to with his clay, doesn't he? He can make something for a special occasion or something for ordinary use from the same lump of clay.

\v{22}Now if God wants to demonstrate his wrath and reveal his power, can't he be extremely patient with the objects of his wrath that are made for destruction? \v{23}Can't he also reveal his glorious riches to the objects of his mercy that he has prepared ahead of time for glory--- \v{24}including us, whom he also called, not only from the Jews but from the gentiles as well? \v{25}As the Scripture\fnote{\fbackref{9:25} Lit. \fbib{As it}} says in Hosea,

\begin{poetry}
\poeml ``Those who are not my people \\
\poemll    I will call my people, \\
\poeml and the one who was not loved \\
\poemll    I will call my loved one.\fnote{\fbackref{9:25} Cf. Hos 2:23} \\
\poeml \v{26}In the very place where it was told them, \\
\poemll    `You are not my people,' \\
\poemlll       they will be called children of the living God.''\fnote{\fbackref{9:26} Cf. Hos 1:10}
\end{poetry}

\v{27}Isaiah also calls out concerning Israel,

\begin{poetry}
\poeml ``Although the descendants of Israel \\
\poemll    are as numerous as the grains of sand on the seashore, \\
\poemlll       only a few will be saved. \\
\poeml \v{28}For the Lord\fnote{\fbackref{9:28} MT source citation reads \fbib{\divine{Lord}}} will carry out his plan decisively, \\
\poemll    bringing it to completion on the earth.''\fnote{\fbackref{9:28} Cf. Isa 10:22-23}
\end{poetry}

\v{29}It is just as Isaiah predicted:

\begin{poetry}
\poeml ``If the Lord of the Heavenly Armies \\
\poemll    had not left us some descendants, \\
\poemlll       we would have become like Sodom \\
\poemlll       and would have been compared to Gomorrah.''\fnote{\fbackref{9:29} Cf. Isa 1:9}
\end{poetry}

\v{30}What can we say, then? Gentiles, who were not pursuing righteousness, have attained righteousness, a righteousness that comes through faith. \v{31}But Israel, who pursued righteousness based on the Law, did not achieve the Law. \v{32}Why not? Because they did not pursue it on the basis of faith, but as if it were based on achievements. They stumbled over the stone that causes people to stumble. \v{33}As it is written,

\begin{poetry}
\poeml ``Look! I am placing a stone in Zion \\
\poemll    over which people will stumble--- \\
\poeml a large rock that will make them fall--- \\
\poemll    and the one who believes in him will never be ashamed.''\fnote{\fbackref{9:33} Cf. Isa 28:16}
\end{poetry}
\labelchapt{10}
\passage{The Person who Believes will be Saved}

\chapt{10}
\v{1}Brothers, my heart's desire and prayer to God about the Jews\fnote{\fbackref{10:1} Lit. \fbib{on behalf of them}} is that they would be saved. \v{2}For I can testify on their behalf that they have a zeal for God, but it is not in keeping with full knowledge. \v{3}For they are ignorant of the righteousness that comes from God while they try to establish their own, and they have not submitted to God's means to attain\fnote{\fbackref{10:3} The Gk. lacks \fbib{means to attain}} righteousness. \v{4}For the Messiah\fnote{\fbackref{10:4} Or \fbib{Christ}} is the culmination\fnote{\fbackref{10:4} Or \fbib{end}} of the Law as far as righteousness is concerned for everyone who believes.

\v{5}For Moses writes about the righteousness that comes from the Law as follows: ``The person who obeys these things will find life by them.''\fnote{\fbackref{10:5} Lev 18:5} \v{6}But the righteousness that comes from faith says, ``Do not say in your heart, `Who will go up to heaven?' (that is, to bring the Messiah\fnote{\fbackref{10:6} Or \fbib{Christ}} down), \v{7}or `Who will go down into the depths?' (that is, to bring the Messiah\fnote{\fbackref{10:7} Or \fbib{Christ}} back from the dead).''

\v{8}But what does it say? ``The message is near you. It is in your mouth and in your heart.''\fnote{\fbackref{10:8} Cf. Deut 9:4; 30:12-14} This is the message about faith that we are proclaiming: \v{9}If you declare with your mouth that Jesus is Lord, and believe in your heart that God raised him from the dead, you will be saved. \v{10}For one believes with his heart and is justified, and declares with his mouth and is saved. \v{11}The Scripture says, ``Everyone who believes in him will never be ashamed.''\fnote{\fbackref{10:11} Isa 28:16} \v{12}There is no difference between Jew and Greek, because they all have the same Lord, who gives richly to all who call on him. \v{13}``Everyone who calls on the name of the Lord\fnote{\fbackref{10:13} MT source citation reads \fbib{\divine{Lord}}} will be saved.''\fnote{\fbackref{10:13} Cf. Joel 2:32}

\v{14}How, then, can people\fnote{\fbackref{10:14} Lit. \fbib{they}} call on someone they have not believed? And how can they believe in someone they have not heard about? And how can they hear without someone preaching? \v{15}And how can people\fnote{\fbackref{10:15} Lit. \fbib{they}} preach unless they are sent? As it is written, ``How beautiful are\fnote{\fbackref{10:15} Lit. \fbib{are the feet of}} those who bring the good news!''\fnote{\fbackref{10:15} Isa 52:7} \v{16}But not everyone has obeyed the gospel, for Isaiah asks, ``Lord, who has believed our message?''\fnote{\fbackref{10:16} Isa 53:1} \v{17}Consequently, faith results from listening, and listening results through the word of the Messiah.\fnote{\fbackref{10:17} Or \fbib{Christ}; other mss. read \fbib{of God}}

\v{18}But I ask, ``Didn't they hear?'' Certainly they did! In fact,

\begin{poetry}
\poeml ``Their voice has gone out into the whole world, \\
\poemll    and their words to the ends of the earth.''\fnote{\fbackref{10:18} Cf. Ps 19:4}
\end{poetry}

\v{19}Again I ask, ``Did Israel not understand?'' Moses was the first to say,

\begin{poetry}
\poeml ``I will make you jealous \\
\poemll    by those who are not a nation; \\
\poeml I will make you angry \\
\poemll    by a nation that doesn't understand.''\fnote{\fbackref{10:19} Cf. Deut 32:21}
\end{poetry}

\v{20}And Isaiah boldly says,

\begin{poetry}
\poeml ``I was found by those who were not looking for me; \\
\poemll    I was revealed to those who were not asking for me.''\fnote{\fbackref{10:20} Cf. Isa 65:1}
\end{poetry}

\v{21}But about Israel he says,

\begin{poetry}
\poeml ``All day long I have held out my hands \\
\poemll    to a disobedient and rebellious people.''\fnote{\fbackref{10:21} Cf. Isa 65:2 (LXX)}
\end{poetry}
\labelchapt{11}
\passage{God's Love for His People}

\chapt{11}
\v{1}So I ask, ``God has not rejected his people, has he?'' Of course not! I am an Israeli myself, a descendant of Abraham from the tribe of Benjamin. \v{2}God has not rejected his people whom he chose\fnote{\fbackref{11:2} Lit. \fbib{knew}} long ago. Do you not know what the Scripture says in the story about Elijah,\fnote{\fbackref{11:2} The Gk. lacks \fbib{the story about}} when he pleads with God against Israel? \v{3}``Lord, they have killed your prophets and demolished your altars. I am the only one left, and they are trying to take my life.''\fnote{\fbackref{11:3} Cf. 1 Kings 19:10, 14} \v{4}But what was the divine reply to him? ``I have reserved for myself 7,000 people who have not knelt to worship Baal.''\fnote{\fbackref{11:4} Cf. 1 Kings 19:18} \v{5}So it is at the present time: there is a remnant, chosen by grace. \v{6}But if this is by grace, then it is no longer on the basis of actions. Otherwise, grace would no longer be grace.

\v{7}What, then, does this mean?\fnote{\fbackref{11:7} The Gk. lacks \fbib{does this mean}} It means that Israel failed to obtain what it was seeking, but the selected group obtained it while the rest were hardened. \v{8}As it is written,

\begin{poetry}
\poeml ``To this day God has put them into\fnote{\fbackref{11:8} Lit. \fbib{has given them a spirit of}} deep sleep. \\
\poemll    Their eyes do not see, and their ears do not hear.''\fnote{\fbackref{11:8} Cf. Deut 29:4; Isa 29:10}
\end{poetry}

\v{9}And David says,

\begin{poetry}
\poeml ``Let their table become a snare and a trap, \\
\poemll    a stumbling block and a punishment for them. \\
\poeml \v{10}Let their eyes be darkened so that they cannot see, \\
\poemll    and keep their backs forever bent.''\fnote{\fbackref{11:10} Cf. Ps 69:22-23; 35:8}
\end{poetry}
\passage{The Salvation of the Gentiles}

\v{11}And so I ask, ``They have not stumbled so as to fall, have they?'' Of course not! On the contrary, because of their stumbling, salvation has come to the gentiles to make the Jews\fnote{\fbackref{11:11} Lit. \fbib{them}} jealous. \v{12}Now if their stumbling means riches for the world, and if their fall means riches for the gentiles, how much more will their full participation mean!

\v{13}I am speaking to you gentiles. Because I am an apostle to the gentiles, I magnify my ministry \v{14}in the hope that I can make my people\fnote{\fbackref{11:14} Lit. \fbib{flesh}} jealous and save some of them. \v{15}For if their rejection results in reconciliation of the world, what will their acceptance bring but life from the dead? \v{16}If the first part of the dough is holy, so is the whole batch. If the root is holy, so are the branches.

\v{17}Now if some of the branches have been broken off, and you, a wild olive branch, have been grafted in their place to share the rich root of the olive tree, \v{18}do not boast about being better than\fnote{\fbackref{11:18} The Gk. lacks \fbib{being better than}} the other\fnote{\fbackref{11:18} The Gk. lacks \fbib{other}} branches. If you boast, remember that you do not support the root, but the root supports you. \v{19}Then you will say, ``Branches were cut off so that I could be grafted in.'' \v{20}That's right! They were broken off because of their unbelief, but you remain only because of faith. Do not be arrogant, but be afraid!\fnote{\fbackref{11:20} Or \fbib{be reverent}} \v{21}For if God did not spare the natural branches, he certainly will not spare you, either.

\v{22}Consider, then, the kindness and severity of God: his severity toward those who fell, but God's kindness toward you---if you continue receiving his kindness. Otherwise, you too will be cut off. \v{23}If the Jews\fnote{\fbackref{11:23} Lit. \fbib{they}} do not persist in their unbelief, they will be grafted in again, because God is able to graft them in. \v{24}After all, if you were cut off from what is naturally a wild olive tree, and contrary to nature were grafted into a cultivated olive tree, how much easier it will be for these natural branches to be grafted back into their own olive tree!
\passage{The Restoration of Israel}

\v{25}For I want to let you know about this secret, brothers, so that you will not claim to be wiser than you are: Stubbornness has come to part of Israel until the full number of the gentiles comes to faith.\fnote{\fbackref{11:25} The Gk. lacks \fbib{to faith}} \v{26}In this way, all Israel will be saved, as it is written,

\begin{poetry}
\poeml ``The Deliverer will come from Zion; \\
\poemll    he will remove ungodliness from Jacob. \\
\poeml \v{27}This is my covenant with them \\
\poemll    when I take away their sins.''\fnote{\fbackref{11:27} Cf. Isa 59:20-21}
\end{poetry}

\v{28}As far as the gospel is concerned, they are enemies for your sake, but as far as election is concerned, they are loved for the sake of their ancestors. \v{29}For God's gifts and calling never change. \v{30}For just as you disobeyed God in the past but now have received his mercy because of their disobedience, \v{31}so they, too, have now disobeyed. As a result, they may\fnote{\fbackref{11:31} Other mss. read \fbib{may now}} receive mercy because of the mercy shown to you. \v{32}For God has locked all people in the prison of their own disobedience so that he may have mercy on them all.
\passage{In Praise of God's Ways}

\begin{poetry}
\poeml \v{33}O how deep are God's riches, \\
\poemll    and wisdom, and knowledge! \\
\poeml How unfathomable are his decisions \\
\poemll    and unexplainable are his ways! \\
\poeml \v{34}Who has known the mind of the Lord? \\
\poemll    Or who has become his advisor?\fnote{\fbackref{11:34} Cf. Isa 40:13 (LXX)} \\
\poeml \v{35}Or who has given him something \\
\poemll    only to have him pay it back?''\fnote{\fbackref{11:35} Cf. Job 41:11} \\
\poeml \v{36}For all things are from him, by him, and for him. \\
\poemll    Glory belongs to him forever! Amen.
\end{poetry}
\labelchapt{12}
\passage{Dedicate Your Lives to God}

\chapt{12}
\v{1}I therefore urge you, brothers, in view of God's mercies, to offer your bodies as living sacrifices that are holy and pleasing to God, for this is the reasonable way for you to worship.\fnote{\fbackref{12:1} Lit. \fbib{to God, your reasonable worship}} \v{2}Do not be conformed to this world, but continuously be transformed by the renewing of your minds so that you may be able to determine what God's will is---what is proper,\fnote{\fbackref{12:2} Or \fbib{good}} pleasing, and perfect.

\v{3}For by the grace given to me I ask every one of you not to think of yourself more highly than you should think, rather to think of yourself with sober judgment on the measure of faith that God has assigned each of you. \v{4}For we have many parts in one body, but these parts do not all have the same function. \v{5}In the same way, even though we are many people, we are one body in the Messiah\fnote{\fbackref{12:5} Or \fbib{Christ}} and individual parts connected to each other. \v{6}We have different gifts based on the grace that was given to us. So if your gift is prophecy, use your gift\fnote{\fbackref{12:6} Lit. \fbib{If prophecy}} in proportion to your faith. \v{7}If your gift is serving, devote yourself to serving others.\fnote{\fbackref{12:7} Lit. \fbib{If serving, in serving}} If it is teaching, devote yourself to teaching others.\fnote{\fbackref{12:7} Lit. \fbib{If teaching, in teaching}} \v{8}If it is encouraging, devote yourself to encouraging others.\fnote{\fbackref{12:8} Lit. \fbib{If encouraging, in encouragement}} If it is sharing, share generously.\fnote{\fbackref{12:8} Lit. \fbib{The one who shares, with generosity}} If it is leading, lead enthusiastically.\fnote{\fbackref{12:8} Lit. \fbib{The one who leads, with enthusiasm}} If it is helping, help cheerfully.\fnote{\fbackref{12:8} Lit. \fbib{The one who helps, with cheerfulness}}

\v{9}Your love must be without hypocrisy. Abhor what is evil; cling to what is good. \v{10}Be devoted to each other with mutual affection. Excel at showing respect for each other. \v{11}Never be lazy in showing such devotion. Be on fire with the Spirit. Serve the Lord.\fnote{\fbackref{12:11} Other mss. read \fbib{the time}} \v{12}Be joyful in hope, patient in trouble, and persistent in prayer. \v{13}Supply the needs of the saints. Extend hospitality to strangers.

\v{14}Bless those who persecute you. Keep on blessing them, and never curse them. \v{15}Rejoice with those who are rejoicing. Cry with those who are crying. \v{16}Live in harmony with each other. Do not be arrogant, but associate with humble people. Do not think that you are wiser than you really are.

\v{17}Do not pay anyone back evil for evil, but\fnote{\fbackref{12:17} The Gk. lacks \fbib{but}} focus your thoughts on what is right in the sight of all people. \v{18}If possible, so far as it depends on you, live in peace with all people. \v{19}Do not take revenge, dear friends, but leave room for God's\fnote{\fbackref{12:19} The Gk. lacks \fbib{God's}} wrath. For it is written, ``Vengeance belongs to me. I will pay them back, declares the Lord.''\fnote{\fbackref{12:19} Cf. Deut 32:35; MT source citation reads \fbib{}\divine{Lord}} \v{20}But ``if your enemy is hungry, feed him. For if he is thirsty, give him a drink. If you do this, you will pile burning coals on his head.''\fnote{\fbackref{12:20} Cf. Prov 25:21-22} \v{21}Do not be conquered by evil, but conquer evil with good.
\labelchapt{13}
\passage{Obey Your Government}

\chapt{13}
\v{1}Every person must be subject to the governing authorities, for no authority exists except by God's permission.\fnote{\fbackref{13:1} Lit. \fbib{except by God}} The existing authorities have been established by God, \v{2}so that whoever resists the authorities opposes what God has established, and those who resist will bring judgment on themselves. \v{3}For the authorities are not a terror to good conduct, but to bad. Would you like to live without being afraid of the authorities? Then do what is right, and you will receive their approval. \v{4}For they are God's servants, working for your good.

But if you do what is wrong, you should be afraid, for it is not without reason that they bear the sword. Indeed, they are God's servants to administer punishment\fnote{\fbackref{13:4} Lit. \fbib{wrath}} to anyone who does wrong. \v{5}Therefore, it is necessary for you to be acquiescent to the authorities,\fnote{\fbackref{13:5} The Gk. lacks \fbib{to the authorities}} not only for the sake of God's\fnote{\fbackref{13:5} The Gk. lacks \fbib{God's}} punishment,\fnote{\fbackref{13:5} Lit. \fbib{wrath}} but also for the sake of your own conscience. \v{6}This is also why you pay taxes. For rulers\fnote{\fbackref{13:6} Lit. \fbib{they}} are God's servants faithfully devoting themselves to their work.\fnote{\fbackref{13:6} Lit. \fbib{to this very thing}} \v{7}Pay everyone whatever you owe them---taxes to whom taxes are due, tolls to whom tolls are due, fear\fnote{\fbackref{13:7} Or \fbib{respect}} to whom fear\fnote{\fbackref{13:7} Or \fbib{respect}} is due, honor to whom honor is due.
\passage{Love One Another}

\v{8}Do not owe anyone anything---except to love one another. For the one who loves another has fulfilled the Law. \v{9}For the commandments, ``You must not commit adultery; you must not murder; you must not steal; you must not covet,''\fnote{\fbackref{13:9} Cf. Exod 20:13-15, 17; Deut 5:17-19, 21} and every other commandment are summed up in this statement: ``You must love your neighbor as yourself.''\fnote{\fbackref{13:9} Lev 19:18} \v{10}Love never does anything that is harmful to its neighbor. Therefore, love is the fulfillment of the Law.
\passage{Live in the Light of the Messiah's Return}

\v{11}This is necessary because you know the times---it's already time for you to wake up from sleep, because our salvation is nearer now than when we became believers. \v{12}The night is almost over, and the day is near. Let's therefore put aside the actions of darkness and put on the armor of light. \v{13}Let's behave decently, as people who live in the light of day.\fnote{\fbackref{13:13} Lit. \fbib{as in the day}} No wild parties, drunkenness, sexual immorality, promiscuity, quarreling, or jealousy! \v{14}Instead, clothe yourselves with the Lord Jesus, the Messiah,\fnote{\fbackref{13:14} Or \fbib{Christ}} and do not obey your flesh and its desires.
\labelchapt{14}
\passage{How to Treat Weak Believers}

\chapt{14}
\v{1}Accept anyone who is weak in faith, but not for the purpose of arguing over differences of opinion. \v{2}One person believes that he may eat anything, while the weak\fnote{\fbackref{14:2} Or \fbib{ill}} person eats only\fnote{\fbackref{14:2} The Gk. lacks \fbib{only}} vegetables. \v{3}The person who eats any kind of food\fnote{\fbackref{14:3} The Gk. lacks \fbib{any kind of food}} must not ridicule the person who does not eat them,\fnote{\fbackref{14:3} The Gk. lacks \fbib{them}} and the person who does not eat certain foods\fnote{\fbackref{14:3} The Gk. lacks \fbib{certain foods}} must not criticize the person who eats them,\fnote{\fbackref{14:3} The Gk. lacks \fbib{them}} for God has accepted him. \v{4}Who are you to criticize someone else's servant? He stands or falls before his own Lord---and stand he will, because the Lord\fnote{\fbackref{14:4} Other mss. read \fbib{because God}} makes him stand.

\v{5}One person decides in favor of one day over another, while another person decides that all days are the same. Let each one be fully convinced in his own mind: \v{6}The one who observes a special day,\fnote{\fbackref{14:6} Lit. \fbib{the day}} observes it to honor the Lord. The one who eats, eats to honor the Lord, since he gives thanks to God. And the one who does not eat, refrains from eating to honor the Lord; yet he, too, gives thanks to God.

\v{7}For none of us lives for himself, and no one dies for himself. \v{8}If we live, we live to honor the Lord; and if we die, we die to honor the Lord. So whether we live or die, we belong to the Lord. \v{9}For this reason the Messiah\fnote{\fbackref{14:9} Or \fbib{Christ}} died and returned to life, so that he might become the Lord of both the dead and the living.

\v{10}Why, then, do you criticize your brother? Or why do you despise your brother? For all of us will stand before the judgment seat of God.\fnote{\fbackref{14:10} Other mss. read \fbib{of the Messiah}} \v{11}For it is written,

\begin{poetry}
\poeml ``As certainly as I live, declares the Lord,\fnote{\fbackref{14:11} MT source citation reads \fbib{\divine{Lord}}} \\
\poemll    every knee will bow to me, \\
\poemlll       and every tongue will praise\fnote{\fbackref{14:11} Or \fbib{confess}} God.''\fnote{\fbackref{14:11} Cf. Isa 49:18; 45:23}
\end{poetry}

\v{12}Consequently, each of us will give an account of himself to God.
\passage{Acting in Love}

\v{13}Therefore, let's no longer criticize\fnote{\fbackref{14:13} Or \fbib{let's not criticize}} each other. Instead, make up your mind not to put a stumbling block or hindrance in the way of a brother. \v{14}I know---and have been persuaded by the Lord Jesus---that nothing is unclean in and of itself, but it is unclean to a person who thinks it is unclean. \v{15}For if your brother is being hurt by what you eat, you are no longer acting in love. Do not destroy the person for whom the Messiah\fnote{\fbackref{14:15} Or \fbib{Christ}} died by what you eat. \v{16}Do not allow what seems good to you to be spoken of as evil. \v{17}For God's kingdom does not consist of food and drink, but of righteousness, peace, and joy produced by the Holy Spirit. \v{18}For the person who serves the Messiah\fnote{\fbackref{14:18} Or \fbib{Christ}} in this way is pleasing to God and approved by people. \v{19}Therefore, let's keep on pursuing those things that bring peace and that lead to building up one another.

\v{20}Do not destroy God's action for the sake of food. Everything is clean, but it is wrong to make another person stumble because of what you eat. \v{21}The right thing to do is to avoid eating meat, drinking wine, or doing anything else that makes your brother stumble, upset, or weak.\fnote{\fbackref{14:21} Other mss. lack \fbib{upset, or weak}} \v{22}As for the faith you do have, have it as your own conviction before God. How blessed is the person who has no reason to condemn himself because of what he approves! \v{23}But the person who has doubts is condemned if he eats, because he does not act in faith; and anything that is not done in faith is sin.
\labelchapt{15}
\passage{Please Others, Not Yourselves}

\chapt{15}
\v{1}Now we who are strong ought to be patient with the weaknesses of those who are not strong and must stop pleasing ourselves. \v{2}Each of us must please our neighbor for the good purpose of building him up. \v{3}For even the Messiah\fnote{\fbackref{15:3} Or \fbib{Christ}} did not please himself. Instead, as it is written, ``The insults of those who insult you have fallen on me.''\fnote{\fbackref{15:3} Cf. Ps 69:9} \v{4}For everything that was written long ago was written to instruct us, so that we might have hope through the endurance and encouragement that the Scriptures give us.\fnote{\fbackref{15:4} Lit. \fbib{of the Scriptures}}

\v{5}Now may God, the source of endurance and encouragement, allow you to live in harmony with each other as you follow the Messiah\fnote{\fbackref{15:5} Or \fbib{Christ}} Jesus,\fnote{\fbackref{15:5} Lit. \fbib{according to the Messiah Jesus}} \v{6}so that with one mind and one voice you might glorify the God and Father of our Lord Jesus, the Messiah.\fnote{\fbackref{15:6} Or \fbib{Christ}}

\v{7}Therefore, accept one another, just as the Messiah\fnote{\fbackref{15:7} Or \fbib{Christ}} accepted you,\fnote{\fbackref{15:7} Other mss. read \fbib{us}} for the glory of God. \v{8}For I tell you that the Messiah\fnote{\fbackref{15:8} Or \fbib{Christ}} became a servant of the circumcised on behalf of God's truth in order to confirm the promises given to our ancestors, \v{9}so that the gentiles may glorify God for his mercy. As it is written,

\begin{poetry}
\poeml ``That is why I will praise\fnote{\fbackref{15:9} Or \fbib{confess}} you among the gentiles; \\
\poemll    I will sing praises to your name.''\fnote{\fbackref{15:9} Cf. Ps 18:49}
\end{poetry}

\v{10}Again he says,\fnote{\fbackref{15:10} Lit. \fbib{It}}

\begin{poetry}
\poeml ``Rejoice, you gentiles, with his people!''\fnote{\fbackref{15:10} Cf. Deut 32:43}
\end{poetry}

\v{11}And again,

\begin{poetry}
\poeml ``Praise the Lord,\fnote{\fbackref{15:11} MT source citation reads \fbib{\divine{Lord}}} all you gentiles! \\
\poemll    Let all the nations\fnote{\fbackref{15:11} Lit. \fbib{all peoples}} praise him.''\fnote{\fbackref{15:11} Cf. Ps 117:1}
\end{poetry}

\v{12}And again, Isaiah says,

\begin{poetry}
\poeml ``There will be a Root\fnote{\fbackref{15:12} I.e. Descendant} from Jesse. \\
\poemll    He will rise up to rule the gentiles, \\
\poemlll       and the gentiles will hope in him.''\fnote{\fbackref{15:12} Cf. Isa 11:10}
\end{poetry}

\v{13}Now may God, the source of hope, fill you with all joy and peace as you believe, so that you may overflow with hope by the power of the Holy Spirit.
\passage{Paul's Desire to Take the Gospel to the Whole World}

\v{14}I myself am convinced,\fnote{\fbackref{15:14} Lit. \fbib{convinced about you}} my brothers, that you yourselves are filled with goodness and full of all the knowledge you need to be able to instruct each other. \v{15}However, on some points I have written to you rather boldly, both as a reminder to you and because of the grace given me by God \v{16}to be a minister of the Messiah\fnote{\fbackref{15:16} Or \fbib{Christ}} Jesus to the gentiles in the priestly service of the gospel of God, so that the offering brought by gentiles may be acceptable, sanctified by the Holy Spirit.

\v{17}Therefore, in the Messiah\fnote{\fbackref{15:17} Or \fbib{Christ}} Jesus I have the right to boast about my work for God. \v{18}For I am bold enough to tell you only about what the Messiah\fnote{\fbackref{15:18} Or \fbib{Christ}} has accomplished through me in bringing gentiles to obedience. By my words and actions, \v{19}by the power of signs and wonders, and by the power of God's Spirit,\fnote{\fbackref{15:19} Other mss. read \fbib{of the Holy Spirit}} I have fully proclaimed the gospel of the Messiah\fnote{\fbackref{15:19} Or \fbib{Christ}} from Jerusalem as far as Illyricum. \v{20}My one ambition is to proclaim the gospel where the name of the Messiah\fnote{\fbackref{15:20} Or \fbib{Christ}} is not known, so I don't\fnote{\fbackref{15:20} Lit. \fbib{known, lest I}} build on someone else's foundation. \v{21}Rather, as it is written,

\begin{poetry}
\poeml ``Those who were never told about him will see, \\
\poemll    and those who have never heard will understand.''\fnote{\fbackref{15:21} Isa 52:15}
\end{poetry}
\passage{Paul's Plan to Visit Rome}

\v{22}This is why I have so often been hindered from coming to you. \v{23}But now, having no further opportunities in these regions, I want to come to you, as I've desired to do for many years. \v{24}Now that I am on my way to Spain, I hope to see you when I come your way and, after I have enjoyed your company for a while, to be sent on by you.

\v{25}Right now, however, I'm going to Jerusalem to minister to the saints, \v{26}because the believers in\fnote{\fbackref{15:26} The Gk. lacks \fbib{the believers in}} Macedonia and Achaia have been eager to share their resources with the poor among the saints in Jerusalem. \v{27}Yes, they were eager to do this, and in fact they are obligated to help them, for if the gentiles have shared in their spiritual blessings, they are obligated to be of service to them in material things.

\v{28}So when I have completed this task and have put my seal on this contribution of theirs, I will visit you on my way to Spain. \v{29}And I know that when I come to you I will come with the full blessing of the Messiah.\fnote{\fbackref{15:29} Or \fbib{Christ}; other mss. read \fbib{the gospel of the Messiah}}

\v{30}Now I urge you, brothers, by our Lord Jesus, the Messiah,\fnote{\fbackref{15:30} Or \fbib{Christ}} and by the love that the Spirit produces, to join me in my struggle, earnestly praying to God for me \v{31}that I may be rescued from the unbelievers in Judea, that my ministry to Jerusalem may be acceptable to the saints, \v{32}and that if it's God's will, I may come to you with joy and be refreshed together with you.
\passage{Doxology}

\v{33}Now may the God who grants\fnote{\fbackref{15:33} Lit. \fbib{God of peace}} peace be with all of you! Amen.
\labelchapt{16}
\passage{Personal Greetings}

\chapt{16}
\v{1}Now I commend to you our sister Phoebe, a deaconess\fnote{\fbackref{16:1} Or \fbib{minister}} in the church at Cenchrea. \v{2}Welcome her in the Lord as is appropriate for saints, and provide her with anything she may need from you, for she has assisted many people, including me.

\v{3}Greet Prisca\fnote{\fbackref{16:3} I.e. Priscilla} and Aquila, who work with me for the Messiah\fnote{\fbackref{16:3} Or \fbib{Christ}} Jesus, \v{4}and who risked their necks for my life. I am thankful to them, and so are all the churches among the gentiles. \v{5}Greet also the church in their house. Greet my dear friend Epaenetus, who was the first convert\fnote{\fbackref{16:5} Lit. \fbib{who was the first fruits}} to the Messiah\fnote{\fbackref{16:5} Or \fbib{Christ}} in Asia. \v{6}Greet Mary, who has worked very hard for you. \v{7}Greet Andronicus and Junia,\fnote{\fbackref{16:7} Or \fbib{Junias}} my fellow Jews who are in prison with me and are prominent among the apostles. They belonged to the Messiah\fnote{\fbackref{16:7} Or \fbib{Christ}} before I did. \v{8}Greet Ampliatus, my dear friend in the Lord. \v{9}Greet Urbanus, our co-worker in the Messiah,\fnote{\fbackref{16:9} Or \fbib{Christ}} and my dear friend Stachys. \v{10}Greet Apelles, who has been approved by the Messiah.\fnote{\fbackref{16:10} Or \fbib{Christ}} Greet those who belong to the family of Aristobulus. \v{11}Greet Herodion, my fellow Jew. Greet those in the family of Narcissus, who belong to the Lord. \v{12}Greet Tryphaena and Tryphosa, who have worked hard for the Lord. Greet my dear friend Persis, who has toiled diligently for the Lord. \v{13}Greet Rufus, the one chosen by the Lord, and his mother, who has been a mother to me, too. \v{14}Greet Asyncritus, Phlegon, Hermes, Patrobas, Hermas, and the brothers who are with them. \v{15}Greet Philologus and Julia, Nereus and his sister, and Olympas and all the saints who are with them. \v{16}Greet one another with a holy kiss.\fnote{\fbackref{16:16} People customarily greeted their friends with a kiss on the cheek.} All the churches of the Messiah\fnote{\fbackref{16:16} Or \fbib{Christ}} greet you.
\passage{Final Warning}

\v{17}Now I urge you, brothers, to watch out for those who create divisions and sinful enticements that oppose the teaching you have learned. Stay away from them, \v{18}because such people are not serving the Messiah\fnote{\fbackref{16:18} Or \fbib{Christ}} our Lord, but their own desires. By their smooth talk and flattering words they deceive the hearts of the unsuspecting. \v{19}For your obedience has become known to everyone, and I am full of joy for you. But I want you to be wise about what is good, and innocent about what is evil. \v{20}The God of peace will soon crush Satan under your feet. May the grace of our Lord Jesus, the Messiah,\fnote{\fbackref{16:20} Or \fbib{Christ}} be with all of you!\fnote{\fbackref{16:20} Other mss. lack \fbib{May the grace of our Lord Jesus the Messiah be with all of you!}}
\passage{Final Greeting}

\v{21}Timothy, my fellow worker, greets you, as do Lucius, Jason, and Sosipater, my fellow Jews. \v{22}I, Tertius, who penned this letter, greet you in the Lord. \v{23}Gaius, who is host to me and the whole church, greets you. Erastus, the city treasurer, and our brother Quartus greet you. \v{24}May the grace of our Lord Jesus, the Messiah,\fnote{\fbackref{16:24} Or \fbib{Christ}} be with all of you!\fnote{\fbackref{16:24} Other mss. lack this vs.}
\passage{Final Doxology}

\v{25}Now to the one who is able to strengthen you with my gospel and the message that I preach about Jesus, the Messiah,\fnote{\fbackref{16:25} Or \fbib{Christ}} by revealing the secret that was kept hidden from long ago \v{26}but now has been made known through the prophets to all the gentiles, in keeping with the decree of the eternal God to bring them to the obedience that springs from faith--- \v{27}to the only wise God, through Jesus the Messiah,\fnote{\fbackref{16:27} Or \fbib{Christ}} be glory forever! Amen.

\bookheader{1 Corinthians}
\labelbook{1Cor}

\bookpretitle{The Letter of Paul Called}
\booktitle{First Corinthians}

\labelchapt{1}
\passage{Paul Greets the Church in Corinth}

\chapt{1}
\v{1}From:\fnote{\fbackref{1:1} The Gk. lacks \fbib{From}} Paul, called to be an apostle of the Messiah\fnote{\fbackref{1:1} Or \fbib{Christ}} Jesus\fnote{\fbackref{1:1} Other mss. read \fbib{Jesus the Messiah}} by the will of God, and from our brother Sosthenes.

\v{2}To: God's church in Corinth, to those who have been sanctified by the Messiah\fnote{\fbackref{1:2} Or \fbib{Christ}} Jesus and called to be holy,\fnote{\fbackref{1:2} Or \fbib{to be saints}} together with all those everywhere who call on the name of our Lord Jesus, the Messiah\fnote{\fbackref{1:2} Or \fbib{Christ}}---their Lord\fnote{\fbackref{1:2} Lit. \fbib{theirs}} and ours.

\v{3}May grace and peace from God our Father and the Lord Jesus, the Messiah,\fnote{\fbackref{1:3} Or \fbib{Christ}} be yours!
\passage{You are Rich}

\v{4}I always thank my\fnote{\fbackref{1:4} Other mss. lack \fbib{my}} God for you because of the grace of God given you by the Messiah\fnote{\fbackref{1:4} Or \fbib{Christ}} Jesus. \v{5}For by him you have become rich in every way---in speech and knowledge of every kind--- \v{6}while our testimony about the Messiah\fnote{\fbackref{1:6} Or \fbib{Christ}} has been confirmed among you. \v{7}Therefore, you don't lack any spiritual gift as you eagerly wait for our Lord Jesus the Messiah\fnote{\fbackref{1:7} Or \fbib{Christ}} to be revealed. \v{8}He will keep you strong until the end, so that you will be blameless on the Day of our Lord Jesus the Messiah.\fnote{\fbackref{1:8} Or \fbib{Christ}} \v{9}Faithful is the God by whom you were called into fellowship with his Son Jesus the Messiah,\fnote{\fbackref{1:9} Or \fbib{Christ}} our Lord.
\passage{Divisions in the Church}

\v{10}Brothers, in the name of our Lord Jesus the Messiah,\fnote{\fbackref{1:10} Or \fbib{Christ}} I urge all of you to be in agreement\fnote{\fbackref{1:10} Lit. \fbib{to say the same thing}} and not to have divisions among you, so that you may be perfectly united in your understanding and opinions. \v{11}My brothers, some members of Chloe's family have made it clear to me that there are quarrels among you. \v{12}This is what I mean: Each of you is saying, ``I belong to Paul,'' or ``I belong to Apollos,'' or ``I belong to Cephas,''\fnote{\fbackref{1:12} I.e. Peter} or ``I belong to the Messiah.''\fnote{\fbackref{1:12} Or \fbib{Christ}}

\v{13}Is the Messiah\fnote{\fbackref{1:13} Or \fbib{Christ}} divided? Paul wasn't crucified for you, was he? You weren't baptized in Paul's name, were you? \v{14}I thank God\fnote{\fbackref{1:14} Other mss. read \fbib{I thank my God}; still other mss. read \fbib{I am thankful}} that I did not baptize any of you except Crispus and Gaius, \v{15}so that no one can say that you were baptized in my name. \v{16}(Oh yes, I also baptized the family of Stephanas. Beyond that, I'm not sure whether I baptized anyone else.) \v{17}For the Messiah\fnote{\fbackref{1:17} Or \fbib{Christ}} did not send me to baptize but to preach the gospel, not with eloquent wisdom, so the cross of the Messiah\fnote{\fbackref{1:17} Or \fbib{Christ}} won't be emptied of its power.
\passage{The Messiah is God's Power and Wisdom}

\v{18}For the message about the cross is nonsense to those who are being destroyed, but it is God's power to us who are being saved. \v{19}For it is written,

\begin{poetry}
\poeml ``I will destroy the wisdom of the wise, \\
\poemll    and the intelligence of the intelligent I will reject.''\fnote{\fbackref{1:19} Isa 29:14}
\end{poetry}

\v{20}Where is the wise person? Where is the scholar? Where is the philosopher of this age? God has turned the wisdom of the world into nonsense, hasn't he? \v{21}For since, in the wisdom of God, the world through its wisdom did not know God,\fnote{\fbackref{1:21} The Gk. lacks \fbib{God}} God was pleased to save those who believe through the nonsense of our preaching. \v{22}Jews ask for signs, and Greeks look for wisdom, \v{23}but we preach the Messiah\fnote{\fbackref{1:23} Or \fbib{Christ}} crucified. He is a stumbling block to Jews and nonsense to gentiles, \v{24}but to those who are called,\fnote{\fbackref{1:24} Or \fbib{chosen}} both Jews and Greeks, the Messiah\fnote{\fbackref{1:24} Or \fbib{Christ}} is God's power and God's wisdom. \v{25}For God's nonsense is wiser than human wisdom,\fnote{\fbackref{1:25} Lit. \fbib{than men}} and God's weakness is stronger than human strength.\fnote{\fbackref{1:25} Lit. \fbib{than men}}

\v{26}Brothers, think about your own calling. Not many of you were wise by human standards,\fnote{\fbackref{1:26} Lit. \fbib{according to the flesh}} not many were powerful, not many were of noble birth. \v{27}But God chose what is nonsense in the world to make the wise feel ashamed. God chose what is weak in the world to make the strong feel ashamed. \v{28}And God chose what is insignificant in the world, what is despised, what is nothing, in order to destroy what is something, \v{29}so that no one\fnote{\fbackref{1:29} Lit. \fbib{flesh}} may boast in God's presence. \v{30}It is because of God\fnote{\fbackref{1:30} Lit. \fbib{him}} that you are in union with the Messiah\fnote{\fbackref{1:30} Or \fbib{Christ}} Jesus, who for us has become wisdom from God, as well as our righteousness, sanctification, and redemption. \v{31}Therefore, as it is written, ``The person who boasts must boast in the Lord.''\fnote{\fbackref{1:31} Jer 9:24; MT source citation reads \fbib{}\divine{Lord}}
\labelchapt{2}
\passage{Preaching in the Power of God}

\chapt{2}
\v{1}When I came to you, brothers, I didn't come and tell you about God's secret\fnote{\fbackref{2:1} Other mss. read \fbib{testimony}} with rhetorical language or wisdom. \v{2}For while I was with you I resolved to know nothing except Jesus the Messiah,\fnote{\fbackref{2:2} Or \fbib{Christ}} and him crucified. \v{3}It was in weakness, fear, and great trembling that I came to you. \v{4}My message and my preaching were not accompanied by clever, wise words, but by a display of the Spirit's power, \v{5}so that your faith would not be based on human wisdom but on God's power.
\passage{God's Spirit Reveals Everything}

\v{6}However, when we are among mature people, we do speak a message of\fnote{\fbackref{2:6} The Gk. lacks \fbib{a message of}} wisdom, but not the wisdom of this world or of the rulers of this world, who are passing off the scene. \v{7}Instead, we speak about God's wisdom in a hidden secret, which God destined before the world began\fnote{\fbackref{2:7} The Gk. lacks \fbib{began}} for our glory. \v{8}None of the rulers of this world understood it, because if they had, they would not have crucified the Lord of glory. \v{9}But as it is written,

\begin{poetry}
\poeml ``No eye has seen, no ear has heard, \\
\poemll    and no mind has imagined \\
\poeml the things that God has prepared \\
\poemll    for those who love him.''\fnote{\fbackref{2:9} Isa 64:4}
\end{poetry}

\v{10}But\fnote{\fbackref{2:10} Other mss. read \fbib{For}} God has revealed those things to us by his Spirit. For the Spirit searches everything, even the deep things of God.

\v{11}Is there anyone who can understand his own thoughts except his own inner spirit? In the same way, no one can know the thoughts of God except God's Spirit. \v{12}Now, we have not received the spirit of the world but the Spirit who comes from God, so that we can understand the things that were freely given to us by God. \v{13}We don't speak about these things with words taught us by human wisdom, but with words\fnote{\fbackref{2:13} Lit. \fbib{in things}} taught by the Spirit, as we explain spiritual things to spiritual people.\fnote{\fbackref{2:13} Or \fbib{in spiritual words}} \v{14}A person who isn't spiritual doesn't accept the things of God's Spirit, for they are nonsense to him. He can't understand them because they are spiritually evaluated. \v{15}The spiritual person evaluates everything but is subject to no one else's evaluation. \v{16}For

\begin{poetry}
\poeml ``Who has known the mind of the Lord\fnote{\fbackref{2:16} MT source citation reads \fbib{}\divine{Lord}} \\
\poemll    so that he can advise him?''\fnote{\fbackref{2:16} Isa 40:13}
\end{poetry}

However, we have the mind of the Messiah.\fnote{\fbackref{2:16} Or \fbib{Christ}}
\labelchapt{3}
\passage{Spiritual Immaturity}

\chapt{3}
\v{1}Brothers, I couldn't talk to you as spiritual people but as worldly people, as mere infants in the Messiah.\fnote{\fbackref{3:1} Or \fbib{Christ}} \v{2}I gave you milk to drink, not solid food, because you weren't ready for it. And you're still not ready! \v{3}That's because you are still worldly. As long as there is jealousy and quarreling among you, you are worldly and living by human standards, aren't you? \v{4}For when one person says, ``I follow Paul,'' and another person says, ``I follow to Apollos,'' you're following\fnote{\fbackref{3:4} The Gk. lacks \fbib{following}} your own human nature, aren't you?

\v{5}Who is Apollos, anyhow? Or who is Paul? They're merely servants through whom you came to believe, as the Lord gave to each of us his task. \v{6}I planted, Apollos watered, but God kept everything growing. \v{7}So neither the one who plants nor the one who waters is significant, but God, who keeps everything growing, is the one who matters. \v{8}The one who plants and the one who waters have the same goal, and each will receive a reward for his own action. \v{9}For we are God's co-workers. You are God's farmland and God's building.
\passage{The Messiah is Our Foundation}

\v{10}As an expert builder using the grace that God gave me, I laid the foundation, and someone else is building on it. But each person must be careful how he builds on it. \v{11}After all, no one can lay any other foundation than the one that is already laid, and that is Jesus the Messiah.\fnote{\fbackref{3:11} Or \fbib{Christ}} \v{12}Whether a person builds on this foundation with gold, silver, expensive stones, wood, hay, or straw, \v{13}the workmanship of each person will become evident, for the day of judgment\fnote{\fbackref{3:13} The Gk. lacks \fbib{of judgment}} will show what it is, because it will be revealed with fire, and the fire will test the quality of each person's action. \v{14}If what a person has built on the foundation survives, he will receive a reward.\fnote{\fbackref{3:14} Or \fbib{receive wages}} \v{15}If his work is burned up, he will suffer loss. However, he himself will be saved, but it will be like going through fire.

\v{16}You know that you are God's sanctuary and that God's Spirit lives in you, don't you? \v{17}If anyone destroys God's sanctuary, God will destroy him, for God's sanctuary is holy. And you are that sanctuary!
\passage{True Wisdom}

\v{18}Let no one deceive himself. If any of you thinks he is wise in the ways of\fnote{\fbackref{3:18} The Gk. lacks \fbib{the ways of}} this world, he must become a fool to become really wise. \v{19}For the wisdom of this world is nonsense in God's sight. For it is written,

\begin{poetry}
\poeml ``He catches the wise with their own trickery,''\fnote{\fbackref{3:19} Job 5:13}
\end{poetry}

\v{20}and again,

\begin{poetry}
\poeml ``The Lord\fnote{\fbackref{3:20} MT source citation reads \fbib{}\divine{Lord}} knows that the thoughts of the wise are worthless.''\fnote{\fbackref{3:20} Ps 94:11}
\end{poetry}

\v{21}So let no one boast about human beings, since everything belongs to you, \v{22}whether Paul, Apollos, Cephas,\fnote{\fbackref{3:22} I.e. Peter} the world, life, death, the present, or the future---everything belongs to you, \v{23}but you belong to the Messiah,\fnote{\fbackref{3:23} Or \fbib{Christ}} and the Messiah\fnote{\fbackref{3:23} Or \fbib{Christ}} belongs to God.
\labelchapt{4}
\passage{Faithful Servants of the Messiah}

\chapt{4}
\v{1}Think of us as servants of the Messiah\fnote{\fbackref{4:1} Or \fbib{Christ}} and as servant managers entrusted with God's secrets. \v{2}Now it is required of servant managers that each one should prove to be trustworthy.\fnote{\fbackref{4:2} Or \fbib{should be found faithful}} \v{3}It is a very small thing to me that I should be examined by you or by any human court. In fact, I don't even evaluate myself. \v{4}For my conscience is clear,\fnote{\fbackref{4:4} Lit. \fbib{I don't know of anything against myself}} but that does not vindicate me. It is the Lord who examines me. \v{5}Therefore, stop judging prematurely, before the Lord comes, for he will bring to light what is now hidden in darkness and reveal the motives of our hearts. Then each person will receive his praise from God.
\passage{Fools for the Messiah's Sake}

\v{6}Brothers, I have applied all this to Apollos and myself for your benefit, so that you may learn from us not to go beyond what the Scriptures say.\fnote{\fbackref{4:6} Lit. \fbib{what is written}} Then you will stop boasting about one person at the expense of another.

\v{7}For who makes you superior? What do you have that you did not receive? And if you did receive it, why do you boast as though you did not receive it? \v{8}You already have all you want! You have already become rich! You have become kings without us! I wish you really were kings so that we could be kings with you! \v{9}For it seems to me that God has put us apostles on display in last place, like men condemned to death. We have become a spectacle for the world, for angels, and for people to stare at. \v{10}We are fools for the Messiah's\fnote{\fbackref{4:10} Or \fbib{Christ's}} sake, but you are wise in the Messiah.\fnote{\fbackref{4:10} Or \fbib{Christ}} We are weak, but you are strong. You are honored, but we are dishonored. \v{11}We are hungry, thirsty, dressed in rags, brutally treated, and homeless, right up to the present. \v{12}We wear ourselves out from working with our own hands. When insulted, we bless. When persecuted, we endure. \v{13}When slandered, we answer with kind words. Even now we have become the filth of the world, the scum of the universe.
\passage{Fatherly Advice}

\v{14}I'm not writing this to make you feel ashamed, but to warn you as my dear children. \v{15}You may have 10,000 mentors who work for the Messiah,\fnote{\fbackref{4:15} Or \fbib{Christ}} but not many fathers. For in the Messiah\fnote{\fbackref{4:15} Or \fbib{Christ}} Jesus I became your father through the gospel. \v{16}So I urge you to imitate me. \v{17}That's why I sent Timothy to you. He is my dear and dependable son in the Lord and will help you remember how I live for the Messiah\fnote{\fbackref{4:17} Or \fbib{Christ}} Jesus as I teach everywhere in every church.

\v{18}Some of you have become arrogant, as though I were not coming to evaluate\fnote{\fbackref{4:18} The Gk. lacks \fbib{evaluate}} you. \v{19}But I will come to you soon if it's the Lord's will. Then I'll discover not only what these arrogant people are saying but also what power they have, \v{20}for the kingdom of God isn't just talk, but also power. \v{21}Which do you prefer? Should I come to you with a stick, or with love and a gentle spirit?
\labelchapt{5}
\passage{Disciplining for Sexual Immorality}

\chapt{5}
\v{1}It is actually reported that sexual immorality exists among you, and of a kind that is not found even among the gentiles. A man is actually living with his father's wife! \v{2}And you are being arrogant instead of being filled with grief and seeing to it that the man who did this is removed from among you. \v{3}Even though I am away from you physically, I am with you in spirit. I have already passed judgment on the man who did this, as though I were present with you. \v{4}In the name of our Lord Jesus, when you are gathered together (and I am there in spirit), and the power of our Lord Jesus is there, too, \v{5}turn this man over to Satan for the destruction of his body,\fnote{\fbackref{5:5} Or \fbib{flesh}} so that his spirit may be saved on the Day of the Lord.\fnote{\fbackref{5:5} Other mss. read \fbib{Lord Jesus}; still other mss. read \fbib{our Lord Jesus, the Messiah}}

\v{6}Your boasting is not good. You know that a little yeast leavens the whole batch of dough, don't you? \v{7}Get rid of the old yeast so that you may be a new batch of dough, since you are to be free from yeast. For the Messiah,\fnote{\fbackref{5:7} Or \fbib{Christ}} our Passover, has been sacrificed. \v{8}So let's keep celebrating the festival, neither with old yeast nor with yeast that is evil and wicked, but with yeast-free bread that is both sincere and true.

\v{9}I wrote to you in my letter to stop associating with people who are sexually immoral--- \v{10}not at all meaning the people of this world who are immoral, greedy, robbers, or idolaters. In that case you would have to leave this world. \v{11}But now I am writing to you to stop associating with any so-called brother if he is sexually immoral, greedy, an idolater, a slanderer, a drunk, or a robber. You must even stop eating with someone like that. \v{12}After all, is it my business to judge outsiders? You are to judge those who are in the community, aren't you? \v{13}God will judge outsiders. ``Expel that wicked man.''\fnote{\fbackref{5:13} Deut 17:7 (LXX)}
\labelchapt{6}
\passage{Morality in Legal Matters}

\chapt{6}
\v{1}When one of you has a complaint against another, does he dare to take the matter before those who are unrighteous and not before the saints? \v{2}You know that the saints will rule the world, don't you? And if the world is going to be ruled by you, can't you handle insignificant cases? \v{3}You know that we will rule angels, not to mention things in this life, don't you? \v{4}So if you have cases dealing with this life, why do you appoint as judges people who have no standing in the church? \v{5}I say this to make you feel ashamed. Has it come to this, that there is not one person among you who is wise enough to settle disagreements between brothers?\fnote{\fbackref{6:5} Lit. \fbib{between his brother}} \v{6}Instead, one brother goes to court against another brother, and before unbelieving judges,\fnote{\fbackref{6:6} The Gk. lacks \fbib{judges}} at that! \v{7}The very fact that you have lawsuits among yourselves is already a defeat for you. Why not rather just accept the wrong? Why not rather be cheated? \v{8}Instead, you yourselves practice doing wrong and cheating others, and brothers at that!

\v{9}You know that wicked people will not inherit the kingdom of God, don't you? Stop deceiving yourselves! Sexually immoral people, idolaters, adulterers, male prostitutes, homosexuals, \v{10}thieves, greedy people, drunks, slanderers, and robbers will not inherit the kingdom of God. \v{11}That is what some of you were! But you were washed, you were sanctified, you were justified in the name of our Lord Jesus the Messiah\fnote{\fbackref{6:11} Or \fbib{Christ}} and by the Spirit of our God.
\passage{Morality in Sexual Matters}

\v{12}Everything is permissible for me, but not everything is helpful. Everything is permissible for me, but I will not allow anything to control me. \v{13}Food is for the stomach, and the stomach is for food, but God will make them both unnecessary. The body is not meant for sexual immorality but for the Lord, and the Lord for the body. \v{14}God raised the Lord, and by his power he will also raise us.

\v{15}You know that your bodies belong to the Messiah,\fnote{\fbackref{6:15} Or \fbib{Christ}} don't you? Should I take what belongs to the Messiah\fnote{\fbackref{6:15} Or \fbib{Christ}} and unite them with a prostitute? Certainly not! \v{16}You know that the person who unites himself with a prostitute becomes one body with her, don't you? For it is said, ``The two will become one flesh.''\fnote{\fbackref{6:16} Gen 2:24} \v{17}But the person who unites himself with the Lord becomes one spirit with him.

\v{18}Keep on running away from sexual immorality. Any other\fnote{\fbackref{6:18} The Gk. lacks \fbib{other}} sin that a person commits is outside his body, but the person who sins sexually sins against his own body. \v{19}You know that your body is a sanctuary of the Holy Spirit who is in you, whom you have received from God, don't you? You do not belong to yourselves, \v{20}because you were bought for a price. Therefore, glorify God with your bodies.
\labelchapt{7}
\passage{Concerning Marriage}

\chapt{7}
\v{1}Now about what you asked: ``Is it advisable for a man not to marry?''\fnote{\fbackref{7:1} Lit. \fbib{to touch a woman}} \v{2}Because sexual immorality is so rampant,\fnote{\fbackref{7:2} Lit. \fbib{because of instances of sexual immorality}} every man should have his own wife, and every woman should have her own husband.

\v{3}A husband should fulfill his obligation to his wife, and a wife should do the same for her husband. \v{4}A wife does not have authority over her own body, but her husband does. In the same way, a husband doesn't have authority over his own body, but his wife does. \v{5}Do not withhold yourselves from each other unless you agree to do so just for a set time, in order to devote yourselves to prayer.\fnote{\fbackref{7:5} Other mss. read \fbib{to fasting and prayer}} Then you should come together again so that Satan does not tempt you through your lack of self-control. \v{6}But I say this as a concession, not as a command. \v{7}I would like everyone to be unmarried,\fnote{\fbackref{7:7} The Gk. lacks \fbib{unmarried}} like I am. However, each person has a special gift from God, one this and another that.

\v{8}I say to those who are unmarried, especially to widows: It is good for them to remain like me. \v{9}However, if they cannot control themselves, they should get married, for it is better to marry than to burn with passion.\fnote{\fbackref{7:9} The Gk. lacks \fbib{with passion}} \v{10}To married people I give this command (not really I, but the Lord): A wife must not leave her husband. \v{11}But if she does leave him, she must remain unmarried or else be reconciled to her husband. Likewise, a husband must not abandon\fnote{\fbackref{7:11} Or \fbib{divorce}} his wife.

\v{12}I (not the Lord) say to the rest of you: If a brother has a wife who is an unbeliever and she is willing to live with him, he must not abandon\fnote{\fbackref{7:12} Or \fbib{divorce}} her. \v{13}And if a woman has a husband who is an unbeliever and he is willing to live with her, she must not abandon\fnote{\fbackref{7:13} Or \fbib{divorce}} him. \v{14}For the unbelieving husband has been sanctified because of his wife, and the unbelieving wife has been sanctified because of her husband.\fnote{\fbackref{7:14} Other mss. read \fbib{brother}} Otherwise, your children would be unclean, but now they are holy. \v{15}But if the unbelieving partner\fnote{\fbackref{7:15} The Gk. lacks \fbib{partner}} leaves, let him go. In such cases the brother or sister is not under obligation. God has called you\fnote{\fbackref{7:15} Other mss. read \fbib{us}} to live in peace. \v{16}Wife, you might be able to save your husband. Husband, you might be able to save your wife.
\passage{Live according to God's Call}

\v{17}Nevertheless, everyone should live the life that the Lord gave him and to which God called him. This is my rule in all the churches. \v{18}Was anyone circumcised when he was called? He should not try to change that. Was anyone uncircumcised when he was called? He should not get circumcised. \v{19}Circumcision is nothing, and uncircumcision is nothing, but obeying God's commandments is everything.\fnote{\fbackref{7:19} The Gk. lacks \fbib{is everything}} \v{20}Everyone should stay in the same condition\fnote{\fbackref{7:20} Lit. \fbib{the calling}} in which he was called. \v{21}Were you a slave when you were called? Do not let that bother you. Of course, if you have a chance to become free, take advantage of the opportunity. \v{22}For the slave who has been called to belong to the Lord is the Lord's free person. In the same way, the free person who has been called is the Messiah's\fnote{\fbackref{7:22} Or \fbib{Christ's}} slave. \v{23}You were bought for a price. Stop becoming slaves of people. \v{24}Brothers, everyone should stay in the same condition\fnote{\fbackref{7:24} Lit. \fbib{the calling}} in which he was called by God.
\passage{Concerning Virgins}

\v{25}Now concerning virgins, although I do not have any command from the Lord, I will give you my opinion as one who by the Lord's mercy is trustworthy. \v{26}In view of the present crisis, I think it is prudent for a man to stay as he is. \v{27}Have you become committed\fnote{\fbackref{7:27} Lit. \fbib{you been bound}} to a wife? Stop trying to get released from your commitment.\fnote{\fbackref{7:27} The Gk. lacks \fbib{from your commitment}} Have you been freed from your commitment to\fnote{\fbackref{7:27} The Gk. lacks \fbib{from your commitment}} a wife? Stop looking for one.\fnote{\fbackref{7:27} Lit. \fbib{for a wife}} \v{28}But if you do get married, you have not sinned. And if a virgin gets married, she has not sinned. However, these people will experience trouble in this life,\fnote{\fbackref{7:28} Lit. \fbib{flesh}} and I want to spare you from that.

\v{29}This is what I mean, brothers: The time is short. From now on, those who have wives should live as though they had none, \v{30}and those who mourn as though they did not mourn, and those who rejoice as though they were not rejoicing, and those who buy as though they did not own a thing, \v{31}and those who use the things in the world as though they were not dependent on them. For the world in its present form is passing away.

\v{32}I want you to be free from concerns. An unmarried man is concerned about the things of the Lord, that is, about how he can please the Lord. \v{33}But a married man is concerned about things of this world, that is, about how he can please his wife, \v{34}and so his attention is divided.

An unmarried woman or virgin is concerned about the affairs of the Lord, so that she may be holy in body and spirit. But a married woman is concerned about the affairs of this world, that is, about how she can please her husband. \v{35}I'm saying this for your benefit, not to put a noose around your necks, but to promote good order and unhindered devotion to the Lord.

\v{36}If a man thinks he is not behaving properly toward his virgin,\fnote{\fbackref{7:36} I.e. virgin fianc\'{e}e, but possibly virgin daughter} and if his passion is so strong that he feels he ought to marry her, let him do what he wants; he isn't sinning. Let them get married. \v{37}However, if a man stands firm in his resolve, feels no necessity, and has made up his mind to keep her a virgin, he will be acting appropriately. \v{38}So then the man who marries the virgin acts appropriately, but the man who refrains from marriage does even better.

\v{39}A wife is bound to her husband as long as he lives. But if her husband dies, she is free to marry anyone she wishes, only in the Lord. \v{40}However, in my opinion she will be happier\fnote{\fbackref{7:40} Or \fbib{more blessed}} if she stays as she is. And in saying this,\fnote{\fbackref{7:40} The Gk. lacks \fbib{in saying this}} I think that I, too, have God's Spirit.
\labelchapt{8}
\passage{Concerning Food Offered to Idols}

\chapt{8}
\v{1}Now concerning food offered to idols: We know that we all possess knowledge. Knowledge puffs up, but love builds up. \v{2}If anyone thinks he really\fnote{\fbackref{8:2} The Gk. lacks \fbib{really}} knows something, he has not yet learned it as he ought to know it. \v{3}But anyone who loves God is known by him.\fnote{\fbackref{8:3} I.e. Other mss. lack \fbib{by him}}

\v{4}Now concerning eating food offered to idols: We know that no idol is real in this world and that there is only one God. \v{5}For even if there are ``gods'' in heaven and on earth (as indeed there are many so-called ``gods'' and ``lords''), \v{6}yet for us

\begin{poetry}
\poeml there is only one God, the Father, \\
\poemll    from whom everything came into being \\
\poemlll       and for whom we live. \\
\poeml And there is only one Lord, Jesus the Messiah,\fnote{\fbackref{8:6} Or \fbib{Christ}} \\
\poemll    through whom everything came into being \\
\poemlll       and through whom we live.
\end{poetry}

\v{7}But not everyone has this knowledge. Some people are so accustomed to idolatry that when they eat food that has been offered to an idol, their conscience becomes contaminated because it is weak. \v{8}However, food will not bring us closer to God. We are no worse off if we do not eat food that has been offered to an idol,\fnote{\fbackref{8:8} The Gk. lacks \fbib{food that has been offered to an idol}} and no better off if we do.

\v{9}But you must see to it that this right of yours does not become a stumbling block for those who are weak. \v{10}For if anyone with a weak conscience sees you, who know better, eating in an idol's temple, he will be encouraged to eat what has been offered to idols, won't he? \v{11}In that case, the weak brother for whom the Messiah\fnote{\fbackref{8:11} Or \fbib{Christ}} died is ruined by your knowledge. \v{12}When you sin against your brothers in this way and wound their weak consciences, you are sinning against the Messiah.\fnote{\fbackref{8:12} Or \fbib{Christ}} \v{13}Therefore, if food that I eat\fnote{\fbackref{8:13} The Gk. lacks \fbib{that I eat}} causes my brother to stumble, I will never eat meat again, in order to keep my brother from stumbling.
\labelchapt{9}
\passage{The Rights of an Apostle}

\chapt{9}
\v{1}I am free, am I not? I am an apostle, am I not? I have seen Jesus our Lord, haven't I? You are the result of\fnote{\fbackref{9:1} The Gk. lacks \fbib{the result of}} my work in the Lord, aren't you? \v{2}If I am not an apostle to other people, surely I am one to you, for you are the evidence of my apostolic authority from the Lord.

\v{3}This is my defense to those who would examine me: \v{4}We have the right to earn our food,\fnote{\fbackref{9:4} Lit. \fbib{to eat and drink}} don't we? \v{5}We have the right to take a believing wife with us like the other apostles, the Lord's brothers, and Cephas,\fnote{\fbackref{9:5} I.e. Peter} don't we? \v{6}Or is it only Barnabas and I who have to keep on working for a living? \v{7}Who ever goes to war at his own expense? Who plants a vineyard and does not eat any of its grapes? Or who takes care of a flock and does not drink any of its milk? \v{8}I am not saying this on human authority, am I? The Law says the same thing, doesn't it? \v{9}For in the Law of Moses it is written, ``You must not muzzle an ox while it is treading out the grain.''\fnote{\fbackref{9:9} Deut 25:4} God is not only concerned about oxen, is he? \v{10}Isn't he really speaking for our benefit? Yes, this was written for our benefit, because the one who plows should plow in hope, and the one who threshes should thresh in hope of sharing in the crop. \v{11}If we have sown spiritual seed among you, is it too much if we reap material benefits from you? \v{12}If others enjoy this right over you, don't we have a stronger claim? But we did not use this right. On the contrary, we tolerate everything in order not to put an obstacle in the way of the gospel of the Messiah.\fnote{\fbackref{9:12} Or \fbib{Christ}}

\v{13}You know that those who work in the Temple get their food from the Temple and that those who serve at the altar get their share of its offerings, don't you? \v{14}In the same way, the Lord has ordered that those who proclaim the gospel should make their living from the gospel.

\v{15}But I have not used any of these rights, and I'm not writing this so that they may be applied in my case. I would rather die than let anyone deprive me of my reason for\fnote{\fbackref{9:15} The Gk. lacks \fbib{reason for}} boasting. \v{16}For if I preach the gospel, I have nothing to boast about, for this obligation has been entrusted to me. How terrible it would be for me if I didn't preach the gospel! \v{17}For if I preach voluntarily, I get a reward, but if I am unwilling to do it, I am still entrusted with that obligation. \v{18}What, then, is my reward? It is\fnote{\fbackref{9:18} The Gk. lacks \fbib{It is}} to be able to preach the gospel free of charge, and so I never resort to demanding my rights when I'm preaching\fnote{\fbackref{9:18} Lit. \fbib{rights in}} the gospel.

\v{19}Although I am free from everyone's expectations, I have made myself a servant to all of them to win more people. \v{20}To the Jews I became like a Jew in order to win Jews. To those under the Law I became like a man under the Law, in order to win those under the Law (although I myself am not under the Law). \v{21}To those who do not have the Law, I became like a man who does not have the Law in order to win those who do not have the Law. However, I am not free from God's Law, but I'm subject to the Messiah's\fnote{\fbackref{9:21} Or \fbib{Christ's}} law. \v{22}To the weak I became weak in order to win the weak. I have become all things to all people so that by all possible means I might save some of them. \v{23}I do all this for the sake of the gospel in order to have a share in its blessings.

\v{24}You know that in a race all the runners run but only one wins the prize, don't you? You must run in such a way that you may be victorious. \v{25}Everyone who enters an athletic contest practices self-control in everything. They do it to win a wreath that withers away, but we run to win a prize that\fnote{\fbackref{9:25} The Gk. lacks \fbib{run to win a prize that}} never fades. \v{26}That is the way I run, with a clear goal in mind. That is the way I fight, not like someone shadow boxing. \v{27}No, I keep on disciplining my body, making it serve me so that after I have preached to others, I myself will not somehow be disqualified.
\labelchapt{10}
\passage{Warnings about Idolatry}

\chapt{10}
\v{1}Now I do not want you to be ignorant, brothers, of the fact that all of our ancestors who left Egypt\fnote{\fbackref{10:1} The Gk. lacks \fbib{who left Egypt}} were under the cloud. They all went through the sea, \v{2}and they all were immersed into Moses in the cloud and in the sea. \v{3}They all ate the same spiritual food \v{4}and drank the same spiritual drink, for they drank from the spiritual rock that went with them. That rock was the Messiah.\fnote{\fbackref{10:4} Or \fbib{Christ}} \v{5}But God wasn't pleased with most of those people,\fnote{\fbackref{10:5} Lit. \fbib{of them}} and so they were struck down in the wilderness.

\v{6}Now their experiences serve as examples for us so that we won't set our hearts on evil as they did. \v{7}Let's stop being idolaters, as some of them were. As it is written, ``The people sat down to eat and drink and got up to play.''\fnote{\fbackref{10:7} Exod 32:6} \v{8}Let's stop sinning sexually, as some of them were doing, and on a single day 23,000 fell dead. \v{9}Let's stop putting the Lord\fnote{\fbackref{10:9} Other mss. read \fbib{Messiah}} to the test, as some of them were doing, and were destroyed by snakes. \v{10}You must stop complaining, as some of them were doing, and were annihilated by the destroyer. \v{11}These things happened to them to serve as an example, and they were written down as a warning for us in whom the culmination of the ages has been attained. \v{12}Therefore, whoever thinks he is standing securely should watch out so he doesn't fall. \v{13}No temptation has overtaken you that is unusual for human beings. But God is faithful, and he will not allow you to be tempted beyond your strength. Instead, along with the temptation he will also provide a way out, so that you may be able to endure it.

\v{14}And so, my dear friends, keep on running away from idolatry. \v{15}I am talking to sensible people. Apply what I am saying to yourselves. \v{16}The cup of blessing that we bless is our fellowship in the blood of the Messiah,\fnote{\fbackref{10:16} Or \fbib{Christ}} isn't it? The bread that we break is our fellowship in the body of the Messiah,\fnote{\fbackref{10:16} Or \fbib{Christ}} isn't it? \v{17}Because there is one loaf, we who are many are one body, because all of us eat from the same loaf.

\v{18}Look at the Israelis from a human point of view.\fnote{\fbackref{10:18} Lit. \fbib{Israel according to the flesh}} Those who eat the sacrifices share in what is on the altar, don't they? \v{19}Am I suggesting that an offering made to idols means anything, or that an idol itself means anything? \v{20}Hardly! What they offer, they offer to demons and not to God, and I do not want you to become partners with demons. \v{21}You cannot drink the cup of the Lord and the cup of demons. You cannot dine with the Lord and dine with demons, \v{22}or you'll provoke the Lord to jealousy, won't you? Are we stronger than he is?
\passage{All to the Glory of God}

\v{23}Everything is permissible, but not everything is helpful. Everything is permissible, but not everything builds up. \v{24}No one should seek his own welfare, but rather his neighbor's.

\v{25}Eat anything that is sold in the meat market without raising any question about it on the grounds of conscience, \v{26}for ``the earth and everything in it belong to the Lord.''\fnote{\fbackref{10:26} Ps 24:1; MT source citation reads \fbib{}\divine{Lord}} \v{27}If an unbeliever invites you to his house and you wish to go, eat whatever is set before you, raising no question on the grounds of conscience. \v{28}However, if someone says to you, ``This was offered as a sacrifice,'' don't eat it, both out of consideration for the one who told you and also for the sake of conscience. \v{29}I mean, of course, his conscience, not yours. For why should my freedom be determined by someone else's conscience? \v{30}If I eat with thankfulness, why should I be denounced because of what I am thankful for?

\v{31}Therefore, whether you eat or drink, or whatever you do, do everything for the glory of God. \v{32}Don't become a stumbling block to Jews or Greeks or to the church of God, \v{33}just as I myself try to please everybody in every way. I don't look out for my own benefit, but rather for the benefit of many people, so that they might be saved.
\labelchapt{11}
\passage{Be Imitators of Me}

\chapt{11}
\v{1}Imitate me, as I do the Messiah.\fnote{\fbackref{11:1} Or \fbib{Christ}} \v{2}I praise you for remembering everything I told you\fnote{\fbackref{11:2} The Gk. lacks \fbib{told you}} and for holding to the traditions\fnote{\fbackref{11:2} I.e. Jewish traditions} that I passed on to you.
\passage{Advice about Head Coverings}

\v{3}Now I want you to realize that the Messiah\fnote{\fbackref{11:3} Or \fbib{Christ}} is the head of every man, and man is the head of the woman, and God is the head of the Messiah.\fnote{\fbackref{11:3} Or \fbib{Christ}} \v{4}Every man who prays or prophesies with something on his head dishonors his head, \v{5}and every woman who prays or prophesies with her head uncovered dishonors her head, which is the same as having her head shaved. \v{6}So if a woman does not cover her head, she should cut off her hair. If it is a disgrace for a woman to cut off her hair or shave her head, let her cover her own head.

\v{7}A man should not cover his head, because he exists as God's image and glory. But the woman is man's glory. \v{8}For man did not come from woman, but woman from man; \v{9}and man was not created for woman, but woman for man. \v{10}This is why a woman should have authority over her own head: because of the angels.

\v{11}In the Lord, however, woman is not independent of man, nor is man of woman. \v{12}For as woman came from man, so man comes through woman. But everything comes from God. \v{13}Decide for yourselves: Is it proper for a woman to pray to God with her head uncovered?\fnote{\fbackref{11:13} Or \fbib{It is proper . . . uncovered, isn't it?}} \v{14}Nature itself teaches you neither that it is disgraceful for a man to have long\fnote{\fbackref{11:14} The Gk. lacks \fbib{long}} hair \v{15}nor that hair is a woman's glory, since hair is given as a substitute for coverings. \v{16}But if anyone wants to argue about this, we do not have any custom like this, nor do any of God's churches.
\passage{Concerning the Lord's Supper}
\passageinfo{(Matthew 26:26-29; Mark 14:22-25; Luke 22:14-20)}

\v{17}Now I am not praising you in giving you the following instructions. When you gather, it is not for the better but for the worse. \v{18}For in the first place, I hear that when you gather as a church there are divisions among you, and I partly believe it. \v{19}Of course, there must be factions among you to show which of you are genuine!

\v{20}When you gather in the same place, it is not to eat the Lord's Supper. \v{21}For as you eat, each of you rushes to eat his own supper, and one person goes hungry while another gets drunk. \v{22}You have homes in which to eat and drink, don't you? Or do you despise God's church and humiliate those who have nothing? What should I say to you? Should I praise you? I will not praise you for this!

\v{23}For I received from the Lord what I also passed on to you---how the Lord Jesus, on the night he was betrayed, took a loaf of bread, \v{24}gave thanks for it, and broke it in pieces, saying, \red{``This is my body that is}\fnote{\fbackref{11:24} Other mss. read \fbib{that is broken}; still other mss. read \fbib{that is given}} \red{for you. Keep doing this in memory of me.''} \v{25}He did the same with the cup after the supper, saying, \red{``This cup is the new covenant in my blood. As often as you drink from it, keep doing this in memory of me.''} \v{26}For as often as you eat this bread and drink from this cup, you proclaim the Lord's death until he comes.

\v{27}Therefore, whoever eats the bread or drinks from the cup in an unworthy manner will be held responsible for the Lord's body and blood. \v{28}A person must examine himself and then eat the bread and drink from the cup, \v{29}because whoever eats and drinks\fnote{\fbackref{11:29} Other mss. read \fbib{drinks in an unworthy manner}} without recognizing the body,\fnote{\fbackref{11:29} Other mss. read \fbib{the Lord's body}} eats and drinks judgment on himself. \v{30}That's why so many of you are weak and sick and a considerable number are dying.\fnote{\fbackref{11:30} Lit. \fbib{are falling asleep}} \v{31}But if we judged ourselves correctly, we would not be judged. \v{32}Now, while we are being judged by the Lord, we are being disciplined so we won't\fnote{\fbackref{11:32} Lit. \fbib{disciplined lest we}} be condemned along with the world.

\v{33}Therefore, my brothers, when you gather to eat, wait for each other. \v{34}If anyone is hungry, he should eat at home, so that when you gather it may not bring judgment on you. And when I come I will give instructions concerning the other matters.
\labelchapt{12}
\passage{Concerning Spiritual Gifts}

\chapt{12}
\v{1}Now concerning spiritual gifts, brothers, I don't want you to be ignorant. \v{2}You know that when you were unbelievers,\fnote{\fbackref{12:2} Or \fbib{pagans}} you were enticed and led astray to worship\fnote{\fbackref{12:2} The Gk. lacks \fbib{worship}} idols that couldn't even speak. \v{3}For this reason I want you to be aware that no one who is speaking by God's Spirit can say, ``Jesus is cursed,'' and no one can say, ``Jesus is Lord,'' except by the Holy Spirit.

\v{4}Now there are varieties of gifts, but the same Spirit, \v{5}and there are varieties of ministries, but the same Lord. \v{6}There are varieties of results, but it is the same God who produces all the results in everyone.

\v{7}To each person has been given the ability to manifest the Spirit for the common good. \v{8}To one has been given a message of wisdom by the Spirit; to another the ability to speak with knowledge according to the same Spirit; \v{9}to another faith by the same Spirit; to another gifts of healing by that one Spirit; \v{10}to another miraculous results; to another prophecy; to another the ability to distinguish between spirits; to another various kinds of languages; and to another the interpretation of languages. \v{11}But one and the same Spirit produces all these results and gives what he wants to each person.
\passage{The Unity and Diversity of Spiritual Gifts}

\v{12}For just as the body is one and yet has many parts, and all the parts of the body, though many, form a single body, so it is with the Messiah.\fnote{\fbackref{12:12} Or \fbib{Christ}} \v{13}For by\fnote{\fbackref{12:13} Or \fbib{in}} one Spirit all of us---Jews and Greeks, slaves and free---were baptized into one body and were all privileged to drink from one Spirit.

\v{14}For the body does not consist of only one part, but of many. \v{15}If the foot says, ``Since I'm not a hand, I'm not part of the body,'' that does not make it any less a part of the body, does it? \v{16}And if the ear says, ``Since I'm not an eye, I'm not part of the body,'' that does not make it any less a part of the body, does it? \v{17}If the whole body were an eye, where would the sense of hearing be? If the whole body\fnote{\fbackref{12:17} The Gk. lacks \fbib{body}} were an ear, where would the sense of smell be? \v{18}But now God has arranged the parts, every one of them, in the body according to his plan.\fnote{\fbackref{12:18} Lit. \fbib{will}} \v{19}Now if all of it were one part, there wouldn't be a body, would there? \v{20}So there are many parts, but one body.

\v{21}The eye cannot say to the hand, ``I don't need you,'' or the head to the feet, ``I don't need you.'' \v{22}On the contrary, those parts of the body that seem to be weaker are in fact indispensable, \v{23}and the parts of the body that we think are less honorable are treated with special honor, and we make our less attractive parts more attractive. \v{24}However, our attractive parts don't need this. But God has put the body together and has given special honor to the parts that lack it, \v{25}so that there might be no disharmony in the body, but that its parts should have the same concern for each other. \v{26}If one part suffers, every part suffers with it. If one part is praised, every part rejoices with it.

\v{27}Now you are the Messiah's\fnote{\fbackref{12:27} Or \fbib{Christ's}} body and individual parts of it. \v{28}God has appointed in the church first of all apostles, second prophets, third teachers, then those who perform miracles, those who have gifts of healing, those who help others, administrators, and those who speak\fnote{\fbackref{12:28} The Gk. lacks \fbib{those who speak}} various kinds of languages. \v{29}Not all are apostles, are they? Not all are prophets, are they? Not all are teachers, are they? Not all perform miracles, do they? \v{30}Not all have the gift of healing, do they? Not all speak in foreign\fnote{\fbackref{12:30} The Gk. lacks \fbib{foreign}; and so through 14:39} languages, do they? Not all interpret, do they? \v{31}Keep on desiring\fnote{\fbackref{12:31} Or \fbib{You are desiring}} the better gifts. And now I will show you the best way of all.
\labelchapt{13}
\passage{The Supremacy of Love}

\chapt{13}
\v{1}If I speak in the languages of humans and angels but have no love, I have become a reverberating gong or a clashing cymbal. \v{2}If I have the gift of prophecy and can understand all secrets and every form of knowledge, and if I have absolute faith so as to move mountains but have no love, I am nothing. \v{3}Even if I give away everything that I have and sacrifice myself,\fnote{\fbackref{13:3} Other mss. read \fbib{sacrifice my body to be burned}; or \fbib{myself so that I may boast}} but have no love, I gain nothing.

\begin{poetry}
\poeml \v{4}Love is always patient; \\
\poemll    love is always kind; \\
\poeml love is never envious \\
\poemll    or arrogant with pride. \\
\poeml Nor is she conceited, \\
\poeml \v{5}and she is never rude; \\
\poeml she never thinks just of herself \\
\poemll    or ever gets annoyed. \\
\poeml She never is resentful; \\
\poeml \v{6}is never glad with sin; \\
\poeml she's always glad to side with truth, \\
\poemll    and pleased that truth will win.\fnote{\fbackref{13:6} The Gk. lacks \fbib{shall win}} \\
\poeml \v{7}She bears up under everything; \\
\poemll    believes the best in all; \\
\poeml there is no limit to her hope, \\
\poemll    and never will she fall.
\end{poetry}

\v{8}Love never fails. Now if there are prophecies, they will be done away with. If there are languages, they will cease. If there is knowledge, it will be done away with. \v{9}For what we know is incomplete and what we prophesy is incomplete. \v{10}But when what is complete\fnote{\fbackref{13:10} Or \fbib{perfect}} comes, then what is incomplete will be done away with.

\v{11}When I was a child, I spoke like a child, thought like a child, and reasoned like a child. When I became a man, I gave up my childish ways. \v{12}Now we see only an indistinct image in a mirror, but then we will be face to face. Now what I know is incomplete, but then I will know fully, even as I have been fully known.

\v{13}Right now three things remain: faith, hope, and love. But the greatest of these is love.
\labelchapt{14}
\passage{Prophecy and Languages}

\chapt{14}
\v{1}Keep on pursuing love, and keep on desiring spiritual gifts, especially the ability to prophesy. \v{2}For the person who speaks in a foreign\fnote{\fbackref{14:2} The Gk. lacks \fbib{foreign}; and so throughout 14:39} language is not actually speaking to people but to God. Indeed, no one understands him, because he is talking about secrets by the Spirit.\fnote{\fbackref{14:2} Or \fbib{with his spirit}} \v{3}But the person who prophesies speaks to people for their upbuilding, encouragement, and comfort. \v{4}The person who speaks in a foreign language builds himself up, but the person who prophesies builds up the church. \v{5}Now I wish that all of you could speak in foreign languages, but especially that you could prophesy. The person who prophesies is more important than the person who speaks in a foreign language, unless he interprets it so that the church may be built up.

\v{6}Indeed, brothers, if I come to you speaking in foreign languages, what good will I be to you unless I speak to you in some revelation, knowledge, prophecy, or teaching? \v{7}In the same way, lifeless instruments like the flute or harp produce sounds. But if there's no difference in the notes, how can a person tell what tune is being played? \v{8}For example, if a bugle doesn't sound a clear call, who will get ready for battle? \v{9}In the same way, unless you speak an intelligible message with your language, how will anyone know what is being said? You'll be talking into the air!

\v{10}There are, I suppose, many different languages\fnote{\fbackref{14:10} Or \fbib{sounds}} in the world, yet none of them is without meaning. \v{11}If I don't know the meaning of the language,\fnote{\fbackref{14:11} Or \fbib{sound}} I will be a foreigner to the speaker and the speaker will be a foreigner to me. \v{12}In the same way, since you're so desirous of spiritual gifts, you must keep on desiring them for building up the church.

\v{13}Therefore, the person who speaks in a foreign language should pray for the ability to interpret it. \v{14}For if I pray in a foreign language, my spirit prays but my mind is not productive. \v{15}What does this mean? I will pray with my spirit, but I will also pray with my mind. I will sing psalms with my spirit, but I will also sing psalms with my mind. \v{16}Otherwise, if you say a blessing with your spirit, how can an otherwise uneducated person\fnote{\fbackref{14:16} Lit. \fbib{the person who occupies the place of the uneducated}} say ``Amen'' to your thanksgiving, since he does not know what you're saying? \v{17}It's good for you to give thanks, but it does not build up the other person. \v{18}I thank God that I speak in foreign languages more than all of you. \v{19}But in church I would rather speak five words with my mind to instruct others than 10,000 words in a foreign language.

\v{20}Brothers, stop being\fnote{\fbackref{14:20} Or \fbib{do not be}} childish in your thinking. Be like infants with respect to evil, but think like adults. \v{21}In the Law it is written,

\begin{poetry}
\poeml ``By means of foreign languages \\
\poemll    and through the mouths of foreigners \\
\poeml I will speak to this people, \\
\poemll    but even then they will not listen to me,''\fnote{\fbackref{14:21} Isa 28:11-12} \\
\poemlll       declares the Lord.
\end{poetry}

\v{22}Foreign languages, then, are meant to be a sign, not for believers, but for unbelievers, while prophecy is meant, not for unbelievers, but for believers. \v{23}Now if the whole church gathers in the same place and everyone is speaking in foreign languages, when uneducated people or unbelievers come in, they will say that you are out of your mind, won't they? \v{24}But if everyone is prophesying, when an unbeliever or an uneducated person comes in he will be convicted and examined by everything that's happening.\fnote{\fbackref{14:24} The Gk. lacks \fbib{that's happening}} \v{25}His secret, inner heart will become known, and so he will bow down to the ground and worship God, declaring, ``God is truly among you!''
\passage{Maintain Order in the Church}

\v{26}What, then, does this mean,\fnote{\fbackref{14:26} The Gk. lacks \fbib{does this mean}} brothers? When you gather, everyone has a psalm, teaching, revelation, foreign language, or interpretation. Everything must be done for upbuilding. \v{27}If anyone speaks in a foreign language, only two or three at the most should do so, one at a time, and somebody must interpret. \v{28}If an interpreter is not present, the speaker\fnote{\fbackref{14:28} Lit. \fbib{present, he}} should remain silent in the church and speak to himself and God.

\v{29}Two or three prophets should speak, and others should weigh carefully what is said. \v{30}If a revelation is made to another person who is seated, the first person should be silent. \v{31}For everyone can prophesy in turn, so that everyone can be instructed and everyone can be encouraged. \v{32}The spirits of prophets are subject to the prophets, \v{33}for God\fnote{\fbackref{14:33} Lit. \fbib{he}} is not a God of disorder but of peace.

As\fnote{\fbackref{14:33-34} Or \fbib{peace, as in all the churches of the saints.} \fbib{\v{34}The}} in all the churches of the saints, \v{34}the women must keep silent in the churches. They are not allowed to speak out, but must place themselves in submission, as the oral\fnote{\fbackref{14:34} The Gk. lacks \fbib{oral}} law also says. \v{35}If they want to learn anything, they should ask their own husbands at home, for it is inappropriate for a woman to speak out in church.\fnote{\fbackref{14:35} Other mss. place vv. 34 and 35 after v. 40.}

\v{36}Did God's word originate with you? Are you the only ones\fnote{\fbackref{14:36} The Gk. is a pl. masc. pronoun} it has reached? \v{37}If anyone thinks he is a prophet or a spiritual person, he must acknowledge that what I am writing to you is the Lord's command. \v{38}But if anyone ignores this, he should be ignored.\fnote{\fbackref{14:38} Other mss. read \fbib{If he is ignorant of this, he should remain ignorant}}

\v{39}Therefore, my brothers, desire the ability to prophesy, and do not prevent others from speaking in foreign languages. \v{40}But everything must be done in a proper and orderly way.
\labelchapt{15}
\passage{The Resurrection of the Messiah}

\chapt{15}
\v{1}Now I'm making known to you, brothers, the gospel that I proclaimed to you, which you accepted, on which you have taken your stand, \v{2}and by which you are also being saved if you hold firmly to the message I proclaimed to you---unless, of course, your faith was worthless.

\v{3}For I passed on to you the most important points that\fnote{\fbackref{15:3} Or \fbib{to you as matters of great importance what}} I received: The Messiah\fnote{\fbackref{15:3} Or \fbib{Christ}} died for our sins according to the Scriptures, \v{4}he was buried, he was raised on the third day according to the Scriptures---and is still alive!--- \v{5}and he was seen by Cephas,\fnote{\fbackref{15:5} I.e. Peter} and then by the Twelve. \v{6}After that, he was seen by more than 500 brothers at one time, most of whom are still alive, though some have died.\fnote{\fbackref{15:6} Lit. \fbib{have fallen asleep}} \v{7}Next he was seen by James, then by all the apostles, \v{8}and finally he was seen by me, as though I were born abnormally late.

\v{9}For I am the least of the apostles and not even fit to be called an apostle because I persecuted God's church. \v{10}But by God's grace I am what I am, and his grace shown to me was not wasted. Instead, I worked harder than all the others---not I, of course, but God's grace that was with me. \v{11}So, whether it was I or the others, this is what we preach, and this is what you believed.
\passage{The Resurrection of the Dead}

\v{12}Now if we preach that the Messiah\fnote{\fbackref{15:12} Or \fbib{Christ}} has been raised from the dead, how can some of you keep claiming there is no resurrection of the dead? \v{13}If there is no resurrection of the dead, then the Messiah\fnote{\fbackref{15:13} Or \fbib{Christ}} has not been raised, \v{14}and if the Messiah\fnote{\fbackref{15:14} Or \fbib{Christ}} has not been raised, then our message means nothing and your\fnote{\fbackref{15:14} Other mss. read \fbib{our}} faith means nothing. \v{15}In addition, we are found to be false witnesses about God because we testified on God's behalf that he raised the Messiah\fnote{\fbackref{15:15} Or \fbib{Christ}}---whom he did not raise if in fact it is true that the dead are not raised. \v{16}For if the dead are not raised, then the Messiah\fnote{\fbackref{15:16} Or \fbib{Christ}} has not been raised, \v{17}and if the Messiah\fnote{\fbackref{15:17} Or \fbib{Christ}} has not been raised, your faith is worthless and you are still imprisoned by your sins. \v{18}Yes, even those who have died\fnote{\fbackref{15:18} Lit. \fbib{have fallen asleep}} believing\fnote{\fbackref{15:18} The Gk. lacks \fbib{believing}} in the Messiah\fnote{\fbackref{15:18} Or \fbib{Christ}} are lost. \v{19}If we have set our hopes on the Messiah\fnote{\fbackref{15:19} Or \fbib{Christ}} in this life only, we deserve more pity than any other people.

\v{20}But at this moment the Messiah\fnote{\fbackref{15:20} Or \fbib{Christ}} stands risen from the dead, the first one offered in the harvest\fnote{\fbackref{15:20} Lit. \fbib{the first fruits}} of those who have died.\fnote{\fbackref{15:20} Lit. \fbib{have fallen asleep}} \v{21}For since death came through a man, the resurrection of the dead also came through a man. \v{22}For as in Adam all die, so also in the Messiah\fnote{\fbackref{15:22} Or \fbib{Christ}} will all be made alive. \v{23}However, this will happen to each person in the proper order: first the Messiah,\fnote{\fbackref{15:23} Or \fbib{Christ}; lit. \fbib{Messiah the first fruits}} then those who belong to the Messiah\fnote{\fbackref{15:23} Or \fbib{Christ}} when he comes. \v{24}Then the end will come, when after he has done away with every ruler and every authority and power, the Messiah\fnote{\fbackref{15:24} Lit. \fbib{power, when he}} hands over the kingdom to God the Father. \v{25}For he must rule until God\fnote{\fbackref{15:25} Lit. \fbib{he}} puts all the Messiah's\fnote{\fbackref{15:25} Lit. \fbib{his}} enemies under his feet. \v{26}The last enemy to be done away with is death, \v{27}for ``God\fnote{\fbackref{15:27} Lit. \fbib{he}} has put everything under his feet.''\fnote{\fbackref{15:27} Ps 8:6} Now when he says, ``Everything has been put under him,'' this clearly excludes the one who put everything under him. \v{28}But when everything has been put under him, then the Son himself will also become subject to the one who put everything under him, so that God may be all in all.

\v{29}Otherwise, what will those people do who are being baptized because of those who have died? If the dead are not raised at all, why are they being baptized because of them? \v{30}And why in fact are we being endangered every hour? \v{31}I face death every day! That is as certain, brothers,\fnote{\fbackref{15:31} Other mss. lack \fbib{brothers}} as it is that I am proud of you in the Messiah,\fnote{\fbackref{15:31} Or \fbib{Christ}} Jesus our Lord. \v{32}If I have fought with wild animals in Ephesus from merely human motives, what do I get out of it? If the dead are not raised,

\begin{poetry}
\poeml ``Let's eat and drink, for tomorrow we die.''\fnote{\fbackref{15:32} Isa 22:13}
\end{poetry}

\v{33}Stop being deceived:

\begin{poetry}
\poeml ``Wicked friends lead to evil ends.''\fnote{\fbackref{15:33} Menander, \fbib{Thais} (218)}
\end{poetry}

\v{34}Come back to your senses as you should, and stop sinning! For some of you---I say this to your shame---don't fully know God.
\passage{The Resurrection Body}

\v{35}But someone will ask, ``How are the dead raised? What kind of body will they have when they come back?'' \v{36}You fool! The seed you plant does not come to life unless it dies, \v{37}and what you plant is not the form that it will be, but a bare kernel, whether it is wheat or something else. \v{38}But God gives the plant\fnote{\fbackref{15:38} The Gk. lacks \fbib{the plant}} the form he wants it to have, and to each kind of seed its own form. \v{39}Not all flesh is the same.\fnote{\fbackref{15:39} Lit. \fbib{the same flesh}} Humans have one kind of flesh,\fnote{\fbackref{15:39} The Gk. lacks \fbib{of flesh}} animals in general have another,\fnote{\fbackref{15:39} Lit. \fbib{another kind of flesh}} birds have another,\fnote{\fbackref{15:39} Lit. \fbib{another kind of flesh}} and fish have still another. \v{40}There are heavenly bodies and earthly bodies, but the splendor of those in heaven is of one kind, and that of those on earth is of another. \v{41}One kind of splendor belongs to the sun, another\fnote{\fbackref{15:41} Lit. \fbib{another kind of splendor}} to the moon, and still another\fnote{\fbackref{15:41} Lit. \fbib{another kind of splendor}} to the stars. In fact, one star differs from another star in splendor.

\v{42}This is how it will be at the resurrection of the dead. What is planted is decaying, what is raised cannot decay. \v{43}The body\fnote{\fbackref{15:43} Lit. \fbib{It}} is planted in a state of dishonor but is raised in a state of splendor. It is planted in weakness but is raised in power. \v{44}It is planted a physical body but is raised a spiritual body. If there is a physical body, there is also a spiritual body.\fnote{\fbackref{15:44} The Gk. lacks \fbib{body}}

\v{45}This, indeed, is what is written: ``The first man, Adam, became a living being.''\fnote{\fbackref{15:45} Gen 2:7} The last Adam became a life-giving spirit. \v{46}The spiritual does not come first, but the physical does, and then comes the spiritual. \v{47}The first man came from the dust of the earth; the second man came from heaven. \v{48}Those who are made of the dust are like the man from the dust; those who are heavenly are like the man who is from heaven. \v{49}Just as we have borne the likeness of the man who was made from dust, we will\fnote{\fbackref{15:49} Other mss. read \fbib{we should}} also bear the likeness of the man from heaven.

\v{50}Brothers, this is what I mean: Mortal bodies\fnote{\fbackref{15:50} Lit. \fbib{mean: Flesh and blood}} cannot inherit the kingdom of God, and what decays cannot inherit what does not decay. \v{51}Let me tell you a secret. Not all of us will die,\fnote{\fbackref{15:51} Lit. \fbib{will fall asleep}} but all of us will be changed--- \v{52}in a moment, faster than an eye can blink, at the sound of the last trumpet. Indeed, that trumpet\fnote{\fbackref{15:52} Lit. \fbib{it}} will sound, and then the dead will be raised never to decay, and we will be changed. \v{53}For what is decaying must be clothed with what cannot decay, and what is dying must be clothed with what cannot die. \v{54}Now, when what is decaying is clothed with what cannot decay, and what is dying is clothed with what cannot die, then the written word will be fulfilled: ``Death has been swallowed up by victory!''\fnote{\fbackref{15:54} Isa 25:8}

\begin{poetry}
\poeml \v{55}``Where, O death, is your victory? \\
\poemll    Where, O death, is your sting?''\fnote{\fbackref{15:55} Hos 13:14}
\end{poetry}

\v{56}Now death's stinger is sin, and sin's power is the Law. \v{57}But thanks be to God, who gives us the victory through our Lord Jesus the Messiah!\fnote{\fbackref{15:57} Or \fbib{Christ}}

\v{58}Therefore, my dear brothers, be steadfast, unmovable, always excelling in the work of the Lord, because you know that the work that you do for the Lord isn't wasted.
\labelchapt{16}
\passage{Concerning the Collection for the Saints}

\chapt{16}
\v{1}Now concerning the collection for the saints, you should follow the directions I gave to the churches in Galatia. \v{2}After the Sabbath ends,\fnote{\fbackref{16:2} Or \fbib{On the first day of the week}} each of you should set aside and save something from your surplus in proportion to what you have, so that no collections will have to be made when I arrive. \v{3}When I arrive, I will send letters along with the men you approve to take your gift to Jerusalem. \v{4}If it is worthwhile for me to go, too, they can go with me.
\passage{Plans for Travel}

\v{5}I will visit you when I go through Macedonia---for I intend to go through Macedonia--- \v{6}and will probably stay with you for a while\fnote{\fbackref{16:6} The Gk. lacks \fbib{for a while}} or even spend the winter with you.\fnote{\fbackref{16:6} The Gk. lacks \fbib{with you}} Then you can send me on my way, wherever I decide to go. \v{7}I do not want to visit with you now just in passing, because I hope to spend a longer time with you if the Lord permits. \v{8}However, I'll stay on in Ephesus until Pentecost, \v{9}because a door has opened wide for me to do effective work, although many people are opposing me.

\v{10}If Timothy comes, see to it that he does not have anything to be afraid of while he is with you, for he is doing the Lord's work as I am. \v{11}Therefore, no one should treat him with contempt. Send him on his way in peace so that he may come to me, because I am expecting him along with the brothers.

\v{12}Now concerning our brother Apollos, I strongly urged him to visit you with the other\fnote{\fbackref{16:12} The Gk. lacks \fbib{other}} brothers, but he was not inclined to do so just now. However, he will visit you\fnote{\fbackref{16:12} The Gk. lacks \fbib{you}} when the time is right.
\passage{Final Instructions}

\v{13}Remain alert. Keep standing firm in your faith. Keep on being courageous and strong. \v{14}Everything you do should be done lovingly. \v{15}Now I urge you, brothers---for you know that the members of the family of Stephanas were the first converts\fnote{\fbackref{16:15} Lit. \fbib{the first fruits}} in Achaia, and that they have devoted themselves to serving the saints--- \v{16}to submit yourselves to people like these and to anyone else who shares their labor and hard work. \v{17}I am glad that Stephanas, Fortunatus, and Achaicus came here, because what was lacking they have supplied through you. \v{18}They refreshed my spirit---and yours, too. Therefore, appreciate men like that.
\passage{Final Greetings}

\v{19}The churches in Asia greet you. Aquila and Prisca\fnote{\fbackref{16:19} I.e. Priscilla} and the church in their house greet you warmly in union with the Lord. \v{20}All the brothers greet you. Greet one another with a holy kiss.\fnote{\fbackref{16:20} People customarily greeted their friends with a kiss.} \v{21}I, Paul, am writing this greeting with my own hand.

\begin{poetry}
\poeml \v{22}If anyone doesn't love the Lord, \\
\poemll    let him be anathema!\fnote{\fbackref{16:22} This term means \fbib{eternally condemned}} \\
\poemlll       Marana tha!\fnote{\fbackref{16:22} This Aram. sentence means \fbib{May our Lord come!}} \\
\poeml \v{23}May the grace of the Lord Jesus be with you! \\
\poeml \v{24}May my love remain\fnote{\fbackref{16:24} Or \fbib{My love is}} with all of you \\
\poemll    in union with the Messiah\fnote{\fbackref{16:24} Or \fbib{Christ}} Jesus.\fnote{\fbackref{16:24} Other mss. read \fbib{Jesus. Amen}}\end{poetry}

\bookheader{2 Corinthians}
\labelbook{2Cor}

\bookpretitle{The Letter from Paul Called}
\booktitle{Second Corinthians}

\labelchapt{1}
\passage{Paul Greets the Church in Corinth}

\chapt{1}
\v{1}From:\fnote{\fbackref{1:1} The Gk. lacks \fbib{From}} Paul, an apostle of the Messiah\fnote{\fbackref{1:1} Or \fbib{Christ}} Jesus by the will of God, and Timothy our brother.

To: God's church in Corinth, and to all the holy people\fnote{\fbackref{1:1} Or \fbib{the saints}} throughout Achaia.

\v{2}May grace and peace from God our Father and the Lord Jesus, the Messiah,\fnote{\fbackref{1:2} Or \fbib{Christ}} be yours!
\passage{The God of All Comfort}

\v{3}Blessed be the God and Father of our Lord Jesus, the Messiah!\fnote{\fbackref{1:3} Or \fbib{Christ}} He is our merciful Father and the God of all comfort, \v{4}who comforts us in all our suffering, so that we may be able to comfort others in all their suffering, as we ourselves are being comforted by God. \v{5}For as the Messiah's\fnote{\fbackref{1:5} Or \fbib{Christ's}} sufferings overflow into us, so also our comfort overflows through the Messiah.\fnote{\fbackref{1:5} Or \fbib{Christ}} \v{6}If we suffer, it is for your comfort and salvation. If we are comforted, it is for your comfort when you patiently endure the same sufferings that we are suffering. \v{7}Our hope for you is unshaken, because we know that as you share our sufferings, you also share our comfort.
\passage{How God Rescued Paul}

\v{8}For we do not want you to be ignorant, brothers, about the suffering we experienced in Asia. We were so crushed beyond our ability to endure that we even despaired of living. \v{9}In fact, we felt that we had received a death sentence so we would not rely on ourselves but on God, who raises the dead. \v{10}He has rescued us from a terrible death, and he will continue to rescue us. Yes, he is the one on whom we have set our hope, and he will rescue us again, \v{11}as you also help us by your prayers for us. Then many people will thank God\fnote{\fbackref{1:11} The Gk. lacks \fbib{God}} on our behalf because of the favor shown us through the prayers of many.
\passage{Paul's Reason for Boasting}

\v{12}For this is what we boast about: Our conscience testifies that we have conducted ourselves in the world with pure motives and godly sincerity, without earthly wisdom but with God's grace---especially toward you. \v{13}For what we are writing you is nothing more than what you can read and also understand. I hope you will understand completely, \v{14}just as you have already understood us partially, so that on the Day of our\fnote{\fbackref{1:14} Other mss. read \fbib{the}} Lord Jesus we can be your reason to boast, even as you are ours.
\passage{Why Paul's Visit Was Postponed}

\v{15}Because I was confident, I planned to come to you first so you might receive a double blessing. \v{16}I planned to leave you in order to go\fnote{\fbackref{1:16} Lit. \fbib{To go through you}} to Macedonia, and then come back to you from Macedonia, and let you send me on to Judea.

\v{17}When I planned this, I did not do it lightly, did I? Are my plans so fickle\fnote{\fbackref{1:17} Lit. \fbib{according to the flesh}} that I can say ``Yes'' and ``No''\fnote{\fbackref{1:17} Lit. ``\fbib{Yes, yes'' and ``No, no''}} at the same time? \v{18}As certainly as God is faithful, we haven't talked to you with mixed messages like that.\fnote{\fbackref{1:18} Lit. \fbib{faithful, our word to you is not ``Yes'' and ``No''}} \v{19}For God's Son, Jesus the Messiah,\fnote{\fbackref{1:19} Or \fbib{Christ}} who was preached among you by us---by me, Silvanus, and Timothy---was not ``Yes'' and ``No.'' But with him it is always ``Yes.'' \v{20}For all God's promises are ``Yes'' in him. And so through him we can say ``Amen,''\fnote{\fbackref{1:20} Lit. \fbib{through him is the ``Amen''}} to the glory of God. \v{21}Now the one who makes us---and you as well---secure in union with the Messiah\fnote{\fbackref{1:21} Or \fbib{Christ}} and has anointed us is God, \v{22}who has placed his seal on us and has given us the Spirit in our hearts as a down payment.

\v{23}I call upon God as a witness on my behalf that it was in order to spare you that I did not return to Corinth. \v{24}It is not that we are trying to rule over your faith, but rather to work with you for your joy, because you have been standing firm in the faith.
\labelchapt{2}
\passage{Paul's Painful Visit}

\chapt{2}
\v{1}Now\fnote{\fbackref{2:1} Other mss. read \fbib{For}} I decided not to pay you another painful visit. \v{2}After all, if I were to grieve you, who should make me happy but the person I am making sad? \v{3}This is the very reason I wrote you, so that when I did come I might not be made sad by those who should have made me happy. For I had confidence that all of you would share the joy that I have. \v{4}I wrote to you out of great sorrow and anguish of heart---along with many tears---not to make you sad but to let you know how much love I have for you.
\passage{Forgive the Person who Sinned}

\v{5}But if anyone has caused grief, he didn't cause me any grief. To some extent---I don't want to emphasize this too much---it has affected\fnote{\fbackref{2:5} The Gk. lacks \fbib{much---it has affected}} all of you. \v{6}This punishment by the majority is severe enough for such a man. \v{7}So forgive and comfort him, or else he will drown in his excessive grief. \v{8}That's why I'm urging you to assure him of your love. \v{9}I had also written to you to see if you would stand the test and be obedient in every way. \v{10}When you forgive someone, I do, too. Indeed, what I have forgiven---if there was anything to forgive---I did\fnote{\fbackref{2:10} The Gk. lacks \fbib{I did}} in the presence of the Messiah\fnote{\fbackref{2:10} Or \fbib{Christ}} for your benefit, \v{11}so that we may not be outsmarted by Satan. After all, we are not unaware of his intentions.
\passage{Paul's Anxiety and Relief}

\v{12}When I went to Troas on behalf of the gospel of the Messiah,\fnote{\fbackref{2:12} Or \fbib{Christ}} the Lord opened a door for me, \v{13}but my spirit could not find any relief, because I couldn't find Titus, my brother. So I said goodbye to them and went on to Macedonia.

\v{14}But thanks be to God! He always leads us triumphantly by the Messiah\fnote{\fbackref{2:14} Or \fbib{Christ}} and through us spreads everywhere the fragrance of knowing him. \v{15}To God we are the aroma of the Messiah\fnote{\fbackref{2:15} Or \fbib{Christ}} among those who are being saved and among those who are being lost. \v{16}To some people we are a deadly fragrance,\fnote{\fbackref{2:16} Lit. \fbib{a fragrance of death to death}} while to others we are a living fragrance.\fnote{\fbackref{2:16} Lit. \fbib{a fragrance of life to life}} Who is qualified for this? \v{17}At least we are not commercializing God's word like so many others. Instead, we speak with sincerity in the Messiah's\fnote{\fbackref{2:17} Or \fbib{Christ's}} name,\fnote{\fbackref{2:17} The Gk. lacks \fbib{name}} like people who are sent from God and are accountable to God.\fnote{\fbackref{2:17} Lit. \fbib{as from God and before God}}
\labelchapt{3}
\passage{Ministers of the New Covenant}

\chapt{3}
\v{1}Are we beginning to recommend ourselves again? Unlike some people, we do not need letters of recommendation to you or from you, do we? \v{2}You are our letter, written in our hearts and known and read by everyone. \v{3}You are demonstrating that you are the Messiah's\fnote{\fbackref{3:3} Or \fbib{Christ's}} letter, produced by our service, written not with ink but with the Spirit of the living God, not on tablets of stone but on tablets of human hearts.

\v{4}Such is the confidence that we have in God through the Messiah.\fnote{\fbackref{3:4} Or \fbib{Christ}} \v{5}By ourselves we are not qualified to claim that anything comes from us. Rather, our credentials come from God, \v{6}who has also qualified us to be ministers of a new covenant, which is not written but spiritual, because the written text\fnote{\fbackref{3:6} Lit. \fbib{what is written}} brings death, but the Spirit gives life.

\v{7}Now if the ministry of death that was inscribed in letters of stone came with such glory that the people of Israel could not gaze on Moses' face (because the glory was fading away from it), \v{8}will not the Spirit's ministry have even more glory? \v{9}For if the ministry of condemnation has glory, then the ministry of justification has an overwhelming glory. \v{10}In fact, that which once had glory lost its glory, because the other glory surpassed it. \v{11}For if that which fades away came\fnote{\fbackref{3:11} The Gk. lacks \fbib{came}} through glory, how much more does that which is permanent have glory?

\v{12}Therefore, since we have such a hope, we speak very boldly, \v{13}not like Moses, who kept covering his face with a veil to keep the people of Israel from gazing at the end of what was fading away. \v{14}However, their minds were hardened, for to this day the same veil is still there when they read the old covenant. Only in union with the Messiah\fnote{\fbackref{3:14} Or \fbib{Christ}} is that veil removed.\fnote{\fbackref{3:14} Lit. \fbib{is it removed}} \v{15}Yet even to this day, when Moses is read, a veil covers their hearts. \v{16}But whenever a person turns to the Lord, the veil is removed. \v{17}Now the Lord is the Spirit, and where the Lord's Spirit is, there is freedom. \v{18}As all of us reflect the glory of the Lord with unveiled faces, we are becoming more like him with ever-increasing glory by the Lord's Spirit.
\labelchapt{4}
\passage{Treasure in Clay Jars}

\chapt{4}
\v{1}Therefore, since we have this ministry through the mercy shown to us, we do not get discouraged. \v{2}Instead, we have renounced secret and shameful ways. We do not use trickery or pervert God's word. By clear statements of the truth we commend ourselves to everyone's conscience before God.

\v{3}So if our gospel is veiled, it is veiled to those who are dying.\fnote{\fbackref{4:3} Or \fbib{being destroyed}} \v{4}In their case, the god of this world has blinded the minds of those who do not believe to keep them from seeing the light of the glorious gospel of the Messiah,\fnote{\fbackref{4:4} Or \fbib{Christ}} who is the image of God.

\v{5}For we do not preach ourselves, but rather Jesus the Messiah\fnote{\fbackref{4:5} Or \fbib{Christ}} as Lord, and ourselves as merely your servants for Jesus' sake. \v{6}For God, who said, ``Let light shine out of darkness,''\fnote{\fbackref{4:6} Gen 1:3} has shone in our hearts to give us the light of the knowledge of God's glory in the face of Jesus\fnote{\fbackref{4:6} Other mss. lack \fbib{Jesus}} the Messiah.\fnote{\fbackref{4:6} Or \fbib{Christ}; other mss. read \fbib{of the Messiah Jesus}}

\v{7}But we have this treasure in clay jars to show that its extraordinary power comes from God and not from us. \v{8}In every way we're troubled but not crushed, frustrated but not in despair, \v{9}persecuted but not abandoned, struck down but not destroyed. \v{10}We are always carrying around the death of Jesus in our bodies, so that the life of Jesus may be clearly shown in our bodies. \v{11}While we are alive, we are constantly being handed over to death for Jesus' sake, so that the life of Jesus may be clearly shown in our mortal bodies. \v{12}And so death is at work in us, but life is at work\fnote{\fbackref{4:12} The Gk. lacks \fbib{is at work}} in you.

\v{13}Now since we have the same spirit of faith in keeping with this Scripture: ``I believed, and so I spoke,''\fnote{\fbackref{4:13} Ps 116:10} we also believe and therefore speak. \v{14}We know that the one who raised the Lord Jesus will also raise us with Jesus and present us to God\fnote{\fbackref{4:14} The Gk. lacks \fbib{to God}} together with you. \v{15}All this is for your sake so that, as his grace spreads, more and more people will give thanks and glorify God.
\passage{Life in an Earthly Tent}

\v{16}That's why we are not discouraged. No, even if outwardly we are wearing out, inwardly we are being renewed each and every day. \v{17}This light, temporary nature of our suffering is producing for us an everlasting weight of glory, far beyond any comparison, \v{18}because we do not look for things that can be seen but for things that cannot be seen. For things that can be seen are temporary, but things that cannot be seen are eternal.
\labelchapt{5}

\chapt{5}
\v{1}We know that if the earthly tent we live in is torn down, we have a building in heaven that comes from God, an eternal house not built by human\fnote{\fbackref{5:1} The Gk. lacks \fbib{human}} hands. \v{2}For in this one we sigh, since we long to put on our heavenly dwelling. \v{3}Of course, if we do put it on, we will not be found without a body.\fnote{\fbackref{5:3} Lit. \fbib{found naked}} \v{4}So while we are still in this tent, we sigh under our burdens, because we do not want to put it off but to put it on, so that our dying bodies may be swallowed up by life. \v{5}God has prepared us for this and has given us his Spirit as a guarantee.

\v{6}Therefore, we are always confident, and we know that as long as we are at home in this body we are away from the Lord. \v{7}For we live by faith, not by sight. \v{8}We are confident, then, and would prefer to be away from this body and to live with the Lord. \v{9}So whether we are at home or away from home, our goal is to be pleasing to him. \v{10}For all of us must appear before the judgment seat of the Messiah,\fnote{\fbackref{5:10} Or \fbib{Christ}} so that each of us may receive what he deserves for what he has done in his body, whether good or worthless.\fnote{\fbackref{5:10} Or \fbib{bad}}
\passage{The Messiah's Love Controls Us}

\v{11}Therefore, since we know what it means to fear the Lord, we try to persuade people. We ourselves are perfectly known to God. I hope we are also really known to your consciences. \v{12}We are not recommending ourselves to you again but are giving you a reason to be proud of us, so that you can answer those who are proud of outward things rather than inward character.\fnote{\fbackref{5:12} Lit. \fbib{rather than the heart}} \v{13}So if we were crazy, it was for God; if we are sane, it is for you. \v{14}The love of the Messiah\fnote{\fbackref{5:14} Or \fbib{Christ}} controls us, for we are convinced of this: that one person died for all people; therefore, all people have died. \v{15}He died for all people, so that those who live should no longer live for themselves but for the one who died and rose for them.

\v{16}So then, from now on we do not think of anyone from a human point of view.\fnote{\fbackref{5:16} Lit. \fbib{according to the flesh}} Even if we did think of the Messiah\fnote{\fbackref{5:16} Or \fbib{Christ}} from a human point of view,\fnote{\fbackref{5:16} Lit. \fbib{according to the flesh}} we don't think of him that way anymore. \v{17}Therefore, if anyone is in the Messiah,\fnote{\fbackref{5:17} Or \fbib{Christ}} he is a new creation. Old things have disappeared, and---look!---all things have become new!

\v{18}All of this comes from God, who has reconciled us to himself through the Messiah\fnote{\fbackref{5:18} Or \fbib{Christ}} and has given us the ministry of reconciliation, \v{19}for through the Messiah,\fnote{\fbackref{5:19} Or \fbib{Christ}} God was reconciling the world to himself by not counting their sins against them. He has committed his message of reconciliation to us. \v{20}Therefore, we are the Messiah's\fnote{\fbackref{5:20} Or \fbib{Christ's}} representatives, as though God were pleading through us. We plead on the Messiah's\fnote{\fbackref{5:20} Or \fbib{Christ's}} behalf: ``Be reconciled to God!'' \v{21}God\fnote{\fbackref{5:21} Lit. \fbib{He}} made the one who did not know sin to be sin for us, so that God's righteousness would be produced in us.\fnote{\fbackref{5:21} Lit. \fbib{that we might become God's righteousness in him}}
\labelchapt{6}
\passage{Workers with God}

\chapt{6}
\v{1}Since, then, we are working with God,\fnote{\fbackref{6:1} Lit. \fbib{working together}} we plead with you not to accept God's grace in vain. \v{2}For he says,

\begin{poetry}
\poeml ``At the right time I heard you, \\
\poeml and on a day of salvation I helped you.''\fnote{\fbackref{6:2} Isa 49:8}
\end{poetry}

Listen, now is really the ``right time''! Now is the ``day of salvation''!
\passage{We are God's Servants}

\v{3}We do not put an obstacle in anyone's way. Otherwise, fault may be found with our ministry. \v{4}Instead, in every way we demonstrate that we are God's servants by tremendous endurance in the midst of difficulties, hardships, and calamities; \v{5}in beatings, imprisonments, and riots; in hard work, sleepless nights, and hunger; \v{6}with purity, knowledge, patience, and kindness; with the Holy Spirit, genuine love, \v{7}truthful speech, and divine power; through the weapons of righteousness in the right and left hands; \v{8}through honor and dishonor; through ill repute and good repute; perceived\fnote{\fbackref{6:8} The Gk. lacks \fbib{perceived}} as deceivers and yet true, \v{9}as unknown and yet well-known, as dying and yet---as you see---very much alive, as punished and yet not killed, \v{10}as sorrowful and yet always rejoicing, as poor and yet enriching many, as having nothing and yet possessing everything.

\v{11}We have spoken frankly\fnote{\fbackref{6:11} Lit. \fbib{Our mouth is open}} to you, Corinthians. Our hearts are wide open. \v{12}We have not cut you off, but you have cut off your own feelings toward us. \v{13}Do us a favor---I ask you as my children---and open wide your hearts.
\passage{Relating with Unbelievers}

\v{14}Stop becoming\fnote{\fbackref{6:14} Or \fbib{Don't become}} unevenly yoked with unbelievers. What partnership can righteousness have with lawlessness? What fellowship can light have with darkness? \v{15}What harmony exists between the Messiah\fnote{\fbackref{6:15} Or \fbib{Christ}} and Beliar,\fnote{\fbackref{6:15} I.e. the devil} or what do a believer and an unbeliever have in common? \v{16}What agreement can a temple of God make with idols? For we\fnote{\fbackref{6:16} Other mss. read \fbib{you} (pl.)} are the temple of the living God, just as God said:

\begin{poetry}
\poeml ``I will live and walk among them. \\
\poemll    I will be their God, \\
\poemlll       and they will be my people.''\fnote{\fbackref{6:16} Lev 26:12; Ezek 37:27}
\end{poetry}

\v{17}Therefore,

\begin{poetry}
\poeml ``Get away from them \\
\poemll    and separate yourselves from them,'' \\
\poemlll       declares the Lord,\fnote{\fbackref{6:17} MT source citation reads \fbib{}\divine{Lord}} \\
\poeml ``and don't touch anything unclean. \\
\poemll    Then I will welcome you. \\
\poeml \v{18}I will be your Father, \\
\poemll    and you will be my sons and daughters,'' \\
\poemlll       declares the Lord\fnote{\fbackref{6:18} MT source citation reads \fbib{}\divine{Lord}} Almighty.\fnote{\fbackref{6:18} Isa 52:11; Ezek 20:34, 41; 2 Sam 7:8, 14}
\end{poetry}
\labelchapt{7}
\passage{Cleanse Yourselves in Holiness}

\chapt{7}
\v{1}Since we have these promises, dear friends, let's cleanse ourselves from everything that contaminates body and spirit by becoming mature in our holy fear of God.
\passage{Encouraged by the Corinthians}

\v{2}Make room for us in your hearts!\fnote{\fbackref{7:2} The Gk. lacks \fbib{in your hearts}} We have not treated anyone unjustly, harmed anyone, or cheated anyone. \v{3}I am not saying this to condemn you. I told you before that you are in our hearts to die together and to live together. \v{4}I have great confidence in you. I am very proud of you. I am very much encouraged. I am overjoyed in all our troubles.

\v{5}For even when we came to Macedonia, our bodies had no rest. We suffered in a number of ways. Outwardly there were conflicts, inwardly there were fears. \v{6}Yet God, who comforts those who feel miserable, comforted us by the arrival of Titus, \v{7}and not only by his arrival but also by the comfort he had received from you. He told us about your longing for me, your sorrow, and your eagerness to take my side, and this made me even happier.

\v{8}If I made you sad with my letter, I do not regret it, although I did regret it then. I see that the letter caused you sorrow, though only for a while. \v{9}Now I am happy, not because you had such sorrow, but because your sorrow led you to repent. For you were sorry in a godly way, and so you were not hurt by us in any way. \v{10}For having sorrow in a godly way results in repentance that leads to salvation and leaves no regrets. But the sorrow of the world produces death.

\v{11}See what great earnestness godly sorrow has produced in you! How ready you are to clear yourselves, how indignant, how alarmed, how full of longing and enthusiasm, how eager to seek justice! In every way you have demonstrated that you are innocent in this matter. \v{12}So, even though I wrote to you, it wasn't because of the man who did the wrong or because of the man who was hurt. Instead, I wrote to you so that your devotion to us might be made perfectly clear to you before God.

\v{13}This is what comforted us. In addition to our own comfort, we were even more delighted at the joy of Titus, because his spirit had been set at rest by all of you. \v{14}For if I have been doing some boasting about you to him, I have never been ashamed of it. Moreover, since everything we told you was true, our boasting to Titus has also proved to be true. \v{15}His affection for you is even greater as he remembers how obedient all of you were and how you welcomed him with fear and trembling. \v{16}I rejoice that I can have complete confidence in you.
\labelchapt{8}
\passage{The Collection for the Christians in Jerusalem}

\chapt{8}
\v{1}We want you to know, brothers, about God's grace that was given to the churches of Macedonia. \v{2}In spite of their terrible ordeal of suffering, their abundant joy and deep poverty have led them to be abundantly generous. \v{3}I can testify that by their own free will they have given to the utmost of their ability, yes, even beyond their ability. \v{4}They begged us earnestly for the privilege\fnote{\fbackref{8:4} Or \fbib{for the grace}} of participating in this ministry to the saints. \v{5}We did not expect that! They gave themselves to the Lord first and then to us, since this was God's will. \v{6}So we urged Titus to finish this work of kindness\fnote{\fbackref{8:6} Or \fbib{this grace}} among you in the same way that he had started it. \v{7}Indeed, the more your faith, speech, knowledge, enthusiasm, and love for us increase, the more we want you to be rich in this work of kindness.\fnote{\fbackref{8:7} Or \fbib{this grace}}

\v{8}I am not commanding you but testing the genuineness of your love by the enthusiasm of others. \v{9}For you know the grace of our Lord Jesus, the Messiah.\fnote{\fbackref{8:9} Or \fbib{Christ}} Although he was rich, for your sakes he became poor, so that you, through his poverty, might become rich.

\v{10}I am giving you my opinion on this matter because it will be helpful to you. Last year you were not only willing to do something, but had already started to do it. \v{11}Now finish what you began, so that your eagerness to do so may be matched by your eagerness\fnote{\fbackref{8:11} Lit. \fbib{matched as you contribute from what you have}} to complete it. \v{12}For if the eagerness is there, the gift\fnote{\fbackref{8:12} Lit. \fbib{it}} is acceptable according to what you have, not according to what you do not have.

\v{13}Not that others should have relief while you have hardship. Rather, it is a question of fairness. \v{14}At the present time, your surplus fills their need, so that their surplus may fill your need. In this way things are fair. \v{15}As it is written,

\begin{poetry}
\poeml ``The person who had much did not have too much, \\
\poemll    and the person who had little did not have too little.''\fnote{\fbackref{8:15} Exod 16:18}
\end{poetry}
\passage{Titus and His Companions}

\v{16}But thanks be to God, who placed in the heart of Titus the same dedication to you that I have. \v{17}He welcomed my request and eagerly went to visit you by his own free will. \v{18}With him we have sent the brother who is praised in all the churches for spreading the gospel.\fnote{\fbackref{8:18} Lit. \fbib{in the gospel}} \v{19}More than that, he has also been selected by the churches to travel with us while we are administering this work of kindness\fnote{\fbackref{8:19} Or \fbib{this grace}} for the glory of the Lord and as evidence of our eagerness to help.\fnote{\fbackref{8:19} The Gk. lacks \fbib{to help}} \v{20}We are trying to avoid any criticism of the way we are administering this great undertaking. \v{21}We intend to do what is right, not only in the sight of the Lord, but also in the sight of people.

\v{22}We have also sent with them our brother whom we have often tested in many ways and found to be dedicated. At present he is more dedicated than ever because he has so much confidence in you.

\v{23}As for Titus, he is my partner and fellow worker on your behalf. Our brothers, emissaries\fnote{\fbackref{8:23} Or \fbib{apostles}} from the churches, are bringing glory to the Messiah.\fnote{\fbackref{8:23} Or \fbib{Christ}} \v{24}Therefore, demonstrate to the churches that you love them and show them publicly why we boast about you.
\labelchapt{9}
\passage{Why Giving is Important}

\chapt{9}
\v{1}I do not need to write to you any further about the ministry to the saints. \v{2}For I know how willing you are, and I boast about you to the people of Macedonia, saying\fnote{\fbackref{9:2} The Gk. lacks \fbib{saying}} that Achaia has been ready since last year, and your enthusiasm has stimulated most of them. \v{3}Now I have sent the brothers so that our boasting about you in this matter may not prove to be an idle boast, and so that you may stand ready, just as I said. \v{4}Otherwise, if any Macedonians come with me and find out that you are not ready, we would be humiliated---to say nothing of you---in this undertaking. \v{5}Therefore, I thought it necessary to urge these brothers to visit you ahead of me, to make arrangements in advance for this gift you promised, and to have it ready as something given generously and not forced.

\v{6}Remember\fnote{\fbackref{9:6} Lit. \fbib{Now}} this: The person who sows sparingly will also reap sparingly, and the person who sows generously will also reap generously. \v{7}Each of you must give what you have decided in your heart, not with regret or under compulsion, since God loves a cheerful giver. \v{8}Besides, God is able to make every blessing of yours overflow for you, so that in every situation you will always have all you need for any good work. \v{9}As it is written,

\begin{poetry}
\poeml ``He scatters everywhere and gives to the poor; \\
\poemll    his righteousness lasts forever.''\fnote{\fbackref{9:9} Ps 112:9}
\end{poetry}

\v{10}Now he who supplies seed to the farmer and bread to eat will also supply you with seed and multiply it and enlarge the harvest that results from your righteousness. \v{11}In every way you will grow richer and become even more generous, and this will cause others to give thanks to God because of us, \v{12}since this ministry you render is not only fully supplying the needs of the saints, it is also overflowing with more and more prayers of thanksgiving to God. \v{13}Because your service in giving proves your love,\fnote{\fbackref{9:13} The Gk. lacks \fbib{your love}} you will be glorifying God as you obey what your confession of the Messiah's\fnote{\fbackref{9:13} Or \fbib{Christ's}} gospel demands,\fnote{\fbackref{9:13} The Gk. lacks \fbib{demands}} since you are generous in sharing with them and with everyone else. \v{14}And so in their prayers for you they will long for you because of God's exceptional grace that was shown to you. \v{15}Thanks be to God for his indescribable gift!
\labelchapt{10}
\passage{Paul's Authority to Speak Forcefully}

\chapt{10}
\v{1}Now I myself, Paul, plead with you with the gentleness and kindness of the Messiah\fnote{\fbackref{10:1} Or \fbib{Christ}}---I who am humble when I am face to face with you but forceful toward you when I am away! \v{2}I beg you that when I come I will not need to be courageous by daring to oppose some people who think that we are living according to the flesh. \v{3}Of course, we are living in the world,\fnote{\fbackref{10:3} Lit. \fbib{flesh}} but we do not wage war in a world-like\fnote{\fbackref{10:3} Lit. \fbib{fleshly}} way. \v{4}For the weapons of our warfare are not those of the world.\fnote{\fbackref{10:4} Lit. \fbib{flesh}} Instead, they have the power of God to demolish fortresses. We tear down arguments \v{5}and every proud obstacle that is raised against the knowledge of God, taking every thought captive in order to obey the Messiah.\fnote{\fbackref{10:5} Or \fbib{Christ}} \v{6}Once your obedience is complete, we will be ready to reprimand every type of disobedience.

\v{7}Look at the plain facts! If anyone is confident that he belongs to the Messiah,\fnote{\fbackref{10:7} Or \fbib{Christ}} he should remind himself of this: Just as he belongs to the Messiah,\fnote{\fbackref{10:7} Or \fbib{Christ}} so do we. \v{8}So if I boast a little too much about our authority, which the Lord gave us to build you up and not to tear you down, I will not be ashamed of it.

\v{9}I do not want you to think that I am trying to frighten you with my letters. \v{10}For someone is saying,\fnote{\fbackref{10:10} The Gk. lacks \fbib{For someone is saying}} ``His letters are impressive and forceful, but his bodily presence is weak and his speech contemptible.'' \v{11}Someone like this should take note of the following: What we say by letter when we are absent is what we will do when present!
\passage{Paul's Reason for Boasting}

\v{12}We would not dare put ourselves in the same class with, or compare ourselves to, those who recommend themselves. Whenever they measure themselves by their own standards or compare themselves among themselves, they show how foolish they are. \v{13}We will not boast about what cannot be evaluated. Instead, we will stay within the field that God assigned us, so as to reach even you. \v{14}For it is not as though we were overstepping our limits when we came to you. We were the first to reach you with the gospel of the Messiah.\fnote{\fbackref{10:14} Or \fbib{Christ}} \v{15}We are not boasting about work done by others that cannot be evaluated. On the contrary, we cherish the hope that your faith may continue to grow and enlarge our sphere of action among you until it overflows. \v{16}Then we can preach the gospel in the regions far beyond you without boasting about things already accomplished by someone else.

\v{17}``The person who boasts should boast in the Lord.''\fnote{\fbackref{10:17} Jer 9:24; MT source citation reads \fbib{}\divine{Lord}} \v{18}It is not the person who commends himself who is approved, but the person whom the Lord commends.
\labelchapt{11}
\passage{Paul Contrasts Himself with False Apostles}

\chapt{11}
\v{1}I wish you would tolerate a little of my foolishness. Yes, please tolerate me! \v{2}I am jealous of you with God's own jealousy, because I promised you in marriage to one husband, to present you as a pure virgin to the Messiah.\fnote{\fbackref{11:2} Or \fbib{Christ}} \v{3}However, I am afraid that just as the serpent deceived Eve by its tricks, so your minds may somehow be lured away from sincere and pure\fnote{\fbackref{11:3} Other mss. lack \fbib{and pure}} devotion to the Messiah.\fnote{\fbackref{11:3} Or \fbib{Christ}}

\v{4}For if someone comes along and preaches another Jesus than the one we preached, or should you receive a different spirit from the one you received or a different gospel from the one you accepted, you are all too willing to listen. \v{5}I do not think I'm inferior in any way to those ``super-apostles.'' \v{6}Even though I may be untrained as an orator, I am not so in the field of knowledge. We have made this clear to all of you in every possible way.

\v{7}Did I commit a sin when I humbled myself by proclaiming to you the gospel of God free of charge, so that you could be exalted? \v{8}I robbed other churches by accepting support from them in order to serve you. \v{9}When I was with you and needed something, I did not bother any of you, because our brothers who came from Macedonia supplied everything I needed. I kept myself from being a burden to you in any way, and I will continue to do so.

\v{10}As surely as the truth of the Messiah\fnote{\fbackref{11:10} Or \fbib{Christ}} is in me, my boasting will not be silenced in the regions of Achaia. \v{11}Why? Because I do not love you? God knows that I do!

\v{12}But I will go on doing what I'm doing in order to deny an opportunity to those people who want an opportunity to be recognized as our equals in the work they are boasting about. \v{13}Such people are false apostles, dishonest workers who are masquerading as apostles of the Messiah.\fnote{\fbackref{11:13} Or \fbib{Christ}} \v{14}And no wonder, since Satan himself masquerades as an angel of light. \v{15}So it is not surprising if his servants also masquerade as servants of righteousness. Their doom\fnote{\fbackref{11:15} Lit. \fbib{end}} will match their deeds!
\passage{Paul's Sufferings as an Apostle}

\v{16}I will say it again: No one should think that I am a fool. But if you do, then treat me like a fool so that I can also boast a little. \v{17}When I talk as a confident boaster, I am not talking with the Lord's authority but like a fool. \v{18}Since many people boast in a fleshly way, I will do it, too. \v{19}You are wise, so you will gladly be tolerant of fools. \v{20}You tolerate anyone who makes you his slaves, devours what you have, takes what is yours, orders you around, or slaps your face!

\v{21}I am ashamed to admit it, but we have been too weak for that. Whatever anyone else dares to claim---I am talking like a fool---I can claim it, too. \v{22}Are they Hebrews? So am I. Are they Israelis? So am I. Are they among Abraham's descendants? So am I. \v{23}Are they the Messiah's\fnote{\fbackref{11:23} Or \fbib{Christ's}} servants? I am insane to talk like this, but I am a far better one! I have been involved in far greater efforts, far more imprisonments, countless beatings, and have faced death more than once. \v{24}Five times I received from the Jews 40 lashes minus one. \v{25}Three times I was beaten with a stick, once I was pelted with stones, three times I was shipwrecked, and I drifted on the sea for a day and a night. \v{26}I have traveled extensively and have been endangered from rivers, robbers, my own people, and gentiles. I've also been in danger in the city, in the open country, at sea, from false brothers, \v{27}in toil and hardship, through many a sleepless night, through hunger, thirst, many periods of fasting, coldness, and nakedness. \v{28}Besides everything else, I have a daily burden because of my anxiety about all the churches. \v{29}Who is weak without me being weak, too? Who is caused to stumble without me becoming indignant?

\v{30}If I must boast, I will boast about the things that show how weak I am. \v{31}The God and Father of the Lord Jesus, who is blessed forever, knows that I am not lying. \v{32}In Damascus, the governor under King Aretas put guards around the city of Damascus to catch me, \v{33}but I was let down in a basket through an opening in the wall and escaped from him.
\labelchapt{12}
\passage{Paul's Thorn}

\chapt{12}
\v{1}I must boast, although it does not do any good. Let's talk about visions and revelations from the Lord. \v{2}I know a man who belongs to the Messiah.\fnote{\fbackref{12:2} Or \fbib{Christ}} Fourteen years ago---whether in his body or outside of his body, I do not know, but God knows---that man was snatched away to the third heaven. \v{3}I know that this man---whether in his body or outside of his body, I do not know, but God knows--- \v{4}was snatched away to Paradise and heard things that cannot be expressed in words, things that no human being has a right even to mention.

\v{5}I will boast about this man, but as for myself I will boast only about my weaknesses. \v{6}However, if I did want to boast, I would not be a fool, because I would be telling the truth. But I am not going to do it in order to keep anyone from thinking more of me than what he sees and hears about me.

\v{7}To keep me from becoming conceited because of the exceptional nature of these revelations, a thorn\fnote{\fbackref{12:7} Or \fbib{stake}} was given to me and placed in my body.\fnote{\fbackref{12:7} Lit. \fbib{was given to me in the flesh}} It was Satan's messenger to keep on tormenting me so that I would not become conceited.

\v{8}I pleaded with the Lord three times to take it away from me, \v{9}but he has told me, \red{``My grace is all you need, because my power is perfected in weakness.''} Therefore, I will most happily boast about my weaknesses, so that the Messiah's\fnote{\fbackref{12:9} Or \fbib{Christ's}} power may rest on me. \v{10}That is why I take such pleasure in weaknesses, insults, hardships, persecutions, and difficulties for the Messiah's\fnote{\fbackref{12:10} Or \fbib{Christ's}} sake, for when I am weak, then I am strong.
\passage{Concern for the Corinthians}

\v{11}I have become a fool. You forced me to be one. Really, I should have been commended by you, for I am not in any way inferior to your ``super-apostles,'' even if I am nothing. \v{12}The signs of an apostle were performed among you with utmost patience---signs, wonders, and powerful actions. \v{13}How were you treated worse than the other churches, except that I did not bother you for help? Forgive me for this wrong! \v{14}Now I'm ready to visit you for a third time, and I will not bother you for help. I do not want your things, but rather you yourselves. Children should not have to support\fnote{\fbackref{12:14} Lit. \fbib{to save up for}} their parents, but parents their children. \v{15}I will be very glad to spend my money and myself for you. Do you love me less because I love you so much?

\v{16}Granting that I have not been a burden to you, was I a clever schemer who trapped you by some trick? \v{17}I did not take advantage of you through any of the men I sent you, did I? \v{18}I encouraged Titus to visit you, and I sent along with him the brother you know so well. Titus didn't take advantage of you, did he? We conducted ourselves with the same spirit, didn't we? We took the very same steps, didn't we?

\v{19}Have you been thinking all along that we are trying to defend ourselves before you? We are speaking before God in the authority of\fnote{\fbackref{12:19} The Gk. lacks \fbib{the authority of}} the Messiah,\fnote{\fbackref{12:19} Or \fbib{Christ}} and everything, dear friends, is meant to build you up. \v{20}I am afraid that I may come and somehow find you not as I want to find you, and that you may find me not as you want to find me. Perhaps there will be quarreling, jealousy, anger, selfishness, slander, gossip, arrogance, and disorderly conduct. \v{21}I am afraid that when I come my God may again humble me before you and that I may have to grieve over many who formerly lived in sin and have not repented of their impurity, sexual immorality, and promiscuity that they once practiced.
\labelchapt{13}
\passage{Final Warnings}

\chapt{13}
\v{1}This will be the third time I am coming to you. ``Every accusation must be verified by two or three witnesses.''\fnote{\fbackref{13:1} Deut 19:15} \v{2}I have already warned those who sinned previously and all the rest. Although I am absent now, I am warning them as I did on my second visit: If I come back, I will not spare you, \v{3}since you want proof that the Messiah\fnote{\fbackref{13:3} Or \fbib{Christ}} is speaking through me. He is not weak in dealing with you but is making his power felt among you. \v{4}Though he was crucified in weakness, he lives by God's power. We are weak with him, but by God's power we will live for you.

\v{5}Keep examining yourselves to see whether you are continuing in the faith. Test yourselves! You know, don't you, that Jesus the Messiah\fnote{\fbackref{13:5} Or \fbib{Christ}} lives in you? Could it be that you are failing the test? \v{6}I hope you will realize that we haven't failed our test. \v{7}We pray to God that you will not do anything wrong---not to show that we have not failed the test, but so that you may do what is right, even if we seem to have failed. \v{8}For we cannot do anything against the truth, but only for the truth. \v{9}We are glad when we are weak and you are strong. That is what we are praying for---your maturity.

\v{10}For this reason I am writing this while I am away from you: When I come I do not want to be severe in using the authority the Lord gave me to build you up and not to tear you down.
\passage{Final Greetings and Benediction}

\v{11}Finally, brothers, goodbye. Keep on growing to maturity. Keep listening to my appeals. Continue agreeing with each other and living in peace. Then the God of love and peace will be with you. \v{12}Greet one another with a holy kiss.\fnote{\fbackref{13:12} People customarily greeted their friends with a kiss.} \v{13}All the saints greet you.

\v{14}May the grace of the Lord Jesus the Messiah,\fnote{\fbackref{13:14} Or \fbib{Christ}} the love of God, and the fellowship of the Holy Spirit be with all of you!

\bookheader{Galatians}
\labelbook{Gal}

\bookpretitle{The Letter from Paul to the}
\booktitle{Galatians}

\labelchapt{1}
\passage{Greetings from Paul}

\chapt{1}
\v{1}From:\fnote{The Gk. lacks \fbib{From}} Paul---an apostle not sent\fnote{The Gk. lacks \fbib{sent}} from men or by a man, but by Jesus the Messiah,\fnote{Or \fbib{Christ}} and God the Father, who raised him from the dead--- \v{2}and all the brothers who are with me.

To: The churches in Galatia.

\v{3}May grace and peace from God our Father and the Lord Jesus, the Messiah,\fnote{Or \fbib{Christ}} be yours! \v{4}He gave himself for our sins in order to rescue us from this present evil age according to the will of our God and Father. \v{5}To him be the glory forever and ever! Amen.
\passage{There is No Other Gospel}

\v{6}I am astonished that you are so quickly deserting the one who called you by the grace of the Messiah\fnote{Or \fbib{Christ}} and, instead, are following\fnote{Lit. \fbib{Messiah for}} a different gospel, \v{7}not that another one really exists. To be sure, there are certain people who are troubling you and want to distort the gospel about the Messiah.\fnote{Or \fbib{Christ}} \v{8}But even if we or an angel from heaven should proclaim to you\fnote{Other mss. lack \fbib{to you}} a gospel contrary to what we proclaimed to you, let that person be condemned! \v{9}What we have told you in the past I am now telling you again: If anyone proclaims to you a gospel contrary to what you received, let that person be condemned! \v{10}Am I now trying to win the approval of people or of God? Or am I trying to please people? If I were still trying to please people, I would not be the Messiah's\fnote{Or \fbib{Christ's}} servant.\fnote{Or \fbib{slave}}
\passage{Jesus Himself Gave Paul His Message}

\v{11}For\fnote{Other mss. read \fbib{Now}} I want you to know, brothers, that the gospel that was proclaimed by me is not of human origin. \v{12}For I did not receive it from a man, nor was I taught it, but it was revealed to me by Jesus the Messiah.\fnote{Or \fbib{Christ}} \v{13}For you have heard about my earlier life in Judaism---how I kept violently persecuting God's church and was trying to destroy it. \v{14}I advanced in Judaism beyond many of my contemporaries, because I was far more zealous for the traditions of my ancestors.

\v{15}But when God, who set me apart before I was born and who called me by his grace, was pleased \v{16}to reveal his Son to me so that I might proclaim him among the gentiles, I did not confer with another human being\fnote{Lit. \fbib{with flesh and blood}} at any time, \v{17}nor did I go up to Jerusalem to see\fnote{The Gk. lacks \fbib{see}} those who were apostles before me. Instead, I went away to Arabia and then came back to Damascus.

\v{18}Then three years later, I went up to Jerusalem to become acquainted with Cephas,\fnote{I.e. Peter} and I stayed with him for fifteen days. \v{19}But I did not see any other apostle except James, the Lord's brother. \v{20}(Before God, what I'm writing to you is the truth.)\fnote{Lit. \fbib{is not a lie}} \v{21}Then I went to the regions of Syria and Cilicia. \v{22}But the churches of the Messiah\fnote{Or \fbib{Christ}} that are in Judea did not yet know me personally. \v{23}The only thing they kept hearing was this: ``The man who used to persecute us is now proclaiming the faith he once tried to destroy!'' \v{24}So they kept glorifying God for what had happened to\fnote{The Gk. lacks \fbib{what had happened to}} me.
\labelchapt{2}
\passage{How Paul Was Accepted by the Apostles in Jerusalem}

\chapt{2}
\v{1}Then fourteen years later, I again went up to Jerusalem with Barnabas, taking Titus with me. \v{2}I went in response to a revelation, and in a private meeting with the reputed leaders, I explained to them the gospel that I'm proclaiming to the gentiles. I did this because I was afraid that\fnote{Lit. \fbib{Lest somehow}} I was running or had run my life's race\fnote{The Gk. lacks \fbib{my life's race}} for nothing. \v{3}But not even Titus, who was with me, was forced to be circumcised, even though he was a Greek. \v{4}However, false brothers were secretly brought in. They slipped in to spy on the freedom we have in the Messiah\fnote{Or \fbib{Christ}} Jesus so that they might enslave us. \v{5}But we did not give in to them for a moment, so that the truth of the gospel might always remain with you.

\v{6}Now those who were reputed to be important added nothing to my message.\fnote{Lit. \fbib{to me}} (What sort of people they were makes no difference to me, since God pays no attention to outward appearances.) \v{7}In fact, they saw that I had been entrusted with the gospel for the uncircumcised, just as Peter had been entrusted with the gospel for the circumcised. \v{8}For the one who worked through Peter by making him an apostle to the circumcised also worked through me by sending me to the gentiles. \v{9}So when James, Cephas,\fnote{I.e. Peter} and John (who were reputed to be leaders)\fnote{Lit. \fbib{pillars}} recognized the grace that had been given me, they gave Barnabas and me the right hand of fellowship, agreeing that we should go to the gentiles and they to the circumcised. \v{10}The only thing they asked us to do was to remember the destitute, the very thing I was eager to do.
\passage{Paul Confronts Cephas in Antioch}

\v{11}But when Cephas\fnote{I.e. Peter} came to Antioch, I opposed him to his face, because he was clearly wrong.\fnote{Or \fbib{was self-condemned}} \v{12}Until some men arrived from James, he was in the habit of eating with the gentiles, but after those men\fnote{Lit. \fbib{after they}} came, he withdrew from the gentiles\fnote{The Gk. lacks \fbib{from the gentiles}} and would not associate with them any longer, because he was afraid of the circumcision party. \v{13}The other Jews also joined him in this hypocritical behavior, to the extent that even Barnabas was caught up in their hypocrisy. \v{14}But when I saw that they were not acting consistently with the truth of the gospel, I told Cephas\fnote{I.e. Peter} in front of everyone, ``Though you are a Jew, you have been living like a gentile and not like a Jew. So how can you insist that the gentiles must live like Jews?''
\passage{Jews, Like Gentiles, are Saved by Faith}

\v{15}We ourselves are Jews by birth, and not gentile sinners, \v{16}yet we know that a person is not justified by doing what the Law requires,\fnote{Lit. \fbib{by works of the law}; and so throughout this verse} but rather by the faithfulness of Jesus\fnote{Or \fbib{by faith in Jesus}} the Messiah.\fnote{Or \fbib{Christ}} We, too, have believed in the Messiah\fnote{Or \fbib{Christ}} Jesus so that we might be justified by the faithfulness of\fnote{Or \fbib{by faith in}} the Messiah\fnote{Or \fbib{Christ}} and not by doing what the Law requires, for no human being\fnote{Lit. \fbib{no flesh}} will be justified by doing what the Law requires. \v{17}Now if we, while trying to be justified by the Messiah,\fnote{Or \fbib{Christ}} have been found to be sinners, does that mean that the Messiah\fnote{Or \fbib{Christ}} is serving the interests of sin? Of course not! \v{18}For if I rebuild something that I tore down, I demonstrate that I am a wrongdoer. \v{19}For through the Law I died to the Law so that I might live for God. I have been crucified with the Messiah.\fnote{Or \fbib{Christ}} \v{20}I no longer live, but the Messiah\fnote{Or \fbib{Christ}} lives in me, and the life that I am now living in this body I live by the faithfulness of the Son of God,\fnote{Or \fbib{by faith in the Son of God}} who loved me and gave himself for me. \v{21}I do not misapply God's grace, for if righteousness comes about by doing what the Law requires, then the Messiah\fnote{Or \fbib{Christ}} died for nothing.
\labelchapt{3}
\passage{Believers are Approved by God}

\chapt{3}
\v{1}You foolish Galatians! Who put you under a spell? Was not Jesus the Messiah\fnote{Or \fbib{Christ}} clearly portrayed before your very eyes as having been crucified? \v{2}I want to learn only one thing from you: Did you receive the Spirit by doing\fnote{Lit. \fbib{Spirit through}} the actions of the Law or by believing what you heard?\fnote{Lit. \fbib{or through the hearing of faith}} \v{3}Are you so foolish? Having started out with the Spirit, are you now ending up with the flesh? \v{4}Did you suffer so much for nothing? (If it really was for nothing!) \v{5}Does God\fnote{Lit. \fbib{he}} supply you with the Spirit and work miracles among you because you do the actions\fnote{Lit. \fbib{you through the works}} of the Law or because you believe what you heard?\fnote{Lit. \fbib{you through the hearing of faith}} \v{6}In the same way, Abraham ``believed God, and it was credited to him as righteousness.''\fnote{Gen 15:6}

\v{7}You see, then, that those who have faith are Abraham's real descendants. \v{8}Because the Scripture saw ahead of time that God would justify the gentiles\fnote{Or \fbib{nations}} by faith, it announced the gospel to Abraham beforehand when it said, ``Through you all nations\fnote{Or \fbib{all the gentiles}} will be blessed.''\fnote{Gen 12:3} \v{9}Therefore, those who believe are blessed together with Abraham, the one who believed.
\passage{No One is Justified by the Law}

\v{10}Certainly all who depend on the actions of the Law are under a curse. For it is written, ``A curse on everyone who does not obey everything that is written in the Book of the Law!''\fnote{Deut 27:26} \v{11}Now it is obvious that no one is justified in the sight of God by the Law, because ``The righteous will live by faith.''\fnote{Hab 2:4} \v{12}But the Law has nothing to do with faith. Instead, ``The person who keeps the commandments\fnote{Lit. \fbib{who does them}} will have life in them.''\fnote{Lev 18:5} \v{13}The Messiah\fnote{Or \fbib{Christ}} redeemed us from the curse of the Law by becoming a curse for us. For it is written, ``A curse on everyone who is hung on a tree!''\fnote{Deut 21:23} \v{14}This happened\fnote{The Gk. lacks \fbib{This happened}} in order that the blessing promised to\fnote{Lit. \fbib{the blessing of}} Abraham would come to the gentiles through the Messiah\fnote{Or \fbib{Christ}} Jesus, so that we might receive the promised Spirit\fnote{Or \fbib{the promise of the Spirit}} through faith.

\v{15}Brothers, let me use an example from everyday life.\fnote{Lit. \fbib{I am speaking according to man}} Once an agreement has been ratified, no one can cancel it or add conditions to it. \v{16}Now the promises were spoken to Abraham and to his descendant. It doesn't say ``descendants,'' referring to many, but ``your descendant,''\fnote{Gen 12:7} referring to one person, who is the Messiah.\fnote{Or \fbib{Christ}} \v{17}This is what I mean: The Law that came 430 years later did not cancel the covenant that God ratified previously. The promise was never nullified. \v{18}For if the inheritance comes about through the Law, it no longer comes about through the promise. But it was through a promise that God so graciously gave it to Abraham.
\passage{The Purpose of the Law}

\v{19}Why, then, was the Law added?\fnote{The Gk. lacks \fbib{added}} Because of transgressions, until the descendant\fnote{Lit. \fbib{seed}} came to whom the promise pertained. It was put into effect through angels by means of a mediator. \v{20}Now a mediator involves more than one party, but God is one. \v{21}So is the Law in conflict with the promises of God? Of course not! For if a law had been given that could give us life, then certainly righteousness would come through the Law. \v{22}But the Scripture has captured everything by means of sin's net, so that what was promised by the faithfulness of\fnote{Or \fbib{by faith in}} the Messiah\fnote{Or \fbib{Christ}} might be granted to those who believe. \v{23}Now before faith came about, we were held in custody and confined under the Law in preparation for the faith that was to be revealed. \v{24}And so the Law was our guardian until the Messiah\fnote{Or \fbib{Christ}} came, so that we might be justified by faith. \v{25}But now that faith has come about, we are no longer under a guardian.
\passage{You are God's Children}

\v{26}For all of you are God's children through faith in the Messiah\fnote{Or \fbib{Christ}} Jesus. \v{27}Indeed, all of you who were baptized into the Messiah\fnote{Or \fbib{Christ}} have clothed yourselves with the Messiah.\fnote{Or \fbib{Christ}} \v{28}Because all of you are one in the Messiah\fnote{Or \fbib{Christ}} Jesus, a person is no longer a Jew or a Greek, a slave or a free person, a male or a female. \v{29}And if you belong to the Messiah,\fnote{Or \fbib{Christ}} then you are Abraham's descendants indeed, and heirs according to the promise.
\labelchapt{4}

\chapt{4}
\v{1}Now what I am saying is this: As long as an heir is a child, he is no better off than a slave, even though he owns everything. \v{2}Instead, he is placed under the care of\fnote{The Gk. lacks \fbib{the care of}} guardians and servant managers until the time set by the father. \v{3}It was the same way with us. While we were children, we were slaves to the basic principles of the world.\fnote{Or \fbib{the elemental spirits of the universe}} \v{4}But when the appropriate time had come, God sent his Son, born by a woman, born under the Law, \v{5}in order to redeem those who were under the Law, and thus to adopt them as his children. \v{6}Now because you are his children, God has sent the Spirit of his Son into our\fnote{Other mss. read \fbib{your}} hearts to cry out, ``Abba!\fnote{\fbib{Abba} is Aram. for \fbib{Father.}} Father!'' \v{7}So you are no longer a slave but a child, and if you are a child, then you are also an heir because of what God did.

\v{8}However, in the past, when you did not know God, you were slaves to things that are not really gods at all.\fnote{Lit. \fbib{gods by nature}} \v{9}But now that you know God, or rather have been known by God, how can you turn back again to those powerless and bankrupt basic principles?\fnote{Or \fbib{elemental spirits}} Why do you want to become their slaves all over again? \v{10}You are observing days, months, seasons, and years. \v{11}I am afraid for you! I don't want my work for you to have\fnote{Lit. \fbib{you, lest somehow my work for you has}} been wasted!
\passage{Paul's Concern for the Galatians}

\v{12}I beg you, brothers, to become like me, since I became like you. You did not do anything wrong to me. \v{13}You know that it was because I was ill\fnote{Lit. \fbib{because of a weakness of the flesh}} that I brought you the gospel the first time. \v{14}Even though my condition put you to the test, you did not despise or reject me. On the contrary, you welcomed me as if I were an angel of God, or as if I were the Messiah\fnote{Or \fbib{Christ}} Jesus. \v{15}What, then, happened to your positive attitude?\fnote{Lit. \fbib{your blessedness}} For I testify that if it had been possible, you would have torn out your eyes and given them to me. \v{16}So have I now become your enemy for telling you the truth?

\v{17}These people who have been instructing you\fnote{Lit. \fbib{They}} are devoted to you, but not in a good way. They want you to avoid me so that you will be devoted to them. \v{18}(Now it is always good to be devoted to a good cause, even when I am not with you.) \v{19}My children, I am suffering birth pains for you again until the Messiah\fnote{Or \fbib{Christ}} is formed in you. \v{20}Indeed, I wish I were with you right now so that I could change the tone of my voice, because I am completely baffled by you!
\passage{You are Children of a Free Woman}

\v{21}Tell me, those of you who want to live under the Law: Are you really listening to what the Law says? \v{22}For it is written that Abraham had two sons, one by a slave woman and the other by a free woman. \v{23}Now the slave woman's son was conceived through human means, while the free woman's son was conceived through divine\fnote{The Gk. lacks \fbib{divine}} promise. \v{24}This is being said as an allegory, for these women represent two covenants. The one woman, Hagar, is from Mount Sinai, and her children are born into slavery. \v{25}Now Hagar is Mount Sinai in Arabia and corresponds to present-day Jerusalem, because she is in slavery along with her children. \v{26}But the heavenly Jerusalem is the free woman, and she is our spiritual mother.\fnote{Other mss. read \fbib{the mother of us all}} \v{27}For it is written,

\begin{poetry}
\poeml ``Rejoice, you childless woman, \\
\poemll    who cannot give birth to any children! \\
\poeml Break into song and shout, \\
\poemll    you who feel no pains of childbirth! \\
\poeml For the children of the deserted woman \\
\poemll    are more numerous than the children \\
\poemlll       of the woman who has a husband.''\fnote{Isa 54:1}
\end{poetry}

\v{28}So you,\fnote{Other mss. read \fbib{we}} brothers, are children of the promise, like Isaac. \v{29}But just as then the son who was conceived according to the flesh persecuted the son who was conceived according to the Spirit, so it is now. \v{30}But what does the Scripture say? ``Drive out the slave woman and her son, for the son of the slave woman must never share the inheritance with the son of the free woman.''\fnote{Gen 21:10} \v{31}So then, brothers, we are not children of the slave woman but of the free woman.
\labelchapt{5}
\passage{Live in the Freedom that the Messiah Provides}

\chapt{5}
\v{1}The Messiah\fnote{Or \fbib{Christ}} has set us free so that we may enjoy the benefits of freedom.\fnote{Lit. \fbib{has set us free for freedom}} So keep on standing firm in it, and stop putting yourselves under the yoke of slavery again. \v{2}Listen! I, Paul, am telling you that if you allow yourselves to be circumcised, the Messiah\fnote{Or \fbib{Christ}} will be of no benefit to you. \v{3}Again, I insist\fnote{Or \fbib{testify}} that everyone who allows himself to be circumcised is obligated to obey the entire Law. \v{4}Those of you who are trying to be justified by the Law have been cut off from the Messiah.\fnote{Or \fbib{Christ}} You have fallen away from grace.

\v{5}Through the Spirit by faith we confidently await the fulfillment of our righteous hope, \v{6}for in union with the Messiah\fnote{Or \fbib{Christ}} Jesus neither circumcision nor uncircumcision matters. What matters is faith\fnote{Lit. \fbib{But faith}} expressed through love.

\v{7}You were running the race beautifully. Who cut in on you and stopped you from obeying the truth? \v{8}Such influence does not come from the one who calls you. \v{9}A little yeast spreads through the whole batch of dough. \v{10}I am confident\fnote{Lit. \fbib{confident about you}} in the Lord that you will take no other view of this. However, the one who is troubling you will suffer God's\fnote{The Gk. lacks \fbib{God's}} judgment, whoever he is. \v{11}As for me, brothers, if I am still preaching the necessity of\fnote{The Gk lacks \fbib{the necessity of}} circumcision, why am I still being persecuted? In that case the offense of the cross has been removed. \v{12}I wish that those who are upsetting you would castrate themselves!

\v{13}For you, brothers, were called to freedom. Only do not turn your freedom into an opportunity to gratify your flesh, but through love make it your habit to serve one another. \v{14}For the whole Law is summarized in a single statement: ``You must love your neighbor as yourself.''\fnote{Lev 19:18} \v{15}But if you bite and devour one another, be careful that you are not destroyed by each other. \v{16}So I say, live by the Spirit, and you will never fulfill the desires of the flesh. \v{17}For what the flesh wants is opposed to the Spirit, and what the Spirit wants is opposed to the flesh. They are opposed to each other, and so you do not do what you want to do. \v{18}But if you are being led by the Spirit, you are not under the Law.

\v{19}Now the actions of the flesh are obvious: sexual immorality, impurity, promiscuity, \v{20}idolatry, witchcraft,\fnote{Or \fbib{sorcery}} hatred, rivalry, jealously, outbursts of anger, quarrels, conflicts, factions, \v{21}envy, murder,\fnote{Other mss. lack \fbib{murder}} drunkenness, wild partying, and things like that. I am telling you now, as I have told you in the past, that people who practice such things will not inherit the kingdom of God. \v{22}But the fruit of the Spirit is love, joy, peace, patience, kindness, goodness, faithfulness,\fnote{Or \fbib{faith}} \v{23}gentleness, and self-control. There is no law against such things. \v{24}Now those who belong to the Messiah\fnote{Or \fbib{Christ}} Jesus have crucified their flesh with its passions and desires. \v{25}Since we live by the Spirit, by the Spirit let us also be guided. \v{26}Let's stop being arrogant, provoking one another and envying one another.
\labelchapt{6}
\passage{Help Each Other}

\chapt{6}
\v{1}Brothers, if a person is caught doing something wrong, those of you who are spiritual should restore that person gently. Watch out for yourself so that you are not tempted as well. \v{2}Practice carrying each other's burdens. In this way you will fulfill the law of the Messiah.\fnote{Or \fbib{Christ}} \v{3}For if anyone thinks he is something when he is really nothing, he is only fooling himself. \v{4}Each person must examine his own actions, and then he can boast about his own accomplishments and not about someone else. \v{5}For everyone must carry his own load.

\v{6}The person who is taught the word should share all his goods with his teacher. \v{7}Stop being\fnote{Or \fbib{Do not be}} deceived; God is not to be ridiculed. A person harvests whatever he plants: \v{8}The person who sows through human means will harvest decay from human means, but the person who sows in the Spirit will harvest eternal life from the Spirit. \v{9}Let's not get tired of doing what is good, for at the right time we will reap a harvest---if we do not give up. \v{10}So then, whenever we have the opportunity, let's practice doing good to everyone, especially to the family of faith.
\passage{A Final Warning against Circumcision}

\v{11}Look at how large these letters are because I am writing with my own hand! \v{12}These people who want to impress others by their external appearance\fnote{Lit. \fbib{their flesh}} are trying to force you to be circumcised, simply to avoid being persecuted for the cross of the Messiah.\fnote{Or \fbib{Christ}} \v{13}Why, not even those who are circumcised obey the Law! They simply want you to be circumcised so that they can boast about your external appearance.\fnote{Lit. \fbib{your flesh}} \v{14}But may I never boast about anything except the cross of our Lord Jesus, the Messiah,\fnote{Or \fbib{Christ}} by which the world has been crucified to me, and I to the world! \v{15}For neither circumcision nor uncircumcision matters. Rather, what matters is being\fnote{The Gk. lacks \fbib{what matters is being}} a new creation. \v{16}Now may peace be on all those who live by this principle, and may mercy be on the Israel of God. \v{17}Let no one make any more trouble for me, because I carry the scars of Jesus on my own body.
\passage{Final Greeting}

\v{18}May the grace of our Lord Jesus, the Messiah,\fnote{Or \fbib{Christ}} be with your spirit, brothers! Amen.

\bookheader{Ephesians}
\labelbook{Eph}

\bookpretitle{The Letter of Paul to the}
\booktitle{Ephesians}

\labelchapt{1}
\passage{Greetings from Paul}

\chapt{1}
\v{1}From:\fnote{\fbackref{1:1} The Gk. lacks \fbib{From}} Paul, an apostle of the Messiah\fnote{\fbackref{1:1} Or \fbib{Christ}} Jesus by God's will.

To: His holy and faithful people\fnote{\fbackref{1:1} Or \fbib{to the saints and faithful}} in Ephesus\fnote{\fbackref{1:1} Other mss. lack \fbib{in Ephesus}} who are in union with the Messiah\fnote{\fbackref{1:1} Or \fbib{Christ}} Jesus.

\v{2}May grace and peace from God our Father and the Lord Jesus, the Messiah,\fnote{\fbackref{1:2} Or \fbib{Christ}} be yours!
\passage{The Many Blessings of Salvation}

\v{3}Blessed be the God and Father of our Lord Jesus, the Messiah!\fnote{\fbackref{1:3} Or \fbib{Christ}} He has blessed us in the Messiah\fnote{\fbackref{1:3} Or \fbib{Christ}} with every spiritual blessing in the heavenly realm, \v{4}just as he chose us in the Messiah\fnote{\fbackref{1:4} Lit. \fbib{in him}} before the creation of the universe\fnote{\fbackref{1:4} Or \fbib{world}} to be holy and blameless in his presence. In love \v{5}he predestined us for adoption to himself through Jesus the Messiah,\fnote{\fbackref{1:5} Or \fbib{Christ}} according to the pleasure of his will, \v{6}so that we would praise\fnote{\fbackref{1:6} Lit. \fbib{to the praise of}} his glorious grace that he gave us in the Beloved One. \v{7}In union with him we have redemption through his blood, the forgiveness of our offenses, according to the riches of God's\fnote{\fbackref{1:7} Lit. \fbib{his}} grace \v{8}that he lavished on us, along with all wisdom and understanding, \v{9}when he made known to us the secret of his will. This was according to his plan that he set forth in the Messiah\fnote{\fbackref{1:9} Lit. \fbib{him}} \v{10}to usher in\fnote{\fbackref{1:10} Or \fbib{administer}} the fullness of the times and to bring together in the Messiah\fnote{\fbackref{1:10} Or \fbib{Christ}} all things in heaven and on earth.

\v{11}In the Messiah\fnote{\fbackref{1:11} Lit. \fbib{him}} we were also chosen when we were predestined according to the purpose of the one who does everything that he wills to do, \v{12}so that we who had already fixed our hope on the Messiah\fnote{\fbackref{1:12} Or \fbib{Christ}} might live for his praise and glory. \v{13}You, too, have heard the word of truth, the gospel of your salvation. When you believed in the Messiah,\fnote{\fbackref{1:13} Lit. \fbib{in him}} you were sealed with the promised Holy Spirit, \v{14}who is the guarantee of our inheritance until God redeems his own possession\fnote{\fbackref{1:14} Lit. \fbib{of the possession}} for his praise and glory.
\passage{Paul's Prayer for the Ephesians}

\v{15}Therefore, because I have heard about your faith in the Lord Jesus and your love\fnote{\fbackref{1:15} Other mss. lack \fbib{your love}} for all the saints, \v{16}I never stop giving thanks for you as I mention you in my prayers. \v{17}I pray\fnote{\fbackref{1:17} The Gk. lacks \fbib{I pray}} that the God of our Lord Jesus, the Messiah,\fnote{\fbackref{1:17} Or \fbib{Christ}} the most glorious Father, would give you a wise spirit, along with revelation that comes through knowing the Messiah\fnote{\fbackref{1:17} Lit. \fbib{knowing him}} fully. \v{18}Then, with the eyes of your hearts enlightened, you will know the confidence\fnote{\fbackref{1:18} Or \fbib{hope}} that is produced by God\fnote{\fbackref{1:18} Lit. \fbib{him}} having called you,\fnote{\fbackref{1:18} The Gk. lacks \fbib{you}} the rich glory that is his inheritance among the saints, \v{19}and the unlimited greatness of his power for us who believe, according to the working of his mighty strength, \v{20}which he brought about in the Messiah\fnote{\fbackref{1:20} Or \fbib{Christ}} when he raised him from the dead and seated him at his right hand in the heavenly realm. \v{21}He is far above every ruler, authority, power, dominion, and every name that can be named, not only in the present age, but also in the one to come. \v{22}God\fnote{\fbackref{1:22} Lit. \fbib{He}} has put everything under the Messiah's\fnote{\fbackref{1:22} Lit. \fbib{under his}} feet and has made him the head of everything for the good of\fnote{\fbackref{1:22} The Gk. lacks \fbib{the good of}} the church, \v{23}which is his body, the fullness of the one who fills everything in every way.\fnote{\fbackref{1:23} Or \fbib{who fills all in all}}
\labelchapt{2}
\passage{God Has Brought Us from Death to Life}

\chapt{2}
\v{1}You used to be dead because of your offenses and sins \v{2}that you once practiced as you lived according to the ways of this present world and according to the ruler of the power of the air, the spirit that is now active in those who are disobedient.\fnote{\fbackref{2:2} Lit. \fbib{the sons of disobedience}} \v{3}Indeed, all of us once behaved like\fnote{\fbackref{2:3} Or \fbib{lived among}} them in the lusts of our flesh, fulfilling the desires of our flesh and senses. By nature we were destined for\fnote{\fbackref{2:3} Lit. \fbib{were children of}} wrath, just like everyone else. \v{4}But God, who is rich in mercy, because of his great love for us\fnote{\fbackref{2:4} Lit. \fbib{love with which he loved us}} \v{5}even when we were dead because of our offenses, made us alive together with\fnote{\fbackref{2:5} Other mss. read \fbib{in}} the Messiah\fnote{\fbackref{2:5} Or \fbib{Christ}} (by grace you have been saved), \v{6}raised us up with him, and seated us with him in the heavenly realm in the Messiah\fnote{\fbackref{2:6} Or \fbib{Christ}} Jesus, \v{7}so that in the coming ages he might display the limitless riches of his grace that comes to us through his kindness in the Messiah\fnote{\fbackref{2:7} Or \fbib{Christ}} Jesus. \v{8}For by such grace you have been saved through faith. This does not come from you; it is the gift of God \v{9}and not the result of actions, to put a stop to all boasting.\fnote{\fbackref{2:9} Lit. \fbib{works, lest anyone boast}} \v{10}For we are God's\fnote{\fbackref{2:10} Lit. \fbib{his}} masterpiece,\fnote{\fbackref{2:10} Or \fbib{workmanship}} created in the Messiah\fnote{\fbackref{2:10} Or \fbib{Christ}} Jesus to perform good actions that God prepared long ago to be our way of life.\fnote{\fbackref{2:10} Lit. \fbib{so that we might walk in them}}
\passage{All Believers are One in the Messiah}

\v{11}So then, remember that at one time you gentiles by birth\fnote{\fbackref{2:11} Lit. \fbib{in the flesh}} were called ``the uncircumcised'' by those who called themselves ``the circumcised.'' They underwent physical circumcision done by human hands. \v{12}At that time you were without the Messiah,\fnote{\fbackref{2:12} Or \fbib{Christ}} excluded from citizenship in Israel,\fnote{\fbackref{2:12} Or \fbib{from the commonwealth of Israel}} and strangers to the covenants of promise. You had no hope and were in the world without God. \v{13}But now, in union with the Messiah\fnote{\fbackref{2:13} Or \fbib{Christ}} Jesus, you who once were far away have been brought near by the blood of the Messiah.\fnote{\fbackref{2:13} Or \fbib{Christ}}

\v{14}For it is he who is our peace. Through his mortality\fnote{\fbackref{2:14} Lit. \fbib{flesh}} he made both groups one by tearing down the wall of hostility that divided them.\fnote{\fbackref{2:14} Lit. \fbib{the dividing wall of hostility}} \v{15}He rendered the Law inoperative, along with its commandments and regulations, thus creating in himself one new humanity from the two, thereby making peace, \v{16}and reconciling both groups to God in one body through the cross, on which he eliminated the hostility. \v{17}He came and proclaimed peace for you who were far away and for you who were near. \v{18}For through him, both of us\fnote{\fbackref{2:18} I.e. both Jews and gentiles} have access to the Father by one Spirit. \v{19}That is why you are no longer strangers and foreigners but fellow citizens with the saints and members of God's household, \v{20}having been built on the foundation of the apostles and prophets, the Messiah\fnote{\fbackref{2:20} Or \fbib{Christ}} Jesus himself being the cornerstone.\fnote{\fbackref{2:20} Or \fbib{capstone}} \v{21}In union with him the whole building is joined together and rises into a holy sanctuary for the Lord. \v{22}You, too, are being built in him, along with the others, into a place for God's Spirit to dwell.
\labelchapt{3}
\passage{Paul's Ministry to the Gentiles}

\chapt{3}
\v{1}For this reason I, Paul, am the prisoner of the Messiah\fnote{\fbackref{3:1} Or \fbib{Christ}} Jesus for the sake of you gentiles. \v{2}Surely you have heard about the responsibility of administering God's grace that was given to me on your behalf, \v{3}and how this secret was made known to me through a revelation, just as I wrote about briefly in the past. \v{4}By reading this, you will be able to grasp my understanding of the secret about the Messiah,\fnote{\fbackref{3:4} Or \fbib{Christ}} \v{5}which in previous generations was not made known to human beings\fnote{\fbackref{3:5} Lit. \fbib{the sons of men}} as it has now been revealed by the Spirit to God's\fnote{\fbackref{3:5} Lit. \fbib{his}} holy apostles and prophets. This is that secret:\fnote{\fbackref{3:5} The Gk. lacks \fbib{This is that secret:}} \v{6}The gentiles are heirs-in-common, members-in-common of the body, and common participants in what was promised\fnote{\fbackref{3:6} Lit. \fbib{of the promise}} by the Messiah\fnote{\fbackref{3:6} Or \fbib{Christ}} Jesus through the gospel.

\v{7}I have become a servant of this gospel\fnote{\fbackref{3:7} Lit. \fbib{of it}} according to the gift of God's grace that was given me by the working of his power. \v{8}To me, the very least of all the saints, this grace was given so that I might proclaim to the gentiles the immeasurable wealth of the Messiah\fnote{\fbackref{3:8} Or \fbib{Christ}} \v{9}and help everyone see how this secret that has been at work was hidden for ages by God, who created all things. \v{10}He did this\fnote{\fbackref{3:10} The Gk. lacks \fbib{He did this}} so that now, through the church, the wisdom of God in all its variety might be made known to the rulers and authorities in the heavenly realm \v{11}in keeping with the eternal purpose that God\fnote{\fbackref{3:11} Lit. \fbib{he}} carried out through the Messiah\fnote{\fbackref{3:11} Or \fbib{Christ}} Jesus our Lord, \v{12}in whom we have boldness and confident access through his faithfulness.\fnote{\fbackref{3:12} Or \fbib{through faith in him}} \v{13}So then, I ask you not to become discouraged because of my troubles on your behalf, which work toward your glory.
\passage{To Know the Messiah's Love}

\v{14}This is the reason I bow my knees before the Father of our Lord Jesus, the Messiah,\fnote{\fbackref{3:14} Or \fbib{Christ}; other mss. lack \fbib{of our Lord Jesus, the Messiah}} \v{15}from whom every family\fnote{\fbackref{3:15} Or \fbib{all fatherhood}} in heaven and on earth receives its name. \v{16}I pray\fnote{\fbackref{3:16} The Gk. lacks \fbib{I pray}} that he would give you, according to his glorious riches, strength in your inner being and power through his Spirit, \v{17}and that the Messiah\fnote{\fbackref{3:17} Or \fbib{Christ}} would make his home in your hearts through faith. Then, having been rooted and grounded in love, \v{18}you will be able to understand, along with all the saints, what is wide, long, high, and deep--- \v{19}that is, you will know the love of the Messiah\fnote{\fbackref{3:19} Or \fbib{Christ}}--- which transcends knowledge, and will be filled with all the fullness of God.

\v{20}Now to the one who can do infinitely more than all we can ask or imagine according to the power that is working among\fnote{\fbackref{3:20} Or \fbib{in}} us--- \v{21}to him be glory in the church and in the Messiah\fnote{\fbackref{3:21} Or \fbib{Christ}} Jesus to all generations, forever and ever! Amen.
\labelchapt{4}
\passage{The Messiah's Gifts to the Church}

\chapt{4}
\v{1}I, therefore, the prisoner of the Lord, urge you to live in a way that is worthy of the calling to which you have been called, \v{2}demonstrating all expressions of humility, gentleness, and patience, accepting one another in love. \v{3}Do your best to maintain the unity of the Spirit by means of the bond of peace. \v{4}There is one body and one Spirit. Likewise, you were called to the one hope of your calling.

\begin{poetry}
\poeml \v{5}There is one Lord, one faith, one baptism, \\
\poeml \v{6}one God and Father of all, \\
\poemlll       who is above all, through all, and in all.
\end{poetry}

\v{7}Now to each one of us grace has been given proportionate to the measure of the Messiah's\fnote{\fbackref{4:7} Or \fbib{Christ's}} gift. \v{8}That is why God\fnote{\fbackref{4:8} Lit. \fbib{he}} says,

\begin{poetry}
\poeml ``When he went up to the highest place, \\
\poemll    he led captives into captivity \\
\poemlll       and gave gifts to people.''\fnote{\fbackref{4:8} Ps 68:18}
\end{poetry}

\v{9}Now what does this ``he went up'' mean except that he also had gone\fnote{\fbackref{4:9} Other mss. read \fbib{had first gone}} down into the lower parts of the earth?\fnote{\fbackref{4:9} Or \fbib{parts, that}} \v{10}The one who went down is the same one who went up above all the heavens so that all things would be fulfilled. \v{11}And it is he who gifted some to be apostles, others to be prophets, others to be evangelists, and still others to be pastors and teachers, \v{12}to equip\fnote{\fbackref{4:12} Or \fbib{perfect}} the saints, to do the work of ministry, and to build up the body of the Messiah\fnote{\fbackref{4:12} Or \fbib{Christ}} \v{13}until all of us are united in the faith and in the full knowledge of God's Son, and until we attain mature adulthood and the full standard of development in the Messiah.\fnote{\fbackref{4:13} Or \fbib{Christ}} \v{14}Then we will no longer be little children, tossed like waves and blown about by every wind of doctrine, by people's trickery, or by clever strategies that would lead us astray. \v{15}Instead, by speaking the truth in love, we will grow up completely and become one with the head, that is, one with the Messiah,\fnote{\fbackref{4:15} Or \fbib{Christ}} \v{16}in whom the whole body is united and held together by every ligament with which it is supplied. As each individual part does its job, the body builds itself up in love.
\passage{The Old Life and the New}

\v{17}Therefore, I tell you and insist on\fnote{\fbackref{4:17} Or \fbib{testify}} in the Lord not to live any longer like the gentiles live, thinking worthless thoughts.\fnote{\fbackref{4:17} Lit. \fbib{in the worthlessness of their mind}} \v{18}They are darkened in their understanding and separated from the life of God because of their ignorance and hardness of heart. \v{19}Since they have lost all sense of shame, they have abandoned themselves to sensuality and practice every kind of sexual perversion without restraint. \v{20}However, that is not the way you came to know the Messiah.\fnote{\fbackref{4:20} Or \fbib{Christ}} \v{21}Surely you have listened to him and have been taught by him, since truth is in Jesus. \v{22}Regarding your former way of life, you were taught\fnote{\fbackref{4:22} The Gk. lacks \fbib{you were taught}} to strip off your old nature, which is being ruined by its deceptive desires, \v{23}to be renewed in your mental attitude, \v{24}and to clothe yourselves with the new nature, which was created according to God's image\fnote{\fbackref{4:24} The Gk. lacks \fbib{image}} in righteousness and true holiness.

\v{25}Therefore, stripping off falsehood, ``let each of us speak the truth to his neighbor,''\fnote{\fbackref{4:25} Zech 8:16} for we belong to one another. \v{26}``Be angry, yet do not sin.''\fnote{\fbackref{4:26} Ps 4:4} Do not let the sun set while you are still angry, \v{27}and do not give the devil an opportunity to work.\fnote{\fbackref{4:27} The Gk. lacks \fbib{to work}} \v{28}The thief must no longer steal but must work hard and do what is good with his own hands, so that he might earn something to give to the needy.

\v{29}Let no filthy talk be heard from your mouths, but only what is good for building up people and meeting the need of the moment.\fnote{\fbackref{4:29} Lit. \fbib{up as the need may be}} This way you will administer grace to those who hear you. \v{30}Do not grieve the Holy Spirit, by whom you were marked with a seal for the day of redemption. \v{31}Let all bitterness, wrath, anger, quarreling, and slander be put away from you, along with all hatred. \v{32}And be kind to one another, compassionate, forgiving one another just as God has forgiven you\fnote{\fbackref{4:32} Other mss. read \fbib{us}} in the Messiah.\fnote{\fbackref{4:32} Or \fbib{Christ}}
\labelchapt{5}

\chapt{5}
\v{1}So be imitators of God, as his dear children. \v{2}Live lovingly, just as the Messiah\fnote{\fbackref{5:2} Or \fbib{Christ}} also loved us\fnote{\fbackref{5:2} Other mss. read \fbib{you}} and gave himself for us as an offering and sacrifice, a fragrant aroma to God. \v{3}Do not let sexual sin, impurity of any kind, or greed even be mentioned among you, as is proper for saints. \v{4}Obscene, flippant, or vulgar talk is totally inappropriate. Instead, let there be thanksgiving. \v{5}For you know very well that no immoral or impure person, or anyone who is greedy (that is, an idolater), has an inheritance in the kingdom of the Messiah\fnote{\fbackref{5:5} Or \fbib{Christ}} and of God.
\passage{Living in the Light}

\v{6}Do not let anyone deceive you with meaningless words, for it is because of these things that God becomes angry with those who disobey.\fnote{\fbackref{5:6} Lit. \fbib{with the sons of disobedience}} \v{7}So do not be partners with them. \v{8}For once you were darkness, but now you are light in the Lord. Live as children of light, \v{9}for the fruit that the light\fnote{\fbackref{5:9} Other mss. read \fbib{fruit of the Spirit}} produces consists of every form of goodness, righteousness, and truth. \v{10}Determine what pleases the Lord, \v{11}and have nothing to do with the unfruitful actions that darkness produces. Instead, expose them for what they are. \v{12}For it is shameful even to mention what is done by these disobedient people\fnote{\fbackref{5:12} Lit. \fbib{by them}} in secret. \v{13}But everything that is exposed to the light becomes visible, \v{14}for the light is making everything visible. That is why it says,

\begin{poetry}
\poeml ``Wake up, sleeper! \\
\poemll    Arise from the dead, \\
\poemlll       and the Messiah\fnote{\fbackref{5:14} Or \fbib{Christ}} will shine on you.'\,'\fnote{\fbackref{5:14} The source of this quote is unknown.}
\end{poetry}
\passage{Wise Behavior}

\v{15}So, then, be careful how you live. Do not be unwise but wise, \v{16}making the best use of your time\fnote{\fbackref{5:16} Or \fbib{buying up the time}} because the times are evil. \v{17}Therefore, do not be foolish, but understand what the Lord's will is. \v{18}Stop getting\fnote{\fbackref{5:18} Or \fbib{Do not get}} drunk with wine, which leads to wild living, but keep on being filled with the Spirit. \v{19}Then you will recite to one another psalms, hymns, and spiritual songs; you will sing and make music to the Lord with your hearts; \v{20}you will consistently give thanks to God the Father for everything in the name of our Lord Jesus, the Messiah;\fnote{\fbackref{5:20} Or \fbib{Christ}} \v{21}and you will submit to one another out of reverence for\fnote{\fbackref{5:21} Or \fbib{another in the fear of}} the Messiah.\fnote{\fbackref{5:21} Or \fbib{Christ}}
\passage{Wives and Husbands}

\v{22}Wives, submit yourselves\fnote{\fbackref{5:22} Other mss. lack \fbib{submit yourselves}} to your husbands as to the Lord. \v{23}For the husband is the head of his wife as the Messiah\fnote{\fbackref{5:23} Or \fbib{Christ}} is the head of the church. It is he who is the Savior of the body. \v{24}Indeed, just as the church is submissive to the Messiah,\fnote{\fbackref{5:24} Or \fbib{Christ}} so wives must be submissive\fnote{\fbackref{5:24} The Gk. lacks \fbib{must be submissive}} to their husbands in everything.

\v{25}Husbands, love your wives as the Messiah\fnote{\fbackref{5:25} Or \fbib{Christ}} loved the church and gave himself for it, \v{26}so that he might make it holy by cleansing it, washing it with water and the word, \v{27}and might present the church to himself in all its glory, without a spot or wrinkle or anything of the kind, but holy and without fault. \v{28}In the same way, husbands must love their wives as they love\fnote{\fbackref{5:28} The Gk. lacks \fbib{they love}} their own bodies. A man who loves his wife loves himself. \v{29}For no one has ever hated his own body, but he nourishes and tenderly cares for it, as the Messiah\fnote{\fbackref{5:29} Or \fbib{Christ}} does\fnote{\fbackref{5:29} The Gk. lacks \fbib{does}} the church.

\v{30}For we are parts of his body---of his flesh and of his bones.\fnote{\fbackref{5:30} Other mss. lack \fbib{of his flesh and of his bones}} \v{31}``That is why a man will leave his father and mother and be united with his wife, and the two will become one flesh.''\fnote{\fbackref{5:31} Gen 2:24} \v{32}This is a great secret, but I am talking about the Messiah\fnote{\fbackref{5:32} Or \fbib{Christ}} and the church. \v{33}But each individual man among you must love his wife as he loves\fnote{\fbackref{5:33} The Gk. lacks \fbib{he loves}} himself; and may the wife fear her husband.
\labelchapt{6}
\passage{Advice for Children and Parents}

\chapt{6}
\v{1}Children, obey your parents in the Lord,\fnote{\fbackref{6:1} Other mss. lack \fbib{in the Lord}} for this is the right thing to do. \v{2}``Honor your father and mother{\ldots}''\fnote{\fbackref{6:2} Exod 20:12; Deut 5:16} (This is a very important commandment with a promise.) \v{3}``{\ldots}so that it may go well for you, and that you may have a long life on the earth.''\fnote{\fbackref{6:3} Exod 20:12; Deut 5:16}

\v{4}Fathers, do not provoke your children to anger, but bring them up by training\fnote{\fbackref{6:4} Or \fbib{discipline}} and instructing them about the Lord.
\passage{Advice for Slaves and Masters}

\v{5}Slaves, obey your earthly masters with fear, trembling, and sincerity, as when you obey\fnote{\fbackref{6:5} Lit. \fbib{as to}} the Messiah.\fnote{\fbackref{6:5} Or \fbib{Christ}} \v{6}Do not do this only while you're being watched in order to please them, but be like slaves of the Messiah,\fnote{\fbackref{6:6} Or \fbib{Christ}} who are determined to obey God's will. \v{7}Serve willingly, as if you were serving the Lord and not merely people,\fnote{\fbackref{6:7} Lit. \fbib{as to the Lord and not people}} \v{8}because you know that everyone will receive a reward from the Lord for whatever good he has done, whether he is a slave or free.

\v{9}Masters, treat your slaves\fnote{\fbackref{6:9} Lit. \fbib{treat them}} the same way. Do not threaten them, for you know that both of you have the same Master in heaven, and there is no favoritism with him.
\passage{Putting on the Whole Armor of God}

\v{10}Finally, be strong in the Lord, relying on his mighty strength. \v{11}Put on the whole armor of God so that you may be able to stand firm against the devil's strategies.\fnote{\fbackref{6:11} Or \fbib{schemes}} \v{12}For our\fnote{\fbackref{6:12} Other mss. read \fbib{your}} struggle is not against human opponents,\fnote{\fbackref{6:12} Lit. \fbib{against flesh and blood}} but against rulers, authorities, cosmic powers in the darkness around us,\fnote{\fbackref{6:12} Lit. \fbib{powers of this darkness}} and evil spiritual forces in the heavenly realm. \v{13}For this reason, take up the whole armor of God so that you may be able to take a stand whenever evil comes. And when you have done everything you could, you will be able to stand firm.

\v{14}Stand firm, therefore, having fastened the belt of truth around your waist, and having put on the breastplate of righteousness, \v{15}and being firm-footed in the gospel of peace.\fnote{\fbackref{6:15} Or \fbib{wear on your feet readiness for the gospel of peace}} \v{16}In addition to having clothed yourselves with these things, having taken up the shield of faith, with which you will be able to put out all the flaming arrows of the evil one, \v{17}also take the helmet of salvation and the sword of the Spirit, which is the word of God. \v{18}Pray in the Spirit at all times with every kind of prayer and request. Likewise, be alert with your most diligent efforts and pray for all the saints. \v{19}Pray\fnote{\fbackref{6:19} The Gk. lacks \fbib{Pray}} also for me, so that, when I begin to speak, the right words will come to me. Then I will boldly make known the secret of the gospel, \v{20}for whose sake I am an ambassador in chains, desiring to declare the gospel\fnote{\fbackref{6:20} Lit. \fbib{declare it}} as boldly as I should.\fnote{\fbackref{6:20} Lit. \fbib{as I should speak}}
\passage{Final Greeting}

\v{21}So that you may know what has happened to me and how I am doing, Tychicus, our dear brother and a faithful minister in service to the Lord, will tell you everything. \v{22}I am sending him to you for this very reason, so that you may know how we are doing and that he may encourage your hearts.

\v{23}May peace and love, with faith, be with the brothers, from God the Father and the Lord Jesus, the Messiah!\fnote{\fbackref{6:23} Or \fbib{Christ}}

\v{24}May grace be with all who sincerely love the Lord Jesus, the Messiah!\fnote{\fbackref{6:24} Or \fbib{Christ}; other mss. read \fbib{Messiah! Amen.}}

\bookheader{Philippians}
\labelbook{Phil}

\bookpretitle{The Letter from Paul to the}
\booktitle{Philippians}

\labelchapt{1}
\passage{Greetings from Paul and Timothy}

\chapt{1}
\v{1}From:\fnote{The Gk. lacks \fbib{From}} Paul and Timothy, servants of the Messiah\fnote{Or \fbib{Christ}} Jesus.

To: All the holy ones\fnote{Or \fbib{saints}} in Philippi, along with their overseers\fnote{Or \fbib{bishops}} and ministers,\fnote{Or \fbib{deacons}} who are in union with the Messiah\fnote{Or \fbib{Christ}} Jesus.

\v{2}May grace and peace from God our Father and the Lord Jesus, the Messiah,\fnote{Or \fbib{Christ}} be yours!
\passage{Paul's Thanksgiving and Prayer for the Philippians}

\v{3}I thank my God every time I remember you,\fnote{Or \fbib{every time you remember me}} \v{4}always praying joyfully in every one of my prayers for all of you \v{5}because of your partnership in the gospel from the first day until now. \v{6}I am convinced of this, that the one who began a good action among\fnote{Or \fbib{in}} you will bring it to completion by the Day of the Messiah\fnote{Or \fbib{Christ}} Jesus. \v{7}For it is only right for me to think this way about all of you, because you're constantly on my mind.\fnote{Lit. \fbib{because I have you in my heart}; or \fbib{you have me in your heart}} Both in my imprisonment and in the defense and confirmation of the gospel, all of you are partners with me in this privilege.\fnote{Or \fbib{grace}} \v{8}For God is my witness how much I long for all of you with the compassion that the Messiah\fnote{Or \fbib{Christ}} Jesus provides.

\v{9}And this is my prayer, that your love will keep on growing more and more with full knowledge and insight, \v{10}so that you may be able to choose what is best and be pure and blameless until the day when the Messiah\fnote{Or \fbib{Christ}} returns, \v{11}having been filled with the fruit of righteousness that comes through Jesus the Messiah\fnote{Or \fbib{Christ}} so that God will be glorified and praised.
\passage{The Priority of the Gospel in Everything}

\v{12}Now I want you to know, brothers, that what has happened to me has actually caused the gospel to advance. \v{13}As a result, it has become clear to the whole imperial guard and to everyone else that I am in prison for preaching about\fnote{The Gk. lacks \fbib{preaching about}} the Messiah.\fnote{Or \fbib{Christ}} \v{14}Moreover, because of my imprisonment the Lord has caused most of the brothers to become confident to speak God's word more boldly and courageously than ever before. \v{15}Some are preaching the Messiah\fnote{Or \fbib{Christ}} because of jealousy and dissension, while others do so\fnote{The Gk. lacks \fbib{do so}} because of their good will. \v{16}The latter are motivated\fnote{The Gk. lacks \fbib{are motivated}} by love, because they know that I have been appointed to defend the gospel. \v{17}The former proclaim the Messiah\fnote{Or \fbib{Christ}} because they are selfishly ambitious and insincere, thinking that they will stir up trouble for me during my imprisonment.

\v{18}But so what? Just this---that in every way, whether by false or true motives, the Messiah\fnote{Or \fbib{Christ}} is being proclaimed. Because of this, I rejoice and will continue to rejoice. \v{19}I know that this will result in my deliverance through your prayers and the help that comes from the Spirit of Jesus the Messiah.\fnote{Or \fbib{Christ}} \v{20}I rejoice because I eagerly expect and hope that I will have nothing to be ashamed of, because through my\fnote{Lit. \fbib{with all}} boldness the Messiah\fnote{Or \fbib{Christ}} will be exalted through me,\fnote{Lit. \fbib{exalted in my body}} now as always, whether I live or die.\fnote{Lit. \fbib{by life or by death}}

\v{21}For to me, to go on living is the Messiah,\fnote{Or \fbib{Christ}} and to die is gain. \v{22}Now if I continue living, fruitful labor is the result, so I do not know which I would prefer. \v{23}Indeed, I cannot decide between the two. I have the desire to leave this life and be with the Messiah,\fnote{Or \fbib{Christ}} for that is far better. \v{24}But for your sake it is better that I remain alive.\fnote{Lit. \fbib{remain in this body}}

\v{25}Since I am convinced of this, I know that I will continue to live and be with all of you, so you will mature in the faith and know joy in it. \v{26}Then your rejoicing in the Messiah\fnote{Or \fbib{Christ}} Jesus will increase along with mine\fnote{Lit. \fbib{in me}} when I visit with you again.
\passage{Standing Firm in One Spirit}

\v{27}The only thing that matters is that you continue to live as good citizens in a manner worthy of the gospel of the Messiah.\fnote{Or \fbib{Christ}} Then, whether I come to see you or whether I stay away, I may hear all about you---that you are standing firm in one spirit, struggling with one mind for the faith of the gospel, \v{28}and that you are not intimidated by your opponents in any way. This is evidence that they will be destroyed and that you will be saved---and all because of\fnote{Lit. \fbib{and that from}} God. \v{29}For you have been given the privilege\fnote{Lit. \fbib{it has been given you}} for the Messiah's\fnote{Or \fbib{Christ's}} sake not only to believe in him but also to suffer for him. \v{30}You have the same struggle that you saw in me and now hear that I am still having.\fnote{Lit. \fbib{hear in me}}
\labelchapt{2}
\passage{Unity through Humility}

\chapt{2}
\v{1}Therefore, if there is any encouragement in the Messiah,\fnote{Or \fbib{Christ}} if there is any comfort of love, if there is any fellowship in the Spirit, if there is any compassion and sympathy, \v{2}then fill me with joy by having the same attitude, sharing the same love, being united in spirit, and keeping one purpose in mind. \v{3}Do not act out of selfish ambition or conceit, but with humility think of others as being better than yourselves. \v{4}Do not be concerned about your own interests, but also be concerned about\fnote{The Gk. lacks \fbib{be concerned about}} the interests of others. \v{5}Have the same attitude among yourselves\fnote{Or \fbib{Have this mind in you}} that was also in the Messiah\fnote{Or \fbib{Christ}} Jesus:\fnote{Verses 6-11 probably represent an early Christian hymn.}

\begin{poetry}
\poeml \v{6}In God's own form existed he, \\
\poemll    and shared with God equality, \\
\poemlll       deemed nothing needed grasping. \\
\poeml \v{7}Instead, poured out in emptiness, \\
\poemll    a servant's form did he possess, \\
\poemlll       a mortal man becoming. \\
\poeml In human form he chose to be, \\
\poeml \v{8}and lived in all humility, \\
\poemlll       death on a cross obeying. \\
\poeml \v{9}Now lifted up by God to heaven, \\
\poemll    a name above all others given, \\
\poemlll       this matchless name possessing. \\
\poeml \v{10}And so, when Jesus' name is called, \\
\poemll    the knees of everyone should fall,\fnote{Or \fbib{every knee should bend}} \\
\poemlll       wherever they're residing.\fnote{Lit. \fbib{in heaven, on earth, and under the earth}} \\
\poeml \v{11}Then every tongue in one accord, \\
\poemll    will say that Jesus the Messiah\fnote{Or \fbib{Christ}} is Lord, \\
\poemlll       while God the Father praising.
\end{poetry}
\passage{Blameless Living}

\v{12}And so, my dear friends, just as you have always obeyed, not only when I was with you but even more now that I am absent, continue to work out your salvation with fear and trembling. \v{13}For it is God who is producing in you both the desire and the ability to do what pleases him. \v{14}Do everything without complaining or arguing \v{15}so that you may be blameless and innocent, God's children without any faults among a crooked and perverse generation, among whom you shine like stars in the world \v{16}as you hold firmly to the word of life. Then I will be proud when the Messiah\fnote{Or \fbib{Christ}} returns\fnote{Lit. \fbib{will boast in the day of the Messiah}} that I did not run in vain or work hard in vain.

\v{17}Yet even if I am being poured out like an offering as part of the sacrifice and service I offer\fnote{The Gk. lacks \fbib{I offer}} for your faith, I rejoice, and I share my joy with all of you. \v{18}In the same way, you also should rejoice and share your joy with me.
\passage{News about Paul's Companions}

\v{19}Now I hope in the Lord Jesus to send Timothy to you soon so that I can be encouraged when I learn of your condition. \v{20}I do not have anyone else like him who takes a genuine interest in your welfare. \v{21}For all the others look after their own interests, not after those of Jesus the Messiah.\fnote{Or \fbib{Christ}} \v{22}But you know his proven worth---how like a son with his father he served with me in the gospel. \v{23}Therefore, I hope to send him as soon as I see how things are going to turn out for me. \v{24}Indeed, I am confident in the Lord that I will come to visit you\fnote{The Gk. lacks \fbib{to visit you}} soon.

\v{25}Meanwhile, I thought it best to send Epaphroditus---my brother, fellow worker, and fellow soldier, but your messenger and minister to my need---back to you. \v{26}For he has been longing for\fnote{Other mss. read \fbib{longing to see}} all of you and is troubled because you heard that he was sick. \v{27}Indeed, he was sick to the point of death, but God had mercy on him, and not only on him but also on me, so that I would not have one sorrow on top of another.\fnote{Lit. \fbib{sorrow on sorrow}} \v{28}Therefore, I am especially eager to send him so that you may have the joy of seeing him again, and so that I may feel relieved. \v{29}So joyfully welcome him in the Lord and make sure you honor such people highly, \v{30}because he came close to death for the work of the Messiah\fnote{Or \fbib{Christ}; other mss. read \fbib{Lord}} by risking his life to complete what remained unfinished in your service to me.
\labelchapt{3}
\passage{Warning against Pride}

\chapt{3}
\v{1}So then,\fnote{Or \fbib{Furthermore}} my brothers, keep on rejoicing in the Lord. It is no trouble for me to write the same things to you; indeed, it is for your safety.

\v{2}Beware of the dogs! Beware of the evil workers! Beware of the mutilators!\fnote{Lit. \fbib{the mutilation}; Lit. \fbib{katatome} (a cutting off)} \v{3}For it is we who are the circumcision\fnote{Lit. \fbib{peritome} (a cutting around)}---we who worship in the Spirit of God\fnote{Other mss. read \fbib{worship God in the Spirit}} and find our joy in the Messiah\fnote{Or \fbib{Christ}} Jesus. We have not placed any confidence in the flesh, \v{4}although I could have confidence in the flesh. If anyone thinks he can place confidence in the flesh, I have more reason to think so.\fnote{Lit. \fbib{I more}} \v{5}Having been circumcised on the eighth day, I am of the nation of Israel, from the tribe of Benjamin, a Hebrew of Hebrews. As far as the Law is concerned, I was a Pharisee. \v{6}As for my zeal, I was a persecutor of the church. As far as righteousness in the Law is concerned, I was blameless.

\v{7}But whatever things were assets to me, these I now consider a loss for the sake of the Messiah.\fnote{Or \fbib{Christ}} \v{8}What is more, I continue to consider all these things to be a loss for the sake of what is far more valuable, knowing the Messiah\fnote{Or \fbib{Christ}} Jesus, my Lord. It is because of him that I have experienced the loss of all those things. Indeed, I consider them rubbish\fnote{Or \fbib{dung}} in order to gain the Messiah\fnote{Or \fbib{Christ}} \v{9}and be found in him, not having a righteousness of my own that comes from the Law, but one that comes through the faithfulness\fnote{Or \fbib{through faith in}} of the Messiah,\fnote{Or \fbib{Christ}} the righteousness that comes from God and that depends on faith. \v{10}I want to know the Messiah\fnote{Lit. \fbib{To know him}}---what his resurrection power is like and what it means to share in his sufferings by becoming like him in his death, \v{11}though I hope to experience the resurrection from the dead.
\passage{Pursuing the Goal}

\v{12}It's not that I have already reached this goal or have already become perfect. But I keep pursuing it, hoping somehow to embrace it just as I have been embraced by the Messiah\fnote{Or \fbib{Christ}} Jesus. \v{13}Brothers, I do not consider myself to have embraced it yet.\fnote{Other mss. omit \fbib{yet}} But this one thing I do: Forgetting what lies behind and straining forward to what lies ahead, \v{14}I keep pursuing the goal to win the prize\fnote{Lit. \fbib{the goal for the prize}} of God's heavenly call in the Messiah\fnote{Or \fbib{Christ}} Jesus.

\v{15}Therefore, those of us who are mature\fnote{Or \fbib{perfect}} should think this way. And if you think differently about anything, God will show you how to think.\fnote{Lit. \fbib{show you this}} \v{16}However, we should live up to what we have achieved so far.
\passage{True and False Teachers}

\v{17}Join together in imitating me, brothers, and pay close attention to those who live by the example we have set for you.\fnote{Lit. \fbib{the example you have in us}} \v{18}For I have often told you, and now tell you even with tears, that many live as enemies of the cross of the Messiah.\fnote{Or \fbib{Christ}} \v{19}Their destiny is destruction, their god is their belly, and their glory is in their shame. Their minds are set on worldly things.

\v{20}Our citizenship, however, is in heaven, and it is from there that we eagerly wait for a Savior, the Lord Jesus, the Messiah.\fnote{Or \fbib{Christ}} \v{21}He will change our unassuming bodies and make them like his glorious body through the power that enables him to bring everything under his authority.
\labelchapt{4}
\passage{Closing Exhortations}

\chapt{4}
\v{1}Therefore, my dear brothers whom I long for, my joy and my victor's crown, this is how you must stand firm in the Lord, dear friends. \v{2}I urge Euodia and Syntyche to have the same attitude in the Lord. \v{3}Yes, I also ask you, my true partner,\fnote{Or \fbib{my loyal Syzygus}} to help these women. They have worked hard with me to advance\fnote{The Gk. lacks \fbib{to advance}} the gospel, along with Clement and the rest of my fellow workers, whose names are in the Book of Life.

\v{4}Keep on rejoicing in the Lord at all times. I will say it again: Keep on rejoicing! \v{5}Let your gracious attitude\fnote{Lit. \fbib{spirit}} be known to all people. The Lord is near: \v{6}Never worry about anything. Instead, in every situation let your petitions be made known to God through prayers and requests, with thanksgiving. \v{7}Then God's peace, which goes far beyond anything we can imagine, will guard your hearts and minds in union with the Messiah\fnote{Or \fbib{Christ}} Jesus.

\v{8}Finally, brothers, whatever is true, whatever is honorable, whatever is fair, whatever is pure, whatever is acceptable, whatever is commendable, if there is anything of excellence and if there is anything praiseworthy---keep thinking about these things. \v{9}Likewise, keep practicing these things: what you have learned, received, heard, and seen in me. Then the God of peace will be with you.
\passage{The Philippians' Gifts}

\v{10}Now I rejoice in the Lord greatly, because once again you have shown your concern for me. Of course, you were concerned for me but you did not have an opportunity to show it.\fnote{The Gk. lacks \fbib{to show it}} \v{11}I am not saying this because I am in any need, for I have learned to be content in whatever situation I am in. \v{12}I know how to be humble, and I know how to prosper. In each and every situation I have learned the secret of being full and of going hungry, of having too much and of having too little. \v{13}I can do all things through him\fnote{Other mss. read \fbib{the Messiah}} who strengthens me. \v{14}Nevertheless, it was kind of you to share my troubles.

\v{15}You Philippians also know that in the early days\fnote{Lit. \fbib{in the beginning}} of the gospel, when I left Macedonia, no church participated with me in the matter of giving and receiving except for you. \v{16}Even while I was in Thessalonica, you provided for my needs not once, but twice. \v{17}It is not that I am looking for a gift. No, I want to see that you receive the fruit that increases to your benefit. \v{18}I have been paid in full and have more than enough. I am fully supplied, now that I have received from Epaphroditus what you sent---a fragrant aroma, a sacrifice acceptable and pleasing to God. \v{19}And my God will fully supply your every need according to his glorious riches in the Messiah\fnote{Or \fbib{Christ}} Jesus. \v{20}Glory belongs to our God and Father forever and ever! Amen.
\passage{Final Greeting}

\v{21}Greet every saint who is in union with the Messiah\fnote{Or \fbib{Christ}} Jesus. The brothers who are with me send their greetings to you. \v{22}All the saints, especially those of the emperor's\fnote{Or \fbib{Caesar's}} household, greet you.

\v{23}May the grace of the Lord Jesus, the Messiah,\fnote{Or \fbib{Christ}} be with your spirit! Amen.\fnote{Other mss. lack \fbib{Amen}}

\bookheader{Colossians}
\labelbook{Col}

\bookpretitle{The Letter of Paul to the}
\booktitle{Colossians}

\labelchapt{1}
\passage{Greetings from Paul}

\chapt{1}
\v{1}From:\fnote{\fbackref{1:1} The Gk. lacks \fbib{From}} Paul, an apostle of the Messiah\fnote{\fbackref{1:1} Or \fbib{Christ}} Jesus by the will of God, and Timothy our brother.

\v{2}To: The holy\fnote{\fbackref{1:2} Or \fbib{to the saints}} and faithful brothers in Colossae who are in union with the Messiah.\fnote{\fbackref{1:2} Or \fbib{Christ}}

May grace and peace from God our Father\fnote{\fbackref{1:2} Other mss. read \fbib{from God our Father and the Lord Jesus, the Messiah}} be yours!
\passage{Paul's Prayer for the Colossians}

\v{3}We give thanks to God, the Father of our Lord Jesus, the Messiah,\fnote{\fbackref{1:3} Or \fbib{Christ}} praying always for you, \v{4}because we have heard about your faith in the Messiah\fnote{\fbackref{1:4} Or \fbib{Christ}} Jesus and the love that you have for all the saints, \v{5}based on the hope laid up for you in heaven. Some time ago you heard about this hope\fnote{\fbackref{1:5} Lit. \fbib{about it}} through the word of truth, the gospel \v{6}that has come to you. Just as it is bearing fruit and spreading all over the world, so it has been doing\fnote{\fbackref{1:6} The Gk. lacks \fbib{it has been doing}} among you from the day you heard it and came to know the grace of God in truth. \v{7}You learned about this gospel\fnote{\fbackref{1:7} Lit. \fbib{Just as you learned}} from Epaphras, our dear fellow servant, who is a faithful minister of the Messiah\fnote{\fbackref{1:7} Or \fbib{Christ}} on your\fnote{\fbackref{1:7} Other mss. read \fbib{our}} behalf. \v{8}He has told us about your love in the Spirit.
\passage{The Messiah is Above All}

\v{9}For this reason, since the day we heard about this, we have not stopped praying for you and asking that you may be filled with the full knowledge of God's\fnote{\fbackref{1:9} Lit. \fbib{his}} will with respect to all spiritual wisdom and understanding, \v{10}so that you might live in a manner worthy of the Lord and be fully pleasing to him\fnote{\fbackref{1:10} Lit. \fbib{to all pleasing}} as you bear fruit while doing all kinds of good things and growing in the full knowledge of God. \v{11}You are being strengthened with all power according to his glorious might, so that you might patiently endure everything with joy \v{12}and might thank the Father, who has enabled us\fnote{\fbackref{1:12} Other mss. read \fbib{you}} to share in the saints' inheritance in the light. \v{13}God\fnote{\fbackref{1:13} Lit. \fbib{He}} has rescued us from the power of darkness and has brought us into the kingdom of the Son whom he loves, \v{14}through whom we have redemption, the forgiveness of sins.
\passage{The Centrality of Jesus}

\begin{poetry}
\poeml \v{15}The Son\fnote{\fbackref{1:15} Lit. \fbib{He}} is the image of the invisible God, \\
\poemll    the firstborn over all creation. \\
\poeml \v{16}For by him all things in heaven and on earth were created, \\
\poemll    things visible and invisible, \\
\poemlll       whether they are kings,\fnote{\fbackref{1:16} Lit. \fbib{thrones}} lords, rulers, or powers. \\
\poeml All things have been created through him and for him. \\
\poeml \v{17}He himself existed before anything else did, \\
\poemll    and he holds all things together. \\
\poeml \v{18}He is also the head of the body, \\
\poemll    which is the church. \\
\poeml He is the beginning, the firstborn from the dead, \\
\poemll    so that he himself might have first place in everything. \\
\poeml \v{19}For God\fnote{\fbackref{1:19} Lit. \fbib{he}} was pleased to have \\
\poemll    all of his divine essence\fnote{\fbackref{1:19} Lit. \fbib{all of the fullness}} inhabit him. \\
\poeml \v{20}Through the Son,\fnote{\fbackref{1:20} Lit. \fbib{Through him}} God\fnote{\fbackref{1:20} Lit. \fbib{he}} also reconciled all things to himself, \\
\poemll    whether things on earth or things in heaven, \\
\poeml thereby making peace \\
\poemll    through the blood of his cross.
\end{poetry}

\v{21}You who were once alienated with a hostile attitude, doing evil,\fnote{\fbackref{1:21} Lit. \fbib{in evil deeds}} \v{22}he has now reconciled by the death of his physical body, so that he may present you holy, blameless, and without fault before him. \v{23}However, you must remain firmly established and steadfast in the faith, without being moved from the hope of the gospel that you heard, which has been proclaimed to every creature under heaven and of which I, Paul, have become a servant.\fnote{\fbackref{1:23} Or \fbib{minister}}
\passage{Paul's Service in the Church}

\v{24}Now I am rejoicing while suffering for you as I complete in my flesh whatever remains of the Messiah's\fnote{\fbackref{1:24} Or \fbib{Christ's}} sufferings on behalf of his body, which is the church. \v{25}I became its servant\fnote{\fbackref{1:25} Or \fbib{minister}} as God commissioned me to work for you, so that I may complete my ministry of\fnote{\fbackref{1:25} The Gk. lacks \fbib{the ministry of}} the word of God. \v{26}This secret was hidden throughout the ages and generations but has now been revealed to his saints, \v{27}to whom God wanted to make known the glorious riches of this secret among the gentiles---which is the Messiah\fnote{\fbackref{1:27} Or \fbib{Christ}} in you, our glorious hope. \v{28}It is he whom we proclaim as we admonish and wisely teach everyone, so that we may present everyone mature\fnote{\fbackref{1:28} Or \fbib{complete}} in the Messiah.\fnote{\fbackref{1:28} Or \fbib{Christ}} \v{29}I work hard and struggle to do this, using the energy that he powerfully provides in me.
\labelchapt{2}

\chapt{2}
\v{1}For I want you to know how much I struggle for you, for those in Laodicea, and for all who have never seen me face to face.\fnote{\fbackref{2:1} Lit. \fbib{my face in the flesh}} \v{2}Because they are united in love, I pray\fnote{\fbackref{2:2} The Gk. lacks \fbib{I pray}} that their hearts may be encouraged by all the riches that come from a complete understanding of the full knowledge of the Messiah,\fnote{\fbackref{2:2} Or \fbib{Christ}} who is\fnote{\fbackref{2:2} The Gk. lacks \fbib{who is}} the mystery of God. \v{3}In him are stored all the treasures of wisdom and knowledge. \v{4}I say this so that no one will mislead you with nice-sounding rhetoric. \v{5}For although I am physically absent, I am with you in spirit, rejoicing to see how stable you are and how firm your faith in the Messiah\fnote{\fbackref{2:5} Or \fbib{Christ}} is.
\passage{Fullness of Life}

\v{6}So then, just as you have received the Messiah\fnote{\fbackref{2:6} Or \fbib{Christ}} Jesus the Lord, continue to live dependent on him. \v{7}For you have been rooted in him and are being built up and strengthened in the faith, just as you were taught, while you continue to be thankful. \v{8}See to it that no one enslaves you through philosophy and empty deceit according to human tradition, according to the basic principles of the world,\fnote{\fbackref{2:8} Or \fbib{the elemental spirits of the universe}} and not according to the Messiah,\fnote{\fbackref{2:8} Or \fbib{Christ}} \v{9}because all the essence\fnote{\fbackref{2:9} Lit. \fbib{all of the fullness}} of deity inhabits him in bodily form. \v{10}And you have been filled by him, who is the head of every ruler and authority. \v{11}In union with him you were also circumcised with a circumcision performed without human\fnote{\fbackref{2:11} The Gk. lacks \fbib{human}} hands by stripping off the corrupt nature by the circumcision performed by the Messiah.\fnote{\fbackref{2:11} Or \fbib{Christ}} \v{12}When you were buried with the Messiah\fnote{\fbackref{2:12} Lit. \fbib{with him}} in baptism, you were also raised with him through faith in the power of God, who raised him from the dead. \v{13}Even when you were dead because of your offenses and the uncircumcision of your flesh, God\fnote{\fbackref{2:13} Lit. \fbib{he}} made you alive with him when he forgave us all of our offenses, \v{14}having erased the charges that were brought against us, along with their obligations that were hostile to us. He took those charges away when he nailed them to the cross. \v{15}And when he had disarmed the rulers and the authorities, he made a public spectacle of them, triumphing over them in the cross.\fnote{\fbackref{2:15} Lit. \fbib{in it}}

\v{16}Therefore, let no one judge you in matters of food and drink or with respect to a festival, a New Moon, or Sabbath days.\fnote{\fbackref{2:16} Lit. \fbib{or Sabbaths}} \v{17}These are a shadow of the things to come, but the reality\fnote{\fbackref{2:17} Or \fbib{substance}} belongs to the Messiah.\fnote{\fbackref{2:17} Or \fbib{Christ}} \v{18}Let no one who delights in humility and the worship of angels cheat you out of the prize by rejoicing about what he has seen.\fnote{\fbackref{2:18} Other mss. read \fbib{what he has not seen}} Such a person is puffed up for no reason by his carnal mind. \v{19}He does not hold on to the head, from whom the whole body, which is nourished and held together by its joints and ligaments, grows as God enables it.
\passage{The New Life in the Messiah}

\v{20}If you have died with the Messiah\fnote{\fbackref{2:20} Or \fbib{Christ}} to the basic principles of the world,\fnote{\fbackref{2:20} Or \fbib{the elemental spirits of the universe}} why are you submitting to its decrees as though you still lived in the world? \v{21}``Don't handle this! Don't taste or touch that!'' \v{22}All of these things will be destroyed as they are used, because they are based on human commands and teachings. \v{23}These things have the appearance of wisdom in promoting self-made religion, humility, and harsh treatment of the body, but they have no value against self-indulgence.
\labelchapt{3}
\passage{Keep Focusing on the Messiah}

\chapt{3}
\v{1}Therefore, if you have been raised with the Messiah,\fnote{\fbackref{3:1} Or \fbib{Christ}} keep focusing on the things that are above, where the Messiah\fnote{\fbackref{3:1} Or \fbib{Christ}} is seated at the right hand of God. \v{2}Keep your minds on things that are above, not on things that are on the earth. \v{3}For you have died, and your life has been safely guarded by the Messiah\fnote{\fbackref{3:3} Or \fbib{Christ}} in God. \v{4}When the Messiah,\fnote{\fbackref{3:4} Or \fbib{Christ}} who is\fnote{\fbackref{3:4} The Gk. lacks \fbib{who is}} your\fnote{\fbackref{3:4} Other mss. read \fbib{our}} life, is revealed, then you, too, will be revealed with him in glory.

\v{5}So put to death your worldly impulses:\fnote{\fbackref{3:5} Lit. \fbib{the parts that are on the earth}} sexual sin, impurity, passion, evil desire, and greed (which is idolatry). \v{6}It is because of these things that the wrath of God is coming on those who are disobedient.\fnote{\fbackref{3:6} Lit. \fbib{on the sons of disobedience}} \v{7}You used to behave like them as you lived among them. \v{8}But now you must also get rid of anger, wrath, malice, slander, obscene speech, and all such sins. \v{9}Do not lie to one another, for you have stripped off the old nature with its practices \v{10}and have clothed yourselves with the new nature, which is being renewed in full knowledge, consistent with the image of the one who created it. \v{11}In him\fnote{\fbackref{3:11} Lit. \fbib{it,} \fbib{\v{11}where}} there is no Greek or Jew, circumcised or uncircumcised, barbarian, Scythian,\fnote{\fbackref{3:11} I.e. uncivilized person} slave, or free person. Instead, the Messiah\fnote{\fbackref{3:11} Or \fbib{Christ}} is all and in all.

\v{12}Therefore, as God's chosen ones, holy and loved, clothe yourselves with compassion, kindness, humility, meekness,\fnote{\fbackref{3:12} Or \fbib{gentleness}} and patience. \v{13}Be tolerant of one another and forgive each other if anyone has a complaint against another. Just as the Lord\fnote{\fbackref{3:13} Other mss. read \fbib{the Messiah}} has forgiven you, you also should forgive.\fnote{\fbackref{3:13} Lit. \fbib{so you also}} \v{14}Above all, clothe yourselves with\fnote{\fbackref{3:14} The Gk. lacks \fbib{clothe yourselves with}} love, which ties everything together in unity. \v{15}Let the peace of the Messiah\fnote{\fbackref{3:15} Or \fbib{Christ}} also rule in your hearts, to which you were called in one body, and be thankful. \v{16}Let the word of the Messiah\fnote{\fbackref{3:16} Or \fbib{Christ}; other mss. read \fbib{of God}; still other mss. read \fbib{of the Lord}} inhabit you richly with wisdom, teaching and admonishing one another with psalms, hymns, and spiritual songs, and singing to God with thankfulness in your hearts. \v{17}And whatever you do, whether by speech or action, do everything in the name of the Lord Jesus, giving thanks to God the Father through him.
\passage{Family Duties}

\v{18}Wives, submit yourselves to your husbands, as is appropriate for those who belong to the Lord. \v{19}Husbands, love your wives, and do not be harsh with\fnote{\fbackref{3:19} Or \fbib{bitter toward}} them.

\v{20}Children, obey your parents in everything, for this is pleasing to the Lord. \v{21}Fathers, do not make your children resentful. Otherwise, they'll become discouraged.

\v{22}Slaves, obey your earthly masters in everything, not only while being watched in order to please them, but with a sincere heart, fearing the Lord. \v{23}Whatever you do, work at it wholeheartedly as though you were doing it\fnote{\fbackref{3:23} The Gk. \fbib{lacks though you were doing it}} for the Lord and not merely for people. \v{24}You know that it is from the Lord that you will receive the inheritance as a reward. It is the Lord Messiah\fnote{\fbackref{3:24} Or \fbib{Christ}} whom you are serving! \v{25}For the person who does what is wrong will be paid back for what he has done without favoritism.
\labelchapt{4}

\chapt{4}
\v{1}Masters, treat your slaves justly and fairly, because you know that you also have a Master in heaven.
\passage{Closing Exhortations}

\v{2}Devote yourselves to prayer. Be alert\fnote{\fbackref{4:2} Lit. \fbib{Be alert in it}} and thankful when you pray. \v{3}At the same time also pray for us---that God would open before us a door for the word so that we may tell the secret about the Messiah,\fnote{\fbackref{4:3} Or \fbib{Christ}} for which I have been imprisoned. \v{4}May I reveal it as clearly as I should!\fnote{\fbackref{4:4} Lit. \fbib{as I should speak}}

\v{5}Behave wisely toward outsiders, making the best use of your time. \v{6}Let your speech always be gracious, seasoned with salt, so that you may know how you ought to answer everyone.
\passage{Greetings from Paul and His Fellow Workers}

\v{7}Tychicus will tell you everything that has happened to me. He is a dear brother, a faithful minister, and a fellow servant in the Lord. \v{8}I am sending him to you for this very reason, so that you may know how we are doing and that he may encourage your hearts. \v{9}He is coming with Onesimus, that faithful and dear brother, who is one of you. They will tell you everything that is happening here.

\v{10}Aristarchus, my fellow prisoner, sends his greetings, as does Mark, the cousin of Barnabas. You have received instructions about him. If he comes to you, welcome him. \v{11}Jesus, who is called Justus, also greets you. These are the only ones of the circumcision who are fellow workers for the kingdom of God. They have been an encouragement to me. \v{12}Epaphras, who is one of you, a servant\fnote{\fbackref{4:12} Or \fbib{slave}} of the Messiah\fnote{\fbackref{4:12} Or \fbib{Christ}} Jesus, sends you his greetings. He is always wrestling in his prayers for you, so that you may stand mature,\fnote{\fbackref{4:12} Or \fbib{complete}} completely convinced of the entire will of God. \v{13}For I can testify on his behalf that he has a deep concern for you and for those in Laodicea and in Hierapolis. \v{14}Luke, the beloved physician, and Demas greet you. \v{15}Give my greetings to the brothers in Laodicea, especially to Nympha and the church that is in her house. \v{16}When this letter has been read among you, have it read also in the church of the Laodiceans, and be sure to read the one from Laodicea. \v{17}Tell Archippus, ``See that you complete the ministry you have received from the Lord.''
\passage{Final Greeting}

\v{18}This greeting is written with my own signature\fnote{\fbackref{4:18} Lit. \fbib{hand}}---``Paul.'' Remember that I remain imprisoned. May grace be with you! Amen.\fnote{\fbackref{4:18} Other mss. lack \fbib{Amen}}

\bookheader{1 Thessalonians}
\labelbook{1Thes}

\bookpretitle{The Letter of Paul Called}
\booktitle{First Thessalonians}

\labelchapt{1}
\passage{Greetings}

\chapt{1}
\v{1}From:\fnote{\fbackref{1:1} The Gk. lacks \fbib{From}} Paul, Silvanus,\fnote{\fbackref{1:1} I.e. Silas} and Timothy.

To: The church of the Thessalonians in union with God the Father and the Lord Jesus, the Messiah.\fnote{\fbackref{1:1} Or \fbib{Christ}}

May grace and peace from God our Father and the Lord Jesus, the Messiah,\fnote{\fbackref{1:1} Or \fbib{Christ}; other mss. lack \fbib{from God our Father and the Lord Jesus, the Messiah}} be yours!
\passage{Paul's Prayer for the Thessalonians}

\v{2}We always thank God for all of you when we mention you in our prayers. \v{3}In the presence of our God and Father, we constantly remember how your faith is active,\fnote{\fbackref{1:3} Lit. \fbib{your work of faith}} your love is hard at work,\fnote{\fbackref{1:3} Lit. \fbib{your labor of love}} and your hope in our Lord Jesus the Messiah\fnote{\fbackref{1:3} Or \fbib{Christ}} is enduring.\fnote{\fbackref{1:3} Lit. \fbib{the endurance of hope of our Lord Jesus, the Messiah}} \v{4}Brothers whom God loves, we know that he has chosen you, \v{5}for the gospel we brought\fnote{\fbackref{1:5} Lit. \fbib{our gospel}} did not come to you in words only, but also with power, with the Holy Spirit, and with deep conviction. Indeed,\fnote{\fbackref{1:5} Lit. \fbib{Just as}} you know what kind of people we proved to be while we were with you, acting on your behalf.

\v{6}You became imitators of us and of the Lord. In spite of a great deal of suffering, you welcomed the word with the joy that the Holy Spirit produces.\fnote{\fbackref{1:6} Or \fbib{the joy of the Holy Spirit}} \v{7}As a result, you became a model for all the believers in Macedonia and Achaia. \v{8}From you the word of the Lord has spread out not only in Macedonia and Achaia, but also in every place where your faith in God has become known. As a result, we do not need to say anything about it.

\v{9}For people\fnote{\fbackref{1:9} Lit. \fbib{they}} keep telling us what kind of welcome you gave us and how you turned away from idols to serve a living and true God \v{10}and to wait for his Son whom he raised from the dead to come back\fnote{\fbackref{1:10} The Gk. lacks \fbib{to come back}} from heaven. This Jesus is the one who rescues us from the coming wrath.
\labelchapt{2}
\passage{Paul Recalls His Visit to the Thessalonians}

\chapt{2}
\v{1}For you yourselves know, brothers, that our visit to you was not a waste of time. \v{2}As you know, we suffered persecution and were mistreated in Philippi. Yet we were encouraged by our God to tell you his\fnote{\fbackref{2:2} Lit. \fbib{God's}} gospel in spite of strong opposition. \v{3}For our appeal to you\fnote{\fbackref{2:3} The Gk. lacks \fbib{to you}} does not spring from deceit, impure motives, or trickery. \v{4}Rather, because we have been approved by God to be entrusted with the gospel, we speak as we do, not trying to please people but God, who tests our motives.

\v{5}As you know, we did not come with flattering words or with a scheme to make money. God is our witness! \v{6}We did not seek praise from people---from you or from anyone else--- \v{7}even though as apostles of the Messiah\fnote{\fbackref{2:7} Or \fbib{Christ}} we might have made such demands. Instead, we were gentle\fnote{\fbackref{2:7} Other mss. read \fbib{infants}} among you, like a nursing mother tenderly caring for her own children. \v{8}We cared so deeply for you that we were determined to share with you not only the gospel of God but our very lives. That is how dear you were to us. \v{9}Brothers, you remember our labor and toil. We worked night and day so that we would not become a burden to any of you while we proclaimed the gospel of God to you. \v{10}You and God are witnesses of how pure, honest, and blameless our conduct was among you who believe. \v{11}You know very well that we treated each of you the way a father treats\fnote{\fbackref{2:11} The Gk. lacks \fbib{treats}} his children. \v{12}We comforted and encouraged you, urging you to live in a manner worthy of God, who calls\fnote{\fbackref{2:12} Other mss. read \fbib{called}} you into his kingdom and glory.
\passage{How the Thessalonians Welcomed the Gospel}

\v{13}Here is another reason why we constantly give thanks to God: When you received God's word, which you heard from us, you did not accept it as the word of humans but for what it really is---the word of God, which is at work in you who believe. \v{14}For you, brothers, became imitators of the churches of God in Judea that are in union with the Messiah\fnote{\fbackref{2:14} Or \fbib{Christ}} Jesus. You suffered the same persecutions from the people of your own country as they did from those Jews \v{15}who killed the Lord Jesus and the\fnote{\fbackref{2:15} Other mss. read \fbib{their own}} prophets, who have persecuted us, and who please neither God nor any group of people, \v{16}as they try to keep us from telling the gentiles how they can be saved. As a result, they are constantly adding to the number of sins they have committed.\fnote{\fbackref{2:16} The Gk. lacks \fbib{they have committed}} However, wrath has overtaken them at last!
\passage{Timothy's Report to Paul}

\v{17}Brothers, although we have been separated from you for a little while---in person but not in heart---we eagerly desire to see you again face to face. \v{18}That is why we wanted to come to you. Certainly I, Paul, wanted to come\fnote{\fbackref{2:18} The Gk. lacks \fbib{wanted to come}} time and again, but Satan blocked our way. \v{19}After all, who is our hope, joy, or reason for\fnote{\fbackref{2:19} Lit. \fbib{or crown of}} rejoicing in the presence of our Lord Jesus at his coming? It is you, isn't it? \v{20}Yes, you are our glory and joy!
\labelchapt{3}

\chapt{3}
\v{1}Therefore, when we could stand it no longer, we decided to remain alone in Athens \v{2}and send Timothy, our brother who works with us for God in the gospel of the Messiah,\fnote{\fbackref{3:2} Or \fbib{Christ}; other mss. lack \fbib{of the Messiah}} to strengthen and encourage you in your faith, \v{3}so that no one would be shaken by these persecutions, for which you are aware that we were destined. \v{4}In fact, when we were with you, we told you ahead of time that we were going to suffer persecution. And as you know, that is what happened. \v{5}But when I could stand it no longer, I sent Timothy\fnote{\fbackref{3:5} The Gk lacks \fbib{Timothy}} to find out about your faith. I was afraid that the tempter had tempted you in some way, and that our work had been a waste of time.
\passage{Paul Rejoices about the Thessalonians}

\v{6}But Timothy has just now returned from visiting\fnote{\fbackref{3:6} The Gk. lacks \fbib{visiting}} you and has told us the good news about your faith and love. He also told us that you always have fond memories of us and want to see us, just as we want to see you. \v{7}That's why, brothers, in all our distress and persecution we have been encouraged about you by your faith. \v{8}For now we can go on living, as long as you continue to stand firm in the Lord. \v{9}How can we thank God enough for you in return for all the joy that we have in God's presence because of you? \v{10}We pray very hard night and day that we may see you again face to face, so that we may equip you with\fnote{\fbackref{3:10} The Gk. lacks \fbib{you with}} whatever is lacking in your faith.

\v{11}Now may our God and Father and our Lord Jesus provide a way for us to visit you. \v{12}May the Lord greatly increase your love\fnote{\fbackref{3:12} Lit. \fbib{cause you to increase and abound in love}} for each other and for all people, just as we love you.\fnote{\fbackref{3:12} Lit. \fbib{as we for you}} \v{13}Then your hearts will be strong, blameless, and holy in the presence of God, who is our Father, when our Lord Jesus appears with all his saints.\fnote{\fbackref{3:13} Or \fbib{holy ones}; other mss. read \fbib{saints. Amen}}
\labelchapt{4}
\passage{Instructions on the Way Christians Should Live}

\chapt{4}
\v{1}Now then, brothers, you learned from us how you ought to live and to please God, as in fact you are doing. We ask and encourage you in the Lord to do so even more. \v{2}You know what instructions we gave you through the Lord Jesus. \v{3}For it is God's will that you be sanctified: You must abstain from sexual immorality. \v{4}Each of you must know how to control his own body\fnote{\fbackref{4:4} Lit. \fbib{vessel}} in a holy and honorable manner, \v{5}not with passion and lust like the gentiles who do not know God. \v{6}Furthermore, you must never take advantage of or exploit a brother in this regard, because the Lord avenges all these things, just as we already told you and warned you. \v{7}For God did not call us to be impure, but to be holy. \v{8}Therefore, whoever rejects this instruction\fnote{\fbackref{4:8} The Gk. lacks \fbib{instruction}} is not rejecting human authority but God, who gives you his Holy Spirit.

\v{9}Now you do not need anyone to write to you about brotherly love, since you have been taught by God to love each other. \v{10}In fact, you are showing love to all the brothers throughout Macedonia, but we urge you, brothers, to keep on doing this even more. \v{11}Also, make it your goal to live quietly, to mind your own business, and to work with your own hands, as we instructed you, \v{12}so that you may win the respect of outsiders, and have need of nothing.
\passage{Comfort about Christians who Have Died}

\v{13}But we do not want you to be ignorant, brothers, about those who have died, so that you may not grieve like other people who have no hope. \v{14}For if we believe that Jesus died and rose again, even so through Jesus God will bring those who have died with him. \v{15}For we declare to you what the Lord has told us to say:\fnote{\fbackref{4:15} Lit. \fbib{you by the word of the Lord:}} We who are alive and remain until the coming of the Lord will by no means precede those who have died. \v{16}With a shout of command, with the archangel's call, and with the sound of God's trumpet, the Lord himself will come down from heaven, and the dead who belong to the Messiah\fnote{\fbackref{4:16} Or \fbib{Christ}} will rise first. \v{17}Then we who are alive and remain will be caught up in the clouds together with them to meet the Lord in the air. And so we will be with the Lord forever. \v{18}So then, encourage one another with these words.
\labelchapt{5}
\passage{Be Ready for the Day of the Lord}

\chapt{5}
\v{1}Now you do not need to have anything written to you about times and dates, brothers, \v{2}for you yourselves know very well that the Day of the Lord will come like a thief in the night. \v{3}When people\fnote{\fbackref{5:3} Lit. \fbib{they}} say, ``There is peace and security,'' destruction will strike them as suddenly as labor pains come\fnote{\fbackref{5:3} The Gk. lacks \fbib{come}} to a pregnant woman, and they will not be able to escape.

\v{4}However, brothers, you are not in the darkness, in order that the Day of the Lord\fnote{\fbackref{5:4} The Gk. lacks \fbib{of the Lord}} might surprise you like a thief. \v{5}For all of you are children of the light and children of the day. We do not belong to the night or to darkness. \v{6}Therefore, let's not fall asleep like others do, but let's stay awake and be sober. \v{7}For people who go to sleep, go to sleep at night; and people who get drunk, get drunk at night. \v{8}But since we belong to the day, let's be sober. We must put on the breastplate of faith and love, and the hope of salvation as a helmet. \v{9}For God has not destined us to receive\fnote{\fbackref{5:9} The Gk. lacks \fbib{receive}} wrath but to obtain salvation through our Lord Jesus, the Messiah,\fnote{\fbackref{5:9} Or \fbib{Christ}} \v{10}who died for us in order that, whether we are awake or asleep, we may live together with him. \v{11}So then, encourage one another and build each other up, as you are doing.
\passage{Paul Gives Final Instructions to the Church}

\v{12}Brothers, we ask you to show your appreciation for those who work among you, set an example for you in the Lord, and instruct\fnote{\fbackref{5:12} Or \fbib{admonish}} you. \v{13}Hold them in the highest regard, loving them because of their work. Live in peace with each other. \v{14}We urge you, brothers, to admonish\fnote{\fbackref{5:14} Or \fbib{instruct}} those who are idle,\fnote{\fbackref{5:14} Or \fbib{disorderly}} cheer up those who are discouraged, and help those who are weak. Be patient with everyone. \v{15}Make sure that no one pays back evil for evil. Instead, always pursue what is good for each other and for everyone else. \v{16}Always be joyful. \v{17}Continually be prayerful. \v{18}In everything be thankful, because this is God's will for you in the Messiah\fnote{\fbackref{5:18} Or \fbib{Christ}} Jesus. \v{19}Do not put out the Spirit's fire.\fnote{\fbackref{5:19} I.e. stifle the Spirit's work} \v{20}Do not despise prophecies. \v{21}Instead, test everything. Hold on to what is good. \v{22}Keep away from every kind of evil.
\passage{Final Greeting}

\v{23}May the God of peace himself make you holy in every way. And may your whole being---spirit, soul, and body---remain blameless when our Lord Jesus, the Messiah,\fnote{\fbackref{5:23} Or \fbib{Christ}} appears. \v{24}The one who calls you is faithful, and he will continue to be faithful.\fnote{\fbackref{5:24} The Gk. lacks \fbib{faithful}} \v{25}Brothers, pray\fnote{\fbackref{5:25} Other mss. read \fbib{also pray}} for us. \v{26}Greet all the brothers with a holy kiss. \v{27}I order you by the Lord to have this letter read to all the brothers. \v{28}May the grace of our Lord Jesus, the Messiah,\fnote{\fbackref{5:28} Or \fbib{Christ}} be with you! Amen.\fnote{\fbackref{5:28} Other mss. lack \fbib{Amen}}

\bookheader{2 Thessalonians}
\labelbook{2Thes}

\bookpretitle{The Letter from Paul Called}
\booktitle{Second Thessalonians}

\labelchapt{1}
\passage{Greetings}

\chapt{1}
\v{1}From:\fnote{The Gk. lacks \fbib{From}} Paul, Silvanus,\fnote{I.e. Silas} and Timothy.

To: The church of the Thessalonians in union with God our Father and the Lord Jesus, the Messiah.\fnote{Or \fbib{Christ}}

\v{2}May grace and peace from God our Father and the Lord Jesus, the Messiah,\fnote{Or \fbib{Christ}} be yours!
\passage{Endurance in Affliction}

\v{3}Brothers, at all times we are obligated to thank God for you. It is right to do this\fnote{Lit. \fbib{As is right}} because your faith is growing all the time and the love every one of you has for each other is increasing. \v{4}As a result, we rejoice about you among God's churches---about your endurance and faith through all the persecutions and afflictions you are experiencing. \v{5}This is evidence of God's righteous judgment and is intended to make you worthy of God's kingdom, for which you are suffering. \v{6}Certainly it is right for God to pay back those who afflict you with affliction, \v{7}and to give us who are afflicted relief when the Lord Jesus is revealed from heaven with his mighty angels \v{8}in blazing fire. He will take revenge on those who do not know God and on those who refuse to obey the gospel of our Lord Jesus. \v{9}Such people will suffer the punishment of eternal destruction by being separated from the Lord's presence and from his glorious power \v{10}when he comes to be glorified by his saints and to be regarded with wonder on that day by all who have believed---including you, because you believed our testimony.

\v{11}With this in mind, we always pray for you, asking\fnote{The Gk. lacks \fbib{asking}} that our God might make you worthy of his calling and that through his power he might help you accomplish every good desire and faithful action. \v{12}That way the name of our Lord Jesus will be glorified by you, and you by him, according to the grace of our God and Lord, Jesus, the Messiah.\fnote{Or \fbib{Christ}}
\labelchapt{2}
\passage{The Lawless One}

\chapt{2}
\v{1}Now we ask you, brothers, regarding the coming of our Lord Jesus, the Messiah,\fnote{Or \fbib{Christ}} and our gathering together to him, \v{2}not to be so quickly upset\fnote{Lit. \fbib{shaken in mind}} or alarmed when someone claims that we said,\fnote{Lit. \fbib{as though by us}} either by some spirit, conversation, or letter that the Day of the Lord has already come. \v{3}Do not let anyone deceive you in any way, for it will not come unless the rebellion\fnote{Or \fbib{apostasy}} takes place first and the man of sin,\fnote{Other mss. read \fbib{man of lawlessness}} who is destined for destruction,\fnote{Lit. \fbib{the son of destruction}} is revealed. \v{4}He opposes and exalts himself above every so-called god and object of worship. As a result, he seats himself in the sanctuary\fnote{Or \fbib{temple}} of God and himself declares that he is God.

\v{5}Don't you remember that I repeatedly told you about these things when I was still with you? \v{6}You know what it is that is now holding him back, so that he will be revealed when his time comes. \v{7}For the secret of this lawlessness is already at work, but only until the person now holding it back gets out of the way. \v{8}Then the lawless one will be revealed, whom the Lord\fnote{Some MSS read \fbib{Lord Jesus}} will destroy with the breath of his mouth, rendering him powerless by the manifestation of his coming.

\v{9}The coming of the lawless one\fnote{Lit. \fbib{His coming}} will be accompanied by the power of Satan. He will use every kind\fnote{Lit. \fbib{In every kind}} of power, including miraculous signs, lying wonders, \v{10}and every type of evil to deceive\fnote{Lit. \fbib{every evil deception}} those who are dying, those who refused to love the truth that would save them.\fnote{Lit. \fbib{so that they might be saved}} \v{11}For this reason, God will send them a powerful delusion so that they will believe the lie. \v{12}Then all who have not believed the truth but have taken pleasure in unrighteousness will be condemned.

\v{13}At all times we are obligated to thank God for you, brothers who are loved by the Lord, because God chose you to be the first fruits\fnote{Other mss. read \fbib{from the beginning}} for salvation through sanctification by the Spirit and through faith in the truth. \v{14}With this purpose in mind, he called you through our proclamation of the\fnote{The Gk. lacks \fbib{proclamation of the}} gospel so that you would obtain the glory of our Lord Jesus, the Messiah.\fnote{Or \fbib{Christ}} \v{15}So then, brothers, stand firm, and cling to the traditions that you were taught by us, either by word of mouth\fnote{Lit. \fbib{by word}} or by our letter. \v{16}May our Lord Jesus, the Messiah\fnote{Or \fbib{Christ}} himself, and may God our Father, who loved us and by his grace gave us eternal comfort\fnote{Or \fbib{encouragement}} and good hope, \v{17}encourage your hearts and strengthen you in every good action and word.
\labelchapt{3}
\passage{An Example to Follow}

\chapt{3}
\v{1}Finally, brothers, pray for us---that the word of the Lord may spread rapidly, and that it may be honored the way it is among you. \v{2}Also pray\fnote{The Gk. lacks \fbib{pray}} that we may be rescued from worthless and evil people, since not everyone holds to the faith.\fnote{Or \fbib{has faith}} \v{3}But the Lord is faithful and will strengthen you and protect you from the evil one. \v{4}We have confidence in the Lord\fnote{Lit. \fbib{in the Lord concerning you}} that you are doing and will continue to do what we command. \v{5}May the Lord direct your hearts to the love of God and to the endurance of the Messiah.\fnote{Or \fbib{Christ}}

\v{6}In the name of our Lord Jesus, the Messiah,\fnote{Or \fbib{Christ}} we command you, brothers, to keep away from every brother who is living in idleness\fnote{Or \fbib{is leading a disorderly life}} and not living\fnote{The Gk. lacks \fbib{living}} according to the tradition that they received\fnote{Other mss. read \fbib{you received}} from us. \v{7}For you yourselves know what you must do to imitate us. We never lived in idleness\fnote{Or \fbib{We did not lead a disorderly life}} among you. \v{8}We did not eat anyone's food without paying for it. Instead, with toil and labor we worked night and day in order not to be a burden to any of you. \v{9}It is not as though we did not have that right, but we wanted\fnote{The Gk. lacks \fbib{we wanted}} to give you an example to follow. \v{10}While we were with you, we gave this order: ``If anyone doesn't want to work, he shouldn't eat.''

\v{11}We hear that some of you are living in idleness.\fnote{Or \fbib{living disorderly lives}} You are not busy working\fnote{Lit. \fbib{ergazomenous} (working)}---you are busy interfering\fnote{Lit. \fbib{periergazomenous} (uselessly working)} in other people's lives! \v{12}We order and encourage such people by the Lord Jesus, the Messiah,\fnote{Or \fbib{Christ}} to do their work quietly and to earn their own living. \v{13}Brothers, do not get tired of doing what is right.

\v{14}If anyone does not obey what we say\fnote{Lit. \fbib{our word}} in this letter, take note of him. Have nothing to do with him so that he will feel ashamed. \v{15}Yet, don't treat him like an enemy, but warn\fnote{Or \fbib{instruct}} him like a brother. \v{16}Now may the Lord of peace give you his peace at all times and in every way. May the Lord be with all of you.
\passage{Final Greeting}

\v{17}I, Paul, am writing this greeting with my own hand. This is the mark in every letter of mine. It is the way I write. \v{18}May the grace of our Lord Jesus, the Messiah,\fnote{Or \fbib{Christ}} be with all of you. Amen.\fnote{Other mss. lack \fbib{Amen}}

\bookheader{1 Timothy}
\labelbook{1Tim}

\bookpretitle{The Letter from Paul Called}
\booktitle{First Timothy}

\labelchapt{1}
\passage{Greetings}

\chapt{1}
\v{1}From:\fnote{\fbackref{1:1} The Gk. lacks \fbib{From}} Paul, an apostle of the Messiah\fnote{\fbackref{1:1} Or \fbib{Christ}} Jesus, by the command of God our Savior and the Messiah\fnote{\fbackref{1:1} Or \fbib{Christ}} Jesus, our hope.

\v{2}To: Timothy, my genuine child in the faith.

May grace, mercy, and peace from God the Father and the Messiah\fnote{\fbackref{1:2} Or \fbib{Christ}} Jesus, our Lord, be yours!
\passage{A Warning against False Teachers}

\v{3}When I was on my way to Macedonia, I urged you to stay in Ephesus so that you could instruct certain people to stop teaching false doctrine \v{4}and occupying themselves with myths and endless genealogies. These things promote controversies rather than God's ongoing purpose, which involves faith. \v{5}The goal of this instruction is love that flows from a pure heart, from a clear conscience, and from a sincere faith. \v{6}Some people have left these qualities behind and have turned to fruitless discussion. \v{7}They want to be teachers of the Law, yet they do not understand either what they are talking about or the things about which they speak so confidently.

\v{8}Of course, we know that the Law is good if a person uses it legitimately, \v{9}that is, if he understands that the Law is not intended for righteous\fnote{\fbackref{1:9} Or \fbib{innocent}} people but for lawbreakers and rebels, for ungodly people and sinners, for those who are unholy and irreverent, for those who kill their fathers, their mothers, or other people, \v{10}for those involved in sexual immorality, for homosexuals, for kidnappers,\fnote{\fbackref{1:10} Or \fbib{slave traders}} for liars, for false witnesses, and for whatever else goes against the healthy teaching \v{11}that agrees with the glorious gospel of the blessed God, which he entrusted to me.

\v{12}I thank the Messiah\fnote{\fbackref{1:12} Or \fbib{Christ}} Jesus, our Lord, who gives me strength, that he has considered me faithful and has appointed me to his service. \v{13}In the past I was a blasphemer, a persecutor, and a violent\fnote{\fbackref{1:13} Or \fbib{an arrogant}} man. But I received mercy because I acted ignorantly in my unbelief, \v{14}and the grace of our Lord overflowed toward me,\fnote{\fbackref{1:14} The Gk. lacks \fbib{toward me}} along with the faith and love that are in the Messiah\fnote{\fbackref{1:14} Or \fbib{Christ}} Jesus. \v{15}This is a trustworthy saying that deserves complete acceptance:\fnote{\fbackref{1:15} This formula accompanied early Christian sayings on which full reliance could be placed.}

\begin{poetry}
\poeml To this world Messiah\fnote{\fbackref{1:15} Or \fbib{Christ}; lit. \fbib{Messiah Jesus}} came, \\
\poemll    sinful people to reclaim.\fnote{\fbackref{1:15} Or \fbib{save}}
\end{poetry}

I am the worst of them. \v{16}But for that very reason I received mercy, so that in me, as the worst sinner,\fnote{\fbackref{1:16} The Gk. lacks \fbib{sinner}} the Messiah\fnote{\fbackref{1:16} Or \fbib{Christ}} Jesus might demonstrate all of his patience as an example for those who would believe in him for eternal life. \v{17}Now to the King Eternal---the immortal, invisible, and only God---be honor and glory forever and ever! Amen.
\passage{Guidelines for Behavior in the Church}

\v{18}Timothy, my child, I am instructing you in keeping with the prophecies made earlier about you, so that by following them you may continue to fight the good fight \v{19}with faith and a good conscience. By ignoring their consciences,\fnote{\fbackref{1:19} Lit. \fbib{By ignoring it}} some people have destroyed their faith like a wrecked ship. \v{20}These include Hymenaeus and Alexander, whom I handed over to Satan so that they may learn not to blaspheme.
\labelchapt{2}
\passage{Prayer and Submission to Authority}

\chapt{2}
\v{1}First of all, then, I urge you to offer to God\fnote{\fbackref{2:1} The Gk. lacks \fbib{to God}} petitions, prayers, intercessions, and expressions of thanks for all people, \v{2}for kings, and for everyone who has authority, so that we might lead a quiet and peaceful life with all godliness and dignity.\fnote{\fbackref{2:2} Or \fbib{seriousness}} \v{3}This is good and acceptable in the sight of God our Savior, \v{4}who wants all people to be saved and to come to know the truth fully. \v{5}There is one God. There is also one mediator between God and human beings---a human, the Messiah\fnote{\fbackref{2:5} Or \fbib{Christ}} Jesus. \v{6}He gave himself as a ransom for everyone, the testimony at the proper time. \v{7}For this reason I was appointed to be an announcer, an apostle, and a faithful and true teacher of the gentiles. (I am telling you the truth.\fnote{\fbackref{2:7} Other mss. read \fbib{the truth in the Messiah}} I am not lying.)
\passage{Instructions to Men and Women}

\v{8}Therefore, I want the men everywhere to pray, lifting up holy hands without being angry or argumentative. \v{9}Women, for their part, should display their beauty by dressing modestly and decently in appropriate clothes, not with elaborate hairstyles or by wearing gold, pearls, or expensive clothes, \v{10}but through good actions. This is proper for women who claim to revere God.

\v{11}Let a woman learn with a quiet spirit, and submissively. \v{12}Moreover, in the area of teaching, I am not allowing a woman to instigate conflict toward a man. Instead, she is to remain calm. \v{13}For Adam was formed first, then Eve, \v{14}and it was not Adam who was deceived. It was the woman who was deceived and became disobedient, \v{15}even though she will be saved through the birth of the Child,\fnote{\fbackref{2:15} I.e. through the redemptive work of the incarnation; or \fbib{through childbearing}} if they continue in faith, love, and holiness, along with good judgment.\fnote{\fbackref{2:15} Or \fbib{modesty}}
\labelchapt{3}
\passage{Qualifications for Leaders in the Church}

\chapt{3}
\v{1}This is a trustworthy saying:\fnote{\fbackref{3:1} This formula accompanied early Christian sayings on which full reliance could be placed.}

\begin{poetry}
\poeml The one who would an elder be, \\
\poeml a noble task desires he.
\end{poetry}

\v{2}Therefore, an elder must be blameless, the husband of one wife,\fnote{\fbackref{3:2} Or \fbib{devoted to his wife}; lit. \fbib{a man of one woman}} stable, sensible, respectable, hospitable to strangers, and teachable.\fnote{\fbackref{3:2} Or \fbib{able to teach}} \v{3}He must not drink excessively or be a violent person, but instead be gentle. He must not be argumentative or love money. \v{4}He must manage his own family well and have children who are submissive and respectful in every way. \v{5}For if a man does not know how to manage his own family, how can he take care of God's church? \v{6}He must not be a recent convert, so that he won't become arrogant and fall into the devil's condemnation. \v{7}He must be well thought of by outsiders, so he doesn't\fnote{\fbackref{3:7} Lit. \fbib{outsiders, lest he}} fall into disgrace and the trap set for him by\fnote{\fbackref{3:7} Lit. \fbib{the trap of}} the devil.

\v{8}Deacons,\fnote{\fbackref{3:8} Or \fbib{Ministers}} too, must be serious. They must not be two-faced,\fnote{\fbackref{3:8} Lit. \fbib{double-worded}} addicted to wine, or greedy for money. \v{9}They must hold firmly to the secret of the faith with clear consciences. \v{10}But they must first be tested. Then, if they prove to be blameless, let them serve in ministry.\fnote{\fbackref{3:10} Or \fbib{them be deacons}} \v{11}Their wives\fnote{\fbackref{3:11} Or \fbib{Women}} must also be serious. They must not be gossips, but instead be stable and trustworthy in everything. \v{12}Deacons\fnote{\fbackref{3:12} Or \fbib{Ministers}} must be husbands of one wife\fnote{\fbackref{3:12} Or \fbib{devoted to their wives}; lit. \fbib{men of one woman}} and must manage their children and their families well. \v{13}Those who serve well in ministry\fnote{\fbackref{3:13} Or \fbib{well as deacons}} gain an excellent reputation for themselves and will have great assurance by their faith in the Messiah\fnote{\fbackref{3:13} Or \fbib{Christ}} Jesus.

\v{14}I hope to come to you soon. However, I'm writing this to you \v{15}in case I am delayed, so that you may know how to behave in God's household, which is the church of the living God, the pillar and foundation of the truth. \v{16}By common confession, the secret of our godly worship is great:\fnote{\fbackref{3:16} What follows probably represents an early Christian hymn or creed.}

\begin{poetry}
\poeml In flesh was he\fnote{\fbackref{3:16} Other mss. read \fbib{God}} revealed to sight, \\
\poemll    kept righteous by the Spirit's might, \\
\poemlll       adored by angels singing.\fnote{\fbackref{3:16} Lit. \fbib{he was seen by angels}} \\
\poeml To nations was he manifest, \\
\poemll    believing souls found peace and rest,\fnote{\fbackref{3:16} Lit. \fbib{he was believed in the world}} \\
\poemlll       our Lord in heaven reigning!\fnote{\fbackref{3:16} Lit. \fbib{he was taken up in glory}}
\end{poetry}
\labelchapt{4}
\passage{A Prophecy about the Future}

\chapt{4}
\v{1}Now the Spirit says clearly that in the last times some people will abandon the faith by following deceitful spirits, the teachings of demons, \v{2}and the hypocrisy of liars, whose consciences have been burned by a hot iron. \v{3}They will try to stop people from marrying and from eating certain foods, which God created to be received with thanksgiving by those who believe and know the truth. \v{4}For everything God created is good, and nothing should be rejected if it is received with thanksgiving, \v{5}because it is sanctified by the word of God and prayer.
\passage{How to be a Good Servant of the Messiah Jesus}

\v{6}If you continue to point these things out to the brothers, you will be a good servant of the Messiah\fnote{\fbackref{4:6} Or \fbib{Christ}} Jesus, nourished by the words of the faith and the healthy teaching that you have followed closely. \v{7}Do not have anything to do with godless myths and fables of old women. Instead, train yourself to be godly. \v{8}Physical exercise is of limited value, but

\begin{poetry}
\poeml Godliness is very dear, \\
\poemll    a pledge of life, both there and here.
\end{poetry}

\v{9}This is a trustworthy saying that deserves complete acceptance.\fnote{\fbackref{4:9} This formula accompanied early Christian sayings on which full reliance could be placed.} \v{10}To this end we work hard and struggle,\fnote{\fbackref{4:10} Other mss. read \fbib{suffer abuse}} because we have set our hope on the living God, who is the Savior of all people, that is, of those who believe.

\v{11}These are the things you must insist on and teach. \v{12}Do not let anyone look down on you because you are young, but be an example for other believers in your speech, behavior, love, faithfulness,\fnote{\fbackref{4:12} Or \fbib{faith}} and purity. \v{13}Until I arrive, give your full concentration to the public reading of Scripture,\fnote{\fbackref{4:13} Lit. \fbib{on the reading}} to exhorting, and to teaching. \v{14}Do not neglect\fnote{\fbackref{4:14} Or \fbib{Stop neglecting}} the gift that is in you, which was given to you through prophecy when the elders laid their hands on you. \v{15}Think on these things. Devote your life to them so that everyone can see your progress. \v{16}Pay close attention to your life and your teaching. Persevere in these things, because if you do so, you will save both yourself and those who listen to you.
\labelchapt{5}
\passage{Treatment of Widows}

\chapt{5}
\v{1}Never speak harshly to an older man, but appeal to him as if he were your father. Treat\fnote{\fbackref{5:1} The Gk. lacks \fbib{Treat}} younger men like brothers, \v{2}older women like mothers, and younger women like sisters, with absolute purity.

\v{3}Honor widows who have no other family members to care for them.\fnote{\fbackref{5:3} Lit. \fbib{who are really widows}} \v{4}But if a widow has children or grandchildren, they must first learn to respect their own family by repaying their parents, for this is pleasing in God's sight. \v{5}A woman who has no other family members to care for her\fnote{\fbackref{5:5} Lit. \fbib{who is really a widow}} and who is left all alone has placed her hope in God and devotes herself to petitions and prayers night and day. \v{6}But the self-indulgent widow\fnote{\fbackref{5:6} Lit. \fbib{the one}} is just as good as dead.

\v{7}Continue to give these instructions, so that they may be blameless. \v{8}If anyone does not take care of his own relatives, especially his immediate family, he has denied the faith and is worse than an unbeliever. \v{9}A widow may be put on the widows'\fnote{\fbackref{5:9} The Gk. lacks \fbib{widows'}} list if she is at least sixty years old and has been the wife of one husband.\fnote{\fbackref{5:9} Or \fbib{devoted to her husband}; lit. \fbib{a woman of one man}} \v{10}She must be well known for her good actions as a woman who has raised children, welcomed strangers, washed the saints' feet, helped the suffering, and devoted herself to doing good in every way.

\v{11}But do not include younger widows on your list.\fnote{\fbackref{5:11} The Gk. lacks \fbib{on your list}} For whenever their natural desires cause them to lose their devotion to the Messiah,\fnote{\fbackref{5:11} Or \fbib{Christ}} they want to remarry. \v{12}They receive condemnation because they have set aside their prior commitment to the Messiah.\fnote{\fbackref{5:12} The Gk. lacks \fbib{to the Messiah}} \v{13}At the same time, they also learn how to be lazy while going from house to house. Not only this, but they even become gossips and keep busy by interfering in other people's lives, saying things they should not say.

\v{14}Therefore, I want younger widows to remarry, have children, manage their homes, and not give the enemy any chance to ridicule them. \v{15}For some widows\fnote{\fbackref{5:15} The Gk. lacks \fbib{widows}} have already turned away to follow Satan. \v{16}If any woman\fnote{\fbackref{5:16} Other mss. read \fbib{man or woman}} is a believer and has relatives who are widows, she should help them. The church should not be burdened, so it can help those widows who have no other family members to care for them.\fnote{\fbackref{5:16} Lit. \fbib{who are really widows}}
\passage{Elders and Their Duties}

\v{17}Elders who handle their duties\fnote{\fbackref{5:17} Or \fbib{who rule}} well should be considered worthy of double compensation,\fnote{\fbackref{5:17} Or \fbib{honor}} especially those who work hard at preaching and teaching. \v{18}For the Scripture says, ``You must not muzzle an ox while it is treading out grain,''\fnote{\fbackref{5:18} Cf. Deut 25:4} and, \red{``A worker deserves his pay.''}\fnote{\fbackref{5:18} Cf. Luke 10:7} \v{19}Do not accept an accusation against an elder unless it is supported ``by two or three witnesses.''\fnote{\fbackref{5:19} Cf. Deut 17:6; 19:15} \v{20}As for those who keep on sinning, rebuke them in front of everyone so that the rest will also be afraid. \v{21}With God as my witness, as well as the Messiah\fnote{\fbackref{5:21} Or \fbib{Christ}} Jesus and the chosen angels, I solemnly call on you to carry out these instructions without prejudice, doing nothing on the basis of partiality. \v{22}Do not ordain\fnote{\fbackref{5:22} Lit. \fbib{lay hands on}} anyone hastily. Do not participate in the sins of others. Keep yourself pure. \v{23}Stop drinking only water, but use a little wine for your stomach because of your frequent illnesses.

\v{24}The sins of some people are obvious, leading them to judgment. The sins\fnote{\fbackref{5:24} Lit. \fbib{Those}} of others follow them there. \v{25}In the same way, good actions are obvious, and those that are not cannot remain hidden.
\labelchapt{6}
\passage{Duties of Servants and Masters}

\chapt{6}
\v{1}All who are under the yoke of slavery should regard their own masters as deserving of the highest respect,\fnote{\fbackref{6:1} Or \fbib{of full honor}} so that the name of God and our teaching may not be discredited.\fnote{\fbackref{6:1} Or \fbib{slandered}} \v{2}Moreover, those who have believing masters should be respectful to them, because they are fellow believers.\fnote{\fbackref{6:2} Lit. \fbib{brothers}} In fact, they must serve them even better, because those who benefit from their service are believers and dear to them. These are the things you must teach and exhort.
\passage{Rules for Godly Living}

\v{3}If anyone teaches false doctrine and refuses to agree with the sound words of our Lord Jesus, the Messiah,\fnote{\fbackref{6:3} Or \fbib{Christ}} and godly teaching, \v{4}he is a conceited person and does not understand anything. He has an unhealthy craving for arguments and debates. This produces jealousy, rivalry, slander, evil suspicions, \v{5}and incessant conflict between people who are depraved in mind and deprived of truth. They think that godliness is a way to make a profit.\fnote{\fbackref{6:5} Other mss. read \fbib{make a profit. Stay away from such people.}} \v{6}Of course, godliness with contentment does bring a great profit.

\begin{poetry}
\poeml \v{7}Nothing to this world we bring; \\
\poemll    from it take we nothing. \\
\poeml \v{8}With food to eat and clothes to wear; \\
\poemll    content we are in everything.
\end{poetry}

\v{9}But people who want to get rich keep toppling into temptation and are trapped by many stupid and harmful desires that plunge them into destruction and ruin. \v{10}For the love of money is a root of all kinds of evil. Some people, in their eagerness to get rich, have wandered away from the faith and caused\fnote{\fbackref{6:10} Lit. \fbib{pierced}} themselves a lot of pain.
\passage{Advice for Timothy}

\begin{poetry}
\poeml \v{11}But you, man of God, must flee from all these things. \\
\poemll    Instead, you must pursue righteousness, godliness, faithfulness,\fnote{\fbackref{6:11} Or \fbib{faith}} \\
\poemlll       love, endurance, and gentleness. \\
\poeml \v{12}Fight the good fight for the faith. \\
\poemll    Keep holding on to eternal life, to which you were called \\
\poeml and about which you gave a good testimony \\
\poemll    in front of many witnesses.
\end{poetry}

\v{13}Since you are\fnote{\fbackref{6:13} The Gk. lacks \fbib{Since you are}} in the presence of God, who gives life to everything, and in the presence of the Messiah\fnote{\fbackref{6:13} Or \fbib{Christ}} Jesus, who gave a good testimony before Pontius Pilate, I solemnly charge you \v{14}to keep these commands stainlessly and blamelessly until the appearance of our Lord Jesus, the Messiah.\fnote{\fbackref{6:14} Or \fbib{Christ}} \v{15}At the right time, he will make him known.

\begin{poetry}
\poeml God\fnote{\fbackref{6:15} Lit. \fbib{He}} is the blessed and only Ruler, \\
\poemll    the King of kings \\
\poemlll       and Lord of lords. \\
\poeml \v{16}He alone has endless life \\
\poemll    and lives in inaccessible light. \\
\poeml No one has ever seen him, \\
\poemll    nor can anyone see him. \\
\poeml Honor and eternal power belong to him! \\
\poemll    Amen.
\end{poetry}
\passage{Advice for the Wealthy}

\v{17}Tell those who are rich in this age not to be arrogant and not to place their confidence in anything as uncertain as riches. Instead, let them place their confidence\fnote{\fbackref{6:17} The Gk. lacks \fbib{let them place their confidence}} in God, who lavishly provides us with everything for our enjoyment. \v{18}They are to do good, to be rich in good actions, to be generous, and to share. \v{19}By doing this they store up a treasure for themselves that is a good foundation for the future, so that they can keep their hold on the life that is real.
\passage{Final Greeting}

\v{20}Timothy, guard what has been entrusted to you. Avoid the pointless discussions and contradictions of what is falsely called knowledge. \v{21}Although some claim to have it, they have abandoned the faith. May grace be with all of you!

\bookheader{2 Timothy}
\labelbook{2Tim}

\bookpretitle{The Letter of Paul Called}
\booktitle{Second Timothy}

\labelchapt{1}
\passage{Greetings}

\chapt{1}
\v{1}From:\fnote{\fbackref{1:1} The Gk. lacks \fbib{From}} Paul, an apostle of the Messiah\fnote{\fbackref{1:1} Or \fbib{Christ}} Jesus by God's will in keeping with the promise of life that is in the Messiah\fnote{\fbackref{1:1} Or \fbib{Christ}} Jesus.

\v{2}To: Timothy, my dear child.

May grace, mercy, and peace from God the Father and the Messiah\fnote{\fbackref{1:2} Or \fbib{Christ}} Jesus our Lord be yours!
\passage{Paul's Advice for Timothy}

\v{3}I constantly thank my God---whom I serve\fnote{\fbackref{1:3} Or \fbib{worship}} with a clear conscience, as my ancestors did---when I remember you in my prayers night and day, \v{4}recalling your tears and longing to see you so that I can be filled with joy. \v{5}I am reminded of your sincere faith, which first existed in your grandmother Lois and your mother Eunice, and I am convinced that this faith\fnote{\fbackref{1:5} Lit. \fbib{it}} also exists in you. \v{6}For this reason, I am reminding you to fan into flames the gift of God that is within you through the laying on of my hands. \v{7}For God did not give us a spirit of timidity but one of power, love, and self-discipline.\fnote{\fbackref{1:7} Or \fbib{good judgment}} \v{8}Therefore, never be ashamed of the testimony about our Lord or of me, his prisoner. Instead, by God's power, join me in suffering for the sake of the gospel.

\begin{poetry}
\poeml \v{9}He saved us \\
\poemll    and called us with a holy calling, \\
\poeml not according to our own accomplishments, \\
\poemll    but according to his own purpose and the grace \\
\poeml that was given to us in the Messiah\fnote{\fbackref{1:9} Or \fbib{Christ}} Jesus \\
\poemll    before time began.\fnote{\fbackref{1:9} Lit. \fbib{before the times of the ages}} \\
\poeml \v{10}Now, however, that grace\fnote{\fbackref{1:10} Lit. \fbib{however, it}} has been revealed \\
\poemll    through the coming of our Savior the Messiah\fnote{\fbackref{1:10} Or \fbib{Christ}} Jesus, \\
\poeml who has destroyed death \\
\poemll    and through the gospel has brought life \\
\poemlll       and release from death into full view.
\end{poetry}

\v{11}For the sake of this gospel\fnote{\fbackref{1:11} Lit. \fbib{For which}} I was appointed to be a preacher, an apostle, and a teacher of the gentiles.\fnote{\fbackref{1:11} Other mss. lack \fbib{of the gentiles}} \v{12}That is why I suffer as I do. However, I am not ashamed, for I know the one in whom I have put my trust, and I'm convinced that he is able to protect what he has entrusted to me\fnote{\fbackref{1:12} Or \fbib{what I have entrusted to him}} until the day that he comes.\fnote{\fbackref{1:12} Lit. \fbib{until that day}} \v{13}Hold on to the pattern of healthy teachings that you have heard from me, along with the faith and love that are in the Messiah\fnote{\fbackref{1:13} Or \fbib{Christ}} Jesus. \v{14}With the help of the Holy Spirit who lives in us, protect the good treasure that has been entrusted to you.
\passage{News about Paul's Helpers}

\v{15}You know that everyone in Asia has abandoned me, including Phygelus and Hermogenes. \v{16}May the Lord grant mercy to the family of Onesiphorus, for he often took care of\fnote{\fbackref{1:16} Or \fbib{refreshed}} me and was not ashamed that I was a prisoner. \v{17}Instead, when he arrived in Rome he searched diligently for me and found me. \v{18}May the Lord grant that he finds mercy on the day he comes again.\fnote{\fbackref{1:18} Lit. \fbib{mercy from the Lord on that day}} You know very well how much he assisted me in Ephesus.
\labelchapt{2}
\passage{Remain Committed to the Messiah Jesus}

\chapt{2}
\v{1}As for you, my child, be strong by the grace that is in the Messiah\fnote{\fbackref{2:1} Or \fbib{Christ}} Jesus. \v{2}What you have heard from me through many witnesses entrust to faithful people who will be able to teach others as well. \v{3}Join me in suffering like a good soldier of the Messiah\fnote{\fbackref{2:3} Or \fbib{Christ}} Jesus. \v{4}No one serving in the military gets mixed up in civilian matters, for his aim is to please his commanding officer. \v{5}Moreover, no one who is an athlete wins a prize unless he competes according to the rules. \v{6}Furthermore, it is the hard working farmer who should have the first share of the crops. \v{7}Think about what I am saying. The Lord will help you to understand all these things.

\v{8}Meditate on\fnote{\fbackref{2:8} Or \fbib{Remember}} Jesus, the Messiah,\fnote{\fbackref{2:8} Or \fbib{Christ}} who was raised from the dead and is a descendant of David. This is the gospel I tell others.\fnote{\fbackref{2:8} Lit. \fbib{of David, according to my gospel}} \v{9}Because of it I am experiencing trouble, even to the point of being chained like a criminal. However, God's word is not chained. \v{10}For that reason, I endure everything for the sake of those who have been chosen so that they, too, may receive the salvation that is in the Messiah\fnote{\fbackref{2:10} Or \fbib{Christ}} Jesus, along with eternal glory. \v{11}This saying is trustworthy:\fnote{\fbackref{2:11} This formula accompanied early Christian sayings on which full reliance could be placed.}

\begin{poetry}
\poeml In dying with the Messiah,\fnote{\fbackref{2:11} Lit. \fbib{him}} \\
\poemll    true life we gain.\fnote{\fbackref{2:11} Lit. \fbib{we will live with him}} \\
\poeml \v{12}Enduring, we with him will reign. \\
\poemll    Who him denies, \\
\poemlll       he will disclaim. \\
\poeml \v{13}Our faith may fail, \\
\poemll    his never wanes--- \\
\poeml That's who he is, \\
\poemll    he cannot change!\fnote{\fbackref{2:13} Lit. \fbib{he cannot deny himself}}
\end{poetry}

\v{14}Remind others about these things, and warn them before God\fnote{\fbackref{2:14} Other mss. read \fbib{of the Lord}} not to argue over words. Arguing\fnote{\fbackref{2:14} Lit. \fbib{It}} does not do any good but only destroys those who are listening. \v{15}Do your best to present yourself to God as an approved worker who has nothing to be ashamed of, handling the word of truth with precision. \v{16}However, avoid pointless discussions. For people will become more and more ungodly, \v{17}and what they say will spread everywhere like gangrene. Hymenaeus and Philetus are like that. \v{18}They have abandoned the truth by claiming that the resurrection has already taken place, and so they destroy the faith of others.

\v{19}However, God's solid foundation still stands. It has this inscription on it: ``The Lord\fnote{\fbackref{:19} MT source citation reads \fbib{}\divine{Lord}} knows those who belong to him,''\fnote{\fbackref{2:19} Cf. Num 16:5} and ``Everyone who calls on the name of the Lord\fnote{\fbackref{2:19} MT source citation reads \fbib{}\divine{Lord}} must turn away from evil.''\fnote{\fbackref{2:19} Cf. Num 16:26} \v{20}In a large house there are not only utensils made of gold and silver, but also those made of wood and clay. Some are for special use, while others are for ordinary use. \v{21}Therefore, if anyone stops associating with\fnote{\fbackref{2:21} Lit. \fbib{cleanses himself from}} these people, he will become a special utensil, set apart for the owner's use, prepared for every good action.

\v{22}Flee from youthful passions. Instead, pursue righteousness, faithfulness,\fnote{\fbackref{2:22} Or \fbib{faith}} love, and peace together with those who call on the Lord with a pure heart. \v{23}Do not have anything to do with foolish and stupid discussions, because you know they breed arguments. \v{24}A servant\fnote{\fbackref{2:24} Or \fbib{slave}} of the Lord must not argue. Instead, he must be kind to everyone, teachable,\fnote{\fbackref{2:24} Or \fbib{able to teach}} willing to suffer wrong, \v{25}and gentle when refuting opponents. After all, maybe God will allow them to repent and to come to a full knowledge of the truth, \v{26}so that they might escape from the devil's snare, even though they've been held captive by him to do his will.
\labelchapt{3}
\passage{What People will be Like in the Last Days}

\chapt{3}
\v{1}You must realize, however, that in the last days difficult times will come. \v{2}People will be lovers of themselves, lovers of money, boastful, arrogant, abusive, disobedient to their parents, ungrateful, unholy, \v{3}unfeeling, uncooperative, slanderous, degenerate, brutal, hateful of what is good, \v{4}traitors, reckless, conceited, and lovers of pleasure rather than lovers of God. \v{5}They will hold to an outward form of godliness but deny its power. Stay away from such people. \v{6}For some of these men go into homes and deceive foolish women who are burdened with sins and swayed by all kinds of desires. \v{7}These women are always studying but are never able to arrive at a full knowledge of the truth. \v{8}Just as Jannes and Jambres opposed Moses, so these men oppose the truth. They are depraved in mind and their faith is a counterfeit. \v{9}But they will not get very far because, as in the case of those two men,\fnote{\fbackref{3:9} The Gk. lacks \fbib{two men}} their stupidity will be plain to everyone.
\passage{Continue to Teach the Truth}

\v{10}But you have observed my teaching, my way of life, my purpose, my faith, my patience, my love, my endurance, \v{11}and how I was persecuted and suffered in Antioch, Iconium, and Lystra. What persecutions I endured! Yet the Lord rescued me from all of them. \v{12}Indeed, all who want to live a godly life in union with the Messiah\fnote{\fbackref{3:12} Or \fbib{Christ}} Jesus will be persecuted. \v{13}But evil people and impostors will go from bad to worse as they deceive others and are themselves deceived.

\v{14}But as for you, continue in what you have learned and found to be true, because you know from whom you learned it. \v{15}From infancy you have known the Holy Scriptures that are able to give you the wisdom you need for salvation through faith in the Messiah\fnote{\fbackref{3:15} Or \fbib{Christ}} Jesus. \v{16}All Scripture is God-breathed\fnote{\fbackref{3:16} Or \fbib{inspired by God}} and is useful for teaching, for reproof, for correction, and for training in righteousness, \v{17}so that the man of God may be complete, thoroughly equipped for every good action.
\labelchapt{4}
\passage{Complete the Task Entrusted to You}

\chapt{4}
\v{1}In the presence of God and the Messiah\fnote{\fbackref{4:1} Or \fbib{Christ}} Jesus, who is going to judge those who are living and those who are dead, and in view of his appearing and his kingdom, I solemnly appeal to you \v{2}to proclaim the message. Be ready to do this\fnote{\fbackref{4:2} The Gk. lacks \fbib{to do this}} whether or not the time is convenient. Refute, warn, and encourage with the utmost patience when you teach. \v{3}For the time will come when people will not tolerate healthy doctrine, but with itching ears will surround themselves with teachers who cater to their people's own desires. \v{4}They will refuse to listen to the truth and will turn to myths. \v{5}But you must be clear-headed about everything. Endure suffering. Do the work of an evangelist. Devote yourself completely to your ministry.

\v{6}I am already being poured out as an offering, and the time for my departure has come. \v{7}I have fought the good fight. I have completed the race. I have kept the faith. \v{8}The victor's crown of righteousness is now waiting for me, which the Lord, the righteous Judge, will give to me on the day that he comes,\fnote{\fbackref{4:8} Lit. \fbib{on that day}} and not only to me but also to all who eagerly wait for his appearing.
\passage{Final Instructions to Timothy}

\v{9}Do your best to come to me soon, \v{10}because Demas, having fallen in love with this present world, has abandoned me and has gone to Thessalonica. Crescens has gone to Galatia, and Titus to Dalmatia. \v{11}Only Luke is with me. Get Mark and bring him with you, for he is useful in my ministry. \v{12}I have sent Tychicus to Ephesus.

\v{13}When you come, bring the coat I left with Carpus in Troas, as well as the scrolls and especially the parchments.\fnote{\fbackref{4:13} Parchments were writing materials made from animal skins.} \v{14}Alexander the metalworker did me a great deal of harm. The Lord will pay him back for what he did. \v{15}You, too, must watch out for him, for he violently opposed our message.

\v{16}At my first trial no one came to my defense. Everyone abandoned me. May it not be held against them! \v{17}However, the Lord stood by me and gave me strength so that through me the message might be fully proclaimed and all the gentiles could hear it. I was rescued out of a lion's mouth. \v{18}The Lord will rescue me from every evil attack\fnote{\fbackref{4:18} Lit. \fbib{evil work}} and will take me safely to\fnote{\fbackref{4:18} Or \fbib{will preserve me for}} his heavenly kingdom. Glory belongs to him forever and ever! Amen.
\passage{Final Greeting}

\v{19}Greet Prisca\fnote{\fbackref{4:19} I.e. Priscilla} and Aquila and the family of Onesiphorus. \v{20}Erastus stayed in Corinth, and I left Trophimus in Miletus because he was sick. \v{21}Do your best to come to me before winter. Eubulus sends you greetings, as do Pudens, Linus, Claudia, and all the brothers.

\v{22}May the Lord be with your spirit. Grace be with all of you! Amen.\fnote{\fbackref{4:22} Other mss. lack \fbib{Amen}}

\bookheader{Titus}
\labelbook{Titus}

\bookpretitle{The Letter of Paul to}
\booktitle{Titus}

\labelchapt{1}
\passage{Greetings}

\chapt{1}
\v{1}From:\fnote{\fbackref{1:1} The Gk. lacks \fbib{From}} Paul, a servant of God, and also an apostle of Jesus the Messiah,\fnote{\fbackref{1:1} Or \fbib{Christ}} to bring the faith to those chosen by God, along with full knowledge of the truth that leads to\fnote{\fbackref{1:1} Lit. \fbib{that is according to}} godliness, \v{2}which is based on the hope of eternal life that God, who cannot lie, promised before the world\fnote{\fbackref{1:2} Or \fbib{the ages}} began. \v{3}At the right time he revealed his message through the proclamation that was entrusted to me by the command of God our Savior.

\v{4}To: Titus, a genuine child in the faith that we share.

May grace and peace\fnote{\fbackref{1:4} Other mss. read \fbib{grace, mercy, and peace}} from God the Father and the Messiah,\fnote{\fbackref{1:4} Or \fbib{Christ}} Jesus our Savior, be yours!
\passage{Qualifications for Leaders in the Church}

\v{5}The reason I left you in Crete was to complete what still needed to be done and to appoint elders in every city, as I myself commanded you. \v{6}An elder must be\fnote{\fbackref{1:6} Lit. \fbib{If anyone is}} blameless. He must be the husband of one wife\fnote{\fbackref{1:6} Or \fbib{devoted to his wife}; lit. \fbib{a man of one woman}} and have children who are believers and who are not accused of having wild lifestyles or of being rebellious. \v{7}Because an overseer is God's servant manager, he must be blameless. He must not be arrogant or irritable. He must not drink too much, be a violent person, or make money in shameful ways. \v{8}Instead, he must be hospitable to strangers, must appreciate what is good, and be sensible, honest, moral, and self-controlled. \v{9}He must be devoted to the trustworthy message that agrees with what we teach, so that he may be able to encourage others with healthy doctrine and refute those who oppose it.
\passage{Guard What is True}

\v{10}For there are many people who are rebellious, especially those who are converts from Judaism.\fnote{\fbackref{1:10} Lit. \fbib{those of the circumcision}} They speak utter nonsense and deceive people. \v{11}They must be silenced, because they are the kind of people who ruin whole families by teaching what they should not teach in order to make money in a shameful way. \v{12}One of their very own prophets said,

\begin{poetry}
\poeml ``Liars ever, men of Crete, \\
\poemll    savage brutes that live to eat.''\fnote{\fbackref{1:12} Epimenides (6\textsuperscript{th} to 5\textsuperscript{th} century BC)}
\end{poetry}

\v{13}That statement is true. For this reason, refute them sharply so that they may become healthy in the faith \v{14}and not pay attention to Jewish myths or commands given by people who reject the truth. \v{15}Everything is clean to those who are clean, but nothing is clean to those who are corrupt and unbelieving. Indeed, their very way of thinking and their consciences have been corrupted. \v{16}They claim to know God, but they deny him by their actions. They are detestable, disobedient, and disqualified to do anything good.
\labelchapt{2}
\passage{Guidelines for Christian Living}

\chapt{2}
\v{1}But as for you, teach what is consistent with healthy doctrine. \v{2}Older men are to be sober, serious, sensible, and sound in faith, love, and endurance. \v{3}Likewise, older women are to show their reverence for God by their behavior. They are not to be gossips or addicted to alcohol, but to be examples\fnote{\fbackref{2:3} Or \fbib{teachers}} of goodness. \v{4}They should encourage the younger women to love their husbands, to love their children, \v{5}to be sensible and pure, to manage their households, to be kind, and to submit themselves to their husbands. Otherwise, the word of God may be discredited.\fnote{\fbackref{2:5} Or \fbib{blasphemed}}

\v{6}Likewise, encourage the younger men to be sensible. \v{7}Always set an example for others by doing good actions. Teach with integrity and dignity. \v{8}Use wholesome speech that cannot be condemned. Then any opponent will be ashamed because he cannot say anything bad about us.

\v{9}Slaves are to submit to their masters in everything, aiming to please them and not argue with them \v{10}or steal from them. Instead, they are to show complete and perfect loyalty, so that in every way they may make the teaching about God our Savior more attractive.

\v{11}For the grace of God has appeared, bringing salvation to all people. \v{12}It trains us to renounce ungodly living and worldly passions so that we might live sensible, honest, and godly lives in the present age \v{13}as we wait for the blessed hope and glorious appearance of our great God and Savior, Jesus the Messiah.\fnote{\fbackref{2:13} Or \fbib{Christ}} \v{14}He gave himself for us to set us free from every wrong and to cleanse us so that we could be his special people who are enthusiastic about doing good deeds.

\v{15}These are the things you should teach. Encourage and refute with full authority. Do not let anyone look down on you.
\labelchapt{3}
\passage{Concentrate on Doing What is Good}

\chapt{3}
\v{1}Remind believers\fnote{\fbackref{3:1} Lit. \fbib{them}} to submit to rulers and authorities, to be obedient, and to be ready to do any honorable kind of work. \v{2}They are not to insult\fnote{\fbackref{3:2} Or \fbib{slander}} anyone or be argumentative. Instead, they are to be gentle and perfectly courteous to everyone. \v{3}After all, we ourselves were once foolish, disobedient, and misled. We were slaves to many kinds of lusts and pleasures, spending our days in malice and jealousy. We were despised, and we hated one another.

\begin{poetry}
\poeml \v{4}In grace our Savior God appeared, \\
\poemll    to make his love for mankind clear. \\
\poeml \v{5}`Twas not for deeds that we had done, \\
\poemll    but by his steadfast love\fnote{\fbackref{3:5} Or \fbib{his mercy}} alone, \\
\poeml he saved us through a second birth, \\
\poemll    renewed us by the Spirit's\fnote{\fbackref{3:5} Lit. \fbib{the Holy Spirit's}} work, \\
\poeml \v{6}and poured him out upon us, too, \\
\poemll    through Jesus the Messiah\fnote{\fbackref{3:6} Or \fbib{Christ}} our Savior true. \\
\poeml \v{7}And so, made heirs by his own grace, \\
\poemll    eternal life we now embrace.\fnote{\fbackref{3:7} Lit. \fbib{we have become heirs according to the hope of eternal life}}
\end{poetry}

\v{8}This saying is trustworthy.\fnote{\fbackref{3:8} This formula accompanied early Christian sayings on which full reliance could be placed.} I want you to insist on these things, so that those who have put their faith in God may devote themselves to good actions. These things are good and helpful to other people.

\v{9}But avoid foolish controversies, arguments about genealogies, quarrels, and fights about the Law. These things are useless and worthless. \v{10}Have nothing to do with a divisive person after you have warned him once or twice. \v{11}For you know that a person like this is corrupt and keeps on sinning, being self-condemned.
\passage{Final Instructions to Titus}

\v{12}As soon as I send Artemas to you, or perhaps Tychicus, do your best to come to me at Nicopolis, for I have decided to spend the winter there. \v{13}Do all you can to send Zenas the expert in the Law and Apollos on their way, and see that they have everything they need. \v{14}Our own people should also learn to make good deeds a priority when urgent needs arise, so they won't be unproductive.
\passage{Final Greeting}

\v{15}All who are with me send you greetings. Greet our fellow believers who love us. May grace be with all of you! Amen.\fnote{\fbackref{3:15} Other mss. lack \fbib{Amen}}

\bookheader{Philemon}
\labelbook{Phlm}

\bookpretitle{The Letter of Paul to}
\booktitle{Philemon}

\passage{Greetings}

\v{1}From:\fnote{\fbackref{1} The Gk. lacks \fbib{From}} Paul, a prisoner of the Messiah\fnote{\fbackref{1} Or \fbib{Christ}} Jesus, and Timothy our brother.

To: Philemon our dear friend\fnote{\fbackref{1} Or \fbib{our beloved Philemon}} and fellow worker, \v{2}to Apphia our sister, to Archippus our fellow soldier, and to the church in your house.

\v{3}May grace and peace from God our Father and the Lord Jesus, the Messiah,\fnote{\fbackref{3} Or \fbib{Christ}} be yours!\fnote{\fbackref{3} The Gk. \fbib{yours} is pl.}
\passage{Paul's Prayer for Philemon}

\v{4}I always thank my God when I mention you\fnote{\fbackref{4} From verse 4 through v. 21, \fbib{you} and \fbib{your} are sing.} in my prayers, \v{5}because I keep hearing about your love for all the saints and the faith that you have in the Lord Jesus. \v{6}I pray\fnote{\fbackref{6} The Gk. lacks \fbib{I pray}} that your partnership in the faith may become effective as you fully acknowledge every blessing that is ours\fnote{\fbackref{6} Other mss. read \fbib{yours} (pl.)} in the Messiah.\fnote{\fbackref{6} Or \fbib{Christ}} \v{7}For I have received considerable joy and encouragement from your love, because the hearts of the saints have been refreshed, brother, through you.
\passage{Paul's Plea for Onesimus}

\v{8}For this reason, although in the Messiah\fnote{\fbackref{8} Or \fbib{Christ}} I have complete freedom to order you to do what is proper, \v{9}I prefer to make my appeal on the basis of love. I, Paul, as an old man and now a prisoner of the Messiah\fnote{\fbackref{9} Or \fbib{Christ}} Jesus, \v{10}appeal to you on behalf of my child Onesimus, whose father I have become during my imprisonment. \v{11}Once he was useless to you, but now he is very useful\fnote{\fbackref{11} The Gk. name \fbib{Onesimus} means \fbib{useful}} both to you and to me. \v{12}As I send him back, it's like I'm coming along with him.\fnote{\fbackref{12} Lit. \fbib{back, it's with my innards}} \v{13}I wanted to keep him with me so that he could serve me in your place during my imprisonment for the gospel. \v{14}Yet I did not want to do anything without your consent, so that your good deed might not be something forced, but voluntary. \v{15}Perhaps this is why he was separated from you for a while, so that you could have him back forever, \v{16}no longer as a slave but better than a slave---as a dear brother, especially to me, but even more so to you, both as a person and as a believer.\fnote{\fbackref{16} Or \fbib{both in the flesh and in the Lord}}

\v{17}So if you consider me a partner, welcome him as you would welcome\fnote{\fbackref{17} The Gk. lacks \fbib{you would welcome}} me. \v{18}If he has wronged you in any way or owes you anything, charge it to my account. \v{19}I, Paul, am writing this with my own hand: I will repay it. (I will not mention to you that you owe me your very life.) \v{20}Yes, brother, I desire this favor from you in the Lord. Refresh my heart in the Messiah!\fnote{\fbackref{20} Or \fbib{Christ}} \v{21}Confident of your obedience, I am writing to you because I know that you will do even more than I ask. \v{22}Meanwhile, prepare a guest room for me, too, for I am hoping through your prayers to be returned to you.
\passage{Greetings from Paul's Fellow Workers}

\v{23}Epaphras, my fellow prisoner in the Messiah\fnote{\fbackref{23} Or \fbib{Christ}} Jesus, sends you\fnote{\fbackref{23} The Gk. \fbib{you} is sing.} greetings, \v{24}as do Mark, Aristarchus, Demas, and Luke, my fellow workers. \v{25}May the grace of our\fnote{\fbackref{25} Other mss. read \fbib{the}} Lord Jesus, the Messiah,\fnote{\fbackref{25} Or \fbib{Christ}} be with your spirit! Amen.\fnote{\fbackref{25} Other mss. lack \fbib{Amen.}}

\addcontentsline{toc}{chapter}{The Application of the Good News}
\bookheader{Hebrews}
\labelbook{Heb}

\bookpretitle{The Letter to the}
\booktitle{Hebrews}

\labelchapt{1}
\passage{God Has Spoken to Us}

\chapt{1}
\v{1}God, having spoken in former times in fragmentary and varied fashion to our forefathers by the prophets, \v{2}has in these last days spoken to us by a Son whom he appointed to be the heir of everything and through whom he also made the universe. \v{3}He is the reflection\fnote{\fbackref{1:3} Or \fbib{radiance}} of God's glory and the exact likeness of his being, and he holds everything together by his powerful word. After he had provided a cleansing from sins, he sat down at the right hand of the Highest Majesty \v{4}and became as much superior to the angels as the name he has inherited is better than theirs.
\passage{God's Son is Superior to the Angels}

\v{5}For to which of the angels did God\fnote{\fbackref{1:5} Lit. \fbib{he}} ever say, ``You are my Son. Today I have become your Father''?\fnote{\fbackref{1:5} Cf. Ps 2:7} Or again, ``I will be his Father, and he will be my Son''?\fnote{\fbackref{1:5} Cf. 2 Sam 7:14} \v{6}And again, when he brings\fnote{\fbackref{1:6} Or \fbib{And when he again brings}} his firstborn into the world, he says, ``Let all God's angels worship him.''\fnote{\fbackref{1:6} Deut 32:43 (LXX); Ps 97:7} \v{7}Now about the angels he says,

\begin{poetry}
\poeml ``He makes his angels winds,
\end{poetry}

and his servants flames of fire.''\fnote{\fbackref{1:7} Cf. Ps 104:4}

\v{8}But about the Son he says,

\begin{poetry}
\poeml ``Your throne, O God, \\
\poemll    is forever and ever, \\
\poeml and the scepter of your kingdom \\
\poemll    is a righteous scepter. \\
\poeml \v{9}You have loved righteousness \\
\poemll    and hated wickedness. \\
\poeml That is why God, your God, \\
\poemll    anointed you rather than your companions \\
\poemlll       with the oil of gladness.''\fnote{\fbackref{1:9} Cf. Ps 45:6-7}
\end{poetry}

\v{10}And,

\begin{poetry}
\poeml ``In the beginning, Lord,\fnote{\fbackref{1:10} MT source citation reads \fbib{}\divine{Lord}} \\
\poemll    you laid the foundation of the earth, \\
\poemlll       and the heavens are the work of your hands. \\
\poeml \v{11}They will come to an end, \\
\poemll    but you will remain forever. \\
\poemlll       They will all wear out like clothes. \\
\poeml \v{12}You will roll them up like a robe, \\
\poemll    and they will be changed like clothes. \\
\poeml But you remain the same, \\
\poemll    and your life\fnote{\fbackref{1:12} Lit. \fbib{years}} will never end.''\fnote{\fbackref{1:12} Cf. Ps 102:25-27}
\end{poetry}

\v{13}But to which of the angels did he ever say,

\begin{poetry}
\poeml ``Sit at my right hand \\
\poemll    until I make your enemies a footstool for your feet''?\fnote{\fbackref{1:13} Cf. Ps 110:1}
\end{poetry}

\v{14}All of them are spirits on a divine mission, sent to serve those who are about to inherit salvation, aren't they?
\labelchapt{2}
\passage{We Must Not Neglect Our Salvation}

\chapt{2}
\v{1}For this reason we must pay closer attention to the things we have heard, or we may drift away, \v{2}because if the message spoken by angels was reliable, and every violation and act of disobedience received its just punishment, \v{3}how will we escape if we neglect a salvation as great as this? It was first proclaimed by the Lord himself, and then it was confirmed to us by those who heard him, \v{4}while God added his testimony through signs, wonders, various miracles, and gifts of the Holy Spirit distributed according to his will.
\passage{Jesus is the Source of Our Salvation}

\v{5}For he did not put the coming world we are talking about under the control of angels. \v{6}Instead, someone has declared somewhere,

\begin{poetry}
\poeml ``What is man that you should remember him, \\
\poemll    or the son of man that you should care for him? \\
\poeml \v{7}You made him a little lower than the angels, \\
\poemll    yet you crowned him with glory and honor \\
\poeml \v{8}and put everything under his feet.''\fnote{\fbackref{2:8} Cf. Ps 8:5-7 (LXX)}
\end{poetry}

Now when God\fnote{\fbackref{2:8} Lit. \fbib{he}} put everything under him, he left nothing outside his control. However, at the present time we do not yet see everything put under him. \v{9}But we do see someone who was made a little lower than the angels. He is Jesus, who is crowned with glory and honor because he suffered death, so that by the grace of\fnote{\fbackref{2:9} Other mss. read \fbib{so that apart from}} God he might experience\fnote{\fbackref{2:9} Lit. \fbib{taste}} death for everyone.

\v{10}It was fitting that God, for whom and through whom everything exists, should make the pioneer of their salvation perfect through suffering as part of his plan to glorify many children, \v{11}because both the one who sanctifies and those who are being sanctified all have the same Father.\fnote{\fbackref{2:11} Lit. \fbib{are all of one}} That is why Jesus\fnote{\fbackref{2:11} Lit. \fbib{he}} is not ashamed to call them brothers \v{12}when he says, ``I will announce your name to my brothers. I will praise you within the congregation.''\fnote{\fbackref{2:12} Cf. Ps 22:22} \v{13}And again, ``I will trust him.''\fnote{\fbackref{2:13} Cf. Isa 8:17 (LXX)} And again, ``I am here with the children God has given me.''\fnote{\fbackref{2:13} Cf. Isa 8:18}

\v{14}Therefore, since the children have flesh and blood, he himself also shared the same things, so that by his death he might destroy the one who has the power of death (that is, the devil) \v{15}and might free those who were slaves all their lives because they were terrified by death. \v{16}For it is clear that he did not come to help angels. No, he came to help Abraham's descendants, \v{17}thereby becoming like his brothers in every way, so that he could be a merciful and faithful high priest in service to God and could atone for the people's sins. \v{18}Because he himself suffered when he was tempted, he is able to help those who are being tempted.
\labelchapt{3}
\passage{The Messiah is Superior to Moses}

\chapt{3}
\v{1}Therefore, holy brothers, partners in a heavenly calling, keep your focus on Jesus, the apostle and high priest of our confession. \v{2}He was faithful to the one who appointed him, just as Moses was in all God's\fnote{\fbackref{3:2} Lit. \fbib{his}} household, \v{3}because he is worthy of greater glory than Moses in the same way that the builder of a house has greater honor than the house itself. \v{4}After all, every house is built by someone, but God is the builder of everything. \v{5}Moses was faithful in all God's\fnote{\fbackref{3:5} Lit. \fbib{his}} household as a servant who was to testify to what would be said later, \v{6}but the Messiah\fnote{\fbackref{3:6} Or \fbib{Christ}} was faithful\fnote{\fbackref{3:6} The Gk. lacks \fbib{was faithful}} as the Son in charge of God's\fnote{\fbackref{3:6} Lit. \fbib{his}} household, and we are his household if we hold on to our courage and the hope in which we rejoice.\fnote{\fbackref{3:6} Lit. \fbib{the boast of our hope}}
\passage{A Rest for the People of God}

\v{7}Therefore, as the Holy Spirit says,

\begin{poetry}
\poeml ``Today, if you hear his voice, \\
\poeml \v{8}do not harden your hearts \\
\poeml as they did when they provoked me \\
\poemll    during the time of testing in the wilderness. \\
\poeml \v{9}There your ancestors tested me, \\
\poemll    even though they had seen my actions \v{10}for 40 years. \\
\poeml That is why I was indignant with that generation and said, \\
\poemll    `They are always going astray in their hearts, \\
\poemlll       and they have not known my ways.' \\
\poeml \v{11}So in my anger I swore a solemn oath \\
\poemll    that they would never enter my rest.''\fnote{\fbackref{3:11} Cf. Ps 95:7-11}
\end{poetry}

\v{12}See to it, my brothers, that no evil, unbelieving heart is found in any of you, as shown by your turning away from the living God. \v{13}Instead, continue to encourage one another every day, as long as it is called ``Today,'' so that none of you may be hardened by the deceitfulness of sin, \v{14}because we are the Messiah's\fnote{\fbackref{3:14} Or \fbib{Christ's}} partners only if we hold on to our original confidence to the end.\fnote{\fbackref{3:14} Other mss. lack \fbib{to the end}} \v{15}As it is said,

\begin{poetry}
\poeml ``Today, if you hear his voice, \\
\poemll    do not harden your hearts \\
\poemlll       as they did when they provoked me.''\fnote{\fbackref{3:15} Cf. Ps 95:7-8}
\end{poetry}

\v{16}Now who heard him and provoked him? Was it not all those who came out of Egypt led\fnote{\fbackref{3:16} The Gk. lacks \fbib{led}} by Moses? \v{17}And with whom was he angry for 40 years? Was it not with those who sinned and whose bodies fell dead in the wilderness? \v{18}And to whom did he swear that they would never enter his rest? It was to those who disobeyed him, was it not? \v{19}So we see that they were unable to enter because of their unbelief.
\labelchapt{4}
\passage{We Must Enter the Rest}

\chapt{4}
\v{1}Therefore, as long as the promise of entering his rest remains valid, let us be afraid! Otherwise, some of you will fail\fnote{\fbackref{4:1} Lit. \fbib{afraid lest someone among you fails}} to reach it, \v{2}because we have had the good news told to us as well as to them. But the message they heard did not help them, because they were not united by faith with those who listened to it. \v{3}We who have believed are entering that rest, just as he has said,

\begin{poetry}
\poeml ``So in my anger I swore a solemn oath \\
\poemll    that they would never enter my rest,''\fnote{\fbackref{4:3} Cf. Ps 95:11}
\end{poetry}

even though his actions had been finished since the creation\fnote{\fbackref{4:3} Lit. \fbib{foundation}; or \fbib{beginning}} of the world. \v{4}Somewhere he has spoken about the seventh day as follows: ``On the seventh day God rested from all his actions,''\fnote{\fbackref{4:4} Cf. Gen 2:2} \v{5}and again in this passage,\fnote{\fbackref{4:5} The Gk. lacks \fbib{passage}} ``They will never enter my rest.''\fnote{\fbackref{4:5} Cf. Ps 95:11} \v{6}Therefore, since it is still true that some will enter it, and since those who once heard the good news failed to enter it because of their disobedience, \v{7}he again fixes a definite day---``Today''---saying long afterward through David, as already quoted,

\begin{poetry}
\poeml ``Today, if you hear his voice, \\
\poemll    do not harden your hearts.''\fnote{\fbackref{4:7} Ps 95:7-8}
\end{poetry}

\v{8}For if Joshua\fnote{\fbackref{4:8} The Gk. name \fbib{Jesus} appears to be a word play on the Heb. name \fbib{Joshua}.} had given them rest, he would not have spoken later about another day.

\v{9}There remains, therefore, a Sabbath rest for the people of God to keep, \v{10}because the one who enters God's\fnote{\fbackref{4:10} Lit. \fbib{his}} rest has himself rested from his own actions, just as God did\fnote{\fbackref{4:10} The Gk. lacks \fbib{did}} from his. \v{11}Let us, therefore, make every effort to enter that rest, so that no one may fail by following their example of disobedience. \v{12}For the word of God is living and active, sharper than any double-edged sword, piercing until it divides soul and spirit, joints and marrow, as it judges the thoughts and purposes of the heart. \v{13}No creature can hide from him, but everyone is exposed and helpless before the eyes of the one to whom we must give a word of explanation.
\passage{Our Compassionate High Priest}

\v{14}Therefore, since we have a great high priest who has gone to heaven, Jesus the Son of God, let us live our lives consistent with\fnote{\fbackref{4:14} Lit. \fbib{us hold tightly to}} our confession of faith.\fnote{\fbackref{4:14} The Heb. lacks \fbib{of faith}} \v{15}For we do not have a high priest who is unable to sympathize with our weaknesses. Instead, we have one who in every respect has been tempted as we are, yet he never sinned. \v{16}So let us keep on coming boldly to the throne of grace, so that we may obtain mercy and find grace to help us in our time of need.
\labelchapt{5}
\passage{Qualifications for the Priesthood}

\chapt{5}
\v{1}For every high priest selected from among men is appointed to officiate on their behalf\fnote{\fbackref{5:1} Lit. \fbib{on behalf of men}} in matters relating to God, that is, to offer gifts and sacrifices for sins. \v{2}He can deal gently with people who are ignorant and easily deceived, since he himself is subject to weakness. \v{3}For that reason he is obligated to offer sacrifices for his own sins as well as for those of the people. \v{4}No one takes this honor upon himself but he is called to it by God, just as Aaron was.
\passage{The Messiah's Qualifications as High Priest}

\v{5}In the same way, the Messiah\fnote{\fbackref{5:5} Or \fbib{Christ}} did not take upon himself the glory of being a high priest. No, it was God who said\fnote{\fbackref{5:5} Lit. \fbib{He said}} to him,

\begin{poetry}
\poeml ``You are my Son. \\
\poemll    Today I have become your Father.''\fnote{\fbackref{5:5} Cf. Ps 2:7}
\end{poetry}

\v{6}As he also says in another place,

\begin{poetry}
\poeml ``You are a priest forever \\
\poemll    according to the order of Melchizedek.''\fnote{\fbackref{5:6} Cf. Ps 110:4}
\end{poetry}

\v{7}As a mortal man,\fnote{\fbackref{5:7} Lit. \fbib{During the days of his flesh}} he offered up prayers and appeals with loud cries and tears to the one who was able to save him from death, and he was heard because of his devotion to God. \v{8}Son though he was, he learned obedience through his sufferings \v{9}and, once made perfect, he became the source of eternal salvation for all who obey him, \v{10}having been designated by God to be a high priest according to the order of Melchizedek.
\passage{You Still Need Someone to Teach You}

\v{11}We have much to say about this,\fnote{\fbackref{5:11} Or \fbib{about him}} but it is difficult to explain because you have become too lazy to understand. \v{12}In fact, though by now you should be teachers, you still need someone to teach you the basic truths of God's word.\fnote{\fbackref{5:12} Or \fbib{oracles}} You have become people who need milk instead of solid food. \v{13}For everyone who lives on milk is still a baby and does not yet know the difference between right and wrong.\fnote{\fbackref{5:13} Lit. \fbib{and is inexperienced in the message of righteousness}} \v{14}But solid food is for mature people, whose minds are trained by practice to distinguish good from evil.
\labelchapt{6}
\passage{The Peril of Immaturity}

\chapt{6}
\v{1}Therefore, leaving behind the elementary teachings about the Messiah,\fnote{\fbackref{6:1} Or \fbib{Christ}} let us continue to be carried along to maturity, not laying again a foundation of repentance from dead actions, faith toward God, \v{2}instruction about baptisms, the laying on of hands, the resurrection of the dead, and eternal judgment. \v{3}And this we will do,\fnote{\fbackref{6:3} Other mss. read \fbib{Let us do this}} if God permits.

\v{4}For it is impossible to keep on restoring to repentance time and again people who have once been enlightened, who have tasted the heavenly gift, who have become partners with the Holy Spirit, \v{5}who have tasted the goodness of God's word and the powers of the coming age, \v{6}and who have fallen away, as long as they continue to crucify the Son of God to their own detriment by exposing him to public ridicule. \v{7}For when the ground soaks up rain that often falls on it and continues producing vegetation useful to those for whom it is cultivated, it receives a blessing from God. \v{8}However, if it continues to produce thorns and thistles, it is worthless and in danger of being cursed, and in the end will be burned.
\passage{Be Diligent}

\v{9}Even though we speak like this, dear friends, we are convinced of better things in your case, things that point to salvation. \v{10}For God is not so unjust as to forget your work and the love you have shown him\fnote{\fbackref{6:10} Lit. \fbib{shown for his name}} as you have ministered to the saints and continue to minister to them. \v{11}But we want each of you to continue to be diligent to the very end, in order to give full assurance to your hope. \v{12}Then, instead of being lazy, you will imitate those who are inheriting the promises through faith and patience.
\passage{God's Promise is Reliable}

\v{13}For when God made his promise to Abraham, he swore an oath by himself, since he had no one greater to swear by. \v{14}He said, ``I will certainly bless you and give you many descendants.''\fnote{\fbackref{6:14} Cf. Gen 22:17} \v{15}And so he obtained what he had been promised, because he patiently waited for it. \v{16}For people swear by someone greater than themselves, and an oath given as confirmation puts an end to all argument. \v{17}In the same way, when God wanted to make the unchangeable character of his purpose perfectly clear to the heirs of his promise, he guaranteed it with an oath, \v{18}so that by these two unchangeable things, in which it is impossible for God to prove false, we who have taken refuge in him might be encouraged to seize the hope set before us. \v{19}That hope,\fnote{\fbackref{6:19} The Gk. lacks \fbib{hope}} firm and secure like an anchor for our souls, reaches behind the curtain \v{20}where Jesus, our forerunner, has gone on our behalf, having become a high priest forever according to the order of Melchizedek.
\labelchapt{7}
\passage{The Messiah is Superior to Melchizedek}

\chapt{7}
\v{1}Now this man Melchizedek, king of Salem and priest of the Most High God, met Abraham and blessed him when he was returning from defeating the kings. \v{2}Abraham gave Melchizedek\fnote{\fbackref{7:2} Lit. \fbib{him}} a tenth of everything.\fnote{\fbackref{7:2} Cf. Gen 14:18-20} In the first place, his name means ``king of righteousness,'' and then he is also king of Salem, that is, ``king of peace.'' \v{3}He has no father, mother, or genealogy, no birth date recorded for him, nor a date of death.\fnote{\fbackref{7:3} Lit. \fbib{had neither beginning of days nor end of life}} Like the Son of God, he continues to be a priest forever.

\v{4}Just look at how great this man was! Even Abraham---the patriarch himself---gave him a tenth of what he had captured! \v{5}The descendants of Levi who accept the priesthood have a commandment in the Law to collect a tenth from the people, that is, from their own brothers, even though they are also descendants of Abraham. \v{6}But this man, whose descent is not traced from them, collected a tenth from Abraham and blessed the man who had received the promises. \v{7}It is beyond dispute that the less important person is blessed by the more important person. \v{8}Mortal men collect tithes, but we are informed by Scripture\fnote{\fbackref{7:8} The Gk. lacks \fbib{by Scripture}} that\fnote{\fbackref{7:8} Or \fbib{it is declared that}} Melchizedek\fnote{\fbackref{7:8} Lit. \fbib{he}} keeps on living. \v{9}One might even say that Levi, who collects the tenth, paid the tenth through Abraham, \v{10}because Levi\fnote{\fbackref{7:10} Lit. \fbib{he}} was still inside his ancestor when Melchizedek met him.

\v{11}Now if perfection could have been attained through the Levitical priesthood---for on this basis the people received the Law---what further need would there be to speak of appointing another kind of priest according to the order of Melchizedek, not one according to the order of Aaron? \v{12}When a change in the priesthood takes place, there must also be a change in the Law. \v{13}For the person we are talking about belonged to a different tribe, and no one from that tribe has ever served\fnote{\fbackref{7:13} Lit. \fbib{from which no one has served}} at the altar. \v{14}Furthermore, it is obvious that our Lord was a descendant of Judah, and Moses said nothing about priests coming from that tribe. \v{15}This point is even more obvious in that another priest who is like Melchizedek has appeared \v{16}who was appointed to be a priest,\fnote{\fbackref{7:16} The Gk. lacks \fbib{to be a priest}} not on the basis of a genealogical registry, but rather on the power of an indestructible life. \v{17}For it is declared about him,

\begin{poetry}
\poeml ``You are a priest forever \\
\poemll    according to the order of Melchizedek.''\fnote{\fbackref{7:17} Cf. Ps 110:4}
\end{poetry}

\v{18}Indeed, because it was weak and ineffective, the former commandment has been annulled, \v{19}since the Law made nothing perfect, and a better hope is presented, by which we approach God.

\v{20}Now none of this happened without an oath. Others became priests without any oath, \v{21}but Jesus\fnote{\fbackref{7:21} Lit. \fbib{he}} became a priest\fnote{\fbackref{7:21} The Gk. lacks \fbib{became a priest}} with an oath when God\fnote{\fbackref{7:21} Lit. \fbib{he}} told him,

\begin{poetry}
\poeml ``The Lord\fnote{\fbackref{7:21} MT source citation reads \fbib{}\divine{Lord}} has taken an oath \\
\poemll    and will not change his mind. \\
\poeml You are a priest forever.''\fnote{\fbackref{7:21} Cf. Ps 110:4}
\end{poetry}

\v{22}In this way, Jesus has become the guarantor of a better covenant.

\v{23}There have been many priests, since each one of them had to stop serving in office when he died. \v{24}But because Jesus\fnote{\fbackref{7:24} Lit. \fbib{he}} lives forever, he has a permanent priesthood. \v{25}Therefore, because he always lives to intercede for them, he is able to save completely\fnote{\fbackref{7:25} Or \fbib{thoroughly}} those who come to God through him.

\v{26}We need such a high priest---one who is holy, innocent, pure, set apart from sinners, exalted above the heavens. \v{27}He has no need to offer sacrifices every day like high priests do, first for his own sins and then for those of the people, since he did this once for all when he sacrificed himself. \v{28}For the Law appoints as high priests men who are weak, but the promised oath, which came after the Law, results in a Son who is eternally perfect.
\labelchapt{8}
\passage{The Messiah Has a Better Ministry}

\chapt{8}
\v{1}Now the main point in what we are saying is this: we do have this kind of high priest, who sat down at the right hand of the throne of the Majesty in heaven \v{2}and who serves in the sanctuary, the true tent set up by the Lord and not by any human. \v{3}For every high priest is appointed to offer both gifts and sacrifices. Therefore, this high priest\fnote{\fbackref{8:3} Lit. \fbib{this one}} had to offer something, too. \v{4}Now if he were on earth, he would not even be a priest, because other men offer the gifts prescribed by the Law. \v{5}They serve in a sanctuary that is a copy, a shadow of the heavenly one. This is why Moses was warned when he was about to build the tent: ``See to it that you make everything according to the pattern that was shown you on the mountain.''\fnote{\fbackref{8:5} Cf. Exod 25:40} \v{6}However, Jesus\fnote{\fbackref{8:6} Lit. \fbib{he}} has now obtained a more superior ministry, since the covenant he mediates is founded on better promises.
\passage{The New Covenant is Better than the Old}

\v{7}If the first covenant had been faultless, there would have been no need to look for a second one, \v{8}but God\fnote{\fbackref{8:8} Lit. \fbib{he}} found something wrong with his people\fnote{\fbackref{8:8} Lit. \fbib{with them}} when he said,

\begin{poetry}
\poeml ``Look! The days are coming, declares the Lord,\fnote{\fbackref{8:8} MT source citation reads \fbib{}\divine{Lord}} \\
\poemll    when I will establish a new covenant \\
\poemlll       with the house of Israel \\
\poemlll       and with the house of Judah. \\
\poeml \v{9}It will not be like the covenant that I made with their ancestors at the time \\
\poemll    when I took them by the hand \\
\poemlll       and brought them out of the land of Egypt. \\
\poeml Because they did not remain loyal to my covenant, \\
\poemll    I ignored them, declares the Lord.\fnote{\fbackref{8:9} MT source citation reads \fbib{}\divine{Lord}} \\
\poeml \v{10}For this is the covenant that I will make with the house of Israel \\
\poemll    after that time, declares the Lord:\fnote{\fbackref{8:10} MT source citation reads \fbib{}\divine{Lord}} \\
\poeml I will put my laws in their minds \\
\poemll    and write them on their hearts. \\
\poeml I will be their God, \\
\poemll    and they will be my people. \\
\poeml \v{11}Never again will everyone teach his neighbor \\
\poemll    or his brother by saying, `Know the Lord,'\fnote{\fbackref{8:11} MT source citation reads \fbib{}\divine{Lord}} \\
\poeml because all of them will know me, \\
\poemll    from the least important to the most important. \\
\poeml \v{12}For I will be merciful regarding their wrong deeds, \\
\poemll    and I will never again remember their sins.''\fnote{\fbackref{8:12} Cf. Jer 31:31-34}
\end{poetry}

\v{13}In speaking of a ``new'' covenant, he has made the first one obsolete, and what is obsolete and aging will soon disappear.
\labelchapt{9}
\passage{The Earthly Sanctuary and Its Ritual}

\chapt{9}
\v{1}Now even the first covenant\fnote{\fbackref{9:1} The Gk. lacks \fbib{covenant}} had regulations for worship and an earthly sanctuary. \v{2}For a tent was set up, and in the first part were the lamp stand, the table, and the bread of the Presence.\fnote{\fbackref{9:2} Lit. \fbib{the presentation of the bread}} This was called the Holy Place. \v{3}Behind the second curtain was the part of the tent called the Most Holy Place, \v{4}which had the gold altar for incense and the Ark of the Covenant completely covered with gold. In it were the gold jar holding the manna, Aaron's staff that had budded, and the Tablets of the Covenant. \v{5}Above it were the cherubim of glory overshadowing the place of atonement. (We cannot discuss these things in detail now.)

\v{6}When everything had been arranged like this, the priests always went into the first part of the tent to perform their duties. \v{7}But only the high priest went\fnote{\fbackref{9:7} The Gk. lacks \fbib{went}} into the second part, and then only once a year, and never without blood, which he offered for himself and for the sins committed by the people in ignorance. \v{8}The Holy Spirit was indicating by this that the way into the Most Holy Place had not yet been disclosed as long as the first part of the tent was still standing. \v{9}This illustration for today indicates that the gifts and sacrifices being offered could not clear the conscience of a worshiper, \v{10}since they deal only with food, drink, and various washings, which are required for the body until the time when things would be set right.
\passage{The Messiah Has Offered a Superior Sacrifice}

\v{11}But when the Messiah\fnote{\fbackref{9:11} Or \fbib{Christ}} came as a high priest of the good things that have come,\fnote{\fbackref{9:11} Other mss. read \fbib{that are to come}} he went\fnote{\fbackref{9:11} The Gk. lacks \fbib{went}} through the greater and more perfect tent that was not made by human\fnote{\fbackref{9:11} The Gk. lacks \fbib{human}} hands and that is not a part of this creation. \v{12}Not with the blood of goats and calves, but with his own blood he went into the Most Holy Place once for all and secured our eternal redemption. \v{13}For if the blood of goats and bulls and the ashes of a heifer sprinkled on those who are unclean purifies them physically, \v{14}how much more will the blood of the Messiah,\fnote{\fbackref{9:14} Or \fbib{Christ}} who through the eternal Spirit\fnote{\fbackref{9:14} Other mss. read \fbib{through the Holy Spirit}} offered himself unblemished to God, cleanse our\fnote{\fbackref{9:14} Other mss. read \fbib{your}} consciences from dead actions so that we may serve the living God!
\passage{The Messiah is the Mediator of a New Covenant}

\v{15}This is why the Messiah\fnote{\fbackref{9:15} Lit. \fbib{why he}} is the mediator of a new covenant; so that those who are called may receive the eternal inheritance promised them, since a death has occurred that redeems them from the offenses committed under the first covenant. \v{16}For where there is a will, the death of the one who made it must be established. \v{17}For a will is in force only when somebody has died, since it never takes effect as long as the one who made it is alive. \v{18}This is why even the first covenant was not put into effect without blood. \v{19}For after every commandment in the Law had been spoken to all the people by Moses, he took the blood of calves and goats,\fnote{\fbackref{9:19} Other mss. lack \fbib{and goats}} together with some water, scarlet wool, and branches of hyssop, and sprinkled the scroll and all the people, \v{20}saying, ``This is the blood of the covenant that God ordained for you.''\fnote{\fbackref{9:20} Cf. Exod 24:8} \v{21}In the same way, he sprinkled with the blood both the tent and everything used in worship. \v{22}In fact, under the Law almost everything is cleansed with blood, and without the shedding of the blood there is no forgiveness.
\passage{The Messiah's Perfect Sacrifice}

\v{23}Thus it was necessary for these earthly\fnote{\fbackref{9:23} The Gk. lacks \fbib{earthly}} copies of the things in heaven to be cleansed by these sacrifices,\fnote{\fbackref{9:23} Lit. \fbib{by these things}} but the heavenly things themselves are made clean\fnote{\fbackref{9:23} The Gk. lacks \fbib{are made clean}} with better sacrifices than these. \v{24}For the Messiah\fnote{\fbackref{9:24} Or \fbib{Christ}} did not go into a sanctuary made by human\fnote{\fbackref{9:24} The Gk. lacks \fbib{human}} hands that is merely a copy of the true one, but into heaven itself, to appear now in God's presence on our behalf. \v{25}Nor did he go into heaven\fnote{\fbackref{9:25} The Gk. lacks \fbib{did he go into heaven}} to sacrifice himself again and again, the way the high priest goes into the Holy Place every year with blood that is not his own. \v{26}Then he would have had to suffer repeatedly since the creation of the world. But now, at the end of the ages, he has appeared once for all to remove sin by his sacrifice. \v{27}Indeed, just as people are destined to die once and after that to be judged,\fnote{\fbackref{9:27} Lit. \fbib{after that the judgment}} \v{28}so the Messiah\fnote{\fbackref{9:28} Or \fbib{Christ}} was sacrificed once to take away the sins of many people. And he will appear a second time, not to deal with sin,\fnote{\fbackref{9:28} Lit. \fbib{a second time without sin}} but to bring salvation to those who eagerly wait for him.
\labelchapt{10}
\passage{The Law is a Reflection}

\chapt{10}
\v{1}For the Law, being only\fnote{\fbackref{10:1} The Gk. lacks \fbib{only}} a reflection\fnote{\fbackref{10:1} Or \fbib{shadow}} of the blessings to come and not their substance, can never make perfect those who come near by the same sacrifices repeatedly offered year after year. \v{2}Otherwise, would they not have stopped offering them, because the worshipers, cleansed once for all, would no longer be aware of any sins? \v{3}Instead, through those sacrifices there is a reminder of sins year after year, \v{4}for it is impossible for the blood of bulls and goats to take away sins.
\passage{The Messiah Offered One Sacrifice}

\v{5}For this reason, the Scriptures\fnote{\fbackref{10:5} The Gk. lacks \fbib{the Scriptures}} say, when the Messiah\fnote{\fbackref{10:5} Lit. \fbib{when he}} was about to come into the world:

\begin{poetry}
\poeml ``You did not want sacrifices and offerings, \\
\poemll    but you prepared a body for me. \\
\poeml \v{6}In burnt offerings and sin offerings \\
\poemll    you never took delight. \\
\poeml \v{7}Then I said, `See, I have come to do your will, O God' \\
\poemll    In the volume of the scroll this is written about me.''\fnote{\fbackref{10:7} Cf. Ps 40:6-8}
\end{poetry}

\v{8}In this passage he says, ``You never wanted or took delight in sacrifices, offerings, burnt offerings, and sin offerings,''\fnote{\fbackref{10:8} Cf. Ps 40:6} which are offered according to the Law. \v{9}Then he says, ``See, I have come to do your will.''\fnote{\fbackref{10:9} Cf. Ps 40:7} He takes away the first in order to establish the second. \v{10}By God's will we have been sanctified once and for all through the sacrifice of the body of Jesus, the Messiah.\fnote{\fbackref{10:10} Or \fbib{Christ}}

\v{11}Day after day every priest stands and repeatedly offers the same sacrifices that can never take away sins. \v{12}But when this priest\fnote{\fbackref{10:12} Lit. \fbib{this one}} had offered for all time one sacrifice for sins, ``he sat down at the right hand of God.''\fnote{\fbackref{10:12} Cf. Ps 110:1} \v{13}Since that time, he has been waiting for his enemies to be made a footstool for his feet. \v{14}For by a single offering he has perfected for all time those who are being sanctified. \v{15}The Holy Spirit also assures us of this, for he said:

\begin{poetry}
\poeml \v{16}``This is the covenant that I will make with them \\
\poemll    after those days, declares the Lord:\fnote{\fbackref{10:16} MT source citation reads \fbib{}\divine{Lord}} \\
\poeml I will put my laws in their hearts \\
\poemll    and will write them on their minds, \\
\poeml \v{17}and I will never again remember their sins \\
\poemll    and their lawless deeds.''\fnote{\fbackref{10:17} Jer 31:34}
\end{poetry}

\v{18}Now where there is forgiveness of these sins,\fnote{\fbackref{10:18} Lit. \fbib{of these things}} there is no longer any offering for sin.
\passage{How We Should Live}

\v{19}Therefore, my brothers, since we have confidence to enter the sanctuary by the blood of Jesus, \v{20}the new and living way that he opened for us through the curtain (that is, through his flesh), \v{21}and since we have a great high priest over the household of God, \v{22}let us continue to come near with sincere hearts in the full assurance that faith provides, because our hearts have been sprinkled clean from a guilty conscience, and our bodies have been washed with pure water. \v{23}Let us continue to hold firmly to the hope that we confess without wavering, for the one who made the promise is faithful. \v{24}And let us continue to consider how to motivate one another to love and good deeds, \v{25}not neglecting to meet together, as is the habit of some, but encouraging one another even more as you see the day of the Lord\fnote{\fbackref{10:25} The Gk. lacks \fbib{of the Lord}} coming nearer.

\v{26}For if we choose to go on sinning after we have learned the full truth, there no longer remains a sacrifice for sins, \v{27}but only a terrifying prospect of judgment and a raging fire that will consume the enemies of God.\fnote{\fbackref{10:27} The Gk. lacks \fbib{of God}} \v{28}Anyone who violates the Law of Moses dies without mercy ``on the testimony of two or three witnesses.''\fnote{\fbackref{10:28} Cf. Deut 17:6} \v{29}How much more severe a punishment do you think that person deserves who tramples on God's Son, treats as common the blood of the covenant by which it\fnote{\fbackref{10:29} Or \fbib{he}} was sanctified, and insults the Spirit of grace? \v{30}For we know the one who said, ``Vengeance belongs to me; I will pay them back,''\fnote{\fbackref{10:30} Cf. Deut 32:35} and again, ``The Lord\fnote{\fbackref{10:30} MT source citation reads \fbib{}\divine{Lord}} will judge his people.''\fnote{\fbackref{10:30} Cf. Deut 32:36; Ps 135:14} \v{31}It is a terrifying thing to fall into the hands of the living God!

\v{32}But you must continue to remember those earlier days, how after you were enlightened you endured a hard and painful struggle. \v{33}At times you were made a public spectacle by means of insults and persecutions, while at other times you associated with people who were treated this way. \v{34}For you sympathized\fnote{\fbackref{10:34} Or \fbib{suffered}} with the prisoners and cheerfully submitted to the violent seizure of your property, because you know that you have a better and more permanent possession.

\v{35}So don't lose your confidence, since it holds a great reward for you. \v{36}For you need endurance, so that after you have done God's will you can receive what he has promised. \v{37}For

\begin{poetry}
\poeml ``in a very little while \\
\poemll    the one who is coming will return--- \\
\poemlll       he will not delay; \\
\poeml \v{38}but my righteous one will live by faith, \\
\poemll    and if he turns back, \\
\poemlll       my soul will take no pleasure in him.''\fnote{\fbackref{10:38} Cf. Isa 26:20 (LXX); Hab 2:3-4 (LXX)}
\end{poetry}

\v{39}Now, we do not belong to those who turn back and are destroyed, but to those who have faith and are saved.
\labelchapt{11}
\passage{The Meaning of Faith}

\chapt{11}
\v{1}Now faith is the assurance that what we hope for will come about\fnote{\fbackref{11:1} The Gk. lacks \fbib{will come about}} and the certainty that what we cannot see exists.\fnote{\fbackref{11:1} The Gk. lacks \fbib{exist}} \v{2}By faith our ancestors won approval.

\v{3}By faith we understand that time was created by the word of God, so that what is seen was made from things that are invisible.

\v{4}By faith Abel offered to God a better sacrifice than Cain did,\fnote{\fbackref{11:4} The Gk. lacks \fbib{did}} and by faith\fnote{\fbackref{11:4} Lit. \fbib{it}} he was declared to be righteous, since God himself accepted his offerings. And by faith\fnote{\fbackref{11:4} Lit. \fbib{it}} he continues to speak, even though he is dead.

\v{5}By faith Enoch was taken away without experiencing death. He could not be found, because God had taken him away. For before he was taken, he won approval as one who pleased God. \v{6}Now without faith it is impossible to please God, for whoever comes to him must believe that he exists and that he rewards those who diligently search for him.

\v{7}By faith Noah, when warned about things not yet seen, reverently prepared an ark to save his family, and by faith\fnote{\fbackref{11:7} Lit. \fbib{it}} he condemned the world and inherited the righteousness that comes by faith.

\v{8}By faith Abraham, when called to go to a place he would later receive as his inheritance, obeyed and went, even though he did not know where he was going.

\v{9}By faith he made his home in the promised land like a stranger, living in tents, as did Isaac and Jacob, who also inherited the same promise, \v{10}because he was waiting for the city with permanent foundations, whose architect and builder is God.

\v{11}By faith Sarah, even though she was old and barren, received the strength to conceive, because she was convinced that the one who had made the promise was faithful. \v{12}Abraham\fnote{\fbackref{11:12} Lit. \fbib{He}} was as good as dead, yet from this one man came descendants as numerous as the stars in the sky and as countless as the sand on the seashore.

\v{13}All these people died having faith. They did not receive the things that were promised, yet they saw them in the distant future and welcomed them, acknowledging that they were strangers and foreigners on earth. \v{14}For people who say such things make it clear that they are looking for a country of their own. \v{15}If they had been thinking about what they had left behind, they would have had an opportunity to go back. \v{16}Instead, they were longing for a better country, that is, a heavenly one. That is why God is not ashamed to be called their God, because he has prepared a city for them.

\v{17}By faith Abraham, when he was tested, offered Isaac---he who had received the promises was about to offer his unique son\fnote{\fbackref{11:17} Lit. \fbib{unique one}} in sacrifice, \v{18}about whom it had been said, ``It is through Isaac that descendants will be named for you.''\fnote{\fbackref{11:18} Cf. Gen 21:12} \v{19}Abraham\fnote{\fbackref{11:19} Lit. \fbib{He}} was certain that God could raise the dead, and figuratively speaking, he did get Isaac\fnote{\fbackref{11:19} Lit. \fbib{him}} back in this way.

\v{20}By faith Isaac blessed Jacob and Esau in regard to their future.

\v{21}By faith Jacob, when he was dying, blessed each of Joseph's sons ``and worshipped while leaning\fnote{\fbackref{11:21} The Gk. lacks \fbib{while leaning}} on the top of his staff.''

\v{22}By faith Joseph, when his end was near, spoke about the exodus of the Israelis and gave them instructions about burying\fnote{\fbackref{11:22} The Gk. lacks \fbib{burying}} his bones.

\v{23}By faith Moses was hidden by his parents for three months after he was born, because they saw that he was a beautiful child and were not afraid of the king's order.

\v{24}By faith Moses, when he had grown up, refused to be called a son of Pharaoh's daughter, \v{25}because he preferred being mistreated with God's people to enjoying the pleasures of sin for a short time. \v{26}He thought that being insulted for the sake of the Messiah\fnote{\fbackref{11:26} Or \fbib{Christ}} was of greater value than the treasures of Egypt, because he was looking ahead to his reward.

\v{27}By faith he left Egypt, without being afraid of the king's anger, and he persevered because he saw the one who is invisible.

\v{28}By faith he established the Passover and the sprinkling of blood to keep the destroyer of the firstborn from touching the people.\fnote{\fbackref{11:28} Lit. \fbib{them}}

\v{29}By faith they went through the Red Sea as if it were dry land. When the Egyptians tried to do this, they were drowned.

\v{30}By faith the walls of Jericho fell down after they had been encircled for seven days.

\v{31}By faith Rahab the prostitute did not die with those who were disobedient, because she had welcomed the spies with a greeting of\fnote{\fbackref{11:31} The Gk. lacks \fbib{a greeting of}} peace.

\v{32}And what more should I say? For time would fail me to tell you about Gideon, Barak, Samson, Jephthah, David, Samuel, and the prophets. \v{33}Through faith they conquered kingdoms, administered justice, received promises, shut the mouths of lions, \v{34}put out raging fires, escaped death by\fnote{\fbackref{11:34} Lit. \fbib{by the edge of}} the sword, found strength in weakness, became powerful in battle, and routed foreign armies. \v{35}Women received their dead raised back to life. Other people were brutally tortured, but refused to be ransomed, so that they might gain a better resurrection. \v{36}Still others endured taunts and floggings, and even chains and imprisonment. \v{37}They were stoned to death, sawed in half, and killed with swords. They went around in sheepskins and goatskins. They were needy, oppressed, and mistreated. \v{38}The world wasn't worthy of them. They wandered in deserts and mountains, and from caves to holes in the ground.

\v{39}All these people won approval for their faith but they did not receive what was promised, \v{40}since God had planned something better for us, so that they would not be perfected without us.
\labelchapt{12}
\passage{We Must Look Off to Jesus}

\chapt{12}
\v{1}Therefore, having so vast a cloud of witnesses surrounding us, and throwing off everything that hinders us and especially the sin that so easily entangles\fnote{\fbackref{12:1} Other mss. read \fbib{distracts}} us, let us keep running with endurance the race set before us, \v{2}fixing our attention on Jesus, the pioneer and perfecter of the faith, who, in view of\fnote{\fbackref{12:2} Or \fbib{instead}} the joy set before him, endured the cross, disregarding its shame, and has sat down at the right hand of the throne of God.
\passage{The Father Disciplines Us}

\v{3}Think about the one who endured such hostility from sinners, so that you may not become tired and give up. \v{4}In your struggle against sin you have not yet resisted to the point of shedding your\fnote{\fbackref{12:4} The Gk. lacks \fbib{the point of shedding your}} blood. \v{5}You have forgotten the encouragement that is addressed to you as sons:

\begin{poetry}
\poeml ``My son, do not think lightly of the Lord's\fnote{\fbackref{12:5} MT source citation reads \fbib{}\divine{Lord}} discipline \\
\poemll    or give up when you are corrected by him. \\
\poeml \v{6}For the Lord\fnote{\fbackref{12:6} MT source citation reads \fbib{}\divine{Lord}} disciplines the one he loves, \\
\poemll    and he punishes\fnote{\fbackref{12:6} Or \fbib{whips}} every son he accepts.''\fnote{\fbackref{12:6} Cf. Prov 3:11-12}
\end{poetry}

\v{7}What you endure disciplines you: God is treating you as sons. Is there a son whom his father does not discipline? \v{8}Now if you are without any discipline, in which all sons share, then you are illegitimate and not God's\fnote{\fbackref{12:8} Lit. \fbib{his}} sons. \v{9}Furthermore, we had earthly fathers who disciplined us, and we respected them for it. We should submit even more to the Father of our spirits and live, shouldn't we? \v{10}For a short time they disciplined us as they thought best, but God\fnote{\fbackref{12:10} Lit. \fbib{he}} does it for our good, so that we may share in his holiness. \v{11}No discipline seems pleasant at the time, but painful. Later on, however, for those who have been trained by it, it produces a harvest of righteousness and peace.
\passage{Live as God's People}

\v{12}Therefore, strengthen your tired arms and your weak knees, \v{13}and straighten the paths of your life,\fnote{\fbackref{12:13} Lit. \fbib{feet}} so that your lameness may not become worse, but instead may be healed.

\v{14}Pursue peace with everyone, as well as holiness, without which no one will see the Lord. \v{15}See to it that no one fails to obtain the grace of God and that no bitter root grows up and causes you trouble, or many of you will become defiled. \v{16}No one should be immoral or godless like Esau, who sold his birthright for a single meal. \v{17}For you know that afterwards, when he wanted to inherit the blessing, he was rejected because he could not find any opportunity to repent, even though he begged to repent\fnote{\fbackref{12:17} Lit. \fbib{begged for it}} with tears.

\v{18}You have not come to something\fnote{\fbackref{12:18} Other mss. read \fbib{to a mountain}} that can be touched, to a blazing fire, to darkness, to gloom, \v{19}to a trumpet's blast, or to a voice that made the hearers beg that not another word be spoken to them. \v{20}For they could not endure the command that was given: ``If even an animal touches the mountain, it must be stoned to death.''\fnote{\fbackref{12:20} Cf. Exod 19:12-13} \v{21}Indeed, the sight was so terrifying that Moses said, ``I am trembling with fear.''\fnote{\fbackref{12:21} Cf. Deut 9:19} \v{22}Instead, you have come to Mount Zion, to the city of the living God, to the heavenly Jerusalem, to tens of thousands of angels joyfully gathered together, \v{23}to the assembly\fnote{\fbackref{12:23} Or \fbib{church}} of the firstborn who are enrolled in heaven, to a judge who is the God of all, to the spirits of righteous people who have been made perfect, \v{24}to Jesus, the mediator of a new covenant, and to the sprinkled blood that speaks a better message than Abel's.

\v{25}See to it that you do not ignore the one who is speaking. For if the hearers\fnote{\fbackref{12:25} Lit. \fbib{if they}} did not escape when they ignored the one who warned them on earth, how much less will we escape\fnote{\fbackref{12:25} The Gk. lacks \fbib{escape}} if we turn away from the one who is from heaven! \v{26}At that time his voice shook the earth, but now he has promised, ``Once more I will shake not only the earth but also heaven.''\fnote{\fbackref{12:26} Cf. Hag 2:6} \v{27}The expression ``once more''\fnote{\fbackref{12:27} Cf. Hag 2:6} signifies the removal of what can be shaken, that is, what he has made, so that what cannot be shaken may remain. \v{28}Therefore, since we are receiving a kingdom that cannot be shaken, let us be thankful and worship God in reverence and fear in a way that pleases him. \v{29}For ``our God is an all-consuming fire.''\fnote{\fbackref{12:29} Cf. Deut 4:24}
\labelchapt{13}
\passage{Concluding Words}

\chapt{13}
\v{1}Let brotherly love continue. \v{2}Stop neglecting to show hospitality to strangers, for by showing hospitality\fnote{\fbackref{13:2} Lit. \fbib{by this}} some have had angels as their guests without being aware of it. \v{3}Continue to remember those in prison as if you were in prison with them, as well as those who are mistreated, since they also are only mortal.\fnote{\fbackref{13:3} Lit. \fbib{are in the body}}

\v{4}Let marriage be kept honorable in every way, and the marriage bed undefiled. For God will judge those who commit sexual sins, especially those who commit adultery.

\v{5}Keep your lives free from the love of money, and be content with what you have, for God\fnote{\fbackref{13:5} Lit. \fbib{he}} has said, ``I will never leave you or abandon you.''\fnote{\fbackref{13:5} Cf. Deut 31:6} \v{6}Hence we can confidently say, ``The Lord\fnote{\fbackref{13:6} MT source citation reads \fbib{}\divine{Lord}} is my helper; I will not be afraid. What can anyone do to me?''\fnote{\fbackref{13:6} Cf. Ps 118:6}

\v{7}Remember your leaders, those who have spoken God's word to you. Think about the impact of their lives, and imitate their faith. \v{8}Jesus, the Messiah,\fnote{\fbackref{13:8} Or \fbib{Christ}} is the same yesterday and today---and forever!

\v{9}Stop being\fnote{\fbackref{13:9} Or \fbib{Do not be}} carried away by all kinds of unusual teachings, for it is good that the heart be strengthened by grace, not by food laws\fnote{\fbackref{13:9} Lit. \fbib{by foods}} that have never helped those who follow them.

\v{10}We have an altar, and those who serve in the tent have no right to eat at it. \v{11}For the bodies of animals, whose blood is taken into the sanctuary by the high priest as an offering for sin, are burned outside the camp. \v{12}That is why Jesus, in order to sanctify the people by his own blood, also suffered outside the city gate. \v{13}Therefore go to him outside the camp and endure the insults he endured. \v{14}For here we have no permanent city but are looking for the one that is coming. \v{15}Therefore, through him let us always bring God a sacrifice of praise, that is, the fruit of our lips that confess his name. \v{16}Do not neglect to do good and to be generous, for God is pleased with such sacrifices.

\v{17}Continue to follow and be submissive to your leaders, since they are watching over your souls as those who will have to give a word of explanation. By doing this, you will be letting them carry out their duties joyfully, and not with grief, for that would be harmful for you.

\v{18}Pray for us, for we are sure that we have a clear conscience and desire to live honorably in every way. \v{19}I especially ask you to do this so that I may be brought back to you sooner.

\v{20}Now may the God of peace, who by the blood of the eternal covenant brought back from the dead our Lord Jesus, the Great Shepherd of the sheep, \v{21}equip you with everything good\fnote{\fbackref{13:21} Other mss. read \fbib{for every good work}} to do his will, accomplishing in us\fnote{\fbackref{13:21} Other mss. read \fbib{you}} what pleases him through Jesus, the Messiah.\fnote{\fbackref{13:21} Or \fbib{Christ}} To him be glory forever and ever!\fnote{\fbackref{13:21} Other mss. lack \fbib{and ever}} Amen.
\passage{Final Greeting}

\v{22}I urge you, brothers, to listen patiently to my encouraging message,\fnote{\fbackref{13:22} Or \fbib{word of exhortation}} for I have written you a short letter.\fnote{\fbackref{13:22} Lit. \fbib{written you briefly}} \v{23}You should know that our brother Timothy has been set free. If he comes soon, he will be with me when I see you.

\v{24}Greet all your leaders and all the saints. Those who are from Italy greet you.

\v{25}May grace be with all of you!\fnote{\fbackref{13:25} Other mss. read \fbib{with all of you! Amen}}

\bookheader{James}
\labelbook{James}

\bookpretitle{The Letter of}
\booktitle{James}

\labelchapt{1}
\passage{Greetings}

\chapt{1}
\v{1}From:\fnote{\fbackref{1:1} The Gk. lacks \fbib{From}} James, a servant of God and of the Lord Jesus, the Messiah.\fnote{\fbackref{1:1} Or \fbib{Christ}}

To: The twelve tribes in the Dispersion.\fnote{\fbackref{1:1} I.e. the Jewish communities outside the land of Israel}

Greetings.
\passage{Responding Wisely to Life}

\v{2}Consider it pure joy, my brothers, when you are involved in various trials, \v{3}because you know that the testing of your faith produces endurance. \v{4}But you must let endurance have its full effect, so that you may be mature and complete, lacking nothing.

\v{5}Now if any of you lacks wisdom, he should ask God, who gives to everyone generously without a rebuke, and it will be given to him. \v{6}But he must ask in faith, without any doubts, for the one who has doubts is like a wave of the sea that is driven and tossed by the wind. \v{7}Such a person should not expect to receive anything from the Lord. \v{8}He is a double-minded man, unstable in all he undertakes.\fnote{\fbackref{1:8} Lit. \fbib{in all his ways}}

\v{9}A brother of humble means should rejoice in his having been exalted, \v{10}and a rich person in his having been humbled, because he will fade away like a wild flower. \v{11}For the sun comes up with its scorching heat and dries up the grass. The flower in it drops off, and its beauty is gone. That is how the rich person will fade away in his pursuits.
\passage{Trial and Temptation}

\v{12}How blessed is the man who endures temptation! When he has passed the test, he will receive the victor's crown of life that God\fnote{\fbackref{1:12} Lit. \fbib{he}; other mss. read \fbib{God}; still other mss. read \fbib{the Lord}} has promised to those who keep on loving him. \v{13}When someone is tempted, he should not say, ``I am being tempted by God,'' because God cannot be tempted by evil, nor does he tempt anyone. \v{14}Instead, each person is tempted by his own desire, being lured and trapped by it. \v{15}When that desire becomes pregnant, it gives birth to sin; and when that sin grows up, it gives birth to death.
\passage{Practical Christian Living}

\v{16}Do not be\fnote{\fbackref{1:16} Or \fbib{Don't be}} deceived, my dear brothers. \v{17}Every generous act of giving and every perfect gift is from above and comes down from the Father who made the heavenly lights,\fnote{\fbackref{1:17} Lit. \fbib{the Father of lights}} in whom there is no inconsistency or shifting shadow. \v{18}In accordance with his will he made us his children by the word of truth, so that we might become the most important of his creatures.\fnote{\fbackref{1:18} Lit. \fbib{a kind of first fruits among his creatures}}

\v{19}You must understand this, my dear brothers. Everyone should be quick to listen, slow to speak, and slow to get angry. \v{20}For human anger does not produce the righteousness that God desires. \v{21}Therefore, rid yourselves of everything impure and every expression of wickedness, and with a gentle spirit welcome the word planted in you that can save your souls.

\v{22}Keep on being obedient to\fnote{\fbackref{1:22} Or \fbib{being doers of}} the word, and not merely being hearers who deceive themselves. \v{23}For if anyone hears the word but is not obedient to it, he is like a man who looks at himself in a mirror \v{24}and studies himself carefully, and then goes off and immediately forgets what he looks like. \v{25}But the one who looks at the perfect law of freedom and remains committed to it---thereby demonstrating that he is not a forgetful hearer but a doer of what that law\fnote{\fbackref{1:25} Lit. \fbib{what it}} requires---will be blessed in what he does.

\v{26}If anyone thinks that he is religious and does not bridle his tongue, but instead deceives himself,\fnote{\fbackref{1:26} Lit. \fbib{deceives his own heart}} his religion is worthless. \v{27}A religion that is pure and stainless according to\fnote{\fbackref{1:27} Lit. \fbib{stainless in the sight of}} God the Father is this: to take care of orphans and widows who are suffering, and to keep oneself unstained by the world.
\labelchapt{2}
\passage{Do Not Show Partiality}

\chapt{2}
\v{1}My brothers, do not let your faith in our glorious Lord Jesus, the Messiah,\fnote{\fbackref{2:1} Or \fbib{Christ}} be tainted by favoritism. \v{2}Suppose a man wearing gold rings and fine clothes comes into your assembly,\fnote{\fbackref{2:2} Or \fbib{synagogue}} and a poor man in dirty clothes also comes in. \v{3}If you give special attention to the man wearing fine clothes and say, ``Please take this seat,'' but you say to the poor man, ``Stand over there'' or ``Sit on the floor at my feet,''\fnote{\fbackref{2:3} Lit. \fbib{Sit at my footstool}} \v{4}then you will have made false distinctions among yourselves and will have judged from evil motives, will you not?

\v{5}Listen, my dear brothers! God has chosen the poor in the world to become rich in faith and to be heirs of the kingdom that he promised to those who keep on loving him, has he not? \v{6}But you have humiliated the man who is poor. Are not rich people the ones who oppress you and drag you into court? \v{7}Are not they the ones who blaspheme the noble Name\fnote{\fbackref{2:7} I.e. God} by which you have been called?

\v{8}Nevertheless, you are doing the right thing if you obey the royal Law in keeping with the Scripture, ``You must love your neighbor as yourself.''\fnote{\fbackref{2:8} Cf. Lev 19:18} \v{9}But if you show partiality, you are committing sin and will be convicted by the Law as violators. \v{10}For whoever keeps the whole Law but fails in one point is guilty of breaking all of it. \v{11}For the one who said, ``Never commit adultery,''\fnote{\fbackref{2:11} Cf. Exod 20:14; Deut 5:18} also said, ``Never murder.''\fnote{\fbackref{2:11} Cf. Exod 20:13; Deut 5:17} Now if you do not commit adultery, but you murder, you become a violator of the Law. \v{12}You must make it your habit to speak and act like people who are going to be judged by the law of liberty. \v{13}For the one who has shown no mercy will be judged without mercy. Mercy triumphs over judgment.
\passage{Faith is Shown by Actions}

\v{14}What good does it do, my brothers, if someone claims to have faith but does not prove it with actions? This kind of faith cannot save him, can it? \v{15}Suppose a brother or sister does not have any clothes or daily food \v{16}and one of you tells them, ``Go in peace! Stay warm and eat heartily.'' If you do not provide for their bodily needs, what good does it do? \v{17}In the same way, faith by itself, if it does not prove itself with actions, is dead.

\v{18}But someone may say, ``You have faith, and I have actions.'' Show me your faith without any actions, and I will show you my faith by my actions. \v{19}You believe that there is one God. That's fine! Even the demons believe that and tremble with fear. \v{20}Do you want proof, you foolish person, that faith without actions is worthless? \v{21}Our ancestor Abraham was justified by his actions when he offered his son Isaac on the altar, wasn't he? \v{22}You\fnote{\fbackref{2:22} The Gk. \fbib{you} is sing.} see that his faith worked together with what he did, and by his actions his faith was made complete. \v{23}And so the Scripture was fulfilled that says, ``Abraham believed God, and it was credited to him as righteousness.''\fnote{\fbackref{2:23} Cf. Gen 15:6} And so he was called God's friend. \v{24}You\fnote{\fbackref{2:24} The Gk. \fbib{you} is pl.} observe that a person is justified through actions and not through faith alone. \v{25}Likewise, Rahab the prostitute was justified through actions when she welcomed the messengers\fnote{\fbackref{2:25} Other mss. read \fbib{spies}} and sent them away on a different road, wasn't she? \v{26}For just as the body without the spirit\fnote{\fbackref{2:26} Or \fbib{without breath}} is dead, so faith without actions is also dead.
\labelchapt{3}
\passage{Speak Wisely}

\chapt{3}
\v{1}Not many of you should become teachers, my brothers, because you know that we who teach\fnote{\fbackref{3:1} The Gk. lacks \fbib{who teach}} will be judged more severely than others.\fnote{\fbackref{3:1} The Gk. lacks \fbib{than others}} \v{2}For all of us make many mistakes. If someone does not make any mistakes when he speaks, he is perfect and able to control his whole body. \v{3}Now if we put bits into horses' mouths to make them obey us, we can guide their whole bodies as well. \v{4}And look at ships! They are so big that it takes strong winds to drive them, yet they are steered by a tiny rudder wherever the helmsman directs.

\v{5}In the same way, the tongue is a small part of the body, yet it can boast of great achievements. A huge forest can be set on fire by a little flame. \v{6}The tongue is a fire, a world of evil. Placed among the parts of our bodies, the tongue contaminates the whole body and sets on fire the course of life, and is itself set on fire by hell.\fnote{\fbackref{3:6} Gk. \fbib{Gehenna}; a Gk. transliteration of the Heb. for \fbib{Valley of Hinnom}} \v{7}For all kinds of animals, birds, reptiles, and sea creatures can be or have been tamed by humans, \v{8}but no one can tame the tongue. It is an uncontrollable evil filled with deadly poison. \v{9}With it we bless the Lord and Father, and with it we curse those who are made in God's likeness. \v{10}From the same mouth come blessing and cursing. It should not be like this, my brothers! \v{11}A spring cannot pour both fresh and brackish water from the same opening, can it? \v{12}My brothers, a fig tree cannot produce olives, nor a grapevine figs, can it? Neither can a salt spring produce fresh water.
\passage{Live Wisely}

\v{13}Who among you is wise and understanding? Let him show by his noble conduct that his actions are done humbly and wisely. \v{14}But if you have bitter jealousy and rivalry in your hearts, stop boasting and slandering the truth. \v{15}That kind of wisdom does not come from above. No, it is worldly, self-centered, and demonic. \v{16}For wherever jealousy and rivalry exist, there is disorder and every kind of evil.

\v{17}However, the wisdom that comes from above is first of all pure, then peace-loving, gentle, willing to yield, full of compassion and good deeds,\fnote{\fbackref{3:17} Lit. \fbib{fruit}} and without a trace of partiality or hypocrisy. \v{18}And a harvest of righteousness is grown from the seed of peace\fnote{\fbackref{3:18} Lit. \fbib{is grown in peace}} planted by peacemakers.
\labelchapt{4}
\passage{Stop Fighting with Each Other}

\chapt{4}
\v{1}Where do those fights and quarrels among you come from? They come from your selfish desires that are at war in your bodies, don't they? \v{2}You want something but do not get it, so you commit murder. You covet something but cannot obtain it, so you quarrel and fight. You do not get things because you do not ask for them! \v{3}You ask for something but do not get it because you ask for it for the wrong reason---for your own pleasure.

\v{4}You adulterers! Don't you know that friendship with the world means hostility with God? So whoever wants to be a friend of this world is an enemy of God. \v{5}Or do you think the Scripture means nothing when it says that the Spirit that God\fnote{\fbackref{4:5} Lit. \fbib{he}} caused to live in us jealously yearns for us?\fnote{\fbackref{4:5} Cf. Exod 20:5; Num 11:29} \v{6}But he gives all the more grace. And so he says,

\begin{poetry}
\poeml ``God opposes the arrogant \\
\poemll    but gives grace to the humble.''\fnote{\fbackref{4:6} Cf. Prov 3:34 (LXX)}
\end{poetry}

\v{7}Therefore, submit yourselves to God. Resist the devil, and he will run away from you. \v{8}Come close to God, and he will come close to you. Cleanse your hands, you sinners, and purify your hearts, you double-minded. \v{9}Be miserable, mourn, and cry. Let your laughter be turned into mourning, and your joy into gloom. \v{10}Humble yourselves in the Lord's presence, and he will exalt you.
\passage{Do Not Criticize Each Other}

\v{11}Do not criticize each other, brothers. Whoever makes it his habit to criticize his brother or to judge his brother is judging the Law and condemning the Law. But if you condemn the Law, you are not a practicer of the Law but its judge. \v{12}There is only one Lawgiver and Judge---the one who can save and destroy. So who are you to judge your neighbor?
\passage{Do Not Boast about Future Plans}

\v{13}Now listen, you who say, ``Today or tomorrow we will go to such and such a town, stay there a year, conduct business, and make money.'' \v{14}You do not know what tomorrow will bring. What is your life? You are a mist that appears for a little while and then vanishes. \v{15}Instead you should say, ``If the Lord wants us to, we will live---and do this or that.'' \v{16}But you boast about your proud intentions. All such boasting is evil.

\v{17}Therefore, anyone who knows what is right but fails to do it is guilty of sin.
\labelchapt{5}
\passage{Advice for Rich People}

\chapt{5}
\v{1}Now listen, you rich people! Cry and moan over the miseries that are overtaking you. \v{2}Your riches are rotten, your clothes have been eaten by moths, \v{3}your gold and silver are corroded, and their corrosion will be used as evidence against you and will eat your flesh like fire. You have stored up treasures in these last days. \v{4}Look! The wages that you kept back from the workers who harvested your fields are shouting out against you, and the cries of the reapers have reached the ears of the Lord of the Heavenly Armies. \v{5}You have lived in luxury and pleasure on earth. You have fattened yourselves\fnote{\fbackref{5:5} Lit. \fbib{fattened your hearts}} for the day of slaughter. \v{6}You have condemned and murdered the one who is righteous, even though he did not rebel against you.
\passage{Be Patient}

\v{7}So be patient, brothers, until the coming of the Lord. See how the farmer waits for the precious crop from his land, being patient with it until it receives the fall and the spring rains. \v{8}You, too, must be patient. Strengthen your hearts, because the coming of the Lord is near. \v{9}Do not complain about each other, brothers, or you will be condemned. Look! The Judge is standing at the door! \v{10}As an example of suffering and patience, brothers, take the prophets, who spoke in the name of the Lord. \v{11}We consider those who endured to be blessed. You have heard about Job's endurance and have seen the purpose of the Lord---that the Lord is compassionate and merciful.
\passage{Do Not Swear Oaths}

\v{12}Above all, brothers, do not swear oaths by heaven, by earth, or by any other object.\fnote{\fbackref{5:12} Lit. \fbib{oath}} Instead, let your ``Yes'' mean yes and your ``No'' mean no! Otherwise,\fnote{\fbackref{5:12} Lit. \fbib{no, lest}} you may fall under condemnation.
\passage{The Power of Prayer}

\v{13}Is anyone among you suffering? He should keep on praying. Is anyone cheerful? He should keep reciting psalms. \v{14}Is anyone among you sick? He should call for the elders of the church, and they should pray for him and anoint him with oil in the name of the Lord. \v{15}And the prayer offered in faith\fnote{\fbackref{5:15} Lit. \fbib{the prayer of faith}} will save the person who is sick. The Lord will raise him up, and if he has committed any sins, he will be forgiven.

\v{16}Therefore, make it your habit to confess your sins to one another and to pray for one another, so that you may be healed. The prayer of a righteous person is powerful and effective. \v{17}Elijah was a person just like us, and he prayed earnestly for it not to rain, and rain never came to the land for three years and six months. \v{18}Then he prayed again, and the skies poured out rain, and the ground produced its crops.

\v{19}My brothers, if one of you wanders away from the truth and somebody brings him back, \v{20}you may be sure that whoever brings a sinner back from his wrong path will save his soul from death and cover a multitude of sins.

\bookheader{1 Peter}
\labelbook{1Pet}

\bookpretitle{The Letter Called}
\booktitle{First Peter}

\labelchapt{1}
\passage{Greetings}

\chapt{1}
\v{1}From:\fnote{The Gk. lacks \fbib{From}} Peter, an apostle of Jesus, the Messiah.\fnote{Or \fbib{Christ}}

To: The exiles of the Dispersion\fnote{I.e. the Jewish communities outside of Israel} in Pontus, Galatia, Cappadocia, Asia, and Bithynia, \v{2}the people chosen according to the foreknowledge of God the Father through the sanctifying action of the Spirit to be obedient to Jesus, the Messiah,\fnote{Or \fbib{Christ}} and to be sprinkled with his blood.

May grace and peace be yours in abundance!
\passage{Our Hope and Joy are in the Messiah}

\v{3}Blessed be the God and Father of our Lord Jesus, the Messiah!\fnote{Or \fbib{Christ}} Because of his great mercy he has granted us a new birth, resulting in an immortal hope through the resurrection of Jesus, the Messiah,\fnote{Or \fbib{Christ}} from the dead \v{4}and to an inheritance kept in heaven for you that can't be destroyed, corrupted, or changed. \v{5}Through faith you are being protected by God's power for a salvation that is ready to be revealed at the end of this era. \v{6}You greatly rejoice in this, even though you have to suffer various kinds of trials for a little while, \v{7}so that your genuine faith, which is more valuable than gold that perishes when tested by fire, may result in praise, glory, and honor when Jesus, the Messiah,\fnote{Or \fbib{Christ}} is revealed.

\v{8}Though you have not seen\fnote{Other mss. read \fbib{known}} him, you love him. And even though you do not see him now, you believe in him and rejoice with an indescribable and glorious joy, \v{9}because you are receiving the goal of your faith, the salvation of your souls.

\v{10}Even the prophets, who prophesied about the grace that was to be yours, carefully researched and investigated this salvation. \v{11}They tried to find out what era or specific time the Spirit of the Messiah\fnote{Or \fbib{Christ}} in them kept referring to when he predicted the sufferings of the Messiah\fnote{Or \fbib{Christ}} and the glories that would follow. \v{12}It was revealed to them that they were not serving themselves but you in regard to the things that have now been announced to you by those who brought you the good news through the Holy Spirit sent from heaven. These are things that even the angels desire to look into.
\passage{Be Holy}

\v{13}Therefore, prepare your minds for action, keep a clear head, and set your hope completely on the grace to be given you when Jesus, the Messiah,\fnote{Or \fbib{Christ}} is revealed. \v{14}As obedient children, do not be shaped by the desires that used to influence you when you were ignorant. \v{15}Instead, be holy in every aspect of your life, just as the one who called you is holy. \v{16}For it is written, ``You must be holy, because I am holy.''\fnote{Cf. Lev 11:44-45; Lev 19:2}

\v{17}If you call ``Father'' the one who judges everyone impartially according to what they have done, you must live in reverent fear as long as you are strangers in a strange land. \v{18}For you know that it was not with perishable things like silver or gold that you have been ransomed from the worthless way of life handed down to you by your ancestors, \v{19}but with the precious blood of the Messiah,\fnote{Or \fbib{Christ}} like that of a lamb without blemish or defect. \v{20}On the one hand, he was foreknown before the creation\fnote{Or \fbib{foundation}} of the world, but on the other hand, he was revealed at the end of time for your sake. \v{21}Through him you believe in God, who raised him from the dead and gave him glory, so that your faith and hope might be in God.
\passage{Love One Another}

\v{22}Now that you have obeyed the truth\fnote{Other mss. read \fbib{the truth through the Spirit}} and have purified your souls to love your brothers sincerely, you must love one another intensely and with a pure heart. \v{23}For you have been born again, not by a seed that perishes but by one that cannot perish---by the living and everlasting word of God.\fnote{Or \fbib{by the word of the living and everlasting God}} \v{24}For

\begin{poetry}
\poeml ``All human life\fnote{Lit. \fbib{all flesh}} is like grass, \\
\poemll    and all its glory is like a flower in the grass. \\
\poeml The grass dries up and the flower drops off, \\
\poeml \v{25}but the word of the Lord\fnote{MT source citation reads \fbib{}\divine{Lord}} lasts forever.''\fnote{Cf. Isa 40:6-8}
\end{poetry}

Now this word is the good news that was announced to you.
\labelchapt{2}
\passage{Live as God's Chosen People}

\chapt{2}
\v{1}Therefore, rid yourselves of every kind of evil and deception, hypocrisy, jealousy, and every kind of slander. \v{2}Like newborn babies, thirst for the pure milk of the word so that by it you may grow in your salvation. \v{3}Surely you have tasted that the Lord is good!

\v{4}As you come to him, the living stone who was rejected by people but was chosen and precious in God's sight, \v{5}you, too, as living stones, are building yourselves up into a spiritual house and a holy priesthood, so that you may offer spiritual sacrifices that are acceptable to God through Jesus, the Messiah.\fnote{Or \fbib{Christ}} \v{6}This is why it says in Scripture:

\begin{poetry}
\poeml ``Look! I am laying a chosen, precious cornerstone\fnote{Or \fbib{capstone}} in Zion. \\
\poemll    The one who believes in him will never be ashamed.''\fnote{Cf. Isa 28:16}
\end{poetry}

\v{7}Therefore he is precious to you who believe, but to those who do not believe,

\begin{poetry}
\poeml ``The stone that the builders rejected \\
\poemll    has become the cornerstone,\fnote{Or \fbib{capstone}} \\
\poeml \v{8}a stone they stumble over \\
\poemll    and a rock they trip on.''\fnote{Cf. Ps 118:22; Isa 8:14}
\end{poetry}

They keep on stumbling because they disobey the word, as they were destined to do.

\v{9}But you are a chosen people, a royal priesthood, a holy nation, a people to be his very own and to proclaim the wonderful deeds\fnote{Or \fbib{the excellence}; cf. Isa 43:21} of the one who called you out of darkness into his marvelous light.

\begin{poetry}
\poeml \v{10}Once you were not a people, \\
\poemll    but now you are the people of God. \\
\poeml Once you had not received mercy, \\
\poemll    but now you have received mercy.
\end{poetry}
\passage{Live as God's Servants}

\v{11}Dear friends, I urge you as aliens and exiles to keep on abstaining from the desires of the flesh that wage war against the soul. \v{12}Continue to live such upright lives among the gentiles that, when they slander you as practicers of evil, they may see your good actions and glorify God when he visits them.\fnote{Lit. \fbib{God on the day of visitation}}

\v{13}For the Lord's sake submit yourselves to every human authority: whether to the king as supreme, \v{14}or to governors who are sent by him to punish those who do wrong and to praise those who do right. \v{15}For it is God's will that by doing right you should silence the ignorant talk\fnote{Lit. \fbib{the ignorance}} of foolish people. \v{16}Live like free people, and do not use your freedom as an excuse for doing evil. Instead, be God's servants. \v{17}Honor everyone. Keep on loving the community of believers,\fnote{Lit. \fbib{the brotherhood}} fearing God, and honoring the king.
\passage{Suffer Patiently}

\v{18}You household servants must submit yourselves to your masters out of respect, not only to those who are kind and fair, but also to those who are unjust. \v{19}For it is a fine thing if, when moved by your conscience to please God, you suffer patiently when wronged. \v{20}What good does it do if, when you sin, you patiently receive punishment for it? But if you suffer for doing good and receive it patiently, you have God's approval. \v{21}This is, in fact, what you were called to do, because:

\begin{poetry}
\poeml The Messiah\fnote{Or \fbib{Christ}} also suffered for you \\
\poemll    and left an example for you \\
\poemlll       to follow in his steps. \\
\poeml \v{22}``He never sinned, \\
\poemll    and he never told a lie.''\fnote{Lit. \fbib{and no deceit was found in his mouth}; cf. Isa 53:9} \\
\poeml \v{23}When he was insulted, \\
\poemll    he did not retaliate. \\
\poeml When he suffered, \\
\poemll    he did not threaten. \\
\poeml It was his habit \\
\poemll    to commit the matter to the one who judges fairly. \\
\poeml \v{24}``He himself bore our sins''\fnote{Cf. Isa 53:4,12} in his body on the tree, \\
\poemll    so that we might die to those sins \\
\poemlll       and live righteously. \\
\poeml ``By his wounds \\
\poemll    you have been healed.''\fnote{Cf. Isa 53:5} \\
\poeml \v{25}You were ``like sheep that kept going astray,''\fnote{Cf. Isa 53:6} \\
\poemll    but now you have returned to the shepherd \\
\poemlll       and overseer of your souls.
\end{poetry}
\labelchapt{3}
\passage{Wives and Husbands}

\chapt{3}
\v{1}In a similar way, you wives must submit yourselves to your husbands so that, even if some of them refuse to obey the word, they may be won over without a word through your conduct as wives \v{2}when they see your pure and reverent lives.

\v{3}Your beauty should not be an external one, consisting of braided hair or the wearing of gold ornaments and dresses. \v{4}Instead, it should be the inner disposition of the heart, consisting in the imperishable quality of a gentle and quiet spirit, which God values greatly.\fnote{Lit. \fbib{which is of great value before God}} \v{5}After all, this is how holy women who set their hope on God used to make themselves beautiful in the past. They submitted themselves to their husbands, \v{6}just as Sarah obeyed Abraham and called him lord. You have become her daughters by doing good and by not letting anything terrify you.

\v{7}In a similar way, you husbands must live with your wives in an understanding manner, as with a most delicate partner.\fnote{Lit. \fbib{with the weaker vessel}} Honor them as heirs with you of the gracious gift of life, so that nothing may interfere with your prayers.
\passage{When You are Wronged}

\v{8}Finally, all of you must live in harmony, be sympathetic, love as brothers, and be compassionate and humble. \v{9}Do not pay others back evil for evil or insult for insult. Instead, keep blessing them, because you were called to inherit a blessing.

\begin{poetry}
\poeml \v{10}``For the person who wants to love life \\
\poemll    and see good days \\
\poeml must keep his tongue from evil \\
\poemll    and his lips from speaking deceit. \\
\poeml \v{11}He must turn away from evil and do good. \\
\poemll    He must seek peace and pursue it. \\
\poeml \v{12}For the Lord\fnote{MT source citation reads \fbib{}\divine{Lord}} watches the righteous,\fnote{Lit. \fbib{the eyes of the Lord are on the righteous}} \\
\poemll    and he pays attention to their prayers.\fnote{Lit. \fbib{and his ears are attentive to their prayer}} \\
\poeml But the Lord\fnote{MT source citation reads \fbib{}\divine{Lord}} opposes those\fnote{Lit. \fbib{the face of the Lord is against}} who do wrong.''\fnote{Cf. Ps 34:12-16}
\end{poetry}

\v{13}Who will harm you if you are devoted to doing what is good? \v{14}But even if you should suffer for doing what is right, you are blessed. ``Never be afraid of their threats, and never get upset. \v{15}Instead, exalt\fnote{Or \fbib{set apart}} the Messiah''\fnote{Or \fbib{Christ}; cf. Isa 8:12} as Lord in your lives.\fnote{Lit. \fbib{hearts}} Always be prepared to give a defense to everyone who asks you to explain the hope you have. \v{16}But do this\fnote{The Gk. lacks \fbib{do this}} gently and respectfully, keeping a clear conscience, so that those who speak evil of your good conduct in the Messiah\fnote{Or \fbib{Christ}} will be ashamed of slandering you. \v{17}After all, if it is the will of God, it is better to suffer for doing right than for doing wrong.

\begin{poetry}
\poeml \v{18}For the Messiah\fnote{Or \fbib{Christ}} also suffered\fnote{Other mss. read \fbib{died}} for sins once for all, \\
\poemll    an innocent person for the guilty, \\
\poemlll       so that he could bring you\fnote{Other mss. read \fbib{us}} to God. \\
\poeml He was put to death in a mortal body \\
\poemll    but was brought to life by the Spirit,
\end{poetry}

\v{19}in which he went and made a proclamation to those imprisoned spirits \v{20}who disobeyed long ago in the days of Noah, when God waited patiently while the ark was being built. In it a few, that is, eight persons, were saved by water. \v{21}Baptism, which is symbolized by that water, now saves you also, not by removing dirt from the body, but by asking God for a clear\fnote{Lit. \fbib{clean}} conscience based on the resurrection of Jesus, the Messiah,\fnote{Or \fbib{Christ}} \v{22}who has gone to heaven and is at the right hand of God, where angels, authorities, and powers have been made subject to him.
\labelchapt{4}
\passage{Good Managers of God's Grace}

\chapt{4}
\v{1}Therefore, since the Messiah\fnote{Or \fbib{Christ}} suffered in a mortal body,\fnote{Other mss. read \fbib{suffered for us}; still other mss. read \fbib{suffered for you}} you, too, must arm yourselves with the same determination, because the person who has suffered in a mortal body has stopped sinning, \v{2}so that he can live the rest of his mortal life\fnote{Lit. \fbib{time}} guided, not by human desires, but by the will of God. \v{3}For you spent enough time in the past doing what the gentiles like to do, living in sensuality, sinful desires, drunkenness, wild celebrations, drinking parties, and detestable idolatry. \v{4}They insult you now because they are surprised that you are no longer joining them in the same excesses of wild living. \v{5}They will give an account to the one who is ready to judge the living and the dead. \v{6}Indeed, this is why the gospel was proclaimed even to those who have died, so that they could be judged in their mortal flesh like all humans and live in the spiritual realm like God.

\v{7}Because everything will soon come to an end, be sensible and clear-headed, so you can pray. \v{8}Above all, continue to love each other deeply, because love covers a multitude of sins. \v{9}Show hospitality to one another without complaining. \v{10}As good servant managers of God's grace in its various forms, serve one another with the gift each of you has received. \v{11}Whoever speaks must speak God's words.\fnote{Lit. \fbib{If anyone speaks as the words of God}} Whoever serves must serve with the strength\fnote{Lit. \fbib{Whoever serves as with the strength}} that God supplies, so that in every way God may be glorified through Jesus, the Messiah.\fnote{Or \fbib{Christ}} Glory and power belong to him forever and ever! Amen.
\passage{Suffering as a Christian}

\v{12}Dear friends, do not be surprised by the fiery ordeal that is taking place among you to test you, as though something strange were happening to you. \v{13}Instead, because you are participating in the sufferings of the Messiah,\fnote{Or \fbib{Christ}} keep on rejoicing, so that you may be glad and shout for joy when his glory is revealed. \v{14}If you are insulted because of the name of the Messiah,\fnote{Or \fbib{Christ}} you are blessed, for the glorious Spirit of God is resting on you.\fnote{Other mss. read \fbib{on you. For their sake he is being blasphemed, but for your sake he is being glorified.}}

\v{15}Of course, none of you should suffer for being a murderer, thief, criminal, or troublemaker. \v{16}But if you suffer for being a Christian, do not feel ashamed, but glorify God with that name. \v{17}For the time has come for judgment to begin with the household of God. And if it begins with us, what will be the outcome for those who refuse to obey the gospel of God?

\begin{poetry}
\poeml \v{18}``If it is hard for the righteous person to be saved, \\
\poemll    what will happen to the ungodly and sinful person?''\fnote{Cf. Prov 11:31 (LXX)}
\end{poetry}

\v{19}So then, those who suffer according to God's will should entrust their souls to their faithful Creator and continue to do what is good.
\labelchapt{5}
\passage{Be Shepherds of God's Flock}

\chapt{5}
\v{1}Therefore, as a fellow elder, a witness of the Messiah's\fnote{Or \fbib{Christ's}} sufferings, and one who shares in the glory to be revealed, I appeal to the elders among you: \v{2}Be shepherds of God's flock that is among you, watching over it, not because you must but because you want to, and not greedily but eagerly, as God desires. \v{3}Do not lord it over the people entrusted to you, but be examples to the flock. \v{4}Then, when the Chief Shepherd appears, you will receive the victor's crown of glory that will never fade away.
\passage{Be Humble and Alert}

\v{5}In a similar way, you young people must submit to the elders.\fnote{Or \fbib{to those who are older}} All of you must clothe yourselves with humility for the sake of each other, because:

\begin{poetry}
\poeml ``God opposes the arrogant, \\
\poemll    but gives grace to the humble.''\fnote{Cf. Prov 3:34 (LXX)}
\end{poetry}

\v{6}Therefore, humble yourselves under the mighty hand of God, so that at the proper time he may exalt you. \v{7}Throw all your worry on him, because he cares for you. \v{8}Be clear-minded and alert. Your opponent, the devil, is prowling around like a roaring lion, looking for someone to devour. \v{9}Resist him and be firm in the faith, because you know that your brothers throughout the world are undergoing the same kinds of suffering. \v{10}After you have suffered for a little while, the God of all grace, who called you by the Messiah\fnote{Or \fbib{Christ}} Jesus\fnote{Other mss. lack \fbib{Jesus}} to his eternal glory, will restore you, establish you, strengthen you, and support you. \v{11}Power belongs\fnote{Other mss. read \fbib{Glory and power belong}} to him forever and ever! Amen.
\passage{Final Greeting}

\v{12}Through Silvanus, whom I regard as a faithful brother, I have written this short letter to encourage you and to testify that this is to be the true grace of God in which you are to stand firm! \v{13}Your sister church\fnote{Lit. \fbib{She who is}} in Babylon, chosen along with you, sends you greetings, as does Mark, whom I regard as a son. \v{14}Greet one another with a loving kiss. Peace be to all of you who are in the Messiah!\fnote{Or \fbib{Christ}; other mss. read \fbib{the Messiah Jesus! Amen}}

\bookheader{2 Peter}
\labelbook{2Pet}

\bookpretitle{The Letter Called}
\booktitle{Second Peter}

\labelchapt{1}
\passage{Greetings}

\chapt{1}
\v{1}From:\fnote{The Gk. lacks \fbib{From}} Simeon\fnote{Other mss. read \fbib{Simon}} Peter, a servant\fnote{Or \fbib{slave}} and apostle of Jesus, the Messiah.\fnote{Or \fbib{Christ}}

To: Those who have received faith that is as valuable as ours through the righteousness of our God and Savior, Jesus the Messiah.\fnote{Or \fbib{Christ}}

\v{2}May grace and peace be yours in abundance through full knowledge of God and of Jesus our Lord!
\passage{We are Called to Holy Living}

\v{3}His divine power has given us everything we need for life and godliness through the full knowledge of the one who called us by his own glory and excellence. \v{4}Through these he has given us his precious and wonderful promises, so that through them you may participate in the divine nature, seeing that you have escaped the corruption that is in the world caused by evil desires. \v{5}For this very reason, you must make every effort to supplement your faith with moral character, your moral character with knowledge, \v{6}your knowledge with self-control, your self-control with endurance, your endurance with godliness, \v{7}your godliness with brotherly kindness, and your brotherly kindness with love. \v{8}For if you possess these qualities, and if they continue to increase among you, they will keep you from being ineffective and unproductive in attaining a full knowledge of our Lord Jesus, the Messiah.\fnote{Or \fbib{Christ}} \v{9}For the person who lacks these qualities is blind and shortsighted, and has forgotten the cleansing that he has received from his past sins.

\v{10}So then, my brothers, be all the more eager to make your calling and election certain, for if you keep on doing this you will never fail. \v{11}For in this way you will be generously granted entry into the eternal kingdom of our Lord and Savior Jesus, the Messiah.\fnote{Or \fbib{Christ}}

\v{12}Therefore, I intend to keep on reminding you about these things, even though you already know them and are firmly established in the truth that you now have. \v{13}Yet I think it is right to refresh your memory as long as I am living in this bodily tent, \v{14}because I know that the removal of my bodily tent will come soon, as indeed our Lord Jesus, the Messiah,\fnote{Or \fbib{Christ}} has shown me. \v{15}And I will make every effort to see that you will always remember these things after I am gone.
\passage{Pay Attention to God's Word}

\v{16}When we told you about the power and coming of our Lord Jesus, the Messiah,\fnote{Or \fbib{Christ}} we did not follow any clever myths. Rather, we were eyewitnesses of his majesty. \v{17}For he received honor and glory from God the Father when these words from the Majestic Glory were spoken about him: ``This is my Son, whom I love. I am pleased with him.'' \v{18}We ourselves heard this voice that came from heaven when we were with him on the holy mountain. \v{19}Therefore we regard the message of the prophets as confirmed beyond doubt, and you will do well to pay attention to it, as to a lamp that is shining in a gloomy place, until the day dawns and the morning star rises in your hearts. \v{20}First of all, you must understand this: No prophecy in Scripture is a matter of one's own interpretation, \v{21}because no prophecy ever originated through a human decision. Instead, men spoke from God as they were carried along by the Holy Spirit.
\labelchapt{2}
\passage{Warning against False Teachers}

\chapt{2}
\v{1}Now there were false prophets among the people, just as there also will be false teachers among you, who will secretly introduce destructive heresies and even deny the Master who bought them, bringing swift destruction on themselves. \v{2}Many people will follow their immoral ways, and because of them the way of truth will be maligned.\fnote{Or \fbib{blasphemed}} \v{3}In their greed they will exploit you with deceptive words. The ancient verdict against them is still in force, and their destruction is not delayed.\fnote{Lit. \fbib{asleep}}

\v{4}For if God did not spare angels when they sinned, but threw them into the lowest hell\fnote{Gk. \fbib{Tartarus}; a reference to the realm of the dead} and imprisoned them in chains\fnote{Other mss. read \fbib{pits}} of deepest darkness, holding them for judgment; \v{5}and if he did not spare the ancient world but protected Noah, a righteous preacher, along with seven others when he brought the flood on the world of ungodly people; \v{6}and if he condemned the cities of Sodom and Gomorrah and destroyed them by burning them to ashes, making them an example to ungodly people of what is going to happen to them; \v{7}and if he rescued Lot, a righteous man who was greatly distressed by the immoral conduct of lawless people--- \v{8}for as long as that righteous man lived among them, day after day he was being tortured in his righteous soul by what he saw and heard in their lawless actions--- \v{9}then the Lord knows how to rescue godly people from their trials and to hold unrighteous people for punishment on the day of judgment, \v{10}especially those who satisfy their flesh by indulging in its passions and who despise authority.

Being bold and arrogant, they are not afraid to slander glorious beings. \v{11}Yet even angels, although they are greater in strength and power, do not bring a slanderous accusation against them from the Lord. \v{12}These people, like irrational animals, are mere creatures of instinct that are born to be caught and killed. They insult what they don't understand, and like animals they, too, will be destroyed, \v{13}suffering harm as punishment for their wrongdoing. They take pleasure in wild parties in broad daylight. They are stains and blemishes, reveling in their deceitful pleasures\fnote{Other mss. read \fbib{in their love feasts}} while they eat with you. \v{14}With eyes full of adultery, they cannot get enough of sin. They seduce unsteady souls and have had their hearts expertly trained in greed. They are doomed to a curse.\fnote{Lit. \fbib{children of a curse}} \v{15}They have left the straight path and wandered off to follow the path of Bosor's\fnote{Other mss. read \fbib{Beor's}} son Balaam, who loved the reward he got for doing wrong. \v{16}But he was rebuked for his offense. A donkey that normally cannot talk spoke with a human voice and restrained the prophet's insanity.

\v{17}These men are dried-up springs, mere clouds driven by a storm. Gloomy darkness is reserved for them. \v{18}By talking high-sounding nonsense and using sinful cravings of the flesh, they entice people who have just escaped from those who live in error. \v{19}Promising them freedom, they themselves are slaves to depravity, for a person is a slave to whatever conquers him.

\v{20}For if, after escaping the world's corruptions through a full knowledge of our Lord and Savior Jesus, the Messiah,\fnote{Or \fbib{Christ}} they are again entangled and conquered by those corruptions,\fnote{Lit. \fbib{by them}} then their last condition is worse than their former one. \v{21}It would have been better for them not to have known the way of righteousness than to know it and turn their backs on the holy commandment that was committed to them. \v{22}The proverb is true that describes what has happened to them: ``A dog returns to its vomit,''\fnote{Cf. Prov 26:11} and ``A pig that is washed goes back to wallow in the mud.''\fnote{The source of this quotation appears to be the Syrian \fbib{Story of Ahikar}.}
\labelchapt{3}
\passage{Be Ready for the Day of the Lord}

\chapt{3}
\v{1}Dear friends, this is now the second of two letters\fnote{Lit. \fbib{second letter}} I am writing to you, in which I have been trying to stimulate your pure minds by reminding you \v{2}to recall the words spoken in the past by the holy prophets and the commandment of our Lord and Savior spoken\fnote{The Gk. lacks \fbib{spoken}} through your apostles.

\v{3}First of all you must understand this: In the last days mockers will come and, following their own desires, will ridicule us\fnote{Cf. Jude 18} \v{4}by saying, ``What happened to the Messiah's\fnote{Lit. \fbib{to his}} promise to return? Ever since our ancestors died,\fnote{Lit. \fbib{fell asleep}} everything continues as it did from the beginning of creation.'' \v{5}But they deliberately ignore the fact that long ago the heavens existed and the earth was formed by God's word out of water and with water, \v{6}by which the world at that time was deluged with water and destroyed. \v{7}Now by that same word, the present heavens and earth have been reserved for fire and are being kept for the day when ungodly people will be judged and destroyed.

\v{8}Don't forget this fact, dear friends: With the Lord a single day is like a thousand years, and a thousand years are like a single day. \v{9}The Lord is not slow about his promise, as some people understand slowness, but is being patient with you. He does not want anyone to perish, but wants\fnote{The Gk. lacks \fbib{wants}} everyone to repent. \v{10}But the Day of the Lord will come like a thief. On that day\fnote{Lit. \fbib{On it}} the heavens will disappear with a roaring sound, the elements will be destroyed by fire, and the earth and everything done on it will be exposed.

\v{11}Since everything will be destroyed in this way, think of the kind of holy and godly people you ought to be \v{12}as you look forward to and hasten the coming of the day of God, when the heavens will be set ablaze and dissolved and the elements will melt with fire. \v{13}But in keeping with his promise, we are looking forward to new heavens and a new earth, where righteousness is at home.

\v{14}So then, dear friends, since you are looking forward to this, make every effort to have the Lord\fnote{Lit. \fbib{have him}} find you at peace and without spot or fault. \v{15}Think of our Lord's patience as facilitating salvation, just as our dear brother Paul also wrote to you according to the wisdom given him. \v{16}He speaks about this subject in all his letters. Some things in them are hard to understand, which ignorant and unstable people distort, leading to their own destruction, as they do the rest of the Scriptures.

\v{17}And so, dear friends, since you already know these things, continuously be on your guard not to be carried away by the deception of lawless people. Otherwise, you may\fnote{Lit. \fbib{people, lest you}} fall from your secure position. \v{18}Instead, continue to grow in the grace and knowledge of our Lord and Savior Jesus, the Messiah.\fnote{Or \fbib{Christ}} Glory belongs to him both now and on that eternal day! Amen.\fnote{Other mss. lack \fbib{Amen}}

\bookheader{1 John}
\labelbook{1John}

\bookpretitle{The Letter Called}
\booktitle{First John}

\labelchapt{1}
\passage{Jesus, the Word of Life}

\chapt{1}
\v{1}What existed from the beginning, what we have heard, what we have seen with our eyes, what we observed and touched with our own hands---this is the\fnote{Lit. \fbib{about the}} Word of life! \v{2}This life was revealed to us, and we have seen it and testify about it. We declare to you this eternal life that was with the Father and was revealed to us. \v{3}What we have seen and heard we declare to you so that you, too, can have fellowship with us. Now this fellowship of ours is with the Father and with his Son, Jesus, the Messiah.\fnote{Or \fbib{Christ}} \v{4}We are writing these things\fnote{Other mss. read \fbib{these things to you}} so that our\fnote{Other mss. read \fbib{your}} joy may be full.
\passage{Living in the Light}

\v{5}This is the message that we have heard from him and declare to you: God is light, and in him there is no darkness---none at all! \v{6}If we claim that we have fellowship with him but keep living in darkness, we are lying and not practicing the truth. \v{7}But if we keep living in the light as he himself is in the light, we have fellowship with one another, and the blood of Jesus his Son cleanses us from all sin. \v{8}If we say that we do not have any sin, we are deceiving ourselves and we're not being truthful to ourselves. \v{9}If we make it our habit to confess our sins, in his faithful righteousness he forgives us for those sins and cleanses us from all unrighteousness. \v{10}If we say that we have never sinned, we make him a liar and his word has no place in us.
\labelchapt{2}
\passage{The Messiah is Our Advocate}

\chapt{2}
\v{1}My little children, I'm writing these things to you so that you might not sin. Yet if anyone does sin, we have an advocate with the Father---Jesus, the Messiah,\fnote{Or \fbib{Christ}} one who is righteous. \v{2}It is he who is the atoning sacrifice for our sins, and not for ours only, but also for the whole world's.

\v{3}This is how we can be sure that we have come to know him: if we continually keep his commandments. \v{4}The person who says, ``I have come to know him,'' but does not continually keep his commandments is a liar, and the truth has no place in that person. \v{5}But whoever continually keeps his commandments is the kind of person in whom God's love has truly been perfected. This is how we can be sure that we are in union with God:\fnote{Lit. \fbib{him}} \v{6}The one who says that he abides in him must live the same way he himself lived.
\passage{We Must Obey God's Commandments}

\v{7}Dear friends, I am not writing to you a new commandment, but an old commandment that you have had from the beginning. This old commandment is the word you have heard. \v{8}On the other hand, I am writing to you a new commandment that is truly in him and in you. For the darkness is fading away, and the true light is already shining.

\v{9}The person who says that he is in the light but hates his brother is still in the darkness. \v{10}The person who loves his brother abides in the light, and there is no reason for him to stumble. \v{11}But the person who hates his brother is in the darkness and lives in the darkness. He does not know where he is going, because the darkness has blinded his eyes.

\begin{poetry}
\poeml \v{12}I am writing to you, little children, \\
\poemll    because your sins have been forgiven \\
\poemlll       on account of his name. \\
\poeml \v{13}I am writing to you, fathers, \\
\poemll    because you have known the one who \\
\poemlll       has existed from the beginning. \\
\poeml I am writing to you, young people, \\
\poemll    because you have overcome the evil one. \\
\poeml \v{14}I have written to you, little children, \\
\poemll    because you have known the Father. \\
\poeml I have written to you, fathers, \\
\poemll    because you have known the one who \\
\poemlll       has existed from the beginning. \\
\poeml I have written to you, young people, \\
\poemll    because you are strong \\
\poemll    and because God's word remains in you \\
\poemlll       and you have overcome the evil one.
\end{poetry}

\v{15}Stop loving\fnote{Or \fbib{Don't love}} the world and the things that are in the world. If anyone persists in loving the world, the Father's love is not in him. \v{16}For everything that is in the world---the desire for fleshly gratification,\fnote{Lit. \fbib{for the flesh}} the desire for possessions,\fnote{Lit. \fbib{of the eyes}} and worldly arrogance---is not from the Father but is from the world. \v{17}And the world and its desires are fading away, but the person who does God's will remains forever.
\passage{Live in the Messiah}

\v{18}Little children, it is the last hour. Just as you heard that an antichrist is coming, so now many antichrists have appeared. This is how we know it is the last hour. \v{19}They left us, but they were not part of us, for if they had been part of us, they would have stayed with us. Their leaving made it clear that none of them was really part of us.

\v{20}You have an anointing from the Holy One and know all things.\fnote{Other mss. read \fbib{and all of you know}} \v{21}I have not written to you because you do not know the truth, but because you do know it and because lies don't come from truth. \v{22}Who is a liar but the person who denies that Jesus is the Messiah?\fnote{Or \fbib{Christ}} The person who denies the Father and the Son is an antichrist. \v{23}No one who denies the Son has the Father. The person who acknowledges the Son also has the Father.

\v{24}What you have heard from the beginning must abide in you. If what you have heard from the beginning abides in you, you will also abide in the Son and in the Father. \v{25}The message that the Son\fnote{Lit. \fbib{that he}} himself declared to us is eternal life. \v{26}I have written\fnote{Lit. \fbib{written these things}} to you about those who are trying to deceive you. \v{27}The anointing you received from God\fnote{Lit. \fbib{him}} abides in you, and you do not need anyone to teach you this.\fnote{The Gk. lacks \fbib{this}} Instead, because God's\fnote{Lit. \fbib{his}} anointing teaches you about everything and is true and not a lie, abide in him, as he taught you to do.\fnote{The Gk. lacks \fbib{to do}}
\passage{Abide in Him}

\v{28}Even now, little children, abide in him. Then, when he appears, we will have confidence and will not turn away from him in shame when he comes. \v{29}Since you know that he is righteous, you also know that everyone who practices righteousness has been fathered by God.\fnote{Lit. \fbib{by him}}
\labelchapt{3}
\passage{We are God's Children}

\chapt{3}
\v{1}See what kind of love the Father has given us: We are called God's children---and that is what we are!\fnote{Other mss. lack \fbib{And that is what we are!}} For this reason the world does not recognize us, because it did not recognize him, either.

\v{2}Dear friends, we are now God's children, but what we will be like has not been revealed yet. We know that when the Messiah\fnote{Lit. \fbib{he}} is revealed, we will be like him, because we will see him as he is. \v{3}And everyone who has this hope based on him keeps himself pure, just as the Messiah\fnote{Lit. \fbib{as he}} is pure. \v{4}Everyone who keeps living in sin also practices disobedience. In fact, sin is disobedience. \v{5}You know that the Messiah\fnote{Lit. \fbib{that he}} was revealed to take away sins,\fnote{Other mss. read \fbib{our sins}} and there is not any sin in him. \v{6}No one who remains in union with him keeps on sinning. The one who keeps on sinning hasn't seen him or known him.

\v{7}Little children, don't let anyone deceive you. The person who practices righteousness is righteous, just as the Messiah\fnote{Lir. \fbib{as he}} is righteous. \v{8}The person who practices sin belongs to the evil one, because the devil has been sinning from the beginning. The reason that the Son of God was revealed was to destroy what the devil has been doing. \v{9}No one who has been born from God practices sin, because God's\fnote{Lit. \fbib{his}} seed abides in him. Indeed, he cannot go on sinning, because he has been born from God. \v{10}This is how God's children and the devil's children are distinguished.\fnote{Lit. \fbib{are revealed}} No person who fails to practice righteousness and to love his brother is from God.
\passage{Love One Another}

\v{11}This is the message that you have heard from the beginning: We should love one another. \v{12}Do not be like Cain,\fnote{Lit. \fbib{Not like Cain}} who was from the evil one and murdered his brother. And why did he murder him? Because what he was doing was evil and his brother's actions\fnote{The Gk. lacks \fbib{actions}} were righteous. \v{13}So do not be surprised, brothers, if the world hates you.

\v{14}We know that we have passed from death to life, because we love one another. The person who does not love\fnote{Other mss. read \fbib{doesn't love his brother}} remains spiritually\fnote{The Gk. lacks \fbib{spiritually}} dead. \v{15}Everyone who hates his brother is a murderer, and you know that no murderer has eternal life present in him. \v{16}This is how we have come to know love: the Messiah\fnote{Lit. \fbib{he}} gave his life for us. We, too, ought to give our lives for our brothers. \v{17}Whoever has earthly possessions and notices a brother in need and yet withholds his compassion from him, how can the love of God be present in him? \v{18}Little children, we must stop expressing love merely by our words and manner of speech; we must love\fnote{The Gk. lacks \fbib{love}} also in action\fnote{Or \fbib{work}} and in truth. \v{19}This is how we will know that we belong to the truth and how we will be able to keep ourselves\fnote{Lit. \fbib{keep our hearts}} strong in his presence.

\v{20}If our hearts condemn us, God is greater than our hearts and knows everything. \v{21}Dear friends, if our hearts do not condemn us, we have confidence in the presence of God. \v{22}Whatever we request we receive from him, because we keep his commandments and do what pleases him. \v{23}And this is his commandment: to believe in the name of his Son, Jesus the Messiah,\fnote{Or \fbib{Christ}} and to love one another as he commanded us. \v{24}The person who keeps his commandments abides in God,\fnote{Lit. \fbib{in him}} and God abides in him.\fnote{Lit. \fbib{and he in him}} This is how we can be sure that he remains in us: he has given us his Spirit.
\labelchapt{4}
\passage{Test What People Say}

\chapt{4}
\v{1}Dear friends, stop believing\fnote{Or \fbib{do not believe}} every spirit. Instead, test the spirits to see whether they are from God, because many false prophets have gone out into the world.

\v{2}This is how you can recognize God's Spirit: Every spirit who acknowledges that Jesus the Messiah\fnote{Or \fbib{Christ}} has become human---and remains so---is from God. \v{3}But every spirit who does not acknowledge Jesus is not from God. This is the spirit of the antichrist. You have heard that he is coming, and now he is already in the world. \v{4}Little children, you belong to God and have overcome them, because the one who is in you is greater than the one who is in the world. \v{5}These people belong to the world. That is why they speak from the world's perspective,\fnote{Lit. \fbib{from the world}} and the world listens to them. \v{6}We belong to God. The person who knows God listens to us. Whoever does not belong to God does not listen to us. This is how we know the Spirit of truth and the spirit of deceit.
\passage{God's Love Lives in Us}

\v{7}Dear friends, let us continuously love one another, because love comes from God. Everyone who loves has been born from God and knows God. \v{8}The person who does not love does not know God, because God is love. \v{9}This is how God's love was revealed among us: God sent his unique Son into the world so that we might live through him. \v{10}This is love: not that we have loved\fnote{Other mss. read \fbib{we loved}} God, but that he loved us and sent his Son to be the atoning sacrifice for our sins. \v{11}Dear friends, if this is the way God loved us, we must also love one another. \v{12}No one has ever seen God. If we love one another, God lives in us, and his love is perfected in us. \v{13}This is how we know that we abide in him and he in us: he has given us his Spirit.

\v{14}We have seen for ourselves and can testify that the Father has sent his Son to be the Savior of the world. \v{15}God abides in the one who acknowledges that Jesus is the Son of God, and he abides in God. \v{16}We have come to know and rely on\fnote{Lit. \fbib{believe in}} the love that God has for us. God is love, and the person who abides in love abides in God, and God abides in him. \v{17}This is how love has been perfected among us: we will have confidence on the day of judgment because, during our time in this world, we are just like him. \v{18}There is no fear where love exists.\fnote{Lit. \fbib{in love}} Rather, perfect love banishes fear, for fear involves punishment, and the person who lives in fear has not been perfected in love.

\v{19}We love\fnote{Other mss. read \fbib{love him}; still other mss. read \fbib{love God}} because God\fnote{Lit. \fbib{he}} first loved us. \v{20}Whoever says, ``I love God,'' but hates his brother is a liar. The one who does not love his brother whom he has seen cannot love the God whom he has not seen. \v{21}And this is the commandment that we have from him: the person who loves God must also love his brother.
\labelchapt{5}
\passage{Faith Overcomes the World}

\chapt{5}
\v{1}Everyone who believes that Jesus is the Messiah\fnote{Or \fbib{Christ}} has been born from God, and everyone who loves the parent also loves the child. \v{2}This is how we know that we love God's children: we love God and keep his commandments. \v{3}For this demonstrates our love for God: We keep his commandments, and his commandments are not difficult, \v{4}because everyone who is born from God has overcome the world. Our faith is the victory that overcomes the world. \v{5}Who overcomes the world? Is it not the person who believes that Jesus is the Son of God?

\v{6}This man, Jesus the Messiah,\fnote{Or \fbib{Christ}} is the one who came by water and blood---not with water only, but with water and with blood. The Spirit is the one who verifies this, because the Spirit is the truth. \v{7}For there are three witnesses in heaven---the Father, the Word, and the Holy Spirit, and these three are one.\fnote{Other mss. lack \fbib{witnesses in heaven---the Father, the Word, and the Holy Spirit, and these three are one.} \fbib{\v{8}And there are three witnesses on earth---}} \v{8}And there are three witness on earth---the Spirit, the water, and the blood---and these three are one.

\v{9}If we accept human testimony, God's testimony is greater, because it is the testimony of God and because he has testified about his Son. \v{10}The person who believes in the Son of God believes this testimony personally.\fnote{Lit. \fbib{God has this testimony in himself}} The person who does not believe God\fnote{Other mss. read \fbib{the Son}} has made him a liar by not believing the testimony that he\fnote{Lit. \fbib{God}} has given about his Son.

\v{11}This is the testimony: God has given us eternal life, and this life is found in his Son. \v{12}The person who has the Son has this life. The person who does not have the Son of God does not have this life.
\passage{Conclusion}

\v{13}I have written these things to you who believe in the name of the Son of God so that you may know that you have eternal life. \v{14}And this is the confidence that we have in him: if we ask for anything according to his will, he listens to us. \v{15}And if we know that he listens to our requests, we can be sure that we have what we ask him for.

\v{16}If anyone sees his brother committing a sin that does not lead to death, he should pray that God\fnote{Lit. \fbib{he}} would give him life. This applies to those who commit sins that do not lead to death. There is a sin that leads to death. I am not telling you to pray about that. \v{17}Every kind of wrongdoing is sin, yet there are sins that do not lead to death.

\v{18}We know that the person who has been born from God does not go on sinning. Rather, the Son\fnote{Lit. \fbib{the one who has been born}} of God protects them, and the evil one cannot harm them. \v{19}We know that we are from God and that the whole world lies under the control of the evil one. \v{20}We also know that the Son of God has come and has given us understanding so that we may know the true God.\fnote{Other mss. read \fbib{the true one}} We are in union with the one who is true, his Son Jesus the Messiah,\fnote{Or \fbib{Christ}} who is the true God and eternal life.

\v{21}Little children, keep yourselves away from idols.\fnote{Other mss. read \fbib{idols. Amen}}

\bookheader{2 John}
\labelbook{2John}

\bookpretitle{The Letter Called}
\booktitle{Second John}

\passage{Greetings from John}

\v{1}From:\fnote{\fbackref{1} The Gk. lacks \fbib{From}} The Elder

To: The chosen lady and her children, whom I genuinely love, and not only I but also all who know the truth, \v{2}that is present in us and will be with us forever.

\v{3}Grace, mercy, and peace will be with us from God the Father and from Jesus\fnote{\fbackref{3} Other mss. read \fbib{the Lord Jesus}} the Messiah,\fnote{\fbackref{3} Or \fbib{Christ}} the Father's Son, in truth and love.
\passage{Living in the Truth}

\v{4}I was overjoyed to find some of your\fnote{\fbackref{4} Lit. \fbib{your} (sing.)} children living truthfully, just as the Father has commanded us. \v{5}Dear lady, I am now requesting of you\fnote{\fbackref{5} Lit. \fbib{you} (sing.)} that we all continue to love one another. It is not as though I am writing to give you\fnote{\fbackref{5} Lit. \fbib{you} (sing.)} a new commandment, but one that we have had from the beginning. \v{6}And this is what demonstrates\fnote{\fbackref{6} The Gk. lacks \fbib{what demonstrates}} love: that we live according to God's\fnote{\fbackref{6} Lit. \fbib{his}} commandments. Just as you\fnote{\fbackref{6} Lit. \fbib{you} (pl.)} have heard from the beginning what he commanded, you\fnote{\fbackref{6} Lit. \fbib{you} (pl.)} must live by it.
\passage{Reject False Teachers}

\v{7}For many deceivers have gone out into the world. They refuse to acknowledge Jesus the Messiah\fnote{\fbackref{7} Or \fbib{Christ}} as having become human. Any such person is a deceiver and an antichrist. \v{8}See\fnote{\fbackref{8} The Gk. verb is pl.} to it that you\fnote{\fbackref{8} Lit. \fbib{you} (pl.)} don't destroy what we have\fnote{\fbackref{8} Other mss. read \fbib{you have}} worked for, but that you\fnote{\fbackref{8} Lit. \fbib{you} (pl.)} receive your\fnote{\fbackref{8} Lit. \fbib{your} (pl.)} full reward. \v{9}Everyone who does not remain true to the teaching of the Messiah,\fnote{\fbackref{9} Or \fbib{Christ}} but goes beyond it, does not have God. The person who remains true to the teaching of the Messiah\fnote{\fbackref{9} Or \fbib{Christ}} has both the Father and the Son. \v{10}If anyone comes to you\fnote{\fbackref{10} Lit. \fbib{you} (pl.)} but does not present his teachings,\fnote{\fbackref{10} Lit. \fbib{this teaching}} do not receive\fnote{\fbackref{10} The Gk. verb is pl.} him into your house or even welcome\fnote{\fbackref{10} The Gk. verb is pl.} him, \v{11}because the one who welcomes him shares in his evil deeds.
\passage{Final Greeting}

\v{12}Although I have a great deal to write to you,\fnote{\fbackref{12} Lit. \fbib{you} (pl.)} I would prefer not to use paper and ink. Instead, I hope to come to you and talk face to face, so that our joy may be complete. \v{13}The children of your\fnote{\fbackref{13} Lit. \fbib{you} (sing.)} chosen sister greet you.\fnote{\fbackref{13} Other mss. read \fbib{you. Amen}}

\bookheader{3 John}
\labelbook{3John}

\bookpretitle{The Letter Called}
\booktitle{Third John}

\passage{Greetings from John}

\v{1}From:\fnote{The Gk. lacks \fbib{From}} The Elder

To: My dear friend Gaius, whom I genuinely love.
\passage{Encouragement for Gaius}

\v{2}Dear friend, I pray that you are doing well in every way and that you are healthy, just as your soul is healthy. \v{3}I was overjoyed when some brothers arrived and testified about your truthfulness and how you live according to the truth. \v{4}I have no greater joy than to hear that my children are living according to the truth.

\v{5}Dear friend, you are faithful in whatever you do for the brothers, especially when they are strangers. \v{6}They have testified before the church about your love. You will do well to send them on their way in a manner worthy of God. \v{7}After all, they went on their trip for the sake of the Name,\fnote{I.e. God} accepting no support from gentiles. \v{8}Therefore, we ought to support such people so that we can become genuine helpers with them.
\passage{Criticism of Diotrephes}

\v{9}I wrote a letter\fnote{Lit. \fbib{wrote something}} to the church, but Diotrephes, who loves to be in charge, will not recognize our authority.\fnote{The Gk. lacks \fbib{authority}} \v{10}For this reason, when I come I will call attention to what he is doing in spreading false charges against us. And not content with that, he refuses to receive the brothers. He even tries to stop those who want to accept them\fnote{The Gk. lacks \fbib{accept them}} and throws them out of the church.
\passage{Praise for Demetrius}

\v{11}Dear friend, do not imitate what is evil, but what is good. The person who does what is good is from God. The person who does what is evil has never seen God. \v{12}Demetrius has received a good report from everyone, including the truth itself. We, too, can testify to this report, and you know that our testimony is true.
\passage{Final Greeting}

\v{13}Although I have a great deal to write to you,\fnote{Lit. \fbib{you} (sing.)} I would rather not write with pen and ink. \v{14}Instead, I hope to see you\fnote{Lit. \fbib{you} (sing.)} soon and speak face to face.

\v{15}May peace be with you!\fnote{Lit. \fbib{you} (sing.)} Your friends greet you.\fnote{Lit. \fbib{you} (sing.)} Greet\fnote{The Gk. verb is sing.} each of our friends by name.

\bookheader{Jude}
\labelbook{Jude}

\bookpretitle{The Letter from}
\booktitle{Jude}

\passage{Greetings}

\v{1}From: Jude, a servant of Jesus the Messiah,\fnote{Or \fbib{Christ}} and yet a brother of James.

To: Those who have been called, who are loved\fnote{Other mss. read \fbib{sanctified}} by God the Father and kept safe by Jesus, the Messiah.\fnote{Or \fbib{Christ}}

\v{2}May mercy, peace, and love be yours in abundance!
\passage{Warning about False Teachers}

\v{3}Dear friends, although I was eager to write to you about the salvation we share, I found it necessary to write to you and urge you to continue your vigorous defense of the faith that was passed down to the saints once and for all. \v{4}For some people have slipped in among you unnoticed. They were written about long ago as being deserving of this condemnation because they are ungodly. They turn the grace of our God into uncontrollable lust and deny our only Master and Lord, Jesus the Messiah.\fnote{Or \fbib{Christ}}

\v{5}Now I want to remind you, even though you are fully aware of these things, that the Lord who once saved his people from the land of Egypt later destroyed those who did not believe. \v{6}He has also held in eternal chains those angels who did not keep their own position but abandoned their assigned place. They are held in deepest darkness for judgment on the great day.\fnote{I.e. the day of judgment} \v{7}Likewise, Sodom and Gomorrah and the cities near them, which like them committed sexual sins and pursued homosexual activities,\fnote{Lit. \fbib{pursued other flesh}} serve as an example of the punishment of eternal fire.

\v{8}In a similar way, these dreamers also defile their flesh, reject the Lord's authority,\fnote{Lit. \fbib{reject dominions}} and slander his glorious beings. \v{9}Even the archangel Michael, when he argued with the devil and fought over the body of Moses, did not dare to bring a slanderous accusation against him. Instead, he said, ``May the Lord rebuke you!''\fnote{This incident is possibly based on \fbib{The Assumption of Moses,} an apocryphal Jewish writing.} \v{10}Whatever these people do not understand, they slander. Like irrational animals, they are destroyed by the very things they know by instinct. \v{11}How terrible it will be for them! For they lived like Cain did\fnote{Lit. \fbib{they followed Cain's path}}, rushed headlong into Balaam's error to make a profit, and destroyed themselves, as happened\fnote{The Gk. lacks \fbib{as happened}} in Korah's rebellion. \v{12}These people are stains on your love feasts.\fnote{Some early Christians had a meal along with the Lord's Supper.} They feast with you without any sense of awe.\fnote{Or \fbib{without fear}} They are shepherds who care only for themselves. They are waterless clouds blown about by the winds. They are autumn trees that are fruitless, totally\fnote{Lit. \fbib{twice}} dead, and uprooted. \v{13}They are wild waves of the sea, churning up the foam of their own shame. They are wandering stars for whom the deepest darkness has been reserved forever.

\v{14}Enoch, in the seventh generation from Adam, prophesied about these people when he said,

\begin{poetry}
\poeml ``Look! The Lord has come with countless thousands of his holy ones. \v{15}He will judge all people and convict everyone of all the ungodly things that they have done in such an ungodly way, including all the harsh things that these ungodly sinners have said about him.''\fnote{Cf. 1 Enoch 1:9 (\fbib{The Apocrypha})}
\end{poetry}

\v{16}These people are complainers and faultfinders, following their own desires. They say arrogant things and flatter people in order to take advantage of them.
\passage{Advice to the Readers}

\v{17}But you, dear friends, must remember the statements and predictions of the apostles of our Lord Jesus, the Messiah.\fnote{Or \fbib{Christ}} \v{18}They kept telling you, ``In the last times there will be mockers, following their own ungodly desires.''\fnote{Cf. 2 Peter 3:3} \v{19}These are the people who cause divisions. They are worldly, devoid of the Spirit.

\v{20}But you, dear friends, must continue to build your most holy faith for your own benefit. Furthermore, continue to pray in the Holy Spirit. \v{21}Remain in God's love as you look for the mercy of our Lord Jesus the Messiah,\fnote{Or \fbib{Christ}} which brings eternal life. \v{22}Show mercy to those who have doubts. \v{23}Save others by snatching them from the fire. To others, show mercy with fear, hating even the clothes stained by their sinful lives.\fnote{Lit. \fbib{by their flesh}}
\passage{Final Prayer}

\v{24}Now to the one who is able to keep you from falling and to make you stand joyful and faultless in his glorious presence, \v{25}to the only God, our Savior, through Jesus the Messiah,\fnote{Or \fbib{Christ}} our Lord, be glory, majesty, power, and authority before all time and for all eternity! Amen.

\addcontentsline{toc}{chapter}{The Consummation of the Good News}
\bookheader{Revelation}
\labelbook{Rev}

\bookpretitle{The Book of}
\booktitle{The Revelation to John}

\labelchapt{1}
\passage{The Revelation of Jesus the Messiah to the Seven Churches}

\chapt{1}
\v{1}This is\fnote{\fbackref{1:1} The Gk. lacks \fbib{This is}} the revelation of Jesus the Messiah,\fnote{\fbackref{1:1} Or \fbib{Christ}} which God gave him to show his servants the things that must happen soon. He made it known by sending his messenger\fnote{\fbackref{1:1} Or \fbib{angel}} to his servant John, \v{2}who testified about this message from God and the testimony about Jesus the Messiah.\fnote{\fbackref{1:2} Or \fbib{Christ}} \v{3}How blessed is the one who reads aloud and those who hear the words of this prophecy and obey what is written in it, for the time is near!

\v{4}From\fnote{\fbackref{1:4} The Gk. lacks \fbib{From}} John to the seven churches in Asia. May grace and peace be yours from the one who is, who was, and who is coming, from the seven spirits who are in front of his throne, \v{5}and from Jesus the Messiah,\fnote{\fbackref{1:5} Or \fbib{Christ}} the witness, the faithful one,\fnote{\fbackref{1:5} Or \fbib{Jesus the Messiah, the faithful witness}} the firstborn from the dead, and the ruler over the kings of the earth. To the one who loves us and has freed\fnote{\fbackref{1:5} Other mss. read \fbib{has washed}} us from our sins by his blood \v{6}and has made us a kingdom, priests for his God and Father, be glory and power forever and ever! Amen.

\begin{poetry}
\poeml \v{7}Look! He is coming in the clouds. \\
\poemll    Every eye will see him, \\
\poemlll       even those who pierced him, \\
\poeml and all the tribes of the earth will mourn \\
\poemll    because of him.
\end{poetry}

So be it! Amen.

\v{8}\red{``I am the Alpha and the Omega,''} declares the Lord God,\red{ ``the one who is, who was, and who is coming, the Almighty.''}
\passage{John's Vision of the Messiah}

\v{9}I am John, your brother and partner in the oppression, kingdom, and patience that comes because of Jesus. I was on the island called Patmos because of the word of God and the testimony about Jesus. \v{10}I came to be in the Spirit on the Day of the Lord, when I heard a loud voice behind me like a trumpet, \v{11}saying, \red{``Write on a scroll what you see, and send it to the seven churches: }\red{Ephesus}\red{, }\red{Smyrna}\red{, }\red{Pergamum}\red{, Thyatira, }\red{Sardis}\red{, }\red{Philadelphia}\red{, and }\red{Laodicea}\red{.''}

\v{12}Then I turned to see who\fnote{\fbackref{1:12} Lit. \fbib{see the voice that}} was talking to me, and when I turned I saw seven gold lamp stands. \v{13}Among the lamp stands there was someone like the Son of Man. He was wearing a long robe with a gold sash around his chest.\fnote{\fbackref{1:13} Or \fbib{waist}} \v{14}His head and his hair were white like wool, in fact, as white as snow. His eyes were like flames of fire, \v{15}his feet were like glowing bronze refined in a furnace, and his voice was like the sound of raging waters. \v{16}In his right hand he held seven stars, and out of his mouth came a sharp, two-edged sword. His face was like the sun when it shines with full force.

\v{17}When I saw him, I fell down at his feet like a dead man. But he placed his right hand on me and said, \red{``Stop being afraid! I am the first and the last,} \v{18}\red{the living one. I was dead}\red{---}\red{but look}\red{!}\red{---I am alive forever and ever! I have the keys of Death and Hades.}\fnote{\fbackref{1:18} I.e. the realm of the dead} \v{19}\red{Therefore, write down what you have seen, what is, and what is going to happen after this.} \v{20}\red{The secret meaning of the seven stars that you saw in my right hand and the seven gold lamp stands is this: the seven stars are the messengers}\fnote{\fbackref{1:20} Or \fbib{angels}} \red{of the seven churches, and the seven lamp stands are the seven churches.''}
\labelchapt{2}
\passage{The Letter to the Church in Ephesus}

\chapt{2}
\v{1}\red{``To the messenger}\fnote{\fbackref{2:1} Or \fbib{angel}} \red{of the church in }\red{Ephesus}\red{, write:}

\red{`The one who holds the seven stars in his right hand, the one who walks among the seven gold lamp stands, says this:}

\v{2}\red{`I know what you've been doing, your toil, and your endurance. I also know that you cannot tolerate evil people. You have tested those who call themselves apostles, but are not, and have found them to be false.} \v{3}\red{You have endured and suffered because of my name, yet you have not grown weary.} \v{4}\red{However, I have this against you: You have abandoned the love you had at first.} \v{5}\red{Therefore, remember how far you have fallen. Repent and go back to what you were doing at first. If you don't, I will come to you and remove your lamp stand from its place---unless you repent.} \v{6}\red{But this is to your credit: You hate the actions of the Nicolaitans, which I also hate.}

\v{7}\red{`Let everyone}\fnote{\fbackref{2:7} Lit. \fbib{Let the one who has an ear}}\red{ listen to what the Spirit says to the churches. To everyone who conquers I will give the privilege of eating from the tree of life that is in God's paradise.'\,''}
\passage{The Letter to the Church in Smyrna}

\v{8}\red{``To the messenger}\fnote{\fbackref{2:8} Or \fbib{angel}} \red{of the church in }\red{Smyrna}\red{, write:}

\red{`The first and the last, who was dead and became alive, says this:}

\v{9}\red{`I know your suffering and your poverty---though you are rich---and the slander committed by those who claim to be Jews but are not. They are the synagogue of Satan.} \v{10}\red{Don't be afraid of what you are going to suffer. Look! The devil is going to throw some of you into prison so that you may be tested. For ten days you will undergo suffering. Be faithful until death, and I will give you the victor's crown of life.}

\v{11}\red{`Let everyone}\fnote{\fbackref{2:11} Lit. \fbib{Let the one who has an ear}}\red{ listen to what the Spirit says to the churches. The one who conquers will never be hurt by the second death.'\,''}
\passage{The Letter to the Church in Pergamum}

\v{12}\red{``To the messenger}\fnote{\fbackref{2:12} Or \fbib{angel}} \red{of the church in }\red{Pergamum}\red{, write:}

\red{`The one who holds the sharp, two-edged sword, says this:}

\v{13}\red{`I know where you live. Satan's throne is there. Yet you hold on to my name and have not denied your faith in me,}\fnote{\fbackref{2:13} Or \fbib{my faith}} \red{even in the days of Antipas, my faithful witness, who was killed in your presence, where Satan lives.} \v{14}\red{But I have a few things against you: You have there some who hold to the teaching of Balaam, the one who taught Balak to put a stumbling block before the people of }\red{Israel}\red{ so that they would eat food sacrificed to idols and practice immorality.} \v{15}\red{You also have some who hold to the teaching of the Nicolaitans.} \v{16}\red{So repent. If you don't, I will come to you quickly and wage war against them with the sword of my mouth.}

\v{17}\red{`Let everyone}\fnote{\fbackref{2:17} Lit. \fbib{Let the one who has an ear}}\red{ listen to what the Spirit says to the churches. To the one who conquers I will give some of the hidden manna. I will also give him a white stone. On the white stone is written a new name that no one knows except the person who receives it.'\,''}
\passage{The Letter to the Church in Thyatira}

\v{18}\red{``To the messenger}\fnote{\fbackref{2:18} Or \fbib{angel}} \red{of the church in Thyatira, write:}

\red{`The Son of God, whose eyes are like flaming fire and whose feet are like glowing bronze, says this:}

\v{19}\red{`I know what you've been doing---your love, faithfulness,}\fnote{\fbackref{2:19} Or \fbib{faith}} \red{service, and endurance---and that your last actions are greater than the first.} \v{20}\red{But I have this against you: You tolerate that woman Jezebel, who calls herself a prophet and who teaches and leads my servants to practice immorality and to eat food sacrificed to idols.} \v{21}\red{I gave her time to repent, but she refused to repent of her immorality.} \v{22}\red{Look! I am going to strike her with illness.}\fnote{\fbackref{2:22} Lit. \fbib{to throw her into a bed}}\red{ Those who commit adultery with her will also suffer greatly, unless they repent from acting like her. }\v{23}\red{I will strike her children dead. Then all the churches will know that I am the one who searches minds and hearts. I will reward each of you as your actions deserve.}

\v{24}\red{`But as for the rest of you in Thyatira---you who do not hold to this teaching and who have not learned what some people call the deep things of Satan---I won't burden you with anything else.} \v{25}\red{Just hold on to what you have until I come.} \v{26}\red{To the person who conquers and continues to do what I've commanded}\fnote{\fbackref{2:26} Lit. \fbib{do my works}}\red{ to the end, I will give authority over the nations.}

\begin{poetry}
\poeml \v{27}\red{`He will rule them with an iron scepter;} \\
\poemll    \red{shattering them like clay pots.'}\fnote{\fbackref{2:27} Cf. Isa 30:14; Jer 19:11}
\end{poetry}

\v{28}\red{`Just as I have received authority from my Father, I will also give him the morning star.}

\v{29}\red{`Let everyone}\fnote{\fbackref{2:29} Lit. \fbib{Let the one who has an ear}}\red{ listen to what the Spirit says to the churches.'\,''}
\labelchapt{3}
\passage{The Letter to the Church in Sardis}

\chapt{3}
\v{1}\red{``To the messenger}\fnote{\fbackref{3:1} Or \fbib{angel}} \red{of the church in }\red{Sardis}\red{, write:}

\red{`The one who has the seven spirits of God and the seven stars says this:}

\red{`I know what you've been doing. You are known for being alive, but you are dead. }\v{2}\red{Be alert, and strengthen the things that are left, which are about to die. I note that your actions are incomplete before my God. }\v{3}\red{So remember what you received and heard. Obey it, and repent. If you are not alert, I will come like a thief, and you won't know the time when I will come to you.} \v{4}\red{But you have a few people in }\red{Sardis}\red{ who have not soiled their clothes. They will walk with me in white clothes because they are worthy.} \v{5}\red{The person who conquers in this way will wear white clothes, and I will never erase his name from the Book of Life. I will acknowledge his name in the presence of my Father and his angels.}

\v{6}\red{`Let everyone}\fnote{\fbackref{3:6} Lit. \fbib{Let the one who has an ear}}\red{ listen to what the Spirit says to the churches.'\,''}
\passage{The Letter to the Church in Philadelphia}

\v{7}\red{``To the messenger}\fnote{\fbackref{3:7} Or \fbib{angel}} \red{of the church in }\red{Philadelphia}\red{, write:}

\begin{poetry}
\poeml \red{`The one who is holy, who is true,} \\
\poemll    \red{who has the key of David,} \\
\poeml \red{who opens a door that}\fnote{\fbackref{3:7} Lit. \fbib{who opens and}} \red{no one can shut,} \\
\poemll    \red{and who shuts a door that}\fnote{\fbackref{3:7} Lit. \fbib{who shuts and}} \red{no one can open,}
\end{poetry}

\red{`says this:}

\v{8}\red{`I know what you've been doing. Look! I have put in front of you an open door that no one can shut. You have only a little strength, but you have obeyed my word and have not denied my name.} \v{9}\red{I will make those who belong to the }\red{synagogue of Satan---those who claim to be Jews and aren't, but are lying---come and bow down at your feet. Then they will realize that I have loved you.} \v{10}\red{Because} \red{you have obeyed my command to endure,}\fnote{\fbackref{3:10} Lit. \fbib{my word of endurance}} \red{I will keep you from the hour of testing that is coming to the whole world to test those living on the earth.} \v{11}\red{I am coming soon! Hold on to what you have so that no one takes your victor's crown.} \v{12}\red{I will make the one who conquers to become a pillar in the sanctuary of my God, and he will never go out of it again. I will write on him the name of my God, the name of the city of my God (the new Jerusalem coming down out of heaven from God), and my own new name.}

\v{13}\red{`Let everyone}\fnote{\fbackref{3:13} Lit. \fbib{Let the one who has an ear}}\red{ listen to what the Spirit says to the churches.'\,''}
\passage{The Letter to the Church in Laodicea}

\v{14}\red{``To the messenger}\fnote{\fbackref{3:14} Or \fbib{angel}} \red{of the church in }\red{Laodicea}\red{, write:}

\red{`The Amen, the witness who is faithful and true, the originator} \red{of God's creation, says this:}

\v{15}\red{`I know your actions, that you are neither cold nor hot. I wish you were cold or hot.} \v{16}\red{Since you are lukewarm and neither hot nor cold, I am going to spit you out of my mouth. }\v{17}\red{You say, ``I am rich. I have become wealthy. I don't need anything.'' Yet you don't realize that you are miserable, pitiful, poor, blind, and naked.} \v{18}\red{Therefore, I advise you to buy from me gold purified in fire so you may be rich, white clothes to wear so your shameful nakedness won't show, and ointment to put on your eyes so you may see.} \v{19}\red{I correct and discipline those whom I love, so be serious and repent!} \v{20}\red{Look! I am standing at the door and knocking. If anyone listens to my voice and opens the door, I will come in to him and eat with him, and he will eat}\fnote{\fbackref{3:20} The Gk. lacks \fbib{will eat}} \red{with me.} \v{21}\red{To the one who conquers I will give a place to sit with me on my throne, just as I have conquered and have sat down with my Father on his throne.}

\v{22}\red{`Let everyone}\fnote{\fbackref{3:22} Lit. \fbib{Let the one who has an ear}}\red{ listen to what the Spirit says to the churches.'\,''}
\labelchapt{4}
\passage{The Vision of God's Throne}

\chapt{4}
\v{1}After these things, I saw a door standing open in heaven. The first voice that I had heard speaking to me like a trumpet said, \red{``Come up here, and I will show you what must happen after this.''} \v{2}Instantly I was in the Spirit, and I saw a throne in heaven with a person seated on the throne. \v{3}The person sitting there looked like jasper and carnelian, and there was a rainbow around the throne that looked like an emerald. \v{4}Around the throne were 24 other thrones, and on these thrones sat 24 elders wearing white robes and gold victor's crowns on their heads. \v{5}Flashes of lightning, noises, and peals of thunder came from the throne. Burning in front of the throne were seven flaming torches, which are the seven spirits of God.

\v{6}In front of the throne was something like a sea of glass as clear as crystal. In the center of the throne and on each side of the throne were four living creatures full of eyes in front and in back. \v{7}The first living creature was like a lion, the second living creature was like an ox, the third living creature had a face like a human, and the fourth living creature was like a flying eagle. \v{8}Each of the four living creatures had six wings and were full of eyes inside and out. Without stopping day or night they were saying,

\begin{poetry}
\poeml ``Holy, holy, holy \\
\poemll    is the Lord God Almighty, \\
\poemlll       who was, who is, and who is coming.''
\end{poetry}

\v{9}Whenever the living creatures give glory, honor, and thanks to the one who sits on the throne, who lives forever and ever, \v{10}the 24 elders bow down and worship in front of the one who sits on the throne, the one who lives forever and ever. They throw their victor's crowns in front of the throne and say,

\begin{poetry}
\poeml \v{11}``You are worthy, our Lord and God, \\
\poemll    to receive glory, honor, and power, \\
\poemlll       because you created all things; \\
\poeml they came into existence \\
\poeml and were created because of your will.''
\end{poetry}
\labelchapt{5}
\passage{The Vision of the Scroll with Seven Seals}

\chapt{5}
\v{1}Then I saw in the right hand of the one who sits on the throne a scroll written on the inside and on the outside, sealed with seven seals. \v{2}I also saw a powerful angel proclaiming with a loud voice, ``Who is worthy to open the scroll and break its seals?'' \v{3}No one in heaven, on earth, or under the earth could open the scroll or look inside it. \v{4}I began to cry bitterly because no one was found worthy to open the scroll or look inside it.

\v{5}``Stop crying,'' one of the elders told me. ``Look! The Lion from the tribe of Judah, the Root of David, has conquered. He can open the scroll and its seven seals.''
\passage{The Vision of the Lamb Taking the Scroll}

\v{6}Then I saw a lamb standing in the middle of the throne, the four living creatures, and the elders. He looked\fnote{\fbackref{5:6} The Gk. lacks \fbib{He looked}} like he had been slaughtered. He had seven horns and seven eyes, which are the seven spirits of God sent into all the earth. \v{7}He went and took the scroll from the right hand of the one who sits on the throne.

\v{8}When the lamb had taken the scroll, the four living creatures and the twenty-four elders bowed down in front of him. Each held a harp and a gold bowl full of incense, the prayers of the saints. \v{9}They sang a new song:

\begin{poetry}
\poeml ``You are worthy to take the scroll and open its seals, \\
\poemll    because you were slaughtered. \\
\poeml With your blood you purchased people\fnote{\fbackref{5:9} Some MSS read \fbib{us}; the Gk. lacks \fbib{people}} for God \\
\poemll    from every tribe, language, people, and nation. \\
\poeml \v{10}You made them a kingdom and priests for our God, \\
\poemll    and they will reign on the earth.''
\end{poetry}
\passage{The Vision of the Song of the Lamb}

\v{11}Then I looked, and I heard the voices of many angels, the living creatures, and the elders surrounding the throne. They numbered 10,000's times 10,000 and thousands times thousands. \v{12}They sang with a loud voice,

\begin{poetry}
\poeml ``Worthy is the lamb who was slaughtered \\
\poemll    to receive power, wealth, wisdom, strength, honor, glory, and praise!''
\end{poetry}

\v{13}I heard every creature in heaven, on earth, under the earth, and on the sea, and everything that is in them, saying,

\begin{poetry}
\poeml ``To the one who sits on the throne and to the lamb \\
\poemll    be praise, honor, glory, and power forever and ever!''
\end{poetry}

\v{14}Then the four living creatures said, ``Amen!'', and the elders bowed down and worshipped.
\labelchapt{6}
\passage{The Vision of the First Seal Opened}

\chapt{6}
\v{1}Then I saw the lamb open the first of the seven seals. I heard one of the four living creatures say with a voice like thunder, ``Go!'' \v{2}Then I looked, and there was a white horse! Its rider had a bow, and a victor's crown had been given to him. He went out as a conqueror to conquer.
\passage{The Vision of the Second Seal Opened}

\v{3}When the lamb\fnote{\fbackref{6:3} Lit. \fbib{he}} opened the second seal, I heard the second living creature say, ``Go!'' \v{4}A second horse went out. It was fiery red, and its rider was given permission to take peace away from the earth and to make people slaughter one another. So he was given a large sword.
\passage{The Vision of the Third Seal Opened}

\v{5}When the lamb\fnote{\fbackref{6:5} Lit. \fbib{he}} opened the third seal, I heard the third living creature say, ``Go!'' I looked, and there was a black horse! Its rider held a scale in his hand. \v{6}I heard what sounded like a voice from among the four living creatures, saying, ``One day's ration of wheat for a day's wage, or three day's ration of barley for a day's wage!\fnote{\fbackref{6:6} A denarius was equivalent to a day's wage for a laborer.} But don't damage the olive oil or the wine!''
\passage{The Vision of the Fourth Seal Opened}

\v{7}When the lamb\fnote{\fbackref{6:7} Lit. \fbib{he}} opened the fourth seal, I heard the voice of the fourth living creature say, ``Go!'' \v{8}I looked, and there was a pale green horse! Its rider's name was Death, and Hades\fnote{\fbackref{6:8} I.e. the realm of the dead} followed him. They were given authority over one-fourth of the earth to kill people using wars, famines, plagues, and the wild animals of the earth.
\passage{The Vision of the Fifth Seal Opened}

\v{9}When the lamb\fnote{\fbackref{6:9} Lit. \fbib{he}} opened the fifth seal, I saw under the altar the souls of those who had been slaughtered because of the word of God and the testimony they had given. \v{10}They cried out in a loud voice,

\begin{poetry}
\poeml ``Holy and true Sovereign, \\
\poemll    how long will it be before you judge \\
\poeml and take revenge on those living on the earth \\
\poemll    who shed our blood?''
\end{poetry}

\v{11}Each of them was given a white robe. They were told to rest a little longer until the number of\fnote{\fbackref{6:11} The Gk. lacks \fbib{the number of}} their fellow servants and their brothers was completed, who would be killed as they themselves had been.
\passage{The Vision of the Sixth Seal Opened}

\v{12}Then I saw the lamb\fnote{\fbackref{6:12} Lit. \fbib{him}} open the sixth seal. There was a powerful earthquake. The sun turned as black as sackcloth made of hair, and the full moon turned as red as blood.\fnote{\fbackref{6:12} Lit. \fbib{became like blood}} \v{13}The stars in the sky fell to the earth like a fig tree drops its fruit when it is shaken by a strong wind. \v{14}The sky vanished like a scroll being rolled up, and every mountain and island was moved from its place. \v{15}Then the kings of the earth, the important people, the generals, the rich, the powerful, and all the slaves and free people concealed themselves in caves and among the rocks in the mountains. \v{16}They told the mountains and rocks, ``Fall on us and hide us from the face of the one who sits on the throne and from the wrath of the lamb. \v{17}For the great day of their wrath has come, and who is able to endure it?''
\labelchapt{7}
\passage{The Vision of the Sealing of 144,000 People}

\chapt{7}
\v{1}After this, I saw four angels standing at the four corners of the earth. They were holding back the four winds of the earth so that no wind could blow on the land, on the sea, or on any tree. \v{2}I saw another angel coming from the east having the seal of the living God. He cried out in a loud voice to the four angels who had been permitted to harm the land and sea, \v{3}``Don't harm the land, the sea, or the trees until we have marked the servants of our God with a seal on their foreheads.''

\v{4}I heard the number of those who were sealed: 144,000. Those who were sealed were from every tribe of Israel: \v{5}12,000 from the tribe of Judah were sealed, 12,000 from the tribe of Reuben, 12,000 from the tribe of Gad, \v{6}12,000 from the tribe of Asher, 12,000 from the tribe of Naphtali, 12,000 from the tribe of Manasseh, \v{7}12,000 from the tribe of Simeon, 12,000 from the tribe of Levi, 12,000 from the tribe of Issachar, \v{8}12,000 from the tribe of Zebulun, 12,000 from the tribe of Joseph, and 12,000 from the tribe of Benjamin were sealed.
\passage{The Vision of Tribulation Saints}

\v{9}After these things, I looked, and there was a crowd so large that no one was able to count it! They were from every nation, tribe, people, and language. They were standing in front of the throne and the lamb and were wearing white robes, with palm branches in their hands. \v{10}They cried out in a loud voice,

\begin{poetry}
\poeml ``Salvation belongs to our God, \\
\poemll    who sits on the throne, \\
\poemlll       and to the lamb!''
\end{poetry}

\v{11}All the angels stood around the throne and around the elders and the four living creatures. They fell on their faces in front of the throne and worshipped God, \v{12}saying,

\begin{poetry}
\poeml ``Amen! Praise, glory, wisdom, thanks, honor, power, and strength \\
\poemll    be to our God forever and ever! Amen!''
\end{poetry}

\v{13}``Who are these people wearing white robes,'' one of the elders asked me, ``and where did they come from?''

\v{14}I told him, ``Sir, you know.''

Then he told me, ``These are the people who are coming out of the terrible suffering.\fnote{\fbackref{7:14} Or \fbib{great tribulation}} They have washed their robes and made them white in the blood of the lamb. \v{15}That is why:

\begin{poetry}
\poeml ``They are in front of the throne of God \\
\poemll    and worship\fnote{\fbackref{7:15} Or \fbib{serve}} him night and day in his Temple. \\
\poeml The one who sits on the throne will shelter them. \\
\poeml \v{16}They will never be hungry or thirsty again. \\
\poemll    Neither the sun nor its heat will ever beat down on them, \\
\poeml \v{17}because the lamb in the center of the throne will be their shepherd. \\
\poemll    He will lead them to springs filled with the water of life, \\
\poemlll       and God will wipe every tear from their eyes.''\fnote{\fbackref{7:17} Isa 25:8}
\end{poetry}
\labelchapt{8}
\passage{The Vision of the Seventh Seal Opened}

\chapt{8}
\v{1}When the lamb\fnote{\fbackref{8:1} Lit. \fbib{he}} opened the seventh seal, there was silence in heaven for about half an hour.
\passage{The Vision of Seven Angels Given Seven Trumpets}

\v{2}Then I saw the seven angels who stand in God's presence, and seven trumpets were given to them. \v{3}Another angel came with a gold censer and stood at the altar. He was given a large quantity of incense to offer on the gold altar before the throne, along with the prayers of all the saints. \v{4}The smoke from the incense and the prayers of the saints went up from the angel's hand to God. \v{5}The angel took the censer, filled it with fire from the altar, and threw it on the earth. Then there were peals of thunder, noises, flashes of lightning, and an earthquake. \v{6}The seven angels who had the seven trumpets got ready to blow them.
\passage{The Vision of the First Four Trumpets}

\v{7}When the first angel blew his trumpet, hail and fire were mixed with blood and thrown on the earth. One-third of the earth was burned up, one-third of the trees was burned up, and all the green grass was burned up.

\v{8}When the second angel blew his trumpet, something like a huge mountain burning with fire was thrown into the sea. One-third of the sea turned into blood, \v{9}one-third of the creatures that were living in the sea died, and one-third of the ships was destroyed.

\v{10}When the third angel blew his trumpet, a huge star blazing like a torch fell from heaven. It fell on one-third of the rivers and on the springs of water. \v{11}The name of the star is Wormwood. One-third of the water turned into wormwood, and many people died from the water because it had turned bitter.

\v{12}When the fourth angel blew his trumpet, one-third of the sun, one-third of the moon, and one-third of the stars were struck so that one-third of them turned dark. One-third of the day was kept from having light, as was the night.
\passage{The Vision of the Eagle Flying}

\v{13}Then I looked, and I heard an eagle flying overhead say in a loud voice,

\begin{poetry}
\poeml ``How terrible, how terrible, how terrible \\
\poemll    for those living on the earth, \\
\poeml because of the blasts of the remaining trumpets \\
\poemll    that the three angels are about to blow!''
\end{poetry}
\labelchapt{9}
\passage{The Vision of the Fifth Trumpet}

\chapt{9}
\v{1}When the fifth angel blew his trumpet, I saw a star that had fallen to earth\fnote{\fbackref{9:1} I.e. Lucifer} from the sky.\fnote{\fbackref{9:1} Or \fbib{from heaven}} The star\fnote{\fbackref{9:1} Lit. \fbib{It}} was given the key to the shaft of the bottomless pit.\fnote{\fbackref{9:1} I.e. the realm of punishment in the afterlife} \v{2}It opened the shaft of the bottomless pit,\fnote{\fbackref{9:2} I.e. the realm of punishment in the afterlife} and smoke came out of the shaft like the smoke from a large furnace. The sun and the air were darkened with the smoke from the shaft. \v{3}Locusts came out of the smoke onto the earth, and they were given power like that of earthly scorpions. \v{4}They were told not to harm the grass on the earth, any green plant, or any tree. They could harm\fnote{\fbackref{9:4} The Gk. lacks \fbib{They could harm}} only the people who do not have the seal of God on their foreheads. \v{5}They were not allowed to kill them, but were only allowed\fnote{\fbackref{9:5} The Gk. lacks \fbib{were only allowed}} to torture them for five months. Their torture was like the pain of a scorpion when it stings someone. \v{6}In those days people will seek death, but never find it. They will long to die, but death will escape them.

\v{7}The locusts looked like horses prepared for battle. On their heads were victor's crowns that looked like gold, and their faces were like human faces. \v{8}They had hair like women's hair and teeth like lions' teeth. \v{9}They had breastplates like iron, and the noise of their wings was like the roar of chariots with many horses rushing into battle. \v{10}They had tails and stingers like scorpions, and they had the power to hurt people with their tails for five months. \v{11}They had the angel of the bottomless pit\fnote{\fbackref{9:11} I.e. the realm of punishment in the afterlife} ruling over them as king. In Hebrew he is called Abaddon,\fnote{\fbackref{9:11} I.e. the realm of destruction in the afterlife} and in Greek he is called Apollyon.\fnote{\fbackref{9:11} I.e. Destroyer}

\v{12}The first catastrophe is over. After these things, there are still two more catastrophes to come.
\passage{The Vision of the Sixth Trumpet}

\v{13}When the sixth angel blew his trumpet, I heard a voice from the four\fnote{\fbackref{9:13} Other mss. lack \fbib{four}} horns of the gold altar in front of God. \v{14}It told the sixth angel who had the trumpet, ``Release the four angels who are held at the great Euphrates River.'' \v{15}So the four angels who were ready for that hour, day, month, and year were released to kill one-third of humanity. \v{16}The number of cavalry troops was 200,000,000. I heard how many there were.\fnote{\fbackref{9:16} Lit. \fbib{heard their number}}

\v{17}This was how I saw the horses in my vision: The riders wore breastplates that had the color of fire, sapphire, and sulfur. The heads of the horses were like lions' heads, and fire, smoke, and sulfur came out of their mouths. \v{18}By these three plagues---the fire, the smoke, and the sulfur that came out of their mouths---one-third of humanity was killed. \v{19}For the power of these horses is in their mouths and their tails. Their tails have heads like snakes, which they use to inflict pain.

\v{20}The rest of the people who survived these plagues did not repent from their evil actions\fnote{\fbackref{9:20} Lit. \fbib{from the works of their hands}} or stop worshiping demons and idols made of gold, silver, bronze, stone, and wood, which cannot see, hear, or walk. \v{21}They did not repent from their murders, their witchcraft, their sexual immorality, or their thefts.
\labelchapt{10}
\passage{The Vision of the Powerful Angel}

\chapt{10}
\v{1}Then I saw another powerful angel come down from heaven. He was dressed in a cloud, and there was a rainbow over his head. His face was like the sun, and his legs were like columns of fire. \v{2}He held a small, opened scroll in his hand. Setting his right foot on the sea and his left foot on the land, \v{3}he shouted in a loud voice as a lion roars. When he shouted, the seven thunders spoke with voices of their own. \v{4}When the seven thunders spoke, I was going to write, but I heard a voice from heaven say, ``Seal up what the seven thunders have said, and don't write it down.''

\v{5}Then the angel whom I saw standing on the sea and on the land raised his right hand to heaven. \v{6}He swore an oath by the one who lives forever and ever, who created heaven and everything in it, the earth and everything in it, and the sea and everything in it: ``There will be no more delay. \v{7}When the time approaches\fnote{\fbackref{10:7} Lit. \fbib{days approach}} for the seventh angel to blow his trumpet, God's secret plan\fnote{\fbackref{10:7} The Gk. lacks \fbib{plan}} will be fulfilled, as he had announced to his servants, the prophets.''

\v{8}Then the voice that I had heard from heaven spoke to me again, saying, ``Go and take the opened scroll from the hand of the angel who is standing on the sea and on the land.''

\v{9}So I went to the angel and asked him to give me the small scroll. ``Take it and eat it,'' he told me. ``It will turn bitter in your stomach, but it will be as sweet as honey in your mouth.''

\v{10}So I took the small scroll from the angel's hand and ate it. It was as sweet as honey in my mouth, but when I had eaten it, it turned bitter in my stomach. \v{11}Then the seven thunders\fnote{\fbackref{10:11} Lit. \fbib{they}} told me, ``You must prophesy again about many people, nations, languages, and kings.''
\labelchapt{11}
\passage{The Vision of the Two Witnesses}

\chapt{11}
\v{1}Then I was given a stick like a measuring rod. I was told, ``Stand up and measure the Temple of God and the altar, and count\fnote{\fbackref{11:1} The Gk. lacks \fbib{count}} those who worship there. \v{2}But don't measure the courtyard outside the Temple. Leave that out, because it is given to the nations, and they will trample the Holy City\fnote{\fbackref{11:2} I.e. Jerusalem} for 42 months. \v{3}I will give my two witnesses who wear sackcloth the authority to prophesy for 1,260 days.''

\v{4}These witnesses\fnote{\fbackref{11:4} Lit. \fbib{These ones}} are the two olive trees and the two lamp stands standing in the presence of the Lord of the earth. \v{5}And if anyone should want to hurt them, fire comes out of their mouths and burns up their enemies. If anyone wants to hurt them, he must be killed in this manner. \v{6}These witnesses\fnote{\fbackref{11:6} Lit. \fbib{They}} have authority to close the heavens in order to keep rain from falling while they are\fnote{\fbackref{11:6} Lit. \fbib{falling during the days of their}} prophesying. They also have authority to turn bodies of water into blood and to strike the earth with any plague, as often as they desire.

\v{7}When they have finished their testimony, the beast that comes up from the bottomless pit\fnote{\fbackref{11:7} I.e. the realm of punishment in the afterlife} will wage war against them, conquer them, and kill them. \v{8}Their dead bodies will lie in the street of the great city that is spiritually called Sodom and Egypt, where their Lord was crucified. \v{9}For three and a half days some members of the people, tribes, languages, and nations will look at their dead bodies and will not allow them to be placed in a tomb. \v{10}Those living on earth will gloat over them, celebrate, and send gifts to each other, because these two prophets had tormented those living on earth.
\passage{The Resurrection of the Witnesses}

\v{11}But after the three and a half days, the breath of life from God entered them, and they stood on their feet. Those who watched them were terrified. \v{12}Then the witnesses\fnote{\fbackref{11:12} Lit. \fbib{they}; other mss. read \fbib{I}} heard a loud voice from heaven calling to them, ``Come up here!'' So they went up to heaven in a cloud, and their enemies watched them. \v{13}At that moment a powerful earthquake struck. One-tenth of the city collapsed, 7,000 people were killed by the earthquake, and the rest were terrified and gave glory to the God of heaven.

\v{14}The second catastrophe is over. The third catastrophe is coming very soon.
\passage{The Vision of the Seventh Angel Blowing His Trumpet}

\v{15}When the seventh angel blew his trumpet, there were loud voices in heaven, saying,

\begin{poetry}
\poeml ``The world's kingdom has become \\
\poemll    the kingdom of our Lord and of his Messiah,\fnote{\fbackref{11:15} Or \fbib{Christ}} \\
\poemlll       and he will rule forever and ever.''
\end{poetry}

\v{16}Then the twenty-four elders who were sitting on their thrones in God's presence fell on their faces and worshipped God. \v{17}They said,

\begin{poetry}
\poeml ``We give thanks to you, Lord God Almighty, \\
\poemll    who is and who was, \\
\poeml because you have taken your great power \\
\poemll    and have begun to rule. \\
\poeml \v{18}The nations were angry, \\
\poemll    but the time for\fnote{\fbackref{11:18} The Gk. lacks \fbib{the time for}} your wrath has come. \\
\poeml It is time for the dead to be judged--- \\
\poemll    to reward your servants, the prophets, the saints, and all who fear your name, \\
\poemlll       both unimportant and important, \\
\poemll    and to destroy those who destroy the earth.''
\end{poetry}

\v{19}Then the Temple of God in heaven was opened, and the ark of his covenant was seen inside his Temple. There were flashes of lightning, noises, peals of thunder, an earthquake, and heavy hail.
\labelchapt{12}
\passage{The Vision of a Woman Dressed with the Sun}

\chapt{12}
\v{1}A spectacular sign appeared in the sky: a woman dressed with the sun, who had the moon under her feet and a victor's crown of twelve stars on her head. \v{2}She was pregnant and was crying out from her labor pains, the agony of giving birth.
\passage{The Vision of the Red Dragon}

\v{3}Then another sign appeared in the sky: a huge red dragon with seven heads, ten horns, and seven royal crowns on its heads. \v{4}Its tail swept away one-third of the stars in the sky and knocked them down to the earth. Then the dragon stood in front of the woman who was about to give birth so that it could devour her child when it was born. \v{5}She gave birth to a son, a boy, who is to rule\fnote{\fbackref{12:5} Or \fbib{shepherd}} all the nations with an iron scepter. But her child was snatched away and taken to God and to his throne. \v{6}Then the woman fled into the wilderness, where a place had been prepared for her by God so that she might be taken care of for 1,260 days.
\passage{The Vision of War in Heaven}

\v{7}Then a war broke out in heaven. Michael and his angels fought with the dragon, and the dragon and its angels fought back. \v{8}But it was not strong enough, and there was no longer any place for them in heaven. \v{9}The huge dragon was hurled down. That ancient serpent, called the Devil and Satan, the deceiver of the whole world, was hurled down to the earth, along with its angels.
\passage{The Vision of the Cry of Victory}

\v{10}Then I heard a loud voice in heaven say,

\begin{poetry}
\poeml ``Now the salvation, the power, \\
\poemll    the kingdom of our God, \\
\poemlll       and the authority of his Messiah\fnote{\fbackref{12:10} Or \fbib{Christ}} have come. \\
\poeml For the one who accuses our brothers, \\
\poemll    who accuses them day and night \\
\poeml in the presence of our God, \\
\poemll    has been thrown out. \\
\poeml \v{11}Our brothers\fnote{\fbackref{12:11} Lit. \fbib{They}} conquered him by the blood of the lamb \\
\poemll    and by the word of their testimony, \\
\poeml for they did not cling to their lives \\
\poemll    even in the face of death. \\
\poeml \v{12}So be glad, heavens, and those who live in them! \\
\poeml How terrible it is for the earth and the sea, \\
\poemll    because the Devil has come down to you, filled with rage, \\
\poemlll       knowing that his time is short!''
\end{poetry}
\passage{The Vision of Persecution of the Woman and Her Children}

\v{13}When the dragon saw that it had been thrown down to the earth, it pursued\fnote{\fbackref{12:13} Or \fbib{persecuted}} the woman who had given birth to the boy. \v{14}However, the woman was given the two wings of a large eagle so that she could fly away from the serpent to her place in the wilderness, where she could be taken care of for a time, times, and half a time. \v{15}From its mouth the serpent spewed water like a river behind the woman in order to sweep her away with the flood. \v{16}But the earth helped the woman by opening its mouth and swallowing the river that the dragon had spewed from its mouth. \v{17}The dragon became angry with the woman and went away to do battle against the rest of her children, the ones who keep God's commandments and hold on to the testimony about Jesus. \v{18}Then the dragon\fnote{\fbackref{12:18} Lit. \fbib{Then it}; other mss. read \fbib{Then I}} stood on the sand of the seashore.\fnote{\fbackref{12:18} Some translations include this sentence within 12:17; other translations include this sentence as part of 13:1.}
\labelchapt{13}
\passage{The Vision of the Beast from the Sea}

\chapt{13}
\v{1}I saw a beast coming out of the sea. It had ten horns, seven heads, and ten royal crowns on its horns. On its heads were blasphemous names. \v{2}The beast that I saw was like a leopard. Its feet were like bear's feet, and its mouth was like a lion's mouth. The dragon gave it his power, his throne, and complete authority.

\v{3}One of the beast's\fnote{\fbackref{13:3} Lit. \fbib{its}} heads looked like it had sustained a mortal wound, but its fatal wound was healed. Rapt with amazement, the whole world followed the beast. \v{4}They worshipped the dragon because it had given authority to the beast. They also worshipped the beast, saying, ``Who is like the beast, and who can fight a war with it?'' \v{5}The beast was allowed\fnote{\fbackref{13:5} Lit. \fbib{was given a mouth}} to speak arrogant and blasphemous things, and it was given authority for 42 months. \v{6}It uttered\fnote{\fbackref{13:6} Lit. \fbib{It opened its mouth to utter}} blasphemies against God, against his name, and against his residence,\fnote{\fbackref{13:6} Lit. \fbib{tent}} that is, against those who are living in heaven. \v{7}It was allowed to wage war against the saints and to conquer them.\fnote{\fbackref{13:7} Other mss. lack \fbib{It was allowed to wage war against the saints and to conquer them.}} It was also given authority over every tribe, people, language, and nation. \v{8}All those who had become settled down and at home, living on the earth, will worship it, everyone whose name had not been written in the Book of Life belonging to the lamb that had been slaughtered since the foundation of the world.

\begin{poetry}
\poeml \v{9}Let everyone listen:\fnote{\fbackref{13:9} Lit. \fbib{Let the one who has an ear}} \\
\poeml \v{10}If anyone is to be taken captive, \\
\poemlll       into captivity he will go. \\
\poeml If anyone is to be killed with a sword, \\
\poemll    with a sword he will be killed.
\end{poetry}
\passage{The Vision of the Beast from the Earth}

Here is a call for endurance and faith of the saints:

\v{11}I saw another beast coming up out of the earth. It had two horns like a lamb and it talked like a dragon. \v{12}It uses all the authority of the first beast on its behalf,\fnote{\fbackref{13:12} Or \fbib{in its presence}} and it makes the earth and those living on it worship the first beast, whose mortal wound was healed. \v{13}It performs spectacular signs, even making fire come down from heaven to earth in front of people. \v{14}It deceives those living on earth with the signs that it is allowed to do on behalf of\fnote{\fbackref{13:14} Or \fbib{in the presence of}} the first\fnote{\fbackref{13:14} The Gk. lacks \fbib{first}} beast, telling them to make an image for the beast who was wounded by a sword and yet lived. \v{15}The second beast\fnote{\fbackref{13:15} Lit. \fbib{It}} was allowed to impart life to the image of the first\fnote{\fbackref{13:15} The Gk. lacks \fbib{first}} beast so that the image of the beast could talk and order the execution of those who would not worship the image of the beast. \v{16}The second beast\fnote{\fbackref{13:16} Lit. \fbib{It}} forces all people---important and unimportant, rich and poor, free and slaves---to be marked on their right hands or on their foreheads, \v{17}so that no one may buy or sell unless he has the mark, which is the beast's name or the number of its name.

\v{18}In this case wisdom is needed: Let the person who has understanding\fnote{\fbackref{13:18} Lit. \fbib{has a mind}} calculate\fnote{\fbackref{13:18} Or \fbib{decide}} the total\fnote{\fbackref{13:18} Or \fbib{the multitude}} of the beast, since it is a human multitude,\fnote{\fbackref{13:18} Or \fbib{is a multitude of a man}; or \fbib{is the number of a human being}} and the sum of the multitude\fnote{\fbackref{13:18} Or \fbib{total number}} is 600, 60, and six.\fnote{\fbackref{13:18} Other mss. read \fbib{616}; some scholars suggest that the Gk. letters {$\chi$}, {$\xi$}, and {$\varsigma$} (totaling 666) may resemble the proto-Arabic term for ``in the name of Allah''}
\labelchapt{14}
\passage{The Vision of the New Song on Mount Zion}

\chapt{14}
\v{1}Then I looked, and there was the lamb, standing on Mount Zion! With him were 144,000 people who had his name and his Father's name written on their foreheads.

\v{2}Then I heard a sound from heaven like that of many waters and like the sound of loud thunder. The sound I heard was like harpists playing on their harps. \v{3}They were singing a new song in front of the throne, the four living creatures, and the elders. No one could learn the song except the 144,000 who had been redeemed from the earth. \v{4}They have not defiled themselves with women, for they are virgins, and they follow the lamb wherever he goes. They have been redeemed from among humanity as the first fruits for God and the lamb. \v{5}In their mouth no lie was found. They are blameless.
\passage{The Vision of Angels Sounding a Warning}

\v{6}Then I saw another angel flying overhead with the eternal gospel to proclaim to those who live\fnote{\fbackref{14:6} Lit. \fbib{sit}} on earth---to every nation, tribe, language, and people. \v{7}He said in a loud voice,

\begin{poetry}
\poeml ``Fear God and give him glory, \\
\poemll    because the time for him to judge has arrived. \\
\poeml Worship the one who made heaven and earth, \\
\poemll    the sea and springs of water.''
\end{poetry}

\v{8}Then another angel, a second one, followed him, saying,

\begin{poetry}
\poeml ``Fallen! Babylon the Great has fallen! \\
\poemll    She has made all nations drink the wine, \\
\poemlll       the wrath earned for her sexual sins.''
\end{poetry}

\v{9}Then another angel, a third one, followed them, saying in a loud voice, ``Whoever worships the beast and its image and receives a mark on his forehead or his hand \v{10}will drink the wine of God's wrath, which has been poured undiluted into the cup of his anger. He will be tortured with fire and sulfur in the presence of the holy angels and the lamb. \v{11}The smoke from their torture goes up forever and ever. There is no rest day or night for those who worship the beast and its image or for anyone who receives the mark of its name.''
\passage{A Call for Endurance}

\v{12}Here is a call for\fnote{\fbackref{14:12} The Gk. lacks \fbib{a call for}} the endurance of the saints, who keep the commandments of God and hold fast to their faithfulness in Jesus:

\v{13}I heard a voice from heaven say, ``Write this:

\begin{poetry}
\poeml How blessed are the dead, \\
\poemll    that is, those who die in the Lord from now on!'' \\
\poeml ``Yes,'' says the Spirit. \\
\poemll    ``Let them rest from their labors, \\
\poemlll       for their actions follow them.''
\end{poetry}
\passage{The Vision of Earth Harvested}

\v{14}Then I looked, and there was a white cloud! On the cloud sat someone who was like the Son of Man, with a gold victor's crown on his head and a sharp sickle in his hand. \v{15}Another angel came out of the Temple, crying out in a loud voice to the one who sat on the cloud,

\begin{poetry}
\poeml ``Swing your sickle, \\
\poemll    and gather the harvest, \\
\poeml for the hour has come to gather it, \\
\poemll    because the harvest of the earth is fully ripe.''
\end{poetry}

\v{16}The one who sat on the cloud swung his sickle across the earth, and the earth was harvested.

\v{17}Then another angel came out of the Temple in heaven. He, too, had a sharp sickle. \v{18}From the altar came another angel who had authority over fire. He called out in a loud voice to the angel\fnote{\fbackref{14:18} Lit. \fbib{to the one}} who had the sharp sickle,

\begin{poetry}
\poeml ``Swing your sharp sickle, \\
\poemll    and gather the bunches of grapes \\
\poeml from the vine of the earth, \\
\poemll    because those grapes are ripe.''
\end{poetry}

\v{19}So the angel swung his sickle in the earth, gathered the grapes from the earth, and threw them into the great winepress of God's wrath. \v{20}The wine press was trampled outside the city, and blood flowed from the wine press as high as a horse's bridle for about 1,600 stadia.\fnote{\fbackref{14:20} I.e. about 194 miles; one Roman stadion was about 640 feet}
\labelchapt{15}
\passage{The Vision of Seven Angels with Seven Plagues}

\chapt{15}
\v{1}I saw another sign in heaven. It was both spectacular and amazing. There were seven angels with the seven last plagues, with which God's wrath is completed.
\passage{The Vision of the Sea of Glass}

\v{2}Then I saw what looked like a sea of glass mixed with fire. Those who had conquered the beast, its image, and the number of its name were standing on the sea of glass holding God's harps in their hands. \v{3}They sang the song of God's servant Moses and the song of the lamb:

\begin{poetry}
\poeml ``Your deeds are both spectacular and amazing, \\
\poemll    Lord God Almighty. \\
\poeml Your ways are just and true, \\
\poemll    King of the nations.\fnote{\fbackref{15:3} Other mss. read \fbib{ages}} \\
\poeml \v{4}Lord, who won't fear and praise your name? \\
\poemll    For you alone are holy, \\
\poemlll       and all the nations will come and worship you \\
\poemll    because your judgments have been revealed.''
\end{poetry}
\passage{The Vision of the Temple Opened}

\v{5}After these things, I looked, and the Temple, which is the Tent of Testimony in heaven, was open! \v{6}The seven angels with the seven plagues came out of the Temple wearing clean, shining linen with gold sashes around their chests.\fnote{\fbackref{15:6} Or \fbib{waists}} \v{7}One of the four living creatures gave to the seven angels seven gold bowls full of the wrath of God, who lives forever and ever. \v{8}The Temple was filled with smoke from the glory of God and his power, and no one could enter the Temple until the seven plagues of the seven angels came to an end.
\labelchapt{16}
\passage{The Vision of the Seven Angels Pouring Out Their Bowls}

\chapt{16}
\v{1}Then I heard a loud voice from the Temple saying to the seven angels, ``Go and pour the seven bowls of God's wrath on the earth.'' \v{2}So the first angel went and poured his bowl on the earth. A horrible, painful sore appeared on the people who had the mark of the beast and worshipped the image.

\v{3}The second angel poured his bowl into the sea. It became like the blood of a dead body, and every living thing in the sea died.

\v{4}The third angel poured his bowl into the rivers and the springs of water, and they turned into blood. \v{5}Then I heard the angel in charge\fnote{\fbackref{16:5} The Gk. lacks \fbib{in charge}} of the water say,

\begin{poetry}
\poeml ``You are just. You are the one who is \\
\poemll    and who was, the Holy One, \\
\poemlll       because you have judged these things. \\
\poeml \v{6}You have given them blood to drink \\
\poemll    because they spilled the blood of saints and prophets. \\
\poemlll       This is what they deserve.''
\end{poetry}

\v{7}Then I heard the altar reply,

\begin{poetry}
\poeml ``Yes, Lord God Almighty, \\
\poemll    your judgments are true and just.''
\end{poetry}

\v{8}The fourth angel poured his bowl on the sun, which then was allowed to burn people with fire, \v{9}and they were burned by the fierce heat. They cursed the name of God, who has the authority over these plagues. They did not repent and give him glory.

\v{10}The fifth angel poured his bowl on the throne of the beast. Its kingdom was plunged into darkness. People\fnote{\fbackref{16:10} Lit. \fbib{They}} gnawed on their tongues in anguish \v{11}and cursed the God of heaven because of their pain-filled sores. But they did not repent of their behavior.

\v{12}The sixth angel poured his bowl on the great Euphrates River. Its water was dried up to prepare the way for the kings from the east. \v{13}Then I saw three disgusting spirits like frogs come out of the mouth of the dragon, out of the mouth of the beast, and out of the mouth of the false prophet. \v{14}They are demonic spirits that perform signs. They go to the kings of the whole earth and gather them for the war of the great Day of God Almighty.

\begin{poetry}
\poeml \v{15}\red{``See, I am coming like a thief.} \\
\poeml \red{How blessed is the person who remains alert} \\
\poemll    \red{and keeps his clothes on!} \\
\poeml \red{He won't have to go naked} \\
\poemll    \red{and let others see his shame.''}
\end{poetry}

\v{16}The spirits\fnote{\fbackref{16:16} Lit. \fbib{They}} gathered the kings\fnote{\fbackref{16:16} Lit. \fbib{them}} to the place that is called Armageddon in Hebrew.

\v{17}The seventh angel threw the contents of\fnote{\fbackref{16:17} The Gk. lacks \fbib{the contents}} his bowl across the sky. A loud voice came from the throne in the Temple and said, ``It has happened!'' \v{18}There were flashes of lightning, noises, peals of thunder, and a powerful earthquake. There has never been such a powerful earthquake since people have been on the earth. \v{19}The great city was split into three parts, and the cities of the nations fell. God remembered to give Babylon the Great the cup of wine filled with the fury of his wrath. \v{20}Every island vanished,\fnote{\fbackref{16:20} Or \fbib{fled}} and the mountains could no longer be found. \v{21}Huge hailstones, each weighing about a talent,\fnote{\fbackref{16:21} I.e. about 75 pounds; a talent weighed about 75 pounds} fell from the sky on people, who cursed God because the plague of hail was so terrible.
\labelchapt{17}
\passage{The Vision of Babylon the Great}

\chapt{17}
\v{1}Then one of the seven angels who held the seven bowls came and told me, ``Come, I will show you how the notorious prostitute who sits on many waters will be judged. \v{2}The kings of the earth committed sexual immorality with her, and those living on earth became drunk with the wine of her immorality.''

\v{3}Then the angel\fnote{\fbackref{17:3} Lit. \fbib{he}} carried me away in the Spirit into a wilderness. I saw a woman sitting on a scarlet beast that was controlled by blasphemy.\fnote{\fbackref{17:3} Lit. \fbib{was filled with blasphemous names}} It had seven heads and ten horns. \v{4}The woman wore purple and scarlet clothes and was adorned with gold, gems, and pearls. In her hand she was holding a gold cup filled with detestable things and the impurities of her immorality. \v{5}On her forehead was written a secret name:

\divine{Babylon the Great},

\divine{the Mother of Prostitutes}

\divine{and Detestable Things of the Earth}

\v{6}I saw that the woman was drunk with the blood of the saints and the blood of the witnesses to Jesus. I was very surprised when I saw her.

\v{7}``Why are you surprised?'' The angel asked me. ``I will tell you the secret of the woman and the beast with the seven heads and the ten horns that carries her. \v{8}The beast that you saw existed once, but is no longer, and is going to crawl out of the bottomless pit\fnote{\fbackref{17:8} I.e. the realm of punishment in the afterlife} and then proceed to its destruction. Those living on earth, whose names were not written in the Book of Life from the foundation of the world, will be surprised when they see the beast because it was, is no longer, and will come again. \v{9}This calls for a mind that has wisdom. The seven heads are seven mountains on which the woman is sitting. They are also seven kings. \v{10}Five of them have fallen, one is living, and the other has not yet come. When he comes, he must remain in power\fnote{\fbackref{17:10} The Gk. lacks \fbib{in power}} for a little while. \v{11}The beast that was and is no longer is the eighth king,\fnote{\fbackref{17:11} The Gk. lacks \fbib{king}} but it belongs with the seven kings\fnote{\fbackref{17:11} The Gk. lacks \fbib{kings}} and goes to its destruction. \v{12}The ten horns that you saw are ten kings who have not yet received a kingdom. They will receive authority to rule\fnote{\fbackref{17:12} The Gk. lacks \fbib{to rule}} as kings with the beast for one hour. \v{13}They have one purpose: to give their power and authority to the beast. \v{14}They will wage war against the lamb, but the lamb will conquer them because he is Lord of lords and King of kings. Those who are called, chosen, and faithful are with him.''

\v{15}The angel\fnote{\fbackref{17:15} Lit. \fbib{He}} also told me, ``The bodies of water you saw, on which the prostitute is sitting, are people, multitudes, nations, and languages. \v{16}The ten horns and the beast you saw will hate the prostitute. They will leave her abandoned and naked. They will eat her flesh and burn her up with fire, \v{17}for God has placed within them a desire to carry out his purpose by uniting to give their kingdom to the beast until God's words are fulfilled. \v{18}The woman you saw is the great city that rules over the kings of the earth.''
\labelchapt{18}
\passage{The Vision of the Fall of Babylon}

\chapt{18}
\v{1}After these things, I saw another angel coming down from heaven. He had great authority,\fnote{\fbackref{18:1} Or \fbib{tremendous power}} and the earth was made bright by his splendor. \v{2}He cried out in a powerful voice,

\begin{poetry}
\poeml ``Fallen! Babylon the Great has fallen! \\
\poemll    She has become a home for demons. \\
\poeml She is a prison for every unclean spirit, \\
\poemll    a prison for every unclean bird, \\
\poemll    and a prison for every unclean \\
\poemlll       and hated beast.
\end{poetry}

\begin{poetry}
\poeml \v{3}For all the nations have drunk \\
\poemll    from the wine of her sexual immorality, \\
\poeml and the kings of the earth have committed sexual immorality with her. \\
\poeml The world's businesses have become rich \\
\poemll    from her luxurious excesses.''
\end{poetry}
\passage{The Warning to Leave Babylon}

\v{4}Then I heard another voice from heaven saying,

\begin{poetry}
\poeml ``Come out of her, my people, \\
\poemll    so that you don't participate in her sins \\
\poemlll       and also suffer from her diseases. \\
\poeml \v{5}For her sins are piled as high as heaven, \\
\poemll    and God has remembered her crimes. \\
\poeml \v{6}Do to her as she herself has done, \\
\poemll    and give her double for her deeds. \\
\poeml Mix a double drink for her in the cup she mixed. \\
\poeml \v{7}Just as she glorified herself and lived in luxury, \\
\poemll    inflict on her just as much torture and misery. \\
\poeml In her heart she says, \\
\poemll    `I am a queen on a throne, not a widow. \\
\poemlll       I will never see misery.' \\
\poeml \v{8}For this reason, her diseases that result in death, misery, and famine \\
\poemll    will come in a single day. \\
\poeml She will be burned up in a fire, \\
\poemll    because powerful is the Lord God who judges her.''
\end{poetry}

\v{9}The kings of the earth, who committed sexual immorality with her and lived in luxury with her, will cry and mourn over her when they see the smoke rising from the fire that consumes her.\fnote{\fbackref{18:9} Lit. \fbib{from her fire}} \v{10}Frightened by the torture that she experiences,\fnote{\fbackref{18:10} Lit. \fbib{by her torture}} they will stand far away and cry out,

\begin{poetry}
\poeml ``How terrible, how terrible it is for that great city, \\
\poemll    the powerful city Babylon, \\
\poemlll       because your judgment arrived in a single hour!''
\end{poetry}

\v{11}The world's businesses cry and mourn over her, because no one buys their cargo anymore--- \v{12}cargo of gold, silver, gems, pearls, fine linen, purple cloth, silk, scarlet cloth, all kinds of scented wood, all articles made of ivory, all articles made of very costly wood, bronze, iron, marble, \v{13}cinnamon, spice, incense, myrrh, frankincense, wine, olive oil, flour, wheat, cattle, sheep, horses, chariots, and slaves (that is, human souls)---

\begin{poetry}
\poeml \v{14}``The fruit that you crafted has abandoned you. \\
\poemll    All your dainties and your splendor are lost,\fnote{\fbackref{18:14} Lit. \fbib{lost to you}} \\
\poemlll       and no one will ever find them again.''
\end{poetry}

\v{15}Frightened by the severity of her punishment, businesses that had become rich because of her will stand at a distance, crying and mourning:

\begin{poetry}
\poeml \v{16}``How terrible, how terrible it is for the great city \\
\poemll    that was clothed in fine linen, purple, and scarlet \\
\poemlll       and was adorned with gold, gems, and pearls, \\
\poeml \v{17}because all this wealth has been destroyed in a single hour!''
\end{poetry}

Every ship's captain, everyone who traveled by ship, sailors, and everyone who made a living from the sea stood far away. \v{18}When they saw the smoke rising from the fire that consumed her,\fnote{\fbackref{18:18} Lit. \fbib{from her fire}} they began to cry out, ``What city was like that great city?'' \v{19}Then they threw dust on their heads and shouted while crying and mourning:

\begin{poetry}
\poeml ``How terrible, how terrible it is for the great city, \\
\poemll    where all who had ships at sea became rich from her wealth, \\
\poemlll       because it has been destroyed in a single hour! \\
\poeml \v{20}Be happy about her, heaven, saints, apostles, and prophets, \\
\poemll    for God has condemned her for you!''
\end{poetry}
\passage{The Vision of the Powerful Angel with the Millstone}

\v{21}Then a powerful angel picked up a stone that was like a large millstone and threw it into the sea, saying,

\begin{poetry}
\poeml ``The great city Babylon will be thrown down violently--- \\
\poemll    and will never be found again. \\
\poeml \v{22}The sound of harpists, musicians, flutists, and trumpeters \\
\poemll    will never be heard within you again. \\
\poeml No artisan of any trade \\
\poemll    will ever be found within you again. \\
\poeml The sound of a millstone \\
\poemll    will never be heard within you again. \\
\poeml \v{23}The light from a lamp \\
\poemll    will never shine within you again. \\
\poeml The voice of a bridegroom and bride \\
\poemll    will never be heard within you again. \\
\poeml For your merchants were the important people of the world, \\
\poemll    and all the nations were deceived by your witchcraft. \\
\poeml \v{24}The blood of the world's prophets, saints, \\
\poemll    and all who had been murdered \\
\poemlll       was found within her.''
\end{poetry}
\labelchapt{19}
\passage{The Vision of Worship}

\chapt{19}
\v{1}After these things, I heard what sounded like the loud voice of a large crowd in heaven, saying,

\begin{poetry}
\poeml ``Hallelujah! \\
\poemll    Salvation, glory, and power belong to our God. \\
\poeml \v{2}His judgments are true and just. \\
\poemll    He has condemned the notorious prostitute \\
\poemlll       who corrupted the world with her immorality. \\
\poeml He has taken revenge on her \\
\poemll    for the blood of his servants.''
\end{poetry}

\v{3}A second time they said,

\begin{poetry}
\poeml ``Hallelujah! \\
\poemll    The smoke goes up from her forever and ever.''
\end{poetry}

\v{4}The twenty-four elders and the four living creatures bowed down and worshipped God, who was sitting on the throne. They said, ``Amen! Hallelujah!'' \v{5}A voice came from the throne, saying,

\begin{poetry}
\poeml ``Praise our God, \\
\poemll    all who serve and fear him, \\
\poeml from the least important \\
\poemll    to the most important.''
\end{poetry}
\passage{The Vision of the Marriage Supper of the Lamb}

\v{6}Then I heard what sounded like the voice of a large crowd, like the sound of raging waters, and like the sound of powerful thunderclaps, saying,

\begin{poetry}
\poeml ``Hallelujah! \\
\poemll    The Lord our God, the Almighty, is reigning. \\
\poeml \v{7}Let us rejoice, be glad, and give him glory, \\
\poemll    because the marriage of the lamb has come \\
\poemlll       and his bride has made herself ready. \\
\poeml \v{8}She has been given the privilege of wearing fine linen, \\
\poemll    dazzling and pure.''
\end{poetry}

(The fine linen represents the righteous deeds of the saints.)

\v{9}Then the angel\fnote{\fbackref{19:9} Lit. \fbib{he}} told me, ``Write this: `How blessed are those who are invited to the marriage supper of the lamb!'\,'' He also told me, ``These are the true words of God.'' \v{10}I bowed down at his feet to worship him, but he told me, ``Don't do that! I am a fellow servant with you and with your brothers who rely on what Jesus is saying. Worship God, because what Jesus is saying is the spirit of prophecy!''
\passage{The Vision of the Coming of the Messiah}

\v{11}Then I saw heaven standing open, and there was a white horse! Its rider is named Faithful and True. He administers justice and wages war righteously. \v{12}His eyes are like a flame of fire, and on his head are many royal crowns. He has a name written on him that nobody knows except himself. \v{13}He is dressed in a robe dipped in\fnote{\fbackref{19:13} Other mss. read \fbib{sprinkled with}} blood, and his name is called the Word of God. \v{14}The armies of heaven, wearing fine linen, white and pure, follow him on white horses. \v{15}A sharp sword comes out of his mouth to strike down the nations. He will rule\fnote{\fbackref{19:15} Or \fbib{shepherd}} them with an iron rod and tread the winepress of the fury of the wrath of God Almighty. \v{16}On his robe that covers\fnote{\fbackref{19:16} Lit. \fbib{robe and on}} his thigh he has a name written:

\divine{King of Kings and Lord of Lords}
\passage{The Vision of the Angel's Invitation}

\v{17}Then I saw an angel standing in the sun. He cried out in a loud voice to all the birds flying overhead, ``Come! Gather for the great supper of God. \v{18}Eat the flesh of kings, the flesh of commanders, the flesh of warriors,\fnote{\fbackref{19:18} Lit. \fbib{of the powerful}} the flesh of horses and their riders, and the flesh of all people, both free and slaves, both unimportant and important.''
\passage{The Vision of the Judgment of the Beast and False Prophet}

\v{19}Then I saw the beast, the kings of the earth, and their armies gathered to wage war against the rider on the horse and his army. \v{20}The beast was captured, along with the false prophet who had performed signs on its behalf.\fnote{\fbackref{19:20} Or \fbib{in its presence}} By these signs\fnote{\fbackref{19:20} Lit. \fbib{By which}} the false prophet\fnote{\fbackref{19:20} Lit. \fbib{he}} had deceived those who had received the mark of the beast and worshipped its image. Both of them were thrown alive into the lake of fire that burns with sulfur. \v{21}The rest were killed by the sword that belonged to the rider on the horse and that came from his mouth, and all the birds gorged themselves with their flesh.
\labelchapt{20}
\passage{The Vision of the Millennial Reign}

\chapt{20}
\v{1}Then I saw an angel coming down from heaven, holding the key to the bottomless pit,\fnote{\fbackref{20:1} I.e. the realm of punishment in the afterlife} with a large chain in his hand. \v{2}He captured the dragon, that ancient serpent, also known as the devil and Satan, and tied him up for a thousand years. \v{3}He threw him into the bottomless pit,\fnote{\fbackref{20:3} I.e. the realm of punishment in the afterlife} locked it, and sealed it over him to keep him from deceiving the nations anymore until the thousand years were over. After that, he must be set free for a little while.
\passage{The Vision of the First Resurrection}

\v{4}Then I saw thrones, and those who sat on them were given authority to judge. I also saw the souls of those who had been beheaded because of their testimony about Jesus and because of the word of God. They had not worshipped the beast or its image and had not received its mark on their foreheads or hands. They came back to life and ruled with the Messiah\fnote{\fbackref{20:4} Or \fbib{Christ}} for a thousand years. \v{5}The rest of the dead did not come back to life until the thousand years were over. This is the first resurrection. \v{6}How blessed and holy are those who participate in the first resurrection! The second death has no power over them. They will be priests of God and the Messiah,\fnote{\fbackref{20:6} Or \fbib{Christ}} and will rule with him for a thousand years.
\passage{The Vision of the Final Judgment}

\v{7}When the thousand years are over, Satan will be freed from his prison. \v{8}He will go out to deceive Gog and Magog, the nations at the four corners of the earth, and gather them for war. They are as numerous as the sands of the seashore. \v{9}They marched over the broad expanse of the earth and surrounded the camp of the saints and the beloved city. Fire came from God\fnote{\fbackref{20:9} Other mss. lack \fbib{from God}} out of heaven and burned them up, \v{10}and the devil who deceived them was thrown into the lake of fire and sulfur, where the beast and the false prophet were. They will be tortured day and night forever and ever.
\passage{The Vision of the White Throne Judgment}

\v{11}Then I saw a large, white throne and the one who was sitting on it. The earth and the heavens fled from his presence, and no place was found for them. \v{12}I saw the dead, both unimportant and important, standing in front of the throne, and books were open. Another book was opened---the Book of Life. The dead were judged according to their actions, as recorded in the books. \v{13}The sea gave up the dead that were in it, and Death and Hades\fnote{\fbackref{20:13} I.e. the realm of the dead} gave up the dead that were in them, and all were judged according to their actions. \v{14}Death and Hades\fnote{\fbackref{20:14} I.e. the realm of the dead} were thrown into the lake of fire. (This is the second death---the lake of fire.) \v{15}Anyone whose name was not found written in the Book of Life was thrown into the lake of fire.
\labelchapt{21}
\passage{The Vision of the New Heaven and the New Earth}

\chapt{21}
\v{1}Then I saw a new heaven and a new earth, because the first heaven and the first earth had disappeared, and the sea was gone. \v{2}I also saw the holy city, New Jerusalem, coming down from God out of heaven, prepared like a bride adorned for her husband.

\v{3}I heard a loud voice from the throne say,

\begin{poetry}
\poeml \red{``See, the tent of God is among humans!} \\
\poemll    \red{He will make his home with them,} \\
\poemlll       \red{and they will be his people.} \\
\poeml \red{God himself will be with them,} \\
\poemll    \red{and he will be their God.}\fnote{\fbackref{21:3} Other mss. lack \fbib{and he will be their God}} \\
\poeml \v{4}\red{He will wipe every tear from their eyes.} \\
\poemll    \red{There won't be death anymore.} \\
\poeml \red{There won't be any grief, crying, or pain,} \\
\poemll    \red{because the first things have disappeared.''}
\end{poetry}

\v{5}The one sitting on the throne said, \red{``See, I am making all things new!''}

He said, \red{``Write this: `These words are trustworthy and true.'\,''}

\v{6}Then he told me, \red{``It has happened! I am the Alpha and the Omega, the beginning and the end. I will freely give a drink from the spring of the water of life to the one who is thirsty. }\v{7}\red{The person who conquers will inherit these things. I will be his God, and he will be my son. }\v{8}\red{But people who are cowardly, unfaithful, detestable, murderers, sexually immoral, sorcerers, idolaters, and all liars will find themselves}\fnote{\fbackref{21:8} Lit. \fbib{will have their part}}\red{ in the lake that burns with fire and sulfur. This is the second death.''}
\passage{The Vision of the New Jerusalem}

\v{9}Then one of the seven angels who had the seven bowls full of the seven last plagues came to me and said, ``Come! I will show you the bride, the wife of the lamb.'' \v{10}He carried me away in the Spirit to a large, high mountain and showed me the holy city, Jerusalem, coming down from God out of heaven. \v{11}The glory of God was its radiance, and its light was like a valuable gem, like jasper, as clear as crystal. \v{12}It had a large, high wall with twelve gates. Twelve angels were at the gates, and the names of the twelve tribes of Israel were written on the gates. \v{13}There were three gates on the east, three gates on the north, three gates on the south, and three gates on the west. \v{14}The wall of the city had twelve foundations, and the twelve names of the twelve apostles of the lamb were written on them.

\v{15}The angel who\fnote{\fbackref{21:15} Lit. \fbib{The one who}} was talking to me had a gold measuring rod to measure the city, its gates, and its walls. \v{16}The city was cubic in shape: its length was the same as its width. He measured the city with his rod, and it measured at 12,000 stadia:\fnote{\fbackref{21:16} I.e. about 1,454 miles long ; one Roman stadion was about 640 feet}Its length, width, and height were the same. \v{17}He also measured its wall. According to the human measurement that the angel was using, it was 144 cubits.\fnote{\fbackref{21:17} I.e. about 84 yards, if a royal cubit (about 21 inches long) is intended; because height, thickness, or width is not specified, since the city structure is three-dimensional, the wall may be a shell that is 144 cubits thick, enclosing the entire structure} \v{18}Its wall was made of jasper. The city was made of pure gold, as clear as glass.

\v{19}The foundations of the city wall were decorated with all kinds of gems: The first foundation\fnote{\fbackref{21:19} The Gk. lacks \fbib{foundation}} was jasper, the second sapphire, the third agate, the fourth emerald, \v{20}the fifth onyx, the sixth carnelian,\fnote{\fbackref{21:20} Or \fbib{olivine}; i.e. an olive-green mineral commonly found in igneous rocks} the seventh chrysolite,\fnote{\fbackref{21:20} I.e. a bright green variety of chalcedony, a transparent silica quartz} the eighth beryl,\fnote{\fbackref{21:20} I.e. an aluminum silicate varying in color from blue, rose, white, and golden, in both opaque and transparent varieties, including emerald and aquamarine} the ninth topaz,\fnote{\fbackref{21:20} I.e. a bright yellow (but occasionally blue, green, brown, pink, or colorless) aluminum silicate variety of sapphire or corundum, and serving as the second stone in the first row in the high priest's breast plate (cf. Exod 28:17)} the tenth chrysoprase,\fnote{\fbackref{21:20} I.e. a bright green variety of chalcedony, a transparent silica quartz} the eleventh jacinth\fnote{\fbackref{21:20} I.e. a red or cinnamon-colored variety of transparent zircon; sometimes also colored deep purple, blue, or violet and serving as the first stone in row three of the high priest's breast plate (cf. Exod 28:19)} and the twelfth amethyst. \v{21}The twelve gates were twelve pearls, and each gate was made of a single pearl. The street of the city was made of pure gold, as clear as glass.

\v{22}I saw no temple in it, because the Lord God Almighty and the lamb are its temple. \v{23}The city doesn't need any sun or moon to give it light, because the glory of God gave it light, and the lamb was its lamp. \v{24}The nations will walk in its light, and the kings of the earth will bring their glory into it. \v{25}Its gates will never be shut at the end of the\fnote{\fbackref{21:25} Lit. \fbib{shut by}} day---because there will be no night there. \v{26}People\fnote{\fbackref{21:26} Lit. \fbib{They}} will bring the glory and wealth\fnote{\fbackref{21:26} Or \fbib{honor}} of the nations into it. \v{27}Nothing unclean, or anyone who does anything detestable, and no one who tells lies will ever enter it. Only those whose names\fnote{\fbackref{21:27} Lit. \fbib{those who}} are written in the lamb's Book of Life will enter it.\fnote{\fbackref{21:27} The Gk. lacks \fbib{will enter it}}
\labelchapt{22}
\passage{The Vision of the River of the Water of Life}

\chapt{22}
\v{1}Then the angel\fnote{\fbackref{22:1} Lit. \fbib{he}} showed me the river of the water of life, as clear as crystal. It was flowing from the throne of God and the lamb. \v{2}Between the city street and the river, the tree of life was visible from each side. It produced twelve kinds of fruit, each month having its own fruit. The leaves of the tree are for the healing of the nations. \v{3}There will no longer be any curse. The throne of God and the lamb will be in the city.\fnote{\fbackref{22:3} Lit. \fbib{in it}} His servants will worship him \v{4}and see his face, and his name will be on their foreheads. \v{5}There will be no more night, and they will not need any light from lamps or the sun because the Lord God will shine on them. They will rule forever and ever.
\passage{He is Coming Soon}

\v{6}He told me, ``These words are trustworthy and true. The Lord God of the spirits and of the prophets has sent his messenger\fnote{\fbackref{22:6} Or \fbib{angel}} to show his servants the things that must happen soon.''

\v{7}\red{``See! I am coming soon! How blessed is the person who keeps the words of the prophecy in this book!''}
\passage{The Vision of an Exhortation to Worship God}

\v{8}I, John, heard and saw these things. When I had heard and seen them, I bowed down to worship at the feet of the angel who had been showing me these things. \v{9}But he told me, ``Don't do that! I am a fellow servant with you, your brothers the prophets, and those who keep the words in this book. Worship God!''

\v{10}Then he told me, ``Don't seal up the words of the prophecy in this book, because the time is near. \v{11}Let the one who does what is evil continue to do what is evil. Let the filthy person continue to be filthy. Let the righteous person continue to do what is right. And let the holy person continue to be holy.''
\passage{Concluding Benediction}

\v{12}\red{``See! I am coming soon! My reward is with me to repay everyone according to his }\red{behavior}\red{.} \v{13}\red{I am the Alpha and the Omega, the first and the last, the beginning and the end.}\fnote{\fbackref{22:13} The speaker possibly concludes his quotation here rather than at the end of vs. 16.}

\v{14}\red{``How blessed are those who wash their robes}\fnote{\fbackref{22:14} Other mss. read \fbib{who do his commandments}} \red{so that they may have the right to the tree of life and may go through the gates into the city!} \v{15}\red{Outside are dogs, sorcerers, immoral people, murderers, idolaters, and everyone who loves and practices falsehood.}

\v{16}\red{``I, Jesus, have sent my angel to give this testimony to you for the churches. I am the root and descend}\red{a}\red{nt of David, the bright morning star.''}
\passage{Concluding Invitation}

\v{17}The Spirit and the bride say, ``Come!''

Let everyone who hears this say, ``Come!''

Let everyone who is thirsty come!

Let anyone who wants the water of life take it as a gift!
\passage{Concluding Warning}

\v{18}I warn everyone who hears the words of the prophecy in this book: If anyone adds anything to them, God will strike him with the plagues that are written in this book. \v{19}If anyone takes away any words from the book of this prophecy, God will take away his portion of the tree of life and the holy city that are described in this book.
\passage{Epilogue}

\v{20}The one who is testifying to these things says, \red{``Yes, I am coming soon!''}

Amen! Come, Lord Jesus!

\v{21}May the grace of the Lord Jesus be with all the saints.

Amen.\fnote{\fbackref{22:21} Other mss. lack \fbib{Amen}}


\end{document}
